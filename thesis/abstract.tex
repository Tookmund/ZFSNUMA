Non-Uniform Memory Access imposes unique challenges on every component of an operating system and the applications that run on it.
One such component is the filesystem which, while not directly impacted by NUMA in most cases, 
typically has some form of cache whose performance is constrained by the latency and bandwidth of the memory that it is stored in.
One such filesystem is ZFS, which contains its own custom caching system, known as the Adaptive Replacement Cache.
This work looks at the impact of NUMA on this cache via sequential read operations, 
shows how current solutions intended to reduce this impact do not adequately account for these caches,
and develops a prototype that reduces the impact of memory affinity by relocating applications to be closer to the caches that they use. 
This prototype is then tested and shown, in some situations, to restore the performance that would otherwise be lost.


% use \newline to start a new paragraph
% I don't know why '\\' (to start newline) not working here. -- Shanhe
