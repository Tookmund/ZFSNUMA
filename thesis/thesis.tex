% Top-level PhD Dissertation file

% Generously contributed by Rance Necaise
%                    PhD, August 1998
%                    Topic:  IMPROVEMENTS TO THE COLOR QUANTIZATION PROCESS
% Published and maintained by Professor William L. Bynum
%                             http://www.cs.wm.edu/~bynum/

% Additional changes by Bob Matthews == rem
%                        PhD
% to conform to Graduate Arts and Sciences Thesis guide of Nov. 2003.
% 

% Further changes done by Ruth Lambrecht
%                    PhD, May 2013
%                    Topic:  Translating Spatial Problems into Lumpable Markov Chains
% with help from Andrew Pyles
%                    PhD, May 2013
%                    Topic:  Network Traffic Aware Smartphone Energy Savings
% to conform to the standards set on 10/08/12.
% 

% Modifications to comply with Physical Standards set on 08/13/2015 done by David T. Nguyen
%            PhD, February 2016 
%            Topic: Enhancing Mobile Device System Using Information from Users and Upper Layers
% Compiling in Ubuntu: use Kile as an editor, and use XeLaTeX button to compile
% Need to instal MS fonts first as follows
%               sudo apt-get install ttf-mscorefonts-installer
%               sudo fc-cache
% After that, check with 
%               fc-match Arial
% Use PDF figures!!! (for some reason EPS figures are not displayed correctly, 
% you can use 'epspdf myfigure.eps' to convert)
%

% Modifications made by Ed Novak
%            PhD, June 2016 
%            Topic: Security And Privacy For Ubiquitous Mobile Devices
% in April 2016 to make the template compile on department machines, via command 
% line (xelatex) and to be easier to use / understand.
%

% At the time (May 2018), please refer 
% http://www.wm.edu/as/graduate/studentresources/thesis-dissertations/physicalstandards/index.php
% for the latest standard.



\documentclass[11pt, final, thesis]{wmthesis}

% Options
% -------
% proposal - if you are writing a proposal, e.g., \documentclass[11pt, proposal]{wmthesis}, no approval page, acknowledge, dedication pages.
% draft - if you are writing a draft, including everything (blank approval page) as ready to be signed
% final - if you are writing a final, will replace approval page with signed_approval_page.pdf
% thesis - if you are a master student, add thesis option, e.g.,  \documentclass[11pt, draft, thesis]{wmthesis}
% dissertation - default

% Refine the toc styple to the latest standard - Shanhe
% \usepackage[titles]{tocloft}
% \newlength\mylength
% \renewcommand\cftchappresnum{\chaptername~}
% \renewcommand\cftchapaftersnum{.}
% \renewcommand{\cftdot}{}
% \settowidth\mylength{\cftchappresnum\cftchapaftersnum\quad}
% \addtolength\cftchapnumwidth{\mylength}


%%----------------------------------------------------------------
% Set packages 
%%----------------------------------------------------------------
%\usepackage{lipsum} % for dummy text
%\usepackage{todonotes} % for dummy image


% Some very useful LaTeX packages include:
% (uncomment the ones you want to load)
% \usepackage{graphicx}
% \usepackage{color}
% \usepackage{url}
% \usepackage{epsfig}
% \usepackage{epstopdf}
% \usepackage{verbatim}
\usepackage[english]{babel}
\usepackage[T1]{fontenc}
\usepackage[sortcites]{biblatex}

\addbibresource{ZFS.bib}

\usepackage{csquotes}
\usepackage[final]{listings}
\usepackage{float}
\usepackage{pstricks}
\usepackage{caption}
\usepackage{subcaption}
\usepackage{graphicx}
\usepackage{fullpage}


%%----------------------------------------------------------------
% Set font 
%%----------------------------------------------------------------
%\ifxetexorluatex
%    \usepackage{fontspec} % only use it with XeLaTeX or LuaLaTeX, recommended. -- Shanhe
%    \setmainfont{Arial}
%\else
    % %%-- This is how to set Arial font using LaTeX or PdfLaTeX, not recommended. -- Shanhe
    % %% If you don't install the font into tex packages, then you will just get Computer Modern font.
    % %% How to install uarial package in Mac OS (required MacTeX, /Library/TeX/texbin in your path). *nix with TexLive should be similar. - Shanhe
    % %%   1. curl --remote-name https://www.tug.org/fonts/getnonfreefonts/install-getnonfreefonts
    % %%   2. sudo texlua install-getnonfreefonts
    % %%   3. sudo getnonfreefonts --sys -a
    % %% Then uncommen the following two lines.
    % \usepackage{uarial}
    % \renewcommand{\familydefault}{\sfdefault}
%\fi


%%----------------------------------------------------------------
% Set indent 
%%----------------------------------------------------------------
% The wmthesis class is based on the latex report class which
% only indents paragraphs if they immediately follow other paragraphs.  The
% dissertation lady says this is wrong.  I tend to give more credence
% to Dr. Knuth (author of TeX) on this issue, since the other way looks really
% crappy.  If you want the first line of every paragraph indented,
% uncomment the next line to include the indentfirst package. -- rem
% Not sure if this is still an option -- Ruth
% Not an issue any more, but I keep this line in case -- Shanhe
% \usepackage{indentfirst}

\definecolor{dkgreen}{rgb}{0,0.6,0}

\lstset{frame=tb,
  upquote=true,
  captionpos=b,
  language=Python,
  aboveskip=3mm,
  belowskip=3mm,
  showstringspaces=false,
  columns=flexible,
  basicstyle={\small\ttfamily},
  numbers=left,
  numberstyle=\tiny\color{gray},
  keywordstyle=\color{blue},
  commentstyle=\color{dkgreen},
  stringstyle=\color{red},
  breaklines=true,
  breakatwhitespace=true,
  tabsize=4
}

% https://tex.stackexchange.com/a/106129
% https://tex.stackexchange.com/a/50263
  \lstdefinelanguage{diff}{
    morecomment=[f][\color{blue}]{@@},
    morecomment=[f][\color{dkgreen}]{+},
    morecomment=[f][\color{red}]{-},
  }




\begin{document}
\doublespacing

%%--Set thesis metadata


% Provide the full title of your thesis or dissertation using the format for upper and lower case as indicated.
\thesisTitle{Performance Implications of Memory Affinity on Filesystem Caches \\ in a Non-Uniform Memory Access Environment}

%  \thesisAuthor macro
%     defines two TeX variables (only usable in this file)
%  \thesis@author  is assumed to be a "short" version of the author's name
%        used on the title page
%  \thesis@authorx is assumed to be the full name of the author
%        used on the approval, the UMI abstract, and the vita pages
%  For example
%     \thesisAuthor{A. Goode Student}
%        sets both \thesis@author and \ thesis@authorx to
%        "A. Goode Student"
%     \thesisAuthor[Aloysius Goode Student]{A. Goode Student}
%        sets \thesis@author  to "A. Goode Student" and
%             \thesis@authorx to "Aloysius Goode Student"
\thesisAuthor{Jacob Allen Adams}

% Enter the expected conferment month and year of your degree, [e.g. January, May, August]. 
% Check with the Office of Graduate Studies and Research to verify the actual graduation month and year.
\thesisMonth{May}
\thesisYear{2021}
\thesisAdvisor{Professor Jim Deverick} % Advisor name with title

% Enter Degree, e.g. Master of Science or Master of Arts or Doctor of Philosophy.
\thesisType{Bachelor of Science}


% "Provide your hometown and state in the following format [e.g. St. Louis, Missouri]. 
% International students should enter their hometown, state/province, and country [e.g. Montreal, Quebec, Canada]"
\thesisHometown{Fairfax, Virginia}

%%-- Degrees earned previous to Ph.D.
%% note that the degree should be spelled out, not abbreviated
% "List all previous degrees with the most recent degree first,
% [e.g. Master of Arts,University of Colorado-Boulder, 1987]."
%\thesisDegreeOne{Master of Science, ABC College, 2013}
%\thesisDegreeTwo{Bachelor of Engineer, ABC College, 2010}

% Enter your department name, [e.g. Department of History, Department of Biology, or American Studies Program].
\thesisDepartment{Department of Computer Science}

%%-- Committee members
\thesisCommittee[Senior Lecturer, Computer Science]{James Deverick}{William \& Mary}
\thesisCommittee[Associate Professor, Linguistics]{Dan Parker}{William \& Mary}
\thesisCommittee[Assistant Professor, Computer Science]{Pradeep Kumar}{William \& Mary}
\thesisCommittee[Assistant Professor, Computer Science]{Jiajia Li}{William \& Mary}


%%-- Insert contents of abstract.tex, acknowledge.tex and the dedication.  Don't
%%forget to check these files for formatting hints.
%% Also, the order they are given right here does not matter
\thesisAbstract{abstract.tex}

% In the actual finished pdf document the TOC comes next
\thesisAcknowledge{acknowledge.tex}

%\thesisDedication{}



%%--Create the thesis Prolog
\makeProlog


%%-- contents of the actual thesis feel free to \input as many files as you want
\chapter{Introduction}
Data access is a critical component of modern computing and often the bottleneck of performance for many applications,
especially as processors have continued to get faster and faster over time.
Because data storage that is the quickest to access is also the most expensive, 
all the data in a system cannot be available at one time equally quickly,
and instead a hierarchy of data access was created, where the higher levels can store less data but are quicker to access and the
lower levels can store more data but are slower to access.
This fundamental principle of computing was a major consideration in the design of
computers as long ago as 1946\cite{burks_preliminary_1946}.

\begin{figure}[H]
    \centering
    \resizebox{0.5\linewidth}{!}{%LaTeX with PSTricks extensions
%%Creator: Inkscape 1.0.2-2 (e86c870879, 2021-01-15)
%%Please note this file requires PSTricks extensions
\psset{xunit=.5pt,yunit=.5pt,runit=.5pt}
\begin{pspicture}(449.38738966,289.12711076)
{
\newrgbcolor{curcolor}{0.7019608 0.7019608 0.7019608}
\pscustom[linestyle=none,fillstyle=solid,fillcolor=curcolor,opacity=0.50196099]
{
\newpath
\moveto(0.94487903,71.65556298)
\lineto(448.44250491,71.65556298)
\lineto(448.44250491,0.94488789)
\lineto(0.94487903,0.94488789)
\closepath
}
}
{
\newrgbcolor{curcolor}{0 0 0}
\pscustom[linewidth=1.88976378,linecolor=curcolor]
{
\newpath
\moveto(0.94487903,71.65556298)
\lineto(448.44250491,71.65556298)
\lineto(448.44250491,0.94488789)
\lineto(0.94487903,0.94488789)
\closepath
}
}
{
\newrgbcolor{curcolor}{0.7019608 0.7019608 0.7019608}
\pscustom[linestyle=none,fillstyle=solid,fillcolor=curcolor,opacity=0.50196099]
{
\newpath
\moveto(96.00523655,143.05181667)
\lineto(353.38217622,143.05181667)
\lineto(353.38217622,72.27599496)
\lineto(96.00523655,72.27599496)
\closepath
}
}
{
\newrgbcolor{curcolor}{0 0 0}
\pscustom[linewidth=1.88976378,linecolor=curcolor]
{
\newpath
\moveto(96.00523655,143.05181667)
\lineto(353.38217622,143.05181667)
\lineto(353.38217622,72.27599496)
\lineto(96.00523655,72.27599496)
\closepath
}
}
{
\newrgbcolor{curcolor}{0.7019608 0.7019608 0.7019608}
\pscustom[linestyle=none,fillstyle=solid,fillcolor=curcolor,opacity=0.50196099]
{
\newpath
\moveto(145.85320615,214.38809251)
\lineto(303.53414895,214.38809251)
\lineto(303.53414895,143.30317579)
\lineto(145.85320615,143.30317579)
\closepath
}
}
{
\newrgbcolor{curcolor}{0 0 0}
\pscustom[linewidth=1.88976378,linecolor=curcolor]
{
\newpath
\moveto(145.85320615,214.38809251)
\lineto(303.53414895,214.38809251)
\lineto(303.53414895,143.30317579)
\lineto(145.85320615,143.30317579)
\closepath
}
}
{
\newrgbcolor{curcolor}{0.7019608 0.7019608 0.7019608}
\pscustom[linestyle=none,fillstyle=solid,fillcolor=curcolor,opacity=0.50196099]
{
\newpath
\moveto(182.07068931,285.85562037)
\lineto(267.31668021,285.85562037)
\lineto(267.31668021,214.47467833)
\lineto(182.07068931,214.47467833)
\closepath
}
}
{
\newrgbcolor{curcolor}{0 0 0}
\pscustom[linewidth=1.88976378,linecolor=curcolor]
{
\newpath
\moveto(182.07068931,285.85562037)
\lineto(267.31668021,285.85562037)
\lineto(267.31668021,214.47467833)
\lineto(182.07068931,214.47467833)
\closepath
}
}
{
\newrgbcolor{curcolor}{0 0 0}
\pscustom[linestyle=none,fillstyle=solid,fillcolor=curcolor]
{
\newpath
\moveto(198.44761957,248.81915835)
\lineto(196.43980876,248.81915835)
\lineto(192.54918703,253.44415446)
\lineto(190.36950136,253.44415446)
\lineto(190.36950136,248.81915835)
\lineto(188.82262766,248.81915835)
\lineto(188.82262766,260.45196106)
\lineto(192.08043742,260.45196106)
\curveto(192.78356183,260.45196106)(193.36949884,260.4050861)(193.83824844,260.31133618)
\curveto(194.30699805,260.22279459)(194.72887269,260.06133639)(195.10387238,259.82696159)
\curveto(195.52574702,259.56133681)(195.85387175,259.2253996)(196.08824655,258.81914994)
\curveto(196.32782968,258.41810861)(196.44762125,257.90769237)(196.44762125,257.28790123)
\curveto(196.44762125,256.44936026)(196.23668393,255.74623586)(195.81480928,255.178528)
\curveto(195.39293464,254.61602847)(194.81220596,254.19154966)(194.07262325,253.90509157)
\closepath
\moveto(194.83043511,257.17852632)
\curveto(194.83043511,257.51185937)(194.77053933,257.80612996)(194.65074776,258.06133808)
\curveto(194.53616452,258.32175452)(194.34345635,258.54050434)(194.07262325,258.71758752)
\curveto(193.8486651,258.86862906)(193.58304033,258.97279564)(193.27574892,259.03008726)
\curveto(192.96845751,259.09258721)(192.60647865,259.12383718)(192.18981233,259.12383718)
\lineto(190.36950136,259.12383718)
\lineto(190.36950136,254.73321587)
\lineto(191.93200005,254.73321587)
\curveto(192.42158297,254.73321587)(192.84866594,254.77488251)(193.21324897,254.85821577)
\curveto(193.577832,254.94675736)(193.88772757,255.10821556)(194.14293569,255.34259036)
\curveto(194.37731049,255.56134018)(194.54918535,255.81133997)(194.65856025,256.09258973)
\curveto(194.77314349,256.37904782)(194.83043511,256.74102669)(194.83043511,257.17852632)
\closepath
}
}
{
\newrgbcolor{curcolor}{0 0 0}
\pscustom[linestyle=none,fillstyle=solid,fillcolor=curcolor]
{
\newpath
\moveto(206.74448779,253.03009231)
\lineto(200.31480569,253.03009231)
\curveto(200.31480569,252.49363442)(200.39553479,252.02488482)(200.55699299,251.62384349)
\curveto(200.71845119,251.22801049)(200.93980517,250.90248993)(201.22105493,250.64728181)
\curveto(201.49188804,250.39728202)(201.81220027,250.20978218)(202.18199162,250.08478228)
\curveto(202.55699131,249.95978239)(202.9684493,249.89728244)(203.41636559,249.89728244)
\curveto(204.01011509,249.89728244)(204.60646875,250.01446984)(205.20542658,250.24884465)
\curveto(205.80959274,250.48842778)(206.23927988,250.72280258)(206.494488,250.95196905)
\lineto(206.57261293,250.95196905)
\lineto(206.57261293,249.3504079)
\curveto(206.07782168,249.14207474)(205.57261377,248.96759572)(205.05698921,248.82697084)
\curveto(204.54136464,248.68634596)(203.99969843,248.61603352)(203.43199057,248.61603352)
\curveto(201.98407512,248.61603352)(200.85386774,249.00665819)(200.04136842,249.78790753)
\curveto(199.22886911,250.57436521)(198.82261945,251.6889476)(198.82261945,253.13165472)
\curveto(198.82261945,254.55873685)(199.21063996,255.6915484)(199.98668097,256.53008936)
\curveto(200.76793031,257.36863032)(201.79397112,257.78790081)(203.06480338,257.78790081)
\curveto(204.24188572,257.78790081)(205.14813496,257.44415109)(205.78355109,256.75665167)
\curveto(206.42417556,256.06915225)(206.74448779,255.09259057)(206.74448779,253.82696664)
\closepath
\moveto(205.31480149,254.15509136)
\curveto(205.30959316,254.92592405)(205.11428082,255.52227771)(204.72886448,255.94415236)
\curveto(204.34865647,256.366027)(203.76792779,256.57696432)(202.98667845,256.57696432)
\curveto(202.20022078,256.57696432)(201.57261714,256.34519369)(201.10386753,255.88165241)
\curveto(200.64032625,255.41811113)(200.37730564,254.84259078)(200.31480569,254.15509136)
\closepath
}
}
{
\newrgbcolor{curcolor}{0 0 0}
\pscustom[linestyle=none,fillstyle=solid,fillcolor=curcolor]
{
\newpath
\moveto(216.04916797,249.81134501)
\curveto(216.04916797,248.33217959)(215.71323075,247.246243)(215.04135632,246.55353525)
\curveto(214.36948188,245.8608275)(213.33562858,245.51447363)(211.93979642,245.51447363)
\curveto(211.47625515,245.51447363)(211.02313053,245.54832777)(210.58042257,245.61603604)
\curveto(210.14292294,245.67853599)(209.71063163,245.76968175)(209.28354866,245.88947331)
\lineto(209.28354866,247.38947205)
\lineto(209.36167359,247.38947205)
\curveto(209.60125672,247.29572213)(209.98146474,247.18113889)(210.50229763,247.04572234)
\curveto(211.02313053,246.90509746)(211.54396342,246.83478502)(212.06479632,246.83478502)
\curveto(212.5647959,246.83478502)(212.97885805,246.8946808)(213.30698277,247.01447237)
\curveto(213.6351075,247.13426393)(213.89031562,247.30093046)(214.07260713,247.51447195)
\curveto(214.25489864,247.71759677)(214.38510687,247.96238824)(214.4632318,248.24884633)
\curveto(214.54135674,248.53530442)(214.5804192,248.85561665)(214.5804192,249.20978302)
\lineto(214.5804192,250.00665735)
\curveto(214.13771124,249.65249098)(213.71323243,249.3868662)(213.30698277,249.20978302)
\curveto(212.90594144,249.03790816)(212.39292104,248.95197074)(211.76792157,248.95197074)
\curveto(210.72625578,248.95197074)(209.89813147,249.32697042)(209.28354866,250.07696979)
\curveto(208.67417417,250.83217749)(208.36948693,251.8946766)(208.36948693,253.26446711)
\curveto(208.36948693,254.01446648)(208.47365351,254.66029927)(208.68198666,255.20196548)
\curveto(208.89552815,255.74884002)(209.18459041,256.22019379)(209.54917343,256.61602679)
\curveto(209.88771482,256.98581815)(210.2991728,257.27227624)(210.7835474,257.47540107)
\curveto(211.26792199,257.68373423)(211.74969242,257.78790081)(212.22885868,257.78790081)
\curveto(212.73406659,257.78790081)(213.15594123,257.73581752)(213.49448262,257.63165094)
\curveto(213.83823233,257.53269269)(214.20021119,257.37904698)(214.5804192,257.17071382)
\lineto(214.67416912,257.54571351)
\lineto(216.04916797,257.54571351)
\closepath
\moveto(214.5804192,251.21759383)
\lineto(214.5804192,255.97540233)
\curveto(214.18979453,256.15248551)(213.8252115,256.27748541)(213.48667012,256.35040201)
\curveto(213.15333707,256.42852695)(212.82000402,256.46758942)(212.48667096,256.46758942)
\curveto(211.67937998,256.46758942)(211.04396384,256.19675631)(210.58042257,255.6550901)
\curveto(210.11688129,255.11342389)(209.88511065,254.32696622)(209.88511065,253.29571708)
\curveto(209.88511065,252.31655124)(210.05698551,251.57436436)(210.40073522,251.06915646)
\curveto(210.74448493,250.56394855)(211.31479695,250.31134459)(212.11167128,250.31134459)
\curveto(212.53875425,250.31134459)(212.96583723,250.39207369)(213.3929202,250.55353189)
\curveto(213.8252115,250.72019842)(214.22104451,250.9415524)(214.5804192,251.21759383)
\closepath
}
}
{
\newrgbcolor{curcolor}{0 0 0}
\pscustom[linestyle=none,fillstyle=solid,fillcolor=curcolor]
{
\newpath
\moveto(220.5179137,259.00664978)
\lineto(218.8616651,259.00664978)
\lineto(218.8616651,260.530086)
\lineto(220.5179137,260.530086)
\closepath
\moveto(220.42416378,248.81915835)
\lineto(218.95541502,248.81915835)
\lineto(218.95541502,257.54571351)
\lineto(220.42416378,257.54571351)
\closepath
}
}
{
\newrgbcolor{curcolor}{0 0 0}
\pscustom[linestyle=none,fillstyle=solid,fillcolor=curcolor]
{
\newpath
\moveto(229.56478022,251.33478123)
\curveto(229.56478022,250.5379069)(229.23405133,249.88426162)(228.57259356,249.37384538)
\curveto(227.91634411,248.86342914)(227.01790736,248.60822103)(225.87728332,248.60822103)
\curveto(225.23145053,248.60822103)(224.63770103,248.6837418)(224.09603482,248.83478334)
\curveto(223.55957694,248.9910332)(223.10905648,249.16030389)(222.74447346,249.34259541)
\lineto(222.74447346,250.99103152)
\lineto(222.82259839,250.99103152)
\curveto(223.28613967,250.64207348)(223.80176423,250.36342788)(224.36947209,250.15509472)
\curveto(224.93717995,249.9519699)(225.48145032,249.85040748)(226.00228322,249.85040748)
\curveto(226.64811601,249.85040748)(227.15332392,249.95457406)(227.51790694,250.16290722)
\curveto(227.88248997,250.37124038)(228.06478148,250.6993651)(228.06478148,251.14728139)
\curveto(228.06478148,251.4910311)(227.96582323,251.75144755)(227.76790673,251.92853073)
\curveto(227.56999023,252.10561392)(227.18978222,252.25665546)(226.62728269,252.38165535)
\curveto(226.41894953,252.42853031)(226.14551226,252.48321777)(225.80697088,252.54571771)
\curveto(225.47363783,252.60821766)(225.16895058,252.67592594)(224.89290915,252.74884254)
\curveto(224.12728479,252.95196737)(223.58301442,253.24884212)(223.26009802,253.63946679)
\curveto(222.94238996,254.03529979)(222.78353592,254.51967439)(222.78353592,255.09259057)
\curveto(222.78353592,255.45196527)(222.85645253,255.79050665)(223.00228574,256.10821472)
\curveto(223.15332728,256.42592278)(223.37988959,256.70977671)(223.68197267,256.9597765)
\curveto(223.97363909,257.20456796)(224.34343045,257.39727613)(224.79134674,257.53790102)
\curveto(225.24447135,257.68373423)(225.74967926,257.75665083)(226.30697046,257.75665083)
\curveto(226.82780336,257.75665083)(227.35384458,257.69154672)(227.88509413,257.5613385)
\curveto(228.42155202,257.4363386)(228.86686414,257.2826929)(229.22103051,257.10040138)
\lineto(229.22103051,255.5300902)
\lineto(229.14290558,255.5300902)
\curveto(228.76790589,255.80613164)(228.31217711,256.03790228)(227.77571923,256.22540212)
\curveto(227.23926134,256.41811029)(226.71322012,256.51446438)(226.19759555,256.51446438)
\curveto(225.66113767,256.51446438)(225.20801305,256.4102978)(224.8382217,256.20196464)
\curveto(224.46843034,255.99883981)(224.28353466,255.69415257)(224.28353466,255.28790291)
\curveto(224.28353466,254.92852821)(224.39551374,254.6576951)(224.61947188,254.47540359)
\curveto(224.8382217,254.29311208)(225.19238806,254.1446747)(225.68197099,254.03009147)
\curveto(225.95280409,253.96759152)(226.25488717,253.90509157)(226.58822022,253.84259162)
\curveto(226.92676161,253.78009168)(227.20801137,253.72280006)(227.43196951,253.67071677)
\curveto(228.11426061,253.5144669)(228.64030183,253.24623796)(229.01009319,252.86602994)
\curveto(229.37988454,252.4806136)(229.56478022,251.97019736)(229.56478022,251.33478123)
\closepath
}
}
{
\newrgbcolor{curcolor}{0 0 0}
\pscustom[linestyle=none,fillstyle=solid,fillcolor=curcolor]
{
\newpath
\moveto(236.19758718,248.89728328)
\curveto(235.92154575,248.82436668)(235.61946267,248.76447089)(235.29133795,248.71759593)
\curveto(234.96842155,248.67072097)(234.67935929,248.64728349)(234.42415118,248.64728349)
\curveto(233.53352693,248.64728349)(232.85644416,248.88686662)(232.39290288,249.36603289)
\curveto(231.92936161,249.84519915)(231.69759097,250.61342767)(231.69759097,251.67071845)
\lineto(231.69759097,256.31133955)
\lineto(230.7054043,256.31133955)
\lineto(230.7054043,257.54571351)
\lineto(231.69759097,257.54571351)
\lineto(231.69759097,260.0535239)
\lineto(233.16633973,260.0535239)
\lineto(233.16633973,257.54571351)
\lineto(236.19758718,257.54571351)
\lineto(236.19758718,256.31133955)
\lineto(233.16633973,256.31133955)
\lineto(233.16633973,252.33478039)
\curveto(233.16633973,251.87644744)(233.17675639,251.51707275)(233.19758971,251.2566563)
\curveto(233.21842302,251.00144818)(233.29133963,250.76186505)(233.41633952,250.5379069)
\curveto(233.53092276,250.32957374)(233.68717263,250.17592804)(233.88508913,250.07696979)
\curveto(234.08821396,249.98321987)(234.39550537,249.93634491)(234.80696335,249.93634491)
\curveto(235.04654649,249.93634491)(235.29654628,249.97019905)(235.55696272,250.03790732)
\curveto(235.81737917,250.11082393)(236.00487901,250.17071971)(236.11946225,250.21759467)
\lineto(236.19758718,250.21759467)
\closepath
}
}
{
\newrgbcolor{curcolor}{0 0 0}
\pscustom[linestyle=none,fillstyle=solid,fillcolor=curcolor]
{
\newpath
\moveto(245.27570408,253.03009231)
\lineto(238.84602199,253.03009231)
\curveto(238.84602199,252.49363442)(238.92675109,252.02488482)(239.08820929,251.62384349)
\curveto(239.24966749,251.22801049)(239.47102147,250.90248993)(239.75227123,250.64728181)
\curveto(240.02310434,250.39728202)(240.34341657,250.20978218)(240.71320792,250.08478228)
\curveto(241.08820761,249.95978239)(241.49966559,249.89728244)(241.94758188,249.89728244)
\curveto(242.54133138,249.89728244)(243.13768505,250.01446984)(243.73664288,250.24884465)
\curveto(244.34080904,250.48842778)(244.77049618,250.72280258)(245.0257043,250.95196905)
\lineto(245.10382923,250.95196905)
\lineto(245.10382923,249.3504079)
\curveto(244.60903798,249.14207474)(244.10383007,248.96759572)(243.5882055,248.82697084)
\curveto(243.07258094,248.68634596)(242.53091473,248.61603352)(241.96320687,248.61603352)
\curveto(240.51529142,248.61603352)(239.38508404,249.00665819)(238.57258472,249.78790753)
\curveto(237.76008541,250.57436521)(237.35383575,251.6889476)(237.35383575,253.13165472)
\curveto(237.35383575,254.55873685)(237.74185625,255.6915484)(238.51789727,256.53008936)
\curveto(239.29914661,257.36863032)(240.32518742,257.78790081)(241.59601968,257.78790081)
\curveto(242.77310202,257.78790081)(243.67935126,257.44415109)(244.31476739,256.75665167)
\curveto(244.95539185,256.06915225)(245.27570408,255.09259057)(245.27570408,253.82696664)
\closepath
\moveto(243.84601779,254.15509136)
\curveto(243.84080946,254.92592405)(243.64549712,255.52227771)(243.26008078,255.94415236)
\curveto(242.87987277,256.366027)(242.29914409,256.57696432)(241.51789475,256.57696432)
\curveto(240.73143707,256.57696432)(240.10383343,256.34519369)(239.63508383,255.88165241)
\curveto(239.17154255,255.41811113)(238.90852194,254.84259078)(238.84602199,254.15509136)
\closepath
}
}
{
\newrgbcolor{curcolor}{0 0 0}
\pscustom[linestyle=none,fillstyle=solid,fillcolor=curcolor]
{
\newpath
\moveto(252.94757224,255.94415236)
\lineto(252.8694473,255.94415236)
\curveto(252.65069749,255.99623565)(252.437156,256.03269395)(252.22882284,256.05352726)
\curveto(252.02569801,256.07956891)(251.78351072,256.09258973)(251.50226095,256.09258973)
\curveto(251.04913634,256.09258973)(250.6116367,255.99102732)(250.18976206,255.78790249)
\curveto(249.76788741,255.58998599)(249.36163775,255.3321737)(248.97101308,255.01446564)
\lineto(248.97101308,248.81915835)
\lineto(247.50226432,248.81915835)
\lineto(247.50226432,257.54571351)
\lineto(248.97101308,257.54571351)
\lineto(248.97101308,256.25665209)
\curveto(249.55434593,256.7254017)(250.06736633,257.05613059)(250.51007429,257.24883876)
\curveto(250.95799058,257.44675526)(251.41371936,257.54571351)(251.87726064,257.54571351)
\curveto(252.13246876,257.54571351)(252.31736444,257.53790102)(252.43194767,257.52227603)
\curveto(252.54653091,257.51185937)(252.71840576,257.48842189)(252.94757224,257.45196359)
\closepath
}
}
{
\newrgbcolor{curcolor}{0 0 0}
\pscustom[linestyle=none,fillstyle=solid,fillcolor=curcolor]
{
\newpath
\moveto(260.56475235,251.33478123)
\curveto(260.56475235,250.5379069)(260.23402346,249.88426162)(259.57256568,249.37384538)
\curveto(258.91631623,248.86342914)(258.01787949,248.60822103)(256.87725545,248.60822103)
\curveto(256.23142266,248.60822103)(255.63767316,248.6837418)(255.09600695,248.83478334)
\curveto(254.55954906,248.9910332)(254.10902861,249.16030389)(253.74444558,249.34259541)
\lineto(253.74444558,250.99103152)
\lineto(253.82257052,250.99103152)
\curveto(254.28611179,250.64207348)(254.80173636,250.36342788)(255.36944422,250.15509472)
\curveto(255.93715207,249.9519699)(256.48142245,249.85040748)(257.00225534,249.85040748)
\curveto(257.64808813,249.85040748)(258.15329604,249.95457406)(258.51787907,250.16290722)
\curveto(258.88246209,250.37124038)(259.06475361,250.6993651)(259.06475361,251.14728139)
\curveto(259.06475361,251.4910311)(258.96579536,251.75144755)(258.76787886,251.92853073)
\curveto(258.56996236,252.10561392)(258.18975434,252.25665546)(257.62725482,252.38165535)
\curveto(257.41892166,252.42853031)(257.14548439,252.48321777)(256.80694301,252.54571771)
\curveto(256.47360995,252.60821766)(256.16892271,252.67592594)(255.89288128,252.74884254)
\curveto(255.12725692,252.95196737)(254.58298654,253.24884212)(254.26007015,253.63946679)
\curveto(253.94236208,254.03529979)(253.78350805,254.51967439)(253.78350805,255.09259057)
\curveto(253.78350805,255.45196527)(253.85642466,255.79050665)(254.00225787,256.10821472)
\curveto(254.15329941,256.42592278)(254.37986171,256.70977671)(254.68194479,256.9597765)
\curveto(254.97361122,257.20456796)(255.34340257,257.39727613)(255.79131886,257.53790102)
\curveto(256.24444348,257.68373423)(256.74965139,257.75665083)(257.30694259,257.75665083)
\curveto(257.82777548,257.75665083)(258.35381671,257.69154672)(258.88506626,257.5613385)
\curveto(259.42152414,257.4363386)(259.86683627,257.2826929)(260.22100264,257.10040138)
\lineto(260.22100264,255.5300902)
\lineto(260.1428777,255.5300902)
\curveto(259.76787802,255.80613164)(259.31214923,256.03790228)(258.77569135,256.22540212)
\curveto(258.23923347,256.41811029)(257.71319224,256.51446438)(257.19756768,256.51446438)
\curveto(256.6611098,256.51446438)(256.20798518,256.4102978)(255.83819382,256.20196464)
\curveto(255.46840247,255.99883981)(255.28350679,255.69415257)(255.28350679,255.28790291)
\curveto(255.28350679,254.92852821)(255.39548586,254.6576951)(255.61944401,254.47540359)
\curveto(255.83819382,254.29311208)(256.19236019,254.1446747)(256.68194311,254.03009147)
\curveto(256.95277622,253.96759152)(257.2548593,253.90509157)(257.58819235,253.84259162)
\curveto(257.92673373,253.78009168)(258.2079835,253.72280006)(258.43194164,253.67071677)
\curveto(259.11423273,253.5144669)(259.64027396,253.24623796)(260.01006531,252.86602994)
\curveto(260.37985667,252.4806136)(260.56475235,251.97019736)(260.56475235,251.33478123)
\closepath
}
}
{
\newrgbcolor{curcolor}{0 0 0}
\pscustom[linestyle=none,fillstyle=solid,fillcolor=curcolor]
{
\newpath
\moveto(187.48668068,174.80378351)
\curveto(187.20022259,174.67878361)(186.93980614,174.56159621)(186.70543134,174.4522213)
\curveto(186.47626487,174.3428464)(186.17418179,174.22826316)(185.7991821,174.10847159)
\curveto(185.48147404,174.00951334)(185.13512016,173.92618008)(184.76012048,173.8584718)
\curveto(184.39032912,173.7855552)(183.9814753,173.74909689)(183.53355901,173.74909689)
\curveto(182.68980972,173.74909689)(181.9215812,173.8662843)(181.22887345,174.1006591)
\curveto(180.54137402,174.34024223)(179.9424162,174.71263775)(179.43199996,175.21784566)
\curveto(178.93200038,175.71263691)(178.54137571,176.34024055)(178.26012594,177.10065658)
\curveto(177.97887618,177.86628093)(177.8382513,178.75430102)(177.8382513,179.76471684)
\curveto(177.8382513,180.72304936)(177.97366785,181.57981948)(178.24450096,182.33502717)
\curveto(178.51533406,183.09023487)(178.90595873,183.72825517)(179.41637497,184.24908806)
\curveto(179.91116622,184.75429597)(180.50751989,185.13971231)(181.20543597,185.40533709)
\curveto(181.90856038,185.67096187)(182.68720555,185.80377426)(183.5413715,185.80377426)
\curveto(184.16637098,185.80377426)(184.78876629,185.72825349)(185.40855743,185.57721195)
\curveto(186.03355691,185.42617041)(186.72626466,185.16054563)(187.48668068,184.78033762)
\lineto(187.48668068,182.94440166)
\lineto(187.36949328,182.94440166)
\curveto(186.72886882,183.48085954)(186.09345269,183.87148421)(185.46324489,184.11627568)
\curveto(184.83303708,184.36106714)(184.15855848,184.48346287)(183.43980909,184.48346287)
\curveto(182.85126792,184.48346287)(182.32001836,184.38710878)(181.84606043,184.19440061)
\curveto(181.37731082,184.00690077)(180.95804034,183.71263018)(180.58824899,183.31158885)
\curveto(180.22887429,182.92096418)(179.94762452,182.42617293)(179.7444997,181.8272151)
\curveto(179.54658319,181.2334656)(179.44762494,180.54596618)(179.44762494,179.76471684)
\curveto(179.44762494,178.94700919)(179.55699985,178.24388478)(179.77574967,177.65534361)
\curveto(179.99970781,177.06680244)(180.28616591,176.58763617)(180.63512395,176.21784482)
\curveto(180.99970697,175.83242848)(181.42418578,175.54597038)(181.90856038,175.35847054)
\curveto(182.3981433,175.17617903)(182.91376786,175.08503327)(183.45543407,175.08503327)
\curveto(184.20022511,175.08503327)(184.89814119,175.21263733)(185.54918231,175.46784545)
\curveto(186.20022343,175.72305357)(186.80959792,176.10586575)(187.37730578,176.61628198)
\lineto(187.48668068,176.61628198)
\closepath
}
}
{
\newrgbcolor{curcolor}{0 0 0}
\pscustom[linestyle=none,fillstyle=solid,fillcolor=curcolor]
{
\newpath
\moveto(197.28354753,182.07721489)
\curveto(197.28354753,181.56159032)(197.19240177,181.08242406)(197.01011026,180.6397161)
\curveto(196.83302707,180.20221647)(196.58302728,179.82200845)(196.26011089,179.49909206)
\curveto(195.85906956,179.09805073)(195.38511163,178.79596765)(194.83823709,178.59284282)
\curveto(194.29136255,178.39492632)(193.60125896,178.29596807)(192.76792633,178.29596807)
\lineto(191.22105263,178.29596807)
\lineto(191.22105263,173.96003422)
\lineto(189.67417893,173.96003422)
\lineto(189.67417893,185.59283693)
\lineto(192.83042627,185.59283693)
\curveto(193.52834235,185.59283693)(194.11948769,185.53294115)(194.60386228,185.41314958)
\curveto(195.08823688,185.29856635)(195.51792401,185.11627483)(195.8929237,184.86627504)
\curveto(196.33563166,184.56940029)(196.67677721,184.19960894)(196.91636034,183.75690098)
\curveto(197.1611518,183.31419302)(197.28354753,182.75429765)(197.28354753,182.07721489)
\closepath
\moveto(195.67417388,182.03815242)
\curveto(195.67417388,182.43919375)(195.60386144,182.78815179)(195.46323656,183.08502654)
\curveto(195.32261168,183.38190129)(195.10907019,183.62408859)(194.8226121,183.81158843)
\curveto(194.57261231,183.97304663)(194.28615422,184.08762987)(193.96323782,184.15533814)
\curveto(193.64552976,184.22825475)(193.24188426,184.26471305)(192.75230134,184.26471305)
\lineto(191.22105263,184.26471305)
\lineto(191.22105263,179.61627946)
\lineto(192.52573903,179.61627946)
\curveto(193.1507385,179.61627946)(193.65855058,179.67096691)(194.04917525,179.78034182)
\curveto(194.43979992,179.89492506)(194.75750799,180.07461241)(195.00229945,180.31940387)
\curveto(195.24709091,180.56940366)(195.41896576,180.83242427)(195.51792401,181.10846571)
\curveto(195.62209059,181.38450714)(195.67417388,181.69440271)(195.67417388,182.03815242)
\closepath
}
}
{
\newrgbcolor{curcolor}{0 0 0}
\pscustom[linestyle=none,fillstyle=solid,fillcolor=curcolor]
{
\newpath
\moveto(208.08041405,178.63190529)
\curveto(208.08041405,177.788156)(207.98666413,177.05117745)(207.79916429,176.42096965)
\curveto(207.61687278,175.79597017)(207.3147897,175.27513728)(206.89291505,174.85847096)
\curveto(206.49187372,174.46263796)(206.02312412,174.1735757)(205.48666623,173.99128419)
\curveto(204.95020835,173.80899268)(204.32520888,173.71784692)(203.61166781,173.71784692)
\curveto(202.88250176,173.71784692)(202.24708563,173.81420101)(201.70541942,174.00690918)
\curveto(201.1637532,174.19961735)(200.70802442,174.48347128)(200.33823307,174.85847096)
\curveto(199.91635842,175.28555394)(199.61167118,175.8011785)(199.42417133,176.40534466)
\curveto(199.24187982,177.00951082)(199.15073406,177.75169769)(199.15073406,178.63190529)
\lineto(199.15073406,185.59283693)
\lineto(200.69760776,185.59283693)
\lineto(200.69760776,178.55378035)
\curveto(200.69760776,177.92357255)(200.73927439,177.42617714)(200.82260766,177.06159411)
\curveto(200.91114925,176.69701108)(201.05698246,176.36628219)(201.26010729,176.06940744)
\curveto(201.48927376,175.73086606)(201.79916934,175.47565794)(202.18979401,175.30378309)
\curveto(202.58562701,175.13190823)(203.05958494,175.0459708)(203.61166781,175.0459708)
\curveto(204.16895901,175.0459708)(204.64291694,175.12930407)(205.03354162,175.29597059)
\curveto(205.42416629,175.46784545)(205.73666602,175.72565773)(205.97104083,176.06940744)
\curveto(206.17416566,176.36628219)(206.3173947,176.70482358)(206.40072797,177.08503159)
\curveto(206.48926956,177.47044793)(206.53354035,177.94701003)(206.53354035,178.51471789)
\lineto(206.53354035,185.59283693)
\lineto(208.08041405,185.59283693)
\closepath
}
}
{
\newrgbcolor{curcolor}{0 0 0}
\pscustom[linestyle=none,fillstyle=solid,fillcolor=curcolor]
{
\newpath
\moveto(225.64289752,174.80378351)
\curveto(225.35643943,174.67878361)(225.09602298,174.56159621)(224.86164818,174.4522213)
\curveto(224.63248171,174.3428464)(224.33039863,174.22826316)(223.95539894,174.10847159)
\curveto(223.63769088,174.00951334)(223.291337,173.92618008)(222.91633732,173.8584718)
\curveto(222.54654596,173.7855552)(222.13769214,173.74909689)(221.68977585,173.74909689)
\curveto(220.84602656,173.74909689)(220.07779804,173.8662843)(219.38509029,174.1006591)
\curveto(218.69759086,174.34024223)(218.09863303,174.71263775)(217.5882168,175.21784566)
\curveto(217.08821722,175.71263691)(216.69759255,176.34024055)(216.41634278,177.10065658)
\curveto(216.13509302,177.86628093)(215.99446814,178.75430102)(215.99446814,179.76471684)
\curveto(215.99446814,180.72304936)(216.12988469,181.57981948)(216.4007178,182.33502717)
\curveto(216.6715509,183.09023487)(217.06217557,183.72825517)(217.57259181,184.24908806)
\curveto(218.06738306,184.75429597)(218.66373673,185.13971231)(219.36165281,185.40533709)
\curveto(220.06477721,185.67096187)(220.84342239,185.80377426)(221.69758834,185.80377426)
\curveto(222.32258782,185.80377426)(222.94498313,185.72825349)(223.56477427,185.57721195)
\curveto(224.18977375,185.42617041)(224.8824815,185.16054563)(225.64289752,184.78033762)
\lineto(225.64289752,182.94440166)
\lineto(225.52571012,182.94440166)
\curveto(224.88508566,183.48085954)(224.24966953,183.87148421)(223.61946172,184.11627568)
\curveto(222.98925392,184.36106714)(222.31477532,184.48346287)(221.59602593,184.48346287)
\curveto(221.00748475,184.48346287)(220.4762352,184.38710878)(220.00227727,184.19440061)
\curveto(219.53352766,184.00690077)(219.11425718,183.71263018)(218.74446582,183.31158885)
\curveto(218.38509113,182.92096418)(218.10384136,182.42617293)(217.90071653,181.8272151)
\curveto(217.70280003,181.2334656)(217.60384178,180.54596618)(217.60384178,179.76471684)
\curveto(217.60384178,178.94700919)(217.71321669,178.24388478)(217.93196651,177.65534361)
\curveto(218.15592465,177.06680244)(218.44238275,176.58763617)(218.79134079,176.21784482)
\curveto(219.15592381,175.83242848)(219.58040262,175.54597038)(220.06477721,175.35847054)
\curveto(220.55436014,175.17617903)(221.0699847,175.08503327)(221.61165091,175.08503327)
\curveto(222.35644195,175.08503327)(223.05435803,175.21263733)(223.70539915,175.46784545)
\curveto(224.35644027,175.72305357)(224.96581476,176.10586575)(225.53352261,176.61628198)
\lineto(225.64289752,176.61628198)
\closepath
}
}
{
\newrgbcolor{curcolor}{0 0 0}
\pscustom[linestyle=none,fillstyle=solid,fillcolor=curcolor]
{
\newpath
\moveto(234.49445356,173.96003422)
\lineto(233.03351729,173.96003422)
\lineto(233.03351729,174.88972094)
\curveto(232.90330907,174.80117934)(232.72622588,174.67617945)(232.50226774,174.51472125)
\curveto(232.28351792,174.35847138)(232.06997644,174.23347149)(231.86164328,174.13972157)
\curveto(231.61685182,174.01993)(231.33560205,173.92097175)(231.01789399,173.84284682)
\curveto(230.70018592,173.75951355)(230.3277904,173.71784692)(229.90070743,173.71784692)
\curveto(229.11424976,173.71784692)(228.44758365,173.97826337)(227.90070911,174.49909626)
\curveto(227.35383457,175.01992916)(227.0803973,175.6839911)(227.0803973,176.49128209)
\curveto(227.0803973,177.15273987)(227.22102218,177.68659358)(227.50227194,178.09284324)
\curveto(227.78873004,178.50430123)(228.1949797,178.82721762)(228.72102092,179.06159243)
\curveto(229.25227047,179.29596723)(229.89029077,179.45482126)(230.63508181,179.53815453)
\curveto(231.37987285,179.62148779)(232.17935134,179.68398774)(233.03351729,179.72565437)
\lineto(233.03351729,179.95221668)
\curveto(233.03351729,180.28554973)(232.97362151,180.56159117)(232.85382994,180.78034098)
\curveto(232.73924671,180.9990908)(232.57258018,181.17096565)(232.35383036,181.29596555)
\curveto(232.14549721,181.41575711)(231.89549742,181.49648621)(231.603831,181.53815284)
\curveto(231.31216457,181.57981948)(231.00747733,181.60065279)(230.68976926,181.60065279)
\curveto(230.30435292,181.60065279)(229.87466578,181.5485695)(229.40070785,181.44440292)
\curveto(228.92674991,181.34544467)(228.43716699,181.19961146)(227.93195908,181.00690329)
\lineto(227.85383415,181.00690329)
\lineto(227.85383415,182.49908954)
\curveto(228.14029224,182.57721447)(228.55435439,182.6631519)(229.0960206,182.75690182)
\curveto(229.63768682,182.85065174)(230.17154053,182.8975267)(230.69758176,182.8975267)
\curveto(231.31216457,182.8975267)(231.84601829,182.84544341)(232.29914291,182.74127683)
\curveto(232.75747586,182.64231858)(233.15330886,182.47044373)(233.48664191,182.22565227)
\curveto(233.81476664,181.98606913)(234.06476643,181.67617356)(234.23664128,181.29596555)
\curveto(234.40851614,180.91575753)(234.49445356,180.44440376)(234.49445356,179.88190424)
\closepath
\moveto(233.03351729,176.10846991)
\lineto(233.03351729,178.53815537)
\curveto(232.585601,178.51211372)(232.05695561,178.47305125)(231.44758113,178.42096797)
\curveto(230.84341497,178.36888468)(230.3642487,178.29336391)(230.01008234,178.19440566)
\curveto(229.58820769,178.07461409)(229.24706214,177.88711425)(228.9866457,177.63190613)
\curveto(228.72622925,177.38190634)(228.59602102,177.03555246)(228.59602102,176.5928445)
\curveto(228.59602102,176.09284492)(228.74706256,175.71524107)(229.04914564,175.46003296)
\curveto(229.35122872,175.21003317)(229.81216584,175.08503327)(230.43195698,175.08503327)
\curveto(230.94758155,175.08503327)(231.41893532,175.18399152)(231.84601829,175.38190802)
\curveto(232.27310127,175.58503285)(232.66893427,175.82722015)(233.03351729,176.10846991)
\closepath
}
}
{
\newrgbcolor{curcolor}{0 0 0}
\pscustom[linestyle=none,fillstyle=solid,fillcolor=curcolor]
{
\newpath
\moveto(243.77569663,174.50690876)
\curveto(243.28611371,174.27253395)(242.81996827,174.09024244)(242.37726031,173.96003422)
\curveto(241.93976068,173.82982599)(241.47361523,173.76472188)(240.97882398,173.76472188)
\curveto(240.34861618,173.76472188)(239.77049167,173.85586764)(239.24445044,174.03815915)
\curveto(238.71840922,174.22565899)(238.26788876,174.50690876)(237.89288908,174.88190844)
\curveto(237.51268107,175.25690813)(237.21841048,175.73086606)(237.01007732,176.30378225)
\curveto(236.80174416,176.87669843)(236.69757758,177.5459687)(236.69757758,178.31159306)
\curveto(236.69757758,179.73867519)(237.08820226,180.85846592)(237.8694516,181.67096523)
\curveto(238.65590927,182.48346455)(239.69236673,182.88971421)(240.97882398,182.88971421)
\curveto(241.47882356,182.88971421)(241.96840648,182.81940177)(242.44757275,182.67877688)
\curveto(242.93194734,182.538152)(243.3746553,182.36627715)(243.77569663,182.16315232)
\lineto(243.77569663,180.53034119)
\lineto(243.6975717,180.53034119)
\curveto(243.24965541,180.87929923)(242.78611413,181.14752817)(242.30694787,181.33502801)
\curveto(241.83298993,181.52252786)(241.36944865,181.61627778)(240.91632404,181.61627778)
\curveto(240.0829914,181.61627778)(239.42413779,181.33502801)(238.9397632,180.77252849)
\curveto(238.46059693,180.21523729)(238.2210138,179.39492548)(238.2210138,178.31159306)
\curveto(238.2210138,177.25951061)(238.45538861,176.44961546)(238.92413821,175.8819076)
\curveto(239.39809615,175.31940807)(240.06215809,175.03815831)(240.91632404,175.03815831)
\curveto(241.21319879,175.03815831)(241.51528187,175.07722078)(241.82257327,175.15534571)
\curveto(242.12986468,175.23347065)(242.40590612,175.33503306)(242.65069758,175.46003296)
\curveto(242.86423906,175.56940786)(243.06475973,175.6839911)(243.25225957,175.80378267)
\curveto(243.43975941,175.92878256)(243.58819679,176.0355533)(243.6975717,176.1240949)
\lineto(243.77569663,176.1240949)
\closepath
}
}
{
\newrgbcolor{curcolor}{0 0 0}
\pscustom[linestyle=none,fillstyle=solid,fillcolor=curcolor]
{
\newpath
\moveto(252.95537645,173.96003422)
\lineto(251.48662769,173.96003422)
\lineto(251.48662769,178.92878004)
\curveto(251.48662769,179.32982137)(251.46319021,179.70482105)(251.41631525,180.05377909)
\curveto(251.36944029,180.40794546)(251.28350286,180.6839869)(251.15850296,180.8819034)
\curveto(251.02829474,181.10065321)(250.8407949,181.26211141)(250.59600344,181.36627799)
\curveto(250.35121198,181.4756529)(250.03350391,181.53034035)(249.64287924,181.53034035)
\curveto(249.24183791,181.53034035)(248.82256743,181.4313821)(248.3850678,181.2334656)
\curveto(247.94756816,181.0355491)(247.52829768,180.78294515)(247.12725635,180.47565374)
\lineto(247.12725635,173.96003422)
\lineto(245.65850759,173.96003422)
\lineto(245.65850759,186.11627399)
\lineto(247.12725635,186.11627399)
\lineto(247.12725635,181.71784019)
\curveto(247.5855893,182.09804821)(248.05954724,182.39492296)(248.54913016,182.60846444)
\curveto(249.03871308,182.82200593)(249.54131682,182.92877667)(250.05694139,182.92877667)
\curveto(250.99964893,182.92877667)(251.71839833,182.64492275)(252.21318958,182.07721489)
\curveto(252.70798083,181.50950703)(252.95537645,180.69179939)(252.95537645,179.62409195)
\closepath
}
}
{
\newrgbcolor{curcolor}{0 0 0}
\pscustom[linestyle=none,fillstyle=solid,fillcolor=curcolor]
{
\newpath
\moveto(263.08817885,178.17096818)
\lineto(256.65849676,178.17096818)
\curveto(256.65849676,177.63451029)(256.73922586,177.16576069)(256.90068406,176.76471936)
\curveto(257.06214226,176.36888636)(257.28349624,176.0433658)(257.564746,175.78815768)
\curveto(257.8355791,175.53815789)(258.15589134,175.35065805)(258.52568269,175.22565815)
\curveto(258.90068238,175.10065826)(259.31214036,175.03815831)(259.76005665,175.03815831)
\curveto(260.35380615,175.03815831)(260.95015982,175.15534571)(261.54911765,175.38972051)
\curveto(262.15328381,175.62930365)(262.58297095,175.86367845)(262.83817906,176.09284492)
\lineto(262.916304,176.09284492)
\lineto(262.916304,174.49128377)
\curveto(262.42151275,174.28295061)(261.91630484,174.10847159)(261.40068027,173.96784671)
\curveto(260.88505571,173.82722183)(260.3433895,173.75690939)(259.77568164,173.75690939)
\curveto(258.32776619,173.75690939)(257.19755881,174.14753406)(256.38505949,174.9287834)
\curveto(255.57256017,175.71524107)(255.16631052,176.82982347)(255.16631052,178.27253059)
\curveto(255.16631052,179.69961272)(255.55433102,180.83242427)(256.33037204,181.67096523)
\curveto(257.11162138,182.50950619)(258.13766218,182.92877667)(259.40849445,182.92877667)
\curveto(260.58557679,182.92877667)(261.49182603,182.58502696)(262.12724216,181.89752754)
\curveto(262.76786662,181.21002812)(263.08817885,180.23346644)(263.08817885,178.96784251)
\closepath
\moveto(261.65849256,179.29596723)
\curveto(261.65328423,180.06679991)(261.45797189,180.66315358)(261.07255555,181.08502822)
\curveto(260.69234754,181.50690287)(260.11161886,181.71784019)(259.33036951,181.71784019)
\curveto(258.54391184,181.71784019)(257.9163082,181.48606955)(257.4475586,181.02252828)
\curveto(256.98401732,180.558987)(256.72099671,179.98346665)(256.65849676,179.29596723)
\closepath
}
}
{
\newrgbcolor{curcolor}{0 0 0}
\pscustom[linestyle=none,fillstyle=solid,fillcolor=curcolor]
{
\newpath
\moveto(271.54911245,176.4756571)
\curveto(271.54911245,175.67878277)(271.21838356,175.02513749)(270.55692578,174.51472125)
\curveto(269.90067633,174.00430501)(269.00223959,173.74909689)(267.86161555,173.74909689)
\curveto(267.21578276,173.74909689)(266.62203326,173.82461766)(266.08036705,173.9756592)
\curveto(265.54390917,174.13190907)(265.09338871,174.30117976)(264.72880568,174.48347128)
\lineto(264.72880568,176.13190739)
\lineto(264.80693062,176.13190739)
\curveto(265.2704719,175.78294935)(265.78609646,175.50430375)(266.35380432,175.29597059)
\curveto(266.92151217,175.09284576)(267.46578255,174.99128335)(267.98661544,174.99128335)
\curveto(268.63244823,174.99128335)(269.13765614,175.09544993)(269.50223917,175.30378309)
\curveto(269.8668222,175.51211625)(270.04911371,175.84024097)(270.04911371,176.28815726)
\curveto(270.04911371,176.63190697)(269.95015546,176.89232342)(269.75223896,177.0694066)
\curveto(269.55432246,177.24648979)(269.17411445,177.39753133)(268.61161492,177.52253122)
\curveto(268.40328176,177.56940618)(268.12984449,177.62409364)(267.79130311,177.68659358)
\curveto(267.45797006,177.74909353)(267.15328281,177.81680181)(266.87724138,177.88971841)
\curveto(266.11161702,178.09284324)(265.56734665,178.38971799)(265.24443025,178.78034266)
\curveto(264.92672218,179.17617566)(264.76786815,179.66055026)(264.76786815,180.23346644)
\curveto(264.76786815,180.59284114)(264.84078476,180.93138252)(264.98661797,181.24909059)
\curveto(265.13765951,181.56679865)(265.36422182,181.85065258)(265.6663049,182.10065237)
\curveto(265.95797132,182.34544383)(266.32776267,182.538152)(266.77567896,182.67877688)
\curveto(267.22880358,182.8246101)(267.73401149,182.8975267)(268.29130269,182.8975267)
\curveto(268.81213558,182.8975267)(269.33817681,182.83242259)(269.86942636,182.70221436)
\curveto(270.40588424,182.57721447)(270.85119637,182.42356877)(271.20536274,182.24127725)
\lineto(271.20536274,180.67096607)
\lineto(271.1272378,180.67096607)
\curveto(270.75223812,180.94700751)(270.29650933,181.17877815)(269.76005145,181.36627799)
\curveto(269.22359357,181.55898616)(268.69755235,181.65534025)(268.18192778,181.65534025)
\curveto(267.6454699,181.65534025)(267.19234528,181.55117367)(266.82255392,181.34284051)
\curveto(266.45276257,181.13971568)(266.26786689,180.83502844)(266.26786689,180.42877878)
\curveto(266.26786689,180.06940408)(266.37984596,179.79857097)(266.60380411,179.61627946)
\curveto(266.82255392,179.43398795)(267.17672029,179.28555057)(267.66630321,179.17096733)
\curveto(267.93713632,179.10846739)(268.2392194,179.04596744)(268.57255245,178.98346749)
\curveto(268.91109383,178.92096754)(269.1923436,178.86367593)(269.41630174,178.81159264)
\curveto(270.09859283,178.65534277)(270.62463406,178.38711383)(270.99442541,178.00690581)
\curveto(271.36421677,177.62148947)(271.54911245,177.11107323)(271.54911245,176.4756571)
\closepath
}
}
{
\newrgbcolor{curcolor}{0 0 0}
\pscustom[linestyle=none,fillstyle=solid,fillcolor=curcolor]
{
\newpath
\moveto(218.10778351,101.84749212)
\lineto(216.0999727,101.84749212)
\lineto(212.20935097,106.47248823)
\lineto(210.0296653,106.47248823)
\lineto(210.0296653,101.84749212)
\lineto(208.4827916,101.84749212)
\lineto(208.4827916,113.48029483)
\lineto(211.74060136,113.48029483)
\curveto(212.44372577,113.48029483)(213.02966278,113.43341987)(213.49841238,113.33966995)
\curveto(213.96716199,113.25112836)(214.38903663,113.08967016)(214.76403632,112.85529536)
\curveto(215.18591096,112.58967058)(215.51403569,112.25373337)(215.74841049,111.84748371)
\curveto(215.98799362,111.44644238)(216.10778519,110.93602614)(216.10778519,110.31623499)
\curveto(216.10778519,109.47769403)(215.89684787,108.77456962)(215.47497322,108.20686177)
\curveto(215.05309858,107.64436224)(214.4723699,107.21988343)(213.73278719,106.93342534)
\closepath
\moveto(214.49059905,110.20686009)
\curveto(214.49059905,110.54019314)(214.43070327,110.83446373)(214.3109117,111.08967184)
\curveto(214.19632846,111.35008829)(214.00362029,111.56883811)(213.73278719,111.74592129)
\curveto(213.50882904,111.89696283)(213.24320426,112.00112941)(212.93591286,112.05842103)
\curveto(212.62862145,112.12092098)(212.26664259,112.15217095)(211.84997627,112.15217095)
\lineto(210.0296653,112.15217095)
\lineto(210.0296653,107.76154964)
\lineto(211.59216399,107.76154964)
\curveto(212.08174691,107.76154964)(212.50882988,107.80321628)(212.87341291,107.88654954)
\curveto(213.23799594,107.97509113)(213.54789151,108.13654933)(213.80309963,108.37092413)
\curveto(214.03747443,108.58967395)(214.20934929,108.83967374)(214.31872419,109.1209235)
\curveto(214.43330743,109.40738159)(214.49059905,109.76936045)(214.49059905,110.20686009)
\closepath
}
}
{
\newrgbcolor{curcolor}{0 0 0}
\pscustom[linestyle=none,fillstyle=solid,fillcolor=curcolor]
{
\newpath
\moveto(228.77964976,101.84749212)
\lineto(227.13121364,101.84749212)
\lineto(225.9905896,105.08967689)
\lineto(220.95934383,105.08967689)
\lineto(219.81871979,101.84749212)
\lineto(218.24840861,101.84749212)
\lineto(222.48278005,113.48029483)
\lineto(224.54527832,113.48029483)
\closepath
\moveto(225.5140275,106.41780077)
\lineto(223.47496672,112.12873347)
\lineto(221.42809344,106.41780077)
\closepath
}
}
{
\newrgbcolor{curcolor}{0 0 0}
\pscustom[linestyle=none,fillstyle=solid,fillcolor=curcolor]
{
\newpath
\moveto(240.90464035,101.84749212)
\lineto(239.35776665,101.84749212)
\lineto(239.35776665,111.87092119)
\lineto(236.12339437,105.05061442)
\lineto(235.20152014,105.05061442)
\lineto(231.99058534,111.87092119)
\lineto(231.99058534,101.84749212)
\lineto(230.54527406,101.84749212)
\lineto(230.54527406,113.48029483)
\lineto(232.65464729,113.48029483)
\lineto(235.75620718,107.00373778)
\lineto(238.75620465,113.48029483)
\lineto(240.90464035,113.48029483)
\closepath
}
}
{
\newrgbcolor{curcolor}{0 0 0}
\pscustom[linestyle=none,fillstyle=solid,fillcolor=curcolor]
{
\newpath
\moveto(218.09213335,36.13224556)
\curveto(218.09213335,35.07495478)(217.86036271,34.11662225)(217.39682143,33.25724797)
\curveto(216.93848848,32.3978737)(216.32650983,31.73120759)(215.56088547,31.25724966)
\curveto(215.02963592,30.92912493)(214.43588642,30.69214596)(213.77963697,30.54631275)
\curveto(213.12859585,30.40047954)(212.26922158,30.32756294)(211.20151414,30.32756294)
\lineto(208.26401661,30.32756294)
\lineto(208.26401661,41.96036565)
\lineto(211.17026417,41.96036565)
\curveto(212.30567988,41.96036565)(213.20672079,41.87703239)(213.87338689,41.71036586)
\curveto(214.54526133,41.54890767)(215.11296918,41.32494952)(215.57651046,41.03849143)
\curveto(216.36817646,40.54370018)(216.98536344,39.88484657)(217.4280714,39.06193059)
\curveto(217.87077936,38.23901462)(218.09213335,37.26245294)(218.09213335,36.13224556)
\closepath
\moveto(216.47494721,36.15568304)
\curveto(216.47494721,37.0671406)(216.31609317,37.83536912)(215.99838511,38.4603686)
\curveto(215.68067704,39.08536807)(215.20671911,39.57755516)(214.5765113,39.93692986)
\curveto(214.11817835,40.1973463)(213.6311996,40.37703365)(213.11557503,40.4759919)
\curveto(212.59995046,40.58015848)(211.98276348,40.63224177)(211.26401409,40.63224177)
\lineto(209.81089031,40.63224177)
\lineto(209.81089031,31.65568682)
\lineto(211.26401409,31.65568682)
\curveto(212.00880513,31.65568682)(212.65724208,31.71037427)(213.20932495,31.81974918)
\curveto(213.76661615,31.92912409)(214.27703239,32.13224892)(214.74057366,32.42912367)
\curveto(215.31869818,32.79891503)(215.75098948,33.28589378)(216.03744757,33.89005994)
\curveto(216.32911399,34.4942261)(216.47494721,35.2494338)(216.47494721,36.15568304)
\closepath
}
}
{
\newrgbcolor{curcolor}{0 0 0}
\pscustom[linestyle=none,fillstyle=solid,fillcolor=curcolor]
{
\newpath
\moveto(222.05306743,40.51505437)
\lineto(220.39681882,40.51505437)
\lineto(220.39681882,42.03849059)
\lineto(222.05306743,42.03849059)
\closepath
\moveto(221.95931751,30.32756294)
\lineto(220.49056874,30.32756294)
\lineto(220.49056874,39.0541181)
\lineto(221.95931751,39.0541181)
\closepath
}
}
{
\newrgbcolor{curcolor}{0 0 0}
\pscustom[linestyle=none,fillstyle=solid,fillcolor=curcolor]
{
\newpath
\moveto(231.09993575,32.84318582)
\curveto(231.09993575,32.04631149)(230.76920686,31.39266621)(230.10774908,30.88224997)
\curveto(229.45149964,30.37183373)(228.55306289,30.11662561)(227.41243885,30.11662561)
\curveto(226.76660606,30.11662561)(226.17285656,30.19214638)(225.63119035,30.34318792)
\curveto(225.09473247,30.49943779)(224.64421201,30.66870848)(224.27962899,30.851)
\lineto(224.27962899,32.49943611)
\lineto(224.35775392,32.49943611)
\curveto(224.8212952,32.15047807)(225.33691976,31.87183247)(225.90462762,31.66349931)
\curveto(226.47233548,31.46037448)(227.01660585,31.35881207)(227.53743875,31.35881207)
\curveto(228.18327154,31.35881207)(228.68847944,31.46297865)(229.05306247,31.67131181)
\curveto(229.4176455,31.87964497)(229.59993701,32.20776969)(229.59993701,32.65568598)
\curveto(229.59993701,32.99943569)(229.50097876,33.25985214)(229.30306226,33.43693532)
\curveto(229.10514576,33.61401851)(228.72493775,33.76506005)(228.16243822,33.89005994)
\curveto(227.95410506,33.9369349)(227.68066779,33.99162236)(227.34212641,34.0541223)
\curveto(227.00879336,34.11662225)(226.70410611,34.18433053)(226.42806468,34.25724713)
\curveto(225.66244032,34.46037196)(225.11816995,34.75724671)(224.79525355,35.14787138)
\curveto(224.47754549,35.54370438)(224.31869145,36.02807898)(224.31869145,36.60099516)
\curveto(224.31869145,36.96036986)(224.39160806,37.29891124)(224.53744127,37.61661931)
\curveto(224.68848281,37.93432737)(224.91504512,38.2181813)(225.2171282,38.46818109)
\curveto(225.50879462,38.71297255)(225.87858597,38.90568072)(226.32650226,39.0463056)
\curveto(226.77962688,39.19213882)(227.28483479,39.26505542)(227.84212599,39.26505542)
\curveto(228.36295889,39.26505542)(228.88900011,39.19995131)(229.42024966,39.06974308)
\curveto(229.95670754,38.94474319)(230.40201967,38.79109749)(230.75618604,38.60880597)
\lineto(230.75618604,37.03849479)
\lineto(230.6780611,37.03849479)
\curveto(230.30306142,37.31453623)(229.84733264,37.54630687)(229.31087475,37.73380671)
\curveto(228.77441687,37.92651488)(228.24837565,38.02286897)(227.73275108,38.02286897)
\curveto(227.1962932,38.02286897)(226.74316858,37.91870239)(226.37337723,37.71036923)
\curveto(226.00358587,37.5072444)(225.81869019,37.20255716)(225.81869019,36.7963075)
\curveto(225.81869019,36.4369328)(225.93066926,36.16609969)(226.15462741,35.98380818)
\curveto(226.37337723,35.80151667)(226.72754359,35.65307929)(227.21712652,35.53849605)
\curveto(227.48795962,35.47599611)(227.7900427,35.41349616)(228.12337575,35.35099621)
\curveto(228.46191714,35.28849627)(228.7431669,35.23120465)(228.96712504,35.17912136)
\curveto(229.64941614,35.02287149)(230.17545736,34.75464255)(230.54524872,34.37443453)
\curveto(230.91504007,33.98901819)(231.09993575,33.47860195)(231.09993575,32.84318582)
\closepath
}
}
{
\newrgbcolor{curcolor}{0 0 0}
\pscustom[linestyle=none,fillstyle=solid,fillcolor=curcolor]
{
\newpath
\moveto(241.12336486,30.32756294)
\lineto(239.18586649,30.32756294)
\lineto(235.68586944,34.14787222)
\lineto(234.73274524,33.24162299)
\lineto(234.73274524,30.32756294)
\lineto(233.26399647,30.32756294)
\lineto(233.26399647,42.48380271)
\lineto(234.73274524,42.48380271)
\lineto(234.73274524,34.68693427)
\lineto(238.97492917,39.0541181)
\lineto(240.82649011,39.0541181)
\lineto(236.77180602,35.02287149)
\closepath
}
}
\end{pspicture}
}
    \captionsetup{width=0.5\linewidth}
    \caption{Memory Hierarchy}
    \label{fig:memoryhierarchy}
\end{figure}

At the top of this hierarchy are registers directly stored on the processor, and at the bottom are disks,
slow but vast in size.
The organization of this lowest level of the memory hierarchy, and efficient use of the higher levels that have become more
complex over time,
is the subject of this thesis.
Specifically the filesystem, which organizes the data stored on disks, has become more powerful over time, allowing
data to be spread out among multiple disks and ensuring it remains exactly as it was stored, despite the limitations
of every layer of the system.
One of these modern filesystems is ZFS, which is the focus of this work.
ZFS handles its own interactions with the higher levels of the memory hierarchy, specifically RAM,
unlike most other filesystems on Linux, an operating system kernel in common use for the most powerful computers today.

The RAM layer of the memory hierarchy has become more complex over time, as processors have become more and more powerful.
Thus ZFS's methods for improving its own performance with RAM-based caching are no longer optimal on some systems.
This work investigates methods for optimizing them on these new systems and one particularly interesting new method showed
promising, though limited, results.

\chapter{Filesystems}
Filesystems are a critical part of modern computing, providing long term storage of data.
Many different filesystems exist today, including FAT\footnote{A filesystem that is still ubiquitious today, 
despite lacking any of the features that are discussed in this paper.},
NTFS, Ext4, HFS+\footnote{NTFS, Ext4, and HFS+ are filesystems known for having journaling capabilities, 
and are the default filesystems for Windows, Linux and MacOS respectively.
NTFS also allowed for a major increase in the maximum size of files, which was previously limited to 4 GB on FAT.}, 
BTRFS, and ZFS\footnote{BTRFS and ZFS are newer modern filesystems that both support generally the same set of features,
though ZFS is known for being more stable and is the focus of this work}.
Some filesystems are much simpler, working with only one disk and providing only a journaling system to recover from system crashes
or other inconsistencies.
Others, like BTRFS and ZFS, are much more complex, allowing users to store vast amounts of data over multiple disks.

Journaling is a method used by filesystems to recover from inconsistencies found on disk.
They record what operations they intend to perform on the disk in a separate area before actually making any changes.
Thus if they are interrupted when writing out data, the journal can be used to complete that operation when the system boots up again.
Working with multiple disks can be done with any filesystem through the use of a volume manager,
but there are distinct advantages to having native support for multiple disks, as discussed in Section \ref{chapter:volumemanagers}.

\section{Blocks}
\label{chapter:blocks}
Blocks are the fundamental unit of a filesystem.
They are typically fixed-sized units on a disk referred to by their offset from the start of the disk.
There are two types of blocks, which filesystems combine to allow them to store variably-sized, non-contiguous structures,
which are most commonly files.
These two types are direct blocks, those that simply store the user's data, and indirect blocks, which store pointers to other blocks,
either indirect or direct.
The root block of a file is then always a form of indirect block, which can then point to either some direct blocks if the file is very small,
or more likely a set of indirect blocks that themselves point to more indirect blocks and eventually direct blocks,
with the number of intermediate layers depending on the size of the file.

\begin{figure}[H]
    \centering
    \resizebox{0.5\linewidth}{!}{%LaTeX with PSTricks extensions
%%Creator: Inkscape 1.0.2-2 (e86c870879, 2021-01-15)
%%Please note this file requires PSTricks extensions
\psset{xunit=.5pt,yunit=.5pt,runit=.5pt}
\begin{pspicture}(1023.940199,714.07810962)
{
\newrgbcolor{curcolor}{0 0 0}
\pscustom[linewidth=1.88976378,linecolor=curcolor]
{
\newpath
\moveto(520.5989178,419.85750861)
\lineto(750.87615874,323.05955208)
}
}
{
\newrgbcolor{curcolor}{0 0 0}
\pscustom[linestyle=none,fillstyle=solid,fillcolor=curcolor]
{
\newpath
\moveto(733.45507734,330.38257312)
\lineto(723.55743637,326.34334897)
\lineto(750.87615874,323.05955208)
\lineto(729.4158532,340.28021409)
\closepath
}
}
{
\newrgbcolor{curcolor}{0 0 0}
\pscustom[linewidth=2.01574809,linecolor=curcolor]
{
\newpath
\moveto(733.45507734,330.38257312)
\lineto(723.55743637,326.34334897)
\lineto(750.87615874,323.05955208)
\lineto(729.4158532,340.28021409)
\closepath
}
}
{
\newrgbcolor{curcolor}{0 0 0}
\pscustom[linewidth=1.88976378,linecolor=curcolor]
{
\newpath
\moveto(520.5989178,419.85750861)
\lineto(290.32171465,323.05958987)
}
}
{
\newrgbcolor{curcolor}{0 0 0}
\pscustom[linestyle=none,fillstyle=solid,fillcolor=curcolor]
{
\newpath
\moveto(307.74279664,330.3826095)
\lineto(311.78202158,340.28025015)
\lineto(290.32171465,323.05958987)
\lineto(317.64043729,326.34338456)
\closepath
}
}
{
\newrgbcolor{curcolor}{0 0 0}
\pscustom[linewidth=2.01574809,linecolor=curcolor]
{
\newpath
\moveto(307.74279664,330.3826095)
\lineto(311.78202158,340.28025015)
\lineto(290.32171465,323.05958987)
\lineto(317.64043729,326.34338456)
\closepath
}
}
{
\newrgbcolor{curcolor}{0.7019608 0.7019608 0.7019608}
\pscustom[linestyle=none,fillstyle=solid,fillcolor=curcolor]
{
\newpath
\moveto(169.81264607,713.03873951)
\lineto(404.66716614,713.03873951)
\lineto(404.66716614,628.08780793)
\lineto(169.81264607,628.08780793)
\closepath
}
}
{
\newrgbcolor{curcolor}{0 0 0}
\pscustom[linewidth=2.0787402,linecolor=curcolor]
{
\newpath
\moveto(169.81264607,713.03873951)
\lineto(404.66716614,713.03873951)
\lineto(404.66716614,628.08780793)
\lineto(169.81264607,628.08780793)
\closepath
}
}
{
\newrgbcolor{curcolor}{0.7019608 0.7019608 0.7019608}
\pscustom[linestyle=none,fillstyle=solid,fillcolor=curcolor,opacity=0.92623001]
{
\newpath
\moveto(14.05395025,528.70290643)
\lineto(137.29255253,528.70290643)
\lineto(137.29255253,417.58612407)
\lineto(14.05395025,417.58612407)
\closepath
}
}
{
\newrgbcolor{curcolor}{0 0 0}
\pscustom[linewidth=1.88975995,linecolor=curcolor]
{
\newpath
\moveto(14.05395025,528.70290643)
\lineto(137.29255253,528.70290643)
\lineto(137.29255253,417.58612407)
\lineto(14.05395025,417.58612407)
\closepath
}
}
{
\newrgbcolor{curcolor}{0 0 0}
\pscustom[linewidth=1.88976378,linecolor=curcolor]
{
\newpath
\moveto(286.44312567,627.82526798)
\lineto(56.16587717,531.02731523)
}
}
{
\newrgbcolor{curcolor}{0 0 0}
\pscustom[linestyle=none,fillstyle=solid,fillcolor=curcolor]
{
\newpath
\moveto(73.58695875,538.35033582)
\lineto(77.62618315,548.24797669)
\lineto(56.16587717,531.02731523)
\lineto(83.48459962,534.31111142)
\closepath
}
}
{
\newrgbcolor{curcolor}{0 0 0}
\pscustom[linewidth=2.01574809,linecolor=curcolor]
{
\newpath
\moveto(73.58695875,538.35033582)
\lineto(77.62618315,548.24797669)
\lineto(56.16587717,531.02731523)
\lineto(83.48459962,534.31111142)
\closepath
}
}
{
\newrgbcolor{curcolor}{0 0 0}
\pscustom[linewidth=1.88976378,linecolor=curcolor]
{
\newpath
\moveto(286.44312567,627.82526798)
\lineto(516.72036661,531.02731523)
}
}
{
\newrgbcolor{curcolor}{0 0 0}
\pscustom[linestyle=none,fillstyle=solid,fillcolor=curcolor]
{
\newpath
\moveto(499.29928511,538.35033602)
\lineto(489.40164419,534.31111174)
\lineto(516.72036661,531.02731523)
\lineto(495.26006083,548.24797694)
\closepath
}
}
{
\newrgbcolor{curcolor}{0 0 0}
\pscustom[linewidth=2.01574809,linecolor=curcolor]
{
\newpath
\moveto(499.29928511,538.35033602)
\lineto(489.40164419,534.31111174)
\lineto(516.72036661,531.02731523)
\lineto(495.26006083,548.24797694)
\closepath
}
}
{
\newrgbcolor{curcolor}{0.7019608 0.7019608 0.7019608}
\pscustom[linestyle=none,fillstyle=solid,fillcolor=curcolor]
{
\newpath
\moveto(240.77417566,321.45473652)
\lineto(364.01277795,321.45473652)
\lineto(364.01277795,210.33795416)
\lineto(240.77417566,210.33795416)
\closepath
}
}
{
\newrgbcolor{curcolor}{0 0 0}
\pscustom[linewidth=1.88975995,linecolor=curcolor]
{
\newpath
\moveto(240.77417566,321.45473652)
\lineto(364.01277795,321.45473652)
\lineto(364.01277795,210.33795416)
\lineto(240.77417566,210.33795416)
\closepath
}
}
{
\newrgbcolor{curcolor}{0.7019608 0.7019608 0.7019608}
\pscustom[linestyle=none,fillstyle=solid,fillcolor=curcolor]
{
\newpath
\moveto(802.41966425,320.73511761)
\lineto(679.18106197,320.73511761)
\lineto(679.18106197,209.61833525)
\lineto(802.41966425,209.61833525)
\closepath
}
}
{
\newrgbcolor{curcolor}{0 0 0}
\pscustom[linewidth=1.88975995,linecolor=curcolor]
{
\newpath
\moveto(802.41966425,320.73511761)
\lineto(679.18106197,320.73511761)
\lineto(679.18106197,209.61833525)
\lineto(802.41966425,209.61833525)
\closepath
}
}
{
\newrgbcolor{curcolor}{0.7019608 0.7019608 0.7019608}
\pscustom[linestyle=none,fillstyle=solid,fillcolor=curcolor,opacity=0.96721298]
{
\newpath
\moveto(581.29319556,528.70289201)
\lineto(458.05459328,528.70289201)
\lineto(458.05459328,417.58610965)
\lineto(581.29319556,417.58610965)
\closepath
}
}
{
\newrgbcolor{curcolor}{0 0 0}
\pscustom[linewidth=1.88975995,linecolor=curcolor]
{
\newpath
\moveto(581.29319556,528.70289201)
\lineto(458.05459328,528.70289201)
\lineto(458.05459328,417.58610965)
\lineto(581.29319556,417.58610965)
\closepath
}
}
{
\newrgbcolor{curcolor}{0 0 0}
\pscustom[linestyle=none,fillstyle=solid,fillcolor=curcolor]
{
\newpath
\moveto(18.39837847,571.10213114)
\lineto(-0.00000045,571.10213114)
\lineto(-0.00000045,600.18406979)
\lineto(3.86717474,600.18406979)
\lineto(3.86717474,574.53962019)
\lineto(18.39837847,574.53962019)
\closepath
}
}
{
\newrgbcolor{curcolor}{0 0 0}
\pscustom[linestyle=none,fillstyle=solid,fillcolor=curcolor]
{
\newpath
\moveto(40.23424705,581.62944137)
\lineto(24.16007948,581.62944137)
\curveto(24.16007948,580.28829981)(24.36190176,579.11642854)(24.76554631,578.11382756)
\curveto(25.16919085,577.12424738)(25.72257451,576.31044789)(26.42569727,575.67242909)
\curveto(27.10277845,575.04743108)(27.90355715,574.57868257)(28.82803337,574.26618357)
\curveto(29.76553039,573.95368456)(30.79417294,573.79743506)(31.91396105,573.79743506)
\curveto(33.39833132,573.79743506)(34.88921199,574.09040287)(36.38660305,574.67633851)
\curveto(37.89701491,575.27529494)(38.97123024,575.86123057)(39.60924904,576.43414541)
\lineto(39.80456092,576.43414541)
\lineto(39.80456092,572.43025191)
\curveto(38.56758569,571.90942024)(37.30456888,571.47322371)(36.01551049,571.12166233)
\curveto(34.72645209,570.77010095)(33.37228974,570.59432026)(31.95302342,570.59432026)
\curveto(28.33324328,570.59432026)(25.50773144,571.57087965)(23.47648791,573.52399843)
\curveto(21.44524438,575.490138)(20.42962261,578.27658746)(20.42962261,581.88334681)
\curveto(20.42962261,585.45104379)(21.39967161,588.28306602)(23.33976959,590.37941351)
\curveto(25.29288838,592.475761)(27.85798438,593.52393475)(31.03505759,593.52393475)
\curveto(33.97775656,593.52393475)(36.24337434,592.66456249)(37.83191095,590.94581796)
\curveto(39.43346835,589.22707343)(40.23424705,586.78567495)(40.23424705,583.62162253)
\closepath
\moveto(36.66003968,584.44193242)
\curveto(36.64701889,586.36900961)(36.1587392,587.85989028)(35.1952006,588.91457443)
\curveto(34.24468279,589.96925857)(32.7928645,590.49660064)(30.83974572,590.49660064)
\curveto(28.87360614,590.49660064)(27.30460072,589.9171754)(26.13272945,588.75832492)
\curveto(24.97387898,587.59947445)(24.31632899,586.16067694)(24.16007948,584.44193242)
\closepath
}
}
{
\newrgbcolor{curcolor}{0 0 0}
\pscustom[linestyle=none,fillstyle=solid,fillcolor=curcolor]
{
\newpath
\moveto(64.66776378,592.91846793)
\lineto(55.83966689,571.10213114)
\lineto(52.1482724,571.10213114)
\lineto(43.37876907,592.91846793)
\lineto(47.36313138,592.91846793)
\lineto(54.12092236,575.55524196)
\lineto(60.82011978,592.91846793)
\closepath
}
}
{
\newrgbcolor{curcolor}{0 0 0}
\pscustom[linestyle=none,fillstyle=solid,fillcolor=curcolor]
{
\newpath
\moveto(87.38253377,581.62944137)
\lineto(71.3083662,581.62944137)
\curveto(71.3083662,580.28829981)(71.51018847,579.11642854)(71.91383302,578.11382756)
\curveto(72.31747757,577.12424738)(72.87086122,576.31044789)(73.57398398,575.67242909)
\curveto(74.25106516,575.04743108)(75.05184386,574.57868257)(75.97632009,574.26618357)
\curveto(76.9138171,573.95368456)(77.94245966,573.79743506)(79.06224776,573.79743506)
\curveto(80.54661803,573.79743506)(82.0374987,574.09040287)(83.53488977,574.67633851)
\curveto(85.04530163,575.27529494)(86.11951696,575.86123057)(86.75753576,576.43414541)
\lineto(86.95284764,576.43414541)
\lineto(86.95284764,572.43025191)
\curveto(85.71587241,571.90942024)(84.4528556,571.47322371)(83.1637972,571.12166233)
\curveto(81.8747388,570.77010095)(80.52057645,570.59432026)(79.10131014,570.59432026)
\curveto(75.48152999,570.59432026)(72.65601816,571.57087965)(70.62477462,573.52399843)
\curveto(68.59353109,575.490138)(67.57790933,578.27658746)(67.57790933,581.88334681)
\curveto(67.57790933,585.45104379)(68.54795832,588.28306602)(70.48805631,590.37941351)
\curveto(72.44117509,592.475761)(75.00627109,593.52393475)(78.18334431,593.52393475)
\curveto(81.12604327,593.52393475)(83.39166106,592.66456249)(84.98019767,590.94581796)
\curveto(86.58175507,589.22707343)(87.38253377,586.78567495)(87.38253377,583.62162253)
\closepath
\moveto(83.8083264,584.44193242)
\curveto(83.79530561,586.36900961)(83.30702591,587.85989028)(82.34348731,588.91457443)
\curveto(81.39296951,589.96925857)(79.94115121,590.49660064)(77.98803243,590.49660064)
\curveto(76.02189286,590.49660064)(74.45288744,589.9171754)(73.28101617,588.75832492)
\curveto(72.12216569,587.59947445)(71.4646157,586.16067694)(71.3083662,584.44193242)
\closepath
}
}
{
\newrgbcolor{curcolor}{0 0 0}
\pscustom[linestyle=none,fillstyle=solid,fillcolor=curcolor]
{
\newpath
\moveto(96.65984695,571.10213114)
\lineto(92.98798364,571.10213114)
\lineto(92.98798364,601.49265938)
\lineto(96.65984695,601.49265938)
\closepath
}
}
{
\newrgbcolor{curcolor}{0 0 0}
\pscustom[linestyle=none,fillstyle=solid,fillcolor=curcolor]
{
\newpath
\moveto(137.16753571,585.65286606)
\curveto(137.16753571,580.43152852)(136.34722582,576.59690531)(134.70660604,574.14899644)
\curveto(133.07900706,571.71410836)(130.54646304,570.49666432)(127.10897399,570.49666432)
\curveto(123.61940176,570.49666432)(121.06732655,571.73363955)(119.45274836,574.20759)
\curveto(117.85119096,576.68154046)(117.05041226,580.48361169)(117.05041226,585.61380369)
\curveto(117.05041226,590.78305806)(117.86421175,594.59815008)(119.49181074,597.05907974)
\curveto(121.11940972,599.5330302)(123.65846414,600.77000543)(127.10897399,600.77000543)
\curveto(130.59854621,600.77000543)(133.14411102,599.51349901)(134.74566842,597.00048618)
\curveto(136.36024661,594.50049414)(137.16753571,590.7179541)(137.16753571,585.65286606)
\closepath
\moveto(132.03083331,576.78570679)
\curveto(132.48656103,577.84039094)(132.79254964,579.07736616)(132.94879914,580.49663248)
\curveto(133.11806944,581.92891958)(133.20270458,583.64766411)(133.20270458,585.65286606)
\curveto(133.20270458,587.63202643)(133.11806944,589.35077095)(132.94879914,590.80909964)
\curveto(132.79254964,592.26742833)(132.48005063,593.50440356)(132.01130213,594.52002533)
\curveto(131.55557441,595.5226263)(130.9305764,596.27783223)(130.1363081,596.78564311)
\curveto(129.35506058,597.293454)(128.34594921,597.54735944)(127.10897399,597.54735944)
\curveto(125.88501955,597.54735944)(124.86939778,597.293454)(124.06210869,596.78564311)
\curveto(123.26784038,596.27783223)(122.63633198,595.50960551)(122.16758347,594.48096295)
\curveto(121.72487654,593.51742435)(121.41888794,592.26091794)(121.24961764,590.7114437)
\curveto(121.09336814,589.16196947)(121.01524339,587.46275613)(121.01524339,585.61380369)
\curveto(121.01524339,583.58256015)(121.08685774,581.88334681)(121.23008645,580.51616367)
\curveto(121.37331516,579.14898052)(121.67930377,577.92502608)(122.14805228,576.84430036)
\curveto(122.57773841,575.82867859)(123.18320523,575.05394147)(123.96445275,574.52008901)
\curveto(124.75872105,573.98623654)(125.8068948,573.71931031)(127.10897399,573.71931031)
\curveto(128.33292842,573.71931031)(129.34855019,573.97321575)(130.15583928,574.48102663)
\curveto(130.96312838,574.98883751)(131.58812639,575.75706424)(132.03083331,576.78570679)
\closepath
}
}
{
\newrgbcolor{curcolor}{0 0 0}
\pscustom[linestyle=none,fillstyle=solid,fillcolor=curcolor]
{
\newpath
\moveto(56.61081451,473.41794619)
\curveto(56.61081451,470.77472544)(56.03138927,468.37889974)(54.8725388,466.23046908)
\curveto(53.72670911,464.08203842)(52.19676607,462.41537706)(50.28270966,461.230485)
\curveto(48.95458889,460.41017511)(47.47021861,459.81772908)(45.82959884,459.45314691)
\curveto(44.20199985,459.08856473)(42.05356919,458.90627365)(39.38430686,458.90627365)
\lineto(32.04058024,458.90627365)
\lineto(32.04058024,487.9882123)
\lineto(39.30618211,487.9882123)
\curveto(42.14471474,487.9882123)(44.39731173,487.77987963)(46.06397309,487.36321429)
\curveto(47.74365524,486.95956975)(49.16292156,486.39967569)(50.32177204,485.68353214)
\curveto(52.3009324,484.44655691)(53.84389624,482.79942674)(54.95066355,480.74214162)
\curveto(56.05743086,478.68485651)(56.61081451,476.24345803)(56.61081451,473.41794619)
\closepath
\moveto(52.56785863,473.47653976)
\curveto(52.56785863,475.75517834)(52.17072448,477.67574514)(51.37645618,479.23824016)
\curveto(50.58218787,480.80073519)(49.39729581,482.03120002)(47.82177999,482.92963466)
\curveto(46.67595031,483.58067425)(45.45850627,484.02989157)(44.16944787,484.27728662)
\curveto(42.88038948,484.53770246)(41.33742564,484.66791037)(39.54055636,484.66791037)
\lineto(35.90775543,484.66791037)
\lineto(35.90775543,462.22657558)
\lineto(39.54055636,462.22657558)
\curveto(41.4025296,462.22657558)(43.02361819,462.36329389)(44.40382213,462.63673052)
\curveto(45.79704686,462.91016715)(47.07308446,463.41797803)(48.23193494,464.16016317)
\curveto(49.67724284,465.08463939)(50.75796856,466.30208343)(51.47411212,467.81249529)
\curveto(52.20327646,469.32290715)(52.56785863,471.21092197)(52.56785863,473.47653976)
\closepath
}
}
{
\newrgbcolor{curcolor}{0 0 0}
\pscustom[linestyle=none,fillstyle=solid,fillcolor=curcolor]
{
\newpath
\moveto(79.52089863,458.90627365)
\lineto(75.86856651,458.90627365)
\lineto(75.86856651,461.230485)
\curveto(75.54304671,461.00913154)(75.10033978,460.69663253)(74.54044573,460.29298798)
\curveto(73.99357248,459.90236423)(73.45972001,459.58986522)(72.93888833,459.35549097)
\curveto(72.32691112,459.05601275)(71.62378835,458.80861771)(70.82952005,458.61330583)
\curveto(70.03525175,458.40497316)(69.10426513,458.30080683)(68.03656019,458.30080683)
\curveto(66.07042062,458.30080683)(64.40375926,458.95184642)(63.03657611,460.25392561)
\curveto(61.66939296,461.5560048)(60.98580139,463.21615576)(60.98580139,465.2343785)
\curveto(60.98580139,466.88801907)(61.33736277,468.22265024)(62.04048553,469.238272)
\curveto(62.75662909,470.26691456)(63.77225085,471.07420366)(65.08735083,471.66013929)
\curveto(66.4154716,472.24607493)(68.01051861,472.64320908)(69.87249185,472.85154175)
\curveto(71.73446508,473.05987442)(73.73315664,473.21612392)(75.86856651,473.32029026)
\lineto(75.86856651,473.8866947)
\curveto(75.86856651,474.72002538)(75.7188274,475.41012735)(75.41934919,475.95700061)
\curveto(75.13289176,476.50387387)(74.71622642,476.93356)(74.16935317,477.24605901)
\curveto(73.64852149,477.54553722)(73.02352348,477.74735949)(72.29435914,477.85152583)
\curveto(71.56519479,477.95569216)(70.80347847,478.00777533)(70.00921016,478.00777533)
\curveto(69.04567156,478.00777533)(67.97145623,477.87756741)(66.78656417,477.61715157)
\curveto(65.60167211,477.36975653)(64.37771767,477.00517436)(63.11470086,476.52340506)
\lineto(62.91938898,476.52340506)
\lineto(62.91938898,480.25386193)
\curveto(63.63553254,480.44917381)(64.67068549,480.66401687)(66.02484785,480.89839113)
\curveto(67.3790102,481.13276538)(68.71364137,481.24995251)(70.02874135,481.24995251)
\curveto(71.56519479,481.24995251)(72.89982596,481.11974459)(74.03263485,480.85932875)
\curveto(75.17846454,480.61193371)(76.16804472,480.18224757)(77.0013754,479.57027036)
\curveto(77.82168529,478.97131393)(78.4466833,478.19657681)(78.87636943,477.24605901)
\curveto(79.30605556,476.2955412)(79.52089863,475.11715953)(79.52089863,473.71091401)
\closepath
\moveto(75.86856651,464.2773503)
\lineto(75.86856651,470.35154971)
\curveto(74.7487784,470.28644575)(73.42716803,470.18878981)(71.90373538,470.05858189)
\curveto(70.39332352,469.92837397)(69.19541067,469.73957249)(68.30999682,469.49217744)
\curveto(67.25531268,469.19269923)(66.40245081,468.72395072)(65.75141122,468.08593192)
\curveto(65.10037162,467.46093391)(64.77485183,466.59505125)(64.77485183,465.48828394)
\curveto(64.77485183,464.23828792)(65.15245479,463.29428051)(65.90766072,462.65626171)
\curveto(66.66286665,462.0312637)(67.81520673,461.71876469)(69.36468096,461.71876469)
\curveto(70.65373936,461.71876469)(71.83212102,461.96615974)(72.89982596,462.46094983)
\curveto(73.96753089,462.96876071)(74.95711107,463.57422754)(75.86856651,464.2773503)
\closepath
}
}
{
\newrgbcolor{curcolor}{0 0 0}
\pscustom[linestyle=none,fillstyle=solid,fillcolor=curcolor]
{
\newpath
\moveto(97.91927834,459.10158553)
\curveto(97.22917637,458.91929444)(96.47397044,458.76955533)(95.65366056,458.65236821)
\curveto(94.84637146,458.53518108)(94.12371751,458.47658752)(93.48569871,458.47658752)
\curveto(91.2591433,458.47658752)(89.56644035,459.07554394)(88.40758988,460.2734568)
\curveto(87.2487394,461.47136965)(86.66931416,463.39193645)(86.66931416,466.0351572)
\lineto(86.66931416,477.63668276)
\lineto(84.18885331,477.63668276)
\lineto(84.18885331,480.72261044)
\lineto(86.66931416,480.72261044)
\lineto(86.66931416,486.99212172)
\lineto(90.34117747,486.99212172)
\lineto(90.34117747,480.72261044)
\lineto(97.91927834,480.72261044)
\lineto(97.91927834,477.63668276)
\lineto(90.34117747,477.63668276)
\lineto(90.34117747,467.69530816)
\curveto(90.34117747,466.54947848)(90.36721906,465.65104384)(90.41930222,465.00000425)
\curveto(90.47138539,464.36198544)(90.65367648,463.76302902)(90.96617548,463.20313497)
\curveto(91.2526329,462.68230329)(91.64325666,462.29818993)(92.13804675,462.05079489)
\curveto(92.64585763,461.81642063)(93.41408435,461.69923351)(94.44272691,461.69923351)
\curveto(95.04168334,461.69923351)(95.66668135,461.78386865)(96.31772094,461.95313895)
\curveto(96.96876054,462.13543003)(97.43750904,462.28516914)(97.72396646,462.40235627)
\lineto(97.91927834,462.40235627)
\closepath
}
}
{
\newrgbcolor{curcolor}{0 0 0}
\pscustom[linestyle=none,fillstyle=solid,fillcolor=curcolor]
{
\newpath
\moveto(119.30592872,458.90627365)
\lineto(115.6535966,458.90627365)
\lineto(115.6535966,461.230485)
\curveto(115.3280768,461.00913154)(114.88536988,460.69663253)(114.32547583,460.29298798)
\curveto(113.77860257,459.90236423)(113.2447501,459.58986522)(112.72391843,459.35549097)
\curveto(112.11194121,459.05601275)(111.40881845,458.80861771)(110.61455014,458.61330583)
\curveto(109.82028184,458.40497316)(108.88929522,458.30080683)(107.82159029,458.30080683)
\curveto(105.85545071,458.30080683)(104.18878935,458.95184642)(102.82160621,460.25392561)
\curveto(101.45442306,461.5560048)(100.77083149,463.21615576)(100.77083149,465.2343785)
\curveto(100.77083149,466.88801907)(101.12239287,468.22265024)(101.82551563,469.238272)
\curveto(102.54165918,470.26691456)(103.55728095,471.07420366)(104.87238093,471.66013929)
\curveto(106.2005017,472.24607493)(107.7955487,472.64320908)(109.65752194,472.85154175)
\curveto(111.51949518,473.05987442)(113.51818673,473.21612392)(115.6535966,473.32029026)
\lineto(115.6535966,473.8866947)
\curveto(115.6535966,474.72002538)(115.50385749,475.41012735)(115.20437928,475.95700061)
\curveto(114.91792186,476.50387387)(114.50125652,476.93356)(113.95438326,477.24605901)
\curveto(113.43355159,477.54553722)(112.80855358,477.74735949)(112.07938923,477.85152583)
\curveto(111.35022489,477.95569216)(110.58850856,478.00777533)(109.79424026,478.00777533)
\curveto(108.83070166,478.00777533)(107.75648633,477.87756741)(106.57159427,477.61715157)
\curveto(105.38670221,477.36975653)(104.16274777,477.00517436)(102.89973096,476.52340506)
\lineto(102.70441908,476.52340506)
\lineto(102.70441908,480.25386193)
\curveto(103.42056263,480.44917381)(104.45571559,480.66401687)(105.80987794,480.89839113)
\curveto(107.1640403,481.13276538)(108.49867146,481.24995251)(109.81377144,481.24995251)
\curveto(111.35022489,481.24995251)(112.68485605,481.11974459)(113.81766495,480.85932875)
\curveto(114.96349463,480.61193371)(115.95307481,480.18224757)(116.78640549,479.57027036)
\curveto(117.60671538,478.97131393)(118.23171339,478.19657681)(118.66139952,477.24605901)
\curveto(119.09108566,476.2955412)(119.30592872,475.11715953)(119.30592872,473.71091401)
\closepath
\moveto(115.6535966,464.2773503)
\lineto(115.6535966,470.35154971)
\curveto(114.5338085,470.28644575)(113.21219812,470.18878981)(111.68876547,470.05858189)
\curveto(110.17835362,469.92837397)(108.98044076,469.73957249)(108.09502692,469.49217744)
\curveto(107.04034277,469.19269923)(106.18748091,468.72395072)(105.53644131,468.08593192)
\curveto(104.88540172,467.46093391)(104.55988192,466.59505125)(104.55988192,465.48828394)
\curveto(104.55988192,464.23828792)(104.93748489,463.29428051)(105.69269082,462.65626171)
\curveto(106.44789674,462.0312637)(107.60023682,461.71876469)(109.14971106,461.71876469)
\curveto(110.43876945,461.71876469)(111.61715112,461.96615974)(112.68485605,462.46094983)
\curveto(113.75256099,462.96876071)(114.74214117,463.57422754)(115.6535966,464.2773503)
\closepath
}
}
{
\newrgbcolor{curcolor}{0 0 0}
\pscustom[linestyle=none,fillstyle=solid,fillcolor=curcolor]
{
\newpath
\moveto(283.33102551,266.16977629)
\curveto(283.33102551,263.52655554)(282.75160027,261.13072983)(281.59274979,258.98229917)
\curveto(280.44692011,256.83386851)(278.91697706,255.16720715)(277.00292066,253.98231509)
\curveto(275.67479989,253.1620052)(274.19042961,252.56955917)(272.54980984,252.204977)
\curveto(270.92221085,251.84039483)(268.77378019,251.65810374)(266.10451786,251.65810374)
\lineto(258.76079124,251.65810374)
\lineto(258.76079124,280.7400424)
\lineto(266.02639311,280.7400424)
\curveto(268.86492574,280.7400424)(271.11752273,280.53170973)(272.78418409,280.11504439)
\curveto(274.46386624,279.71139984)(275.88313256,279.15150579)(277.04198303,278.43536223)
\curveto(279.0211434,277.19838701)(280.56410724,275.55125683)(281.67087455,273.49397172)
\curveto(282.77764186,271.4366866)(283.33102551,268.99528812)(283.33102551,266.16977629)
\closepath
\moveto(279.28806963,266.22836985)
\curveto(279.28806963,268.50700843)(278.89093548,270.42757523)(278.09666718,271.99007026)
\curveto(277.30239887,273.55256528)(276.11750681,274.78303011)(274.54199099,275.68146475)
\curveto(273.39616131,276.33250435)(272.17871727,276.78172167)(270.88965887,277.02911671)
\curveto(269.60060048,277.28953255)(268.05763664,277.41974047)(266.26076736,277.41974047)
\lineto(262.62796643,277.41974047)
\lineto(262.62796643,254.97840567)
\lineto(266.26076736,254.97840567)
\curveto(268.1227406,254.97840567)(269.74382919,255.11512398)(271.12403313,255.38856061)
\curveto(272.51725786,255.66199724)(273.79329546,256.16980813)(274.95214594,256.91199326)
\curveto(276.39745384,257.83646949)(277.47817956,259.05391353)(278.19432312,260.56432538)
\curveto(278.92348746,262.07473724)(279.28806963,263.96275206)(279.28806963,266.22836985)
\closepath
}
}
{
\newrgbcolor{curcolor}{0 0 0}
\pscustom[linestyle=none,fillstyle=solid,fillcolor=curcolor]
{
\newpath
\moveto(306.24110963,251.65810374)
\lineto(302.5887775,251.65810374)
\lineto(302.5887775,253.98231509)
\curveto(302.26325771,253.76096163)(301.82055078,253.44846262)(301.26065673,253.04481808)
\curveto(300.71378347,252.65419432)(300.17993101,252.34169532)(299.65909933,252.10732106)
\curveto(299.04712211,251.80784285)(298.34399935,251.5604478)(297.54973105,251.36513592)
\curveto(296.75546274,251.15680325)(295.82447612,251.05263692)(294.75677119,251.05263692)
\curveto(292.79063162,251.05263692)(291.12397026,251.70367651)(289.75678711,253.0057557)
\curveto(288.38960396,254.30783489)(287.70601239,255.96798585)(287.70601239,257.98620859)
\curveto(287.70601239,259.63984916)(288.05757377,260.97448033)(288.76069653,261.9901021)
\curveto(289.47684009,263.01874465)(290.49246185,263.82603375)(291.80756183,264.41196938)
\curveto(293.1356826,264.99790502)(294.73072961,265.39503917)(296.59270285,265.60337184)
\curveto(298.45467608,265.81170451)(300.45336764,265.96795401)(302.5887775,266.07212035)
\lineto(302.5887775,266.6385248)
\curveto(302.5887775,267.47185548)(302.4390384,268.16195744)(302.13956018,268.7088307)
\curveto(301.85310276,269.25570396)(301.43643742,269.68539009)(300.88956416,269.9978891)
\curveto(300.36873249,270.29736731)(299.74373448,270.49918959)(299.01457013,270.60335592)
\curveto(298.28540579,270.70752226)(297.52368946,270.75960542)(296.72942116,270.75960542)
\curveto(295.76588256,270.75960542)(294.69166723,270.62939751)(293.50677517,270.36898167)
\curveto(292.32188311,270.12158662)(291.09792867,269.75700445)(289.83491186,269.27523515)
\lineto(289.63959998,269.27523515)
\lineto(289.63959998,273.00569202)
\curveto(290.35574354,273.2010039)(291.39089649,273.41584697)(292.74505885,273.65022122)
\curveto(294.0992212,273.88459547)(295.43385237,274.0017826)(296.74895235,274.0017826)
\curveto(298.28540579,274.0017826)(299.62003696,273.87157468)(300.75284585,273.61115884)
\curveto(301.89867554,273.3637638)(302.88825572,272.93407767)(303.7215864,272.32210045)
\curveto(304.54189629,271.72314402)(305.1668943,270.94840691)(305.59658043,269.9978891)
\curveto(306.02626656,269.04737129)(306.24110963,267.86898963)(306.24110963,266.4627441)
\closepath
\moveto(302.5887775,257.02918039)
\lineto(302.5887775,263.1033798)
\curveto(301.4689894,263.03827584)(300.14737903,262.9406199)(298.62394638,262.81041198)
\curveto(297.11353452,262.68020406)(295.91562167,262.49140258)(295.03020782,262.24400754)
\curveto(293.97552368,261.94452932)(293.12266181,261.47578082)(292.47162222,260.83776201)
\curveto(291.82058262,260.212764)(291.49506283,259.34688134)(291.49506283,258.24011404)
\curveto(291.49506283,256.99011801)(291.87266579,256.0461106)(292.62787172,255.4080918)
\curveto(293.38307765,254.78309379)(294.53541773,254.47059479)(296.08489196,254.47059479)
\curveto(297.37395036,254.47059479)(298.55233202,254.71798983)(299.62003696,255.21277992)
\curveto(300.68774189,255.72059081)(301.67732207,256.32605763)(302.5887775,257.02918039)
\closepath
}
}
{
\newrgbcolor{curcolor}{0 0 0}
\pscustom[linestyle=none,fillstyle=solid,fillcolor=curcolor]
{
\newpath
\moveto(324.63948934,251.85341562)
\curveto(323.94938737,251.67112453)(323.19418144,251.52138543)(322.37387156,251.4041983)
\curveto(321.56658246,251.28701117)(320.84392851,251.22841761)(320.20590971,251.22841761)
\curveto(317.9793543,251.22841761)(316.28665135,251.82737404)(315.12780088,253.02528689)
\curveto(313.9689504,254.22319974)(313.38952516,256.14376654)(313.38952516,258.78698729)
\lineto(313.38952516,270.38851286)
\lineto(310.90906431,270.38851286)
\lineto(310.90906431,273.47444053)
\lineto(313.38952516,273.47444053)
\lineto(313.38952516,279.74395182)
\lineto(317.06138847,279.74395182)
\lineto(317.06138847,273.47444053)
\lineto(324.63948934,273.47444053)
\lineto(324.63948934,270.38851286)
\lineto(317.06138847,270.38851286)
\lineto(317.06138847,260.44713826)
\curveto(317.06138847,259.30130857)(317.08743005,258.40287393)(317.13951322,257.75183434)
\curveto(317.19159639,257.11381554)(317.37388748,256.51485911)(317.68638648,255.95496506)
\curveto(317.9728439,255.43413339)(318.36346766,255.05002003)(318.85825775,254.80262498)
\curveto(319.36606863,254.56825073)(320.13429535,254.4510636)(321.16293791,254.4510636)
\curveto(321.76189434,254.4510636)(322.38689235,254.53569875)(323.03793194,254.70496904)
\curveto(323.68897154,254.88726013)(324.15772004,255.03699923)(324.44417746,255.15418636)
\lineto(324.63948934,255.15418636)
\closepath
}
}
{
\newrgbcolor{curcolor}{0 0 0}
\pscustom[linestyle=none,fillstyle=solid,fillcolor=curcolor]
{
\newpath
\moveto(346.02613972,251.65810374)
\lineto(342.3738076,251.65810374)
\lineto(342.3738076,253.98231509)
\curveto(342.0482878,253.76096163)(341.60558088,253.44846262)(341.04568683,253.04481808)
\curveto(340.49881357,252.65419432)(339.9649611,252.34169532)(339.44412943,252.10732106)
\curveto(338.83215221,251.80784285)(338.12902945,251.5604478)(337.33476114,251.36513592)
\curveto(336.54049284,251.15680325)(335.60950622,251.05263692)(334.54180129,251.05263692)
\curveto(332.57566171,251.05263692)(330.90900035,251.70367651)(329.54181721,253.0057557)
\curveto(328.17463406,254.30783489)(327.49104248,255.96798585)(327.49104248,257.98620859)
\curveto(327.49104248,259.63984916)(327.84260387,260.97448033)(328.54572663,261.9901021)
\curveto(329.26187018,263.01874465)(330.27749195,263.82603375)(331.59259193,264.41196938)
\curveto(332.9207127,264.99790502)(334.5157597,265.39503917)(336.37773294,265.60337184)
\curveto(338.23970618,265.81170451)(340.23839773,265.96795401)(342.3738076,266.07212035)
\lineto(342.3738076,266.6385248)
\curveto(342.3738076,267.47185548)(342.22406849,268.16195744)(341.92459028,268.7088307)
\curveto(341.63813286,269.25570396)(341.22146752,269.68539009)(340.67459426,269.9978891)
\curveto(340.15376258,270.29736731)(339.52876457,270.49918959)(338.79960023,270.60335592)
\curveto(338.07043588,270.70752226)(337.30871956,270.75960542)(336.51445125,270.75960542)
\curveto(335.55091266,270.75960542)(334.47669733,270.62939751)(333.29180527,270.36898167)
\curveto(332.1069132,270.12158662)(330.88295877,269.75700445)(329.61994196,269.27523515)
\lineto(329.42463008,269.27523515)
\lineto(329.42463008,273.00569202)
\curveto(330.14077363,273.2010039)(331.17592659,273.41584697)(332.53008894,273.65022122)
\curveto(333.8842513,273.88459547)(335.21888246,274.0017826)(336.53398244,274.0017826)
\curveto(338.07043588,274.0017826)(339.40506705,273.87157468)(340.53787594,273.61115884)
\curveto(341.68370563,273.3637638)(342.67328581,272.93407767)(343.50661649,272.32210045)
\curveto(344.32692638,271.72314402)(344.95192439,270.94840691)(345.38161052,269.9978891)
\curveto(345.81129665,269.04737129)(346.02613972,267.86898963)(346.02613972,266.4627441)
\closepath
\moveto(342.3738076,257.02918039)
\lineto(342.3738076,263.1033798)
\curveto(341.2540195,263.03827584)(339.93240912,262.9406199)(338.40897647,262.81041198)
\curveto(336.89856462,262.68020406)(335.70065176,262.49140258)(334.81523792,262.24400754)
\curveto(333.76055377,261.94452932)(332.90769191,261.47578082)(332.25665231,260.83776201)
\curveto(331.60561272,260.212764)(331.28009292,259.34688134)(331.28009292,258.24011404)
\curveto(331.28009292,256.99011801)(331.65769589,256.0461106)(332.41290181,255.4080918)
\curveto(333.16810774,254.78309379)(334.32044782,254.47059479)(335.86992206,254.47059479)
\curveto(337.15898045,254.47059479)(338.33736212,254.71798983)(339.40506705,255.21277992)
\curveto(340.47277199,255.72059081)(341.46235217,256.32605763)(342.3738076,257.02918039)
\closepath
}
}
{
\newrgbcolor{curcolor}{0 0 0}
\pscustom[linestyle=none,fillstyle=solid,fillcolor=curcolor]
{
\newpath
\moveto(264.62919085,371.38027363)
\lineto(246.23081192,371.38027363)
\lineto(246.23081192,400.46221229)
\lineto(250.09798711,400.46221229)
\lineto(250.09798711,374.81776269)
\lineto(264.62919085,374.81776269)
\closepath
}
}
{
\newrgbcolor{curcolor}{0 0 0}
\pscustom[linestyle=none,fillstyle=solid,fillcolor=curcolor]
{
\newpath
\moveto(286.46505943,381.90758386)
\lineto(270.39089186,381.90758386)
\curveto(270.39089186,380.5664423)(270.59271413,379.39457103)(270.99635868,378.39197006)
\curveto(271.40000323,377.40238988)(271.95338688,376.58859038)(272.65650964,375.95057158)
\curveto(273.33359082,375.32557357)(274.13436952,374.85682506)(275.05884574,374.54432606)
\curveto(275.99634276,374.23182705)(277.02498532,374.07557755)(278.14477342,374.07557755)
\curveto(279.62914369,374.07557755)(281.12002436,374.36854537)(282.61741543,374.954481)
\curveto(284.12782729,375.55343743)(285.20204262,376.13937306)(285.84006142,376.71228791)
\lineto(286.0353733,376.71228791)
\lineto(286.0353733,372.7083944)
\curveto(284.79839807,372.18756273)(283.53538126,371.7513662)(282.24632286,371.39980482)
\curveto(280.95726446,371.04824344)(279.60310211,370.87246275)(278.18383579,370.87246275)
\curveto(274.56405565,370.87246275)(271.73854382,371.84902214)(269.70730028,373.80214092)
\curveto(267.67605675,375.7682805)(266.66043498,378.55472996)(266.66043498,382.16148931)
\curveto(266.66043498,385.72918628)(267.63048398,388.56120851)(269.57058197,390.65755601)
\curveto(271.52370075,392.7539035)(274.08879675,393.80207724)(277.26586997,393.80207724)
\curveto(280.20856893,393.80207724)(282.47418672,392.94270498)(284.06272333,391.22396045)
\curveto(285.66428073,389.50521592)(286.46505943,387.06381745)(286.46505943,383.89976502)
\closepath
\moveto(282.89085206,384.72007491)
\curveto(282.87783127,386.64715211)(282.38955157,388.13803278)(281.42601297,389.19271692)
\curveto(280.47549516,390.24740106)(279.02367687,390.77474313)(277.07055809,390.77474313)
\curveto(275.10441852,390.77474313)(273.53541309,390.19531789)(272.36354183,389.03646742)
\curveto(271.20469135,387.87761694)(270.54714136,386.43881944)(270.39089186,384.72007491)
\closepath
}
}
{
\newrgbcolor{curcolor}{0 0 0}
\pscustom[linestyle=none,fillstyle=solid,fillcolor=curcolor]
{
\newpath
\moveto(310.89857616,393.19661042)
\lineto(302.07047927,371.38027363)
\lineto(298.37908477,371.38027363)
\lineto(289.60958144,393.19661042)
\lineto(293.59394375,393.19661042)
\lineto(300.35173474,375.83338445)
\lineto(307.05093216,393.19661042)
\closepath
}
}
{
\newrgbcolor{curcolor}{0 0 0}
\pscustom[linestyle=none,fillstyle=solid,fillcolor=curcolor]
{
\newpath
\moveto(333.61334614,381.90758386)
\lineto(317.53917857,381.90758386)
\curveto(317.53917857,380.5664423)(317.74100084,379.39457103)(318.14464539,378.39197006)
\curveto(318.54828994,377.40238988)(319.1016736,376.58859038)(319.80479636,375.95057158)
\curveto(320.48187753,375.32557357)(321.28265623,374.85682506)(322.20713246,374.54432606)
\curveto(323.14462947,374.23182705)(324.17327203,374.07557755)(325.29306013,374.07557755)
\curveto(326.77743041,374.07557755)(328.26831108,374.36854537)(329.76570214,374.954481)
\curveto(331.276114,375.55343743)(332.35032933,376.13937306)(332.98834813,376.71228791)
\lineto(333.18366001,376.71228791)
\lineto(333.18366001,372.7083944)
\curveto(331.94668478,372.18756273)(330.68366797,371.7513662)(329.39460957,371.39980482)
\curveto(328.10555118,371.04824344)(326.75138882,370.87246275)(325.33212251,370.87246275)
\curveto(321.71234237,370.87246275)(318.88683053,371.84902214)(316.855587,373.80214092)
\curveto(314.82434346,375.7682805)(313.8087217,378.55472996)(313.8087217,382.16148931)
\curveto(313.8087217,385.72918628)(314.77877069,388.56120851)(316.71886868,390.65755601)
\curveto(318.67198746,392.7539035)(321.23708346,393.80207724)(324.41415668,393.80207724)
\curveto(327.35685565,393.80207724)(329.62247343,392.94270498)(331.21101004,391.22396045)
\curveto(332.81256744,389.50521592)(333.61334614,387.06381745)(333.61334614,383.89976502)
\closepath
\moveto(330.03913877,384.72007491)
\curveto(330.02611798,386.64715211)(329.53783828,388.13803278)(328.57429969,389.19271692)
\curveto(327.62378188,390.24740106)(326.17196358,390.77474313)(324.2188448,390.77474313)
\curveto(322.25270523,390.77474313)(320.68369981,390.19531789)(319.51182854,389.03646742)
\curveto(318.35297806,387.87761694)(317.69542807,386.43881944)(317.53917857,384.72007491)
\closepath
}
}
{
\newrgbcolor{curcolor}{0 0 0}
\pscustom[linestyle=none,fillstyle=solid,fillcolor=curcolor]
{
\newpath
\moveto(342.89065933,371.38027363)
\lineto(339.21879602,371.38027363)
\lineto(339.21879602,401.77080187)
\lineto(342.89065933,401.77080187)
\closepath
}
}
{
\newrgbcolor{curcolor}{0 0 0}
\pscustom[linestyle=none,fillstyle=solid,fillcolor=curcolor]
{
\newpath
\moveto(381.77725949,371.38027363)
\lineto(366.03512212,371.38027363)
\lineto(366.03512212,374.34901418)
\lineto(372.08979034,374.34901418)
\lineto(372.08979034,393.84113962)
\lineto(366.03512212,393.84113962)
\lineto(366.03512212,396.49738116)
\curveto(366.855432,396.49738116)(367.73433546,396.56248512)(368.67183247,396.69269304)
\curveto(369.60932949,396.83592175)(370.31896264,397.03774402)(370.80073194,397.29815986)
\curveto(371.39968837,397.62367966)(371.86843688,398.0338346)(372.20697746,398.52862469)
\curveto(372.55853885,399.03643558)(372.76036112,399.71351676)(372.81244429,400.55986823)
\lineto(375.8397784,400.55986823)
\lineto(375.8397784,374.34901418)
\lineto(381.77725949,374.34901418)
\closepath
}
}
{
\newrgbcolor{curcolor}{0 0 0}
\pscustom[linewidth=1.88976378,linecolor=curcolor]
{
\newpath
\moveto(741.17471622,210.46440247)
\lineto(971.45195717,113.66646861)
}
}
{
\newrgbcolor{curcolor}{0 0 0}
\pscustom[linestyle=none,fillstyle=solid,fillcolor=curcolor]
{
\newpath
\moveto(954.03087515,120.98948819)
\lineto(944.13323452,116.95026322)
\lineto(971.45195717,113.66646861)
\lineto(949.99165018,130.88712883)
\closepath
}
}
{
\newrgbcolor{curcolor}{0 0 0}
\pscustom[linewidth=2.01574809,linecolor=curcolor]
{
\newpath
\moveto(954.03087515,120.98948819)
\lineto(944.13323452,116.95026322)
\lineto(971.45195717,113.66646861)
\lineto(949.99165018,130.88712883)
\closepath
}
}
{
\newrgbcolor{curcolor}{0.7019608 0.7019608 0.7019608}
\pscustom[linestyle=none,fillstyle=solid,fillcolor=curcolor]
{
\newpath
\moveto(1022.99537476,112.06163744)
\lineto(899.75677247,112.06163744)
\lineto(899.75677247,0.94485508)
\lineto(1022.99537476,0.94485508)
\closepath
}
}
{
\newrgbcolor{curcolor}{0 0 0}
\pscustom[linewidth=1.88975995,linecolor=curcolor]
{
\newpath
\moveto(1022.99537476,112.06163744)
\lineto(899.75677247,112.06163744)
\lineto(899.75677247,0.94485508)
\lineto(1022.99537476,0.94485508)
\closepath
}
}
{
\newrgbcolor{curcolor}{0 0 0}
\pscustom[linestyle=none,fillstyle=solid,fillcolor=curcolor]
{
\newpath
\moveto(942.31365116,56.7766772)
\curveto(942.31365116,54.13345645)(941.73422592,51.73763075)(940.57537544,49.58920009)
\curveto(939.42954576,47.44076943)(937.89960271,45.77410807)(935.9855463,44.58921601)
\curveto(934.65742553,43.76890612)(933.17305526,43.17646009)(931.53243548,42.81187791)
\curveto(929.9048365,42.44729574)(927.75640584,42.26500466)(925.0871435,42.26500466)
\lineto(917.74341689,42.26500466)
\lineto(917.74341689,71.34694331)
\lineto(925.00901875,71.34694331)
\curveto(927.84755138,71.34694331)(930.10014838,71.13861064)(931.76680974,70.7219453)
\curveto(933.44649189,70.31830075)(934.8657582,69.7584067)(936.02460868,69.04226315)
\curveto(938.00376905,67.80528792)(939.54673288,66.15815775)(940.65350019,64.10087263)
\curveto(941.7602675,62.04358751)(942.31365116,59.60218904)(942.31365116,56.7766772)
\closepath
\moveto(938.27069528,56.83527076)
\curveto(938.27069528,59.11390934)(937.87356113,61.03447614)(937.07929282,62.59697117)
\curveto(936.28502452,64.15946619)(935.10013246,65.38993103)(933.52461664,66.28836567)
\curveto(932.37878695,66.93940526)(931.16134291,67.38862258)(929.87228452,67.63601763)
\curveto(928.58322612,67.89643346)(927.04026229,68.02664138)(925.24339301,68.02664138)
\lineto(921.61059207,68.02664138)
\lineto(921.61059207,45.58530658)
\lineto(925.24339301,45.58530658)
\curveto(927.10536625,45.58530658)(928.72645483,45.7220249)(930.10665877,45.99546153)
\curveto(931.4998835,46.26889816)(932.77592111,46.77670904)(933.93477158,47.51889418)
\curveto(935.38007948,48.4433704)(936.46080521,49.66081444)(937.17694876,51.1712263)
\curveto(937.90611311,52.68163816)(938.27069528,54.56965298)(938.27069528,56.83527076)
\closepath
}
}
{
\newrgbcolor{curcolor}{0 0 0}
\pscustom[linestyle=none,fillstyle=solid,fillcolor=curcolor]
{
\newpath
\moveto(965.22373527,42.26500466)
\lineto(961.57140315,42.26500466)
\lineto(961.57140315,44.58921601)
\curveto(961.24588335,44.36786254)(960.80317643,44.05536354)(960.24328238,43.65171899)
\curveto(959.69640912,43.26109523)(959.16255665,42.94859623)(958.64172498,42.71422197)
\curveto(958.02974776,42.41474376)(957.326625,42.16734872)(956.53235669,41.97203684)
\curveto(955.73808839,41.76370417)(954.80710177,41.65953783)(953.73939684,41.65953783)
\curveto(951.77325726,41.65953783)(950.1065959,42.31057743)(948.73941276,43.61265661)
\curveto(947.37222961,44.9147358)(946.68863804,46.57488677)(946.68863804,48.59310951)
\curveto(946.68863804,50.24675008)(947.04019942,51.58138124)(947.74332218,52.59700301)
\curveto(948.45946573,53.62564557)(949.4750875,54.43293466)(950.79018748,55.0188703)
\curveto(952.11830825,55.60480593)(953.71335525,56.00194008)(955.57532849,56.21027275)
\curveto(957.43730173,56.41860542)(959.43599328,56.57485493)(961.57140315,56.67902126)
\lineto(961.57140315,57.24542571)
\curveto(961.57140315,58.07875639)(961.42166404,58.76885836)(961.12218583,59.31573162)
\curveto(960.83572841,59.86260488)(960.41906307,60.29229101)(959.87218981,60.60479001)
\curveto(959.35135814,60.90426823)(958.72636013,61.1060905)(957.99719578,61.21025683)
\curveto(957.26803144,61.31442317)(956.50631511,61.36650634)(955.71204681,61.36650634)
\curveto(954.74850821,61.36650634)(953.67429288,61.23629842)(952.48940082,60.97588258)
\curveto(951.30450876,60.72848754)(950.08055432,60.36390536)(948.81753751,59.88213606)
\lineto(948.62222563,59.88213606)
\lineto(948.62222563,63.61259294)
\curveto(949.33836918,63.80790481)(950.37352214,64.02274788)(951.72768449,64.25712213)
\curveto(953.08184685,64.49149639)(954.41647801,64.60868351)(955.73157799,64.60868351)
\curveto(957.26803144,64.60868351)(958.6026626,64.4784756)(959.7354715,64.21805976)
\curveto(960.88130118,63.97066471)(961.87088136,63.54097858)(962.70421204,62.92900136)
\curveto(963.52452193,62.33004494)(964.14951994,61.55530782)(964.57920607,60.60479001)
\curveto(965.00889221,59.65427221)(965.22373527,58.47589054)(965.22373527,57.06964502)
\closepath
\moveto(961.57140315,47.6360813)
\lineto(961.57140315,53.71028071)
\curveto(960.45161505,53.64517675)(959.13000467,53.54752082)(957.60657202,53.4173129)
\curveto(956.09616017,53.28710498)(954.89824731,53.0983035)(954.01283347,52.85090845)
\curveto(952.95814932,52.55143024)(952.10528746,52.08268173)(951.45424786,51.44466293)
\curveto(950.80320827,50.81966492)(950.47768847,49.95378226)(950.47768847,48.84701495)
\curveto(950.47768847,47.59701893)(950.85529144,46.65301152)(951.61049737,46.01499272)
\curveto(952.36570329,45.38999471)(953.51804338,45.0774957)(955.06751761,45.0774957)
\curveto(956.356576,45.0774957)(957.53495767,45.32489075)(958.6026626,45.81968084)
\curveto(959.67036754,46.32749172)(960.65994772,46.93295854)(961.57140315,47.6360813)
\closepath
}
}
{
\newrgbcolor{curcolor}{0 0 0}
\pscustom[linestyle=none,fillstyle=solid,fillcolor=curcolor]
{
\newpath
\moveto(983.62211499,42.46031653)
\curveto(982.93201302,42.27802545)(982.17680709,42.12828634)(981.3564972,42.01109921)
\curveto(980.54920811,41.89391209)(979.82655416,41.83531852)(979.18853535,41.83531852)
\curveto(976.96197994,41.83531852)(975.269277,42.43427495)(974.11042652,43.6321878)
\curveto(972.95157605,44.83010065)(972.37215081,46.75066746)(972.37215081,49.39388821)
\lineto(972.37215081,60.99541377)
\lineto(969.89168995,60.99541377)
\lineto(969.89168995,64.08134144)
\lineto(972.37215081,64.08134144)
\lineto(972.37215081,70.35085273)
\lineto(976.04401412,70.35085273)
\lineto(976.04401412,64.08134144)
\lineto(983.62211499,64.08134144)
\lineto(983.62211499,60.99541377)
\lineto(976.04401412,60.99541377)
\lineto(976.04401412,51.05403917)
\curveto(976.04401412,49.90820949)(976.0700557,49.00977485)(976.12213887,48.35873525)
\curveto(976.17422203,47.72071645)(976.35651312,47.12176003)(976.66901213,46.56186597)
\curveto(976.95546955,46.0410343)(977.3460933,45.65692094)(977.8408834,45.40952589)
\curveto(978.34869428,45.17515164)(979.116921,45.05796451)(980.14556356,45.05796451)
\curveto(980.74451998,45.05796451)(981.36951799,45.14259966)(982.02055759,45.31186995)
\curveto(982.67159718,45.49416104)(983.14034569,45.64390015)(983.42680311,45.76108727)
\lineto(983.62211499,45.76108727)
\closepath
}
}
{
\newrgbcolor{curcolor}{0 0 0}
\pscustom[linestyle=none,fillstyle=solid,fillcolor=curcolor]
{
\newpath
\moveto(1005.00876537,42.26500466)
\lineto(1001.35643325,42.26500466)
\lineto(1001.35643325,44.58921601)
\curveto(1001.03091345,44.36786254)(1000.58820652,44.05536354)(1000.02831247,43.65171899)
\curveto(999.48143921,43.26109523)(998.94758675,42.94859623)(998.42675507,42.71422197)
\curveto(997.81477785,42.41474376)(997.11165509,42.16734872)(996.31738679,41.97203684)
\curveto(995.52311848,41.76370417)(994.59213187,41.65953783)(993.52442693,41.65953783)
\curveto(991.55828736,41.65953783)(989.891626,42.31057743)(988.52444285,43.61265661)
\curveto(987.1572597,44.9147358)(986.47366813,46.57488677)(986.47366813,48.59310951)
\curveto(986.47366813,50.24675008)(986.82522951,51.58138124)(987.52835227,52.59700301)
\curveto(988.24449583,53.62564557)(989.26011759,54.43293466)(990.57521757,55.0188703)
\curveto(991.90333834,55.60480593)(993.49838535,56.00194008)(995.36035859,56.21027275)
\curveto(997.22233182,56.41860542)(999.22102338,56.57485493)(1001.35643325,56.67902126)
\lineto(1001.35643325,57.24542571)
\curveto(1001.35643325,58.07875639)(1001.20669414,58.76885836)(1000.90721593,59.31573162)
\curveto(1000.6207585,59.86260488)(1000.20409316,60.29229101)(999.65721991,60.60479001)
\curveto(999.13638823,60.90426823)(998.51139022,61.1060905)(997.78222588,61.21025683)
\curveto(997.05306153,61.31442317)(996.29134521,61.36650634)(995.4970769,61.36650634)
\curveto(994.5335383,61.36650634)(993.45932297,61.23629842)(992.27443091,60.97588258)
\curveto(991.08953885,60.72848754)(989.86558441,60.36390536)(988.6025676,59.88213606)
\lineto(988.40725572,59.88213606)
\lineto(988.40725572,63.61259294)
\curveto(989.12339928,63.80790481)(990.15855223,64.02274788)(991.51271459,64.25712213)
\curveto(992.86687694,64.49149639)(994.20150811,64.60868351)(995.51660809,64.60868351)
\curveto(997.05306153,64.60868351)(998.3876927,64.4784756)(999.52050159,64.21805976)
\curveto(1000.66633128,63.97066471)(1001.65591146,63.54097858)(1002.48924214,62.92900136)
\curveto(1003.30955203,62.33004494)(1003.93455004,61.55530782)(1004.36423617,60.60479001)
\curveto(1004.7939223,59.65427221)(1005.00876537,58.47589054)(1005.00876537,57.06964502)
\closepath
\moveto(1001.35643325,47.6360813)
\lineto(1001.35643325,53.71028071)
\curveto(1000.23664514,53.64517675)(998.91503477,53.54752082)(997.39160212,53.4173129)
\curveto(995.88119026,53.28710498)(994.68327741,53.0983035)(993.79786356,52.85090845)
\curveto(992.74317942,52.55143024)(991.89031755,52.08268173)(991.23927796,51.44466293)
\curveto(990.58823836,50.81966492)(990.26271857,49.95378226)(990.26271857,48.84701495)
\curveto(990.26271857,47.59701893)(990.64032153,46.65301152)(991.39552746,46.01499272)
\curveto(992.15073339,45.38999471)(993.30307347,45.0774957)(994.8525477,45.0774957)
\curveto(996.1416061,45.0774957)(997.31998776,45.32489075)(998.3876927,45.81968084)
\curveto(999.45539763,46.32749172)(1000.44497781,46.93295854)(1001.35643325,47.6360813)
\closepath
}
}
{
\newrgbcolor{curcolor}{0 0 0}
\pscustom[linewidth=1.88976378,linecolor=curcolor]
{
\newpath
\moveto(741.17471622,210.46440247)
\lineto(510.89747528,113.66646861)
}
}
{
\newrgbcolor{curcolor}{0 0 0}
\pscustom[linestyle=none,fillstyle=solid,fillcolor=curcolor]
{
\newpath
\moveto(528.31855729,120.98948819)
\lineto(532.35778226,130.88712883)
\lineto(510.89747528,113.66646861)
\lineto(538.21619792,116.95026322)
\closepath
}
}
{
\newrgbcolor{curcolor}{0 0 0}
\pscustom[linewidth=2.01574809,linecolor=curcolor]
{
\newpath
\moveto(528.31855729,120.98948819)
\lineto(532.35778226,130.88712883)
\lineto(510.89747528,113.66646861)
\lineto(538.21619792,116.95026322)
\closepath
}
}
{
\newrgbcolor{curcolor}{0.7019608 0.7019608 0.7019608}
\pscustom[linestyle=none,fillstyle=solid,fillcolor=curcolor]
{
\newpath
\moveto(459.3540654,112.06163744)
\lineto(582.59266768,112.06163744)
\lineto(582.59266768,0.94485508)
\lineto(459.3540654,0.94485508)
\closepath
}
}
{
\newrgbcolor{curcolor}{0 0 0}
\pscustom[linewidth=1.88975995,linecolor=curcolor]
{
\newpath
\moveto(459.3540654,112.06163744)
\lineto(582.59266768,112.06163744)
\lineto(582.59266768,0.94485508)
\lineto(459.3540654,0.94485508)
\closepath
}
}
{
\newrgbcolor{curcolor}{0 0 0}
\pscustom[linestyle=none,fillstyle=solid,fillcolor=curcolor]
{
\newpath
\moveto(501.91091524,56.7766772)
\curveto(501.91091524,54.13345645)(501.33149,51.73763075)(500.17263953,49.58920009)
\curveto(499.02680984,47.44076943)(497.4968668,45.77410807)(495.58281039,44.58921601)
\curveto(494.25468962,43.76890612)(492.77031935,43.17646009)(491.12969957,42.81187791)
\curveto(489.50210059,42.44729574)(487.35366993,42.26500466)(484.68440759,42.26500466)
\lineto(477.34068097,42.26500466)
\lineto(477.34068097,71.34694331)
\lineto(484.60628284,71.34694331)
\curveto(487.44481547,71.34694331)(489.69741246,71.13861064)(491.36407382,70.7219453)
\curveto(493.04375598,70.31830075)(494.46302229,69.7584067)(495.62187277,69.04226315)
\curveto(497.60103313,67.80528792)(499.14399697,66.15815775)(500.25076428,64.10087263)
\curveto(501.35753159,62.04358751)(501.91091524,59.60218904)(501.91091524,56.7766772)
\closepath
\moveto(497.86795937,56.83527076)
\curveto(497.86795937,59.11390934)(497.47082521,61.03447614)(496.67655691,62.59697117)
\curveto(495.8822886,64.15946619)(494.69739654,65.38993103)(493.12188073,66.28836567)
\curveto(491.97605104,66.93940526)(490.758607,67.38862258)(489.46954861,67.63601763)
\curveto(488.18049021,67.89643346)(486.63752637,68.02664138)(484.84065709,68.02664138)
\lineto(481.20785616,68.02664138)
\lineto(481.20785616,45.58530658)
\lineto(484.84065709,45.58530658)
\curveto(486.70263033,45.58530658)(488.32371892,45.7220249)(489.70392286,45.99546153)
\curveto(491.09714759,46.26889816)(492.37318519,46.77670904)(493.53203567,47.51889418)
\curveto(494.97734357,48.4433704)(496.0580693,49.66081444)(496.77421285,51.1712263)
\curveto(497.50337719,52.68163816)(497.86795937,54.56965298)(497.86795937,56.83527076)
\closepath
}
}
{
\newrgbcolor{curcolor}{0 0 0}
\pscustom[linestyle=none,fillstyle=solid,fillcolor=curcolor]
{
\newpath
\moveto(524.82099936,42.26500466)
\lineto(521.16866724,42.26500466)
\lineto(521.16866724,44.58921601)
\curveto(520.84314744,44.36786254)(520.40044052,44.05536354)(519.84054647,43.65171899)
\curveto(519.29367321,43.26109523)(518.75982074,42.94859623)(518.23898907,42.71422197)
\curveto(517.62701185,42.41474376)(516.92388909,42.16734872)(516.12962078,41.97203684)
\curveto(515.33535248,41.76370417)(514.40436586,41.65953783)(513.33666092,41.65953783)
\curveto(511.37052135,41.65953783)(509.70385999,42.31057743)(508.33667684,43.61265661)
\curveto(506.9694937,44.9147358)(506.28590212,46.57488677)(506.28590212,48.59310951)
\curveto(506.28590212,50.24675008)(506.6374635,51.58138124)(507.34058627,52.59700301)
\curveto(508.05672982,53.62564557)(509.07235158,54.43293466)(510.38745156,55.0188703)
\curveto(511.71557234,55.60480593)(513.31061934,56.00194008)(515.17259258,56.21027275)
\curveto(517.03456582,56.41860542)(519.03325737,56.57485493)(521.16866724,56.67902126)
\lineto(521.16866724,57.24542571)
\curveto(521.16866724,58.07875639)(521.01892813,58.76885836)(520.71944992,59.31573162)
\curveto(520.4329925,59.86260488)(520.01632716,60.29229101)(519.4694539,60.60479001)
\curveto(518.94862222,60.90426823)(518.32362421,61.1060905)(517.59445987,61.21025683)
\curveto(516.86529552,61.31442317)(516.1035792,61.36650634)(515.30931089,61.36650634)
\curveto(514.34577229,61.36650634)(513.27155696,61.23629842)(512.0866649,60.97588258)
\curveto(510.90177284,60.72848754)(509.67781841,60.36390536)(508.41480159,59.88213606)
\lineto(508.21948972,59.88213606)
\lineto(508.21948972,63.61259294)
\curveto(508.93563327,63.80790481)(509.97078622,64.02274788)(511.32494858,64.25712213)
\curveto(512.67911093,64.49149639)(514.0137421,64.60868351)(515.32884208,64.60868351)
\curveto(516.86529552,64.60868351)(518.19992669,64.4784756)(519.33273558,64.21805976)
\curveto(520.47856527,63.97066471)(521.46814545,63.54097858)(522.30147613,62.92900136)
\curveto(523.12178602,62.33004494)(523.74678403,61.55530782)(524.17647016,60.60479001)
\curveto(524.60615629,59.65427221)(524.82099936,58.47589054)(524.82099936,57.06964502)
\closepath
\moveto(521.16866724,47.6360813)
\lineto(521.16866724,53.71028071)
\curveto(520.04887914,53.64517675)(518.72726876,53.54752082)(517.20383611,53.4173129)
\curveto(515.69342425,53.28710498)(514.4955114,53.0983035)(513.61009755,52.85090845)
\curveto(512.55541341,52.55143024)(511.70255154,52.08268173)(511.05151195,51.44466293)
\curveto(510.40047236,50.81966492)(510.07495256,49.95378226)(510.07495256,48.84701495)
\curveto(510.07495256,47.59701893)(510.45255552,46.65301152)(511.20776145,46.01499272)
\curveto(511.96296738,45.38999471)(513.11530746,45.0774957)(514.6647817,45.0774957)
\curveto(515.95384009,45.0774957)(517.13222176,45.32489075)(518.19992669,45.81968084)
\curveto(519.26763162,46.32749172)(520.25721181,46.93295854)(521.16866724,47.6360813)
\closepath
}
}
{
\newrgbcolor{curcolor}{0 0 0}
\pscustom[linestyle=none,fillstyle=solid,fillcolor=curcolor]
{
\newpath
\moveto(543.21937908,42.46031653)
\curveto(542.52927711,42.27802545)(541.77407118,42.12828634)(540.95376129,42.01109921)
\curveto(540.14647219,41.89391209)(539.42381824,41.83531852)(538.78579944,41.83531852)
\curveto(536.55924403,41.83531852)(534.86654109,42.43427495)(533.70769061,43.6321878)
\curveto(532.54884013,44.83010065)(531.96941489,46.75066746)(531.96941489,49.39388821)
\lineto(531.96941489,60.99541377)
\lineto(529.48895404,60.99541377)
\lineto(529.48895404,64.08134144)
\lineto(531.96941489,64.08134144)
\lineto(531.96941489,70.35085273)
\lineto(535.6412782,70.35085273)
\lineto(535.6412782,64.08134144)
\lineto(543.21937908,64.08134144)
\lineto(543.21937908,60.99541377)
\lineto(535.6412782,60.99541377)
\lineto(535.6412782,51.05403917)
\curveto(535.6412782,49.90820949)(535.66731979,49.00977485)(535.71940295,48.35873525)
\curveto(535.77148612,47.72071645)(535.95377721,47.12176003)(536.26627621,46.56186597)
\curveto(536.55273363,46.0410343)(536.94335739,45.65692094)(537.43814748,45.40952589)
\curveto(537.94595837,45.17515164)(538.71418509,45.05796451)(539.74282764,45.05796451)
\curveto(540.34178407,45.05796451)(540.96678208,45.14259966)(541.61782167,45.31186995)
\curveto(542.26886127,45.49416104)(542.73760978,45.64390015)(543.0240672,45.76108727)
\lineto(543.21937908,45.76108727)
\closepath
}
}
{
\newrgbcolor{curcolor}{0 0 0}
\pscustom[linestyle=none,fillstyle=solid,fillcolor=curcolor]
{
\newpath
\moveto(564.60602945,42.26500466)
\lineto(560.95369733,42.26500466)
\lineto(560.95369733,44.58921601)
\curveto(560.62817754,44.36786254)(560.18547061,44.05536354)(559.62557656,43.65171899)
\curveto(559.0787033,43.26109523)(558.54485084,42.94859623)(558.02401916,42.71422197)
\curveto(557.41204194,42.41474376)(556.70891918,42.16734872)(555.91465088,41.97203684)
\curveto(555.12038257,41.76370417)(554.18939595,41.65953783)(553.12169102,41.65953783)
\curveto(551.15555145,41.65953783)(549.48889009,42.31057743)(548.12170694,43.61265661)
\curveto(546.75452379,44.9147358)(546.07093222,46.57488677)(546.07093222,48.59310951)
\curveto(546.07093222,50.24675008)(546.4224936,51.58138124)(547.12561636,52.59700301)
\curveto(547.84175991,53.62564557)(548.85738168,54.43293466)(550.17248166,55.0188703)
\curveto(551.50060243,55.60480593)(553.09564944,56.00194008)(554.95762267,56.21027275)
\curveto(556.81959591,56.41860542)(558.81828746,56.57485493)(560.95369733,56.67902126)
\lineto(560.95369733,57.24542571)
\curveto(560.95369733,58.07875639)(560.80395823,58.76885836)(560.50448001,59.31573162)
\curveto(560.21802259,59.86260488)(559.80135725,60.29229101)(559.25448399,60.60479001)
\curveto(558.73365232,60.90426823)(558.10865431,61.1060905)(557.37948996,61.21025683)
\curveto(556.65032562,61.31442317)(555.88860929,61.36650634)(555.09434099,61.36650634)
\curveto(554.13080239,61.36650634)(553.05658706,61.23629842)(551.871695,60.97588258)
\curveto(550.68680294,60.72848754)(549.4628485,60.36390536)(548.19983169,59.88213606)
\lineto(548.00451981,59.88213606)
\lineto(548.00451981,63.61259294)
\curveto(548.72066336,63.80790481)(549.75581632,64.02274788)(551.10997867,64.25712213)
\curveto(552.46414103,64.49149639)(553.7987722,64.60868351)(555.11387218,64.60868351)
\curveto(556.65032562,64.60868351)(557.98495678,64.4784756)(559.11776568,64.21805976)
\curveto(560.26359536,63.97066471)(561.25317555,63.54097858)(562.08650623,62.92900136)
\curveto(562.90681611,62.33004494)(563.53181412,61.55530782)(563.96150026,60.60479001)
\curveto(564.39118639,59.65427221)(564.60602945,58.47589054)(564.60602945,57.06964502)
\closepath
\moveto(560.95369733,47.6360813)
\lineto(560.95369733,53.71028071)
\curveto(559.83390923,53.64517675)(558.51229886,53.54752082)(556.98886621,53.4173129)
\curveto(555.47845435,53.28710498)(554.2805415,53.0983035)(553.39512765,52.85090845)
\curveto(552.34044351,52.55143024)(551.48758164,52.08268173)(550.83654204,51.44466293)
\curveto(550.18550245,50.81966492)(549.85998265,49.95378226)(549.85998265,48.84701495)
\curveto(549.85998265,47.59701893)(550.23758562,46.65301152)(550.99279155,46.01499272)
\curveto(551.74799748,45.38999471)(552.90033756,45.0774957)(554.44981179,45.0774957)
\curveto(555.73887019,45.0774957)(556.91725185,45.32489075)(557.98495678,45.81968084)
\curveto(559.05266172,46.32749172)(560.0422419,46.93295854)(560.95369733,47.6360813)
\closepath
}
}
{
\newrgbcolor{curcolor}{0 0 0}
\pscustom[linestyle=none,fillstyle=solid,fillcolor=curcolor]
{
\newpath
\moveto(479.06544707,160.55915374)
\lineto(460.66706815,160.55915374)
\lineto(460.66706815,189.64109239)
\lineto(464.53424334,189.64109239)
\lineto(464.53424334,163.99664279)
\lineto(479.06544707,163.99664279)
\closepath
}
}
{
\newrgbcolor{curcolor}{0 0 0}
\pscustom[linestyle=none,fillstyle=solid,fillcolor=curcolor]
{
\newpath
\moveto(500.90131566,171.08646397)
\lineto(484.82714808,171.08646397)
\curveto(484.82714808,169.74532241)(485.02897036,168.57345114)(485.43261491,167.57085016)
\curveto(485.83625946,166.58126998)(486.38964311,165.76747049)(487.09276587,165.12945169)
\curveto(487.76984705,164.50445368)(488.57062575,164.03570517)(489.49510197,163.72320616)
\curveto(490.43259899,163.41070716)(491.46124155,163.25445766)(492.58102965,163.25445766)
\curveto(494.06539992,163.25445766)(495.55628059,163.54742547)(497.05367166,164.13336111)
\curveto(498.56408351,164.73231753)(499.63829884,165.31825317)(500.27631765,165.89116801)
\lineto(500.47162952,165.89116801)
\lineto(500.47162952,161.88727451)
\curveto(499.2346543,161.36644283)(497.97163748,160.93024631)(496.68257909,160.57868493)
\curveto(495.39352069,160.22712355)(494.03935834,160.05134286)(492.62009202,160.05134286)
\curveto(489.00031188,160.05134286)(486.17480004,161.02790225)(484.14355651,162.98102103)
\curveto(482.11231298,164.9471606)(481.09669121,167.73361006)(481.09669121,171.34036941)
\curveto(481.09669121,174.90806639)(482.06674021,177.74008862)(484.0068382,179.83643611)
\curveto(485.95995698,181.9327836)(488.52505298,182.98095735)(491.7021262,182.98095735)
\curveto(494.64482516,182.98095735)(496.91044295,182.12158509)(498.49897955,180.40284056)
\curveto(500.10053696,178.68409603)(500.90131566,176.24269755)(500.90131566,173.07864513)
\closepath
\moveto(497.32710829,173.89895502)
\curveto(497.31408749,175.82603221)(496.8258078,177.31691288)(495.8622692,178.37159703)
\curveto(494.91175139,179.42628117)(493.4599331,179.95362324)(491.50681432,179.95362324)
\curveto(489.54067474,179.95362324)(487.97166932,179.374198)(486.79979805,178.21534752)
\curveto(485.64094758,177.05649705)(484.98339759,175.61769954)(484.82714808,173.89895502)
\closepath
}
}
{
\newrgbcolor{curcolor}{0 0 0}
\pscustom[linestyle=none,fillstyle=solid,fillcolor=curcolor]
{
\newpath
\moveto(525.33483239,182.37549053)
\lineto(516.50673549,160.55915374)
\lineto(512.815341,160.55915374)
\lineto(504.04583767,182.37549053)
\lineto(508.03019998,182.37549053)
\lineto(514.78799097,165.01226456)
\lineto(521.48718839,182.37549053)
\closepath
}
}
{
\newrgbcolor{curcolor}{0 0 0}
\pscustom[linestyle=none,fillstyle=solid,fillcolor=curcolor]
{
\newpath
\moveto(548.04960237,171.08646397)
\lineto(531.9754348,171.08646397)
\curveto(531.9754348,169.74532241)(532.17725707,168.57345114)(532.58090162,167.57085016)
\curveto(532.98454617,166.58126998)(533.53792982,165.76747049)(534.24105259,165.12945169)
\curveto(534.91813376,164.50445368)(535.71891246,164.03570517)(536.64338869,163.72320616)
\curveto(537.5808857,163.41070716)(538.60952826,163.25445766)(539.72931636,163.25445766)
\curveto(541.21368664,163.25445766)(542.7045673,163.54742547)(544.20195837,164.13336111)
\curveto(545.71237023,164.73231753)(546.78658556,165.31825317)(547.42460436,165.89116801)
\lineto(547.61991624,165.89116801)
\lineto(547.61991624,161.88727451)
\curveto(546.38294101,161.36644283)(545.1199242,160.93024631)(543.8308658,160.57868493)
\curveto(542.54180741,160.22712355)(541.18764505,160.05134286)(539.76837874,160.05134286)
\curveto(536.1485986,160.05134286)(533.32308676,161.02790225)(531.29184323,162.98102103)
\curveto(529.26059969,164.9471606)(528.24497793,167.73361006)(528.24497793,171.34036941)
\curveto(528.24497793,174.90806639)(529.21502692,177.74008862)(531.15512491,179.83643611)
\curveto(533.10824369,181.9327836)(535.67333969,182.98095735)(538.85041291,182.98095735)
\curveto(541.79311187,182.98095735)(544.05872966,182.12158509)(545.64726627,180.40284056)
\curveto(547.24882367,178.68409603)(548.04960237,176.24269755)(548.04960237,173.07864513)
\closepath
\moveto(544.475395,173.89895502)
\curveto(544.46237421,175.82603221)(543.97409451,177.31691288)(543.01055591,178.37159703)
\curveto(542.06003811,179.42628117)(540.60821981,179.95362324)(538.65510103,179.95362324)
\curveto(536.68896146,179.95362324)(535.11995604,179.374198)(533.94808477,178.21534752)
\curveto(532.78923429,177.05649705)(532.1316843,175.61769954)(531.9754348,173.89895502)
\closepath
}
}
{
\newrgbcolor{curcolor}{0 0 0}
\pscustom[linestyle=none,fillstyle=solid,fillcolor=curcolor]
{
\newpath
\moveto(557.32691555,160.55915374)
\lineto(553.65505224,160.55915374)
\lineto(553.65505224,190.94968198)
\lineto(557.32691555,190.94968198)
\closepath
}
}
{
\newrgbcolor{curcolor}{0 0 0}
\pscustom[linestyle=none,fillstyle=solid,fillcolor=curcolor]
{
\newpath
\moveto(597.87366669,160.55915374)
\lineto(578.18622937,160.55915374)
\lineto(578.18622937,164.64117199)
\lineto(582.28777881,168.1567858)
\curveto(583.66798275,169.32865707)(584.95053075,170.49401794)(586.13542281,171.65286842)
\curveto(588.63541485,174.07473571)(590.34764898,175.99530251)(591.27212521,177.41456882)
\curveto(592.19660143,178.84685593)(592.65883954,180.38981977)(592.65883954,182.04346033)
\curveto(592.65883954,183.55387219)(592.15753905,184.73225386)(591.15493808,185.57860533)
\curveto(590.1653579,186.43797759)(588.77864356,186.86766372)(586.99479507,186.86766372)
\curveto(585.80990301,186.86766372)(584.52735501,186.65933105)(583.14715107,186.24266571)
\curveto(581.76694714,185.82600037)(580.41929518,185.18798157)(579.1041952,184.32860931)
\lineto(578.90888332,184.32860931)
\lineto(578.90888332,188.43015875)
\curveto(579.83335954,188.88588646)(581.06382437,189.30255181)(582.60027782,189.68015477)
\curveto(584.14975205,190.05775773)(585.64714311,190.24655922)(587.09245101,190.24655922)
\curveto(590.07421235,190.24655922)(592.41144449,189.52390527)(594.10414744,188.07859737)
\curveto(595.79685038,186.64631026)(596.64320185,184.69970188)(596.64320185,182.23877221)
\curveto(596.64320185,181.1320049)(596.49997314,180.09685195)(596.21351572,179.13331335)
\curveto(595.94007909,178.18279554)(595.52992415,177.27785051)(594.98305089,176.41847824)
\curveto(594.47524001,175.61118915)(593.87628358,174.81692084)(593.18618161,174.03567333)
\curveto(592.50910043,173.25442582)(591.68228015,172.38854316)(590.70572076,171.43802535)
\curveto(589.31249603,170.0708422)(587.87369853,168.74272143)(586.38932825,167.45366304)
\curveto(584.90495798,166.17762543)(583.51824364,164.99273337)(582.22918525,163.89898685)
\lineto(597.87366669,163.89898685)
\closepath
}
}
{
\newrgbcolor{curcolor}{0 0 0}
\pscustom[linestyle=none,fillstyle=solid,fillcolor=curcolor]
{
\newpath
\moveto(268.0895682,656.32503922)
\lineto(263.07005294,656.32503922)
\lineto(253.3435214,667.88750241)
\lineto(247.89432,667.88750241)
\lineto(247.89432,656.32503922)
\lineto(244.02714482,656.32503922)
\lineto(244.02714482,685.40697787)
\lineto(252.17165014,685.40697787)
\curveto(253.92945704,685.40697787)(255.39429613,685.28979075)(256.56616739,685.05541649)
\curveto(257.73803866,684.83406303)(258.7927228,684.43041848)(259.73021982,683.84448285)
\curveto(260.78490396,683.18042246)(261.60521385,682.34058139)(262.19114948,681.32495962)
\curveto(262.79010591,680.32235865)(263.08958412,679.04632104)(263.08958412,677.49684681)
\curveto(263.08958412,675.40049932)(262.56224205,673.64269241)(261.50755791,672.2234261)
\curveto(260.45287377,670.81718058)(259.00105547,669.75598604)(257.15210303,669.03984249)
\closepath
\moveto(259.04662825,677.22341018)
\curveto(259.04662825,678.05674086)(258.89688914,678.7924156)(258.59741093,679.4304344)
\curveto(258.31095351,680.081474)(257.82918421,680.62834726)(257.15210303,681.07105418)
\curveto(256.59220898,681.44865714)(255.92814859,681.70907298)(255.15992187,681.85230169)
\curveto(254.39169515,682.00855119)(253.48675012,682.08667595)(252.44508677,682.08667595)
\lineto(247.89432,682.08667595)
\lineto(247.89432,671.11014839)
\lineto(251.80055757,671.11014839)
\curveto(253.024512,671.11014839)(254.09221694,671.21431473)(255.00367237,671.4226474)
\curveto(255.9151278,671.64400086)(256.68986492,672.04764541)(257.32788372,672.63358104)
\curveto(257.91381935,673.1804543)(258.34350549,673.80545231)(258.61694211,674.50857507)
\curveto(258.90339954,675.22471863)(259.04662825,676.12966366)(259.04662825,677.22341018)
\closepath
}
}
{
\newrgbcolor{curcolor}{0 0 0}
\pscustom[linestyle=none,fillstyle=solid,fillcolor=curcolor]
{
\newpath
\moveto(289.16372059,667.22344202)
\curveto(289.16372059,663.66876584)(288.25226516,660.86278519)(286.4293543,658.80550007)
\curveto(284.60644344,656.74821496)(282.16504496,655.7195724)(279.10515887,655.7195724)
\curveto(276.0192312,655.7195724)(273.56481193,656.74821496)(271.74190106,658.80550007)
\curveto(269.93201099,660.86278519)(269.02706596,663.66876584)(269.02706596,667.22344202)
\curveto(269.02706596,670.7781182)(269.93201099,673.58409885)(271.74190106,675.64138397)
\curveto(273.56481193,677.71168988)(276.0192312,678.74684283)(279.10515887,678.74684283)
\curveto(282.16504496,678.74684283)(284.60644344,677.71168988)(286.4293543,675.64138397)
\curveto(288.25226516,673.58409885)(289.16372059,670.7781182)(289.16372059,667.22344202)
\closepath
\moveto(285.37467016,667.22344202)
\curveto(285.37467016,670.04895386)(284.8212865,672.14530135)(283.71451919,673.5124845)
\curveto(282.60775189,674.89268843)(281.07129844,675.5827904)(279.10515887,675.5827904)
\curveto(277.11297771,675.5827904)(275.56350348,674.89268843)(274.45673617,673.5124845)
\curveto(273.36298965,672.14530135)(272.81611639,670.04895386)(272.81611639,667.22344202)
\curveto(272.81611639,664.48907573)(273.36950005,662.41225942)(274.47626736,660.99299311)
\curveto(275.58303467,659.58674758)(277.12599851,658.88362482)(279.10515887,658.88362482)
\curveto(281.05827765,658.88362482)(282.5882207,659.58023719)(283.69498801,660.97346192)
\curveto(284.81477611,662.37970744)(285.37467016,664.46303414)(285.37467016,667.22344202)
\closepath
}
}
{
\newrgbcolor{curcolor}{0 0 0}
\pscustom[linestyle=none,fillstyle=solid,fillcolor=curcolor]
{
\newpath
\moveto(313.44098805,667.22344202)
\curveto(313.44098805,663.66876584)(312.52953262,660.86278519)(310.70662176,658.80550007)
\curveto(308.88371089,656.74821496)(306.44231242,655.7195724)(303.38242633,655.7195724)
\curveto(300.29649865,655.7195724)(297.84207938,656.74821496)(296.01916852,658.80550007)
\curveto(294.20927845,660.86278519)(293.30433341,663.66876584)(293.30433341,667.22344202)
\curveto(293.30433341,670.7781182)(294.20927845,673.58409885)(296.01916852,675.64138397)
\curveto(297.84207938,677.71168988)(300.29649865,678.74684283)(303.38242633,678.74684283)
\curveto(306.44231242,678.74684283)(308.88371089,677.71168988)(310.70662176,675.64138397)
\curveto(312.52953262,673.58409885)(313.44098805,670.7781182)(313.44098805,667.22344202)
\closepath
\moveto(309.65193761,667.22344202)
\curveto(309.65193761,670.04895386)(309.09855396,672.14530135)(307.99178665,673.5124845)
\curveto(306.88501934,674.89268843)(305.3485659,675.5827904)(303.38242633,675.5827904)
\curveto(301.39024517,675.5827904)(299.84077094,674.89268843)(298.73400363,673.5124845)
\curveto(297.64025711,672.14530135)(297.09338385,670.04895386)(297.09338385,667.22344202)
\curveto(297.09338385,664.48907573)(297.64676751,662.41225942)(298.75353481,660.99299311)
\curveto(299.86030212,659.58674758)(301.40326596,658.88362482)(303.38242633,658.88362482)
\curveto(305.33554511,658.88362482)(306.86548815,659.58023719)(307.97225546,660.97346192)
\curveto(309.09204356,662.37970744)(309.65193761,664.46303414)(309.65193761,667.22344202)
\closepath
}
}
{
\newrgbcolor{curcolor}{0 0 0}
\pscustom[linestyle=none,fillstyle=solid,fillcolor=curcolor]
{
\newpath
\moveto(330.45264823,656.5203511)
\curveto(329.76254626,656.33806001)(329.00734034,656.1883209)(328.18703045,656.07113378)
\curveto(327.37974135,655.95394665)(326.6570874,655.89535309)(326.0190686,655.89535309)
\curveto(323.79251319,655.89535309)(322.09981024,656.49430951)(320.94095977,657.69222237)
\curveto(319.78210929,658.89013522)(319.20268405,660.81070202)(319.20268405,663.45392277)
\lineto(319.20268405,675.05544833)
\lineto(316.7222232,675.05544833)
\lineto(316.7222232,678.14137601)
\lineto(319.20268405,678.14137601)
\lineto(319.20268405,684.4108873)
\lineto(322.87454736,684.4108873)
\lineto(322.87454736,678.14137601)
\lineto(330.45264823,678.14137601)
\lineto(330.45264823,675.05544833)
\lineto(322.87454736,675.05544833)
\lineto(322.87454736,665.11407374)
\curveto(322.87454736,663.96824405)(322.90058895,663.06980941)(322.95267211,662.41876982)
\curveto(323.00475528,661.78075102)(323.18704637,661.18179459)(323.49954537,660.62190054)
\curveto(323.78600279,660.10106886)(324.17662655,659.7169555)(324.67141664,659.46956046)
\curveto(325.17922752,659.2351862)(325.94745424,659.11799908)(326.9760968,659.11799908)
\curveto(327.57505323,659.11799908)(328.20005124,659.20263422)(328.85109083,659.37190452)
\curveto(329.50213043,659.5541956)(329.97087893,659.70393471)(330.25733636,659.82112184)
\lineto(330.45264823,659.82112184)
\closepath
}
}
\end{pspicture}
}
    \captionsetup{width=0.5\linewidth}
    \caption{Direct and Indirect Blocks. Directly pointing to a data block makes it a level 0 block,
    while pointing to one indirect block before the data block makes it level 1, two indirect blocks
    makes it level 2, and so on.}

    \label{fig:directindirectblocks}
\end{figure}

Using blocks allows filesystems to have variably-sized non-contiguous data structures, which is critical when storing structures
like files that can be widely varying in size.

\section{Volume Managers} 
\label{chapter:volumemanagers}
Typical filesystems require a volume manager when working with multiple disks.
This is because the typical filesystem can only work with one disk as a time.
Volume managers solve this problem by acting as an intermediate layer between the filesystem and a set of disks,
making them appear as one logical disk to the filesystem.
They can be organized in a variety of ways to provide redundancy, as discussed in the next chapter.
They can also subdivide one disk, or even one disk partition into smaller units that can be extended as needed.

Modern filesystems, like ZFS and BTRFS, no longer require this intermediate layer, instead building in support for multiple disks
and disk sharing between filesystems directly.
This allows for more fine-grained control of where data will be stored, at the expense of each filesystem having to design its own method
for spreading its data across multiple disks in various configurations, such as mirroring and RAID.

\section{Mirrors and RAID}
When a filesystem is made up of more than one disk, data can be arranged in different strategies to prevent any one disk
failure from losing all of your data.
For simpler filesystems without multiple disk support this is all handled by the volume layer,
but some more advanced filesystems handle this themselves.
The most conservative of these strategies is mirroring, or RAID-1, where all data is duplicated exactly on multiple disks,
with the filesystem only having access to one disk's worth of storage\cite{patterson_case_1988}.
This solution dramatically reduces the amount of storage available but ensures all data is easily recoverable in
the event of a disk failure.
Beyond this are the levels of RAID, that trade off various amounts of disk storage for redundancy, adding parity bits and other
tricks in order to ensure data recovery if only some number of disks are lost.
Some RAID configurations have parity information all on one disk or a set of disks, 
allowing any one disk, or some number of disks, but not all the parity disks, to fail while still recovering all data.
Others spread the parity information out among all the disks, reducing the redundancy but allowing for many writes at once,
not being constrained by having to always write to the parity disk.

When this is handled by the filesystem, it can use what it knows about the redundancy of the data to transparently repair
the filesystem, rewriting data if it detects that one of the redundant copies has become corrupted or was incorrectly
written to disk.
This requires some kind of corruption detection mechanism, but such a mechanism is typically a standard part of a RAID configuration.

\section{State Consistency}
One goal of every filesystem is to never be in an unrecoverable state on disk
\cite{ahrens_openzfs_basics,mckusick_zfs_2015_presentation}.
Most filesystems handle this situation through a journal system and a filesystem check, also known as \texttt{fsck},
that can restore the filesystem to a working state, even if a write is interrupted.
Other filesystems, such as ZFS and BTRFS, try to categorically eliminate this source of failure through a Copy on Write
model.
Copy on Write means that any changes to a file will result in a write to an entirely different part of the disk, instead of 
overwriting the previous contents of the file.
This means that even if a write fails, the filesystem will simply appear to be in exactly the same state as before, because
only overwriting the root block, or \textit{superblock}, of the filesystem actually changes its state in a destructive way, 
linking up all the new, and usually some old, blocks into the filesystem all at once.

\begin{figure}[H]
    \centering
    \resizebox{0.75\linewidth}{!}{%LaTeX with PSTricks extensions
%%Creator: Inkscape 1.0.2-2 (e86c870879, 2021-01-15)
%%Please note this file requires PSTricks extensions
\psset{xunit=.5pt,yunit=.5pt,runit=.5pt}
\begin{pspicture}(901.16387939,425.9927063)
{
\newrgbcolor{curcolor}{0 0 0}
\pscustom[linestyle=none,fillstyle=solid,fillcolor=curcolor,opacity=0]
{
\newpath
\moveto(0.00023118,486.11849263)
\lineto(894.54960082,486.11849263)
\lineto(894.54960082,425.99263014)
\lineto(0.00023118,425.99263014)
\closepath
}
}
{
\newrgbcolor{curcolor}{0.53725493 0.61176473 0.90196079}
\pscustom[linestyle=none,fillstyle=solid,fillcolor=curcolor]
{
\newpath
\moveto(26.97966736,260.51837205)
\lineto(26.97966736,260.51837205)
\curveto(26.97966736,260.53280771)(26.99137338,260.54451373)(27.00581429,260.54451373)
\lineto(104.49666787,260.54451373)
\curveto(104.50359699,260.54451373)(104.51025577,260.54175783)(104.5151639,260.5368497)
\curveto(104.52006415,260.53194157)(104.52282267,260.52530117)(104.52282267,260.51837205)
\lineto(104.52282267,237.01164799)
\curveto(104.52282267,236.99721233)(104.51111404,236.98550632)(104.49666787,236.98550632)
\lineto(27.00581429,236.98550632)
\curveto(26.99137338,236.98550632)(26.97966736,236.99721233)(26.97966736,237.01164799)
\closepath
}
}
{
\newrgbcolor{curcolor}{0 0 0}
\pscustom[linewidth=1.92650529,linecolor=curcolor]
{
\newpath
\moveto(26.97966736,260.51837205)
\lineto(26.97966736,260.51837205)
\curveto(26.97966736,260.53280771)(26.99137338,260.54451373)(27.00581429,260.54451373)
\lineto(104.49666787,260.54451373)
\curveto(104.50359699,260.54451373)(104.51025577,260.54175783)(104.5151639,260.5368497)
\curveto(104.52006415,260.53194157)(104.52282267,260.52530117)(104.52282267,260.51837205)
\lineto(104.52282267,237.01164799)
\curveto(104.52282267,236.99721233)(104.51111404,236.98550632)(104.49666787,236.98550632)
\lineto(27.00581429,236.98550632)
\curveto(26.99137338,236.98550632)(26.97966736,236.99721233)(26.97966736,237.01164799)
\closepath
}
}
{
\newrgbcolor{curcolor}{0.53725493 0.61176473 0.90196079}
\pscustom[linestyle=none,fillstyle=solid,fillcolor=curcolor]
{
\newpath
\moveto(128.33928101,260.51837205)
\lineto(128.33928101,260.51837205)
\curveto(128.33928101,260.53280771)(128.35099489,260.54451373)(128.36543581,260.54451373)
\lineto(205.85627102,260.54451373)
\curveto(205.86322638,260.54451373)(205.86986679,260.54175783)(205.87477492,260.5368497)
\curveto(205.87968304,260.53194157)(205.88243894,260.52530117)(205.88243894,260.51837205)
\lineto(205.88243894,237.01164799)
\curveto(205.88243894,236.99721233)(205.87073293,236.98550632)(205.85627102,236.98550632)
\lineto(128.36543581,236.98550632)
\curveto(128.35099489,236.98550632)(128.33928101,236.99721233)(128.33928101,237.01164799)
\closepath
}
}
{
\newrgbcolor{curcolor}{0 0 0}
\pscustom[linewidth=1.92650529,linecolor=curcolor]
{
\newpath
\moveto(128.33928101,260.51837205)
\lineto(128.33928101,260.51837205)
\curveto(128.33928101,260.53280771)(128.35099489,260.54451373)(128.36543581,260.54451373)
\lineto(205.85627102,260.54451373)
\curveto(205.86322638,260.54451373)(205.86986679,260.54175783)(205.87477492,260.5368497)
\curveto(205.87968304,260.53194157)(205.88243894,260.52530117)(205.88243894,260.51837205)
\lineto(205.88243894,237.01164799)
\curveto(205.88243894,236.99721233)(205.87073293,236.98550632)(205.85627102,236.98550632)
\lineto(128.36543581,236.98550632)
\curveto(128.35099489,236.98550632)(128.33928101,236.99721233)(128.33928101,237.01164799)
\closepath
}
}
{
\newrgbcolor{curcolor}{0.53725493 0.61176473 0.90196079}
\pscustom[linestyle=none,fillstyle=solid,fillcolor=curcolor]
{
\newpath
\moveto(228.20829924,260.51837205)
\lineto(228.20829924,260.51837205)
\curveto(228.20829924,260.53280771)(228.22001576,260.54451373)(228.23445142,260.54451373)
\lineto(305.72530763,260.54451373)
\curveto(305.73223675,260.54451373)(305.73887715,260.54175783)(305.74381152,260.5368497)
\curveto(305.74871965,260.53194157)(305.7514493,260.52530117)(305.7514493,260.51837205)
\lineto(305.7514493,237.01164799)
\curveto(305.7514493,236.99721233)(305.73974329,236.98550632)(305.72530763,236.98550632)
\lineto(228.23445142,236.98550632)
\curveto(228.22001576,236.98550632)(228.20829924,236.99721233)(228.20829924,237.01164799)
\closepath
}
}
{
\newrgbcolor{curcolor}{0 0 0}
\pscustom[linewidth=1.92650529,linecolor=curcolor]
{
\newpath
\moveto(228.20829924,260.51837205)
\lineto(228.20829924,260.51837205)
\curveto(228.20829924,260.53280771)(228.22001576,260.54451373)(228.23445142,260.54451373)
\lineto(305.72530763,260.54451373)
\curveto(305.73223675,260.54451373)(305.73887715,260.54175783)(305.74381152,260.5368497)
\curveto(305.74871965,260.53194157)(305.7514493,260.52530117)(305.7514493,260.51837205)
\lineto(305.7514493,237.01164799)
\curveto(305.7514493,236.99721233)(305.73974329,236.98550632)(305.72530763,236.98550632)
\lineto(228.23445142,236.98550632)
\curveto(228.22001576,236.98550632)(228.20829924,236.99721233)(228.20829924,237.01164799)
\closepath
}
}
{
\newrgbcolor{curcolor}{0.53725493 0.61176473 0.90196079}
\pscustom[linestyle=none,fillstyle=solid,fillcolor=curcolor]
{
\newpath
\moveto(328.0773621,260.51837205)
\lineto(328.0773621,260.51837205)
\curveto(328.0773621,260.53280771)(328.08906811,260.54451373)(328.10353003,260.54451373)
\lineto(405.59434424,260.54451373)
\curveto(405.60127336,260.54451373)(405.60796626,260.54175783)(405.61284813,260.5368497)
\curveto(405.61778251,260.53194157)(405.62051216,260.52530117)(405.62051216,260.51837205)
\lineto(405.62051216,237.01164799)
\curveto(405.62051216,236.99721233)(405.6087799,236.98550632)(405.59434424,236.98550632)
\lineto(328.10353003,236.98550632)
\curveto(328.08906811,236.98550632)(328.0773621,236.99721233)(328.0773621,237.01164799)
\closepath
}
}
{
\newrgbcolor{curcolor}{0 0 0}
\pscustom[linewidth=1.92650529,linecolor=curcolor]
{
\newpath
\moveto(328.0773621,260.51837205)
\lineto(328.0773621,260.51837205)
\curveto(328.0773621,260.53280771)(328.08906811,260.54451373)(328.10353003,260.54451373)
\lineto(405.59434424,260.54451373)
\curveto(405.60127336,260.54451373)(405.60796626,260.54175783)(405.61284813,260.5368497)
\curveto(405.61778251,260.53194157)(405.62051216,260.52530117)(405.62051216,260.51837205)
\lineto(405.62051216,237.01164799)
\curveto(405.62051216,236.99721233)(405.6087799,236.98550632)(405.59434424,236.98550632)
\lineto(328.10353003,236.98550632)
\curveto(328.08906811,236.98550632)(328.0773621,236.99721233)(328.0773621,237.01164799)
\closepath
}
}
{
\newrgbcolor{curcolor}{0.53725493 0.61176473 0.90196079}
\pscustom[linestyle=none,fillstyle=solid,fillcolor=curcolor]
{
\newpath
\moveto(76.01988564,303.03442768)
\lineto(76.01988564,303.03442768)
\curveto(76.01988564,303.04886335)(76.03159428,303.06056936)(76.04602994,303.06056936)
\lineto(114.92278586,303.06056936)
\curveto(114.92971761,303.06056936)(114.93637114,303.05781346)(114.94128452,303.05291846)
\curveto(114.94618214,303.04802346)(114.94894067,303.0413568)(114.94894067,303.03442768)
\lineto(114.94894067,279.52772988)
\curveto(114.94894067,279.51326797)(114.93723203,279.50156195)(114.92278586,279.50156195)
\lineto(76.04602994,279.50156195)
\curveto(76.03159428,279.50156195)(76.01988564,279.51326797)(76.01988564,279.52772988)
\closepath
}
}
{
\newrgbcolor{curcolor}{0 0 0}
\pscustom[linewidth=1.92650529,linecolor=curcolor]
{
\newpath
\moveto(76.01988564,303.03442768)
\lineto(76.01988564,303.03442768)
\curveto(76.01988564,303.04886335)(76.03159428,303.06056936)(76.04602994,303.06056936)
\lineto(114.92278586,303.06056936)
\curveto(114.92971761,303.06056936)(114.93637114,303.05781346)(114.94128452,303.05291846)
\curveto(114.94618214,303.04802346)(114.94894067,303.0413568)(114.94894067,303.03442768)
\lineto(114.94894067,279.52772988)
\curveto(114.94894067,279.51326797)(114.93723203,279.50156195)(114.92278586,279.50156195)
\lineto(76.04602994,279.50156195)
\curveto(76.03159428,279.50156195)(76.01988564,279.51326797)(76.01988564,279.52772988)
\closepath
}
}
{
\newrgbcolor{curcolor}{0.53725493 0.61176473 0.90196079}
\pscustom[linestyle=none,fillstyle=solid,fillcolor=curcolor]
{
\newpath
\moveto(196.75707684,387.21625197)
\lineto(196.75707684,387.21625197)
\curveto(196.75707684,387.23070338)(196.76878286,387.24241202)(196.78321852,387.24241202)
\lineto(235.65999019,387.24241202)
\curveto(235.66691931,387.24241202)(235.67355972,387.23965349)(235.67846784,387.23475587)
\curveto(235.68337597,387.22984774)(235.68613187,387.22319683)(235.68613187,387.21625197)
\lineto(235.68613187,363.70955416)
\curveto(235.68613187,363.69511324)(235.67442586,363.68340461)(235.65999019,363.68340461)
\lineto(196.78321852,363.68340461)
\curveto(196.76878286,363.68340461)(196.75707684,363.69511324)(196.75707684,363.70955416)
\closepath
}
}
{
\newrgbcolor{curcolor}{0 0 0}
\pscustom[linewidth=1.92650529,linecolor=curcolor]
{
\newpath
\moveto(196.75707684,387.21625197)
\lineto(196.75707684,387.21625197)
\curveto(196.75707684,387.23070338)(196.76878286,387.24241202)(196.78321852,387.24241202)
\lineto(235.65999019,387.24241202)
\curveto(235.66691931,387.24241202)(235.67355972,387.23965349)(235.67846784,387.23475587)
\curveto(235.68337597,387.22984774)(235.68613187,387.22319683)(235.68613187,387.21625197)
\lineto(235.68613187,363.70955416)
\curveto(235.68613187,363.69511324)(235.67442586,363.68340461)(235.65999019,363.68340461)
\lineto(196.78321852,363.68340461)
\curveto(196.76878286,363.68340461)(196.75707684,363.69511324)(196.75707684,363.70955416)
\closepath
}
}
{
\newrgbcolor{curcolor}{0.53725493 0.61176473 0.90196079}
\pscustom[linestyle=none,fillstyle=solid,fillcolor=curcolor]
{
\newpath
\moveto(115.22216321,303.03442768)
\lineto(115.22216321,303.03442768)
\curveto(115.22216321,303.04886335)(115.23386922,303.06056936)(115.24830489,303.06056936)
\lineto(154.12506344,303.06056936)
\curveto(154.13199256,303.06056936)(154.13865921,303.05781346)(154.14356733,303.05291846)
\curveto(154.14844921,303.04802346)(154.15120511,303.0413568)(154.15120511,303.03442768)
\lineto(154.15120511,279.52772988)
\curveto(154.15120511,279.51326797)(154.1394991,279.50156195)(154.12506344,279.50156195)
\lineto(115.24830489,279.50156195)
\curveto(115.23386922,279.50156195)(115.22216321,279.51326797)(115.22216321,279.52772988)
\closepath
}
}
{
\newrgbcolor{curcolor}{0 0 0}
\pscustom[linewidth=1.92650529,linecolor=curcolor]
{
\newpath
\moveto(115.22216321,303.03442768)
\lineto(115.22216321,303.03442768)
\curveto(115.22216321,303.04886335)(115.23386922,303.06056936)(115.24830489,303.06056936)
\lineto(154.12506344,303.06056936)
\curveto(154.13199256,303.06056936)(154.13865921,303.05781346)(154.14356733,303.05291846)
\curveto(154.14844921,303.04802346)(154.15120511,303.0413568)(154.15120511,303.03442768)
\lineto(154.15120511,279.52772988)
\curveto(154.15120511,279.51326797)(154.1394991,279.50156195)(154.12506344,279.50156195)
\lineto(115.24830489,279.50156195)
\curveto(115.23386922,279.50156195)(115.22216321,279.51326797)(115.22216321,279.52772988)
\closepath
}
}
{
\newrgbcolor{curcolor}{0.53725493 0.61176473 0.90196079}
\pscustom[linestyle=none,fillstyle=solid,fillcolor=curcolor]
{
\newpath
\moveto(277.24854114,303.03442768)
\lineto(277.24854114,303.03442768)
\curveto(277.24854114,303.04886335)(277.26024716,303.06056936)(277.27468282,303.06056936)
\lineto(316.15145449,303.06056936)
\curveto(316.15838361,303.06056936)(316.16502402,303.05781346)(316.1699059,303.05291846)
\curveto(316.17484027,303.04802346)(316.17756992,303.0413568)(316.17756992,303.03442768)
\lineto(316.17756992,279.52772988)
\curveto(316.17756992,279.51326797)(316.16589016,279.50156195)(316.15145449,279.50156195)
\lineto(277.27468282,279.50156195)
\curveto(277.26024716,279.50156195)(277.24854114,279.51326797)(277.24854114,279.52772988)
\closepath
}
}
{
\newrgbcolor{curcolor}{0 0 0}
\pscustom[linewidth=1.92650529,linecolor=curcolor]
{
\newpath
\moveto(277.24854114,303.03442768)
\lineto(277.24854114,303.03442768)
\curveto(277.24854114,303.04886335)(277.26024716,303.06056936)(277.27468282,303.06056936)
\lineto(316.15145449,303.06056936)
\curveto(316.15838361,303.06056936)(316.16502402,303.05781346)(316.1699059,303.05291846)
\curveto(316.17484027,303.04802346)(316.17756992,303.0413568)(316.17756992,303.03442768)
\lineto(316.17756992,279.52772988)
\curveto(316.17756992,279.51326797)(316.16589016,279.50156195)(316.15145449,279.50156195)
\lineto(277.27468282,279.50156195)
\curveto(277.26024716,279.50156195)(277.24854114,279.51326797)(277.24854114,279.52772988)
\closepath
}
}
{
\newrgbcolor{curcolor}{0.53725493 0.61176473 0.90196079}
\pscustom[linestyle=none,fillstyle=solid,fillcolor=curcolor]
{
\newpath
\moveto(316.45082396,303.03442768)
\lineto(316.45082396,303.03442768)
\curveto(316.45082396,303.04886335)(316.46252998,303.06056936)(316.47696564,303.06056936)
\lineto(355.35371106,303.06056936)
\curveto(355.36066643,303.06056936)(355.36728059,303.05781346)(355.37221496,303.05291846)
\curveto(355.37714934,303.04802346)(355.37987899,303.0413568)(355.37987899,303.03442768)
\lineto(355.37987899,279.52772988)
\curveto(355.37987899,279.51326797)(355.36814673,279.50156195)(355.35371106,279.50156195)
\lineto(316.47696564,279.50156195)
\curveto(316.46252998,279.50156195)(316.45082396,279.51326797)(316.45082396,279.52772988)
\closepath
}
}
{
\newrgbcolor{curcolor}{0 0 0}
\pscustom[linewidth=1.92650529,linecolor=curcolor]
{
\newpath
\moveto(316.45082396,303.03442768)
\lineto(316.45082396,303.03442768)
\curveto(316.45082396,303.04886335)(316.46252998,303.06056936)(316.47696564,303.06056936)
\lineto(355.35371106,303.06056936)
\curveto(355.36066643,303.06056936)(355.36728059,303.05781346)(355.37221496,303.05291846)
\curveto(355.37714934,303.04802346)(355.37987899,303.0413568)(355.37987899,303.03442768)
\lineto(355.37987899,279.52772988)
\curveto(355.37987899,279.51326797)(355.36814673,279.50156195)(355.35371106,279.50156195)
\lineto(316.47696564,279.50156195)
\curveto(316.46252998,279.50156195)(316.45082396,279.51326797)(316.45082396,279.52772988)
\closepath
}
}
{
\newrgbcolor{curcolor}{0.53725493 0.61176473 0.90196079}
\pscustom[linestyle=none,fillstyle=solid,fillcolor=curcolor]
{
\newpath
\moveto(177.37950453,344.57870051)
\lineto(177.37950453,344.57870051)
\curveto(177.37950453,344.59313618)(177.39121055,344.60484219)(177.40564621,344.60484219)
\lineto(216.28239163,344.60484219)
\curveto(216.289347,344.60484219)(216.29598741,344.60208629)(216.30089553,344.59716767)
\curveto(216.30579316,344.59227004)(216.30855956,344.58562963)(216.30855956,344.57870051)
\lineto(216.30855956,321.07197646)
\curveto(216.30855956,321.05754079)(216.29685355,321.04583478)(216.28239163,321.04583478)
\lineto(177.40564621,321.04583478)
\curveto(177.39121055,321.04583478)(177.37950453,321.05754079)(177.37950453,321.07197646)
\closepath
}
}
{
\newrgbcolor{curcolor}{0 0 0}
\pscustom[linewidth=1.92650529,linecolor=curcolor]
{
\newpath
\moveto(177.37950453,344.57870051)
\lineto(177.37950453,344.57870051)
\curveto(177.37950453,344.59313618)(177.39121055,344.60484219)(177.40564621,344.60484219)
\lineto(216.28239163,344.60484219)
\curveto(216.289347,344.60484219)(216.29598741,344.60208629)(216.30089553,344.59716767)
\curveto(216.30579316,344.59227004)(216.30855956,344.58562963)(216.30855956,344.57870051)
\lineto(216.30855956,321.07197646)
\curveto(216.30855956,321.05754079)(216.29685355,321.04583478)(216.28239163,321.04583478)
\lineto(177.40564621,321.04583478)
\curveto(177.39121055,321.04583478)(177.37950453,321.05754079)(177.37950453,321.07197646)
\closepath
}
}
{
\newrgbcolor{curcolor}{0.53725493 0.61176473 0.90196079}
\pscustom[linestyle=none,fillstyle=solid,fillcolor=curcolor]
{
\newpath
\moveto(216.73089466,344.57870051)
\lineto(216.73089466,344.57870051)
\curveto(216.73089466,344.59313618)(216.74260068,344.60484219)(216.75703634,344.60484219)
\lineto(255.63378176,344.60484219)
\curveto(255.64073713,344.60484219)(255.64737754,344.60208629)(255.65228566,344.59716767)
\curveto(255.65716754,344.59227004)(255.65994969,344.58562963)(255.65994969,344.57870051)
\lineto(255.65994969,321.07197646)
\curveto(255.65994969,321.05754079)(255.64824368,321.04583478)(255.63378176,321.04583478)
\lineto(216.75703634,321.04583478)
\curveto(216.74260068,321.04583478)(216.73089466,321.05754079)(216.73089466,321.07197646)
\closepath
}
}
{
\newrgbcolor{curcolor}{0 0 0}
\pscustom[linewidth=1.92650529,linecolor=curcolor]
{
\newpath
\moveto(216.73089466,344.57870051)
\lineto(216.73089466,344.57870051)
\curveto(216.73089466,344.59313618)(216.74260068,344.60484219)(216.75703634,344.60484219)
\lineto(255.63378176,344.60484219)
\curveto(255.64073713,344.60484219)(255.64737754,344.60208629)(255.65228566,344.59716767)
\curveto(255.65716754,344.59227004)(255.65994969,344.58562963)(255.65994969,344.57870051)
\lineto(255.65994969,321.07197646)
\curveto(255.65994969,321.05754079)(255.64824368,321.04583478)(255.63378176,321.04583478)
\lineto(216.75703634,321.04583478)
\curveto(216.74260068,321.04583478)(216.73089466,321.05754079)(216.73089466,321.07197646)
\closepath
}
}
{
\newrgbcolor{curcolor}{0 0 0}
\pscustom[linestyle=none,fillstyle=solid,fillcolor=curcolor,opacity=0]
{
\newpath
\moveto(95.99359847,291.52051399)
\lineto(64.56059126,260.21349077)
}
}
{
\newrgbcolor{curcolor}{0 0 0}
\pscustom[linewidth=1.92650529,linecolor=curcolor]
{
\newpath
\moveto(95.99359847,291.52051399)
\lineto(72.75045132,268.37053462)
}
}
{
\newrgbcolor{curcolor}{0 0 0}
\pscustom[linestyle=none,fillstyle=solid,fillcolor=curcolor]
{
\newpath
\moveto(70.50491781,270.62512323)
\lineto(66.55605441,262.20096706)
\lineto(74.99599795,266.11597226)
\closepath
}
}
{
\newrgbcolor{curcolor}{0 0 0}
\pscustom[linewidth=1.92650529,linecolor=curcolor]
{
\newpath
\moveto(70.50491781,270.62512323)
\lineto(66.55605441,262.20096706)
\lineto(74.99599795,266.11597226)
\closepath
}
}
{
\newrgbcolor{curcolor}{0 0 0}
\pscustom[linestyle=none,fillstyle=solid,fillcolor=curcolor,opacity=0]
{
\newpath
\moveto(134.00349788,291.8849227)
\lineto(168.14516628,260.1684515)
}
}
{
\newrgbcolor{curcolor}{0 0 0}
\pscustom[linewidth=1.92650529,linecolor=curcolor]
{
\newpath
\moveto(134.00349788,291.8849227)
\lineto(159.67645902,268.03560092)
}
}
{
\newrgbcolor{curcolor}{0 0 0}
\pscustom[linestyle=none,fillstyle=solid,fillcolor=curcolor]
{
\newpath
\moveto(157.51072587,265.70426706)
\lineto(166.08177151,262.08527176)
\lineto(161.84219741,270.36693478)
\closepath
}
}
{
\newrgbcolor{curcolor}{0 0 0}
\pscustom[linewidth=1.92650529,linecolor=curcolor]
{
\newpath
\moveto(157.51072587,265.70426706)
\lineto(166.08177151,262.08527176)
\lineto(161.84219741,270.36693478)
\closepath
}
}
{
\newrgbcolor{curcolor}{0 0 0}
\pscustom[linestyle=none,fillstyle=solid,fillcolor=curcolor,opacity=0]
{
\newpath
\moveto(297.07325166,291.8849227)
\lineto(266.23866843,260.57789949)
}
}
{
\newrgbcolor{curcolor}{0 0 0}
\pscustom[linewidth=1.92650529,linecolor=curcolor]
{
\newpath
\moveto(297.07325166,291.8849227)
\lineto(274.34975436,268.81326339)
}
}
{
\newrgbcolor{curcolor}{0 0 0}
\pscustom[linestyle=none,fillstyle=solid,fillcolor=curcolor]
{
\newpath
\moveto(272.08264609,271.046146)
\lineto(268.21493739,262.5844571)
\lineto(276.61686263,266.58035452)
\closepath
}
}
{
\newrgbcolor{curcolor}{0 0 0}
\pscustom[linewidth=1.92650529,linecolor=curcolor]
{
\newpath
\moveto(272.08264609,271.046146)
\lineto(268.21493739,262.5844571)
\lineto(276.61686263,266.58035452)
\closepath
}
}
{
\newrgbcolor{curcolor}{0 0 0}
\pscustom[linestyle=none,fillstyle=solid,fillcolor=curcolor,opacity=0]
{
\newpath
\moveto(335.08312219,291.8849227)
\lineto(367.71297738,260.1684515)
}
}
{
\newrgbcolor{curcolor}{0 0 0}
\pscustom[linewidth=1.92650529,linecolor=curcolor]
{
\newpath
\moveto(335.08312219,291.8849227)
\lineto(359.42433274,268.2250756)
}
}
{
\newrgbcolor{curcolor}{0 0 0}
\pscustom[linestyle=none,fillstyle=solid,fillcolor=curcolor]
{
\newpath
\moveto(357.20643697,265.94332169)
\lineto(365.69345391,262.13146589)
\lineto(361.64222851,270.50686888)
\closepath
}
}
{
\newrgbcolor{curcolor}{0 0 0}
\pscustom[linewidth=1.92650529,linecolor=curcolor]
{
\newpath
\moveto(357.20643697,265.94332169)
\lineto(365.69345391,262.13146589)
\lineto(361.64222851,270.50686888)
\closepath
}
}
{
\newrgbcolor{curcolor}{0 0 0}
\pscustom[linestyle=none,fillstyle=solid,fillcolor=curcolor,opacity=0]
{
\newpath
\moveto(196.7569981,332.94327788)
\lineto(114.93037378,303.65199861)
}
}
{
\newrgbcolor{curcolor}{0 0 0}
\pscustom[linewidth=1.92650529,linecolor=curcolor]
{
\newpath
\moveto(196.7569981,332.94327788)
\lineto(125.81315226,307.54767314)
}
}
{
\newrgbcolor{curcolor}{0 0 0}
\pscustom[linestyle=none,fillstyle=solid,fillcolor=curcolor]
{
\newpath
\moveto(124.74071349,310.54359883)
\lineto(117.58196158,304.60118567)
\lineto(126.88558579,304.55177369)
\closepath
}
}
{
\newrgbcolor{curcolor}{0 0 0}
\pscustom[linewidth=1.92650529,linecolor=curcolor]
{
\newpath
\moveto(124.74071349,310.54359883)
\lineto(117.58196158,304.60118567)
\lineto(126.88558579,304.55177369)
\closepath
}
}
{
\newrgbcolor{curcolor}{0 0 0}
\pscustom[linestyle=none,fillstyle=solid,fillcolor=curcolor,opacity=0]
{
\newpath
\moveto(234.9160022,333.79365673)
\lineto(316.45911791,303.68348149)
}
}
{
\newrgbcolor{curcolor}{0 0 0}
\pscustom[linewidth=1.92650529,linecolor=curcolor]
{
\newpath
\moveto(234.9160022,333.79365673)
\lineto(305.61570155,307.68743663)
}
}
{
\newrgbcolor{curcolor}{0 0 0}
\pscustom[linestyle=none,fillstyle=solid,fillcolor=curcolor]
{
\newpath
\moveto(304.51344656,304.70237706)
\lineto(313.81710226,304.65904382)
\lineto(306.71798278,310.67250933)
\closepath
}
}
{
\newrgbcolor{curcolor}{0 0 0}
\pscustom[linewidth=1.92650529,linecolor=curcolor]
{
\newpath
\moveto(304.51344656,304.70237706)
\lineto(313.81710226,304.65904382)
\lineto(306.71798278,310.67250933)
\closepath
}
}
{
\newrgbcolor{curcolor}{0 0 0}
\pscustom[linestyle=none,fillstyle=solid,fillcolor=curcolor,opacity=0]
{
\newpath
\moveto(215.9855156,375.94524067)
\lineto(216.17449159,344.63822795)
}
}
{
\newrgbcolor{curcolor}{0 0 0}
\pscustom[linewidth=1.92650529,linecolor=curcolor]
{
\newpath
\moveto(215.9855156,375.94524067)
\lineto(216.10472795,356.19704969)
}
}
{
\newrgbcolor{curcolor}{0 0 0}
\pscustom[linestyle=none,fillstyle=solid,fillcolor=curcolor]
{
\newpath
\moveto(212.9227134,356.17783713)
\lineto(216.1574995,347.45452146)
\lineto(219.28674251,356.216236)
\closepath
}
}
{
\newrgbcolor{curcolor}{0 0 0}
\pscustom[linewidth=1.92650529,linecolor=curcolor]
{
\newpath
\moveto(212.9227134,356.17783713)
\lineto(216.1574995,347.45452146)
\lineto(219.28674251,356.216236)
\closepath
}
}
{
\newrgbcolor{curcolor}{0.53725493 0.61176473 0.90196079}
\pscustom[linestyle=none,fillstyle=solid,fillcolor=curcolor]
{
\newpath
\moveto(494.57547638,260.63989411)
\lineto(494.57547638,260.63989411)
\curveto(494.57547638,260.65432978)(494.58720864,260.66603579)(494.60164431,260.66603579)
\lineto(572.09251101,260.66603579)
\curveto(572.09941389,260.66603579)(572.10608054,260.66325364)(572.11098867,260.65837176)
\curveto(572.11587055,260.65346364)(572.11862645,260.64682323)(572.11862645,260.63989411)
\lineto(572.11862645,237.35361581)
\curveto(572.11862645,237.33918014)(572.10694668,237.32750037)(572.09251101,237.32750037)
\lineto(494.60164431,237.32750037)
\curveto(494.58720864,237.32750037)(494.57547638,237.33918014)(494.57547638,237.35361581)
\closepath
}
}
{
\newrgbcolor{curcolor}{0 0 0}
\pscustom[linewidth=1.92650529,linecolor=curcolor]
{
\newpath
\moveto(494.57547638,260.63989411)
\lineto(494.57547638,260.63989411)
\curveto(494.57547638,260.65432978)(494.58720864,260.66603579)(494.60164431,260.66603579)
\lineto(572.09251101,260.66603579)
\curveto(572.09941389,260.66603579)(572.10608054,260.66325364)(572.11098867,260.65837176)
\curveto(572.11587055,260.65346364)(572.11862645,260.64682323)(572.11862645,260.63989411)
\lineto(572.11862645,237.35361581)
\curveto(572.11862645,237.33918014)(572.10694668,237.32750037)(572.09251101,237.32750037)
\lineto(494.60164431,237.32750037)
\curveto(494.58720864,237.32750037)(494.57547638,237.33918014)(494.57547638,237.35361581)
\closepath
}
}
{
\newrgbcolor{curcolor}{0.53725493 0.61176473 0.90196079}
\pscustom[linestyle=none,fillstyle=solid,fillcolor=curcolor]
{
\newpath
\moveto(596.08406348,260.63989411)
\lineto(596.08406348,260.63989411)
\curveto(596.08406348,260.65432978)(596.09576949,260.66603579)(596.11020516,260.66603579)
\lineto(673.60107186,260.66603579)
\curveto(673.60805348,260.66603579)(673.61466763,260.66325364)(673.61960201,260.65837176)
\curveto(673.62443139,260.65346364)(673.62721354,260.64682323)(673.62721354,260.63989411)
\lineto(673.62721354,237.35361581)
\curveto(673.62721354,237.33918014)(673.61550753,237.32750037)(673.60107186,237.32750037)
\lineto(596.11020516,237.32750037)
\curveto(596.09576949,237.32750037)(596.08406348,237.33918014)(596.08406348,237.35361581)
\closepath
}
}
{
\newrgbcolor{curcolor}{0 0 0}
\pscustom[linewidth=1.92650529,linecolor=curcolor]
{
\newpath
\moveto(596.08406348,260.63989411)
\lineto(596.08406348,260.63989411)
\curveto(596.08406348,260.65432978)(596.09576949,260.66603579)(596.11020516,260.66603579)
\lineto(673.60107186,260.66603579)
\curveto(673.60805348,260.66603579)(673.61466763,260.66325364)(673.61960201,260.65837176)
\curveto(673.62443139,260.65346364)(673.62721354,260.64682323)(673.62721354,260.63989411)
\lineto(673.62721354,237.35361581)
\curveto(673.62721354,237.33918014)(673.61550753,237.32750037)(673.60107186,237.32750037)
\lineto(596.11020516,237.32750037)
\curveto(596.09576949,237.32750037)(596.08406348,237.33918014)(596.08406348,237.35361581)
\closepath
}
}
{
\newrgbcolor{curcolor}{0.53725493 0.61176473 0.90196079}
\pscustom[linestyle=none,fillstyle=solid,fillcolor=curcolor]
{
\newpath
\moveto(695.80407152,260.63989411)
\lineto(695.80407152,260.63989411)
\curveto(695.80407152,260.65432978)(695.81583003,260.66603579)(695.83026569,260.66603579)
\lineto(773.32118489,260.66603579)
\curveto(773.32806152,260.66603579)(773.33470192,260.66325364)(773.33961005,260.65837176)
\curveto(773.34454442,260.65346364)(773.34722158,260.64682323)(773.34722158,260.63989411)
\lineto(773.34722158,237.35361581)
\curveto(773.34722158,237.33918014)(773.33562056,237.32750037)(773.32118489,237.32750037)
\lineto(695.83026569,237.32750037)
\curveto(695.81583003,237.32750037)(695.80407152,237.33918014)(695.80407152,237.35361581)
\closepath
}
}
{
\newrgbcolor{curcolor}{0 0 0}
\pscustom[linewidth=1.92650529,linecolor=curcolor]
{
\newpath
\moveto(695.80407152,260.63989411)
\lineto(695.80407152,260.63989411)
\curveto(695.80407152,260.65432978)(695.81583003,260.66603579)(695.83026569,260.66603579)
\lineto(773.32118489,260.66603579)
\curveto(773.32806152,260.66603579)(773.33470192,260.66325364)(773.33961005,260.65837176)
\curveto(773.34454442,260.65346364)(773.34722158,260.64682323)(773.34722158,260.63989411)
\lineto(773.34722158,237.35361581)
\curveto(773.34722158,237.33918014)(773.33562056,237.32750037)(773.32118489,237.32750037)
\lineto(695.83026569,237.32750037)
\curveto(695.81583003,237.32750037)(695.80407152,237.33918014)(695.80407152,237.35361581)
\closepath
}
}
{
\newrgbcolor{curcolor}{0.53725493 0.61176473 0.90196079}
\pscustom[linestyle=none,fillstyle=solid,fillcolor=curcolor]
{
\newpath
\moveto(795.67316062,260.63989411)
\lineto(795.67316062,260.63989411)
\curveto(795.67316062,260.65432978)(795.68489288,260.66603579)(795.69932855,260.66603579)
\lineto(873.19014276,260.66603579)
\curveto(873.19715062,260.66603579)(873.20376478,260.66325364)(873.20869915,260.65837176)
\curveto(873.21352854,260.65346364)(873.21631068,260.64682323)(873.21631068,260.63989411)
\lineto(873.21631068,237.35361581)
\curveto(873.21631068,237.33918014)(873.20457843,237.32750037)(873.19014276,237.32750037)
\lineto(795.69932855,237.32750037)
\curveto(795.68489288,237.32750037)(795.67316062,237.33918014)(795.67316062,237.35361581)
\closepath
}
}
{
\newrgbcolor{curcolor}{0 0 0}
\pscustom[linewidth=1.92650529,linecolor=curcolor]
{
\newpath
\moveto(795.67316062,260.63989411)
\lineto(795.67316062,260.63989411)
\curveto(795.67316062,260.65432978)(795.68489288,260.66603579)(795.69932855,260.66603579)
\lineto(873.19014276,260.66603579)
\curveto(873.19715062,260.66603579)(873.20376478,260.66325364)(873.20869915,260.65837176)
\curveto(873.21352854,260.65346364)(873.21631068,260.64682323)(873.21631068,260.63989411)
\lineto(873.21631068,237.35361581)
\curveto(873.21631068,237.33918014)(873.20457843,237.32750037)(873.19014276,237.32750037)
\lineto(795.69932855,237.32750037)
\curveto(795.68489288,237.32750037)(795.67316062,237.33918014)(795.67316062,237.35361581)
\closepath
}
}
{
\newrgbcolor{curcolor}{0.53725493 0.61176473 0.90196079}
\pscustom[linestyle=none,fillstyle=solid,fillcolor=curcolor]
{
\newpath
\moveto(543.6156133,302.67004522)
\lineto(543.6156133,302.67004522)
\curveto(543.6156133,302.68448089)(543.62731931,302.69616065)(543.64175498,302.69616065)
\lineto(582.51852665,302.69616065)
\curveto(582.52545577,302.69616065)(582.53212242,302.693431)(582.53697805,302.68852287)
\curveto(582.54191243,302.68361475)(582.54466832,302.67697434)(582.54466832,302.67004522)
\lineto(582.54466832,279.38377741)
\curveto(582.54466832,279.36934175)(582.53296231,279.35762524)(582.51852665,279.35762524)
\lineto(543.64175498,279.35762524)
\curveto(543.62731931,279.35762524)(543.6156133,279.36934175)(543.6156133,279.38377741)
\closepath
}
}
{
\newrgbcolor{curcolor}{0 0 0}
\pscustom[linewidth=1.92650529,linecolor=curcolor]
{
\newpath
\moveto(543.6156133,302.67004522)
\lineto(543.6156133,302.67004522)
\curveto(543.6156133,302.68448089)(543.62731931,302.69616065)(543.64175498,302.69616065)
\lineto(582.51852665,302.69616065)
\curveto(582.52545577,302.69616065)(582.53212242,302.693431)(582.53697805,302.68852287)
\curveto(582.54191243,302.68361475)(582.54466832,302.67697434)(582.54466832,302.67004522)
\lineto(582.54466832,279.38377741)
\curveto(582.54466832,279.36934175)(582.53296231,279.35762524)(582.51852665,279.35762524)
\lineto(543.64175498,279.35762524)
\curveto(543.62731931,279.35762524)(543.6156133,279.36934175)(543.6156133,279.38377741)
\closepath
}
}
{
\newrgbcolor{curcolor}{0.53725493 0.61176473 0.90196079}
\pscustom[linestyle=none,fillstyle=solid,fillcolor=curcolor]
{
\newpath
\moveto(664.35269164,385.758583)
\lineto(664.35269164,385.758583)
\curveto(664.35269164,385.77301867)(664.36445015,385.78471681)(664.37888581,385.78471681)
\lineto(703.25570997,385.78471681)
\curveto(703.2625866,385.78471681)(703.26922701,385.78196878)(703.27413513,385.77705803)
\curveto(703.27906951,385.7721604)(703.28174667,385.76551475)(703.28174667,385.758583)
\lineto(703.28174667,362.47232569)
\curveto(703.28174667,362.45789003)(703.27014564,362.44618139)(703.25570997,362.44618139)
\lineto(664.37888581,362.44618139)
\curveto(664.36445015,362.44618139)(664.35269164,362.45789003)(664.35269164,362.47232569)
\closepath
}
}
{
\newrgbcolor{curcolor}{0 0 0}
\pscustom[linewidth=1.92650529,linecolor=curcolor]
{
\newpath
\moveto(664.35269164,385.758583)
\lineto(664.35269164,385.758583)
\curveto(664.35269164,385.77301867)(664.36445015,385.78471681)(664.37888581,385.78471681)
\lineto(703.25570997,385.78471681)
\curveto(703.2625866,385.78471681)(703.26922701,385.78196878)(703.27413513,385.77705803)
\curveto(703.27906951,385.7721604)(703.28174667,385.76551475)(703.28174667,385.758583)
\lineto(703.28174667,362.47232569)
\curveto(703.28174667,362.45789003)(703.27014564,362.44618139)(703.25570997,362.44618139)
\lineto(664.37888581,362.44618139)
\curveto(664.36445015,362.44618139)(664.35269164,362.45789003)(664.35269164,362.47232569)
\closepath
}
}
{
\newrgbcolor{curcolor}{0.53725493 0.61176473 0.90196079}
\pscustom[linestyle=none,fillstyle=solid,fillcolor=curcolor]
{
\newpath
\moveto(582.96697718,302.67004522)
\lineto(582.96697718,302.67004522)
\curveto(582.96697718,302.68448089)(582.97870944,302.69616065)(582.99314511,302.69616065)
\lineto(621.86991678,302.69616065)
\curveto(621.87684589,302.69616065)(621.8834863,302.693431)(621.88836818,302.68852287)
\curveto(621.89327631,302.68361475)(621.89603221,302.67697434)(621.89603221,302.67004522)
\lineto(621.89603221,279.38377741)
\curveto(621.89603221,279.36934175)(621.88435244,279.35762524)(621.86991678,279.35762524)
\lineto(582.99314511,279.35762524)
\curveto(582.97870944,279.35762524)(582.96697718,279.36934175)(582.96697718,279.38377741)
\closepath
}
}
{
\newrgbcolor{curcolor}{0 0 0}
\pscustom[linewidth=1.92650529,linecolor=curcolor]
{
\newpath
\moveto(582.96697718,302.67004522)
\lineto(582.96697718,302.67004522)
\curveto(582.96697718,302.68448089)(582.97870944,302.69616065)(582.99314511,302.69616065)
\lineto(621.86991678,302.69616065)
\curveto(621.87684589,302.69616065)(621.8834863,302.693431)(621.88836818,302.68852287)
\curveto(621.89327631,302.68361475)(621.89603221,302.67697434)(621.89603221,302.67004522)
\lineto(621.89603221,279.38377741)
\curveto(621.89603221,279.36934175)(621.88435244,279.35762524)(621.86991678,279.35762524)
\lineto(582.99314511,279.35762524)
\curveto(582.97870944,279.35762524)(582.96697718,279.36934175)(582.96697718,279.38377741)
\closepath
}
}
{
\newrgbcolor{curcolor}{0.53725493 0.61176473 0.90196079}
\pscustom[linestyle=none,fillstyle=solid,fillcolor=curcolor]
{
\newpath
\moveto(744.84418219,302.67004522)
\lineto(744.84418219,302.67004522)
\curveto(744.84418219,302.68448089)(744.85594069,302.69616065)(744.87037636,302.69616065)
\lineto(783.74720052,302.69616065)
\curveto(783.75402465,302.69616065)(783.76069131,302.693431)(783.76562568,302.68852287)
\curveto(783.77056005,302.68361475)(783.77323721,302.67697434)(783.77323721,302.67004522)
\lineto(783.77323721,279.38377741)
\curveto(783.77323721,279.36934175)(783.76163619,279.35762524)(783.74720052,279.35762524)
\lineto(744.87037636,279.35762524)
\curveto(744.85594069,279.35762524)(744.84418219,279.36934175)(744.84418219,279.38377741)
\closepath
}
}
{
\newrgbcolor{curcolor}{0 0 0}
\pscustom[linewidth=1.92650529,linecolor=curcolor]
{
\newpath
\moveto(744.84418219,302.67004522)
\lineto(744.84418219,302.67004522)
\curveto(744.84418219,302.68448089)(744.85594069,302.69616065)(744.87037636,302.69616065)
\lineto(783.74720052,302.69616065)
\curveto(783.75402465,302.69616065)(783.76069131,302.693431)(783.76562568,302.68852287)
\curveto(783.77056005,302.68361475)(783.77323721,302.67697434)(783.77323721,302.67004522)
\lineto(783.77323721,279.38377741)
\curveto(783.77323721,279.36934175)(783.76163619,279.35762524)(783.74720052,279.35762524)
\lineto(744.87037636,279.35762524)
\curveto(744.85594069,279.35762524)(744.84418219,279.36934175)(744.84418219,279.38377741)
\closepath
}
}
{
\newrgbcolor{curcolor}{0.53725493 0.61176473 0.90196079}
\pscustom[linestyle=none,fillstyle=solid,fillcolor=curcolor]
{
\newpath
\moveto(784.19557232,302.67004522)
\lineto(784.19557232,302.67004522)
\curveto(784.19557232,302.68448089)(784.20730458,302.69616065)(784.22174024,302.69616065)
\lineto(823.0985644,302.69616065)
\curveto(823.10544103,302.69616065)(823.11208144,302.693431)(823.11698956,302.68852287)
\curveto(823.12192394,302.68361475)(823.12462734,302.67697434)(823.12462734,302.67004522)
\lineto(823.12462734,279.38377741)
\curveto(823.12462734,279.36934175)(823.11300007,279.35762524)(823.0985644,279.35762524)
\lineto(784.22174024,279.35762524)
\curveto(784.20730458,279.35762524)(784.19557232,279.36934175)(784.19557232,279.38377741)
\closepath
}
}
{
\newrgbcolor{curcolor}{0 0 0}
\pscustom[linewidth=1.92650529,linecolor=curcolor]
{
\newpath
\moveto(784.19557232,302.67004522)
\lineto(784.19557232,302.67004522)
\curveto(784.19557232,302.68448089)(784.20730458,302.69616065)(784.22174024,302.69616065)
\lineto(823.0985644,302.69616065)
\curveto(823.10544103,302.69616065)(823.11208144,302.693431)(823.11698956,302.68852287)
\curveto(823.12192394,302.68361475)(823.12462734,302.67697434)(823.12462734,302.67004522)
\lineto(823.12462734,279.38377741)
\curveto(823.12462734,279.36934175)(823.11300007,279.35762524)(823.0985644,279.35762524)
\lineto(784.22174024,279.35762524)
\curveto(784.20730458,279.35762524)(784.19557232,279.36934175)(784.19557232,279.38377741)
\closepath
}
}
{
\newrgbcolor{curcolor}{0.53725493 0.61176473 0.90196079}
\pscustom[linestyle=none,fillstyle=solid,fillcolor=curcolor]
{
\newpath
\moveto(645.12433163,343.6069177)
\lineto(645.12433163,343.6069177)
\curveto(645.12433163,343.62135337)(645.13601139,343.63303314)(645.15044706,343.63303314)
\lineto(684.02719248,343.63303314)
\curveto(684.03420034,343.63303314)(684.04084075,343.63027724)(684.04567013,343.62538486)
\curveto(684.05060451,343.62048723)(684.05338665,343.61384682)(684.05338665,343.6069177)
\lineto(684.05338665,320.3206394)
\curveto(684.05338665,320.30620373)(684.04162815,320.29449772)(684.02719248,320.29449772)
\lineto(645.15044706,320.29449772)
\curveto(645.13601139,320.29449772)(645.12433163,320.30620373)(645.12433163,320.3206394)
\closepath
}
}
{
\newrgbcolor{curcolor}{0 0 0}
\pscustom[linewidth=1.92650529,linecolor=curcolor]
{
\newpath
\moveto(645.12433163,343.6069177)
\lineto(645.12433163,343.6069177)
\curveto(645.12433163,343.62135337)(645.13601139,343.63303314)(645.15044706,343.63303314)
\lineto(684.02719248,343.63303314)
\curveto(684.03420034,343.63303314)(684.04084075,343.63027724)(684.04567013,343.62538486)
\curveto(684.05060451,343.62048723)(684.05338665,343.61384682)(684.05338665,343.6069177)
\lineto(684.05338665,320.3206394)
\curveto(684.05338665,320.30620373)(684.04162815,320.29449772)(684.02719248,320.29449772)
\lineto(645.15044706,320.29449772)
\curveto(645.13601139,320.29449772)(645.12433163,320.30620373)(645.12433163,320.3206394)
\closepath
}
}
{
\newrgbcolor{curcolor}{0.53725493 0.61176473 0.90196079}
\pscustom[linestyle=none,fillstyle=solid,fillcolor=curcolor]
{
\newpath
\moveto(684.32661445,343.6069177)
\lineto(684.32661445,343.6069177)
\curveto(684.32661445,343.62135337)(684.33821547,343.63303314)(684.35265114,343.63303314)
\lineto(723.2294753,343.63303314)
\curveto(723.23637817,343.63303314)(723.24301858,343.63027724)(723.24795295,343.62538486)
\curveto(723.25283483,343.62048723)(723.25566947,343.61384682)(723.25566947,343.6069177)
\lineto(723.25566947,320.3206394)
\curveto(723.25566947,320.30620373)(723.24391097,320.29449772)(723.2294753,320.29449772)
\lineto(684.35265114,320.29449772)
\curveto(684.33821547,320.29449772)(684.32661445,320.30620373)(684.32661445,320.3206394)
\closepath
}
}
{
\newrgbcolor{curcolor}{0 0 0}
\pscustom[linewidth=1.92650529,linecolor=curcolor]
{
\newpath
\moveto(684.32661445,343.6069177)
\lineto(684.32661445,343.6069177)
\curveto(684.32661445,343.62135337)(684.33821547,343.63303314)(684.35265114,343.63303314)
\lineto(723.2294753,343.63303314)
\curveto(723.23637817,343.63303314)(723.24301858,343.63027724)(723.24795295,343.62538486)
\curveto(723.25283483,343.62048723)(723.25566947,343.61384682)(723.25566947,343.6069177)
\lineto(723.25566947,320.3206394)
\curveto(723.25566947,320.30620373)(723.24391097,320.29449772)(723.2294753,320.29449772)
\lineto(684.35265114,320.29449772)
\curveto(684.33821547,320.29449772)(684.32661445,320.30620373)(684.32661445,320.3206394)
\closepath
}
}
{
\newrgbcolor{curcolor}{0 0 0}
\pscustom[linestyle=none,fillstyle=solid,fillcolor=curcolor,opacity=0]
{
\newpath
\moveto(563.58937863,291.39902867)
\lineto(532.15637141,260.53293895)
}
}
{
\newrgbcolor{curcolor}{0 0 0}
\pscustom[linewidth=1.92650529,linecolor=curcolor]
{
\newpath
\moveto(563.58937863,291.39902867)
\lineto(540.40391377,268.6317152)
}
}
{
\newrgbcolor{curcolor}{0 0 0}
\pscustom[linestyle=none,fillstyle=solid,fillcolor=curcolor]
{
\newpath
\moveto(538.17439072,270.90215679)
\lineto(534.1658949,262.50621579)
\lineto(542.63341056,266.3612736)
\closepath
}
}
{
\newrgbcolor{curcolor}{0 0 0}
\pscustom[linewidth=1.92650529,linecolor=curcolor]
{
\newpath
\moveto(538.17439072,270.90215679)
\lineto(534.1658949,262.50621579)
\lineto(542.63341056,266.3612736)
\closepath
}
}
{
\newrgbcolor{curcolor}{0 0 0}
\pscustom[linestyle=none,fillstyle=solid,fillcolor=curcolor,opacity=0]
{
\newpath
\moveto(601.5993279,291.76346888)
\lineto(635.74099105,260.45645616)
}
}
{
\newrgbcolor{curcolor}{0 0 0}
\pscustom[linewidth=1.92650529,linecolor=curcolor]
{
\newpath
\moveto(601.5993279,291.76346888)
\lineto(627.22153323,268.26859257)
}
}
{
\newrgbcolor{curcolor}{0 0 0}
\pscustom[linestyle=none,fillstyle=solid,fillcolor=curcolor]
{
\newpath
\moveto(625.07088142,265.92326924)
\lineto(633.66519473,262.35989063)
\lineto(629.37210631,270.61390541)
\closepath
}
}
{
\newrgbcolor{curcolor}{0 0 0}
\pscustom[linewidth=1.92650529,linecolor=curcolor]
{
\newpath
\moveto(625.07088142,265.92326924)
\lineto(633.66519473,262.35989063)
\lineto(629.37210631,270.61390541)
\closepath
}
}
{
\newrgbcolor{curcolor}{0 0 0}
\pscustom[linestyle=none,fillstyle=solid,fillcolor=curcolor,opacity=0]
{
\newpath
\moveto(764.81800001,291.76346888)
\lineto(733.98341678,260.80291215)
}
}
{
\newrgbcolor{curcolor}{0 0 0}
\pscustom[linewidth=1.92650529,linecolor=curcolor]
{
\newpath
\moveto(764.81800001,291.76346888)
\lineto(742.14019816,268.99300056)
}
}
{
\newrgbcolor{curcolor}{0 0 0}
\pscustom[linestyle=none,fillstyle=solid,fillcolor=curcolor]
{
\newpath
\moveto(739.88555706,271.2384947)
\lineto(735.97076184,262.79841992)
\lineto(744.39478677,266.74754578)
\closepath
}
}
{
\newrgbcolor{curcolor}{0 0 0}
\pscustom[linewidth=1.92650529,linecolor=curcolor]
{
\newpath
\moveto(739.88555706,271.2384947)
\lineto(735.97076184,262.79841992)
\lineto(744.39478677,266.74754578)
\closepath
}
}
{
\newrgbcolor{curcolor}{0 0 0}
\pscustom[linestyle=none,fillstyle=solid,fillcolor=curcolor,opacity=0]
{
\newpath
\moveto(802.82792304,291.76346888)
\lineto(835.45777822,260.45645616)
}
}
{
\newrgbcolor{curcolor}{0 0 0}
\pscustom[linewidth=1.92650529,linecolor=curcolor]
{
\newpath
\moveto(802.82792304,291.76346888)
\lineto(827.11698146,268.45909088)
}
}
{
\newrgbcolor{curcolor}{0 0 0}
\pscustom[linestyle=none,fillstyle=solid,fillcolor=curcolor]
{
\newpath
\moveto(824.91391506,266.16295379)
\lineto(833.42557761,262.40629473)
\lineto(829.31999538,270.75522796)
\closepath
}
}
{
\newrgbcolor{curcolor}{0 0 0}
\pscustom[linewidth=1.92650529,linecolor=curcolor]
{
\newpath
\moveto(824.91391506,266.16295379)
\lineto(833.42557761,262.40629473)
\lineto(829.31999538,270.75522796)
\closepath
}
}
{
\newrgbcolor{curcolor}{0 0 0}
\pscustom[linestyle=none,fillstyle=solid,fillcolor=curcolor,opacity=0]
{
\newpath
\moveto(664.50193018,332.09300401)
\lineto(581.03747978,301.51038878)
}
}
{
\newrgbcolor{curcolor}{0 0 0}
\pscustom[linewidth=1.92650529,linecolor=curcolor]
{
\newpath
\moveto(664.50193018,332.09300401)
\lineto(591.89084362,305.48723112)
}
}
{
\newrgbcolor{curcolor}{0 0 0}
\pscustom[linestyle=none,fillstyle=solid,fillcolor=curcolor]
{
\newpath
\moveto(590.79606894,308.47504659)
\lineto(583.68191012,302.47933694)
\lineto(592.98564456,302.49941564)
\closepath
}
}
{
\newrgbcolor{curcolor}{0 0 0}
\pscustom[linewidth=1.92650529,linecolor=curcolor]
{
\newpath
\moveto(590.79606894,308.47504659)
\lineto(583.68191012,302.47933694)
\lineto(592.98564456,302.49941564)
\closepath
}
}
{
\newrgbcolor{curcolor}{0 0 0}
\pscustom[linestyle=none,fillstyle=solid,fillcolor=curcolor,opacity=0]
{
\newpath
\moveto(702.51166949,332.94327788)
\lineto(784.05481145,302.48664664)
}
}
{
\newrgbcolor{curcolor}{0 0 0}
\pscustom[linewidth=1.92650529,linecolor=curcolor]
{
\newpath
\moveto(702.51166949,332.94327788)
\lineto(773.22643442,306.53107415)
}
}
{
\newrgbcolor{curcolor}{0 0 0}
\pscustom[linestyle=none,fillstyle=solid,fillcolor=curcolor]
{
\newpath
\moveto(772.11310335,303.55014842)
\lineto(781.41649658,303.47206459)
\lineto(774.33987049,309.51202612)
\closepath
}
}
{
\newrgbcolor{curcolor}{0 0 0}
\pscustom[linewidth=1.92650529,linecolor=curcolor]
{
\newpath
\moveto(772.11310335,303.55014842)
\lineto(781.41649658,303.47206459)
\lineto(774.33987049,309.51202612)
\closepath
}
}
{
\newrgbcolor{curcolor}{0 0 0}
\pscustom[linestyle=none,fillstyle=solid,fillcolor=curcolor,opacity=0]
{
\newpath
\moveto(683.58136661,374.60907015)
\lineto(683.77034261,343.74298042)
}
}
{
\newrgbcolor{curcolor}{0 0 0}
\pscustom[linewidth=1.92650529,linecolor=curcolor]
{
\newpath
\moveto(683.58136661,374.60907015)
\lineto(683.69955535,355.30178641)
}
}
{
\newrgbcolor{curcolor}{0 0 0}
\pscustom[linestyle=none,fillstyle=solid,fillcolor=curcolor]
{
\newpath
\moveto(680.5175408,355.28230088)
\lineto(683.75312479,346.55928968)
\lineto(686.88151741,355.32127718)
\closepath
}
}
{
\newrgbcolor{curcolor}{0 0 0}
\pscustom[linewidth=1.92650529,linecolor=curcolor]
{
\newpath
\moveto(680.5175408,355.28230088)
\lineto(683.75312479,346.55928968)
\lineto(686.88151741,355.32127718)
\closepath
}
}
{
\newrgbcolor{curcolor}{0 1 0}
\pscustom[linestyle=none,fillstyle=solid,fillcolor=curcolor]
{
\newpath
\moveto(603.3879857,254.68751771)
\lineto(603.3879857,254.68751771)
\curveto(603.3879857,254.70200587)(603.39977045,254.71373813)(603.41423236,254.71373813)
\lineto(680.90494159,254.71373813)
\curveto(680.9119232,254.71373813)(680.91853736,254.71098223)(680.92347173,254.70607411)
\curveto(680.92840611,254.70113973)(680.93118825,254.69447308)(680.93118825,254.68751771)
\lineto(680.93118825,231.52740713)
\curveto(680.93118825,231.51294522)(680.91937725,231.50118671)(680.90494159,231.50118671)
\lineto(603.41423236,231.50118671)
\curveto(603.39977045,231.50118671)(603.3879857,231.51294522)(603.3879857,231.52740713)
\closepath
}
}
{
\newrgbcolor{curcolor}{0 0 0}
\pscustom[linewidth=1.92650529,linecolor=curcolor]
{
\newpath
\moveto(603.3879857,254.68751771)
\lineto(603.3879857,254.68751771)
\curveto(603.3879857,254.70200587)(603.39977045,254.71373813)(603.41423236,254.71373813)
\lineto(680.90494159,254.71373813)
\curveto(680.9119232,254.71373813)(680.91853736,254.71098223)(680.92347173,254.70607411)
\curveto(680.92840611,254.70113973)(680.93118825,254.69447308)(680.93118825,254.68751771)
\lineto(680.93118825,231.52740713)
\curveto(680.93118825,231.51294522)(680.91937725,231.50118671)(680.90494159,231.50118671)
\lineto(603.41423236,231.50118671)
\curveto(603.39977045,231.50118671)(603.3879857,231.51294522)(603.3879857,231.52740713)
\closepath
}
}
{
\newrgbcolor{curcolor}{0.53725493 0.61176473 0.90196079}
\pscustom[linestyle=none,fillstyle=solid,fillcolor=curcolor]
{
\newpath
\moveto(25.04195475,35.30473362)
\lineto(25.04195475,35.30473362)
\curveto(25.04195475,35.31916928)(25.05365814,35.33090154)(25.06809381,35.33090154)
\lineto(102.55896576,35.33090154)
\curveto(102.5658975,35.33090154)(102.57255103,35.32814564)(102.57745128,35.32323752)
\curveto(102.58235154,35.31830314)(102.58511006,35.31166274)(102.58511006,35.30473362)
\lineto(102.58511006,12.01848156)
\curveto(102.58511006,12.00404589)(102.57340142,11.99236613)(102.55896576,11.99236613)
\lineto(25.06809381,11.99236613)
\curveto(25.05365814,11.99236613)(25.04195475,12.00404589)(25.04195475,12.01848156)
\closepath
}
}
{
\newrgbcolor{curcolor}{0 0 0}
\pscustom[linewidth=1.92650529,linecolor=curcolor]
{
\newpath
\moveto(25.04195475,35.30473362)
\lineto(25.04195475,35.30473362)
\curveto(25.04195475,35.31916928)(25.05365814,35.33090154)(25.06809381,35.33090154)
\lineto(102.55896576,35.33090154)
\curveto(102.5658975,35.33090154)(102.57255103,35.32814564)(102.57745128,35.32323752)
\curveto(102.58235154,35.31830314)(102.58511006,35.31166274)(102.58511006,35.30473362)
\lineto(102.58511006,12.01848156)
\curveto(102.58511006,12.00404589)(102.57340142,11.99236613)(102.55896576,11.99236613)
\lineto(25.06809381,11.99236613)
\curveto(25.05365814,11.99236613)(25.04195475,12.00404589)(25.04195475,12.01848156)
\closepath
}
}
{
\newrgbcolor{curcolor}{0.53725493 0.61176473 0.90196079}
\pscustom[linestyle=none,fillstyle=solid,fillcolor=curcolor]
{
\newpath
\moveto(126.55057859,35.30473362)
\lineto(126.55057859,35.30473362)
\curveto(126.55057859,35.31916928)(126.56228985,35.33090154)(126.57672552,35.33090154)
\lineto(204.06758697,35.33090154)
\curveto(204.07451609,35.33090154)(204.08118275,35.32814564)(204.08609087,35.32323752)
\curveto(204.09098325,35.31830314)(204.09372865,35.31166274)(204.09372865,35.30473362)
\lineto(204.09372865,12.01848156)
\curveto(204.09372865,12.00404589)(204.08202264,11.99236613)(204.06758697,11.99236613)
\lineto(126.57672552,11.99236613)
\curveto(126.56228985,11.99236613)(126.55057859,12.00404589)(126.55057859,12.01848156)
\closepath
}
}
{
\newrgbcolor{curcolor}{0 0 0}
\pscustom[linewidth=1.92650529,linecolor=curcolor]
{
\newpath
\moveto(126.55057859,35.30473362)
\lineto(126.55057859,35.30473362)
\curveto(126.55057859,35.31916928)(126.56228985,35.33090154)(126.57672552,35.33090154)
\lineto(204.06758697,35.33090154)
\curveto(204.07451609,35.33090154)(204.08118275,35.32814564)(204.08609087,35.32323752)
\curveto(204.09098325,35.31830314)(204.09372865,35.31166274)(204.09372865,35.30473362)
\lineto(204.09372865,12.01848156)
\curveto(204.09372865,12.00404589)(204.08202264,11.99236613)(204.06758697,11.99236613)
\lineto(126.57672552,11.99236613)
\curveto(126.56228985,11.99236613)(126.55057859,12.00404589)(126.55057859,12.01848156)
\closepath
}
}
{
\newrgbcolor{curcolor}{0.53725493 0.61176473 0.90196079}
\pscustom[linestyle=none,fillstyle=solid,fillcolor=curcolor]
{
\newpath
\moveto(226.27061288,35.30473362)
\lineto(226.27061288,35.30473362)
\curveto(226.27061288,35.31916928)(226.28231889,35.33090154)(226.29675456,35.33090154)
\lineto(303.78762126,35.33090154)
\curveto(303.79455038,35.33090154)(303.80121703,35.32814564)(303.80609891,35.32323752)
\curveto(303.81100704,35.31830314)(303.81376294,35.31166274)(303.81376294,35.30473362)
\lineto(303.81376294,12.01848156)
\curveto(303.81376294,12.00404589)(303.80205693,11.99236613)(303.78762126,11.99236613)
\lineto(226.29675456,11.99236613)
\curveto(226.28231889,11.99236613)(226.27061288,12.00404589)(226.27061288,12.01848156)
\closepath
}
}
{
\newrgbcolor{curcolor}{0 0 0}
\pscustom[linewidth=1.92650529,linecolor=curcolor]
{
\newpath
\moveto(226.27061288,35.30473362)
\lineto(226.27061288,35.30473362)
\curveto(226.27061288,35.31916928)(226.28231889,35.33090154)(226.29675456,35.33090154)
\lineto(303.78762126,35.33090154)
\curveto(303.79455038,35.33090154)(303.80121703,35.32814564)(303.80609891,35.32323752)
\curveto(303.81100704,35.31830314)(303.81376294,35.31166274)(303.81376294,35.30473362)
\lineto(303.81376294,12.01848156)
\curveto(303.81376294,12.00404589)(303.80205693,11.99236613)(303.78762126,11.99236613)
\lineto(226.29675456,11.99236613)
\curveto(226.28231889,11.99236613)(226.27061288,12.00404589)(226.27061288,12.01848156)
\closepath
}
}
{
\newrgbcolor{curcolor}{0.53725493 0.61176473 0.90196079}
\pscustom[linestyle=none,fillstyle=solid,fillcolor=curcolor]
{
\newpath
\moveto(326.13962324,35.30473362)
\lineto(326.13962324,35.30473362)
\curveto(326.13962324,35.31916928)(326.1513555,35.33090154)(326.16579117,35.33090154)
\lineto(403.65663162,35.33090154)
\curveto(403.66358699,35.33090154)(403.6702274,35.32814564)(403.67513552,35.32323752)
\curveto(403.6800174,35.31830314)(403.6827733,35.31166274)(403.6827733,35.30473362)
\lineto(403.6827733,12.01848156)
\curveto(403.6827733,12.00404589)(403.67106729,11.99236613)(403.65663162,11.99236613)
\lineto(326.16579117,11.99236613)
\curveto(326.1513555,11.99236613)(326.13962324,12.00404589)(326.13962324,12.01848156)
\closepath
}
}
{
\newrgbcolor{curcolor}{0 0 0}
\pscustom[linewidth=1.92650529,linecolor=curcolor]
{
\newpath
\moveto(326.13962324,35.30473362)
\lineto(326.13962324,35.30473362)
\curveto(326.13962324,35.31916928)(326.1513555,35.33090154)(326.16579117,35.33090154)
\lineto(403.65663162,35.33090154)
\curveto(403.66358699,35.33090154)(403.6702274,35.32814564)(403.67513552,35.32323752)
\curveto(403.6800174,35.31830314)(403.6827733,35.31166274)(403.6827733,35.30473362)
\lineto(403.6827733,12.01848156)
\curveto(403.6827733,12.00404589)(403.67106729,11.99236613)(403.65663162,11.99236613)
\lineto(326.16579117,11.99236613)
\curveto(326.1513555,11.99236613)(326.13962324,12.00404589)(326.13962324,12.01848156)
\closepath
}
}
{
\newrgbcolor{curcolor}{0.53725493 0.61176473 0.90196079}
\pscustom[linestyle=none,fillstyle=solid,fillcolor=curcolor]
{
\newpath
\moveto(74.23118323,77.4563018)
\lineto(74.23118323,77.4563018)
\curveto(74.23118323,77.47076372)(74.24293386,77.48254847)(74.25742202,77.48254847)
\lineto(113.13399159,77.48254847)
\curveto(113.14095483,77.48254847)(113.14762935,77.47979257)(113.15254798,77.47488444)
\curveto(113.15747185,77.46992382)(113.16023825,77.46328342)(113.16023825,77.4563018)
\lineto(113.16023825,54.07577172)
\curveto(113.16023825,54.06125731)(113.14849025,54.04952506)(113.13399159,54.04952506)
\lineto(74.25742202,54.04952506)
\curveto(74.24293386,54.04952506)(74.23118323,54.06125731)(74.23118323,54.07577172)
\closepath
}
}
{
\newrgbcolor{curcolor}{0 0 0}
\pscustom[linewidth=1.92650529,linecolor=curcolor]
{
\newpath
\moveto(74.23118323,77.4563018)
\lineto(74.23118323,77.4563018)
\curveto(74.23118323,77.47076372)(74.24293386,77.48254847)(74.25742202,77.48254847)
\lineto(113.13399159,77.48254847)
\curveto(113.14095483,77.48254847)(113.14762935,77.47979257)(113.15254798,77.47488444)
\curveto(113.15747185,77.46992382)(113.16023825,77.46328342)(113.16023825,77.4563018)
\lineto(113.16023825,54.07577172)
\curveto(113.16023825,54.06125731)(113.14849025,54.04952506)(113.13399159,54.04952506)
\lineto(74.25742202,54.04952506)
\curveto(74.24293386,54.04952506)(74.23118323,54.06125731)(74.23118323,54.07577172)
\closepath
}
}
{
\newrgbcolor{curcolor}{0.53725493 0.61176473 0.90196079}
\pscustom[linestyle=none,fillstyle=solid,fillcolor=curcolor]
{
\newpath
\moveto(194.9683823,161.03072574)
\lineto(194.9683823,161.03072574)
\curveto(194.9683823,161.0452139)(194.98012506,161.05697241)(194.99461322,161.05697241)
\lineto(233.87120116,161.05697241)
\curveto(233.87813028,161.05697241)(233.88482318,161.05419026)(233.88975755,161.04928213)
\curveto(233.89466568,161.04434776)(233.89744783,161.03770735)(233.89744783,161.03072574)
\lineto(233.89744783,137.65019566)
\curveto(233.89744783,137.63568125)(233.88568932,137.62394899)(233.87120116,137.62394899)
\lineto(194.99461322,137.62394899)
\curveto(194.98012506,137.62394899)(194.9683823,137.63568125)(194.9683823,137.65019566)
\closepath
}
}
{
\newrgbcolor{curcolor}{0 0 0}
\pscustom[linewidth=1.92650529,linecolor=curcolor]
{
\newpath
\moveto(194.9683823,161.03072574)
\lineto(194.9683823,161.03072574)
\curveto(194.9683823,161.0452139)(194.98012506,161.05697241)(194.99461322,161.05697241)
\lineto(233.87120116,161.05697241)
\curveto(233.87813028,161.05697241)(233.88482318,161.05419026)(233.88975755,161.04928213)
\curveto(233.89466568,161.04434776)(233.89744783,161.03770735)(233.89744783,161.03072574)
\lineto(233.89744783,137.65019566)
\curveto(233.89744783,137.63568125)(233.88568932,137.62394899)(233.87120116,137.62394899)
\lineto(194.99461322,137.62394899)
\curveto(194.98012506,137.62394899)(194.9683823,137.63568125)(194.9683823,137.65019566)
\closepath
}
}
{
\newrgbcolor{curcolor}{0.53725493 0.61176473 0.90196079}
\pscustom[linestyle=none,fillstyle=solid,fillcolor=curcolor]
{
\newpath
\moveto(113.43346605,77.4563018)
\lineto(113.43346605,77.4563018)
\curveto(113.43346605,77.47076372)(113.44521143,77.48254847)(113.45969959,77.48254847)
\lineto(152.33627441,77.48254847)
\curveto(152.34322977,77.48254847)(152.34989643,77.47979257)(152.3548308,77.47488444)
\curveto(152.35973893,77.46992382)(152.36252107,77.46328342)(152.36252107,77.4563018)
\lineto(152.36252107,54.07577172)
\curveto(152.36252107,54.06125731)(152.35076257,54.04952506)(152.33627441,54.04952506)
\lineto(113.45969959,54.04952506)
\curveto(113.44521143,54.04952506)(113.43346605,54.06125731)(113.43346605,54.07577172)
\closepath
}
}
{
\newrgbcolor{curcolor}{0 0 0}
\pscustom[linewidth=1.92650529,linecolor=curcolor]
{
\newpath
\moveto(113.43346605,77.4563018)
\lineto(113.43346605,77.4563018)
\curveto(113.43346605,77.47076372)(113.44521143,77.48254847)(113.45969959,77.48254847)
\lineto(152.33627441,77.48254847)
\curveto(152.34322977,77.48254847)(152.34989643,77.47979257)(152.3548308,77.47488444)
\curveto(152.35973893,77.46992382)(152.36252107,77.46328342)(152.36252107,77.4563018)
\lineto(152.36252107,54.07577172)
\curveto(152.36252107,54.06125731)(152.35076257,54.04952506)(152.33627441,54.04952506)
\lineto(113.45969959,54.04952506)
\curveto(113.44521143,54.04952506)(113.43346605,54.06125731)(113.43346605,54.07577172)
\closepath
}
}
{
\newrgbcolor{curcolor}{0.53725493 0.61176473 0.90196079}
\pscustom[linestyle=none,fillstyle=solid,fillcolor=curcolor]
{
\newpath
\moveto(275.31072355,77.4563018)
\lineto(275.31072355,77.4563018)
\curveto(275.31072355,77.47076372)(275.32245581,77.48254847)(275.33697021,77.48254847)
\lineto(314.21355815,77.48254847)
\curveto(314.22048727,77.48254847)(314.22715393,77.47979257)(314.2320883,77.47488444)
\curveto(314.23699643,77.46992382)(314.23980482,77.46328342)(314.23980482,77.4563018)
\lineto(314.23980482,54.07577172)
\curveto(314.23980482,54.06125731)(314.22802007,54.04952506)(314.21355815,54.04952506)
\lineto(275.33697021,54.04952506)
\curveto(275.32245581,54.04952506)(275.31072355,54.06125731)(275.31072355,54.07577172)
\closepath
}
}
{
\newrgbcolor{curcolor}{0 0 0}
\pscustom[linewidth=1.92650529,linecolor=curcolor]
{
\newpath
\moveto(275.31072355,77.4563018)
\lineto(275.31072355,77.4563018)
\curveto(275.31072355,77.47076372)(275.32245581,77.48254847)(275.33697021,77.48254847)
\lineto(314.21355815,77.48254847)
\curveto(314.22048727,77.48254847)(314.22715393,77.47979257)(314.2320883,77.47488444)
\curveto(314.23699643,77.46992382)(314.23980482,77.46328342)(314.23980482,77.4563018)
\lineto(314.23980482,54.07577172)
\curveto(314.23980482,54.06125731)(314.22802007,54.04952506)(314.21355815,54.04952506)
\lineto(275.33697021,54.04952506)
\curveto(275.32245581,54.04952506)(275.31072355,54.06125731)(275.31072355,54.07577172)
\closepath
}
}
{
\newrgbcolor{curcolor}{0.53725493 0.61176473 0.90196079}
\pscustom[linestyle=none,fillstyle=solid,fillcolor=curcolor]
{
\newpath
\moveto(314.66211368,77.4563018)
\lineto(314.66211368,77.4563018)
\curveto(314.66211368,77.47076372)(314.67387218,77.48254847)(314.68836034,77.48254847)
\lineto(353.56492204,77.48254847)
\curveto(353.57190365,77.48254847)(353.5785703,77.47979257)(353.58350468,77.47488444)
\curveto(353.58843905,77.46992382)(353.5911687,77.46328342)(353.5911687,77.4563018)
\lineto(353.5911687,54.07577172)
\curveto(353.5911687,54.06125731)(353.5794102,54.04952506)(353.56492204,54.04952506)
\lineto(314.68836034,54.04952506)
\curveto(314.67387218,54.04952506)(314.66211368,54.06125731)(314.66211368,54.07577172)
\closepath
}
}
{
\newrgbcolor{curcolor}{0 0 0}
\pscustom[linewidth=1.92650529,linecolor=curcolor]
{
\newpath
\moveto(314.66211368,77.4563018)
\lineto(314.66211368,77.4563018)
\curveto(314.66211368,77.47076372)(314.67387218,77.48254847)(314.68836034,77.48254847)
\lineto(353.56492204,77.48254847)
\curveto(353.57190365,77.48254847)(353.5785703,77.47979257)(353.58350468,77.47488444)
\curveto(353.58843905,77.46992382)(353.5911687,77.46328342)(353.5911687,77.4563018)
\lineto(353.5911687,54.07577172)
\curveto(353.5911687,54.06125731)(353.5794102,54.04952506)(353.56492204,54.04952506)
\lineto(314.68836034,54.04952506)
\curveto(314.67387218,54.04952506)(314.66211368,54.06125731)(314.66211368,54.07577172)
\closepath
}
}
{
\newrgbcolor{curcolor}{0.53725493 0.61176473 0.90196079}
\pscustom[linestyle=none,fillstyle=solid,fillcolor=curcolor]
{
\newpath
\moveto(175.59079425,118.75760924)
\lineto(175.59079425,118.75760924)
\curveto(175.59079425,118.77212365)(175.60255275,118.78385591)(175.61704091,118.78385591)
\lineto(214.49360261,118.78385591)
\curveto(214.50058422,118.78385591)(214.50724037,118.78110001)(214.512159,118.77619188)
\curveto(214.51708287,118.77125751)(214.51984927,118.76459086)(214.51984927,118.75760924)
\lineto(214.51984927,95.37707916)
\curveto(214.51984927,95.362591)(214.50810126,95.3508325)(214.49360261,95.3508325)
\lineto(175.61704091,95.3508325)
\curveto(175.60255275,95.3508325)(175.59079425,95.362591)(175.59079425,95.37707916)
\closepath
}
}
{
\newrgbcolor{curcolor}{0 0 0}
\pscustom[linewidth=1.92650529,linecolor=curcolor]
{
\newpath
\moveto(175.59079425,118.75760924)
\lineto(175.59079425,118.75760924)
\curveto(175.59079425,118.77212365)(175.60255275,118.78385591)(175.61704091,118.78385591)
\lineto(214.49360261,118.78385591)
\curveto(214.50058422,118.78385591)(214.50724037,118.78110001)(214.512159,118.77619188)
\curveto(214.51708287,118.77125751)(214.51984927,118.76459086)(214.51984927,118.75760924)
\lineto(214.51984927,95.37707916)
\curveto(214.51984927,95.362591)(214.50810126,95.3508325)(214.49360261,95.3508325)
\lineto(175.61704091,95.3508325)
\curveto(175.60255275,95.3508325)(175.59079425,95.362591)(175.59079425,95.37707916)
\closepath
}
}
{
\newrgbcolor{curcolor}{0.53725493 0.61176473 0.90196079}
\pscustom[linestyle=none,fillstyle=solid,fillcolor=curcolor]
{
\newpath
\moveto(214.79306132,118.75760924)
\lineto(214.79306132,118.75760924)
\curveto(214.79306132,118.77212365)(214.80480933,118.78385591)(214.81930798,118.78385591)
\lineto(253.69585918,118.78385591)
\curveto(253.70284079,118.78385591)(253.70950745,118.78110001)(253.71444182,118.77619188)
\curveto(253.71934995,118.77125751)(253.72210585,118.76459086)(253.72210585,118.75760924)
\lineto(253.72210585,95.37707916)
\curveto(253.72210585,95.362591)(253.71037359,95.3508325)(253.69585918,95.3508325)
\lineto(214.81930798,95.3508325)
\curveto(214.80480933,95.3508325)(214.79306132,95.362591)(214.79306132,95.37707916)
\closepath
}
}
{
\newrgbcolor{curcolor}{0 0 0}
\pscustom[linewidth=1.92650529,linecolor=curcolor]
{
\newpath
\moveto(214.79306132,118.75760924)
\lineto(214.79306132,118.75760924)
\curveto(214.79306132,118.77212365)(214.80480933,118.78385591)(214.81930798,118.78385591)
\lineto(253.69585918,118.78385591)
\curveto(253.70284079,118.78385591)(253.70950745,118.78110001)(253.71444182,118.77619188)
\curveto(253.71934995,118.77125751)(253.72210585,118.76459086)(253.72210585,118.75760924)
\lineto(253.72210585,95.37707916)
\curveto(253.72210585,95.362591)(253.71037359,95.3508325)(253.69585918,95.3508325)
\lineto(214.81930798,95.3508325)
\curveto(214.80480933,95.3508325)(214.79306132,95.362591)(214.79306132,95.37707916)
\closepath
}
}
{
\newrgbcolor{curcolor}{0 0 0}
\pscustom[linestyle=none,fillstyle=solid,fillcolor=curcolor,opacity=0]
{
\newpath
\moveto(94.05588586,66.18540599)
\lineto(62.74886265,35.06735877)
}
}
{
\newrgbcolor{curcolor}{0 0 0}
\pscustom[linewidth=1.92650529,linecolor=curcolor]
{
\newpath
\moveto(94.05588586,66.18540599)
\lineto(70.94704289,43.21605618)
}
}
{
\newrgbcolor{curcolor}{0 0 0}
\pscustom[linestyle=none,fillstyle=solid,fillcolor=curcolor]
{
\newpath
\moveto(68.70379284,45.47292825)
\lineto(64.74634679,37.05278782)
\lineto(73.19027982,40.95921036)
\closepath
}
}
{
\newrgbcolor{curcolor}{0 0 0}
\pscustom[linewidth=1.92650529,linecolor=curcolor]
{
\newpath
\moveto(68.70379284,45.47292825)
\lineto(64.74634679,37.05278782)
\lineto(73.19027982,40.95921036)
\closepath
}
}
{
\newrgbcolor{curcolor}{0 0 0}
\pscustom[linestyle=none,fillstyle=solid,fillcolor=curcolor,opacity=0]
{
\newpath
\moveto(132.06579052,66.67131052)
\lineto(170.49090956,34.10444733)
}
}
{
\newrgbcolor{curcolor}{0 0 0}
\pscustom[linewidth=1.92650529,linecolor=curcolor]
{
\newpath
\moveto(132.06579052,66.67131052)
\lineto(161.67292217,41.57800175)
}
}
{
\newrgbcolor{curcolor}{0 0 0}
\pscustom[linestyle=none,fillstyle=solid,fillcolor=curcolor]
{
\newpath
\moveto(159.61553788,39.15052635)
\lineto(168.34240997,35.92538853)
\lineto(163.73034582,44.0055034)
\closepath
}
}
{
\newrgbcolor{curcolor}{0 0 0}
\pscustom[linewidth=1.92650529,linecolor=curcolor]
{
\newpath
\moveto(159.61553788,39.15052635)
\lineto(168.34240997,35.92538853)
\lineto(163.73034582,44.0055034)
\closepath
}
}
{
\newrgbcolor{curcolor}{0 0 0}
\pscustom[linestyle=none,fillstyle=solid,fillcolor=curcolor,opacity=0]
{
\newpath
\moveto(295.28456761,66.67131052)
\lineto(264.44998438,35.5532633)
}
}
{
\newrgbcolor{curcolor}{0 0 0}
\pscustom[linewidth=1.92650529,linecolor=curcolor]
{
\newpath
\moveto(295.28456761,66.67131052)
\lineto(272.58597841,43.76403407)
}
}
{
\newrgbcolor{curcolor}{0 0 0}
\pscustom[linestyle=none,fillstyle=solid,fillcolor=curcolor]
{
\newpath
\moveto(270.32561553,46.00376707)
\lineto(266.43231633,37.55383667)
\lineto(274.84628879,41.52430108)
\closepath
}
}
{
\newrgbcolor{curcolor}{0 0 0}
\pscustom[linewidth=1.92650529,linecolor=curcolor]
{
\newpath
\moveto(270.32561553,46.00376707)
\lineto(266.43231633,37.55383667)
\lineto(274.84628879,41.52430108)
\closepath
}
}
{
\newrgbcolor{curcolor}{0 0 0}
\pscustom[linestyle=none,fillstyle=solid,fillcolor=curcolor,opacity=0]
{
\newpath
\moveto(333.2944119,66.67131052)
\lineto(365.92426709,35.20680731)
}
}
{
\newrgbcolor{curcolor}{0 0 0}
\pscustom[linewidth=1.92650529,linecolor=curcolor]
{
\newpath
\moveto(333.2944119,66.67131052)
\lineto(357.60360152,43.23033437)
}
}
{
\newrgbcolor{curcolor}{0 0 0}
\pscustom[linestyle=none,fillstyle=solid,fillcolor=curcolor]
{
\newpath
\moveto(355.3947871,40.93973533)
\lineto(363.89692211,37.16171148)
\lineto(359.81236345,45.52090716)
\closepath
}
}
{
\newrgbcolor{curcolor}{0 0 0}
\pscustom[linewidth=1.92650529,linecolor=curcolor]
{
\newpath
\moveto(355.3947871,40.93973533)
\lineto(363.89692211,37.16171148)
\lineto(359.81236345,45.52090716)
\closepath
}
}
{
\newrgbcolor{curcolor}{0 0 0}
\pscustom[linestyle=none,fillstyle=solid,fillcolor=curcolor,opacity=0]
{
\newpath
\moveto(194.9683823,107.12230472)
\lineto(113.58270459,76.53968949)
}
}
{
\newrgbcolor{curcolor}{0 0 0}
\pscustom[linewidth=1.92650529,linecolor=curcolor]
{
\newpath
\moveto(194.9683823,107.12230472)
\lineto(124.40301076,80.6056655)
}
}
{
\newrgbcolor{curcolor}{0 0 0}
\pscustom[linestyle=none,fillstyle=solid,fillcolor=curcolor]
{
\newpath
\moveto(123.28369282,83.58437338)
\lineto(116.21908508,77.53036989)
\lineto(125.52232871,77.62693138)
\closepath
}
}
{
\newrgbcolor{curcolor}{0 0 0}
\pscustom[linewidth=1.92650529,linecolor=curcolor]
{
\newpath
\moveto(123.28369282,83.58437338)
\lineto(116.21908508,77.53036989)
\lineto(125.52232871,77.62693138)
\closepath
}
}
{
\newrgbcolor{curcolor}{0 0 0}
\pscustom[linestyle=none,fillstyle=solid,fillcolor=curcolor,opacity=0]
{
\newpath
\moveto(232.97815835,107.97259171)
\lineto(315.59219053,76.5080885)
}
}
{
\newrgbcolor{curcolor}{0 0 0}
\pscustom[linewidth=1.92650529,linecolor=curcolor]
{
\newpath
\moveto(232.97815835,107.97259171)
\lineto(304.79008642,80.62222715)
}
}
{
\newrgbcolor{curcolor}{0 0 0}
\pscustom[linestyle=none,fillstyle=solid,fillcolor=curcolor]
{
\newpath
\moveto(303.65751654,77.64850614)
\lineto(312.96027985,77.51052742)
\lineto(305.92265631,83.59589567)
\closepath
}
}
{
\newrgbcolor{curcolor}{0 0 0}
\pscustom[linewidth=1.92650529,linecolor=curcolor]
{
\newpath
\moveto(303.65751654,77.64850614)
\lineto(312.96027985,77.51052742)
\lineto(305.92265631,83.59589567)
\closepath
}
}
{
\newrgbcolor{curcolor}{0 0 0}
\pscustom[linestyle=none,fillstyle=solid,fillcolor=curcolor,opacity=0]
{
\newpath
\moveto(214.19683155,149.88132575)
\lineto(214.38580755,118.76327852)
}
}
{
\newrgbcolor{curcolor}{0 0 0}
\pscustom[linewidth=1.92650529,linecolor=curcolor]
{
\newpath
\moveto(214.19683155,149.88132575)
\lineto(214.31562396,130.32210026)
}
}
{
\newrgbcolor{curcolor}{0 0 0}
\pscustom[linestyle=none,fillstyle=solid,fillcolor=curcolor]
{
\newpath
\moveto(211.13360941,130.30278272)
\lineto(214.36870785,121.57957203)
\lineto(217.49761227,130.34141781)
\closepath
}
}
{
\newrgbcolor{curcolor}{0 0 0}
\pscustom[linewidth=1.92650529,linecolor=curcolor]
{
\newpath
\moveto(211.13360941,130.30278272)
\lineto(214.36870785,121.57957203)
\lineto(217.49761227,130.34141781)
\closepath
}
}
{
\newrgbcolor{curcolor}{0 1 0}
\pscustom[linestyle=none,fillstyle=solid,fillcolor=curcolor]
{
\newpath
\moveto(133.85448506,29.35256719)
\lineto(133.85448506,29.35256719)
\curveto(133.85448506,29.36700286)(133.86618058,29.37868263)(133.88062936,29.37868263)
\lineto(211.37150919,29.37868263)
\curveto(211.37843831,29.37868263)(211.38507872,29.37592673)(211.38998684,29.37107109)
\curveto(211.39486872,29.36613672)(211.39765087,29.35949631)(211.39765087,29.35256719)
\lineto(211.39765087,6.06631513)
\curveto(211.39765087,6.05187947)(211.38594486,6.04014721)(211.37150919,6.04014721)
\lineto(133.88062936,6.04014721)
\curveto(133.86618058,6.04014721)(133.85448506,6.05187947)(133.85448506,6.06631513)
\closepath
}
}
{
\newrgbcolor{curcolor}{0 0 0}
\pscustom[linewidth=1.92650529,linecolor=curcolor]
{
\newpath
\moveto(133.85448506,29.35256719)
\lineto(133.85448506,29.35256719)
\curveto(133.85448506,29.36700286)(133.86618058,29.37868263)(133.88062936,29.37868263)
\lineto(211.37150919,29.37868263)
\curveto(211.37843831,29.37868263)(211.38507872,29.37592673)(211.38998684,29.37107109)
\curveto(211.39486872,29.36613672)(211.39765087,29.35949631)(211.39765087,29.35256719)
\lineto(211.39765087,6.06631513)
\curveto(211.39765087,6.05187947)(211.38594486,6.04014721)(211.37150919,6.04014721)
\lineto(133.88062936,6.04014721)
\curveto(133.86618058,6.04014721)(133.85448506,6.05187947)(133.85448506,6.06631513)
\closepath
}
}
{
\newrgbcolor{curcolor}{0 1 0}
\pscustom[linestyle=none,fillstyle=solid,fillcolor=curcolor]
{
\newpath
\moveto(81.53498733,71.50400415)
\lineto(81.53498733,71.50400415)
\curveto(81.53498733,71.5185448)(81.54673272,71.53025081)(81.561234,71.53025081)
\lineto(120.43780882,71.53025081)
\curveto(120.44475631,71.53025081)(120.45144134,71.52752116)(120.45636521,71.52258679)
\curveto(120.46128384,71.51767866)(120.46404236,71.51098576)(120.46404236,71.50400415)
\lineto(120.46404236,48.12347406)
\curveto(120.46404236,48.10901215)(120.45229698,48.0972274)(120.43780882,48.0972274)
\lineto(81.561234,48.0972274)
\curveto(81.54673272,48.0972274)(81.53498733,48.10901215)(81.53498733,48.12347406)
\closepath
}
}
{
\newrgbcolor{curcolor}{0 0 0}
\pscustom[linewidth=1.92650529,linecolor=curcolor]
{
\newpath
\moveto(81.53498733,71.50400415)
\lineto(81.53498733,71.50400415)
\curveto(81.53498733,71.5185448)(81.54673272,71.53025081)(81.561234,71.53025081)
\lineto(120.43780882,71.53025081)
\curveto(120.44475631,71.53025081)(120.45144134,71.52752116)(120.45636521,71.52258679)
\curveto(120.46128384,71.51767866)(120.46404236,71.51098576)(120.46404236,71.50400415)
\lineto(120.46404236,48.12347406)
\curveto(120.46404236,48.10901215)(120.45229698,48.0972274)(120.43780882,48.0972274)
\lineto(81.561234,48.0972274)
\curveto(81.54673272,48.0972274)(81.53498733,48.10901215)(81.53498733,48.12347406)
\closepath
}
}
{
\newrgbcolor{curcolor}{0 1 0}
\pscustom[linestyle=none,fillstyle=solid,fillcolor=curcolor]
{
\newpath
\moveto(120.73735677,71.50400415)
\lineto(120.73735677,71.50400415)
\curveto(120.73735677,71.5185448)(120.74910478,71.53025081)(120.76360343,71.53025081)
\lineto(159.64017038,71.53025081)
\curveto(159.64712574,71.53025081)(159.6537924,71.52752116)(159.65872677,71.52258679)
\curveto(159.6636349,71.51767866)(159.66641704,71.51098576)(159.66641704,71.50400415)
\lineto(159.66641704,48.12347406)
\curveto(159.66641704,48.10901215)(159.65465854,48.0972274)(159.64017038,48.0972274)
\lineto(120.76360343,48.0972274)
\curveto(120.74910478,48.0972274)(120.73735677,48.10901215)(120.73735677,48.12347406)
\closepath
}
}
{
\newrgbcolor{curcolor}{0 0 0}
\pscustom[linewidth=1.92650529,linecolor=curcolor]
{
\newpath
\moveto(120.73735677,71.50400415)
\lineto(120.73735677,71.50400415)
\curveto(120.73735677,71.5185448)(120.74910478,71.53025081)(120.76360343,71.53025081)
\lineto(159.64017038,71.53025081)
\curveto(159.64712574,71.53025081)(159.6537924,71.52752116)(159.65872677,71.52258679)
\curveto(159.6636349,71.51767866)(159.66641704,71.51098576)(159.66641704,71.50400415)
\lineto(159.66641704,48.12347406)
\curveto(159.66641704,48.10901215)(159.65465854,48.0972274)(159.64017038,48.0972274)
\lineto(120.76360343,48.0972274)
\curveto(120.74910478,48.0972274)(120.73735677,48.10901215)(120.73735677,48.12347406)
\closepath
}
}
{
\newrgbcolor{curcolor}{0 0 0}
\pscustom[linestyle=none,fillstyle=solid,fillcolor=curcolor,opacity=0]
{
\newpath
\moveto(101.21067715,60.3546304)
\lineto(70.06113393,28.89012718)
}
}
{
\newrgbcolor{curcolor}{0 0 0}
\pscustom[linewidth=1.92650529,linecolor=curcolor]
{
\newpath
\moveto(101.21067715,60.3546304)
\lineto(78.19338518,37.10459874)
}
}
{
\newrgbcolor{curcolor}{0 0 0}
\pscustom[linestyle=none,fillstyle=solid,fillcolor=curcolor]
{
\newpath
\moveto(75.93203018,39.34330812)
\lineto(72.04256299,30.89159295)
\lineto(80.45474018,34.86586311)
\closepath
}
}
{
\newrgbcolor{curcolor}{0 0 0}
\pscustom[linewidth=1.92650529,linecolor=curcolor]
{
\newpath
\moveto(75.93203018,39.34330812)
\lineto(72.04256299,30.89159295)
\lineto(80.45474018,34.86586311)
\closepath
}
}
{
\newrgbcolor{curcolor}{0 0 0}
\pscustom[linestyle=none,fillstyle=solid,fillcolor=curcolor,opacity=0]
{
\newpath
\moveto(142.05264694,59.01843888)
\lineto(160.09985449,29.25471962)
}
}
{
\newrgbcolor{curcolor}{0 0 0}
\pscustom[linewidth=1.92650529,linecolor=curcolor]
{
\newpath
\moveto(142.05264694,59.01843888)
\lineto(154.10669077,39.13871535)
}
}
{
\newrgbcolor{curcolor}{0 0 0}
\pscustom[linestyle=none,fillstyle=solid,fillcolor=curcolor]
{
\newpath
\moveto(151.3857514,37.48887618)
\lineto(158.63962122,31.66298246)
\lineto(156.82765639,40.78855453)
\closepath
}
}
{
\newrgbcolor{curcolor}{0 0 0}
\pscustom[linewidth=1.92650529,linecolor=curcolor]
{
\newpath
\moveto(151.3857514,37.48887618)
\lineto(158.63962122,31.66298246)
\lineto(156.82765639,40.78855453)
\closepath
}
}
{
\newrgbcolor{curcolor}{0 1 0}
\pscustom[linestyle=none,fillstyle=solid,fillcolor=curcolor]
{
\newpath
\moveto(182.89461148,112.80533783)
\lineto(182.89461148,112.80533783)
\curveto(182.89461148,112.81982599)(182.90635686,112.8315845)(182.9208319,112.8315845)
\lineto(221.79741984,112.8315845)
\curveto(221.8043752,112.8315845)(221.81104186,112.82880235)(221.81597623,112.82389423)
\curveto(221.82088436,112.8189861)(221.8236665,112.8122932)(221.8236665,112.80533783)
\lineto(221.8236665,89.42480775)
\curveto(221.8236665,89.41031959)(221.811908,89.39856108)(221.79741984,89.39856108)
\lineto(182.9208319,89.39856108)
\curveto(182.90635686,89.39856108)(182.89461148,89.41031959)(182.89461148,89.42480775)
\closepath
}
}
{
\newrgbcolor{curcolor}{0 0 0}
\pscustom[linewidth=1.92650529,linecolor=curcolor]
{
\newpath
\moveto(182.89461148,112.80533783)
\lineto(182.89461148,112.80533783)
\curveto(182.89461148,112.81982599)(182.90635686,112.8315845)(182.9208319,112.8315845)
\lineto(221.79741984,112.8315845)
\curveto(221.8043752,112.8315845)(221.81104186,112.82880235)(221.81597623,112.82389423)
\curveto(221.82088436,112.8189861)(221.8236665,112.8122932)(221.8236665,112.80533783)
\lineto(221.8236665,89.42480775)
\curveto(221.8236665,89.41031959)(221.811908,89.39856108)(221.79741984,89.39856108)
\lineto(182.9208319,89.39856108)
\curveto(182.90635686,89.39856108)(182.89461148,89.41031959)(182.89461148,89.42480775)
\closepath
}
}
{
\newrgbcolor{curcolor}{0 1 0}
\pscustom[linestyle=none,fillstyle=solid,fillcolor=curcolor]
{
\newpath
\moveto(222.24598848,112.80533783)
\lineto(222.24598848,112.80533783)
\curveto(222.24598848,112.81982599)(222.25773387,112.8315845)(222.27223515,112.8315845)
\lineto(261.14880997,112.8315845)
\curveto(261.15573909,112.8315845)(261.16243199,112.82880235)(261.16736636,112.82389423)
\curveto(261.17227449,112.8189861)(261.17505663,112.8122932)(261.17505663,112.80533783)
\lineto(261.17505663,89.42480775)
\curveto(261.17505663,89.41031959)(261.16329813,89.39856108)(261.14880997,89.39856108)
\lineto(222.27223515,89.39856108)
\curveto(222.25773387,89.39856108)(222.24598848,89.41031959)(222.24598848,89.42480775)
\closepath
}
}
{
\newrgbcolor{curcolor}{0 0 0}
\pscustom[linewidth=1.92650529,linecolor=curcolor]
{
\newpath
\moveto(222.24598848,112.80533783)
\lineto(222.24598848,112.80533783)
\curveto(222.24598848,112.81982599)(222.25773387,112.8315845)(222.27223515,112.8315845)
\lineto(261.14880997,112.8315845)
\curveto(261.15573909,112.8315845)(261.16243199,112.82880235)(261.16736636,112.82389423)
\curveto(261.17227449,112.8189861)(261.17505663,112.8122932)(261.17505663,112.80533783)
\lineto(261.17505663,89.42480775)
\curveto(261.17505663,89.41031959)(261.16329813,89.39856108)(261.14880997,89.39856108)
\lineto(222.27223515,89.39856108)
\curveto(222.25773387,89.39856108)(222.24598848,89.41031959)(222.24598848,89.42480775)
\closepath
}
}
{
\newrgbcolor{curcolor}{0 0 0}
\pscustom[linestyle=none,fillstyle=solid,fillcolor=curcolor,opacity=0]
{
\newpath
\moveto(202.57029342,100.92709417)
\lineto(119.67281824,71.03739092)
}
}
{
\newrgbcolor{curcolor}{0 0 0}
\pscustom[linewidth=1.92650529,linecolor=curcolor]
{
\newpath
\moveto(202.57029342,100.92709417)
\lineto(130.54661512,74.95806539)
}
}
{
\newrgbcolor{curcolor}{0 0 0}
\pscustom[linestyle=none,fillstyle=solid,fillcolor=curcolor]
{
\newpath
\moveto(129.46729185,77.95149765)
\lineto(122.32222232,71.99263833)
\lineto(131.62593839,71.96463313)
\closepath
}
}
{
\newrgbcolor{curcolor}{0 0 0}
\pscustom[linewidth=1.92650529,linecolor=curcolor]
{
\newpath
\moveto(129.46729185,77.95149765)
\lineto(122.32222232,71.99263833)
\lineto(131.62593839,71.96463313)
\closepath
}
}
{
\newrgbcolor{curcolor}{0 0 0}
\pscustom[linestyle=none,fillstyle=solid,fillcolor=curcolor,opacity=0]
{
\newpath
\moveto(241.62354767,100.92709417)
\lineto(300.4895702,76.51769478)
}
}
{
\newrgbcolor{curcolor}{0 0 0}
\pscustom[linewidth=1.92650529,linecolor=curcolor]
{
\newpath
\moveto(241.62354767,100.92709417)
\lineto(289.8121115,80.94521862)
}
}
{
\newrgbcolor{curcolor}{0 0 0}
\pscustom[linestyle=none,fillstyle=solid,fillcolor=curcolor]
{
\newpath
\moveto(288.59324258,78.0058545)
\lineto(297.88800066,77.59648525)
\lineto(291.03095418,83.884609)
\closepath
}
}
{
\newrgbcolor{curcolor}{0 0 0}
\pscustom[linewidth=1.92650529,linecolor=curcolor]
{
\newpath
\moveto(288.59324258,78.0058545)
\lineto(297.88800066,77.59648525)
\lineto(291.03095418,83.884609)
\closepath
}
}
{
\newrgbcolor{curcolor}{0 1 0}
\pscustom[linestyle=none,fillstyle=solid,fillcolor=curcolor]
{
\newpath
\moveto(494.57547638,35.42628193)
\lineto(494.57547638,35.42628193)
\curveto(494.57547638,35.44071759)(494.58720864,35.45242361)(494.60164431,35.45242361)
\lineto(572.09251101,35.45242361)
\curveto(572.09941389,35.45242361)(572.10608054,35.44966771)(572.11098867,35.44473333)
\curveto(572.11587055,35.4398777)(572.11862645,35.43321105)(572.11862645,35.42628193)
\lineto(572.11862645,12.14002987)
\curveto(572.11862645,12.1255942)(572.10694668,12.11388819)(572.09251101,12.11388819)
\lineto(494.60164431,12.11388819)
\curveto(494.58720864,12.11388819)(494.57547638,12.1255942)(494.57547638,12.14002987)
\closepath
}
}
{
\newrgbcolor{curcolor}{0 0 0}
\pscustom[linewidth=1.92650529,linecolor=curcolor]
{
\newpath
\moveto(494.57547638,35.42628193)
\lineto(494.57547638,35.42628193)
\curveto(494.57547638,35.44071759)(494.58720864,35.45242361)(494.60164431,35.45242361)
\lineto(572.09251101,35.45242361)
\curveto(572.09941389,35.45242361)(572.10608054,35.44966771)(572.11098867,35.44473333)
\curveto(572.11587055,35.4398777)(572.11862645,35.43321105)(572.11862645,35.42628193)
\lineto(572.11862645,12.14002987)
\curveto(572.11862645,12.1255942)(572.10694668,12.11388819)(572.09251101,12.11388819)
\lineto(494.60164431,12.11388819)
\curveto(494.58720864,12.11388819)(494.57547638,12.1255942)(494.57547638,12.14002987)
\closepath
}
}
{
\newrgbcolor{curcolor}{0 1 0}
\pscustom[linestyle=none,fillstyle=solid,fillcolor=curcolor]
{
\newpath
\moveto(596.08406348,35.42628193)
\lineto(596.08406348,35.42628193)
\curveto(596.08406348,35.44071759)(596.09576949,35.45242361)(596.11020516,35.45242361)
\lineto(673.60107186,35.45242361)
\curveto(673.60805348,35.45242361)(673.61466763,35.44966771)(673.61960201,35.44473333)
\curveto(673.62443139,35.4398777)(673.62721354,35.43321105)(673.62721354,35.42628193)
\lineto(673.62721354,12.14002987)
\curveto(673.62721354,12.1255942)(673.61550753,12.11388819)(673.60107186,12.11388819)
\lineto(596.11020516,12.11388819)
\curveto(596.09576949,12.11388819)(596.08406348,12.1255942)(596.08406348,12.14002987)
\closepath
}
}
{
\newrgbcolor{curcolor}{0 0 0}
\pscustom[linewidth=1.92650529,linecolor=curcolor]
{
\newpath
\moveto(596.08406348,35.42628193)
\lineto(596.08406348,35.42628193)
\curveto(596.08406348,35.44071759)(596.09576949,35.45242361)(596.11020516,35.45242361)
\lineto(673.60107186,35.45242361)
\curveto(673.60805348,35.45242361)(673.61466763,35.44966771)(673.61960201,35.44473333)
\curveto(673.62443139,35.4398777)(673.62721354,35.43321105)(673.62721354,35.42628193)
\lineto(673.62721354,12.14002987)
\curveto(673.62721354,12.1255942)(673.61550753,12.11388819)(673.60107186,12.11388819)
\lineto(596.11020516,12.11388819)
\curveto(596.09576949,12.11388819)(596.08406348,12.1255942)(596.08406348,12.14002987)
\closepath
}
}
{
\newrgbcolor{curcolor}{0.53725493 0.61176473 0.90196079}
\pscustom[linestyle=none,fillstyle=solid,fillcolor=curcolor]
{
\newpath
\moveto(695.80407152,35.42628193)
\lineto(695.80407152,35.42628193)
\curveto(695.80407152,35.44071759)(695.81583003,35.45242361)(695.83026569,35.45242361)
\lineto(773.32118489,35.45242361)
\curveto(773.32806152,35.45242361)(773.33470192,35.44966771)(773.33961005,35.44473333)
\curveto(773.34454442,35.4398777)(773.34722158,35.43321105)(773.34722158,35.42628193)
\lineto(773.34722158,12.14002987)
\curveto(773.34722158,12.1255942)(773.33562056,12.11388819)(773.32118489,12.11388819)
\lineto(695.83026569,12.11388819)
\curveto(695.81583003,12.11388819)(695.80407152,12.1255942)(695.80407152,12.14002987)
\closepath
}
}
{
\newrgbcolor{curcolor}{0 0 0}
\pscustom[linewidth=1.92650529,linecolor=curcolor]
{
\newpath
\moveto(695.80407152,35.42628193)
\lineto(695.80407152,35.42628193)
\curveto(695.80407152,35.44071759)(695.81583003,35.45242361)(695.83026569,35.45242361)
\lineto(773.32118489,35.45242361)
\curveto(773.32806152,35.45242361)(773.33470192,35.44966771)(773.33961005,35.44473333)
\curveto(773.34454442,35.4398777)(773.34722158,35.43321105)(773.34722158,35.42628193)
\lineto(773.34722158,12.14002987)
\curveto(773.34722158,12.1255942)(773.33562056,12.11388819)(773.32118489,12.11388819)
\lineto(695.83026569,12.11388819)
\curveto(695.81583003,12.11388819)(695.80407152,12.1255942)(695.80407152,12.14002987)
\closepath
}
}
{
\newrgbcolor{curcolor}{0.53725493 0.61176473 0.90196079}
\pscustom[linestyle=none,fillstyle=solid,fillcolor=curcolor]
{
\newpath
\moveto(795.67316062,35.42628193)
\lineto(795.67316062,35.42628193)
\curveto(795.67316062,35.44071759)(795.68489288,35.45242361)(795.69932855,35.45242361)
\lineto(873.19014276,35.45242361)
\curveto(873.19715062,35.45242361)(873.20376478,35.44966771)(873.20869915,35.44473333)
\curveto(873.21352854,35.4398777)(873.21631068,35.43321105)(873.21631068,35.42628193)
\lineto(873.21631068,12.14002987)
\curveto(873.21631068,12.1255942)(873.20457843,12.11388819)(873.19014276,12.11388819)
\lineto(795.69932855,12.11388819)
\curveto(795.68489288,12.11388819)(795.67316062,12.1255942)(795.67316062,12.14002987)
\closepath
}
}
{
\newrgbcolor{curcolor}{0 0 0}
\pscustom[linewidth=1.92650529,linecolor=curcolor]
{
\newpath
\moveto(795.67316062,35.42628193)
\lineto(795.67316062,35.42628193)
\curveto(795.67316062,35.44071759)(795.68489288,35.45242361)(795.69932855,35.45242361)
\lineto(873.19014276,35.45242361)
\curveto(873.19715062,35.45242361)(873.20376478,35.44966771)(873.20869915,35.44473333)
\curveto(873.21352854,35.4398777)(873.21631068,35.43321105)(873.21631068,35.42628193)
\lineto(873.21631068,12.14002987)
\curveto(873.21631068,12.1255942)(873.20457843,12.11388819)(873.19014276,12.11388819)
\lineto(795.69932855,12.11388819)
\curveto(795.68489288,12.11388819)(795.67316062,12.1255942)(795.67316062,12.14002987)
\closepath
}
}
{
\newrgbcolor{curcolor}{0 1 0}
\pscustom[linestyle=none,fillstyle=solid,fillcolor=curcolor]
{
\newpath
\moveto(543.6156133,77.57785011)
\lineto(543.6156133,77.57785011)
\curveto(543.6156133,77.59228578)(543.62731931,77.60399179)(543.64175498,77.60399179)
\lineto(582.51852665,77.60399179)
\curveto(582.52545577,77.60399179)(582.53212242,77.60123589)(582.53697805,77.59635401)
\curveto(582.54191243,77.59144589)(582.54466832,77.58477923)(582.54466832,77.57785011)
\lineto(582.54466832,54.29159806)
\curveto(582.54466832,54.27716239)(582.53296231,54.26545638)(582.51852665,54.26545638)
\lineto(543.64175498,54.26545638)
\curveto(543.62731931,54.26545638)(543.6156133,54.27716239)(543.6156133,54.29159806)
\closepath
}
}
{
\newrgbcolor{curcolor}{0 0 0}
\pscustom[linewidth=1.92650529,linecolor=curcolor]
{
\newpath
\moveto(543.6156133,77.57785011)
\lineto(543.6156133,77.57785011)
\curveto(543.6156133,77.59228578)(543.62731931,77.60399179)(543.64175498,77.60399179)
\lineto(582.51852665,77.60399179)
\curveto(582.52545577,77.60399179)(582.53212242,77.60123589)(582.53697805,77.59635401)
\curveto(582.54191243,77.59144589)(582.54466832,77.58477923)(582.54466832,77.57785011)
\lineto(582.54466832,54.29159806)
\curveto(582.54466832,54.27716239)(582.53296231,54.26545638)(582.51852665,54.26545638)
\lineto(543.64175498,54.26545638)
\curveto(543.62731931,54.26545638)(543.6156133,54.27716239)(543.6156133,54.29159806)
\closepath
}
}
{
\newrgbcolor{curcolor}{0 1 0}
\pscustom[linestyle=none,fillstyle=solid,fillcolor=curcolor]
{
\newpath
\moveto(664.35269164,161.03080448)
\lineto(664.35269164,161.03080448)
\curveto(664.35269164,161.04524015)(664.36445015,161.05697241)(664.37888581,161.05697241)
\lineto(703.25570997,161.05697241)
\curveto(703.2625866,161.05697241)(703.26922701,161.05424275)(703.27413513,161.04930838)
\curveto(703.27906951,161.0444265)(703.28174667,161.0377336)(703.28174667,161.03080448)
\lineto(703.28174667,137.74457867)
\curveto(703.28174667,137.730143)(703.27014564,137.71843699)(703.25570997,137.71843699)
\lineto(664.37888581,137.71843699)
\curveto(664.36445015,137.71843699)(664.35269164,137.730143)(664.35269164,137.74457867)
\closepath
}
}
{
\newrgbcolor{curcolor}{0 0 0}
\pscustom[linewidth=1.92650529,linecolor=curcolor]
{
\newpath
\moveto(664.35269164,161.03080448)
\lineto(664.35269164,161.03080448)
\curveto(664.35269164,161.04524015)(664.36445015,161.05697241)(664.37888581,161.05697241)
\lineto(703.25570997,161.05697241)
\curveto(703.2625866,161.05697241)(703.26922701,161.05424275)(703.27413513,161.04930838)
\curveto(703.27906951,161.0444265)(703.28174667,161.0377336)(703.28174667,161.03080448)
\lineto(703.28174667,137.74457867)
\curveto(703.28174667,137.730143)(703.27014564,137.71843699)(703.25570997,137.71843699)
\lineto(664.37888581,137.71843699)
\curveto(664.36445015,137.71843699)(664.35269164,137.730143)(664.35269164,137.74457867)
\closepath
}
}
{
\newrgbcolor{curcolor}{0 1 0}
\pscustom[linestyle=none,fillstyle=solid,fillcolor=curcolor]
{
\newpath
\moveto(582.96697718,77.57785011)
\lineto(582.96697718,77.57785011)
\curveto(582.96697718,77.59228578)(582.97870944,77.60399179)(582.99314511,77.60399179)
\lineto(621.86991678,77.60399179)
\curveto(621.87684589,77.60399179)(621.8834863,77.60123589)(621.88836818,77.59635401)
\curveto(621.89327631,77.59144589)(621.89603221,77.58477923)(621.89603221,77.57785011)
\lineto(621.89603221,54.29159806)
\curveto(621.89603221,54.27716239)(621.88435244,54.26545638)(621.86991678,54.26545638)
\lineto(582.99314511,54.26545638)
\curveto(582.97870944,54.26545638)(582.96697718,54.27716239)(582.96697718,54.29159806)
\closepath
}
}
{
\newrgbcolor{curcolor}{0 0 0}
\pscustom[linewidth=1.92650529,linecolor=curcolor]
{
\newpath
\moveto(582.96697718,77.57785011)
\lineto(582.96697718,77.57785011)
\curveto(582.96697718,77.59228578)(582.97870944,77.60399179)(582.99314511,77.60399179)
\lineto(621.86991678,77.60399179)
\curveto(621.87684589,77.60399179)(621.8834863,77.60123589)(621.88836818,77.59635401)
\curveto(621.89327631,77.59144589)(621.89603221,77.58477923)(621.89603221,77.57785011)
\lineto(621.89603221,54.29159806)
\curveto(621.89603221,54.27716239)(621.88435244,54.26545638)(621.86991678,54.26545638)
\lineto(582.99314511,54.26545638)
\curveto(582.97870944,54.26545638)(582.96697718,54.27716239)(582.96697718,54.29159806)
\closepath
}
}
{
\newrgbcolor{curcolor}{0.53725493 0.61176473 0.90196079}
\pscustom[linestyle=none,fillstyle=solid,fillcolor=curcolor]
{
\newpath
\moveto(744.84418219,77.57785011)
\lineto(744.84418219,77.57785011)
\curveto(744.84418219,77.59228578)(744.85594069,77.60399179)(744.87037636,77.60399179)
\lineto(783.74720052,77.60399179)
\curveto(783.75402465,77.60399179)(783.76069131,77.60123589)(783.76562568,77.59635401)
\curveto(783.77056005,77.59144589)(783.77323721,77.58477923)(783.77323721,77.57785011)
\lineto(783.77323721,54.29159806)
\curveto(783.77323721,54.27716239)(783.76163619,54.26545638)(783.74720052,54.26545638)
\lineto(744.87037636,54.26545638)
\curveto(744.85594069,54.26545638)(744.84418219,54.27716239)(744.84418219,54.29159806)
\closepath
}
}
{
\newrgbcolor{curcolor}{0 0 0}
\pscustom[linewidth=1.92650529,linecolor=curcolor]
{
\newpath
\moveto(744.84418219,77.57785011)
\lineto(744.84418219,77.57785011)
\curveto(744.84418219,77.59228578)(744.85594069,77.60399179)(744.87037636,77.60399179)
\lineto(783.74720052,77.60399179)
\curveto(783.75402465,77.60399179)(783.76069131,77.60123589)(783.76562568,77.59635401)
\curveto(783.77056005,77.59144589)(783.77323721,77.58477923)(783.77323721,77.57785011)
\lineto(783.77323721,54.29159806)
\curveto(783.77323721,54.27716239)(783.76163619,54.26545638)(783.74720052,54.26545638)
\lineto(744.87037636,54.26545638)
\curveto(744.85594069,54.26545638)(744.84418219,54.27716239)(744.84418219,54.29159806)
\closepath
}
}
{
\newrgbcolor{curcolor}{0.53725493 0.61176473 0.90196079}
\pscustom[linestyle=none,fillstyle=solid,fillcolor=curcolor]
{
\newpath
\moveto(784.19557232,77.57785011)
\lineto(784.19557232,77.57785011)
\curveto(784.19557232,77.59228578)(784.20730458,77.60399179)(784.22174024,77.60399179)
\lineto(823.0985644,77.60399179)
\curveto(823.10544103,77.60399179)(823.11208144,77.60123589)(823.11698956,77.59635401)
\curveto(823.12192394,77.59144589)(823.12462734,77.58477923)(823.12462734,77.57785011)
\lineto(823.12462734,54.29159806)
\curveto(823.12462734,54.27716239)(823.11300007,54.26545638)(823.0985644,54.26545638)
\lineto(784.22174024,54.26545638)
\curveto(784.20730458,54.26545638)(784.19557232,54.27716239)(784.19557232,54.29159806)
\closepath
}
}
{
\newrgbcolor{curcolor}{0 0 0}
\pscustom[linewidth=1.92650529,linecolor=curcolor]
{
\newpath
\moveto(784.19557232,77.57785011)
\lineto(784.19557232,77.57785011)
\curveto(784.19557232,77.59228578)(784.20730458,77.60399179)(784.22174024,77.60399179)
\lineto(823.0985644,77.60399179)
\curveto(823.10544103,77.60399179)(823.11208144,77.60123589)(823.11698956,77.59635401)
\curveto(823.12192394,77.59144589)(823.12462734,77.58477923)(823.12462734,77.57785011)
\lineto(823.12462734,54.29159806)
\curveto(823.12462734,54.27716239)(823.11300007,54.26545638)(823.0985644,54.26545638)
\lineto(784.22174024,54.26545638)
\curveto(784.20730458,54.26545638)(784.19557232,54.27716239)(784.19557232,54.29159806)
\closepath
}
}
{
\newrgbcolor{curcolor}{0 1 0}
\pscustom[linestyle=none,fillstyle=solid,fillcolor=curcolor]
{
\newpath
\moveto(645.12433163,118.75774048)
\lineto(645.12433163,118.75774048)
\curveto(645.12433163,118.77217614)(645.13601139,118.78385591)(645.15044706,118.78385591)
\lineto(684.02719248,118.78385591)
\curveto(684.03420034,118.78385591)(684.04084075,118.78110001)(684.04567013,118.77621813)
\curveto(684.05060451,118.77131)(684.05338665,118.76464335)(684.05338665,118.75774048)
\lineto(684.05338665,95.47148842)
\curveto(684.05338665,95.45705275)(684.04162815,95.44532049)(684.02719248,95.44532049)
\lineto(645.15044706,95.44532049)
\curveto(645.13601139,95.44532049)(645.12433163,95.45705275)(645.12433163,95.47148842)
\closepath
}
}
{
\newrgbcolor{curcolor}{0 0 0}
\pscustom[linewidth=1.92650529,linecolor=curcolor]
{
\newpath
\moveto(645.12433163,118.75774048)
\lineto(645.12433163,118.75774048)
\curveto(645.12433163,118.77217614)(645.13601139,118.78385591)(645.15044706,118.78385591)
\lineto(684.02719248,118.78385591)
\curveto(684.03420034,118.78385591)(684.04084075,118.78110001)(684.04567013,118.77621813)
\curveto(684.05060451,118.77131)(684.05338665,118.76464335)(684.05338665,118.75774048)
\lineto(684.05338665,95.47148842)
\curveto(684.05338665,95.45705275)(684.04162815,95.44532049)(684.02719248,95.44532049)
\lineto(645.15044706,95.44532049)
\curveto(645.13601139,95.44532049)(645.12433163,95.45705275)(645.12433163,95.47148842)
\closepath
}
}
{
\newrgbcolor{curcolor}{0 1 0}
\pscustom[linestyle=none,fillstyle=solid,fillcolor=curcolor]
{
\newpath
\moveto(684.32661445,118.75774048)
\lineto(684.32661445,118.75774048)
\curveto(684.32661445,118.77217614)(684.33821547,118.78385591)(684.35265114,118.78385591)
\lineto(723.2294753,118.78385591)
\curveto(723.23637817,118.78385591)(723.24301858,118.78110001)(723.24795295,118.77621813)
\curveto(723.25283483,118.77131)(723.25566947,118.76464335)(723.25566947,118.75774048)
\lineto(723.25566947,95.47148842)
\curveto(723.25566947,95.45705275)(723.24391097,95.44532049)(723.2294753,95.44532049)
\lineto(684.35265114,95.44532049)
\curveto(684.33821547,95.44532049)(684.32661445,95.45705275)(684.32661445,95.47148842)
\closepath
}
}
{
\newrgbcolor{curcolor}{0 0 0}
\pscustom[linewidth=1.92650529,linecolor=curcolor]
{
\newpath
\moveto(684.32661445,118.75774048)
\lineto(684.32661445,118.75774048)
\curveto(684.32661445,118.77217614)(684.33821547,118.78385591)(684.35265114,118.78385591)
\lineto(723.2294753,118.78385591)
\curveto(723.23637817,118.78385591)(723.24301858,118.78110001)(723.24795295,118.77621813)
\curveto(723.25283483,118.77131)(723.25566947,118.76464335)(723.25566947,118.75774048)
\lineto(723.25566947,95.47148842)
\curveto(723.25566947,95.45705275)(723.24391097,95.44532049)(723.2294753,95.44532049)
\lineto(684.35265114,95.44532049)
\curveto(684.33821547,95.44532049)(684.32661445,95.45705275)(684.32661445,95.47148842)
\closepath
}
}
{
\newrgbcolor{curcolor}{0 0 0}
\pscustom[linestyle=none,fillstyle=solid,fillcolor=curcolor,opacity=0]
{
\newpath
\moveto(563.58937863,66.30687556)
\lineto(532.28235541,35.18882834)
}
}
{
\newrgbcolor{curcolor}{0 0 0}
\pscustom[linewidth=1.92650529,linecolor=curcolor]
{
\newpath
\moveto(563.58937863,66.30687556)
\lineto(540.48055403,43.33752575)
}
}
{
\newrgbcolor{curcolor}{0 0 0}
\pscustom[linestyle=none,fillstyle=solid,fillcolor=curcolor]
{
\newpath
\moveto(538.23727773,45.59439782)
\lineto(534.27985792,37.17425739)
\lineto(542.72380408,41.08067993)
\closepath
}
}
{
\newrgbcolor{curcolor}{0 0 0}
\pscustom[linewidth=1.92650529,linecolor=curcolor]
{
\newpath
\moveto(538.23727773,45.59439782)
\lineto(534.27985792,37.17425739)
\lineto(542.72380408,41.08067993)
\closepath
}
}
{
\newrgbcolor{curcolor}{0 0 0}
\pscustom[linestyle=none,fillstyle=solid,fillcolor=curcolor,opacity=0]
{
\newpath
\moveto(601.5993279,66.67131052)
\lineto(635.74099105,35.20680731)
}
}
{
\newrgbcolor{curcolor}{0 0 0}
\pscustom[linewidth=1.92650529,linecolor=curcolor]
{
\newpath
\moveto(601.5993279,66.67131052)
\lineto(627.24103451,43.04020352)
}
}
{
\newrgbcolor{curcolor}{0 0 0}
\pscustom[linestyle=none,fillstyle=solid,fillcolor=curcolor]
{
\newpath
\moveto(625.08458218,40.70026075)
\lineto(633.66997162,37.11541237)
\lineto(629.39748683,45.38017253)
\closepath
}
}
{
\newrgbcolor{curcolor}{0 0 0}
\pscustom[linewidth=1.92650529,linecolor=curcolor]
{
\newpath
\moveto(625.08458218,40.70026075)
\lineto(633.66997162,37.11541237)
\lineto(629.39748683,45.38017253)
\closepath
}
}
{
\newrgbcolor{curcolor}{0 0 0}
\pscustom[linestyle=none,fillstyle=solid,fillcolor=curcolor,opacity=0]
{
\newpath
\moveto(764.81800001,66.67131052)
\lineto(733.98341678,35.5532633)
}
}
{
\newrgbcolor{curcolor}{0 0 0}
\pscustom[linewidth=1.92650529,linecolor=curcolor]
{
\newpath
\moveto(764.81800001,66.67131052)
\lineto(742.11938456,43.76403407)
}
}
{
\newrgbcolor{curcolor}{0 0 0}
\pscustom[linestyle=none,fillstyle=solid,fillcolor=curcolor]
{
\newpath
\moveto(739.85904793,46.00376707)
\lineto(735.96577497,37.55383667)
\lineto(744.37972119,41.52430108)
\closepath
}
}
{
\newrgbcolor{curcolor}{0 0 0}
\pscustom[linewidth=1.92650529,linecolor=curcolor]
{
\newpath
\moveto(739.85904793,46.00376707)
\lineto(735.96577497,37.55383667)
\lineto(744.37972119,41.52430108)
\closepath
}
}
{
\newrgbcolor{curcolor}{0 0 0}
\pscustom[linestyle=none,fillstyle=solid,fillcolor=curcolor,opacity=0]
{
\newpath
\moveto(802.82792304,66.67131052)
\lineto(835.45777822,35.20680731)
}
}
{
\newrgbcolor{curcolor}{0 0 0}
\pscustom[linewidth=1.92650529,linecolor=curcolor]
{
\newpath
\moveto(802.82792304,66.67131052)
\lineto(827.13708641,43.23033437)
}
}
{
\newrgbcolor{curcolor}{0 0 0}
\pscustom[linestyle=none,fillstyle=solid,fillcolor=curcolor]
{
\newpath
\moveto(824.92827199,40.93973533)
\lineto(833.43045949,37.16171148)
\lineto(829.34579585,45.52090716)
\closepath
}
}
{
\newrgbcolor{curcolor}{0 0 0}
\pscustom[linewidth=1.92650529,linecolor=curcolor]
{
\newpath
\moveto(824.92827199,40.93973533)
\lineto(833.43045949,37.16171148)
\lineto(829.34579585,45.52090716)
\closepath
}
}
{
\newrgbcolor{curcolor}{0 0 0}
\pscustom[linestyle=none,fillstyle=solid,fillcolor=curcolor,opacity=0]
{
\newpath
\moveto(664.35269164,107.24377429)
\lineto(583.25057241,76.47218306)
}
}
{
\newrgbcolor{curcolor}{0 0 0}
\pscustom[linewidth=1.92650529,linecolor=curcolor]
{
\newpath
\moveto(664.35269164,107.24377429)
\lineto(594.05782086,80.5726472)
}
}
{
\newrgbcolor{curcolor}{0 0 0}
\pscustom[linestyle=none,fillstyle=solid,fillcolor=curcolor]
{
\newpath
\moveto(592.9290305,83.54778553)
\lineto(585.88376917,77.47128865)
\lineto(595.18663748,77.59753511)
\closepath
}
}
{
\newrgbcolor{curcolor}{0 0 0}
\pscustom[linewidth=1.92650529,linecolor=curcolor]
{
\newpath
\moveto(592.9290305,83.54778553)
\lineto(585.88376917,77.47128865)
\lineto(595.18663748,77.59753511)
\closepath
}
}
{
\newrgbcolor{curcolor}{0 0 0}
\pscustom[linestyle=none,fillstyle=solid,fillcolor=curcolor,opacity=0]
{
\newpath
\moveto(702.51166949,108.09411378)
\lineto(785.40913941,77.51149854)
}
}
{
\newrgbcolor{curcolor}{0 0 0}
\pscustom[linewidth=1.92650529,linecolor=curcolor]
{
\newpath
\moveto(702.51166949,108.09411378)
\lineto(774.56459444,81.51230409)
}
}
{
\newrgbcolor{curcolor}{0 0 0}
\pscustom[linestyle=none,fillstyle=solid,fillcolor=curcolor]
{
\newpath
\moveto(773.46323185,78.52692956)
\lineto(782.76688755,78.48627347)
\lineto(775.66598329,84.49767862)
\closepath
}
}
{
\newrgbcolor{curcolor}{0 0 0}
\pscustom[linewidth=1.92650529,linecolor=curcolor]
{
\newpath
\moveto(773.46323185,78.52692956)
\lineto(782.76688755,78.48627347)
\lineto(775.66598329,84.49767862)
\closepath
}
}
{
\newrgbcolor{curcolor}{0 0 0}
\pscustom[linestyle=none,fillstyle=solid,fillcolor=curcolor,opacity=0]
{
\newpath
\moveto(683.58136661,149.88132575)
\lineto(683.77034261,118.92075852)
}
}
{
\newrgbcolor{curcolor}{0 0 0}
\pscustom[linewidth=1.92650529,linecolor=curcolor]
{
\newpath
\moveto(683.58136661,149.88132575)
\lineto(683.69979157,130.47958026)
}
}
{
\newrgbcolor{curcolor}{0 0 0}
\pscustom[linestyle=none,fillstyle=solid,fillcolor=curcolor]
{
\newpath
\moveto(680.51780326,130.46013148)
\lineto(683.75312479,121.73705203)
\lineto(686.88177988,130.49897654)
\closepath
}
}
{
\newrgbcolor{curcolor}{0 0 0}
\pscustom[linewidth=1.92650529,linecolor=curcolor]
{
\newpath
\moveto(680.51780326,130.46013148)
\lineto(683.75312479,121.73705203)
\lineto(686.88177988,130.49897654)
\closepath
}
}
{
\newrgbcolor{curcolor}{0 0 0}
\pscustom[linestyle=none,fillstyle=solid,fillcolor=curcolor,opacity=0]
{
\newpath
\moveto(49.18939667,425.99270888)
\lineto(371.6139406,425.99270888)
\lineto(371.6139406,394.02426968)
\lineto(49.18939667,394.02426968)
\closepath
}
}
{
\newrgbcolor{curcolor}{0 0 0}
\pscustom[linestyle=none,fillstyle=solid,fillcolor=curcolor]
{
\newpath
\moveto(130.46522854,402.87436698)
\lineto(121.01993517,402.87436698)
\lineto(121.01993517,404.65561337)
\lineto(124.65274031,404.65561337)
\lineto(124.65274031,416.35090219)
\lineto(121.01993517,416.35090219)
\lineto(121.01993517,417.94464896)
\curveto(121.51212167,417.94464896)(122.03946435,417.98371138)(122.60196321,418.06183623)
\curveto(123.16446208,418.14777355)(123.59024246,418.26886706)(123.87930438,418.42511674)
\curveto(124.23867865,418.62042884)(124.51992808,418.8665221)(124.72305267,419.16339649)
\curveto(124.93398974,419.46808338)(125.05508325,419.87433256)(125.08633318,420.38214403)
\lineto(126.90273575,420.38214403)
\lineto(126.90273575,404.65561337)
\lineto(130.46522854,404.65561337)
\closepath
}
}
{
\newrgbcolor{curcolor}{0 0 0}
\pscustom[linestyle=none,fillstyle=solid,fillcolor=curcolor]
{
\newpath
\moveto(133.19972293,402.87436698)
\lineto(130.39894736,402.87436698)
\lineto(130.39894736,406.21420397)
\lineto(133.19972293,406.21420397)
\closepath
}
}
{
\newrgbcolor{curcolor}{0 0 0}
\pscustom[linestyle=none,fillstyle=solid,fillcolor=curcolor]
{
\newpath
\moveto(145.89016602,402.87436698)
\lineto(138.99955498,402.87436698)
\lineto(138.99955498,404.65561337)
\lineto(141.2847066,404.65561337)
\lineto(141.2847066,418.542304)
\lineto(138.99955498,418.542304)
\lineto(138.99955498,420.3235504)
\lineto(145.89016602,420.3235504)
\lineto(145.89016602,418.542304)
\lineto(143.6050144,418.542304)
\lineto(143.6050144,404.65561337)
\lineto(145.89016602,404.65561337)
\closepath
}
}
{
\newrgbcolor{curcolor}{0 0 0}
\pscustom[linestyle=none,fillstyle=solid,fillcolor=curcolor]
{
\newpath
\moveto(158.71532774,402.87436698)
\lineto(156.5122072,402.87436698)
\lineto(156.5122072,410.32747689)
\curveto(156.5122072,410.92903817)(156.47705102,411.49153703)(156.40673866,412.01497347)
\curveto(156.3364263,412.54622239)(156.20752032,412.96028406)(156.02002069,413.25715845)
\curveto(155.82470859,413.58528279)(155.54345916,413.8274698)(155.1762724,413.98371948)
\curveto(154.80908565,414.14778165)(154.33252411,414.22981274)(153.7465878,414.22981274)
\curveto(153.14502652,414.22981274)(152.51612154,414.08137554)(151.85987287,413.78450114)
\curveto(151.2036242,413.48762674)(150.57471922,413.10872126)(149.97315794,412.64778469)
\lineto(149.97315794,402.87436698)
\lineto(147.7700374,402.87436698)
\lineto(147.7700374,415.96418422)
\lineto(149.97315794,415.96418422)
\lineto(149.97315794,414.51106217)
\curveto(150.66065655,415.08137351)(151.37159261,415.52668511)(152.10596612,415.84699696)
\curveto(152.84033963,416.16730881)(153.59424436,416.32746474)(154.36768029,416.32746474)
\curveto(155.78173993,416.32746474)(156.85986274,415.90168435)(157.60204874,415.05012357)
\curveto(158.34423474,414.1985628)(158.71532774,412.97200278)(158.71532774,411.37044353)
\closepath
}
}
{
\newrgbcolor{curcolor}{0 0 0}
\pscustom[linestyle=none,fillstyle=solid,fillcolor=curcolor]
{
\newpath
\moveto(163.57719152,418.15558604)
\lineto(161.09282155,418.15558604)
\lineto(161.09282155,420.44073766)
\lineto(163.57719152,420.44073766)
\closepath
\moveto(163.4365668,402.87436698)
\lineto(161.23344626,402.87436698)
\lineto(161.23344626,415.96418422)
\lineto(163.4365668,415.96418422)
\closepath
}
}
{
\newrgbcolor{curcolor}{0 0 0}
\pscustom[linestyle=none,fillstyle=solid,fillcolor=curcolor]
{
\newpath
\moveto(174.60685897,402.99155424)
\curveto(174.19279731,402.88217946)(173.73967323,402.7923359)(173.24748672,402.72202354)
\curveto(172.76311271,402.65171118)(172.32951983,402.616555)(171.94670811,402.616555)
\curveto(170.61077331,402.616555)(169.59515037,402.97592927)(168.89983928,403.69467782)
\curveto(168.20452819,404.41342636)(167.85687264,405.56576778)(167.85687264,407.15170207)
\lineto(167.85687264,414.11262547)
\lineto(166.3685944,414.11262547)
\lineto(166.3685944,415.96418422)
\lineto(167.85687264,415.96418422)
\lineto(167.85687264,419.72589536)
\lineto(170.05999318,419.72589536)
\lineto(170.05999318,415.96418422)
\lineto(174.60685897,415.96418422)
\lineto(174.60685897,414.11262547)
\lineto(170.05999318,414.11262547)
\lineto(170.05999318,408.1477938)
\curveto(170.05999318,407.46029519)(170.07561815,406.92123379)(170.10686808,406.53060958)
\curveto(170.13811802,406.14779785)(170.2474928,405.78842358)(170.43499242,405.45248676)
\curveto(170.60686707,405.13998739)(170.8412416,404.90951911)(171.138116,404.76108191)
\curveto(171.44280288,404.62045719)(171.90373945,404.55014484)(172.5209257,404.55014484)
\curveto(172.88029997,404.55014484)(173.25529921,404.60092598)(173.64592342,404.70248828)
\curveto(174.03654763,404.81186306)(174.31779706,404.90170662)(174.48967171,404.97201898)
\lineto(174.60685897,404.97201898)
\closepath
}
}
{
\newrgbcolor{curcolor}{0 0 0}
\pscustom[linestyle=none,fillstyle=solid,fillcolor=curcolor]
{
\newpath
\moveto(178.02516412,418.15558604)
\lineto(175.54079415,418.15558604)
\lineto(175.54079415,420.44073766)
\lineto(178.02516412,420.44073766)
\closepath
\moveto(177.88453941,402.87436698)
\lineto(175.68141887,402.87436698)
\lineto(175.68141887,415.96418422)
\lineto(177.88453941,415.96418422)
\closepath
}
}
{
\newrgbcolor{curcolor}{0 0 0}
\pscustom[linestyle=none,fillstyle=solid,fillcolor=curcolor]
{
\newpath
\moveto(192.42982093,402.87436698)
\lineto(190.23841911,402.87436698)
\lineto(190.23841911,404.26889541)
\curveto(190.04310701,404.13608318)(189.77748255,403.94858355)(189.44154573,403.70639655)
\curveto(189.11342139,403.47202202)(188.79310954,403.2845224)(188.48061017,403.14389768)
\curveto(188.11342342,402.96421055)(187.69154927,402.81577335)(187.21498774,402.69858609)
\curveto(186.7384262,402.57358634)(186.17983358,402.51108647)(185.53920988,402.51108647)
\curveto(184.35952477,402.51108647)(183.35952679,402.90171067)(182.53921596,403.68295909)
\curveto(181.71890512,404.46420751)(181.3087497,405.46029924)(181.3087497,406.67123429)
\curveto(181.3087497,407.66341978)(181.51968677,408.46419941)(181.94156092,409.07357318)
\curveto(182.37124755,409.69075943)(182.98062131,410.17513345)(183.76968221,410.52669523)
\curveto(184.5665556,410.87825702)(185.52358491,411.11653779)(186.64077015,411.24153754)
\curveto(187.75795539,411.36653728)(188.95717171,411.46028709)(190.23841911,411.52278697)
\lineto(190.23841911,411.86263003)
\curveto(190.23841911,412.36262902)(190.14857555,412.77669068)(189.96888841,413.10481501)
\curveto(189.79701376,413.43293935)(189.54701427,413.69075133)(189.21888993,413.87825095)
\curveto(188.90639056,414.05793808)(188.53139132,414.17903159)(188.09389221,414.24153146)
\curveto(187.65639309,414.30403134)(187.19936277,414.33528127)(186.72280123,414.33528127)
\curveto(186.1446774,414.33528127)(185.50014746,414.25715643)(184.7892114,414.10090675)
\curveto(184.07827534,413.95246955)(183.34390183,413.73371999)(182.58609086,413.44465808)
\lineto(182.4689036,413.44465808)
\lineto(182.4689036,415.68293479)
\curveto(182.89859023,415.80012206)(183.51968272,415.92902804)(184.33218108,416.06965276)
\curveto(185.14467943,416.21027747)(185.94545906,416.28058983)(186.73451996,416.28058983)
\curveto(187.65639309,416.28058983)(188.45717272,416.20246499)(189.13685885,416.04621531)
\curveto(189.82435745,415.89777811)(190.41810625,415.63996613)(190.91810524,415.27277937)
\curveto(191.41029174,414.9134051)(191.78529098,414.44856229)(192.04310296,413.87825095)
\curveto(192.30091494,413.3079396)(192.42982093,412.60090978)(192.42982093,411.75716149)
\closepath
\moveto(190.23841911,406.0970167)
\lineto(190.23841911,409.74154057)
\curveto(189.56654548,409.70247815)(188.77357833,409.64388452)(187.85951768,409.56575968)
\curveto(186.95326952,409.48763484)(186.23452097,409.37435382)(185.70327205,409.22591662)
\curveto(185.07046083,409.04622948)(184.55874312,408.76498005)(184.16811891,408.38216833)
\curveto(183.7774947,408.00716909)(183.58218259,407.48763889)(183.58218259,406.82357773)
\curveto(183.58218259,406.07357925)(183.80874464,405.50717415)(184.26186872,405.12436242)
\curveto(184.7149928,404.74936318)(185.40639765,404.56186356)(186.33608327,404.56186356)
\curveto(187.1095192,404.56186356)(187.81654902,404.71030076)(188.45717272,405.00717516)
\curveto(189.09779642,405.31186204)(189.69154522,405.67514256)(190.23841911,406.0970167)
\closepath
}
}
{
\newrgbcolor{curcolor}{0 0 0}
\pscustom[linestyle=none,fillstyle=solid,fillcolor=curcolor]
{
\newpath
\moveto(195.45250081,402.87436698)
\lineto(193.24938027,402.87436698)
\lineto(193.24938027,421.10870506)
\lineto(195.45250081,421.10870506)
\closepath
}
}
{
\newrgbcolor{curcolor}{0 0 0}
\pscustom[linestyle=none,fillstyle=solid,fillcolor=curcolor]
{
\newpath
\moveto(217.27342525,409.51888477)
\curveto(217.27342525,408.42513699)(217.11717557,407.44076398)(216.8046762,406.56576576)
\curveto(216.49998932,405.69076753)(216.08592766,404.95639401)(215.56249122,404.36264522)
\curveto(215.00780484,403.74545897)(214.39843107,403.28061616)(213.73436992,402.96811679)
\curveto(213.07030876,402.66342991)(212.33984149,402.51108647)(211.54296811,402.51108647)
\curveto(210.80078211,402.51108647)(210.15234592,402.60093003)(209.59765954,402.78061717)
\curveto(209.04297317,402.95249182)(208.49609928,403.18686635)(207.95703787,403.48374075)
\lineto(207.81641315,402.87436698)
\lineto(205.75391733,402.87436698)
\lineto(205.75391733,421.10870506)
\lineto(207.95703787,421.10870506)
\lineto(207.95703787,414.59309325)
\curveto(208.57422412,415.10090472)(209.23047279,415.51496638)(209.92578388,415.83527823)
\curveto(210.62109497,416.16340257)(211.40234339,416.32746474)(212.26952913,416.32746474)
\curveto(213.816401,416.32746474)(215.03514853,415.73371594)(215.92577173,414.54621834)
\curveto(216.82420741,413.35872075)(217.27342525,411.68294289)(217.27342525,409.51888477)
\closepath
\moveto(214.99999236,409.46029114)
\curveto(214.99999236,411.02278798)(214.74218038,412.20637933)(214.22655642,413.0110652)
\curveto(213.71093247,413.82356356)(212.8789029,414.22981274)(211.73046773,414.22981274)
\curveto(211.08984402,414.22981274)(210.44140784,414.08918802)(209.78515917,413.80793859)
\curveto(209.12891049,413.53450164)(208.51953673,413.17903361)(207.95703787,412.7415345)
\lineto(207.95703787,405.24154969)
\curveto(208.5820366,404.96030026)(209.11719177,404.76498815)(209.56250337,404.65561337)
\curveto(210.01562745,404.54623859)(210.52734516,404.49155121)(211.09765651,404.49155121)
\curveto(212.31640404,404.49155121)(213.26952711,404.8899879)(213.95702572,405.68686128)
\curveto(214.65233681,406.49154716)(214.99999236,407.74935711)(214.99999236,409.46029114)
\closepath
}
}
{
\newrgbcolor{curcolor}{0 0 0}
\pscustom[linestyle=none,fillstyle=solid,fillcolor=curcolor]
{
\newpath
\moveto(220.84445531,402.87436698)
\lineto(218.64133477,402.87436698)
\lineto(218.64133477,421.10870506)
\lineto(220.84445531,421.10870506)
\closepath
}
}
{
\newrgbcolor{curcolor}{0 0 0}
\pscustom[linestyle=none,fillstyle=solid,fillcolor=curcolor]
{
\newpath
\moveto(236.30211139,409.41341624)
\curveto(236.30211139,407.28060806)(235.7552375,405.59701772)(234.66148971,404.36264522)
\curveto(233.56774193,403.12827272)(232.10290114,402.51108647)(230.26696736,402.51108647)
\curveto(228.41540861,402.51108647)(226.94275534,403.12827272)(225.84900756,404.36264522)
\curveto(224.76307226,405.59701772)(224.2201046,407.28060806)(224.2201046,409.41341624)
\curveto(224.2201046,411.54622442)(224.76307226,413.22981476)(225.84900756,414.46418726)
\curveto(226.94275534,415.70637225)(228.41540861,416.32746474)(230.26696736,416.32746474)
\curveto(232.10290114,416.32746474)(233.56774193,415.70637225)(234.66148971,414.46418726)
\curveto(235.7552375,413.22981476)(236.30211139,411.54622442)(236.30211139,409.41341624)
\closepath
\moveto(234.02867849,409.41341624)
\curveto(234.02867849,411.10872531)(233.69664791,412.36653526)(233.03258676,413.1868461)
\curveto(232.3685256,414.01496942)(231.44665247,414.42903108)(230.26696736,414.42903108)
\curveto(229.07165728,414.42903108)(228.14197166,414.01496942)(227.47791051,413.1868461)
\curveto(226.82166184,412.36653526)(226.4935375,411.10872531)(226.4935375,409.41341624)
\curveto(226.4935375,407.77279456)(226.82556808,406.52670333)(227.48962923,405.67514256)
\curveto(228.15369039,404.83139427)(229.07946976,404.40952012)(230.26696736,404.40952012)
\curveto(231.43883999,404.40952012)(232.35680688,404.82748803)(233.02086803,405.66342383)
\curveto(233.69274167,406.50717212)(234.02867849,407.75716959)(234.02867849,409.41341624)
\closepath
}
}
{
\newrgbcolor{curcolor}{0 0 0}
\pscustom[linestyle=none,fillstyle=solid,fillcolor=curcolor]
{
\newpath
\moveto(247.52152531,403.69467782)
\curveto(246.7871518,403.34311603)(246.08793447,403.06967908)(245.42387331,402.87436698)
\curveto(244.76762464,402.67905488)(244.06840731,402.58139882)(243.32622131,402.58139882)
\curveto(242.38091072,402.58139882)(241.51372498,402.7181173)(240.72466408,402.99155424)
\curveto(239.93560318,403.27280367)(239.25982329,403.69467782)(238.69732443,404.25717668)
\curveto(238.12701309,404.81967554)(237.68560773,405.5306116)(237.37310837,406.38998486)
\curveto(237.060609,407.24935812)(236.90435931,408.25326234)(236.90435931,409.40169751)
\curveto(236.90435931,411.54231818)(237.49029563,413.22200228)(238.66216826,414.44074981)
\curveto(239.84185337,415.65949734)(241.39653772,416.26887111)(243.32622131,416.26887111)
\curveto(244.07621979,416.26887111)(244.8105933,416.16340257)(245.52934185,415.9524655)
\curveto(246.25590288,415.74152842)(246.91996403,415.48371645)(247.52152531,415.17902956)
\lineto(247.52152531,412.72981577)
\lineto(247.40433805,412.72981577)
\curveto(246.73246441,413.25325221)(246.03715332,413.65559515)(245.31840478,413.93684458)
\curveto(244.60746872,414.21809401)(243.91215762,414.35871872)(243.2324715,414.35871872)
\curveto(241.98247403,414.35871872)(240.99419478,413.93684458)(240.26763375,413.09309629)
\curveto(239.54888521,412.25716048)(239.18951094,411.02669422)(239.18951094,409.40169751)
\curveto(239.18951094,407.82357571)(239.54107273,406.60873442)(240.2441963,405.75717364)
\curveto(240.95513236,404.91342535)(241.95122409,404.49155121)(243.2324715,404.49155121)
\curveto(243.6777831,404.49155121)(244.13090718,404.55014484)(244.59184375,404.6673321)
\curveto(245.05278031,404.78451936)(245.46684198,404.9368628)(245.83402873,405.12436242)
\curveto(246.15434058,405.28842459)(246.45512122,405.46029924)(246.73637065,405.63998638)
\curveto(247.01762009,405.827486)(247.24027588,405.98764193)(247.40433805,406.12045416)
\lineto(247.52152531,406.12045416)
\closepath
}
}
{
\newrgbcolor{curcolor}{0 0 0}
\pscustom[linestyle=none,fillstyle=solid,fillcolor=curcolor]
{
\newpath
\moveto(260.35662811,402.87436698)
\lineto(257.45038399,402.87436698)
\lineto(252.20039462,408.60482413)
\lineto(250.77071002,407.24545188)
\lineto(250.77071002,402.87436698)
\lineto(248.56758948,402.87436698)
\lineto(248.56758948,421.10870506)
\lineto(250.77071002,421.10870506)
\lineto(250.77071002,409.41341624)
\lineto(257.13397838,415.96418422)
\lineto(259.91131651,415.96418422)
\lineto(253.82929758,409.91732147)
\closepath
}
}
{
\newrgbcolor{curcolor}{0 0 0}
\pscustom[linestyle=none,fillstyle=solid,fillcolor=curcolor]
{
\newpath
\moveto(273.67865313,402.99155424)
\curveto(273.26459147,402.88217946)(272.81146739,402.7923359)(272.31928089,402.72202354)
\curveto(271.83490687,402.65171118)(271.40131399,402.616555)(271.01850227,402.616555)
\curveto(269.68256747,402.616555)(268.66694453,402.97592927)(267.97163344,403.69467782)
\curveto(267.27632235,404.41342636)(266.9286668,405.56576778)(266.9286668,407.15170207)
\lineto(266.9286668,414.11262547)
\lineto(265.44038857,414.11262547)
\lineto(265.44038857,415.96418422)
\lineto(266.9286668,415.96418422)
\lineto(266.9286668,419.72589536)
\lineto(269.13178734,419.72589536)
\lineto(269.13178734,415.96418422)
\lineto(273.67865313,415.96418422)
\lineto(273.67865313,414.11262547)
\lineto(269.13178734,414.11262547)
\lineto(269.13178734,408.1477938)
\curveto(269.13178734,407.46029519)(269.14741231,406.92123379)(269.17866225,406.53060958)
\curveto(269.20991218,406.14779785)(269.31928696,405.78842358)(269.50678658,405.45248676)
\curveto(269.67866123,405.13998739)(269.91303576,404.90951911)(270.20991016,404.76108191)
\curveto(270.51459704,404.62045719)(270.97553361,404.55014484)(271.59271986,404.55014484)
\curveto(271.95209413,404.55014484)(272.32709337,404.60092598)(272.71771758,404.70248828)
\curveto(273.10834179,404.81186306)(273.38959122,404.90170662)(273.56146587,404.97201898)
\lineto(273.67865313,404.97201898)
\closepath
}
}
{
\newrgbcolor{curcolor}{0 0 0}
\pscustom[linestyle=none,fillstyle=solid,fillcolor=curcolor]
{
\newpath
\moveto(282.89773542,413.56184534)
\lineto(282.78054815,413.56184534)
\curveto(282.45242382,413.63997018)(282.13211197,413.69465757)(281.8196126,413.72590751)
\curveto(281.51492572,413.76496993)(281.1516452,413.78450114)(280.72977106,413.78450114)
\curveto(280.05008493,413.78450114)(279.39383626,413.6321577)(278.76102504,413.32747081)
\curveto(278.12821382,413.03059641)(277.51884006,412.64387845)(276.93290375,412.16731691)
\lineto(276.93290375,402.87436698)
\lineto(274.72978321,402.87436698)
\lineto(274.72978321,415.96418422)
\lineto(276.93290375,415.96418422)
\lineto(276.93290375,414.03059439)
\curveto(277.80790197,414.73371796)(278.57743166,415.22981071)(279.24149282,415.51887262)
\curveto(279.91336646,415.81574702)(280.59695883,415.96418422)(281.29226992,415.96418422)
\curveto(281.67508164,415.96418422)(281.95242483,415.9524655)(282.12429948,415.92902804)
\curveto(282.29617413,415.91340308)(282.55398611,415.8782469)(282.89773542,415.82355951)
\closepath
}
}
{
\newrgbcolor{curcolor}{0 0 0}
\pscustom[linestyle=none,fillstyle=solid,fillcolor=curcolor]
{
\newpath
\moveto(294.13477321,409.19076044)
\lineto(284.49026149,409.19076044)
\curveto(284.49026149,408.38607457)(284.61135499,407.68295099)(284.853542,407.08138971)
\curveto(285.09572901,406.48764091)(285.42775959,405.99936065)(285.84963374,405.61654893)
\curveto(286.25588291,405.24154969)(286.73635069,404.96030026)(287.29103707,404.77280064)
\curveto(287.85353593,404.58530102)(288.47072218,404.49155121)(289.14259582,404.49155121)
\curveto(290.03321901,404.49155121)(290.92774845,404.6673321)(291.82618413,405.01889389)
\curveto(292.7324323,405.37826816)(293.37696224,405.72982995)(293.75977397,406.07357925)
\lineto(293.87696123,406.07357925)
\lineto(293.87696123,403.67124037)
\curveto(293.13477523,403.358741)(292.37696427,403.09702278)(291.60352833,402.88608571)
\curveto(290.8300924,402.67514863)(290.01759405,402.5696801)(289.16603327,402.5696801)
\curveto(286.99416267,402.5696801)(285.2988536,403.15561641)(284.08010607,404.32748904)
\curveto(282.86135854,405.50717415)(282.25198477,407.17904576)(282.25198477,409.34310388)
\curveto(282.25198477,411.48372455)(282.83401484,413.18293986)(283.99807498,414.44074981)
\curveto(285.16994761,415.69855976)(286.709007,416.32746474)(288.61525314,416.32746474)
\curveto(290.38087456,416.32746474)(291.74024681,415.81184078)(292.69336988,414.78059287)
\curveto(293.65430543,413.74934496)(294.13477321,412.28450417)(294.13477321,410.38607052)
\closepath
\moveto(291.9902463,410.87825702)
\curveto(291.98243382,412.03450468)(291.68946566,412.92903412)(291.11134183,413.56184534)
\curveto(290.54103049,414.19465656)(289.6699385,414.51106217)(288.49806587,414.51106217)
\curveto(287.31838076,414.51106217)(286.37697642,414.16340662)(285.67385284,413.46809553)
\curveto(284.97854175,412.77278444)(284.5840113,411.90950493)(284.49026149,410.87825702)
\closepath
}
}
{
\newrgbcolor{curcolor}{0 0 0}
\pscustom[linestyle=none,fillstyle=solid,fillcolor=curcolor]
{
\newpath
\moveto(305.15075737,409.19076044)
\lineto(295.50624565,409.19076044)
\curveto(295.50624565,408.38607457)(295.62733916,407.68295099)(295.86952616,407.08138971)
\curveto(296.11171317,406.48764091)(296.44374375,405.99936065)(296.8656179,405.61654893)
\curveto(297.27186708,405.24154969)(297.75233485,404.96030026)(298.30702123,404.77280064)
\curveto(298.86952009,404.58530102)(299.48670634,404.49155121)(300.15857998,404.49155121)
\curveto(301.04920318,404.49155121)(301.94373261,404.6673321)(302.8421683,405.01889389)
\curveto(303.74841646,405.37826816)(304.3929464,405.72982995)(304.77575813,406.07357925)
\lineto(304.89294539,406.07357925)
\lineto(304.89294539,403.67124037)
\curveto(304.1507594,403.358741)(303.39294843,403.09702278)(302.6195125,402.88608571)
\curveto(301.84607656,402.67514863)(301.03357821,402.5696801)(300.18201743,402.5696801)
\curveto(298.01014683,402.5696801)(296.31483776,403.15561641)(295.09609023,404.32748904)
\curveto(293.8773427,405.50717415)(293.26796893,407.17904576)(293.26796893,409.34310388)
\curveto(293.26796893,411.48372455)(293.849999,413.18293986)(295.01405915,414.44074981)
\curveto(296.18593177,415.69855976)(297.72499116,416.32746474)(299.6312373,416.32746474)
\curveto(301.39685872,416.32746474)(302.75623097,415.81184078)(303.70935404,414.78059287)
\curveto(304.67028959,413.74934496)(305.15075737,412.28450417)(305.15075737,410.38607052)
\closepath
\moveto(303.00623046,410.87825702)
\curveto(302.99841798,412.03450468)(302.70544982,412.92903412)(302.12732599,413.56184534)
\curveto(301.55701465,414.19465656)(300.68592266,414.51106217)(299.51405003,414.51106217)
\curveto(298.33436492,414.51106217)(297.39296058,414.16340662)(296.689837,413.46809553)
\curveto(295.99452591,412.77278444)(295.59999546,411.90950493)(295.50624565,410.87825702)
\closepath
}
}
{
\newrgbcolor{curcolor}{0 0 0}
\pscustom[linestyle=none,fillstyle=solid,fillcolor=curcolor,opacity=0]
{
\newpath
\moveto(518.72293405,425.99270888)
\lineto(841.14747799,425.99270888)
\lineto(841.14747799,394.02426968)
\lineto(518.72293405,394.02426968)
\closepath
}
}
{
\newrgbcolor{curcolor}{0 0 0}
\pscustom[linestyle=none,fillstyle=solid,fillcolor=curcolor]
{
\newpath
\moveto(588.09485459,402.87436698)
\lineto(576.28237851,402.87436698)
\lineto(576.28237851,405.32358077)
\lineto(578.74331103,407.4329515)
\curveto(579.57143435,408.13607508)(580.34096404,408.83529241)(581.0519001,409.5306035)
\curveto(582.55189706,410.98372556)(583.57923873,412.13606698)(584.13392511,412.98762775)
\curveto(584.68861149,413.84700101)(584.96595468,414.77278039)(584.96595468,415.76496588)
\curveto(584.96595468,416.67121404)(584.66517403,417.37824386)(584.06361275,417.88605533)
\curveto(583.46986395,418.40167929)(582.63783439,418.65949127)(581.56752406,418.65949127)
\curveto(580.856588,418.65949127)(580.0870583,418.53449152)(579.25893498,418.28449202)
\curveto(578.43081166,418.03449253)(577.62221955,417.65168081)(576.83315864,417.13605685)
\lineto(576.71597138,417.13605685)
\lineto(576.71597138,419.59698937)
\curveto(577.27065776,419.87042631)(578.00893751,420.12042581)(578.93081065,420.34698785)
\curveto(579.86049626,420.57354989)(580.75893194,420.68683091)(581.62611769,420.68683091)
\curveto(583.41517657,420.68683091)(584.81751748,420.25323804)(585.83314042,419.38605229)
\curveto(586.84876336,418.52667903)(587.35657483,417.35871265)(587.35657483,415.88215314)
\curveto(587.35657483,415.21809198)(587.27063751,414.59699949)(587.09876286,414.01887566)
\curveto(586.93470069,413.44856432)(586.68860744,412.90559667)(586.3604831,412.38997271)
\curveto(586.05579622,411.90559869)(585.69642195,411.42903716)(585.28236028,410.96028811)
\curveto(584.87611111,410.49153906)(584.38001836,409.97200886)(583.79408205,409.40169751)
\curveto(582.95814624,408.58138667)(582.09486674,407.78451329)(581.20424354,407.01107735)
\curveto(580.31362035,406.2454539)(579.48159078,405.53451784)(578.70815485,404.87826917)
\lineto(588.09485459,404.87826917)
\closepath
}
}
{
\newrgbcolor{curcolor}{0 0 0}
\pscustom[linestyle=none,fillstyle=solid,fillcolor=curcolor]
{
\newpath
\moveto(592.56925177,402.87436698)
\lineto(589.76847619,402.87436698)
\lineto(589.76847619,406.21420397)
\lineto(592.56925177,406.21420397)
\closepath
}
}
{
\newrgbcolor{curcolor}{0 0 0}
\pscustom[linestyle=none,fillstyle=solid,fillcolor=curcolor]
{
\newpath
\moveto(612.58389878,404.13998942)
\curveto(612.15421215,403.9524898)(611.76358794,403.7767089)(611.41202615,403.61264674)
\curveto(611.06827685,403.44858457)(610.61515277,403.27670992)(610.0526539,403.09702278)
\curveto(609.57609237,402.94858558)(609.05656217,402.82358583)(608.49406331,402.72202354)
\curveto(607.93937693,402.61264876)(607.32609693,402.55796137)(606.65422329,402.55796137)
\curveto(605.38860085,402.55796137)(604.23625943,402.73374226)(603.19719904,403.08530405)
\curveto(602.16595113,403.44467833)(601.26751544,404.00327094)(600.501892,404.76108191)
\curveto(599.75189351,405.50326791)(599.1659572,406.44467225)(598.74408305,407.58529494)
\curveto(598.32220891,408.73373011)(598.11127184,410.06575867)(598.11127184,411.5813806)
\curveto(598.11127184,413.01887769)(598.31439642,414.30403134)(598.7206456,415.43684154)
\curveto(599.12689478,416.56965175)(599.71283109,417.52668106)(600.47845454,418.30792948)
\curveto(601.22064054,419.06574044)(602.11516998,419.64386427)(603.16204286,420.04230096)
\curveto(604.21672822,420.44073766)(605.38469461,420.639956)(606.66594201,420.639956)
\curveto(607.60344011,420.639956)(608.53703197,420.52667498)(609.46671759,420.30011294)
\curveto(610.40421569,420.0735509)(611.44327609,419.67511421)(612.58389878,419.10480286)
\lineto(612.58389878,416.35090219)
\lineto(612.40811789,416.35090219)
\curveto(611.44718233,417.15558806)(610.49405926,417.74152437)(609.54874868,418.10871113)
\curveto(608.60343809,418.47589789)(607.59172139,418.65949127)(606.51359857,418.65949127)
\curveto(605.63078786,418.65949127)(604.83391447,418.51496031)(604.12297841,418.22589839)
\curveto(603.41985484,417.94464896)(602.79094986,417.50324361)(602.23626348,416.90168232)
\curveto(601.69720207,416.31574601)(601.27532793,415.57356001)(600.97064105,414.67512433)
\curveto(600.67376665,413.78450114)(600.52532945,412.75325323)(600.52532945,411.5813806)
\curveto(600.52532945,410.35482058)(600.68939162,409.30013522)(601.01751595,408.41732451)
\curveto(601.35345277,407.53451379)(601.7831394,406.81576525)(602.30657584,406.26107887)
\curveto(602.85344973,405.68295504)(603.49016719,405.25326841)(604.21672822,404.97201898)
\curveto(604.95110174,404.69858204)(605.72453767,404.56186356)(606.53703602,404.56186356)
\curveto(607.65422126,404.56186356)(608.70109414,404.75326943)(609.67765466,405.13608115)
\curveto(610.65421519,405.51889287)(611.56827584,406.09311046)(612.41983661,406.85873391)
\lineto(612.58389878,406.85873391)
\closepath
}
}
{
\newrgbcolor{curcolor}{0 0 0}
\pscustom[linestyle=none,fillstyle=solid,fillcolor=curcolor]
{
\newpath
\moveto(628.91511495,418.3196482)
\curveto(629.62605101,417.53839979)(630.16901866,416.58137047)(630.5440179,415.44856027)
\curveto(630.92682962,414.31575006)(631.11823548,413.03059641)(631.11823548,411.59309932)
\curveto(631.11823548,410.15560224)(630.92292338,408.86654235)(630.53229917,407.72591966)
\curveto(630.14948745,406.59310945)(629.61042604,405.64779886)(628.91511495,404.8899879)
\curveto(628.1963664,404.100927)(627.34480563,403.5071782)(626.36043262,403.10874151)
\curveto(625.3838721,402.71030481)(624.26668686,402.51108647)(623.00887691,402.51108647)
\curveto(621.78231689,402.51108647)(620.66513165,402.71421105)(619.65732119,403.12046023)
\curveto(618.65732322,403.52670941)(617.80576244,404.11655196)(617.10263887,404.8899879)
\curveto(616.39951529,405.66342383)(615.85654764,406.61264066)(615.47373591,407.73763838)
\curveto(615.09873667,408.8626361)(614.91123705,410.14778975)(614.91123705,411.59309932)
\curveto(614.91123705,413.01497145)(615.09873667,414.28840637)(615.47373591,415.41340409)
\curveto(615.84873515,416.54621429)(616.39560905,417.51496233)(617.11435759,418.3196482)
\curveto(617.8018562,419.08527165)(618.65341698,419.67120797)(619.66903992,420.07745714)
\curveto(620.69247535,420.48370632)(621.80575434,420.68683091)(623.00887691,420.68683091)
\curveto(624.25887437,420.68683091)(625.37996585,420.47980008)(626.37215134,420.06573842)
\curveto(627.37214932,419.65948924)(628.21980385,419.07745917)(628.91511495,418.3196482)
\closepath
\moveto(628.70417787,411.59309932)
\curveto(628.70417787,413.85871974)(628.1963664,415.60480995)(627.18074346,416.83136997)
\curveto(626.16512051,418.06574247)(624.77840457,418.68292872)(623.02059563,418.68292872)
\curveto(621.24716172,418.68292872)(619.8526333,418.06574247)(618.83701035,416.83136997)
\curveto(617.82919989,415.60480995)(617.32529466,413.85871974)(617.32529466,411.59309932)
\curveto(617.32529466,409.30404146)(617.84091862,407.55013876)(618.87216653,406.33139123)
\curveto(619.90341444,405.12045618)(621.28622414,404.51498866)(623.02059563,404.51498866)
\curveto(624.75496712,404.51498866)(626.13387058,405.12045618)(627.15730601,406.33139123)
\curveto(628.18855392,407.55013876)(628.70417787,409.30404146)(628.70417787,411.59309932)
\closepath
}
}
{
\newrgbcolor{curcolor}{0 0 0}
\pscustom[linestyle=none,fillstyle=solid,fillcolor=curcolor]
{
\newpath
\moveto(655.07984544,420.3235504)
\lineto(650.54469837,402.87436698)
\lineto(647.93142242,402.87436698)
\lineto(644.26346109,417.35871265)
\lineto(640.67753085,402.87436698)
\lineto(638.12284853,402.87436698)
\lineto(633.50567038,420.3235504)
\lineto(635.88457181,420.3235504)
\lineto(639.55253313,405.81576727)
\lineto(643.16190082,420.3235504)
\lineto(645.5173648,420.3235504)
\lineto(649.16188867,405.67514256)
\lineto(652.80641254,420.3235504)
\closepath
}
}
{
\newrgbcolor{curcolor}{0 0 0}
\pscustom[linestyle=none,fillstyle=solid,fillcolor=curcolor]
{
\newpath
\moveto(673.70699354,406.64779684)
\curveto(673.70699354,405.45248676)(673.2109008,404.47201999)(672.2187153,403.70639655)
\curveto(671.2343423,402.9407731)(669.88668878,402.55796137)(668.17575474,402.55796137)
\curveto(667.2070067,402.55796137)(666.31638351,402.67124239)(665.50388515,402.89780443)
\curveto(664.69919928,403.13217896)(664.0234194,403.38608469)(663.47654551,403.65952164)
\lineto(663.47654551,406.13217288)
\lineto(663.59373277,406.13217288)
\curveto(664.28904386,405.60873644)(665.0624798,405.19076854)(665.91404057,404.87826917)
\curveto(666.76560135,404.57358229)(667.58200594,404.42123885)(668.36325436,404.42123885)
\curveto(669.3320024,404.42123885)(670.08981337,404.57748853)(670.63668726,404.8899879)
\curveto(671.18356115,405.20248727)(671.4569981,405.69467377)(671.4569981,406.36654741)
\curveto(671.4569981,406.88217136)(671.3085609,407.27279557)(671.0116865,407.53842004)
\curveto(670.7148121,407.8040445)(670.14450076,408.03060654)(669.30075246,408.21810616)
\curveto(668.9882531,408.28841852)(668.57809768,408.3704496)(668.07028621,408.46419941)
\curveto(667.57028722,408.55794922)(667.11325689,408.65951152)(666.69919523,408.76888629)
\curveto(665.55076006,409.07357318)(664.73435546,409.51888477)(664.24998144,410.10482109)
\curveto(663.77341991,410.69856989)(663.53513914,411.42513091)(663.53513914,412.28450417)
\curveto(663.53513914,412.82356558)(663.64451392,413.33137705)(663.86326347,413.80793859)
\curveto(664.08982552,414.28450012)(664.42966858,414.71028051)(664.88279266,415.08527975)
\curveto(665.32029177,415.45246651)(665.87497815,415.74152842)(666.54685179,415.9524655)
\curveto(667.22653791,416.17121505)(667.98434888,416.28058983)(668.82028469,416.28058983)
\curveto(669.6015331,416.28058983)(670.39059401,416.18293378)(671.18746739,415.98762168)
\curveto(671.99215326,415.80012206)(672.66012066,415.56965377)(673.19136959,415.29621683)
\lineto(673.19136959,412.94075285)
\lineto(673.07418232,412.94075285)
\curveto(672.51168346,413.35481451)(671.8280911,413.70247005)(671.02340523,413.98371948)
\curveto(670.21871935,414.2727814)(669.42965845,414.41731236)(668.65622252,414.41731236)
\curveto(667.85153665,414.41731236)(667.17185052,414.26106267)(666.61716415,413.9485633)
\curveto(666.06247777,413.64387642)(665.78513458,413.1868461)(665.78513458,412.57747233)
\curveto(665.78513458,412.03841092)(665.95310299,411.63216175)(666.28903981,411.3587248)
\curveto(666.61716415,411.08528785)(667.14841307,410.86263205)(667.88278659,410.6907574)
\curveto(668.28903576,410.59700759)(668.74215984,410.50325778)(669.24215883,410.40950797)
\curveto(669.7499703,410.31575816)(670.17184445,410.22982084)(670.50778127,410.15169599)
\curveto(671.5312167,409.91732147)(672.3202776,409.51497853)(672.87496398,408.94466719)
\curveto(673.42965035,408.36654336)(673.70699354,407.60091991)(673.70699354,406.64779684)
\closepath
}
}
{
\newrgbcolor{curcolor}{0 0 0}
\pscustom[linestyle=none,fillstyle=solid,fillcolor=curcolor]
{
\newpath
\moveto(684.67965809,409.41341624)
\curveto(684.67965809,407.28060806)(684.1327842,405.59701772)(683.03903641,404.36264522)
\curveto(681.94528863,403.12827272)(680.48044784,402.51108647)(678.64451406,402.51108647)
\curveto(676.79295531,402.51108647)(675.32030204,403.12827272)(674.22655426,404.36264522)
\curveto(673.14061896,405.59701772)(672.59765131,407.28060806)(672.59765131,409.41341624)
\curveto(672.59765131,411.54622442)(673.14061896,413.22981476)(674.22655426,414.46418726)
\curveto(675.32030204,415.70637225)(676.79295531,416.32746474)(678.64451406,416.32746474)
\curveto(680.48044784,416.32746474)(681.94528863,415.70637225)(683.03903641,414.46418726)
\curveto(684.1327842,413.22981476)(684.67965809,411.54622442)(684.67965809,409.41341624)
\closepath
\moveto(682.40622519,409.41341624)
\curveto(682.40622519,411.10872531)(682.07419462,412.36653526)(681.41013346,413.1868461)
\curveto(680.7460723,414.01496942)(679.82419917,414.42903108)(678.64451406,414.42903108)
\curveto(677.44920398,414.42903108)(676.51951836,414.01496942)(675.85545721,413.1868461)
\curveto(675.19920854,412.36653526)(674.8710842,411.10872531)(674.8710842,409.41341624)
\curveto(674.8710842,407.77279456)(675.20311478,406.52670333)(675.86717593,405.67514256)
\curveto(676.53123709,404.83139427)(677.45701646,404.40952012)(678.64451406,404.40952012)
\curveto(679.81638669,404.40952012)(680.73435358,404.82748803)(681.39841473,405.66342383)
\curveto(682.07028837,406.50717212)(682.40622519,407.75716959)(682.40622519,409.41341624)
\closepath
}
}
{
\newrgbcolor{curcolor}{0 0 0}
\pscustom[linestyle=none,fillstyle=solid,fillcolor=curcolor]
{
\newpath
\moveto(705.32092794,402.87436698)
\lineto(703.1178074,402.87436698)
\lineto(703.1178074,410.32747689)
\curveto(703.1178074,410.88997575)(703.0904637,411.4329434)(703.03577631,411.95637984)
\curveto(702.98890141,412.47981628)(702.88343287,412.89778418)(702.7193707,413.21028355)
\curveto(702.53968357,413.54622037)(702.28187159,413.80012611)(701.94593477,413.97200076)
\curveto(701.60999795,414.14387541)(701.12562393,414.22981274)(700.49281271,414.22981274)
\curveto(699.87562646,414.22981274)(699.25844021,414.07356305)(698.64125396,413.76106368)
\curveto(698.02406771,413.4563768)(697.40688146,413.06575259)(696.78969521,412.58919106)
\curveto(696.81313266,412.40950392)(696.83266387,412.19856685)(696.84828884,411.95637984)
\curveto(696.86391381,411.72200531)(696.8717263,411.48763079)(696.8717263,411.25325626)
\lineto(696.8717263,402.87436698)
\lineto(694.66860576,402.87436698)
\lineto(694.66860576,410.32747689)
\curveto(694.66860576,410.90560072)(694.64126206,411.45247461)(694.58657467,411.96809857)
\curveto(694.53969977,412.49153501)(694.43423123,412.90950291)(694.27016906,413.22200228)
\curveto(694.09048193,413.5579391)(693.83266995,413.80793859)(693.49673313,413.97200076)
\curveto(693.16079631,414.14387541)(692.67642229,414.22981274)(692.04361107,414.22981274)
\curveto(691.44204979,414.22981274)(690.83658227,414.08137554)(690.2272085,413.78450114)
\curveto(689.62564722,413.48762674)(689.02408594,413.10872126)(688.42252466,412.64778469)
\lineto(688.42252466,402.87436698)
\lineto(686.21940412,402.87436698)
\lineto(686.21940412,415.96418422)
\lineto(688.42252466,415.96418422)
\lineto(688.42252466,414.51106217)
\curveto(689.11002326,415.08137351)(689.79361563,415.52668511)(690.47330175,415.84699696)
\curveto(691.16080036,416.16730881)(691.89126763,416.32746474)(692.66470356,416.32746474)
\curveto(693.55532676,416.32746474)(694.30923148,416.13996512)(694.92641773,415.76496588)
\curveto(695.55141647,415.38996664)(696.01625928,414.87043644)(696.32094616,414.20637528)
\curveto(697.21156936,414.95637376)(698.02406771,415.49543517)(698.75844122,415.82355951)
\curveto(699.49281474,416.15949633)(700.2779694,416.32746474)(701.11390521,416.32746474)
\curveto(702.55140229,416.32746474)(703.6099939,415.88996562)(704.28968002,415.0149674)
\curveto(704.97717863,414.14778165)(705.32092794,412.93294036)(705.32092794,411.37044353)
\closepath
}
}
{
\newrgbcolor{curcolor}{0 0 0}
\pscustom[linestyle=none,fillstyle=solid,fillcolor=curcolor]
{
\newpath
\moveto(717.28837959,409.19076044)
\lineto(707.64386787,409.19076044)
\curveto(707.64386787,408.38607457)(707.76496138,407.68295099)(708.00714839,407.08138971)
\curveto(708.2493354,406.48764091)(708.58136597,405.99936065)(709.00324012,405.61654893)
\curveto(709.4094893,405.24154969)(709.88995707,404.96030026)(710.44464345,404.77280064)
\curveto(711.00714231,404.58530102)(711.62432856,404.49155121)(712.2962022,404.49155121)
\curveto(713.1868254,404.49155121)(714.08135484,404.6673321)(714.97979052,405.01889389)
\curveto(715.88603868,405.37826816)(716.53056863,405.72982995)(716.91338035,406.07357925)
\lineto(717.03056761,406.07357925)
\lineto(717.03056761,403.67124037)
\curveto(716.28838162,403.358741)(715.53057065,403.09702278)(714.75713472,402.88608571)
\curveto(713.98369878,402.67514863)(713.17120043,402.5696801)(712.31963965,402.5696801)
\curveto(710.14776905,402.5696801)(708.45245999,403.15561641)(707.23371245,404.32748904)
\curveto(706.01496492,405.50717415)(705.40559116,407.17904576)(705.40559116,409.34310388)
\curveto(705.40559116,411.48372455)(705.98762123,413.18293986)(707.15168137,414.44074981)
\curveto(708.323554,415.69855976)(709.86261338,416.32746474)(711.76885952,416.32746474)
\curveto(713.53448094,416.32746474)(714.89385319,415.81184078)(715.84697626,414.78059287)
\curveto(716.80791182,413.74934496)(717.28837959,412.28450417)(717.28837959,410.38607052)
\closepath
\moveto(715.14385269,410.87825702)
\curveto(715.1360402,412.03450468)(714.84307204,412.92903412)(714.26494822,413.56184534)
\curveto(713.69463687,414.19465656)(712.82354488,414.51106217)(711.65167226,414.51106217)
\curveto(710.47198715,414.51106217)(709.5305828,414.16340662)(708.82745923,413.46809553)
\curveto(708.13214813,412.77278444)(707.73761768,411.90950493)(707.64386787,410.87825702)
\closepath
}
}
{
\newrgbcolor{curcolor}{0 0 0}
\pscustom[linestyle=none,fillstyle=solid,fillcolor=curcolor]
{
\newpath
\moveto(734.86683521,409.51888477)
\curveto(734.86683521,408.42513699)(734.71058552,407.44076398)(734.39808616,406.56576576)
\curveto(734.09339927,405.69076753)(733.67933761,404.95639401)(733.15590117,404.36264522)
\curveto(732.6012148,403.74545897)(731.99184103,403.28061616)(731.32777987,402.96811679)
\curveto(730.66371872,402.66342991)(729.93325145,402.51108647)(729.13637806,402.51108647)
\curveto(728.39419206,402.51108647)(727.74575588,402.60093003)(727.1910695,402.78061717)
\curveto(726.63638312,402.95249182)(726.08950923,403.18686635)(725.55044782,403.48374075)
\lineto(725.40982311,402.87436698)
\lineto(723.34732728,402.87436698)
\lineto(723.34732728,421.10870506)
\lineto(725.55044782,421.10870506)
\lineto(725.55044782,414.59309325)
\curveto(726.16763407,415.10090472)(726.82388274,415.51496638)(727.51919384,415.83527823)
\curveto(728.21450493,416.16340257)(728.99575335,416.32746474)(729.86293909,416.32746474)
\curveto(731.40981096,416.32746474)(732.62855849,415.73371594)(733.51918169,414.54621834)
\curveto(734.41761737,413.35872075)(734.86683521,411.68294289)(734.86683521,409.51888477)
\closepath
\moveto(732.59340231,409.46029114)
\curveto(732.59340231,411.02278798)(732.33559033,412.20637933)(731.81996638,413.0110652)
\curveto(731.30434242,413.82356356)(730.47231286,414.22981274)(729.32387768,414.22981274)
\curveto(728.68325398,414.22981274)(728.03481779,414.08918802)(727.37856912,413.80793859)
\curveto(726.72232045,413.53450164)(726.11294668,413.17903361)(725.55044782,412.7415345)
\lineto(725.55044782,405.24154969)
\curveto(726.17544656,404.96030026)(726.71060172,404.76498815)(727.15591332,404.65561337)
\curveto(727.6090374,404.54623859)(728.12075512,404.49155121)(728.69106646,404.49155121)
\curveto(729.909814,404.49155121)(730.86293707,404.8899879)(731.55043567,405.68686128)
\curveto(732.24574676,406.49154716)(732.59340231,407.74935711)(732.59340231,409.46029114)
\closepath
}
}
{
\newrgbcolor{curcolor}{0 0 0}
\pscustom[linestyle=none,fillstyle=solid,fillcolor=curcolor]
{
\newpath
\moveto(738.43785764,402.87436698)
\lineto(736.2347371,402.87436698)
\lineto(736.2347371,421.10870506)
\lineto(738.43785764,421.10870506)
\closepath
}
}
{
\newrgbcolor{curcolor}{0 0 0}
\pscustom[linestyle=none,fillstyle=solid,fillcolor=curcolor]
{
\newpath
\moveto(753.89551372,409.41341624)
\curveto(753.89551372,407.28060806)(753.34863982,405.59701772)(752.25489204,404.36264522)
\curveto(751.16114425,403.12827272)(749.69630347,402.51108647)(747.86036969,402.51108647)
\curveto(746.00881094,402.51108647)(744.53615767,403.12827272)(743.44240988,404.36264522)
\curveto(742.35647458,405.59701772)(741.81350693,407.28060806)(741.81350693,409.41341624)
\curveto(741.81350693,411.54622442)(742.35647458,413.22981476)(743.44240988,414.46418726)
\curveto(744.53615767,415.70637225)(746.00881094,416.32746474)(747.86036969,416.32746474)
\curveto(749.69630347,416.32746474)(751.16114425,415.70637225)(752.25489204,414.46418726)
\curveto(753.34863982,413.22981476)(753.89551372,411.54622442)(753.89551372,409.41341624)
\closepath
\moveto(751.62208082,409.41341624)
\curveto(751.62208082,411.10872531)(751.29005024,412.36653526)(750.62598909,413.1868461)
\curveto(749.96192793,414.01496942)(749.0400548,414.42903108)(747.86036969,414.42903108)
\curveto(746.66505961,414.42903108)(745.73537399,414.01496942)(745.07131283,413.1868461)
\curveto(744.41506416,412.36653526)(744.08693983,411.10872531)(744.08693983,409.41341624)
\curveto(744.08693983,407.77279456)(744.41897041,406.52670333)(745.08303156,405.67514256)
\curveto(745.74709272,404.83139427)(746.67287209,404.40952012)(747.86036969,404.40952012)
\curveto(749.03224231,404.40952012)(749.9502092,404.82748803)(750.61427036,405.66342383)
\curveto(751.286144,406.50717212)(751.62208082,407.75716959)(751.62208082,409.41341624)
\closepath
}
}
{
\newrgbcolor{curcolor}{0 0 0}
\pscustom[linestyle=none,fillstyle=solid,fillcolor=curcolor]
{
\newpath
\moveto(765.11492764,403.69467782)
\curveto(764.38055413,403.34311603)(763.68133679,403.06967908)(763.01727564,402.87436698)
\curveto(762.36102697,402.67905488)(761.66180963,402.58139882)(760.91962364,402.58139882)
\curveto(759.97431305,402.58139882)(759.10712731,402.7181173)(758.3180664,402.99155424)
\curveto(757.5290055,403.27280367)(756.85322562,403.69467782)(756.29072676,404.25717668)
\curveto(755.72041542,404.81967554)(755.27901006,405.5306116)(754.96651069,406.38998486)
\curveto(754.65401132,407.24935812)(754.49776164,408.25326234)(754.49776164,409.40169751)
\curveto(754.49776164,411.54231818)(755.08369795,413.22200228)(756.25557058,414.44074981)
\curveto(757.43525569,415.65949734)(758.98994004,416.26887111)(760.91962364,416.26887111)
\curveto(761.66962212,416.26887111)(762.40399563,416.16340257)(763.12274418,415.9524655)
\curveto(763.8493052,415.74152842)(764.51336636,415.48371645)(765.11492764,415.17902956)
\lineto(765.11492764,412.72981577)
\lineto(764.99774038,412.72981577)
\curveto(764.32586674,413.25325221)(763.63055565,413.65559515)(762.9118071,413.93684458)
\curveto(762.20087104,414.21809401)(761.50555995,414.35871872)(760.82587383,414.35871872)
\curveto(759.57587636,414.35871872)(758.58759711,413.93684458)(757.86103608,413.09309629)
\curveto(757.14228754,412.25716048)(756.78291326,411.02669422)(756.78291326,409.40169751)
\curveto(756.78291326,407.82357571)(757.13447505,406.60873442)(757.83759863,405.75717364)
\curveto(758.54853469,404.91342535)(759.54462642,404.49155121)(760.82587383,404.49155121)
\curveto(761.27118542,404.49155121)(761.72430951,404.55014484)(762.18524607,404.6673321)
\curveto(762.64618264,404.78451936)(763.0602443,404.9368628)(763.42743106,405.12436242)
\curveto(763.74774291,405.28842459)(764.04852355,405.46029924)(764.32977298,405.63998638)
\curveto(764.61102241,405.827486)(764.83367821,405.98764193)(764.99774038,406.12045416)
\lineto(765.11492764,406.12045416)
\closepath
}
}
{
\newrgbcolor{curcolor}{0 0 0}
\pscustom[linestyle=none,fillstyle=solid,fillcolor=curcolor]
{
\newpath
\moveto(777.95003043,402.87436698)
\lineto(775.04378632,402.87436698)
\lineto(769.79379695,408.60482413)
\lineto(768.36411235,407.24545188)
\lineto(768.36411235,402.87436698)
\lineto(766.16099181,402.87436698)
\lineto(766.16099181,421.10870506)
\lineto(768.36411235,421.10870506)
\lineto(768.36411235,409.41341624)
\lineto(774.72738071,415.96418422)
\lineto(777.50471884,415.96418422)
\lineto(771.4226999,409.91732147)
\closepath
}
}
{
\newrgbcolor{curcolor}{0 0 0}
\pscustom[linestyle=none,fillstyle=solid,fillcolor=curcolor]
{
\newpath
\moveto(787.82676519,406.64779684)
\curveto(787.82676519,405.45248676)(787.33067245,404.47201999)(786.33848696,403.70639655)
\curveto(785.35411395,402.9407731)(784.00646043,402.55796137)(782.2955264,402.55796137)
\curveto(781.32677836,402.55796137)(780.43615516,402.67124239)(779.62365681,402.89780443)
\curveto(778.81897094,403.13217896)(778.14319105,403.38608469)(777.59631716,403.65952164)
\lineto(777.59631716,406.13217288)
\lineto(777.71350442,406.13217288)
\curveto(778.40881552,405.60873644)(779.18225145,405.19076854)(780.03381223,404.87826917)
\curveto(780.885373,404.57358229)(781.7017776,404.42123885)(782.48302602,404.42123885)
\curveto(783.45177405,404.42123885)(784.20958502,404.57748853)(784.75645891,404.8899879)
\curveto(785.3033328,405.20248727)(785.57676975,405.69467377)(785.57676975,406.36654741)
\curveto(785.57676975,406.88217136)(785.42833255,407.27279557)(785.13145815,407.53842004)
\curveto(784.83458375,407.8040445)(784.26427241,408.03060654)(783.42052412,408.21810616)
\curveto(783.10802475,408.28841852)(782.69786933,408.3704496)(782.19005786,408.46419941)
\curveto(781.69005887,408.55794922)(781.23302855,408.65951152)(780.81896689,408.76888629)
\curveto(779.67053171,409.07357318)(778.85412711,409.51888477)(778.3697531,410.10482109)
\curveto(777.89319156,410.69856989)(777.65491079,411.42513091)(777.65491079,412.28450417)
\curveto(777.65491079,412.82356558)(777.76428557,413.33137705)(777.98303513,413.80793859)
\curveto(778.20959717,414.28450012)(778.54944023,414.71028051)(779.00256431,415.08527975)
\curveto(779.44006343,415.45246651)(779.9947498,415.74152842)(780.66662344,415.9524655)
\curveto(781.34630957,416.17121505)(782.10412053,416.28058983)(782.94005634,416.28058983)
\curveto(783.72130476,416.28058983)(784.51036566,416.18293378)(785.30723905,415.98762168)
\curveto(786.11192492,415.80012206)(786.77989231,415.56965377)(787.31114124,415.29621683)
\lineto(787.31114124,412.94075285)
\lineto(787.19395398,412.94075285)
\curveto(786.63145512,413.35481451)(785.94786275,413.70247005)(785.14317688,413.98371948)
\curveto(784.33849101,414.2727814)(783.54943011,414.41731236)(782.77599417,414.41731236)
\curveto(781.9713083,414.41731236)(781.29162218,414.26106267)(780.7369358,413.9485633)
\curveto(780.18224942,413.64387642)(779.90490624,413.1868461)(779.90490624,412.57747233)
\curveto(779.90490624,412.03841092)(780.07287465,411.63216175)(780.40881147,411.3587248)
\curveto(780.7369358,411.08528785)(781.26818473,410.86263205)(782.00255824,410.6907574)
\curveto(782.40880742,410.59700759)(782.8619315,410.50325778)(783.36193049,410.40950797)
\curveto(783.86974196,410.31575816)(784.2916161,410.22982084)(784.62755292,410.15169599)
\curveto(785.65098835,409.91732147)(786.44004925,409.51497853)(786.99473563,408.94466719)
\curveto(787.54942201,408.36654336)(787.82676519,407.60091991)(787.82676519,406.64779684)
\closepath
}
}
{
\newrgbcolor{curcolor}{0 0 0}
\pscustom[linestyle=none,fillstyle=solid,fillcolor=curcolor,opacity=0]
{
\newpath
\moveto(469.83179946,201.26497497)
\lineto(894.36639909,201.26497497)
\lineto(894.36639909,169.29656202)
\lineto(469.83179946,169.29656202)
\closepath
}
}
{
\newrgbcolor{curcolor}{0 0 0}
\pscustom[linestyle=none,fillstyle=solid,fillcolor=curcolor]
{
\newpath
\moveto(514.26722026,186.48352654)
\lineto(511.67738175,186.48352654)
\lineto(511.67738175,181.57338023)
\lineto(509.42738631,181.57338023)
\lineto(509.42738631,186.48352654)
\lineto(501.07193448,186.48352654)
\lineto(501.07193448,189.17883358)
\lineto(509.52113612,199.02256365)
\lineto(511.67738175,199.02256365)
\lineto(511.67738175,188.35852274)
\lineto(514.26722026,188.35852274)
\closepath
\moveto(509.42738631,188.35852274)
\lineto(509.42738631,196.2335068)
\lineto(502.66568125,188.35852274)
\closepath
}
}
{
\newrgbcolor{curcolor}{0 0 0}
\pscustom[linestyle=none,fillstyle=solid,fillcolor=curcolor]
{
\newpath
\moveto(521.19298748,181.57338023)
\lineto(518.3922119,181.57338023)
\lineto(518.3922119,184.91321722)
\lineto(521.19298748,184.91321722)
\closepath
}
}
{
\newrgbcolor{curcolor}{0 0 0}
\pscustom[linestyle=none,fillstyle=solid,fillcolor=curcolor]
{
\newpath
\moveto(549.37652416,181.57338023)
\lineto(546.36481151,181.57338023)
\lineto(540.52888583,188.51086619)
\lineto(537.2593612,188.51086619)
\lineto(537.2593612,181.57338023)
\lineto(534.9390534,181.57338023)
\lineto(534.9390534,199.02256365)
\lineto(539.82576225,199.02256365)
\curveto(540.88044762,199.02256365)(541.75935209,198.95225129)(542.46247566,198.81162658)
\curveto(543.16559924,198.67881435)(543.79841046,198.43662734)(544.36090932,198.08506555)
\curveto(544.99372054,197.68662885)(545.48590704,197.18272362)(545.83746883,196.57334986)
\curveto(546.1968431,195.97178858)(546.37653024,195.20616513)(546.37653024,194.27647951)
\curveto(546.37653024,193.01866956)(546.06012463,191.96398419)(545.42731341,191.11242342)
\curveto(544.79450219,190.26867513)(543.9234102,189.63195766)(542.81403745,189.20227104)
\closepath
\moveto(543.9507539,194.11241734)
\curveto(543.9507539,194.61241633)(543.86091033,195.05382169)(543.68122319,195.43663341)
\curveto(543.50934854,195.82725762)(543.22028663,196.15538196)(542.81403745,196.42100642)
\curveto(542.47810063,196.64756846)(542.07966394,196.80381814)(541.61872737,196.88975547)
\curveto(541.1577908,196.98350528)(540.61482315,197.03038018)(539.98982442,197.03038018)
\lineto(537.2593612,197.03038018)
\lineto(537.2593612,190.44445602)
\lineto(539.60310645,190.44445602)
\curveto(540.33747996,190.44445602)(540.97810367,190.50695589)(541.52497756,190.63195564)
\curveto(542.07185145,190.76476787)(542.53669426,191.00695488)(542.91950599,191.35851667)
\curveto(543.27106777,191.686641)(543.52887975,192.06164024)(543.69294192,192.48351439)
\curveto(543.86481657,192.91320102)(543.9507539,193.45616867)(543.9507539,194.11241734)
\closepath
}
}
{
\newrgbcolor{curcolor}{0 0 0}
\pscustom[linestyle=none,fillstyle=solid,fillcolor=curcolor]
{
\newpath
\moveto(561.82181146,187.88977369)
\lineto(552.17729974,187.88977369)
\curveto(552.17729974,187.08508782)(552.29839324,186.38196425)(552.54058025,185.78040296)
\curveto(552.78276726,185.18665417)(553.11479784,184.69837391)(553.53667199,184.31556218)
\curveto(553.94292116,183.94056294)(554.42338894,183.65931351)(554.97807532,183.47181389)
\curveto(555.54057418,183.28431427)(556.15776043,183.19056446)(556.82963407,183.19056446)
\curveto(557.72025726,183.19056446)(558.6147867,183.36634535)(559.51322238,183.71790714)
\curveto(560.41947055,184.07728141)(561.06400049,184.4288432)(561.44681222,184.77259251)
\lineto(561.56399948,184.77259251)
\lineto(561.56399948,182.37025362)
\curveto(560.82181348,182.05775425)(560.06400252,181.79603603)(559.29056658,181.58509896)
\curveto(558.51713065,181.37416189)(557.7046323,181.26869335)(556.85307152,181.26869335)
\curveto(554.68120092,181.26869335)(552.98589185,181.85462966)(551.76714432,183.02650229)
\curveto(550.54839679,184.2061874)(549.93902302,185.87805902)(549.93902302,188.04211713)
\curveto(549.93902302,190.1827378)(550.52105309,191.88195311)(551.68511324,193.13976306)
\curveto(552.85698586,194.39757301)(554.39604525,195.02647799)(556.30229139,195.02647799)
\curveto(558.06791281,195.02647799)(559.42728506,194.51085404)(560.38040813,193.47960612)
\curveto(561.34134368,192.44835821)(561.82181146,190.98351743)(561.82181146,189.08508377)
\closepath
\moveto(559.67728455,189.57727028)
\curveto(559.66947207,190.73351793)(559.37650391,191.62804737)(558.79838008,192.26085859)
\curveto(558.22806874,192.89366981)(557.35697675,193.21007542)(556.18510412,193.21007542)
\curveto(555.00541901,193.21007542)(554.06401467,192.86241987)(553.36089109,192.16710878)
\curveto(552.66558,191.47179769)(552.27104955,190.60851819)(552.17729974,189.57727028)
\closepath
}
}
{
\newrgbcolor{curcolor}{0 0 0}
\pscustom[linestyle=none,fillstyle=solid,fillcolor=curcolor]
{
\newpath
\moveto(581.62645885,194.66319748)
\lineto(578.21630951,181.57338023)
\lineto(576.17725114,181.57338023)
\lineto(572.8139767,191.66320355)
\lineto(569.47413971,181.57338023)
\lineto(567.44680007,181.57338023)
\lineto(564.00149454,194.66319748)
\lineto(566.29836489,194.66319748)
\lineto(568.70070378,184.52649925)
\lineto(571.97022841,194.66319748)
\lineto(573.78663098,194.66319748)
\lineto(577.13818669,184.52649925)
\lineto(579.41161959,194.66319748)
\closepath
}
}
{
\newrgbcolor{curcolor}{0 0 0}
\pscustom[linestyle=none,fillstyle=solid,fillcolor=curcolor]
{
\newpath
\moveto(592.97018588,192.26085859)
\lineto(592.85299862,192.26085859)
\curveto(592.52487428,192.33898343)(592.20456243,192.39367082)(591.89206307,192.42492076)
\curveto(591.58737618,192.46398318)(591.22409567,192.48351439)(590.80222152,192.48351439)
\curveto(590.1225354,192.48351439)(589.46628673,192.33117095)(588.83347551,192.02648407)
\curveto(588.20066429,191.72960967)(587.59129053,191.3428917)(587.00535421,190.86633017)
\lineto(587.00535421,181.57338023)
\lineto(584.80223367,181.57338023)
\lineto(584.80223367,194.66319748)
\lineto(587.00535421,194.66319748)
\lineto(587.00535421,192.72960764)
\curveto(587.88035244,193.43273122)(588.64988213,193.92882396)(589.31394329,194.21788588)
\curveto(589.98581693,194.51476028)(590.66940929,194.66319748)(591.36472038,194.66319748)
\curveto(591.74753211,194.66319748)(592.0248753,194.65147875)(592.19674995,194.6280413)
\curveto(592.3686246,194.61241633)(592.62643658,194.57726015)(592.97018588,194.52257276)
\closepath
}
}
{
\newrgbcolor{curcolor}{0 0 0}
\pscustom[linestyle=none,fillstyle=solid,fillcolor=curcolor]
{
\newpath
\moveto(597.41158314,196.85459929)
\lineto(594.92721317,196.85459929)
\lineto(594.92721317,199.13975091)
\lineto(597.41158314,199.13975091)
\closepath
\moveto(597.27095842,181.57338023)
\lineto(595.06783789,181.57338023)
\lineto(595.06783789,194.66319748)
\lineto(597.27095842,194.66319748)
\closepath
}
}
{
\newrgbcolor{curcolor}{0 0 0}
\pscustom[linestyle=none,fillstyle=solid,fillcolor=curcolor]
{
\newpath
\moveto(608.42718583,181.6905675)
\curveto(608.01312417,181.58119272)(607.56000009,181.49134915)(607.06781359,181.42103679)
\curveto(606.58343957,181.35072443)(606.14984669,181.31556826)(605.76703497,181.31556826)
\curveto(604.43110017,181.31556826)(603.41547723,181.67494253)(602.72016614,182.39369107)
\curveto(602.02485505,183.11243962)(601.6771995,184.26478103)(601.6771995,185.85071532)
\lineto(601.6771995,192.81163873)
\lineto(600.18892127,192.81163873)
\lineto(600.18892127,194.66319748)
\lineto(601.6771995,194.66319748)
\lineto(601.6771995,198.42490861)
\lineto(603.88032004,198.42490861)
\lineto(603.88032004,194.66319748)
\lineto(608.42718583,194.66319748)
\lineto(608.42718583,192.81163873)
\lineto(603.88032004,192.81163873)
\lineto(603.88032004,186.84680705)
\curveto(603.88032004,186.15930845)(603.89594501,185.62024704)(603.92719495,185.22962283)
\curveto(603.95844488,184.8468111)(604.06781966,184.48743683)(604.25531928,184.15150001)
\curveto(604.42719393,183.83900065)(604.66156846,183.60853236)(604.95844286,183.46009516)
\curveto(605.26312974,183.31947045)(605.72406631,183.24915809)(606.34125256,183.24915809)
\curveto(606.70062683,183.24915809)(607.07562607,183.29993924)(607.46625028,183.40150153)
\curveto(607.85687449,183.51087631)(608.13812392,183.60071988)(608.30999857,183.67103224)
\lineto(608.42718583,183.67103224)
\closepath
}
}
{
\newrgbcolor{curcolor}{0 0 0}
\pscustom[linestyle=none,fillstyle=solid,fillcolor=curcolor]
{
\newpath
\moveto(622.04434576,187.88977369)
\lineto(612.39983404,187.88977369)
\curveto(612.39983404,187.08508782)(612.52092754,186.38196425)(612.76311455,185.78040296)
\curveto(613.00530156,185.18665417)(613.33733214,184.69837391)(613.75920629,184.31556218)
\curveto(614.16545546,183.94056294)(614.64592324,183.65931351)(615.20060962,183.47181389)
\curveto(615.76310848,183.28431427)(616.38029473,183.19056446)(617.05216837,183.19056446)
\curveto(617.94279156,183.19056446)(618.837321,183.36634535)(619.73575668,183.71790714)
\curveto(620.64200485,184.07728141)(621.28653479,184.4288432)(621.66934652,184.77259251)
\lineto(621.78653378,184.77259251)
\lineto(621.78653378,182.37025362)
\curveto(621.04434778,182.05775425)(620.28653682,181.79603603)(619.51310088,181.58509896)
\curveto(618.73966495,181.37416189)(617.9271666,181.26869335)(617.07560582,181.26869335)
\curveto(614.90373522,181.26869335)(613.20842615,181.85462966)(611.98967862,183.02650229)
\curveto(610.77093109,184.2061874)(610.16155732,185.87805902)(610.16155732,188.04211713)
\curveto(610.16155732,190.1827378)(610.74358739,191.88195311)(611.90764753,193.13976306)
\curveto(613.07952016,194.39757301)(614.61857955,195.02647799)(616.52482568,195.02647799)
\curveto(618.29044711,195.02647799)(619.64981936,194.51085404)(620.60294243,193.47960612)
\curveto(621.56387798,192.44835821)(622.04434576,190.98351743)(622.04434576,189.08508377)
\closepath
\moveto(619.89981885,189.57727028)
\curveto(619.89200637,190.73351793)(619.59903821,191.62804737)(619.02091438,192.26085859)
\curveto(618.45060304,192.89366981)(617.57951105,193.21007542)(616.40763842,193.21007542)
\curveto(615.22795331,193.21007542)(614.28654897,192.86241987)(613.58342539,192.16710878)
\curveto(612.8881143,191.47179769)(612.49358385,190.60851819)(612.39983404,189.57727028)
\closepath
}
}
{
\newrgbcolor{curcolor}{0 0 0}
\pscustom[linestyle=none,fillstyle=solid,fillcolor=curcolor]
{
\newpath
\moveto(643.17320922,185.34681009)
\curveto(643.17320922,184.15150001)(642.67711648,183.17103325)(641.68493099,182.4054098)
\curveto(640.70055798,181.63978635)(639.35290446,181.25697462)(637.64197042,181.25697462)
\curveto(636.67322238,181.25697462)(635.78259919,181.37025564)(634.97010083,181.59681769)
\curveto(634.16541496,181.83119221)(633.48963508,182.08509795)(632.94276119,182.35853489)
\lineto(632.94276119,184.83118614)
\lineto(633.05994845,184.83118614)
\curveto(633.75525954,184.3077497)(634.52869548,183.88978179)(635.38025625,183.57728243)
\curveto(636.23181703,183.27259554)(637.04822162,183.1202521)(637.82947004,183.1202521)
\curveto(638.79821808,183.1202521)(639.55602905,183.27650178)(640.10290294,183.58900115)
\curveto(640.64977683,183.90150052)(640.92321378,184.39368702)(640.92321378,185.06556066)
\curveto(640.92321378,185.58118462)(640.77477658,185.97180883)(640.47790218,186.23743329)
\curveto(640.18102778,186.50305775)(639.61071644,186.72961979)(638.76696814,186.91711941)
\curveto(638.45446878,186.98743177)(638.04431336,187.06946285)(637.53650189,187.16321266)
\curveto(637.0365029,187.25696247)(636.57947257,187.35852477)(636.16541091,187.46789955)
\curveto(635.01697574,187.77258643)(634.20057114,188.21789803)(633.71619712,188.80383434)
\curveto(633.23963559,189.39758314)(633.00135482,190.12414417)(633.00135482,190.98351743)
\curveto(633.00135482,191.52257884)(633.1107296,192.03039031)(633.32947916,192.50695184)
\curveto(633.5560412,192.98351338)(633.89588426,193.40929377)(634.34900834,193.78429301)
\curveto(634.78650745,194.15147976)(635.34119383,194.44054168)(636.01306747,194.65147875)
\curveto(636.69275359,194.87022831)(637.45056456,194.97960309)(638.28650037,194.97960309)
\curveto(639.06774879,194.97960309)(639.85680969,194.88194703)(640.65368307,194.68663493)
\curveto(641.45836894,194.49913531)(642.12633634,194.26866703)(642.65758527,193.99523008)
\lineto(642.65758527,191.6397661)
\lineto(642.540398,191.6397661)
\curveto(641.97789914,192.05382776)(641.29430678,192.40148331)(640.48962091,192.68273274)
\curveto(639.68493504,192.97179465)(638.89587413,193.11632561)(638.1224382,193.11632561)
\curveto(637.31775233,193.11632561)(636.63806621,192.96007593)(636.08337983,192.64757656)
\curveto(635.52869345,192.34288968)(635.25135026,191.88585935)(635.25135026,191.27648558)
\curveto(635.25135026,190.73742418)(635.41931867,190.331175)(635.75525549,190.05773805)
\curveto(636.08337983,189.78430111)(636.61462875,189.56164531)(637.34900227,189.38977066)
\curveto(637.75525144,189.29602085)(638.20837553,189.20227104)(638.70837451,189.10852122)
\curveto(639.21618598,189.01477141)(639.63806013,188.92883409)(639.97399695,188.85070925)
\curveto(640.99743238,188.61633472)(641.78649328,188.21399179)(642.34117966,187.64368044)
\curveto(642.89586603,187.06555661)(643.17320922,186.29993316)(643.17320922,185.34681009)
\closepath
}
}
{
\newrgbcolor{curcolor}{0 0 0}
\pscustom[linestyle=none,fillstyle=solid,fillcolor=curcolor]
{
\newpath
\moveto(657.17708711,181.57338023)
\lineto(654.97396658,181.57338023)
\lineto(654.97396658,183.02650229)
\curveto(654.23178058,182.44056598)(653.52084452,181.99134814)(652.84115839,181.67884877)
\curveto(652.16147227,181.3663494)(651.41147379,181.21009972)(650.59116295,181.21009972)
\curveto(649.21616573,181.21009972)(648.1458554,181.62806762)(647.38023195,182.46400343)
\curveto(646.6146085,183.30775172)(646.23179678,184.54212422)(646.23179678,186.16712093)
\lineto(646.23179678,194.66319748)
\lineto(648.43491732,194.66319748)
\lineto(648.43491732,187.21008757)
\curveto(648.43491732,186.54602641)(648.46616725,185.97571507)(648.52866713,185.49915353)
\curveto(648.591167,185.03040448)(648.72397923,184.62806155)(648.92710382,184.29212473)
\curveto(649.13804089,183.94837542)(649.41147784,183.69837593)(649.74741466,183.54212625)
\curveto(650.08335148,183.38587656)(650.57163174,183.30775172)(651.21225544,183.30775172)
\curveto(651.78256679,183.30775172)(652.40365928,183.45618892)(653.07553292,183.75306332)
\curveto(653.75521904,184.04993772)(654.38803026,184.4288432)(654.97396658,184.88977977)
\lineto(654.97396658,194.66319748)
\lineto(657.17708711,194.66319748)
\closepath
}
}
{
\newrgbcolor{curcolor}{0 0 0}
\pscustom[linestyle=none,fillstyle=solid,fillcolor=curcolor]
{
\newpath
\moveto(673.03252376,188.27649166)
\curveto(673.03252376,187.21399381)(672.88018032,186.24133953)(672.57549343,185.35852882)
\curveto(672.27080655,184.48353059)(671.84111992,183.74134459)(671.28643354,183.13197083)
\curveto(670.77080959,182.553847)(670.16143582,182.10462916)(669.45831224,181.78431731)
\curveto(668.76300115,181.47181794)(668.0247214,181.31556826)(667.24347298,181.31556826)
\curveto(666.56378686,181.31556826)(665.94660061,181.38978686)(665.39191423,181.53822405)
\curveto(664.84504034,181.68666125)(664.28644772,181.91712954)(663.71613637,182.2296289)
\lineto(663.71613637,176.74526501)
\lineto(661.51301583,176.74526501)
\lineto(661.51301583,194.66319748)
\lineto(663.71613637,194.66319748)
\lineto(663.71613637,193.2921065)
\curveto(664.30207269,193.78429301)(664.95832136,194.19444843)(665.68488239,194.52257276)
\curveto(666.4192559,194.85850958)(667.20050432,195.02647799)(668.02862764,195.02647799)
\curveto(669.60674944,195.02647799)(670.83330946,194.42882295)(671.70830769,193.23351287)
\curveto(672.5911184,192.04601528)(673.03252376,190.39367487)(673.03252376,188.27649166)
\closepath
\moveto(670.75909086,188.21789803)
\curveto(670.75909086,189.79601983)(670.48956016,190.97570494)(669.95049875,191.75695336)
\curveto(669.41143734,192.53820178)(668.58331402,192.92882599)(667.46612878,192.92882599)
\curveto(666.83331756,192.92882599)(666.1966001,192.79210752)(665.5559764,192.51867057)
\curveto(664.91535269,192.24523362)(664.30207269,191.88585935)(663.71613637,191.44054775)
\lineto(663.71613637,184.02259402)
\curveto(664.34113511,183.74134459)(664.87629027,183.54993873)(665.32160187,183.44837644)
\curveto(665.77472595,183.34681414)(666.28644367,183.296033)(666.85675501,183.296033)
\curveto(668.08331503,183.296033)(669.04034434,183.71009466)(669.72784295,184.53821798)
\curveto(670.41534156,185.3663413)(670.75909086,186.59290132)(670.75909086,188.21789803)
\closepath
}
}
{
\newrgbcolor{curcolor}{0 0 0}
\pscustom[linestyle=none,fillstyle=solid,fillcolor=curcolor]
{
\newpath
\moveto(687.42311962,187.88977369)
\lineto(677.7786079,187.88977369)
\curveto(677.7786079,187.08508782)(677.8997014,186.38196425)(678.14188841,185.78040296)
\curveto(678.38407542,185.18665417)(678.716106,184.69837391)(679.13798014,184.31556218)
\curveto(679.54422932,183.94056294)(680.0246971,183.65931351)(680.57938347,183.47181389)
\curveto(681.14188234,183.28431427)(681.75906859,183.19056446)(682.43094222,183.19056446)
\curveto(683.32156542,183.19056446)(684.21609486,183.36634535)(685.11453054,183.71790714)
\curveto(686.02077871,184.07728141)(686.66530865,184.4288432)(687.04812037,184.77259251)
\lineto(687.16530764,184.77259251)
\lineto(687.16530764,182.37025362)
\curveto(686.42312164,182.05775425)(685.66531068,181.79603603)(684.89187474,181.58509896)
\curveto(684.11843881,181.37416189)(683.30594045,181.26869335)(682.45437968,181.26869335)
\curveto(680.28250908,181.26869335)(678.58720001,181.85462966)(677.36845248,183.02650229)
\curveto(676.14970494,184.2061874)(675.54033118,185.87805902)(675.54033118,188.04211713)
\curveto(675.54033118,190.1827378)(676.12236125,191.88195311)(677.28642139,193.13976306)
\curveto(678.45829402,194.39757301)(679.9973534,195.02647799)(681.90359954,195.02647799)
\curveto(683.66922097,195.02647799)(685.02859321,194.51085404)(685.98171628,193.47960612)
\curveto(686.94265184,192.44835821)(687.42311962,190.98351743)(687.42311962,189.08508377)
\closepath
\moveto(685.27859271,189.57727028)
\curveto(685.27078022,190.73351793)(684.97781207,191.62804737)(684.39968824,192.26085859)
\curveto(683.82937689,192.89366981)(682.95828491,193.21007542)(681.78641228,193.21007542)
\curveto(680.60672717,193.21007542)(679.66532283,192.86241987)(678.96219925,192.16710878)
\curveto(678.26688816,191.47179769)(677.87235771,190.60851819)(677.7786079,189.57727028)
\closepath
}
}
{
\newrgbcolor{curcolor}{0 0 0}
\pscustom[linestyle=none,fillstyle=solid,fillcolor=curcolor]
{
\newpath
\moveto(698.93090881,192.26085859)
\lineto(698.81372155,192.26085859)
\curveto(698.48559721,192.33898343)(698.16528536,192.39367082)(697.852786,192.42492076)
\curveto(697.54809911,192.46398318)(697.1848186,192.48351439)(696.76294445,192.48351439)
\curveto(696.08325833,192.48351439)(695.42700966,192.33117095)(694.79419844,192.02648407)
\curveto(694.16138722,191.72960967)(693.55201345,191.3428917)(692.96607714,190.86633017)
\lineto(692.96607714,181.57338023)
\lineto(690.7629566,181.57338023)
\lineto(690.7629566,194.66319748)
\lineto(692.96607714,194.66319748)
\lineto(692.96607714,192.72960764)
\curveto(693.84107537,193.43273122)(694.61060506,193.92882396)(695.27466622,194.21788588)
\curveto(695.94653986,194.51476028)(696.63013222,194.66319748)(697.32544331,194.66319748)
\curveto(697.70825504,194.66319748)(697.98559823,194.65147875)(698.15747288,194.6280413)
\curveto(698.32934753,194.61241633)(698.58715951,194.57726015)(698.93090881,194.52257276)
\closepath
}
}
{
\newrgbcolor{curcolor}{0 0 0}
\pscustom[linestyle=none,fillstyle=solid,fillcolor=curcolor]
{
\newpath
\moveto(712.52463128,188.21789803)
\curveto(712.52463128,187.12415024)(712.3683816,186.13977724)(712.05588223,185.26477901)
\curveto(711.75119535,184.38978078)(711.33713369,183.65540727)(710.81369725,183.06165847)
\curveto(710.25901087,182.44447222)(709.64963711,181.97962941)(708.98557595,181.66713004)
\curveto(708.3215148,181.36244316)(707.59104753,181.21009972)(706.79417414,181.21009972)
\curveto(706.05198814,181.21009972)(705.40355196,181.29994329)(704.84886558,181.47963042)
\curveto(704.2941792,181.65150508)(703.74730531,181.8858796)(703.2082439,182.182754)
\lineto(703.06761919,181.57338023)
\lineto(701.00512336,181.57338023)
\lineto(701.00512336,199.80771831)
\lineto(703.2082439,199.80771831)
\lineto(703.2082439,193.2921065)
\curveto(703.82543015,193.79991797)(704.48167882,194.21397964)(705.17698991,194.53429149)
\curveto(705.87230101,194.86241582)(706.65354942,195.02647799)(707.52073517,195.02647799)
\curveto(709.06760704,195.02647799)(710.28635457,194.43272919)(711.17697776,193.2452316)
\curveto(712.07541344,192.057734)(712.52463128,190.38195615)(712.52463128,188.21789803)
\closepath
\moveto(710.25119839,188.1593044)
\curveto(710.25119839,189.72180123)(709.99338641,190.90539259)(709.47776245,191.71007846)
\curveto(708.9621385,192.52257681)(708.13010893,192.92882599)(706.98167376,192.92882599)
\curveto(706.34105006,192.92882599)(705.69261387,192.78820127)(705.0363652,192.50695184)
\curveto(704.38011653,192.2335149)(703.77074276,191.87804687)(703.2082439,191.44054775)
\lineto(703.2082439,183.94056294)
\curveto(703.83324264,183.65931351)(704.3683978,183.46400141)(704.8137094,183.35462663)
\curveto(705.26683348,183.24525185)(705.7785512,183.19056446)(706.34886254,183.19056446)
\curveto(707.56761007,183.19056446)(708.52073314,183.58900115)(709.20823175,184.38587454)
\curveto(709.90354284,185.19056041)(710.25119839,186.44837036)(710.25119839,188.1593044)
\closepath
}
}
{
\newrgbcolor{curcolor}{0 0 0}
\pscustom[linestyle=none,fillstyle=solid,fillcolor=curcolor]
{
\newpath
\moveto(718.18477607,181.57338023)
\lineto(715.98165553,181.57338023)
\lineto(715.98165553,199.80771831)
\lineto(718.18477607,199.80771831)
\closepath
}
}
{
\newrgbcolor{curcolor}{0 0 0}
\pscustom[linestyle=none,fillstyle=solid,fillcolor=curcolor]
{
\newpath
\moveto(733.70036965,188.11242949)
\curveto(733.70036965,185.97962131)(733.15349576,184.29603097)(732.05974798,183.06165847)
\curveto(730.96600019,181.82728597)(729.50115941,181.21009972)(727.66522563,181.21009972)
\curveto(725.81366687,181.21009972)(724.34101361,181.82728597)(723.24726582,183.06165847)
\curveto(722.16133052,184.29603097)(721.61836287,185.97962131)(721.61836287,188.11242949)
\curveto(721.61836287,190.24523767)(722.16133052,191.92882801)(723.24726582,193.16320051)
\curveto(724.34101361,194.4053855)(725.81366687,195.02647799)(727.66522563,195.02647799)
\curveto(729.50115941,195.02647799)(730.96600019,194.4053855)(732.05974798,193.16320051)
\curveto(733.15349576,191.92882801)(733.70036965,190.24523767)(733.70036965,188.11242949)
\closepath
\moveto(731.42693676,188.11242949)
\curveto(731.42693676,189.80773856)(731.09490618,191.06554851)(730.43084502,191.88585935)
\curveto(729.76678387,192.71398267)(728.84491074,193.12804434)(727.66522563,193.12804434)
\curveto(726.46991555,193.12804434)(725.54022993,192.71398267)(724.87616877,191.88585935)
\curveto(724.2199201,191.06554851)(723.89179577,189.80773856)(723.89179577,188.11242949)
\curveto(723.89179577,186.47180781)(724.22382634,185.22571659)(724.8878875,184.37415581)
\curveto(725.55194865,183.53040752)(726.47772803,183.10853337)(727.66522563,183.10853337)
\curveto(728.83709825,183.10853337)(729.75506514,183.52650128)(730.4191263,184.36243709)
\curveto(731.09099994,185.20618538)(731.42693676,186.45618285)(731.42693676,188.11242949)
\closepath
}
}
{
\newrgbcolor{curcolor}{0 0 0}
\pscustom[linestyle=none,fillstyle=solid,fillcolor=curcolor]
{
\newpath
\moveto(746.7901869,182.39369107)
\curveto(746.05581338,182.04212928)(745.35659605,181.76869234)(744.69253489,181.57338023)
\curveto(744.03628622,181.37806813)(743.33706889,181.28041208)(742.59488289,181.28041208)
\curveto(741.64957231,181.28041208)(740.78238656,181.41713055)(739.99332566,181.6905675)
\curveto(739.20426476,181.97181693)(738.52848488,182.39369107)(737.96598602,182.95618993)
\curveto(737.39567467,183.51868879)(736.95426931,184.22962485)(736.64176995,185.08899811)
\curveto(736.32927058,185.94837137)(736.1730209,186.95227559)(736.1730209,188.10071077)
\curveto(736.1730209,190.24133143)(736.75895721,191.92101553)(737.93082984,193.13976306)
\curveto(739.11051495,194.35851059)(740.6651993,194.96788436)(742.59488289,194.96788436)
\curveto(743.34488137,194.96788436)(744.07925489,194.86241582)(744.79800343,194.65147875)
\curveto(745.52456446,194.44054168)(746.18862562,194.1827297)(746.7901869,193.87804282)
\lineto(746.7901869,191.42882903)
\lineto(746.67299963,191.42882903)
\curveto(746.00112599,191.95226547)(745.3058149,192.3546084)(744.58706636,192.63585783)
\curveto(743.8761303,192.91710726)(743.18081921,193.05773198)(742.50113308,193.05773198)
\curveto(741.25113561,193.05773198)(740.26285636,192.63585783)(739.53629534,191.79210954)
\curveto(738.81754679,190.95617373)(738.45817252,189.72570748)(738.45817252,188.10071077)
\curveto(738.45817252,186.52258896)(738.80973431,185.30774767)(739.51285788,184.4561869)
\curveto(740.22379394,183.6124386)(741.21988568,183.19056446)(742.50113308,183.19056446)
\curveto(742.94644468,183.19056446)(743.39956876,183.24915809)(743.86050533,183.36634535)
\curveto(744.3214419,183.48353262)(744.73550356,183.63587606)(745.10269031,183.82337568)
\curveto(745.42300217,183.98743785)(745.72378281,184.1593125)(746.00503224,184.33899963)
\curveto(746.28628167,184.52649925)(746.50893747,184.68665518)(746.67299963,184.81946741)
\lineto(746.7901869,184.81946741)
\closepath
}
}
{
\newrgbcolor{curcolor}{0 0 0}
\pscustom[linestyle=none,fillstyle=solid,fillcolor=curcolor]
{
\newpath
\moveto(761.49718837,181.57338023)
\lineto(758.59094425,181.57338023)
\lineto(753.34095488,187.30383738)
\lineto(751.91127028,185.94446513)
\lineto(751.91127028,181.57338023)
\lineto(749.70814974,181.57338023)
\lineto(749.70814974,199.80771831)
\lineto(751.91127028,199.80771831)
\lineto(751.91127028,188.11242949)
\lineto(758.27453864,194.66319748)
\lineto(761.05187677,194.66319748)
\lineto(754.96985783,188.61633472)
\closepath
}
}
{
\newrgbcolor{curcolor}{0 0 0}
\pscustom[linestyle=none,fillstyle=solid,fillcolor=curcolor]
{
\newpath
\moveto(779.26277739,176.74526501)
\lineto(776.57918907,176.74526501)
\curveto(775.19637937,178.3311993)(774.1221628,180.06166454)(773.35653935,181.93666075)
\curveto(772.5909159,183.81165695)(772.20810418,185.92493392)(772.20810418,188.27649166)
\curveto(772.20810418,190.6280494)(772.5909159,192.74132637)(773.35653935,194.61632257)
\curveto(774.1221628,196.49131877)(775.19637937,198.22178402)(776.57918907,199.80771831)
\lineto(779.26277739,199.80771831)
\lineto(779.26277739,199.69053105)
\curveto(778.62996617,199.1202197)(778.02449865,198.46006479)(777.44637482,197.71006631)
\curveto(776.87606347,196.96788031)(776.34481455,196.10069457)(775.85262805,195.10850907)
\curveto(775.38387899,194.14757352)(775.00106727,193.08898191)(774.70419287,191.93273426)
\curveto(774.41513096,190.7764866)(774.2706,189.55773907)(774.2706,188.27649166)
\curveto(774.2706,186.94055687)(774.41122471,185.71790309)(774.69247414,184.60853034)
\curveto(774.98153606,183.49915758)(775.36825403,182.44447222)(775.85262805,181.44447424)
\curveto(776.3213771,180.48353869)(776.85653226,179.61635295)(777.45809354,178.84291701)
\curveto(778.05965483,178.06166859)(778.66121611,177.40151368)(779.26277739,176.86245227)
\closepath
}
}
{
\newrgbcolor{curcolor}{0 0 0}
\pscustom[linestyle=none,fillstyle=solid,fillcolor=curcolor]
{
\newpath
\moveto(793.32524891,181.57338023)
\lineto(791.1338471,181.57338023)
\lineto(791.1338471,182.96790866)
\curveto(790.938535,182.83509643)(790.67291053,182.64759681)(790.33697371,182.4054098)
\curveto(790.00884938,182.17103527)(789.68853753,181.98353565)(789.37603816,181.84291094)
\curveto(789.0088514,181.6632238)(788.58697726,181.5147866)(788.11041572,181.39759934)
\curveto(787.63385419,181.27259959)(787.07526157,181.21009972)(786.43463787,181.21009972)
\curveto(785.25495276,181.21009972)(784.25495478,181.60072393)(783.43464394,182.38197235)
\curveto(782.6143331,183.16322076)(782.20417768,184.1593125)(782.20417768,185.37024754)
\curveto(782.20417768,186.36243304)(782.41511476,187.16321266)(782.8369889,187.77258643)
\curveto(783.26667553,188.38977268)(783.8760493,188.8741467)(784.6651102,189.22570849)
\curveto(785.46198359,189.57727028)(786.4190129,189.81555104)(787.53619814,189.94055079)
\curveto(788.65338337,190.06555054)(789.8525997,190.15930035)(791.1338471,190.22180022)
\lineto(791.1338471,190.56164328)
\curveto(791.1338471,191.06164227)(791.04400353,191.47570393)(790.8643164,191.80382827)
\curveto(790.69244174,192.1319526)(790.44244225,192.38976458)(790.11431792,192.5772642)
\curveto(789.80181855,192.75695134)(789.42681931,192.87804484)(788.98932019,192.94054472)
\curveto(788.55182108,193.00304459)(788.09479076,193.03429453)(787.61822922,193.03429453)
\curveto(787.04010539,193.03429453)(786.39557545,192.95616968)(785.68463939,192.79992)
\curveto(784.97370333,192.6514828)(784.23932981,192.43273324)(783.48151885,192.14367133)
\lineto(783.36433158,192.14367133)
\lineto(783.36433158,194.38194805)
\curveto(783.79401821,194.49913531)(784.41511071,194.6280413)(785.22760906,194.76866601)
\curveto(786.04010742,194.90929073)(786.84088704,194.97960309)(787.62994795,194.97960309)
\curveto(788.55182108,194.97960309)(789.35260071,194.90147824)(790.03228683,194.74522856)
\curveto(790.71978544,194.59679136)(791.31353424,194.33897938)(791.81353322,193.97179263)
\curveto(792.30571973,193.61241835)(792.68071897,193.14757555)(792.93853095,192.5772642)
\curveto(793.19634292,192.00695286)(793.32524891,191.29992304)(793.32524891,190.45617475)
\closepath
\moveto(791.1338471,184.79602996)
\lineto(791.1338471,188.44055383)
\curveto(790.46197346,188.40149141)(789.66900632,188.34289778)(788.75494567,188.26477293)
\curveto(787.8486975,188.18664809)(787.12994896,188.07336707)(786.59870003,187.92492987)
\curveto(785.96588882,187.74524274)(785.4541711,187.46399331)(785.06354689,187.08118158)
\curveto(784.67292268,186.70618234)(784.47761058,186.18665214)(784.47761058,185.52259099)
\curveto(784.47761058,184.77259251)(784.70417262,184.2061874)(785.1572967,183.82337568)
\curveto(785.61042079,183.44837644)(786.30182564,183.26087682)(787.23151125,183.26087682)
\curveto(788.00494719,183.26087682)(788.71197701,183.40931402)(789.35260071,183.70618841)
\curveto(789.99322441,184.0108753)(790.58697321,184.37415581)(791.1338471,184.79602996)
\closepath
}
}
{
\newrgbcolor{curcolor}{0 0 0}
\pscustom[linestyle=none,fillstyle=solid,fillcolor=curcolor]
{
\newpath
\moveto(804.36428906,181.6905675)
\curveto(803.9502274,181.58119272)(803.49710332,181.49134915)(803.00491681,181.42103679)
\curveto(802.52054279,181.35072443)(802.08694992,181.31556826)(801.7041382,181.31556826)
\curveto(800.3682034,181.31556826)(799.35258046,181.67494253)(798.65726937,182.39369107)
\curveto(797.96195827,183.11243962)(797.61430273,184.26478103)(797.61430273,185.85071532)
\lineto(797.61430273,192.81163873)
\lineto(796.12602449,192.81163873)
\lineto(796.12602449,194.66319748)
\lineto(797.61430273,194.66319748)
\lineto(797.61430273,198.42490861)
\lineto(799.81742327,198.42490861)
\lineto(799.81742327,194.66319748)
\lineto(804.36428906,194.66319748)
\lineto(804.36428906,192.81163873)
\lineto(799.81742327,192.81163873)
\lineto(799.81742327,186.84680705)
\curveto(799.81742327,186.15930845)(799.83304824,185.62024704)(799.86429817,185.22962283)
\curveto(799.89554811,184.8468111)(800.00492289,184.48743683)(800.19242251,184.15150001)
\curveto(800.36429716,183.83900065)(800.59867168,183.60853236)(800.89554608,183.46009516)
\curveto(801.20023297,183.31947045)(801.66116953,183.24915809)(802.27835578,183.24915809)
\curveto(802.63773006,183.24915809)(803.0127293,183.29993924)(803.40335351,183.40150153)
\curveto(803.79397771,183.51087631)(804.07522714,183.60071988)(804.2471018,183.67103224)
\lineto(804.36428906,183.67103224)
\closepath
}
}
{
\newrgbcolor{curcolor}{0 0 0}
\pscustom[linestyle=none,fillstyle=solid,fillcolor=curcolor]
{
\newpath
\moveto(818.18066733,188.11242949)
\curveto(818.18066733,185.97962131)(817.63379344,184.29603097)(816.54004565,183.06165847)
\curveto(815.44629787,181.82728597)(813.98145708,181.21009972)(812.1455233,181.21009972)
\curveto(810.29396455,181.21009972)(808.82131128,181.82728597)(807.7275635,183.06165847)
\curveto(806.6416282,184.29603097)(806.09866055,185.97962131)(806.09866055,188.11242949)
\curveto(806.09866055,190.24523767)(806.6416282,191.92882801)(807.7275635,193.16320051)
\curveto(808.82131128,194.4053855)(810.29396455,195.02647799)(812.1455233,195.02647799)
\curveto(813.98145708,195.02647799)(815.44629787,194.4053855)(816.54004565,193.16320051)
\curveto(817.63379344,191.92882801)(818.18066733,190.24523767)(818.18066733,188.11242949)
\closepath
\moveto(815.90723443,188.11242949)
\curveto(815.90723443,189.80773856)(815.57520386,191.06554851)(814.9111427,191.88585935)
\curveto(814.24708155,192.71398267)(813.32520841,193.12804434)(812.1455233,193.12804434)
\curveto(810.95021322,193.12804434)(810.02052761,192.71398267)(809.35646645,191.88585935)
\curveto(808.70021778,191.06554851)(808.37209344,189.80773856)(808.37209344,188.11242949)
\curveto(808.37209344,186.47180781)(808.70412402,185.22571659)(809.36818518,184.37415581)
\curveto(810.03224633,183.53040752)(810.95802571,183.10853337)(812.1455233,183.10853337)
\curveto(813.31739593,183.10853337)(814.23536282,183.52650128)(814.89942398,184.36243709)
\curveto(815.57129762,185.20618538)(815.90723443,186.45618285)(815.90723443,188.11242949)
\closepath
}
}
{
\newrgbcolor{curcolor}{0 0 0}
\pscustom[linestyle=none,fillstyle=solid,fillcolor=curcolor]
{
\newpath
\moveto(840.69234049,181.57338023)
\lineto(838.48921996,181.57338023)
\lineto(838.48921996,189.02649014)
\curveto(838.48921996,189.588989)(838.46187626,190.13195665)(838.40718887,190.65539309)
\curveto(838.36031397,191.17882953)(838.25484543,191.59679744)(838.09078326,191.9092968)
\curveto(837.91109613,192.24523362)(837.65328415,192.49913936)(837.31734733,192.67101401)
\curveto(836.98141051,192.84288866)(836.49703649,192.92882599)(835.86422527,192.92882599)
\curveto(835.24703902,192.92882599)(834.62985277,192.77257631)(834.01266652,192.46007694)
\curveto(833.39548027,192.15539006)(832.77829402,191.76476585)(832.16110777,191.28820431)
\curveto(832.18454522,191.10851717)(832.20407643,190.8975801)(832.2197014,190.65539309)
\curveto(832.23532637,190.42101857)(832.24313885,190.18664404)(832.24313885,189.95226952)
\lineto(832.24313885,181.57338023)
\lineto(830.04001832,181.57338023)
\lineto(830.04001832,189.02649014)
\curveto(830.04001832,189.60461397)(830.01267462,190.15148786)(829.95798723,190.66711182)
\curveto(829.91111233,191.19054826)(829.80564379,191.60851616)(829.64158162,191.92101553)
\curveto(829.46189449,192.25695235)(829.20408251,192.50695184)(828.86814569,192.67101401)
\curveto(828.53220887,192.84288866)(828.04783485,192.92882599)(827.41502363,192.92882599)
\curveto(826.81346235,192.92882599)(826.20799483,192.78038879)(825.59862106,192.48351439)
\curveto(824.99705978,192.18663999)(824.3954985,191.80773451)(823.79393721,191.34679794)
\lineto(823.79393721,181.57338023)
\lineto(821.59081668,181.57338023)
\lineto(821.59081668,194.66319748)
\lineto(823.79393721,194.66319748)
\lineto(823.79393721,193.21007542)
\curveto(824.48143582,193.78038676)(825.16502819,194.22569836)(825.84471431,194.54601021)
\curveto(826.53221292,194.86632207)(827.26268019,195.02647799)(828.03611612,195.02647799)
\curveto(828.92673932,195.02647799)(829.68064404,194.83897837)(830.29783029,194.46397913)
\curveto(830.92282903,194.08897989)(831.38767184,193.56944969)(831.69235872,192.90538854)
\curveto(832.58298192,193.65538702)(833.39548027,194.19444843)(834.12985378,194.52257276)
\curveto(834.8642273,194.85850958)(835.64938196,195.02647799)(836.48531776,195.02647799)
\curveto(837.92281485,195.02647799)(838.98140646,194.58897888)(839.66109258,193.71398065)
\curveto(840.34859119,192.8467949)(840.69234049,191.63195361)(840.69234049,190.06945678)
\closepath
}
}
{
\newrgbcolor{curcolor}{0 0 0}
\pscustom[linestyle=none,fillstyle=solid,fillcolor=curcolor]
{
\newpath
\moveto(847.30170211,196.85459929)
\lineto(844.81733214,196.85459929)
\lineto(844.81733214,199.13975091)
\lineto(847.30170211,199.13975091)
\closepath
\moveto(847.1610774,181.57338023)
\lineto(844.95795686,181.57338023)
\lineto(844.95795686,194.66319748)
\lineto(847.1610774,194.66319748)
\closepath
}
}
{
\newrgbcolor{curcolor}{0 0 0}
\pscustom[linestyle=none,fillstyle=solid,fillcolor=curcolor]
{
\newpath
\moveto(861.20011147,182.39369107)
\curveto(860.46573795,182.04212928)(859.76652062,181.76869234)(859.10245946,181.57338023)
\curveto(858.44621079,181.37806813)(857.74699346,181.28041208)(857.00480746,181.28041208)
\curveto(856.05949688,181.28041208)(855.19231113,181.41713055)(854.40325023,181.6905675)
\curveto(853.61418933,181.97181693)(852.93840945,182.39369107)(852.37591059,182.95618993)
\curveto(851.80559924,183.51868879)(851.36419388,184.22962485)(851.05169452,185.08899811)
\curveto(850.73919515,185.94837137)(850.58294547,186.95227559)(850.58294547,188.10071077)
\curveto(850.58294547,190.24133143)(851.16888178,191.92101553)(852.34075441,193.13976306)
\curveto(853.52043952,194.35851059)(855.07512387,194.96788436)(857.00480746,194.96788436)
\curveto(857.75480594,194.96788436)(858.48917946,194.86241582)(859.207928,194.65147875)
\curveto(859.93448903,194.44054168)(860.59855018,194.1827297)(861.20011147,193.87804282)
\lineto(861.20011147,191.42882903)
\lineto(861.0829242,191.42882903)
\curveto(860.41105056,191.95226547)(859.71573947,192.3546084)(858.99699093,192.63585783)
\curveto(858.28605487,192.91710726)(857.59074378,193.05773198)(856.91105765,193.05773198)
\curveto(855.66106018,193.05773198)(854.67278093,192.63585783)(853.94621991,191.79210954)
\curveto(853.22747136,190.95617373)(852.86809709,189.72570748)(852.86809709,188.10071077)
\curveto(852.86809709,186.52258896)(853.21965888,185.30774767)(853.92278245,184.4561869)
\curveto(854.63371851,183.6124386)(855.62981025,183.19056446)(856.91105765,183.19056446)
\curveto(857.35636925,183.19056446)(857.80949333,183.24915809)(858.2704299,183.36634535)
\curveto(858.73136647,183.48353262)(859.14542813,183.63587606)(859.51261488,183.82337568)
\curveto(859.83292674,183.98743785)(860.13370738,184.1593125)(860.41495681,184.33899963)
\curveto(860.69620624,184.52649925)(860.91886204,184.68665518)(861.0829242,184.81946741)
\lineto(861.20011147,184.81946741)
\closepath
}
}
{
\newrgbcolor{curcolor}{0 0 0}
\pscustom[linestyle=none,fillstyle=solid,fillcolor=curcolor]
{
\newpath
\moveto(870.63368611,188.27649166)
\curveto(870.63368611,185.92493392)(870.25087439,183.81165695)(869.48525094,181.93666075)
\curveto(868.71962749,180.06166454)(867.64541091,178.3311993)(866.26260121,176.74526501)
\lineto(863.5790129,176.74526501)
\lineto(863.5790129,176.86245227)
\curveto(864.18057418,177.40151368)(864.78213546,178.06166859)(865.38369674,178.84291701)
\curveto(865.99307051,179.61635295)(866.52822568,180.48353869)(866.98916224,181.44447424)
\curveto(867.47353626,182.44447222)(867.85634799,183.49915758)(868.13759742,184.60853034)
\curveto(868.42665933,185.71790309)(868.57119029,186.94055687)(868.57119029,188.27649166)
\curveto(868.57119029,189.55773907)(868.42665933,190.7764866)(868.13759742,191.93273426)
\curveto(867.8485355,193.08898191)(867.46572378,194.14757352)(866.98916224,195.10850907)
\curveto(866.49697574,196.10069457)(865.96182057,196.96788031)(865.38369674,197.71006631)
\curveto(864.8133854,198.46006479)(864.21182412,199.1202197)(863.5790129,199.69053105)
\lineto(863.5790129,199.80771831)
\lineto(866.26260121,199.80771831)
\curveto(867.64541091,198.22178402)(868.71962749,196.49131877)(869.48525094,194.61632257)
\curveto(870.25087439,192.74132637)(870.63368611,190.6280494)(870.63368611,188.27649166)
\closepath
}
}
{
\newrgbcolor{curcolor}{0 0 0}
\pscustom[linestyle=none,fillstyle=solid,fillcolor=curcolor,opacity=0]
{
\newpath
\moveto(49.33902891,201.26560489)
\lineto(403.63752405,201.26560489)
\lineto(403.63752405,169.29716569)
\lineto(49.33902891,169.29716569)
\closepath
}
}
{
\newrgbcolor{curcolor}{0 0 0}
\pscustom[linestyle=none,fillstyle=solid,fillcolor=curcolor]
{
\newpath
\moveto(121.23288874,186.54956879)
\curveto(121.60788798,186.21363197)(121.9164811,185.79175782)(122.15866811,185.28394635)
\curveto(122.40085512,184.77613488)(122.52194863,184.11988621)(122.52194863,183.31520034)
\curveto(122.52194863,182.51832695)(122.37741767,181.78785968)(122.08835575,181.12379853)
\curveto(121.79929384,180.45973737)(121.39304466,179.88161354)(120.86960822,179.38942704)
\curveto(120.28367191,178.84255315)(119.59226706,178.43630397)(118.79539367,178.17067951)
\curveto(118.00633277,177.91286753)(117.13914703,177.78396154)(116.19383644,177.78396154)
\curveto(115.2250884,177.78396154)(114.27196533,177.9011488)(113.33446723,178.13552333)
\curveto(112.39696913,178.36208537)(111.62743944,178.61208486)(111.02587816,178.88552181)
\lineto(111.02587816,181.3347356)
\lineto(111.20165905,181.3347356)
\curveto(111.86572021,180.89723648)(112.64696862,180.53395597)(113.5454043,180.24489406)
\curveto(114.44383998,179.95583214)(115.31102573,179.81130118)(116.14696154,179.81130118)
\curveto(116.63914804,179.81130118)(117.16258448,179.89333227)(117.71727086,180.05739444)
\curveto(118.27195723,180.2214566)(118.72117507,180.46364361)(119.06492438,180.78395546)
\curveto(119.42429865,181.12770477)(119.68992311,181.50661025)(119.86179776,181.92067191)
\curveto(120.0414849,182.33473357)(120.13132847,182.85817001)(120.13132847,183.49098123)
\curveto(120.13132847,184.11597997)(120.02976617,184.63160392)(119.82664158,185.0378531)
\curveto(119.63132948,185.45191476)(119.35789253,185.77613285)(119.00633075,186.01050738)
\curveto(118.65476896,186.25269439)(118.22898857,186.41675656)(117.72898958,186.50269388)
\curveto(117.22899059,186.59644369)(116.68992919,186.6433186)(116.11180536,186.6433186)
\lineto(115.05711999,186.6433186)
\lineto(115.05711999,188.58862716)
\lineto(115.87743083,188.58862716)
\curveto(117.06492843,188.58862716)(118.01023901,188.83472041)(118.71336259,189.32690691)
\curveto(119.42429865,189.8269059)(119.77976668,190.55346693)(119.77976668,191.50659)
\curveto(119.77976668,191.92846415)(119.68992311,192.2956509)(119.51023597,192.60815027)
\curveto(119.33054884,192.92846212)(119.08054934,193.19018034)(118.76023749,193.39330493)
\curveto(118.42430067,193.59642952)(118.0649264,193.73705423)(117.68211468,193.81517908)
\curveto(117.29930295,193.89330392)(116.86571008,193.93236634)(116.38133606,193.93236634)
\curveto(115.63915006,193.93236634)(114.85008916,193.79955411)(114.01415335,193.53392965)
\curveto(113.17821755,193.26830518)(112.38915665,192.89330594)(111.64697065,192.40893192)
\lineto(111.52978339,192.40893192)
\lineto(111.52978339,194.85814571)
\curveto(112.08446976,195.13158266)(112.82274952,195.38158215)(113.74462265,195.6081442)
\curveto(114.67430827,195.84251872)(115.57274395,195.95970598)(116.43992969,195.95970598)
\curveto(117.29149047,195.95970598)(118.04148895,195.88158114)(118.68992514,195.72533146)
\curveto(119.33836132,195.56908177)(119.92429764,195.31908228)(120.44773408,194.97533298)
\curveto(121.01023294,194.60033374)(121.43601333,194.14720965)(121.72507524,193.61596073)
\curveto(122.01413715,193.08471181)(122.15866811,192.46361931)(122.15866811,191.75268325)
\curveto(122.15866811,190.78393521)(121.81491881,189.93628068)(121.1274202,189.20971965)
\curveto(120.44773408,188.49097111)(119.64304821,188.03784702)(118.71336259,187.8503474)
\lineto(118.71336259,187.68628524)
\curveto(119.08836183,187.62378536)(119.51804846,187.49097313)(120.00242248,187.28784854)
\curveto(120.4867965,187.09253644)(120.89695192,186.84644319)(121.23288874,186.54956879)
\closepath
}
}
{
\newrgbcolor{curcolor}{0 0 0}
\pscustom[linestyle=none,fillstyle=solid,fillcolor=curcolor]
{
\newpath
\moveto(126.35444132,178.14724205)
\lineto(123.55366574,178.14724205)
\lineto(123.55366574,181.48707904)
\lineto(126.35444132,181.48707904)
\closepath
}
}
{
\newrgbcolor{curcolor}{0 0 0}
\pscustom[linestyle=none,fillstyle=solid,fillcolor=curcolor]
{
\newpath
\moveto(146.36908738,179.41286449)
\curveto(145.93940075,179.22536487)(145.54877654,179.04958398)(145.19721475,178.88552181)
\curveto(144.85346545,178.72145964)(144.40034136,178.54958499)(143.8378425,178.36989785)
\curveto(143.36128097,178.22146065)(142.84175077,178.09646091)(142.27925191,177.99489861)
\curveto(141.72456553,177.88552383)(141.11128552,177.83083644)(140.43941189,177.83083644)
\curveto(139.17378945,177.83083644)(138.02144803,178.00661734)(136.98238764,178.35817913)
\curveto(135.95113972,178.7175534)(135.05270404,179.27614602)(134.28708059,180.03395698)
\curveto(133.53708211,180.77614298)(132.9511458,181.71754732)(132.52927165,182.85817001)
\curveto(132.10739751,184.00660519)(131.89646043,185.33863374)(131.89646043,186.85425567)
\curveto(131.89646043,188.29175276)(132.09958502,189.57690641)(132.5058342,190.70971661)
\curveto(132.91208338,191.84252682)(133.49801969,192.79955613)(134.26364314,193.58080455)
\curveto(135.00582914,194.33861552)(135.90035858,194.91673935)(136.94723146,195.31517604)
\curveto(138.00191682,195.71361273)(139.16988321,195.91283108)(140.45113061,195.91283108)
\curveto(141.38862871,195.91283108)(142.32222057,195.79955006)(143.25190619,195.57298802)
\curveto(144.18940429,195.34642598)(145.22846469,194.94798928)(146.36908738,194.37767794)
\lineto(146.36908738,191.62377726)
\lineto(146.19330648,191.62377726)
\curveto(145.23237093,192.42846313)(144.27924786,193.01439945)(143.33393727,193.3815862)
\curveto(142.38862669,193.74877296)(141.37690999,193.93236634)(140.29878717,193.93236634)
\curveto(139.41597646,193.93236634)(138.61910307,193.78783538)(137.90816701,193.49877347)
\curveto(137.20504344,193.21752404)(136.57613846,192.77611868)(136.02145208,192.1745574)
\curveto(135.48239067,191.58862108)(135.06051653,190.84643509)(134.75582964,189.94799941)
\curveto(134.45895525,189.05737621)(134.31051805,188.0261283)(134.31051805,186.85425567)
\curveto(134.31051805,185.62769566)(134.47458021,184.57301029)(134.80270455,183.69019958)
\curveto(135.13864137,182.80738887)(135.568328,182.08864032)(136.09176444,181.53395395)
\curveto(136.63863833,180.95583012)(137.27535579,180.52614349)(138.00191682,180.24489406)
\curveto(138.73629033,179.97145711)(139.50972627,179.83473864)(140.32222462,179.83473864)
\curveto(141.43940986,179.83473864)(142.48628274,180.0261445)(143.46284326,180.40895622)
\curveto(144.43940379,180.79176795)(145.35346443,181.36598554)(146.20502521,182.13160899)
\lineto(146.36908738,182.13160899)
\closepath
}
}
{
\newrgbcolor{curcolor}{0 0 0}
\pscustom[linestyle=none,fillstyle=solid,fillcolor=curcolor]
{
\newpath
\moveto(162.70030354,193.59252328)
\curveto(163.4112396,192.81127486)(163.95420725,191.85424555)(164.3292065,190.72143534)
\curveto(164.71201822,189.58862513)(164.90342408,188.30347149)(164.90342408,186.8659744)
\curveto(164.90342408,185.42847731)(164.70811198,184.13941742)(164.31748777,182.99879473)
\curveto(163.93467604,181.86598452)(163.39561464,180.92067394)(162.70030354,180.16286297)
\curveto(161.981555,179.37380207)(161.12999422,178.78005327)(160.14562122,178.38161658)
\curveto(159.16906069,177.98317989)(158.05187546,177.78396154)(156.7940655,177.78396154)
\curveto(155.56750549,177.78396154)(154.45032025,177.98708613)(153.44250979,178.39333531)
\curveto(152.44251182,178.79958448)(151.59095104,179.38942704)(150.88782746,180.16286297)
\curveto(150.18470389,180.93629891)(149.64173624,181.88551573)(149.25892451,183.01051346)
\curveto(148.88392527,184.13551118)(148.69642565,185.42066482)(148.69642565,186.8659744)
\curveto(148.69642565,188.28784652)(148.88392527,189.56128144)(149.25892451,190.68627916)
\curveto(149.63392375,191.81908937)(150.18079765,192.78783741)(150.89954619,193.59252328)
\curveto(151.5870448,194.35814673)(152.43860557,194.94408304)(153.45422852,195.35033222)
\curveto(154.47766394,195.75658139)(155.59094294,195.95970598)(156.7940655,195.95970598)
\curveto(158.04406297,195.95970598)(159.16515445,195.75267515)(160.15733994,195.33861349)
\curveto(161.15733792,194.93236431)(162.00499245,194.35033424)(162.70030354,193.59252328)
\closepath
\moveto(162.48936647,186.8659744)
\curveto(162.48936647,189.13159481)(161.981555,190.87768502)(160.96593206,192.10424504)
\curveto(159.95030911,193.33861754)(158.56359317,193.95580379)(156.80578423,193.95580379)
\curveto(155.03235032,193.95580379)(153.6378219,193.33861754)(152.62219895,192.10424504)
\curveto(151.61438849,190.87768502)(151.11048326,189.13159481)(151.11048326,186.8659744)
\curveto(151.11048326,184.57691653)(151.62610722,182.82301384)(152.65735513,181.6042663)
\curveto(153.68860304,180.39333126)(155.07141274,179.78786373)(156.80578423,179.78786373)
\curveto(158.54015572,179.78786373)(159.91905918,180.39333126)(160.9424946,181.6042663)
\curveto(161.97374252,182.82301384)(162.48936647,184.57691653)(162.48936647,186.8659744)
\closepath
}
}
{
\newrgbcolor{curcolor}{0 0 0}
\pscustom[linestyle=none,fillstyle=solid,fillcolor=curcolor]
{
\newpath
\moveto(188.86503404,195.59642547)
\lineto(184.32988697,178.14724205)
\lineto(181.71661101,178.14724205)
\lineto(178.04864969,192.63158772)
\lineto(174.46271945,178.14724205)
\lineto(171.90803713,178.14724205)
\lineto(167.29085898,195.59642547)
\lineto(169.66976041,195.59642547)
\lineto(173.33772173,181.08864235)
\lineto(176.94708942,195.59642547)
\lineto(179.3025534,195.59642547)
\lineto(182.94707727,180.94801763)
\lineto(186.59160114,195.59642547)
\closepath
}
}
{
\newrgbcolor{curcolor}{0 0 0}
\pscustom[linestyle=none,fillstyle=solid,fillcolor=curcolor]
{
\newpath
\moveto(200.50782128,193.42846111)
\lineto(198.02345131,193.42846111)
\lineto(198.02345131,195.71361273)
\lineto(200.50782128,195.71361273)
\closepath
\moveto(200.36719657,178.14724205)
\lineto(198.16407603,178.14724205)
\lineto(198.16407603,191.2370593)
\lineto(200.36719657,191.2370593)
\closepath
}
}
{
\newrgbcolor{curcolor}{0 0 0}
\pscustom[linestyle=none,fillstyle=solid,fillcolor=curcolor]
{
\newpath
\moveto(215.68591402,178.14724205)
\lineto(213.48279348,178.14724205)
\lineto(213.48279348,185.60035196)
\curveto(213.48279348,186.20191324)(213.4476373,186.7644121)(213.37732495,187.28784854)
\curveto(213.30701259,187.81909747)(213.1781066,188.23315913)(212.99060698,188.53003353)
\curveto(212.79529487,188.85815786)(212.51404544,189.10034487)(212.14685869,189.25659456)
\curveto(211.77967193,189.42065672)(211.3031104,189.50268781)(210.71717408,189.50268781)
\curveto(210.1156128,189.50268781)(209.48670782,189.35425061)(208.83045915,189.05737621)
\curveto(208.17421048,188.76050181)(207.54530551,188.38159633)(206.94374422,187.92065976)
\lineto(206.94374422,178.14724205)
\lineto(204.74062369,178.14724205)
\lineto(204.74062369,191.2370593)
\lineto(206.94374422,191.2370593)
\lineto(206.94374422,189.78393724)
\curveto(207.63124283,190.35424858)(208.34217889,190.79956018)(209.07655241,191.11987203)
\curveto(209.81092592,191.44018389)(210.56483064,191.60033981)(211.33826658,191.60033981)
\curveto(212.75232621,191.60033981)(213.83044903,191.17455942)(214.57263503,190.32299865)
\curveto(215.31482102,189.47143787)(215.68591402,188.24487786)(215.68591402,186.6433186)
\closepath
}
}
{
\newrgbcolor{curcolor}{0 0 0}
\pscustom[linestyle=none,fillstyle=solid,fillcolor=curcolor]
{
\newpath
\moveto(228.79776491,178.14724205)
\lineto(226.59464437,178.14724205)
\lineto(226.59464437,179.51833303)
\curveto(225.96183315,178.97145913)(225.30167824,178.54567875)(224.61417963,178.24099186)
\curveto(223.92668103,177.93630498)(223.18058879,177.78396154)(222.37590292,177.78396154)
\curveto(220.81340608,177.78396154)(219.5712211,178.38552282)(218.64934796,179.58864538)
\curveto(217.73528731,180.79176795)(217.27825699,182.45973332)(217.27825699,184.5925415)
\curveto(217.27825699,185.70191426)(217.43450667,186.6901935)(217.74700604,187.55737925)
\curveto(218.06731789,188.42456499)(218.49700452,189.16284475)(219.03606593,189.77221851)
\curveto(219.56731485,190.36596731)(220.1845011,190.81909139)(220.88762468,191.13159076)
\curveto(221.59856074,191.44409013)(222.33293425,191.60033981)(223.09074522,191.60033981)
\curveto(223.77824383,191.60033981)(224.38761759,191.52612121)(224.91886652,191.37768401)
\curveto(225.45011544,191.2370593)(226.00870806,191.0144035)(226.59464437,190.70971661)
\lineto(226.59464437,196.38158013)
\lineto(228.79776491,196.38158013)
\closepath
\moveto(226.59464437,181.36989178)
\lineto(226.59464437,188.88159532)
\curveto(226.00089557,189.14721978)(225.46964665,189.33081316)(225.0008976,189.43237545)
\curveto(224.53214855,189.53393775)(224.02043084,189.58471889)(223.46574446,189.58471889)
\curveto(222.23137196,189.58471889)(221.2704364,189.15503226)(220.5829378,188.295659)
\curveto(219.89543919,187.43628574)(219.55168988,186.21753821)(219.55168988,184.63941641)
\curveto(219.55168988,183.08473206)(219.81731435,181.9011407)(220.34856327,181.08864235)
\curveto(220.8798122,180.28395648)(221.73137297,179.88161354)(222.9032456,179.88161354)
\curveto(223.52824433,179.88161354)(224.16105555,180.01833201)(224.80167925,180.29176896)
\curveto(225.44230296,180.57301839)(226.039958,180.93239266)(226.59464437,181.36989178)
\closepath
}
}
{
\newrgbcolor{curcolor}{0 0 0}
\pscustom[linestyle=none,fillstyle=solid,fillcolor=curcolor]
{
\newpath
\moveto(233.60375493,193.42846111)
\lineto(231.11938497,193.42846111)
\lineto(231.11938497,195.71361273)
\lineto(233.60375493,195.71361273)
\closepath
\moveto(233.46313022,178.14724205)
\lineto(231.26000968,178.14724205)
\lineto(231.26000968,191.2370593)
\lineto(233.46313022,191.2370593)
\closepath
}
}
{
\newrgbcolor{curcolor}{0 0 0}
\pscustom[linestyle=none,fillstyle=solid,fillcolor=curcolor]
{
\newpath
\moveto(246.00450955,188.83472041)
\lineto(245.88732228,188.83472041)
\curveto(245.55919795,188.91284525)(245.2388861,188.96753264)(244.92638673,188.99878258)
\curveto(244.62169985,189.037845)(244.25841933,189.05737621)(243.83654519,189.05737621)
\curveto(243.15685906,189.05737621)(242.50061039,188.90503277)(241.86779917,188.60034589)
\curveto(241.23498796,188.30347149)(240.62561419,187.91675352)(240.03967788,187.44019199)
\lineto(240.03967788,178.14724205)
\lineto(237.83655734,178.14724205)
\lineto(237.83655734,191.2370593)
\lineto(240.03967788,191.2370593)
\lineto(240.03967788,189.30346946)
\curveto(240.9146761,190.00659304)(241.6842058,190.50268578)(242.34826695,190.7917477)
\curveto(243.02014059,191.0886221)(243.70373296,191.2370593)(244.39904405,191.2370593)
\curveto(244.78185577,191.2370593)(245.05919896,191.22534057)(245.23107361,191.20190312)
\curveto(245.40294827,191.18627815)(245.66076024,191.15112197)(246.00450955,191.09643458)
\closepath
}
}
{
\newrgbcolor{curcolor}{0 0 0}
\pscustom[linestyle=none,fillstyle=solid,fillcolor=curcolor]
{
\newpath
\moveto(257.24155497,184.46363551)
\lineto(247.59704325,184.46363551)
\curveto(247.59704325,183.65894964)(247.71813675,182.95582607)(247.96032376,182.35426478)
\curveto(248.20251077,181.76051599)(248.53454135,181.27223573)(248.9564155,180.889424)
\curveto(249.36266467,180.51442476)(249.84313245,180.23317533)(250.39781883,180.04567571)
\curveto(250.96031769,179.85817609)(251.57750394,179.76442628)(252.24937758,179.76442628)
\curveto(253.14000077,179.76442628)(254.03453021,179.94020717)(254.93296589,180.29176896)
\curveto(255.83921406,180.65114323)(256.483744,181.00270502)(256.86655573,181.34645433)
\lineto(256.98374299,181.34645433)
\lineto(256.98374299,178.94411544)
\curveto(256.24155699,178.63161607)(255.48374603,178.36989785)(254.71031009,178.15896078)
\curveto(253.93687416,177.94802371)(253.12437581,177.84255517)(252.27281503,177.84255517)
\curveto(250.10094443,177.84255517)(248.40563536,178.42849148)(247.18688783,179.60036411)
\curveto(245.9681403,180.78004922)(245.35876653,182.45192084)(245.35876653,184.61597895)
\curveto(245.35876653,186.75659962)(245.9407966,188.45581493)(247.10485675,189.71362488)
\curveto(248.27672937,190.97143483)(249.81578876,191.60033981)(251.7220349,191.60033981)
\curveto(253.48765632,191.60033981)(254.84702857,191.08471586)(255.80015164,190.05346794)
\curveto(256.76108719,189.02222003)(257.24155497,187.55737925)(257.24155497,185.65894559)
\closepath
\moveto(255.09702806,186.1511321)
\curveto(255.08921558,187.30737975)(254.79624742,188.20190919)(254.21812359,188.83472041)
\curveto(253.64781225,189.46753163)(252.77672026,189.78393724)(251.60484763,189.78393724)
\curveto(250.42516252,189.78393724)(249.48375818,189.43628169)(248.7806346,188.7409706)
\curveto(248.08532351,188.04565951)(247.69079306,187.18238001)(247.59704325,186.1511321)
\closepath
}
}
{
\newrgbcolor{curcolor}{0 0 0}
\pscustom[linestyle=none,fillstyle=solid,fillcolor=curcolor]
{
\newpath
\moveto(266.98018271,178.96755289)
\curveto(266.2458092,178.6159911)(265.54659186,178.34255416)(264.88253071,178.14724205)
\curveto(264.22628204,177.95192995)(263.5270647,177.8542739)(262.7848787,177.8542739)
\curveto(261.83956812,177.8542739)(260.97238237,177.99099237)(260.18332147,178.26442932)
\curveto(259.39426057,178.54567875)(258.71848069,178.96755289)(258.15598183,179.53005175)
\curveto(257.58567048,180.09255061)(257.14426513,180.80348667)(256.83176576,181.66285993)
\curveto(256.51926639,182.52223319)(256.36301671,183.52613741)(256.36301671,184.67457259)
\curveto(256.36301671,186.81519325)(256.94895302,188.49487735)(258.12082565,189.71362488)
\curveto(259.30051076,190.93237241)(260.85519511,191.54174618)(262.7848787,191.54174618)
\curveto(263.53487719,191.54174618)(264.2692507,191.43627764)(264.98799924,191.22534057)
\curveto(265.71456027,191.0144035)(266.37862143,190.75659152)(266.98018271,190.45190464)
\lineto(266.98018271,188.00269085)
\lineto(266.86299545,188.00269085)
\curveto(266.19112181,188.52612729)(265.49581071,188.92847022)(264.77706217,189.20971965)
\curveto(264.06612611,189.49096908)(263.37081502,189.6315938)(262.69112889,189.6315938)
\curveto(261.44113143,189.6315938)(260.45285218,189.20971965)(259.72629115,188.36597136)
\curveto(259.0075426,187.53003555)(258.64816833,186.29956929)(258.64816833,184.67457259)
\curveto(258.64816833,183.09645078)(258.99973012,181.88160949)(259.7028537,181.03004872)
\curveto(260.41378976,180.18630042)(261.40988149,179.76442628)(262.69112889,179.76442628)
\curveto(263.13644049,179.76442628)(263.58956457,179.82301991)(264.05050114,179.94020717)
\curveto(264.51143771,180.05739444)(264.92549937,180.20973788)(265.29268613,180.3972375)
\curveto(265.61299798,180.56129966)(265.91377862,180.73317432)(266.19502805,180.91286145)
\curveto(266.47627748,181.10036107)(266.69893328,181.260517)(266.86299545,181.39332923)
\lineto(266.98018271,181.39332923)
\closepath
}
}
{
\newrgbcolor{curcolor}{0 0 0}
\pscustom[linestyle=none,fillstyle=solid,fillcolor=curcolor]
{
\newpath
\moveto(274.7293583,178.26442932)
\curveto(274.31529664,178.15505454)(273.86217256,178.06521097)(273.36998605,177.99489861)
\curveto(272.88561203,177.92458625)(272.45201916,177.88943008)(272.06920744,177.88943008)
\curveto(270.73327264,177.88943008)(269.7176497,178.24880435)(269.02233861,178.96755289)
\curveto(268.32702752,179.68630144)(267.97937197,180.83864285)(267.97937197,182.42457714)
\lineto(267.97937197,189.38550055)
\lineto(266.49109373,189.38550055)
\lineto(266.49109373,191.2370593)
\lineto(267.97937197,191.2370593)
\lineto(267.97937197,194.99877043)
\lineto(270.18249251,194.99877043)
\lineto(270.18249251,191.2370593)
\lineto(274.7293583,191.2370593)
\lineto(274.7293583,189.38550055)
\lineto(270.18249251,189.38550055)
\lineto(270.18249251,183.42066887)
\curveto(270.18249251,182.73317027)(270.19811748,182.19410886)(270.22936741,181.80348465)
\curveto(270.26061735,181.42067292)(270.36999213,181.06129865)(270.55749175,180.72536183)
\curveto(270.7293664,180.41286247)(270.96374093,180.18239418)(271.26061533,180.03395698)
\curveto(271.56530221,179.89333227)(272.02623877,179.82301991)(272.64342503,179.82301991)
\curveto(273.0027993,179.82301991)(273.37779854,179.87380106)(273.76842275,179.97536335)
\curveto(274.15904696,180.08473813)(274.44029639,180.1745817)(274.61217104,180.24489406)
\lineto(274.7293583,180.24489406)
\closepath
}
}
{
\newrgbcolor{curcolor}{0 0 0}
\pscustom[linestyle=none,fillstyle=solid,fillcolor=curcolor]
{
\newpath
\moveto(293.29998415,184.79175985)
\curveto(293.29998415,183.69801206)(293.14373446,182.71363906)(292.8312351,181.83864083)
\curveto(292.52654821,180.9636426)(292.11248655,180.22926909)(291.58905011,179.63552029)
\curveto(291.03436374,179.01833404)(290.42498997,178.55349123)(289.76092881,178.24099186)
\curveto(289.09686766,177.93630498)(288.36640039,177.78396154)(287.569527,177.78396154)
\curveto(286.827341,177.78396154)(286.17890482,177.87380511)(285.62421844,178.05349224)
\curveto(285.06953206,178.2253669)(284.52265817,178.45974142)(283.98359676,178.75661582)
\lineto(283.84297205,178.14724205)
\lineto(281.78047622,178.14724205)
\lineto(281.78047622,196.38158013)
\lineto(283.98359676,196.38158013)
\lineto(283.98359676,189.86596832)
\curveto(284.60078301,190.37377979)(285.25703168,190.78784146)(285.95234278,191.10815331)
\curveto(286.64765387,191.43627764)(287.42890229,191.60033981)(288.29608803,191.60033981)
\curveto(289.8429599,191.60033981)(291.06170743,191.00659101)(291.95233063,189.81909342)
\curveto(292.85076631,188.63159582)(293.29998415,186.95581797)(293.29998415,184.79175985)
\closepath
\moveto(291.02655125,184.73316622)
\curveto(291.02655125,186.29566305)(290.76873927,187.47925441)(290.25311532,188.28394028)
\curveto(289.73749136,189.09643863)(288.9054618,189.50268781)(287.75702662,189.50268781)
\curveto(287.11640292,189.50268781)(286.46796673,189.36206309)(285.81171806,189.08081366)
\curveto(285.15546939,188.80737672)(284.54609562,188.45190869)(283.98359676,188.01440957)
\lineto(283.98359676,180.51442476)
\curveto(284.6085955,180.23317533)(285.14375066,180.03786322)(285.58906226,179.92848845)
\curveto(286.04218634,179.81911367)(286.55390406,179.76442628)(287.1242154,179.76442628)
\curveto(288.34296294,179.76442628)(289.29608601,180.16286297)(289.98358461,180.95973636)
\curveto(290.67889571,181.76442223)(291.02655125,183.02223218)(291.02655125,184.73316622)
\closepath
}
}
{
\newrgbcolor{curcolor}{0 0 0}
\pscustom[linestyle=none,fillstyle=solid,fillcolor=curcolor]
{
\newpath
\moveto(296.87100658,178.14724205)
\lineto(294.66788604,178.14724205)
\lineto(294.66788604,196.38158013)
\lineto(296.87100658,196.38158013)
\closepath
}
}
{
\newrgbcolor{curcolor}{0 0 0}
\pscustom[linestyle=none,fillstyle=solid,fillcolor=curcolor]
{
\newpath
\moveto(312.32866266,184.68629131)
\curveto(312.32866266,182.55348313)(311.78178876,180.86989279)(310.68804098,179.63552029)
\curveto(309.59429319,178.40114779)(308.12945241,177.78396154)(306.29351863,177.78396154)
\curveto(304.44195988,177.78396154)(302.96930661,178.40114779)(301.87555882,179.63552029)
\curveto(300.78962352,180.86989279)(300.24665587,182.55348313)(300.24665587,184.68629131)
\curveto(300.24665587,186.81909949)(300.78962352,188.50268983)(301.87555882,189.73706233)
\curveto(302.96930661,190.97924732)(304.44195988,191.60033981)(306.29351863,191.60033981)
\curveto(308.12945241,191.60033981)(309.59429319,190.97924732)(310.68804098,189.73706233)
\curveto(311.78178876,188.50268983)(312.32866266,186.81909949)(312.32866266,184.68629131)
\closepath
\moveto(310.05522976,184.68629131)
\curveto(310.05522976,186.38160038)(309.72319918,187.63941033)(309.05913803,188.45972117)
\curveto(308.39507687,189.28784449)(307.47320374,189.70190616)(306.29351863,189.70190616)
\curveto(305.09820855,189.70190616)(304.16852293,189.28784449)(303.50446177,188.45972117)
\curveto(302.8482131,187.63941033)(302.52008877,186.38160038)(302.52008877,184.68629131)
\curveto(302.52008877,183.04566963)(302.85211935,181.79957841)(303.5161805,180.94801763)
\curveto(304.18024166,180.10426934)(305.10602103,179.68239519)(306.29351863,179.68239519)
\curveto(307.46539125,179.68239519)(308.38335814,180.1003631)(309.0474193,180.93629891)
\curveto(309.71929294,181.7800472)(310.05522976,183.03004467)(310.05522976,184.68629131)
\closepath
}
}
{
\newrgbcolor{curcolor}{0 0 0}
\pscustom[linestyle=none,fillstyle=solid,fillcolor=curcolor]
{
\newpath
\moveto(323.54807658,178.96755289)
\curveto(322.81370307,178.6159911)(322.11448573,178.34255416)(321.45042458,178.14724205)
\curveto(320.79417591,177.95192995)(320.09495857,177.8542739)(319.35277258,177.8542739)
\curveto(318.40746199,177.8542739)(317.54027625,177.99099237)(316.75121535,178.26442932)
\curveto(315.96215444,178.54567875)(315.28637456,178.96755289)(314.7238757,179.53005175)
\curveto(314.15356436,180.09255061)(313.712159,180.80348667)(313.39965963,181.66285993)
\curveto(313.08716026,182.52223319)(312.93091058,183.52613741)(312.93091058,184.67457259)
\curveto(312.93091058,186.81519325)(313.51684689,188.49487735)(314.68871952,189.71362488)
\curveto(315.86840463,190.93237241)(317.42308898,191.54174618)(319.35277258,191.54174618)
\curveto(320.10277106,191.54174618)(320.83714457,191.43627764)(321.55589312,191.22534057)
\curveto(322.28245414,191.0144035)(322.9465153,190.75659152)(323.54807658,190.45190464)
\lineto(323.54807658,188.00269085)
\lineto(323.43088932,188.00269085)
\curveto(322.75901568,188.52612729)(322.06370459,188.92847022)(321.34495604,189.20971965)
\curveto(320.63401998,189.49096908)(319.93870889,189.6315938)(319.25902277,189.6315938)
\curveto(318.0090253,189.6315938)(317.02074605,189.20971965)(316.29418502,188.36597136)
\curveto(315.57543648,187.53003555)(315.2160622,186.29956929)(315.2160622,184.67457259)
\curveto(315.2160622,183.09645078)(315.56762399,181.88160949)(316.27074757,181.03004872)
\curveto(316.98168363,180.18630042)(317.97777536,179.76442628)(319.25902277,179.76442628)
\curveto(319.70433437,179.76442628)(320.15745845,179.82301991)(320.61839501,179.94020717)
\curveto(321.07933158,180.05739444)(321.49339324,180.20973788)(321.86058,180.3972375)
\curveto(322.18089185,180.56129966)(322.48167249,180.73317432)(322.76292192,180.91286145)
\curveto(323.04417135,181.10036107)(323.26682715,181.260517)(323.43088932,181.39332923)
\lineto(323.54807658,181.39332923)
\closepath
}
}
{
\newrgbcolor{curcolor}{0 0 0}
\pscustom[linestyle=none,fillstyle=solid,fillcolor=curcolor]
{
\newpath
\moveto(336.38317937,178.14724205)
\lineto(333.47693526,178.14724205)
\lineto(328.22694589,183.8776992)
\lineto(326.79726129,182.51832695)
\lineto(326.79726129,178.14724205)
\lineto(324.59414075,178.14724205)
\lineto(324.59414075,196.38158013)
\lineto(326.79726129,196.38158013)
\lineto(326.79726129,184.68629131)
\lineto(333.16052965,191.2370593)
\lineto(335.93786778,191.2370593)
\lineto(329.85584884,185.19019654)
\closepath
}
}
{
\newrgbcolor{curcolor}{0 0 0}
\pscustom[linestyle=none,fillstyle=solid,fillcolor=curcolor]
{
\newpath
\moveto(346.25989888,181.92067191)
\curveto(346.25989888,180.72536183)(345.76380613,179.74489507)(344.77162064,178.97927162)
\curveto(343.78724763,178.21364817)(342.43959411,177.83083644)(340.72866008,177.83083644)
\curveto(339.75991204,177.83083644)(338.86928884,177.94411746)(338.05679049,178.17067951)
\curveto(337.25210462,178.40505403)(336.57632474,178.65895977)(336.02945084,178.93239671)
\lineto(336.02945084,181.40504796)
\lineto(336.14663811,181.40504796)
\curveto(336.8419492,180.88161152)(337.61538513,180.46364361)(338.46694591,180.15114425)
\curveto(339.31850668,179.84645736)(340.13491128,179.69411392)(340.9161597,179.69411392)
\curveto(341.88490774,179.69411392)(342.6427187,179.8503636)(343.18959259,180.16286297)
\curveto(343.73646649,180.47536234)(344.00990343,180.96754884)(344.00990343,181.63942248)
\curveto(344.00990343,182.15504644)(343.86146623,182.54567065)(343.56459183,182.81129511)
\curveto(343.26771744,183.07691957)(342.69740609,183.30348161)(341.8536578,183.49098123)
\curveto(341.54115843,183.56129359)(341.13100301,183.64332467)(340.62319154,183.73707448)
\curveto(340.12319255,183.83082429)(339.66616223,183.93238659)(339.25210057,184.04176137)
\curveto(338.10366539,184.34644825)(337.2872608,184.79175985)(336.80288678,185.37769616)
\curveto(336.32632524,185.97144496)(336.08804447,186.69800599)(336.08804447,187.55737925)
\curveto(336.08804447,188.09644066)(336.19741925,188.60425213)(336.41616881,189.08081366)
\curveto(336.64273085,189.5573752)(336.98257391,189.98315559)(337.435698,190.35815483)
\curveto(337.87319711,190.72534158)(338.42788349,191.0144035)(339.09975713,191.22534057)
\curveto(339.77944325,191.44409013)(340.53725421,191.55346491)(341.37319002,191.55346491)
\curveto(342.15443844,191.55346491)(342.94349934,191.45580885)(343.74037273,191.26049675)
\curveto(344.5450586,191.07299713)(345.213026,190.84252885)(345.74427492,190.5690919)
\lineto(345.74427492,188.21362792)
\lineto(345.62708766,188.21362792)
\curveto(345.0645888,188.62768958)(344.38099643,188.97534513)(343.57631056,189.25659456)
\curveto(342.77162469,189.54565647)(341.98256379,189.69018743)(341.20912785,189.69018743)
\curveto(340.40444198,189.69018743)(339.72475586,189.53393775)(339.17006948,189.22143838)
\curveto(338.61538311,188.9167515)(338.33803992,188.45972117)(338.33803992,187.8503474)
\curveto(338.33803992,187.311286)(338.50600833,186.90503682)(338.84194515,186.63159987)
\curveto(339.17006948,186.35816293)(339.70131841,186.13550713)(340.43569192,185.96363248)
\curveto(340.8419411,185.86988267)(341.29506518,185.77613285)(341.79506417,185.68238304)
\curveto(342.30287564,185.58863323)(342.72474978,185.50269591)(343.0606866,185.42457107)
\curveto(344.08412203,185.19019654)(344.87318293,184.78785361)(345.42786931,184.21754226)
\curveto(345.98255569,183.63941843)(346.25989888,182.87379498)(346.25989888,181.92067191)
\closepath
}
}
{
\newrgbcolor{curcolor}{0 0 0}
\pscustom[linestyle=none,fillstyle=solid,fillcolor=curcolor,opacity=0]
{
\newpath
\moveto(453.58440454,425.38538727)
\lineto(453.77338053,0.00042191)
}
}
{
\newrgbcolor{curcolor}{0 0 0}
\pscustom[linewidth=1.92650529,linecolor=curcolor]
{
\newpath
\moveto(453.58440454,425.38538727)
\lineto(453.77338053,0.00042191)
}
}
{
\newrgbcolor{curcolor}{0 0 0}
\pscustom[linestyle=none,fillstyle=solid,fillcolor=curcolor,opacity=0]
{
\newpath
\moveto(0.00016819,215.47751837)
\lineto(901.16368191,215.32003838)
}
}
{
\newrgbcolor{curcolor}{0 0 0}
\pscustom[linewidth=1.92650529,linecolor=curcolor]
{
\newpath
\moveto(0.00016819,215.47751837)
\lineto(901.16368191,215.32003838)
}
}
{
\newrgbcolor{curcolor}{0 1 0}
\pscustom[linestyle=none,fillstyle=solid,fillcolor=curcolor]
{
\newpath
\moveto(502.77364876,254.68751771)
\lineto(502.77364876,254.68751771)
\curveto(502.77364876,254.70200587)(502.78538102,254.71373813)(502.79986918,254.71373813)
\lineto(580.29055215,254.71373813)
\curveto(580.29753377,254.71373813)(580.30417417,254.71098223)(580.3090823,254.70607411)
\curveto(580.31401667,254.70113973)(580.31679882,254.69447308)(580.31679882,254.68751771)
\lineto(580.31679882,231.52740713)
\curveto(580.31679882,231.51294522)(580.30504031,231.50118671)(580.29055215,231.50118671)
\lineto(502.79986918,231.50118671)
\curveto(502.78538102,231.50118671)(502.77364876,231.51294522)(502.77364876,231.52740713)
\closepath
}
}
{
\newrgbcolor{curcolor}{0 0 0}
\pscustom[linewidth=1.92650529,linecolor=curcolor]
{
\newpath
\moveto(502.77364876,254.68751771)
\lineto(502.77364876,254.68751771)
\curveto(502.77364876,254.70200587)(502.78538102,254.71373813)(502.79986918,254.71373813)
\lineto(580.29055215,254.71373813)
\curveto(580.29753377,254.71373813)(580.30417417,254.71098223)(580.3090823,254.70607411)
\curveto(580.31401667,254.70113973)(580.31679882,254.69447308)(580.31679882,254.68751771)
\lineto(580.31679882,231.52740713)
\curveto(580.31679882,231.51294522)(580.30504031,231.50118671)(580.29055215,231.50118671)
\lineto(502.79986918,231.50118671)
\curveto(502.78538102,231.50118671)(502.77364876,231.51294522)(502.77364876,231.52740713)
\closepath
}
}
{
\newrgbcolor{curcolor}{0 1 0}
\pscustom[linestyle=none,fillstyle=solid,fillcolor=curcolor]
{
\newpath
\moveto(33.24011663,29.35256719)
\lineto(33.24011663,29.35256719)
\curveto(33.24011663,29.36700286)(33.25181739,29.37868263)(33.26625306,29.37868263)
\lineto(110.75711976,29.37868263)
\curveto(110.76404888,29.37868263)(110.77070766,29.37592673)(110.77560791,29.37107109)
\curveto(110.78050554,29.36613672)(110.78326144,29.35949631)(110.78326144,29.35256719)
\lineto(110.78326144,6.06631513)
\curveto(110.78326144,6.05187947)(110.77155543,6.04014721)(110.75711976,6.04014721)
\lineto(33.26625306,6.04014721)
\curveto(33.25181739,6.04014721)(33.24011663,6.05187947)(33.24011663,6.06631513)
\closepath
}
}
{
\newrgbcolor{curcolor}{0 0 0}
\pscustom[linewidth=1.92650529,linecolor=curcolor]
{
\newpath
\moveto(33.24011663,29.35256719)
\lineto(33.24011663,29.35256719)
\curveto(33.24011663,29.36700286)(33.25181739,29.37868263)(33.26625306,29.37868263)
\lineto(110.75711976,29.37868263)
\curveto(110.76404888,29.37868263)(110.77070766,29.37592673)(110.77560791,29.37107109)
\curveto(110.78050554,29.36613672)(110.78326144,29.35949631)(110.78326144,29.35256719)
\lineto(110.78326144,6.06631513)
\curveto(110.78326144,6.05187947)(110.77155543,6.04014721)(110.75711976,6.04014721)
\lineto(33.26625306,6.04014721)
\curveto(33.25181739,6.04014721)(33.24011663,6.05187947)(33.24011663,6.06631513)
\closepath
}
}
{
\newrgbcolor{curcolor}{0 0 0}
\pscustom[linestyle=none,fillstyle=solid,fillcolor=curcolor,opacity=0]
{
\newpath
\moveto(606.82839868,529.07641089)
\lineto(921.7883908,529.07641089)
\lineto(921.7883908,493.51742778)
\lineto(606.82839868,493.51742778)
\closepath
}
}
\end{pspicture}
}
    \captionsetup{width=0.75\linewidth}
    \caption{Copy-on-write filesystems avoid changing any of the existing data on disk, and instead write all changes to new locations on disk\cite{ahrens_openzfs_basics}.}
    \label{fig:cow}
\end{figure}

\chapter{ZFS}
ZFS, formerly known as the Zettabyte Filesystem, is a filesystem developed originally for the Solaris operating system
by Jeff Bonwick and Matt Ahrens \cite{ahrens_read_write}. 
Its most innovative feature is combining the roles of volume manager and filesystem, which allows ZFS to pool storage in a way that 
works much like allocating memory in a program\cite{mckusick_zfs_2015_presentation}.
A storage pool can consist of multiple disks in various configurations, and all of the available storage of those disks
can be used by any of the filesystems that are present in the pool.
ZFS is also Copy on Write, as discussed above, which both ensures that the filesystem can never be in an inconsistent state on disk
and allows for snapshots, as old versions can be kept around with ease by simply not deleting old blocks.
Snapshots also only take up the extra space of their difference between the current state of the filesystem,
as the blocks used by both are not duplicated, but instead simply pointed to by both the snapshot and live filesystem.

\section{Block Pointers}
Block pointers are the fundamental structure on which ZFS's most powerful features are built. 
They are 128-byte structures that uniquely identify a block in a storage pool
\cite{ahrens_openzfs_basics,ahrens_read_write,mckusick_zfs_2015_presentation}. 
They can contain up to three different addresses for three different copies on different virtual devices, or VDEVs,
that represent logical, not physical, disks.
VDEVs can be just one physical disk, 
or they can be a mirror, where multiple disks contain exactly the same data,
or a RAIDZ pool,  where data are striped across multiple disks and a parity block is appended 
so that data can be recovered in the event of a disk failure.
These addresses are Data Virtual Addresses, and they contain a VDEV number, a size, as blocks can be variably-sized in ZFS, 
though they default to 128K, and an offset, the distance in bytes from the start of the logical disk
to the start of the block.
By default, metadata are stored twice, and pool-wide metadata are stored three times.
However, there can only be two copies if the pool is encrypted, as encryption key information is stored in the third DVA in the block pointer. 
Pool-wide metadata are not encrypted and thus are not affected by this limitation, even though they
are stored three times in the pool.

\begin{figure}[H]
    \centering
    \resizebox{0.75\linewidth}{!}{%LaTeX with PSTricks extensions
%%Creator: Inkscape 1.0.2-2 (e86c870879, 2021-01-15)
%%Please note this file requires PSTricks extensions
\psset{xunit=.5pt,yunit=.5pt,runit=.5pt}
\begin{pspicture}(937.28155518,560.4675293)
{
\newrgbcolor{curcolor}{0.90980393 0.90980393 0.90980393}
\pscustom[linestyle=none,fillstyle=solid,fillcolor=curcolor]
{
\newpath
\moveto(736.78197308,527.17080068)
\lineto(736.78197308,527.17080068)
\curveto(736.78197308,530.89624721)(739.80207195,533.91632508)(743.52752372,533.91632508)
\lineto(930.03601743,533.91632508)
\curveto(931.82504268,533.91632508)(933.54083973,533.20563623)(934.80585029,531.9406073)
\curveto(936.07086085,530.67557049)(936.78156808,528.95982593)(936.78156808,527.17080068)
\lineto(936.78156808,500.18950362)
\curveto(936.78156808,496.46403872)(933.7614692,493.44396609)(930.03601743,493.44396609)
\lineto(743.52752372,493.44396609)
\lineto(743.52752372,493.44396609)
\curveto(739.80207195,493.44396609)(736.78197308,496.46403872)(736.78197308,500.18950362)
\closepath
}
}
{
\newrgbcolor{curcolor}{0.10196079 0.8392157 0.96078432}
\pscustom[linewidth=0.99999798,linecolor=curcolor]
{
\newpath
\moveto(736.78197308,527.17080068)
\lineto(736.78197308,527.17080068)
\curveto(736.78197308,530.89624721)(739.80207195,533.91632508)(743.52752372,533.91632508)
\lineto(930.03601743,533.91632508)
\curveto(931.82504268,533.91632508)(933.54083973,533.20563623)(934.80585029,531.9406073)
\curveto(936.07086085,530.67557049)(936.78156808,528.95982593)(936.78156808,527.17080068)
\lineto(936.78156808,500.18950362)
\curveto(936.78156808,496.46403872)(933.7614692,493.44396609)(930.03601743,493.44396609)
\lineto(743.52752372,493.44396609)
\lineto(743.52752372,493.44396609)
\curveto(739.80207195,493.44396609)(736.78197308,496.46403872)(736.78197308,500.18950362)
\closepath
}
}
{
\newrgbcolor{curcolor}{0 0 0}
\pscustom[linestyle=none,fillstyle=solid,fillcolor=curcolor]
{
\newpath
\moveto(749.28891059,506.76016591)
\lineto(749.28891059,520.12214214)
\lineto(758.30323153,520.12214214)
\lineto(758.30323153,518.54531957)
\lineto(751.05713937,518.54531957)
\lineto(751.05713937,514.40729964)
\lineto(757.32797133,514.40729964)
\lineto(757.32797133,512.83047707)
\lineto(751.05713937,512.83047707)
\lineto(751.05713937,506.76016591)
\closepath
}
}
{
\newrgbcolor{curcolor}{0 0 0}
\pscustom[linestyle=none,fillstyle=solid,fillcolor=curcolor]
{
\newpath
\moveto(760.38388723,518.2354238)
\lineto(760.38388723,520.12214214)
\lineto(762.02451187,520.12214214)
\lineto(762.02451187,518.2354238)
\closepath
\moveto(760.38388723,506.76016591)
\lineto(760.38388723,516.43985128)
\lineto(762.02451187,516.43985128)
\lineto(762.02451187,506.76016591)
\closepath
}
}
{
\newrgbcolor{curcolor}{0 0 0}
\pscustom[linestyle=none,fillstyle=solid,fillcolor=curcolor]
{
\newpath
\moveto(764.50053515,506.76016591)
\lineto(764.50053515,516.43985128)
\lineto(765.97709732,516.43985128)
\lineto(765.97709732,514.97240369)
\curveto(766.35383335,515.65903548)(766.70018744,516.11172635)(767.01615959,516.3304763)
\curveto(767.33820813,516.54922626)(767.69063861,516.65860123)(768.07345103,516.65860123)
\curveto(768.6264023,516.65860123)(769.18846814,516.48238599)(769.75964857,516.12995552)
\lineto(769.19454453,514.60782043)
\curveto(768.79350295,514.84479955)(768.39246138,514.9632891)(767.9914198,514.9632891)
\curveto(767.63291293,514.9632891)(767.31086439,514.85391413)(767.02527418,514.63516418)
\curveto(766.73968396,514.42249061)(766.53612498,514.12474762)(766.41459723,513.74193521)
\curveto(766.2323056,513.158602)(766.14115979,512.52058131)(766.14115979,511.82787313)
\lineto(766.14115979,506.76016591)
\closepath
}
}
{
\newrgbcolor{curcolor}{0 0 0}
\pscustom[linestyle=none,fillstyle=solid,fillcolor=curcolor]
{
\newpath
\moveto(770.07850147,509.64948819)
\lineto(771.70089695,509.90469647)
\curveto(771.79204276,509.254523)(772.04421284,508.75625922)(772.4574072,508.40990513)
\curveto(772.87667794,508.06355104)(773.46001114,507.89037399)(774.20740681,507.89037399)
\curveto(774.96087887,507.89037399)(775.51990652,508.04228368)(775.88448978,508.34610306)
\curveto(776.24907303,508.65599882)(776.43136466,509.01754388)(776.43136466,509.43073824)
\curveto(776.43136466,509.80139788)(776.27034039,510.09306448)(775.94829185,510.30573805)
\curveto(775.72346551,510.45157135)(775.16443785,510.63690117)(774.27120888,510.86172751)
\curveto(773.06808415,511.16554688)(772.23258086,511.42683155)(771.76469902,511.6455815)
\curveto(771.30289356,511.87040784)(770.95046308,512.17726541)(770.70740758,512.56615422)
\curveto(770.47042847,512.96111941)(770.35193891,513.39558112)(770.35193891,513.86953935)
\curveto(770.35193891,514.30096286)(770.44916111,514.69896625)(770.64360551,515.0635495)
\curveto(770.8441263,515.43420914)(771.11452555,515.74106671)(771.45480325,515.98412221)
\curveto(771.71001153,516.17249023)(772.05636562,516.3304763)(772.49386552,516.45808044)
\curveto(772.93744181,516.59176097)(773.41140004,516.65860123)(773.91574021,516.65860123)
\curveto(774.67528865,516.65860123)(775.34065309,516.54922626)(775.91183352,516.3304763)
\curveto(776.48909034,516.11172635)(776.91443747,515.81398336)(777.18787491,515.43724733)
\curveto(777.46131235,515.06658769)(777.64968036,514.56832391)(777.75297895,513.942456)
\lineto(776.14881264,513.72370604)
\curveto(776.07589599,514.22196982)(775.86322242,514.61085863)(775.51079194,514.89037245)
\curveto(775.16443785,515.16988628)(774.67225046,515.3096432)(774.03422977,515.3096432)
\curveto(773.28075771,515.3096432)(772.74299741,515.18507725)(772.42094887,514.93594536)
\curveto(772.09890033,514.68681347)(771.93787606,514.39514687)(771.93787606,514.06094555)
\curveto(771.93787606,513.84827199)(772.00471632,513.65686578)(772.13839685,513.48672693)
\curveto(772.27207738,513.31051169)(772.48171275,513.16467839)(772.76730296,513.04922703)
\curveto(772.93136543,512.98846315)(773.41443824,512.84870624)(774.21652139,512.62995628)
\curveto(775.37711142,512.32006052)(776.18527096,512.06485224)(776.64100003,511.86433145)
\curveto(777.10280548,511.66988705)(777.46435054,511.38429684)(777.72563521,511.00756081)
\curveto(777.98691987,510.63082478)(778.1175622,510.16294294)(778.1175622,509.60391528)
\curveto(778.1175622,509.0570404)(777.95653793,508.54054746)(777.63448939,508.05443646)
\curveto(777.31851724,507.57440184)(776.85974998,507.20070401)(776.25818761,506.93334295)
\curveto(775.65662524,506.67205829)(774.97606984,506.54141596)(774.21652139,506.54141596)
\curveto(772.95870917,506.54141596)(771.99863994,506.80270062)(771.33631369,507.32526995)
\curveto(770.68006384,507.84783928)(770.2607931,508.62257869)(770.07850147,509.64948819)
\closepath
}
}
{
\newrgbcolor{curcolor}{0 0 0}
\pscustom[linestyle=none,fillstyle=solid,fillcolor=curcolor]
{
\newpath
\moveto(783.65009495,508.2276135)
\lineto(783.88707406,506.77839507)
\curveto(783.42526861,506.68117287)(783.01207426,506.63256177)(782.647491,506.63256177)
\curveto(782.05200502,506.63256177)(781.59019957,506.72674578)(781.26207464,506.91511379)
\curveto(780.93394971,507.10348181)(780.70304699,507.3495755)(780.56936646,507.65339488)
\curveto(780.43568593,507.96329064)(780.36884567,508.61042592)(780.36884567,509.5948007)
\lineto(780.36884567,515.16380989)
\lineto(779.16572094,515.16380989)
\lineto(779.16572094,516.43985128)
\lineto(780.36884567,516.43985128)
\lineto(780.36884567,518.83698617)
\lineto(782.00035573,519.82136095)
\lineto(782.00035573,516.43985128)
\lineto(783.65009495,516.43985128)
\lineto(783.65009495,515.16380989)
\lineto(782.00035573,515.16380989)
\lineto(782.00035573,509.50365489)
\curveto(782.00035573,509.03577305)(782.02769947,508.73499186)(782.08238696,508.60131134)
\curveto(782.14315084,508.46763081)(782.23733484,508.36129403)(782.36493898,508.28230099)
\curveto(782.49861951,508.20330795)(782.68698752,508.16381143)(782.93004303,508.16381143)
\curveto(783.11233465,508.16381143)(783.35235196,508.18507879)(783.65009495,508.2276135)
\closepath
}
}
{
\newrgbcolor{curcolor}{0 0 0}
\pscustom[linestyle=none,fillstyle=solid,fillcolor=curcolor]
{
\newpath
\moveto(796.7257837,510.30573805)
\lineto(798.3390646,510.09610267)
\curveto(798.16284936,508.98412375)(797.71015849,508.11216214)(796.98099198,507.48021783)
\curveto(796.25790186,506.85434992)(795.36771109,506.54141596)(794.31041965,506.54141596)
\curveto(792.98576717,506.54141596)(791.91936115,506.97283947)(791.11120161,507.83568651)
\curveto(790.30911845,508.70460993)(789.90807687,509.94723118)(789.90807687,511.56355027)
\curveto(789.90807687,512.60868893)(790.08125392,513.52318525)(790.42760801,514.30703925)
\curveto(790.7739621,515.09089324)(791.29956962,515.67726464)(792.00443058,516.06615345)
\curveto(792.71536792,516.46111864)(793.48706914,516.65860123)(794.31953423,516.65860123)
\curveto(795.37074928,516.65860123)(796.23055812,516.39124018)(796.89896075,515.85651808)
\curveto(797.56736338,515.32787236)(797.9957487,514.5744003)(798.18411672,513.59610191)
\lineto(796.58906498,513.35000821)
\curveto(796.4371553,514.00018168)(796.16675605,514.48933088)(795.77786725,514.8174558)
\curveto(795.39505483,515.14558073)(794.93021118,515.3096432)(794.3833363,515.3096432)
\curveto(793.5569476,515.3096432)(792.88550677,515.01190021)(792.36901383,514.41641423)
\curveto(791.85252089,513.82700463)(791.59427442,512.89124095)(791.59427442,511.60912318)
\curveto(791.59427442,510.30877624)(791.84340631,509.36389797)(792.34167009,508.77448838)
\curveto(792.83993386,508.18507879)(793.49010733,507.89037399)(794.29219049,507.89037399)
\curveto(794.93628757,507.89037399)(795.47404787,508.08785659)(795.90547139,508.48282178)
\curveto(796.3368949,508.87778697)(796.61033234,509.48542573)(796.7257837,510.30573805)
\closepath
}
}
{
\newrgbcolor{curcolor}{0 0 0}
\pscustom[linestyle=none,fillstyle=solid,fillcolor=curcolor]
{
\newpath
\moveto(799.13201506,511.60000859)
\curveto(799.13201506,513.39254292)(799.63027884,514.7202336)(800.62680639,515.58308064)
\curveto(801.45927149,516.30009437)(802.47402821,516.65860123)(803.67107656,516.65860123)
\curveto(805.00180543,516.65860123)(806.0894788,516.22110133)(806.93409667,515.34610152)
\curveto(807.77871454,514.4771781)(808.20102348,513.27405337)(808.20102348,511.73672731)
\curveto(808.20102348,510.49106787)(808.01265546,509.50973128)(807.63591944,508.79271754)
\curveto(807.2652598,508.0817802)(806.72142311,507.52882893)(806.00440938,507.13386374)
\curveto(805.29347203,506.73889855)(804.51569443,506.54141596)(803.67107656,506.54141596)
\curveto(802.31604213,506.54141596)(801.21925418,506.97587767)(800.3807127,507.84480109)
\curveto(799.5482476,508.71372451)(799.13201506,509.96546034)(799.13201506,511.60000859)
\closepath
\moveto(800.8182126,511.60000859)
\curveto(800.8182126,510.36042553)(801.08861185,509.43073824)(801.62941034,508.81094671)
\curveto(802.17020883,508.19723156)(802.85076424,507.89037399)(803.67107656,507.89037399)
\curveto(804.48531249,507.89037399)(805.1628297,508.20026976)(805.70362819,508.82006129)
\curveto(806.24442669,509.43985282)(806.51482593,510.38473108)(806.51482593,511.65469608)
\curveto(806.51482593,512.85174443)(806.24138849,513.75712618)(805.69451361,514.37084132)
\curveto(805.15371512,514.99063285)(804.4792361,515.30052861)(803.67107656,515.30052861)
\curveto(802.85076424,515.30052861)(802.17020883,514.99367104)(801.62941034,514.3799559)
\curveto(801.08861185,513.76624076)(800.8182126,512.83959166)(800.8182126,511.60000859)
\closepath
}
}
{
\newrgbcolor{curcolor}{0 0 0}
\pscustom[linestyle=none,fillstyle=solid,fillcolor=curcolor]
{
\newpath
\moveto(810.12133691,503.05053131)
\lineto(810.12133691,516.43985128)
\lineto(811.61612825,516.43985128)
\lineto(811.61612825,515.18203906)
\curveto(811.96855873,515.67422645)(812.36656211,516.0418479)(812.8101384,516.2849034)
\curveto(813.2537147,516.53403529)(813.79147499,516.65860123)(814.4234193,516.65860123)
\curveto(815.24980801,516.65860123)(815.97897451,516.44592767)(816.61091882,516.02058054)
\curveto(817.24286312,515.59523341)(817.71985955,514.99367104)(818.04190809,514.21589344)
\curveto(818.36395663,513.44419222)(818.5249809,512.59653615)(818.5249809,511.67292524)
\curveto(818.5249809,510.68247407)(818.34572747,509.7892451)(817.9872206,508.99323833)
\curveto(817.63479012,508.20330795)(817.11829718,507.5956692)(816.43774177,507.17032207)
\curveto(815.76326276,506.75105133)(815.05232541,506.54141596)(814.30492974,506.54141596)
\curveto(813.75805486,506.54141596)(813.26586747,506.65686732)(812.82836757,506.88777005)
\curveto(812.39694405,507.11867277)(812.04147538,507.41033938)(811.76196155,507.76276985)
\lineto(811.76196155,503.05053131)
\closepath
\moveto(811.60701367,511.54532111)
\curveto(811.60701367,510.29966166)(811.85918375,509.37908894)(812.36352392,508.78360296)
\curveto(812.86786409,508.18811698)(813.47854104,507.89037399)(814.19555477,507.89037399)
\curveto(814.92472127,507.89037399)(815.547551,508.19723156)(816.06404394,508.81094671)
\curveto(816.58661327,509.43073824)(816.84789793,510.38776928)(816.84789793,511.68203983)
\curveto(816.84789793,512.9155465)(816.59268966,513.83915741)(816.0822731,514.45287255)
\curveto(815.57793293,515.06658769)(814.97333237,515.37344526)(814.26847142,515.37344526)
\curveto(813.56968685,515.37344526)(812.94989532,515.04532034)(812.40909683,514.38907048)
\curveto(811.87437472,513.73889701)(811.60701367,512.79098055)(811.60701367,511.54532111)
\closepath
}
}
{
\newrgbcolor{curcolor}{0 0 0}
\pscustom[linestyle=none,fillstyle=solid,fillcolor=curcolor]
{
\newpath
\moveto(820.42706898,503.03230214)
\lineto(820.24477736,504.57266639)
\curveto(820.60328422,504.47544419)(820.91621818,504.42683309)(821.18357923,504.42683309)
\curveto(821.54816249,504.42683309)(821.83982909,504.48759696)(822.05857904,504.60912471)
\curveto(822.27732899,504.73065247)(822.45658243,504.90079132)(822.59633934,505.11954127)
\curveto(822.69963793,505.28360373)(822.86673859,505.6907217)(823.09764131,506.34089517)
\curveto(823.12802325,506.43204098)(823.17663435,506.56572151)(823.24347461,506.74193675)
\lineto(819.57029834,516.43985128)
\lineto(821.33852712,516.43985128)
\lineto(823.35284959,510.83438376)
\curveto(823.61413426,510.12344642)(823.84807518,509.37605075)(824.05467235,508.59219676)
\curveto(824.24304037,509.34566881)(824.46786671,510.08091171)(824.72915137,510.79792544)
\lineto(826.79816133,516.43985128)
\lineto(828.43878597,516.43985128)
\lineto(824.75649512,506.59610344)
\curveto(824.36152992,505.53273562)(824.05467235,504.80053092)(823.8359224,504.39948934)
\curveto(823.5442558,503.85869085)(823.21005448,503.46372566)(822.83331845,503.21459377)
\curveto(822.45658243,502.95938549)(822.00692975,502.83178136)(821.48436042,502.83178136)
\curveto(821.16838827,502.83178136)(820.81595779,502.89862162)(820.42706898,503.03230214)
\closepath
}
}
{
\newrgbcolor{curcolor}{0 0 0}
\pscustom[linestyle=none,fillstyle=solid,fillcolor=curcolor]
{
\newpath
\moveto(834.39327712,511.60000859)
\curveto(834.39327712,513.39254292)(834.8915409,514.7202336)(835.88806846,515.58308064)
\curveto(836.72053355,516.30009437)(837.73529027,516.65860123)(838.93233862,516.65860123)
\curveto(840.26306749,516.65860123)(841.35074087,516.22110133)(842.19535874,515.34610152)
\curveto(843.03997661,514.4771781)(843.46228554,513.27405337)(843.46228554,511.73672731)
\curveto(843.46228554,510.49106787)(843.27391753,509.50973128)(842.8971815,508.79271754)
\curveto(842.52652186,508.0817802)(841.98268517,507.52882893)(841.26567144,507.13386374)
\curveto(840.5547341,506.73889855)(839.77695649,506.54141596)(838.93233862,506.54141596)
\curveto(837.5773042,506.54141596)(836.48051624,506.97587767)(835.64197476,507.84480109)
\curveto(834.80950966,508.71372451)(834.39327712,509.96546034)(834.39327712,511.60000859)
\closepath
\moveto(836.07947466,511.60000859)
\curveto(836.07947466,510.36042553)(836.34987391,509.43073824)(836.8906724,508.81094671)
\curveto(837.43147089,508.19723156)(838.1120263,507.89037399)(838.93233862,507.89037399)
\curveto(839.74657455,507.89037399)(840.42409176,508.20026976)(840.96489026,508.82006129)
\curveto(841.50568875,509.43985282)(841.77608799,510.38473108)(841.77608799,511.65469608)
\curveto(841.77608799,512.85174443)(841.50265055,513.75712618)(840.95577567,514.37084132)
\curveto(840.41497718,514.99063285)(839.74049816,515.30052861)(838.93233862,515.30052861)
\curveto(838.1120263,515.30052861)(837.43147089,514.99367104)(836.8906724,514.3799559)
\curveto(836.34987391,513.76624076)(836.07947466,512.83959166)(836.07947466,511.60000859)
\closepath
}
}
{
\newrgbcolor{curcolor}{0 0 0}
\pscustom[linestyle=none,fillstyle=solid,fillcolor=curcolor]
{
\newpath
\moveto(845.77452216,506.76016591)
\lineto(845.77452216,515.16380989)
\lineto(844.32530372,515.16380989)
\lineto(844.32530372,516.43985128)
\lineto(845.77452216,516.43985128)
\lineto(845.77452216,517.46979897)
\curveto(845.77452216,518.11997244)(845.83224784,518.60304525)(845.9476992,518.9190174)
\curveto(846.10568528,519.34436453)(846.38216091,519.68768043)(846.7771261,519.94896509)
\curveto(847.17816768,520.21632614)(847.73719534,520.35000667)(848.45420907,520.35000667)
\curveto(848.91601452,520.35000667)(849.42643108,520.29531918)(849.98545873,520.18594421)
\lineto(849.73936503,518.75495494)
\curveto(849.39908733,518.81571881)(849.07703879,518.84610075)(848.77321941,518.84610075)
\curveto(848.27495563,518.84610075)(847.92252516,518.73976397)(847.71592798,518.5270904)
\curveto(847.5093308,518.31441684)(847.40603221,517.91641346)(847.40603221,517.33308025)
\lineto(847.40603221,516.43985128)
\lineto(849.29275055,516.43985128)
\lineto(849.29275055,515.16380989)
\lineto(847.40603221,515.16380989)
\lineto(847.40603221,506.76016591)
\closepath
}
}
{
\newrgbcolor{curcolor}{0 0 0}
\pscustom[linestyle=none,fillstyle=solid,fillcolor=curcolor]
{
\newpath
\moveto(862.00385224,506.76016591)
\lineto(862.00385224,507.98151981)
\curveto(861.39013709,507.02145057)(860.48779354,506.54141596)(859.29682158,506.54141596)
\curveto(858.52512036,506.54141596)(857.81418302,506.75408952)(857.16400955,507.17943665)
\curveto(856.51991247,507.60478378)(856.0186105,508.19723156)(855.66010363,508.95678001)
\curveto(855.30767315,509.72240484)(855.13145791,510.60044284)(855.13145791,511.59089401)
\curveto(855.13145791,512.55703963)(855.29248218,513.43203944)(855.61453072,514.21589344)
\curveto(855.93657926,515.00582382)(856.41965207,515.61042438)(857.06374916,516.02969512)
\curveto(857.70784624,516.44896586)(858.42789816,516.65860123)(859.22390493,516.65860123)
\curveto(859.80723814,516.65860123)(860.32676927,516.53403529)(860.78249834,516.2849034)
\curveto(861.2382274,516.0418479)(861.60888704,515.72283755)(861.89447726,515.32787236)
\lineto(861.89447726,520.12214214)
\lineto(863.52598732,520.12214214)
\lineto(863.52598732,506.76016591)
\closepath
\moveto(856.81765546,511.59089401)
\curveto(856.81765546,510.35131095)(857.07894012,509.42466185)(857.60150945,508.81094671)
\curveto(858.12407878,508.19723156)(858.74083212,507.89037399)(859.45176946,507.89037399)
\curveto(860.16878319,507.89037399)(860.77642195,508.1820406)(861.27468573,508.7653738)
\curveto(861.7790259,509.35478339)(862.03119598,510.25105056)(862.03119598,511.45417529)
\curveto(862.03119598,512.77882778)(861.7759877,513.75104979)(861.26557115,514.37084132)
\curveto(860.75515459,514.99063285)(860.12624848,515.30052861)(859.37885281,515.30052861)
\curveto(858.64968631,515.30052861)(858.03900936,515.00278562)(857.54682197,514.40729964)
\curveto(857.06071096,513.81181366)(856.81765546,512.87301179)(856.81765546,511.59089401)
\closepath
}
}
{
\newrgbcolor{curcolor}{0 0 0}
\pscustom[linestyle=none,fillstyle=solid,fillcolor=curcolor]
{
\newpath
\moveto(872.41895928,507.95417606)
\curveto(871.81132053,507.43768312)(871.22494913,507.07309987)(870.65984509,506.8604263)
\curveto(870.10081743,506.64775274)(869.49925506,506.54141596)(868.85515798,506.54141596)
\curveto(867.79179016,506.54141596)(866.97451604,506.79966243)(866.40333561,507.31615537)
\curveto(865.83215518,507.8387247)(865.54656496,508.50408914)(865.54656496,509.31224868)
\curveto(865.54656496,509.78620691)(865.65290174,510.21763043)(865.86557531,510.60651923)
\curveto(866.08432526,511.00148442)(866.36687728,511.31745657)(866.71323137,511.55443569)
\curveto(867.06566185,511.7914148)(867.46062704,511.97066824)(867.89812694,512.09219599)
\curveto(868.22017548,512.17726541)(868.70628649,512.25929664)(869.35645996,512.33828968)
\curveto(870.68111244,512.49627576)(871.65637265,512.68464377)(872.28224056,512.90339372)
\curveto(872.28831695,513.12822006)(872.29135514,513.27101517)(872.29135514,513.33177905)
\curveto(872.29135514,514.00018168)(872.13640726,514.47110171)(871.8265115,514.74453915)
\curveto(871.40724076,515.11519879)(870.78441103,515.30052861)(869.95802232,515.30052861)
\curveto(869.1863211,515.30052861)(868.61514067,515.16380989)(868.24448103,514.89037245)
\curveto(867.87989778,514.6230114)(867.60949853,514.14601498)(867.4332833,513.45938319)
\lineto(865.82911698,513.67813314)
\curveto(865.97495028,514.36476493)(866.21496759,514.9177162)(866.54916891,515.33698694)
\curveto(866.88337022,515.76233407)(867.36644303,516.0874208)(867.99838734,516.31224714)
\curveto(868.63033164,516.54314987)(869.36253634,516.65860123)(870.19500144,516.65860123)
\curveto(871.02139015,516.65860123)(871.69283097,516.56137903)(872.20932391,516.36693463)
\curveto(872.72581685,516.17249023)(873.10559108,515.92639653)(873.34864658,515.62865354)
\curveto(873.59170208,515.33698694)(873.76184093,514.9663273)(873.85906313,514.51667462)
\curveto(873.91375062,514.23716079)(873.94109437,513.73282063)(873.94109437,513.00365412)
\lineto(873.94109437,510.8161546)
\curveto(873.94109437,509.29098132)(873.9745145,508.3248357)(874.04135476,507.91771774)
\curveto(874.11427141,507.51667616)(874.25402832,507.13082555)(874.4606255,506.76016591)
\lineto(872.74708421,506.76016591)
\curveto(872.57694536,507.10044361)(872.46757038,507.498447)(872.41895928,507.95417606)
\closepath
\moveto(872.28224056,511.61823776)
\curveto(871.68675458,511.37518225)(870.79352561,511.16858508)(869.60255365,510.99844623)
\curveto(868.92807463,510.90122403)(868.45107821,510.79184905)(868.17156438,510.6703213)
\curveto(867.89205056,510.54879355)(867.6763388,510.36954011)(867.52442911,510.132561)
\curveto(867.37251942,509.90165827)(867.29656458,509.6434118)(867.29656458,509.35782159)
\curveto(867.29656458,508.92032168)(867.46062704,508.55573843)(867.78875197,508.26407183)
\curveto(868.12295328,507.97240523)(868.60906429,507.82657192)(869.24708498,507.82657192)
\curveto(869.87902929,507.82657192)(870.44109513,507.96329064)(870.93328253,508.23672808)
\curveto(871.42546992,508.51624191)(871.78701498,508.89601613)(872.0179177,509.37605075)
\curveto(872.19413294,509.74671039)(872.28224056,510.29358527)(872.28224056,511.01667539)
\closepath
}
}
{
\newrgbcolor{curcolor}{0 0 0}
\pscustom[linestyle=none,fillstyle=solid,fillcolor=curcolor]
{
\newpath
\moveto(880.06323361,508.2276135)
\lineto(880.30021272,506.77839507)
\curveto(879.83840727,506.68117287)(879.42521291,506.63256177)(879.06062966,506.63256177)
\curveto(878.46514368,506.63256177)(878.00333823,506.72674578)(877.6752133,506.91511379)
\curveto(877.34708837,507.10348181)(877.11618564,507.3495755)(876.98250512,507.65339488)
\curveto(876.84882459,507.96329064)(876.78198433,508.61042592)(876.78198433,509.5948007)
\lineto(876.78198433,515.16380989)
\lineto(875.57885959,515.16380989)
\lineto(875.57885959,516.43985128)
\lineto(876.78198433,516.43985128)
\lineto(876.78198433,518.83698617)
\lineto(878.41349439,519.82136095)
\lineto(878.41349439,516.43985128)
\lineto(880.06323361,516.43985128)
\lineto(880.06323361,515.16380989)
\lineto(878.41349439,515.16380989)
\lineto(878.41349439,509.50365489)
\curveto(878.41349439,509.03577305)(878.44083813,508.73499186)(878.49552562,508.60131134)
\curveto(878.55628949,508.46763081)(878.6504735,508.36129403)(878.77807764,508.28230099)
\curveto(878.91175816,508.20330795)(879.10012618,508.16381143)(879.34318168,508.16381143)
\curveto(879.52547331,508.16381143)(879.76549062,508.18507879)(880.06323361,508.2276135)
\closepath
}
}
{
\newrgbcolor{curcolor}{0 0 0}
\pscustom[linestyle=none,fillstyle=solid,fillcolor=curcolor]
{
\newpath
\moveto(887.96826709,507.95417606)
\curveto(887.36062833,507.43768312)(886.77425694,507.07309987)(886.20915289,506.8604263)
\curveto(885.65012524,506.64775274)(885.04856287,506.54141596)(884.40446579,506.54141596)
\curveto(883.34109797,506.54141596)(882.52382384,506.79966243)(881.95264341,507.31615537)
\curveto(881.38146298,507.8387247)(881.09587277,508.50408914)(881.09587277,509.31224868)
\curveto(881.09587277,509.78620691)(881.20220955,510.21763043)(881.41488311,510.60651923)
\curveto(881.63363307,511.00148442)(881.91618509,511.31745657)(882.26253918,511.55443569)
\curveto(882.61496966,511.7914148)(883.00993485,511.97066824)(883.44743475,512.09219599)
\curveto(883.76948329,512.17726541)(884.2555943,512.25929664)(884.90576776,512.33828968)
\curveto(886.23042025,512.49627576)(887.20568045,512.68464377)(887.83154837,512.90339372)
\curveto(887.83762476,513.12822006)(887.84066295,513.27101517)(887.84066295,513.33177905)
\curveto(887.84066295,514.00018168)(887.68571507,514.47110171)(887.3758193,514.74453915)
\curveto(886.95654856,515.11519879)(886.33371884,515.30052861)(885.50733013,515.30052861)
\curveto(884.73562891,515.30052861)(884.16444848,515.16380989)(883.79378884,514.89037245)
\curveto(883.42920559,514.6230114)(883.15880634,514.14601498)(882.9825911,513.45938319)
\lineto(881.37842479,513.67813314)
\curveto(881.52425809,514.36476493)(881.7642754,514.9177162)(882.09847671,515.33698694)
\curveto(882.43267803,515.76233407)(882.91575084,516.0874208)(883.54769515,516.31224714)
\curveto(884.17963945,516.54314987)(884.91184415,516.65860123)(885.74430925,516.65860123)
\curveto(886.57069795,516.65860123)(887.24213878,516.56137903)(887.75863172,516.36693463)
\curveto(888.27512466,516.17249023)(888.65489888,515.92639653)(888.89795439,515.62865354)
\curveto(889.14100989,515.33698694)(889.31114874,514.9663273)(889.40837094,514.51667462)
\curveto(889.46305843,514.23716079)(889.49040217,513.73282063)(889.49040217,513.00365412)
\lineto(889.49040217,510.8161546)
\curveto(889.49040217,509.29098132)(889.5238223,508.3248357)(889.59066257,507.91771774)
\curveto(889.66357922,507.51667616)(889.80333613,507.13082555)(890.00993331,506.76016591)
\lineto(888.29639202,506.76016591)
\curveto(888.12625317,507.10044361)(888.01687819,507.498447)(887.96826709,507.95417606)
\closepath
\moveto(887.83154837,511.61823776)
\curveto(887.23606239,511.37518225)(886.34283342,511.16858508)(885.15186146,510.99844623)
\curveto(884.47738244,510.90122403)(884.00038602,510.79184905)(883.72087219,510.6703213)
\curveto(883.44135836,510.54879355)(883.2256466,510.36954011)(883.07373692,510.132561)
\curveto(882.92182723,509.90165827)(882.84587238,509.6434118)(882.84587238,509.35782159)
\curveto(882.84587238,508.92032168)(883.00993485,508.55573843)(883.33805977,508.26407183)
\curveto(883.67226109,507.97240523)(884.15837209,507.82657192)(884.79639279,507.82657192)
\curveto(885.42833709,507.82657192)(885.99040294,507.96329064)(886.48259033,508.23672808)
\curveto(886.97477773,508.51624191)(887.33632278,508.89601613)(887.56722551,509.37605075)
\curveto(887.74344075,509.74671039)(887.83154837,510.29358527)(887.83154837,511.01667539)
\closepath
}
}
{
\newrgbcolor{curcolor}{0.90980393 0.90980393 0.90980393}
\pscustom[linestyle=none,fillstyle=solid,fillcolor=curcolor]
{
\newpath
\moveto(0.49999948,261.35490852)
\lineto(0.49999948,261.35490852)
\curveto(0.49999948,269.49289729)(7.09715935,276.09007553)(15.23517174,276.09007553)
\lineto(136.81963682,276.09007553)
\curveto(140.72765253,276.09007553)(144.47559245,274.53763773)(147.23897793,271.7742575)
\curveto(150.00236341,269.01085102)(151.55481171,265.26290585)(151.55481171,261.35490852)
\lineto(151.55481171,202.41597275)
\curveto(151.55481171,194.27798398)(144.95765184,187.68080574)(136.81963682,187.68080574)
\lineto(15.23517174,187.68080574)
\curveto(7.09715935,187.68080574)(0.49999948,194.27798398)(0.49999948,202.41597275)
\closepath
}
}
{
\newrgbcolor{curcolor}{0.10196079 0.8392157 0.96078432}
\pscustom[linewidth=0.99999798,linecolor=curcolor]
{
\newpath
\moveto(0.49999948,261.35490852)
\lineto(0.49999948,261.35490852)
\curveto(0.49999948,269.49289729)(7.09715935,276.09007553)(15.23517174,276.09007553)
\lineto(136.81963682,276.09007553)
\curveto(140.72765253,276.09007553)(144.47559245,274.53763773)(147.23897793,271.7742575)
\curveto(150.00236341,269.01085102)(151.55481171,265.26290585)(151.55481171,261.35490852)
\lineto(151.55481171,202.41597275)
\curveto(151.55481171,194.27798398)(144.95765184,187.68080574)(136.81963682,187.68080574)
\lineto(15.23517174,187.68080574)
\curveto(7.09715935,187.68080574)(0.49999948,194.27798398)(0.49999948,202.41597275)
\closepath
}
}
{
\newrgbcolor{curcolor}{0 0 0}
\pscustom[linestyle=none,fillstyle=solid,fillcolor=curcolor]
{
\newpath
\moveto(17.58920183,246.96541899)
\lineto(14.0436297,260.32739522)
\lineto(15.85743138,260.32739522)
\lineto(17.88998302,251.56828257)
\curveto(18.10873297,250.65074804)(18.29710098,249.73928991)(18.45508706,248.83390817)
\curveto(18.79536476,250.26185924)(18.99588555,251.08520975)(19.05664943,251.30395971)
\lineto(21.59961762,260.32739522)
\lineto(23.73242965,260.32739522)
\lineto(25.64649173,253.56437588)
\curveto(26.12652635,251.88729291)(26.47288044,250.31047034)(26.685554,248.83390817)
\curveto(26.85569285,249.67852604)(27.077481,250.64770985)(27.35091844,251.74145961)
\lineto(29.44727214,260.32739522)
\lineto(31.2246155,260.32739522)
\lineto(27.56055381,246.96541899)
\lineto(25.8561271,246.96541899)
\lineto(23.03972147,257.14640634)
\curveto(22.80274236,257.9971006)(22.66298544,258.51966993)(22.62045073,258.71411433)
\curveto(22.48069381,258.10039918)(22.35005148,257.57782986)(22.22852373,257.14640634)
\lineto(19.39388894,246.96541899)
\closepath
}
}
{
\newrgbcolor{curcolor}{0 0 0}
\pscustom[linestyle=none,fillstyle=solid,fillcolor=curcolor]
{
\newpath
\moveto(32.64886485,246.96541899)
\lineto(32.64886485,260.32739522)
\lineto(34.28948949,260.32739522)
\lineto(34.28948949,255.53312544)
\curveto(35.05511432,256.42027803)(36.02125995,256.86385432)(37.18792636,256.86385432)
\curveto(37.90494009,256.86385432)(38.52776981,256.72105921)(39.05641553,256.435469)
\curveto(39.58506125,256.15595517)(39.96179727,255.76706636)(40.18662361,255.26880258)
\curveto(40.41752634,254.77053881)(40.5329777,254.04744869)(40.5329777,253.09953223)
\lineto(40.5329777,246.96541899)
\lineto(38.89235306,246.96541899)
\lineto(38.89235306,253.09953223)
\curveto(38.89235306,253.91984455)(38.71309963,254.51533053)(38.35459277,254.88599017)
\curveto(38.00216229,255.2627262)(37.50086031,255.45109421)(36.85068685,255.45109421)
\curveto(36.36457584,255.45109421)(35.90580858,255.32349007)(35.47438507,255.0682818)
\curveto(35.04903794,254.81914991)(34.74521856,254.4788722)(34.56292693,254.04744869)
\curveto(34.38063531,253.61602517)(34.28948949,253.02053919)(34.28948949,252.26099075)
\lineto(34.28948949,246.96541899)
\closepath
}
}
{
\newrgbcolor{curcolor}{0 0 0}
\pscustom[linestyle=none,fillstyle=solid,fillcolor=curcolor]
{
\newpath
\moveto(49.65381039,250.08260581)
\lineto(51.34912251,249.87297044)
\curveto(51.08176146,248.88251927)(50.58653588,248.11385624)(49.86344576,247.56698136)
\curveto(49.14035564,247.02010648)(48.21674473,246.74666904)(47.09261303,246.74666904)
\curveto(45.67681473,246.74666904)(44.55268304,247.18113075)(43.72021794,248.05005417)
\curveto(42.89382923,248.92505398)(42.48063488,250.14944607)(42.48063488,251.72323045)
\curveto(42.48063488,253.35170231)(42.89990562,254.61559092)(43.7384471,255.51489628)
\curveto(44.57698859,256.41420164)(45.66466196,256.86385432)(47.00146722,256.86385432)
\curveto(48.29573777,256.86385432)(49.3530292,256.42331622)(50.17334152,255.54224002)
\curveto(50.99365384,254.66116383)(51.40381,253.42158077)(51.40381,251.82349084)
\curveto(51.40381,251.72626864)(51.40077181,251.58043534)(51.39469542,251.38599094)
\lineto(44.17594701,251.38599094)
\curveto(44.23671088,250.32262312)(44.53749207,249.50838718)(45.07829056,248.94328314)
\curveto(45.61908905,248.3781791)(46.29356807,248.09562708)(47.10172761,248.09562708)
\curveto(47.70328998,248.09562708)(48.21674473,248.25361315)(48.64209186,248.56958531)
\curveto(49.06743899,248.88555746)(49.4046785,249.38989763)(49.65381039,250.08260581)
\closepath
\moveto(44.26709282,252.73494898)
\lineto(49.67203955,252.73494898)
\curveto(49.5991229,253.54918491)(49.39252572,254.15986186)(49.05224802,254.56697982)
\curveto(48.52967869,255.19892413)(47.85216148,255.51489628)(47.01969638,255.51489628)
\curveto(46.26622433,255.51489628)(45.63124183,255.2627262)(45.11474888,254.75838603)
\curveto(44.60433233,254.25404586)(44.32178031,253.57956685)(44.26709282,252.73494898)
\closepath
}
}
{
\newrgbcolor{curcolor}{0 0 0}
\pscustom[linestyle=none,fillstyle=solid,fillcolor=curcolor]
{
\newpath
\moveto(53.40615276,246.96541899)
\lineto(53.40615276,256.64510437)
\lineto(54.88271494,256.64510437)
\lineto(54.88271494,255.26880258)
\curveto(55.59365228,256.33217041)(56.62056178,256.86385432)(57.96344343,256.86385432)
\curveto(58.54677663,256.86385432)(59.08149874,256.75751754)(59.56760974,256.54484397)
\curveto(60.05979713,256.33824679)(60.42741858,256.06480935)(60.67047408,255.72453165)
\curveto(60.91352958,255.38425395)(61.08366843,254.98017418)(61.18089063,254.51229233)
\curveto(61.24165451,254.20847296)(61.27203645,253.67678905)(61.27203645,252.9172406)
\lineto(61.27203645,246.96541899)
\lineto(59.63141181,246.96541899)
\lineto(59.63141181,252.85343853)
\curveto(59.63141181,253.52184116)(59.56760974,254.02010494)(59.4400056,254.34822987)
\curveto(59.31240146,254.68243119)(59.08453693,254.94675404)(58.756412,255.14119845)
\curveto(58.43436346,255.34171924)(58.05458924,255.44197963)(57.61708933,255.44197963)
\curveto(56.91830477,255.44197963)(56.3137042,255.22019148)(55.80328765,254.77661519)
\curveto(55.29894748,254.3330389)(55.0467774,253.49145923)(55.0467774,252.25187616)
\lineto(55.0467774,246.96541899)
\closepath
}
}
{
\newrgbcolor{curcolor}{0 0 0}
\pscustom[linestyle=none,fillstyle=solid,fillcolor=curcolor]
{
\newpath
\moveto(72.53748722,248.43286659)
\lineto(72.77446633,246.98364816)
\curveto(72.31266088,246.88642596)(71.89946652,246.83781486)(71.53488327,246.83781486)
\curveto(70.93939729,246.83781486)(70.47759183,246.93199886)(70.14946691,247.12036688)
\curveto(69.82134198,247.30873489)(69.59043925,247.55482859)(69.45675873,247.85864796)
\curveto(69.3230782,248.16854373)(69.25623794,248.815679)(69.25623794,249.80005379)
\lineto(69.25623794,255.36906298)
\lineto(68.0531132,255.36906298)
\lineto(68.0531132,256.64510437)
\lineto(69.25623794,256.64510437)
\lineto(69.25623794,259.04223926)
\lineto(70.88774799,260.02661404)
\lineto(70.88774799,256.64510437)
\lineto(72.53748722,256.64510437)
\lineto(72.53748722,255.36906298)
\lineto(70.88774799,255.36906298)
\lineto(70.88774799,249.70890797)
\curveto(70.88774799,249.24102613)(70.91509174,248.94024495)(70.96977923,248.80656442)
\curveto(71.0305431,248.6728839)(71.12472711,248.56654711)(71.25233125,248.48755408)
\curveto(71.38601177,248.40856104)(71.57437979,248.36906452)(71.81743529,248.36906452)
\curveto(71.99972692,248.36906452)(72.23974423,248.39033187)(72.53748722,248.43286659)
\closepath
}
}
{
\newrgbcolor{curcolor}{0 0 0}
\pscustom[linestyle=none,fillstyle=solid,fillcolor=curcolor]
{
\newpath
\moveto(74.12611202,246.96541899)
\lineto(74.12611202,260.32739522)
\lineto(75.76673666,260.32739522)
\lineto(75.76673666,255.53312544)
\curveto(76.53236149,256.42027803)(77.49850712,256.86385432)(78.66517353,256.86385432)
\curveto(79.38218726,256.86385432)(80.00501698,256.72105921)(80.5336627,256.435469)
\curveto(81.06230841,256.15595517)(81.43904444,255.76706636)(81.66387078,255.26880258)
\curveto(81.89477351,254.77053881)(82.01022487,254.04744869)(82.01022487,253.09953223)
\lineto(82.01022487,246.96541899)
\lineto(80.36960023,246.96541899)
\lineto(80.36960023,253.09953223)
\curveto(80.36960023,253.91984455)(80.1903468,254.51533053)(79.83183994,254.88599017)
\curveto(79.47940946,255.2627262)(78.97810748,255.45109421)(78.32793402,255.45109421)
\curveto(77.84182301,255.45109421)(77.38305575,255.32349007)(76.95163224,255.0682818)
\curveto(76.52628511,254.81914991)(76.22246573,254.4788722)(76.0401741,254.04744869)
\curveto(75.85788248,253.61602517)(75.76673666,253.02053919)(75.76673666,252.26099075)
\lineto(75.76673666,246.96541899)
\closepath
}
}
{
\newrgbcolor{curcolor}{0 0 0}
\pscustom[linestyle=none,fillstyle=solid,fillcolor=curcolor]
{
\newpath
\moveto(91.13105756,250.08260581)
\lineto(92.82636968,249.87297044)
\curveto(92.55900863,248.88251927)(92.06378305,248.11385624)(91.34069293,247.56698136)
\curveto(90.61760281,247.02010648)(89.6939919,246.74666904)(88.5698602,246.74666904)
\curveto(87.1540619,246.74666904)(86.02993021,247.18113075)(85.19746511,248.05005417)
\curveto(84.3710764,248.92505398)(83.95788205,250.14944607)(83.95788205,251.72323045)
\curveto(83.95788205,253.35170231)(84.37715279,254.61559092)(85.21569427,255.51489628)
\curveto(86.05423576,256.41420164)(87.14190913,256.86385432)(88.47871439,256.86385432)
\curveto(89.77298494,256.86385432)(90.83027637,256.42331622)(91.65058869,255.54224002)
\curveto(92.47090101,254.66116383)(92.88105717,253.42158077)(92.88105717,251.82349084)
\curveto(92.88105717,251.72626864)(92.87801898,251.58043534)(92.87194259,251.38599094)
\lineto(85.65319418,251.38599094)
\curveto(85.71395805,250.32262312)(86.01473924,249.50838718)(86.55553773,248.94328314)
\curveto(87.09633622,248.3781791)(87.77081524,248.09562708)(88.57897478,248.09562708)
\curveto(89.18053715,248.09562708)(89.6939919,248.25361315)(90.11933903,248.56958531)
\curveto(90.54468616,248.88555746)(90.88192567,249.38989763)(91.13105756,250.08260581)
\closepath
\moveto(85.74433999,252.73494898)
\lineto(91.14928672,252.73494898)
\curveto(91.07637007,253.54918491)(90.86977289,254.15986186)(90.52949519,254.56697982)
\curveto(90.00692586,255.19892413)(89.32940865,255.51489628)(88.49694355,255.51489628)
\curveto(87.7434715,255.51489628)(87.108489,255.2627262)(86.59199605,254.75838603)
\curveto(86.0815795,254.25404586)(85.79902748,253.57956685)(85.74433999,252.73494898)
\closepath
}
}
{
\newrgbcolor{curcolor}{0 0 0}
\pscustom[linestyle=none,fillstyle=solid,fillcolor=curcolor]
{
\newpath
\moveto(16.55925414,224.96546354)
\lineto(15.03711906,224.96546354)
\lineto(15.03711906,238.32743977)
\lineto(16.6777437,238.32743977)
\lineto(16.6777437,233.56051374)
\curveto(17.37045188,234.42943716)(18.25456627,234.86389887)(19.33008687,234.86389887)
\curveto(19.92557285,234.86389887)(20.4876387,234.74237112)(21.01628441,234.49931561)
\curveto(21.55100652,234.2623365)(21.98850642,233.92509699)(22.32878413,233.48759709)
\curveto(22.67513822,233.05617357)(22.94553746,232.53360424)(23.13998186,231.9198891)
\curveto(23.33442627,231.30617396)(23.43164847,230.6499241)(23.43164847,229.95113953)
\curveto(23.43164847,228.29228573)(23.02149231,227.01016796)(22.20117999,226.10478621)
\curveto(21.38086767,225.19940446)(20.39649288,224.74671359)(19.24805564,224.74671359)
\curveto(18.10569478,224.74671359)(17.20942761,225.22371001)(16.55925414,226.17770286)
\closepath
\moveto(16.54102498,229.87822288)
\curveto(16.54102498,228.71763286)(16.69901106,227.87909138)(17.01498321,227.36259843)
\curveto(17.53147615,226.51798056)(18.23026072,226.09567163)(19.11133692,226.09567163)
\curveto(19.82835065,226.09567163)(20.44814218,226.40556739)(20.97071151,227.02535892)
\curveto(21.49328084,227.65122684)(21.7545655,228.58091414)(21.7545655,229.81442081)
\curveto(21.7545655,231.07830942)(21.50239542,232.01103491)(20.99805525,232.61259728)
\curveto(20.49979147,233.21415965)(19.89519091,233.51494083)(19.18425357,233.51494083)
\curveto(18.46723984,233.51494083)(17.84744831,233.20200687)(17.32487898,232.57613895)
\curveto(16.80230965,231.95634742)(16.54102498,231.05704207)(16.54102498,229.87822288)
\closepath
}
}
{
\newrgbcolor{curcolor}{0 0 0}
\pscustom[linestyle=none,fillstyle=solid,fillcolor=curcolor]
{
\newpath
\moveto(25.38842023,224.96546354)
\lineto(25.38842023,238.32743977)
\lineto(27.02904487,238.32743977)
\lineto(27.02904487,224.96546354)
\closepath
}
}
{
\newrgbcolor{curcolor}{0 0 0}
\pscustom[linestyle=none,fillstyle=solid,fillcolor=curcolor]
{
\newpath
\moveto(28.95819326,229.80530623)
\curveto(28.95819326,231.59784056)(29.45645704,232.92553124)(30.4529846,233.78837827)
\curveto(31.2854497,234.505392)(32.30020642,234.86389887)(33.49725477,234.86389887)
\curveto(34.82798364,234.86389887)(35.91565701,234.42639896)(36.76027488,233.55139916)
\curveto(37.60489275,232.68247574)(38.02720169,231.479351)(38.02720169,229.94202495)
\curveto(38.02720169,228.6963655)(37.83883367,227.71502891)(37.46209764,226.99801518)
\curveto(37.091438,226.28707784)(36.54760132,225.73412657)(35.83058759,225.33916138)
\curveto(35.11965024,224.94419619)(34.34187264,224.74671359)(33.49725477,224.74671359)
\curveto(32.14222034,224.74671359)(31.04543239,225.1811753)(30.20689091,226.05009872)
\curveto(29.37442581,226.91902214)(28.95819326,228.17075798)(28.95819326,229.80530623)
\closepath
\moveto(30.64439081,229.80530623)
\curveto(30.64439081,228.56572317)(30.91479006,227.63603587)(31.45558855,227.01624434)
\curveto(31.99638704,226.4025292)(32.67694245,226.09567163)(33.49725477,226.09567163)
\curveto(34.3114907,226.09567163)(34.98900791,226.40556739)(35.5298064,227.02535892)
\curveto(36.07060489,227.64515045)(36.34100414,228.59002872)(36.34100414,229.85999372)
\curveto(36.34100414,231.05704207)(36.0675667,231.96242381)(35.52069182,232.57613895)
\curveto(34.97989333,233.19593048)(34.30541431,233.50582625)(33.49725477,233.50582625)
\curveto(32.67694245,233.50582625)(31.99638704,233.19896868)(31.45558855,232.58525354)
\curveto(30.91479006,231.97153839)(30.64439081,231.04488929)(30.64439081,229.80530623)
\closepath
}
}
{
\newrgbcolor{curcolor}{0 0 0}
\pscustom[linestyle=none,fillstyle=solid,fillcolor=curcolor]
{
\newpath
\moveto(46.26392094,228.51103568)
\lineto(47.87720183,228.30140031)
\curveto(47.70098659,227.18942139)(47.24829572,226.31745977)(46.51912921,225.68551547)
\curveto(45.79603909,225.05964755)(44.90584832,224.74671359)(43.84855688,224.74671359)
\curveto(42.5239044,224.74671359)(41.45749838,225.17813711)(40.64933884,226.04098414)
\curveto(39.84725568,226.90990756)(39.4462141,228.15252882)(39.4462141,229.7688479)
\curveto(39.4462141,230.81398656)(39.61939115,231.72848289)(39.96574524,232.51233688)
\curveto(40.31209933,233.29619088)(40.83770685,233.88256228)(41.54256781,234.27145108)
\curveto(42.25350515,234.66641627)(43.02520637,234.86389887)(43.85767146,234.86389887)
\curveto(44.90888651,234.86389887)(45.76869535,234.59653782)(46.43709798,234.06181571)
\curveto(47.10550061,233.53316999)(47.53388593,232.77969794)(47.72225395,231.80139954)
\lineto(46.12720222,231.55530585)
\curveto(45.97529253,232.20547931)(45.70489328,232.69462851)(45.31600448,233.02275344)
\curveto(44.93319206,233.35087837)(44.46834841,233.51494083)(43.92147353,233.51494083)
\curveto(43.09508483,233.51494083)(42.423644,233.21719784)(41.90715106,232.62171186)
\curveto(41.39065812,232.03230227)(41.13241165,231.09653858)(41.13241165,229.81442081)
\curveto(41.13241165,228.51407387)(41.38154354,227.56919561)(41.87980732,226.97978602)
\curveto(42.3780711,226.39037642)(43.02824456,226.09567163)(43.83032772,226.09567163)
\curveto(44.4744248,226.09567163)(45.0121851,226.29315422)(45.44360862,226.68811941)
\curveto(45.87503213,227.08308461)(46.14846957,227.69072336)(46.26392094,228.51103568)
\closepath
}
}
{
\newrgbcolor{curcolor}{0 0 0}
\pscustom[linestyle=none,fillstyle=solid,fillcolor=curcolor]
{
\newpath
\moveto(49.28994382,224.96546354)
\lineto(49.28994382,238.32743977)
\lineto(50.93056846,238.32743977)
\lineto(50.93056846,230.70764978)
\lineto(54.8133801,234.64514892)
\lineto(56.93707755,234.64514892)
\lineto(53.23655753,231.05400387)
\lineto(57.31077539,224.96546354)
\lineto(55.28733833,224.96546354)
\lineto(52.08812029,229.91468121)
\lineto(50.93056846,228.80270228)
\lineto(50.93056846,224.96546354)
\closepath
}
}
{
\newrgbcolor{curcolor}{0 0 0}
\pscustom[linestyle=none,fillstyle=solid,fillcolor=curcolor]
{
\newpath
\moveto(65.57125942,224.96546354)
\lineto(62.60902049,234.64514892)
\lineto(64.30433262,234.64514892)
\lineto(65.84469686,229.05791056)
\lineto(66.41891549,226.97978602)
\curveto(66.44322104,227.08308461)(66.61032169,227.74844904)(66.92021746,228.97587933)
\lineto(68.4605817,234.64514892)
\lineto(70.14677925,234.64514892)
\lineto(71.59599768,229.03056682)
\lineto(72.07907049,227.18030681)
\lineto(72.63505995,229.04879598)
\lineto(74.29391376,234.64514892)
\lineto(75.88896549,234.64514892)
\lineto(72.86292449,224.96546354)
\lineto(71.15849778,224.96546354)
\lineto(69.61813353,230.76233727)
\lineto(69.2444357,232.41207649)
\lineto(67.28480071,224.96546354)
\closepath
}
}
{
\newrgbcolor{curcolor}{0 0 0}
\pscustom[linestyle=none,fillstyle=solid,fillcolor=curcolor]
{
\newpath
\moveto(83.57851212,226.1594737)
\curveto(82.97087337,225.64298076)(82.38450197,225.2783975)(81.81939792,225.06572394)
\curveto(81.26037027,224.85305037)(80.6588079,224.74671359)(80.01471082,224.74671359)
\curveto(78.951343,224.74671359)(78.13406887,225.00496006)(77.56288844,225.521453)
\curveto(76.99170801,226.04402233)(76.7061178,226.70938677)(76.7061178,227.51754632)
\curveto(76.7061178,227.99150455)(76.81245458,228.42292806)(77.02512815,228.81181686)
\curveto(77.2438781,229.20678206)(77.52643012,229.52275421)(77.87278421,229.75973332)
\curveto(78.22521469,229.99671244)(78.62017988,230.17596587)(79.05767978,230.29749362)
\curveto(79.37972832,230.38256305)(79.86583933,230.46459428)(80.51601279,230.54358732)
\curveto(81.84066528,230.70157339)(82.81592548,230.88994141)(83.4417934,231.10869136)
\curveto(83.44786979,231.3335177)(83.45090798,231.47631281)(83.45090798,231.53707668)
\curveto(83.45090798,232.20547931)(83.2959601,232.67639935)(82.98606433,232.94983679)
\curveto(82.56679359,233.32049643)(81.94396387,233.50582625)(81.11757516,233.50582625)
\curveto(80.34587394,233.50582625)(79.77469351,233.36910753)(79.40403387,233.09567009)
\curveto(79.03945062,232.82830904)(78.76905137,232.35131261)(78.59283613,231.66468082)
\lineto(76.98866982,231.88343077)
\curveto(77.13450312,232.57006257)(77.37452043,233.12301383)(77.70872174,233.54228457)
\curveto(78.04292306,233.9676317)(78.52599587,234.29271844)(79.15794018,234.51754478)
\curveto(79.78988448,234.7484475)(80.52208918,234.86389887)(81.35455428,234.86389887)
\curveto(82.18094298,234.86389887)(82.85238381,234.76667667)(83.36887675,234.57223227)
\curveto(83.88536969,234.37778786)(84.26514391,234.13169417)(84.50819942,233.83395118)
\curveto(84.75125492,233.54228457)(84.92139377,233.17162493)(85.01861597,232.72197226)
\curveto(85.07330346,232.44245843)(85.1006472,231.93811826)(85.1006472,231.20895175)
\lineto(85.1006472,229.02145224)
\curveto(85.1006472,227.49627896)(85.13406733,226.53013334)(85.2009076,226.12301537)
\curveto(85.27382425,225.72197379)(85.41358116,225.33612318)(85.62017834,224.96546354)
\lineto(83.90663705,224.96546354)
\curveto(83.7364982,225.30574125)(83.62712322,225.70374463)(83.57851212,226.1594737)
\closepath
\moveto(83.4417934,229.82353539)
\curveto(82.84630742,229.58047989)(81.95307845,229.37388271)(80.76210649,229.20374386)
\curveto(80.08762747,229.10652166)(79.61063105,228.99714669)(79.33111722,228.87561893)
\curveto(79.05160339,228.75409118)(78.83589164,228.57483775)(78.68398195,228.33785864)
\curveto(78.53207226,228.10695591)(78.45611741,227.84870944)(78.45611741,227.56311922)
\curveto(78.45611741,227.12561932)(78.62017988,226.76103607)(78.94830481,226.46936946)
\curveto(79.28250612,226.17770286)(79.76861713,226.03186956)(80.40663782,226.03186956)
\curveto(81.03858212,226.03186956)(81.60064797,226.16858828)(82.09283536,226.44202572)
\curveto(82.58502276,226.72153955)(82.94656782,227.10131377)(83.17747054,227.58134839)
\curveto(83.35368578,227.95200803)(83.4417934,228.49888291)(83.4417934,229.22197302)
\closepath
}
}
{
\newrgbcolor{curcolor}{0 0 0}
\pscustom[linestyle=none,fillstyle=solid,fillcolor=curcolor]
{
\newpath
\moveto(86.98450231,227.85478583)
\lineto(88.60689779,228.1099941)
\curveto(88.6980436,227.45982063)(88.95021368,226.96155685)(89.36340804,226.61520276)
\curveto(89.78267878,226.26884867)(90.36601198,226.09567163)(91.11340765,226.09567163)
\curveto(91.86687971,226.09567163)(92.42590736,226.24758132)(92.79049062,226.5514007)
\curveto(93.15507387,226.86129646)(93.3373655,227.22284152)(93.3373655,227.63603587)
\curveto(93.3373655,228.00669551)(93.17634123,228.29836212)(92.85429269,228.51103568)
\curveto(92.62946635,228.65686898)(92.07043869,228.8421988)(91.17720972,229.06702514)
\curveto(89.97408499,229.37084452)(89.1385817,229.63212918)(88.67069986,229.85087914)
\curveto(88.2088944,230.07570548)(87.85646393,230.38256305)(87.61340842,230.77145185)
\curveto(87.37642931,231.16641704)(87.25793975,231.60087875)(87.25793975,232.07483698)
\curveto(87.25793975,232.5062605)(87.35516195,232.90426388)(87.54960635,233.26884713)
\curveto(87.75012714,233.63950678)(88.02052639,233.94636435)(88.36080409,234.18941985)
\curveto(88.61601237,234.37778786)(88.96236646,234.53577394)(89.39986636,234.66337808)
\curveto(89.84344265,234.7970586)(90.31740088,234.86389887)(90.82174105,234.86389887)
\curveto(91.5812895,234.86389887)(92.24665393,234.75452389)(92.81783436,234.53577394)
\curveto(93.39509118,234.31702399)(93.82043831,234.019281)(94.09387575,233.64254497)
\curveto(94.36731319,233.27188533)(94.5556812,232.77362155)(94.65897979,232.14775363)
\lineto(93.05481348,231.92900368)
\curveto(92.98189683,232.42726746)(92.76922326,232.81615626)(92.41679278,233.09567009)
\curveto(92.07043869,233.37518392)(91.5782513,233.51494083)(90.94023061,233.51494083)
\curveto(90.18675855,233.51494083)(89.64899825,233.39037489)(89.32694971,233.141243)
\curveto(89.00490117,232.89211111)(88.8438769,232.6004445)(88.8438769,232.26624319)
\curveto(88.8438769,232.05356962)(88.91071717,231.86216342)(89.04439769,231.69202456)
\curveto(89.17807822,231.51580933)(89.38771359,231.36997602)(89.6733038,231.25452466)
\curveto(89.83736627,231.19376079)(90.32043908,231.05400387)(91.12252223,230.83525392)
\curveto(92.28311226,230.52535815)(93.0912718,230.27014988)(93.54700087,230.06962909)
\curveto(94.00880632,229.87518469)(94.37035138,229.58959447)(94.63163605,229.21285844)
\curveto(94.89292071,228.83612242)(95.02356304,228.36824057)(95.02356304,227.80921292)
\curveto(95.02356304,227.26233804)(94.86253877,226.7458451)(94.54049023,226.25973409)
\curveto(94.22451808,225.77969948)(93.76575082,225.40600164)(93.16418845,225.13864059)
\curveto(92.56262608,224.87735592)(91.88207068,224.74671359)(91.12252223,224.74671359)
\curveto(89.86471001,224.74671359)(88.90464078,225.00799826)(88.24231453,225.53056759)
\curveto(87.58606468,226.05313692)(87.16679394,226.82787633)(86.98450231,227.85478583)
\closepath
}
}
{
\newrgbcolor{curcolor}{0 0 0}
\pscustom[linestyle=none,fillstyle=solid,fillcolor=curcolor]
{
\newpath
\moveto(16.83269158,202.96550809)
\lineto(13.87045265,212.64519347)
\lineto(15.56576478,212.64519347)
\lineto(17.10612902,207.05795511)
\lineto(17.68034765,204.97983057)
\curveto(17.7046532,205.08312916)(17.87175386,205.74849359)(18.18164962,206.97592388)
\lineto(19.72201387,212.64519347)
\lineto(21.40821141,212.64519347)
\lineto(22.85742984,207.03061137)
\lineto(23.34050265,205.18035136)
\lineto(23.89649211,207.04884053)
\lineto(25.55534592,212.64519347)
\lineto(27.15039765,212.64519347)
\lineto(24.12435665,202.96550809)
\lineto(22.41992994,202.96550809)
\lineto(20.87956569,208.76238182)
\lineto(20.50586786,210.41212104)
\lineto(18.54623287,202.96550809)
\closepath
}
}
{
\newrgbcolor{curcolor}{0 0 0}
\pscustom[linestyle=none,fillstyle=solid,fillcolor=curcolor]
{
\newpath
\moveto(28.50531026,202.96550809)
\lineto(28.50531026,212.64519347)
\lineto(29.98187243,212.64519347)
\lineto(29.98187243,211.17774587)
\curveto(30.35860846,211.86437767)(30.70496255,212.31706854)(31.02093471,212.53581849)
\curveto(31.34298325,212.75456844)(31.69541372,212.86394342)(32.07822614,212.86394342)
\curveto(32.63117741,212.86394342)(33.19324326,212.68772818)(33.76442369,212.3352977)
\lineto(33.19931964,210.81316262)
\curveto(32.79827806,211.05014173)(32.39723649,211.16863129)(31.99619491,211.16863129)
\curveto(31.63768804,211.16863129)(31.3156395,211.05925631)(31.03004929,210.84050636)
\curveto(30.74445907,210.6278328)(30.54090009,210.33008981)(30.41937234,209.94727739)
\curveto(30.23708071,209.36394419)(30.1459349,208.72592349)(30.1459349,208.03321531)
\lineto(30.1459349,202.96550809)
\closepath
}
}
{
\newrgbcolor{curcolor}{0 0 0}
\pscustom[linestyle=none,fillstyle=solid,fillcolor=curcolor]
{
\newpath
\moveto(34.74864102,214.44076599)
\lineto(34.74864102,216.32748432)
\lineto(36.38926566,216.32748432)
\lineto(36.38926566,214.44076599)
\closepath
\moveto(34.74864102,202.96550809)
\lineto(34.74864102,212.64519347)
\lineto(36.38926566,212.64519347)
\lineto(36.38926566,202.96550809)
\closepath
}
}
{
\newrgbcolor{curcolor}{0 0 0}
\pscustom[linestyle=none,fillstyle=solid,fillcolor=curcolor]
{
\newpath
\moveto(42.46554761,204.43295569)
\lineto(42.70252672,202.98373726)
\curveto(42.24072127,202.88651506)(41.82752691,202.83790396)(41.46294366,202.83790396)
\curveto(40.86745768,202.83790396)(40.40565223,202.93208796)(40.0775273,203.12045598)
\curveto(39.74940237,203.30882399)(39.51849964,203.55491769)(39.38481912,203.85873706)
\curveto(39.25113859,204.16863283)(39.18429833,204.8157681)(39.18429833,205.80014289)
\lineto(39.18429833,211.36915208)
\lineto(37.98117359,211.36915208)
\lineto(37.98117359,212.64519347)
\lineto(39.18429833,212.64519347)
\lineto(39.18429833,215.04232836)
\lineto(40.81580839,216.02670314)
\lineto(40.81580839,212.64519347)
\lineto(42.46554761,212.64519347)
\lineto(42.46554761,211.36915208)
\lineto(40.81580839,211.36915208)
\lineto(40.81580839,205.70899707)
\curveto(40.81580839,205.24111523)(40.84315213,204.94033405)(40.89783962,204.80665352)
\curveto(40.95860349,204.672973)(41.0527875,204.56663621)(41.18039164,204.48764318)
\curveto(41.31407217,204.40865014)(41.50244018,204.36915362)(41.74549568,204.36915362)
\curveto(41.92778731,204.36915362)(42.16780462,204.39042097)(42.46554761,204.43295569)
\closepath
}
}
{
\newrgbcolor{curcolor}{0 0 0}
\pscustom[linestyle=none,fillstyle=solid,fillcolor=curcolor]
{
\newpath
\moveto(47.63620479,204.43295569)
\lineto(47.8731839,202.98373726)
\curveto(47.41137845,202.88651506)(46.99818409,202.83790396)(46.63360084,202.83790396)
\curveto(46.03811486,202.83790396)(45.5763094,202.93208796)(45.24818448,203.12045598)
\curveto(44.92005955,203.30882399)(44.68915682,203.55491769)(44.5554763,203.85873706)
\curveto(44.42179577,204.16863283)(44.35495551,204.8157681)(44.35495551,205.80014289)
\lineto(44.35495551,211.36915208)
\lineto(43.15183077,211.36915208)
\lineto(43.15183077,212.64519347)
\lineto(44.35495551,212.64519347)
\lineto(44.35495551,215.04232836)
\lineto(45.98646556,216.02670314)
\lineto(45.98646556,212.64519347)
\lineto(47.63620479,212.64519347)
\lineto(47.63620479,211.36915208)
\lineto(45.98646556,211.36915208)
\lineto(45.98646556,205.70899707)
\curveto(45.98646556,205.24111523)(46.01380931,204.94033405)(46.0684968,204.80665352)
\curveto(46.12926067,204.672973)(46.22344468,204.56663621)(46.35104882,204.48764318)
\curveto(46.48472934,204.40865014)(46.67309736,204.36915362)(46.91615286,204.36915362)
\curveto(47.09844449,204.36915362)(47.3384618,204.39042097)(47.63620479,204.43295569)
\closepath
}
}
{
\newrgbcolor{curcolor}{0 0 0}
\pscustom[linestyle=none,fillstyle=solid,fillcolor=curcolor]
{
\newpath
\moveto(55.85113022,206.08269491)
\lineto(57.54644235,205.87305954)
\curveto(57.27908129,204.88260837)(56.78385571,204.11394534)(56.06076559,203.56707046)
\curveto(55.33767547,203.02019558)(54.41406456,202.74675814)(53.28993287,202.74675814)
\curveto(51.87413457,202.74675814)(50.75000287,203.18121985)(49.91753777,204.05014327)
\curveto(49.09114907,204.92514308)(48.67795471,206.14953517)(48.67795471,207.72331955)
\curveto(48.67795471,209.35179141)(49.09722545,210.61568002)(49.93576694,211.51498538)
\curveto(50.77430842,212.41429074)(51.86198179,212.86394342)(53.19878705,212.86394342)
\curveto(54.4930576,212.86394342)(55.55034904,212.42340532)(56.37066136,211.54232912)
\curveto(57.19097368,210.66125293)(57.60112984,209.42166987)(57.60112984,207.82357994)
\curveto(57.60112984,207.72635774)(57.59809164,207.58052444)(57.59201525,207.38608004)
\lineto(50.37326684,207.38608004)
\curveto(50.43403072,206.32271222)(50.7348119,205.50847628)(51.27561039,204.94337224)
\curveto(51.81640888,204.3782682)(52.4908879,204.09571618)(53.29904745,204.09571618)
\curveto(53.90060982,204.09571618)(54.41406456,204.25370225)(54.83941169,204.56967441)
\curveto(55.26475882,204.88564656)(55.60199833,205.38998673)(55.85113022,206.08269491)
\closepath
\moveto(50.46441265,208.73503808)
\lineto(55.86935938,208.73503808)
\curveto(55.79644273,209.54927401)(55.58984555,210.15995096)(55.24956785,210.56706892)
\curveto(54.72699852,211.19901323)(54.04948131,211.51498538)(53.21701622,211.51498538)
\curveto(52.46354416,211.51498538)(51.82856166,211.2628153)(51.31206872,210.75847513)
\curveto(50.80165216,210.25413496)(50.51910014,209.57965595)(50.46441265,208.73503808)
\closepath
}
}
{
\newrgbcolor{curcolor}{0 0 0}
\pscustom[linestyle=none,fillstyle=solid,fillcolor=curcolor]
{
\newpath
\moveto(59.6034745,202.96550809)
\lineto(59.6034745,212.64519347)
\lineto(61.08003668,212.64519347)
\lineto(61.08003668,211.26889168)
\curveto(61.79097402,212.33225951)(62.81788352,212.86394342)(64.16076517,212.86394342)
\curveto(64.74409837,212.86394342)(65.27882047,212.75760664)(65.76493148,212.54493307)
\curveto(66.25711887,212.33833589)(66.62474032,212.06489845)(66.86779582,211.72462075)
\curveto(67.11085132,211.38434305)(67.28099017,210.98026328)(67.37821237,210.51238143)
\curveto(67.43897625,210.20856206)(67.46935819,209.67687815)(67.46935819,208.9173297)
\lineto(67.46935819,202.96550809)
\lineto(65.82873355,202.96550809)
\lineto(65.82873355,208.85352763)
\curveto(65.82873355,209.52193026)(65.76493148,210.02019404)(65.63732734,210.34831897)
\curveto(65.5097232,210.68252029)(65.28185867,210.94684314)(64.95373374,211.14128755)
\curveto(64.6316852,211.34180834)(64.25191098,211.44206873)(63.81441107,211.44206873)
\curveto(63.11562651,211.44206873)(62.51102594,211.22028058)(62.00060939,210.77670429)
\curveto(61.49626922,210.333128)(61.24409914,209.49154833)(61.24409914,208.25196526)
\lineto(61.24409914,202.96550809)
\closepath
}
}
{
\newrgbcolor{curcolor}{0 0 0}
\pscustom[linestyle=none,fillstyle=solid,fillcolor=curcolor,opacity=0]
{
\newpath
\moveto(151.55481171,231.88544063)
\curveto(173.88547515,231.88544063)(185.05080687,221.76735089)(196.21613859,211.64926114)
\curveto(207.38147031,201.53117139)(218.54680203,191.41308164)(240.87746547,191.41308164)
}
}
{
\newrgbcolor{curcolor}{0.49803922 0.49803922 0.49803922}
\pscustom[linewidth=2.99999393,linecolor=curcolor]
{
\newpath
\moveto(151.55481171,231.88544063)
\curveto(173.88547515,231.88544063)(185.05080687,221.76735089)(196.21613859,211.64926114)
\curveto(201.79880445,206.59021627)(207.38147031,201.53117139)(214.35980264,197.73690086)
\curveto(216.10439228,196.7883201)(217.93619959,195.91882055)(219.87706181,195.14806096)
\curveto(220.84746666,194.76273365)(221.84514181,194.40207822)(222.87283002,194.06858808)
\lineto(223.06549367,194.00782705)
}
}
{
\newrgbcolor{curcolor}{0.49803922 0.49803922 0.49803922}
\pscustom[linestyle=none,fillstyle=solid,fillcolor=curcolor]
{
\newpath
\moveto(222.35120378,189.10437241)
\lineto(236.53757925,192.04531134)
\lineto(223.77978356,198.91122919)
\closepath
}
}
{
\newrgbcolor{curcolor}{0.49803922 0.49803922 0.49803922}
\pscustom[linewidth=2.99999393,linecolor=curcolor]
{
\newpath
\moveto(222.35120378,189.10437241)
\lineto(236.53757925,192.04531134)
\lineto(223.77978356,198.91122919)
\closepath
}
}
{
\newrgbcolor{curcolor}{0 0 0}
\pscustom[linestyle=none,fillstyle=solid,fillcolor=curcolor,opacity=0]
{
\newpath
\moveto(689.69561173,535.45953718)
\curveto(701.46724144,535.45953718)(707.35305629,530.01860332)(713.23887114,524.57766945)
\curveto(719.124686,519.13673559)(725.01050085,513.69580172)(736.78213056,513.69580172)
}
}
{
\newrgbcolor{curcolor}{0.49803922 0.49803922 0.49803922}
\pscustom[linewidth=2.99999393,linecolor=curcolor]
{
\newpath
\moveto(689.69561173,535.45953718)
\curveto(701.46724144,535.45953718)(707.35305629,530.01860332)(713.23887114,524.57766945)
\curveto(719.124686,519.13673559)(725.01050085,513.69580172)(736.78213056,513.69580172)
}
}
{
\newrgbcolor{curcolor}{0 0 0}
\pscustom[linestyle=none,fillstyle=solid,fillcolor=curcolor,opacity=0]
{
\newpath
\moveto(689.69561173,501.05416066)
\curveto(701.46724144,501.05416066)(707.35305629,504.21163458)(713.23887114,507.3691085)
\curveto(719.124686,510.52658242)(725.01050085,513.68407209)(736.78213056,513.68407209)
}
}
{
\newrgbcolor{curcolor}{0.49803922 0.49803922 0.49803922}
\pscustom[linewidth=2.99999393,linecolor=curcolor]
{
\newpath
\moveto(689.69561173,501.05416066)
\curveto(701.46724144,501.05416066)(707.35305629,504.21163458)(713.23887114,507.3691085)
\curveto(719.124686,510.52658242)(725.01050085,513.68407209)(736.78213056,513.68407209)
}
}
{
\newrgbcolor{curcolor}{0.90980393 0.90980393 0.90980393}
\pscustom[linestyle=none,fillstyle=solid,fillcolor=curcolor]
{
\newpath
\moveto(0.49999948,104.83291576)
\lineto(0.49999948,104.83291576)
\curveto(0.49999948,112.28961982)(6.54485338,118.33448947)(14.00155482,118.33448947)
\lineto(152.0059919,118.33448947)
\curveto(155.58681142,118.33448947)(159.02099081,116.91199891)(161.55303293,114.37998304)
\curveto(164.0850488,111.84794093)(165.50753936,108.41377465)(165.50753936,104.83291576)
\lineto(165.50753936,50.82835321)
\curveto(165.50753936,43.37164915)(159.46269596,37.3267795)(152.0059919,37.3267795)
\lineto(14.00155482,37.3267795)
\curveto(6.54485338,37.3267795)(0.49999948,43.37164915)(0.49999948,50.82835321)
\closepath
}
}
{
\newrgbcolor{curcolor}{0.10196079 0.8392157 0.96078432}
\pscustom[linewidth=0.99999798,linecolor=curcolor]
{
\newpath
\moveto(0.49999948,104.83291576)
\lineto(0.49999948,104.83291576)
\curveto(0.49999948,112.28961982)(6.54485338,118.33448947)(14.00155482,118.33448947)
\lineto(152.0059919,118.33448947)
\curveto(155.58681142,118.33448947)(159.02099081,116.91199891)(161.55303293,114.37998304)
\curveto(164.0850488,111.84794093)(165.50753936,108.41377465)(165.50753936,104.83291576)
\lineto(165.50753936,50.82835321)
\curveto(165.50753936,43.37164915)(159.46269596,37.3267795)(152.0059919,37.3267795)
\lineto(14.00155482,37.3267795)
\curveto(6.54485338,37.3267795)(0.49999948,43.37164915)(0.49999948,50.82835321)
\closepath
}
}
{
\newrgbcolor{curcolor}{0 0 0}
\pscustom[linestyle=none,fillstyle=solid,fillcolor=curcolor]
{
\newpath
\moveto(24.42840682,97.59550976)
\lineto(26.19663559,97.14889527)
\curveto(25.82597595,95.69663865)(25.15757332,94.58769792)(24.1914277,93.82207309)
\curveto(23.23135847,93.06252464)(22.05557748,92.68275042)(20.66408473,92.68275042)
\curveto(19.22398088,92.68275042)(18.05123808,92.97441702)(17.14585633,93.55775023)
\curveto(16.24655098,94.14715982)(15.55991918,94.99785408)(15.08596095,96.109833)
\curveto(14.61807911,97.22181192)(14.38413819,98.41582208)(14.38413819,99.69186346)
\curveto(14.38413819,101.08335621)(14.64846105,102.29559553)(15.17710677,103.32858142)
\curveto(15.71182887,104.36764369)(16.46833912,105.15453587)(17.44663752,105.68925798)
\curveto(18.4310123,106.23005647)(19.51260929,106.50045572)(20.69142847,106.50045572)
\curveto(22.02823373,106.50045572)(23.15236543,106.16017801)(24.06382356,105.47962261)
\curveto(24.9752817,104.7990672)(25.6102642,103.84203616)(25.96877106,102.60852949)
\lineto(24.22788603,102.19837333)
\curveto(23.91799026,103.17059534)(23.46833758,103.87849449)(22.87892799,104.32207078)
\curveto(22.2895184,104.76564707)(21.54819912,104.98743522)(20.65497015,104.98743522)
\curveto(19.62806065,104.98743522)(18.76825181,104.74134152)(18.07554363,104.24915413)
\curveto(17.38891184,103.75696674)(16.90583903,103.09464049)(16.6263252,102.2621754)
\curveto(16.34681137,101.43578669)(16.20705446,100.58205424)(16.20705446,99.70097805)
\curveto(16.20705446,98.56469357)(16.37111692,97.57120421)(16.69924185,96.72050995)
\curveto(17.03344316,95.87589208)(17.54993611,95.24394778)(18.24872068,94.82467703)
\curveto(18.94750524,94.40540629)(19.70401549,94.19577092)(20.51825143,94.19577092)
\curveto(21.5087026,94.19577092)(22.34724408,94.48136114)(23.03387587,95.05254157)
\curveto(23.72050767,95.623722)(24.18535131,96.47137806)(24.42840682,97.59550976)
\closepath
}
}
{
\newrgbcolor{curcolor}{0 0 0}
\pscustom[linestyle=none,fillstyle=solid,fillcolor=curcolor]
{
\newpath
\moveto(28.16222515,92.91061496)
\lineto(28.16222515,106.27259118)
\lineto(29.80284979,106.27259118)
\lineto(29.80284979,101.47832141)
\curveto(30.56847462,102.36547399)(31.53462024,102.80905028)(32.70128665,102.80905028)
\curveto(33.41830039,102.80905028)(34.04113011,102.66625517)(34.56977583,102.38066496)
\curveto(35.09842154,102.10115113)(35.47515757,101.71226233)(35.69998391,101.21399855)
\curveto(35.93088664,100.71573477)(36.046338,99.99264465)(36.046338,99.04472819)
\lineto(36.046338,92.91061496)
\lineto(34.40571336,92.91061496)
\lineto(34.40571336,99.04472819)
\curveto(34.40571336,99.86504051)(34.22645993,100.46052649)(33.86795307,100.83118613)
\curveto(33.51552259,101.20792216)(33.01422061,101.39629017)(32.36404715,101.39629017)
\curveto(31.87793614,101.39629017)(31.41916888,101.26868603)(30.98774536,101.01347776)
\curveto(30.56239824,100.76434587)(30.25857886,100.42406816)(30.07628723,99.99264465)
\curveto(29.89399561,99.56122113)(29.80284979,98.96573515)(29.80284979,98.20618671)
\lineto(29.80284979,92.91061496)
\closepath
}
}
{
\newrgbcolor{curcolor}{0 0 0}
\pscustom[linestyle=none,fillstyle=solid,fillcolor=curcolor]
{
\newpath
\moveto(45.16717069,96.02780177)
\lineto(46.86248281,95.8181664)
\curveto(46.59512176,94.82771523)(46.09989618,94.0590522)(45.37680606,93.51217732)
\curveto(44.65371594,92.96530244)(43.73010503,92.691865)(42.60597333,92.691865)
\curveto(41.19017503,92.691865)(40.06604334,93.12632671)(39.23357824,93.99525013)
\curveto(38.40718953,94.87024994)(37.99399518,96.09464203)(37.99399518,97.66842641)
\curveto(37.99399518,99.29689827)(38.41326592,100.56078688)(39.2518074,101.46009224)
\curveto(40.09034889,102.3593976)(41.17802226,102.80905028)(42.51482752,102.80905028)
\curveto(43.80909807,102.80905028)(44.8663895,102.36851218)(45.68670182,101.48743599)
\curveto(46.50701414,100.60635979)(46.9171703,99.36677673)(46.9171703,97.7686868)
\curveto(46.9171703,97.6714646)(46.91413211,97.5256313)(46.90805572,97.3311869)
\lineto(39.68930731,97.3311869)
\curveto(39.75007118,96.26781908)(40.05085237,95.45358315)(40.59165086,94.8884791)
\curveto(41.13244935,94.32337506)(41.80692837,94.04082304)(42.61508791,94.04082304)
\curveto(43.21665028,94.04082304)(43.73010503,94.19880912)(44.15545216,94.51478127)
\curveto(44.58079929,94.83075342)(44.9180388,95.33509359)(45.16717069,96.02780177)
\closepath
\moveto(39.78045312,98.68014494)
\lineto(45.18539985,98.68014494)
\curveto(45.1124832,99.49438087)(44.90588602,100.10505782)(44.56560832,100.51217578)
\curveto(44.04303899,101.14412009)(43.36552178,101.46009224)(42.53305668,101.46009224)
\curveto(41.77958463,101.46009224)(41.14460213,101.20792216)(40.62810918,100.70358199)
\curveto(40.11769263,100.19924183)(39.83514061,99.52476281)(39.78045312,98.68014494)
\closepath
}
}
{
\newrgbcolor{curcolor}{0 0 0}
\pscustom[linestyle=none,fillstyle=solid,fillcolor=curcolor]
{
\newpath
\moveto(55.23591983,96.45618709)
\lineto(56.84920072,96.24655172)
\curveto(56.67298548,95.1345728)(56.22029461,94.26261119)(55.49112811,93.63066688)
\curveto(54.76803799,93.00479896)(53.87784721,92.691865)(52.82055578,92.691865)
\curveto(51.49590329,92.691865)(50.42949727,93.12328852)(49.62133773,93.98613555)
\curveto(48.81925457,94.85505897)(48.41821299,96.09768023)(48.41821299,97.71399932)
\curveto(48.41821299,98.75913798)(48.59139004,99.6736343)(48.93774413,100.4574883)
\curveto(49.28409822,101.24134229)(49.80970574,101.82771369)(50.5145667,102.21660249)
\curveto(51.22550404,102.61156768)(51.99720526,102.80905028)(52.82967036,102.80905028)
\curveto(53.8808854,102.80905028)(54.74069424,102.54168923)(55.40909687,102.00696712)
\curveto(56.0774995,101.47832141)(56.50588483,100.72484935)(56.69425284,99.74655095)
\lineto(55.09920111,99.50045726)
\curveto(54.94729142,100.15063072)(54.67689217,100.63977992)(54.28800337,100.96790485)
\curveto(53.90519095,101.29602978)(53.44034731,101.46009224)(52.89347243,101.46009224)
\curveto(52.06708372,101.46009224)(51.3956429,101.16234925)(50.87914995,100.56686327)
\curveto(50.36265701,99.97745368)(50.10441054,99.04169)(50.10441054,97.75957222)
\curveto(50.10441054,96.45922529)(50.35354243,95.51434702)(50.85180621,94.92493743)
\curveto(51.35006999,94.33552784)(52.00024346,94.04082304)(52.80232661,94.04082304)
\curveto(53.44642369,94.04082304)(53.98418399,94.23830564)(54.41560751,94.63327083)
\curveto(54.84703103,95.02823602)(55.12046847,95.63587477)(55.23591983,96.45618709)
\closepath
}
}
{
\newrgbcolor{curcolor}{0 0 0}
\pscustom[linestyle=none,fillstyle=solid,fillcolor=curcolor]
{
\newpath
\moveto(58.26194271,92.91061496)
\lineto(58.26194271,106.27259118)
\lineto(59.90256735,106.27259118)
\lineto(59.90256735,98.65280119)
\lineto(63.785379,102.59030033)
\lineto(65.90907645,102.59030033)
\lineto(62.20855643,98.99915528)
\lineto(66.28277428,92.91061496)
\lineto(64.25933723,92.91061496)
\lineto(61.06011918,97.85983262)
\lineto(59.90256735,96.7478537)
\lineto(59.90256735,92.91061496)
\closepath
}
}
{
\newrgbcolor{curcolor}{0 0 0}
\pscustom[linestyle=none,fillstyle=solid,fillcolor=curcolor]
{
\newpath
\moveto(66.92989525,95.79993724)
\lineto(68.55229073,96.05514551)
\curveto(68.64343654,95.40497205)(68.89560662,94.90670827)(69.30880098,94.56035418)
\curveto(69.72807172,94.21400009)(70.31140492,94.04082304)(71.05880059,94.04082304)
\curveto(71.81227265,94.04082304)(72.3713003,94.19273273)(72.73588356,94.49655211)
\curveto(73.10046681,94.80644787)(73.28275844,95.16799293)(73.28275844,95.58118728)
\curveto(73.28275844,95.95184693)(73.12173417,96.24351353)(72.79968563,96.45618709)
\curveto(72.57485929,96.60202039)(72.01583163,96.78735021)(71.12260266,97.01217655)
\curveto(69.91947793,97.31599593)(69.08397464,97.5772806)(68.6160928,97.79603055)
\curveto(68.15428734,98.02085689)(67.80185686,98.32771446)(67.55880136,98.71660326)
\curveto(67.32182225,99.11156845)(67.20333269,99.54603016)(67.20333269,100.01998839)
\curveto(67.20333269,100.45141191)(67.30055489,100.84941529)(67.49499929,101.21399855)
\curveto(67.69552008,101.58465819)(67.96591933,101.89151576)(68.30619703,102.13457126)
\curveto(68.56140531,102.32293928)(68.9077594,102.48092535)(69.3452593,102.60852949)
\curveto(69.78883559,102.74221002)(70.26279382,102.80905028)(70.76713399,102.80905028)
\curveto(71.52668243,102.80905028)(72.19204687,102.6996753)(72.7632273,102.48092535)
\curveto(73.34048412,102.2621754)(73.76583125,101.96443241)(74.03926869,101.58769638)
\curveto(74.31270613,101.21703674)(74.50107414,100.71877296)(74.60437273,100.09290504)
\lineto(73.00020642,99.87415509)
\curveto(72.92728976,100.37241887)(72.7146162,100.76130767)(72.36218572,101.0408215)
\curveto(72.01583163,101.32033533)(71.52364424,101.46009224)(70.88562355,101.46009224)
\curveto(70.13215149,101.46009224)(69.59439119,101.3355263)(69.27234265,101.08639441)
\curveto(68.95029411,100.83726252)(68.78926984,100.54559592)(68.78926984,100.2113946)
\curveto(68.78926984,99.99872104)(68.8561101,99.80731483)(68.98979063,99.63717598)
\curveto(69.12347116,99.46096074)(69.33310653,99.31512744)(69.61869674,99.19967607)
\curveto(69.78275921,99.1389122)(70.26583202,98.99915528)(71.06791517,98.78040533)
\curveto(72.2285052,98.47050957)(73.03666474,98.21530129)(73.49239381,98.0147805)
\curveto(73.95419926,97.8203361)(74.31574432,97.53474588)(74.57702899,97.15800985)
\curveto(74.83831365,96.78127383)(74.96895598,96.31339199)(74.96895598,95.75436433)
\curveto(74.96895598,95.20748945)(74.80793171,94.69099651)(74.48588317,94.2048855)
\curveto(74.16991102,93.72485089)(73.71114376,93.35115305)(73.10958139,93.083792)
\curveto(72.50801902,92.82250734)(71.82746362,92.691865)(71.06791517,92.691865)
\curveto(69.81010295,92.691865)(68.85003372,92.95314967)(68.18770747,93.475719)
\curveto(67.53145762,93.99828833)(67.11218688,94.77302774)(66.92989525,95.79993724)
\closepath
}
}
{
\newrgbcolor{curcolor}{0 0 0}
\pscustom[linestyle=none,fillstyle=solid,fillcolor=curcolor]
{
\newpath
\moveto(83.26320687,92.91061496)
\lineto(83.26320687,94.33248964)
\curveto(82.50973482,93.23873988)(81.48586351,92.691865)(80.19159296,92.691865)
\curveto(79.62041253,92.691865)(79.08569043,92.80123998)(78.58742665,93.01998993)
\curveto(78.09523926,93.23873988)(77.72761781,93.51217732)(77.48456231,93.84030225)
\curveto(77.2475832,94.17450357)(77.08048254,94.58162153)(76.98326034,95.06165615)
\curveto(76.91642007,95.38370469)(76.88299994,95.89412124)(76.88299994,96.59290581)
\lineto(76.88299994,102.59030033)
\lineto(78.52362458,102.59030033)
\lineto(78.52362458,97.22181192)
\curveto(78.52362458,96.36504128)(78.55704471,95.78778446)(78.62388498,95.49004147)
\curveto(78.72718356,95.05861796)(78.94593352,94.71834025)(79.28013483,94.46920836)
\curveto(79.61433615,94.22615286)(80.0275305,94.10462511)(80.51971789,94.10462511)
\curveto(81.01190528,94.10462511)(81.47371074,94.22919105)(81.90513425,94.47832294)
\curveto(82.33655777,94.73353122)(82.64037715,95.07684712)(82.81659239,95.50827063)
\curveto(82.99888401,95.94577054)(83.09002983,96.57771484)(83.09002983,97.40410355)
\lineto(83.09002983,102.59030033)
\lineto(84.73065447,102.59030033)
\lineto(84.73065447,92.91061496)
\closepath
}
}
{
\newrgbcolor{curcolor}{0 0 0}
\pscustom[linestyle=none,fillstyle=solid,fillcolor=curcolor]
{
\newpath
\moveto(87.29809936,92.91061496)
\lineto(87.29809936,102.59030033)
\lineto(88.76554695,102.59030033)
\lineto(88.76554695,101.23222771)
\curveto(89.06936633,101.70618594)(89.4734461,102.08596016)(89.97778627,102.37155038)
\curveto(90.48212644,102.66321698)(91.05634506,102.80905028)(91.70044214,102.80905028)
\curveto(92.41745587,102.80905028)(93.00382727,102.66017878)(93.45955634,102.36243579)
\curveto(93.92136179,102.0646928)(94.24644853,101.64846026)(94.43481654,101.11373815)
\curveto(95.20044137,102.24394624)(96.19696893,102.80905028)(97.42439922,102.80905028)
\curveto(98.38446845,102.80905028)(99.12274954,102.54168923)(99.63924248,102.00696712)
\curveto(100.15573542,101.47832141)(100.41398189,100.66104728)(100.41398189,99.55514474)
\lineto(100.41398189,92.91061496)
\lineto(98.78247184,92.91061496)
\lineto(98.78247184,99.00826986)
\curveto(98.78247184,99.66451972)(98.72778435,100.13543976)(98.61840937,100.42102997)
\curveto(98.51511078,100.71269657)(98.32370457,100.94663749)(98.04419075,101.12285273)
\curveto(97.76467692,101.29906797)(97.43655199,101.38717559)(97.05981596,101.38717559)
\curveto(96.37926056,101.38717559)(95.81415652,101.15931106)(95.36450384,100.70358199)
\curveto(94.91485116,100.25392931)(94.69002482,99.53083919)(94.69002482,98.53431164)
\lineto(94.69002482,92.91061496)
\lineto(93.04940018,92.91061496)
\lineto(93.04940018,99.19967607)
\curveto(93.04940018,99.92884258)(92.91571965,100.47571746)(92.6483586,100.84030071)
\curveto(92.38099755,101.20488397)(91.94349764,101.38717559)(91.33585889,101.38717559)
\curveto(90.87405343,101.38717559)(90.44566811,101.26564784)(90.05070292,101.02259234)
\curveto(89.66181412,100.77953684)(89.3792621,100.42406816)(89.20304686,99.95618632)
\curveto(89.02683162,99.48830448)(88.938724,98.81382546)(88.938724,97.93274927)
\lineto(88.938724,92.91061496)
\closepath
}
}
{
\newrgbcolor{curcolor}{0 0 0}
\pscustom[linestyle=none,fillstyle=solid,fillcolor=curcolor]
{
\newpath
\moveto(107.40738167,97.75045764)
\curveto(107.40738167,99.54299197)(107.90564545,100.87068265)(108.90217301,101.73352968)
\curveto(109.73463811,102.45054341)(110.74939483,102.80905028)(111.94644318,102.80905028)
\curveto(113.27717205,102.80905028)(114.36484542,102.37155038)(115.20946329,101.49655057)
\curveto(116.05408116,100.62762715)(116.4763901,99.42450241)(116.4763901,97.88717636)
\curveto(116.4763901,96.64151691)(116.28802208,95.66018032)(115.91128605,94.94316659)
\curveto(115.54062641,94.23222925)(114.99678973,93.67927798)(114.279776,93.28431279)
\curveto(113.56883865,92.8893476)(112.79106105,92.691865)(111.94644318,92.691865)
\curveto(110.59140875,92.691865)(109.4946208,93.12632671)(108.65607932,93.99525013)
\curveto(107.82361422,94.86417355)(107.40738167,96.11590939)(107.40738167,97.75045764)
\closepath
\moveto(109.09357922,97.75045764)
\curveto(109.09357922,96.51087458)(109.36397847,95.58118728)(109.90477696,94.96139575)
\curveto(110.44557545,94.34768061)(111.12613086,94.04082304)(111.94644318,94.04082304)
\curveto(112.76067911,94.04082304)(113.43819632,94.35071881)(113.97899481,94.97051034)
\curveto(114.5197933,95.59030187)(114.79019255,96.53518013)(114.79019255,97.80514513)
\curveto(114.79019255,99.00219348)(114.51675511,99.90757522)(113.96988023,100.52129037)
\curveto(113.42908174,101.1410819)(112.75460272,101.45097766)(111.94644318,101.45097766)
\curveto(111.12613086,101.45097766)(110.44557545,101.14412009)(109.90477696,100.53040495)
\curveto(109.36397847,99.9166898)(109.09357922,98.9900407)(109.09357922,97.75045764)
\closepath
}
}
{
\newrgbcolor{curcolor}{0 0 0}
\pscustom[linestyle=none,fillstyle=solid,fillcolor=curcolor]
{
\newpath
\moveto(118.78863434,92.91061496)
\lineto(118.78863434,101.31425894)
\lineto(117.33941591,101.31425894)
\lineto(117.33941591,102.59030033)
\lineto(118.78863434,102.59030033)
\lineto(118.78863434,103.62024802)
\curveto(118.78863434,104.27042149)(118.84636002,104.7534943)(118.96181139,105.06946645)
\curveto(119.11979746,105.49481358)(119.3962731,105.83812947)(119.79123829,106.09941414)
\curveto(120.19227987,106.36677519)(120.75130752,106.50045572)(121.46832125,106.50045572)
\curveto(121.93012671,106.50045572)(122.44054326,106.44576823)(122.99957092,106.33639325)
\lineto(122.75347722,104.90540399)
\curveto(122.41319952,104.96616786)(122.09115098,104.9965498)(121.7873316,104.9965498)
\curveto(121.28906782,104.9965498)(120.93663734,104.89021302)(120.73004016,104.67753945)
\curveto(120.52344299,104.46486589)(120.4201444,104.0668625)(120.4201444,103.4835293)
\lineto(120.4201444,102.59030033)
\lineto(122.30686273,102.59030033)
\lineto(122.30686273,101.31425894)
\lineto(120.4201444,101.31425894)
\lineto(120.4201444,92.91061496)
\closepath
}
}
{
\newrgbcolor{curcolor}{0 0 0}
\pscustom[linestyle=none,fillstyle=solid,fillcolor=curcolor]
{
\newpath
\moveto(20.96486591,70.91065951)
\lineto(20.96486591,72.1320134)
\curveto(20.35115077,71.17194417)(19.44880722,70.69190955)(18.25783526,70.69190955)
\curveto(17.48613404,70.69190955)(16.77519669,70.90458312)(16.12502323,71.32993025)
\curveto(15.48092614,71.75527738)(14.97962417,72.34772516)(14.62111731,73.10727361)
\curveto(14.26868683,73.87289844)(14.09247159,74.75093644)(14.09247159,75.74138761)
\curveto(14.09247159,76.70753323)(14.25349586,77.58253304)(14.5755444,78.36638703)
\curveto(14.89759294,79.15631741)(15.38066575,79.76091798)(16.02476283,80.18018872)
\curveto(16.66885991,80.59945946)(17.38891184,80.80909483)(18.18491861,80.80909483)
\curveto(18.76825181,80.80909483)(19.28778295,80.68452888)(19.74351201,80.43539699)
\curveto(20.19924108,80.19234149)(20.56990072,79.87333115)(20.85549094,79.47836596)
\lineto(20.85549094,84.27263573)
\lineto(22.48700099,84.27263573)
\lineto(22.48700099,70.91065951)
\closepath
\moveto(15.77866913,75.74138761)
\curveto(15.77866913,74.50180455)(16.0399538,73.57515545)(16.56252313,72.9614403)
\curveto(17.08509246,72.34772516)(17.7018458,72.04086759)(18.41278314,72.04086759)
\curveto(19.12979687,72.04086759)(19.73743563,72.33253419)(20.23569941,72.9158674)
\curveto(20.74003957,73.50527699)(20.99220966,74.40154415)(20.99220966,75.60466889)
\curveto(20.99220966,76.92932138)(20.73700138,77.90154339)(20.22658482,78.52133492)
\curveto(19.71616827,79.14112645)(19.08726216,79.45102221)(18.33986649,79.45102221)
\curveto(17.61069998,79.45102221)(17.00002303,79.15327922)(16.50783564,78.55779324)
\curveto(16.02172464,77.96230726)(15.77866913,77.02350538)(15.77866913,75.74138761)
\closepath
}
}
{
\newrgbcolor{curcolor}{0 0 0}
\pscustom[linestyle=none,fillstyle=solid,fillcolor=curcolor]
{
\newpath
\moveto(31.37996914,72.10466966)
\curveto(30.77233039,71.58817672)(30.18595899,71.22359346)(29.62085495,71.0109199)
\curveto(29.06182729,70.79824634)(28.46026492,70.69190955)(27.81616784,70.69190955)
\curveto(26.75280002,70.69190955)(25.9355259,70.95015602)(25.36434547,71.46664897)
\curveto(24.79316504,71.9892183)(24.50757482,72.65458273)(24.50757482,73.46274228)
\curveto(24.50757482,73.93670051)(24.6139116,74.36812402)(24.82658517,74.75701283)
\curveto(25.04533512,75.15197802)(25.32788714,75.46795017)(25.67424123,75.70492928)
\curveto(26.02667171,75.9419084)(26.4216369,76.12116183)(26.8591368,76.24268958)
\curveto(27.18118534,76.32775901)(27.66729635,76.40979024)(28.31746982,76.48878328)
\curveto(29.6421223,76.64676936)(30.61738251,76.83513737)(31.24325042,77.05388732)
\curveto(31.24932681,77.27871366)(31.25236501,77.42150877)(31.25236501,77.48227264)
\curveto(31.25236501,78.15067527)(31.09741712,78.62159531)(30.78752136,78.89503275)
\curveto(30.36825062,79.26569239)(29.74542089,79.45102221)(28.91903219,79.45102221)
\curveto(28.14733097,79.45102221)(27.57615054,79.31430349)(27.2054909,79.04086605)
\curveto(26.84090764,78.773505)(26.5705084,78.29650858)(26.39429316,77.60987678)
\lineto(24.79012684,77.82862673)
\curveto(24.93596014,78.51525853)(25.17597745,79.0682098)(25.51017877,79.48748054)
\curveto(25.84438008,79.91282767)(26.32745289,80.2379144)(26.9593972,80.46274074)
\curveto(27.5913415,80.69364347)(28.3235462,80.80909483)(29.1560113,80.80909483)
\curveto(29.98240001,80.80909483)(30.65384083,80.71187263)(31.17033377,80.51742823)
\curveto(31.68682672,80.32298383)(32.06660094,80.07689013)(32.30965644,79.77914714)
\curveto(32.55271194,79.48748054)(32.72285079,79.1168209)(32.82007299,78.66716822)
\curveto(32.87476048,78.38765439)(32.90210423,77.88331422)(32.90210423,77.15414772)
\lineto(32.90210423,74.9666482)
\curveto(32.90210423,73.44147492)(32.93552436,72.4753293)(33.00236462,72.06821133)
\curveto(33.07528127,71.66716976)(33.21503819,71.28131915)(33.42163536,70.91065951)
\lineto(31.70809407,70.91065951)
\curveto(31.53795522,71.25093721)(31.42858024,71.64894059)(31.37996914,72.10466966)
\closepath
\moveto(31.24325042,75.76873135)
\curveto(30.64776444,75.52567585)(29.75453547,75.31907868)(28.56356351,75.14893982)
\curveto(27.88908449,75.05171762)(27.41208807,74.94234265)(27.13257424,74.8208149)
\curveto(26.85306042,74.69928714)(26.63734866,74.52003371)(26.48543897,74.2830546)
\curveto(26.33352928,74.05215187)(26.25757444,73.7939054)(26.25757444,73.50831518)
\curveto(26.25757444,73.07081528)(26.4216369,72.70623203)(26.74976183,72.41456542)
\curveto(27.08396314,72.12289882)(27.57007415,71.97706552)(28.20809484,71.97706552)
\curveto(28.84003915,71.97706552)(29.402105,72.11378424)(29.89429239,72.38722168)
\curveto(30.38647978,72.66673551)(30.74802484,73.04650973)(30.97892757,73.52654435)
\curveto(31.1551428,73.89720399)(31.24325042,74.44407887)(31.24325042,75.16716899)
\closepath
}
}
{
\newrgbcolor{curcolor}{0 0 0}
\pscustom[linestyle=none,fillstyle=solid,fillcolor=curcolor]
{
\newpath
\moveto(39.02423965,72.3781071)
\lineto(39.26121877,70.92888867)
\curveto(38.79941331,70.83166647)(38.38621896,70.78305537)(38.02163571,70.78305537)
\curveto(37.42614973,70.78305537)(36.96434427,70.87723937)(36.63621934,71.06560739)
\curveto(36.30809442,71.2539754)(36.07719169,71.5000691)(35.94351116,71.80388848)
\curveto(35.80983064,72.11378424)(35.74299037,72.76091952)(35.74299037,73.7452943)
\lineto(35.74299037,79.31430349)
\lineto(34.53986564,79.31430349)
\lineto(34.53986564,80.59034488)
\lineto(35.74299037,80.59034488)
\lineto(35.74299037,82.98747977)
\lineto(37.37450043,83.97185455)
\lineto(37.37450043,80.59034488)
\lineto(39.02423965,80.59034488)
\lineto(39.02423965,79.31430349)
\lineto(37.37450043,79.31430349)
\lineto(37.37450043,73.65414849)
\curveto(37.37450043,73.18626664)(37.40184418,72.88548546)(37.45653166,72.75180493)
\curveto(37.51729554,72.61812441)(37.61147955,72.51178763)(37.73908369,72.43279459)
\curveto(37.87276421,72.35380155)(38.06113223,72.31430503)(38.30418773,72.31430503)
\curveto(38.48647935,72.31430503)(38.72649666,72.33557239)(39.02423965,72.3781071)
\closepath
}
}
{
\newrgbcolor{curcolor}{0 0 0}
\pscustom[linestyle=none,fillstyle=solid,fillcolor=curcolor]
{
\newpath
\moveto(46.92927123,72.10466966)
\curveto(46.32163247,71.58817672)(45.73526107,71.22359346)(45.17015703,71.0109199)
\curveto(44.61112938,70.79824634)(44.00956701,70.69190955)(43.36546993,70.69190955)
\curveto(42.30210211,70.69190955)(41.48482798,70.95015602)(40.91364755,71.46664897)
\curveto(40.34246712,71.9892183)(40.05687691,72.65458273)(40.05687691,73.46274228)
\curveto(40.05687691,73.93670051)(40.16321369,74.36812402)(40.37588725,74.75701283)
\curveto(40.59463721,75.15197802)(40.87718923,75.46795017)(41.22354332,75.70492928)
\curveto(41.57597379,75.9419084)(41.97093899,76.12116183)(42.40843889,76.24268958)
\curveto(42.73048743,76.32775901)(43.21659843,76.40979024)(43.8667719,76.48878328)
\curveto(45.19142439,76.64676936)(46.16668459,76.83513737)(46.79255251,77.05388732)
\curveto(46.7986289,77.27871366)(46.80166709,77.42150877)(46.80166709,77.48227264)
\curveto(46.80166709,78.15067527)(46.64671921,78.62159531)(46.33682344,78.89503275)
\curveto(45.9175527,79.26569239)(45.29472298,79.45102221)(44.46833427,79.45102221)
\curveto(43.69663305,79.45102221)(43.12545262,79.31430349)(42.75479298,79.04086605)
\curveto(42.39020973,78.773505)(42.11981048,78.29650858)(41.94359524,77.60987678)
\lineto(40.33942893,77.82862673)
\curveto(40.48526223,78.51525853)(40.72527954,79.0682098)(41.05948085,79.48748054)
\curveto(41.39368217,79.91282767)(41.87675498,80.2379144)(42.50869928,80.46274074)
\curveto(43.14064359,80.69364347)(43.87284829,80.80909483)(44.70531338,80.80909483)
\curveto(45.53170209,80.80909483)(46.20314292,80.71187263)(46.71963586,80.51742823)
\curveto(47.2361288,80.32298383)(47.61590302,80.07689013)(47.85895852,79.77914714)
\curveto(48.10201403,79.48748054)(48.27215288,79.1168209)(48.36937508,78.66716822)
\curveto(48.42406257,78.38765439)(48.45140631,77.88331422)(48.45140631,77.15414772)
\lineto(48.45140631,74.9666482)
\curveto(48.45140631,73.44147492)(48.48482644,72.4753293)(48.55166671,72.06821133)
\curveto(48.62458336,71.66716976)(48.76434027,71.28131915)(48.97093745,70.91065951)
\lineto(47.25739616,70.91065951)
\curveto(47.08725731,71.25093721)(46.97788233,71.64894059)(46.92927123,72.10466966)
\closepath
\moveto(46.79255251,75.76873135)
\curveto(46.19706653,75.52567585)(45.30383756,75.31907868)(44.1128656,75.14893982)
\curveto(43.43838658,75.05171762)(42.96139016,74.94234265)(42.68187633,74.8208149)
\curveto(42.4023625,74.69928714)(42.18665074,74.52003371)(42.03474106,74.2830546)
\curveto(41.88283137,74.05215187)(41.80687652,73.7939054)(41.80687652,73.50831518)
\curveto(41.80687652,73.07081528)(41.97093899,72.70623203)(42.29906391,72.41456542)
\curveto(42.63326523,72.12289882)(43.11937623,71.97706552)(43.75739693,71.97706552)
\curveto(44.38934123,71.97706552)(44.95140708,72.11378424)(45.44359447,72.38722168)
\curveto(45.93578186,72.66673551)(46.29732692,73.04650973)(46.52822965,73.52654435)
\curveto(46.70444489,73.89720399)(46.79255251,74.44407887)(46.79255251,75.16716899)
\closepath
}
}
{
\newrgbcolor{curcolor}{0 0 0}
\pscustom[linestyle=none,fillstyle=solid,fillcolor=curcolor]
{
\newpath
\moveto(59.74419892,72.3781071)
\lineto(59.98117803,70.92888867)
\curveto(59.51937258,70.83166647)(59.10617822,70.78305537)(58.74159497,70.78305537)
\curveto(58.14610899,70.78305537)(57.68430353,70.87723937)(57.35617861,71.06560739)
\curveto(57.02805368,71.2539754)(56.79715095,71.5000691)(56.66347043,71.80388848)
\curveto(56.5297899,72.11378424)(56.46294964,72.76091952)(56.46294964,73.7452943)
\lineto(56.46294964,79.31430349)
\lineto(55.2598249,79.31430349)
\lineto(55.2598249,80.59034488)
\lineto(56.46294964,80.59034488)
\lineto(56.46294964,82.98747977)
\lineto(58.09445969,83.97185455)
\lineto(58.09445969,80.59034488)
\lineto(59.74419892,80.59034488)
\lineto(59.74419892,79.31430349)
\lineto(58.09445969,79.31430349)
\lineto(58.09445969,73.65414849)
\curveto(58.09445969,73.18626664)(58.12180344,72.88548546)(58.17649093,72.75180493)
\curveto(58.2372548,72.61812441)(58.33143881,72.51178763)(58.45904295,72.43279459)
\curveto(58.59272347,72.35380155)(58.78109149,72.31430503)(59.02414699,72.31430503)
\curveto(59.20643862,72.31430503)(59.44645593,72.33557239)(59.74419892,72.3781071)
\closepath
}
}
{
\newrgbcolor{curcolor}{0 0 0}
\pscustom[linestyle=none,fillstyle=solid,fillcolor=curcolor]
{
\newpath
\moveto(61.33282372,70.91065951)
\lineto(61.33282372,84.27263573)
\lineto(62.97344836,84.27263573)
\lineto(62.97344836,79.47836596)
\curveto(63.73907319,80.36551854)(64.70521882,80.80909483)(65.87188523,80.80909483)
\curveto(66.58889896,80.80909483)(67.21172868,80.66629972)(67.7403744,80.38070951)
\curveto(68.26902011,80.10119568)(68.64575614,79.71230688)(68.87058248,79.2140431)
\curveto(69.10148521,78.71577932)(69.21693657,77.9926892)(69.21693657,77.04477274)
\lineto(69.21693657,70.91065951)
\lineto(67.57631193,70.91065951)
\lineto(67.57631193,77.04477274)
\curveto(67.57631193,77.86508506)(67.3970585,78.46057104)(67.03855164,78.83123068)
\curveto(66.68612116,79.20796671)(66.18481918,79.39633472)(65.53464572,79.39633472)
\curveto(65.04853471,79.39633472)(64.58976745,79.26873058)(64.15834394,79.01352231)
\curveto(63.73299681,78.76439042)(63.42917743,78.42411271)(63.2468858,77.9926892)
\curveto(63.06459418,77.56126568)(62.97344836,76.9657797)(62.97344836,76.20623126)
\lineto(62.97344836,70.91065951)
\closepath
}
}
{
\newrgbcolor{curcolor}{0 0 0}
\pscustom[linestyle=none,fillstyle=solid,fillcolor=curcolor]
{
\newpath
\moveto(71.72058321,82.3859174)
\lineto(71.72058321,84.27263573)
\lineto(73.36120785,84.27263573)
\lineto(73.36120785,82.3859174)
\closepath
\moveto(71.72058321,70.91065951)
\lineto(71.72058321,80.59034488)
\lineto(73.36120785,80.59034488)
\lineto(73.36120785,70.91065951)
\closepath
}
}
{
\newrgbcolor{curcolor}{0 0 0}
\pscustom[linestyle=none,fillstyle=solid,fillcolor=curcolor]
{
\newpath
\moveto(75.19921139,73.79998179)
\lineto(76.82160687,74.05519006)
\curveto(76.91275268,73.4050166)(77.16492276,72.90675282)(77.57811712,72.56039873)
\curveto(77.99738786,72.21404464)(78.58072106,72.04086759)(79.32811673,72.04086759)
\curveto(80.08158879,72.04086759)(80.64061644,72.19277728)(81.0051997,72.49659666)
\curveto(81.36978295,72.80649242)(81.55207458,73.16803748)(81.55207458,73.58123183)
\curveto(81.55207458,73.95189148)(81.39105031,74.24355808)(81.06900177,74.45623164)
\curveto(80.84417543,74.60206494)(80.28514777,74.78739476)(79.3919188,75.0122211)
\curveto(78.18879407,75.31604048)(77.35329078,75.57732515)(76.88540894,75.7960751)
\curveto(76.42360348,76.02090144)(76.071173,76.32775901)(75.8281175,76.71664781)
\curveto(75.59113839,77.111613)(75.47264883,77.54607471)(75.47264883,78.02003294)
\curveto(75.47264883,78.45145646)(75.56987103,78.84945984)(75.76431543,79.2140431)
\curveto(75.96483622,79.58470274)(76.23523547,79.89156031)(76.57551317,80.13461581)
\curveto(76.83072145,80.32298383)(77.17707554,80.4809699)(77.61457544,80.60857404)
\curveto(78.05815173,80.74225457)(78.53210996,80.80909483)(79.03645013,80.80909483)
\curveto(79.79599857,80.80909483)(80.46136301,80.69971985)(81.03254344,80.4809699)
\curveto(81.60980026,80.26221995)(82.03514739,79.96447696)(82.30858483,79.58774093)
\curveto(82.58202227,79.21708129)(82.77039028,78.71881751)(82.87368887,78.09294959)
\lineto(81.26952256,77.87419964)
\curveto(81.1966059,78.37246342)(80.98393234,78.76135222)(80.63150186,79.04086605)
\curveto(80.28514777,79.32037988)(79.79296038,79.46013679)(79.15493969,79.46013679)
\curveto(78.40146763,79.46013679)(77.86370733,79.33557085)(77.54165879,79.08643896)
\curveto(77.21961025,78.83730707)(77.05858598,78.54564047)(77.05858598,78.21143915)
\curveto(77.05858598,77.99876559)(77.12542624,77.80735938)(77.25910677,77.63722053)
\curveto(77.3927873,77.46100529)(77.60242267,77.31517199)(77.88801288,77.19972062)
\curveto(78.05207535,77.13895675)(78.53514816,76.99919983)(79.33723131,76.78044988)
\curveto(80.49782134,76.47055412)(81.30598088,76.21534584)(81.76170995,76.01482505)
\curveto(82.2235154,75.82038065)(82.58506046,75.53479043)(82.84634513,75.1580544)
\curveto(83.10762979,74.78131838)(83.23827212,74.31343654)(83.23827212,73.75440888)
\curveto(83.23827212,73.207534)(83.07724785,72.69104106)(82.75519931,72.20493005)
\curveto(82.43922716,71.72489544)(81.9804599,71.3511976)(81.37889753,71.08383655)
\curveto(80.77733516,70.82255189)(80.09677976,70.69190955)(79.33723131,70.69190955)
\curveto(78.07941909,70.69190955)(77.11934986,70.95319422)(76.45702361,71.47576355)
\curveto(75.80077376,71.99833288)(75.38150302,72.77307229)(75.19921139,73.79998179)
\closepath
}
}
{
\newrgbcolor{curcolor}{0 0 0}
\pscustom[linestyle=none,fillstyle=solid,fillcolor=curcolor]
{
\newpath
\moveto(91.87245399,70.91065951)
\lineto(90.35031891,70.91065951)
\lineto(90.35031891,84.27263573)
\lineto(91.99094355,84.27263573)
\lineto(91.99094355,79.5057097)
\curveto(92.68365173,80.37463312)(93.56776612,80.80909483)(94.64328672,80.80909483)
\curveto(95.2387727,80.80909483)(95.80083855,80.68756708)(96.32948426,80.44451158)
\curveto(96.86420637,80.20753246)(97.30170627,79.87029295)(97.64198398,79.43279305)
\curveto(97.98833807,79.00136953)(98.25873731,78.4788002)(98.45318171,77.86508506)
\curveto(98.64762612,77.25136992)(98.74484832,76.59512006)(98.74484832,75.89633549)
\curveto(98.74484832,74.23748169)(98.33469216,72.95536392)(97.51437984,72.04998217)
\curveto(96.69406752,71.14460043)(95.70969273,70.69190955)(94.56125549,70.69190955)
\curveto(93.41889463,70.69190955)(92.52262746,71.16890598)(91.87245399,72.12289882)
\closepath
\moveto(91.85422483,75.82341884)
\curveto(91.85422483,74.66282882)(92.01221091,73.82428734)(92.32818306,73.30779439)
\curveto(92.844676,72.46317653)(93.54346057,72.04086759)(94.42453677,72.04086759)
\curveto(95.1415505,72.04086759)(95.76134203,72.35076336)(96.28391136,72.97055489)
\curveto(96.80648069,73.5964228)(97.06776535,74.5261101)(97.06776535,75.75961677)
\curveto(97.06776535,77.02350538)(96.81559527,77.95623087)(96.3112551,78.55779324)
\curveto(95.81299132,79.15935561)(95.20839076,79.46013679)(94.49745342,79.46013679)
\curveto(93.78043969,79.46013679)(93.16064815,79.14720283)(92.63807883,78.52133492)
\curveto(92.1155095,77.90154339)(91.85422483,77.00223803)(91.85422483,75.82341884)
\closepath
}
}
{
\newrgbcolor{curcolor}{0 0 0}
\pscustom[linestyle=none,fillstyle=solid,fillcolor=curcolor]
{
\newpath
\moveto(100.70161626,70.91065951)
\lineto(100.70161626,84.27263573)
\lineto(102.3422409,84.27263573)
\lineto(102.3422409,70.91065951)
\closepath
}
}
{
\newrgbcolor{curcolor}{0 0 0}
\pscustom[linestyle=none,fillstyle=solid,fillcolor=curcolor]
{
\newpath
\moveto(104.27138644,75.75050219)
\curveto(104.27138644,77.54303652)(104.76965022,78.8707272)(105.76617777,79.73357423)
\curveto(106.59864287,80.45058796)(107.61339959,80.80909483)(108.81044794,80.80909483)
\curveto(110.14117681,80.80909483)(111.22885019,80.37159493)(112.07346806,79.49659512)
\curveto(112.91808592,78.6276717)(113.34039486,77.42454696)(113.34039486,75.88722091)
\curveto(113.34039486,74.64156146)(113.15202685,73.66022487)(112.77529082,72.94321114)
\curveto(112.40463118,72.2322738)(111.86079449,71.67932253)(111.14378076,71.28435734)
\curveto(110.43284342,70.88939215)(109.65506581,70.69190955)(108.81044794,70.69190955)
\curveto(107.45541351,70.69190955)(106.35862556,71.12637126)(105.52008408,71.99529468)
\curveto(104.68761898,72.8642181)(104.27138644,74.11595394)(104.27138644,75.75050219)
\closepath
\moveto(105.95758398,75.75050219)
\curveto(105.95758398,74.51091913)(106.22798323,73.58123183)(106.76878172,72.9614403)
\curveto(107.30958021,72.34772516)(107.99013562,72.04086759)(108.81044794,72.04086759)
\curveto(109.62468387,72.04086759)(110.30220108,72.35076336)(110.84299958,72.97055489)
\curveto(111.38379807,73.59034642)(111.65419731,74.53522468)(111.65419731,75.80518968)
\curveto(111.65419731,77.00223803)(111.38075987,77.90761977)(110.83388499,78.52133492)
\curveto(110.2930865,79.14112645)(109.61860748,79.45102221)(108.81044794,79.45102221)
\curveto(107.99013562,79.45102221)(107.30958021,79.14416464)(106.76878172,78.5304495)
\curveto(106.22798323,77.91673435)(105.95758398,76.99008525)(105.95758398,75.75050219)
\closepath
}
}
{
\newrgbcolor{curcolor}{0 0 0}
\pscustom[linestyle=none,fillstyle=solid,fillcolor=curcolor]
{
\newpath
\moveto(121.57711697,74.45623164)
\lineto(123.19039786,74.24659627)
\curveto(123.01418263,73.13461735)(122.56149175,72.26265574)(121.83232525,71.63071143)
\curveto(121.10923513,71.00484351)(120.21904435,70.69190955)(119.16175292,70.69190955)
\curveto(117.83710043,70.69190955)(116.77069441,71.12333307)(115.96253487,71.9861801)
\curveto(115.16045171,72.85510352)(114.75941013,74.09772478)(114.75941013,75.71404387)
\curveto(114.75941013,76.75918253)(114.93258718,77.67367885)(115.27894127,78.45753285)
\curveto(115.62529536,79.24138684)(116.15090288,79.82775824)(116.85576384,80.21664704)
\curveto(117.56670118,80.61161223)(118.3384024,80.80909483)(119.1708675,80.80909483)
\curveto(120.22208255,80.80909483)(121.08189138,80.54173378)(121.75029401,80.00701167)
\curveto(122.41869665,79.47836596)(122.84708197,78.7248939)(123.03544998,77.7465955)
\lineto(121.44039825,77.50050181)
\curveto(121.28848856,78.15067527)(121.01808931,78.63982447)(120.62920051,78.9679494)
\curveto(120.2463881,79.29607433)(119.78154445,79.46013679)(119.23466957,79.46013679)
\curveto(118.40828086,79.46013679)(117.73684004,79.1623938)(117.22034709,78.56690782)
\curveto(116.70385415,77.97749823)(116.44560768,77.04173455)(116.44560768,75.75961677)
\curveto(116.44560768,74.45926984)(116.69473957,73.51439157)(117.19300335,72.92498198)
\curveto(117.69126713,72.33557239)(118.3414406,72.04086759)(119.14352375,72.04086759)
\curveto(119.78762084,72.04086759)(120.32538113,72.23835019)(120.75680465,72.63331538)
\curveto(121.18822817,73.02828057)(121.46166561,73.63591932)(121.57711697,74.45623164)
\closepath
}
}
{
\newrgbcolor{curcolor}{0 0 0}
\pscustom[linestyle=none,fillstyle=solid,fillcolor=curcolor]
{
\newpath
\moveto(124.60314367,70.91065951)
\lineto(124.60314367,84.27263573)
\lineto(126.24376831,84.27263573)
\lineto(126.24376831,76.65284574)
\lineto(130.12657995,80.59034488)
\lineto(132.2502774,80.59034488)
\lineto(128.54975738,76.99919983)
\lineto(132.62397524,70.91065951)
\lineto(130.60053818,70.91065951)
\lineto(127.40132013,75.85987717)
\lineto(126.24376831,74.74789825)
\lineto(126.24376831,70.91065951)
\closepath
}
}
{
\newrgbcolor{curcolor}{0 0 0}
\pscustom[linestyle=none,fillstyle=solid,fillcolor=curcolor]
{
\newpath
\moveto(14.68491938,45.20107545)
\lineto(14.68491938,58.59039543)
\lineto(16.17971071,58.59039543)
\lineto(16.17971071,57.3325832)
\curveto(16.53214119,57.8247706)(16.93014458,58.19239204)(17.37372087,58.43544754)
\curveto(17.81729716,58.68457943)(18.35505746,58.80914538)(18.98700176,58.80914538)
\curveto(19.81339047,58.80914538)(20.54255698,58.59647181)(21.17450128,58.17112469)
\curveto(21.80644559,57.74577756)(22.28344201,57.14421519)(22.60549055,56.36643758)
\curveto(22.92753909,55.59473636)(23.08856336,54.7470803)(23.08856336,53.82346939)
\curveto(23.08856336,52.83301822)(22.90930993,51.93978925)(22.55080306,51.14378248)
\curveto(22.19837258,50.3538521)(21.68187964,49.74621334)(21.00132424,49.32086622)
\curveto(20.32684522,48.90159547)(19.61590787,48.6919601)(18.86851221,48.6919601)
\curveto(18.32163733,48.6919601)(17.82944993,48.80741147)(17.39195003,49.03831419)
\curveto(16.96052651,49.26921692)(16.60505784,49.56088352)(16.32554401,49.913314)
\lineto(16.32554401,45.20107545)
\closepath
\moveto(16.17059613,53.69586525)
\curveto(16.17059613,52.4502058)(16.42276622,51.52963309)(16.92710638,50.93414711)
\curveto(17.43144655,50.33866113)(18.0421235,50.04091814)(18.75913723,50.04091814)
\curveto(19.48830374,50.04091814)(20.11113346,50.34777571)(20.6276264,50.96149085)
\curveto(21.15019573,51.58128238)(21.4114804,52.53831342)(21.4114804,53.83258397)
\curveto(21.4114804,55.06609065)(21.15627212,55.98970155)(20.64585557,56.6034167)
\curveto(20.1415154,57.21713184)(19.53691484,57.52398941)(18.83205388,57.52398941)
\curveto(18.13326931,57.52398941)(17.51347778,57.19586448)(16.97267929,56.53961463)
\curveto(16.43795718,55.88944116)(16.17059613,54.9415247)(16.17059613,53.69586525)
\closepath
}
}
{
\newrgbcolor{curcolor}{0 0 0}
\pscustom[linestyle=none,fillstyle=solid,fillcolor=curcolor]
{
\newpath
\moveto(24.45288733,53.75055274)
\curveto(24.45288733,55.54308707)(24.95115111,56.87077775)(25.94767867,57.73362478)
\curveto(26.78014377,58.45063851)(27.79490049,58.80914538)(28.99194884,58.80914538)
\curveto(30.32267771,58.80914538)(31.41035108,58.37164548)(32.25496895,57.49664567)
\curveto(33.09958682,56.62772225)(33.52189576,55.42459751)(33.52189576,53.88727146)
\curveto(33.52189576,52.64161201)(33.33352774,51.66027542)(32.95679171,50.94326169)
\curveto(32.58613207,50.23232435)(32.04229539,49.67937308)(31.32528166,49.28440789)
\curveto(30.61434431,48.8894427)(29.83656671,48.6919601)(28.99194884,48.6919601)
\curveto(27.63691441,48.6919601)(26.54012646,49.12642181)(25.70158498,49.99534523)
\curveto(24.86911988,50.86426865)(24.45288733,52.11600449)(24.45288733,53.75055274)
\closepath
\moveto(26.13908488,53.75055274)
\curveto(26.13908488,52.51096968)(26.40948413,51.58128238)(26.95028262,50.96149085)
\curveto(27.49108111,50.34777571)(28.17163652,50.04091814)(28.99194884,50.04091814)
\curveto(29.80618477,50.04091814)(30.48370198,50.35081391)(31.02450047,50.97060544)
\curveto(31.56529896,51.59039697)(31.83569821,52.53527523)(31.83569821,53.80524023)
\curveto(31.83569821,55.00228858)(31.56226077,55.90767032)(31.01538589,56.52138547)
\curveto(30.4745874,57.141177)(29.80010838,57.45107276)(28.99194884,57.45107276)
\curveto(28.17163652,57.45107276)(27.49108111,57.14421519)(26.95028262,56.53050005)
\curveto(26.40948413,55.9167849)(26.13908488,54.9901358)(26.13908488,53.75055274)
\closepath
}
}
{
\newrgbcolor{curcolor}{0 0 0}
\pscustom[linestyle=none,fillstyle=solid,fillcolor=curcolor]
{
\newpath
\moveto(35.45132377,60.38596795)
\lineto(35.45132377,62.27268628)
\lineto(37.09194841,62.27268628)
\lineto(37.09194841,60.38596795)
\closepath
\moveto(35.45132377,48.91071006)
\lineto(35.45132377,58.59039543)
\lineto(37.09194841,58.59039543)
\lineto(37.09194841,48.91071006)
\closepath
}
}
{
\newrgbcolor{curcolor}{0 0 0}
\pscustom[linestyle=none,fillstyle=solid,fillcolor=curcolor]
{
\newpath
\moveto(39.58620181,48.91071006)
\lineto(39.58620181,58.59039543)
\lineto(41.06276398,58.59039543)
\lineto(41.06276398,57.21409365)
\curveto(41.77370132,58.27746147)(42.80061082,58.80914538)(44.14349247,58.80914538)
\curveto(44.72682568,58.80914538)(45.26154778,58.7028086)(45.74765878,58.49013503)
\curveto(46.23984618,58.28353786)(46.60746762,58.01010042)(46.85052313,57.66982271)
\curveto(47.09357863,57.32954501)(47.26371748,56.92546524)(47.36093968,56.4575834)
\curveto(47.42170356,56.15376402)(47.45208549,55.62208011)(47.45208549,54.86253166)
\lineto(47.45208549,48.91071006)
\lineto(45.81146085,48.91071006)
\lineto(45.81146085,54.79872959)
\curveto(45.81146085,55.46713223)(45.74765878,55.965396)(45.62005465,56.29352093)
\curveto(45.49245051,56.62772225)(45.26458597,56.89204511)(44.93646105,57.08648951)
\curveto(44.61441251,57.2870103)(44.23463828,57.38727069)(43.79713838,57.38727069)
\curveto(43.09835381,57.38727069)(42.49375325,57.16548255)(41.9833367,56.72190625)
\curveto(41.47899653,56.27832996)(41.22682644,55.43675029)(41.22682644,54.19716723)
\lineto(41.22682644,48.91071006)
\closepath
}
}
{
\newrgbcolor{curcolor}{0 0 0}
\pscustom[linestyle=none,fillstyle=solid,fillcolor=curcolor]
{
\newpath
\moveto(53.54687718,50.37815765)
\lineto(53.78385629,48.92893922)
\curveto(53.32205084,48.83171702)(52.90885648,48.78310592)(52.54427323,48.78310592)
\curveto(51.94878725,48.78310592)(51.48698179,48.87728992)(51.15885687,49.06565794)
\curveto(50.83073194,49.25402595)(50.59982921,49.50011965)(50.46614869,49.80393903)
\curveto(50.33246816,50.11383479)(50.2656279,50.76097007)(50.2656279,51.74534485)
\lineto(50.2656279,57.31435404)
\lineto(49.06250316,57.31435404)
\lineto(49.06250316,58.59039543)
\lineto(50.2656279,58.59039543)
\lineto(50.2656279,60.98753032)
\lineto(51.89713795,61.9719051)
\lineto(51.89713795,58.59039543)
\lineto(53.54687718,58.59039543)
\lineto(53.54687718,57.31435404)
\lineto(51.89713795,57.31435404)
\lineto(51.89713795,51.65419904)
\curveto(51.89713795,51.18631719)(51.9244817,50.88553601)(51.97916919,50.75185548)
\curveto(52.03993306,50.61817496)(52.13411707,50.51183818)(52.26172121,50.43284514)
\curveto(52.39540173,50.3538521)(52.58376975,50.31435558)(52.82682525,50.31435558)
\curveto(53.00911688,50.31435558)(53.24913419,50.33562294)(53.54687718,50.37815765)
\closepath
}
}
{
\newrgbcolor{curcolor}{0 0 0}
\pscustom[linestyle=none,fillstyle=solid,fillcolor=curcolor]
{
\newpath
\moveto(54.47925213,51.80003234)
\lineto(56.1016476,52.05524061)
\curveto(56.19279342,51.40506715)(56.4449635,50.90680337)(56.85815785,50.56044928)
\curveto(57.2774286,50.21409519)(57.8607618,50.04091814)(58.60815747,50.04091814)
\curveto(59.36162953,50.04091814)(59.92065718,50.19282783)(60.28524043,50.49664721)
\curveto(60.64982369,50.80654297)(60.83211531,51.16808803)(60.83211531,51.58128238)
\curveto(60.83211531,51.95194203)(60.67109104,52.24360863)(60.3490425,52.45628219)
\curveto(60.12421616,52.60211549)(59.56518851,52.78744531)(58.67195954,53.01227165)
\curveto(57.4688348,53.31609103)(56.63333151,53.5773757)(56.16544967,53.79612565)
\curveto(55.70364422,54.02095199)(55.35121374,54.32780956)(55.10815824,54.71669836)
\curveto(54.87117912,55.11166355)(54.75268957,55.54612526)(54.75268957,56.02008349)
\curveto(54.75268957,56.45150701)(54.84991177,56.84951039)(55.04435617,57.21409365)
\curveto(55.24487696,57.58475329)(55.51527621,57.89161086)(55.85555391,58.13466636)
\curveto(56.11076219,58.32303438)(56.45711628,58.48102045)(56.89461618,58.60862459)
\curveto(57.33819247,58.74230512)(57.8121507,58.80914538)(58.31649087,58.80914538)
\curveto(59.07603931,58.80914538)(59.74140375,58.6997704)(60.31258418,58.48102045)
\curveto(60.889841,58.2622705)(61.31518812,57.96452751)(61.58862556,57.58779148)
\curveto(61.862063,57.21713184)(62.05043102,56.71886806)(62.15372961,56.09300014)
\lineto(60.54956329,55.87425019)
\curveto(60.47664664,56.37251397)(60.26397308,56.76140277)(59.9115426,57.0409166)
\curveto(59.56518851,57.32043043)(59.07300112,57.46018734)(58.43498042,57.46018734)
\curveto(57.68150837,57.46018734)(57.14374807,57.3356214)(56.82169953,57.08648951)
\curveto(56.49965099,56.83735762)(56.33862672,56.54569102)(56.33862672,56.2114897)
\curveto(56.33862672,55.99881614)(56.40546698,55.80740993)(56.53914751,55.63727108)
\curveto(56.67282803,55.46105584)(56.8824634,55.31522254)(57.16805362,55.19977117)
\curveto(57.33211608,55.1390073)(57.81518889,54.99925038)(58.61727205,54.78050043)
\curveto(59.77786207,54.47060467)(60.58602162,54.21539639)(61.04175068,54.0148756)
\curveto(61.50355614,53.8204312)(61.8651012,53.53484098)(62.12638586,53.15810495)
\curveto(62.38767053,52.78136893)(62.51831286,52.31348709)(62.51831286,51.75445943)
\curveto(62.51831286,51.20758455)(62.35728859,50.69109161)(62.03524005,50.2049806)
\curveto(61.7192679,49.72494599)(61.26050064,49.35124815)(60.65893827,49.0838871)
\curveto(60.0573759,48.82260244)(59.3768205,48.6919601)(58.61727205,48.6919601)
\curveto(57.35945983,48.6919601)(56.39939059,48.95324477)(55.73706435,49.4758141)
\curveto(55.0808145,49.99838343)(54.66154375,50.77312284)(54.47925213,51.80003234)
\closepath
}
}
{
\newrgbcolor{curcolor}{0 0 0}
\pscustom[linestyle=none,fillstyle=solid,fillcolor=curcolor]
{
\newpath
\moveto(73.22150469,50.37815765)
\lineto(73.45848381,48.92893922)
\curveto(72.99667835,48.83171702)(72.583484,48.78310592)(72.21890075,48.78310592)
\curveto(71.62341477,48.78310592)(71.16160931,48.87728992)(70.83348438,49.06565794)
\curveto(70.50535946,49.25402595)(70.27445673,49.50011965)(70.1407762,49.80393903)
\curveto(70.00709568,50.11383479)(69.94025541,50.76097007)(69.94025541,51.74534485)
\lineto(69.94025541,57.31435404)
\lineto(68.73713068,57.31435404)
\lineto(68.73713068,58.59039543)
\lineto(69.94025541,58.59039543)
\lineto(69.94025541,60.98753032)
\lineto(71.57176547,61.9719051)
\lineto(71.57176547,58.59039543)
\lineto(73.22150469,58.59039543)
\lineto(73.22150469,57.31435404)
\lineto(71.57176547,57.31435404)
\lineto(71.57176547,51.65419904)
\curveto(71.57176547,51.18631719)(71.59910922,50.88553601)(71.6537967,50.75185548)
\curveto(71.71456058,50.61817496)(71.80874459,50.51183818)(71.93634872,50.43284514)
\curveto(72.07002925,50.3538521)(72.25839727,50.31435558)(72.50145277,50.31435558)
\curveto(72.68374439,50.31435558)(72.9237617,50.33562294)(73.22150469,50.37815765)
\closepath
}
}
{
\newrgbcolor{curcolor}{0 0 0}
\pscustom[linestyle=none,fillstyle=solid,fillcolor=curcolor]
{
\newpath
\moveto(74.19945255,53.75055274)
\curveto(74.19945255,55.54308707)(74.69771633,56.87077775)(75.69424389,57.73362478)
\curveto(76.52670898,58.45063851)(77.54146571,58.80914538)(78.73851405,58.80914538)
\curveto(80.06924293,58.80914538)(81.1569163,58.37164548)(82.00153417,57.49664567)
\curveto(82.84615204,56.62772225)(83.26846097,55.42459751)(83.26846097,53.88727146)
\curveto(83.26846097,52.64161201)(83.08009296,51.66027542)(82.70335693,50.94326169)
\curveto(82.33269729,50.23232435)(81.78886061,49.67937308)(81.07184687,49.28440789)
\curveto(80.36090953,48.8894427)(79.58313192,48.6919601)(78.73851405,48.6919601)
\curveto(77.38347963,48.6919601)(76.28669168,49.12642181)(75.44815019,49.99534523)
\curveto(74.6156851,50.86426865)(74.19945255,52.11600449)(74.19945255,53.75055274)
\closepath
\moveto(75.8856501,53.75055274)
\curveto(75.8856501,52.51096968)(76.15604934,51.58128238)(76.69684784,50.96149085)
\curveto(77.23764633,50.34777571)(77.91820173,50.04091814)(78.73851405,50.04091814)
\curveto(79.55274999,50.04091814)(80.2302672,50.35081391)(80.77106569,50.97060544)
\curveto(81.31186418,51.59039697)(81.58226343,52.53527523)(81.58226343,53.80524023)
\curveto(81.58226343,55.00228858)(81.30882599,55.90767032)(80.76195111,56.52138547)
\curveto(80.22115262,57.141177)(79.5466736,57.45107276)(78.73851405,57.45107276)
\curveto(77.91820173,57.45107276)(77.23764633,57.14421519)(76.69684784,56.53050005)
\curveto(76.15604934,55.9167849)(75.8856501,54.9901358)(75.8856501,53.75055274)
\closepath
}
}
{
\newrgbcolor{curcolor}{0 0 0}
\pscustom[linestyle=none,fillstyle=solid,fillcolor=curcolor,opacity=0]
{
\newpath
\moveto(240.87746547,122.58644937)
\curveto(222.03498394,122.58644937)(212.61374318,111.39749565)(203.19250242,100.20854192)
\curveto(193.77126165,89.0195882)(184.35002089,77.83063448)(165.50753936,77.83063448)
}
}
{
\newrgbcolor{curcolor}{0.49803922 0.49803922 0.49803922}
\pscustom[linewidth=2.99999393,linecolor=curcolor]
{
\newpath
\moveto(240.87746547,122.58644937)
\curveto(222.03498394,122.58644937)(212.61374318,111.39749565)(203.19250242,100.20854192)
\curveto(193.77126165,89.0195882)(184.35002089,77.83063448)(165.50753936,77.83063448)
}
}
{
\newrgbcolor{curcolor}{0 0 0}
\pscustom[linestyle=none,fillstyle=solid,fillcolor=curcolor,opacity=0]
{
\newpath
\moveto(239.9325855,18.74413996)
\curveto(221.32632396,18.74413996)(212.02319319,33.51576359)(202.72006243,48.28738722)
\curveto(193.41693166,63.05901085)(184.11380089,77.83063448)(165.50753936,77.83063448)
}
}
{
\newrgbcolor{curcolor}{0.49803922 0.49803922 0.49803922}
\pscustom[linewidth=2.99999393,linecolor=curcolor]
{
\newpath
\moveto(239.9325855,18.74413996)
\curveto(221.32632396,18.74413996)(212.02319319,33.51576359)(202.72006243,48.28738722)
\curveto(193.41693166,63.05901085)(184.11380089,77.83063448)(165.50753936,77.83063448)
}
}
{
\newrgbcolor{curcolor}{0.12156863 0.28627452 0.49019608}
\pscustom[linestyle=none,fillstyle=solid,fillcolor=curcolor]
{
\newpath
\moveto(240.87760983,553.85320072)
\lineto(689.69561173,553.85320072)
\lineto(689.69561173,517.06587364)
\lineto(240.87760983,517.06587364)
\closepath
}
}
{
\newrgbcolor{curcolor}{0.10196079 0.8392157 0.96078432}
\pscustom[linewidth=1.99999595,linecolor=curcolor]
{
\newpath
\moveto(240.87760983,553.85320072)
\lineto(689.69561173,553.85320072)
\lineto(689.69561173,517.06587364)
\lineto(240.87760983,517.06587364)
\closepath
}
}
{
\newrgbcolor{curcolor}{0.12156863 0.28627452 0.49019608}
\pscustom[linestyle=none,fillstyle=solid,fillcolor=curcolor]
{
\newpath
\moveto(240.87760983,519.44783995)
\lineto(689.69561173,519.44783995)
\lineto(689.69561173,482.66049712)
\lineto(240.87760983,482.66049712)
\closepath
}
}
{
\newrgbcolor{curcolor}{0.10196079 0.8392157 0.96078432}
\pscustom[linewidth=1.99999595,linecolor=curcolor]
{
\newpath
\moveto(240.87760983,519.44783995)
\lineto(689.69561173,519.44783995)
\lineto(689.69561173,482.66049712)
\lineto(240.87760983,482.66049712)
\closepath
}
}
{
\newrgbcolor{curcolor}{0.12156863 0.28627452 0.49019608}
\pscustom[linestyle=none,fillstyle=solid,fillcolor=curcolor]
{
\newpath
\moveto(240.87760983,485.0424083)
\lineto(689.69561173,485.0424083)
\lineto(689.69561173,448.25508122)
\lineto(240.87760983,448.25508122)
\closepath
}
}
{
\newrgbcolor{curcolor}{0.10196079 0.8392157 0.96078432}
\pscustom[linewidth=1.99999595,linecolor=curcolor]
{
\newpath
\moveto(240.87760983,485.0424083)
\lineto(689.69561173,485.0424083)
\lineto(689.69561173,448.25508122)
\lineto(240.87760983,448.25508122)
\closepath
}
}
{
\newrgbcolor{curcolor}{0.12156863 0.28627452 0.49019608}
\pscustom[linestyle=none,fillstyle=solid,fillcolor=curcolor]
{
\newpath
\moveto(240.87760983,450.6370449)
\lineto(689.69561173,450.6370449)
\lineto(689.69561173,413.84971782)
\lineto(240.87760983,413.84971782)
\closepath
}
}
{
\newrgbcolor{curcolor}{0.10196079 0.8392157 0.96078432}
\pscustom[linewidth=1.99999595,linecolor=curcolor]
{
\newpath
\moveto(240.87760983,450.6370449)
\lineto(689.69561173,450.6370449)
\lineto(689.69561173,413.84971782)
\lineto(240.87760983,413.84971782)
\closepath
}
}
{
\newrgbcolor{curcolor}{0.12156863 0.28627452 0.49019608}
\pscustom[linestyle=none,fillstyle=solid,fillcolor=curcolor]
{
\newpath
\moveto(240.87760983,416.2316815)
\lineto(689.69561173,416.2316815)
\lineto(689.69561173,379.44435442)
\lineto(240.87760983,379.44435442)
\closepath
}
}
{
\newrgbcolor{curcolor}{0.10196079 0.8392157 0.96078432}
\pscustom[linewidth=1.99999595,linecolor=curcolor]
{
\newpath
\moveto(240.87760983,416.2316815)
\lineto(689.69561173,416.2316815)
\lineto(689.69561173,379.44435442)
\lineto(240.87760983,379.44435442)
\closepath
}
}
{
\newrgbcolor{curcolor}{0.12156863 0.28627452 0.49019608}
\pscustom[linestyle=none,fillstyle=solid,fillcolor=curcolor]
{
\newpath
\moveto(240.87760983,381.8263181)
\lineto(689.69561173,381.8263181)
\lineto(689.69561173,345.03899102)
\lineto(240.87760983,345.03899102)
\closepath
}
}
{
\newrgbcolor{curcolor}{0.10196079 0.8392157 0.96078432}
\pscustom[linewidth=1.99999595,linecolor=curcolor]
{
\newpath
\moveto(240.87760983,381.8263181)
\lineto(689.69561173,381.8263181)
\lineto(689.69561173,345.03899102)
\lineto(240.87760983,345.03899102)
\closepath
}
}
{
\newrgbcolor{curcolor}{0.92941177 0.60784316 0.03529412}
\pscustom[linestyle=none,fillstyle=solid,fillcolor=curcolor]
{
\newpath
\moveto(240.87760983,347.42090221)
\lineto(689.69561173,347.42090221)
\lineto(689.69561173,310.63357513)
\lineto(240.87760983,310.63357513)
\closepath
}
}
{
\newrgbcolor{curcolor}{0.10196079 0.8392157 0.96078432}
\pscustom[linewidth=1.99999595,linecolor=curcolor]
{
\newpath
\moveto(240.87760983,347.42090221)
\lineto(689.69561173,347.42090221)
\lineto(689.69561173,310.63357513)
\lineto(240.87760983,310.63357513)
\closepath
}
}
{
\newrgbcolor{curcolor}{0.53333336 0.53333336 0.53333336}
\pscustom[linestyle=none,fillstyle=solid,fillcolor=curcolor]
{
\newpath
\moveto(240.87746547,313.01737608)
\lineto(689.69545425,313.01737608)
\lineto(689.69545425,244.1986178)
\lineto(240.87746547,244.1986178)
\closepath
}
}
{
\newrgbcolor{curcolor}{0.10196079 0.8392157 0.96078432}
\pscustom[linewidth=1.99999595,linecolor=curcolor]
{
\newpath
\moveto(240.87746547,313.01737608)
\lineto(689.69545425,313.01737608)
\lineto(689.69545425,244.1986178)
\lineto(240.87746547,244.1986178)
\closepath
}
}
{
\newrgbcolor{curcolor}{1 1 1}
\pscustom[linestyle=none,fillstyle=solid,fillcolor=curcolor]
{
\newpath
\moveto(446.70445306,276.56833455)
\curveto(446.70445306,275.95375246)(446.63674487,275.40167024)(446.50132847,274.9120879)
\curveto(446.3711204,274.42250556)(446.1758083,274.00583974)(445.91539216,273.66209043)
\curveto(445.66018434,273.32354945)(445.34247665,273.06052915)(444.96226909,272.87302953)
\curveto(444.58206153,272.69073823)(444.14716657,272.59959258)(443.65758423,272.59959258)
\curveto(443.44925132,272.59959258)(443.25654338,272.62042588)(443.0794604,272.66209246)
\curveto(442.90237743,272.70375904)(442.72789862,272.76886307)(442.55602396,272.85740456)
\curveto(442.38935763,272.94594605)(442.22269131,273.05792499)(442.05602498,273.19334138)
\curveto(441.88935865,273.32875777)(441.71227567,273.48761162)(441.52477605,273.66990292)
\lineto(441.52477605,270.15428504)
\curveto(441.52477605,270.11261845)(441.51435941,270.07616019)(441.49352612,270.04491026)
\curveto(441.47269282,270.01366032)(441.43883873,269.98761871)(441.39196382,269.96678542)
\curveto(441.34508892,269.94595212)(441.27998488,269.93032716)(441.19665172,269.91991051)
\curveto(441.11331855,269.90949387)(441.00654793,269.90428554)(440.87633986,269.90428554)
\curveto(440.75134012,269.90428554)(440.64717366,269.90949387)(440.5638405,269.91991051)
\curveto(440.48050733,269.93032716)(440.41279914,269.94595212)(440.36071591,269.96678542)
\curveto(440.313841,269.98761871)(440.27998691,270.01366032)(440.25915361,270.04491026)
\curveto(440.24352865,270.07616019)(440.23571616,270.11261845)(440.23571616,270.15428504)
\lineto(440.23571616,279.97457765)
\curveto(440.23571616,280.02145256)(440.24352865,280.05791081)(440.25915361,280.08395243)
\curveto(440.27477858,280.11520237)(440.30602852,280.14124398)(440.35290342,280.16207727)
\curveto(440.39977833,280.18291056)(440.45967404,280.19593137)(440.53259056,280.20113969)
\curveto(440.60550708,280.21155634)(440.69404857,280.21676466)(440.79821502,280.21676466)
\curveto(440.9075898,280.21676466)(440.99613129,280.21155634)(441.06383949,280.20113969)
\curveto(441.136756,280.19593137)(441.19665172,280.18291056)(441.24352662,280.16207727)
\curveto(441.29040153,280.14124398)(441.32165146,280.11520237)(441.33727643,280.08395243)
\curveto(441.35810972,280.05791081)(441.36852637,280.02145256)(441.36852637,279.97457765)
\lineto(441.36852637,279.02926706)
\curveto(441.5820676,279.24801662)(441.78779635,279.4381204)(441.98571262,279.59957841)
\curveto(442.18362888,279.76103642)(442.38154515,279.89384865)(442.57946142,279.9980151)
\curveto(442.782586,280.10738988)(442.98831475,280.18811888)(443.19664767,280.24020211)
\curveto(443.4101889,280.29749366)(443.63414678,280.32613944)(443.86852131,280.32613944)
\curveto(444.37893694,280.32613944)(444.81383189,280.22718131)(445.17320616,280.02926504)
\curveto(445.53258044,279.83134877)(445.82424651,279.56051599)(446.04820439,279.21676668)
\curveto(446.27737059,278.87301738)(446.44403692,278.47197653)(446.54820338,278.01364412)
\curveto(446.65236984,277.56052004)(446.70445306,277.07875018)(446.70445306,276.56833455)
\closepath
\moveto(445.34508082,276.41989735)
\curveto(445.34508082,276.77927162)(445.31643504,277.12562509)(445.25914349,277.45895774)
\curveto(445.20706026,277.79749873)(445.11331045,278.09697729)(444.97789406,278.35739342)
\curveto(444.84768599,278.61780956)(444.67060301,278.82614248)(444.44664513,278.98239216)
\curveto(444.22268726,279.13864184)(443.94404199,279.21676668)(443.61070933,279.21676668)
\curveto(443.444043,279.21676668)(443.27998083,279.19072507)(443.11852282,279.13864184)
\curveto(442.95706482,279.09176694)(442.79300265,279.0136421)(442.62633632,278.90426732)
\curveto(442.45966999,278.80010086)(442.28519118,278.65947615)(442.10289988,278.48239317)
\curveto(441.92060858,278.31051852)(441.72790064,278.09697729)(441.52477605,277.84176947)
\lineto(441.52477605,275.04490013)
\curveto(441.878942,274.61260934)(442.21487882,274.28188084)(442.53258651,274.05271464)
\curveto(442.8502942,273.82354844)(443.18362686,273.70896534)(443.53258449,273.70896534)
\curveto(443.8555005,273.70896534)(444.13154161,273.78709018)(444.36070781,273.94333986)
\curveto(444.59508233,274.09958955)(444.78258195,274.30792246)(444.92320667,274.5683386)
\curveto(445.06903971,274.82875474)(445.17581033,275.12042081)(445.24351852,275.44333683)
\curveto(445.31122672,275.76625284)(445.34508082,276.09177301)(445.34508082,276.41989735)
\closepath
}
}
{
\newrgbcolor{curcolor}{1 1 1}
\pscustom[linestyle=none,fillstyle=solid,fillcolor=curcolor]
{
\newpath
\moveto(453.93256304,272.9355294)
\curveto(453.93256304,272.87302953)(453.91172975,272.82615463)(453.87006317,272.79490469)
\curveto(453.82839659,272.76365475)(453.77110504,272.7402173)(453.69818852,272.72459233)
\curveto(453.625272,272.70896736)(453.51850138,272.70115488)(453.37787667,272.70115488)
\curveto(453.24246028,272.70115488)(453.1330855,272.70896736)(453.04975233,272.72459233)
\curveto(452.97162749,272.7402173)(452.91433594,272.76365475)(452.87787768,272.79490469)
\curveto(452.84141942,272.82615463)(452.82319029,272.87302953)(452.82319029,272.9355294)
\lineto(452.82319029,273.63865298)
\curveto(452.51589925,273.31052864)(452.17214994,273.05532083)(451.79194238,272.87302953)
\curveto(451.41694314,272.69073823)(451.01850645,272.59959258)(450.5966323,272.59959258)
\curveto(450.22684138,272.59959258)(449.89090456,272.64907165)(449.58882184,272.74802978)
\curveto(449.29194744,272.84177959)(449.03673963,272.97980015)(448.82319839,273.16209145)
\curveto(448.61486548,273.34438274)(448.45080331,273.56834062)(448.33101189,273.83396508)
\curveto(448.21642879,274.09958955)(448.15913724,274.40167227)(448.15913724,274.74021325)
\curveto(448.15913724,275.13604578)(448.23986624,275.47979509)(448.40132425,275.77146116)
\curveto(448.56278225,276.06312724)(448.79455262,276.30531425)(449.09663534,276.49802219)
\curveto(449.39871806,276.69073013)(449.76850898,276.83395901)(450.20600809,276.92770882)
\curveto(450.6435072,277.02666695)(451.13569371,277.07614602)(451.6825676,277.07614602)
\lineto(452.65131564,277.07614602)
\lineto(452.65131564,277.62301991)
\curveto(452.65131564,277.8938527)(452.62266986,278.13343555)(452.56537831,278.34176846)
\curveto(452.50808676,278.55010137)(452.41433695,278.72197602)(452.28412888,278.85739241)
\curveto(452.15912914,278.99801713)(451.99506697,279.10218358)(451.79194238,279.16989178)
\curveto(451.58881779,279.2428083)(451.3388183,279.27926656)(451.0419439,279.27926656)
\curveto(450.72423621,279.27926656)(450.43777845,279.24020414)(450.18257064,279.1620793)
\curveto(449.93257114,279.08916278)(449.71121743,279.00582961)(449.51850948,278.9120798)
\curveto(449.33100986,278.82353831)(449.17215602,278.74020515)(449.04194795,278.66208031)
\curveto(448.9169482,278.58916379)(448.82319839,278.55270553)(448.76069852,278.55270553)
\curveto(448.71903194,278.55270553)(448.68257368,278.56312217)(448.65132374,278.58395547)
\curveto(448.6200738,278.60478876)(448.59142803,278.63603869)(448.56538641,278.67770528)
\curveto(448.54455312,278.71937186)(448.52892815,278.77145509)(448.51851151,278.83395496)
\curveto(448.50809486,278.90166316)(448.50288654,278.97457968)(448.50288654,279.05270452)
\curveto(448.50288654,279.18291259)(448.51069902,279.28447488)(448.52632399,279.3573914)
\curveto(448.54715728,279.43551624)(448.59142803,279.50843276)(448.65913622,279.57614096)
\curveto(448.73205274,279.64384915)(448.85444833,279.72197399)(449.02632298,279.81051548)
\curveto(449.19819763,279.90426529)(449.3961139,279.98759846)(449.62007178,280.06051498)
\curveto(449.84402966,280.13863982)(450.08882083,280.20113969)(450.35444529,280.2480146)
\curveto(450.62006975,280.30009782)(450.88829838,280.32613944)(451.15913116,280.32613944)
\curveto(451.66433847,280.32613944)(452.0940251,280.26884789)(452.44819105,280.15426479)
\curveto(452.802357,280.03968168)(453.08881475,279.87041119)(453.30756431,279.64645331)
\curveto(453.52631387,279.42770376)(453.68516771,279.15426681)(453.78412585,278.82614248)
\curveto(453.88308398,278.49801814)(453.93256304,278.11520642)(453.93256304,277.6777073)
\closepath
\moveto(452.65131564,276.1464604)
\lineto(451.54975537,276.1464604)
\curveto(451.19558942,276.1464604)(450.88829838,276.11521047)(450.62788224,276.05271059)
\curveto(450.3674661,275.99541904)(450.1513207,275.90687755)(449.97944605,275.78708613)
\curveto(449.8075714,275.67250303)(449.67996749,275.53187831)(449.59663432,275.36521198)
\curveto(449.51850948,275.20375398)(449.47944706,275.01625436)(449.47944706,274.80271312)
\curveto(449.47944706,274.43813053)(449.59403016,274.14646445)(449.82319637,273.92771489)
\curveto(450.05757089,273.71417366)(450.38309107,273.60740304)(450.79975689,273.60740304)
\curveto(451.13829787,273.60740304)(451.45079724,273.69334037)(451.73725499,273.86521502)
\curveto(452.02892107,274.03708967)(452.33360795,274.30010997)(452.65131564,274.65427592)
\closepath
}
}
{
\newrgbcolor{curcolor}{1 1 1}
\pscustom[linestyle=none,fillstyle=solid,fillcolor=curcolor]
{
\newpath
\moveto(462.32310811,272.94334189)
\curveto(462.32310811,272.90167531)(462.31269147,272.86521705)(462.29185817,272.83396711)
\curveto(462.27623321,272.80271717)(462.24498327,272.77667556)(462.19810836,272.75584227)
\curveto(462.15644178,272.7402173)(462.09915023,272.72719649)(462.02623371,272.71677985)
\curveto(461.95331719,272.7063632)(461.86477571,272.70115488)(461.76060925,272.70115488)
\curveto(461.65123447,272.70115488)(461.56008882,272.7063632)(461.4871723,272.71677985)
\curveto(461.41425578,272.72719649)(461.35436007,272.7402173)(461.30748517,272.75584227)
\curveto(461.26061026,272.77667556)(461.22675616,272.80271717)(461.20592287,272.83396711)
\curveto(461.18508958,272.86521705)(461.17467294,272.90167531)(461.17467294,272.94334189)
\lineto(461.17467294,273.87302751)
\curveto(460.80488202,273.47198665)(460.41946613,273.15948728)(460.01842528,272.9355294)
\curveto(459.62259275,272.71157152)(459.18769779,272.59959258)(458.71374042,272.59959258)
\curveto(458.19811646,272.59959258)(457.75801319,272.69855072)(457.39343059,272.89646698)
\curveto(457.028848,273.09959157)(456.7319736,273.37042436)(456.5028074,273.70896534)
\curveto(456.27884952,274.05271464)(456.11478735,274.4537555)(456.01062089,274.9120879)
\curveto(455.90645444,275.37562863)(455.85437121,275.86260681)(455.85437121,276.37302244)
\curveto(455.85437121,276.97718789)(455.91947525,277.52145762)(456.04968331,278.00583164)
\curveto(456.17989138,278.49541398)(456.37259933,278.9120798)(456.62780714,279.25582911)
\curveto(456.88301496,279.59957841)(457.19811849,279.86259871)(457.57311773,280.04489001)
\curveto(457.95332529,280.23238963)(458.39082441,280.32613944)(458.88561507,280.32613944)
\curveto(459.29707257,280.32613944)(459.67207181,280.23499379)(460.01061279,280.05270249)
\curveto(460.3543621,279.87561952)(460.69290308,279.61259922)(461.02623574,279.26364159)
\lineto(461.02623574,283.34957082)
\curveto(461.02623574,283.38602908)(461.03404822,283.41988317)(461.04967319,283.45113311)
\curveto(461.07050648,283.48759137)(461.10696474,283.51363298)(461.15904797,283.52925795)
\curveto(461.2111312,283.55009124)(461.27623523,283.56571621)(461.35436007,283.57613286)
\curveto(461.43769324,283.59175783)(461.54185969,283.59957031)(461.66685944,283.59957031)
\curveto(461.79706751,283.59957031)(461.90383813,283.59175783)(461.98717129,283.57613286)
\curveto(462.07050446,283.56571621)(462.13560849,283.55009124)(462.1824834,283.52925795)
\curveto(462.2293583,283.51363298)(462.2632124,283.48759137)(462.28404569,283.45113311)
\curveto(462.3100873,283.41988317)(462.32310811,283.38602908)(462.32310811,283.34957082)
\closepath
\moveto(461.02623574,277.88864437)
\curveto(460.67727811,278.32093517)(460.33873713,278.6490595)(460.01061279,278.87301738)
\curveto(459.68769678,279.10218358)(459.3491558,279.21676668)(458.99498985,279.21676668)
\curveto(458.66686552,279.21676668)(458.38822025,279.13864184)(458.15905404,278.98239216)
\curveto(457.92988784,278.82614248)(457.74238822,278.62041373)(457.59655518,278.36520591)
\curveto(457.45593047,278.10999809)(457.35176401,277.82093618)(457.28405582,277.49802017)
\curveto(457.22155594,277.17510415)(457.19030601,276.84697982)(457.19030601,276.51364716)
\curveto(457.19030601,276.15948121)(457.21634762,275.81312774)(457.26843085,275.47458676)
\curveto(457.3257224,275.13604578)(457.42207637,274.83396306)(457.55749276,274.5683386)
\curveto(457.69290915,274.30792246)(457.87259629,274.09698539)(458.09655417,273.93552738)
\curveto(458.32051205,273.7792777)(458.60176148,273.70115285)(458.94030246,273.70115285)
\curveto(459.11217711,273.70115285)(459.27623928,273.72459031)(459.43248896,273.77146521)
\curveto(459.59394697,273.81834012)(459.75800914,273.89646496)(459.92467547,274.00583974)
\curveto(460.0913418,274.11521452)(460.26582061,274.25583923)(460.44811191,274.42771388)
\curveto(460.63040321,274.60479686)(460.82311115,274.82094225)(461.02623574,275.07615007)
\closepath
}
}
{
\newrgbcolor{curcolor}{1 1 1}
\pscustom[linestyle=none,fillstyle=solid,fillcolor=curcolor]
{
\newpath
\moveto(470.72309263,272.94334189)
\curveto(470.72309263,272.90167531)(470.71267598,272.86521705)(470.69184269,272.83396711)
\curveto(470.67621772,272.80271717)(470.64496779,272.77667556)(470.59809288,272.75584227)
\curveto(470.5564263,272.7402173)(470.49913475,272.72719649)(470.42621823,272.71677985)
\curveto(470.35330171,272.7063632)(470.26476022,272.70115488)(470.16059377,272.70115488)
\curveto(470.05121899,272.70115488)(469.96007334,272.7063632)(469.88715682,272.71677985)
\curveto(469.8142403,272.72719649)(469.75434459,272.7402173)(469.70746968,272.75584227)
\curveto(469.66059478,272.77667556)(469.62674068,272.80271717)(469.60590739,272.83396711)
\curveto(469.5850741,272.86521705)(469.57465745,272.90167531)(469.57465745,272.94334189)
\lineto(469.57465745,273.87302751)
\curveto(469.20486653,273.47198665)(468.81945065,273.15948728)(468.41840979,272.9355294)
\curveto(468.02257726,272.71157152)(467.58768231,272.59959258)(467.11372494,272.59959258)
\curveto(466.59810098,272.59959258)(466.1579977,272.69855072)(465.79341511,272.89646698)
\curveto(465.42883251,273.09959157)(465.13195812,273.37042436)(464.90279191,273.70896534)
\curveto(464.67883403,274.05271464)(464.51477187,274.4537555)(464.41060541,274.9120879)
\curveto(464.30643895,275.37562863)(464.25435573,275.86260681)(464.25435573,276.37302244)
\curveto(464.25435573,276.97718789)(464.31945976,277.52145762)(464.44966783,278.00583164)
\curveto(464.5798759,278.49541398)(464.77258384,278.9120798)(465.02779166,279.25582911)
\curveto(465.28299948,279.59957841)(465.59810301,279.86259871)(465.97310225,280.04489001)
\curveto(466.35330981,280.23238963)(466.79080892,280.32613944)(467.28559959,280.32613944)
\curveto(467.69705709,280.32613944)(468.07205633,280.23499379)(468.41059731,280.05270249)
\curveto(468.75434661,279.87561952)(469.09288759,279.61259922)(469.42622025,279.26364159)
\lineto(469.42622025,283.34957082)
\curveto(469.42622025,283.38602908)(469.43403274,283.41988317)(469.44965771,283.45113311)
\curveto(469.470491,283.48759137)(469.50694926,283.51363298)(469.55903248,283.52925795)
\curveto(469.61111571,283.55009124)(469.67621975,283.56571621)(469.75434459,283.57613286)
\curveto(469.83767775,283.59175783)(469.94184421,283.59957031)(470.06684396,283.59957031)
\curveto(470.19705203,283.59957031)(470.30382264,283.59175783)(470.38715581,283.57613286)
\curveto(470.47048897,283.56571621)(470.53559301,283.55009124)(470.58246791,283.52925795)
\curveto(470.62934282,283.51363298)(470.66319691,283.48759137)(470.68403021,283.45113311)
\curveto(470.71007182,283.41988317)(470.72309263,283.38602908)(470.72309263,283.34957082)
\closepath
\moveto(469.42622025,277.88864437)
\curveto(469.07726263,278.32093517)(468.73872165,278.6490595)(468.41059731,278.87301738)
\curveto(468.0876813,279.10218358)(467.74914032,279.21676668)(467.39497437,279.21676668)
\curveto(467.06685003,279.21676668)(466.78820476,279.13864184)(466.55903856,278.98239216)
\curveto(466.32987236,278.82614248)(466.14237274,278.62041373)(465.9965397,278.36520591)
\curveto(465.85591498,278.10999809)(465.75174853,277.82093618)(465.68404033,277.49802017)
\curveto(465.62154046,277.17510415)(465.59029052,276.84697982)(465.59029052,276.51364716)
\curveto(465.59029052,276.15948121)(465.61633213,275.81312774)(465.66841536,275.47458676)
\curveto(465.72570691,275.13604578)(465.82206088,274.83396306)(465.95747728,274.5683386)
\curveto(466.09289367,274.30792246)(466.27258081,274.09698539)(466.49653869,273.93552738)
\curveto(466.72049657,273.7792777)(467.001746,273.70115285)(467.34028698,273.70115285)
\curveto(467.51216163,273.70115285)(467.6762238,273.72459031)(467.83247348,273.77146521)
\curveto(467.99393149,273.81834012)(468.15799365,273.89646496)(468.32465998,274.00583974)
\curveto(468.49132631,274.11521452)(468.66580513,274.25583923)(468.84809642,274.42771388)
\curveto(469.03038772,274.60479686)(469.22309566,274.82094225)(469.42622025,275.07615007)
\closepath
}
}
{
\newrgbcolor{curcolor}{1 1 1}
\pscustom[linestyle=none,fillstyle=solid,fillcolor=curcolor]
{
\newpath
\moveto(474.38870792,272.94334189)
\curveto(474.38870792,272.90167531)(474.37829127,272.86521705)(474.35745798,272.83396711)
\curveto(474.33662469,272.8079255)(474.30277059,272.78448804)(474.25589568,272.76365475)
\curveto(474.20902078,272.74282146)(474.14391674,272.72719649)(474.06058358,272.71677985)
\curveto(473.97725042,272.7063632)(473.8704798,272.70115488)(473.74027173,272.70115488)
\curveto(473.61527198,272.70115488)(473.51110553,272.7063632)(473.42777236,272.71677985)
\curveto(473.3444392,272.72719649)(473.276731,272.74282146)(473.22464777,272.76365475)
\curveto(473.17777287,272.78448804)(473.14391877,272.8079255)(473.12308548,272.83396711)
\curveto(473.10746051,272.86521705)(473.09964803,272.90167531)(473.09964803,272.94334189)
\lineto(473.09964803,279.97457765)
\curveto(473.09964803,280.01103591)(473.10746051,280.04489001)(473.12308548,280.07613994)
\curveto(473.14391877,280.10738988)(473.17777287,280.1334315)(473.22464777,280.15426479)
\curveto(473.276731,280.17509808)(473.3444392,280.19072305)(473.42777236,280.20113969)
\curveto(473.51110553,280.21155634)(473.61527198,280.21676466)(473.74027173,280.21676466)
\curveto(473.8704798,280.21676466)(473.97725042,280.21155634)(474.06058358,280.20113969)
\curveto(474.14391674,280.19072305)(474.20902078,280.17509808)(474.25589568,280.15426479)
\curveto(474.30277059,280.1334315)(474.33662469,280.10738988)(474.35745798,280.07613994)
\curveto(474.37829127,280.04489001)(474.38870792,280.01103591)(474.38870792,279.97457765)
\closepath
\moveto(474.53714511,282.34957284)
\curveto(474.53714511,282.04749012)(474.47985356,281.84176137)(474.36527046,281.73238659)
\curveto(474.25068736,281.62301181)(474.03975029,281.56832442)(473.73245924,281.56832442)
\curveto(473.43037652,281.56832442)(473.22204361,281.62040765)(473.10746051,281.72457411)
\curveto(472.99808573,281.83394888)(472.94339834,282.03707347)(472.94339834,282.33394787)
\curveto(472.94339834,282.63603059)(473.00068989,282.84175934)(473.11527299,282.95113412)
\curveto(473.2298561,283.0605089)(473.44079317,283.11519629)(473.74808421,283.11519629)
\curveto(474.05016693,283.11519629)(474.25589568,283.0605089)(474.36527046,282.95113412)
\curveto(474.47985356,282.84696767)(474.53714511,282.64644724)(474.53714511,282.34957284)
\closepath
}
}
{
\newrgbcolor{curcolor}{1 1 1}
\pscustom[linestyle=none,fillstyle=solid,fillcolor=curcolor]
{
\newpath
\moveto(482.81831731,272.94334189)
\curveto(482.81831731,272.90167531)(482.80790066,272.86521705)(482.78706737,272.83396711)
\curveto(482.76623408,272.8079255)(482.73237998,272.78448804)(482.68550508,272.76365475)
\curveto(482.63863017,272.74282146)(482.57352614,272.72719649)(482.49019297,272.71677985)
\curveto(482.40685981,272.7063632)(482.30269335,272.70115488)(482.17769361,272.70115488)
\curveto(482.04748554,272.70115488)(481.94071492,272.7063632)(481.85738176,272.71677985)
\curveto(481.77404859,272.72719649)(481.70894456,272.74282146)(481.66206965,272.76365475)
\curveto(481.61519475,272.78448804)(481.58134065,272.8079255)(481.56050736,272.83396711)
\curveto(481.53967407,272.86521705)(481.52925742,272.90167531)(481.52925742,272.94334189)
\lineto(481.52925742,277.06052105)
\curveto(481.52925742,277.46156191)(481.49800748,277.78447792)(481.43550761,278.02926909)
\curveto(481.37300774,278.27406026)(481.28186209,278.48499733)(481.16207066,278.66208031)
\curveto(481.04227924,278.83916328)(480.88602956,278.97457968)(480.69332161,279.06832949)
\curveto(480.50582199,279.1620793)(480.28707244,279.2089542)(480.03707294,279.2089542)
\curveto(479.71415693,279.2089542)(479.39124092,279.0943711)(479.0683249,278.8652049)
\curveto(478.74540889,278.63603869)(478.40686791,278.30010187)(478.05270196,277.85739444)
\lineto(478.05270196,272.94334189)
\curveto(478.05270196,272.90167531)(478.04228531,272.86521705)(478.02145202,272.83396711)
\curveto(478.00061873,272.8079255)(477.96676463,272.78448804)(477.91988973,272.76365475)
\curveto(477.87301482,272.74282146)(477.80791079,272.72719649)(477.72457762,272.71677985)
\curveto(477.64124446,272.7063632)(477.53447384,272.70115488)(477.40426577,272.70115488)
\curveto(477.27926603,272.70115488)(477.17509957,272.7063632)(477.09176641,272.71677985)
\curveto(477.00843324,272.72719649)(476.94072505,272.74282146)(476.88864182,272.76365475)
\curveto(476.84176691,272.78448804)(476.80791281,272.8079255)(476.78707952,272.83396711)
\curveto(476.77145456,272.86521705)(476.76364207,272.90167531)(476.76364207,272.94334189)
\lineto(476.76364207,279.97457765)
\curveto(476.76364207,280.01624423)(476.77145456,280.05009833)(476.78707952,280.07613994)
\curveto(476.80270449,280.10738988)(476.83395443,280.1334315)(476.88082933,280.15426479)
\curveto(476.92770424,280.1803064)(476.98759995,280.19593137)(477.06051647,280.20113969)
\curveto(477.13343299,280.21155634)(477.22978696,280.21676466)(477.34957838,280.21676466)
\curveto(477.46416149,280.21676466)(477.5579113,280.21155634)(477.63082781,280.20113969)
\curveto(477.70895266,280.19593137)(477.76884837,280.1803064)(477.81051495,280.15426479)
\curveto(477.85218153,280.1334315)(477.88082731,280.10738988)(477.89645228,280.07613994)
\curveto(477.91728557,280.05009833)(477.92770221,280.01624423)(477.92770221,279.97457765)
\lineto(477.92770221,279.04489203)
\curveto(478.32353475,279.48759947)(478.71676312,279.81051548)(479.10738732,280.01364007)
\curveto(479.50321986,280.22197298)(479.90165655,280.32613944)(480.3026974,280.32613944)
\curveto(480.77144646,280.32613944)(481.16467483,280.24541043)(481.48238252,280.08395243)
\curveto(481.80529853,279.92770274)(482.06571467,279.71676567)(482.26363093,279.45114121)
\curveto(482.4615472,279.18551675)(482.60217191,278.87301738)(482.68550508,278.51364311)
\curveto(482.77404657,278.15947716)(482.81831731,277.73239469)(482.81831731,277.2323957)
\closepath
}
}
{
\newrgbcolor{curcolor}{1 1 1}
\pscustom[linestyle=none,fillstyle=solid,fillcolor=curcolor]
{
\newpath
\moveto(491.03861469,279.6542658)
\curveto(491.03861469,279.4719745)(491.01257308,279.33916227)(490.96048985,279.25582911)
\curveto(490.91361494,279.17770426)(490.85111507,279.13864184)(490.77299023,279.13864184)
\lineto(489.76517977,279.13864184)
\curveto(489.94747107,278.95114222)(490.07507497,278.74280931)(490.14799149,278.51364311)
\curveto(490.22090801,278.28968523)(490.25736627,278.0553107)(490.25736627,277.81051953)
\curveto(490.25736627,277.40427035)(490.19226224,277.04489608)(490.06205417,276.73239672)
\curveto(489.9318461,276.41989735)(489.74434648,276.15427289)(489.49955531,275.93552333)
\curveto(489.25997246,275.72198209)(488.97351471,275.55791993)(488.64018205,275.44333683)
\curveto(488.30684939,275.32875372)(487.93705847,275.27146217)(487.53080929,275.27146217)
\curveto(487.24435154,275.27146217)(486.97091459,275.30792043)(486.71049845,275.38083695)
\curveto(486.45529064,275.45896179)(486.25737437,275.55531577)(486.11674966,275.66989887)
\curveto(486.02299985,275.57614906)(485.944875,275.46937844)(485.88237513,275.34958702)
\curveto(485.82508358,275.22979559)(485.79643781,275.09177504)(485.79643781,274.93552535)
\curveto(485.79643781,274.75323406)(485.87977097,274.6021927)(486.0464373,274.48240127)
\curveto(486.21831195,274.36260985)(486.44487399,274.29750581)(486.72612342,274.28708917)
\lineto(488.5620572,274.20896433)
\curveto(488.91101483,274.19854768)(489.23132668,274.14906861)(489.52299276,274.06052713)
\curveto(489.81465884,273.97719396)(490.06726249,273.85479838)(490.28080372,273.69334037)
\curveto(490.49434496,273.53709069)(490.66101129,273.34177858)(490.78080271,273.10740406)
\curveto(490.90059414,272.87823785)(490.96048985,272.61000923)(490.96048985,272.30271819)
\curveto(490.96048985,271.97980217)(490.89278165,271.67251113)(490.75736526,271.38084505)
\curveto(490.62194887,271.08917898)(490.41361596,270.83397116)(490.13236653,270.6152216)
\curveto(489.85632542,270.39126372)(489.50215947,270.21678491)(489.06986868,270.09178516)
\curveto(488.63757789,269.96157709)(488.12716225,269.89647306)(487.53862178,269.89647306)
\curveto(486.97091459,269.89647306)(486.48654057,269.94595212)(486.08549972,270.04491026)
\curveto(485.68966719,270.13866007)(485.36414701,270.26886814)(485.1089392,270.43553447)
\curveto(484.85373138,270.6022008)(484.66883592,270.80272122)(484.55425282,271.03709575)
\curveto(484.43966972,271.26626195)(484.38237817,271.51626144)(484.38237817,271.78709423)
\curveto(484.38237817,271.95896888)(484.40321146,272.12563521)(484.44487804,272.28709322)
\curveto(484.48654462,272.44855122)(484.5490445,272.60219675)(484.63237766,272.74802978)
\curveto(484.72091915,272.89386282)(484.82768977,273.03188338)(484.95268951,273.16209145)
\curveto(485.08289758,273.29750784)(485.23133478,273.43032007)(485.39800111,273.56052814)
\curveto(485.1427933,273.69073621)(484.95268951,273.85479838)(484.82768977,274.05271464)
\curveto(484.70789834,274.25063091)(484.64800263,274.46417214)(484.64800263,274.69333834)
\curveto(484.64800263,275.01104603)(484.71310667,275.29489963)(484.84331474,275.54489912)
\curveto(484.97352281,275.79489861)(485.13498081,276.01885649)(485.32768875,276.21677276)
\curveto(485.16623075,276.4094807)(485.03862684,276.6256261)(484.94487703,276.86520895)
\curveto(484.85112722,277.11000012)(484.80425231,277.40427035)(484.80425231,277.74801966)
\curveto(484.80425231,278.14906051)(484.87196051,278.50843479)(485.0073769,278.82614248)
\curveto(485.1427933,279.14385017)(485.33029292,279.41207879)(485.56987576,279.63082835)
\curveto(485.80945861,279.8495779)(486.09591637,280.01624423)(486.42924902,280.13082733)
\curveto(486.76779,280.25061876)(487.13497676,280.31051447)(487.53080929,280.31051447)
\curveto(487.74435053,280.31051447)(487.94226679,280.29749366)(488.12455809,280.27145205)
\curveto(488.31205771,280.25061876)(488.48653652,280.21936882)(488.64799453,280.17770224)
\lineto(490.77299023,280.17770224)
\curveto(490.86153172,280.17770224)(490.92663575,280.1334315)(490.96830233,280.04489001)
\curveto(491.01517724,279.96155684)(491.03861469,279.83134877)(491.03861469,279.6542658)
\closepath
\moveto(489.02299377,277.80270705)
\curveto(489.02299377,278.28187274)(488.89018154,278.65426782)(488.62455708,278.91989229)
\curveto(488.36414094,279.19072507)(487.99174586,279.32614146)(487.50737184,279.32614146)
\curveto(487.25737235,279.32614146)(487.03862279,279.28447488)(486.85112317,279.20114172)
\curveto(486.66883187,279.11780855)(486.51518635,279.00322545)(486.3901866,278.85739241)
\curveto(486.27039518,278.71155937)(486.17924953,278.54228888)(486.11674966,278.34958094)
\curveto(486.05945811,278.16208132)(486.03081233,277.96416505)(486.03081233,277.75583214)
\curveto(486.03081233,277.29229141)(486.1610204,276.92770882)(486.42143654,276.66208436)
\curveto(486.687061,276.3964599)(487.05685192,276.26364766)(487.53080929,276.26364766)
\curveto(487.78601711,276.26364766)(488.00737083,276.30271009)(488.19487045,276.38083493)
\curveto(488.38237007,276.46416809)(488.53601559,276.57614703)(488.65580702,276.71677175)
\curveto(488.78080676,276.86260479)(488.87195241,277.02927111)(488.92924396,277.21677073)
\curveto(488.99174383,277.40427035)(489.02299377,277.59958246)(489.02299377,277.80270705)
\closepath
\moveto(489.67142996,272.23240583)
\curveto(489.67142996,272.53448855)(489.54643021,272.76625891)(489.29643072,272.92771692)
\curveto(489.05163955,273.09438325)(488.71830689,273.18292474)(488.29643274,273.19334138)
\lineto(486.47612393,273.25584126)
\curveto(486.3094576,273.12563319)(486.17143705,273.00063344)(486.06206227,272.88084201)
\curveto(485.95789581,272.76625891)(485.87456265,272.65688414)(485.81206277,272.55271768)
\curveto(485.7495629,272.4433429)(485.70529216,272.33657228)(485.67925054,272.23240583)
\curveto(485.65841725,272.12823937)(485.64800061,272.02146876)(485.64800061,271.91209398)
\curveto(485.64800061,271.573553)(485.81987526,271.31834518)(486.16362456,271.14647053)
\curveto(486.50737387,270.96938755)(486.98653956,270.88084606)(487.60112165,270.88084606)
\curveto(487.99174586,270.88084606)(488.31726603,270.91990849)(488.57768217,270.99803333)
\curveto(488.84330664,271.07094985)(489.05684787,271.16990798)(489.21830588,271.29490773)
\curveto(489.37976388,271.41990747)(489.49434698,271.56313635)(489.56205518,271.72459436)
\curveto(489.6349717,271.88605236)(489.67142996,272.05532285)(489.67142996,272.23240583)
\closepath
}
}
{
\newrgbcolor{curcolor}{0 0 0}
\pscustom[linestyle=none,fillstyle=solid,fillcolor=curcolor,opacity=0]
{
\newpath
\moveto(465.28661734,519.44783995)
\lineto(465.28661734,553.84147109)
}
}
{
\newrgbcolor{curcolor}{0.10196079 0.8392157 0.96078432}
\pscustom[linewidth=1.99999595,linecolor=curcolor]
{
\newpath
\moveto(465.28661734,519.44783995)
\lineto(465.28661734,553.84147109)
}
}
{
\newrgbcolor{curcolor}{0 0 0}
\pscustom[linestyle=none,fillstyle=solid,fillcolor=curcolor,opacity=0]
{
\newpath
\moveto(465.28661734,450.6370449)
\lineto(465.28661734,485.03067604)
}
}
{
\newrgbcolor{curcolor}{0.10196079 0.8392157 0.96078432}
\pscustom[linewidth=1.99999595,linecolor=curcolor]
{
\newpath
\moveto(465.28661734,450.6370449)
\lineto(465.28661734,485.03067604)
}
}
{
\newrgbcolor{curcolor}{0 0 0}
\pscustom[linestyle=none,fillstyle=solid,fillcolor=curcolor,opacity=0]
{
\newpath
\moveto(465.28661734,416.2316815)
\lineto(465.28661734,381.83805036)
}
}
{
\newrgbcolor{curcolor}{0.10196079 0.8392157 0.96078432}
\pscustom[linewidth=1.99999595,linecolor=curcolor]
{
\newpath
\moveto(465.28661734,416.2316815)
\lineto(465.28661734,381.83805036)
}
}
{
\newrgbcolor{curcolor}{0.81960785 0.12941177 0.14117648}
\pscustom[linestyle=none,fillstyle=solid,fillcolor=curcolor]
{
\newpath
\moveto(240.87760983,244.20475952)
\lineto(689.69561173,244.20475952)
\lineto(689.69561173,207.41743244)
\lineto(240.87760983,207.41743244)
\closepath
}
}
{
\newrgbcolor{curcolor}{0.10196079 0.8392157 0.96078432}
\pscustom[linewidth=1.99999595,linecolor=curcolor]
{
\newpath
\moveto(240.87760983,244.20475952)
\lineto(689.69561173,244.20475952)
\lineto(689.69561173,207.41743244)
\lineto(240.87760983,207.41743244)
\closepath
}
}
{
\newrgbcolor{curcolor}{1 1 1}
\pscustom[linestyle=none,fillstyle=solid,fillcolor=curcolor]
{
\newpath
\moveto(417.95859127,223.77138846)
\curveto(417.95859127,223.15680637)(417.89088308,222.60472416)(417.75546668,222.11514182)
\curveto(417.62525862,221.62555947)(417.42994651,221.20889365)(417.16953037,220.86514435)
\curveto(416.91432255,220.52660337)(416.59661486,220.26358307)(416.2164073,220.07608344)
\curveto(415.83619974,219.89379215)(415.40130479,219.8026465)(414.91172244,219.8026465)
\curveto(414.70338953,219.8026465)(414.51068159,219.82347979)(414.33359861,219.86514637)
\curveto(414.15651564,219.90681295)(413.98203683,219.97191699)(413.81016217,220.06045848)
\curveto(413.64349584,220.14899996)(413.47682952,220.2609789)(413.31016319,220.3963953)
\curveto(413.14349686,220.53181169)(412.96641388,220.69066553)(412.77891426,220.87295683)
\lineto(412.77891426,217.35733895)
\curveto(412.77891426,217.31567237)(412.76849762,217.27921411)(412.74766433,217.24796417)
\curveto(412.72683103,217.21671423)(412.69297694,217.19067262)(412.64610203,217.16983933)
\curveto(412.59922713,217.14900604)(412.53412309,217.13338107)(412.45078993,217.12296442)
\curveto(412.36745676,217.11254778)(412.26068615,217.10733946)(412.13047808,217.10733946)
\curveto(412.00547833,217.10733946)(411.90131187,217.11254778)(411.81797871,217.12296442)
\curveto(411.73464554,217.13338107)(411.66693735,217.14900604)(411.61485412,217.16983933)
\curveto(411.56797921,217.19067262)(411.53412512,217.21671423)(411.51329183,217.24796417)
\curveto(411.49766686,217.27921411)(411.48985437,217.31567237)(411.48985437,217.35733895)
\lineto(411.48985437,227.17763156)
\curveto(411.48985437,227.22450647)(411.49766686,227.26096473)(411.51329183,227.28700634)
\curveto(411.52891679,227.31825628)(411.56016673,227.34429789)(411.60704164,227.36513118)
\curveto(411.65391654,227.38596448)(411.71381225,227.39898528)(411.78672877,227.40419361)
\curveto(411.85964529,227.41461025)(411.94818678,227.41981857)(412.05235323,227.41981857)
\curveto(412.16172801,227.41981857)(412.2502695,227.41461025)(412.3179777,227.40419361)
\curveto(412.39089421,227.39898528)(412.45078993,227.38596448)(412.49766483,227.36513118)
\curveto(412.54453974,227.34429789)(412.57578967,227.31825628)(412.59141464,227.28700634)
\curveto(412.61224793,227.26096473)(412.62266458,227.22450647)(412.62266458,227.17763156)
\lineto(412.62266458,226.23232098)
\curveto(412.83620581,226.45107054)(413.04193456,226.64117432)(413.23985083,226.80263232)
\curveto(413.43776709,226.96409033)(413.63568336,227.09690256)(413.83359963,227.20106902)
\curveto(414.03672422,227.3104438)(414.24245297,227.3911728)(414.45078588,227.44325603)
\curveto(414.66432711,227.50054758)(414.88828499,227.52919335)(415.12265952,227.52919335)
\curveto(415.63307515,227.52919335)(416.0679701,227.43023522)(416.42734437,227.23231895)
\curveto(416.78671865,227.03440269)(417.07838472,226.7635699)(417.3023426,226.4198206)
\curveto(417.5315088,226.07607129)(417.69817513,225.67503044)(417.80234159,225.21669803)
\curveto(417.90650805,224.76357395)(417.95859127,224.28180409)(417.95859127,223.77138846)
\closepath
\moveto(416.59921903,223.62295126)
\curveto(416.59921903,223.98232553)(416.57057325,224.328679)(416.5132817,224.66201166)
\curveto(416.46119847,225.00055264)(416.36744866,225.3000312)(416.23203227,225.56044734)
\curveto(416.1018242,225.82086348)(415.92474123,226.02919639)(415.70078335,226.18544607)
\curveto(415.47682547,226.34169576)(415.1981802,226.4198206)(414.86484754,226.4198206)
\curveto(414.69818121,226.4198206)(414.53411904,226.39377898)(414.37266103,226.34169576)
\curveto(414.21120303,226.29482085)(414.04714086,226.21669601)(413.88047453,226.10732123)
\curveto(413.7138082,226.00315478)(413.53932939,225.86253006)(413.35703809,225.68544709)
\curveto(413.17474679,225.51357243)(412.98203885,225.3000312)(412.77891426,225.04482338)
\lineto(412.77891426,222.24795405)
\curveto(413.13308021,221.81566326)(413.46901703,221.48493476)(413.78672472,221.25576856)
\curveto(414.10443241,221.02660235)(414.43776507,220.91201925)(414.7867227,220.91201925)
\curveto(415.10963871,220.91201925)(415.38567982,220.99014409)(415.61484602,221.14639378)
\curveto(415.84922054,221.30264346)(416.03672017,221.51097637)(416.17734488,221.77139251)
\curveto(416.32317792,222.03180865)(416.42994854,222.32347473)(416.49765673,222.64639074)
\curveto(416.56536493,222.96930675)(416.59921903,223.29482693)(416.59921903,223.62295126)
\closepath
}
}
{
\newrgbcolor{curcolor}{1 1 1}
\pscustom[linestyle=none,fillstyle=solid,fillcolor=curcolor]
{
\newpath
\moveto(425.94451222,220.1463958)
\curveto(425.94451222,220.10472922)(425.93409557,220.06827096)(425.91326228,220.03702102)
\curveto(425.89242899,220.01097941)(425.85857489,219.98754196)(425.81169999,219.96670867)
\curveto(425.76482508,219.94587538)(425.69972105,219.93025041)(425.61638788,219.91983376)
\curveto(425.53305472,219.90941712)(425.42888826,219.90420879)(425.30388852,219.90420879)
\curveto(425.17368045,219.90420879)(425.06690983,219.90941712)(424.98357667,219.91983376)
\curveto(424.9002435,219.93025041)(424.83513947,219.94587538)(424.78826456,219.96670867)
\curveto(424.74138966,219.98754196)(424.70753556,220.01097941)(424.68670227,220.03702102)
\curveto(424.66586898,220.06827096)(424.65545233,220.10472922)(424.65545233,220.1463958)
\lineto(424.65545233,224.26357497)
\curveto(424.65545233,224.66461582)(424.62420239,224.98753183)(424.56170252,225.232323)
\curveto(424.49920265,225.47711417)(424.408057,225.68805125)(424.28826557,225.86513422)
\curveto(424.16847415,226.0422172)(424.01222447,226.17763359)(423.81951652,226.2713834)
\curveto(423.6320169,226.36513321)(423.41326735,226.41200811)(423.16326785,226.41200811)
\curveto(422.84035184,226.41200811)(422.51743583,226.29742501)(422.19451981,226.06825881)
\curveto(421.8716038,225.83909261)(421.53306282,225.50315579)(421.17889687,225.06044835)
\lineto(421.17889687,220.1463958)
\curveto(421.17889687,220.10472922)(421.16848023,220.06827096)(421.14764693,220.03702102)
\curveto(421.12681364,220.01097941)(421.09295954,219.98754196)(421.04608464,219.96670867)
\curveto(420.99920973,219.94587538)(420.9341057,219.93025041)(420.85077254,219.91983376)
\curveto(420.76743937,219.90941712)(420.66066875,219.90420879)(420.53046068,219.90420879)
\curveto(420.40546094,219.90420879)(420.30129448,219.90941712)(420.21796132,219.91983376)
\curveto(420.13462815,219.93025041)(420.06691996,219.94587538)(420.01483673,219.96670867)
\curveto(419.96796182,219.98754196)(419.93410772,220.01097941)(419.91327443,220.03702102)
\curveto(419.89764947,220.06827096)(419.88983698,220.10472922)(419.88983698,220.1463958)
\lineto(419.88983698,230.58387467)
\curveto(419.88983698,230.62554125)(419.89764947,230.66199951)(419.91327443,230.69324944)
\curveto(419.93410772,230.72449938)(419.96796182,230.750541)(420.01483673,230.77137429)
\curveto(420.06691996,230.79220758)(420.13462815,230.80783255)(420.21796132,230.81824919)
\curveto(420.30129448,230.82866584)(420.40546094,230.83387416)(420.53046068,230.83387416)
\curveto(420.66066875,230.83387416)(420.76743937,230.82866584)(420.85077254,230.81824919)
\curveto(420.9341057,230.80783255)(420.99920973,230.79220758)(421.04608464,230.77137429)
\curveto(421.09295954,230.750541)(421.12681364,230.72449938)(421.14764693,230.69324944)
\curveto(421.16848023,230.66199951)(421.17889687,230.62554125)(421.17889687,230.58387467)
\lineto(421.17889687,226.37294569)
\curveto(421.54868779,226.7635699)(421.92108287,227.05263182)(422.29608211,227.24013144)
\curveto(422.67108135,227.43283938)(423.04868475,227.52919335)(423.42889231,227.52919335)
\curveto(423.89764137,227.52919335)(424.29086974,227.44846435)(424.60857743,227.28700634)
\curveto(424.93149344,227.13075666)(425.19190958,226.91981959)(425.38982584,226.65419512)
\curveto(425.58774211,226.38857066)(425.72836682,226.07607129)(425.81169999,225.71669702)
\curveto(425.90024148,225.36253107)(425.94451222,224.93284444)(425.94451222,224.42763713)
\closepath
}
}
{
\newrgbcolor{curcolor}{1 1 1}
\pscustom[linestyle=none,fillstyle=solid,fillcolor=curcolor]
{
\newpath
\moveto(431.20493858,219.9510837)
\lineto(430.26744048,217.36515143)
\curveto(430.23619054,217.28181827)(430.15546154,217.2193184)(430.02525347,217.17765181)
\curveto(429.90025372,217.13077691)(429.70754578,217.10733946)(429.44712964,217.10733946)
\curveto(429.31171325,217.10733946)(429.20233847,217.11515194)(429.1190053,217.13077691)
\curveto(429.03567214,217.14119355)(428.9705681,217.16202685)(428.9236932,217.19327678)
\curveto(428.88202662,217.22452672)(428.85858916,217.2661933)(428.85338084,217.31827653)
\curveto(428.84817252,217.37035976)(428.86119333,217.43285963)(428.89244326,217.50577615)
\lineto(429.8611913,219.9510837)
\curveto(429.8143164,219.97191699)(429.77004565,220.00577109)(429.72837907,220.05264599)
\curveto(429.68671249,220.0995209)(429.65806671,220.14899996)(429.64244174,220.20108319)
\lineto(427.13463432,226.91981959)
\curveto(427.09296774,227.02919436)(427.07213445,227.11513169)(427.07213445,227.17763156)
\curveto(427.07213445,227.24013144)(427.09296774,227.2896105)(427.13463432,227.32606876)
\curveto(427.1763009,227.36252702)(427.2440091,227.38596448)(427.33775891,227.39638112)
\curveto(427.43150872,227.41200609)(427.55650847,227.41981857)(427.71275815,227.41981857)
\curveto(427.86900784,227.41981857)(427.99140342,227.41461025)(428.07994491,227.40419361)
\curveto(428.1684864,227.39898528)(428.23879875,227.38596448)(428.29088198,227.36513118)
\curveto(428.34296521,227.34429789)(428.37942347,227.31304796)(428.40025676,227.27138137)
\curveto(428.42629837,227.23492311)(428.45233999,227.18283989)(428.4783816,227.11513169)
\lineto(430.48619004,221.47451811)
\lineto(430.50962749,221.47451811)
\lineto(432.44712356,227.14638163)
\curveto(432.4783735,227.24533976)(432.51483176,227.30783963)(432.55649834,227.33388125)
\curveto(432.60337325,227.36513118)(432.67108144,227.38596448)(432.75962293,227.39638112)
\curveto(432.84816442,227.41200609)(432.97576833,227.41981857)(433.14243466,227.41981857)
\curveto(433.28826769,227.41981857)(433.40805912,227.41200609)(433.50180893,227.39638112)
\curveto(433.59555874,227.38596448)(433.66326693,227.36252702)(433.70493352,227.32606876)
\curveto(433.75180842,227.2896105)(433.77524587,227.24013144)(433.77524587,227.17763156)
\curveto(433.77524587,227.11513169)(433.75962091,227.03700685)(433.72837097,226.94325704)
\closepath
}
}
{
\newrgbcolor{curcolor}{1 1 1}
\pscustom[linestyle=none,fillstyle=solid,fillcolor=curcolor]
{
\newpath
\moveto(439.50386053,222.06045443)
\curveto(439.50386053,221.70108015)(439.43615233,221.3807683)(439.30073594,221.09951887)
\curveto(439.17052787,220.81826944)(438.98302825,220.58129075)(438.73823708,220.38858281)
\curveto(438.49344591,220.19587487)(438.20177983,220.05004183)(437.86323885,219.9510837)
\curveto(437.52469787,219.85212556)(437.15230279,219.8026465)(436.74605361,219.8026465)
\curveto(436.49605412,219.8026465)(436.25647127,219.82347979)(436.02730507,219.86514637)
\curveto(435.80334719,219.90160463)(435.6002226,219.94847954)(435.4179313,220.00577109)
\curveto(435.24084832,220.06827096)(435.08980696,220.13077083)(434.96480722,220.19327071)
\curveto(434.83980747,220.2609789)(434.74866182,220.32087462)(434.69137027,220.37295784)
\curveto(434.63407872,220.42504107)(434.59241214,220.49795759)(434.56637052,220.5917074)
\curveto(434.54032891,220.68545721)(434.5273081,220.81306112)(434.5273081,220.97451913)
\curveto(434.5273081,221.07347726)(434.53251643,221.15681042)(434.54293307,221.22451862)
\curveto(434.55334972,221.29222682)(434.56637052,221.3469142)(434.58199549,221.38858079)
\curveto(434.59762046,221.43024737)(434.61845375,221.45889314)(434.64449537,221.47451811)
\curveto(434.6757453,221.4953514)(434.7095994,221.50576805)(434.74605766,221.50576805)
\curveto(434.80334921,221.50576805)(434.88668238,221.46930979)(434.99605715,221.39639327)
\curveto(435.11064026,221.32868507)(435.24866081,221.25316439)(435.41011882,221.16983123)
\curveto(435.57678514,221.08649807)(435.77209725,221.00837322)(435.99605513,220.9354567)
\curveto(436.22001301,220.86774851)(436.47782499,220.83389441)(436.76949106,220.83389441)
\curveto(436.98824062,220.83389441)(437.18615689,220.85733186)(437.36323986,220.90420677)
\curveto(437.54032284,220.95108167)(437.69396836,221.01878987)(437.82417643,221.10733136)
\curveto(437.9543845,221.20108117)(438.05334263,221.31826843)(438.12105083,221.45889314)
\curveto(438.19396734,221.59951786)(438.2304256,221.76618419)(438.2304256,221.95889213)
\curveto(438.2304256,222.1568084)(438.17834238,222.32347473)(438.07417592,222.45889112)
\curveto(437.97521779,222.59430751)(437.84240556,222.71409894)(437.67573923,222.81826539)
\curveto(437.5090729,222.92243185)(437.32157328,223.0135775)(437.11324037,223.09170234)
\curveto(436.90490746,223.1750355)(436.68876206,223.26097283)(436.46480418,223.34951432)
\curveto(436.24605462,223.4380558)(436.02990923,223.53701394)(435.81636799,223.64638871)
\curveto(435.60803508,223.76097182)(435.42053546,223.89899237)(435.25386913,224.06045038)
\curveto(435.0872028,224.22190838)(434.95178641,224.41461633)(434.84761995,224.63857421)
\curveto(434.74866182,224.86253209)(434.69918276,225.13076071)(434.69918276,225.44326008)
\curveto(434.69918276,225.71930118)(434.75126598,225.98232148)(434.85543244,226.23232098)
\curveto(434.96480722,226.48752879)(435.12626522,226.70888251)(435.33980646,226.89638213)
\curveto(435.55334769,227.08909008)(435.81897215,227.2427356)(436.13667984,227.3573187)
\curveto(436.45959586,227.4719018)(436.8345951,227.52919335)(437.26167757,227.52919335)
\curveto(437.44917719,227.52919335)(437.63667681,227.51356838)(437.82417643,227.48231845)
\curveto(438.01167605,227.45106851)(438.18094654,227.41200609)(438.3319879,227.36513118)
\curveto(438.48302926,227.31825628)(438.61063317,227.26617305)(438.71479962,227.2088815)
\curveto(438.8241744,227.15679827)(438.90490341,227.10992337)(438.95698663,227.06825679)
\curveto(439.01427818,227.0265902)(439.05073644,226.99013194)(439.06636141,226.95888201)
\curveto(439.0871947,226.92763207)(439.10021551,226.89117381)(439.10542383,226.84950723)
\curveto(439.11584048,226.81304897)(439.12365296,226.76617406)(439.12886128,226.70888251)
\curveto(439.13927793,226.65159096)(439.14448625,226.5812786)(439.14448625,226.49794544)
\curveto(439.14448625,226.40940395)(439.13927793,226.33127911)(439.12886128,226.26357092)
\curveto(439.12365296,226.20107104)(439.11063216,226.14898781)(439.08979886,226.10732123)
\curveto(439.0741739,226.06565465)(439.0533406,226.03440471)(439.02729899,226.01357142)
\curveto(439.00125738,225.99794645)(438.9726116,225.99013397)(438.94136166,225.99013397)
\curveto(438.89448676,225.99013397)(438.82677856,226.01877974)(438.73823708,226.07607129)
\curveto(438.64969559,226.13336285)(438.53511249,226.19325856)(438.39448777,226.25575843)
\curveto(438.25386306,226.32346663)(438.08719673,226.3859665)(437.89448878,226.44325805)
\curveto(437.70698916,226.5005496)(437.49084377,226.52919538)(437.2460526,226.52919538)
\curveto(437.02730304,226.52919538)(436.8345951,226.50315376)(436.66792877,226.45107054)
\curveto(436.50126244,226.40419563)(436.36324189,226.33388327)(436.25386711,226.24013346)
\curveto(436.14970065,226.15159198)(436.06897165,226.04482136)(436.0116801,225.91982161)
\curveto(435.95959687,225.79482186)(435.93355526,225.65940547)(435.93355526,225.51357243)
\curveto(435.93355526,225.31044785)(435.98563848,225.13857319)(436.08980494,224.99794848)
\curveto(436.19397139,224.86253209)(436.32938779,224.74274066)(436.49605412,224.63857421)
\curveto(436.66272045,224.53440775)(436.85282423,224.44065794)(437.06636546,224.35732478)
\curveto(437.2799067,224.27399161)(437.49605209,224.18805428)(437.71480165,224.0995128)
\curveto(437.93875953,224.01097131)(438.15750909,223.91201318)(438.37105032,223.8026384)
\curveto(438.58979988,223.69326362)(438.78250782,223.56045139)(438.94917415,223.40420171)
\curveto(439.11584048,223.24795202)(439.24865271,223.0604524)(439.34761084,222.84170284)
\curveto(439.4517773,222.62295329)(439.50386053,222.36253715)(439.50386053,222.06045443)
\closepath
}
}
{
\newrgbcolor{curcolor}{1 1 1}
\pscustom[linestyle=none,fillstyle=solid,fillcolor=curcolor]
{
\newpath
\moveto(442.6348547,220.1463958)
\curveto(442.6348547,220.10472922)(442.62443806,220.06827096)(442.60360477,220.03702102)
\curveto(442.58277148,220.01097941)(442.54891738,219.98754196)(442.50204247,219.96670867)
\curveto(442.45516757,219.94587538)(442.39006353,219.93025041)(442.30673037,219.91983376)
\curveto(442.2233972,219.90941712)(442.11662659,219.90420879)(441.98641852,219.90420879)
\curveto(441.86141877,219.90420879)(441.75725231,219.90941712)(441.67391915,219.91983376)
\curveto(441.59058599,219.93025041)(441.52287779,219.94587538)(441.47079456,219.96670867)
\curveto(441.42391966,219.98754196)(441.39006556,220.01097941)(441.36923227,220.03702102)
\curveto(441.3536073,220.06827096)(441.34579481,220.10472922)(441.34579481,220.1463958)
\lineto(441.34579481,227.17763156)
\curveto(441.34579481,227.21408982)(441.3536073,227.24794392)(441.36923227,227.27919386)
\curveto(441.39006556,227.3104438)(441.42391966,227.33648541)(441.47079456,227.3573187)
\curveto(441.52287779,227.37815199)(441.59058599,227.39377696)(441.67391915,227.40419361)
\curveto(441.75725231,227.41461025)(441.86141877,227.41981857)(441.98641852,227.41981857)
\curveto(442.11662659,227.41981857)(442.2233972,227.41461025)(442.30673037,227.40419361)
\curveto(442.39006353,227.39377696)(442.45516757,227.37815199)(442.50204247,227.3573187)
\curveto(442.54891738,227.33648541)(442.58277148,227.3104438)(442.60360477,227.27919386)
\curveto(442.62443806,227.24794392)(442.6348547,227.21408982)(442.6348547,227.17763156)
\closepath
\moveto(442.7832919,229.55262675)
\curveto(442.7832919,229.25054403)(442.72600035,229.04481528)(442.61141725,228.9354405)
\curveto(442.49683415,228.82606573)(442.28589708,228.77137834)(441.97860603,228.77137834)
\curveto(441.67652331,228.77137834)(441.4681904,228.82346156)(441.3536073,228.92762802)
\curveto(441.24423252,229.0370028)(441.18954513,229.24012739)(441.18954513,229.53700179)
\curveto(441.18954513,229.83908451)(441.24683668,230.04481326)(441.36141978,230.15418804)
\curveto(441.47600288,230.26356282)(441.68693996,230.3182502)(441.994231,230.3182502)
\curveto(442.29631372,230.3182502)(442.50204247,230.26356282)(442.61141725,230.15418804)
\curveto(442.72600035,230.05002158)(442.7832919,229.84950115)(442.7832919,229.55262675)
\closepath
}
}
{
\newrgbcolor{curcolor}{1 1 1}
\pscustom[linestyle=none,fillstyle=solid,fillcolor=curcolor]
{
\newpath
\moveto(450.06446422,221.21670613)
\curveto(450.06446422,221.12816465)(450.06186006,221.05003981)(450.05665173,220.98233161)
\curveto(450.05144341,220.91983174)(450.04102676,220.86514435)(450.0254018,220.81826944)
\curveto(450.01498515,220.77660286)(449.99936018,220.73754044)(449.97852689,220.70108218)
\curveto(449.96290192,220.66983224)(449.92123534,220.62035318)(449.85352714,220.55264498)
\curveto(449.79102727,220.49014511)(449.68165249,220.4094161)(449.52540281,220.31045797)
\curveto(449.36915312,220.21670816)(449.19207015,220.13077083)(448.99415388,220.05264599)
\curveto(448.80144594,219.97972947)(448.59050887,219.91983376)(448.36134267,219.87295886)
\curveto(448.13217646,219.82608395)(447.89519778,219.8026465)(447.65040661,219.8026465)
\curveto(447.14519929,219.8026465)(446.69728354,219.88597966)(446.30665933,220.05264599)
\curveto(445.91603512,220.21931232)(445.58791078,220.46149933)(445.32228632,220.77920702)
\curveto(445.06187018,221.10212303)(444.86134975,221.4953514)(444.72072504,221.95889213)
\curveto(444.58530865,222.42764118)(444.51760045,222.96670259)(444.51760045,223.57607636)
\curveto(444.51760045,224.26878329)(444.60093361,224.86253209)(444.76759994,225.35732275)
\curveto(444.93947459,225.85732174)(445.17124496,226.26617508)(445.46291103,226.58388277)
\curveto(445.75978543,226.90159046)(446.1061389,227.13596498)(446.50197143,227.28700634)
\curveto(446.90301229,227.44325603)(447.33530308,227.52138087)(447.7988438,227.52138087)
\curveto(448.02280168,227.52138087)(448.23894708,227.50054758)(448.44727999,227.45888099)
\curveto(448.66082123,227.41721441)(448.85613333,227.36252702)(449.0332163,227.29481883)
\curveto(449.21029928,227.22711063)(449.36654896,227.14898579)(449.50196536,227.0604443)
\curveto(449.64259007,226.97190281)(449.74415237,226.89638213)(449.80665224,226.83388226)
\curveto(449.86915211,226.77138239)(449.91081869,226.72190332)(449.93165199,226.68544506)
\curveto(449.9576936,226.6489868)(449.97852689,226.60471606)(449.99415186,226.55263283)
\curveto(450.00977683,226.50575792)(450.02019347,226.45107054)(450.0254018,226.38857066)
\curveto(450.03061012,226.32607079)(450.03321428,226.24794595)(450.03321428,226.15419614)
\curveto(450.03321428,225.95107155)(450.00977683,225.80784267)(449.96290192,225.72450951)
\curveto(449.91602702,225.64638466)(449.85873547,225.60732224)(449.79102727,225.60732224)
\curveto(449.71290243,225.60732224)(449.62175678,225.64898883)(449.51759032,225.73232199)
\curveto(449.41863219,225.82086348)(449.29102828,225.91721745)(449.1347786,226.02138391)
\curveto(448.97852892,226.12555036)(448.78842513,226.21930017)(448.56446725,226.30263334)
\curveto(448.3457177,226.39117482)(448.08530156,226.43544557)(447.78321884,226.43544557)
\curveto(447.16342842,226.43544557)(446.68686689,226.19586272)(446.35353423,225.71669702)
\curveto(446.0254099,225.24273965)(445.86134773,224.55263688)(445.86134773,223.64638871)
\curveto(445.86134773,223.19326463)(445.90301431,222.79482794)(445.98634747,222.45107864)
\curveto(446.07488896,222.11253765)(446.20249287,221.82868406)(446.3691592,221.59951786)
\curveto(446.53582553,221.37035166)(446.73895012,221.19847701)(446.97853297,221.0838939)
\curveto(447.22332414,220.97451913)(447.50196941,220.91983174)(447.81446877,220.91983174)
\curveto(448.11134317,220.91983174)(448.37175931,220.96670664)(448.59571719,221.06045645)
\curveto(448.81967507,221.15420626)(449.01238301,221.25576856)(449.17384102,221.36514333)
\curveto(449.34050735,221.47972644)(449.4785279,221.58128873)(449.58790268,221.66983022)
\curveto(449.70248578,221.76358003)(449.79102727,221.81045493)(449.85352714,221.81045493)
\curveto(449.8899854,221.81045493)(449.92123534,221.80003829)(449.94727695,221.779205)
\curveto(449.97331857,221.7583717)(449.99415186,221.72191345)(450.00977683,221.66983022)
\curveto(450.03061012,221.62295531)(450.04363093,221.56045544)(450.04883925,221.4823306)
\curveto(450.05925589,221.40941408)(450.06446422,221.32087259)(450.06446422,221.21670613)
\closepath
}
}
{
\newrgbcolor{curcolor}{1 1 1}
\pscustom[linestyle=none,fillstyle=solid,fillcolor=curcolor]
{
\newpath
\moveto(456.93063836,220.13858332)
\curveto(456.93063836,220.07608344)(456.90980507,220.02920854)(456.86813849,219.9979586)
\curveto(456.82647191,219.96670867)(456.76918036,219.94327121)(456.69626384,219.92764625)
\curveto(456.62334732,219.91202128)(456.5165767,219.90420879)(456.37595199,219.90420879)
\curveto(456.24053559,219.90420879)(456.13116081,219.91202128)(456.04782765,219.92764625)
\curveto(455.96970281,219.94327121)(455.91241126,219.96670867)(455.875953,219.9979586)
\curveto(455.83949474,220.02920854)(455.82126561,220.07608344)(455.82126561,220.13858332)
\lineto(455.82126561,220.84170689)
\curveto(455.51397456,220.51358256)(455.17022526,220.25837474)(454.7900177,220.07608344)
\curveto(454.41501846,219.89379215)(454.01658176,219.8026465)(453.59470762,219.8026465)
\curveto(453.2249167,219.8026465)(452.88897988,219.85212556)(452.58689716,219.9510837)
\curveto(452.29002276,220.04483351)(452.03481494,220.18285406)(451.82127371,220.36514536)
\curveto(451.6129408,220.54743666)(451.44887863,220.77139454)(451.32908721,221.037019)
\curveto(451.2145041,221.30264346)(451.15721255,221.60472618)(451.15721255,221.94326716)
\curveto(451.15721255,222.3390997)(451.23794156,222.682849)(451.39939956,222.97451508)
\curveto(451.56085757,223.26618115)(451.79262793,223.50836816)(452.09471065,223.7010761)
\curveto(452.39679338,223.89378405)(452.76658429,224.03701292)(453.20408341,224.13076273)
\curveto(453.64158252,224.22972087)(454.13376903,224.27919993)(454.68064292,224.27919993)
\lineto(455.64939096,224.27919993)
\lineto(455.64939096,224.82607383)
\curveto(455.64939096,225.09690661)(455.62074518,225.33648946)(455.56345363,225.54482237)
\curveto(455.50616208,225.75315528)(455.41241227,225.92502993)(455.2822042,226.06044633)
\curveto(455.15720445,226.20107104)(454.99314229,226.3052375)(454.7900177,226.37294569)
\curveto(454.58689311,226.44586221)(454.33689361,226.48232047)(454.04001922,226.48232047)
\curveto(453.72231153,226.48232047)(453.43585377,226.44325805)(453.18064596,226.36513321)
\curveto(452.93064646,226.29221669)(452.70929274,226.20888353)(452.5165848,226.11513372)
\curveto(452.32908518,226.02659223)(452.17023134,225.94325906)(452.04002327,225.86513422)
\curveto(451.91502352,225.7922177)(451.82127371,225.75575944)(451.75877383,225.75575944)
\curveto(451.71710725,225.75575944)(451.68064899,225.76617609)(451.64939906,225.78700938)
\curveto(451.61814912,225.80784267)(451.58950334,225.83909261)(451.56346173,225.88075919)
\curveto(451.54262844,225.92242577)(451.52700347,225.974509)(451.51658683,226.03700887)
\curveto(451.50617018,226.10471707)(451.50096186,226.17763359)(451.50096186,226.25575843)
\curveto(451.50096186,226.3859665)(451.50877434,226.48752879)(451.52439931,226.56044531)
\curveto(451.5452326,226.63857016)(451.58950334,226.71148667)(451.65721154,226.77919487)
\curveto(451.73012806,226.84690307)(451.85252365,226.92502791)(452.0243983,227.0135694)
\curveto(452.19627295,227.10731921)(452.39418921,227.19065237)(452.61814709,227.26356889)
\curveto(452.84210497,227.34169373)(453.08689615,227.40419361)(453.35252061,227.45106851)
\curveto(453.61814507,227.50315174)(453.88637369,227.52919335)(454.15720648,227.52919335)
\curveto(454.66241379,227.52919335)(455.09210042,227.4719018)(455.44626637,227.3573187)
\curveto(455.80043232,227.2427356)(456.08689007,227.07346511)(456.30563963,226.84950723)
\curveto(456.52438918,226.63075767)(456.68324303,226.35732073)(456.78220116,226.02919639)
\curveto(456.8811593,225.70107205)(456.93063836,225.31826033)(456.93063836,224.88076122)
\closepath
\moveto(455.64939096,223.34951432)
\lineto(454.54783069,223.34951432)
\curveto(454.19366474,223.34951432)(453.88637369,223.31826438)(453.62595755,223.25576451)
\curveto(453.36554141,223.19847296)(453.14939602,223.10993147)(452.97752137,222.99014004)
\curveto(452.80564672,222.87555694)(452.67804281,222.73493223)(452.59470964,222.5682659)
\curveto(452.5165848,222.40680789)(452.47752238,222.21930827)(452.47752238,222.00576704)
\curveto(452.47752238,221.64118444)(452.59210548,221.34951837)(452.82127168,221.13076881)
\curveto(453.05564621,220.91722757)(453.38116638,220.81045696)(453.79783221,220.81045696)
\curveto(454.13637319,220.81045696)(454.44887255,220.89639428)(454.73533031,221.06826894)
\curveto(455.02699638,221.24014359)(455.33168327,221.50316389)(455.64939096,221.85732984)
\closepath
}
}
{
\newrgbcolor{curcolor}{1 1 1}
\pscustom[linestyle=none,fillstyle=solid,fillcolor=curcolor]
{
\newpath
\moveto(460.58681611,220.1463958)
\curveto(460.58681611,220.10472922)(460.57639946,220.06827096)(460.55556617,220.03702102)
\curveto(460.53473288,220.01097941)(460.50087878,219.98754196)(460.45400388,219.96670867)
\curveto(460.40712897,219.94587538)(460.34202494,219.93025041)(460.25869177,219.91983376)
\curveto(460.17535861,219.90941712)(460.06858799,219.90420879)(459.93837992,219.90420879)
\curveto(459.81338017,219.90420879)(459.70921372,219.90941712)(459.62588055,219.91983376)
\curveto(459.54254739,219.93025041)(459.47483919,219.94587538)(459.42275597,219.96670867)
\curveto(459.37588106,219.98754196)(459.34202696,220.01097941)(459.32119367,220.03702102)
\curveto(459.3055687,220.06827096)(459.29775622,220.10472922)(459.29775622,220.1463958)
\lineto(459.29775622,230.58387467)
\curveto(459.29775622,230.62554125)(459.3055687,230.66199951)(459.32119367,230.69324944)
\curveto(459.34202696,230.72449938)(459.37588106,230.750541)(459.42275597,230.77137429)
\curveto(459.47483919,230.79220758)(459.54254739,230.80783255)(459.62588055,230.81824919)
\curveto(459.70921372,230.82866584)(459.81338017,230.83387416)(459.93837992,230.83387416)
\curveto(460.06858799,230.83387416)(460.17535861,230.82866584)(460.25869177,230.81824919)
\curveto(460.34202494,230.80783255)(460.40712897,230.79220758)(460.45400388,230.77137429)
\curveto(460.50087878,230.750541)(460.53473288,230.72449938)(460.55556617,230.69324944)
\curveto(460.57639946,230.66199951)(460.58681611,230.62554125)(460.58681611,230.58387467)
\closepath
}
}
{
\newrgbcolor{curcolor}{1 1 1}
\pscustom[linestyle=none,fillstyle=solid,fillcolor=curcolor]
{
\newpath
\moveto(473.04648097,223.75576349)
\curveto(473.04648097,223.14638973)(472.97877278,222.59691167)(472.84335638,222.10732933)
\curveto(472.71314831,221.62295531)(472.51783621,221.20889365)(472.25742007,220.86514435)
\curveto(472.00221225,220.52139504)(471.68710872,220.25837474)(471.31210948,220.07608344)
\curveto(470.93711024,219.89379215)(470.50742361,219.8026465)(470.02304959,219.8026465)
\curveto(469.79909171,219.8026465)(469.5907588,219.82608395)(469.39805086,219.87295886)
\curveto(469.21055124,219.91462544)(469.02565578,219.9849378)(468.84336448,220.08389593)
\curveto(468.66107319,220.18285406)(468.47878189,220.30785381)(468.29649059,220.45889517)
\curveto(468.11419929,220.60993653)(467.92149135,220.79222783)(467.71836676,221.00576906)
\lineto(467.71836676,220.1463958)
\curveto(467.71836676,220.10472922)(467.70795012,220.06827096)(467.68711682,220.03702102)
\curveto(467.66628353,220.00577109)(467.63242943,219.97972947)(467.58555453,219.95889618)
\curveto(467.53867962,219.94327121)(467.47878391,219.93025041)(467.40586739,219.91983376)
\curveto(467.3381592,219.90941712)(467.24961771,219.90420879)(467.14024293,219.90420879)
\curveto(467.03607648,219.90420879)(466.94753499,219.90941712)(466.87461847,219.91983376)
\curveto(466.80170195,219.93025041)(466.74180624,219.94327121)(466.69493133,219.95889618)
\curveto(466.64805643,219.97972947)(466.61680649,220.00577109)(466.60118152,220.03702102)
\curveto(466.58555655,220.06827096)(466.57774407,220.10472922)(466.57774407,220.1463958)
\lineto(466.57774407,230.58387467)
\curveto(466.57774407,230.62554125)(466.58555655,230.66199951)(466.60118152,230.69324944)
\curveto(466.62201481,230.72449938)(466.65586891,230.750541)(466.70274382,230.77137429)
\curveto(466.75482705,230.79220758)(466.82253524,230.80783255)(466.90586841,230.81824919)
\curveto(466.98920157,230.82866584)(467.09336803,230.83387416)(467.21836777,230.83387416)
\curveto(467.34857584,230.83387416)(467.45534646,230.82866584)(467.53867962,230.81824919)
\curveto(467.62201279,230.80783255)(467.68711682,230.79220758)(467.73399173,230.77137429)
\curveto(467.78086663,230.750541)(467.81472073,230.72449938)(467.83555402,230.69324944)
\curveto(467.85638731,230.66199951)(467.86680396,230.62554125)(467.86680396,230.58387467)
\lineto(467.86680396,226.37294569)
\curveto(468.07513687,226.58648693)(468.2756573,226.76617406)(468.46836524,226.9120071)
\curveto(468.66628151,227.05784014)(468.85898945,227.1750274)(469.04648907,227.26356889)
\curveto(469.23398869,227.3573187)(469.42148831,227.4250269)(469.60898793,227.46669348)
\curveto(469.79648755,227.50836006)(469.99440382,227.52919335)(470.20273673,227.52919335)
\curveto(470.71315236,227.52919335)(471.14804732,227.42763106)(471.50742159,227.22450647)
\curveto(471.87200418,227.02138188)(472.16627442,226.74794493)(472.3902323,226.40419563)
\curveto(472.6193985,226.06565465)(472.78606483,225.66721796)(472.89023129,225.20888555)
\curveto(472.99439774,224.75055315)(473.04648097,224.26617913)(473.04648097,223.75576349)
\closepath
\moveto(471.68710872,223.60732629)
\curveto(471.68710872,223.96670057)(471.65846295,224.31565819)(471.6011714,224.65419917)
\curveto(471.54908817,224.99274016)(471.45533836,225.29221872)(471.31992197,225.55263485)
\curveto(471.18450558,225.81305099)(471.00481844,226.02138391)(470.78086056,226.17763359)
\curveto(470.55690268,226.3390916)(470.27825741,226.4198206)(469.94492475,226.4198206)
\curveto(469.77825842,226.4198206)(469.61419626,226.39638315)(469.45273825,226.34950824)
\curveto(469.29128024,226.30263334)(469.12721807,226.22450849)(468.96055175,226.11513372)
\curveto(468.79388542,226.00575894)(468.6194066,225.86513422)(468.43711531,225.69325957)
\curveto(468.26003233,225.52138492)(468.06992855,225.30523952)(467.86680396,225.04482338)
\lineto(467.86680396,222.24014156)
\curveto(468.22096991,221.80785077)(468.55951089,221.47712227)(468.8824269,221.24795607)
\curveto(469.20534292,221.02399819)(469.54127974,220.91201925)(469.89023736,220.91201925)
\curveto(470.21315338,220.91201925)(470.48919448,220.99014409)(470.71836069,221.14639378)
\curveto(470.94752689,221.30264346)(471.13242235,221.50837221)(471.27304706,221.76358003)
\curveto(471.4188801,222.02399617)(471.52304656,222.31305808)(471.58554643,222.63076577)
\curveto(471.65325463,222.95368178)(471.68710872,223.27920196)(471.68710872,223.60732629)
\closepath
}
}
{
\newrgbcolor{curcolor}{1 1 1}
\pscustom[linestyle=none,fillstyle=solid,fillcolor=curcolor]
{
\newpath
\moveto(476.26678466,220.1463958)
\curveto(476.26678466,220.10472922)(476.25636802,220.06827096)(476.23553472,220.03702102)
\curveto(476.21470143,220.01097941)(476.18084734,219.98754196)(476.13397243,219.96670867)
\curveto(476.08709753,219.94587538)(476.02199349,219.93025041)(475.93866033,219.91983376)
\curveto(475.85532716,219.90941712)(475.74855654,219.90420879)(475.61834847,219.90420879)
\curveto(475.49334873,219.90420879)(475.38918227,219.90941712)(475.30584911,219.91983376)
\curveto(475.22251594,219.93025041)(475.15480775,219.94587538)(475.10272452,219.96670867)
\curveto(475.05584961,219.98754196)(475.02199552,220.01097941)(475.00116222,220.03702102)
\curveto(474.98553726,220.06827096)(474.97772477,220.10472922)(474.97772477,220.1463958)
\lineto(474.97772477,227.17763156)
\curveto(474.97772477,227.21408982)(474.98553726,227.24794392)(475.00116222,227.27919386)
\curveto(475.02199552,227.3104438)(475.05584961,227.33648541)(475.10272452,227.3573187)
\curveto(475.15480775,227.37815199)(475.22251594,227.39377696)(475.30584911,227.40419361)
\curveto(475.38918227,227.41461025)(475.49334873,227.41981857)(475.61834847,227.41981857)
\curveto(475.74855654,227.41981857)(475.85532716,227.41461025)(475.93866033,227.40419361)
\curveto(476.02199349,227.39377696)(476.08709753,227.37815199)(476.13397243,227.3573187)
\curveto(476.18084734,227.33648541)(476.21470143,227.3104438)(476.23553472,227.27919386)
\curveto(476.25636802,227.24794392)(476.26678466,227.21408982)(476.26678466,227.17763156)
\closepath
\moveto(476.41522186,229.55262675)
\curveto(476.41522186,229.25054403)(476.35793031,229.04481528)(476.24334721,228.9354405)
\curveto(476.12876411,228.82606573)(475.91782703,228.77137834)(475.61053599,228.77137834)
\curveto(475.30845327,228.77137834)(475.10012036,228.82346156)(474.98553726,228.92762802)
\curveto(474.87616248,229.0370028)(474.82147509,229.24012739)(474.82147509,229.53700179)
\curveto(474.82147509,229.83908451)(474.87876664,230.04481326)(474.99334974,230.15418804)
\curveto(475.10793284,230.26356282)(475.31886991,230.3182502)(475.62616096,230.3182502)
\curveto(475.92824368,230.3182502)(476.13397243,230.26356282)(476.24334721,230.15418804)
\curveto(476.35793031,230.05002158)(476.41522186,229.84950115)(476.41522186,229.55262675)
\closepath
}
}
{
\newrgbcolor{curcolor}{1 1 1}
\pscustom[linestyle=none,fillstyle=solid,fillcolor=curcolor]
{
\newpath
\moveto(482.7510855,226.716695)
\curveto(482.7510855,226.6021119)(482.74848133,226.50575792)(482.74327301,226.42763308)
\curveto(482.73806469,226.34950824)(482.72764804,226.28700837)(482.71202307,226.24013346)
\curveto(482.69639811,226.19846688)(482.67556482,226.16461278)(482.6495232,226.13857117)
\curveto(482.62868991,226.11773788)(482.59743997,226.10732123)(482.55577339,226.10732123)
\curveto(482.51410681,226.10732123)(482.46202358,226.11773788)(482.39952371,226.13857117)
\curveto(482.34223216,226.16461278)(482.27452396,226.18805023)(482.19639912,226.20888353)
\curveto(482.1234826,226.23492514)(482.04014944,226.25836259)(481.94639963,226.27919588)
\curveto(481.85264981,226.30002917)(481.75108752,226.31044582)(481.64171274,226.31044582)
\curveto(481.51150467,226.31044582)(481.38390076,226.28440421)(481.25890102,226.23232098)
\curveto(481.13390127,226.18023775)(481.00108904,226.09430042)(480.86046432,225.974509)
\curveto(480.72504793,225.85471758)(480.58181906,225.69586373)(480.43077769,225.49794747)
\curveto(480.27973633,225.3000312)(480.11307,225.05784419)(479.93077871,224.77138644)
\lineto(479.93077871,220.1463958)
\curveto(479.93077871,220.10472922)(479.92036206,220.06827096)(479.89952877,220.03702102)
\curveto(479.87869548,220.01097941)(479.84484138,219.98754196)(479.79796648,219.96670867)
\curveto(479.75109157,219.94587538)(479.68598754,219.93025041)(479.60265437,219.91983376)
\curveto(479.51932121,219.90941712)(479.41255059,219.90420879)(479.28234252,219.90420879)
\curveto(479.15734277,219.90420879)(479.05317632,219.90941712)(478.96984315,219.91983376)
\curveto(478.88650999,219.93025041)(478.81880179,219.94587538)(478.76671856,219.96670867)
\curveto(478.71984366,219.98754196)(478.68598956,220.01097941)(478.66515627,220.03702102)
\curveto(478.6495313,220.06827096)(478.64171882,220.10472922)(478.64171882,220.1463958)
\lineto(478.64171882,227.17763156)
\curveto(478.64171882,227.21929815)(478.6495313,227.25315224)(478.66515627,227.27919386)
\curveto(478.68078124,227.3104438)(478.71203117,227.33648541)(478.75890608,227.3573187)
\curveto(478.80578098,227.38336031)(478.8656767,227.39898528)(478.93859322,227.40419361)
\curveto(479.01150973,227.41461025)(479.10786371,227.41981857)(479.22765513,227.41981857)
\curveto(479.34223823,227.41981857)(479.43598804,227.41461025)(479.50890456,227.40419361)
\curveto(479.5870294,227.39898528)(479.64692511,227.38336031)(479.6885917,227.3573187)
\curveto(479.73025828,227.33648541)(479.75890405,227.3104438)(479.77452902,227.27919386)
\curveto(479.79536231,227.25315224)(479.80577896,227.21929815)(479.80577896,227.17763156)
\lineto(479.80577896,226.15419614)
\curveto(479.9984869,226.43544557)(480.17817404,226.66461177)(480.34484037,226.84169474)
\curveto(480.51671502,227.01877772)(480.67817303,227.15679827)(480.82921439,227.25575641)
\curveto(480.98025575,227.35992286)(481.12869295,227.43023522)(481.27452599,227.46669348)
\curveto(481.42556735,227.50836006)(481.57660871,227.52919335)(481.72765007,227.52919335)
\curveto(481.79535826,227.52919335)(481.87087894,227.52398503)(481.95421211,227.51356838)
\curveto(482.0427536,227.50836006)(482.13389925,227.49533925)(482.22764906,227.47450596)
\curveto(482.32139887,227.45367267)(482.40473203,227.43023522)(482.47764855,227.40419361)
\curveto(482.55577339,227.37815199)(482.61046078,227.35211038)(482.64171072,227.32606876)
\curveto(482.67296065,227.30002715)(482.69379394,227.27398554)(482.70421059,227.24794392)
\curveto(482.71462724,227.22711063)(482.72243972,227.19846486)(482.72764804,227.1620066)
\curveto(482.73806469,227.12554834)(482.74327301,227.07086095)(482.74327301,226.99794443)
\curveto(482.74848133,226.93023623)(482.7510855,226.83648642)(482.7510855,226.716695)
\closepath
}
}
{
\newrgbcolor{curcolor}{1 1 1}
\pscustom[linestyle=none,fillstyle=solid,fillcolor=curcolor]
{
\newpath
\moveto(487.86595093,220.66201976)
\curveto(487.86595093,220.5109784)(487.85553429,220.39118697)(487.83470099,220.30264549)
\curveto(487.8138677,220.214104)(487.78261777,220.14899996)(487.74095118,220.10733338)
\curveto(487.6992846,220.0656668)(487.63678473,220.02660438)(487.55345156,219.99014612)
\curveto(487.4701184,219.95368786)(487.37376443,219.92504208)(487.26438965,219.90420879)
\curveto(487.16022319,219.87816718)(487.04824425,219.85733389)(486.92845283,219.84170892)
\curveto(486.80866141,219.82608395)(486.68886998,219.81827147)(486.56907856,219.81827147)
\curveto(486.20449596,219.81827147)(485.8919966,219.86514637)(485.63158046,219.95889618)
\curveto(485.37116432,220.05785431)(485.15762308,220.20368735)(484.99095675,220.3963953)
\curveto(484.82429042,220.59431156)(484.70189484,220.84170689)(484.62377,221.13858129)
\curveto(484.55085348,221.44066401)(484.51439522,221.79482996)(484.51439522,222.20107914)
\lineto(484.51439522,226.31044582)
\lineto(483.53002221,226.31044582)
\curveto(483.45189737,226.31044582)(483.3893975,226.3521124)(483.34252259,226.43544557)
\curveto(483.29564769,226.51877873)(483.27221023,226.65419512)(483.27221023,226.84169474)
\curveto(483.27221023,226.94065288)(483.27741856,227.02398604)(483.2878352,227.09169424)
\curveto(483.30346017,227.15940243)(483.3216893,227.21408982)(483.34252259,227.25575641)
\curveto(483.36335588,227.30263131)(483.3893975,227.33388125)(483.42064743,227.34950622)
\curveto(483.45710569,227.37033951)(483.49616811,227.38075615)(483.5378347,227.38075615)
\lineto(484.51439522,227.38075615)
\lineto(484.51439522,229.05262777)
\curveto(484.51439522,229.08908603)(484.5222077,229.12294012)(484.53783267,229.15419006)
\curveto(484.55866596,229.18544)(484.59252006,229.21148161)(484.63939497,229.2323149)
\curveto(484.69147819,229.25835652)(484.75918639,229.27658565)(484.84251955,229.28700229)
\curveto(484.92585272,229.29741894)(485.03001917,229.30262726)(485.15501892,229.30262726)
\curveto(485.28522699,229.30262726)(485.39199761,229.29741894)(485.47533077,229.28700229)
\curveto(485.55866394,229.27658565)(485.62376797,229.25835652)(485.67064288,229.2323149)
\curveto(485.71751778,229.21148161)(485.75137188,229.18544)(485.77220517,229.15419006)
\curveto(485.79303846,229.12294012)(485.80345511,229.08908603)(485.80345511,229.05262777)
\lineto(485.80345511,227.38075615)
\lineto(487.60813895,227.38075615)
\curveto(487.64980554,227.38075615)(487.6862638,227.37033951)(487.71751373,227.34950622)
\curveto(487.74876367,227.33388125)(487.77480528,227.30263131)(487.79563857,227.25575641)
\curveto(487.82168019,227.21408982)(487.83990932,227.15940243)(487.85032596,227.09169424)
\curveto(487.86074261,227.02398604)(487.86595093,226.94065288)(487.86595093,226.84169474)
\curveto(487.86595093,226.65419512)(487.84251348,226.51877873)(487.79563857,226.43544557)
\curveto(487.74876367,226.3521124)(487.6862638,226.31044582)(487.60813895,226.31044582)
\lineto(485.80345511,226.31044582)
\lineto(485.80345511,222.38857876)
\curveto(485.80345511,221.90420474)(485.87376747,221.53701799)(486.01439218,221.28701849)
\curveto(486.16022522,221.04222732)(486.4180372,220.91983174)(486.78782811,220.91983174)
\curveto(486.90761954,220.91983174)(487.01439016,220.93024838)(487.10813997,220.95108167)
\curveto(487.20188978,220.97712329)(487.28522294,221.0031649)(487.35813946,221.02920651)
\curveto(487.43105598,221.05524813)(487.49355585,221.07868558)(487.54563908,221.09951887)
\curveto(487.59772231,221.12556049)(487.64459721,221.13858129)(487.6862638,221.13858129)
\curveto(487.71230541,221.13858129)(487.73574286,221.13076881)(487.75657615,221.11514384)
\curveto(487.78261777,221.1047272)(487.8008469,221.08128974)(487.81126354,221.04483148)
\curveto(487.82688851,221.00837322)(487.83990932,220.95889416)(487.85032596,220.89639428)
\curveto(487.86074261,220.83389441)(487.86595093,220.75576957)(487.86595093,220.66201976)
\closepath
}
}
{
\newrgbcolor{curcolor}{1 1 1}
\pscustom[linestyle=none,fillstyle=solid,fillcolor=curcolor]
{
\newpath
\moveto(495.60837452,220.1463958)
\curveto(495.60837452,220.10472922)(495.59795788,220.06827096)(495.57712459,220.03702102)
\curveto(495.5562913,220.01097941)(495.5224372,219.98754196)(495.47556229,219.96670867)
\curveto(495.42868739,219.94587538)(495.36358335,219.93025041)(495.28025019,219.91983376)
\curveto(495.19691702,219.90941712)(495.09275057,219.90420879)(494.96775082,219.90420879)
\curveto(494.83754275,219.90420879)(494.73077213,219.90941712)(494.64743897,219.91983376)
\curveto(494.5641058,219.93025041)(494.49900177,219.94587538)(494.45212686,219.96670867)
\curveto(494.40525196,219.98754196)(494.37139786,220.01097941)(494.35056457,220.03702102)
\curveto(494.32973128,220.06827096)(494.31931463,220.10472922)(494.31931463,220.1463958)
\lineto(494.31931463,224.26357497)
\curveto(494.31931463,224.66461582)(494.2880647,224.98753183)(494.22556482,225.232323)
\curveto(494.16306495,225.47711417)(494.0719193,225.68805125)(493.95212788,225.86513422)
\curveto(493.83233645,226.0422172)(493.67608677,226.17763359)(493.48337883,226.2713834)
\curveto(493.29587921,226.36513321)(493.07712965,226.41200811)(492.82713016,226.41200811)
\curveto(492.50421414,226.41200811)(492.18129813,226.29742501)(491.85838212,226.06825881)
\curveto(491.5354661,225.83909261)(491.19692512,225.50315579)(490.84275917,225.06044835)
\lineto(490.84275917,220.1463958)
\curveto(490.84275917,220.10472922)(490.83234253,220.06827096)(490.81150924,220.03702102)
\curveto(490.79067595,220.01097941)(490.75682185,219.98754196)(490.70994694,219.96670867)
\curveto(490.66307204,219.94587538)(490.597968,219.93025041)(490.51463484,219.91983376)
\curveto(490.43130167,219.90941712)(490.32453106,219.90420879)(490.19432299,219.90420879)
\curveto(490.06932324,219.90420879)(489.96515678,219.90941712)(489.88182362,219.91983376)
\curveto(489.79849045,219.93025041)(489.73078226,219.94587538)(489.67869903,219.96670867)
\curveto(489.63182413,219.98754196)(489.59797003,220.01097941)(489.57713674,220.03702102)
\curveto(489.56151177,220.06827096)(489.55369928,220.10472922)(489.55369928,220.1463958)
\lineto(489.55369928,230.58387467)
\curveto(489.55369928,230.62554125)(489.56151177,230.66199951)(489.57713674,230.69324944)
\curveto(489.59797003,230.72449938)(489.63182413,230.750541)(489.67869903,230.77137429)
\curveto(489.73078226,230.79220758)(489.79849045,230.80783255)(489.88182362,230.81824919)
\curveto(489.96515678,230.82866584)(490.06932324,230.83387416)(490.19432299,230.83387416)
\curveto(490.32453106,230.83387416)(490.43130167,230.82866584)(490.51463484,230.81824919)
\curveto(490.597968,230.80783255)(490.66307204,230.79220758)(490.70994694,230.77137429)
\curveto(490.75682185,230.750541)(490.79067595,230.72449938)(490.81150924,230.69324944)
\curveto(490.83234253,230.66199951)(490.84275917,230.62554125)(490.84275917,230.58387467)
\lineto(490.84275917,226.37294569)
\curveto(491.21255009,226.7635699)(491.58494517,227.05263182)(491.95994441,227.24013144)
\curveto(492.33494365,227.43283938)(492.71254705,227.52919335)(493.09275462,227.52919335)
\curveto(493.56150367,227.52919335)(493.95473204,227.44846435)(494.27243973,227.28700634)
\curveto(494.59535574,227.13075666)(494.85577188,226.91981959)(495.05368815,226.65419512)
\curveto(495.25160441,226.38857066)(495.39222913,226.07607129)(495.47556229,225.71669702)
\curveto(495.56410378,225.36253107)(495.60837452,224.93284444)(495.60837452,224.42763713)
\closepath
}
}
{
\newrgbcolor{curcolor}{1 1 1}
\pscustom[linestyle=none,fillstyle=solid,fillcolor=curcolor]
{
\newpath
\moveto(505.22591639,220.66201976)
\curveto(505.22591639,220.5109784)(505.21549974,220.39118697)(505.19466645,220.30264549)
\curveto(505.17383316,220.214104)(505.14258322,220.14899996)(505.10091664,220.10733338)
\curveto(505.05925006,220.0656668)(504.99675019,220.02660438)(504.91341702,219.99014612)
\curveto(504.83008386,219.95368786)(504.73372988,219.92504208)(504.62435511,219.90420879)
\curveto(504.52018865,219.87816718)(504.40820971,219.85733389)(504.28841829,219.84170892)
\curveto(504.16862686,219.82608395)(504.04883544,219.81827147)(503.92904401,219.81827147)
\curveto(503.56446142,219.81827147)(503.25196205,219.86514637)(502.99154591,219.95889618)
\curveto(502.73112977,220.05785431)(502.51758854,220.20368735)(502.35092221,220.3963953)
\curveto(502.18425588,220.59431156)(502.06186029,220.84170689)(501.98373545,221.13858129)
\curveto(501.91081893,221.44066401)(501.87436067,221.79482996)(501.87436067,222.20107914)
\lineto(501.87436067,226.31044582)
\lineto(500.88998767,226.31044582)
\curveto(500.81186283,226.31044582)(500.74936295,226.3521124)(500.70248805,226.43544557)
\curveto(500.65561314,226.51877873)(500.63217569,226.65419512)(500.63217569,226.84169474)
\curveto(500.63217569,226.94065288)(500.63738401,227.02398604)(500.64780066,227.09169424)
\curveto(500.66342563,227.15940243)(500.68165476,227.21408982)(500.70248805,227.25575641)
\curveto(500.72332134,227.30263131)(500.74936295,227.33388125)(500.78061289,227.34950622)
\curveto(500.81707115,227.37033951)(500.85613357,227.38075615)(500.89780015,227.38075615)
\lineto(501.87436067,227.38075615)
\lineto(501.87436067,229.05262777)
\curveto(501.87436067,229.08908603)(501.88217316,229.12294012)(501.89779813,229.15419006)
\curveto(501.91863142,229.18544)(501.95248552,229.21148161)(501.99936042,229.2323149)
\curveto(502.05144365,229.25835652)(502.11915185,229.27658565)(502.20248501,229.28700229)
\curveto(502.28581817,229.29741894)(502.38998463,229.30262726)(502.51498438,229.30262726)
\curveto(502.64519245,229.30262726)(502.75196306,229.29741894)(502.83529623,229.28700229)
\curveto(502.91862939,229.27658565)(502.98373343,229.25835652)(503.03060833,229.2323149)
\curveto(503.07748324,229.21148161)(503.11133734,229.18544)(503.13217063,229.15419006)
\curveto(503.15300392,229.12294012)(503.16342056,229.08908603)(503.16342056,229.05262777)
\lineto(503.16342056,227.38075615)
\lineto(504.96810441,227.38075615)
\curveto(505.00977099,227.38075615)(505.04622925,227.37033951)(505.07747919,227.34950622)
\curveto(505.10872913,227.33388125)(505.13477074,227.30263131)(505.15560403,227.25575641)
\curveto(505.18164564,227.21408982)(505.19987477,227.15940243)(505.21029142,227.09169424)
\curveto(505.22070806,227.02398604)(505.22591639,226.94065288)(505.22591639,226.84169474)
\curveto(505.22591639,226.65419512)(505.20247894,226.51877873)(505.15560403,226.43544557)
\curveto(505.10872913,226.3521124)(505.04622925,226.31044582)(504.96810441,226.31044582)
\lineto(503.16342056,226.31044582)
\lineto(503.16342056,222.38857876)
\curveto(503.16342056,221.90420474)(503.23373292,221.53701799)(503.37435764,221.28701849)
\curveto(503.52019068,221.04222732)(503.77800265,220.91983174)(504.14779357,220.91983174)
\curveto(504.267585,220.91983174)(504.37435561,220.93024838)(504.46810542,220.95108167)
\curveto(504.56185523,220.97712329)(504.6451884,221.0031649)(504.71810492,221.02920651)
\curveto(504.79102144,221.05524813)(504.85352131,221.07868558)(504.90560454,221.09951887)
\curveto(504.95768776,221.12556049)(505.00456267,221.13858129)(505.04622925,221.13858129)
\curveto(505.07227087,221.13858129)(505.09570832,221.13076881)(505.11654161,221.11514384)
\curveto(505.14258322,221.1047272)(505.16081235,221.08128974)(505.171229,221.04483148)
\curveto(505.18685397,221.00837322)(505.19987477,220.95889416)(505.21029142,220.89639428)
\curveto(505.22070806,220.83389441)(505.22591639,220.75576957)(505.22591639,220.66201976)
\closepath
}
}
{
\newrgbcolor{curcolor}{1 1 1}
\pscustom[linestyle=none,fillstyle=solid,fillcolor=curcolor]
{
\newpath
\moveto(512.24177895,220.294833)
\curveto(512.28344553,220.22191648)(512.30427882,220.15941661)(512.30427882,220.10733338)
\curveto(512.30427882,220.06045848)(512.27823721,220.02139606)(512.22615398,219.99014612)
\curveto(512.17927908,219.95889618)(512.1037584,219.93806289)(511.99959194,219.92764625)
\curveto(511.90063381,219.91202128)(511.77042574,219.90420879)(511.60896773,219.90420879)
\curveto(511.45271805,219.90420879)(511.3277183,219.90941712)(511.23396849,219.91983376)
\curveto(511.145427,219.92504208)(511.07251049,219.93545873)(511.01521893,219.9510837)
\curveto(510.96313571,219.97191699)(510.92146912,219.99535444)(510.89021919,220.02139606)
\curveto(510.86417757,220.05264599)(510.84074012,220.08650009)(510.81990683,220.12295835)
\lineto(509.13241025,222.9120152)
\lineto(507.45272615,220.12295835)
\curveto(507.43189286,220.08650009)(507.40585124,220.05264599)(507.37460131,220.02139606)
\curveto(507.34855969,219.99535444)(507.30689311,219.97191699)(507.24960156,219.9510837)
\curveto(507.19751833,219.93545873)(507.12720597,219.92504208)(507.03866449,219.91983376)
\curveto(506.950123,219.90941712)(506.83293574,219.90420879)(506.6871027,219.90420879)
\curveto(506.53606134,219.90420879)(506.41106159,219.91202128)(506.31210346,219.92764625)
\curveto(506.21835365,219.93806289)(506.14804129,219.95889618)(506.10116639,219.99014612)
\curveto(506.0594998,220.02139606)(506.03866651,220.06045848)(506.03866651,220.10733338)
\curveto(506.04387483,220.15941661)(506.06991645,220.22191648)(506.11679135,220.294833)
\lineto(508.30428692,223.72451356)
\lineto(506.23397862,227.02919436)
\curveto(506.19231203,227.10211088)(506.16887458,227.1620066)(506.16366626,227.2088815)
\curveto(506.16366626,227.26096473)(506.18710371,227.30263131)(506.23397862,227.33388125)
\curveto(506.28606184,227.37033951)(506.36158252,227.39377696)(506.46054066,227.40419361)
\curveto(506.56470711,227.41461025)(506.70012351,227.41981857)(506.86678984,227.41981857)
\curveto(507.0178312,227.41981857)(507.13762262,227.41461025)(507.22616411,227.40419361)
\curveto(507.31991392,227.39898528)(507.39283044,227.38856864)(507.44491366,227.37294367)
\curveto(507.49699689,227.3573187)(507.53605931,227.33648541)(507.56210093,227.3104438)
\curveto(507.58814254,227.28440218)(507.61157999,227.25315224)(507.63241328,227.21669398)
\lineto(509.23397254,224.59951178)
\lineto(510.85896925,227.21669398)
\curveto(510.87980254,227.24794392)(510.90323999,227.2765897)(510.92928161,227.30263131)
\curveto(510.95532322,227.32867292)(510.98917732,227.34950622)(511.0308439,227.36513118)
\curveto(511.07771881,227.38596448)(511.14021868,227.39898528)(511.21834352,227.40419361)
\curveto(511.30167669,227.41461025)(511.41365563,227.41981857)(511.55428034,227.41981857)
\curveto(511.7053217,227.41981857)(511.82771729,227.41461025)(511.9214671,227.40419361)
\curveto(512.02042523,227.39377696)(512.09334175,227.37294367)(512.14021666,227.34169373)
\curveto(512.18709156,227.31565212)(512.20792485,227.2765897)(512.20271653,227.22450647)
\curveto(512.19750821,227.17242324)(512.17146659,227.10731921)(512.12459169,227.02919436)
\lineto(510.06990835,223.76357598)
\closepath
}
}
{
\newrgbcolor{curcolor}{1 1 1}
\pscustom[linestyle=none,fillstyle=solid,fillcolor=curcolor]
{
\newpath
\moveto(519.71663259,226.85731971)
\curveto(519.71663259,226.67502842)(519.69059097,226.54221618)(519.63850775,226.45888302)
\curveto(519.59163284,226.38075818)(519.52913297,226.34169576)(519.45100813,226.34169576)
\lineto(518.44319767,226.34169576)
\curveto(518.62548896,226.15419614)(518.75309287,225.94586323)(518.82600939,225.71669702)
\curveto(518.89892591,225.49273914)(518.93538417,225.25836462)(518.93538417,225.01357345)
\curveto(518.93538417,224.60732427)(518.87028014,224.24795)(518.74007207,223.93545063)
\curveto(518.609864,223.62295126)(518.42236438,223.3573268)(518.17757321,223.13857724)
\curveto(517.93799036,222.92503601)(517.6515326,222.76097384)(517.31819995,222.64639074)
\curveto(516.98486729,222.53180764)(516.61507637,222.47451609)(516.20882719,222.47451609)
\curveto(515.92236944,222.47451609)(515.64893249,222.51097435)(515.38851635,222.58389087)
\curveto(515.13330854,222.66201571)(514.93539227,222.75836968)(514.79476756,222.87295278)
\curveto(514.70101775,222.77920297)(514.6228929,222.67243235)(514.56039303,222.55264093)
\curveto(514.50310148,222.43284951)(514.4744557,222.29482895)(514.4744557,222.13857927)
\curveto(514.4744557,221.95628797)(514.55778887,221.80524661)(514.7244552,221.68545519)
\curveto(514.89632985,221.56566376)(515.12289189,221.50055973)(515.40414132,221.49014308)
\lineto(517.2400751,221.41201824)
\curveto(517.58903273,221.40160159)(517.90934458,221.35212253)(518.20101066,221.26358104)
\curveto(518.49267673,221.18024788)(518.74528039,221.05785229)(518.95882162,220.89639428)
\curveto(519.17236286,220.7401446)(519.33902919,220.5448325)(519.45882061,220.31045797)
\curveto(519.57861203,220.08129177)(519.63850775,219.81306314)(519.63850775,219.5057721)
\curveto(519.63850775,219.18285609)(519.57079955,218.87556504)(519.43538316,218.58389897)
\curveto(519.29996677,218.29223289)(519.09163385,218.03702507)(518.81038442,217.81827552)
\curveto(518.53434332,217.59431764)(518.18017737,217.41983882)(517.74788658,217.29483908)
\curveto(517.31559578,217.16463101)(516.80518015,217.09952697)(516.21663968,217.09952697)
\curveto(515.64893249,217.09952697)(515.16455847,217.14900604)(514.76351762,217.24796417)
\curveto(514.36768509,217.34171398)(514.04216491,217.47192205)(513.7869571,217.63858838)
\curveto(513.53174928,217.80525471)(513.34685382,218.00577514)(513.23227072,218.24014966)
\curveto(513.11768762,218.46931587)(513.06039607,218.71931536)(513.06039607,218.99014814)
\curveto(513.06039607,219.1620228)(513.08122936,219.32868912)(513.12289594,219.49014713)
\curveto(513.16456252,219.65160514)(513.2270624,219.80525066)(513.31039556,219.9510837)
\curveto(513.39893705,220.09691674)(513.50570767,220.23493729)(513.63070741,220.36514536)
\curveto(513.76091548,220.50056175)(513.90935268,220.63337398)(514.07601901,220.76358205)
\curveto(513.82081119,220.89379012)(513.63070741,221.05785229)(513.50570767,221.25576856)
\curveto(513.38591624,221.45368482)(513.32602053,221.66722606)(513.32602053,221.89639226)
\curveto(513.32602053,222.21409995)(513.39112456,222.49795354)(513.52133263,222.74795303)
\curveto(513.6515407,222.99795253)(513.81299871,223.22191041)(514.00570665,223.41982667)
\curveto(513.84424865,223.61253462)(513.71664474,223.82868001)(513.62289493,224.06826286)
\curveto(513.52914512,224.31305403)(513.48227021,224.60732427)(513.48227021,224.95107357)
\curveto(513.48227021,225.35211443)(513.54997841,225.7114887)(513.6853948,226.02919639)
\curveto(513.82081119,226.34690408)(514.00831081,226.6151327)(514.24789366,226.83388226)
\curveto(514.48747651,227.05263182)(514.77393426,227.21929815)(515.10726692,227.33388125)
\curveto(515.4458079,227.45367267)(515.81299466,227.51356838)(516.20882719,227.51356838)
\curveto(516.42236843,227.51356838)(516.62028469,227.50054758)(516.80257599,227.47450596)
\curveto(516.99007561,227.45367267)(517.16455442,227.42242274)(517.32601243,227.38075615)
\lineto(519.45100813,227.38075615)
\curveto(519.53954961,227.38075615)(519.60465365,227.33648541)(519.64632023,227.24794392)
\curveto(519.69319514,227.16461076)(519.71663259,227.03440269)(519.71663259,226.85731971)
\closepath
\moveto(517.70101167,225.00576096)
\curveto(517.70101167,225.48492666)(517.56819944,225.85732174)(517.30257498,226.1229462)
\curveto(517.04215884,226.39377898)(516.66976376,226.52919538)(516.18538974,226.52919538)
\curveto(515.93539025,226.52919538)(515.71664069,226.48752879)(515.52914107,226.40419563)
\curveto(515.34684977,226.32086247)(515.19320425,226.20627936)(515.0682045,226.06044633)
\curveto(514.94841308,225.91461329)(514.85726743,225.7453428)(514.79476756,225.55263485)
\curveto(514.737476,225.36513523)(514.70883023,225.16721897)(514.70883023,224.95888606)
\curveto(514.70883023,224.49534533)(514.8390383,224.13076273)(515.09945444,223.86513827)
\curveto(515.3650789,223.59951381)(515.73486982,223.46670158)(516.20882719,223.46670158)
\curveto(516.46403501,223.46670158)(516.68538873,223.505764)(516.87288835,223.58388884)
\curveto(517.06038797,223.66722201)(517.21403349,223.77920095)(517.33382491,223.91982566)
\curveto(517.45882466,224.0656587)(517.54997031,224.23232503)(517.60726186,224.41982465)
\curveto(517.66976173,224.60732427)(517.70101167,224.80263637)(517.70101167,225.00576096)
\closepath
\moveto(518.34944786,219.43545974)
\curveto(518.34944786,219.73754246)(518.22444811,219.96931283)(517.97444862,220.13077083)
\curveto(517.72965745,220.29743716)(517.39632479,220.38597865)(516.97445064,220.3963953)
\lineto(515.15414183,220.45889517)
\curveto(514.9874755,220.3286871)(514.84945494,220.20368735)(514.74008017,220.08389593)
\curveto(514.63591371,219.96931283)(514.55258055,219.85993805)(514.49008067,219.75577159)
\curveto(514.4275808,219.64639681)(514.38331006,219.5396262)(514.35726844,219.43545974)
\curveto(514.33643515,219.33129329)(514.3260185,219.22452267)(514.3260185,219.11514789)
\curveto(514.3260185,218.77660691)(514.49789316,218.52139909)(514.84164246,218.34952444)
\curveto(515.18539176,218.17244147)(515.66455746,218.08389998)(516.27913955,218.08389998)
\curveto(516.66976376,218.08389998)(516.99528393,218.1229624)(517.25570007,218.20108724)
\curveto(517.52132453,218.27400376)(517.73486577,218.37296189)(517.89632377,218.49796164)
\curveto(518.05778178,218.62296139)(518.17236488,218.76619026)(518.24007308,218.92764827)
\curveto(518.3129896,219.08910628)(518.34944786,219.25837677)(518.34944786,219.43545974)
\closepath
}
}
{
\newrgbcolor{curcolor}{0.81960785 0.12941177 0.14117648}
\pscustom[linestyle=none,fillstyle=solid,fillcolor=curcolor]
{
\newpath
\moveto(240.87760983,209.79936987)
\lineto(689.69561173,209.79936987)
\lineto(689.69561173,173.01204279)
\lineto(240.87760983,173.01204279)
\closepath
}
}
{
\newrgbcolor{curcolor}{0.10196079 0.8392157 0.96078432}
\pscustom[linewidth=1.99999595,linecolor=curcolor]
{
\newpath
\moveto(240.87760983,209.79936987)
\lineto(689.69561173,209.79936987)
\lineto(689.69561173,173.01204279)
\lineto(240.87760983,173.01204279)
\closepath
}
}
{
\newrgbcolor{curcolor}{1 1 1}
\pscustom[linestyle=none,fillstyle=solid,fillcolor=curcolor]
{
\newpath
\moveto(417.89218391,185.74106547)
\curveto(417.89218391,185.69939889)(417.88176726,185.66294063)(417.86093397,185.63169069)
\curveto(417.84010068,185.60564908)(417.80624658,185.58221163)(417.75937168,185.56137834)
\curveto(417.71249677,185.54054505)(417.64739274,185.52492008)(417.56405957,185.51450343)
\curveto(417.48072641,185.50408679)(417.37395579,185.49887846)(417.24374772,185.49887846)
\curveto(417.11874797,185.49887846)(417.01458152,185.50408679)(416.93124835,185.51450343)
\curveto(416.84791519,185.52492008)(416.78020699,185.54054505)(416.72812377,185.56137834)
\curveto(416.68124886,185.58221163)(416.64739476,185.60564908)(416.62656147,185.63169069)
\curveto(416.6109365,185.66294063)(416.60312402,185.69939889)(416.60312402,185.74106547)
\lineto(416.60312402,196.17854434)
\curveto(416.60312402,196.22021092)(416.6109365,196.25666918)(416.62656147,196.28791912)
\curveto(416.64739476,196.31916905)(416.68124886,196.34521067)(416.72812377,196.36604396)
\curveto(416.78020699,196.38687725)(416.84791519,196.40250222)(416.93124835,196.41291886)
\curveto(417.01458152,196.42333551)(417.11874797,196.42854383)(417.24374772,196.42854383)
\curveto(417.37395579,196.42854383)(417.48072641,196.42333551)(417.56405957,196.41291886)
\curveto(417.64739274,196.40250222)(417.71249677,196.38687725)(417.75937168,196.36604396)
\curveto(417.80624658,196.34521067)(417.84010068,196.31916905)(417.86093397,196.28791912)
\curveto(417.88176726,196.25666918)(417.89218391,196.22021092)(417.89218391,196.17854434)
\closepath
}
}
{
\newrgbcolor{curcolor}{1 1 1}
\pscustom[linestyle=none,fillstyle=solid,fillcolor=curcolor]
{
\newpath
\moveto(426.80616589,189.3348082)
\curveto(426.80616589,188.76189269)(426.73064521,188.23324793)(426.57960385,187.74887391)
\curveto(426.42856249,187.26970821)(426.20200045,186.85564655)(425.89991773,186.50668892)
\curveto(425.60304333,186.1577313)(425.22804409,185.88429435)(424.77492,185.68637808)
\curveto(424.32700425,185.49367014)(423.80617197,185.39731617)(423.21242317,185.39731617)
\curveto(422.63429934,185.39731617)(422.12909203,185.4832535)(421.69680124,185.65512815)
\curveto(421.26971877,185.8270028)(420.91294866,186.07700229)(420.62649091,186.40512663)
\curveto(420.34003315,186.73325096)(420.12649192,187.13168766)(419.9858672,187.60043671)
\curveto(419.84524249,188.06918576)(419.77493013,188.60043468)(419.77493013,189.19418348)
\curveto(419.77493013,189.76709899)(419.84784665,190.29313959)(419.99367969,190.77230528)
\curveto(420.14472105,191.2566793)(420.36867893,191.67334513)(420.66555333,192.02230275)
\curveto(420.96763605,192.37126038)(421.34263529,192.64209317)(421.79055105,192.83480111)
\curveto(422.23846681,193.02750905)(422.76190325,193.12386302)(423.36086037,193.12386302)
\curveto(423.9389842,193.12386302)(424.44158735,193.0379257)(424.86866981,192.86605105)
\curveto(425.30096061,192.69417639)(425.66033488,192.4441769)(425.94679263,192.11605256)
\curveto(426.23325038,191.78792823)(426.44679162,191.38949154)(426.58741633,190.92074248)
\curveto(426.73324937,190.45199343)(426.80616589,189.92334867)(426.80616589,189.3348082)
\closepath
\moveto(425.44679364,189.24887087)
\curveto(425.44679364,189.62907843)(425.41033538,189.98845271)(425.33741887,190.32699369)
\curveto(425.26971067,190.66553467)(425.15512757,190.96240907)(424.99366956,191.21761688)
\curveto(424.83221156,191.4728247)(424.613462,191.67334513)(424.33742089,191.81917817)
\curveto(424.06137978,191.97021953)(423.71763048,192.04574021)(423.30617298,192.04574021)
\curveto(422.92596542,192.04574021)(422.59784108,191.97803201)(422.32179997,191.84261562)
\curveto(422.05096719,191.70719923)(421.82700931,191.51449128)(421.64992633,191.26449179)
\curveto(421.47284336,191.01970062)(421.34003113,190.72803454)(421.25148964,190.38949356)
\curveto(421.16815648,190.05095258)(421.12648989,189.68116166)(421.12648989,189.28012081)
\curveto(421.12648989,188.89470492)(421.16034399,188.53272649)(421.22805219,188.19418551)
\curveto(421.30096871,187.85564452)(421.41815597,187.55877013)(421.57961398,187.30356231)
\curveto(421.7462803,187.05356282)(421.96763402,186.85304239)(422.24367513,186.70200103)
\curveto(422.51971624,186.55616799)(422.86346554,186.48325147)(423.27492304,186.48325147)
\curveto(423.64992228,186.48325147)(423.97544246,186.55095967)(424.25148356,186.68637606)
\curveto(424.52752467,186.82179245)(424.75408671,187.01189623)(424.93116969,187.2566874)
\curveto(425.10825266,187.50147858)(425.23846073,187.79314465)(425.3217939,188.13168563)
\curveto(425.40512706,188.47022661)(425.44679364,188.84262169)(425.44679364,189.24887087)
\closepath
}
}
{
\newrgbcolor{curcolor}{1 1 1}
\pscustom[linestyle=none,fillstyle=solid,fillcolor=curcolor]
{
\newpath
\moveto(434.57408734,192.45198938)
\curveto(434.57408734,192.26969809)(434.54804573,192.13688586)(434.4959625,192.05355269)
\curveto(434.4490876,191.97542785)(434.38658772,191.93636543)(434.30846288,191.93636543)
\lineto(433.30065242,191.93636543)
\curveto(433.48294372,191.74886581)(433.61054763,191.5405329)(433.68346415,191.31136669)
\curveto(433.75638067,191.08740881)(433.79283893,190.85303429)(433.79283893,190.60824312)
\curveto(433.79283893,190.20199394)(433.72773489,189.84261967)(433.59752682,189.5301203)
\curveto(433.46731875,189.21762093)(433.27981913,188.95199647)(433.03502796,188.73324691)
\curveto(432.79544511,188.51970568)(432.50898736,188.35564351)(432.1756547,188.24106041)
\curveto(431.84232204,188.12647731)(431.47253113,188.06918576)(431.06628195,188.06918576)
\curveto(430.77982419,188.06918576)(430.50638725,188.10564402)(430.24597111,188.17856054)
\curveto(429.99076329,188.25668538)(429.79284703,188.35303935)(429.65222231,188.46762245)
\curveto(429.5584725,188.37387264)(429.48034766,188.26710202)(429.41784779,188.1473106)
\curveto(429.36055624,188.02751918)(429.33191046,187.88949862)(429.33191046,187.73324894)
\curveto(429.33191046,187.55095764)(429.41524362,187.39991628)(429.58190995,187.28012486)
\curveto(429.75378461,187.16033343)(429.98034665,187.0952294)(430.26159608,187.08481275)
\lineto(432.09752986,187.00668791)
\curveto(432.44648749,186.99627126)(432.76679934,186.9467922)(433.05846541,186.85825071)
\curveto(433.35013149,186.77491755)(433.60273514,186.65252196)(433.81627638,186.49106395)
\curveto(434.02981761,186.33481427)(434.19648394,186.13950217)(434.31627537,185.90512764)
\curveto(434.43606679,185.67596144)(434.4959625,185.40773281)(434.4959625,185.10044177)
\curveto(434.4959625,184.77752576)(434.42825431,184.47023471)(434.29283791,184.17856864)
\curveto(434.15742152,183.88690256)(433.94908861,183.63169474)(433.66783918,183.41294519)
\curveto(433.39179807,183.18898731)(433.03763212,183.01450849)(432.60534133,182.88950875)
\curveto(432.17305054,182.75930068)(431.66263491,182.69419664)(431.07409443,182.69419664)
\curveto(430.50638725,182.69419664)(430.02201323,182.74367571)(429.62097237,182.84263384)
\curveto(429.22513984,182.93638365)(428.89961967,183.06659172)(428.64441185,183.23325805)
\curveto(428.38920404,183.39992438)(428.20430858,183.60044481)(428.08972548,183.83481933)
\curveto(427.97514237,184.06398554)(427.91785082,184.31398503)(427.91785082,184.58481781)
\curveto(427.91785082,184.75669247)(427.93868411,184.9233588)(427.9803507,185.0848168)
\curveto(428.02201728,185.24627481)(428.08451715,185.39992033)(428.16785032,185.54575337)
\curveto(428.2563918,185.69158641)(428.36316242,185.82960696)(428.48816217,185.95981503)
\curveto(428.61837024,186.09523142)(428.76680744,186.22804365)(428.93347377,186.35825172)
\curveto(428.67826595,186.48845979)(428.48816217,186.65252196)(428.36316242,186.85043823)
\curveto(428.243371,187.04835449)(428.18347529,187.26189573)(428.18347529,187.49106193)
\curveto(428.18347529,187.80876962)(428.24857932,188.09262321)(428.37878739,188.34262271)
\curveto(428.50899546,188.5926222)(428.67045347,188.81658008)(428.86316141,189.01449634)
\curveto(428.7017034,189.20720429)(428.57409949,189.42334968)(428.48034968,189.66293253)
\curveto(428.38659987,189.9077237)(428.33972497,190.20199394)(428.33972497,190.54574324)
\curveto(428.33972497,190.9467841)(428.40743317,191.30615837)(428.54284956,191.62386606)
\curveto(428.67826595,191.94157375)(428.86576557,192.20980237)(429.10534842,192.42855193)
\curveto(429.34493127,192.64730149)(429.63138902,192.81396782)(429.96472168,192.92855092)
\curveto(430.30326266,193.04834234)(430.67044942,193.10823805)(431.06628195,193.10823805)
\curveto(431.27982318,193.10823805)(431.47773945,193.09521725)(431.66003075,193.06917563)
\curveto(431.84753037,193.04834234)(432.02200918,193.01709241)(432.18346719,192.97542582)
\lineto(434.30846288,192.97542582)
\curveto(434.39700437,192.97542582)(434.4621084,192.93115508)(434.50377499,192.84261359)
\curveto(434.55064989,192.75928043)(434.57408734,192.62907236)(434.57408734,192.45198938)
\closepath
\moveto(432.55846643,190.60043063)
\curveto(432.55846643,191.07959633)(432.4256542,191.45199141)(432.16002973,191.71761587)
\curveto(431.89961359,191.98844866)(431.52721851,192.12386505)(431.0428445,192.12386505)
\curveto(430.792845,192.12386505)(430.57409544,192.08219847)(430.38659582,191.9988653)
\curveto(430.20430453,191.91553214)(430.050659,191.80094904)(429.92565926,191.655116)
\curveto(429.80586783,191.50928296)(429.71472218,191.34001247)(429.65222231,191.14730453)
\curveto(429.59493076,190.95980491)(429.56628499,190.76188864)(429.56628499,190.55355573)
\curveto(429.56628499,190.090015)(429.69649306,189.7254324)(429.95690919,189.45980794)
\curveto(430.22253366,189.19418348)(430.59232457,189.06137125)(431.06628195,189.06137125)
\curveto(431.32148976,189.06137125)(431.54284348,189.10043367)(431.7303431,189.17855851)
\curveto(431.91784272,189.26189168)(432.07148825,189.37387062)(432.19127967,189.51449533)
\curveto(432.31627942,189.66032837)(432.40742507,189.8269947)(432.46471662,190.01449432)
\curveto(432.52721649,190.20199394)(432.55846643,190.39730604)(432.55846643,190.60043063)
\closepath
\moveto(433.20690261,185.03012941)
\curveto(433.20690261,185.33221213)(433.08190287,185.5639825)(432.83190337,185.7254405)
\curveto(432.5871122,185.89210683)(432.25377954,185.98064832)(431.8319054,185.99106497)
\lineto(430.01159658,186.05356484)
\curveto(429.84493025,185.92335677)(429.7069097,185.79835702)(429.59753492,185.6785656)
\curveto(429.49336847,185.5639825)(429.4100353,185.45460772)(429.34753543,185.35044126)
\curveto(429.28503556,185.24106649)(429.24076481,185.13429587)(429.2147232,185.03012941)
\curveto(429.19388991,184.92596296)(429.18347326,184.81919234)(429.18347326,184.70981756)
\curveto(429.18347326,184.37127658)(429.35534791,184.11606876)(429.69909722,183.94419411)
\curveto(430.04284652,183.76711114)(430.52201222,183.67856965)(431.13659431,183.67856965)
\curveto(431.52721851,183.67856965)(431.85273869,183.71763207)(432.11315483,183.79575691)
\curveto(432.37877929,183.86867343)(432.59232052,183.96763156)(432.75377853,184.09263131)
\curveto(432.91523654,184.21763106)(433.02981964,184.36085993)(433.09752783,184.52231794)
\curveto(433.17044435,184.68377595)(433.20690261,184.85304644)(433.20690261,185.03012941)
\closepath
}
}
{
\newrgbcolor{curcolor}{1 1 1}
\pscustom[linestyle=none,fillstyle=solid,fillcolor=curcolor]
{
\newpath
\moveto(437.50814341,185.74106547)
\curveto(437.50814341,185.69939889)(437.49772676,185.66294063)(437.47689347,185.63169069)
\curveto(437.45606018,185.60564908)(437.42220608,185.58221163)(437.37533118,185.56137834)
\curveto(437.32845627,185.54054505)(437.26335224,185.52492008)(437.18001907,185.51450343)
\curveto(437.09668591,185.50408679)(436.98991529,185.49887846)(436.85970722,185.49887846)
\curveto(436.73470747,185.49887846)(436.63054102,185.50408679)(436.54720785,185.51450343)
\curveto(436.46387469,185.52492008)(436.39616649,185.54054505)(436.34408326,185.56137834)
\curveto(436.29720836,185.58221163)(436.26335426,185.60564908)(436.24252097,185.63169069)
\curveto(436.226896,185.66294063)(436.21908352,185.69939889)(436.21908352,185.74106547)
\lineto(436.21908352,192.77230123)
\curveto(436.21908352,192.80875949)(436.226896,192.84261359)(436.24252097,192.87386353)
\curveto(436.26335426,192.90511347)(436.29720836,192.93115508)(436.34408326,192.95198837)
\curveto(436.39616649,192.97282166)(436.46387469,192.98844663)(436.54720785,192.99886328)
\curveto(436.63054102,193.00927992)(436.73470747,193.01448824)(436.85970722,193.01448824)
\curveto(436.98991529,193.01448824)(437.09668591,193.00927992)(437.18001907,192.99886328)
\curveto(437.26335224,192.98844663)(437.32845627,192.97282166)(437.37533118,192.95198837)
\curveto(437.42220608,192.93115508)(437.45606018,192.90511347)(437.47689347,192.87386353)
\curveto(437.49772676,192.84261359)(437.50814341,192.80875949)(437.50814341,192.77230123)
\closepath
\moveto(437.65658061,195.14729643)
\curveto(437.65658061,194.8452137)(437.59928906,194.63948495)(437.48470595,194.53011018)
\curveto(437.37012285,194.4207354)(437.15918578,194.36604801)(436.85189474,194.36604801)
\curveto(436.54981201,194.36604801)(436.3414791,194.41813124)(436.226896,194.52229769)
\curveto(436.11752122,194.63167247)(436.06283383,194.83479706)(436.06283383,195.13167146)
\curveto(436.06283383,195.43375418)(436.12012538,195.63948293)(436.23470849,195.74885771)
\curveto(436.34929159,195.85823249)(436.56022866,195.91291988)(436.8675197,195.91291988)
\curveto(437.16960243,195.91291988)(437.37533118,195.85823249)(437.48470595,195.74885771)
\curveto(437.59928906,195.64469125)(437.65658061,195.44417082)(437.65658061,195.14729643)
\closepath
}
}
{
\newrgbcolor{curcolor}{1 1 1}
\pscustom[linestyle=none,fillstyle=solid,fillcolor=curcolor]
{
\newpath
\moveto(444.93775483,186.81137581)
\curveto(444.93775483,186.72283432)(444.93515067,186.64470948)(444.92994234,186.57700128)
\curveto(444.92473402,186.51450141)(444.91431737,186.45981402)(444.89869241,186.41293911)
\curveto(444.88827576,186.37127253)(444.87265079,186.33221011)(444.8518175,186.29575185)
\curveto(444.83619253,186.26450191)(444.79452595,186.21502285)(444.72681775,186.14731465)
\curveto(444.66431788,186.08481478)(444.5549431,186.00408577)(444.39869342,185.90512764)
\curveto(444.24244374,185.81137783)(444.06536076,185.7254405)(443.86744449,185.64731566)
\curveto(443.67473655,185.57439914)(443.46379948,185.51450343)(443.23463328,185.46762853)
\curveto(443.00546707,185.42075362)(442.76848839,185.39731617)(442.52369722,185.39731617)
\curveto(442.01848991,185.39731617)(441.57057415,185.48064933)(441.17994994,185.64731566)
\curveto(440.78932573,185.81398199)(440.46120139,186.056169)(440.19557693,186.37387669)
\curveto(439.93516079,186.6967927)(439.73464036,187.09002107)(439.59401565,187.5535618)
\curveto(439.45859926,188.02231085)(439.39089106,188.56137226)(439.39089106,189.17074603)
\curveto(439.39089106,189.86345296)(439.47422422,190.45720176)(439.64089055,190.95199242)
\curveto(439.81276521,191.45199141)(440.04453557,191.86084475)(440.33620165,192.17855244)
\curveto(440.63307604,192.49626013)(440.97942951,192.73063465)(441.37526204,192.88167601)
\curveto(441.7763029,193.0379257)(442.20859369,193.11605054)(442.67213442,193.11605054)
\curveto(442.89609229,193.11605054)(443.11223769,193.09521725)(443.3205706,193.05355067)
\curveto(443.53411184,193.01188408)(443.72942394,192.95719669)(443.90650692,192.8894885)
\curveto(444.08358989,192.8217803)(444.23983957,192.74365546)(444.37525597,192.65511397)
\curveto(444.51588068,192.56657248)(444.61744298,192.4910518)(444.67994285,192.42855193)
\curveto(444.74244272,192.36605206)(444.7841093,192.31657299)(444.8049426,192.28011473)
\curveto(444.83098421,192.24365647)(444.8518175,192.19938573)(444.86744247,192.1473025)
\curveto(444.88306744,192.1004276)(444.89348408,192.04574021)(444.89869241,191.98324033)
\curveto(444.90390073,191.92074046)(444.90650489,191.84261562)(444.90650489,191.74886581)
\curveto(444.90650489,191.54574122)(444.88306744,191.40251234)(444.83619253,191.31917918)
\curveto(444.78931763,191.24105434)(444.73202608,191.20199191)(444.66431788,191.20199191)
\curveto(444.58619304,191.20199191)(444.49504739,191.2436585)(444.39088093,191.32699166)
\curveto(444.2919228,191.41553315)(444.16431889,191.51188712)(444.00806921,191.61605358)
\curveto(443.85181953,191.72022003)(443.66171574,191.81396984)(443.43775786,191.89730301)
\curveto(443.21900831,191.98584449)(442.95859217,192.03011524)(442.65650945,192.03011524)
\curveto(442.03671904,192.03011524)(441.5601575,191.79053239)(441.22682484,191.31136669)
\curveto(440.89870051,190.83740932)(440.73463834,190.14730655)(440.73463834,189.24105839)
\curveto(440.73463834,188.7879343)(440.77630492,188.38949761)(440.85963809,188.04574831)
\curveto(440.94817957,187.70720733)(441.07578348,187.42335373)(441.24244981,187.19418753)
\curveto(441.40911614,186.96502133)(441.61224073,186.79314668)(441.85182358,186.67856357)
\curveto(442.09661475,186.5691888)(442.37526002,186.51450141)(442.68775938,186.51450141)
\curveto(442.98463378,186.51450141)(443.24504992,186.56137631)(443.4690078,186.65512612)
\curveto(443.69296568,186.74887593)(443.88567362,186.85043823)(444.04713163,186.95981301)
\curveto(444.21379796,187.07439611)(444.35181851,187.1759584)(444.46119329,187.26449989)
\curveto(444.57577639,187.3582497)(444.66431788,187.4051246)(444.72681775,187.4051246)
\curveto(444.76327601,187.4051246)(444.79452595,187.39470796)(444.82056756,187.37387467)
\curveto(444.84660918,187.35304138)(444.86744247,187.31658312)(444.88306744,187.26449989)
\curveto(444.90390073,187.21762498)(444.91692154,187.15512511)(444.92212986,187.07700027)
\curveto(444.9325465,187.00408375)(444.93775483,186.91554226)(444.93775483,186.81137581)
\closepath
}
}
{
\newrgbcolor{curcolor}{1 1 1}
\pscustom[linestyle=none,fillstyle=solid,fillcolor=curcolor]
{
\newpath
\moveto(451.80392707,185.73325299)
\curveto(451.80392707,185.67075312)(451.78309377,185.62387821)(451.74142719,185.59262827)
\curveto(451.69976061,185.56137834)(451.64246906,185.53794088)(451.56955254,185.52231592)
\curveto(451.49663602,185.50669095)(451.3898654,185.49887846)(451.24924069,185.49887846)
\curveto(451.1138243,185.49887846)(451.00444952,185.50669095)(450.92111635,185.52231592)
\curveto(450.84299151,185.53794088)(450.78569996,185.56137834)(450.7492417,185.59262827)
\curveto(450.71278344,185.62387821)(450.69455431,185.67075312)(450.69455431,185.73325299)
\lineto(450.69455431,186.43637657)
\curveto(450.38726327,186.10825223)(450.04351396,185.85304441)(449.6633064,185.67075312)
\curveto(449.28830716,185.48846182)(448.88987047,185.39731617)(448.46799632,185.39731617)
\curveto(448.0982054,185.39731617)(447.76226858,185.44679524)(447.46018586,185.54575337)
\curveto(447.16331146,185.63950318)(446.90810365,185.77752373)(446.69456241,185.95981503)
\curveto(446.4862295,186.14210633)(446.32216733,186.36606421)(446.20237591,186.63168867)
\curveto(446.08779281,186.89731313)(446.03050126,187.19939585)(446.03050126,187.53793683)
\curveto(446.03050126,187.93376937)(446.11123026,188.27751867)(446.27268827,188.56918475)
\curveto(446.43414627,188.86085082)(446.66591664,189.10303783)(446.96799936,189.29574578)
\curveto(447.27008208,189.48845372)(447.639873,189.63168259)(448.07737211,189.7254324)
\curveto(448.51487123,189.82439054)(449.00705773,189.8738696)(449.55393162,189.8738696)
\lineto(450.52267966,189.8738696)
\lineto(450.52267966,190.4207435)
\curveto(450.52267966,190.69157628)(450.49403388,190.93115913)(450.43674233,191.13949204)
\curveto(450.37945078,191.34782495)(450.28570097,191.5196996)(450.1554929,191.655116)
\curveto(450.03049316,191.79574071)(449.86643099,191.89990717)(449.6633064,191.96761536)
\curveto(449.46018181,192.04053188)(449.21018232,192.07699014)(448.91330792,192.07699014)
\curveto(448.59560023,192.07699014)(448.30914248,192.03792772)(448.05393466,191.95980288)
\curveto(447.80393517,191.88688636)(447.58258145,191.8035532)(447.3898735,191.70980339)
\curveto(447.20237388,191.6212619)(447.04352004,191.53792873)(446.91331197,191.45980389)
\curveto(446.78831222,191.38688737)(446.69456241,191.35042911)(446.63206254,191.35042911)
\curveto(446.59039596,191.35042911)(446.5539377,191.36084576)(446.52268776,191.38167905)
\curveto(446.49143782,191.40251234)(446.46279205,191.43376228)(446.43675043,191.47542886)
\curveto(446.41591714,191.51709544)(446.40029217,191.56917867)(446.38987553,191.63167854)
\curveto(446.37945888,191.69938674)(446.37425056,191.77230326)(446.37425056,191.8504281)
\curveto(446.37425056,191.98063617)(446.38206304,192.08219847)(446.39768801,192.15511498)
\curveto(446.4185213,192.23323983)(446.46279205,192.30615635)(446.53050024,192.37386454)
\curveto(446.60341676,192.44157274)(446.72581235,192.51969758)(446.897687,192.60823907)
\curveto(447.06956165,192.70198888)(447.26747792,192.78532204)(447.4914358,192.85823856)
\curveto(447.71539368,192.9363634)(447.96018485,192.99886328)(448.22580931,193.04573818)
\curveto(448.49143377,193.09782141)(448.7596624,193.12386302)(449.03049518,193.12386302)
\curveto(449.53570249,193.12386302)(449.96538912,193.06657147)(450.31955507,192.95198837)
\curveto(450.67372102,192.83740527)(450.96017877,192.66813478)(451.17892833,192.4441769)
\curveto(451.39767789,192.22542734)(451.55653173,191.9519904)(451.65548987,191.62386606)
\curveto(451.754448,191.29574173)(451.80392707,190.91293)(451.80392707,190.47543089)
\closepath
\moveto(450.52267966,188.94418399)
\lineto(449.42111939,188.94418399)
\curveto(449.06695344,188.94418399)(448.7596624,188.91293405)(448.49924626,188.85043418)
\curveto(448.23883012,188.79314263)(448.02268472,188.70460114)(447.85081007,188.58480971)
\curveto(447.67893542,188.47022661)(447.55133151,188.3296019)(447.46799835,188.16293557)
\curveto(447.3898735,188.00147756)(447.35081108,187.81397794)(447.35081108,187.60043671)
\curveto(447.35081108,187.23585411)(447.46539418,186.94418804)(447.69456039,186.72543848)
\curveto(447.92893491,186.51189725)(448.25445509,186.40512663)(448.67112091,186.40512663)
\curveto(449.00966189,186.40512663)(449.32216126,186.49106395)(449.60861901,186.66293861)
\curveto(449.90028509,186.83481326)(450.20497197,187.09783356)(450.52267966,187.45199951)
\closepath
}
}
{
\newrgbcolor{curcolor}{1 1 1}
\pscustom[linestyle=none,fillstyle=solid,fillcolor=curcolor]
{
\newpath
\moveto(455.46010863,185.74106547)
\curveto(455.46010863,185.69939889)(455.44969198,185.66294063)(455.42885869,185.63169069)
\curveto(455.4080254,185.60564908)(455.3741713,185.58221163)(455.3272964,185.56137834)
\curveto(455.28042149,185.54054505)(455.21531746,185.52492008)(455.13198429,185.51450343)
\curveto(455.04865113,185.50408679)(454.94188051,185.49887846)(454.81167244,185.49887846)
\curveto(454.68667269,185.49887846)(454.58250624,185.50408679)(454.49917307,185.51450343)
\curveto(454.41583991,185.52492008)(454.34813171,185.54054505)(454.29604848,185.56137834)
\curveto(454.24917358,185.58221163)(454.21531948,185.60564908)(454.19448619,185.63169069)
\curveto(454.17886122,185.66294063)(454.17104874,185.69939889)(454.17104874,185.74106547)
\lineto(454.17104874,196.17854434)
\curveto(454.17104874,196.22021092)(454.17886122,196.25666918)(454.19448619,196.28791912)
\curveto(454.21531948,196.31916905)(454.24917358,196.34521067)(454.29604848,196.36604396)
\curveto(454.34813171,196.38687725)(454.41583991,196.40250222)(454.49917307,196.41291886)
\curveto(454.58250624,196.42333551)(454.68667269,196.42854383)(454.81167244,196.42854383)
\curveto(454.94188051,196.42854383)(455.04865113,196.42333551)(455.13198429,196.41291886)
\curveto(455.21531746,196.40250222)(455.28042149,196.38687725)(455.3272964,196.36604396)
\curveto(455.3741713,196.34521067)(455.4080254,196.31916905)(455.42885869,196.28791912)
\curveto(455.44969198,196.25666918)(455.46010863,196.22021092)(455.46010863,196.17854434)
\closepath
}
}
{
\newrgbcolor{curcolor}{1 1 1}
\pscustom[linestyle=none,fillstyle=solid,fillcolor=curcolor]
{
\newpath
\moveto(467.91976967,189.35043316)
\curveto(467.91976967,188.7410594)(467.85206148,188.19158134)(467.71664509,187.701999)
\curveto(467.58643702,187.21762498)(467.39112491,186.80356332)(467.13070877,186.45981402)
\curveto(466.87550096,186.11606471)(466.56039743,185.85304441)(466.18539819,185.67075312)
\curveto(465.81039895,185.48846182)(465.38071232,185.39731617)(464.8963383,185.39731617)
\curveto(464.67238042,185.39731617)(464.46404751,185.42075362)(464.27133956,185.46762853)
\curveto(464.08383994,185.50929511)(463.89894448,185.57960747)(463.71665319,185.6785656)
\curveto(463.53436189,185.77752373)(463.35207059,185.90252348)(463.16977929,186.05356484)
\curveto(462.987488,186.2046062)(462.79478005,186.3868975)(462.59165546,186.60043873)
\lineto(462.59165546,185.74106547)
\curveto(462.59165546,185.69939889)(462.58123882,185.66294063)(462.56040553,185.63169069)
\curveto(462.53957224,185.60044076)(462.50571814,185.57439914)(462.45884323,185.55356585)
\curveto(462.41196833,185.53794088)(462.35207262,185.52492008)(462.2791561,185.51450343)
\curveto(462.2114479,185.50408679)(462.12290641,185.49887846)(462.01353163,185.49887846)
\curveto(461.90936518,185.49887846)(461.82082369,185.50408679)(461.74790717,185.51450343)
\curveto(461.67499065,185.52492008)(461.61509494,185.53794088)(461.56822004,185.55356585)
\curveto(461.52134513,185.57439914)(461.49009519,185.60044076)(461.47447023,185.63169069)
\curveto(461.45884526,185.66294063)(461.45103277,185.69939889)(461.45103277,185.74106547)
\lineto(461.45103277,196.17854434)
\curveto(461.45103277,196.22021092)(461.45884526,196.25666918)(461.47447023,196.28791912)
\curveto(461.49530352,196.31916905)(461.52915762,196.34521067)(461.57603252,196.36604396)
\curveto(461.62811575,196.38687725)(461.69582394,196.40250222)(461.77915711,196.41291886)
\curveto(461.86249027,196.42333551)(461.96665673,196.42854383)(462.09165648,196.42854383)
\curveto(462.22186455,196.42854383)(462.32863516,196.42333551)(462.41196833,196.41291886)
\curveto(462.49530149,196.40250222)(462.56040553,196.38687725)(462.60728043,196.36604396)
\curveto(462.65415534,196.34521067)(462.68800944,196.31916905)(462.70884273,196.28791912)
\curveto(462.72967602,196.25666918)(462.74009266,196.22021092)(462.74009266,196.17854434)
\lineto(462.74009266,191.96761536)
\curveto(462.94842557,192.1811566)(463.148946,192.36084373)(463.34165395,192.50667677)
\curveto(463.53957021,192.65250981)(463.73227815,192.76969707)(463.91977777,192.85823856)
\curveto(464.10727739,192.95198837)(464.29477702,193.01969657)(464.48227664,193.06136315)
\curveto(464.66977626,193.10302973)(464.86769252,193.12386302)(465.07602543,193.12386302)
\curveto(465.58644107,193.12386302)(466.02133602,193.02230073)(466.38071029,192.81917614)
\curveto(466.74529289,192.61605155)(467.03956312,192.34261461)(467.263521,191.9988653)
\curveto(467.49268721,191.66032432)(467.65935354,191.26188763)(467.76351999,190.80355522)
\curveto(467.86768645,190.34522282)(467.91976967,189.8608488)(467.91976967,189.35043316)
\closepath
\moveto(466.56039743,189.20199596)
\curveto(466.56039743,189.56137024)(466.53175165,189.91032786)(466.4744601,190.24886884)
\curveto(466.42237687,190.58740983)(466.32862706,190.88688839)(466.19321067,191.14730453)
\curveto(466.05779428,191.40772066)(465.87810714,191.61605358)(465.65414926,191.77230326)
\curveto(465.43019138,191.93376127)(465.15154611,192.01449027)(464.81821346,192.01449027)
\curveto(464.65154713,192.01449027)(464.48748496,191.99105282)(464.32602695,191.94417791)
\curveto(464.16456895,191.89730301)(464.00050678,191.81917817)(463.83384045,191.70980339)
\curveto(463.66717412,191.60042861)(463.49269531,191.45980389)(463.31040401,191.28792924)
\curveto(463.13332103,191.11605459)(462.94321725,190.89990919)(462.74009266,190.63949305)
\lineto(462.74009266,187.83481123)
\curveto(463.09425861,187.40252044)(463.43279959,187.07179195)(463.75571561,186.84262574)
\curveto(464.07863162,186.61866786)(464.41456844,186.50668892)(464.76352607,186.50668892)
\curveto(465.08644208,186.50668892)(465.36248319,186.58481376)(465.59164939,186.74106345)
\curveto(465.82081559,186.89731313)(466.00571105,187.10304188)(466.14633577,187.3582497)
\curveto(466.2921688,187.61866584)(466.39633526,187.90772775)(466.45883513,188.22543544)
\curveto(466.52654333,188.54835146)(466.56039743,188.87387163)(466.56039743,189.20199596)
\closepath
}
}
{
\newrgbcolor{curcolor}{1 1 1}
\pscustom[linestyle=none,fillstyle=solid,fillcolor=curcolor]
{
\newpath
\moveto(471.14007718,185.74106547)
\curveto(471.14007718,185.69939889)(471.12966053,185.66294063)(471.10882724,185.63169069)
\curveto(471.08799395,185.60564908)(471.05413985,185.58221163)(471.00726495,185.56137834)
\curveto(470.96039004,185.54054505)(470.89528601,185.52492008)(470.81195284,185.51450343)
\curveto(470.72861968,185.50408679)(470.62184906,185.49887846)(470.49164099,185.49887846)
\curveto(470.36664125,185.49887846)(470.26247479,185.50408679)(470.17914163,185.51450343)
\curveto(470.09580846,185.52492008)(470.02810026,185.54054505)(469.97601704,185.56137834)
\curveto(469.92914213,185.58221163)(469.89528803,185.60564908)(469.87445474,185.63169069)
\curveto(469.85882977,185.66294063)(469.85101729,185.69939889)(469.85101729,185.74106547)
\lineto(469.85101729,192.77230123)
\curveto(469.85101729,192.80875949)(469.85882977,192.84261359)(469.87445474,192.87386353)
\curveto(469.89528803,192.90511347)(469.92914213,192.93115508)(469.97601704,192.95198837)
\curveto(470.02810026,192.97282166)(470.09580846,192.98844663)(470.17914163,192.99886328)
\curveto(470.26247479,193.00927992)(470.36664125,193.01448824)(470.49164099,193.01448824)
\curveto(470.62184906,193.01448824)(470.72861968,193.00927992)(470.81195284,192.99886328)
\curveto(470.89528601,192.98844663)(470.96039004,192.97282166)(471.00726495,192.95198837)
\curveto(471.05413985,192.93115508)(471.08799395,192.90511347)(471.10882724,192.87386353)
\curveto(471.12966053,192.84261359)(471.14007718,192.80875949)(471.14007718,192.77230123)
\closepath
\moveto(471.28851438,195.14729643)
\curveto(471.28851438,194.8452137)(471.23122283,194.63948495)(471.11663973,194.53011018)
\curveto(471.00205663,194.4207354)(470.79111955,194.36604801)(470.48382851,194.36604801)
\curveto(470.18174579,194.36604801)(469.97341288,194.41813124)(469.85882977,194.52229769)
\curveto(469.749455,194.63167247)(469.69476761,194.83479706)(469.69476761,195.13167146)
\curveto(469.69476761,195.43375418)(469.75205916,195.63948293)(469.86664226,195.74885771)
\curveto(469.98122536,195.85823249)(470.19216243,195.91291988)(470.49945348,195.91291988)
\curveto(470.8015362,195.91291988)(471.00726495,195.85823249)(471.11663973,195.74885771)
\curveto(471.23122283,195.64469125)(471.28851438,195.44417082)(471.28851438,195.14729643)
\closepath
}
}
{
\newrgbcolor{curcolor}{1 1 1}
\pscustom[linestyle=none,fillstyle=solid,fillcolor=curcolor]
{
\newpath
\moveto(477.6243742,192.31136467)
\curveto(477.6243742,192.19678157)(477.62177004,192.1004276)(477.61656171,192.02230275)
\curveto(477.61135339,191.94417791)(477.60093675,191.88167804)(477.58531178,191.83480313)
\curveto(477.56968681,191.79313655)(477.54885352,191.75928245)(477.5228119,191.73324084)
\curveto(477.50197861,191.71240755)(477.47072868,191.7019909)(477.42906209,191.7019909)
\curveto(477.38739551,191.7019909)(477.33531228,191.71240755)(477.27281241,191.73324084)
\curveto(477.21552086,191.75928245)(477.14781266,191.78271991)(477.06968782,191.8035532)
\curveto(476.9967713,191.82959481)(476.91343814,191.85303226)(476.81968833,191.87386555)
\curveto(476.72593852,191.89469885)(476.62437622,191.90511549)(476.51500145,191.90511549)
\curveto(476.38479338,191.90511549)(476.25718947,191.87907388)(476.13218972,191.82699065)
\curveto(476.00718997,191.77490742)(475.87437774,191.6889701)(475.73375303,191.56917867)
\curveto(475.59833663,191.44938725)(475.45510776,191.2905334)(475.3040664,191.09261714)
\curveto(475.15302504,190.89470087)(474.98635871,190.65251386)(474.80406741,190.36605611)
\lineto(474.80406741,185.74106547)
\curveto(474.80406741,185.69939889)(474.79365076,185.66294063)(474.77281747,185.63169069)
\curveto(474.75198418,185.60564908)(474.71813008,185.58221163)(474.67125518,185.56137834)
\curveto(474.62438027,185.54054505)(474.55927624,185.52492008)(474.47594307,185.51450343)
\curveto(474.39260991,185.50408679)(474.28583929,185.49887846)(474.15563122,185.49887846)
\curveto(474.03063148,185.49887846)(473.92646502,185.50408679)(473.84313186,185.51450343)
\curveto(473.75979869,185.52492008)(473.69209049,185.54054505)(473.64000727,185.56137834)
\curveto(473.59313236,185.58221163)(473.55927826,185.60564908)(473.53844497,185.63169069)
\curveto(473.52282,185.66294063)(473.51500752,185.69939889)(473.51500752,185.74106547)
\lineto(473.51500752,192.77230123)
\curveto(473.51500752,192.81396782)(473.52282,192.84782192)(473.53844497,192.87386353)
\curveto(473.55406994,192.90511347)(473.58531988,192.93115508)(473.63219478,192.95198837)
\curveto(473.67906969,192.97802999)(473.7389654,192.99365495)(473.81188192,192.99886328)
\curveto(473.88479844,193.00927992)(473.98115241,193.01448824)(474.10094383,193.01448824)
\curveto(474.21552693,193.01448824)(474.30927675,193.00927992)(474.38219326,192.99886328)
\curveto(474.46031811,192.99365495)(474.52021382,192.97802999)(474.5618804,192.95198837)
\curveto(474.60354698,192.93115508)(474.63219276,192.90511347)(474.64781773,192.87386353)
\curveto(474.66865102,192.84782192)(474.67906766,192.81396782)(474.67906766,192.77230123)
\lineto(474.67906766,191.74886581)
\curveto(474.87177561,192.03011524)(475.05146274,192.25928144)(475.21812907,192.43636442)
\curveto(475.39000372,192.61344739)(475.55146173,192.75146794)(475.70250309,192.85042608)
\curveto(475.85354445,192.95459253)(476.00198165,193.02490489)(476.14781469,193.06136315)
\curveto(476.29885605,193.10302973)(476.44989741,193.12386302)(476.60093877,193.12386302)
\curveto(476.66864697,193.12386302)(476.74416765,193.1186547)(476.82750081,193.10823805)
\curveto(476.9160423,193.10302973)(477.00718795,193.09000892)(477.10093776,193.06917563)
\curveto(477.19468757,193.04834234)(477.27802073,193.02490489)(477.35093725,192.99886328)
\curveto(477.42906209,192.97282166)(477.48374948,192.94678005)(477.51499942,192.92073843)
\curveto(477.54624936,192.89469682)(477.56708265,192.86865521)(477.57749929,192.84261359)
\curveto(477.58791594,192.8217803)(477.59572842,192.79313453)(477.60093675,192.75667627)
\curveto(477.61135339,192.72021801)(477.61656171,192.66553062)(477.61656171,192.5926141)
\curveto(477.62177004,192.5249059)(477.6243742,192.43115609)(477.6243742,192.31136467)
\closepath
}
}
{
\newrgbcolor{curcolor}{1 1 1}
\pscustom[linestyle=none,fillstyle=solid,fillcolor=curcolor]
{
\newpath
\moveto(482.73923963,186.25668943)
\curveto(482.73923963,186.10564807)(482.72882299,185.98585664)(482.7079897,185.89731516)
\curveto(482.68715641,185.80877367)(482.65590647,185.74366963)(482.61423989,185.70200305)
\curveto(482.57257331,185.66033647)(482.51007343,185.62127405)(482.42674027,185.58481579)
\curveto(482.3434071,185.54835753)(482.24705313,185.51971175)(482.13767835,185.49887846)
\curveto(482.0335119,185.47283685)(481.92153296,185.45200356)(481.80174153,185.43637859)
\curveto(481.68195011,185.42075362)(481.56215868,185.41294114)(481.44236726,185.41294114)
\curveto(481.07778467,185.41294114)(480.7652853,185.45981604)(480.50486916,185.55356585)
\curveto(480.24445302,185.65252399)(480.03091179,185.79835702)(479.86424546,185.99106497)
\curveto(479.69757913,186.18898123)(479.57518354,186.43637657)(479.4970587,186.73325096)
\curveto(479.42414218,187.03533369)(479.38768392,187.38949964)(479.38768392,187.79574881)
\lineto(479.38768392,191.90511549)
\lineto(478.40331091,191.90511549)
\curveto(478.32518607,191.90511549)(478.2626862,191.94678207)(478.21581129,192.03011524)
\curveto(478.16893639,192.1134484)(478.14549894,192.24886479)(478.14549894,192.43636442)
\curveto(478.14549894,192.53532255)(478.15070726,192.61865571)(478.16112391,192.68636391)
\curveto(478.17674887,192.75407211)(478.194978,192.80875949)(478.21581129,192.85042608)
\curveto(478.23664459,192.89730098)(478.2626862,192.92855092)(478.29393614,192.94417589)
\curveto(478.3303944,192.96500918)(478.36945682,192.97542582)(478.4111234,192.97542582)
\lineto(479.38768392,192.97542582)
\lineto(479.38768392,194.64729744)
\curveto(479.38768392,194.6837557)(479.39549641,194.7176098)(479.41112137,194.74885973)
\curveto(479.43195467,194.78010967)(479.46580876,194.80615128)(479.51268367,194.82698457)
\curveto(479.5647669,194.85302619)(479.63247509,194.87125532)(479.71580826,194.88167196)
\curveto(479.79914142,194.89208861)(479.90330788,194.89729693)(480.02830762,194.89729693)
\curveto(480.15851569,194.89729693)(480.26528631,194.89208861)(480.34861948,194.88167196)
\curveto(480.43195264,194.87125532)(480.49705667,194.85302619)(480.54393158,194.82698457)
\curveto(480.59080649,194.80615128)(480.62466058,194.78010967)(480.64549387,194.74885973)
\curveto(480.66632717,194.7176098)(480.67674381,194.6837557)(480.67674381,194.64729744)
\lineto(480.67674381,192.97542582)
\lineto(482.48142766,192.97542582)
\curveto(482.52309424,192.97542582)(482.5595525,192.96500918)(482.59080244,192.94417589)
\curveto(482.62205237,192.92855092)(482.64809399,192.89730098)(482.66892728,192.85042608)
\curveto(482.69496889,192.80875949)(482.71319802,192.75407211)(482.72361467,192.68636391)
\curveto(482.73403131,192.61865571)(482.73923963,192.53532255)(482.73923963,192.43636442)
\curveto(482.73923963,192.24886479)(482.71580218,192.1134484)(482.66892728,192.03011524)
\curveto(482.62205237,191.94678207)(482.5595525,191.90511549)(482.48142766,191.90511549)
\lineto(480.67674381,191.90511549)
\lineto(480.67674381,187.98324843)
\curveto(480.67674381,187.49887441)(480.74705617,187.13168766)(480.88768088,186.88168816)
\curveto(481.03351392,186.63689699)(481.2913259,186.51450141)(481.66111682,186.51450141)
\curveto(481.78090824,186.51450141)(481.88767886,186.52491805)(481.98142867,186.54575134)
\curveto(482.07517848,186.57179296)(482.15851164,186.59783457)(482.23142816,186.62387619)
\curveto(482.30434468,186.6499178)(482.36684456,186.67335525)(482.41892778,186.69418854)
\curveto(482.47101101,186.72023016)(482.51788592,186.73325096)(482.5595525,186.73325096)
\curveto(482.58559411,186.73325096)(482.60903156,186.72543848)(482.62986486,186.70981351)
\curveto(482.65590647,186.69939687)(482.6741356,186.67595941)(482.68455225,186.63950115)
\curveto(482.70017721,186.60304289)(482.71319802,186.55356383)(482.72361467,186.49106395)
\curveto(482.73403131,186.42856408)(482.73923963,186.35043924)(482.73923963,186.25668943)
\closepath
}
}
{
\newrgbcolor{curcolor}{1 1 1}
\pscustom[linestyle=none,fillstyle=solid,fillcolor=curcolor]
{
\newpath
\moveto(490.48165941,185.74106547)
\curveto(490.48165941,185.69939889)(490.47124277,185.66294063)(490.45040947,185.63169069)
\curveto(490.42957618,185.60564908)(490.39572209,185.58221163)(490.34884718,185.56137834)
\curveto(490.30197228,185.54054505)(490.23686824,185.52492008)(490.15353508,185.51450343)
\curveto(490.07020191,185.50408679)(489.96603546,185.49887846)(489.84103571,185.49887846)
\curveto(489.71082764,185.49887846)(489.60405702,185.50408679)(489.52072386,185.51450343)
\curveto(489.43739069,185.52492008)(489.37228666,185.54054505)(489.32541175,185.56137834)
\curveto(489.27853685,185.58221163)(489.24468275,185.60564908)(489.22384946,185.63169069)
\curveto(489.20301617,185.66294063)(489.19259952,185.69939889)(489.19259952,185.74106547)
\lineto(489.19259952,189.85824464)
\curveto(489.19259952,190.25928549)(489.16134959,190.5822015)(489.09884971,190.82699267)
\curveto(489.03634984,191.07178385)(488.94520419,191.28272092)(488.82541277,191.45980389)
\curveto(488.70562134,191.63688687)(488.54937166,191.77230326)(488.35666371,191.86605307)
\curveto(488.16916409,191.95980288)(487.95041454,192.00667779)(487.70041504,192.00667779)
\curveto(487.37749903,192.00667779)(487.05458302,191.89209468)(486.73166701,191.66292848)
\curveto(486.40875099,191.43376228)(486.07021001,191.09782546)(485.71604406,190.65511802)
\lineto(485.71604406,185.74106547)
\curveto(485.71604406,185.69939889)(485.70562742,185.66294063)(485.68479413,185.63169069)
\curveto(485.66396083,185.60564908)(485.63010674,185.58221163)(485.58323183,185.56137834)
\curveto(485.53635693,185.54054505)(485.47125289,185.52492008)(485.38791973,185.51450343)
\curveto(485.30458656,185.50408679)(485.19781594,185.49887846)(485.06760787,185.49887846)
\curveto(484.94260813,185.49887846)(484.83844167,185.50408679)(484.75510851,185.51450343)
\curveto(484.67177534,185.52492008)(484.60406715,185.54054505)(484.55198392,185.56137834)
\curveto(484.50510901,185.58221163)(484.47125492,185.60564908)(484.45042162,185.63169069)
\curveto(484.43479666,185.66294063)(484.42698417,185.69939889)(484.42698417,185.74106547)
\lineto(484.42698417,196.17854434)
\curveto(484.42698417,196.22021092)(484.43479666,196.25666918)(484.45042162,196.28791912)
\curveto(484.47125492,196.31916905)(484.50510901,196.34521067)(484.55198392,196.36604396)
\curveto(484.60406715,196.38687725)(484.67177534,196.40250222)(484.75510851,196.41291886)
\curveto(484.83844167,196.42333551)(484.94260813,196.42854383)(485.06760787,196.42854383)
\curveto(485.19781594,196.42854383)(485.30458656,196.42333551)(485.38791973,196.41291886)
\curveto(485.47125289,196.40250222)(485.53635693,196.38687725)(485.58323183,196.36604396)
\curveto(485.63010674,196.34521067)(485.66396083,196.31916905)(485.68479413,196.28791912)
\curveto(485.70562742,196.25666918)(485.71604406,196.22021092)(485.71604406,196.17854434)
\lineto(485.71604406,191.96761536)
\curveto(486.08583498,192.35823957)(486.45823006,192.64730149)(486.8332293,192.83480111)
\curveto(487.20822854,193.02750905)(487.58583194,193.12386302)(487.96603951,193.12386302)
\curveto(488.43478856,193.12386302)(488.82801693,193.04313402)(489.14572462,192.88167601)
\curveto(489.46864063,192.72542633)(489.72905677,192.51448926)(489.92697303,192.24886479)
\curveto(490.1248893,191.98324033)(490.26551402,191.67074097)(490.34884718,191.31136669)
\curveto(490.43738867,190.95720074)(490.48165941,190.52751411)(490.48165941,190.0223068)
\closepath
}
}
{
\newrgbcolor{curcolor}{1 1 1}
\pscustom[linestyle=none,fillstyle=solid,fillcolor=curcolor]
{
\newpath
\moveto(500.09920128,186.25668943)
\curveto(500.09920128,186.10564807)(500.08878463,185.98585664)(500.06795134,185.89731516)
\curveto(500.04711805,185.80877367)(500.01586811,185.74366963)(499.97420153,185.70200305)
\curveto(499.93253495,185.66033647)(499.87003507,185.62127405)(499.78670191,185.58481579)
\curveto(499.70336874,185.54835753)(499.60701477,185.51971175)(499.49763999,185.49887846)
\curveto(499.39347354,185.47283685)(499.2814946,185.45200356)(499.16170317,185.43637859)
\curveto(499.04191175,185.42075362)(498.92212033,185.41294114)(498.8023289,185.41294114)
\curveto(498.43774631,185.41294114)(498.12524694,185.45981604)(497.8648308,185.55356585)
\curveto(497.60441466,185.65252399)(497.39087343,185.79835702)(497.2242071,185.99106497)
\curveto(497.05754077,186.18898123)(496.93514518,186.43637657)(496.85702034,186.73325096)
\curveto(496.78410382,187.03533369)(496.74764556,187.38949964)(496.74764556,187.79574881)
\lineto(496.74764556,191.90511549)
\lineto(495.76327256,191.90511549)
\curveto(495.68514771,191.90511549)(495.62264784,191.94678207)(495.57577294,192.03011524)
\curveto(495.52889803,192.1134484)(495.50546058,192.24886479)(495.50546058,192.43636442)
\curveto(495.50546058,192.53532255)(495.5106689,192.61865571)(495.52108555,192.68636391)
\curveto(495.53671052,192.75407211)(495.55493965,192.80875949)(495.57577294,192.85042608)
\curveto(495.59660623,192.89730098)(495.62264784,192.92855092)(495.65389778,192.94417589)
\curveto(495.69035604,192.96500918)(495.72941846,192.97542582)(495.77108504,192.97542582)
\lineto(496.74764556,192.97542582)
\lineto(496.74764556,194.64729744)
\curveto(496.74764556,194.6837557)(496.75545805,194.7176098)(496.77108302,194.74885973)
\curveto(496.79191631,194.78010967)(496.8257704,194.80615128)(496.87264531,194.82698457)
\curveto(496.92472854,194.85302619)(496.99243673,194.87125532)(497.0757699,194.88167196)
\curveto(497.15910306,194.89208861)(497.26326952,194.89729693)(497.38826927,194.89729693)
\curveto(497.51847734,194.89729693)(497.62524795,194.89208861)(497.70858112,194.88167196)
\curveto(497.79191428,194.87125532)(497.85701832,194.85302619)(497.90389322,194.82698457)
\curveto(497.95076813,194.80615128)(497.98462222,194.78010967)(498.00545552,194.74885973)
\curveto(498.02628881,194.7176098)(498.03670545,194.6837557)(498.03670545,194.64729744)
\lineto(498.03670545,192.97542582)
\lineto(499.8413893,192.97542582)
\curveto(499.88305588,192.97542582)(499.91951414,192.96500918)(499.95076408,192.94417589)
\curveto(499.98201401,192.92855092)(500.00805563,192.89730098)(500.02888892,192.85042608)
\curveto(500.05493053,192.80875949)(500.07315966,192.75407211)(500.08357631,192.68636391)
\curveto(500.09399295,192.61865571)(500.09920128,192.53532255)(500.09920128,192.43636442)
\curveto(500.09920128,192.24886479)(500.07576382,192.1134484)(500.02888892,192.03011524)
\curveto(499.98201401,191.94678207)(499.91951414,191.90511549)(499.8413893,191.90511549)
\lineto(498.03670545,191.90511549)
\lineto(498.03670545,187.98324843)
\curveto(498.03670545,187.49887441)(498.10701781,187.13168766)(498.24764253,186.88168816)
\curveto(498.39347556,186.63689699)(498.65128754,186.51450141)(499.02107846,186.51450141)
\curveto(499.14086988,186.51450141)(499.2476405,186.52491805)(499.34139031,186.54575134)
\curveto(499.43514012,186.57179296)(499.51847329,186.59783457)(499.5913898,186.62387619)
\curveto(499.66430632,186.6499178)(499.7268062,186.67335525)(499.77888942,186.69418854)
\curveto(499.83097265,186.72023016)(499.87784756,186.73325096)(499.91951414,186.73325096)
\curveto(499.94555575,186.73325096)(499.96899321,186.72543848)(499.9898265,186.70981351)
\curveto(500.01586811,186.69939687)(500.03409724,186.67595941)(500.04451389,186.63950115)
\curveto(500.06013886,186.60304289)(500.07315966,186.55356383)(500.08357631,186.49106395)
\curveto(500.09399295,186.42856408)(500.09920128,186.35043924)(500.09920128,186.25668943)
\closepath
}
}
{
\newrgbcolor{curcolor}{1 1 1}
\pscustom[linestyle=none,fillstyle=solid,fillcolor=curcolor]
{
\newpath
\moveto(507.11506384,185.88950267)
\curveto(507.15673042,185.81658615)(507.17756371,185.75408628)(507.17756371,185.70200305)
\curveto(507.17756371,185.65512815)(507.1515221,185.61606573)(507.09943887,185.58481579)
\curveto(507.05256397,185.55356585)(506.97704329,185.53273256)(506.87287683,185.52231592)
\curveto(506.7739187,185.50669095)(506.64371063,185.49887846)(506.48225262,185.49887846)
\curveto(506.32600294,185.49887846)(506.20100319,185.50408679)(506.10725338,185.51450343)
\curveto(506.01871189,185.51971175)(505.94579537,185.5301284)(505.88850382,185.54575337)
\curveto(505.8364206,185.56658666)(505.79475401,185.59002411)(505.76350408,185.61606573)
\curveto(505.73746246,185.64731566)(505.71402501,185.68116976)(505.69319172,185.71762802)
\lineto(504.00569514,188.50668487)
\lineto(502.32601104,185.71762802)
\curveto(502.30517775,185.68116976)(502.27913613,185.64731566)(502.2478862,185.61606573)
\curveto(502.22184458,185.59002411)(502.180178,185.56658666)(502.12288645,185.54575337)
\curveto(502.07080322,185.5301284)(502.00049086,185.51971175)(501.91194938,185.51450343)
\curveto(501.82340789,185.50408679)(501.70622063,185.49887846)(501.56038759,185.49887846)
\curveto(501.40934623,185.49887846)(501.28434648,185.50669095)(501.18538835,185.52231592)
\curveto(501.09163854,185.53273256)(501.02132618,185.55356585)(500.97445127,185.58481579)
\curveto(500.93278469,185.61606573)(500.9119514,185.65512815)(500.9119514,185.70200305)
\curveto(500.91715972,185.75408628)(500.94320134,185.81658615)(500.99007624,185.88950267)
\lineto(503.17757181,189.31918323)
\lineto(501.1072635,192.62386404)
\curveto(501.06559692,192.69678055)(501.04215947,192.75667627)(501.03695115,192.80355117)
\curveto(501.03695115,192.8556344)(501.0603886,192.89730098)(501.1072635,192.92855092)
\curveto(501.15934673,192.96500918)(501.23486741,192.98844663)(501.33382555,192.99886328)
\curveto(501.437992,193.00927992)(501.57340839,193.01448824)(501.74007472,193.01448824)
\curveto(501.89111608,193.01448824)(502.01090751,193.00927992)(502.099449,192.99886328)
\curveto(502.19319881,192.99365495)(502.26611532,192.98323831)(502.31819855,192.96761334)
\curveto(502.37028178,192.95198837)(502.4093442,192.93115508)(502.43538582,192.90511347)
\curveto(502.46142743,192.87907185)(502.48486488,192.84782192)(502.50569817,192.81136366)
\lineto(504.10725743,190.19418146)
\lineto(505.73225414,192.81136366)
\curveto(505.75308743,192.84261359)(505.77652488,192.87125937)(505.8025665,192.89730098)
\curveto(505.82860811,192.9233426)(505.86246221,192.94417589)(505.90412879,192.95980086)
\curveto(505.9510037,192.98063415)(506.01350357,192.99365495)(506.09162841,192.99886328)
\curveto(506.17496158,193.00927992)(506.28694052,193.01448824)(506.42756523,193.01448824)
\curveto(506.57860659,193.01448824)(506.70100218,193.00927992)(506.79475199,192.99886328)
\curveto(506.89371012,192.98844663)(506.96662664,192.96761334)(507.01350154,192.9363634)
\curveto(507.06037645,192.91032179)(507.08120974,192.87125937)(507.07600142,192.81917614)
\curveto(507.0707931,192.76709291)(507.04475148,192.70198888)(506.99787658,192.62386404)
\lineto(504.94319324,189.35824565)
\closepath
}
}
{
\newrgbcolor{curcolor}{1 1 1}
\pscustom[linestyle=none,fillstyle=solid,fillcolor=curcolor]
{
\newpath
\moveto(514.58992511,192.45198938)
\curveto(514.58992511,192.26969809)(514.56388349,192.13688586)(514.51180026,192.05355269)
\curveto(514.46492536,191.97542785)(514.40242549,191.93636543)(514.32430064,191.93636543)
\lineto(513.31649018,191.93636543)
\curveto(513.49878148,191.74886581)(513.62638539,191.5405329)(513.69930191,191.31136669)
\curveto(513.77221843,191.08740881)(513.80867669,190.85303429)(513.80867669,190.60824312)
\curveto(513.80867669,190.20199394)(513.74357265,189.84261967)(513.61336458,189.5301203)
\curveto(513.48315651,189.21762093)(513.29565689,188.95199647)(513.05086572,188.73324691)
\curveto(512.81128287,188.51970568)(512.52482512,188.35564351)(512.19149246,188.24106041)
\curveto(511.8581598,188.12647731)(511.48836889,188.06918576)(511.08211971,188.06918576)
\curveto(510.79566196,188.06918576)(510.52222501,188.10564402)(510.26180887,188.17856054)
\curveto(510.00660105,188.25668538)(509.80868479,188.35303935)(509.66806007,188.46762245)
\curveto(509.57431026,188.37387264)(509.49618542,188.26710202)(509.43368555,188.1473106)
\curveto(509.376394,188.02751918)(509.34774822,187.88949862)(509.34774822,187.73324894)
\curveto(509.34774822,187.55095764)(509.43108139,187.39991628)(509.59774772,187.28012486)
\curveto(509.76962237,187.16033343)(509.99618441,187.0952294)(510.27743384,187.08481275)
\lineto(512.11336762,187.00668791)
\curveto(512.46232525,186.99627126)(512.7826371,186.9467922)(513.07430318,186.85825071)
\curveto(513.36596925,186.77491755)(513.61857291,186.65252196)(513.83211414,186.49106395)
\curveto(514.04565538,186.33481427)(514.2123217,186.13950217)(514.33211313,185.90512764)
\curveto(514.45190455,185.67596144)(514.51180026,185.40773281)(514.51180026,185.10044177)
\curveto(514.51180026,184.77752576)(514.44409207,184.47023471)(514.30867568,184.17856864)
\curveto(514.17325928,183.88690256)(513.96492637,183.63169474)(513.68367694,183.41294519)
\curveto(513.40763583,183.18898731)(513.05346988,183.01450849)(512.62117909,182.88950875)
\curveto(512.1888883,182.75930068)(511.67847267,182.69419664)(511.08993219,182.69419664)
\curveto(510.52222501,182.69419664)(510.03785099,182.74367571)(509.63681014,182.84263384)
\curveto(509.2409776,182.93638365)(508.91545743,183.06659172)(508.66024961,183.23325805)
\curveto(508.4050418,183.39992438)(508.22014634,183.60044481)(508.10556324,183.83481933)
\curveto(507.99098014,184.06398554)(507.93368859,184.31398503)(507.93368859,184.58481781)
\curveto(507.93368859,184.75669247)(507.95452188,184.9233588)(507.99618846,185.0848168)
\curveto(508.03785504,185.24627481)(508.10035491,185.39992033)(508.18368808,185.54575337)
\curveto(508.27222957,185.69158641)(508.37900018,185.82960696)(508.50399993,185.95981503)
\curveto(508.634208,186.09523142)(508.7826452,186.22804365)(508.94931153,186.35825172)
\curveto(508.69410371,186.48845979)(508.50399993,186.65252196)(508.37900018,186.85043823)
\curveto(508.25920876,187.04835449)(508.19931305,187.26189573)(508.19931305,187.49106193)
\curveto(508.19931305,187.80876962)(508.26441708,188.09262321)(508.39462515,188.34262271)
\curveto(508.52483322,188.5926222)(508.68629123,188.81658008)(508.87899917,189.01449634)
\curveto(508.71754116,189.20720429)(508.58993726,189.42334968)(508.49618745,189.66293253)
\curveto(508.40243764,189.9077237)(508.35556273,190.20199394)(508.35556273,190.54574324)
\curveto(508.35556273,190.9467841)(508.42327093,191.30615837)(508.55868732,191.62386606)
\curveto(508.69410371,191.94157375)(508.88160333,192.20980237)(509.12118618,192.42855193)
\curveto(509.36076903,192.64730149)(509.64722678,192.81396782)(509.98055944,192.92855092)
\curveto(510.31910042,193.04834234)(510.68628718,193.10823805)(511.08211971,193.10823805)
\curveto(511.29566094,193.10823805)(511.49357721,193.09521725)(511.67586851,193.06917563)
\curveto(511.86336813,193.04834234)(512.03784694,193.01709241)(512.19930495,192.97542582)
\lineto(514.32430064,192.97542582)
\curveto(514.41284213,192.97542582)(514.47794617,192.93115508)(514.51961275,192.84261359)
\curveto(514.56648765,192.75928043)(514.58992511,192.62907236)(514.58992511,192.45198938)
\closepath
\moveto(512.57430419,190.60043063)
\curveto(512.57430419,191.07959633)(512.44149196,191.45199141)(512.17586749,191.71761587)
\curveto(511.91545136,191.98844866)(511.54305628,192.12386505)(511.05868226,192.12386505)
\curveto(510.80868276,192.12386505)(510.58993321,192.08219847)(510.40243359,191.9988653)
\curveto(510.22014229,191.91553214)(510.06649677,191.80094904)(509.94149702,191.655116)
\curveto(509.8217056,191.50928296)(509.73055995,191.34001247)(509.66806007,191.14730453)
\curveto(509.61076852,190.95980491)(509.58212275,190.76188864)(509.58212275,190.55355573)
\curveto(509.58212275,190.090015)(509.71233082,189.7254324)(509.97274696,189.45980794)
\curveto(510.23837142,189.19418348)(510.60816234,189.06137125)(511.08211971,189.06137125)
\curveto(511.33732753,189.06137125)(511.55868124,189.10043367)(511.74618086,189.17855851)
\curveto(511.93368049,189.26189168)(512.08732601,189.37387062)(512.20711743,189.51449533)
\curveto(512.33211718,189.66032837)(512.42326283,189.8269947)(512.48055438,190.01449432)
\curveto(512.54305425,190.20199394)(512.57430419,190.39730604)(512.57430419,190.60043063)
\closepath
\moveto(513.22274037,185.03012941)
\curveto(513.22274037,185.33221213)(513.09774063,185.5639825)(512.84774113,185.7254405)
\curveto(512.60294996,185.89210683)(512.2696173,185.98064832)(511.84774316,185.99106497)
\lineto(510.02743435,186.05356484)
\curveto(509.86076802,185.92335677)(509.72274746,185.79835702)(509.61337268,185.6785656)
\curveto(509.50920623,185.5639825)(509.42587306,185.45460772)(509.36337319,185.35044126)
\curveto(509.30087332,185.24106649)(509.25660257,185.13429587)(509.23056096,185.03012941)
\curveto(509.20972767,184.92596296)(509.19931102,184.81919234)(509.19931102,184.70981756)
\curveto(509.19931102,184.37127658)(509.37118567,184.11606876)(509.71493498,183.94419411)
\curveto(510.05868428,183.76711114)(510.53784998,183.67856965)(511.15243207,183.67856965)
\curveto(511.54305628,183.67856965)(511.86857645,183.71763207)(512.12899259,183.79575691)
\curveto(512.39461705,183.86867343)(512.60815829,183.96763156)(512.76961629,184.09263131)
\curveto(512.9310743,184.21763106)(513.0456574,184.36085993)(513.1133656,184.52231794)
\curveto(513.18628212,184.68377595)(513.22274037,184.85304644)(513.22274037,185.03012941)
\closepath
}
}
{
\newrgbcolor{curcolor}{0.53725493 0.61176473 0.90196079}
\pscustom[linestyle=none,fillstyle=solid,fillcolor=curcolor]
{
\newpath
\moveto(240.87760983,175.39400647)
\lineto(689.69561173,175.39400647)
\lineto(689.69561173,138.60667939)
\lineto(240.87760983,138.60667939)
\closepath
}
}
{
\newrgbcolor{curcolor}{0.10196079 0.8392157 0.96078432}
\pscustom[linewidth=1.99999595,linecolor=curcolor]
{
\newpath
\moveto(240.87760983,175.39400647)
\lineto(689.69561173,175.39400647)
\lineto(689.69561173,138.60667939)
\lineto(240.87760983,138.60667939)
\closepath
}
}
{
\newrgbcolor{curcolor}{1 1 1}
\pscustom[linestyle=none,fillstyle=solid,fillcolor=curcolor]
{
\newpath
\moveto(442.02885503,161.25752005)
\curveto(442.02885503,161.15856192)(442.02364671,161.07783292)(442.01323006,161.01533304)
\curveto(442.00281342,160.95804149)(441.98979261,160.91116659)(441.97416764,160.87470833)
\curveto(441.963751,160.83825007)(441.94812603,160.81220845)(441.92729274,160.79658349)
\curveto(441.90645945,160.78616684)(441.88041783,160.78095852)(441.84916789,160.78095852)
\curveto(441.81270964,160.78095852)(441.76583473,160.79137516)(441.70854318,160.81220845)
\curveto(441.65645995,160.83825007)(441.59135592,160.86429168)(441.51323108,160.8903333)
\curveto(441.43510623,160.92158323)(441.34135642,160.94762485)(441.23198164,160.96845814)
\curveto(441.12781519,160.99449975)(441.0054196,161.00752056)(440.86479489,161.00752056)
\curveto(440.67208695,161.00752056)(440.50802478,160.97627062)(440.37260838,160.91377075)
\curveto(440.23719199,160.85127088)(440.12781721,160.75231274)(440.04448405,160.61689635)
\curveto(439.96115088,160.48668828)(439.90125517,160.31481363)(439.86479691,160.10127239)
\curveto(439.82833865,159.89293948)(439.81010952,159.63773167)(439.81010952,159.33564894)
\lineto(439.81010952,158.57002549)
\lineto(441.38041884,158.57002549)
\curveto(441.42208543,158.57002549)(441.45593952,158.55960885)(441.48198114,158.53877556)
\curveto(441.51323108,158.52315059)(441.53927269,158.49190065)(441.56010598,158.44502575)
\curveto(441.58614759,158.40335917)(441.60437672,158.34867178)(441.61479337,158.28096358)
\curveto(441.63041834,158.21325538)(441.63823082,158.12992222)(441.63823082,158.03096409)
\curveto(441.63823082,157.84346447)(441.61479337,157.70804807)(441.56791846,157.62471491)
\curveto(441.52104356,157.54138174)(441.45854369,157.49971516)(441.38041884,157.49971516)
\lineto(439.81010952,157.49971516)
\lineto(439.81010952,151.33566514)
\curveto(439.81010952,151.29399856)(439.79969288,151.2575403)(439.77885959,151.22629037)
\curveto(439.7580263,151.20024875)(439.72156804,151.1768113)(439.66948481,151.15597801)
\curveto(439.6226099,151.13514472)(439.55750587,151.11951975)(439.4741727,151.1091031)
\curveto(439.39083954,151.09868646)(439.28667308,151.09347813)(439.16167334,151.09347813)
\curveto(439.03667359,151.09347813)(438.93250713,151.09868646)(438.84917397,151.1091031)
\curveto(438.76584081,151.11951975)(438.69813261,151.13514472)(438.64604938,151.15597801)
\curveto(438.59917448,151.1768113)(438.56532038,151.20024875)(438.54448709,151.22629037)
\curveto(438.5236538,151.2575403)(438.51323715,151.29399856)(438.51323715,151.33566514)
\lineto(438.51323715,157.49971516)
\lineto(437.52105166,157.49971516)
\curveto(437.43771849,157.49971516)(437.37521862,157.54138174)(437.33355204,157.62471491)
\curveto(437.29188546,157.70804807)(437.27105217,157.84346447)(437.27105217,158.03096409)
\curveto(437.27105217,158.12992222)(437.27626049,158.21325538)(437.28667713,158.28096358)
\curveto(437.29709378,158.34867178)(437.31271875,158.40335917)(437.33355204,158.44502575)
\curveto(437.35438533,158.49190065)(437.38042694,158.52315059)(437.41167688,158.53877556)
\curveto(437.44292682,158.55960885)(437.47938508,158.57002549)(437.52105166,158.57002549)
\lineto(438.51323715,158.57002549)
\lineto(438.51323715,159.29658652)
\curveto(438.51323715,159.78616887)(438.55750789,160.20543885)(438.64604938,160.55439648)
\curveto(438.73979919,160.90856243)(438.88042391,161.19762434)(439.06792353,161.42158222)
\curveto(439.25542315,161.6455401)(439.48979767,161.80960227)(439.7710471,161.91376872)
\curveto(440.05750486,162.0231435)(440.39344168,162.07783089)(440.77885756,162.07783089)
\curveto(440.96114886,162.07783089)(441.13823183,162.05960176)(441.31010649,162.0231435)
\curveto(441.48198114,161.99189357)(441.61479337,161.95543531)(441.70854318,161.91376872)
\curveto(441.80229299,161.87731046)(441.86479286,161.84345637)(441.8960428,161.81220643)
\curveto(441.92729274,161.78095649)(441.95333435,161.73928991)(441.97416764,161.68720668)
\curveto(441.99500093,161.64033178)(442.00802174,161.58043607)(442.01323006,161.50751955)
\curveto(442.02364671,161.43981135)(442.02885503,161.35647819)(442.02885503,161.25752005)
\closepath
}
}
{
\newrgbcolor{curcolor}{1 1 1}
\pscustom[linestyle=none,fillstyle=solid,fillcolor=curcolor]
{
\newpath
\moveto(444.34635045,151.33566514)
\curveto(444.34635045,151.29399856)(444.33593381,151.2575403)(444.31510052,151.22629037)
\curveto(444.29426722,151.20024875)(444.26041313,151.1768113)(444.21353822,151.15597801)
\curveto(444.16666332,151.13514472)(444.10155928,151.11951975)(444.01822612,151.1091031)
\curveto(443.93489295,151.09868646)(443.82812234,151.09347813)(443.69791427,151.09347813)
\curveto(443.57291452,151.09347813)(443.46874806,151.09868646)(443.3854149,151.1091031)
\curveto(443.30208173,151.11951975)(443.23437354,151.13514472)(443.18229031,151.15597801)
\curveto(443.1354154,151.1768113)(443.10156131,151.20024875)(443.08072802,151.22629037)
\curveto(443.06510305,151.2575403)(443.05729056,151.29399856)(443.05729056,151.33566514)
\lineto(443.05729056,158.36690091)
\curveto(443.05729056,158.40335917)(443.06510305,158.43721326)(443.08072802,158.4684632)
\curveto(443.10156131,158.49971314)(443.1354154,158.52575475)(443.18229031,158.54658804)
\curveto(443.23437354,158.56742133)(443.30208173,158.5830463)(443.3854149,158.59346295)
\curveto(443.46874806,158.60387959)(443.57291452,158.60908792)(443.69791427,158.60908792)
\curveto(443.82812234,158.60908792)(443.93489295,158.60387959)(444.01822612,158.59346295)
\curveto(444.10155928,158.5830463)(444.16666332,158.56742133)(444.21353822,158.54658804)
\curveto(444.26041313,158.52575475)(444.29426722,158.49971314)(444.31510052,158.4684632)
\curveto(444.33593381,158.43721326)(444.34635045,158.40335917)(444.34635045,158.36690091)
\closepath
\moveto(444.49478765,160.7418961)
\curveto(444.49478765,160.43981338)(444.4374961,160.23408462)(444.322913,160.12470985)
\curveto(444.2083299,160.01533507)(443.99739283,159.96064768)(443.69010178,159.96064768)
\curveto(443.38801906,159.96064768)(443.17968615,160.01273091)(443.06510305,160.11689736)
\curveto(442.95572827,160.22627214)(442.90104088,160.42939673)(442.90104088,160.72627113)
\curveto(442.90104088,161.02835385)(442.95833243,161.2340826)(443.07291553,161.34345738)
\curveto(443.18749863,161.45283216)(443.39843571,161.50751955)(443.70572675,161.50751955)
\curveto(444.00780947,161.50751955)(444.21353822,161.45283216)(444.322913,161.34345738)
\curveto(444.4374961,161.23929092)(444.49478765,161.0387705)(444.49478765,160.7418961)
\closepath
}
}
{
\newrgbcolor{curcolor}{1 1 1}
\pscustom[linestyle=none,fillstyle=solid,fillcolor=curcolor]
{
\newpath
\moveto(448.01034259,151.33566514)
\curveto(448.01034259,151.29399856)(447.99992594,151.2575403)(447.97909265,151.22629037)
\curveto(447.95825936,151.20024875)(447.92440526,151.1768113)(447.87753036,151.15597801)
\curveto(447.83065545,151.13514472)(447.76555142,151.11951975)(447.68221825,151.1091031)
\curveto(447.59888509,151.09868646)(447.49211447,151.09347813)(447.3619064,151.09347813)
\curveto(447.23690666,151.09347813)(447.1327402,151.09868646)(447.04940704,151.1091031)
\curveto(446.96607387,151.11951975)(446.89836568,151.13514472)(446.84628245,151.15597801)
\curveto(446.79940754,151.1768113)(446.76555344,151.20024875)(446.74472015,151.22629037)
\curveto(446.72909518,151.2575403)(446.7212827,151.29399856)(446.7212827,151.33566514)
\lineto(446.7212827,161.77314401)
\curveto(446.7212827,161.81481059)(446.72909518,161.85126885)(446.74472015,161.88251879)
\curveto(446.76555344,161.91376872)(446.79940754,161.93981034)(446.84628245,161.96064363)
\curveto(446.89836568,161.98147692)(446.96607387,161.99710189)(447.04940704,162.00751853)
\curveto(447.1327402,162.01793518)(447.23690666,162.0231435)(447.3619064,162.0231435)
\curveto(447.49211447,162.0231435)(447.59888509,162.01793518)(447.68221825,162.00751853)
\curveto(447.76555142,161.99710189)(447.83065545,161.98147692)(447.87753036,161.96064363)
\curveto(447.92440526,161.93981034)(447.95825936,161.91376872)(447.97909265,161.88251879)
\curveto(447.99992594,161.85126885)(448.01034259,161.81481059)(448.01034259,161.77314401)
\closepath
}
}
{
\newrgbcolor{curcolor}{1 1 1}
\pscustom[linestyle=none,fillstyle=solid,fillcolor=curcolor]
{
\newpath
\moveto(451.67433568,151.33566514)
\curveto(451.67433568,151.29399856)(451.66391904,151.2575403)(451.64308575,151.22629037)
\curveto(451.62225245,151.20024875)(451.58839836,151.1768113)(451.54152345,151.15597801)
\curveto(451.49464855,151.13514472)(451.42954451,151.11951975)(451.34621135,151.1091031)
\curveto(451.26287818,151.09868646)(451.15610756,151.09347813)(451.0258995,151.09347813)
\curveto(450.90089975,151.09347813)(450.79673329,151.09868646)(450.71340013,151.1091031)
\curveto(450.63006696,151.11951975)(450.56235877,151.13514472)(450.51027554,151.15597801)
\curveto(450.46340063,151.1768113)(450.42954654,151.20024875)(450.40871324,151.22629037)
\curveto(450.39308828,151.2575403)(450.38527579,151.29399856)(450.38527579,151.33566514)
\lineto(450.38527579,161.77314401)
\curveto(450.38527579,161.81481059)(450.39308828,161.85126885)(450.40871324,161.88251879)
\curveto(450.42954654,161.91376872)(450.46340063,161.93981034)(450.51027554,161.96064363)
\curveto(450.56235877,161.98147692)(450.63006696,161.99710189)(450.71340013,162.00751853)
\curveto(450.79673329,162.01793518)(450.90089975,162.0231435)(451.0258995,162.0231435)
\curveto(451.15610756,162.0231435)(451.26287818,162.01793518)(451.34621135,162.00751853)
\curveto(451.42954451,161.99710189)(451.49464855,161.98147692)(451.54152345,161.96064363)
\curveto(451.58839836,161.93981034)(451.62225245,161.91376872)(451.64308575,161.88251879)
\curveto(451.66391904,161.85126885)(451.67433568,161.81481059)(451.67433568,161.77314401)
\closepath
}
}
{
\newrgbcolor{curcolor}{1 1 1}
\pscustom[linestyle=none,fillstyle=solid,fillcolor=curcolor]
{
\newpath
\moveto(462.719939,152.40597548)
\curveto(462.719939,152.31743399)(462.71733484,152.23930915)(462.71212652,152.17160095)
\curveto(462.70691819,152.10910108)(462.69650155,152.05441369)(462.68087658,152.00753878)
\curveto(462.67045993,151.9658722)(462.65483497,151.92680978)(462.63400168,151.89035152)
\curveto(462.61837671,151.85910158)(462.57671012,151.80962252)(462.50900193,151.74191432)
\curveto(462.44650205,151.67941445)(462.33712728,151.59868545)(462.18087759,151.49972731)
\curveto(462.02462791,151.4059775)(461.84754493,151.32004018)(461.64962867,151.24191533)
\curveto(461.45692073,151.16899882)(461.24598365,151.1091031)(461.01681745,151.0622282)
\curveto(460.78765125,151.01535329)(460.55067256,150.99191584)(460.30588139,150.99191584)
\curveto(459.80067408,150.99191584)(459.35275832,151.07524901)(458.96213411,151.24191533)
\curveto(458.5715099,151.40858166)(458.24338557,151.65076867)(457.9777611,151.96847636)
\curveto(457.71734496,152.29139238)(457.51682454,152.68462075)(457.37619982,153.14816147)
\curveto(457.24078343,153.61691052)(457.17307523,154.15597193)(457.17307523,154.7653457)
\curveto(457.17307523,155.45805263)(457.2564084,156.05180143)(457.42307473,156.54659209)
\curveto(457.59494938,157.04659108)(457.82671974,157.45544442)(458.11838582,157.77315211)
\curveto(458.41526022,158.0908598)(458.76161368,158.32523432)(459.15744622,158.47627568)
\curveto(459.55848707,158.63252537)(459.99077786,158.71065021)(460.45431859,158.71065021)
\curveto(460.67827647,158.71065021)(460.89442186,158.68981692)(461.10275478,158.64815034)
\curveto(461.31629601,158.60648375)(461.51160811,158.55179636)(461.68869109,158.48408817)
\curveto(461.86577406,158.41637997)(462.02202375,158.33825513)(462.15744014,158.24971364)
\curveto(462.29806486,158.16117216)(462.39962715,158.08565148)(462.46212702,158.0231516)
\curveto(462.5246269,157.96065173)(462.56629348,157.91117266)(462.58712677,157.8747144)
\curveto(462.61316838,157.83825614)(462.63400168,157.7939854)(462.64962664,157.74190217)
\curveto(462.66525161,157.69502727)(462.67566826,157.64033988)(462.68087658,157.57784)
\curveto(462.6860849,157.51534013)(462.68868906,157.43721529)(462.68868906,157.34346548)
\curveto(462.68868906,157.14034089)(462.66525161,156.99711201)(462.61837671,156.91377885)
\curveto(462.5715018,156.83565401)(462.51421025,156.79659159)(462.44650205,156.79659159)
\curveto(462.36837721,156.79659159)(462.27723156,156.83825817)(462.17306511,156.92159133)
\curveto(462.07410698,157.01013282)(461.94650307,157.10648679)(461.79025338,157.21065325)
\curveto(461.6340037,157.3148197)(461.44389992,157.40856951)(461.21994204,157.49190268)
\curveto(461.00119248,157.58044417)(460.74077634,157.62471491)(460.43869362,157.62471491)
\curveto(459.81890321,157.62471491)(459.34234167,157.38513206)(459.00900902,156.90596636)
\curveto(458.68088468,156.43200899)(458.51682251,155.74190622)(458.51682251,154.83565806)
\curveto(458.51682251,154.38253397)(458.55848909,153.98409728)(458.64182226,153.64034798)
\curveto(458.73036375,153.301807)(458.85796766,153.0179534)(459.02463398,152.7887872)
\curveto(459.19130031,152.559621)(459.3944249,152.38774635)(459.63400775,152.27316325)
\curveto(459.87879892,152.16378847)(460.15744419,152.10910108)(460.46994356,152.10910108)
\curveto(460.76681796,152.10910108)(461.0272341,152.15597598)(461.25119198,152.24972579)
\curveto(461.47514986,152.3434756)(461.6678578,152.4450379)(461.8293158,152.55441268)
\curveto(461.99598213,152.66899578)(462.13400269,152.77055807)(462.24337747,152.85909956)
\curveto(462.35796057,152.95284937)(462.44650205,152.99972427)(462.50900193,152.99972427)
\curveto(462.54546019,152.99972427)(462.57671012,152.98930763)(462.60275174,152.96847434)
\curveto(462.62879335,152.94764105)(462.64962664,152.91118279)(462.66525161,152.85909956)
\curveto(462.6860849,152.81222465)(462.69910571,152.74972478)(462.70431403,152.67159994)
\curveto(462.71473068,152.59868342)(462.719939,152.51014193)(462.719939,152.40597548)
\closepath
}
}
{
\newrgbcolor{curcolor}{1 1 1}
\pscustom[linestyle=none,fillstyle=solid,fillcolor=curcolor]
{
\newpath
\moveto(470.82829622,154.92940787)
\curveto(470.82829622,154.35649236)(470.75277554,153.8278476)(470.60173418,153.34347358)
\curveto(470.45069282,152.86430788)(470.22413078,152.45024622)(469.92204806,152.10128859)
\curveto(469.62517366,151.75233097)(469.25017442,151.47889402)(468.79705034,151.28097776)
\curveto(468.34913458,151.08826981)(467.8283023,150.99191584)(467.2345535,150.99191584)
\curveto(466.65642967,150.99191584)(466.15122236,151.07785317)(465.71893157,151.24972782)
\curveto(465.2918491,151.42160247)(464.93507899,151.67160196)(464.64862124,151.9997263)
\curveto(464.36216348,152.32785064)(464.14862225,152.72628733)(464.00799753,153.19503638)
\curveto(463.86737282,153.66378543)(463.79706046,154.19503435)(463.79706046,154.78878315)
\curveto(463.79706046,155.36169866)(463.86997698,155.88773926)(464.01581002,156.36690496)
\curveto(464.16685138,156.85127898)(464.39080926,157.2679448)(464.68768366,157.61690242)
\curveto(464.98976638,157.96586005)(465.36476562,158.23669284)(465.81268138,158.42940078)
\curveto(466.26059714,158.62210872)(466.78403358,158.71846269)(467.3829907,158.71846269)
\curveto(467.96111453,158.71846269)(468.46371768,158.63252537)(468.89080015,158.46065072)
\curveto(469.32309094,158.28877606)(469.68246521,158.03877657)(469.96892296,157.71065223)
\curveto(470.25538072,157.3825279)(470.46892195,156.98409121)(470.60954667,156.51534216)
\curveto(470.7553797,156.0465931)(470.82829622,155.51794834)(470.82829622,154.92940787)
\closepath
\moveto(469.46892398,154.84347054)
\curveto(469.46892398,155.2236781)(469.43246572,155.58305238)(469.3595492,155.92159336)
\curveto(469.291841,156.26013434)(469.1772579,156.55700874)(469.01579989,156.81221655)
\curveto(468.85434189,157.06742437)(468.63559233,157.2679448)(468.35955122,157.41377784)
\curveto(468.08351012,157.5648192)(467.73976081,157.64033988)(467.32830331,157.64033988)
\curveto(466.94809575,157.64033988)(466.61997141,157.57263168)(466.3439303,157.43721529)
\curveto(466.07309752,157.3017989)(465.84913964,157.10909095)(465.67205667,156.85909146)
\curveto(465.49497369,156.61430029)(465.36216146,156.32263421)(465.27361997,155.98409323)
\curveto(465.19028681,155.64555225)(465.14862023,155.27576133)(465.14862023,154.87472048)
\curveto(465.14862023,154.48930459)(465.18247432,154.12732616)(465.25018252,153.78878518)
\curveto(465.32309904,153.4502442)(465.4402863,153.1533698)(465.60174431,152.89816198)
\curveto(465.76841064,152.64816249)(465.98976436,152.44764206)(466.26580546,152.2966007)
\curveto(466.54184657,152.15076766)(466.88559587,152.07785114)(467.29705337,152.07785114)
\curveto(467.67205262,152.07785114)(467.99757279,152.14555934)(468.2736139,152.28097573)
\curveto(468.549655,152.41639212)(468.77621705,152.6064959)(468.95330002,152.85128708)
\curveto(469.130383,153.09607825)(469.26059107,153.38774432)(469.34392423,153.7262853)
\curveto(469.42725739,154.06482628)(469.46892398,154.43722136)(469.46892398,154.84347054)
\closepath
}
}
{
\newrgbcolor{curcolor}{1 1 1}
\pscustom[linestyle=none,fillstyle=solid,fillcolor=curcolor]
{
\newpath
\moveto(478.7446544,151.33566514)
\curveto(478.7446544,151.29399856)(478.73423775,151.2575403)(478.71340446,151.22629037)
\curveto(478.69777949,151.20024875)(478.66652956,151.1768113)(478.61965465,151.15597801)
\curveto(478.57277975,151.13514472)(478.51027987,151.11951975)(478.43215503,151.1091031)
\curveto(478.35923851,151.09868646)(478.26809286,151.09347813)(478.15871809,151.09347813)
\curveto(478.03892666,151.09347813)(477.93996853,151.09868646)(477.86184369,151.1091031)
\curveto(477.78892717,151.11951975)(477.72903146,151.13514472)(477.68215655,151.15597801)
\curveto(477.64048997,151.1768113)(477.61184419,151.20024875)(477.59621923,151.22629037)
\curveto(477.58059426,151.2575403)(477.57278177,151.29399856)(477.57278177,151.33566514)
\lineto(477.57278177,152.26535076)
\curveto(477.17174092,151.82264332)(476.77590839,151.49972731)(476.38528418,151.29660272)
\curveto(475.99465997,151.09347813)(475.59882744,150.99191584)(475.19778658,150.99191584)
\curveto(474.72903753,150.99191584)(474.333205,151.07004068)(474.01028899,151.22629037)
\curveto(473.6925813,151.38254005)(473.43476932,151.59347712)(473.23685305,151.85910158)
\curveto(473.03893679,152.12993437)(472.89570791,152.44243374)(472.80716642,152.79659969)
\curveto(472.72383326,153.15597396)(472.68216668,153.59086891)(472.68216668,154.10128454)
\lineto(472.68216668,158.36690091)
\curveto(472.68216668,158.40856749)(472.68997916,158.44242159)(472.70560413,158.4684632)
\curveto(472.72643742,158.49971314)(472.76289568,158.52575475)(472.81497891,158.54658804)
\curveto(472.86706214,158.57262966)(472.93477033,158.58825462)(473.0181035,158.59346295)
\curveto(473.10143666,158.60387959)(473.20560312,158.60908792)(473.33060286,158.60908792)
\curveto(473.45560261,158.60908792)(473.55976907,158.60387959)(473.64310223,158.59346295)
\curveto(473.72643539,158.58825462)(473.79153943,158.57262966)(473.83841433,158.54658804)
\curveto(473.89049756,158.52575475)(473.92695582,158.49971314)(473.94778911,158.4684632)
\curveto(473.9686224,158.44242159)(473.97903905,158.40856749)(473.97903905,158.36690091)
\lineto(473.97903905,154.2731592)
\curveto(473.97903905,153.8617017)(474.00768483,153.5309732)(474.06497638,153.28097371)
\curveto(474.12747625,153.03618253)(474.2186219,152.82524546)(474.33841332,152.64816249)
\curveto(474.46341307,152.47628783)(474.61966275,152.34087144)(474.80716237,152.24191331)
\curveto(474.99466199,152.1481635)(475.21341155,152.10128859)(475.46341104,152.10128859)
\curveto(475.78632706,152.10128859)(476.10663891,152.2158717)(476.4243466,152.4450379)
\curveto(476.74726261,152.6742041)(477.08840775,153.01014092)(477.44778203,153.45284836)
\lineto(477.44778203,158.36690091)
\curveto(477.44778203,158.40856749)(477.45559451,158.44242159)(477.47121948,158.4684632)
\curveto(477.49205277,158.49971314)(477.52851103,158.52575475)(477.58059426,158.54658804)
\curveto(477.63267748,158.57262966)(477.69778152,158.58825462)(477.77590636,158.59346295)
\curveto(477.85923953,158.60387959)(477.96601014,158.60908792)(478.09621821,158.60908792)
\curveto(478.22121796,158.60908792)(478.32538442,158.60387959)(478.40871758,158.59346295)
\curveto(478.49205074,158.58825462)(478.55715478,158.57262966)(478.60402968,158.54658804)
\curveto(478.65090459,158.52575475)(478.68475869,158.49971314)(478.70559198,158.4684632)
\curveto(478.73163359,158.44242159)(478.7446544,158.40856749)(478.7446544,158.36690091)
\closepath
}
}
{
\newrgbcolor{curcolor}{1 1 1}
\pscustom[linestyle=none,fillstyle=solid,fillcolor=curcolor]
{
\newpath
\moveto(487.17588885,151.33566514)
\curveto(487.17588885,151.29399856)(487.16547221,151.2575403)(487.14463892,151.22629037)
\curveto(487.12380562,151.20024875)(487.08995153,151.1768113)(487.04307662,151.15597801)
\curveto(486.99620172,151.13514472)(486.93109768,151.11951975)(486.84776452,151.1091031)
\curveto(486.76443135,151.09868646)(486.6602649,151.09347813)(486.53526515,151.09347813)
\curveto(486.40505708,151.09347813)(486.29828646,151.09868646)(486.2149533,151.1091031)
\curveto(486.13162013,151.11951975)(486.0665161,151.13514472)(486.01964119,151.15597801)
\curveto(485.97276629,151.1768113)(485.93891219,151.20024875)(485.9180789,151.22629037)
\curveto(485.89724561,151.2575403)(485.88682896,151.29399856)(485.88682896,151.33566514)
\lineto(485.88682896,155.45284431)
\curveto(485.88682896,155.85388516)(485.85557903,156.17680117)(485.79307915,156.42159235)
\curveto(485.73057928,156.66638352)(485.63943363,156.87732059)(485.51964221,157.05440356)
\curveto(485.39985078,157.23148654)(485.2436011,157.36690293)(485.05089316,157.46065274)
\curveto(484.86339354,157.55440255)(484.64464398,157.60127746)(484.39464448,157.60127746)
\curveto(484.07172847,157.60127746)(483.74881246,157.48669436)(483.42589645,157.25752815)
\curveto(483.10298043,157.02836195)(482.76443945,156.69242513)(482.4102735,156.24971769)
\lineto(482.4102735,151.33566514)
\curveto(482.4102735,151.29399856)(482.39985686,151.2575403)(482.37902357,151.22629037)
\curveto(482.35819027,151.20024875)(482.32433618,151.1768113)(482.27746127,151.15597801)
\curveto(482.23058637,151.13514472)(482.16548233,151.11951975)(482.08214917,151.1091031)
\curveto(481.998816,151.09868646)(481.89204539,151.09347813)(481.76183732,151.09347813)
\curveto(481.63683757,151.09347813)(481.53267111,151.09868646)(481.44933795,151.1091031)
\curveto(481.36600478,151.11951975)(481.29829659,151.13514472)(481.24621336,151.15597801)
\curveto(481.19933845,151.1768113)(481.16548436,151.20024875)(481.14465107,151.22629037)
\curveto(481.1290261,151.2575403)(481.12121361,151.29399856)(481.12121361,151.33566514)
\lineto(481.12121361,158.36690091)
\curveto(481.12121361,158.40856749)(481.1290261,158.44242159)(481.14465107,158.4684632)
\curveto(481.16027603,158.49971314)(481.19152597,158.52575475)(481.23840088,158.54658804)
\curveto(481.28527578,158.57262966)(481.34517149,158.58825462)(481.41808801,158.59346295)
\curveto(481.49100453,158.60387959)(481.5873585,158.60908792)(481.70714993,158.60908792)
\curveto(481.82173303,158.60908792)(481.91548284,158.60387959)(481.98839936,158.59346295)
\curveto(482.0665242,158.58825462)(482.12641991,158.57262966)(482.16808649,158.54658804)
\curveto(482.20975308,158.52575475)(482.23839885,158.49971314)(482.25402382,158.4684632)
\curveto(482.27485711,158.44242159)(482.28527376,158.40856749)(482.28527376,158.36690091)
\lineto(482.28527376,157.43721529)
\curveto(482.68110629,157.87992273)(483.07433466,158.20283874)(483.46495887,158.40596333)
\curveto(483.8607914,158.61429624)(484.25922809,158.71846269)(484.66026895,158.71846269)
\curveto(485.129018,158.71846269)(485.52224637,158.63773369)(485.83995406,158.47627568)
\curveto(486.16287007,158.320026)(486.42328621,158.10908893)(486.62120248,157.84346447)
\curveto(486.81911874,157.57784)(486.95974346,157.26534064)(487.04307662,156.90596636)
\curveto(487.13161811,156.55180041)(487.17588885,156.12471795)(487.17588885,155.62471896)
\closepath
}
}
{
\newrgbcolor{curcolor}{1 1 1}
\pscustom[linestyle=none,fillstyle=solid,fillcolor=curcolor]
{
\newpath
\moveto(493.03344001,151.8512891)
\curveto(493.03344001,151.70024774)(493.02302336,151.58045632)(493.00219007,151.49191483)
\curveto(492.98135678,151.40337334)(492.95010684,151.33826931)(492.90844026,151.29660272)
\curveto(492.86677368,151.25493614)(492.80427381,151.21587372)(492.72094064,151.17941546)
\curveto(492.63760748,151.1429572)(492.54125351,151.11431143)(492.43187873,151.09347813)
\curveto(492.32771227,151.06743652)(492.21573333,151.04660323)(492.09594191,151.03097826)
\curveto(491.97615048,151.01535329)(491.85635906,151.00754081)(491.73656764,151.00754081)
\curveto(491.37198504,151.00754081)(491.05948567,151.05441571)(490.79906953,151.14816552)
\curveto(490.53865339,151.24712366)(490.32511216,151.39295669)(490.15844583,151.58566464)
\curveto(489.9917795,151.7835809)(489.86938392,152.03097624)(489.79125907,152.32785064)
\curveto(489.71834256,152.62993336)(489.6818843,152.98409931)(489.6818843,153.39034848)
\lineto(489.6818843,157.49971516)
\lineto(488.69751129,157.49971516)
\curveto(488.61938645,157.49971516)(488.55688657,157.54138174)(488.51001167,157.62471491)
\curveto(488.46313676,157.70804807)(488.43969931,157.84346447)(488.43969931,158.03096409)
\curveto(488.43969931,158.12992222)(488.44490763,158.21325538)(488.45532428,158.28096358)
\curveto(488.47094925,158.34867178)(488.48917838,158.40335917)(488.51001167,158.44502575)
\curveto(488.53084496,158.49190065)(488.55688657,158.52315059)(488.58813651,158.53877556)
\curveto(488.62459477,158.55960885)(488.66365719,158.57002549)(488.70532377,158.57002549)
\lineto(489.6818843,158.57002549)
\lineto(489.6818843,160.24189711)
\curveto(489.6818843,160.27835537)(489.68969678,160.31220947)(489.70532175,160.3434594)
\curveto(489.72615504,160.37470934)(489.76000914,160.40075095)(489.80688404,160.42158425)
\curveto(489.85896727,160.44762586)(489.92667547,160.46585499)(490.01000863,160.47627163)
\curveto(490.0933418,160.48668828)(490.19750825,160.4918966)(490.322508,160.4918966)
\curveto(490.45271607,160.4918966)(490.55948669,160.48668828)(490.64281985,160.47627163)
\curveto(490.72615302,160.46585499)(490.79125705,160.44762586)(490.83813195,160.42158425)
\curveto(490.88500686,160.40075095)(490.91886096,160.37470934)(490.93969425,160.3434594)
\curveto(490.96052754,160.31220947)(490.97094419,160.27835537)(490.97094419,160.24189711)
\lineto(490.97094419,158.57002549)
\lineto(492.77562803,158.57002549)
\curveto(492.81729461,158.57002549)(492.85375287,158.55960885)(492.88500281,158.53877556)
\curveto(492.91625275,158.52315059)(492.94229436,158.49190065)(492.96312765,158.44502575)
\curveto(492.98916927,158.40335917)(493.0073984,158.34867178)(493.01781504,158.28096358)
\curveto(493.02823169,158.21325538)(493.03344001,158.12992222)(493.03344001,158.03096409)
\curveto(493.03344001,157.84346447)(493.01000256,157.70804807)(492.96312765,157.62471491)
\curveto(492.91625275,157.54138174)(492.85375287,157.49971516)(492.77562803,157.49971516)
\lineto(490.97094419,157.49971516)
\lineto(490.97094419,153.5778481)
\curveto(490.97094419,153.09347408)(491.04125654,152.72628733)(491.18188126,152.47628783)
\curveto(491.3277143,152.23149666)(491.58552627,152.10910108)(491.95531719,152.10910108)
\curveto(492.07510862,152.10910108)(492.18187923,152.11951772)(492.27562904,152.14035101)
\curveto(492.36937885,152.16639263)(492.45271202,152.19243424)(492.52562854,152.21847586)
\curveto(492.59854506,152.24451747)(492.66104493,152.26795492)(492.71312816,152.28878821)
\curveto(492.76521139,152.31482983)(492.81208629,152.32785064)(492.85375287,152.32785064)
\curveto(492.87979449,152.32785064)(492.90323194,152.32003815)(492.92406523,152.30441318)
\curveto(492.95010684,152.29399654)(492.96833597,152.27055908)(492.97875262,152.23410082)
\curveto(492.99437759,152.19764257)(493.0073984,152.1481635)(493.01781504,152.08566363)
\curveto(493.02823169,152.02316375)(493.03344001,151.94503891)(493.03344001,151.8512891)
\closepath
}
}
{
\newrgbcolor{curcolor}{0 0.50196081 0}
\pscustom[linestyle=none,fillstyle=solid,fillcolor=curcolor]
{
\newpath
\moveto(240.87746547,136.87376952)
\lineto(689.69545425,136.87376952)
\lineto(689.69545425,1.00002891)
\lineto(240.87746547,1.00002891)
\closepath
}
}
{
\newrgbcolor{curcolor}{0.10196079 0.8392157 0.96078432}
\pscustom[linewidth=1.99999595,linecolor=curcolor]
{
\newpath
\moveto(240.87746547,136.87376952)
\lineto(689.69545425,136.87376952)
\lineto(689.69545425,1.00002891)
\lineto(240.87746547,1.00002891)
\closepath
}
}
{
\newrgbcolor{curcolor}{1 1 1}
\pscustom[linestyle=none,fillstyle=solid,fillcolor=curcolor]
{
\newpath
\moveto(415.1302945,63.6550202)
\curveto(415.1302945,63.56127039)(415.12248202,63.47793722)(415.10685705,63.4050207)
\curveto(415.0964404,63.33210419)(415.07821127,63.26960431)(415.05216966,63.21752108)
\curveto(415.03133637,63.17064618)(415.00008643,63.13418792)(414.95841985,63.10814631)
\curveto(414.92196159,63.08731301)(414.88029501,63.07689637)(414.8334201,63.07689637)
\lineto(409.24749391,63.07689637)
\curveto(409.17457739,63.07689637)(409.10947336,63.08470885)(409.05218181,63.10033382)
\curveto(409.00009858,63.12116711)(408.95322368,63.15241705)(408.91155709,63.19408363)
\curveto(408.87509883,63.23575021)(408.84645306,63.29564593)(408.82561977,63.37377077)
\curveto(408.8099948,63.45189561)(408.80218232,63.54824958)(408.80218232,63.66283268)
\curveto(408.80218232,63.76699914)(408.80478648,63.85814479)(408.8099948,63.93626963)
\curveto(408.82041144,64.01439447)(408.83864057,64.08210267)(408.86468219,64.13939422)
\curveto(408.8907238,64.20189409)(408.92197374,64.2617898)(408.958432,64.31908135)
\curveto(409.00009858,64.38158123)(409.05218181,64.44668526)(409.11468168,64.51439346)
\lineto(411.07561521,66.59251425)
\curveto(411.52873929,67.07167995)(411.89071773,67.50136658)(412.16155051,67.88157414)
\curveto(412.43759162,68.2617817)(412.64852869,68.60813517)(412.79436173,68.92063454)
\curveto(412.94540309,69.2331339)(413.04436122,69.51698749)(413.09123613,69.77219531)
\curveto(413.13811103,70.02740313)(413.16154849,70.26698598)(413.16154849,70.49094386)
\curveto(413.16154849,70.71490174)(413.12509023,70.92583881)(413.05217371,71.12375507)
\curveto(412.97925719,71.32687966)(412.87248657,71.50396264)(412.73186186,71.655004)
\curveto(412.59644547,71.80604536)(412.42457081,71.92583678)(412.2162379,72.01437827)
\curveto(412.00790499,72.10291976)(411.76832214,72.1471905)(411.49748936,72.1471905)
\curveto(411.17978167,72.1471905)(410.89332391,72.10291976)(410.6381161,72.01437827)
\curveto(410.3881166,71.92583678)(410.16676289,71.82948281)(409.97405494,71.72531636)
\curveto(409.78655532,71.6211499)(409.62770148,71.52479593)(409.49749341,71.43625444)
\curveto(409.37249366,71.34771295)(409.27874385,71.30344221)(409.21624398,71.30344221)
\curveto(409.17978572,71.30344221)(409.14593162,71.31385886)(409.11468168,71.33469215)
\curveto(409.08864007,71.35552544)(409.06520262,71.38937954)(409.04436932,71.43625444)
\curveto(409.02874436,71.48312935)(409.01572355,71.54562922)(409.0053069,71.62375406)
\curveto(408.99489026,71.7018789)(408.98968194,71.79562871)(408.98968194,71.90500349)
\curveto(408.98968194,71.98312833)(408.9922861,72.05083653)(408.99749442,72.10812808)
\curveto(409.00270274,72.16541963)(409.01051523,72.2148987)(409.02093187,72.25656528)
\curveto(409.03655684,72.29823186)(409.05478597,72.33729428)(409.07561926,72.37375254)
\curveto(409.09645255,72.4102108)(409.13811913,72.45448155)(409.20061901,72.50656477)
\curveto(409.26311888,72.56385632)(409.3698895,72.639377)(409.52093086,72.73312681)
\curveto(409.67718054,72.82687662)(409.86988849,72.91802227)(410.09905469,73.00656376)
\curveto(410.33342921,73.10031357)(410.58863703,73.17843841)(410.86467814,73.24093829)
\curveto(411.14592757,73.30343816)(411.44019781,73.3346881)(411.74748885,73.3346881)
\curveto(412.23707119,73.3346881)(412.66415366,73.26437574)(413.02873626,73.12375102)
\curveto(413.39852717,72.98833463)(413.70321406,72.80083501)(413.94279691,72.56125216)
\curveto(414.18758808,72.32166931)(414.36987937,72.04302405)(414.4896708,71.72531636)
\curveto(414.60946222,71.40760867)(414.66935793,71.06906768)(414.66935793,70.70969341)
\curveto(414.66935793,70.3867774)(414.64071216,70.06386139)(414.58342061,69.74094537)
\curveto(414.52612906,69.42323768)(414.40373347,69.07688422)(414.21623385,68.70188498)
\curveto(414.03394255,68.33209406)(413.76831809,67.9180324)(413.41936047,67.45969999)
\curveto(413.07040284,67.00657591)(412.60686211,66.48313947)(412.02873828,65.88939067)
\lineto(410.42717902,64.21751906)
\lineto(414.82560762,64.21751906)
\curveto(414.8672742,64.21751906)(414.90633662,64.20449825)(414.94279488,64.17845664)
\curveto(414.98446146,64.15762335)(415.01831556,64.12376925)(415.04435717,64.07689434)
\curveto(415.07560711,64.03001944)(415.0964404,63.97012373)(415.10685705,63.89720721)
\curveto(415.12248202,63.82949901)(415.1302945,63.74877001)(415.1302945,63.6550202)
\closepath
}
}
{
\newrgbcolor{curcolor}{1 1 1}
\pscustom[linestyle=none,fillstyle=solid,fillcolor=curcolor]
{
\newpath
\moveto(423.17940292,66.32688979)
\curveto(423.17940292,65.78522422)(423.08825727,65.30345436)(422.90596597,64.88158021)
\curveto(422.72367468,64.45970607)(422.4658627,64.10293596)(422.13253004,63.81126988)
\curveto(421.79919738,63.52481213)(421.40076069,63.30606257)(420.93721996,63.15502121)
\curveto(420.47888755,63.00397985)(419.96847192,62.92845917)(419.40597306,62.92845917)
\curveto(419.09347369,62.92845917)(418.79659929,62.95189662)(418.51534986,62.99877153)
\curveto(418.23410043,63.04043811)(417.98410094,63.09252134)(417.76535138,63.15502121)
\curveto(417.54660183,63.22272941)(417.36691469,63.28783344)(417.22628997,63.35033331)
\curveto(417.08566526,63.41283319)(416.99451961,63.46231225)(416.95285303,63.49877051)
\curveto(416.91639477,63.53522877)(416.89035315,63.56908287)(416.87472819,63.60033281)
\curveto(416.85910322,63.63158275)(416.84347825,63.668041)(416.82785328,63.70970759)
\curveto(416.81743664,63.75658249)(416.80962415,63.81387404)(416.80441583,63.88158224)
\curveto(416.79920751,63.94929044)(416.79660334,64.03001944)(416.79660334,64.12376925)
\curveto(416.79660334,64.21231074)(416.79920751,64.29043558)(416.80441583,64.35814377)
\curveto(416.81483247,64.43106029)(416.83045744,64.48835184)(416.85129073,64.53001843)
\curveto(416.87212402,64.57689333)(416.89556148,64.61074743)(416.92160309,64.63158072)
\curveto(416.95285303,64.65241401)(416.98670713,64.66283066)(417.02316539,64.66283066)
\curveto(417.07524861,64.66283066)(417.15597762,64.62897656)(417.26535239,64.56126836)
\curveto(417.37472717,64.49876849)(417.52056021,64.42845613)(417.70285151,64.35033129)
\curveto(417.89035113,64.27741477)(418.11951733,64.20710241)(418.39035012,64.13939422)
\curveto(418.66639122,64.07168602)(418.9919114,64.03783192)(419.36691064,64.03783192)
\curveto(419.71586827,64.03783192)(420.03618012,64.07949851)(420.32784619,64.16283167)
\curveto(420.61951227,64.24616483)(420.86951176,64.3763729)(421.07784467,64.55345588)
\curveto(421.29138591,64.73053885)(421.45805224,64.95189257)(421.57784366,65.21751703)
\curveto(421.69763509,65.48834982)(421.7575308,65.81647415)(421.7575308,66.20189004)
\curveto(421.7575308,66.52480605)(421.70544757,66.80865965)(421.60128111,67.05345082)
\curveto(421.50232298,67.30345031)(421.3460733,67.50917906)(421.13253206,67.67063707)
\curveto(420.92419915,67.8373034)(420.65857469,67.95969898)(420.33565868,68.03782382)
\curveto(420.01274266,68.12115699)(419.62732678,68.16282357)(419.17941102,68.16282357)
\curveto(418.86170333,68.16282357)(418.58826638,68.1471986)(418.35910018,68.11594866)
\curveto(418.12993398,68.08469873)(417.91639274,68.06907376)(417.71847648,68.06907376)
\curveto(417.57264344,68.06907376)(417.46847698,68.10292786)(417.40597711,68.17063605)
\curveto(417.34347724,68.23834425)(417.3122273,68.36855232)(417.3122273,68.56126026)
\lineto(417.3122273,72.67843943)
\curveto(417.3122273,72.85031408)(417.35128972,72.97791799)(417.42941456,73.06125115)
\curveto(417.51274773,73.14458431)(417.62993499,73.1862509)(417.78097635,73.1862509)
\lineto(422.25752979,73.1862509)
\curveto(422.29919637,73.1862509)(422.33825879,73.17323009)(422.37471705,73.14718848)
\curveto(422.41638363,73.12635519)(422.45023773,73.09250109)(422.47627934,73.04562618)
\curveto(422.50232096,72.99875128)(422.52055009,72.93885556)(422.53096673,72.86593905)
\curveto(422.5465917,72.79302253)(422.55440418,72.7070852)(422.55440418,72.60812707)
\curveto(422.55440418,72.42062745)(422.52836257,72.27479441)(422.47627934,72.17062795)
\curveto(422.42419611,72.0664615)(422.3512796,72.01437827)(422.25752979,72.01437827)
\lineto(418.48409993,72.01437827)
\lineto(418.48409993,69.17844651)
\curveto(418.66639122,69.20448813)(418.85128668,69.2201131)(419.0387863,69.22532142)
\curveto(419.23149425,69.23052974)(419.45284797,69.2331339)(419.70284746,69.2331339)
\curveto(420.28097129,69.2331339)(420.7861786,69.16282154)(421.21846939,69.02219683)
\curveto(421.65076018,68.88678044)(422.01013445,68.69146833)(422.29659221,68.43626052)
\curveto(422.58825828,68.18626102)(422.80700784,67.88157414)(422.95284088,67.52219987)
\curveto(423.10388224,67.16282559)(423.17940292,66.7643889)(423.17940292,66.32688979)
\closepath
}
}
{
\newrgbcolor{curcolor}{1 1 1}
\pscustom[linestyle=none,fillstyle=solid,fillcolor=curcolor]
{
\newpath
\moveto(431.58007313,66.35032724)
\curveto(431.58007313,65.9076198)(431.51236493,65.47793317)(431.37694854,65.06126735)
\curveto(431.24153214,64.64980985)(431.03319923,64.28522726)(430.7519498,63.96751957)
\curveto(430.47070037,63.6550202)(430.11393026,63.40241654)(429.68163947,63.2097086)
\curveto(429.24934868,63.02220898)(428.73893305,62.92845917)(428.15039257,62.92845917)
\curveto(427.72851842,62.92845917)(427.35612335,62.9805424)(427.03320733,63.08470885)
\curveto(426.71029132,63.18887531)(426.42904189,63.33731251)(426.18945904,63.53002045)
\curveto(425.94987619,63.72272839)(425.74935577,63.95970708)(425.58789776,64.24095651)
\curveto(425.43164808,64.52220594)(425.30404417,64.84251779)(425.20508603,65.20189207)
\curveto(425.11133622,65.56126634)(425.04362803,65.95449471)(425.00196145,66.38157718)
\curveto(424.96029486,66.80865965)(424.93946157,67.26699205)(424.93946157,67.75657439)
\curveto(424.93946157,68.18886518)(424.96289903,68.6263643)(425.00977393,69.06907173)
\curveto(425.05664884,69.51177917)(425.139982,69.93886164)(425.25977342,70.35031914)
\curveto(425.37956485,70.76177664)(425.54102285,71.14719253)(425.74414744,71.5065668)
\curveto(425.95248035,71.87114939)(426.21029233,72.18625292)(426.51758338,72.45187738)
\curveto(426.83008274,72.72271017)(427.20247782,72.9362514)(427.63476861,73.09250109)
\curveto(428.06705941,73.24875077)(428.56966255,73.32687561)(429.14257806,73.32687561)
\curveto(429.335286,73.32687561)(429.53059811,73.31385481)(429.72851437,73.28781319)
\curveto(429.92643064,73.2669799)(430.10872194,73.23572996)(430.27538827,73.19406338)
\curveto(430.4420546,73.15760512)(430.58267931,73.11593854)(430.69726241,73.06906363)
\curveto(430.81184551,73.02218873)(430.88736619,72.98312631)(430.92382445,72.95187637)
\curveto(430.96028271,72.92583476)(430.98632433,72.89458482)(431.0019493,72.85812656)
\curveto(431.02278259,72.82687662)(431.03840756,72.79041837)(431.0488242,72.74875178)
\curveto(431.05924085,72.71229352)(431.06705333,72.66802278)(431.07226165,72.61593955)
\curveto(431.07746998,72.56906465)(431.08007414,72.50916893)(431.08007414,72.43625242)
\curveto(431.08007414,72.34250261)(431.07746998,72.2617736)(431.07226165,72.19406541)
\curveto(431.07226165,72.12635721)(431.06184501,72.07166982)(431.04101172,72.03000324)
\curveto(431.02538675,71.98833666)(431.0019493,71.95708672)(430.97069936,71.93625343)
\curveto(430.94465775,71.91542014)(430.90819949,71.90500349)(430.86132458,71.90500349)
\curveto(430.80403303,71.90500349)(430.72590819,71.92323262)(430.62695006,71.95969088)
\curveto(430.52799192,71.99614914)(430.4082005,72.03521156)(430.26757578,72.07687814)
\curveto(430.12695107,72.12375305)(429.95507642,72.16541963)(429.75195183,72.20187789)
\curveto(429.55403556,72.23833615)(429.3222652,72.25656528)(429.05664074,72.25656528)
\curveto(428.57226672,72.25656528)(428.15560089,72.15760715)(427.80664327,71.95969088)
\curveto(427.45768564,71.76177462)(427.17122789,71.49615015)(426.94727001,71.16281749)
\curveto(426.72852045,70.82948484)(426.56445828,70.44146479)(426.4550835,69.99875735)
\curveto(426.35091705,69.56125824)(426.29102134,69.10292583)(426.27539637,68.62376014)
\curveto(426.41081276,68.70188498)(426.56185412,68.78000982)(426.72852045,68.85813466)
\curveto(426.9003951,68.9362595)(427.08529056,69.00657186)(427.28320683,69.06907173)
\curveto(427.48633142,69.13157161)(427.69987265,69.18105067)(427.92383053,69.21750893)
\curveto(428.14778841,69.25917552)(428.38737126,69.28000881)(428.64257907,69.28000881)
\curveto(429.18945297,69.28000881)(429.65038953,69.20448813)(430.02538877,69.05344677)
\curveto(430.40038801,68.90761373)(430.70247074,68.70188498)(430.93163694,68.43626052)
\curveto(431.16601146,68.17584438)(431.33267779,67.86594917)(431.43163593,67.5065749)
\curveto(431.53059406,67.15240895)(431.58007313,66.76699306)(431.58007313,66.35032724)
\closepath
\moveto(430.22070088,66.22532749)
\curveto(430.22070088,66.52741021)(430.18945094,66.80084716)(430.12695107,67.04563833)
\curveto(430.06965952,67.29563783)(429.97070138,67.5065749)(429.83007667,67.67844955)
\curveto(429.68945195,67.8503242)(429.50195233,67.98313643)(429.26757781,68.07688624)
\curveto(429.03320328,68.17063605)(428.74414137,68.21751096)(428.40039206,68.21751096)
\curveto(428.20768412,68.21751096)(428.01497618,68.19928183)(427.82226823,68.16282357)
\curveto(427.62956029,68.13157363)(427.44206067,68.08469873)(427.25976937,68.02219885)
\curveto(427.0826864,67.9649073)(426.91081175,67.89459495)(426.74414542,67.81126178)
\curveto(426.58268741,67.73313694)(426.43425021,67.64719961)(426.29883382,67.5534498)
\curveto(426.29883382,66.88157616)(426.3405004,66.3190773)(426.42383357,65.86595322)
\curveto(426.51237505,65.41282914)(426.6373748,65.04824654)(426.79883281,64.77220544)
\curveto(426.96549914,64.50137265)(427.17122789,64.30606055)(427.41601906,64.18626912)
\curveto(427.66081023,64.07168602)(427.94726798,64.01439447)(428.27539232,64.01439447)
\curveto(428.60872498,64.01439447)(428.89778689,64.07689434)(429.14257806,64.20189409)
\curveto(429.38736923,64.32689384)(429.59049382,64.49356017)(429.75195183,64.70189308)
\curveto(429.91340983,64.91543431)(430.0305971,65.152413)(430.10351362,65.41282914)
\curveto(430.18163846,65.6784536)(430.22070088,65.94928639)(430.22070088,66.22532749)
\closepath
}
}
{
\newrgbcolor{curcolor}{1 1 1}
\pscustom[linestyle=none,fillstyle=solid,fillcolor=curcolor]
{
\newpath
\moveto(436.55887621,67.02220088)
\curveto(436.55887621,66.81907629)(436.53543876,66.67845158)(436.48856386,66.60032673)
\curveto(436.44168895,66.52220189)(436.37398076,66.48313947)(436.28543927,66.48313947)
\lineto(433.01982088,66.48313947)
\curveto(432.92607107,66.48313947)(432.85575871,66.52220189)(432.80888381,66.60032673)
\curveto(432.7620089,66.6836599)(432.73857145,66.82428461)(432.73857145,67.02220088)
\curveto(432.73857145,67.22011715)(432.7620089,67.3581377)(432.80888381,67.43626254)
\curveto(432.85575871,67.51438738)(432.92607107,67.5534498)(433.01982088,67.5534498)
\lineto(436.28543927,67.5534498)
\curveto(436.32710585,67.5534498)(436.36356411,67.54563732)(436.39481405,67.53001235)
\curveto(436.43127231,67.51438738)(436.45991808,67.48313745)(436.48075137,67.43626254)
\curveto(436.50679299,67.39459596)(436.52502212,67.33990857)(436.53543876,67.27220037)
\curveto(436.55106373,67.20449218)(436.55887621,67.12115901)(436.55887621,67.02220088)
\closepath
}
}
{
\newrgbcolor{curcolor}{1 1 1}
\pscustom[linestyle=none,fillstyle=solid,fillcolor=curcolor]
{
\newpath
\moveto(444.75954702,66.88157616)
\curveto(444.75954702,66.2722024)(444.69183882,65.72272434)(444.55642243,65.233142)
\curveto(444.42621436,64.74876798)(444.23090225,64.33470632)(443.97048611,63.99095702)
\curveto(443.7152783,63.64720771)(443.40017477,63.38418741)(443.02517553,63.20189612)
\curveto(442.65017629,63.01960482)(442.22048966,62.92845917)(441.73611564,62.92845917)
\curveto(441.51215776,62.92845917)(441.30382485,62.95189662)(441.1111169,62.99877153)
\curveto(440.92361728,63.04043811)(440.73872183,63.11075047)(440.55643053,63.2097086)
\curveto(440.37413923,63.30866673)(440.19184793,63.43366648)(440.00955664,63.58470784)
\curveto(439.82726534,63.7357492)(439.63455739,63.9180405)(439.43143281,64.13158173)
\lineto(439.43143281,63.27220847)
\curveto(439.43143281,63.23054189)(439.42101616,63.19408363)(439.40018287,63.16283369)
\curveto(439.37934958,63.13158376)(439.34549548,63.10554214)(439.29862058,63.08470885)
\curveto(439.25174567,63.06908388)(439.19184996,63.05606308)(439.11893344,63.04564643)
\curveto(439.05122524,63.03522979)(438.96268376,63.03002146)(438.85330898,63.03002146)
\curveto(438.74914252,63.03002146)(438.66060103,63.03522979)(438.58768451,63.04564643)
\curveto(438.514768,63.05606308)(438.45487228,63.06908388)(438.40799738,63.08470885)
\curveto(438.36112247,63.10554214)(438.32987254,63.13158376)(438.31424757,63.16283369)
\curveto(438.2986226,63.19408363)(438.29081012,63.23054189)(438.29081012,63.27220847)
\lineto(438.29081012,73.70968734)
\curveto(438.29081012,73.75135392)(438.2986226,73.78781218)(438.31424757,73.81906212)
\curveto(438.33508086,73.85031205)(438.36893496,73.87635367)(438.41580986,73.89718696)
\curveto(438.46789309,73.91802025)(438.53560129,73.93364522)(438.61893445,73.94406186)
\curveto(438.70226762,73.95447851)(438.80643407,73.95968683)(438.93143382,73.95968683)
\curveto(439.06164189,73.95968683)(439.16841251,73.95447851)(439.25174567,73.94406186)
\curveto(439.33507883,73.93364522)(439.40018287,73.91802025)(439.44705777,73.89718696)
\curveto(439.49393268,73.87635367)(439.52778678,73.85031205)(439.54862007,73.81906212)
\curveto(439.56945336,73.78781218)(439.57987001,73.75135392)(439.57987001,73.70968734)
\lineto(439.57987001,69.49875836)
\curveto(439.78820292,69.7122996)(439.98872334,69.89198673)(440.18143129,70.03781977)
\curveto(440.37934755,70.18365281)(440.5720555,70.30084007)(440.75955512,70.38938156)
\curveto(440.94705474,70.48313137)(441.13455436,70.55083957)(441.32205398,70.59250615)
\curveto(441.5095536,70.63417273)(441.70746986,70.65500602)(441.91580278,70.65500602)
\curveto(442.42621841,70.65500602)(442.86111336,70.55344373)(443.22048763,70.35031914)
\curveto(443.58507023,70.14719455)(443.87934047,69.87375761)(444.10329835,69.5300083)
\curveto(444.33246455,69.19146732)(444.49913088,68.79303063)(444.60329733,68.33469822)
\curveto(444.70746379,67.87636582)(444.75954702,67.3919918)(444.75954702,66.88157616)
\closepath
\moveto(443.40017477,66.73313896)
\curveto(443.40017477,67.09251324)(443.37152899,67.44147086)(443.31423744,67.78001184)
\curveto(443.26215422,68.11855283)(443.16840441,68.41803139)(443.03298801,68.67844753)
\curveto(442.89757162,68.93886366)(442.71788448,69.14719658)(442.4939266,69.30344626)
\curveto(442.26996872,69.46490427)(441.99132346,69.54563327)(441.6579908,69.54563327)
\curveto(441.49132447,69.54563327)(441.3272623,69.52219582)(441.16580429,69.47532091)
\curveto(441.00434629,69.42844601)(440.84028412,69.35032116)(440.67361779,69.24094639)
\curveto(440.50695146,69.13157161)(440.33247265,68.99094689)(440.15018135,68.81907224)
\curveto(439.97309838,68.64719759)(439.78299459,68.43105219)(439.57987001,68.17063605)
\lineto(439.57987001,65.36595423)
\curveto(439.93403595,64.93366344)(440.27257694,64.60293495)(440.59549295,64.37376874)
\curveto(440.91840896,64.14981086)(441.25434578,64.03783192)(441.60330341,64.03783192)
\curveto(441.92621942,64.03783192)(442.20226053,64.11595676)(442.43142673,64.27220645)
\curveto(442.66059293,64.42845613)(442.84548839,64.63418488)(442.98611311,64.8893927)
\curveto(443.13194615,65.14980884)(443.2361126,65.43887075)(443.29861248,65.75657844)
\curveto(443.36632067,66.07949446)(443.40017477,66.40501463)(443.40017477,66.73313896)
\closepath
}
}
{
\newrgbcolor{curcolor}{1 1 1}
\pscustom[linestyle=none,fillstyle=solid,fillcolor=curcolor]
{
\newpath
\moveto(447.97985261,63.27220847)
\curveto(447.97985261,63.23054189)(447.96943597,63.19408363)(447.94860268,63.16283369)
\curveto(447.92776939,63.13679208)(447.89391529,63.11335463)(447.84704038,63.09252134)
\curveto(447.80016548,63.07168805)(447.73506144,63.05606308)(447.65172828,63.04564643)
\curveto(447.56839511,63.03522979)(447.4616245,63.03002146)(447.33141643,63.03002146)
\curveto(447.20641668,63.03002146)(447.10225022,63.03522979)(447.01891706,63.04564643)
\curveto(446.9355839,63.05606308)(446.8678757,63.07168805)(446.81579247,63.09252134)
\curveto(446.76891757,63.11335463)(446.73506347,63.13679208)(446.71423018,63.16283369)
\curveto(446.69860521,63.19408363)(446.69079272,63.23054189)(446.69079272,63.27220847)
\lineto(446.69079272,70.30344423)
\curveto(446.69079272,70.33990249)(446.69860521,70.37375659)(446.71423018,70.40500653)
\curveto(446.73506347,70.43625647)(446.76891757,70.46229808)(446.81579247,70.48313137)
\curveto(446.8678757,70.50396466)(446.9355839,70.51958963)(447.01891706,70.53000628)
\curveto(447.10225022,70.54042292)(447.20641668,70.54563124)(447.33141643,70.54563124)
\curveto(447.4616245,70.54563124)(447.56839511,70.54042292)(447.65172828,70.53000628)
\curveto(447.73506144,70.51958963)(447.80016548,70.50396466)(447.84704038,70.48313137)
\curveto(447.89391529,70.46229808)(447.92776939,70.43625647)(447.94860268,70.40500653)
\curveto(447.96943597,70.37375659)(447.97985261,70.33990249)(447.97985261,70.30344423)
\closepath
\moveto(448.12828981,72.67843943)
\curveto(448.12828981,72.3763567)(448.07099826,72.17062795)(447.95641516,72.06125318)
\curveto(447.84183206,71.9518784)(447.63089499,71.89719101)(447.32360394,71.89719101)
\curveto(447.02152122,71.89719101)(446.81318831,71.94927424)(446.69860521,72.05344069)
\curveto(446.58923043,72.16281547)(446.53454304,72.36594006)(446.53454304,72.66281446)
\curveto(446.53454304,72.96489718)(446.59183459,73.17062593)(446.70641769,73.28000071)
\curveto(446.82100079,73.38937549)(447.03193787,73.44406288)(447.33922891,73.44406288)
\curveto(447.64131163,73.44406288)(447.84704038,73.38937549)(447.95641516,73.28000071)
\curveto(448.07099826,73.17583425)(448.12828981,72.97531382)(448.12828981,72.67843943)
\closepath
}
}
{
\newrgbcolor{curcolor}{1 1 1}
\pscustom[linestyle=none,fillstyle=solid,fillcolor=curcolor]
{
\newpath
\moveto(454.01102937,63.78783243)
\curveto(454.01102937,63.63679107)(454.00061272,63.51699964)(453.97977943,63.42845816)
\curveto(453.95894614,63.33991667)(453.9276962,63.27481263)(453.88602962,63.23314605)
\curveto(453.84436304,63.19147947)(453.78186316,63.15241705)(453.69853,63.11595879)
\curveto(453.61519683,63.07950053)(453.51884286,63.05085475)(453.40946808,63.03002146)
\curveto(453.30530163,63.00397985)(453.19332269,62.98314656)(453.07353126,62.96752159)
\curveto(452.95373984,62.95189662)(452.83394842,62.94408414)(452.71415699,62.94408414)
\curveto(452.3495744,62.94408414)(452.03707503,62.99095904)(451.77665889,63.08470885)
\curveto(451.51624275,63.18366699)(451.30270152,63.32950002)(451.13603519,63.52220797)
\curveto(450.96936886,63.72012423)(450.84697327,63.96751957)(450.76884843,64.26439396)
\curveto(450.69593191,64.56647669)(450.65947365,64.92064264)(450.65947365,65.32689181)
\lineto(450.65947365,69.43625849)
\lineto(449.67510065,69.43625849)
\curveto(449.5969758,69.43625849)(449.53447593,69.47792507)(449.48760103,69.56125824)
\curveto(449.44072612,69.6445914)(449.41728867,69.78000779)(449.41728867,69.96750742)
\curveto(449.41728867,70.06646555)(449.42249699,70.14979871)(449.43291364,70.21750691)
\curveto(449.4485386,70.28521511)(449.46676773,70.33990249)(449.48760103,70.38156908)
\curveto(449.50843432,70.42844398)(449.53447593,70.45969392)(449.56572587,70.47531889)
\curveto(449.60218413,70.49615218)(449.64124655,70.50656882)(449.68291313,70.50656882)
\lineto(450.65947365,70.50656882)
\lineto(450.65947365,72.17844044)
\curveto(450.65947365,72.2148987)(450.66728614,72.2487528)(450.68291111,72.28000273)
\curveto(450.7037444,72.31125267)(450.73759849,72.33729428)(450.7844734,72.35812757)
\curveto(450.83655663,72.38416919)(450.90426482,72.40239832)(450.98759799,72.41281496)
\curveto(451.07093115,72.42323161)(451.17509761,72.42843993)(451.30009736,72.42843993)
\curveto(451.43030543,72.42843993)(451.53707604,72.42323161)(451.62040921,72.41281496)
\curveto(451.70374237,72.40239832)(451.76884641,72.38416919)(451.81572131,72.35812757)
\curveto(451.86259622,72.33729428)(451.89645031,72.31125267)(451.91728361,72.28000273)
\curveto(451.9381169,72.2487528)(451.94853354,72.2148987)(451.94853354,72.17844044)
\lineto(451.94853354,70.50656882)
\lineto(453.75321739,70.50656882)
\curveto(453.79488397,70.50656882)(453.83134223,70.49615218)(453.86259217,70.47531889)
\curveto(453.8938421,70.45969392)(453.91988372,70.42844398)(453.94071701,70.38156908)
\curveto(453.96675862,70.33990249)(453.98498775,70.28521511)(453.9954044,70.21750691)
\curveto(454.00582104,70.14979871)(454.01102937,70.06646555)(454.01102937,69.96750742)
\curveto(454.01102937,69.78000779)(453.98759191,69.6445914)(453.94071701,69.56125824)
\curveto(453.8938421,69.47792507)(453.83134223,69.43625849)(453.75321739,69.43625849)
\lineto(451.94853354,69.43625849)
\lineto(451.94853354,65.51439143)
\curveto(451.94853354,65.03001741)(452.0188459,64.66283066)(452.15947062,64.41283116)
\curveto(452.30530365,64.16803999)(452.56311563,64.04564441)(452.93290655,64.04564441)
\curveto(453.05269797,64.04564441)(453.15946859,64.05606105)(453.2532184,64.07689434)
\curveto(453.34696821,64.10293596)(453.43030138,64.12897757)(453.50321789,64.15501919)
\curveto(453.57613441,64.1810608)(453.63863429,64.20449825)(453.69071751,64.22533154)
\curveto(453.74280074,64.25137316)(453.78967565,64.26439396)(453.83134223,64.26439396)
\curveto(453.85738384,64.26439396)(453.8808213,64.25658148)(453.90165459,64.24095651)
\curveto(453.9276962,64.23053987)(453.94592533,64.20710241)(453.95634198,64.17064415)
\curveto(453.97196694,64.13418589)(453.98498775,64.08470683)(453.9954044,64.02220695)
\curveto(454.00582104,63.95970708)(454.01102937,63.88158224)(454.01102937,63.78783243)
\closepath
}
}
{
\newrgbcolor{curcolor}{1 1 1}
\pscustom[linestyle=none,fillstyle=solid,fillcolor=curcolor]
{
\newpath
\moveto(464.36944497,64.34251881)
\curveto(464.36944497,64.25397732)(464.36684081,64.17585248)(464.36163249,64.10814428)
\curveto(464.35642417,64.04564441)(464.34600752,63.99095702)(464.33038255,63.94408211)
\curveto(464.31996591,63.90241553)(464.30434094,63.86335311)(464.28350765,63.82689485)
\curveto(464.26788268,63.79564491)(464.2262161,63.74616585)(464.1585079,63.67845765)
\curveto(464.09600803,63.61595778)(463.98663325,63.53522877)(463.83038357,63.43627064)
\curveto(463.67413388,63.34252083)(463.49705091,63.2565835)(463.29913464,63.17845866)
\curveto(463.1064267,63.10554214)(462.89548963,63.04564643)(462.66632342,62.99877153)
\curveto(462.43715722,62.95189662)(462.20017853,62.92845917)(461.95538736,62.92845917)
\curveto(461.45018005,62.92845917)(461.00226429,63.01179233)(460.61164008,63.17845866)
\curveto(460.22101587,63.34512499)(459.89289154,63.587312)(459.62726708,63.90501969)
\curveto(459.36685094,64.2279357)(459.16633051,64.62116407)(459.0257058,65.0847048)
\curveto(458.8902894,65.55345385)(458.82258121,66.09251526)(458.82258121,66.70188903)
\curveto(458.82258121,67.39459596)(458.90591437,67.98834476)(459.0725807,68.48313542)
\curveto(459.24445535,68.98313441)(459.47622572,69.39198775)(459.76789179,69.70969544)
\curveto(460.06476619,70.02740313)(460.41111966,70.26177765)(460.80695219,70.41281901)
\curveto(461.20799304,70.5690687)(461.64028383,70.64719354)(462.10382456,70.64719354)
\curveto(462.32778244,70.64719354)(462.54392784,70.62636025)(462.75226075,70.58469367)
\curveto(462.96580198,70.54302708)(463.16111409,70.48833969)(463.33819706,70.4206315)
\curveto(463.51528004,70.3529233)(463.67152972,70.27479846)(463.80694611,70.18625697)
\curveto(463.94757083,70.09771548)(464.04913312,70.0221948)(464.111633,69.95969493)
\curveto(464.17413287,69.89719506)(464.21579945,69.84771599)(464.23663274,69.81125773)
\curveto(464.26267436,69.77479947)(464.28350765,69.73052873)(464.29913262,69.6784455)
\curveto(464.31475759,69.6315706)(464.32517423,69.57688321)(464.33038255,69.51438333)
\curveto(464.33559088,69.45188346)(464.33819504,69.37375862)(464.33819504,69.28000881)
\curveto(464.33819504,69.07688422)(464.31475759,68.93365534)(464.26788268,68.85032218)
\curveto(464.22100777,68.77219734)(464.16371622,68.73313491)(464.09600803,68.73313491)
\curveto(464.01788319,68.73313491)(463.92673754,68.7748015)(463.82257108,68.85813466)
\curveto(463.72361295,68.94667615)(463.59600904,69.04303012)(463.43975936,69.14719658)
\curveto(463.28350967,69.25136303)(463.09340589,69.34511284)(462.86944801,69.42844601)
\curveto(462.65069845,69.51698749)(462.39028232,69.56125824)(462.08819959,69.56125824)
\curveto(461.46840918,69.56125824)(460.99184765,69.32167539)(460.65851499,68.84250969)
\curveto(460.33039065,68.36855232)(460.16632849,67.67844955)(460.16632849,66.77220139)
\curveto(460.16632849,66.3190773)(460.20799507,65.92064061)(460.29132823,65.57689131)
\curveto(460.37986972,65.23835033)(460.50747363,64.95449673)(460.67413996,64.72533053)
\curveto(460.84080629,64.49616433)(461.04393088,64.32428968)(461.28351372,64.20970657)
\curveto(461.52830489,64.1003318)(461.80695016,64.04564441)(462.11944953,64.04564441)
\curveto(462.41632393,64.04564441)(462.67674007,64.09251931)(462.90069795,64.18626912)
\curveto(463.12465583,64.28001893)(463.31736377,64.38158123)(463.47882178,64.49095601)
\curveto(463.64548811,64.60553911)(463.78350866,64.7071014)(463.89288344,64.79564289)
\curveto(464.00746654,64.8893927)(464.09600803,64.9362676)(464.1585079,64.9362676)
\curveto(464.19496616,64.9362676)(464.2262161,64.92585096)(464.25225771,64.90501767)
\curveto(464.27829933,64.88418438)(464.29913262,64.84772612)(464.31475759,64.79564289)
\curveto(464.33559088,64.74876798)(464.34861168,64.68626811)(464.35382001,64.60814327)
\curveto(464.36423665,64.53522675)(464.36944497,64.44668526)(464.36944497,64.34251881)
\closepath
}
}
{
\newrgbcolor{curcolor}{1 1 1}
\pscustom[linestyle=none,fillstyle=solid,fillcolor=curcolor]
{
\newpath
\moveto(472.12142818,63.27220847)
\curveto(472.12142818,63.23054189)(472.11101153,63.19408363)(472.09017824,63.16283369)
\curveto(472.06934495,63.13679208)(472.03549085,63.11335463)(471.98861595,63.09252134)
\curveto(471.94174104,63.07168805)(471.87663701,63.05606308)(471.79330384,63.04564643)
\curveto(471.70997068,63.03522979)(471.60580422,63.03002146)(471.48080448,63.03002146)
\curveto(471.35059641,63.03002146)(471.24382579,63.03522979)(471.16049262,63.04564643)
\curveto(471.07715946,63.05606308)(471.01205542,63.07168805)(470.96518052,63.09252134)
\curveto(470.91830561,63.11335463)(470.88445152,63.13679208)(470.86361822,63.16283369)
\curveto(470.84278493,63.19408363)(470.83236829,63.23054189)(470.83236829,63.27220847)
\lineto(470.83236829,67.38938764)
\curveto(470.83236829,67.79042849)(470.80111835,68.1133445)(470.73861848,68.35813567)
\curveto(470.6761186,68.60292685)(470.58497296,68.81386392)(470.46518153,68.99094689)
\curveto(470.34539011,69.16802987)(470.18914042,69.30344626)(469.99643248,69.39719607)
\curveto(469.80893286,69.49094588)(469.5901833,69.53782079)(469.34018381,69.53782079)
\curveto(469.0172678,69.53782079)(468.69435178,69.42323768)(468.37143577,69.19407148)
\curveto(468.04851976,68.96490528)(467.70997878,68.62896846)(467.35581283,68.18626102)
\lineto(467.35581283,63.27220847)
\curveto(467.35581283,63.23054189)(467.34539618,63.19408363)(467.32456289,63.16283369)
\curveto(467.3037296,63.13679208)(467.2698755,63.11335463)(467.2230006,63.09252134)
\curveto(467.17612569,63.07168805)(467.11102166,63.05606308)(467.02768849,63.04564643)
\curveto(466.94435533,63.03522979)(466.83758471,63.03002146)(466.70737664,63.03002146)
\curveto(466.58237689,63.03002146)(466.47821044,63.03522979)(466.39487727,63.04564643)
\curveto(466.31154411,63.05606308)(466.24383591,63.07168805)(466.19175269,63.09252134)
\curveto(466.14487778,63.11335463)(466.11102368,63.13679208)(466.09019039,63.16283369)
\curveto(466.07456542,63.19408363)(466.06675294,63.23054189)(466.06675294,63.27220847)
\lineto(466.06675294,73.70968734)
\curveto(466.06675294,73.75135392)(466.07456542,73.78781218)(466.09019039,73.81906212)
\curveto(466.11102368,73.85031205)(466.14487778,73.87635367)(466.19175269,73.89718696)
\curveto(466.24383591,73.91802025)(466.31154411,73.93364522)(466.39487727,73.94406186)
\curveto(466.47821044,73.95447851)(466.58237689,73.95968683)(466.70737664,73.95968683)
\curveto(466.83758471,73.95968683)(466.94435533,73.95447851)(467.02768849,73.94406186)
\curveto(467.11102166,73.93364522)(467.17612569,73.91802025)(467.2230006,73.89718696)
\curveto(467.2698755,73.87635367)(467.3037296,73.85031205)(467.32456289,73.81906212)
\curveto(467.34539618,73.78781218)(467.35581283,73.75135392)(467.35581283,73.70968734)
\lineto(467.35581283,69.49875836)
\curveto(467.72560375,69.88938257)(468.09799883,70.17844449)(468.47299807,70.36594411)
\curveto(468.84799731,70.55865205)(469.22560071,70.65500602)(469.60580827,70.65500602)
\curveto(470.07455732,70.65500602)(470.46778569,70.57427702)(470.78549338,70.41281901)
\curveto(471.1084094,70.25656933)(471.36882554,70.04563226)(471.5667418,69.78000779)
\curveto(471.76465807,69.51438333)(471.90528278,69.20188397)(471.98861595,68.84250969)
\curveto(472.07715743,68.48834374)(472.12142818,68.05865711)(472.12142818,67.5534498)
\closepath
}
}
{
\newrgbcolor{curcolor}{1 1 1}
\pscustom[linestyle=none,fillstyle=solid,fillcolor=curcolor]
{
\newpath
\moveto(480.4667253,67.09251324)
\curveto(480.4667253,66.88938865)(480.41464208,66.74355561)(480.31047562,66.65501412)
\curveto(480.21151749,66.57168096)(480.09693439,66.53001438)(479.96672632,66.53001438)
\lineto(475.35736065,66.53001438)
\curveto(475.35736065,66.13939017)(475.39642307,65.78782838)(475.47454791,65.47532901)
\curveto(475.55267276,65.16282964)(475.68288083,64.89460102)(475.86517212,64.67064314)
\curveto(476.04746342,64.44668526)(476.28444211,64.27481061)(476.57610818,64.15501919)
\curveto(476.86777426,64.03522776)(477.22454437,63.97533205)(477.64641852,63.97533205)
\curveto(477.97975117,63.97533205)(478.27662557,64.00137366)(478.53704171,64.05345689)
\curveto(478.79745785,64.11074844)(479.02141573,64.17324832)(479.20891535,64.24095651)
\curveto(479.40162329,64.30866471)(479.55787298,64.36856042)(479.6776644,64.42064365)
\curveto(479.80266415,64.4779352)(479.89641396,64.50658097)(479.95891383,64.50658097)
\curveto(479.99537209,64.50658097)(480.02662203,64.49616433)(480.05266364,64.47533104)
\curveto(480.08391358,64.45970607)(480.10735103,64.43366445)(480.122976,64.3972062)
\curveto(480.13860097,64.36074794)(480.14901761,64.30866471)(480.15422594,64.24095651)
\curveto(480.16464258,64.17845664)(480.16985091,64.1003318)(480.16985091,64.00658199)
\curveto(480.16985091,63.93887379)(480.16724674,63.87897808)(480.16203842,63.82689485)
\curveto(480.1568301,63.78001994)(480.14901761,63.7357492)(480.13860097,63.69408262)
\curveto(480.13339265,63.65762436)(480.12037184,63.62377026)(480.09953855,63.59252032)
\curveto(480.08391358,63.56127039)(480.06047613,63.53002045)(480.02922619,63.49877051)
\curveto(480.00318458,63.4727289)(479.91985141,63.425854)(479.7792267,63.3581458)
\curveto(479.63860198,63.29564593)(479.45631068,63.23314605)(479.2323528,63.17064618)
\curveto(479.00839492,63.10814631)(478.74797878,63.05345892)(478.45110439,63.00658401)
\curveto(478.15943831,62.95450078)(477.84693894,62.92845917)(477.51360628,62.92845917)
\curveto(476.93548246,62.92845917)(476.42767098,63.00918817)(475.99017187,63.17064618)
\curveto(475.55788108,63.33210419)(475.19329848,63.57168703)(474.89642408,63.88939472)
\curveto(474.59954969,64.20710241)(474.37559181,64.60553911)(474.22455044,65.0847048)
\curveto(474.07350908,65.5638705)(473.9979884,66.12116104)(473.9979884,66.75657642)
\curveto(473.9979884,67.36074186)(474.07611325,67.90240743)(474.23236293,68.38157313)
\curveto(474.38861261,68.86594715)(474.61257049,69.27480048)(474.90423657,69.60813314)
\curveto(475.20111097,69.94667412)(475.55788108,70.2044861)(475.9745469,70.38156908)
\curveto(476.39121272,70.56386037)(476.85735761,70.65500602)(477.37298157,70.65500602)
\curveto(477.92506378,70.65500602)(478.39381284,70.56646454)(478.77922872,70.38938156)
\curveto(479.16985293,70.21229859)(479.49016478,69.97271574)(479.74016428,69.67063302)
\curveto(479.99016377,69.37375862)(480.17245507,69.02219683)(480.28703817,68.61594765)
\curveto(480.40682959,68.2149068)(480.4667253,67.78522017)(480.4667253,67.32688776)
\closepath
\moveto(479.16985293,67.47532496)
\curveto(479.1854779,68.15240692)(479.03443654,68.68365585)(478.71672885,69.06907173)
\curveto(478.40422948,69.45448762)(477.93808459,69.64719556)(477.31829418,69.64719556)
\curveto(477.00058649,69.64719556)(476.72194122,69.58729985)(476.48235837,69.46750843)
\curveto(476.24277552,69.347717)(476.0422551,69.18886316)(475.88079709,68.99094689)
\curveto(475.71933908,68.79303063)(475.59433934,68.56126026)(475.50579785,68.2956358)
\curveto(475.41725636,68.03521966)(475.3677773,67.76178272)(475.35736065,67.47532496)
\closepath
}
}
{
\newrgbcolor{curcolor}{1 1 1}
\pscustom[linestyle=none,fillstyle=solid,fillcolor=curcolor]
{
\newpath
\moveto(487.47339447,64.34251881)
\curveto(487.47339447,64.25397732)(487.4707903,64.17585248)(487.46558198,64.10814428)
\curveto(487.46037366,64.04564441)(487.44995701,63.99095702)(487.43433204,63.94408211)
\curveto(487.4239154,63.90241553)(487.40829043,63.86335311)(487.38745714,63.82689485)
\curveto(487.37183217,63.79564491)(487.33016559,63.74616585)(487.26245739,63.67845765)
\curveto(487.19995752,63.61595778)(487.09058274,63.53522877)(486.93433306,63.43627064)
\curveto(486.77808337,63.34252083)(486.6010004,63.2565835)(486.40308413,63.17845866)
\curveto(486.21037619,63.10554214)(485.99943912,63.04564643)(485.77027291,62.99877153)
\curveto(485.54110671,62.95189662)(485.30412803,62.92845917)(485.05933685,62.92845917)
\curveto(484.55412954,62.92845917)(484.10621378,63.01179233)(483.71558958,63.17845866)
\curveto(483.32496537,63.34512499)(482.99684103,63.587312)(482.73121657,63.90501969)
\curveto(482.47080043,64.2279357)(482.27028,64.62116407)(482.12965529,65.0847048)
\curveto(481.99423889,65.55345385)(481.9265307,66.09251526)(481.9265307,66.70188903)
\curveto(481.9265307,67.39459596)(482.00986386,67.98834476)(482.17653019,68.48313542)
\curveto(482.34840484,68.98313441)(482.58017521,69.39198775)(482.87184128,69.70969544)
\curveto(483.16871568,70.02740313)(483.51506915,70.26177765)(483.91090168,70.41281901)
\curveto(484.31194253,70.5690687)(484.74423333,70.64719354)(485.20777405,70.64719354)
\curveto(485.43173193,70.64719354)(485.64787733,70.62636025)(485.85621024,70.58469367)
\curveto(486.06975147,70.54302708)(486.26506358,70.48833969)(486.44214655,70.4206315)
\curveto(486.61922953,70.3529233)(486.77547921,70.27479846)(486.9108956,70.18625697)
\curveto(487.05152032,70.09771548)(487.15308261,70.0221948)(487.21558249,69.95969493)
\curveto(487.27808236,69.89719506)(487.31974894,69.84771599)(487.34058223,69.81125773)
\curveto(487.36662385,69.77479947)(487.38745714,69.73052873)(487.40308211,69.6784455)
\curveto(487.41870708,69.6315706)(487.42912372,69.57688321)(487.43433204,69.51438333)
\curveto(487.43954037,69.45188346)(487.44214453,69.37375862)(487.44214453,69.28000881)
\curveto(487.44214453,69.07688422)(487.41870708,68.93365534)(487.37183217,68.85032218)
\curveto(487.32495727,68.77219734)(487.26766572,68.73313491)(487.19995752,68.73313491)
\curveto(487.12183268,68.73313491)(487.03068703,68.7748015)(486.92652057,68.85813466)
\curveto(486.82756244,68.94667615)(486.69995853,69.04303012)(486.54370885,69.14719658)
\curveto(486.38745916,69.25136303)(486.19735538,69.34511284)(485.9733975,69.42844601)
\curveto(485.75464795,69.51698749)(485.49423181,69.56125824)(485.19214909,69.56125824)
\curveto(484.57235867,69.56125824)(484.09579714,69.32167539)(483.76246448,68.84250969)
\curveto(483.43434014,68.36855232)(483.27027798,67.67844955)(483.27027798,66.77220139)
\curveto(483.27027798,66.3190773)(483.31194456,65.92064061)(483.39527772,65.57689131)
\curveto(483.48381921,65.23835033)(483.61142312,64.95449673)(483.77808945,64.72533053)
\curveto(483.94475578,64.49616433)(484.14788037,64.32428968)(484.38746321,64.20970657)
\curveto(484.63225439,64.1003318)(484.91089965,64.04564441)(485.22339902,64.04564441)
\curveto(485.52027342,64.04564441)(485.78068956,64.09251931)(486.00464744,64.18626912)
\curveto(486.22860532,64.28001893)(486.42131326,64.38158123)(486.58277127,64.49095601)
\curveto(486.7494376,64.60553911)(486.88745815,64.7071014)(486.99683293,64.79564289)
\curveto(487.11141603,64.8893927)(487.19995752,64.9362676)(487.26245739,64.9362676)
\curveto(487.29891565,64.9362676)(487.33016559,64.92585096)(487.3562072,64.90501767)
\curveto(487.38224882,64.88418438)(487.40308211,64.84772612)(487.41870708,64.79564289)
\curveto(487.43954037,64.74876798)(487.45256117,64.68626811)(487.4577695,64.60814327)
\curveto(487.46818614,64.53522675)(487.47339447,64.44668526)(487.47339447,64.34251881)
\closepath
}
}
{
\newrgbcolor{curcolor}{1 1 1}
\pscustom[linestyle=none,fillstyle=solid,fillcolor=curcolor]
{
\newpath
\moveto(494.94413587,63.28002096)
\curveto(494.94413587,63.23835438)(494.93371922,63.20189612)(494.91288593,63.17064618)
\curveto(494.89205264,63.13939624)(494.85559438,63.11335463)(494.80351115,63.09252134)
\curveto(494.75663625,63.07168805)(494.68892805,63.05606308)(494.60038656,63.04564643)
\curveto(494.51184508,63.03522979)(494.39986614,63.03002146)(494.26444974,63.03002146)
\curveto(494.12382503,63.03002146)(494.00663777,63.03522979)(493.91288796,63.04564643)
\curveto(493.82434647,63.05085475)(493.74882579,63.0612714)(493.68632592,63.07689637)
\curveto(493.62382604,63.09772966)(493.57174281,63.12377127)(493.53007623,63.15502121)
\curveto(493.49361797,63.19147947)(493.45976387,63.23314605)(493.42851394,63.28002096)
\lineto(490.45976995,67.17063808)
\lineto(490.45976995,63.27220847)
\curveto(490.45976995,63.23054189)(490.4493533,63.19408363)(490.42852001,63.16283369)
\curveto(490.40768672,63.13679208)(490.37383262,63.11335463)(490.32695772,63.09252134)
\curveto(490.28008281,63.07168805)(490.21497878,63.05606308)(490.13164561,63.04564643)
\curveto(490.04831245,63.03522979)(489.94154183,63.03002146)(489.81133376,63.03002146)
\curveto(489.68633402,63.03002146)(489.58216756,63.03522979)(489.49883439,63.04564643)
\curveto(489.41550123,63.05606308)(489.34779303,63.07168805)(489.29570981,63.09252134)
\curveto(489.2488349,63.11335463)(489.2149808,63.13679208)(489.19414751,63.16283369)
\curveto(489.17852254,63.19408363)(489.17071006,63.23054189)(489.17071006,63.27220847)
\lineto(489.17071006,73.70968734)
\curveto(489.17071006,73.75135392)(489.17852254,73.78781218)(489.19414751,73.81906212)
\curveto(489.2149808,73.85031205)(489.2488349,73.87635367)(489.29570981,73.89718696)
\curveto(489.34779303,73.91802025)(489.41550123,73.93364522)(489.49883439,73.94406186)
\curveto(489.58216756,73.95447851)(489.68633402,73.95968683)(489.81133376,73.95968683)
\curveto(489.94154183,73.95968683)(490.04831245,73.95447851)(490.13164561,73.94406186)
\curveto(490.21497878,73.93364522)(490.28008281,73.91802025)(490.32695772,73.89718696)
\curveto(490.37383262,73.87635367)(490.40768672,73.85031205)(490.42852001,73.81906212)
\curveto(490.4493533,73.78781218)(490.45976995,73.75135392)(490.45976995,73.70968734)
\lineto(490.45976995,67.35032521)
\lineto(493.11601457,70.2721943)
\curveto(493.15768115,70.32427753)(493.2019519,70.36594411)(493.2488268,70.39719405)
\curveto(493.29570171,70.4336523)(493.3503891,70.46229808)(493.41288897,70.48313137)
\curveto(493.48059717,70.50917298)(493.55872201,70.52479795)(493.64726349,70.53000628)
\curveto(493.73580498,70.54042292)(493.8425756,70.54563124)(493.96757535,70.54563124)
\curveto(494.09778342,70.54563124)(494.20715819,70.54042292)(494.29569968,70.53000628)
\curveto(494.38424117,70.52479795)(494.45455353,70.51177715)(494.50663675,70.49094386)
\curveto(494.5639283,70.47531889)(494.60299073,70.45188143)(494.62382402,70.4206315)
\curveto(494.64986563,70.39458988)(494.66288644,70.35813162)(494.66288644,70.31125672)
\curveto(494.66288644,70.24875685)(494.64465731,70.18625697)(494.60819905,70.1237571)
\curveto(494.57694911,70.06125723)(494.52226172,69.98834071)(494.44413688,69.90500754)
\lineto(491.89726704,67.3581377)
\lineto(494.75663625,63.64720771)
\curveto(494.82434444,63.55866623)(494.87121935,63.48574971)(494.89726096,63.42845816)
\curveto(494.9285109,63.37637493)(494.94413587,63.32689586)(494.94413587,63.28002096)
\closepath
}
}
{
\newrgbcolor{curcolor}{1 1 1}
\pscustom[linestyle=none,fillstyle=solid,fillcolor=curcolor]
{
\newpath
\moveto(500.70474432,65.1862671)
\curveto(500.70474432,64.82689282)(500.63703612,64.50658097)(500.50161973,64.22533154)
\curveto(500.37141166,63.94408211)(500.18391204,63.70710343)(499.93912087,63.51439548)
\curveto(499.6943297,63.32168754)(499.40266362,63.1758545)(499.06412264,63.07689637)
\curveto(498.72558166,62.97793824)(498.35318658,62.92845917)(497.9469374,62.92845917)
\curveto(497.69693791,62.92845917)(497.45735506,62.94929246)(497.22818886,62.99095904)
\curveto(497.00423098,63.0274173)(496.80110639,63.07429221)(496.61881509,63.13158376)
\curveto(496.44173212,63.19408363)(496.29069076,63.2565835)(496.16569101,63.31908338)
\curveto(496.04069126,63.38679157)(495.94954562,63.44668729)(495.89225407,63.49877051)
\curveto(495.83496251,63.55085374)(495.79329593,63.62377026)(495.76725432,63.71752007)
\curveto(495.7412127,63.81126988)(495.7281919,63.93887379)(495.7281919,64.1003318)
\curveto(495.7281919,64.19928993)(495.73340022,64.28262309)(495.74381687,64.35033129)
\curveto(495.75423351,64.41803949)(495.76725432,64.47272688)(495.78287929,64.51439346)
\curveto(495.79850426,64.55606004)(495.81933755,64.58470582)(495.84537916,64.60033078)
\curveto(495.8766291,64.62116407)(495.9104832,64.63158072)(495.94694145,64.63158072)
\curveto(496.00423301,64.63158072)(496.08756617,64.59512246)(496.19694095,64.52220594)
\curveto(496.31152405,64.45449775)(496.4495446,64.37897707)(496.61100261,64.2956439)
\curveto(496.77766894,64.21231074)(496.97298104,64.13418589)(497.19693892,64.06126938)
\curveto(497.4208968,63.99356118)(497.67870878,63.95970708)(497.97037486,63.95970708)
\curveto(498.18912441,63.95970708)(498.38704068,63.98314453)(498.56412365,64.03001944)
\curveto(498.74120663,64.07689434)(498.89485215,64.14460254)(499.02506022,64.23314403)
\curveto(499.15526829,64.32689384)(499.25422642,64.4440811)(499.32193462,64.58470582)
\curveto(499.39485114,64.72533053)(499.4313094,64.89199686)(499.4313094,65.0847048)
\curveto(499.4313094,65.28262107)(499.37922617,65.4492874)(499.27505972,65.58470379)
\curveto(499.17610158,65.72012018)(499.04328935,65.83991161)(498.87662302,65.94407806)
\curveto(498.70995669,66.04824452)(498.52245707,66.13939017)(498.31412416,66.21751501)
\curveto(498.10579125,66.30084817)(497.88964585,66.3867855)(497.66568797,66.47532699)
\curveto(497.44693842,66.56386847)(497.23079302,66.66282661)(497.01725179,66.77220139)
\curveto(496.80891888,66.88678449)(496.62141926,67.02480504)(496.45475293,67.18626305)
\curveto(496.2880866,67.34772105)(496.1526702,67.540429)(496.04850375,67.76438688)
\curveto(495.94954562,67.98834476)(495.90006655,68.25657338)(495.90006655,68.56907275)
\curveto(495.90006655,68.84511385)(495.95214978,69.10813416)(496.05631623,69.35813365)
\curveto(496.16569101,69.61334147)(496.32714902,69.83469518)(496.54069025,70.0221948)
\curveto(496.75423149,70.21490275)(497.01985595,70.36854827)(497.33756364,70.48313137)
\curveto(497.66047965,70.59771447)(498.03547889,70.65500602)(498.46256136,70.65500602)
\curveto(498.65006098,70.65500602)(498.8375606,70.63938105)(499.02506022,70.60813112)
\curveto(499.21255984,70.57688118)(499.38183033,70.53781876)(499.53287169,70.49094386)
\curveto(499.68391305,70.44406895)(499.81151696,70.39198572)(499.91568342,70.33469417)
\curveto(500.0250582,70.28261094)(500.1057872,70.23573604)(500.15787043,70.19406946)
\curveto(500.21516198,70.15240287)(500.25162024,70.11594461)(500.26724521,70.08469468)
\curveto(500.2880785,70.05344474)(500.3010993,70.01698648)(500.30630763,69.9753199)
\curveto(500.31672427,69.93886164)(500.32453676,69.89198673)(500.32974508,69.83469518)
\curveto(500.34016172,69.77740363)(500.34537005,69.70709128)(500.34537005,69.62375811)
\curveto(500.34537005,69.53521662)(500.34016172,69.45709178)(500.32974508,69.38938359)
\curveto(500.32453676,69.32688371)(500.31151595,69.27480048)(500.29068266,69.2331339)
\curveto(500.27505769,69.19146732)(500.2542244,69.16021738)(500.22818279,69.13938409)
\curveto(500.20214117,69.12375912)(500.1734954,69.11594664)(500.14224546,69.11594664)
\curveto(500.09537055,69.11594664)(500.02766236,69.14459241)(499.93912087,69.20188397)
\curveto(499.85057938,69.25917552)(499.73599628,69.31907123)(499.59537157,69.3815711)
\curveto(499.45474685,69.4492793)(499.28808052,69.51177917)(499.09537258,69.56907072)
\curveto(498.90787296,69.62636227)(498.69172756,69.65500805)(498.44693639,69.65500805)
\curveto(498.22818684,69.65500805)(498.03547889,69.62896643)(497.86881256,69.57688321)
\curveto(497.70214623,69.5300083)(497.56412568,69.45969594)(497.4547509,69.36594613)
\curveto(497.35058445,69.27740465)(497.26985544,69.17063403)(497.21256389,69.04563428)
\curveto(497.16048066,68.92063454)(497.13443905,68.78521814)(497.13443905,68.6393851)
\curveto(497.13443905,68.43626052)(497.18652228,68.26438586)(497.29068873,68.12376115)
\curveto(497.39485519,67.98834476)(497.53027158,67.86855333)(497.69693791,67.76438688)
\curveto(497.86360424,67.66022042)(498.05370802,67.56647061)(498.26724926,67.48313745)
\curveto(498.48079049,67.39980428)(498.69693589,67.31386696)(498.91568544,67.22532547)
\curveto(499.13964332,67.13678398)(499.35839288,67.03782585)(499.57193411,66.92845107)
\curveto(499.79068367,66.81907629)(499.98339161,66.68626406)(500.15005794,66.53001438)
\curveto(500.31672427,66.37376469)(500.4495365,66.18626507)(500.54849464,65.96751552)
\curveto(500.65266109,65.74876596)(500.70474432,65.48834982)(500.70474432,65.1862671)
\closepath
}
}
{
\newrgbcolor{curcolor}{1 1 1}
\pscustom[linestyle=none,fillstyle=solid,fillcolor=curcolor]
{
\newpath
\moveto(508.57010773,63.27220847)
\curveto(508.57010773,63.23054189)(508.55969108,63.19408363)(508.53885779,63.16283369)
\curveto(508.52323282,63.13679208)(508.49198288,63.11335463)(508.44510798,63.09252134)
\curveto(508.39823307,63.07168805)(508.3357332,63.05606308)(508.25760836,63.04564643)
\curveto(508.18469184,63.03522979)(508.09354619,63.03002146)(507.98417141,63.03002146)
\curveto(507.86437999,63.03002146)(507.76542186,63.03522979)(507.68729701,63.04564643)
\curveto(507.61438049,63.05606308)(507.55448478,63.07168805)(507.50760988,63.09252134)
\curveto(507.4659433,63.11335463)(507.43729752,63.13679208)(507.42167255,63.16283369)
\curveto(507.40604758,63.19408363)(507.3982351,63.23054189)(507.3982351,63.27220847)
\lineto(507.3982351,64.20189409)
\curveto(506.99719424,63.75918665)(506.60136171,63.43627064)(506.2107375,63.23314605)
\curveto(505.82011329,63.03002146)(505.42428076,62.92845917)(505.02323991,62.92845917)
\curveto(504.55449086,62.92845917)(504.15865833,63.00658401)(503.83574231,63.16283369)
\curveto(503.51803462,63.31908338)(503.26022265,63.53002045)(503.06230638,63.79564491)
\curveto(502.86439011,64.0664777)(502.72116124,64.37897707)(502.63261975,64.73314301)
\curveto(502.54928658,65.09251729)(502.50762,65.52741224)(502.50762,66.03782787)
\lineto(502.50762,70.30344423)
\curveto(502.50762,70.34511082)(502.51543249,70.37896492)(502.53105746,70.40500653)
\curveto(502.55189075,70.43625647)(502.58834901,70.46229808)(502.64043223,70.48313137)
\curveto(502.69251546,70.50917298)(502.76022366,70.52479795)(502.84355682,70.53000628)
\curveto(502.92688999,70.54042292)(503.03105644,70.54563124)(503.15605619,70.54563124)
\curveto(503.28105594,70.54563124)(503.38522239,70.54042292)(503.46855556,70.53000628)
\curveto(503.55188872,70.52479795)(503.61699276,70.50917298)(503.66386766,70.48313137)
\curveto(503.71595089,70.46229808)(503.75240915,70.43625647)(503.77324244,70.40500653)
\curveto(503.79407573,70.37896492)(503.80449238,70.34511082)(503.80449238,70.30344423)
\lineto(503.80449238,66.20970252)
\curveto(503.80449238,65.79824502)(503.83313815,65.46751653)(503.8904297,65.21751703)
\curveto(503.95292958,64.97272586)(504.04407522,64.76178879)(504.16386665,64.58470582)
\curveto(504.2888664,64.41283116)(504.44511608,64.27741477)(504.6326157,64.17845664)
\curveto(504.82011532,64.08470683)(505.03886488,64.03783192)(505.28886437,64.03783192)
\curveto(505.61178038,64.03783192)(505.93209223,64.15241502)(506.24979992,64.38158123)
\curveto(506.57271594,64.61074743)(506.91386108,64.94668425)(507.27323535,65.38939169)
\lineto(507.27323535,70.30344423)
\curveto(507.27323535,70.34511082)(507.28104784,70.37896492)(507.2966728,70.40500653)
\curveto(507.3175061,70.43625647)(507.35396436,70.46229808)(507.40604758,70.48313137)
\curveto(507.45813081,70.50917298)(507.52323485,70.52479795)(507.60135969,70.53000628)
\curveto(507.68469285,70.54042292)(507.79146347,70.54563124)(507.92167154,70.54563124)
\curveto(508.04667129,70.54563124)(508.15083774,70.54042292)(508.23417091,70.53000628)
\curveto(508.31750407,70.52479795)(508.38260811,70.50917298)(508.42948301,70.48313137)
\curveto(508.47635792,70.46229808)(508.51021201,70.43625647)(508.53104531,70.40500653)
\curveto(508.55708692,70.37896492)(508.57010773,70.34511082)(508.57010773,70.30344423)
\closepath
}
}
{
\newrgbcolor{curcolor}{1 1 1}
\pscustom[linestyle=none,fillstyle=solid,fillcolor=curcolor]
{
\newpath
\moveto(521.37632569,63.27220847)
\curveto(521.37632569,63.23054189)(521.36590904,63.19408363)(521.34507575,63.16283369)
\curveto(521.32424246,63.13679208)(521.29038836,63.11335463)(521.24351346,63.09252134)
\curveto(521.19663855,63.07168805)(521.13153452,63.05606308)(521.04820135,63.04564643)
\curveto(520.96486819,63.03522979)(520.86070173,63.03002146)(520.73570199,63.03002146)
\curveto(520.60549392,63.03002146)(520.4987233,63.03522979)(520.41539014,63.04564643)
\curveto(520.33205697,63.05606308)(520.26434878,63.07168805)(520.21226555,63.09252134)
\curveto(520.16539064,63.11335463)(520.13153654,63.13679208)(520.11070325,63.16283369)
\curveto(520.08986996,63.19408363)(520.07945332,63.23054189)(520.07945332,63.27220847)
\lineto(520.07945332,67.54563732)
\curveto(520.07945332,67.84251172)(520.0534117,68.1133445)(520.00132847,68.35813567)
\curveto(519.94924525,68.60292685)(519.86591208,68.81386392)(519.75132898,68.99094689)
\curveto(519.63674588,69.16802987)(519.49091284,69.30344626)(519.31382987,69.39719607)
\curveto(519.13674689,69.49094588)(518.92841398,69.53782079)(518.68883113,69.53782079)
\curveto(518.39195673,69.53782079)(518.09247817,69.42323768)(517.79039545,69.19407148)
\curveto(517.49352105,68.96490528)(517.16539672,68.62896846)(516.80602244,68.18626102)
\lineto(516.80602244,63.27220847)
\curveto(516.80602244,63.23054189)(516.7956058,63.19408363)(516.77477251,63.16283369)
\curveto(516.75393922,63.13679208)(516.71748096,63.11335463)(516.66539773,63.09252134)
\curveto(516.61852282,63.07168805)(516.55341879,63.05606308)(516.47008563,63.04564643)
\curveto(516.38675246,63.03522979)(516.282586,63.03002146)(516.15758626,63.03002146)
\curveto(516.03779483,63.03002146)(515.93362838,63.03522979)(515.84508689,63.04564643)
\curveto(515.76175373,63.05606308)(515.69404553,63.07168805)(515.6419623,63.09252134)
\curveto(515.5950874,63.11335463)(515.5612333,63.13679208)(515.54040001,63.16283369)
\curveto(515.52477504,63.19408363)(515.51696256,63.23054189)(515.51696256,63.27220847)
\lineto(515.51696256,67.54563732)
\curveto(515.51696256,67.84251172)(515.48831678,68.1133445)(515.43102523,68.35813567)
\curveto(515.37373368,68.60292685)(515.28779635,68.81386392)(515.17321325,68.99094689)
\curveto(515.05863015,69.16802987)(514.91279711,69.30344626)(514.73571414,69.39719607)
\curveto(514.56383949,69.49094588)(514.35811074,69.53782079)(514.11852789,69.53782079)
\curveto(513.82165349,69.53782079)(513.52217493,69.42323768)(513.22009221,69.19407148)
\curveto(512.91800948,68.96490528)(512.58988515,68.62896846)(512.2357192,68.18626102)
\lineto(512.2357192,63.27220847)
\curveto(512.2357192,63.23054189)(512.22530255,63.19408363)(512.20446926,63.16283369)
\curveto(512.18363597,63.13679208)(512.14978187,63.11335463)(512.10290697,63.09252134)
\curveto(512.05603206,63.07168805)(511.99092803,63.05606308)(511.90759486,63.04564643)
\curveto(511.8242617,63.03522979)(511.71749108,63.03002146)(511.58728301,63.03002146)
\curveto(511.46228327,63.03002146)(511.35811681,63.03522979)(511.27478365,63.04564643)
\curveto(511.19145048,63.05606308)(511.12374228,63.07168805)(511.07165906,63.09252134)
\curveto(511.02478415,63.11335463)(510.99093005,63.13679208)(510.97009676,63.16283369)
\curveto(510.95447179,63.19408363)(510.94665931,63.23054189)(510.94665931,63.27220847)
\lineto(510.94665931,70.30344423)
\curveto(510.94665931,70.34511082)(510.95447179,70.37896492)(510.97009676,70.40500653)
\curveto(510.98572173,70.43625647)(511.01697167,70.46229808)(511.06384657,70.48313137)
\curveto(511.11072148,70.50917298)(511.17061719,70.52479795)(511.24353371,70.53000628)
\curveto(511.31645023,70.54042292)(511.4128042,70.54563124)(511.53259562,70.54563124)
\curveto(511.64717872,70.54563124)(511.74092854,70.54042292)(511.81384505,70.53000628)
\curveto(511.8919699,70.52479795)(511.95186561,70.50917298)(511.99353219,70.48313137)
\curveto(512.03519877,70.46229808)(512.06384455,70.43625647)(512.07946952,70.40500653)
\curveto(512.10030281,70.37896492)(512.11071945,70.34511082)(512.11071945,70.30344423)
\lineto(512.11071945,69.37375862)
\curveto(512.50655198,69.81646605)(512.88936371,70.13938207)(513.25915463,70.34250666)
\curveto(513.63415387,70.55083957)(514.01175727,70.65500602)(514.39196483,70.65500602)
\curveto(514.68363091,70.65500602)(514.94404705,70.62115192)(515.17321325,70.55344373)
\curveto(515.40758778,70.48573553)(515.61331653,70.38938156)(515.7903995,70.26438181)
\curveto(515.96748248,70.14459039)(516.11852384,69.99875735)(516.24352358,69.8268827)
\curveto(516.36852333,69.66021637)(516.47268979,69.47271675)(516.55602295,69.26438384)
\curveto(516.79039748,69.51959166)(517.01175119,69.73573705)(517.22008411,69.91282003)
\curveto(517.43362534,70.089903)(517.63674993,70.23313188)(517.82945787,70.34250666)
\curveto(518.02737414,70.45188143)(518.21747792,70.53000628)(518.39976922,70.57688118)
\curveto(518.58726884,70.62896441)(518.77476846,70.65500602)(518.96226808,70.65500602)
\curveto(519.41539216,70.65500602)(519.79559972,70.57427702)(520.10289077,70.41281901)
\curveto(520.41018181,70.25656933)(520.65757715,70.04563226)(520.84507677,69.78000779)
\curveto(521.03778471,69.51438333)(521.1732011,69.20188397)(521.25132594,68.84250969)
\curveto(521.33465911,68.48834374)(521.37632569,68.1133445)(521.37632569,67.71751197)
\closepath
}
}
{
\newrgbcolor{curcolor}{0 0 0}
\pscustom[linestyle=none,fillstyle=solid,fillcolor=curcolor,opacity=0]
{
\newpath
\moveto(269.59714485,347.91759412)
\lineto(269.59714485,314.75230695)
}
}
{
\newrgbcolor{curcolor}{0.10196079 0.8392157 0.96078432}
\pscustom[linewidth=1.99999595,linecolor=curcolor]
{
\newpath
\moveto(269.59714485,347.91759412)
\lineto(269.59714485,314.75230695)
}
}
{
\newrgbcolor{curcolor}{0 0 0}
\pscustom[linestyle=none,fillstyle=solid,fillcolor=curcolor,opacity=0]
{
\newpath
\moveto(301.59755249,347.91759412)
\lineto(301.59755249,314.75230695)
}
}
{
\newrgbcolor{curcolor}{0.10196079 0.8392157 0.96078432}
\pscustom[linewidth=1.99999595,linecolor=curcolor]
{
\newpath
\moveto(301.59755249,347.91759412)
\lineto(301.59755249,314.75230695)
}
}
{
\newrgbcolor{curcolor}{0 0 0}
\pscustom[linestyle=none,fillstyle=solid,fillcolor=curcolor,opacity=0]
{
\newpath
\moveto(341.59804892,347.91759412)
\lineto(341.59804892,314.75230695)
}
}
{
\newrgbcolor{curcolor}{0.10196079 0.8392157 0.96078432}
\pscustom[linewidth=1.99999595,linecolor=curcolor]
{
\newpath
\moveto(341.59804892,347.91759412)
\lineto(341.59804892,314.75230695)
}
}
{
\newrgbcolor{curcolor}{0 0 0}
\pscustom[linestyle=none,fillstyle=solid,fillcolor=curcolor,opacity=0]
{
\newpath
\moveto(381.59854534,347.91759412)
\lineto(381.59854534,314.75230695)
}
}
{
\newrgbcolor{curcolor}{0.10196079 0.8392157 0.96078432}
\pscustom[linewidth=1.99999595,linecolor=curcolor]
{
\newpath
\moveto(381.59854534,347.91759412)
\lineto(381.59854534,314.75230695)
}
}
{
\newrgbcolor{curcolor}{0 0 0}
\pscustom[linestyle=none,fillstyle=solid,fillcolor=curcolor,opacity=0]
{
\newpath
\moveto(392.35206687,347.91759412)
\lineto(392.35206687,314.75230695)
}
}
{
\newrgbcolor{curcolor}{0.10196079 0.8392157 0.96078432}
\pscustom[linewidth=1.99999595,linecolor=curcolor]
{
\newpath
\moveto(392.35206687,347.91759412)
\lineto(392.35206687,314.75230695)
}
}
{
\newrgbcolor{curcolor}{0 0 0}
\pscustom[linestyle=none,fillstyle=solid,fillcolor=curcolor,opacity=0]
{
\newpath
\moveto(437.59927184,347.91759412)
\lineto(437.59927184,314.75230695)
}
}
{
\newrgbcolor{curcolor}{0.10196079 0.8392157 0.96078432}
\pscustom[linewidth=1.99999595,linecolor=curcolor]
{
\newpath
\moveto(437.59927184,347.91759412)
\lineto(437.59927184,314.75230695)
}
}
{
\newrgbcolor{curcolor}{0 0 0}
\pscustom[linestyle=none,fillstyle=solid,fillcolor=curcolor,opacity=0]
{
\newpath
\moveto(565.60092864,347.91759412)
\lineto(565.60092864,314.75230695)
}
}
{
\newrgbcolor{curcolor}{0.10196079 0.8392157 0.96078432}
\pscustom[linewidth=1.99999595,linecolor=curcolor]
{
\newpath
\moveto(565.60092864,347.91759412)
\lineto(565.60092864,314.75230695)
}
}
{
\newrgbcolor{curcolor}{0 0 0}
\pscustom[linestyle=none,fillstyle=solid,fillcolor=curcolor,opacity=0]
{
\newpath
\moveto(234.77638596,348.91580732)
\lineto(278.71329437,348.91580732)
\lineto(278.71329437,312.12848024)
\lineto(234.77638596,312.12848024)
\closepath
}
}
{
\newrgbcolor{curcolor}{0.94509804 0.99215686 0.99607843}
\pscustom[linestyle=none,fillstyle=solid,fillcolor=curcolor]
{
\newpath
\moveto(250.4495182,329.53122028)
\curveto(250.4495182,329.26646454)(250.41479613,329.02123996)(250.34535201,328.79554654)
\curveto(250.28024814,328.57419338)(250.18693259,328.37454151)(250.06540536,328.19659094)
\curveto(249.94387814,328.01864036)(249.79630937,327.86239107)(249.62269905,327.72784307)
\curveto(249.45342898,327.59329507)(249.2602875,327.48044836)(249.0432746,327.38930294)
\curveto(248.83060196,327.29815753)(248.60056828,327.2287134)(248.35317358,327.18097056)
\curveto(248.11011913,327.13756798)(247.83017248,327.11586669)(247.51333365,327.11586669)
\lineto(245.31282283,327.11586669)
\curveto(245.2129969,327.11586669)(245.11751122,327.14841862)(245.0263658,327.21352249)
\curveto(244.93956064,327.28296662)(244.89615806,327.40232372)(244.89615806,327.57159378)
\lineto(244.89615806,335.08458041)
\curveto(244.89615806,335.25385048)(244.93956064,335.37103744)(245.0263658,335.43614131)
\curveto(245.11751122,335.50558544)(245.2129969,335.54030751)(245.31282283,335.54030751)
\lineto(247.22687662,335.54030751)
\curveto(247.73034655,335.54030751)(248.14267106,335.49256467)(248.46385015,335.39707899)
\curveto(248.78502925,335.30159331)(249.05195512,335.16270506)(249.26462776,334.98041422)
\curveto(249.48164066,334.79812338)(249.64440034,334.5746001)(249.75290679,334.30984436)
\curveto(249.86141324,334.04508862)(249.91566646,333.74561081)(249.91566646,333.41141095)
\curveto(249.91566646,333.21175908)(249.89179504,333.02078772)(249.8440522,332.83849689)
\curveto(249.79630937,332.65620605)(249.72469511,332.48693599)(249.62920943,332.3306867)
\curveto(249.53806401,332.17877767)(249.42087705,332.04205954)(249.27764853,331.92053232)
\curveto(249.13876028,331.79900509)(248.97817073,331.69917916)(248.79587989,331.62105451)
\curveto(249.02591357,331.57765193)(249.24075634,331.49735716)(249.44040821,331.38017019)
\curveto(249.64006008,331.26732348)(249.8136704,331.12192484)(249.96123917,330.94397426)
\curveto(250.1131482,330.76602368)(250.2325053,330.5576913)(250.31931046,330.31897711)
\curveto(250.40611562,330.08026292)(250.4495182,329.81767731)(250.4495182,329.53122028)
\closepath
\moveto(248.76332796,333.30073437)
\curveto(248.76332796,333.50472649)(248.73511628,333.68918746)(248.67869293,333.85411726)
\curveto(248.62226957,334.01904707)(248.53329428,334.15793533)(248.41176706,334.27078203)
\curveto(248.29023983,334.387969)(248.13182042,334.47694429)(247.93650881,334.5377079)
\curveto(247.74119719,334.59847151)(247.48295184,334.62885332)(247.16177275,334.62885332)
\lineto(246.00292386,334.62885332)
\lineto(246.00292386,331.90100116)
\lineto(247.27895972,331.90100116)
\curveto(247.569757,331.90100116)(247.80413094,331.93789335)(247.98208151,332.01167774)
\curveto(248.16003209,332.08980238)(248.30760087,332.19179844)(248.42478783,332.31766593)
\curveto(248.5419748,332.44787367)(248.62660983,332.59761257)(248.67869293,332.76688263)
\curveto(248.73511628,332.94049295)(248.76332796,333.11844353)(248.76332796,333.30073437)
\closepath
\moveto(249.27113815,329.46611641)
\curveto(249.27113815,329.71785137)(249.2299057,329.93920453)(249.14744079,330.13017588)
\curveto(249.06931615,330.32114724)(248.95212918,330.48173678)(248.79587989,330.61194452)
\curveto(248.64397086,330.74215226)(248.45082938,330.83980807)(248.21645545,330.90491194)
\curveto(247.98642177,330.97435607)(247.69562449,331.00907813)(247.34406359,331.00907813)
\lineto(246.00292386,331.00907813)
\lineto(246.00292386,328.04034165)
\lineto(247.63052061,328.04034165)
\curveto(247.88659584,328.04034165)(248.11011913,328.07072345)(248.30109048,328.13148707)
\curveto(248.49206183,328.19225068)(248.65916177,328.28122597)(248.80239028,328.39841293)
\curveto(248.94995905,328.51994016)(249.06497589,328.66967906)(249.14744079,328.84762964)
\curveto(249.2299057,329.02558022)(249.27113815,329.23174247)(249.27113815,329.46611641)
\closepath
}
}
{
\newrgbcolor{curcolor}{0.94509804 0.99215686 0.99607843}
\pscustom[linestyle=none,fillstyle=solid,fillcolor=curcolor]
{
\newpath
\moveto(258.61397827,331.45178445)
\curveto(258.61397827,330.7226211)(258.52066272,330.08460317)(258.33403162,329.53773066)
\curveto(258.14740053,328.99519841)(257.87396427,328.54381158)(257.51372286,328.18357016)
\curveto(257.1578217,327.827669)(256.71945564,327.56074314)(256.19862468,327.38279256)
\curveto(255.67779372,327.20484198)(255.05713682,327.11586669)(254.33665399,327.11586669)
\lineto(252.55280794,327.11586669)
\curveto(252.45298201,327.11586669)(252.35749633,327.14841862)(252.26635091,327.21352249)
\curveto(252.17954575,327.28296662)(252.13614317,327.40232372)(252.13614317,327.57159378)
\lineto(252.13614317,335.08458041)
\curveto(252.13614317,335.25385048)(252.17954575,335.37103744)(252.26635091,335.43614131)
\curveto(252.35749633,335.50558544)(252.45298201,335.54030751)(252.55280794,335.54030751)
\lineto(254.46035134,335.54030751)
\curveto(255.18951469,335.54030751)(255.80583133,335.44482183)(256.30930126,335.25385048)
\curveto(256.81277119,335.06721938)(257.23377622,334.79595325)(257.57231634,334.4400521)
\curveto(257.91519673,334.0884912)(258.17344208,333.66097578)(258.3470524,333.15750585)
\curveto(258.52500298,332.65403592)(258.61397827,332.08546212)(258.61397827,331.45178445)
\closepath
\moveto(257.43559821,331.40621174)
\curveto(257.43559821,331.86193883)(257.37917486,332.28511399)(257.26632815,332.67573721)
\curveto(257.15348144,333.07070069)(256.97553086,333.41141095)(256.73247642,333.69786798)
\curveto(256.49376222,333.98432501)(256.18994416,334.20784829)(255.82102223,334.36843784)
\curveto(255.4521003,334.52902739)(254.97684205,334.60932216)(254.39524747,334.60932216)
\lineto(253.25592974,334.60932216)
\lineto(253.25592974,328.05336242)
\lineto(254.40826825,328.05336242)
\curveto(254.94646024,328.05336242)(255.4000172,328.12063642)(255.76893913,328.25518442)
\curveto(256.13786107,328.38973242)(256.44601939,328.59589467)(256.69341409,328.87367119)
\curveto(256.9408088,329.1514477)(257.12526977,329.49866834)(257.24679699,329.91533311)
\curveto(257.37266447,330.33633814)(257.43559821,330.83329768)(257.43559821,331.40621174)
\closepath
}
}
{
\newrgbcolor{curcolor}{0.94509804 0.99215686 0.99607843}
\pscustom[linestyle=none,fillstyle=solid,fillcolor=curcolor]
{
\newpath
\moveto(265.50068588,327.51951068)
\curveto(265.55710924,327.42402501)(265.58966117,327.34807049)(265.59834169,327.29164714)
\curveto(265.61136246,327.23522378)(265.59834169,327.18965108)(265.55927937,327.15492901)
\curveto(265.52021705,327.1245472)(265.45511318,327.10501604)(265.36396776,327.09633553)
\curveto(265.2771626,327.08331475)(265.15997563,327.07680437)(265.01240686,327.07680437)
\curveto(264.8735186,327.07680437)(264.76501215,327.08114462)(264.68688751,327.08982514)
\curveto(264.60876286,327.0941654)(264.54582912,327.10501604)(264.49808628,327.12237708)
\curveto(264.4546837,327.13973811)(264.41996164,327.15926927)(264.39392009,327.18097056)
\curveto(264.3722188,327.20701211)(264.35268764,327.23956404)(264.33532661,327.27862636)
\lineto(262.46033514,330.63147568)
\lineto(260.56581252,327.27862636)
\curveto(260.54411123,327.23956404)(260.52023981,327.20701211)(260.49419826,327.18097056)
\curveto(260.46815671,327.15492901)(260.43126452,327.13322772)(260.38352168,327.11586669)
\curveto(260.3401191,327.10284591)(260.27935549,327.0941654)(260.20123084,327.08982514)
\curveto(260.1231062,327.08114462)(260.01894001,327.07680437)(259.88873227,327.07680437)
\curveto(259.74550375,327.07680437)(259.63265704,327.08331475)(259.55019214,327.09633553)
\curveto(259.46772724,327.10501604)(259.40913375,327.1245472)(259.37441169,327.15492901)
\curveto(259.34402988,327.18965108)(259.33317924,327.23522378)(259.34185975,327.29164714)
\curveto(259.35488053,327.34807049)(259.38960259,327.42402501)(259.44602595,327.51951068)
\lineto(261.74419257,331.39319097)
\lineto(259.55019214,335.13666351)
\curveto(259.49810904,335.23214919)(259.46338698,335.3081037)(259.44602595,335.36452706)
\curveto(259.43300517,335.42529067)(259.44385582,335.47086338)(259.47857788,335.50124518)
\curveto(259.51329995,335.53596725)(259.57406356,335.55766854)(259.66086872,335.56634905)
\curveto(259.74767388,335.57502957)(259.86703098,335.57936983)(260.01894001,335.57936983)
\curveto(260.14914775,335.57936983)(260.25548407,335.57502957)(260.33794897,335.56634905)
\curveto(260.42041387,335.5620088)(260.48551774,335.55115815)(260.53326058,335.53379712)
\curveto(260.58100342,335.52077634)(260.61572548,335.49907505)(260.63742677,335.46869325)
\curveto(260.65912806,335.4426517)(260.68299948,335.41226989)(260.70904103,335.37754783)
\lineto(262.51892863,332.22652051)
\lineto(264.31579545,335.37754783)
\curveto(264.33749674,335.41226989)(264.35919803,335.4426517)(264.38089932,335.46869325)
\curveto(264.40260061,335.49907505)(264.43298241,335.52077634)(264.47204473,335.53379712)
\curveto(264.51544731,335.55115815)(264.5740408,335.5620088)(264.64782518,335.56634905)
\curveto(264.72160957,335.57502957)(264.8214355,335.57936983)(264.94730299,335.57936983)
\curveto(265.08185099,335.57936983)(265.19035744,335.57285944)(265.27282234,335.55983867)
\curveto(265.3596275,335.55115815)(265.42039111,335.53162699)(265.45511318,335.50124518)
\curveto(265.4941755,335.47086338)(265.5093664,335.42529067)(265.50068588,335.36452706)
\curveto(265.49200537,335.3081037)(265.46162356,335.23214919)(265.40954047,335.13666351)
\lineto(263.22205042,331.41923252)
\closepath
}
}
{
\newrgbcolor{curcolor}{0 0 0}
\pscustom[linestyle=none,fillstyle=solid,fillcolor=curcolor,opacity=0]
{
\newpath
\moveto(268.16016613,348.91580732)
\lineto(312.09708503,348.91580732)
\lineto(312.09708503,312.12848024)
\lineto(268.16016613,312.12848024)
\closepath
}
}
{
\newrgbcolor{curcolor}{0.94509804 0.99215686 0.99607843}
\pscustom[linestyle=none,fillstyle=solid,fillcolor=curcolor]
{
\newpath
\moveto(279.23047697,327.27862636)
\curveto(279.23047697,327.2439043)(279.22179645,327.21352249)(279.20443542,327.18748095)
\curveto(279.18707439,327.16577966)(279.15886271,327.14624849)(279.11980039,327.12888746)
\curveto(279.08073806,327.11152643)(279.02648484,327.09850566)(278.95704071,327.08982514)
\curveto(278.88759658,327.08114462)(278.79862129,327.07680437)(278.69011484,327.07680437)
\curveto(278.58594865,327.07680437)(278.49914349,327.08114462)(278.42969936,327.08982514)
\curveto(278.36025523,327.09850566)(278.30383188,327.11152643)(278.2604293,327.12888746)
\curveto(278.22136698,327.14624849)(278.1931553,327.16577966)(278.17579427,327.18748095)
\curveto(278.16277349,327.21352249)(278.15626311,327.2439043)(278.15626311,327.27862636)
\lineto(278.15626311,335.97650344)
\curveto(278.15626311,336.0112255)(278.16277349,336.04160731)(278.17579427,336.06764885)
\curveto(278.1931553,336.0936904)(278.22136698,336.11539169)(278.2604293,336.13275272)
\curveto(278.30383188,336.15011376)(278.36025523,336.16313453)(278.42969936,336.17181505)
\curveto(278.49914349,336.18049556)(278.58594865,336.18483582)(278.69011484,336.18483582)
\curveto(278.79862129,336.18483582)(278.88759658,336.18049556)(278.95704071,336.17181505)
\curveto(279.02648484,336.16313453)(279.08073806,336.15011376)(279.11980039,336.13275272)
\curveto(279.15886271,336.11539169)(279.18707439,336.0936904)(279.20443542,336.06764885)
\curveto(279.22179645,336.04160731)(279.23047697,336.0112255)(279.23047697,335.97650344)
\closepath
}
}
{
\newrgbcolor{curcolor}{0.94509804 0.99215686 0.99607843}
\pscustom[linestyle=none,fillstyle=solid,fillcolor=curcolor]
{
\newpath
\moveto(286.01425583,333.15099547)
\curveto(286.01425583,333.13363443)(286.0120857,333.11410327)(286.00774545,333.09240198)
\curveto(286.00774545,333.07504095)(286.00557532,333.05550979)(286.00123506,333.0338085)
\curveto(285.9968948,333.01210721)(285.99038441,332.98823579)(285.9817039,332.96219424)
\curveto(285.97736364,332.93615269)(285.97085325,332.90794102)(285.96217274,332.87755921)
\lineto(283.96348392,327.3111783)
\curveto(283.94612289,327.26343546)(283.92225147,327.22437314)(283.89186966,327.19399133)
\curveto(283.86582811,327.16360953)(283.82459566,327.13973811)(283.76817231,327.12237708)
\curveto(283.71174895,327.10501604)(283.6401347,327.0941654)(283.55332954,327.08982514)
\curveto(283.46652438,327.08114462)(283.35801793,327.07680437)(283.22781019,327.07680437)
\curveto(283.09760244,327.07680437)(282.98909599,327.08114462)(282.90229083,327.08982514)
\curveto(282.81548567,327.09850566)(282.74387142,327.11152643)(282.68744806,327.12888746)
\curveto(282.63536497,327.14624849)(282.59413251,327.17011991)(282.56375071,327.20050172)
\curveto(282.5333689,327.23088353)(282.50949748,327.26777572)(282.49213645,327.3111783)
\lineto(280.49995802,332.87755921)
\lineto(280.45438531,333.01427734)
\curveto(280.44570479,333.05333966)(280.43919441,333.08155134)(280.43485415,333.09891237)
\lineto(280.43485415,333.15099547)
\curveto(280.43485415,333.18571753)(280.44353467,333.21609934)(280.4608957,333.24214088)
\curveto(280.47825673,333.26818243)(280.50646841,333.28771359)(280.54553073,333.30073437)
\curveto(280.58893331,333.3180954)(280.64318653,333.32894604)(280.7082904,333.3332863)
\curveto(280.77773453,333.33762656)(280.86236956,333.33979669)(280.9621955,333.33979669)
\curveto(281.08806298,333.33979669)(281.18788892,333.33545643)(281.2616733,333.32677592)
\curveto(281.33979795,333.32243566)(281.39839143,333.31158501)(281.43745375,333.29422398)
\curveto(281.48085633,333.27686295)(281.51123814,333.25516166)(281.52859917,333.22912011)
\curveto(281.55030046,333.20307856)(281.56983162,333.17052663)(281.58719265,333.1314643)
\lineto(283.24083096,328.30075713)
\lineto(283.26687251,328.22263248)
\lineto(283.28640367,328.30075713)
\lineto(284.92051081,333.1314643)
\curveto(284.92919133,333.17052663)(284.94438223,333.20307856)(284.96608352,333.22912011)
\curveto(284.98778481,333.25516166)(285.01816662,333.27686295)(285.05722894,333.29422398)
\curveto(285.10063152,333.31158501)(285.15705487,333.32243566)(285.226499,333.32677592)
\curveto(285.30028339,333.33545643)(285.39576907,333.33979669)(285.51295603,333.33979669)
\curveto(285.61278197,333.33979669)(285.69524687,333.33762656)(285.76035074,333.3332863)
\curveto(285.82545461,333.32894604)(285.87536758,333.3180954)(285.91008964,333.30073437)
\curveto(285.94915196,333.28337333)(285.97519351,333.26167204)(285.98821429,333.2356305)
\curveto(286.00557532,333.21392921)(286.01425583,333.18571753)(286.01425583,333.15099547)
\closepath
}
}
{
\newrgbcolor{curcolor}{0.94509804 0.99215686 0.99607843}
\pscustom[linestyle=none,fillstyle=solid,fillcolor=curcolor]
{
\newpath
\moveto(288.29712521,327.27862636)
\curveto(288.29712521,327.2439043)(288.28844469,327.21352249)(288.27108366,327.18748095)
\curveto(288.25372263,327.16577966)(288.22551095,327.14624849)(288.18644863,327.12888746)
\curveto(288.14738631,327.11152643)(288.09313308,327.09850566)(288.02368895,327.08982514)
\curveto(287.95424483,327.08114462)(287.86526954,327.07680437)(287.75676309,327.07680437)
\curveto(287.65259689,327.07680437)(287.56579173,327.08114462)(287.4963476,327.08982514)
\curveto(287.42690348,327.09850566)(287.37048012,327.11152643)(287.32707754,327.12888746)
\curveto(287.28801522,327.14624849)(287.25980354,327.16577966)(287.24244251,327.18748095)
\curveto(287.22942174,327.21352249)(287.22291135,327.2439043)(287.22291135,327.27862636)
\lineto(287.22291135,335.97650344)
\curveto(287.22291135,336.0112255)(287.22942174,336.04160731)(287.24244251,336.06764885)
\curveto(287.25980354,336.0936904)(287.28801522,336.11539169)(287.32707754,336.13275272)
\curveto(287.37048012,336.15011376)(287.42690348,336.16313453)(287.4963476,336.17181505)
\curveto(287.56579173,336.18049556)(287.65259689,336.18483582)(287.75676309,336.18483582)
\curveto(287.86526954,336.18483582)(287.95424483,336.18049556)(288.02368895,336.17181505)
\curveto(288.09313308,336.16313453)(288.14738631,336.15011376)(288.18644863,336.13275272)
\curveto(288.22551095,336.11539169)(288.25372263,336.0936904)(288.27108366,336.06764885)
\curveto(288.28844469,336.04160731)(288.29712521,336.0112255)(288.29712521,335.97650344)
\closepath
}
}
{
\newrgbcolor{curcolor}{0 0 0}
\pscustom[linestyle=none,fillstyle=solid,fillcolor=curcolor,opacity=0]
{
\newpath
\moveto(299.04393561,348.91580732)
\lineto(342.98085451,348.91580732)
\lineto(342.98085451,312.12848024)
\lineto(299.04393561,312.12848024)
\closepath
}
}
{
\newrgbcolor{curcolor}{0.94509804 0.99215686 0.99607843}
\pscustom[linestyle=none,fillstyle=solid,fillcolor=curcolor]
{
\newpath
\moveto(312.0868617,327.70831191)
\curveto(312.0868617,327.58244443)(312.07818118,327.48261849)(312.06082015,327.40883411)
\curveto(312.04345912,327.33504972)(312.01741757,327.28079649)(311.9826955,327.24607443)
\curveto(311.94797344,327.21135237)(311.89589034,327.17880043)(311.82644622,327.14841862)
\curveto(311.75700209,327.11803682)(311.67670731,327.0941654)(311.5855619,327.07680437)
\curveto(311.49875673,327.05510308)(311.40544119,327.03774204)(311.30561525,327.02472127)
\curveto(311.20578932,327.0117005)(311.10596338,327.00519011)(311.00613745,327.00519011)
\curveto(310.70231939,327.00519011)(310.44190391,327.04425243)(310.22489101,327.12237708)
\curveto(310.00787811,327.20484198)(309.82992753,327.3263692)(309.69103927,327.48695875)
\curveto(309.55215101,327.65188855)(309.45015495,327.85805081)(309.38505108,328.10544552)
\curveto(309.32428747,328.35718048)(309.29390566,328.65231803)(309.29390566,328.99085815)
\lineto(309.29390566,332.41532173)
\lineto(308.4735969,332.41532173)
\curveto(308.40849302,332.41532173)(308.35640993,332.4500438)(308.31734761,332.51948792)
\curveto(308.27828528,332.58893205)(308.25875412,332.70177876)(308.25875412,332.85802805)
\curveto(308.25875412,332.94049295)(308.26309438,333.00993708)(308.2717749,333.06636043)
\curveto(308.28479567,333.12278379)(308.29998657,333.1683565)(308.31734761,333.20307856)
\curveto(308.33470864,333.24214088)(308.35640993,333.26818243)(308.38245148,333.28120321)
\curveto(308.41283328,333.29856424)(308.44538522,333.30724475)(308.48010728,333.30724475)
\lineto(309.29390566,333.30724475)
\lineto(309.29390566,334.70046758)
\curveto(309.29390566,334.73084938)(309.30041605,334.75906106)(309.31343682,334.78510261)
\curveto(309.33079785,334.81114416)(309.35900953,334.83284545)(309.39807185,334.85020648)
\curveto(309.44147443,334.87190777)(309.49789779,334.88709867)(309.56734192,334.89577919)
\curveto(309.63678604,334.90445971)(309.7235912,334.90879996)(309.8277574,334.90879996)
\curveto(309.93626385,334.90879996)(310.02523914,334.90445971)(310.09468327,334.89577919)
\curveto(310.16412739,334.88709867)(310.21838062,334.87190777)(310.25744294,334.85020648)
\curveto(310.29650526,334.83284545)(310.32471694,334.81114416)(310.34207797,334.78510261)
\curveto(310.359439,334.75906106)(310.36811952,334.73084938)(310.36811952,334.70046758)
\lineto(310.36811952,333.30724475)
\lineto(311.87201892,333.30724475)
\curveto(311.90674099,333.30724475)(311.93712279,333.29856424)(311.96316434,333.28120321)
\curveto(311.98920589,333.26818243)(312.01090718,333.24214088)(312.02826821,333.20307856)
\curveto(312.0499695,333.1683565)(312.06516041,333.12278379)(312.07384092,333.06636043)
\curveto(312.08252144,333.00993708)(312.0868617,332.94049295)(312.0868617,332.85802805)
\curveto(312.0868617,332.70177876)(312.06733054,332.58893205)(312.02826821,332.51948792)
\curveto(311.98920589,332.4500438)(311.93712279,332.41532173)(311.87201892,332.41532173)
\lineto(310.36811952,332.41532173)
\lineto(310.36811952,329.14710744)
\curveto(310.36811952,328.74346345)(310.426713,328.43747526)(310.54389997,328.22914287)
\curveto(310.6654272,328.02515074)(310.88026997,327.92315468)(311.18842829,327.92315468)
\curveto(311.28825422,327.92315468)(311.37722951,327.9318352)(311.45535415,327.94919623)
\curveto(311.5334788,327.97089752)(311.60292293,327.99259881)(311.66368654,328.0143001)
\curveto(311.72445015,328.03600139)(311.77653325,328.05553255)(311.81993583,328.07289358)
\curveto(311.86333841,328.09459487)(311.90240073,328.10544552)(311.93712279,328.10544552)
\curveto(311.95882408,328.10544552)(311.97835525,328.09893513)(311.99571628,328.08591436)
\curveto(312.01741757,328.07723384)(312.03260847,328.05770268)(312.04128899,328.02732087)
\curveto(312.05430976,327.99693907)(312.06516041,327.95570662)(312.07384092,327.90362352)
\curveto(312.08252144,327.85154042)(312.0868617,327.78643655)(312.0868617,327.70831191)
\closepath
}
}
{
\newrgbcolor{curcolor}{0.94509804 0.99215686 0.99607843}
\pscustom[linestyle=none,fillstyle=solid,fillcolor=curcolor]
{
\newpath
\moveto(316.16258361,327.11586669)
\lineto(315.38133717,324.96092858)
\curveto(315.35529562,324.89148445)(315.28802162,324.83940136)(315.17951517,324.80467929)
\curveto(315.07534897,324.76561697)(314.91475943,324.74608581)(314.69774653,324.74608581)
\curveto(314.58489982,324.74608581)(314.4937544,324.7525962)(314.42431027,324.76561697)
\curveto(314.35486614,324.77429749)(314.30061292,324.79165852)(314.2615506,324.81770007)
\curveto(314.22682853,324.84374162)(314.20729737,324.87846368)(314.20295711,324.92186626)
\curveto(314.19861685,324.96526884)(314.2094675,325.01735194)(314.23550905,325.07811555)
\lineto(315.04279704,327.11586669)
\curveto(315.00373472,327.13322772)(314.96684252,327.1614394)(314.93212046,327.20050172)
\curveto(314.8973984,327.23956404)(314.87352698,327.28079649)(314.8605062,327.32419907)
\lineto(312.77067197,332.92313192)
\curveto(312.7359499,333.01427734)(312.71858887,333.08589159)(312.71858887,333.13797469)
\curveto(312.71858887,333.19005779)(312.7359499,333.23129024)(312.77067197,333.26167204)
\curveto(312.80539403,333.29205385)(312.86181738,333.31158501)(312.93994203,333.32026553)
\curveto(313.01806667,333.3332863)(313.12223287,333.33979669)(313.25244061,333.33979669)
\curveto(313.38264835,333.33979669)(313.48464441,333.33545643)(313.5584288,333.32677592)
\curveto(313.63221318,333.32243566)(313.69080667,333.31158501)(313.73420925,333.29422398)
\curveto(313.77761183,333.27686295)(313.80799363,333.2508214)(313.82535466,333.21609934)
\curveto(313.84705596,333.18571753)(313.86875725,333.14231495)(313.89045854,333.08589159)
\lineto(315.563628,328.38539216)
\lineto(315.58315916,328.38539216)
\lineto(317.19773515,333.11193314)
\curveto(317.22377669,333.19439805)(317.2541585,333.24648114)(317.28888056,333.26818243)
\curveto(317.32794289,333.29422398)(317.38436624,333.31158501)(317.45815063,333.32026553)
\curveto(317.53193501,333.3332863)(317.63827134,333.33979669)(317.77715959,333.33979669)
\curveto(317.89868682,333.33979669)(317.99851275,333.3332863)(318.0766374,333.32026553)
\curveto(318.15476204,333.31158501)(318.21118539,333.29205385)(318.24590746,333.26167204)
\curveto(318.28496978,333.23129024)(318.30450094,333.19005779)(318.30450094,333.13797469)
\curveto(318.30450094,333.08589159)(318.29148017,333.02078772)(318.26543862,332.94266308)
\closepath
}
}
{
\newrgbcolor{curcolor}{0.94509804 0.99215686 0.99607843}
\pscustom[linestyle=none,fillstyle=solid,fillcolor=curcolor]
{
\newpath
\moveto(324.91057935,330.29944595)
\curveto(324.91057935,329.7872955)(324.85415599,329.32722815)(324.74130928,328.9192439)
\curveto(324.63280283,328.51125964)(324.47004316,328.164039)(324.25303026,327.87758197)
\curveto(324.04035761,327.5954652)(323.77560188,327.37628217)(323.45876304,327.22003288)
\curveto(323.1419242,327.06812385)(322.77951266,326.99216934)(322.37152841,326.99216934)
\curveto(322.19791809,326.99216934)(322.03732854,327.00953037)(321.88975977,327.04425243)
\curveto(321.74219099,327.0789745)(321.59679235,327.13322772)(321.45356383,327.20701211)
\curveto(321.31467558,327.28079649)(321.17578732,327.37411204)(321.03689906,327.48695875)
\curveto(320.89801081,327.59980546)(320.75044204,327.73218333)(320.59419275,327.88409236)
\lineto(320.59419275,324.95441819)
\curveto(320.59419275,324.91969613)(320.58551223,324.88931432)(320.5681512,324.86327278)
\curveto(320.55079017,324.83723123)(320.52257849,324.81552994)(320.48351617,324.79816891)
\curveto(320.44445385,324.78080787)(320.39020062,324.7677871)(320.32075649,324.75910658)
\curveto(320.25131236,324.75042607)(320.16233707,324.74608581)(320.05383062,324.74608581)
\curveto(319.94966443,324.74608581)(319.86285927,324.75042607)(319.79341514,324.75910658)
\curveto(319.72397101,324.7677871)(319.66754766,324.78080787)(319.62414508,324.79816891)
\curveto(319.58508276,324.81552994)(319.55687108,324.83723123)(319.53951005,324.86327278)
\curveto(319.52648927,324.88931432)(319.51997889,324.91969613)(319.51997889,324.95441819)
\lineto(319.51997889,333.13797469)
\curveto(319.51997889,333.17703701)(319.52648927,333.20741882)(319.53951005,333.22912011)
\curveto(319.55253082,333.25516166)(319.57857237,333.27686295)(319.61763469,333.29422398)
\curveto(319.65669701,333.31158501)(319.70660998,333.32243566)(319.76737359,333.32677592)
\curveto(319.82813721,333.33545643)(319.90192159,333.33979669)(319.98872675,333.33979669)
\curveto(320.07987217,333.33979669)(320.15365656,333.33545643)(320.21007991,333.32677592)
\curveto(320.27084352,333.32243566)(320.32075649,333.31158501)(320.35981881,333.29422398)
\curveto(320.39888114,333.27686295)(320.42492268,333.25516166)(320.43794346,333.22912011)
\curveto(320.45530449,333.20741882)(320.46398501,333.17703701)(320.46398501,333.13797469)
\lineto(320.46398501,332.35021786)
\curveto(320.64193558,332.5325087)(320.81337578,332.69092812)(320.97830558,332.82547611)
\curveto(321.14323539,332.96002411)(321.30816519,333.07070069)(321.473095,333.15750585)
\curveto(321.64236506,333.24865127)(321.81380525,333.31592527)(321.98741557,333.35932785)
\curveto(322.16536615,333.40707069)(322.35199724,333.43094211)(322.54730886,333.43094211)
\curveto(322.97265414,333.43094211)(323.33506569,333.34847721)(323.63454349,333.1835474)
\curveto(323.93402129,333.0186176)(324.17707574,332.79292418)(324.36370684,332.50646715)
\curveto(324.55467819,332.22001012)(324.69356645,331.88581025)(324.78037161,331.50386755)
\curveto(324.86717677,331.1262651)(324.91057935,330.72479123)(324.91057935,330.29944595)
\closepath
\moveto(323.777772,330.17574859)
\curveto(323.777772,330.4752264)(323.75390059,330.76385355)(323.70615775,331.04163007)
\curveto(323.66275517,331.32374684)(323.58463052,331.57331168)(323.47178381,331.79032458)
\curveto(323.36327736,332.00733748)(323.21570859,332.1809478)(323.0290775,332.31115554)
\curveto(322.8424464,332.44136328)(322.6102426,332.50646715)(322.33246608,332.50646715)
\curveto(322.19357783,332.50646715)(322.0568597,332.48476586)(321.9223117,332.44136328)
\curveto(321.7877637,332.40230096)(321.65104557,332.33719709)(321.51215732,332.24605167)
\curveto(321.37326906,332.15924651)(321.22787042,332.04205954)(321.07596139,331.89449077)
\curveto(320.92405236,331.75126225)(320.76346281,331.57331168)(320.59419275,331.36063903)
\lineto(320.59419275,329.02992048)
\curveto(320.88933029,328.66967906)(321.16927693,328.39407268)(321.43403267,328.20310132)
\curveto(321.69878841,328.01212997)(321.97656493,327.91664429)(322.26736221,327.91664429)
\curveto(322.53645821,327.91664429)(322.76649189,327.98174816)(322.95746324,328.1119559)
\curveto(323.15277485,328.24216364)(323.30902414,328.41577397)(323.4262111,328.63278687)
\curveto(323.54773833,328.84979977)(323.63671362,329.09285422)(323.69313697,329.36195021)
\curveto(323.74956033,329.63104621)(323.777772,329.90231234)(323.777772,330.17574859)
\closepath
}
}
{
\newrgbcolor{curcolor}{0.94509804 0.99215686 0.99607843}
\pscustom[linestyle=none,fillstyle=solid,fillcolor=curcolor]
{
\newpath
\moveto(331.51994195,330.46220562)
\curveto(331.51994195,330.29293556)(331.47653937,330.17140833)(331.38973421,330.09762395)
\curveto(331.30726931,330.02817982)(331.21178363,329.99345776)(331.10327718,329.99345776)
\lineto(327.26214883,329.99345776)
\curveto(327.26214883,329.6679384)(327.29470077,329.37497099)(327.35980464,329.11455551)
\curveto(327.42490851,328.85414003)(327.53341496,328.63061674)(327.68532399,328.44398564)
\curveto(327.83723302,328.25735455)(328.03471476,328.11412603)(328.27776921,328.0143001)
\curveto(328.52082366,327.91447416)(328.81813133,327.8645612)(329.16969223,327.8645612)
\curveto(329.44746875,327.8645612)(329.69486345,327.88626249)(329.91187635,327.92966507)
\curveto(330.12888926,327.97740791)(330.31552035,328.029491)(330.47176964,328.08591436)
\curveto(330.63235919,328.14233771)(330.76256693,328.19225068)(330.86239286,328.23565326)
\curveto(330.96655905,328.2833961)(331.0446837,328.30726752)(331.09676679,328.30726752)
\curveto(331.1271486,328.30726752)(331.15319015,328.298587)(331.17489144,328.28122597)
\curveto(331.20093299,328.26820519)(331.22046415,328.2465039)(331.23348492,328.2161221)
\curveto(331.2465057,328.18574029)(331.25518621,328.14233771)(331.25952647,328.08591436)
\curveto(331.26820699,328.03383126)(331.27254724,327.96872739)(331.27254724,327.89060275)
\curveto(331.27254724,327.83417939)(331.27037711,327.78426642)(331.26603686,327.74086384)
\curveto(331.2616966,327.70180152)(331.25518621,327.66490933)(331.2465057,327.63018726)
\curveto(331.24216544,327.59980546)(331.23131479,327.57159378)(331.21395376,327.54555223)
\curveto(331.20093299,327.51951068)(331.18140183,327.49346914)(331.15536028,327.46742759)
\curveto(331.13365899,327.4457263)(331.06421486,327.40666398)(330.94702789,327.35024062)
\curveto(330.82984093,327.29815753)(330.67793189,327.24607443)(330.4913008,327.19399133)
\curveto(330.30466971,327.14190824)(330.0876568,327.09633553)(329.8402621,327.05727321)
\curveto(329.59720765,327.01387063)(329.33679217,326.99216934)(329.05901565,326.99216934)
\curveto(328.57724701,326.99216934)(328.15407186,327.05944333)(327.78949018,327.19399133)
\curveto(327.42924877,327.32853933)(327.12543071,327.5281912)(326.878036,327.79294694)
\curveto(326.63064129,328.05770268)(326.4440102,328.38973242)(326.31814271,328.78903616)
\curveto(326.19227523,329.18833989)(326.12934149,329.6527475)(326.12934149,330.18225898)
\curveto(326.12934149,330.68572891)(326.19444536,331.13711574)(326.3246531,331.53641948)
\curveto(326.45486084,331.94006348)(326.64149194,332.28077373)(326.88454639,332.55855025)
\curveto(327.13194109,332.84066702)(327.42924877,333.05550979)(327.77646941,333.20307856)
\curveto(328.12369005,333.35498759)(328.51214314,333.43094211)(328.94182869,333.43094211)
\curveto(329.40189604,333.43094211)(329.79251926,333.35715772)(330.11369835,333.20958895)
\curveto(330.4392177,333.06202018)(330.70614357,332.86236831)(330.91447596,332.61063334)
\curveto(331.12280834,332.36323863)(331.27471737,332.07027122)(331.37020305,331.73173109)
\curveto(331.47002898,331.39753123)(331.51994195,331.03945994)(331.51994195,330.65751723)
\closepath
\moveto(330.4392177,330.78121459)
\curveto(330.45223848,331.34544813)(330.326371,331.78815445)(330.06161526,332.10933354)
\curveto(329.80119977,332.43051263)(329.41274668,332.59110218)(328.89625598,332.59110218)
\curveto(328.63150024,332.59110218)(328.39929643,332.54118921)(328.19964457,332.44136328)
\curveto(327.9999927,332.34153734)(327.83289276,332.20915947)(327.69834476,332.04422967)
\curveto(327.56379677,331.87929987)(327.45963057,331.68615838)(327.38584619,331.46480522)
\curveto(327.3120618,331.24779232)(327.27082935,331.01992878)(327.26214883,330.78121459)
\closepath
}
}
{
\newrgbcolor{curcolor}{0 0 0}
\pscustom[linestyle=none,fillstyle=solid,fillcolor=curcolor,opacity=0]
{
\newpath
\moveto(374.051579,348.91580732)
\lineto(417.9884979,348.91580732)
\lineto(417.9884979,312.12848024)
\lineto(374.051579,312.12848024)
\closepath
}
}
{
\newrgbcolor{curcolor}{0.94509804 0.99215686 0.99607843}
\pscustom[linestyle=none,fillstyle=solid,fillcolor=curcolor]
{
\newpath
\moveto(388.85884469,327.57810417)
\curveto(388.85884469,327.49997952)(388.85450443,327.4305354)(388.84582392,327.36977178)
\curveto(388.8371434,327.31334843)(388.8219525,327.26560559)(388.80025121,327.22654327)
\curveto(388.77854992,327.18748095)(388.75250837,327.15926927)(388.72212656,327.14190824)
\curveto(388.69608502,327.1245472)(388.66570321,327.11586669)(388.63098114,327.11586669)
\lineto(384.5880308,327.11586669)
\curveto(384.48820486,327.11586669)(384.39271919,327.14841862)(384.30157377,327.21352249)
\curveto(384.21476861,327.28296662)(384.17136603,327.40232372)(384.17136603,327.57159378)
\lineto(384.17136603,335.08458041)
\curveto(384.17136603,335.25385048)(384.21476861,335.37103744)(384.30157377,335.43614131)
\curveto(384.39271919,335.50558544)(384.48820486,335.54030751)(384.5880308,335.54030751)
\lineto(388.58540844,335.54030751)
\curveto(388.6201305,335.54030751)(388.65051231,335.53162699)(388.67655385,335.51426596)
\curveto(388.70693566,335.49690492)(388.73080708,335.46869325)(388.74816811,335.42963093)
\curveto(388.76552914,335.3905686)(388.77854992,335.34065564)(388.78723043,335.27989202)
\curveto(388.80025121,335.22346867)(388.80676159,335.15185441)(388.80676159,335.06504925)
\curveto(388.80676159,334.98692461)(388.80025121,334.91748048)(388.78723043,334.85671687)
\curveto(388.77854992,334.80029351)(388.76552914,334.75255067)(388.74816811,334.71348835)
\curveto(388.73080708,334.67876629)(388.70693566,334.65272474)(388.67655385,334.63536371)
\curveto(388.65051231,334.61800268)(388.6201305,334.60932216)(388.58540844,334.60932216)
\lineto(385.2911526,334.60932216)
\lineto(385.2911526,331.96610503)
\lineto(388.11666057,331.96610503)
\curveto(388.15138263,331.96610503)(388.18176444,331.95525438)(388.20780599,331.93355309)
\curveto(388.23818779,331.91619206)(388.26205921,331.89015051)(388.27942025,331.85542845)
\curveto(388.30112154,331.82070638)(388.31631244,331.77296354)(388.32499295,331.71219993)
\curveto(388.33367347,331.65143632)(388.33801373,331.57982206)(388.33801373,331.49735716)
\curveto(388.33801373,331.41923252)(388.33367347,331.35195852)(388.32499295,331.29553516)
\curveto(388.31631244,331.23911181)(388.30112154,331.1935391)(388.27942025,331.15881703)
\curveto(388.26205921,331.12409497)(388.23818779,331.09805342)(388.20780599,331.08069239)
\curveto(388.18176444,331.06767162)(388.15138263,331.06116123)(388.11666057,331.06116123)
\lineto(385.2911526,331.06116123)
\lineto(385.2911526,328.04685203)
\lineto(388.63098114,328.04685203)
\curveto(388.66570321,328.04685203)(388.69608502,328.03817152)(388.72212656,328.02081049)
\curveto(388.75250837,328.00344945)(388.77854992,327.97523778)(388.80025121,327.93617545)
\curveto(388.8219525,327.90145339)(388.8371434,327.85371055)(388.84582392,327.79294694)
\curveto(388.85450443,327.73652359)(388.85884469,327.66490933)(388.85884469,327.57810417)
\closepath
}
}
{
\newrgbcolor{curcolor}{0 0 0}
\pscustom[linestyle=none,fillstyle=solid,fillcolor=curcolor,opacity=0]
{
\newpath
\moveto(470.05424548,348.91580732)
\lineto(533.6131719,348.91580732)
\lineto(533.6131719,312.12848024)
\lineto(470.05424548,312.12848024)
\closepath
}
}
{
\newrgbcolor{curcolor}{0.94509804 0.99215686 0.99607843}
\pscustom[linestyle=none,fillstyle=solid,fillcolor=curcolor]
{
\newpath
\moveto(485.34980931,333.07287082)
\curveto(485.34980931,332.65186579)(485.28036518,332.27209322)(485.14147693,331.93355309)
\curveto(485.00258867,331.59501297)(484.8029368,331.30638581)(484.54252132,331.06767162)
\curveto(484.2864461,330.82895742)(483.96960726,330.64449646)(483.59200481,330.51428872)
\curveto(483.21874262,330.38408098)(482.77169605,330.31897711)(482.25086509,330.31897711)
\lineto(481.29383819,330.31897711)
\lineto(481.29383819,327.28513675)
\curveto(481.29383819,327.25041469)(481.28298755,327.22003288)(481.26128626,327.19399133)
\curveto(481.24392523,327.16794978)(481.21354342,327.14624849)(481.17014084,327.12888746)
\curveto(481.13107852,327.11586669)(481.07465516,327.10501604)(481.00087078,327.09633553)
\curveto(480.93142665,327.08331475)(480.84245136,327.07680437)(480.73394491,327.07680437)
\curveto(480.62543846,327.07680437)(480.53429304,327.08331475)(480.46050865,327.09633553)
\curveto(480.39106452,327.10501604)(480.33464117,327.11586669)(480.29123859,327.12888746)
\curveto(480.24783601,327.14624849)(480.2174542,327.16794978)(480.20009317,327.19399133)
\curveto(480.18273214,327.22003288)(480.17405162,327.25041469)(480.17405162,327.28513675)
\lineto(480.17405162,335.05853886)
\curveto(480.17405162,335.23214919)(480.21962433,335.35584654)(480.31076975,335.42963093)
\curveto(480.40191517,335.50341531)(480.50391123,335.54030751)(480.61675794,335.54030751)
\lineto(482.42013515,335.54030751)
\curveto(482.60242599,335.54030751)(482.77603631,335.53162699)(482.94096611,335.51426596)
\curveto(483.11023617,335.50124518)(483.30771791,335.46869325)(483.53341133,335.41661015)
\curveto(483.76344501,335.36886731)(483.99564881,335.27555177)(484.23002274,335.13666351)
\curveto(484.46873693,335.00211551)(484.67055893,334.83501558)(484.83548874,334.63536371)
\curveto(485.00041854,334.43571184)(485.12628602,334.20350804)(485.21309118,333.9387523)
\curveto(485.3042366,333.67833681)(485.34980931,333.38970966)(485.34980931,333.07287082)
\closepath
\moveto(484.17142926,332.9817254)
\curveto(484.17142926,333.32460579)(484.10632539,333.61106282)(483.97611765,333.84109649)
\curveto(483.85025017,334.07113017)(483.69183075,334.24257036)(483.5008594,334.35541707)
\curveto(483.3142283,334.46826377)(483.11891669,334.53987803)(482.91492456,334.57025984)
\curveto(482.71527269,334.60064164)(482.51996108,334.61583255)(482.32898973,334.61583255)
\lineto(481.29383819,334.61583255)
\lineto(481.29383819,331.23694168)
\lineto(482.30294818,331.23694168)
\curveto(482.64148831,331.23694168)(482.92143495,331.28034426)(483.14278811,331.36714942)
\curveto(483.36848153,331.45395458)(483.55728275,331.57331168)(483.70919178,331.72522071)
\curveto(483.86110081,331.88146999)(483.97611765,332.06593096)(484.05424229,332.2786036)
\curveto(484.13236694,332.4956165)(484.17142926,332.72999044)(484.17142926,332.9817254)
\closepath
}
}
{
\newrgbcolor{curcolor}{0.94509804 0.99215686 0.99607843}
\pscustom[linestyle=none,fillstyle=solid,fillcolor=curcolor]
{
\newpath
\moveto(491.57875631,329.47913718)
\curveto(491.57875631,329.0841737)(491.50497193,328.7326128)(491.35740316,328.42445448)
\curveto(491.21417464,328.11629616)(491.01235264,327.85371055)(490.75193716,327.63669765)
\curveto(490.49586194,327.42402501)(490.19204388,327.26343546)(489.84048298,327.15492901)
\curveto(489.49326234,327.04642256)(489.11783002,326.99216934)(488.71418602,326.99216934)
\curveto(488.43206925,326.99216934)(488.16948364,327.01604075)(487.92642919,327.06378359)
\curveto(487.687715,327.11152643)(487.47287223,327.17011991)(487.28190088,327.23956404)
\curveto(487.09526978,327.30900817)(486.93685036,327.38062243)(486.80664262,327.45440681)
\curveto(486.68077514,327.5281912)(486.59179985,327.59112494)(486.53971675,327.64320804)
\curveto(486.49197392,327.69529113)(486.45508172,327.760395)(486.42904017,327.83851965)
\curveto(486.40733888,327.92098455)(486.39648824,328.029491)(486.39648824,328.164039)
\curveto(486.39648824,328.25952468)(486.4008285,328.33764932)(486.40950901,328.39841293)
\curveto(486.41818953,328.4635168)(486.4312103,328.5155999)(486.44857134,328.55466222)
\curveto(486.46593237,328.59372454)(486.48763366,328.61976609)(486.51367521,328.63278687)
\curveto(486.53971675,328.6501479)(486.57009856,328.65882841)(486.60482062,328.65882841)
\curveto(486.66558424,328.65882841)(486.75021927,328.62193622)(486.85872572,328.54815184)
\curveto(486.97157243,328.47436745)(487.11480094,328.39407268)(487.28841126,328.30726752)
\curveto(487.46202158,328.22046235)(487.67035397,328.13799745)(487.91340842,328.05987281)
\curveto(488.16080312,327.98608842)(488.44509002,327.94919623)(488.76626912,327.94919623)
\curveto(489.00932357,327.94919623)(489.23067673,327.98174816)(489.43032859,328.04685203)
\curveto(489.63432072,328.1119559)(489.80793104,328.20310132)(489.95115956,328.32028829)
\curveto(490.09872833,328.44181551)(490.21157504,328.58938429)(490.28969968,328.76299461)
\curveto(490.36782433,328.93660493)(490.40688665,329.13408667)(490.40688665,329.35543983)
\curveto(490.40688665,329.59415402)(490.35263342,329.79814614)(490.24412697,329.96741621)
\curveto(490.13562052,330.13668627)(489.99239201,330.28425504)(489.81444143,330.41012253)
\curveto(489.63649085,330.54033027)(489.43249872,330.65751723)(489.20246505,330.76168343)
\curveto(488.97677163,330.87018988)(488.74456783,330.97869633)(488.50585364,331.08720278)
\curveto(488.26713945,331.20004949)(488.03493564,331.32374684)(487.80924222,331.45829484)
\curveto(487.58354881,331.59284284)(487.38172681,331.75126225)(487.20377623,331.93355309)
\curveto(487.02582565,332.11584393)(486.88042701,332.32851657)(486.7675803,332.57157102)
\curveto(486.65907385,332.81896573)(486.60482062,333.11410327)(486.60482062,333.45698366)
\curveto(486.60482062,333.80854456)(486.66775437,334.12104313)(486.79362185,334.39447939)
\curveto(486.92382959,334.6722559)(487.10178017,334.90445971)(487.32747358,335.0910908)
\curveto(487.55750726,335.27772189)(487.82877339,335.41878028)(488.14127196,335.51426596)
\curveto(488.4581108,335.61409189)(488.79882105,335.66400486)(489.16340273,335.66400486)
\curveto(489.35003382,335.66400486)(489.53666492,335.64664383)(489.72329601,335.61192176)
\curveto(489.91426736,335.58153996)(490.09221794,335.53813738)(490.25714775,335.48171402)
\curveto(490.42641781,335.42963093)(490.57615671,335.36886731)(490.70636445,335.29942319)
\curveto(490.83657219,335.23431931)(490.92120722,335.18006609)(490.96026955,335.13666351)
\curveto(491.00367213,335.09760119)(491.0318838,335.06504925)(491.04490458,335.0390077)
\curveto(491.05792535,335.01730641)(491.068776,334.98692461)(491.07745651,334.94786229)
\curveto(491.08613703,334.91314022)(491.09264742,334.86973764)(491.09698767,334.81765454)
\curveto(491.10132793,334.76557145)(491.10349806,334.69829745)(491.10349806,334.61583255)
\curveto(491.10349806,334.5377079)(491.0991578,334.46826377)(491.09047729,334.40750016)
\curveto(491.08613703,334.34673655)(491.07745651,334.29465345)(491.06443574,334.25125087)
\curveto(491.05141497,334.21218855)(491.0318838,334.18180674)(491.00584226,334.16010545)
\curveto(490.98414097,334.14274442)(490.95809942,334.13406391)(490.92771761,334.13406391)
\curveto(490.87997477,334.13406391)(490.80402026,334.16444571)(490.69985407,334.22520933)
\curveto(490.60002813,334.28597294)(490.47633078,334.35324694)(490.32876201,334.42703132)
\curveto(490.18119323,334.50515597)(490.00541278,334.5746001)(489.80142066,334.63536371)
\curveto(489.60176879,334.70046758)(489.37607537,334.73301951)(489.1243404,334.73301951)
\curveto(488.88996647,334.73301951)(488.68597434,334.70046758)(488.51236402,334.63536371)
\curveto(488.3387537,334.5746001)(488.19552519,334.49213519)(488.08267848,334.387969)
\curveto(487.96983177,334.28380281)(487.88519674,334.16010545)(487.82877339,334.01687694)
\curveto(487.77235003,333.87364843)(487.74413835,333.72173939)(487.74413835,333.56114985)
\curveto(487.74413835,333.32677592)(487.79839158,333.12495392)(487.90689803,332.95568385)
\curveto(488.01540448,332.78641379)(488.158633,332.63667489)(488.33658357,332.50646715)
\curveto(488.51887441,332.37625941)(488.72503667,332.25690231)(488.95507034,332.14839586)
\curveto(489.18510402,332.03988941)(489.41947795,331.92921283)(489.65819214,331.81636612)
\curveto(489.89690633,331.70785967)(490.13128027,331.58633245)(490.36131394,331.45178445)
\curveto(490.59134762,331.32157671)(490.79533974,331.16532742)(490.97329032,330.98303658)
\curveto(491.15558116,330.80508601)(491.3009798,330.59241336)(491.40948625,330.34501866)
\curveto(491.52233296,330.10196421)(491.57875631,329.81333705)(491.57875631,329.47913718)
\closepath
}
}
{
\newrgbcolor{curcolor}{0.94509804 0.99215686 0.99607843}
\pscustom[linestyle=none,fillstyle=solid,fillcolor=curcolor]
{
\newpath
\moveto(494.29381187,327.28513675)
\curveto(494.29381187,327.25041469)(494.28513135,327.22003288)(494.26777032,327.19399133)
\curveto(494.25040929,327.16794978)(494.22002748,327.14624849)(494.1766249,327.12888746)
\curveto(494.13322232,327.11586669)(494.07462884,327.10501604)(494.00084445,327.09633553)
\curveto(493.93140032,327.08331475)(493.84242503,327.07680437)(493.73391858,327.07680437)
\curveto(493.62975239,327.07680437)(493.5407771,327.08331475)(493.46699271,327.09633553)
\curveto(493.39320833,327.10501604)(493.33461485,327.11586669)(493.29121227,327.12888746)
\curveto(493.24780968,327.14624849)(493.21742788,327.16794978)(493.20006685,327.19399133)
\curveto(493.18270581,327.22003288)(493.1740253,327.25041469)(493.1740253,327.28513675)
\lineto(493.1740253,335.37103744)
\curveto(493.1740253,335.40575951)(493.18270581,335.43614131)(493.20006685,335.46218286)
\curveto(493.22176814,335.48822441)(493.25432007,335.50775557)(493.29772265,335.52077634)
\curveto(493.34546549,335.53813738)(493.40405897,335.55115815)(493.4735031,335.55983867)
\curveto(493.54728749,335.57285944)(493.63409265,335.57936983)(493.73391858,335.57936983)
\curveto(493.84242503,335.57936983)(493.93140032,335.57285944)(494.00084445,335.55983867)
\curveto(494.07462884,335.55115815)(494.13322232,335.53813738)(494.1766249,335.52077634)
\curveto(494.22002748,335.50775557)(494.25040929,335.48822441)(494.26777032,335.46218286)
\curveto(494.28513135,335.43614131)(494.29381187,335.40575951)(494.29381187,335.37103744)
\closepath
}
}
{
\newrgbcolor{curcolor}{0.94509804 0.99215686 0.99607843}
\pscustom[linestyle=none,fillstyle=solid,fillcolor=curcolor]
{
\newpath
\moveto(501.36441281,327.58461455)
\curveto(501.36441281,327.50214965)(501.36007255,327.4305354)(501.35139203,327.36977178)
\curveto(501.34271152,327.31334843)(501.32752061,327.26560559)(501.30581932,327.22654327)
\curveto(501.28845829,327.18748095)(501.26458687,327.15926927)(501.23420507,327.14190824)
\curveto(501.20816352,327.1245472)(501.17561158,327.11586669)(501.13654926,327.11586669)
\lineto(496.16261357,327.11586669)
\curveto(496.04976686,327.11586669)(495.95862144,327.15058875)(495.88917732,327.22003288)
\curveto(495.81973319,327.28947701)(495.78501112,327.40015359)(495.78501112,327.55206262)
\lineto(495.78501112,327.79294694)
\curveto(495.78501112,327.84937029)(495.78718125,327.90145339)(495.79152151,327.94919623)
\curveto(495.80020203,327.99693907)(495.81539293,328.04685203)(495.83709422,328.09893513)
\curveto(495.85879551,328.15535848)(495.88917732,328.2161221)(495.92823964,328.28122597)
\curveto(495.96730196,328.3506701)(496.0150448,328.43096487)(496.07146815,328.52211029)
\lineto(499.95816921,334.60281177)
\lineto(496.11704086,334.60281177)
\curveto(496.07363828,334.60281177)(496.03674609,334.61149229)(496.00636428,334.62885332)
\curveto(495.97598248,334.64621435)(495.94994093,334.6722559)(495.92823964,334.70697797)
\curveto(495.91087861,334.74604029)(495.89785783,334.79378313)(495.88917732,334.85020648)
\curveto(495.8804968,334.91097009)(495.87615654,334.98041422)(495.87615654,335.05853886)
\curveto(495.87615654,335.14534403)(495.8804968,335.21912841)(495.88917732,335.27989202)
\curveto(495.89785783,335.34065564)(495.91087861,335.3905686)(495.92823964,335.42963093)
\curveto(495.94994093,335.46869325)(495.97598248,335.49690492)(496.00636428,335.51426596)
\curveto(496.03674609,335.53162699)(496.07363828,335.54030751)(496.11704086,335.54030751)
\lineto(500.80451952,335.54030751)
\curveto(500.92170649,335.54030751)(501.01285191,335.50558544)(501.07795578,335.43614131)
\curveto(501.14739991,335.37103744)(501.18212197,335.26904138)(501.18212197,335.13015312)
\lineto(501.18212197,334.87624803)
\curveto(501.18212197,334.8068039)(501.17778171,334.74387016)(501.1691012,334.6874468)
\curveto(501.16042068,334.63536371)(501.14522978,334.58111048)(501.12352849,334.52468713)
\curveto(501.1018272,334.46826377)(501.07144539,334.40750016)(501.03238307,334.34239629)
\curveto(500.99766101,334.27729242)(500.94991817,334.20133791)(500.88915455,334.11453275)
\lineto(497.02198466,328.05987281)
\lineto(501.13654926,328.05987281)
\curveto(501.21033365,328.05987281)(501.266757,328.02298061)(501.30581932,327.94919623)
\curveto(501.34488165,327.87541184)(501.36441281,327.75388462)(501.36441281,327.58461455)
\closepath
}
}
{
\newrgbcolor{curcolor}{0.94509804 0.99215686 0.99607843}
\pscustom[linestyle=none,fillstyle=solid,fillcolor=curcolor]
{
\newpath
\moveto(507.44815065,327.57810417)
\curveto(507.44815065,327.49997952)(507.4438104,327.4305354)(507.43512988,327.36977178)
\curveto(507.42644936,327.31334843)(507.41125846,327.26560559)(507.38955717,327.22654327)
\curveto(507.36785588,327.18748095)(507.34181433,327.15926927)(507.31143253,327.14190824)
\curveto(507.28539098,327.1245472)(507.25500917,327.11586669)(507.22028711,327.11586669)
\lineto(503.17733676,327.11586669)
\curveto(503.07751083,327.11586669)(502.98202515,327.14841862)(502.89087973,327.21352249)
\curveto(502.80407457,327.28296662)(502.76067199,327.40232372)(502.76067199,327.57159378)
\lineto(502.76067199,335.08458041)
\curveto(502.76067199,335.25385048)(502.80407457,335.37103744)(502.89087973,335.43614131)
\curveto(502.98202515,335.50558544)(503.07751083,335.54030751)(503.17733676,335.54030751)
\lineto(507.1747144,335.54030751)
\curveto(507.20943646,335.54030751)(507.23981827,335.53162699)(507.26585982,335.51426596)
\curveto(507.29624162,335.49690492)(507.32011304,335.46869325)(507.33747407,335.42963093)
\curveto(507.35483511,335.3905686)(507.36785588,335.34065564)(507.3765364,335.27989202)
\curveto(507.38955717,335.22346867)(507.39606756,335.15185441)(507.39606756,335.06504925)
\curveto(507.39606756,334.98692461)(507.38955717,334.91748048)(507.3765364,334.85671687)
\curveto(507.36785588,334.80029351)(507.35483511,334.75255067)(507.33747407,334.71348835)
\curveto(507.32011304,334.67876629)(507.29624162,334.65272474)(507.26585982,334.63536371)
\curveto(507.23981827,334.61800268)(507.20943646,334.60932216)(507.1747144,334.60932216)
\lineto(503.88045856,334.60932216)
\lineto(503.88045856,331.96610503)
\lineto(506.70596653,331.96610503)
\curveto(506.7406886,331.96610503)(506.7710704,331.95525438)(506.79711195,331.93355309)
\curveto(506.82749376,331.91619206)(506.85136518,331.89015051)(506.86872621,331.85542845)
\curveto(506.8904275,331.82070638)(506.9056184,331.77296354)(506.91429892,331.71219993)
\curveto(506.92297943,331.65143632)(506.92731969,331.57982206)(506.92731969,331.49735716)
\curveto(506.92731969,331.41923252)(506.92297943,331.35195852)(506.91429892,331.29553516)
\curveto(506.9056184,331.23911181)(506.8904275,331.1935391)(506.86872621,331.15881703)
\curveto(506.85136518,331.12409497)(506.82749376,331.09805342)(506.79711195,331.08069239)
\curveto(506.7710704,331.06767162)(506.7406886,331.06116123)(506.70596653,331.06116123)
\lineto(503.88045856,331.06116123)
\lineto(503.88045856,328.04685203)
\lineto(507.22028711,328.04685203)
\curveto(507.25500917,328.04685203)(507.28539098,328.03817152)(507.31143253,328.02081049)
\curveto(507.34181433,328.00344945)(507.36785588,327.97523778)(507.38955717,327.93617545)
\curveto(507.41125846,327.90145339)(507.42644936,327.85371055)(507.43512988,327.79294694)
\curveto(507.4438104,327.73652359)(507.44815065,327.66490933)(507.44815065,327.57810417)
\closepath
}
}
{
\newrgbcolor{curcolor}{0 0 0}
\pscustom[linestyle=none,fillstyle=solid,fillcolor=curcolor,opacity=0]
{
\newpath
\moveto(391.54353833,348.91580732)
\lineto(445.65366497,348.91580732)
\lineto(445.65366497,312.12848024)
\lineto(391.54353833,312.12848024)
\closepath
}
}
{
\newrgbcolor{curcolor}{0.94509804 0.99215686 0.99607843}
\pscustom[linestyle=none,fillstyle=solid,fillcolor=curcolor]
{
\newpath
\moveto(405.75183366,328.17054939)
\curveto(405.75183366,328.096765)(405.74966353,328.03166113)(405.74532328,327.97523778)
\curveto(405.74098302,327.92315468)(405.7323025,327.87758197)(405.71928173,327.83851965)
\curveto(405.71060121,327.80379758)(405.69758044,327.77124565)(405.68021941,327.74086384)
\curveto(405.66719863,327.7148223)(405.63247657,327.67358984)(405.57605321,327.61716649)
\curveto(405.52397012,327.56508339)(405.4328247,327.49780939)(405.30261696,327.41534449)
\curveto(405.17240922,327.33721985)(405.02484044,327.26560559)(404.85991064,327.20050172)
\curveto(404.69932109,327.13973811)(404.52354064,327.08982514)(404.33256929,327.05076282)
\curveto(404.14159794,327.0117005)(403.9441162,326.99216934)(403.74012407,326.99216934)
\curveto(403.31911904,326.99216934)(402.94585685,327.06161346)(402.6203375,327.20050172)
\curveto(402.29481815,327.33938998)(402.02138189,327.54121197)(401.80002874,327.80596771)
\curveto(401.58301583,328.07506371)(401.4159159,328.40275319)(401.29872893,328.78903616)
\curveto(401.18588223,329.17965938)(401.12945887,329.62887608)(401.12945887,330.13668627)
\curveto(401.12945887,330.71394059)(401.198903,331.20873)(401.33779126,331.62105451)
\curveto(401.48101977,332.03771928)(401.67416125,332.37842954)(401.9172157,332.64318528)
\curveto(402.16461041,332.90794102)(402.45323757,333.10325263)(402.78309718,333.22912011)
\curveto(403.11729704,333.35932785)(403.47753846,333.42443172)(403.86382142,333.42443172)
\curveto(404.05045252,333.42443172)(404.23057323,333.40707069)(404.40418355,333.37234862)
\curveto(404.58213413,333.33762656)(404.7448938,333.29205385)(404.89246257,333.2356305)
\curveto(405.04003135,333.17920714)(405.17023909,333.11410327)(405.2830858,333.04031889)
\curveto(405.40027276,332.9665345)(405.48490779,332.90360076)(405.53699089,332.85151766)
\curveto(405.58907399,332.79943457)(405.62379605,332.75820211)(405.64115708,332.72782031)
\curveto(405.66285837,332.6974385)(405.68021941,332.66054631)(405.69324018,332.61714373)
\curveto(405.70626095,332.57808141)(405.71494147,332.5325087)(405.71928173,332.4804256)
\curveto(405.72362199,332.4283425)(405.72579211,332.36323863)(405.72579211,332.28511399)
\curveto(405.72579211,332.11584393)(405.70626095,331.99648683)(405.66719863,331.9270427)
\curveto(405.62813631,331.86193883)(405.58039347,331.8293869)(405.52397012,331.8293869)
\curveto(405.45886625,331.8293869)(405.38291173,331.86410896)(405.29610657,331.93355309)
\curveto(405.21364167,332.00733748)(405.10730535,332.08763225)(404.97709761,332.17443741)
\curveto(404.84688987,332.26124257)(404.68847045,332.33936722)(404.50183935,332.40881134)
\curveto(404.31954852,332.48259573)(404.10253562,332.51948792)(403.85080065,332.51948792)
\curveto(403.33430995,332.51948792)(402.93717634,332.31983605)(402.65939982,331.92053232)
\curveto(402.38596357,331.52556884)(402.24924544,330.95048465)(402.24924544,330.19527975)
\curveto(402.24924544,329.81767731)(402.2839675,329.48564757)(402.35341163,329.19919054)
\curveto(402.42719602,328.91707377)(402.53353234,328.6805297)(402.6724206,328.48955835)
\curveto(402.81130885,328.298587)(402.98057892,328.15535848)(403.18023079,328.05987281)
\curveto(403.38422291,327.96872739)(403.61642672,327.92315468)(403.8768422,327.92315468)
\curveto(404.12423691,327.92315468)(404.34124981,327.962217)(404.5278809,328.04034165)
\curveto(404.714512,328.11846629)(404.87510154,328.20310132)(405.00964954,328.29424674)
\curveto(405.1485378,328.38973242)(405.26355464,328.47436745)(405.35470005,328.54815184)
\curveto(405.45018573,328.62627648)(405.52397012,328.6653388)(405.57605321,328.6653388)
\curveto(405.60643502,328.6653388)(405.63247657,328.65665829)(405.65417786,328.63929725)
\curveto(405.67587915,328.62193622)(405.69324018,328.59155442)(405.70626095,328.54815184)
\curveto(405.72362199,328.50908951)(405.73447263,328.45700642)(405.73881289,328.39190255)
\curveto(405.7474934,328.33113893)(405.75183366,328.25735455)(405.75183366,328.17054939)
\closepath
}
}
{
\newrgbcolor{curcolor}{0.94509804 0.99215686 0.99607843}
\pscustom[linestyle=none,fillstyle=solid,fillcolor=curcolor]
{
\newpath
\moveto(412.508796,330.2734044)
\curveto(412.508796,329.79597602)(412.44586226,329.35543983)(412.31999478,328.95179583)
\curveto(412.19412729,328.55249209)(412.00532607,328.20744158)(411.75359111,327.91664429)
\curveto(411.5061964,327.62584701)(411.19369782,327.39798346)(410.81609537,327.23305366)
\curveto(410.44283318,327.07246411)(410.00880738,326.99216934)(409.51401797,326.99216934)
\curveto(409.03224933,326.99216934)(408.6112443,327.06378359)(408.25100288,327.20701211)
\curveto(407.89510173,327.35024062)(407.59779405,327.55857301)(407.35907986,327.83200926)
\curveto(407.12036567,328.10544552)(406.94241509,328.43747526)(406.82522812,328.82809848)
\curveto(406.70804116,329.2187217)(406.64944767,329.66142802)(406.64944767,330.15621743)
\curveto(406.64944767,330.63364581)(406.71021129,331.07201187)(406.83173851,331.47131561)
\curveto(406.95760599,331.87495961)(407.14423709,332.22218025)(407.3916318,332.51297754)
\curveto(407.64336676,332.80377482)(407.95586534,333.02946824)(408.32912753,333.19005779)
\curveto(408.70238972,333.35064733)(409.13858565,333.43094211)(409.63771532,333.43094211)
\curveto(410.11948396,333.43094211)(410.53831886,333.35932785)(410.89422002,333.21609934)
\curveto(411.25446143,333.07287082)(411.55393924,332.86453844)(411.79265343,332.59110218)
\curveto(412.03136762,332.31766593)(412.2093182,331.98563619)(412.32650516,331.59501297)
\curveto(412.44803239,331.20438974)(412.508796,330.76385355)(412.508796,330.2734044)
\closepath
\moveto(411.37598866,330.20179014)
\curveto(411.37598866,330.51862898)(411.34560685,330.81810678)(411.28484324,331.10022355)
\curveto(411.22841989,331.38234032)(411.13293421,331.62973503)(410.99838621,331.84240767)
\curveto(410.86383821,332.05508032)(410.68154738,332.22218025)(410.4515137,332.34370747)
\curveto(410.22148002,332.46957496)(409.935023,332.5325087)(409.59214261,332.5325087)
\curveto(409.27530378,332.5325087)(409.00186752,332.47608534)(408.77183385,332.36323863)
\curveto(408.54614043,332.25039193)(408.35950933,332.08980238)(408.21194056,331.88146999)
\curveto(408.06437179,331.67747787)(407.95369521,331.43442342)(407.87991082,331.15230665)
\curveto(407.81046669,330.87018988)(407.77574463,330.56203156)(407.77574463,330.22783169)
\curveto(407.77574463,329.9066526)(407.80395631,329.60500466)(407.86037966,329.32288789)
\curveto(407.92114327,329.04077112)(408.01879908,328.79337641)(408.15334708,328.58070377)
\curveto(408.29223534,328.37237139)(408.4766963,328.20527145)(408.70672998,328.07940397)
\curveto(408.93676365,327.95787674)(409.22322068,327.89711313)(409.56610106,327.89711313)
\curveto(409.87859964,327.89711313)(410.14986577,327.95353649)(410.37989944,328.0663832)
\curveto(410.60993312,328.1792299)(410.79873434,328.33764932)(410.94630311,328.54164145)
\curveto(411.09387189,328.74563358)(411.20237834,328.98868802)(411.27182247,329.2708048)
\curveto(411.34126659,329.55292157)(411.37598866,329.86325002)(411.37598866,330.20179014)
\closepath
}
}
{
\newrgbcolor{curcolor}{0.94509804 0.99215686 0.99607843}
\pscustom[linestyle=none,fillstyle=solid,fillcolor=curcolor]
{
\newpath
\moveto(422.77762068,327.27862636)
\curveto(422.77762068,327.2439043)(422.76894016,327.21352249)(422.75157913,327.18748095)
\curveto(422.7342181,327.16577966)(422.70600642,327.14624849)(422.6669441,327.12888746)
\curveto(422.62788178,327.11152643)(422.57362855,327.09850566)(422.50418442,327.08982514)
\curveto(422.43474029,327.08114462)(422.34793513,327.07680437)(422.24376894,327.07680437)
\curveto(422.13526249,327.07680437)(422.0462872,327.08114462)(421.97684307,327.08982514)
\curveto(421.90739894,327.09850566)(421.85097559,327.11152643)(421.80757301,327.12888746)
\curveto(421.76851069,327.14624849)(421.74029901,327.16577966)(421.72293798,327.18748095)
\curveto(421.70557695,327.21352249)(421.69689643,327.2439043)(421.69689643,327.27862636)
\lineto(421.69689643,330.83980807)
\curveto(421.69689643,331.08720278)(421.67519514,331.31289619)(421.63179256,331.51688832)
\curveto(421.58838998,331.72088045)(421.51894585,331.8966609)(421.42346018,332.04422967)
\curveto(421.3279745,332.19179844)(421.20644727,332.30464515)(421.0588785,332.3827698)
\curveto(420.91130973,332.46089444)(420.73769941,332.49995676)(420.53804754,332.49995676)
\curveto(420.29065283,332.49995676)(420.041088,332.40447109)(419.78935303,332.21349973)
\curveto(419.54195832,332.02252838)(419.26852207,331.74258174)(418.96904427,331.37365981)
\lineto(418.96904427,327.27862636)
\curveto(418.96904427,327.2439043)(418.96036375,327.21352249)(418.94300272,327.18748095)
\curveto(418.92564168,327.16577966)(418.89525988,327.14624849)(418.8518573,327.12888746)
\curveto(418.81279498,327.11152643)(418.75854175,327.09850566)(418.68909762,327.08982514)
\curveto(418.61965349,327.08114462)(418.53284833,327.07680437)(418.42868214,327.07680437)
\curveto(418.32885621,327.07680437)(418.24205105,327.08114462)(418.16826666,327.08982514)
\curveto(418.09882253,327.09850566)(418.04239918,327.11152643)(417.9989966,327.12888746)
\curveto(417.95993428,327.14624849)(417.9317226,327.16577966)(417.91436157,327.18748095)
\curveto(417.90134079,327.21352249)(417.89483041,327.2439043)(417.89483041,327.27862636)
\lineto(417.89483041,330.83980807)
\curveto(417.89483041,331.08720278)(417.87095899,331.31289619)(417.82321615,331.51688832)
\curveto(417.77547331,331.72088045)(417.70385905,331.8966609)(417.60837338,332.04422967)
\curveto(417.5128877,332.19179844)(417.39136047,332.30464515)(417.2437917,332.3827698)
\curveto(417.10056319,332.46089444)(416.929123,332.49995676)(416.72947113,332.49995676)
\curveto(416.48207642,332.49995676)(416.23251158,332.40447109)(415.98077662,332.21349973)
\curveto(415.72904165,332.02252838)(415.4556054,331.74258174)(415.16046785,331.37365981)
\lineto(415.16046785,327.27862636)
\curveto(415.16046785,327.2439043)(415.15178734,327.21352249)(415.1344263,327.18748095)
\curveto(415.11706527,327.16577966)(415.0888536,327.14624849)(415.04979127,327.12888746)
\curveto(415.01072895,327.11152643)(414.95647573,327.09850566)(414.8870316,327.08982514)
\curveto(414.81758747,327.08114462)(414.72861218,327.07680437)(414.62010573,327.07680437)
\curveto(414.51593954,327.07680437)(414.42913438,327.08114462)(414.35969025,327.08982514)
\curveto(414.29024612,327.09850566)(414.23382277,327.11152643)(414.19042019,327.12888746)
\curveto(414.15135786,327.14624849)(414.12314619,327.16577966)(414.10578515,327.18748095)
\curveto(414.09276438,327.21352249)(414.08625399,327.2439043)(414.08625399,327.27862636)
\lineto(414.08625399,333.13797469)
\curveto(414.08625399,333.17269676)(414.09276438,333.20090843)(414.10578515,333.22260972)
\curveto(414.11880593,333.24865127)(414.14484748,333.27035256)(414.1839098,333.28771359)
\curveto(414.22297212,333.30941488)(414.27288509,333.32243566)(414.3336487,333.32677592)
\curveto(414.39441231,333.33545643)(414.47470709,333.33979669)(414.57453302,333.33979669)
\curveto(414.6700187,333.33979669)(414.74814334,333.33545643)(414.80890695,333.32677592)
\curveto(414.87401082,333.32243566)(414.92392379,333.30941488)(414.95864585,333.28771359)
\curveto(414.99336792,333.27035256)(415.01723934,333.24865127)(415.03026011,333.22260972)
\curveto(415.04762114,333.20090843)(415.05630166,333.17269676)(415.05630166,333.13797469)
\lineto(415.05630166,332.36323863)
\curveto(415.38616127,332.73216057)(415.70517023,333.00125656)(416.01332855,333.17052663)
\curveto(416.32582713,333.34413695)(416.64049584,333.43094211)(416.95733467,333.43094211)
\curveto(417.20038912,333.43094211)(417.41740202,333.40273043)(417.60837338,333.34630708)
\curveto(417.80368499,333.28988372)(417.97512518,333.20958895)(418.12269395,333.10542276)
\curveto(418.27026272,333.00559682)(418.39613021,332.8840696)(418.5002964,332.74084108)
\curveto(418.60446259,332.60195283)(418.69126775,332.44570354)(418.76071188,332.27209322)
\curveto(418.95602349,332.48476586)(419.14048446,332.66488657)(419.31409478,332.81245534)
\curveto(419.49204536,332.96002411)(419.66131542,333.07938121)(419.82190497,333.17052663)
\curveto(419.98683477,333.26167204)(420.14525419,333.32677592)(420.29716322,333.36583824)
\curveto(420.45341251,333.40924082)(420.6096618,333.43094211)(420.76591109,333.43094211)
\curveto(421.14351353,333.43094211)(421.46035237,333.36366811)(421.71642759,333.22912011)
\curveto(421.97250281,333.09891237)(422.17866507,332.92313192)(422.33491436,332.70177876)
\curveto(422.49550391,332.4804256)(422.60835061,332.22001012)(422.67345449,331.92053232)
\curveto(422.74289861,331.62539477)(422.77762068,331.31289619)(422.77762068,330.98303658)
\closepath
}
}
{
\newrgbcolor{curcolor}{0.94509804 0.99215686 0.99607843}
\pscustom[linestyle=none,fillstyle=solid,fillcolor=curcolor]
{
\newpath
\moveto(430.11683325,330.29944595)
\curveto(430.11683325,329.7872955)(430.0604099,329.32722815)(429.94756319,328.9192439)
\curveto(429.83905674,328.51125964)(429.67629706,328.164039)(429.45928416,327.87758197)
\curveto(429.24661152,327.5954652)(428.98185578,327.37628217)(428.66501694,327.22003288)
\curveto(428.34817811,327.06812385)(427.98576656,326.99216934)(427.57778231,326.99216934)
\curveto(427.40417199,326.99216934)(427.24358244,327.00953037)(427.09601367,327.04425243)
\curveto(426.9484449,327.0789745)(426.80304625,327.13322772)(426.65981774,327.20701211)
\curveto(426.52092948,327.28079649)(426.38204122,327.37411204)(426.24315297,327.48695875)
\curveto(426.10426471,327.59980546)(425.95669594,327.73218333)(425.80044665,327.88409236)
\lineto(425.80044665,324.95441819)
\curveto(425.80044665,324.91969613)(425.79176613,324.88931432)(425.7744051,324.86327278)
\curveto(425.75704407,324.83723123)(425.72883239,324.81552994)(425.68977007,324.79816891)
\curveto(425.65070775,324.78080787)(425.59645452,324.7677871)(425.52701039,324.75910658)
\curveto(425.45756627,324.75042607)(425.36859098,324.74608581)(425.26008453,324.74608581)
\curveto(425.15591833,324.74608581)(425.06911317,324.75042607)(424.99966905,324.75910658)
\curveto(424.93022492,324.7677871)(424.87380156,324.78080787)(424.83039898,324.79816891)
\curveto(424.79133666,324.81552994)(424.76312498,324.83723123)(424.74576395,324.86327278)
\curveto(424.73274318,324.88931432)(424.72623279,324.91969613)(424.72623279,324.95441819)
\lineto(424.72623279,333.13797469)
\curveto(424.72623279,333.17703701)(424.73274318,333.20741882)(424.74576395,333.22912011)
\curveto(424.75878473,333.25516166)(424.78482627,333.27686295)(424.8238886,333.29422398)
\curveto(424.86295092,333.31158501)(424.91286389,333.32243566)(424.9736275,333.32677592)
\curveto(425.03439111,333.33545643)(425.1081755,333.33979669)(425.19498066,333.33979669)
\curveto(425.28612607,333.33979669)(425.35991046,333.33545643)(425.41633382,333.32677592)
\curveto(425.47709743,333.32243566)(425.52701039,333.31158501)(425.56607272,333.29422398)
\curveto(425.60513504,333.27686295)(425.63117659,333.25516166)(425.64419736,333.22912011)
\curveto(425.66155839,333.20741882)(425.67023891,333.17703701)(425.67023891,333.13797469)
\lineto(425.67023891,332.35021786)
\curveto(425.84818949,332.5325087)(426.01962968,332.69092812)(426.18455948,332.82547611)
\curveto(426.34948929,332.96002411)(426.51441909,333.07070069)(426.6793489,333.15750585)
\curveto(426.84861896,333.24865127)(427.02005915,333.31592527)(427.19366947,333.35932785)
\curveto(427.37162005,333.40707069)(427.55825115,333.43094211)(427.75356276,333.43094211)
\curveto(428.17890804,333.43094211)(428.54131959,333.34847721)(428.84079739,333.1835474)
\curveto(429.1402752,333.0186176)(429.38332965,332.79292418)(429.56996074,332.50646715)
\curveto(429.76093209,332.22001012)(429.89982035,331.88581025)(429.98662551,331.50386755)
\curveto(430.07343067,331.1262651)(430.11683325,330.72479123)(430.11683325,330.29944595)
\closepath
\moveto(428.98402591,330.17574859)
\curveto(428.98402591,330.4752264)(428.96015449,330.76385355)(428.91241165,331.04163007)
\curveto(428.86900907,331.32374684)(428.79088443,331.57331168)(428.67803772,331.79032458)
\curveto(428.56953127,332.00733748)(428.42196249,332.1809478)(428.2353314,332.31115554)
\curveto(428.0487003,332.44136328)(427.8164965,332.50646715)(427.53871999,332.50646715)
\curveto(427.39983173,332.50646715)(427.2631136,332.48476586)(427.1285656,332.44136328)
\curveto(426.99401761,332.40230096)(426.85729948,332.33719709)(426.71841122,332.24605167)
\curveto(426.57952296,332.15924651)(426.43412432,332.04205954)(426.28221529,331.89449077)
\curveto(426.13030626,331.75126225)(425.96971671,331.57331168)(425.80044665,331.36063903)
\lineto(425.80044665,329.02992048)
\curveto(426.0955842,328.66967906)(426.37553084,328.39407268)(426.64028658,328.20310132)
\curveto(426.90504232,328.01212997)(427.18281883,327.91664429)(427.47361612,327.91664429)
\curveto(427.74271211,327.91664429)(427.97274579,327.98174816)(428.16371714,328.1119559)
\curveto(428.35902875,328.24216364)(428.51527804,328.41577397)(428.63246501,328.63278687)
\curveto(428.75399223,328.84979977)(428.84296752,329.09285422)(428.89939088,329.36195021)
\curveto(428.95581423,329.63104621)(428.98402591,329.90231234)(428.98402591,330.17574859)
\closepath
}
}
{
\newrgbcolor{curcolor}{0 0 0}
\pscustom[linestyle=none,fillstyle=solid,fillcolor=curcolor,opacity=0]
{
\newpath
\moveto(610.55606071,348.91580732)
\lineto(674.11490838,348.91580732)
\lineto(674.11490838,312.12848024)
\lineto(610.55606071,312.12848024)
\closepath
}
}
{
\newrgbcolor{curcolor}{0.94509804 0.99215686 0.99607843}
\pscustom[linestyle=none,fillstyle=solid,fillcolor=curcolor]
{
\newpath
\moveto(625.0442768,327.60414572)
\curveto(625.0442768,327.51734056)(625.03993655,327.44355617)(625.03125603,327.38279256)
\curveto(625.02257551,327.3263692)(625.00738461,327.27645624)(624.98568332,327.23305366)
\curveto(624.96832229,327.19399133)(624.94445087,327.16360953)(624.91406906,327.14190824)
\curveto(624.88802752,327.1245472)(624.85547558,327.11586669)(624.81641326,327.11586669)
\lineto(621.09247188,327.11586669)
\curveto(620.99264594,327.11586669)(620.89716027,327.14841862)(620.80601485,327.21352249)
\curveto(620.71920969,327.28296662)(620.67580711,327.40232372)(620.67580711,327.57159378)
\lineto(620.67580711,335.37103744)
\curveto(620.67580711,335.40575951)(620.68448762,335.43614131)(620.70184866,335.46218286)
\curveto(620.71920969,335.48822441)(620.74959149,335.50775557)(620.79299407,335.52077634)
\curveto(620.83639665,335.53813738)(620.89499014,335.55115815)(620.96877452,335.55983867)
\curveto(621.04255891,335.57285944)(621.1315342,335.57936983)(621.23570039,335.57936983)
\curveto(621.34420684,335.57936983)(621.43318213,335.57285944)(621.50262626,335.55983867)
\curveto(621.57641065,335.55115815)(621.63500413,335.53813738)(621.67840671,335.52077634)
\curveto(621.72180929,335.50775557)(621.7521911,335.48822441)(621.76955213,335.46218286)
\curveto(621.78691316,335.43614131)(621.79559368,335.40575951)(621.79559368,335.37103744)
\lineto(621.79559368,328.08591436)
\lineto(624.81641326,328.08591436)
\curveto(624.85547558,328.08591436)(624.88802752,328.07506371)(624.91406906,328.05336242)
\curveto(624.94445087,328.03600139)(624.96832229,328.00778971)(624.98568332,327.96872739)
\curveto(625.00738461,327.93400533)(625.02257551,327.88626249)(625.03125603,327.82549887)
\curveto(625.03993655,327.76473526)(625.0442768,327.69095088)(625.0442768,327.60414572)
\closepath
}
}
{
\newrgbcolor{curcolor}{0.94509804 0.99215686 0.99607843}
\pscustom[linestyle=none,fillstyle=solid,fillcolor=curcolor]
{
\newpath
\moveto(630.80051466,329.47913718)
\curveto(630.80051466,329.0841737)(630.72673027,328.7326128)(630.5791615,328.42445448)
\curveto(630.43593298,328.11629616)(630.23411099,327.85371055)(629.9736955,327.63669765)
\curveto(629.71762028,327.42402501)(629.41380222,327.26343546)(629.06224132,327.15492901)
\curveto(628.71502068,327.04642256)(628.33958836,326.99216934)(627.93594436,326.99216934)
\curveto(627.65382759,326.99216934)(627.39124198,327.01604075)(627.14818753,327.06378359)
\curveto(626.90947334,327.11152643)(626.69463057,327.17011991)(626.50365922,327.23956404)
\curveto(626.31702812,327.30900817)(626.15860871,327.38062243)(626.02840096,327.45440681)
\curveto(625.90253348,327.5281912)(625.81355819,327.59112494)(625.7614751,327.64320804)
\curveto(625.71373226,327.69529113)(625.67684006,327.760395)(625.65079852,327.83851965)
\curveto(625.62909723,327.92098455)(625.61824658,328.029491)(625.61824658,328.164039)
\curveto(625.61824658,328.25952468)(625.62258684,328.33764932)(625.63126736,328.39841293)
\curveto(625.63994787,328.4635168)(625.65296865,328.5155999)(625.67032968,328.55466222)
\curveto(625.68769071,328.59372454)(625.709392,328.61976609)(625.73543355,328.63278687)
\curveto(625.7614751,328.6501479)(625.7918569,328.65882841)(625.82657897,328.65882841)
\curveto(625.88734258,328.65882841)(625.97197761,328.62193622)(626.08048406,328.54815184)
\curveto(626.19333077,328.47436745)(626.33655928,328.39407268)(626.5101696,328.30726752)
\curveto(626.68377993,328.22046235)(626.89211231,328.13799745)(627.13516676,328.05987281)
\curveto(627.38256147,327.98608842)(627.66684837,327.94919623)(627.98802746,327.94919623)
\curveto(628.23108191,327.94919623)(628.45243507,327.98174816)(628.65208694,328.04685203)
\curveto(628.85607906,328.1119559)(629.02968939,328.20310132)(629.1729179,328.32028829)
\curveto(629.32048667,328.44181551)(629.43333338,328.58938429)(629.51145803,328.76299461)
\curveto(629.58958267,328.93660493)(629.62864499,329.13408667)(629.62864499,329.35543983)
\curveto(629.62864499,329.59415402)(629.57439177,329.79814614)(629.46588532,329.96741621)
\curveto(629.35737887,330.13668627)(629.21415035,330.28425504)(629.03619977,330.41012253)
\curveto(628.85824919,330.54033027)(628.65425707,330.65751723)(628.42422339,330.76168343)
\curveto(628.19852997,330.87018988)(627.96632617,330.97869633)(627.72761198,331.08720278)
\curveto(627.48889779,331.20004949)(627.25669398,331.32374684)(627.03100057,331.45829484)
\curveto(626.80530715,331.59284284)(626.60348515,331.75126225)(626.42553457,331.93355309)
\curveto(626.24758399,332.11584393)(626.10218535,332.32851657)(625.98933864,332.57157102)
\curveto(625.88083219,332.81896573)(625.82657897,333.11410327)(625.82657897,333.45698366)
\curveto(625.82657897,333.80854456)(625.88951271,334.12104313)(626.01538019,334.39447939)
\curveto(626.14558793,334.6722559)(626.32353851,334.90445971)(626.54923193,335.0910908)
\curveto(626.7792656,335.27772189)(627.05053173,335.41878028)(627.36303031,335.51426596)
\curveto(627.67986914,335.61409189)(628.0205794,335.66400486)(628.38516107,335.66400486)
\curveto(628.57179216,335.66400486)(628.75842326,335.64664383)(628.94505435,335.61192176)
\curveto(629.13602571,335.58153996)(629.31397629,335.53813738)(629.47890609,335.48171402)
\curveto(629.64817615,335.42963093)(629.79791505,335.36886731)(629.9281228,335.29942319)
\curveto(630.05833054,335.23431931)(630.14296557,335.18006609)(630.18202789,335.13666351)
\curveto(630.22543047,335.09760119)(630.25364215,335.06504925)(630.26666292,335.0390077)
\curveto(630.27968369,335.01730641)(630.29053434,334.98692461)(630.29921486,334.94786229)
\curveto(630.30789537,334.91314022)(630.31440576,334.86973764)(630.31874602,334.81765454)
\curveto(630.32308628,334.76557145)(630.3252564,334.69829745)(630.3252564,334.61583255)
\curveto(630.3252564,334.5377079)(630.32091615,334.46826377)(630.31223563,334.40750016)
\curveto(630.30789537,334.34673655)(630.29921486,334.29465345)(630.28619408,334.25125087)
\curveto(630.27317331,334.21218855)(630.25364215,334.18180674)(630.2276006,334.16010545)
\curveto(630.20589931,334.14274442)(630.17985776,334.13406391)(630.14947595,334.13406391)
\curveto(630.10173312,334.13406391)(630.0257786,334.16444571)(629.92161241,334.22520933)
\curveto(629.82178647,334.28597294)(629.69808912,334.35324694)(629.55052035,334.42703132)
\curveto(629.40295157,334.50515597)(629.22717113,334.5746001)(629.023179,334.63536371)
\curveto(628.82352713,334.70046758)(628.59783371,334.73301951)(628.34609875,334.73301951)
\curveto(628.11172481,334.73301951)(627.90773269,334.70046758)(627.73412237,334.63536371)
\curveto(627.56051205,334.5746001)(627.41728353,334.49213519)(627.30443682,334.387969)
\curveto(627.19159011,334.28380281)(627.10695508,334.16010545)(627.05053173,334.01687694)
\curveto(626.99410837,333.87364843)(626.9658967,333.72173939)(626.9658967,333.56114985)
\curveto(626.9658967,333.32677592)(627.02014992,333.12495392)(627.12865637,332.95568385)
\curveto(627.23716282,332.78641379)(627.38039134,332.63667489)(627.55834192,332.50646715)
\curveto(627.74063275,332.37625941)(627.94679501,332.25690231)(628.17682868,332.14839586)
\curveto(628.40686236,332.03988941)(628.64123629,331.92921283)(628.87995048,331.81636612)
\curveto(629.11866467,331.70785967)(629.35303861,331.58633245)(629.58307228,331.45178445)
\curveto(629.81310596,331.32157671)(630.01709808,331.16532742)(630.19504866,330.98303658)
\curveto(630.3773395,330.80508601)(630.52273814,330.59241336)(630.63124459,330.34501866)
\curveto(630.7440913,330.10196421)(630.80051466,329.81333705)(630.80051466,329.47913718)
\closepath
}
}
{
\newrgbcolor{curcolor}{0.94509804 0.99215686 0.99607843}
\pscustom[linestyle=none,fillstyle=solid,fillcolor=curcolor]
{
\newpath
\moveto(633.51556926,327.28513675)
\curveto(633.51556926,327.25041469)(633.50688874,327.22003288)(633.48952771,327.19399133)
\curveto(633.47216668,327.16794978)(633.44178487,327.14624849)(633.39838229,327.12888746)
\curveto(633.35497971,327.11586669)(633.29638623,327.10501604)(633.22260184,327.09633553)
\curveto(633.15315771,327.08331475)(633.06418242,327.07680437)(632.95567597,327.07680437)
\curveto(632.85150978,327.07680437)(632.76253449,327.08331475)(632.6887501,327.09633553)
\curveto(632.61496572,327.10501604)(632.55637223,327.11586669)(632.51296965,327.12888746)
\curveto(632.46956707,327.14624849)(632.43918527,327.16794978)(632.42182424,327.19399133)
\curveto(632.4044632,327.22003288)(632.39578269,327.25041469)(632.39578269,327.28513675)
\lineto(632.39578269,335.37103744)
\curveto(632.39578269,335.40575951)(632.4044632,335.43614131)(632.42182424,335.46218286)
\curveto(632.44352553,335.48822441)(632.47607746,335.50775557)(632.51948004,335.52077634)
\curveto(632.56722288,335.53813738)(632.62581636,335.55115815)(632.69526049,335.55983867)
\curveto(632.76904488,335.57285944)(632.85585004,335.57936983)(632.95567597,335.57936983)
\curveto(633.06418242,335.57936983)(633.15315771,335.57285944)(633.22260184,335.55983867)
\curveto(633.29638623,335.55115815)(633.35497971,335.53813738)(633.39838229,335.52077634)
\curveto(633.44178487,335.50775557)(633.47216668,335.48822441)(633.48952771,335.46218286)
\curveto(633.50688874,335.43614131)(633.51556926,335.40575951)(633.51556926,335.37103744)
\closepath
}
}
{
\newrgbcolor{curcolor}{0.94509804 0.99215686 0.99607843}
\pscustom[linestyle=none,fillstyle=solid,fillcolor=curcolor]
{
\newpath
\moveto(640.58617115,327.58461455)
\curveto(640.58617115,327.50214965)(640.58183089,327.4305354)(640.57315038,327.36977178)
\curveto(640.56446986,327.31334843)(640.54927896,327.26560559)(640.52757767,327.22654327)
\curveto(640.51021663,327.18748095)(640.48634522,327.15926927)(640.45596341,327.14190824)
\curveto(640.42992186,327.1245472)(640.39736993,327.11586669)(640.3583076,327.11586669)
\lineto(635.38437191,327.11586669)
\curveto(635.2715252,327.11586669)(635.18037979,327.15058875)(635.11093566,327.22003288)
\curveto(635.04149153,327.28947701)(635.00676947,327.40015359)(635.00676947,327.55206262)
\lineto(635.00676947,327.79294694)
\curveto(635.00676947,327.84937029)(635.00893959,327.90145339)(635.01327985,327.94919623)
\curveto(635.02196037,327.99693907)(635.03715127,328.04685203)(635.05885256,328.09893513)
\curveto(635.08055385,328.15535848)(635.11093566,328.2161221)(635.14999798,328.28122597)
\curveto(635.1890603,328.3506701)(635.23680314,328.43096487)(635.29322649,328.52211029)
\lineto(639.17992755,334.60281177)
\lineto(635.3387992,334.60281177)
\curveto(635.29539662,334.60281177)(635.25850443,334.61149229)(635.22812262,334.62885332)
\curveto(635.19774082,334.64621435)(635.17169927,334.6722559)(635.14999798,334.70697797)
\curveto(635.13263695,334.74604029)(635.11961617,334.79378313)(635.11093566,334.85020648)
\curveto(635.10225514,334.91097009)(635.09791488,334.98041422)(635.09791488,335.05853886)
\curveto(635.09791488,335.14534403)(635.10225514,335.21912841)(635.11093566,335.27989202)
\curveto(635.11961617,335.34065564)(635.13263695,335.3905686)(635.14999798,335.42963093)
\curveto(635.17169927,335.46869325)(635.19774082,335.49690492)(635.22812262,335.51426596)
\curveto(635.25850443,335.53162699)(635.29539662,335.54030751)(635.3387992,335.54030751)
\lineto(640.02627787,335.54030751)
\curveto(640.14346483,335.54030751)(640.23461025,335.50558544)(640.29971412,335.43614131)
\curveto(640.36915825,335.37103744)(640.40388031,335.26904138)(640.40388031,335.13015312)
\lineto(640.40388031,334.87624803)
\curveto(640.40388031,334.8068039)(640.39954006,334.74387016)(640.39085954,334.6874468)
\curveto(640.38217902,334.63536371)(640.36698812,334.58111048)(640.34528683,334.52468713)
\curveto(640.32358554,334.46826377)(640.29320373,334.40750016)(640.25414141,334.34239629)
\curveto(640.21941935,334.27729242)(640.17167651,334.20133791)(640.1109129,334.11453275)
\lineto(636.243743,328.05987281)
\lineto(640.3583076,328.05987281)
\curveto(640.43209199,328.05987281)(640.48851534,328.02298061)(640.52757767,327.94919623)
\curveto(640.56663999,327.87541184)(640.58617115,327.75388462)(640.58617115,327.58461455)
\closepath
}
}
{
\newrgbcolor{curcolor}{0.94509804 0.99215686 0.99607843}
\pscustom[linestyle=none,fillstyle=solid,fillcolor=curcolor]
{
\newpath
\moveto(646.66990804,327.57810417)
\curveto(646.66990804,327.49997952)(646.66556778,327.4305354)(646.65688727,327.36977178)
\curveto(646.64820675,327.31334843)(646.63301585,327.26560559)(646.61131456,327.22654327)
\curveto(646.58961327,327.18748095)(646.56357172,327.15926927)(646.53318992,327.14190824)
\curveto(646.50714837,327.1245472)(646.47676656,327.11586669)(646.4420445,327.11586669)
\lineto(642.39909415,327.11586669)
\curveto(642.29926822,327.11586669)(642.20378254,327.14841862)(642.11263712,327.21352249)
\curveto(642.02583196,327.28296662)(641.98242938,327.40232372)(641.98242938,327.57159378)
\lineto(641.98242938,335.08458041)
\curveto(641.98242938,335.25385048)(642.02583196,335.37103744)(642.11263712,335.43614131)
\curveto(642.20378254,335.50558544)(642.29926822,335.54030751)(642.39909415,335.54030751)
\lineto(646.39647179,335.54030751)
\curveto(646.43119385,335.54030751)(646.46157566,335.53162699)(646.48761721,335.51426596)
\curveto(646.51799901,335.49690492)(646.54187043,335.46869325)(646.55923146,335.42963093)
\curveto(646.5765925,335.3905686)(646.58961327,335.34065564)(646.59829379,335.27989202)
\curveto(646.61131456,335.22346867)(646.61782495,335.15185441)(646.61782495,335.06504925)
\curveto(646.61782495,334.98692461)(646.61131456,334.91748048)(646.59829379,334.85671687)
\curveto(646.58961327,334.80029351)(646.5765925,334.75255067)(646.55923146,334.71348835)
\curveto(646.54187043,334.67876629)(646.51799901,334.65272474)(646.48761721,334.63536371)
\curveto(646.46157566,334.61800268)(646.43119385,334.60932216)(646.39647179,334.60932216)
\lineto(643.10221595,334.60932216)
\lineto(643.10221595,331.96610503)
\lineto(645.92772392,331.96610503)
\curveto(645.96244599,331.96610503)(645.99282779,331.95525438)(646.01886934,331.93355309)
\curveto(646.04925115,331.91619206)(646.07312257,331.89015051)(646.0904836,331.85542845)
\curveto(646.11218489,331.82070638)(646.12737579,331.77296354)(646.13605631,331.71219993)
\curveto(646.14473682,331.65143632)(646.14907708,331.57982206)(646.14907708,331.49735716)
\curveto(646.14907708,331.41923252)(646.14473682,331.35195852)(646.13605631,331.29553516)
\curveto(646.12737579,331.23911181)(646.11218489,331.1935391)(646.0904836,331.15881703)
\curveto(646.07312257,331.12409497)(646.04925115,331.09805342)(646.01886934,331.08069239)
\curveto(645.99282779,331.06767162)(645.96244599,331.06116123)(645.92772392,331.06116123)
\lineto(643.10221595,331.06116123)
\lineto(643.10221595,328.04685203)
\lineto(646.4420445,328.04685203)
\curveto(646.47676656,328.04685203)(646.50714837,328.03817152)(646.53318992,328.02081049)
\curveto(646.56357172,328.00344945)(646.58961327,327.97523778)(646.61131456,327.93617545)
\curveto(646.63301585,327.90145339)(646.64820675,327.85371055)(646.65688727,327.79294694)
\curveto(646.66556778,327.73652359)(646.66990804,327.66490933)(646.66990804,327.57810417)
\closepath
}
}
{
\newrgbcolor{curcolor}{0 0 0}
\pscustom[linestyle=none,fillstyle=solid,fillcolor=curcolor,opacity=0]
{
\newpath
\moveto(376.66154747,521.91632838)
\lineto(465.25979325,521.91632838)
\lineto(465.25979325,485.12899606)
\lineto(376.66154747,485.12899606)
\closepath
}
}
{
\newrgbcolor{curcolor}{1 1 1}
\pscustom[linestyle=none,fillstyle=solid,fillcolor=curcolor]
{
\newpath
\moveto(393.39589351,501.34543637)
\curveto(393.39589351,500.77252087)(393.32037283,500.2438761)(393.16933147,499.75950208)
\curveto(393.01829011,499.28033639)(392.79172807,498.86627473)(392.48964535,498.5173171)
\curveto(392.19277095,498.16835947)(391.81777171,497.89492253)(391.36464763,497.69700626)
\curveto(390.91673187,497.50429832)(390.39589959,497.40794435)(389.80215079,497.40794435)
\curveto(389.22402696,497.40794435)(388.71881965,497.49388167)(388.28652886,497.66575632)
\curveto(387.85944639,497.83763098)(387.50267628,498.08763047)(387.21621853,498.41575481)
\curveto(386.92976077,498.74387914)(386.71621954,499.14231583)(386.57559482,499.61106489)
\curveto(386.43497011,500.07981394)(386.36465775,500.61106286)(386.36465775,501.20481166)
\curveto(386.36465775,501.77772716)(386.43757427,502.30376777)(386.58340731,502.78293346)
\curveto(386.73444867,503.26730748)(386.95840655,503.6839733)(387.25528095,504.03293093)
\curveto(387.55736367,504.38188856)(387.93236291,504.65272134)(388.38027867,504.84542929)
\curveto(388.82819443,505.03813723)(389.35163087,505.1344912)(389.95058799,505.1344912)
\curveto(390.52871182,505.1344912)(391.03131497,505.04855387)(391.45839744,504.87667922)
\curveto(391.89068823,504.70480457)(392.2500625,504.45480508)(392.53652025,504.12668074)
\curveto(392.82297801,503.79855641)(393.03651924,503.40011971)(393.17714396,502.93137066)
\curveto(393.32297699,502.46262161)(393.39589351,501.93397685)(393.39589351,501.34543637)
\closepath
\moveto(392.03652127,501.25949905)
\curveto(392.03652127,501.63970661)(392.00006301,501.99908088)(391.92714649,502.33762186)
\curveto(391.85943829,502.67616285)(391.74485519,502.97303724)(391.58339718,503.22824506)
\curveto(391.42193918,503.48345288)(391.20318962,503.6839733)(390.92714851,503.82980634)
\curveto(390.6511074,503.9808477)(390.3073581,504.05636838)(389.8959006,504.05636838)
\curveto(389.51569304,504.05636838)(389.1875687,503.98866019)(388.91152759,503.85324379)
\curveto(388.64069481,503.7178274)(388.41673693,503.52511946)(388.23965395,503.27511997)
\curveto(388.06257098,503.03032879)(387.92975875,502.73866272)(387.84121726,502.40012174)
\curveto(387.7578841,502.06158076)(387.71621751,501.69178984)(387.71621751,501.29074898)
\curveto(387.71621751,500.9053331)(387.75007161,500.54335466)(387.81777981,500.20481368)
\curveto(387.89069633,499.8662727)(388.00788359,499.5693983)(388.1693416,499.31419049)
\curveto(388.33600793,499.06419099)(388.55736164,498.86367057)(388.83340275,498.7126292)
\curveto(389.10944386,498.56679617)(389.45319316,498.49387965)(389.86465066,498.49387965)
\curveto(390.2396499,498.49387965)(390.56517008,498.56158784)(390.84121119,498.69700424)
\curveto(391.11725229,498.83242063)(391.34381434,499.02252441)(391.52089731,499.26731558)
\curveto(391.69798028,499.51210675)(391.82818835,499.80377283)(391.91152152,500.14231381)
\curveto(391.99485468,500.48085479)(392.03652127,500.85324987)(392.03652127,501.25949905)
\closepath
}
}
{
\newrgbcolor{curcolor}{1 1 1}
\pscustom[linestyle=none,fillstyle=solid,fillcolor=curcolor]
{
\newpath
\moveto(399.14038204,507.67354856)
\curveto(399.14038204,507.57459043)(399.13517372,507.49386142)(399.12475707,507.43136155)
\curveto(399.11434043,507.37407)(399.10131962,507.32719509)(399.08569465,507.29073683)
\curveto(399.07527801,507.25427857)(399.05965304,507.22823696)(399.03881975,507.21261199)
\curveto(399.01798646,507.20219535)(398.99194484,507.19698702)(398.9606949,507.19698702)
\curveto(398.92423664,507.19698702)(398.87736174,507.20740367)(398.82007019,507.22823696)
\curveto(398.76798696,507.25427857)(398.70288293,507.28032019)(398.62475808,507.3063618)
\curveto(398.54663324,507.33761174)(398.45288343,507.36365335)(398.34350865,507.38448664)
\curveto(398.2393422,507.41052826)(398.11694661,507.42354906)(397.9763219,507.42354906)
\curveto(397.78361395,507.42354906)(397.61955179,507.39229913)(397.48413539,507.32979925)
\curveto(397.348719,507.26729938)(397.23934422,507.16834125)(397.15601106,507.03292486)
\curveto(397.07267789,506.90271679)(397.01278218,506.73084213)(396.97632392,506.5173009)
\curveto(396.93986566,506.30896799)(396.92163653,506.05376017)(396.92163653,505.75167745)
\lineto(396.92163653,504.986054)
\lineto(398.49194585,504.986054)
\curveto(398.53361244,504.986054)(398.56746653,504.97563736)(398.59350815,504.95480406)
\curveto(398.62475808,504.9391791)(398.6507997,504.90792916)(398.67163299,504.86105425)
\curveto(398.6976746,504.81938767)(398.71590373,504.76470028)(398.72632038,504.69699209)
\curveto(398.74194535,504.62928389)(398.74975783,504.54595073)(398.74975783,504.44699259)
\curveto(398.74975783,504.25949297)(398.72632038,504.12407658)(398.67944547,504.04074342)
\curveto(398.63257057,503.95741025)(398.5700707,503.91574367)(398.49194585,503.91574367)
\lineto(396.92163653,503.91574367)
\lineto(396.92163653,497.75169365)
\curveto(396.92163653,497.71002707)(396.91121989,497.67356881)(396.8903866,497.64231887)
\curveto(396.86955331,497.61627726)(396.83309505,497.59283981)(396.78101182,497.57200651)
\curveto(396.73413691,497.55117322)(396.66903288,497.53554825)(396.58569971,497.52513161)
\curveto(396.50236655,497.51471496)(396.39820009,497.50950664)(396.27320035,497.50950664)
\curveto(396.1482006,497.50950664)(396.04403414,497.51471496)(395.96070098,497.52513161)
\curveto(395.87736781,497.53554825)(395.80965962,497.55117322)(395.75757639,497.57200651)
\curveto(395.71070149,497.59283981)(395.67684739,497.61627726)(395.6560141,497.64231887)
\curveto(395.63518081,497.67356881)(395.62476416,497.71002707)(395.62476416,497.75169365)
\lineto(395.62476416,503.91574367)
\lineto(394.63257867,503.91574367)
\curveto(394.5492455,503.91574367)(394.48674563,503.95741025)(394.44507905,504.04074342)
\curveto(394.40341247,504.12407658)(394.38257918,504.25949297)(394.38257918,504.44699259)
\curveto(394.38257918,504.54595073)(394.3877875,504.62928389)(394.39820414,504.69699209)
\curveto(394.40862079,504.76470028)(394.42424576,504.81938767)(394.44507905,504.86105425)
\curveto(394.46591234,504.90792916)(394.49195395,504.9391791)(394.52320389,504.95480406)
\curveto(394.55445383,504.97563736)(394.59091209,504.986054)(394.63257867,504.986054)
\lineto(395.62476416,504.986054)
\lineto(395.62476416,505.71261503)
\curveto(395.62476416,506.20219737)(395.6690349,506.62146736)(395.75757639,506.97042498)
\curveto(395.8513262,507.32459093)(395.99195092,507.61365285)(396.17945054,507.83761073)
\curveto(396.36695016,508.06156861)(396.60132468,508.22563077)(396.88257411,508.32979723)
\curveto(397.16903187,508.43917201)(397.50496869,508.4938594)(397.89038457,508.4938594)
\curveto(398.07267587,508.4938594)(398.24975884,508.47563027)(398.4216335,508.43917201)
\curveto(398.59350815,508.40792207)(398.72632038,508.37146381)(398.82007019,508.32979723)
\curveto(398.91382,508.29333897)(398.97631987,508.25948487)(399.00756981,508.22823494)
\curveto(399.03881975,508.196985)(399.06486136,508.15531842)(399.08569465,508.10323519)
\curveto(399.10652794,508.05636028)(399.11954875,507.99646457)(399.12475707,507.92354805)
\curveto(399.13517372,507.85583986)(399.14038204,507.77250669)(399.14038204,507.67354856)
\closepath
}
}
{
\newrgbcolor{curcolor}{1 1 1}
\pscustom[linestyle=none,fillstyle=solid,fillcolor=curcolor]
{
\newpath
\moveto(403.87637251,507.67354856)
\curveto(403.87637251,507.57459043)(403.87116419,507.49386142)(403.86074754,507.43136155)
\curveto(403.8503309,507.37407)(403.83731009,507.32719509)(403.82168512,507.29073683)
\curveto(403.81126848,507.25427857)(403.79564351,507.22823696)(403.77481022,507.21261199)
\curveto(403.75397693,507.20219535)(403.72793531,507.19698702)(403.69668538,507.19698702)
\curveto(403.66022712,507.19698702)(403.61335221,507.20740367)(403.55606066,507.22823696)
\curveto(403.50397743,507.25427857)(403.4388734,507.28032019)(403.36074856,507.3063618)
\curveto(403.28262371,507.33761174)(403.1888739,507.36365335)(403.07949912,507.38448664)
\curveto(402.97533267,507.41052826)(402.85293708,507.42354906)(402.71231237,507.42354906)
\curveto(402.51960443,507.42354906)(402.35554226,507.39229913)(402.22012587,507.32979925)
\curveto(402.08470947,507.26729938)(401.97533469,507.16834125)(401.89200153,507.03292486)
\curveto(401.80866836,506.90271679)(401.74877265,506.73084213)(401.71231439,506.5173009)
\curveto(401.67585613,506.30896799)(401.657627,506.05376017)(401.657627,505.75167745)
\lineto(401.657627,504.986054)
\lineto(403.22793632,504.986054)
\curveto(403.26960291,504.986054)(403.303457,504.97563736)(403.32949862,504.95480406)
\curveto(403.36074856,504.9391791)(403.38679017,504.90792916)(403.40762346,504.86105425)
\curveto(403.43366507,504.81938767)(403.4518942,504.76470028)(403.46231085,504.69699209)
\curveto(403.47793582,504.62928389)(403.4857483,504.54595073)(403.4857483,504.44699259)
\curveto(403.4857483,504.25949297)(403.46231085,504.12407658)(403.41543594,504.04074342)
\curveto(403.36856104,503.95741025)(403.30606117,503.91574367)(403.22793632,503.91574367)
\lineto(401.657627,503.91574367)
\lineto(401.657627,497.75169365)
\curveto(401.657627,497.71002707)(401.64721036,497.67356881)(401.62637707,497.64231887)
\curveto(401.60554378,497.61627726)(401.56908552,497.59283981)(401.51700229,497.57200651)
\curveto(401.47012738,497.55117322)(401.40502335,497.53554825)(401.32169018,497.52513161)
\curveto(401.23835702,497.51471496)(401.13419056,497.50950664)(401.00919082,497.50950664)
\curveto(400.88419107,497.50950664)(400.78002461,497.51471496)(400.69669145,497.52513161)
\curveto(400.61335829,497.53554825)(400.54565009,497.55117322)(400.49356686,497.57200651)
\curveto(400.44669196,497.59283981)(400.41283786,497.61627726)(400.39200457,497.64231887)
\curveto(400.37117128,497.67356881)(400.36075463,497.71002707)(400.36075463,497.75169365)
\lineto(400.36075463,503.91574367)
\lineto(399.36856914,503.91574367)
\curveto(399.28523597,503.91574367)(399.2227361,503.95741025)(399.18106952,504.04074342)
\curveto(399.13940294,504.12407658)(399.11856965,504.25949297)(399.11856965,504.44699259)
\curveto(399.11856965,504.54595073)(399.12377797,504.62928389)(399.13419461,504.69699209)
\curveto(399.14461126,504.76470028)(399.16023623,504.81938767)(399.18106952,504.86105425)
\curveto(399.20190281,504.90792916)(399.22794442,504.9391791)(399.25919436,504.95480406)
\curveto(399.2904443,504.97563736)(399.32690256,504.986054)(399.36856914,504.986054)
\lineto(400.36075463,504.986054)
\lineto(400.36075463,505.71261503)
\curveto(400.36075463,506.20219737)(400.40502537,506.62146736)(400.49356686,506.97042498)
\curveto(400.58731667,507.32459093)(400.72794139,507.61365285)(400.91544101,507.83761073)
\curveto(401.10294063,508.06156861)(401.33731515,508.22563077)(401.61856458,508.32979723)
\curveto(401.90502234,508.43917201)(402.24095916,508.4938594)(402.62637504,508.4938594)
\curveto(402.80866634,508.4938594)(402.98574931,508.47563027)(403.15762397,508.43917201)
\curveto(403.32949862,508.40792207)(403.46231085,508.37146381)(403.55606066,508.32979723)
\curveto(403.64981047,508.29333897)(403.71231034,508.25948487)(403.74356028,508.22823494)
\curveto(403.77481022,508.196985)(403.80085183,508.15531842)(403.82168512,508.10323519)
\curveto(403.84251841,508.05636028)(403.85553922,507.99646457)(403.86074754,507.92354805)
\curveto(403.87116419,507.85583986)(403.87637251,507.77250669)(403.87637251,507.67354856)
\closepath
}
}
{
\newrgbcolor{curcolor}{1 1 1}
\pscustom[linestyle=none,fillstyle=solid,fillcolor=curcolor]
{
\newpath
\moveto(409.11086184,499.66575227)
\curveto(409.11086184,499.306378)(409.04315365,498.98606615)(408.90773725,498.70481672)
\curveto(408.77752918,498.42356729)(408.59002956,498.1865886)(408.34523839,497.99388066)
\curveto(408.10044722,497.80117272)(407.80878115,497.65533968)(407.47024016,497.55638155)
\curveto(407.13169918,497.45742341)(406.7593041,497.40794435)(406.35305493,497.40794435)
\curveto(406.10305543,497.40794435)(405.86347259,497.42877764)(405.63430638,497.47044422)
\curveto(405.4103485,497.50690248)(405.20722391,497.55377738)(405.02493262,497.61106894)
\curveto(404.84784964,497.67356881)(404.69680828,497.73606868)(404.57180853,497.79856856)
\curveto(404.44680879,497.86627675)(404.35566314,497.92617246)(404.29837159,497.97825569)
\curveto(404.24108004,498.03033892)(404.19941346,498.10325544)(404.17337184,498.19700525)
\curveto(404.14733023,498.29075506)(404.13430942,498.41835897)(404.13430942,498.57981697)
\curveto(404.13430942,498.67877511)(404.13951774,498.76210827)(404.14993439,498.82981647)
\curveto(404.16035103,498.89752466)(404.17337184,498.95221205)(404.18899681,498.99387864)
\curveto(404.20462178,499.03554522)(404.22545507,499.06419099)(404.25149668,499.07981596)
\curveto(404.28274662,499.10064925)(404.31660072,499.1110659)(404.35305898,499.1110659)
\curveto(404.41035053,499.1110659)(404.49368369,499.07460764)(404.60305847,499.00169112)
\curveto(404.71764157,498.93398292)(404.85566213,498.85846224)(405.01712013,498.77512908)
\curveto(405.18378646,498.69179591)(405.37909857,498.61367107)(405.60305645,498.54075455)
\curveto(405.82701433,498.47304636)(406.0848263,498.43919226)(406.37649238,498.43919226)
\curveto(406.59524194,498.43919226)(406.7931582,498.46262971)(406.97024118,498.50950462)
\curveto(407.14732415,498.55637952)(407.30096967,498.62408772)(407.43117774,498.7126292)
\curveto(407.56138581,498.80637901)(407.66034395,498.92356628)(407.72805214,499.06419099)
\curveto(407.80096866,499.20481571)(407.83742692,499.37148204)(407.83742692,499.56418998)
\curveto(407.83742692,499.76210625)(407.78534369,499.92877258)(407.68117724,500.06418897)
\curveto(407.5822191,500.19960536)(407.44940687,500.31939678)(407.28274054,500.42356324)
\curveto(407.11607422,500.5277297)(406.9285746,500.61887534)(406.72024168,500.69700019)
\curveto(406.51190877,500.78033335)(406.29576338,500.86627068)(406.0718055,500.95481216)
\curveto(405.85305594,501.04335365)(405.63691054,501.14231178)(405.42336931,501.25168656)
\curveto(405.2150364,501.36626966)(405.02753678,501.50429022)(404.86087045,501.66574822)
\curveto(404.69420412,501.82720623)(404.55878773,502.01991417)(404.45462127,502.24387205)
\curveto(404.35566314,502.46782993)(404.30618407,502.73605856)(404.30618407,503.04855792)
\curveto(404.30618407,503.32459903)(404.3582673,503.58761933)(404.46243376,503.83761883)
\curveto(404.57180853,504.09282664)(404.73326654,504.31418036)(404.94680777,504.50167998)
\curveto(405.16034901,504.69438792)(405.42597347,504.84803345)(405.74368116,504.96261655)
\curveto(406.06659717,505.07719965)(406.44159641,505.1344912)(406.86867888,505.1344912)
\curveto(407.0561785,505.1344912)(407.24367812,505.11886623)(407.43117774,505.0876163)
\curveto(407.61867736,505.05636636)(407.78794785,505.01730394)(407.93898922,504.97042903)
\curveto(408.09003058,504.92355413)(408.21763448,504.8714709)(408.32180094,504.81417935)
\curveto(408.43117572,504.76209612)(408.51190472,504.71522122)(408.56398795,504.67355463)
\curveto(408.6212795,504.63188805)(408.65773776,504.59542979)(408.67336273,504.56417986)
\curveto(408.69419602,504.53292992)(408.70721683,504.49647166)(408.71242515,504.45480508)
\curveto(408.7228418,504.41834682)(408.73065428,504.37147191)(408.7358626,504.31418036)
\curveto(408.74627925,504.25688881)(408.75148757,504.18657645)(408.75148757,504.10324329)
\curveto(408.75148757,504.0147018)(408.74627925,503.93657696)(408.7358626,503.86886876)
\curveto(408.73065428,503.80636889)(408.71763347,503.75428566)(408.69680018,503.71261908)
\curveto(408.68117521,503.6709525)(408.66034192,503.63970256)(408.63430031,503.61886927)
\curveto(408.60825869,503.6032443)(408.57961292,503.59543182)(408.54836298,503.59543182)
\curveto(408.50148808,503.59543182)(408.43377988,503.62407759)(408.34523839,503.68136914)
\curveto(408.25669691,503.73866069)(408.1421138,503.79855641)(408.00148909,503.86105628)
\curveto(407.86086437,503.92876448)(407.69419804,503.99126435)(407.5014901,504.0485559)
\curveto(407.31399048,504.10584745)(407.09784509,504.13449323)(406.85305391,504.13449323)
\curveto(406.63430436,504.13449323)(406.44159641,504.10845161)(406.27493009,504.05636838)
\curveto(406.10826376,504.00949348)(405.9702432,503.93918112)(405.86086842,503.84543131)
\curveto(405.75670197,503.75688982)(405.67597297,503.65011921)(405.61868141,503.52511946)
\curveto(405.56659819,503.40011971)(405.54055657,503.26470332)(405.54055657,503.11887028)
\curveto(405.54055657,502.91574569)(405.5926398,502.74387104)(405.69680626,502.60324633)
\curveto(405.80097271,502.46782993)(405.9363891,502.34803851)(406.10305543,502.24387205)
\curveto(406.26972176,502.1397056)(406.45982554,502.04595579)(406.67336678,501.96262262)
\curveto(406.88690801,501.87928946)(407.10305341,501.79335213)(407.32180297,501.70481065)
\curveto(407.54576085,501.61626916)(407.7645104,501.51731103)(407.97805164,501.40793625)
\curveto(408.19680119,501.29856147)(408.38950914,501.16574924)(408.55617547,501.00949955)
\curveto(408.7228418,500.85324987)(408.85565403,500.66575025)(408.95461216,500.44700069)
\curveto(409.05877861,500.22825114)(409.11086184,499.967835)(409.11086184,499.66575227)
\closepath
}
}
{
\newrgbcolor{curcolor}{1 1 1}
\pscustom[linestyle=none,fillstyle=solid,fillcolor=curcolor]
{
\newpath
\moveto(416.95278398,501.57199841)
\curveto(416.95278398,501.36887383)(416.90070075,501.22304079)(416.7965343,501.1344993)
\curveto(416.69757617,501.05116614)(416.58299306,501.00949955)(416.45278499,501.00949955)
\lineto(411.84341933,501.00949955)
\curveto(411.84341933,500.61887534)(411.88248175,500.26731356)(411.96060659,499.95481419)
\curveto(412.03873143,499.64231482)(412.1689395,499.3740862)(412.3512308,499.15012832)
\curveto(412.5335221,498.92617044)(412.77050078,498.75429579)(413.06216686,498.63450436)
\curveto(413.35383294,498.51471294)(413.71060305,498.45481723)(414.13247719,498.45481723)
\curveto(414.46580985,498.45481723)(414.76268425,498.48085884)(415.02310039,498.53294207)
\curveto(415.28351653,498.59023362)(415.50747441,498.65273349)(415.69497403,498.72044169)
\curveto(415.88768197,498.78814988)(416.04393166,498.8480456)(416.16372308,498.90012882)
\curveto(416.28872283,498.95742038)(416.38247264,498.98606615)(416.44497251,498.98606615)
\curveto(416.48143077,498.98606615)(416.51268071,498.97564951)(416.53872232,498.95481621)
\curveto(416.56997226,498.93919125)(416.59340971,498.91314963)(416.60903468,498.87669137)
\curveto(416.62465965,498.84023311)(416.63507629,498.78814988)(416.64028461,498.72044169)
\curveto(416.65070126,498.65794182)(416.65590958,498.57981697)(416.65590958,498.48606716)
\curveto(416.65590958,498.41835897)(416.65330542,498.35846326)(416.6480971,498.30638003)
\curveto(416.64288878,498.25950512)(416.63507629,498.21523438)(416.62465965,498.1735678)
\curveto(416.61945132,498.13710954)(416.60643052,498.10325544)(416.58559723,498.0720055)
\curveto(416.56997226,498.04075557)(416.5465348,498.00950563)(416.51528487,497.97825569)
\curveto(416.48924325,497.95221408)(416.40591009,497.90533917)(416.26528537,497.83763098)
\curveto(416.12466066,497.7751311)(415.94236936,497.71263123)(415.71841148,497.65013136)
\curveto(415.4944536,497.58763148)(415.23403746,497.53294409)(414.93716306,497.48606919)
\curveto(414.64549699,497.43398596)(414.33299762,497.40794435)(413.99966496,497.40794435)
\curveto(413.42154113,497.40794435)(412.91372966,497.48867335)(412.47623055,497.65013136)
\curveto(412.04393976,497.81158936)(411.67935716,498.05117221)(411.38248276,498.3688799)
\curveto(411.08560836,498.68658759)(410.86165048,499.08502428)(410.71060912,499.56418998)
\curveto(410.55956776,500.04335568)(410.48404708,500.60064621)(410.48404708,501.23606159)
\curveto(410.48404708,501.84022704)(410.56217192,502.38189261)(410.71842161,502.8610583)
\curveto(410.87467129,503.34543232)(411.09862917,503.75428566)(411.39029525,504.08761832)
\curveto(411.68716964,504.4261593)(412.04393976,504.68397128)(412.46060558,504.86105425)
\curveto(412.8772714,505.04334555)(413.34341629,505.1344912)(413.85904025,505.1344912)
\curveto(414.41112246,505.1344912)(414.87987151,505.04594971)(415.2652874,504.86886674)
\curveto(415.65591161,504.69178376)(415.97622346,504.45220092)(416.22622295,504.15011819)
\curveto(416.47622245,503.85324379)(416.65851374,503.50168201)(416.77309685,503.09543283)
\curveto(416.89288827,502.69439197)(416.95278398,502.26470534)(416.95278398,501.80637294)
\closepath
\moveto(415.65591161,501.95481014)
\curveto(415.67153658,502.6318921)(415.52049522,503.16314103)(415.20278753,503.54855691)
\curveto(414.89028816,503.9339728)(414.42414327,504.12668074)(413.80435286,504.12668074)
\curveto(413.48664517,504.12668074)(413.2079999,504.06678503)(412.96841705,503.9469936)
\curveto(412.7288342,503.82720218)(412.52831377,503.66834834)(412.36685577,503.47043207)
\curveto(412.20539776,503.2725158)(412.08039801,503.04074544)(411.99185653,502.77512098)
\curveto(411.90331504,502.51470484)(411.85383597,502.24126789)(411.84341933,501.95481014)
\closepath
}
}
{
\newrgbcolor{curcolor}{1 1 1}
\pscustom[linestyle=none,fillstyle=solid,fillcolor=curcolor]
{
\newpath
\moveto(422.48102253,498.26731761)
\curveto(422.48102253,498.11627625)(422.47060588,497.99648482)(422.44977259,497.90794333)
\curveto(422.4289393,497.81940185)(422.39768936,497.75429781)(422.35602278,497.71263123)
\curveto(422.3143562,497.67096465)(422.25185632,497.63190223)(422.16852316,497.59544397)
\curveto(422.08519,497.55898571)(421.98883602,497.53033993)(421.87946125,497.50950664)
\curveto(421.77529479,497.48346503)(421.66331585,497.46263174)(421.54352443,497.44700677)
\curveto(421.423733,497.4313818)(421.30394158,497.42356931)(421.18415015,497.42356931)
\curveto(420.81956756,497.42356931)(420.50706819,497.47044422)(420.24665205,497.56419403)
\curveto(419.98623591,497.66315216)(419.77269468,497.8089852)(419.60602835,498.00169314)
\curveto(419.43936202,498.19960941)(419.31696643,498.44700474)(419.23884159,498.74387914)
\curveto(419.16592507,499.04596186)(419.12946681,499.40012781)(419.12946681,499.80637699)
\lineto(419.12946681,503.91574367)
\lineto(418.14509381,503.91574367)
\curveto(418.06696897,503.91574367)(418.00446909,503.95741025)(417.95759419,504.04074342)
\curveto(417.91071928,504.12407658)(417.88728183,504.25949297)(417.88728183,504.44699259)
\curveto(417.88728183,504.54595073)(417.89249015,504.62928389)(417.9029068,504.69699209)
\curveto(417.91853177,504.76470028)(417.9367609,504.81938767)(417.95759419,504.86105425)
\curveto(417.97842748,504.90792916)(418.00446909,504.9391791)(418.03571903,504.95480406)
\curveto(418.07217729,504.97563736)(418.11123971,504.986054)(418.15290629,504.986054)
\lineto(419.12946681,504.986054)
\lineto(419.12946681,506.65792562)
\curveto(419.12946681,506.69438387)(419.1372793,506.72823797)(419.15290427,506.75948791)
\curveto(419.17373756,506.79073785)(419.20759166,506.81677946)(419.25446656,506.83761275)
\curveto(419.30654979,506.86365437)(419.37425799,506.8818835)(419.45759115,506.89230014)
\curveto(419.54092431,506.90271679)(419.64509077,506.90792511)(419.77009052,506.90792511)
\curveto(419.90029859,506.90792511)(420.0070692,506.90271679)(420.09040237,506.89230014)
\curveto(420.17373553,506.8818835)(420.23883957,506.86365437)(420.28571447,506.83761275)
\curveto(420.33258938,506.81677946)(420.36644348,506.79073785)(420.38727677,506.75948791)
\curveto(420.40811006,506.72823797)(420.4185267,506.69438387)(420.4185267,506.65792562)
\lineto(420.4185267,504.986054)
\lineto(422.22321055,504.986054)
\curveto(422.26487713,504.986054)(422.30133539,504.97563736)(422.33258533,504.95480406)
\curveto(422.36383526,504.9391791)(422.38987688,504.90792916)(422.41071017,504.86105425)
\curveto(422.43675178,504.81938767)(422.45498091,504.76470028)(422.46539756,504.69699209)
\curveto(422.4758142,504.62928389)(422.48102253,504.54595073)(422.48102253,504.44699259)
\curveto(422.48102253,504.25949297)(422.45758507,504.12407658)(422.41071017,504.04074342)
\curveto(422.36383526,503.95741025)(422.30133539,503.91574367)(422.22321055,503.91574367)
\lineto(420.4185267,503.91574367)
\lineto(420.4185267,499.99387661)
\curveto(420.4185267,499.50950259)(420.48883906,499.14231583)(420.62946378,498.89231634)
\curveto(420.77529681,498.64752517)(421.03310879,498.52512958)(421.40289971,498.52512958)
\curveto(421.52269113,498.52512958)(421.62946175,498.53554623)(421.72321156,498.55637952)
\curveto(421.81696137,498.58242113)(421.90029454,498.60846275)(421.97321106,498.63450436)
\curveto(422.04612757,498.66054598)(422.10862745,498.68398343)(422.16071068,498.70481672)
\curveto(422.2127939,498.73085833)(422.25966881,498.74387914)(422.30133539,498.74387914)
\curveto(422.32737701,498.74387914)(422.35081446,498.73606666)(422.37164775,498.72044169)
\curveto(422.39768936,498.71002504)(422.41591849,498.68658759)(422.42633514,498.65012933)
\curveto(422.44196011,498.61367107)(422.45498091,498.56419201)(422.46539756,498.50169213)
\curveto(422.4758142,498.43919226)(422.48102253,498.36106742)(422.48102253,498.26731761)
\closepath
}
}
{
\newrgbcolor{curcolor}{1 1 1}
\pscustom[linestyle=none,fillstyle=solid,fillcolor=curcolor]
{
\newpath
\moveto(430.21563173,498.08763047)
\curveto(430.21563173,497.98867234)(430.20781924,497.90533917)(430.19219427,497.83763098)
\curveto(430.17656931,497.76992278)(430.15573602,497.71523539)(430.1296944,497.67356881)
\curveto(430.10365279,497.63190223)(430.07240285,497.60065229)(430.03594459,497.579819)
\curveto(430.00469465,497.56419403)(429.97084056,497.55638155)(429.9343823,497.55638155)
\lineto(424.65314299,497.55638155)
\curveto(424.61668473,497.55638155)(424.58283063,497.56419403)(424.5515807,497.579819)
\curveto(424.52033076,497.60065229)(424.48908082,497.63190223)(424.45783089,497.67356881)
\curveto(424.43178927,497.71523539)(424.41095598,497.76992278)(424.39533101,497.83763098)
\curveto(424.37970605,497.90533917)(424.37189356,497.98867234)(424.37189356,498.08763047)
\curveto(424.37189356,498.18138028)(424.37970605,498.26210928)(424.39533101,498.32981748)
\curveto(424.41095598,498.39752568)(424.42918511,498.45221307)(424.4500184,498.49387965)
\curveto(424.47606002,498.54075455)(424.50470579,498.57460865)(424.53595573,498.59544194)
\curveto(424.57241399,498.62148356)(424.61147641,498.63450436)(424.65314299,498.63450436)
\lineto(426.76251372,498.63450436)
\lineto(426.76251372,506.3219888)
\lineto(424.80939268,505.15792865)
\curveto(424.71043454,505.10584542)(424.62970554,505.07459549)(424.56720567,505.06417884)
\curveto(424.50991411,505.0537622)(424.46303921,505.06417884)(424.42658095,505.09542878)
\curveto(424.39012269,505.13188704)(424.36408108,505.18917859)(424.34845611,505.26730343)
\curveto(424.33803946,505.34542827)(424.33283114,505.44438641)(424.33283114,505.56417783)
\curveto(424.33283114,505.65271932)(424.3354353,505.72824)(424.34064362,505.79073987)
\curveto(424.35106027,505.85323974)(424.36408108,505.90532297)(424.37970605,505.94698955)
\curveto(424.39533101,505.98865614)(424.4161643,506.0251144)(424.44220592,506.05636433)
\curveto(424.47345586,506.08761427)(424.51251828,506.11886421)(424.55939318,506.15011414)
\lineto(426.88751347,507.64229862)
\curveto(426.90834676,507.65792359)(426.93438837,507.6709444)(426.96563831,507.68136104)
\curveto(426.99688825,507.69177769)(427.03595067,507.70219433)(427.08282557,507.71261098)
\curveto(427.12970048,507.72302763)(427.18438787,507.72823595)(427.24688774,507.72823595)
\curveto(427.30938761,507.73344427)(427.38751245,507.73604843)(427.48126226,507.73604843)
\curveto(427.60626201,507.73604843)(427.71042847,507.73084011)(427.79376163,507.72042346)
\curveto(427.8770948,507.71000682)(427.94219883,507.69438185)(427.98907374,507.67354856)
\curveto(428.03594864,507.65792359)(428.06719858,507.63448614)(428.08282355,507.6032362)
\curveto(428.09844851,507.57719459)(428.106261,507.54854881)(428.106261,507.51729888)
\lineto(428.106261,498.63450436)
\lineto(429.9343823,498.63450436)
\curveto(429.97604888,498.63450436)(430.0151113,498.62148356)(430.05156956,498.59544194)
\curveto(430.08802782,498.57460865)(430.11667359,498.54075455)(430.13750689,498.49387965)
\curveto(430.1635485,498.45221307)(430.18177763,498.39752568)(430.19219427,498.32981748)
\curveto(430.20781924,498.26210928)(430.21563173,498.18138028)(430.21563173,498.08763047)
\closepath
}
}
{
\newrgbcolor{curcolor}{0 0 0}
\pscustom[linestyle=none,fillstyle=solid,fillcolor=curcolor,opacity=0]
{
\newpath
\moveto(376.66154747,453.60554283)
\lineto(465.25979325,453.60554283)
\lineto(465.25979325,416.81821575)
\lineto(376.66154747,416.81821575)
\closepath
}
}
{
\newrgbcolor{curcolor}{1 1 1}
\pscustom[linestyle=none,fillstyle=solid,fillcolor=curcolor]
{
\newpath
\moveto(393.39589351,433.0346447)
\curveto(393.39589351,432.4617292)(393.32037283,431.93308443)(393.16933147,431.44871041)
\curveto(393.01829011,430.96954472)(392.79172807,430.55548306)(392.48964535,430.20652543)
\curveto(392.19277095,429.8575678)(391.81777171,429.58413086)(391.36464763,429.38621459)
\curveto(390.91673187,429.19350665)(390.39589959,429.09715268)(389.80215079,429.09715268)
\curveto(389.22402696,429.09715268)(388.71881965,429.18309)(388.28652886,429.35496465)
\curveto(387.85944639,429.52683931)(387.50267628,429.7768388)(387.21621853,430.10496314)
\curveto(386.92976077,430.43308747)(386.71621954,430.83152416)(386.57559482,431.30027321)
\curveto(386.43497011,431.76902227)(386.36465775,432.30027119)(386.36465775,432.89401999)
\curveto(386.36465775,433.46693549)(386.43757427,433.9929761)(386.58340731,434.47214179)
\curveto(386.73444867,434.95651581)(386.95840655,435.37318163)(387.25528095,435.72213926)
\curveto(387.55736367,436.07109689)(387.93236291,436.34192967)(388.38027867,436.53463762)
\curveto(388.82819443,436.72734556)(389.35163087,436.82369953)(389.95058799,436.82369953)
\curveto(390.52871182,436.82369953)(391.03131497,436.7377622)(391.45839744,436.56588755)
\curveto(391.89068823,436.3940129)(392.2500625,436.14401341)(392.53652025,435.81588907)
\curveto(392.82297801,435.48776474)(393.03651924,435.08932804)(393.17714396,434.62057899)
\curveto(393.32297699,434.15182994)(393.39589351,433.62318518)(393.39589351,433.0346447)
\closepath
\moveto(392.03652127,432.94870738)
\curveto(392.03652127,433.32891494)(392.00006301,433.68828921)(391.92714649,434.02683019)
\curveto(391.85943829,434.36537117)(391.74485519,434.66224557)(391.58339718,434.91745339)
\curveto(391.42193918,435.17266121)(391.20318962,435.37318163)(390.92714851,435.51901467)
\curveto(390.6511074,435.67005603)(390.3073581,435.74557671)(389.8959006,435.74557671)
\curveto(389.51569304,435.74557671)(389.1875687,435.67786852)(388.91152759,435.54245212)
\curveto(388.64069481,435.40703573)(388.41673693,435.21432779)(388.23965395,434.9643283)
\curveto(388.06257098,434.71953712)(387.92975875,434.42787105)(387.84121726,434.08933007)
\curveto(387.7578841,433.75078909)(387.71621751,433.38099817)(387.71621751,432.97995731)
\curveto(387.71621751,432.59454143)(387.75007161,432.23256299)(387.81777981,431.89402201)
\curveto(387.89069633,431.55548103)(388.00788359,431.25860663)(388.1693416,431.00339882)
\curveto(388.33600793,430.75339932)(388.55736164,430.5528789)(388.83340275,430.40183753)
\curveto(389.10944386,430.2560045)(389.45319316,430.18308798)(389.86465066,430.18308798)
\curveto(390.2396499,430.18308798)(390.56517008,430.25079617)(390.84121119,430.38621257)
\curveto(391.11725229,430.52162896)(391.34381434,430.71173274)(391.52089731,430.95652391)
\curveto(391.69798028,431.20131508)(391.82818835,431.49298116)(391.91152152,431.83152214)
\curveto(391.99485468,432.17006312)(392.03652127,432.5424582)(392.03652127,432.94870738)
\closepath
}
}
{
\newrgbcolor{curcolor}{1 1 1}
\pscustom[linestyle=none,fillstyle=solid,fillcolor=curcolor]
{
\newpath
\moveto(399.14038204,439.36275689)
\curveto(399.14038204,439.26379876)(399.13517372,439.18306975)(399.12475707,439.12056988)
\curveto(399.11434043,439.06327833)(399.10131962,439.01640342)(399.08569465,438.97994516)
\curveto(399.07527801,438.9434869)(399.05965304,438.91744529)(399.03881975,438.90182032)
\curveto(399.01798646,438.89140368)(398.99194484,438.88619535)(398.9606949,438.88619535)
\curveto(398.92423664,438.88619535)(398.87736174,438.896612)(398.82007019,438.91744529)
\curveto(398.76798696,438.9434869)(398.70288293,438.96952852)(398.62475808,438.99557013)
\curveto(398.54663324,439.02682007)(398.45288343,439.05286168)(398.34350865,439.07369497)
\curveto(398.2393422,439.09973659)(398.11694661,439.11275739)(397.9763219,439.11275739)
\curveto(397.78361395,439.11275739)(397.61955179,439.08150746)(397.48413539,439.01900758)
\curveto(397.348719,438.95650771)(397.23934422,438.85754958)(397.15601106,438.72213319)
\curveto(397.07267789,438.59192512)(397.01278218,438.42005046)(396.97632392,438.20650923)
\curveto(396.93986566,437.99817632)(396.92163653,437.7429685)(396.92163653,437.44088578)
\lineto(396.92163653,436.67526233)
\lineto(398.49194585,436.67526233)
\curveto(398.53361244,436.67526233)(398.56746653,436.66484568)(398.59350815,436.64401239)
\curveto(398.62475808,436.62838743)(398.6507997,436.59713749)(398.67163299,436.55026258)
\curveto(398.6976746,436.508596)(398.71590373,436.45390861)(398.72632038,436.38620042)
\curveto(398.74194535,436.31849222)(398.74975783,436.23515906)(398.74975783,436.13620092)
\curveto(398.74975783,435.9487013)(398.72632038,435.81328491)(398.67944547,435.72995174)
\curveto(398.63257057,435.64661858)(398.5700707,435.604952)(398.49194585,435.604952)
\lineto(396.92163653,435.604952)
\lineto(396.92163653,429.44090198)
\curveto(396.92163653,429.3992354)(396.91121989,429.36277714)(396.8903866,429.3315272)
\curveto(396.86955331,429.30548559)(396.83309505,429.28204814)(396.78101182,429.26121484)
\curveto(396.73413691,429.24038155)(396.66903288,429.22475658)(396.58569971,429.21433994)
\curveto(396.50236655,429.20392329)(396.39820009,429.19871497)(396.27320035,429.19871497)
\curveto(396.1482006,429.19871497)(396.04403414,429.20392329)(395.96070098,429.21433994)
\curveto(395.87736781,429.22475658)(395.80965962,429.24038155)(395.75757639,429.26121484)
\curveto(395.71070149,429.28204814)(395.67684739,429.30548559)(395.6560141,429.3315272)
\curveto(395.63518081,429.36277714)(395.62476416,429.3992354)(395.62476416,429.44090198)
\lineto(395.62476416,435.604952)
\lineto(394.63257867,435.604952)
\curveto(394.5492455,435.604952)(394.48674563,435.64661858)(394.44507905,435.72995174)
\curveto(394.40341247,435.81328491)(394.38257918,435.9487013)(394.38257918,436.13620092)
\curveto(394.38257918,436.23515906)(394.3877875,436.31849222)(394.39820414,436.38620042)
\curveto(394.40862079,436.45390861)(394.42424576,436.508596)(394.44507905,436.55026258)
\curveto(394.46591234,436.59713749)(394.49195395,436.62838743)(394.52320389,436.64401239)
\curveto(394.55445383,436.66484568)(394.59091209,436.67526233)(394.63257867,436.67526233)
\lineto(395.62476416,436.67526233)
\lineto(395.62476416,437.40182336)
\curveto(395.62476416,437.8914057)(395.6690349,438.31067569)(395.75757639,438.65963331)
\curveto(395.8513262,439.01379926)(395.99195092,439.30286118)(396.17945054,439.52681906)
\curveto(396.36695016,439.75077694)(396.60132468,439.9148391)(396.88257411,440.01900556)
\curveto(397.16903187,440.12838034)(397.50496869,440.18306773)(397.89038457,440.18306773)
\curveto(398.07267587,440.18306773)(398.24975884,440.1648386)(398.4216335,440.12838034)
\curveto(398.59350815,440.0971304)(398.72632038,440.06067214)(398.82007019,440.01900556)
\curveto(398.91382,439.9825473)(398.97631987,439.9486932)(399.00756981,439.91744327)
\curveto(399.03881975,439.88619333)(399.06486136,439.84452675)(399.08569465,439.79244352)
\curveto(399.10652794,439.74556861)(399.11954875,439.6856729)(399.12475707,439.61275638)
\curveto(399.13517372,439.54504819)(399.14038204,439.46171502)(399.14038204,439.36275689)
\closepath
}
}
{
\newrgbcolor{curcolor}{1 1 1}
\pscustom[linestyle=none,fillstyle=solid,fillcolor=curcolor]
{
\newpath
\moveto(403.87637251,439.36275689)
\curveto(403.87637251,439.26379876)(403.87116419,439.18306975)(403.86074754,439.12056988)
\curveto(403.8503309,439.06327833)(403.83731009,439.01640342)(403.82168512,438.97994516)
\curveto(403.81126848,438.9434869)(403.79564351,438.91744529)(403.77481022,438.90182032)
\curveto(403.75397693,438.89140368)(403.72793531,438.88619535)(403.69668538,438.88619535)
\curveto(403.66022712,438.88619535)(403.61335221,438.896612)(403.55606066,438.91744529)
\curveto(403.50397743,438.9434869)(403.4388734,438.96952852)(403.36074856,438.99557013)
\curveto(403.28262371,439.02682007)(403.1888739,439.05286168)(403.07949912,439.07369497)
\curveto(402.97533267,439.09973659)(402.85293708,439.11275739)(402.71231237,439.11275739)
\curveto(402.51960443,439.11275739)(402.35554226,439.08150746)(402.22012587,439.01900758)
\curveto(402.08470947,438.95650771)(401.97533469,438.85754958)(401.89200153,438.72213319)
\curveto(401.80866836,438.59192512)(401.74877265,438.42005046)(401.71231439,438.20650923)
\curveto(401.67585613,437.99817632)(401.657627,437.7429685)(401.657627,437.44088578)
\lineto(401.657627,436.67526233)
\lineto(403.22793632,436.67526233)
\curveto(403.26960291,436.67526233)(403.303457,436.66484568)(403.32949862,436.64401239)
\curveto(403.36074856,436.62838743)(403.38679017,436.59713749)(403.40762346,436.55026258)
\curveto(403.43366507,436.508596)(403.4518942,436.45390861)(403.46231085,436.38620042)
\curveto(403.47793582,436.31849222)(403.4857483,436.23515906)(403.4857483,436.13620092)
\curveto(403.4857483,435.9487013)(403.46231085,435.81328491)(403.41543594,435.72995174)
\curveto(403.36856104,435.64661858)(403.30606117,435.604952)(403.22793632,435.604952)
\lineto(401.657627,435.604952)
\lineto(401.657627,429.44090198)
\curveto(401.657627,429.3992354)(401.64721036,429.36277714)(401.62637707,429.3315272)
\curveto(401.60554378,429.30548559)(401.56908552,429.28204814)(401.51700229,429.26121484)
\curveto(401.47012738,429.24038155)(401.40502335,429.22475658)(401.32169018,429.21433994)
\curveto(401.23835702,429.20392329)(401.13419056,429.19871497)(401.00919082,429.19871497)
\curveto(400.88419107,429.19871497)(400.78002461,429.20392329)(400.69669145,429.21433994)
\curveto(400.61335829,429.22475658)(400.54565009,429.24038155)(400.49356686,429.26121484)
\curveto(400.44669196,429.28204814)(400.41283786,429.30548559)(400.39200457,429.3315272)
\curveto(400.37117128,429.36277714)(400.36075463,429.3992354)(400.36075463,429.44090198)
\lineto(400.36075463,435.604952)
\lineto(399.36856914,435.604952)
\curveto(399.28523597,435.604952)(399.2227361,435.64661858)(399.18106952,435.72995174)
\curveto(399.13940294,435.81328491)(399.11856965,435.9487013)(399.11856965,436.13620092)
\curveto(399.11856965,436.23515906)(399.12377797,436.31849222)(399.13419461,436.38620042)
\curveto(399.14461126,436.45390861)(399.16023623,436.508596)(399.18106952,436.55026258)
\curveto(399.20190281,436.59713749)(399.22794442,436.62838743)(399.25919436,436.64401239)
\curveto(399.2904443,436.66484568)(399.32690256,436.67526233)(399.36856914,436.67526233)
\lineto(400.36075463,436.67526233)
\lineto(400.36075463,437.40182336)
\curveto(400.36075463,437.8914057)(400.40502537,438.31067569)(400.49356686,438.65963331)
\curveto(400.58731667,439.01379926)(400.72794139,439.30286118)(400.91544101,439.52681906)
\curveto(401.10294063,439.75077694)(401.33731515,439.9148391)(401.61856458,440.01900556)
\curveto(401.90502234,440.12838034)(402.24095916,440.18306773)(402.62637504,440.18306773)
\curveto(402.80866634,440.18306773)(402.98574931,440.1648386)(403.15762397,440.12838034)
\curveto(403.32949862,440.0971304)(403.46231085,440.06067214)(403.55606066,440.01900556)
\curveto(403.64981047,439.9825473)(403.71231034,439.9486932)(403.74356028,439.91744327)
\curveto(403.77481022,439.88619333)(403.80085183,439.84452675)(403.82168512,439.79244352)
\curveto(403.84251841,439.74556861)(403.85553922,439.6856729)(403.86074754,439.61275638)
\curveto(403.87116419,439.54504819)(403.87637251,439.46171502)(403.87637251,439.36275689)
\closepath
}
}
{
\newrgbcolor{curcolor}{1 1 1}
\pscustom[linestyle=none,fillstyle=solid,fillcolor=curcolor]
{
\newpath
\moveto(409.11086184,431.3549606)
\curveto(409.11086184,430.99558633)(409.04315365,430.67527448)(408.90773725,430.39402505)
\curveto(408.77752918,430.11277562)(408.59002956,429.87579693)(408.34523839,429.68308899)
\curveto(408.10044722,429.49038105)(407.80878115,429.34454801)(407.47024016,429.24558988)
\curveto(407.13169918,429.14663174)(406.7593041,429.09715268)(406.35305493,429.09715268)
\curveto(406.10305543,429.09715268)(405.86347259,429.11798597)(405.63430638,429.15965255)
\curveto(405.4103485,429.19611081)(405.20722391,429.24298571)(405.02493262,429.30027726)
\curveto(404.84784964,429.36277714)(404.69680828,429.42527701)(404.57180853,429.48777689)
\curveto(404.44680879,429.55548508)(404.35566314,429.61538079)(404.29837159,429.66746402)
\curveto(404.24108004,429.71954725)(404.19941346,429.79246377)(404.17337184,429.88621358)
\curveto(404.14733023,429.97996339)(404.13430942,430.1075673)(404.13430942,430.2690253)
\curveto(404.13430942,430.36798344)(404.13951774,430.4513166)(404.14993439,430.5190248)
\curveto(404.16035103,430.58673299)(404.17337184,430.64142038)(404.18899681,430.68308696)
\curveto(404.20462178,430.72475355)(404.22545507,430.75339932)(404.25149668,430.76902429)
\curveto(404.28274662,430.78985758)(404.31660072,430.80027423)(404.35305898,430.80027423)
\curveto(404.41035053,430.80027423)(404.49368369,430.76381597)(404.60305847,430.69089945)
\curveto(404.71764157,430.62319125)(404.85566213,430.54767057)(405.01712013,430.46433741)
\curveto(405.18378646,430.38100424)(405.37909857,430.3028794)(405.60305645,430.22996288)
\curveto(405.82701433,430.16225469)(406.0848263,430.12840059)(406.37649238,430.12840059)
\curveto(406.59524194,430.12840059)(406.7931582,430.15183804)(406.97024118,430.19871295)
\curveto(407.14732415,430.24558785)(407.30096967,430.31329605)(407.43117774,430.40183753)
\curveto(407.56138581,430.49558734)(407.66034395,430.61277461)(407.72805214,430.75339932)
\curveto(407.80096866,430.89402404)(407.83742692,431.06069037)(407.83742692,431.25339831)
\curveto(407.83742692,431.45131458)(407.78534369,431.6179809)(407.68117724,431.7533973)
\curveto(407.5822191,431.88881369)(407.44940687,432.00860511)(407.28274054,432.11277157)
\curveto(407.11607422,432.21693803)(406.9285746,432.30808367)(406.72024168,432.38620852)
\curveto(406.51190877,432.46954168)(406.29576338,432.55547901)(406.0718055,432.64402049)
\curveto(405.85305594,432.73256198)(405.63691054,432.83152011)(405.42336931,432.94089489)
\curveto(405.2150364,433.05547799)(405.02753678,433.19349855)(404.86087045,433.35495655)
\curveto(404.69420412,433.51641456)(404.55878773,433.7091225)(404.45462127,433.93308038)
\curveto(404.35566314,434.15703826)(404.30618407,434.42526689)(404.30618407,434.73776625)
\curveto(404.30618407,435.01380736)(404.3582673,435.27682766)(404.46243376,435.52682716)
\curveto(404.57180853,435.78203497)(404.73326654,436.00338869)(404.94680777,436.19088831)
\curveto(405.16034901,436.38359625)(405.42597347,436.53724178)(405.74368116,436.65182488)
\curveto(406.06659717,436.76640798)(406.44159641,436.82369953)(406.86867888,436.82369953)
\curveto(407.0561785,436.82369953)(407.24367812,436.80807456)(407.43117774,436.77682462)
\curveto(407.61867736,436.74557469)(407.78794785,436.70651227)(407.93898922,436.65963736)
\curveto(408.09003058,436.61276246)(408.21763448,436.56067923)(408.32180094,436.50338768)
\curveto(408.43117572,436.45130445)(408.51190472,436.40442955)(408.56398795,436.36276296)
\curveto(408.6212795,436.32109638)(408.65773776,436.28463812)(408.67336273,436.25338818)
\curveto(408.69419602,436.22213825)(408.70721683,436.18567999)(408.71242515,436.14401341)
\curveto(408.7228418,436.10755515)(408.73065428,436.06068024)(408.7358626,436.00338869)
\curveto(408.74627925,435.94609714)(408.75148757,435.87578478)(408.75148757,435.79245162)
\curveto(408.75148757,435.70391013)(408.74627925,435.62578529)(408.7358626,435.55807709)
\curveto(408.73065428,435.49557722)(408.71763347,435.44349399)(408.69680018,435.40182741)
\curveto(408.68117521,435.36016083)(408.66034192,435.32891089)(408.63430031,435.3080776)
\curveto(408.60825869,435.29245263)(408.57961292,435.28464015)(408.54836298,435.28464015)
\curveto(408.50148808,435.28464015)(408.43377988,435.31328592)(408.34523839,435.37057747)
\curveto(408.25669691,435.42786902)(408.1421138,435.48776474)(408.00148909,435.55026461)
\curveto(407.86086437,435.6179728)(407.69419804,435.68047268)(407.5014901,435.73776423)
\curveto(407.31399048,435.79505578)(407.09784509,435.82370155)(406.85305391,435.82370155)
\curveto(406.63430436,435.82370155)(406.44159641,435.79765994)(406.27493009,435.74557671)
\curveto(406.10826376,435.69870181)(405.9702432,435.62838945)(405.86086842,435.53463964)
\curveto(405.75670197,435.44609815)(405.67597297,435.33932754)(405.61868141,435.21432779)
\curveto(405.56659819,435.08932804)(405.54055657,434.95391165)(405.54055657,434.80807861)
\curveto(405.54055657,434.60495402)(405.5926398,434.43307937)(405.69680626,434.29245466)
\curveto(405.80097271,434.15703826)(405.9363891,434.03724684)(406.10305543,433.93308038)
\curveto(406.26972176,433.82891393)(406.45982554,433.73516412)(406.67336678,433.65183095)
\curveto(406.88690801,433.56849779)(407.10305341,433.48256046)(407.32180297,433.39401898)
\curveto(407.54576085,433.30547749)(407.7645104,433.20651935)(407.97805164,433.09714458)
\curveto(408.19680119,432.9877698)(408.38950914,432.85495757)(408.55617547,432.69870788)
\curveto(408.7228418,432.5424582)(408.85565403,432.35495858)(408.95461216,432.13620902)
\curveto(409.05877861,431.91745947)(409.11086184,431.65704333)(409.11086184,431.3549606)
\closepath
}
}
{
\newrgbcolor{curcolor}{1 1 1}
\pscustom[linestyle=none,fillstyle=solid,fillcolor=curcolor]
{
\newpath
\moveto(416.95278398,433.26120674)
\curveto(416.95278398,433.05808216)(416.90070075,432.91224912)(416.7965343,432.82370763)
\curveto(416.69757617,432.74037447)(416.58299306,432.69870788)(416.45278499,432.69870788)
\lineto(411.84341933,432.69870788)
\curveto(411.84341933,432.30808367)(411.88248175,431.95652189)(411.96060659,431.64402252)
\curveto(412.03873143,431.33152315)(412.1689395,431.06329453)(412.3512308,430.83933665)
\curveto(412.5335221,430.61537877)(412.77050078,430.44350412)(413.06216686,430.32371269)
\curveto(413.35383294,430.20392127)(413.71060305,430.14402556)(414.13247719,430.14402556)
\curveto(414.46580985,430.14402556)(414.76268425,430.17006717)(415.02310039,430.2221504)
\curveto(415.28351653,430.27944195)(415.50747441,430.34194182)(415.69497403,430.40965002)
\curveto(415.88768197,430.47735821)(416.04393166,430.53725393)(416.16372308,430.58933715)
\curveto(416.28872283,430.64662871)(416.38247264,430.67527448)(416.44497251,430.67527448)
\curveto(416.48143077,430.67527448)(416.51268071,430.66485783)(416.53872232,430.64402454)
\curveto(416.56997226,430.62839958)(416.59340971,430.60235796)(416.60903468,430.5658997)
\curveto(416.62465965,430.52944144)(416.63507629,430.47735821)(416.64028461,430.40965002)
\curveto(416.65070126,430.34715014)(416.65590958,430.2690253)(416.65590958,430.17527549)
\curveto(416.65590958,430.1075673)(416.65330542,430.04767158)(416.6480971,429.99558836)
\curveto(416.64288878,429.94871345)(416.63507629,429.90444271)(416.62465965,429.86277613)
\curveto(416.61945132,429.82631787)(416.60643052,429.79246377)(416.58559723,429.76121383)
\curveto(416.56997226,429.72996389)(416.5465348,429.69871396)(416.51528487,429.66746402)
\curveto(416.48924325,429.64142241)(416.40591009,429.5945475)(416.26528537,429.52683931)
\curveto(416.12466066,429.46433943)(415.94236936,429.40183956)(415.71841148,429.33933969)
\curveto(415.4944536,429.27683981)(415.23403746,429.22215242)(414.93716306,429.17527752)
\curveto(414.64549699,429.12319429)(414.33299762,429.09715268)(413.99966496,429.09715268)
\curveto(413.42154113,429.09715268)(412.91372966,429.17788168)(412.47623055,429.33933969)
\curveto(412.04393976,429.50079769)(411.67935716,429.74038054)(411.38248276,430.05808823)
\curveto(411.08560836,430.37579592)(410.86165048,430.77423261)(410.71060912,431.25339831)
\curveto(410.55956776,431.73256401)(410.48404708,432.28985454)(410.48404708,432.92526992)
\curveto(410.48404708,433.52943537)(410.56217192,434.07110094)(410.71842161,434.55026663)
\curveto(410.87467129,435.03464065)(411.09862917,435.44349399)(411.39029525,435.77682665)
\curveto(411.68716964,436.11536763)(412.04393976,436.37317961)(412.46060558,436.55026258)
\curveto(412.8772714,436.73255388)(413.34341629,436.82369953)(413.85904025,436.82369953)
\curveto(414.41112246,436.82369953)(414.87987151,436.73515804)(415.2652874,436.55807507)
\curveto(415.65591161,436.38099209)(415.97622346,436.14140924)(416.22622295,435.83932652)
\curveto(416.47622245,435.54245212)(416.65851374,435.19089034)(416.77309685,434.78464116)
\curveto(416.89288827,434.3836003)(416.95278398,433.95391367)(416.95278398,433.49558127)
\closepath
\moveto(415.65591161,433.64401847)
\curveto(415.67153658,434.32110043)(415.52049522,434.85234936)(415.20278753,435.23776524)
\curveto(414.89028816,435.62318113)(414.42414327,435.81588907)(413.80435286,435.81588907)
\curveto(413.48664517,435.81588907)(413.2079999,435.75599336)(412.96841705,435.63620193)
\curveto(412.7288342,435.51641051)(412.52831377,435.35755667)(412.36685577,435.1596404)
\curveto(412.20539776,434.96172413)(412.08039801,434.72995377)(411.99185653,434.46432931)
\curveto(411.90331504,434.20391317)(411.85383597,433.93047622)(411.84341933,433.64401847)
\closepath
}
}
{
\newrgbcolor{curcolor}{1 1 1}
\pscustom[linestyle=none,fillstyle=solid,fillcolor=curcolor]
{
\newpath
\moveto(422.48102253,429.95652594)
\curveto(422.48102253,429.80548458)(422.47060588,429.68569315)(422.44977259,429.59715166)
\curveto(422.4289393,429.50861018)(422.39768936,429.44350614)(422.35602278,429.40183956)
\curveto(422.3143562,429.36017298)(422.25185632,429.32111056)(422.16852316,429.2846523)
\curveto(422.08519,429.24819404)(421.98883602,429.21954826)(421.87946125,429.19871497)
\curveto(421.77529479,429.17267336)(421.66331585,429.15184007)(421.54352443,429.1362151)
\curveto(421.423733,429.12059013)(421.30394158,429.11277764)(421.18415015,429.11277764)
\curveto(420.81956756,429.11277764)(420.50706819,429.15965255)(420.24665205,429.25340236)
\curveto(419.98623591,429.35236049)(419.77269468,429.49819353)(419.60602835,429.69090147)
\curveto(419.43936202,429.88881774)(419.31696643,430.13621307)(419.23884159,430.43308747)
\curveto(419.16592507,430.73517019)(419.12946681,431.08933614)(419.12946681,431.49558532)
\lineto(419.12946681,435.604952)
\lineto(418.14509381,435.604952)
\curveto(418.06696897,435.604952)(418.00446909,435.64661858)(417.95759419,435.72995174)
\curveto(417.91071928,435.81328491)(417.88728183,435.9487013)(417.88728183,436.13620092)
\curveto(417.88728183,436.23515906)(417.89249015,436.31849222)(417.9029068,436.38620042)
\curveto(417.91853177,436.45390861)(417.9367609,436.508596)(417.95759419,436.55026258)
\curveto(417.97842748,436.59713749)(418.00446909,436.62838743)(418.03571903,436.64401239)
\curveto(418.07217729,436.66484568)(418.11123971,436.67526233)(418.15290629,436.67526233)
\lineto(419.12946681,436.67526233)
\lineto(419.12946681,438.34713394)
\curveto(419.12946681,438.3835922)(419.1372793,438.4174463)(419.15290427,438.44869624)
\curveto(419.17373756,438.47994618)(419.20759166,438.50598779)(419.25446656,438.52682108)
\curveto(419.30654979,438.5528627)(419.37425799,438.57109182)(419.45759115,438.58150847)
\curveto(419.54092431,438.59192512)(419.64509077,438.59713344)(419.77009052,438.59713344)
\curveto(419.90029859,438.59713344)(420.0070692,438.59192512)(420.09040237,438.58150847)
\curveto(420.17373553,438.57109182)(420.23883957,438.5528627)(420.28571447,438.52682108)
\curveto(420.33258938,438.50598779)(420.36644348,438.47994618)(420.38727677,438.44869624)
\curveto(420.40811006,438.4174463)(420.4185267,438.3835922)(420.4185267,438.34713394)
\lineto(420.4185267,436.67526233)
\lineto(422.22321055,436.67526233)
\curveto(422.26487713,436.67526233)(422.30133539,436.66484568)(422.33258533,436.64401239)
\curveto(422.36383526,436.62838743)(422.38987688,436.59713749)(422.41071017,436.55026258)
\curveto(422.43675178,436.508596)(422.45498091,436.45390861)(422.46539756,436.38620042)
\curveto(422.4758142,436.31849222)(422.48102253,436.23515906)(422.48102253,436.13620092)
\curveto(422.48102253,435.9487013)(422.45758507,435.81328491)(422.41071017,435.72995174)
\curveto(422.36383526,435.64661858)(422.30133539,435.604952)(422.22321055,435.604952)
\lineto(420.4185267,435.604952)
\lineto(420.4185267,431.68308494)
\curveto(420.4185267,431.19871092)(420.48883906,430.83152416)(420.62946378,430.58152467)
\curveto(420.77529681,430.3367335)(421.03310879,430.21433791)(421.40289971,430.21433791)
\curveto(421.52269113,430.21433791)(421.62946175,430.22475456)(421.72321156,430.24558785)
\curveto(421.81696137,430.27162946)(421.90029454,430.29767108)(421.97321106,430.32371269)
\curveto(422.04612757,430.34975431)(422.10862745,430.37319176)(422.16071068,430.39402505)
\curveto(422.2127939,430.42006666)(422.25966881,430.43308747)(422.30133539,430.43308747)
\curveto(422.32737701,430.43308747)(422.35081446,430.42527499)(422.37164775,430.40965002)
\curveto(422.39768936,430.39923337)(422.41591849,430.37579592)(422.42633514,430.33933766)
\curveto(422.44196011,430.3028794)(422.45498091,430.25340033)(422.46539756,430.19090046)
\curveto(422.4758142,430.12840059)(422.48102253,430.05027575)(422.48102253,429.95652594)
\closepath
}
}
{
\newrgbcolor{curcolor}{1 1 1}
\pscustom[linestyle=none,fillstyle=solid,fillcolor=curcolor]
{
\newpath
\moveto(430.19219427,429.8237137)
\curveto(430.19219427,429.72996389)(430.18438179,429.64663073)(430.16875682,429.57371421)
\curveto(430.15834018,429.50079769)(430.14011105,429.43829782)(430.11406943,429.38621459)
\curveto(430.09323614,429.33933969)(430.06198621,429.30288143)(430.02031962,429.27683981)
\curveto(429.98386136,429.25600652)(429.94219478,429.24558988)(429.89531988,429.24558988)
\lineto(424.30939369,429.24558988)
\curveto(424.23647717,429.24558988)(424.17137313,429.25340236)(424.11408158,429.26902733)
\curveto(424.06199836,429.28986062)(424.01512345,429.32111056)(423.97345687,429.36277714)
\curveto(423.93699861,429.40444372)(423.90835283,429.46433943)(423.88751954,429.54246427)
\curveto(423.87189457,429.62058912)(423.86408209,429.71694309)(423.86408209,429.83152619)
\curveto(423.86408209,429.93569264)(423.86668625,430.02683829)(423.87189457,430.10496314)
\curveto(423.88231122,430.18308798)(423.90054035,430.25079617)(423.92658196,430.30808772)
\curveto(423.95262358,430.3705876)(423.98387351,430.43048331)(424.02033177,430.48777486)
\curveto(424.06199836,430.55027473)(424.11408158,430.61537877)(424.17658146,430.68308696)
\lineto(426.13751499,432.76120776)
\curveto(426.59063907,433.24037345)(426.9526175,433.67006008)(427.22345029,434.05026765)
\curveto(427.49949139,434.43047521)(427.71042847,434.77682867)(427.85626151,435.08932804)
\curveto(428.00730287,435.40182741)(428.106261,435.685681)(428.1531359,435.94088882)
\curveto(428.20001081,436.19609663)(428.22344826,436.43567948)(428.22344826,436.65963736)
\curveto(428.22344826,436.88359524)(428.18699,437.09453231)(428.11407348,437.29244858)
\curveto(428.04115696,437.49557317)(427.93438635,437.67265614)(427.79376163,437.8236975)
\curveto(427.65834524,437.97473887)(427.48647059,438.09453029)(427.27813768,438.18307178)
\curveto(427.06980476,438.27161326)(426.83022192,438.31588401)(426.55938913,438.31588401)
\curveto(426.24168144,438.31588401)(425.95522369,438.27161326)(425.70001587,438.18307178)
\curveto(425.45001638,438.09453029)(425.22866266,437.99817632)(425.03595472,437.89400986)
\curveto(424.8484551,437.78984341)(424.68960125,437.69348944)(424.55939318,437.60494795)
\curveto(424.43439343,437.51640646)(424.34064362,437.47213572)(424.27814375,437.47213572)
\curveto(424.24168549,437.47213572)(424.20783139,437.48255236)(424.17658146,437.50338565)
\curveto(424.15053984,437.52421894)(424.12710239,437.55807304)(424.1062691,437.60494795)
\curveto(424.09064413,437.65182285)(424.07762332,437.71432273)(424.06720668,437.79244757)
\curveto(424.05679003,437.87057241)(424.05158171,437.96432222)(424.05158171,438.073697)
\curveto(424.05158171,438.15182184)(424.05418587,438.21953004)(424.05939419,438.27682159)
\curveto(424.06460252,438.33411314)(424.072415,438.3835922)(424.08283165,438.42525879)
\curveto(424.09845661,438.46692537)(424.11668574,438.50598779)(424.13751904,438.54244605)
\curveto(424.15835233,438.57890431)(424.20001891,438.62317505)(424.26251878,438.67525828)
\curveto(424.32501866,438.73254983)(424.43178927,438.80807051)(424.58283063,438.90182032)
\curveto(424.73908032,438.99557013)(424.93178826,439.08671578)(425.16095446,439.17525727)
\curveto(425.39532899,439.26900708)(425.65053681,439.34713192)(425.92657791,439.40963179)
\curveto(426.20782734,439.47213167)(426.50209758,439.5033816)(426.80938863,439.5033816)
\curveto(427.29897097,439.5033816)(427.72605344,439.43306925)(428.09063603,439.29244453)
\curveto(428.46042695,439.15702814)(428.76511383,438.96952852)(429.00469668,438.72994567)
\curveto(429.24948785,438.49036282)(429.43177915,438.21171755)(429.55157057,437.89400986)
\curveto(429.671362,437.57630217)(429.73125771,437.23776119)(429.73125771,436.87838692)
\curveto(429.73125771,436.55547091)(429.70261193,436.23255489)(429.64532038,435.90963888)
\curveto(429.58802883,435.59193119)(429.46563325,435.24557773)(429.27813363,434.87057848)
\curveto(429.09584233,434.50078757)(428.83021787,434.08672591)(428.48126024,433.6283935)
\curveto(428.13230261,433.17526942)(427.66876188,432.65183298)(427.09063806,432.05808418)
\lineto(425.4890788,430.38621257)
\lineto(429.88750739,430.38621257)
\curveto(429.92917397,430.38621257)(429.9682364,430.37319176)(430.00469465,430.34715014)
\curveto(430.04636124,430.32631685)(430.08021534,430.29246276)(430.10625695,430.24558785)
\curveto(430.13750689,430.19871295)(430.15834018,430.13881723)(430.16875682,430.06590071)
\curveto(430.18438179,429.99819252)(430.19219427,429.91746352)(430.19219427,429.8237137)
\closepath
}
}
{
\newrgbcolor{curcolor}{0 0 0}
\pscustom[linestyle=none,fillstyle=solid,fillcolor=curcolor,opacity=0]
{
\newpath
\moveto(376.66154747,385.29476252)
\lineto(465.25979325,385.29476252)
\lineto(465.25979325,348.50743544)
\lineto(376.66154747,348.50743544)
\closepath
}
}
{
\newrgbcolor{curcolor}{1 1 1}
\pscustom[linestyle=none,fillstyle=solid,fillcolor=curcolor]
{
\newpath
\moveto(393.39589351,364.72387303)
\curveto(393.39589351,364.15095753)(393.32037283,363.62231276)(393.16933147,363.13793874)
\curveto(393.01829011,362.65877305)(392.79172807,362.24471139)(392.48964535,361.89575376)
\curveto(392.19277095,361.54679613)(391.81777171,361.27335919)(391.36464763,361.07544292)
\curveto(390.91673187,360.88273498)(390.39589959,360.78638101)(389.80215079,360.78638101)
\curveto(389.22402696,360.78638101)(388.71881965,360.87231833)(388.28652886,361.04419298)
\curveto(387.85944639,361.21606764)(387.50267628,361.46606713)(387.21621853,361.79419146)
\curveto(386.92976077,362.1223158)(386.71621954,362.52075249)(386.57559482,362.98950154)
\curveto(386.43497011,363.4582506)(386.36465775,363.98949952)(386.36465775,364.58324832)
\curveto(386.36465775,365.15616382)(386.43757427,365.68220443)(386.58340731,366.16137012)
\curveto(386.73444867,366.64574414)(386.95840655,367.06240996)(387.25528095,367.41136759)
\curveto(387.55736367,367.76032522)(387.93236291,368.031158)(388.38027867,368.22386594)
\curveto(388.82819443,368.41657389)(389.35163087,368.51292786)(389.95058799,368.51292786)
\curveto(390.52871182,368.51292786)(391.03131497,368.42699053)(391.45839744,368.25511588)
\curveto(391.89068823,368.08324123)(392.2500625,367.83324174)(392.53652025,367.5051174)
\curveto(392.82297801,367.17699306)(393.03651924,366.77855637)(393.17714396,366.30980732)
\curveto(393.32297699,365.84105827)(393.39589351,365.31241351)(393.39589351,364.72387303)
\closepath
\moveto(392.03652127,364.63793571)
\curveto(392.03652127,365.01814327)(392.00006301,365.37751754)(391.92714649,365.71605852)
\curveto(391.85943829,366.0545995)(391.74485519,366.3514739)(391.58339718,366.60668172)
\curveto(391.42193918,366.86188954)(391.20318962,367.06240996)(390.92714851,367.208243)
\curveto(390.6511074,367.35928436)(390.3073581,367.43480504)(389.8959006,367.43480504)
\curveto(389.51569304,367.43480504)(389.1875687,367.36709685)(388.91152759,367.23168045)
\curveto(388.64069481,367.09626406)(388.41673693,366.90355612)(388.23965395,366.65355662)
\curveto(388.06257098,366.40876545)(387.92975875,366.11709938)(387.84121726,365.7785584)
\curveto(387.7578841,365.44001742)(387.71621751,365.0702265)(387.71621751,364.66918564)
\curveto(387.71621751,364.28376976)(387.75007161,363.92179132)(387.81777981,363.58325034)
\curveto(387.89069633,363.24470936)(388.00788359,362.94783496)(388.1693416,362.69262715)
\curveto(388.33600793,362.44262765)(388.55736164,362.24210722)(388.83340275,362.09106586)
\curveto(389.10944386,361.94523283)(389.45319316,361.87231631)(389.86465066,361.87231631)
\curveto(390.2396499,361.87231631)(390.56517008,361.9400245)(390.84121119,362.0754409)
\curveto(391.11725229,362.21085729)(391.34381434,362.40096107)(391.52089731,362.64575224)
\curveto(391.69798028,362.89054341)(391.82818835,363.18220949)(391.91152152,363.52075047)
\curveto(391.99485468,363.85929145)(392.03652127,364.23168653)(392.03652127,364.63793571)
\closepath
}
}
{
\newrgbcolor{curcolor}{1 1 1}
\pscustom[linestyle=none,fillstyle=solid,fillcolor=curcolor]
{
\newpath
\moveto(399.14038204,371.05198522)
\curveto(399.14038204,370.95302708)(399.13517372,370.87229808)(399.12475707,370.80979821)
\curveto(399.11434043,370.75250666)(399.10131962,370.70563175)(399.08569465,370.66917349)
\curveto(399.07527801,370.63271523)(399.05965304,370.60667362)(399.03881975,370.59104865)
\curveto(399.01798646,370.58063201)(398.99194484,370.57542368)(398.9606949,370.57542368)
\curveto(398.92423664,370.57542368)(398.87736174,370.58584033)(398.82007019,370.60667362)
\curveto(398.76798696,370.63271523)(398.70288293,370.65875685)(398.62475808,370.68479846)
\curveto(398.54663324,370.7160484)(398.45288343,370.74209001)(398.34350865,370.7629233)
\curveto(398.2393422,370.78896492)(398.11694661,370.80198572)(397.9763219,370.80198572)
\curveto(397.78361395,370.80198572)(397.61955179,370.77073579)(397.48413539,370.70823591)
\curveto(397.348719,370.64573604)(397.23934422,370.54677791)(397.15601106,370.41136152)
\curveto(397.07267789,370.28115345)(397.01278218,370.10927879)(396.97632392,369.89573756)
\curveto(396.93986566,369.68740465)(396.92163653,369.43219683)(396.92163653,369.13011411)
\lineto(396.92163653,368.36449066)
\lineto(398.49194585,368.36449066)
\curveto(398.53361244,368.36449066)(398.56746653,368.35407401)(398.59350815,368.33324072)
\curveto(398.62475808,368.31761576)(398.6507997,368.28636582)(398.67163299,368.23949091)
\curveto(398.6976746,368.19782433)(398.71590373,368.14313694)(398.72632038,368.07542875)
\curveto(398.74194535,368.00772055)(398.74975783,367.92438738)(398.74975783,367.82542925)
\curveto(398.74975783,367.63792963)(398.72632038,367.50251324)(398.67944547,367.41918007)
\curveto(398.63257057,367.33584691)(398.5700707,367.29418033)(398.49194585,367.29418033)
\lineto(396.92163653,367.29418033)
\lineto(396.92163653,361.13013031)
\curveto(396.92163653,361.08846373)(396.91121989,361.05200547)(396.8903866,361.02075553)
\curveto(396.86955331,360.99471392)(396.83309505,360.97127646)(396.78101182,360.95044317)
\curveto(396.73413691,360.92960988)(396.66903288,360.91398491)(396.58569971,360.90356827)
\curveto(396.50236655,360.89315162)(396.39820009,360.8879433)(396.27320035,360.8879433)
\curveto(396.1482006,360.8879433)(396.04403414,360.89315162)(395.96070098,360.90356827)
\curveto(395.87736781,360.91398491)(395.80965962,360.92960988)(395.75757639,360.95044317)
\curveto(395.71070149,360.97127646)(395.67684739,360.99471392)(395.6560141,361.02075553)
\curveto(395.63518081,361.05200547)(395.62476416,361.08846373)(395.62476416,361.13013031)
\lineto(395.62476416,367.29418033)
\lineto(394.63257867,367.29418033)
\curveto(394.5492455,367.29418033)(394.48674563,367.33584691)(394.44507905,367.41918007)
\curveto(394.40341247,367.50251324)(394.38257918,367.63792963)(394.38257918,367.82542925)
\curveto(394.38257918,367.92438738)(394.3877875,368.00772055)(394.39820414,368.07542875)
\curveto(394.40862079,368.14313694)(394.42424576,368.19782433)(394.44507905,368.23949091)
\curveto(394.46591234,368.28636582)(394.49195395,368.31761576)(394.52320389,368.33324072)
\curveto(394.55445383,368.35407401)(394.59091209,368.36449066)(394.63257867,368.36449066)
\lineto(395.62476416,368.36449066)
\lineto(395.62476416,369.09105169)
\curveto(395.62476416,369.58063403)(395.6690349,369.99990402)(395.75757639,370.34886164)
\curveto(395.8513262,370.70302759)(395.99195092,370.99208951)(396.17945054,371.21604739)
\curveto(396.36695016,371.44000527)(396.60132468,371.60406743)(396.88257411,371.70823389)
\curveto(397.16903187,371.81760867)(397.50496869,371.87229606)(397.89038457,371.87229606)
\curveto(398.07267587,371.87229606)(398.24975884,371.85406693)(398.4216335,371.81760867)
\curveto(398.59350815,371.78635873)(398.72632038,371.74990047)(398.82007019,371.70823389)
\curveto(398.91382,371.67177563)(398.97631987,371.63792153)(399.00756981,371.60667159)
\curveto(399.03881975,371.57542166)(399.06486136,371.53375508)(399.08569465,371.48167185)
\curveto(399.10652794,371.43479694)(399.11954875,371.37490123)(399.12475707,371.30198471)
\curveto(399.13517372,371.23427652)(399.14038204,371.15094335)(399.14038204,371.05198522)
\closepath
}
}
{
\newrgbcolor{curcolor}{1 1 1}
\pscustom[linestyle=none,fillstyle=solid,fillcolor=curcolor]
{
\newpath
\moveto(403.87637251,371.05198522)
\curveto(403.87637251,370.95302708)(403.87116419,370.87229808)(403.86074754,370.80979821)
\curveto(403.8503309,370.75250666)(403.83731009,370.70563175)(403.82168512,370.66917349)
\curveto(403.81126848,370.63271523)(403.79564351,370.60667362)(403.77481022,370.59104865)
\curveto(403.75397693,370.58063201)(403.72793531,370.57542368)(403.69668538,370.57542368)
\curveto(403.66022712,370.57542368)(403.61335221,370.58584033)(403.55606066,370.60667362)
\curveto(403.50397743,370.63271523)(403.4388734,370.65875685)(403.36074856,370.68479846)
\curveto(403.28262371,370.7160484)(403.1888739,370.74209001)(403.07949912,370.7629233)
\curveto(402.97533267,370.78896492)(402.85293708,370.80198572)(402.71231237,370.80198572)
\curveto(402.51960443,370.80198572)(402.35554226,370.77073579)(402.22012587,370.70823591)
\curveto(402.08470947,370.64573604)(401.97533469,370.54677791)(401.89200153,370.41136152)
\curveto(401.80866836,370.28115345)(401.74877265,370.10927879)(401.71231439,369.89573756)
\curveto(401.67585613,369.68740465)(401.657627,369.43219683)(401.657627,369.13011411)
\lineto(401.657627,368.36449066)
\lineto(403.22793632,368.36449066)
\curveto(403.26960291,368.36449066)(403.303457,368.35407401)(403.32949862,368.33324072)
\curveto(403.36074856,368.31761576)(403.38679017,368.28636582)(403.40762346,368.23949091)
\curveto(403.43366507,368.19782433)(403.4518942,368.14313694)(403.46231085,368.07542875)
\curveto(403.47793582,368.00772055)(403.4857483,367.92438738)(403.4857483,367.82542925)
\curveto(403.4857483,367.63792963)(403.46231085,367.50251324)(403.41543594,367.41918007)
\curveto(403.36856104,367.33584691)(403.30606117,367.29418033)(403.22793632,367.29418033)
\lineto(401.657627,367.29418033)
\lineto(401.657627,361.13013031)
\curveto(401.657627,361.08846373)(401.64721036,361.05200547)(401.62637707,361.02075553)
\curveto(401.60554378,360.99471392)(401.56908552,360.97127646)(401.51700229,360.95044317)
\curveto(401.47012738,360.92960988)(401.40502335,360.91398491)(401.32169018,360.90356827)
\curveto(401.23835702,360.89315162)(401.13419056,360.8879433)(401.00919082,360.8879433)
\curveto(400.88419107,360.8879433)(400.78002461,360.89315162)(400.69669145,360.90356827)
\curveto(400.61335829,360.91398491)(400.54565009,360.92960988)(400.49356686,360.95044317)
\curveto(400.44669196,360.97127646)(400.41283786,360.99471392)(400.39200457,361.02075553)
\curveto(400.37117128,361.05200547)(400.36075463,361.08846373)(400.36075463,361.13013031)
\lineto(400.36075463,367.29418033)
\lineto(399.36856914,367.29418033)
\curveto(399.28523597,367.29418033)(399.2227361,367.33584691)(399.18106952,367.41918007)
\curveto(399.13940294,367.50251324)(399.11856965,367.63792963)(399.11856965,367.82542925)
\curveto(399.11856965,367.92438738)(399.12377797,368.00772055)(399.13419461,368.07542875)
\curveto(399.14461126,368.14313694)(399.16023623,368.19782433)(399.18106952,368.23949091)
\curveto(399.20190281,368.28636582)(399.22794442,368.31761576)(399.25919436,368.33324072)
\curveto(399.2904443,368.35407401)(399.32690256,368.36449066)(399.36856914,368.36449066)
\lineto(400.36075463,368.36449066)
\lineto(400.36075463,369.09105169)
\curveto(400.36075463,369.58063403)(400.40502537,369.99990402)(400.49356686,370.34886164)
\curveto(400.58731667,370.70302759)(400.72794139,370.99208951)(400.91544101,371.21604739)
\curveto(401.10294063,371.44000527)(401.33731515,371.60406743)(401.61856458,371.70823389)
\curveto(401.90502234,371.81760867)(402.24095916,371.87229606)(402.62637504,371.87229606)
\curveto(402.80866634,371.87229606)(402.98574931,371.85406693)(403.15762397,371.81760867)
\curveto(403.32949862,371.78635873)(403.46231085,371.74990047)(403.55606066,371.70823389)
\curveto(403.64981047,371.67177563)(403.71231034,371.63792153)(403.74356028,371.60667159)
\curveto(403.77481022,371.57542166)(403.80085183,371.53375508)(403.82168512,371.48167185)
\curveto(403.84251841,371.43479694)(403.85553922,371.37490123)(403.86074754,371.30198471)
\curveto(403.87116419,371.23427652)(403.87637251,371.15094335)(403.87637251,371.05198522)
\closepath
}
}
{
\newrgbcolor{curcolor}{1 1 1}
\pscustom[linestyle=none,fillstyle=solid,fillcolor=curcolor]
{
\newpath
\moveto(409.11086184,363.04418893)
\curveto(409.11086184,362.68481466)(409.04315365,362.36450281)(408.90773725,362.08325338)
\curveto(408.77752918,361.80200395)(408.59002956,361.56502526)(408.34523839,361.37231732)
\curveto(408.10044722,361.17960938)(407.80878115,361.03377634)(407.47024016,360.93481821)
\curveto(407.13169918,360.83586007)(406.7593041,360.78638101)(406.35305493,360.78638101)
\curveto(406.10305543,360.78638101)(405.86347259,360.8072143)(405.63430638,360.84888088)
\curveto(405.4103485,360.88533914)(405.20722391,360.93221404)(405.02493262,360.98950559)
\curveto(404.84784964,361.05200547)(404.69680828,361.11450534)(404.57180853,361.17700521)
\curveto(404.44680879,361.24471341)(404.35566314,361.30460912)(404.29837159,361.35669235)
\curveto(404.24108004,361.40877558)(404.19941346,361.4816921)(404.17337184,361.57544191)
\curveto(404.14733023,361.66919172)(404.13430942,361.79679563)(404.13430942,361.95825363)
\curveto(404.13430942,362.05721177)(404.13951774,362.14054493)(404.14993439,362.20825313)
\curveto(404.16035103,362.27596132)(404.17337184,362.33064871)(404.18899681,362.37231529)
\curveto(404.20462178,362.41398188)(404.22545507,362.44262765)(404.25149668,362.45825262)
\curveto(404.28274662,362.47908591)(404.31660072,362.48950256)(404.35305898,362.48950256)
\curveto(404.41035053,362.48950256)(404.49368369,362.4530443)(404.60305847,362.38012778)
\curveto(404.71764157,362.31241958)(404.85566213,362.2368989)(405.01712013,362.15356574)
\curveto(405.18378646,362.07023257)(405.37909857,361.99210773)(405.60305645,361.91919121)
\curveto(405.82701433,361.85148302)(406.0848263,361.81762892)(406.37649238,361.81762892)
\curveto(406.59524194,361.81762892)(406.7931582,361.84106637)(406.97024118,361.88794128)
\curveto(407.14732415,361.93481618)(407.30096967,362.00252438)(407.43117774,362.09106586)
\curveto(407.56138581,362.18481567)(407.66034395,362.30200294)(407.72805214,362.44262765)
\curveto(407.80096866,362.58325237)(407.83742692,362.7499187)(407.83742692,362.94262664)
\curveto(407.83742692,363.14054291)(407.78534369,363.30720923)(407.68117724,363.44262563)
\curveto(407.5822191,363.57804202)(407.44940687,363.69783344)(407.28274054,363.8019999)
\curveto(407.11607422,363.90616635)(406.9285746,363.997312)(406.72024168,364.07543685)
\curveto(406.51190877,364.15877001)(406.29576338,364.24470734)(406.0718055,364.33324882)
\curveto(405.85305594,364.42179031)(405.63691054,364.52074844)(405.42336931,364.63012322)
\curveto(405.2150364,364.74470632)(405.02753678,364.88272688)(404.86087045,365.04418488)
\curveto(404.69420412,365.20564289)(404.55878773,365.39835083)(404.45462127,365.62230871)
\curveto(404.35566314,365.84626659)(404.30618407,366.11449522)(404.30618407,366.42699458)
\curveto(404.30618407,366.70303569)(404.3582673,366.96605599)(404.46243376,367.21605549)
\curveto(404.57180853,367.4712633)(404.73326654,367.69261702)(404.94680777,367.88011664)
\curveto(405.16034901,368.07282458)(405.42597347,368.22647011)(405.74368116,368.34105321)
\curveto(406.06659717,368.45563631)(406.44159641,368.51292786)(406.86867888,368.51292786)
\curveto(407.0561785,368.51292786)(407.24367812,368.49730289)(407.43117774,368.46605295)
\curveto(407.61867736,368.43480302)(407.78794785,368.3957406)(407.93898922,368.34886569)
\curveto(408.09003058,368.30199079)(408.21763448,368.24990756)(408.32180094,368.19261601)
\curveto(408.43117572,368.14053278)(408.51190472,368.09365788)(408.56398795,368.05199129)
\curveto(408.6212795,368.01032471)(408.65773776,367.97386645)(408.67336273,367.94261651)
\curveto(408.69419602,367.91136658)(408.70721683,367.87490832)(408.71242515,367.83324174)
\curveto(408.7228418,367.79678348)(408.73065428,367.74990857)(408.7358626,367.69261702)
\curveto(408.74627925,367.63532547)(408.75148757,367.56501311)(408.75148757,367.48167995)
\curveto(408.75148757,367.39313846)(408.74627925,367.31501362)(408.7358626,367.24730542)
\curveto(408.73065428,367.18480555)(408.71763347,367.13272232)(408.69680018,367.09105574)
\curveto(408.68117521,367.04938916)(408.66034192,367.01813922)(408.63430031,366.99730593)
\curveto(408.60825869,366.98168096)(408.57961292,366.97386848)(408.54836298,366.97386848)
\curveto(408.50148808,366.97386848)(408.43377988,367.00251425)(408.34523839,367.0598058)
\curveto(408.25669691,367.11709735)(408.1421138,367.17699306)(408.00148909,367.23949294)
\curveto(407.86086437,367.30720113)(407.69419804,367.36970101)(407.5014901,367.42699256)
\curveto(407.31399048,367.48428411)(407.09784509,367.51292988)(406.85305391,367.51292988)
\curveto(406.63430436,367.51292988)(406.44159641,367.48688827)(406.27493009,367.43480504)
\curveto(406.10826376,367.38793014)(405.9702432,367.31761778)(405.86086842,367.22386797)
\curveto(405.75670197,367.13532648)(405.67597297,367.02855587)(405.61868141,366.90355612)
\curveto(405.56659819,366.77855637)(405.54055657,366.64313998)(405.54055657,366.49730694)
\curveto(405.54055657,366.29418235)(405.5926398,366.1223077)(405.69680626,365.98168299)
\curveto(405.80097271,365.84626659)(405.9363891,365.72647517)(406.10305543,365.62230871)
\curveto(406.26972176,365.51814226)(406.45982554,365.42439245)(406.67336678,365.34105928)
\curveto(406.88690801,365.25772612)(407.10305341,365.17178879)(407.32180297,365.0832473)
\curveto(407.54576085,364.99470582)(407.7645104,364.89574768)(407.97805164,364.78637291)
\curveto(408.19680119,364.67699813)(408.38950914,364.5441859)(408.55617547,364.38793621)
\curveto(408.7228418,364.23168653)(408.85565403,364.04418691)(408.95461216,363.82543735)
\curveto(409.05877861,363.60668779)(409.11086184,363.34627166)(409.11086184,363.04418893)
\closepath
}
}
{
\newrgbcolor{curcolor}{1 1 1}
\pscustom[linestyle=none,fillstyle=solid,fillcolor=curcolor]
{
\newpath
\moveto(416.95278398,364.95043507)
\curveto(416.95278398,364.74731048)(416.90070075,364.60147745)(416.7965343,364.51293596)
\curveto(416.69757617,364.42960279)(416.58299306,364.38793621)(416.45278499,364.38793621)
\lineto(411.84341933,364.38793621)
\curveto(411.84341933,363.997312)(411.88248175,363.64575022)(411.96060659,363.33325085)
\curveto(412.03873143,363.02075148)(412.1689395,362.75252286)(412.3512308,362.52856498)
\curveto(412.5335221,362.3046071)(412.77050078,362.13273245)(413.06216686,362.01294102)
\curveto(413.35383294,361.8931496)(413.71060305,361.83325389)(414.13247719,361.83325389)
\curveto(414.46580985,361.83325389)(414.76268425,361.8592955)(415.02310039,361.91137873)
\curveto(415.28351653,361.96867028)(415.50747441,362.03117015)(415.69497403,362.09887835)
\curveto(415.88768197,362.16658654)(416.04393166,362.22648226)(416.16372308,362.27856548)
\curveto(416.28872283,362.33585703)(416.38247264,362.36450281)(416.44497251,362.36450281)
\curveto(416.48143077,362.36450281)(416.51268071,362.35408616)(416.53872232,362.33325287)
\curveto(416.56997226,362.31762791)(416.59340971,362.29158629)(416.60903468,362.25512803)
\curveto(416.62465965,362.21866977)(416.63507629,362.16658654)(416.64028461,362.09887835)
\curveto(416.65070126,362.03637847)(416.65590958,361.95825363)(416.65590958,361.86450382)
\curveto(416.65590958,361.79679563)(416.65330542,361.73689991)(416.6480971,361.68481669)
\curveto(416.64288878,361.63794178)(416.63507629,361.59367104)(416.62465965,361.55200446)
\curveto(416.61945132,361.5155462)(416.60643052,361.4816921)(416.58559723,361.45044216)
\curveto(416.56997226,361.41919222)(416.5465348,361.38794229)(416.51528487,361.35669235)
\curveto(416.48924325,361.33065074)(416.40591009,361.28377583)(416.26528537,361.21606764)
\curveto(416.12466066,361.15356776)(415.94236936,361.09106789)(415.71841148,361.02856802)
\curveto(415.4944536,360.96606814)(415.23403746,360.91138075)(414.93716306,360.86450585)
\curveto(414.64549699,360.81242262)(414.33299762,360.78638101)(413.99966496,360.78638101)
\curveto(413.42154113,360.78638101)(412.91372966,360.86711001)(412.47623055,361.02856802)
\curveto(412.04393976,361.19002602)(411.67935716,361.42960887)(411.38248276,361.74731656)
\curveto(411.08560836,362.06502425)(410.86165048,362.46346094)(410.71060912,362.94262664)
\curveto(410.55956776,363.42179234)(410.48404708,363.97908287)(410.48404708,364.61449825)
\curveto(410.48404708,365.2186637)(410.56217192,365.76032927)(410.71842161,366.23949496)
\curveto(410.87467129,366.72386898)(411.09862917,367.13272232)(411.39029525,367.46605498)
\curveto(411.68716964,367.80459596)(412.04393976,368.06240794)(412.46060558,368.23949091)
\curveto(412.8772714,368.42178221)(413.34341629,368.51292786)(413.85904025,368.51292786)
\curveto(414.41112246,368.51292786)(414.87987151,368.42438637)(415.2652874,368.2473034)
\curveto(415.65591161,368.07022042)(415.97622346,367.83063757)(416.22622295,367.52855485)
\curveto(416.47622245,367.23168045)(416.65851374,366.88011867)(416.77309685,366.47386949)
\curveto(416.89288827,366.07282863)(416.95278398,365.643142)(416.95278398,365.1848096)
\closepath
\moveto(415.65591161,365.3332468)
\curveto(415.67153658,366.01032876)(415.52049522,366.54157768)(415.20278753,366.92699357)
\curveto(414.89028816,367.31240946)(414.42414327,367.5051174)(413.80435286,367.5051174)
\curveto(413.48664517,367.5051174)(413.2079999,367.44522169)(412.96841705,367.32543026)
\curveto(412.7288342,367.20563884)(412.52831377,367.046785)(412.36685577,366.84886873)
\curveto(412.20539776,366.65095246)(412.08039801,366.4191821)(411.99185653,366.15355764)
\curveto(411.90331504,365.8931415)(411.85383597,365.61970455)(411.84341933,365.3332468)
\closepath
}
}
{
\newrgbcolor{curcolor}{1 1 1}
\pscustom[linestyle=none,fillstyle=solid,fillcolor=curcolor]
{
\newpath
\moveto(422.48102253,361.64575427)
\curveto(422.48102253,361.4947129)(422.47060588,361.37492148)(422.44977259,361.28637999)
\curveto(422.4289393,361.19783851)(422.39768936,361.13273447)(422.35602278,361.09106789)
\curveto(422.3143562,361.04940131)(422.25185632,361.01033889)(422.16852316,360.97388063)
\curveto(422.08519,360.93742237)(421.98883602,360.90877659)(421.87946125,360.8879433)
\curveto(421.77529479,360.86190169)(421.66331585,360.8410684)(421.54352443,360.82544343)
\curveto(421.423733,360.80981846)(421.30394158,360.80200597)(421.18415015,360.80200597)
\curveto(420.81956756,360.80200597)(420.50706819,360.84888088)(420.24665205,360.94263069)
\curveto(419.98623591,361.04158882)(419.77269468,361.18742186)(419.60602835,361.3801298)
\curveto(419.43936202,361.57804607)(419.31696643,361.8254414)(419.23884159,362.1223158)
\curveto(419.16592507,362.42439852)(419.12946681,362.77856447)(419.12946681,363.18481365)
\lineto(419.12946681,367.29418033)
\lineto(418.14509381,367.29418033)
\curveto(418.06696897,367.29418033)(418.00446909,367.33584691)(417.95759419,367.41918007)
\curveto(417.91071928,367.50251324)(417.88728183,367.63792963)(417.88728183,367.82542925)
\curveto(417.88728183,367.92438738)(417.89249015,368.00772055)(417.9029068,368.07542875)
\curveto(417.91853177,368.14313694)(417.9367609,368.19782433)(417.95759419,368.23949091)
\curveto(417.97842748,368.28636582)(418.00446909,368.31761576)(418.03571903,368.33324072)
\curveto(418.07217729,368.35407401)(418.11123971,368.36449066)(418.15290629,368.36449066)
\lineto(419.12946681,368.36449066)
\lineto(419.12946681,370.03636227)
\curveto(419.12946681,370.07282053)(419.1372793,370.10667463)(419.15290427,370.13792457)
\curveto(419.17373756,370.16917451)(419.20759166,370.19521612)(419.25446656,370.21604941)
\curveto(419.30654979,370.24209102)(419.37425799,370.26032015)(419.45759115,370.2707368)
\curveto(419.54092431,370.28115345)(419.64509077,370.28636177)(419.77009052,370.28636177)
\curveto(419.90029859,370.28636177)(420.0070692,370.28115345)(420.09040237,370.2707368)
\curveto(420.17373553,370.26032015)(420.23883957,370.24209102)(420.28571447,370.21604941)
\curveto(420.33258938,370.19521612)(420.36644348,370.16917451)(420.38727677,370.13792457)
\curveto(420.40811006,370.10667463)(420.4185267,370.07282053)(420.4185267,370.03636227)
\lineto(420.4185267,368.36449066)
\lineto(422.22321055,368.36449066)
\curveto(422.26487713,368.36449066)(422.30133539,368.35407401)(422.33258533,368.33324072)
\curveto(422.36383526,368.31761576)(422.38987688,368.28636582)(422.41071017,368.23949091)
\curveto(422.43675178,368.19782433)(422.45498091,368.14313694)(422.46539756,368.07542875)
\curveto(422.4758142,368.00772055)(422.48102253,367.92438738)(422.48102253,367.82542925)
\curveto(422.48102253,367.63792963)(422.45758507,367.50251324)(422.41071017,367.41918007)
\curveto(422.36383526,367.33584691)(422.30133539,367.29418033)(422.22321055,367.29418033)
\lineto(420.4185267,367.29418033)
\lineto(420.4185267,363.37231327)
\curveto(420.4185267,362.88793925)(420.48883906,362.52075249)(420.62946378,362.270753)
\curveto(420.77529681,362.02596183)(421.03310879,361.90356624)(421.40289971,361.90356624)
\curveto(421.52269113,361.90356624)(421.62946175,361.91398289)(421.72321156,361.93481618)
\curveto(421.81696137,361.96085779)(421.90029454,361.98689941)(421.97321106,362.01294102)
\curveto(422.04612757,362.03898264)(422.10862745,362.06242009)(422.16071068,362.08325338)
\curveto(422.2127939,362.10929499)(422.25966881,362.1223158)(422.30133539,362.1223158)
\curveto(422.32737701,362.1223158)(422.35081446,362.11450332)(422.37164775,362.09887835)
\curveto(422.39768936,362.0884617)(422.41591849,362.06502425)(422.42633514,362.02856599)
\curveto(422.44196011,361.99210773)(422.45498091,361.94262866)(422.46539756,361.88012879)
\curveto(422.4758142,361.81762892)(422.48102253,361.73950408)(422.48102253,361.64575427)
\closepath
}
}
{
\newrgbcolor{curcolor}{1 1 1}
\pscustom[linestyle=none,fillstyle=solid,fillcolor=curcolor]
{
\newpath
\moveto(430.11406943,363.82543735)
\curveto(430.11406943,363.36189662)(430.03334043,362.94262664)(429.87188242,362.5676274)
\curveto(429.71042442,362.19783648)(429.47865405,361.88012879)(429.17657133,361.61450433)
\curveto(428.87448861,361.34887987)(428.50209353,361.14315112)(428.05938609,360.99731808)
\curveto(427.61667866,360.85669336)(427.11667967,360.78638101)(426.55938913,360.78638101)
\curveto(426.22084815,360.78638101)(425.90314046,360.81502678)(425.60626606,360.87231833)
\curveto(425.31459999,360.92440156)(425.05418385,360.98690143)(424.82501764,361.05981795)
\curveto(424.59585144,361.13794279)(424.40574766,361.21606764)(424.2547063,361.29419248)
\curveto(424.10366494,361.37752564)(424.00731097,361.43742135)(423.96564438,361.47387961)
\curveto(423.92918612,361.51033787)(423.90054035,361.54679613)(423.87970706,361.58325439)
\curveto(423.85887377,361.61971265)(423.84064464,361.6639834)(423.82501967,361.71606662)
\curveto(423.8093947,361.76814985)(423.79637389,361.83064972)(423.78595725,361.90356624)
\curveto(423.78074892,361.98169109)(423.77814476,362.0754409)(423.77814476,362.18481567)
\curveto(423.77814476,362.37231529)(423.79637389,362.50252336)(423.83283215,362.57543988)
\curveto(423.86929041,362.6483564)(423.92137364,362.68481466)(423.98908184,362.68481466)
\curveto(424.03595674,362.68481466)(424.12710239,362.64314808)(424.26251878,362.55981491)
\curveto(424.4031435,362.47648175)(424.58022647,362.3853361)(424.79376771,362.28637797)
\curveto(425.01251726,362.19262816)(425.26512092,362.10408667)(425.55157867,362.02075351)
\curveto(425.84324475,361.93742034)(426.16616076,361.89575376)(426.52032671,361.89575376)
\curveto(426.86407601,361.89575376)(427.16615874,361.9400245)(427.42657488,362.02856599)
\curveto(427.68699101,362.11710748)(427.90574057,362.24210722)(428.08282355,362.40356523)
\curveto(428.25990652,362.56502324)(428.39271875,362.75512702)(428.48126024,362.97387658)
\curveto(428.56980173,363.19783446)(428.61407247,363.44002147)(428.61407247,363.7004376)
\curveto(428.61407247,363.98689536)(428.55678092,364.24470734)(428.44219782,364.47387354)
\curveto(428.33282304,364.70303974)(428.16876087,364.90095601)(427.95001132,365.06762234)
\curveto(427.73647008,365.23428867)(427.47084562,365.36189257)(427.15313793,365.45043406)
\curveto(426.84063856,365.53897555)(426.48386845,365.58324629)(426.0828276,365.58324629)
\lineto(425.12189204,365.58324629)
\curveto(425.08022546,365.58324629)(425.03855888,365.59105878)(424.9968923,365.60668374)
\curveto(424.96043404,365.62751704)(424.92657994,365.65876697)(424.89533,365.70043355)
\curveto(424.86928839,365.74210014)(424.84585093,365.79678753)(424.82501764,365.86449572)
\curveto(424.80939268,365.93220392)(424.80158019,366.02074541)(424.80158019,366.13012018)
\curveto(424.80158019,366.22907832)(424.80939268,366.30980732)(424.82501764,366.37230719)
\curveto(424.84064261,366.44001539)(424.8614759,366.49209862)(424.88751752,366.52855688)
\curveto(424.91876745,366.57022346)(424.95262155,366.59886924)(424.98907981,366.6144942)
\curveto(425.02553807,366.63532749)(425.06720465,366.64574414)(425.11407956,366.64574414)
\lineto(425.99689027,366.64574414)
\curveto(426.34063957,366.64574414)(426.64793062,366.69001488)(426.9187634,366.77855637)
\curveto(427.19480451,366.86709786)(427.42657488,366.99470177)(427.6140745,367.1613681)
\curveto(427.80678244,367.32803443)(427.95261548,367.52595069)(428.05157361,367.75511689)
\curveto(428.15574007,367.98949142)(428.20782329,368.24990756)(428.20782329,368.53636531)
\curveto(428.20782329,368.74469822)(428.1739692,368.94261449)(428.106261,369.13011411)
\curveto(428.0385528,369.32282205)(427.93699051,369.48948838)(427.80157412,369.6301131)
\curveto(427.66615772,369.77073781)(427.49167891,369.88011259)(427.27813768,369.95823743)
\curveto(427.06980476,370.0415706)(426.82761775,370.08323718)(426.55157665,370.08323718)
\curveto(426.24949393,370.08323718)(425.97084866,370.03636227)(425.71564084,369.94261246)
\curveto(425.46564135,369.85407098)(425.24168347,369.75771701)(425.0437672,369.65355055)
\curveto(424.84585093,369.54938409)(424.68178877,369.45042596)(424.5515807,369.35667615)
\curveto(424.42137263,369.26813466)(424.33022698,369.22386392)(424.27814375,369.22386392)
\curveto(424.24168549,369.22386392)(424.20783139,369.22907224)(424.17658146,369.23948889)
\curveto(424.15053984,369.25511386)(424.12710239,369.28115547)(424.1062691,369.31761373)
\curveto(424.09064413,369.35928031)(424.07762332,369.4139677)(424.06720668,369.4816759)
\curveto(424.06199836,369.55459242)(424.05939419,369.64834223)(424.05939419,369.76292533)
\curveto(424.05939419,369.84105017)(424.06199836,369.90875837)(424.06720668,369.96604992)
\curveto(424.072415,370.02854979)(424.08283165,370.08063302)(424.09845661,370.1222996)
\curveto(424.11408158,370.16917451)(424.13231071,370.21084109)(424.153144,370.24729935)
\curveto(424.1739773,370.28375761)(424.20783139,370.32542419)(424.2547063,370.37229909)
\curveto(424.30678953,370.42438232)(424.40835182,370.49729884)(424.55939318,370.59104865)
\curveto(424.71043454,370.68479846)(424.89533,370.77594411)(425.11407956,370.8644856)
\curveto(425.33803744,370.95823541)(425.59324525,371.03636025)(425.87970301,371.09886012)
\curveto(426.17136908,371.16136)(426.48386845,371.19260993)(426.81720111,371.19260993)
\curveto(427.28595016,371.19260993)(427.69740766,371.13271422)(428.05157361,371.0129228)
\curveto(428.40573956,370.89313137)(428.7000098,370.72386088)(428.93438432,370.50511133)
\curveto(429.17396717,370.28636177)(429.35105014,370.02334147)(429.46563325,369.71605042)
\curveto(429.58542467,369.4139677)(429.64532038,369.07803088)(429.64532038,368.70823996)
\curveto(429.64532038,368.39053227)(429.6036538,368.09365788)(429.52032064,367.81761677)
\curveto(429.43698747,367.54678398)(429.31459189,367.30459697)(429.15313388,367.09105574)
\curveto(428.99167587,366.88272283)(428.79375961,366.70303569)(428.55938508,366.55199433)
\curveto(428.32501056,366.40616129)(428.05417777,366.304599)(427.74688673,366.24730745)
\lineto(427.74688673,366.23168248)
\curveto(428.09584435,366.19522422)(428.41355204,366.10668273)(428.7000098,365.96605802)
\curveto(428.99167587,365.83064162)(429.24167537,365.65876697)(429.45000828,365.45043406)
\curveto(429.65834119,365.24210115)(429.8197992,364.99730998)(429.9343823,364.71606055)
\curveto(430.05417372,364.44001944)(430.11406943,364.14314504)(430.11406943,363.82543735)
\closepath
}
}
{
\newrgbcolor{curcolor}{0 0 0}
\pscustom[linestyle=none,fillstyle=solid,fillcolor=curcolor,opacity=0]
{
\newpath
\moveto(321.29861496,560.46753116)
\lineto(409.89688699,560.46753116)
\lineto(409.89688699,523.68020408)
\lineto(321.29861496,523.68020408)
\closepath
}
}
{
\newrgbcolor{curcolor}{1 1 1}
\pscustom[linestyle=none,fillstyle=solid,fillcolor=curcolor]
{
\newpath
\moveto(337.25952969,543.34975631)
\curveto(337.25952969,543.32892302)(337.25692553,543.30548557)(337.25171721,543.27944396)
\curveto(337.25171721,543.25861067)(337.24911304,543.23517321)(337.24390472,543.2091316)
\curveto(337.2386964,543.18308999)(337.23088391,543.15444421)(337.22046727,543.12319427)
\curveto(337.21525895,543.09194434)(337.20744646,543.05809024)(337.19702982,543.02163198)
\lineto(334.79859717,536.34195801)
\curveto(334.77776388,536.28466645)(334.74911811,536.23779155)(334.71265985,536.20133329)
\curveto(334.68140991,536.16487503)(334.63193084,536.13622925)(334.56422265,536.11539596)
\curveto(334.49651445,536.09456267)(334.41057713,536.08154187)(334.30641067,536.07633354)
\curveto(334.20224421,536.0659169)(334.07203614,536.06070857)(333.91578646,536.06070857)
\curveto(333.75953678,536.06070857)(333.62932871,536.0659169)(333.52516225,536.07633354)
\curveto(333.4209958,536.08675019)(333.33505847,536.10237516)(333.26735027,536.12320845)
\curveto(333.2048504,536.14404174)(333.15537133,536.17268751)(333.11891307,536.20914577)
\curveto(333.08245481,536.24560403)(333.05380904,536.28987478)(333.03297575,536.34195801)
\lineto(330.64235559,543.02163198)
\lineto(330.5876682,543.18569415)
\curveto(330.57725155,543.23256905)(330.56943907,543.26642315)(330.56423075,543.28725644)
\lineto(330.56423075,543.34975631)
\curveto(330.56423075,543.3914229)(330.57464739,543.42788116)(330.59548068,543.45913109)
\curveto(330.61631398,543.49038103)(330.65016807,543.51381848)(330.69704298,543.52944345)
\curveto(330.74912621,543.55027674)(330.81423024,543.56329755)(330.89235508,543.56850587)
\curveto(330.97568825,543.57371419)(331.07725054,543.57631836)(331.19704197,543.57631836)
\curveto(331.34808333,543.57631836)(331.46787475,543.57111003)(331.55641624,543.56069339)
\curveto(331.65016605,543.55548506)(331.72047841,543.54246426)(331.76735331,543.52163097)
\curveto(331.81943654,543.50079768)(331.8558948,543.47475606)(331.87672809,543.44350612)
\curveto(331.9027697,543.41225619)(331.92620716,543.37319377)(331.94704045,543.32631886)
\lineto(333.93141143,537.5294556)
\lineto(333.96266137,537.43570579)
\lineto(333.98609882,537.5294556)
\lineto(335.94703235,543.32631886)
\curveto(335.95744899,543.37319377)(335.97567812,543.41225619)(336.00171974,543.44350612)
\curveto(336.02776135,543.47475606)(336.06421961,543.50079768)(336.11109451,543.52163097)
\curveto(336.16317774,543.54246426)(336.23088594,543.55548506)(336.3142191,543.56069339)
\curveto(336.40276059,543.57111003)(336.51734369,543.57631836)(336.65796841,543.57631836)
\curveto(336.77775983,543.57631836)(336.87671796,543.57371419)(336.95484281,543.56850587)
\curveto(337.03296765,543.56329755)(337.09286336,543.55027674)(337.13452994,543.52944345)
\curveto(337.18140485,543.50861016)(337.21265478,543.48256855)(337.22827975,543.45131861)
\curveto(337.24911304,543.42527699)(337.25952969,543.3914229)(337.25952969,543.34975631)
\closepath
}
}
{
\newrgbcolor{curcolor}{1 1 1}
\pscustom[linestyle=none,fillstyle=solid,fillcolor=curcolor]
{
\newpath
\moveto(344.58932737,536.30289558)
\curveto(344.58932737,536.261229)(344.57891073,536.22477074)(344.55807744,536.19352081)
\curveto(344.54245247,536.16227087)(344.51120253,536.13622925)(344.46432763,536.11539596)
\curveto(344.42266104,536.099771)(344.36536949,536.08675019)(344.29245297,536.07633354)
\curveto(344.21953646,536.0659169)(344.13099497,536.06070857)(344.02682851,536.06070857)
\curveto(343.91745373,536.06070857)(343.82630808,536.0659169)(343.75339157,536.07633354)
\curveto(343.68047505,536.08675019)(343.62057933,536.099771)(343.57370443,536.11539596)
\curveto(343.52682952,536.13622925)(343.49297543,536.16227087)(343.47214214,536.19352081)
\curveto(343.45130884,536.22477074)(343.4408922,536.261229)(343.4408922,536.30289558)
\lineto(343.4408922,537.2325812)
\curveto(343.07110128,536.83154035)(342.68568539,536.51904098)(342.28464454,536.2950831)
\curveto(341.88881201,536.07112522)(341.45391706,535.95914628)(340.97995968,535.95914628)
\curveto(340.46433573,535.95914628)(340.02423245,536.05810441)(339.65964986,536.25602068)
\curveto(339.29506726,536.45914527)(338.99819286,536.72997805)(338.76902666,537.06851903)
\curveto(338.54506878,537.41226834)(338.38100661,537.81330919)(338.27684016,538.2716416)
\curveto(338.1726737,538.73518233)(338.12059047,539.22216051)(338.12059047,539.73257614)
\curveto(338.12059047,540.33674158)(338.18569451,540.88101131)(338.31590258,541.36538533)
\curveto(338.44611065,541.85496767)(338.63881859,542.2716335)(338.89402641,542.6153828)
\curveto(339.14923422,542.95913211)(339.46433775,543.22215241)(339.83933699,543.4044437)
\curveto(340.21954456,543.59194332)(340.65704367,543.68569313)(341.15183433,543.68569313)
\curveto(341.56329183,543.68569313)(341.93829107,543.59454749)(342.27683206,543.41225619)
\curveto(342.62058136,543.23517321)(342.95912234,542.97215291)(343.292455,542.62319529)
\lineto(343.292455,546.70912451)
\curveto(343.292455,546.74558277)(343.30026748,546.77943687)(343.31589245,546.81068681)
\curveto(343.33672574,546.84714507)(343.373184,546.87318668)(343.42526723,546.88881165)
\curveto(343.47735046,546.90964494)(343.54245449,546.92526991)(343.62057933,546.93568655)
\curveto(343.7039125,546.95131152)(343.80807896,546.95912401)(343.9330787,546.95912401)
\curveto(344.06328677,546.95912401)(344.17005739,546.95131152)(344.25339055,546.93568655)
\curveto(344.33672372,546.92526991)(344.40182775,546.90964494)(344.44870266,546.88881165)
\curveto(344.49557756,546.87318668)(344.52943166,546.84714507)(344.55026495,546.81068681)
\curveto(344.57630657,546.77943687)(344.58932737,546.74558277)(344.58932737,546.70912451)
\closepath
\moveto(343.292455,541.24819807)
\curveto(342.94349737,541.68048886)(342.60495639,542.0086132)(342.27683206,542.23257108)
\curveto(341.95391604,542.46173728)(341.61537506,542.57632038)(341.26120911,542.57632038)
\curveto(340.93308478,542.57632038)(340.65443951,542.49819554)(340.42527331,542.34194586)
\curveto(340.1961071,542.18569617)(340.00860748,541.97996742)(339.86277444,541.7247596)
\curveto(339.72214973,541.46955179)(339.61798327,541.18048987)(339.55027508,540.85757386)
\curveto(339.4877752,540.53465785)(339.45652527,540.20653351)(339.45652527,539.87320085)
\curveto(339.45652527,539.5190349)(339.48256688,539.17268144)(339.53465011,538.83414046)
\curveto(339.59194166,538.49559948)(339.68829563,538.19351676)(339.82371202,537.92789229)
\curveto(339.95912842,537.66747615)(340.13881555,537.45653908)(340.36277343,537.29508107)
\curveto(340.58673131,537.13883139)(340.86798074,537.06070655)(341.20652172,537.06070655)
\curveto(341.37839638,537.06070655)(341.54245854,537.084144)(341.69870823,537.13101891)
\curveto(341.86016623,537.17789381)(342.0242284,537.25601865)(342.19089473,537.36539343)
\curveto(342.35756106,537.47476821)(342.53203987,537.61539293)(342.71433117,537.78726758)
\curveto(342.89662247,537.96435055)(343.08933041,538.18049595)(343.292455,538.43570377)
\closepath
}
}
{
\newrgbcolor{curcolor}{1 1 1}
\pscustom[linestyle=none,fillstyle=solid,fillcolor=curcolor]
{
\newpath
\moveto(352.96587301,540.12320035)
\curveto(352.96587301,539.92007576)(352.91378978,539.77424272)(352.80962332,539.68570123)
\curveto(352.71066519,539.60236807)(352.59608209,539.56070149)(352.46587402,539.56070149)
\lineto(347.85650835,539.56070149)
\curveto(347.85650835,539.17007728)(347.89557077,538.81851549)(347.97369562,538.50601612)
\curveto(348.05182046,538.19351676)(348.18202853,537.92528813)(348.36431982,537.70133025)
\curveto(348.54661112,537.47737237)(348.78358981,537.30549772)(349.07525588,537.1857063)
\curveto(349.36692196,537.06591487)(349.72369207,537.00601916)(350.14556622,537.00601916)
\curveto(350.47889888,537.00601916)(350.77577327,537.03206077)(351.03618941,537.084144)
\curveto(351.29660555,537.14143555)(351.52056343,537.20393543)(351.70806305,537.27164362)
\curveto(351.900771,537.33935182)(352.05702068,537.39924753)(352.1768121,537.45133076)
\curveto(352.30181185,537.50862231)(352.39556166,537.53726808)(352.45806153,537.53726808)
\curveto(352.49451979,537.53726808)(352.52576973,537.52685144)(352.55181134,537.50601815)
\curveto(352.58306128,537.49039318)(352.60649873,537.46435157)(352.6221237,537.42789331)
\curveto(352.63774867,537.39143505)(352.64816532,537.33935182)(352.65337364,537.27164362)
\curveto(352.66379028,537.20914375)(352.66899861,537.13101891)(352.66899861,537.0372691)
\curveto(352.66899861,536.9695609)(352.66639445,536.90966519)(352.66118612,536.85758196)
\curveto(352.6559778,536.81070706)(352.64816532,536.76643631)(352.63774867,536.72476973)
\curveto(352.63254035,536.68831147)(352.61951954,536.65445737)(352.59868625,536.62320744)
\curveto(352.58306128,536.5919575)(352.55962383,536.56070756)(352.52837389,536.52945763)
\curveto(352.50233228,536.50341601)(352.41899911,536.45654111)(352.2783744,536.38883291)
\curveto(352.13774968,536.32633304)(351.95545839,536.26383316)(351.73150051,536.20133329)
\curveto(351.50754263,536.13883342)(351.24712649,536.08414603)(350.95025209,536.03727112)
\curveto(350.65858601,535.98518789)(350.34608664,535.95914628)(350.01275399,535.95914628)
\curveto(349.43463016,535.95914628)(348.92681869,536.03987528)(348.48931957,536.20133329)
\curveto(348.05702878,536.3627913)(347.69244618,536.60237414)(347.39557179,536.92008183)
\curveto(347.09869739,537.23778952)(346.87473951,537.63622622)(346.72369815,538.11539191)
\curveto(346.57265679,538.59455761)(346.49713611,539.15184815)(346.49713611,539.78726353)
\curveto(346.49713611,540.39142897)(346.57526095,540.93309454)(346.73151063,541.41226024)
\curveto(346.88776031,541.89663426)(347.11171819,542.3054876)(347.40338427,542.63882025)
\curveto(347.70025867,542.97736123)(348.05702878,543.23517321)(348.4736946,543.41225619)
\curveto(348.89036043,543.59454749)(349.35650531,543.68569313)(349.87212927,543.68569313)
\curveto(350.42421149,543.68569313)(350.89296054,543.59715165)(351.27837642,543.42006867)
\curveto(351.66900063,543.2429857)(351.98931248,543.00340285)(352.23931198,542.70132013)
\curveto(352.48931147,542.40444573)(352.67160277,542.05288394)(352.78618587,541.64663476)
\curveto(352.90597729,541.24559391)(352.96587301,540.81590728)(352.96587301,540.35757487)
\closepath
\moveto(351.66900063,540.50601207)
\curveto(351.6846256,541.18309404)(351.53358424,541.71434296)(351.21587655,542.09975885)
\curveto(350.90337718,542.48517473)(350.43723229,542.67788267)(349.81744188,542.67788267)
\curveto(349.49973419,542.67788267)(349.22108892,542.61798696)(348.98150607,542.49819554)
\curveto(348.74192323,542.37840411)(348.5414028,542.21955027)(348.37994479,542.021634)
\curveto(348.21848679,541.82371774)(348.09348704,541.59194737)(348.00494555,541.32632291)
\curveto(347.91640406,541.06590677)(347.866925,540.79246983)(347.85650835,540.50601207)
\closepath
}
}
{
\newrgbcolor{curcolor}{1 1 1}
\pscustom[linestyle=none,fillstyle=solid,fillcolor=curcolor]
{
\newpath
\moveto(360.619483,543.34975631)
\curveto(360.619483,543.32892302)(360.61687883,543.30548557)(360.61167051,543.27944396)
\curveto(360.61167051,543.25861067)(360.60906635,543.23517321)(360.60385803,543.2091316)
\curveto(360.5986497,543.18308999)(360.59083722,543.15444421)(360.58042057,543.12319427)
\curveto(360.57521225,543.09194434)(360.56739977,543.05809024)(360.55698312,543.02163198)
\lineto(358.15855048,536.34195801)
\curveto(358.13771719,536.28466645)(358.10907141,536.23779155)(358.07261315,536.20133329)
\curveto(358.04136322,536.16487503)(357.99188415,536.13622925)(357.92417595,536.11539596)
\curveto(357.85646776,536.09456267)(357.77053043,536.08154187)(357.66636398,536.07633354)
\curveto(357.56219752,536.0659169)(357.43198945,536.06070857)(357.27573977,536.06070857)
\curveto(357.11949008,536.06070857)(356.98928201,536.0659169)(356.88511556,536.07633354)
\curveto(356.7809491,536.08675019)(356.69501178,536.10237516)(356.62730358,536.12320845)
\curveto(356.56480371,536.14404174)(356.51532464,536.17268751)(356.47886638,536.20914577)
\curveto(356.44240812,536.24560403)(356.41376235,536.28987478)(356.39292905,536.34195801)
\lineto(354.0023089,543.02163198)
\lineto(353.94762151,543.18569415)
\curveto(353.93720486,543.23256905)(353.92939238,543.26642315)(353.92418405,543.28725644)
\lineto(353.92418405,543.34975631)
\curveto(353.92418405,543.3914229)(353.9346007,543.42788116)(353.95543399,543.45913109)
\curveto(353.97626728,543.49038103)(354.01012138,543.51381848)(354.05699628,543.52944345)
\curveto(354.10907951,543.55027674)(354.17418355,543.56329755)(354.25230839,543.56850587)
\curveto(354.33564155,543.57371419)(354.43720385,543.57631836)(354.55699527,543.57631836)
\curveto(354.70803663,543.57631836)(354.82782806,543.57111003)(354.91636954,543.56069339)
\curveto(355.01011935,543.55548506)(355.08043171,543.54246426)(355.12730662,543.52163097)
\curveto(355.17938985,543.50079768)(355.2158481,543.47475606)(355.2366814,543.44350612)
\curveto(355.26272301,543.41225619)(355.28616046,543.37319377)(355.30699375,543.32631886)
\lineto(357.29136474,537.5294556)
\lineto(357.32261467,537.43570579)
\lineto(357.34605212,537.5294556)
\lineto(359.30698565,543.32631886)
\curveto(359.3174023,543.37319377)(359.33563143,543.41225619)(359.36167304,543.44350612)
\curveto(359.38771466,543.47475606)(359.42417292,543.50079768)(359.47104782,543.52163097)
\curveto(359.52313105,543.54246426)(359.59083925,543.55548506)(359.67417241,543.56069339)
\curveto(359.7627139,543.57111003)(359.877297,543.57631836)(360.01792171,543.57631836)
\curveto(360.13771314,543.57631836)(360.23667127,543.57371419)(360.31479611,543.56850587)
\curveto(360.39292095,543.56329755)(360.45281667,543.55027674)(360.49448325,543.52944345)
\curveto(360.54135815,543.50861016)(360.57260809,543.48256855)(360.58823306,543.45131861)
\curveto(360.60906635,543.42527699)(360.619483,543.3914229)(360.619483,543.34975631)
\closepath
}
}
{
\newrgbcolor{curcolor}{1 1 1}
\pscustom[linestyle=none,fillstyle=solid,fillcolor=curcolor]
{
\newpath
\moveto(368.11671742,536.6388324)
\curveto(368.11671742,536.53987427)(368.10890493,536.45654111)(368.09327996,536.38883291)
\curveto(368.077655,536.32112471)(368.0568217,536.26643732)(368.03078009,536.22477074)
\curveto(368.00473848,536.18310416)(367.97348854,536.15185422)(367.93703028,536.13102093)
\curveto(367.90578034,536.11539596)(367.87192625,536.10758348)(367.83546799,536.10758348)
\lineto(362.55422868,536.10758348)
\curveto(362.51777042,536.10758348)(362.48391632,536.11539596)(362.45266639,536.13102093)
\curveto(362.42141645,536.15185422)(362.39016651,536.18310416)(362.35891658,536.22477074)
\curveto(362.33287496,536.26643732)(362.31204167,536.32112471)(362.2964167,536.38883291)
\curveto(362.28079173,536.45654111)(362.27297925,536.53987427)(362.27297925,536.6388324)
\curveto(362.27297925,536.73258221)(362.28079173,536.81331122)(362.2964167,536.88101941)
\curveto(362.31204167,536.94872761)(362.3302708,537.003415)(362.35110409,537.04508158)
\curveto(362.37714571,537.09195649)(362.40579148,537.12581058)(362.43704142,537.14664388)
\curveto(362.47349968,537.17268549)(362.5125621,537.1857063)(362.55422868,537.1857063)
\lineto(364.66359941,537.1857063)
\lineto(364.66359941,544.87319073)
\lineto(362.71047836,543.70913059)
\curveto(362.61152023,543.65704736)(362.53079123,543.62579742)(362.46829136,543.61538078)
\curveto(362.4109998,543.60496413)(362.3641249,543.61538078)(362.32766664,543.64663071)
\curveto(362.29120838,543.68308897)(362.26516677,543.74038052)(362.2495418,543.81850536)
\curveto(362.23912515,543.89663021)(362.23391683,543.99558834)(362.23391683,544.11537976)
\curveto(362.23391683,544.20392125)(362.23652099,544.27944193)(362.24172931,544.34194181)
\curveto(362.25214596,544.40444168)(362.26516677,544.45652491)(362.28079173,544.49819149)
\curveto(362.2964167,544.53985807)(362.31724999,544.57631633)(362.34329161,544.60756627)
\curveto(362.37454154,544.6388162)(362.41360397,544.67006614)(362.46047887,544.70131608)
\lineto(364.78859916,546.19350056)
\curveto(364.80943245,546.20912552)(364.83547406,546.22214633)(364.866724,546.23256298)
\curveto(364.89797393,546.24297962)(364.93703636,546.25339627)(364.98391126,546.26381291)
\curveto(365.03078617,546.27422956)(365.08547356,546.27943788)(365.14797343,546.27943788)
\curveto(365.2104733,546.2846462)(365.28859814,546.28725037)(365.38234795,546.28725037)
\curveto(365.5073477,546.28725037)(365.61151416,546.28204204)(365.69484732,546.2716254)
\curveto(365.77818049,546.26120875)(365.84328452,546.24558378)(365.89015943,546.22475049)
\curveto(365.93703433,546.20912552)(365.96828427,546.18568807)(365.98390924,546.15443813)
\curveto(365.9995342,546.12839652)(366.00734669,546.09975075)(366.00734669,546.06850081)
\lineto(366.00734669,537.1857063)
\lineto(367.83546799,537.1857063)
\curveto(367.87713457,537.1857063)(367.91619699,537.17268549)(367.95265525,537.14664388)
\curveto(367.98911351,537.12581058)(368.01775928,537.09195649)(368.03859258,537.04508158)
\curveto(368.06463419,537.003415)(368.08286332,536.94872761)(368.09327996,536.88101941)
\curveto(368.10890493,536.81331122)(368.11671742,536.73258221)(368.11671742,536.6388324)
\closepath
}
}
{
\newrgbcolor{curcolor}{0 0 0}
\pscustom[linestyle=none,fillstyle=solid,fillcolor=curcolor,opacity=0]
{
\newpath
\moveto(321.29861496,489.15680942)
\lineto(409.89688699,489.15680942)
\lineto(409.89688699,452.36948234)
\lineto(321.29861496,452.36948234)
\closepath
}
}
{
\newrgbcolor{curcolor}{1 1 1}
\pscustom[linestyle=none,fillstyle=solid,fillcolor=curcolor]
{
\newpath
\moveto(337.25952969,472.03903072)
\curveto(337.25952969,472.01819743)(337.25692553,471.99475998)(337.25171721,471.96871836)
\curveto(337.25171721,471.94788507)(337.24911304,471.92444762)(337.24390472,471.898406)
\curveto(337.2386964,471.87236439)(337.23088391,471.84371861)(337.22046727,471.81246868)
\curveto(337.21525895,471.78121874)(337.20744646,471.74736464)(337.19702982,471.71090638)
\lineto(334.79859717,465.03123241)
\curveto(334.77776388,464.97394086)(334.74911811,464.92706595)(334.71265985,464.89060769)
\curveto(334.68140991,464.85414943)(334.63193084,464.82550366)(334.56422265,464.80467037)
\curveto(334.49651445,464.78383708)(334.41057713,464.77081627)(334.30641067,464.76560795)
\curveto(334.20224421,464.7551913)(334.07203614,464.74998298)(333.91578646,464.74998298)
\curveto(333.75953678,464.74998298)(333.62932871,464.7551913)(333.52516225,464.76560795)
\curveto(333.4209958,464.77602459)(333.33505847,464.79164956)(333.26735027,464.81248285)
\curveto(333.2048504,464.83331614)(333.15537133,464.86196192)(333.11891307,464.89842018)
\curveto(333.08245481,464.93487844)(333.05380904,464.97914918)(333.03297575,465.03123241)
\lineto(330.64235559,471.71090638)
\lineto(330.5876682,471.87496855)
\curveto(330.57725155,471.92184346)(330.56943907,471.95569755)(330.56423075,471.97653085)
\lineto(330.56423075,472.03903072)
\curveto(330.56423075,472.0806973)(330.57464739,472.11715556)(330.59548068,472.1484055)
\curveto(330.61631398,472.17965543)(330.65016807,472.20309289)(330.69704298,472.21871785)
\curveto(330.74912621,472.23955115)(330.81423024,472.25257195)(330.89235508,472.25778028)
\curveto(330.97568825,472.2629886)(331.07725054,472.26559276)(331.19704197,472.26559276)
\curveto(331.34808333,472.26559276)(331.46787475,472.26038444)(331.55641624,472.24996779)
\curveto(331.65016605,472.24475947)(331.72047841,472.23173866)(331.76735331,472.21090537)
\curveto(331.81943654,472.19007208)(331.8558948,472.16403047)(331.87672809,472.13278053)
\curveto(331.9027697,472.10153059)(331.92620716,472.06246817)(331.94704045,472.01559327)
\lineto(333.93141143,466.21873)
\lineto(333.96266137,466.12498019)
\lineto(333.98609882,466.21873)
\lineto(335.94703235,472.01559327)
\curveto(335.95744899,472.06246817)(335.97567812,472.10153059)(336.00171974,472.13278053)
\curveto(336.02776135,472.16403047)(336.06421961,472.19007208)(336.11109451,472.21090537)
\curveto(336.16317774,472.23173866)(336.23088594,472.24475947)(336.3142191,472.24996779)
\curveto(336.40276059,472.26038444)(336.51734369,472.26559276)(336.65796841,472.26559276)
\curveto(336.77775983,472.26559276)(336.87671796,472.2629886)(336.95484281,472.25778028)
\curveto(337.03296765,472.25257195)(337.09286336,472.23955115)(337.13452994,472.21871785)
\curveto(337.18140485,472.19788456)(337.21265478,472.17184295)(337.22827975,472.14059301)
\curveto(337.24911304,472.1145514)(337.25952969,472.0806973)(337.25952969,472.03903072)
\closepath
}
}
{
\newrgbcolor{curcolor}{1 1 1}
\pscustom[linestyle=none,fillstyle=solid,fillcolor=curcolor]
{
\newpath
\moveto(344.58932737,464.99216999)
\curveto(344.58932737,464.95050341)(344.57891073,464.91404515)(344.55807744,464.88279521)
\curveto(344.54245247,464.85154527)(344.51120253,464.82550366)(344.46432763,464.80467037)
\curveto(344.42266104,464.7890454)(344.36536949,464.77602459)(344.29245297,464.76560795)
\curveto(344.21953646,464.7551913)(344.13099497,464.74998298)(344.02682851,464.74998298)
\curveto(343.91745373,464.74998298)(343.82630808,464.7551913)(343.75339157,464.76560795)
\curveto(343.68047505,464.77602459)(343.62057933,464.7890454)(343.57370443,464.80467037)
\curveto(343.52682952,464.82550366)(343.49297543,464.85154527)(343.47214214,464.88279521)
\curveto(343.45130884,464.91404515)(343.4408922,464.95050341)(343.4408922,464.99216999)
\lineto(343.4408922,465.92185561)
\curveto(343.07110128,465.52081475)(342.68568539,465.20831538)(342.28464454,464.9843575)
\curveto(341.88881201,464.76039962)(341.45391706,464.64842068)(340.97995968,464.64842068)
\curveto(340.46433573,464.64842068)(340.02423245,464.74737882)(339.65964986,464.94529508)
\curveto(339.29506726,465.14841967)(338.99819286,465.41925246)(338.76902666,465.75779344)
\curveto(338.54506878,466.10154274)(338.38100661,466.5025836)(338.27684016,466.960916)
\curveto(338.1726737,467.42445673)(338.12059047,467.91143491)(338.12059047,468.42185054)
\curveto(338.12059047,469.02601599)(338.18569451,469.57028572)(338.31590258,470.05465974)
\curveto(338.44611065,470.54424208)(338.63881859,470.9609079)(338.89402641,471.30465721)
\curveto(339.14923422,471.64840651)(339.46433775,471.91142681)(339.83933699,472.09371811)
\curveto(340.21954456,472.28121773)(340.65704367,472.37496754)(341.15183433,472.37496754)
\curveto(341.56329183,472.37496754)(341.93829107,472.28382189)(342.27683206,472.10153059)
\curveto(342.62058136,471.92444762)(342.95912234,471.66142732)(343.292455,471.31246969)
\lineto(343.292455,475.39839892)
\curveto(343.292455,475.43485718)(343.30026748,475.46871127)(343.31589245,475.49996121)
\curveto(343.33672574,475.53641947)(343.373184,475.56246108)(343.42526723,475.57808605)
\curveto(343.47735046,475.59891934)(343.54245449,475.61454431)(343.62057933,475.62496096)
\curveto(343.7039125,475.64058593)(343.80807896,475.64839841)(343.9330787,475.64839841)
\curveto(344.06328677,475.64839841)(344.17005739,475.64058593)(344.25339055,475.62496096)
\curveto(344.33672372,475.61454431)(344.40182775,475.59891934)(344.44870266,475.57808605)
\curveto(344.49557756,475.56246108)(344.52943166,475.53641947)(344.55026495,475.49996121)
\curveto(344.57630657,475.46871127)(344.58932737,475.43485718)(344.58932737,475.39839892)
\closepath
\moveto(343.292455,469.93747247)
\curveto(342.94349737,470.36976327)(342.60495639,470.6978876)(342.27683206,470.92184548)
\curveto(341.95391604,471.15101168)(341.61537506,471.26559478)(341.26120911,471.26559478)
\curveto(340.93308478,471.26559478)(340.65443951,471.18746994)(340.42527331,471.03122026)
\curveto(340.1961071,470.87497058)(340.00860748,470.66924183)(339.86277444,470.41403401)
\curveto(339.72214973,470.15882619)(339.61798327,469.86976428)(339.55027508,469.54684827)
\curveto(339.4877752,469.22393225)(339.45652527,468.89580792)(339.45652527,468.56247526)
\curveto(339.45652527,468.20830931)(339.48256688,467.86195584)(339.53465011,467.52341486)
\curveto(339.59194166,467.18487388)(339.68829563,466.88279116)(339.82371202,466.6171667)
\curveto(339.95912842,466.35675056)(340.13881555,466.14581349)(340.36277343,465.98435548)
\curveto(340.58673131,465.8281058)(340.86798074,465.74998095)(341.20652172,465.74998095)
\curveto(341.37839638,465.74998095)(341.54245854,465.77341841)(341.69870823,465.82029331)
\curveto(341.86016623,465.86716822)(342.0242284,465.94529306)(342.19089473,466.05466784)
\curveto(342.35756106,466.16404262)(342.53203987,466.30466733)(342.71433117,466.47654198)
\curveto(342.89662247,466.65362496)(343.08933041,466.86977035)(343.292455,467.12497817)
\closepath
}
}
{
\newrgbcolor{curcolor}{1 1 1}
\pscustom[linestyle=none,fillstyle=solid,fillcolor=curcolor]
{
\newpath
\moveto(352.96587301,468.81247475)
\curveto(352.96587301,468.60935016)(352.91378978,468.46351713)(352.80962332,468.37497564)
\curveto(352.71066519,468.29164247)(352.59608209,468.24997589)(352.46587402,468.24997589)
\lineto(347.85650835,468.24997589)
\curveto(347.85650835,467.85935168)(347.89557077,467.50778989)(347.97369562,467.19529053)
\curveto(348.05182046,466.88279116)(348.18202853,466.61456254)(348.36431982,466.39060466)
\curveto(348.54661112,466.16664678)(348.78358981,465.99477213)(349.07525588,465.8749807)
\curveto(349.36692196,465.75518928)(349.72369207,465.69529356)(350.14556622,465.69529356)
\curveto(350.47889888,465.69529356)(350.77577327,465.72133518)(351.03618941,465.77341841)
\curveto(351.29660555,465.83070996)(351.52056343,465.89320983)(351.70806305,465.96091803)
\curveto(351.900771,466.02862622)(352.05702068,466.08852194)(352.1768121,466.14060516)
\curveto(352.30181185,466.19789671)(352.39556166,466.22654249)(352.45806153,466.22654249)
\curveto(352.49451979,466.22654249)(352.52576973,466.21612584)(352.55181134,466.19529255)
\curveto(352.58306128,466.17966758)(352.60649873,466.15362597)(352.6221237,466.11716771)
\curveto(352.63774867,466.08070945)(352.64816532,466.02862622)(352.65337364,465.96091803)
\curveto(352.66379028,465.89841815)(352.66899861,465.82029331)(352.66899861,465.7265435)
\curveto(352.66899861,465.65883531)(352.66639445,465.59893959)(352.66118612,465.54685637)
\curveto(352.6559778,465.49998146)(352.64816532,465.45571072)(352.63774867,465.41404413)
\curveto(352.63254035,465.37758587)(352.61951954,465.34373178)(352.59868625,465.31248184)
\curveto(352.58306128,465.2812319)(352.55962383,465.24998197)(352.52837389,465.21873203)
\curveto(352.50233228,465.19269042)(352.41899911,465.14581551)(352.2783744,465.07810731)
\curveto(352.13774968,465.01560744)(351.95545839,464.95310757)(351.73150051,464.89060769)
\curveto(351.50754263,464.82810782)(351.24712649,464.77342043)(350.95025209,464.72654553)
\curveto(350.65858601,464.6744623)(350.34608664,464.64842068)(350.01275399,464.64842068)
\curveto(349.43463016,464.64842068)(348.92681869,464.72914969)(348.48931957,464.89060769)
\curveto(348.05702878,465.0520657)(347.69244618,465.29164855)(347.39557179,465.60935624)
\curveto(347.09869739,465.92706393)(346.87473951,466.32550062)(346.72369815,466.80466632)
\curveto(346.57265679,467.28383201)(346.49713611,467.84112255)(346.49713611,468.47653793)
\curveto(346.49713611,469.08070338)(346.57526095,469.62236895)(346.73151063,470.10153464)
\curveto(346.88776031,470.58590866)(347.11171819,470.994762)(347.40338427,471.32809466)
\curveto(347.70025867,471.66663564)(348.05702878,471.92444762)(348.4736946,472.10153059)
\curveto(348.89036043,472.28382189)(349.35650531,472.37496754)(349.87212927,472.37496754)
\curveto(350.42421149,472.37496754)(350.89296054,472.28642605)(351.27837642,472.10934308)
\curveto(351.66900063,471.9322601)(351.98931248,471.69267725)(352.23931198,471.39059453)
\curveto(352.48931147,471.09372013)(352.67160277,470.74215834)(352.78618587,470.33590917)
\curveto(352.90597729,469.93486831)(352.96587301,469.50518168)(352.96587301,469.04684928)
\closepath
\moveto(351.66900063,469.19528648)
\curveto(351.6846256,469.87236844)(351.53358424,470.40361736)(351.21587655,470.78903325)
\curveto(350.90337718,471.17444914)(350.43723229,471.36715708)(349.81744188,471.36715708)
\curveto(349.49973419,471.36715708)(349.22108892,471.30726137)(348.98150607,471.18746994)
\curveto(348.74192323,471.06767852)(348.5414028,470.90882467)(348.37994479,470.71090841)
\curveto(348.21848679,470.51299214)(348.09348704,470.28122178)(348.00494555,470.01559732)
\curveto(347.91640406,469.75518118)(347.866925,469.48174423)(347.85650835,469.19528648)
\closepath
}
}
{
\newrgbcolor{curcolor}{1 1 1}
\pscustom[linestyle=none,fillstyle=solid,fillcolor=curcolor]
{
\newpath
\moveto(360.619483,472.03903072)
\curveto(360.619483,472.01819743)(360.61687883,471.99475998)(360.61167051,471.96871836)
\curveto(360.61167051,471.94788507)(360.60906635,471.92444762)(360.60385803,471.898406)
\curveto(360.5986497,471.87236439)(360.59083722,471.84371861)(360.58042057,471.81246868)
\curveto(360.57521225,471.78121874)(360.56739977,471.74736464)(360.55698312,471.71090638)
\lineto(358.15855048,465.03123241)
\curveto(358.13771719,464.97394086)(358.10907141,464.92706595)(358.07261315,464.89060769)
\curveto(358.04136322,464.85414943)(357.99188415,464.82550366)(357.92417595,464.80467037)
\curveto(357.85646776,464.78383708)(357.77053043,464.77081627)(357.66636398,464.76560795)
\curveto(357.56219752,464.7551913)(357.43198945,464.74998298)(357.27573977,464.74998298)
\curveto(357.11949008,464.74998298)(356.98928201,464.7551913)(356.88511556,464.76560795)
\curveto(356.7809491,464.77602459)(356.69501178,464.79164956)(356.62730358,464.81248285)
\curveto(356.56480371,464.83331614)(356.51532464,464.86196192)(356.47886638,464.89842018)
\curveto(356.44240812,464.93487844)(356.41376235,464.97914918)(356.39292905,465.03123241)
\lineto(354.0023089,471.71090638)
\lineto(353.94762151,471.87496855)
\curveto(353.93720486,471.92184346)(353.92939238,471.95569755)(353.92418405,471.97653085)
\lineto(353.92418405,472.03903072)
\curveto(353.92418405,472.0806973)(353.9346007,472.11715556)(353.95543399,472.1484055)
\curveto(353.97626728,472.17965543)(354.01012138,472.20309289)(354.05699628,472.21871785)
\curveto(354.10907951,472.23955115)(354.17418355,472.25257195)(354.25230839,472.25778028)
\curveto(354.33564155,472.2629886)(354.43720385,472.26559276)(354.55699527,472.26559276)
\curveto(354.70803663,472.26559276)(354.82782806,472.26038444)(354.91636954,472.24996779)
\curveto(355.01011935,472.24475947)(355.08043171,472.23173866)(355.12730662,472.21090537)
\curveto(355.17938985,472.19007208)(355.2158481,472.16403047)(355.2366814,472.13278053)
\curveto(355.26272301,472.10153059)(355.28616046,472.06246817)(355.30699375,472.01559327)
\lineto(357.29136474,466.21873)
\lineto(357.32261467,466.12498019)
\lineto(357.34605212,466.21873)
\lineto(359.30698565,472.01559327)
\curveto(359.3174023,472.06246817)(359.33563143,472.10153059)(359.36167304,472.13278053)
\curveto(359.38771466,472.16403047)(359.42417292,472.19007208)(359.47104782,472.21090537)
\curveto(359.52313105,472.23173866)(359.59083925,472.24475947)(359.67417241,472.24996779)
\curveto(359.7627139,472.26038444)(359.877297,472.26559276)(360.01792171,472.26559276)
\curveto(360.13771314,472.26559276)(360.23667127,472.2629886)(360.31479611,472.25778028)
\curveto(360.39292095,472.25257195)(360.45281667,472.23955115)(360.49448325,472.21871785)
\curveto(360.54135815,472.19788456)(360.57260809,472.17184295)(360.58823306,472.14059301)
\curveto(360.60906635,472.1145514)(360.619483,472.0806973)(360.619483,472.03903072)
\closepath
}
}
{
\newrgbcolor{curcolor}{1 1 1}
\pscustom[linestyle=none,fillstyle=solid,fillcolor=curcolor]
{
\newpath
\moveto(368.09327996,465.37498171)
\curveto(368.09327996,465.2812319)(368.08546748,465.19789874)(368.06984251,465.12498222)
\curveto(368.05942587,465.0520657)(368.04119674,464.98956583)(368.01515512,464.9374826)
\curveto(367.99432183,464.89060769)(367.96307189,464.85414943)(367.92140531,464.82810782)
\curveto(367.88494705,464.80727453)(367.84328047,464.79685788)(367.79640557,464.79685788)
\lineto(362.21047938,464.79685788)
\curveto(362.13756286,464.79685788)(362.07245882,464.80467037)(362.01516727,464.82029534)
\curveto(361.96308404,464.84112863)(361.91620914,464.87237856)(361.87454256,464.91404515)
\curveto(361.8380843,464.95571173)(361.80943852,465.01560744)(361.78860523,465.09373228)
\curveto(361.77298026,465.17185712)(361.76516778,465.2682111)(361.76516778,465.3827942)
\curveto(361.76516778,465.48696065)(361.76777194,465.5781063)(361.77298026,465.65623114)
\curveto(361.78339691,465.73435599)(361.80162604,465.80206418)(361.82766765,465.85935573)
\curveto(361.85370927,465.92185561)(361.8849592,465.98175132)(361.92141746,466.03904287)
\curveto(361.96308404,466.10154274)(362.01516727,466.16664678)(362.07766715,466.23435497)
\lineto(364.03860068,468.31247577)
\curveto(364.49172476,468.79164146)(364.85370319,469.22132809)(365.12453598,469.60153565)
\curveto(365.40057708,469.98174322)(365.61151416,470.32809668)(365.75734719,470.64059605)
\curveto(365.90838856,470.95309542)(366.00734669,471.23694901)(366.05422159,471.49215683)
\curveto(366.1010965,471.74736464)(366.12453395,471.98694749)(366.12453395,472.21090537)
\curveto(366.12453395,472.43486325)(366.08807569,472.64580032)(366.01515917,472.84371659)
\curveto(365.94224265,473.04684118)(365.83547204,473.22392415)(365.69484732,473.37496551)
\curveto(365.55943093,473.52600687)(365.38755628,473.6457983)(365.17922337,473.73433979)
\curveto(364.97089045,473.82288127)(364.73130761,473.86715202)(364.46047482,473.86715202)
\curveto(364.14276713,473.86715202)(363.85630938,473.82288127)(363.60110156,473.73433979)
\curveto(363.35110207,473.6457983)(363.12974835,473.54944433)(362.93704041,473.44527787)
\curveto(362.74954079,473.34111142)(362.59068694,473.24475744)(362.46047887,473.15621596)
\curveto(362.33547912,473.06767447)(362.24172931,473.02340373)(362.17922944,473.02340373)
\curveto(362.14277118,473.02340373)(362.10891708,473.03382037)(362.07766715,473.05465366)
\curveto(362.05162553,473.07548695)(362.02818808,473.10934105)(362.00735479,473.15621596)
\curveto(361.99172982,473.20309086)(361.97870901,473.26559073)(361.96829237,473.34371558)
\curveto(361.95787572,473.42184042)(361.9526674,473.51559023)(361.9526674,473.62496501)
\curveto(361.9526674,473.70308985)(361.95527156,473.77079805)(361.96047988,473.8280896)
\curveto(361.96568821,473.88538115)(361.97350069,473.93486021)(361.98391734,473.9765268)
\curveto(361.9995423,474.01819338)(362.01777143,474.0572558)(362.03860473,474.09371406)
\curveto(362.05943802,474.13017232)(362.1011046,474.17444306)(362.16360447,474.22652629)
\curveto(362.22610435,474.28381784)(362.33287496,474.35933852)(362.48391632,474.45308833)
\curveto(362.64016601,474.54683814)(362.83287395,474.63798379)(363.06204015,474.72652528)
\curveto(363.29641468,474.82027509)(363.55162249,474.89839993)(363.8276636,474.9608998)
\curveto(364.10891303,475.02339968)(364.40318327,475.05464961)(364.71047431,475.05464961)
\curveto(365.20005666,475.05464961)(365.62713913,474.98433725)(365.99172172,474.84371254)
\curveto(366.36151264,474.70829615)(366.66619952,474.52079653)(366.90578237,474.28121368)
\curveto(367.15057354,474.04163083)(367.33286484,473.76298556)(367.45265626,473.44527787)
\curveto(367.57244769,473.12757018)(367.6323434,472.7890292)(367.6323434,472.42965493)
\curveto(367.6323434,472.10673891)(367.60369762,471.7838229)(367.54640607,471.46090689)
\curveto(367.48911452,471.1431992)(367.36671894,470.79684573)(367.17921932,470.42184649)
\curveto(366.99692802,470.05205558)(366.73130356,469.63799391)(366.38234593,469.17966151)
\curveto(366.0333883,468.72653743)(365.56984757,468.20310099)(364.99172375,467.60935219)
\lineto(363.39016449,465.93748057)
\lineto(367.78859308,465.93748057)
\curveto(367.83025966,465.93748057)(367.86932208,465.92445977)(367.90578034,465.89841815)
\curveto(367.94744693,465.87758486)(367.98130102,465.84373076)(368.00734264,465.79685586)
\curveto(368.03859258,465.74998095)(368.05942587,465.69008524)(368.06984251,465.61716872)
\curveto(368.08546748,465.54946053)(368.09327996,465.46873152)(368.09327996,465.37498171)
\closepath
}
}
{
\newrgbcolor{curcolor}{0 0 0}
\pscustom[linestyle=none,fillstyle=solid,fillcolor=curcolor,opacity=0]
{
\newpath
\moveto(321.29861496,421.34600186)
\lineto(409.89688699,421.34600186)
\lineto(409.89688699,384.55867478)
\lineto(321.29861496,384.55867478)
\closepath
}
}
{
\newrgbcolor{curcolor}{1 1 1}
\pscustom[linestyle=none,fillstyle=solid,fillcolor=curcolor]
{
\newpath
\moveto(337.25952969,404.22824804)
\curveto(337.25952969,404.20741474)(337.25692553,404.18397729)(337.25171721,404.15793568)
\curveto(337.25171721,404.13710239)(337.24911304,404.11366493)(337.24390472,404.08762332)
\curveto(337.2386964,404.06158171)(337.23088391,404.03293593)(337.22046727,404.00168599)
\curveto(337.21525895,403.97043606)(337.20744646,403.93658196)(337.19702982,403.9001237)
\lineto(334.79859717,397.22044973)
\curveto(334.77776388,397.16315818)(334.74911811,397.11628327)(334.71265985,397.07982501)
\curveto(334.68140991,397.04336675)(334.63193084,397.01472098)(334.56422265,396.99388769)
\curveto(334.49651445,396.97305439)(334.41057713,396.96003359)(334.30641067,396.95482526)
\curveto(334.20224421,396.94440862)(334.07203614,396.9392003)(333.91578646,396.9392003)
\curveto(333.75953678,396.9392003)(333.62932871,396.94440862)(333.52516225,396.95482526)
\curveto(333.4209958,396.96524191)(333.33505847,396.98086688)(333.26735027,397.00170017)
\curveto(333.2048504,397.02253346)(333.15537133,397.05117924)(333.11891307,397.0876375)
\curveto(333.08245481,397.12409576)(333.05380904,397.1683665)(333.03297575,397.22044973)
\lineto(330.64235559,403.9001237)
\lineto(330.5876682,404.06418587)
\curveto(330.57725155,404.11106077)(330.56943907,404.14491487)(330.56423075,404.16574816)
\lineto(330.56423075,404.22824804)
\curveto(330.56423075,404.26991462)(330.57464739,404.30637288)(330.59548068,404.33762281)
\curveto(330.61631398,404.36887275)(330.65016807,404.3923102)(330.69704298,404.40793517)
\curveto(330.74912621,404.42876846)(330.81423024,404.44178927)(330.89235508,404.44699759)
\curveto(330.97568825,404.45220592)(331.07725054,404.45481008)(331.19704197,404.45481008)
\curveto(331.34808333,404.45481008)(331.46787475,404.44960175)(331.55641624,404.43918511)
\curveto(331.65016605,404.43397679)(331.72047841,404.42095598)(331.76735331,404.40012269)
\curveto(331.81943654,404.3792894)(331.8558948,404.35324778)(331.87672809,404.32199785)
\curveto(331.9027697,404.29074791)(331.92620716,404.25168549)(331.94704045,404.20481058)
\lineto(333.93141143,398.40794732)
\lineto(333.96266137,398.31419751)
\lineto(333.98609882,398.40794732)
\lineto(335.94703235,404.20481058)
\curveto(335.95744899,404.25168549)(335.97567812,404.29074791)(336.00171974,404.32199785)
\curveto(336.02776135,404.35324778)(336.06421961,404.3792894)(336.11109451,404.40012269)
\curveto(336.16317774,404.42095598)(336.23088594,404.43397679)(336.3142191,404.43918511)
\curveto(336.40276059,404.44960175)(336.51734369,404.45481008)(336.65796841,404.45481008)
\curveto(336.77775983,404.45481008)(336.87671796,404.45220592)(336.95484281,404.44699759)
\curveto(337.03296765,404.44178927)(337.09286336,404.42876846)(337.13452994,404.40793517)
\curveto(337.18140485,404.38710188)(337.21265478,404.36106027)(337.22827975,404.32981033)
\curveto(337.24911304,404.30376872)(337.25952969,404.26991462)(337.25952969,404.22824804)
\closepath
}
}
{
\newrgbcolor{curcolor}{1 1 1}
\pscustom[linestyle=none,fillstyle=solid,fillcolor=curcolor]
{
\newpath
\moveto(344.58932737,397.18138731)
\curveto(344.58932737,397.13972072)(344.57891073,397.10326246)(344.55807744,397.07201253)
\curveto(344.54245247,397.04076259)(344.51120253,397.01472098)(344.46432763,396.99388769)
\curveto(344.42266104,396.97826272)(344.36536949,396.96524191)(344.29245297,396.95482526)
\curveto(344.21953646,396.94440862)(344.13099497,396.9392003)(344.02682851,396.9392003)
\curveto(343.91745373,396.9392003)(343.82630808,396.94440862)(343.75339157,396.95482526)
\curveto(343.68047505,396.96524191)(343.62057933,396.97826272)(343.57370443,396.99388769)
\curveto(343.52682952,397.01472098)(343.49297543,397.04076259)(343.47214214,397.07201253)
\curveto(343.45130884,397.10326246)(343.4408922,397.13972072)(343.4408922,397.18138731)
\lineto(343.4408922,398.11107292)
\curveto(343.07110128,397.71003207)(342.68568539,397.3975327)(342.28464454,397.17357482)
\curveto(341.88881201,396.94961694)(341.45391706,396.837638)(340.97995968,396.837638)
\curveto(340.46433573,396.837638)(340.02423245,396.93659613)(339.65964986,397.1345124)
\curveto(339.29506726,397.33763699)(338.99819286,397.60846977)(338.76902666,397.94701076)
\curveto(338.54506878,398.29076006)(338.38100661,398.69180091)(338.27684016,399.15013332)
\curveto(338.1726737,399.61367405)(338.12059047,400.10065223)(338.12059047,400.61106786)
\curveto(338.12059047,401.2152333)(338.18569451,401.75950304)(338.31590258,402.24387705)
\curveto(338.44611065,402.7334594)(338.63881859,403.15012522)(338.89402641,403.49387452)
\curveto(339.14923422,403.83762383)(339.46433775,404.10064413)(339.83933699,404.28293543)
\curveto(340.21954456,404.47043505)(340.65704367,404.56418486)(341.15183433,404.56418486)
\curveto(341.56329183,404.56418486)(341.93829107,404.47303921)(342.27683206,404.29074791)
\curveto(342.62058136,404.11366493)(342.95912234,403.85064463)(343.292455,403.50168701)
\lineto(343.292455,407.58761623)
\curveto(343.292455,407.62407449)(343.30026748,407.65792859)(343.31589245,407.68917853)
\curveto(343.33672574,407.72563679)(343.373184,407.7516784)(343.42526723,407.76730337)
\curveto(343.47735046,407.78813666)(343.54245449,407.80376163)(343.62057933,407.81417827)
\curveto(343.7039125,407.82980324)(343.80807896,407.83761573)(343.9330787,407.83761573)
\curveto(344.06328677,407.83761573)(344.17005739,407.82980324)(344.25339055,407.81417827)
\curveto(344.33672372,407.80376163)(344.40182775,407.78813666)(344.44870266,407.76730337)
\curveto(344.49557756,407.7516784)(344.52943166,407.72563679)(344.55026495,407.68917853)
\curveto(344.57630657,407.65792859)(344.58932737,407.62407449)(344.58932737,407.58761623)
\closepath
\moveto(343.292455,402.12668979)
\curveto(342.94349737,402.55898058)(342.60495639,402.88710492)(342.27683206,403.1110628)
\curveto(341.95391604,403.340229)(341.61537506,403.4548121)(341.26120911,403.4548121)
\curveto(340.93308478,403.4548121)(340.65443951,403.37668726)(340.42527331,403.22043758)
\curveto(340.1961071,403.06418789)(340.00860748,402.85845914)(339.86277444,402.60325133)
\curveto(339.72214973,402.34804351)(339.61798327,402.0589816)(339.55027508,401.73606558)
\curveto(339.4877752,401.41314957)(339.45652527,401.08502523)(339.45652527,400.75169258)
\curveto(339.45652527,400.39752663)(339.48256688,400.05117316)(339.53465011,399.71263218)
\curveto(339.59194166,399.3740912)(339.68829563,399.07200848)(339.82371202,398.80638402)
\curveto(339.95912842,398.54596788)(340.13881555,398.3350308)(340.36277343,398.1735728)
\curveto(340.58673131,398.01732311)(340.86798074,397.93919827)(341.20652172,397.93919827)
\curveto(341.37839638,397.93919827)(341.54245854,397.96263572)(341.69870823,398.00951063)
\curveto(341.86016623,398.05638553)(342.0242284,398.13451038)(342.19089473,398.24388515)
\curveto(342.35756106,398.35325993)(342.53203987,398.49388465)(342.71433117,398.6657593)
\curveto(342.89662247,398.84284227)(343.08933041,399.05898767)(343.292455,399.31419549)
\closepath
}
}
{
\newrgbcolor{curcolor}{1 1 1}
\pscustom[linestyle=none,fillstyle=solid,fillcolor=curcolor]
{
\newpath
\moveto(352.96587301,401.00169207)
\curveto(352.96587301,400.79856748)(352.91378978,400.65273444)(352.80962332,400.56419296)
\curveto(352.71066519,400.48085979)(352.59608209,400.43919321)(352.46587402,400.43919321)
\lineto(347.85650835,400.43919321)
\curveto(347.85650835,400.048569)(347.89557077,399.69700721)(347.97369562,399.38450784)
\curveto(348.05182046,399.07200848)(348.18202853,398.80377985)(348.36431982,398.57982197)
\curveto(348.54661112,398.35586409)(348.78358981,398.18398944)(349.07525588,398.06419802)
\curveto(349.36692196,397.94440659)(349.72369207,397.88451088)(350.14556622,397.88451088)
\curveto(350.47889888,397.88451088)(350.77577327,397.9105525)(351.03618941,397.96263572)
\curveto(351.29660555,398.01992727)(351.52056343,398.08242715)(351.70806305,398.15013534)
\curveto(351.900771,398.21784354)(352.05702068,398.27773925)(352.1768121,398.32982248)
\curveto(352.30181185,398.38711403)(352.39556166,398.41575981)(352.45806153,398.41575981)
\curveto(352.49451979,398.41575981)(352.52576973,398.40534316)(352.55181134,398.38450987)
\curveto(352.58306128,398.3688849)(352.60649873,398.34284329)(352.6221237,398.30638503)
\curveto(352.63774867,398.26992677)(352.64816532,398.21784354)(352.65337364,398.15013534)
\curveto(352.66379028,398.08763547)(352.66899861,398.00951063)(352.66899861,397.91576082)
\curveto(352.66899861,397.84805262)(352.66639445,397.78815691)(352.66118612,397.73607368)
\curveto(352.6559778,397.68919878)(352.64816532,397.64492803)(352.63774867,397.60326145)
\curveto(352.63254035,397.56680319)(352.61951954,397.53294909)(352.59868625,397.50169916)
\curveto(352.58306128,397.47044922)(352.55962383,397.43919928)(352.52837389,397.40794935)
\curveto(352.50233228,397.38190773)(352.41899911,397.33503283)(352.2783744,397.26732463)
\curveto(352.13774968,397.20482476)(351.95545839,397.14232488)(351.73150051,397.07982501)
\curveto(351.50754263,397.01732514)(351.24712649,396.96263775)(350.95025209,396.91576284)
\curveto(350.65858601,396.86367962)(350.34608664,396.837638)(350.01275399,396.837638)
\curveto(349.43463016,396.837638)(348.92681869,396.91836701)(348.48931957,397.07982501)
\curveto(348.05702878,397.24128302)(347.69244618,397.48086587)(347.39557179,397.79857356)
\curveto(347.09869739,398.11628125)(346.87473951,398.51471794)(346.72369815,398.99388364)
\curveto(346.57265679,399.47304933)(346.49713611,400.03033987)(346.49713611,400.66575525)
\curveto(346.49713611,401.26992069)(346.57526095,401.81158626)(346.73151063,402.29075196)
\curveto(346.88776031,402.77512598)(347.11171819,403.18397932)(347.40338427,403.51731198)
\curveto(347.70025867,403.85585296)(348.05702878,404.11366493)(348.4736946,404.29074791)
\curveto(348.89036043,404.47303921)(349.35650531,404.56418486)(349.87212927,404.56418486)
\curveto(350.42421149,404.56418486)(350.89296054,404.47564337)(351.27837642,404.29856039)
\curveto(351.66900063,404.12147742)(351.98931248,403.88189457)(352.23931198,403.57981185)
\curveto(352.48931147,403.28293745)(352.67160277,402.93137566)(352.78618587,402.52512648)
\curveto(352.90597729,402.12408563)(352.96587301,401.694399)(352.96587301,401.2360666)
\closepath
\moveto(351.66900063,401.38450379)
\curveto(351.6846256,402.06158576)(351.53358424,402.59283468)(351.21587655,402.97825057)
\curveto(350.90337718,403.36366645)(350.43723229,403.5563744)(349.81744188,403.5563744)
\curveto(349.49973419,403.5563744)(349.22108892,403.49647868)(348.98150607,403.37668726)
\curveto(348.74192323,403.25689584)(348.5414028,403.09804199)(348.37994479,402.90012573)
\curveto(348.21848679,402.70220946)(348.09348704,402.4704391)(348.00494555,402.20481463)
\curveto(347.91640406,401.94439849)(347.866925,401.67096155)(347.85650835,401.38450379)
\closepath
}
}
{
\newrgbcolor{curcolor}{1 1 1}
\pscustom[linestyle=none,fillstyle=solid,fillcolor=curcolor]
{
\newpath
\moveto(360.619483,404.22824804)
\curveto(360.619483,404.20741474)(360.61687883,404.18397729)(360.61167051,404.15793568)
\curveto(360.61167051,404.13710239)(360.60906635,404.11366493)(360.60385803,404.08762332)
\curveto(360.5986497,404.06158171)(360.59083722,404.03293593)(360.58042057,404.00168599)
\curveto(360.57521225,403.97043606)(360.56739977,403.93658196)(360.55698312,403.9001237)
\lineto(358.15855048,397.22044973)
\curveto(358.13771719,397.16315818)(358.10907141,397.11628327)(358.07261315,397.07982501)
\curveto(358.04136322,397.04336675)(357.99188415,397.01472098)(357.92417595,396.99388769)
\curveto(357.85646776,396.97305439)(357.77053043,396.96003359)(357.66636398,396.95482526)
\curveto(357.56219752,396.94440862)(357.43198945,396.9392003)(357.27573977,396.9392003)
\curveto(357.11949008,396.9392003)(356.98928201,396.94440862)(356.88511556,396.95482526)
\curveto(356.7809491,396.96524191)(356.69501178,396.98086688)(356.62730358,397.00170017)
\curveto(356.56480371,397.02253346)(356.51532464,397.05117924)(356.47886638,397.0876375)
\curveto(356.44240812,397.12409576)(356.41376235,397.1683665)(356.39292905,397.22044973)
\lineto(354.0023089,403.9001237)
\lineto(353.94762151,404.06418587)
\curveto(353.93720486,404.11106077)(353.92939238,404.14491487)(353.92418405,404.16574816)
\lineto(353.92418405,404.22824804)
\curveto(353.92418405,404.26991462)(353.9346007,404.30637288)(353.95543399,404.33762281)
\curveto(353.97626728,404.36887275)(354.01012138,404.3923102)(354.05699628,404.40793517)
\curveto(354.10907951,404.42876846)(354.17418355,404.44178927)(354.25230839,404.44699759)
\curveto(354.33564155,404.45220592)(354.43720385,404.45481008)(354.55699527,404.45481008)
\curveto(354.70803663,404.45481008)(354.82782806,404.44960175)(354.91636954,404.43918511)
\curveto(355.01011935,404.43397679)(355.08043171,404.42095598)(355.12730662,404.40012269)
\curveto(355.17938985,404.3792894)(355.2158481,404.35324778)(355.2366814,404.32199785)
\curveto(355.26272301,404.29074791)(355.28616046,404.25168549)(355.30699375,404.20481058)
\lineto(357.29136474,398.40794732)
\lineto(357.32261467,398.31419751)
\lineto(357.34605212,398.40794732)
\lineto(359.30698565,404.20481058)
\curveto(359.3174023,404.25168549)(359.33563143,404.29074791)(359.36167304,404.32199785)
\curveto(359.38771466,404.35324778)(359.42417292,404.3792894)(359.47104782,404.40012269)
\curveto(359.52313105,404.42095598)(359.59083925,404.43397679)(359.67417241,404.43918511)
\curveto(359.7627139,404.44960175)(359.877297,404.45481008)(360.01792171,404.45481008)
\curveto(360.13771314,404.45481008)(360.23667127,404.45220592)(360.31479611,404.44699759)
\curveto(360.39292095,404.44178927)(360.45281667,404.42876846)(360.49448325,404.40793517)
\curveto(360.54135815,404.38710188)(360.57260809,404.36106027)(360.58823306,404.32981033)
\curveto(360.60906635,404.30376872)(360.619483,404.26991462)(360.619483,404.22824804)
\closepath
}
}
{
\newrgbcolor{curcolor}{1 1 1}
\pscustom[linestyle=none,fillstyle=solid,fillcolor=curcolor]
{
\newpath
\moveto(368.01515512,399.87669435)
\curveto(368.01515512,399.41315362)(367.93442612,398.99388364)(367.77296811,398.61888439)
\curveto(367.61151011,398.24909348)(367.37973974,397.93138579)(367.07765702,397.66576132)
\curveto(366.7755743,397.40013686)(366.40317922,397.19440811)(365.96047178,397.04857507)
\curveto(365.51776435,396.90795036)(365.01776536,396.837638)(364.46047482,396.837638)
\curveto(364.12193384,396.837638)(363.80422615,396.86628378)(363.50735175,396.92357533)
\curveto(363.21568567,396.97565856)(362.95526954,397.03815843)(362.72610333,397.11107495)
\curveto(362.49693713,397.18919979)(362.30683335,397.26732463)(362.15579199,397.34544947)
\curveto(362.00475063,397.42878264)(361.90839666,397.48867835)(361.86673007,397.52513661)
\curveto(361.83027181,397.56159487)(361.80162604,397.59805313)(361.78079275,397.63451139)
\curveto(361.75995946,397.67096965)(361.74173033,397.71524039)(361.72610536,397.76732362)
\curveto(361.71048039,397.81940685)(361.69745958,397.88190672)(361.68704294,397.95482324)
\curveto(361.68183461,398.03294808)(361.67923045,398.12669789)(361.67923045,398.23607267)
\curveto(361.67923045,398.42357229)(361.69745958,398.55378036)(361.73391784,398.62669688)
\curveto(361.7703761,398.6996134)(361.82245933,398.73607166)(361.89016753,398.73607166)
\curveto(361.93704243,398.73607166)(362.02818808,398.69440508)(362.16360447,398.61107191)
\curveto(362.30422919,398.52773875)(362.48131216,398.4365931)(362.6948534,398.33763496)
\curveto(362.91360295,398.24388515)(363.16620661,398.15534367)(363.45266436,398.0720105)
\curveto(363.74433044,397.98867734)(364.06724645,397.94701076)(364.4214124,397.94701076)
\curveto(364.7651617,397.94701076)(365.06724443,397.9912815)(365.32766056,398.07982299)
\curveto(365.5880767,398.16836447)(365.80682626,398.29336422)(365.98390924,398.45482223)
\curveto(366.16099221,398.61628023)(366.29380444,398.80638402)(366.38234593,399.02513357)
\curveto(366.47088742,399.24909145)(366.51515816,399.49127846)(366.51515816,399.7516946)
\curveto(366.51515816,400.03815235)(366.45786661,400.29596433)(366.34328351,400.52513053)
\curveto(366.23390873,400.75429674)(366.06984656,400.952213)(365.851097,401.11887933)
\curveto(365.63755577,401.28554566)(365.37193131,401.41314957)(365.05422362,401.50169106)
\curveto(364.74172425,401.59023254)(364.38495414,401.63450329)(363.98391329,401.63450329)
\lineto(363.02297773,401.63450329)
\curveto(362.98131115,401.63450329)(362.93964457,401.64231577)(362.89797798,401.65794074)
\curveto(362.86151973,401.67877403)(362.82766563,401.71002397)(362.79641569,401.75169055)
\curveto(362.77037408,401.79335713)(362.74693662,401.84804452)(362.72610333,401.91575272)
\curveto(362.71047836,401.98346091)(362.70266588,402.0720024)(362.70266588,402.18137718)
\curveto(362.70266588,402.28033531)(362.71047836,402.36106432)(362.72610333,402.42356419)
\curveto(362.7417283,402.49127239)(362.76256159,402.54335561)(362.78860321,402.57981387)
\curveto(362.81985314,402.62148046)(362.85370724,402.65012623)(362.8901655,402.6657512)
\curveto(362.92662376,402.68658449)(362.96829034,402.69700114)(363.01516525,402.69700114)
\lineto(363.89797596,402.69700114)
\curveto(364.24172526,402.69700114)(364.54901631,402.74127188)(364.81984909,402.82981337)
\curveto(365.0958902,402.91835486)(365.32766056,403.04595876)(365.51516019,403.21262509)
\curveto(365.70786813,403.37929142)(365.85370117,403.57720769)(365.9526593,403.80637389)
\curveto(366.05682575,404.04074842)(366.10890898,404.30116455)(366.10890898,404.58762231)
\curveto(366.10890898,404.79595522)(366.07505488,404.99387149)(366.00734669,405.18137111)
\curveto(365.93963849,405.37407905)(365.8380762,405.54074538)(365.70265981,405.68137009)
\curveto(365.56724341,405.82199481)(365.3927646,405.93136959)(365.17922337,406.00949443)
\curveto(364.97089045,406.09282759)(364.72870344,406.13449418)(364.45266234,406.13449418)
\curveto(364.15057962,406.13449418)(363.87193435,406.08761927)(363.61672653,405.99386946)
\curveto(363.36672704,405.90532797)(363.14276916,405.808974)(362.94485289,405.70480755)
\curveto(362.74693662,405.60064109)(362.58287446,405.50168296)(362.45266639,405.40793315)
\curveto(362.32245832,405.31939166)(362.23131267,405.27512092)(362.17922944,405.27512092)
\curveto(362.14277118,405.27512092)(362.10891708,405.28032924)(362.07766715,405.29074588)
\curveto(362.05162553,405.30637085)(362.02818808,405.33241247)(362.00735479,405.36887073)
\curveto(361.99172982,405.41053731)(361.97870901,405.4652247)(361.96829237,405.53293289)
\curveto(361.96308404,405.60584941)(361.96047988,405.69959922)(361.96047988,405.81418232)
\curveto(361.96047988,405.89230717)(361.96308404,405.96001536)(361.96829237,406.01730691)
\curveto(361.97350069,406.07980679)(361.98391734,406.13189001)(361.9995423,406.1735566)
\curveto(362.01516727,406.2204315)(362.0333964,406.26209808)(362.05422969,406.29855634)
\curveto(362.07506298,406.3350146)(362.10891708,406.37668119)(362.15579199,406.42355609)
\curveto(362.20787522,406.47563932)(362.30943751,406.54855584)(362.46047887,406.64230565)
\curveto(362.61152023,406.73605546)(362.79641569,406.82720111)(363.01516525,406.91574259)
\curveto(363.23912313,407.0094924)(363.49433094,407.08761725)(363.7807887,407.15011712)
\curveto(364.07245477,407.21261699)(364.38495414,407.24386693)(364.7182868,407.24386693)
\curveto(365.18703585,407.24386693)(365.59849335,407.18397122)(365.9526593,407.06417979)
\curveto(366.30682525,406.94438837)(366.60109549,406.77511788)(366.83547001,406.55636832)
\curveto(367.07505286,406.33761876)(367.25213583,406.07459846)(367.36671894,405.76730742)
\curveto(367.48651036,405.4652247)(367.54640607,405.12928788)(367.54640607,404.75949696)
\curveto(367.54640607,404.44178927)(367.50473949,404.14491487)(367.42140633,403.86887376)
\curveto(367.33807316,403.59804098)(367.21567757,403.35585397)(367.05421957,403.14231273)
\curveto(366.89276156,402.93397982)(366.6948453,402.75429269)(366.46047077,402.60325133)
\curveto(366.22609625,402.45741829)(365.95526346,402.35585599)(365.64797242,402.29856444)
\lineto(365.64797242,402.28293948)
\curveto(365.99693004,402.24648122)(366.31463773,402.15793973)(366.60109549,402.01731501)
\curveto(366.89276156,401.88189862)(367.14276106,401.71002397)(367.35109397,401.50169106)
\curveto(367.55942688,401.29335815)(367.72088489,401.04856697)(367.83546799,400.76731754)
\curveto(367.95525941,400.49127644)(368.01515512,400.19440204)(368.01515512,399.87669435)
\closepath
}
}
{
\newrgbcolor{curcolor}{0 0 0}
\pscustom[linestyle=none,fillstyle=solid,fillcolor=curcolor,opacity=0]
{
\newpath
\moveto(577.30184983,560.46753116)
\lineto(665.90009561,560.46753116)
\lineto(665.90009561,523.68020408)
\lineto(577.30184983,523.68020408)
\closepath
}
}
{
\newrgbcolor{curcolor}{0.94509804 0.99215686 0.99607843}
\pscustom[linestyle=none,fillstyle=solid,fillcolor=curcolor]
{
\newpath
\moveto(595.21589233,536.58414501)
\curveto(595.25755891,536.46956191)(595.2783922,536.3758121)(595.2783922,536.30289558)
\curveto(595.28360052,536.23518739)(595.26537139,536.18310416)(595.22370481,536.1466459)
\curveto(595.18203823,536.11018764)(595.11172587,536.08675019)(595.01276774,536.07633354)
\curveto(594.91901793,536.0659169)(594.79141402,536.06070857)(594.62995601,536.06070857)
\curveto(594.46849801,536.06070857)(594.33828994,536.0659169)(594.2393318,536.07633354)
\curveto(594.14558199,536.08154187)(594.07266548,536.09195851)(594.02058225,536.10758348)
\curveto(593.97370734,536.12841677)(593.93724908,536.15445838)(593.91120747,536.18570832)
\curveto(593.89037418,536.21695826)(593.86954089,536.25602068)(593.8487076,536.30289558)
\lineto(592.98152185,538.7638281)
\lineto(588.77840536,538.7638281)
\lineto(587.95028204,536.33414552)
\curveto(587.93465707,536.28727062)(587.91382378,536.24560403)(587.88778217,536.20914577)
\curveto(587.86174055,536.17789584)(587.82267813,536.14925006)(587.7705949,536.12320845)
\curveto(587.72372,536.10237516)(587.65340764,536.08675019)(587.55965783,536.07633354)
\curveto(587.47111634,536.0659169)(587.35392908,536.06070857)(587.20809604,536.06070857)
\curveto(587.05705468,536.06070857)(586.93205493,536.06852106)(586.8330968,536.08414603)
\curveto(586.73934699,536.09456267)(586.6716388,536.11800013)(586.62997221,536.15445838)
\curveto(586.59351395,536.19091664)(586.57528482,536.24299987)(586.57528482,536.31070807)
\curveto(586.58049315,536.38362459)(586.6039306,536.4773744)(586.64559718,536.5919575)
\lineto(590.03621532,545.98256348)
\curveto(590.05704861,546.03985503)(590.08309022,546.08672994)(590.11434016,546.1231882)
\curveto(590.15079842,546.15964646)(590.20027748,546.18829223)(590.26277736,546.20912552)
\curveto(590.33048555,546.22995882)(590.41381872,546.24297962)(590.51277685,546.24818794)
\curveto(590.61694331,546.25860459)(590.74715138,546.26381291)(590.90340106,546.26381291)
\curveto(591.07006739,546.26381291)(591.20808794,546.25860459)(591.31746272,546.24818794)
\curveto(591.4268375,546.24297962)(591.51537899,546.22995882)(591.58308718,546.20912552)
\curveto(591.65079538,546.18829223)(591.70287861,546.1570423)(591.73933687,546.11537571)
\curveto(591.77579513,546.07891745)(591.8044409,546.03204255)(591.82527419,545.974751)
\closepath
\moveto(590.86433864,544.87319073)
\lineto(590.85652615,544.87319073)
\lineto(589.11434218,539.83413843)
\lineto(592.62996006,539.83413843)
\closepath
}
}
{
\newrgbcolor{curcolor}{0.94509804 0.99215686 0.99607843}
\pscustom[linestyle=none,fillstyle=solid,fillcolor=curcolor]
{
\newpath
\moveto(602.32325308,538.94351524)
\curveto(602.32325308,538.46955786)(602.23471159,538.04768372)(602.05762862,537.6778928)
\curveto(601.88575397,537.30810188)(601.64356696,536.99299835)(601.33106759,536.73258221)
\curveto(601.02377654,536.4773744)(600.65919395,536.28466645)(600.2373198,536.15445838)
\curveto(599.82065398,536.02425032)(599.37013406,535.95914628)(598.88576004,535.95914628)
\curveto(598.54721906,535.95914628)(598.23211553,535.98779206)(597.94044945,536.04508361)
\curveto(597.6539917,536.10237516)(597.39617972,536.17268751)(597.16701352,536.25602068)
\curveto(596.94305564,536.33935384)(596.75295186,536.42529117)(596.59670218,536.51383266)
\curveto(596.44566081,536.60237414)(596.3388902,536.67789482)(596.27639032,536.7403947)
\curveto(596.21909877,536.80289457)(596.17482803,536.88101941)(596.14357809,536.97476922)
\curveto(596.11753648,537.07372736)(596.10451567,537.20393543)(596.10451567,537.36539343)
\curveto(596.10451567,537.47997653)(596.10972399,537.57372634)(596.12014064,537.64664286)
\curveto(596.13055729,537.7247677)(596.14618225,537.78726758)(596.16701555,537.83414248)
\curveto(596.18784884,537.88101739)(596.21389045,537.91226733)(596.24514039,537.92789229)
\curveto(596.27639032,537.94872558)(596.31284858,537.95914223)(596.35451517,537.95914223)
\curveto(596.42743168,537.95914223)(596.52899398,537.91487149)(596.65920205,537.82633)
\curveto(596.79461844,537.73778851)(596.96649309,537.64143454)(597.174826,537.53726808)
\curveto(597.38315892,537.43310163)(597.63315841,537.3341435)(597.92482449,537.24039369)
\curveto(598.22169888,537.1518522)(598.56284403,537.10758145)(598.94825991,537.10758145)
\curveto(599.23992599,537.10758145)(599.50555045,537.14664388)(599.7451333,537.22476872)
\curveto(599.98992447,537.30289356)(600.19825738,537.41226834)(600.37013203,537.55289305)
\curveto(600.54721501,537.69872609)(600.6826314,537.87580907)(600.77638121,538.08414198)
\curveto(600.87013102,538.29247489)(600.91700593,538.52945358)(600.91700593,538.79507804)
\curveto(600.91700593,539.08153579)(600.85190189,539.32632696)(600.72169382,539.52945155)
\curveto(600.59148575,539.73257614)(600.4196111,539.90965911)(600.20606987,540.06070047)
\curveto(599.99252863,540.21695016)(599.74773746,540.35757487)(599.47169635,540.48257462)
\curveto(599.20086357,540.61278269)(598.9222183,540.74299076)(598.63576055,540.87319883)
\curveto(598.34930279,541.00861522)(598.07065752,541.15705242)(597.79982474,541.31851043)
\curveto(597.52899195,541.47996843)(597.28680494,541.67007222)(597.07326371,541.88882177)
\curveto(596.85972248,542.10757133)(596.68524366,542.36277915)(596.54982727,542.65444522)
\curveto(596.4196192,542.95131962)(596.35451517,543.30548557)(596.35451517,543.71694307)
\curveto(596.35451517,544.13881722)(596.43003585,544.51381646)(596.58107721,544.84194079)
\curveto(596.73732689,545.17527345)(596.95086812,545.45391872)(597.22170091,545.6778766)
\curveto(597.49774202,545.90183448)(597.82326219,546.07110497)(598.19826143,546.18568807)
\curveto(598.578469,546.3054795)(598.98732233,546.36537521)(599.42482145,546.36537521)
\curveto(599.64877933,546.36537521)(599.87273721,546.34454192)(600.09669509,546.30287533)
\curveto(600.32586129,546.26641707)(600.53940252,546.21433385)(600.73731879,546.14662565)
\curveto(600.94044338,546.08412578)(601.12013052,546.01120926)(601.2763802,545.92787609)
\curveto(601.43262988,545.84975125)(601.53419218,545.78464722)(601.58106708,545.73256399)
\curveto(601.63315031,545.68568908)(601.66700441,545.64662666)(601.68262938,545.61537673)
\curveto(601.69825434,545.58933511)(601.71127515,545.55287685)(601.7216918,545.50600195)
\curveto(601.73210844,545.46433537)(601.73992093,545.41225214)(601.74512925,545.34975226)
\curveto(601.75033757,545.28725239)(601.75294173,545.20652339)(601.75294173,545.10756525)
\curveto(601.75294173,545.01381544)(601.74773341,544.93048228)(601.73731677,544.85756576)
\curveto(601.73210844,544.78464924)(601.7216918,544.72214937)(601.70606683,544.67006614)
\curveto(601.69044186,544.62319124)(601.66700441,544.58673298)(601.63575447,544.56069136)
\curveto(601.60971286,544.53985807)(601.57846292,544.52944143)(601.54200466,544.52944143)
\curveto(601.48471311,544.52944143)(601.39356746,544.56589968)(601.26856771,544.6388162)
\curveto(601.14877629,544.71173272)(601.00033909,544.79246173)(600.82325612,544.88100321)
\curveto(600.64617314,544.97475302)(600.43523607,545.05808619)(600.1904449,545.13100271)
\curveto(599.95086205,545.20912755)(599.68002927,545.24818997)(599.37794654,545.24818997)
\curveto(599.09669711,545.24818997)(598.85190594,545.20912755)(598.64357303,545.13100271)
\curveto(598.43524012,545.05808619)(598.26336547,544.95912806)(598.12794907,544.83412831)
\curveto(597.99253268,544.70912856)(597.89097039,544.56069136)(597.82326219,544.38881671)
\curveto(597.755554,544.21694206)(597.7216999,544.03465076)(597.7216999,543.84194282)
\curveto(597.7216999,543.56069339)(597.78680393,543.31850638)(597.917012,543.11538179)
\curveto(598.04722007,542.9122572)(598.21909472,542.73257006)(598.43263596,542.57632038)
\curveto(598.65138551,542.4200707)(598.89878085,542.27684182)(599.17482195,542.14663375)
\curveto(599.45086306,542.01642568)(599.73211249,541.88361345)(600.01857025,541.74819706)
\curveto(600.305028,541.61798899)(600.58627743,541.47215595)(600.86231854,541.31069794)
\curveto(601.13835965,541.15444826)(601.38315082,540.96694864)(601.59669205,540.74819908)
\curveto(601.81544161,540.53465785)(601.98992042,540.27945003)(602.12012849,539.98257563)
\curveto(602.25554488,539.69090956)(602.32325308,539.34455609)(602.32325308,538.94351524)
\closepath
}
}
{
\newrgbcolor{curcolor}{0.94509804 0.99215686 0.99607843}
\pscustom[linestyle=none,fillstyle=solid,fillcolor=curcolor]
{
\newpath
\moveto(605.58130789,536.31070807)
\curveto(605.58130789,536.26904149)(605.57089125,536.23258323)(605.55005795,536.20133329)
\curveto(605.52922466,536.17008335)(605.4927664,536.14404174)(605.44068318,536.12320845)
\curveto(605.38859995,536.10758348)(605.31828759,536.09456267)(605.2297461,536.08414603)
\curveto(605.14641294,536.06852106)(605.03964232,536.06070857)(604.90943425,536.06070857)
\curveto(604.7844345,536.06070857)(604.67766389,536.06852106)(604.5891224,536.08414603)
\curveto(604.50058091,536.09456267)(604.43026855,536.10758348)(604.37818533,536.12320845)
\curveto(604.3261021,536.14404174)(604.28964384,536.17008335)(604.26881055,536.20133329)
\curveto(604.24797726,536.23258323)(604.23756061,536.26904149)(604.23756061,536.31070807)
\lineto(604.23756061,546.01381342)
\curveto(604.23756061,546.05548)(604.24797726,546.09193826)(604.26881055,546.1231882)
\curveto(604.29485216,546.15443813)(604.33391458,546.17787559)(604.38599781,546.19350056)
\curveto(604.44328936,546.21433385)(604.51360172,546.22995882)(604.59693488,546.24037546)
\curveto(604.68547637,546.25600043)(604.78964283,546.26381291)(604.90943425,546.26381291)
\curveto(605.03964232,546.26381291)(605.14641294,546.25600043)(605.2297461,546.24037546)
\curveto(605.31828759,546.22995882)(605.38859995,546.21433385)(605.44068318,546.19350056)
\curveto(605.4927664,546.17787559)(605.52922466,546.15443813)(605.55005795,546.1231882)
\curveto(605.57089125,546.09193826)(605.58130789,546.05548)(605.58130789,546.01381342)
\closepath
}
}
{
\newrgbcolor{curcolor}{0.94509804 0.99215686 0.99607843}
\pscustom[linestyle=none,fillstyle=solid,fillcolor=curcolor]
{
\newpath
\moveto(614.06604146,536.67008234)
\curveto(614.06604146,536.57112421)(614.06083313,536.48518688)(614.05041649,536.41227036)
\curveto(614.03999984,536.34456217)(614.02177071,536.28727062)(613.9957291,536.24039571)
\curveto(613.97489581,536.19352081)(613.94625003,536.15966671)(613.90979177,536.13883342)
\curveto(613.87854184,536.11800013)(613.83947942,536.10758348)(613.79260451,536.10758348)
\lineto(607.8238666,536.10758348)
\curveto(607.6884502,536.10758348)(607.57907543,536.14925006)(607.49574226,536.23258323)
\curveto(607.4124091,536.31591639)(607.37074251,536.44872862)(607.37074251,536.63101992)
\lineto(607.37074251,536.92008183)
\curveto(607.37074251,536.98779003)(607.37334668,537.0502899)(607.378555,537.10758145)
\curveto(607.38897164,537.16487301)(607.40720077,537.22476872)(607.43324239,537.28726859)
\curveto(607.459284,537.35497679)(607.49574226,537.42789331)(607.54261717,537.50601815)
\curveto(607.58949207,537.58935131)(607.64678362,537.68570528)(607.71449182,537.79508006)
\lineto(612.37854487,545.09194029)
\lineto(607.76917921,545.09194029)
\curveto(607.71709598,545.09194029)(607.67282524,545.10235693)(607.63636698,545.12319022)
\curveto(607.59990872,545.14402351)(607.56865878,545.17527345)(607.54261717,545.21694003)
\curveto(607.52178388,545.26381494)(607.50615891,545.32110649)(607.49574226,545.38881469)
\curveto(607.48532562,545.4617312)(607.48011729,545.54506437)(607.48011729,545.63881418)
\curveto(607.48011729,545.74298063)(607.48532562,545.83152212)(607.49574226,545.90443864)
\curveto(607.50615891,545.97735516)(607.52178388,546.03725087)(607.54261717,546.08412578)
\curveto(607.56865878,546.13100068)(607.59990872,546.16485478)(607.63636698,546.18568807)
\curveto(607.67282524,546.20652136)(607.71709598,546.21693801)(607.76917921,546.21693801)
\lineto(613.39416782,546.21693801)
\curveto(613.53479253,546.21693801)(613.64416731,546.17527143)(613.72229215,546.09193826)
\curveto(613.80562532,546.01381342)(613.8472919,545.89141783)(613.8472919,545.7247515)
\lineto(613.8472919,545.42006462)
\curveto(613.8472919,545.33673146)(613.84208358,545.26121078)(613.83166693,545.19350258)
\curveto(613.82125029,545.13100271)(613.80302116,545.06589867)(613.77697954,544.99819048)
\curveto(613.75093793,544.93048228)(613.71447967,544.85756576)(613.66760476,544.77944092)
\curveto(613.62593818,544.70131608)(613.56864663,544.61017043)(613.49573011,544.50600397)
\lineto(608.85511451,537.24039369)
\lineto(613.79260451,537.24039369)
\curveto(613.881146,537.24039369)(613.94885419,537.19612294)(613.9957291,537.10758145)
\curveto(614.042604,537.01903997)(614.06604146,536.87320693)(614.06604146,536.67008234)
\closepath
}
}
{
\newrgbcolor{curcolor}{0.94509804 0.99215686 0.99607843}
\pscustom[linestyle=none,fillstyle=solid,fillcolor=curcolor]
{
\newpath
\moveto(621.36652754,536.66226986)
\curveto(621.36652754,536.56852005)(621.36131922,536.48518688)(621.35090257,536.41227036)
\curveto(621.34048593,536.34456217)(621.3222568,536.28727062)(621.29621519,536.24039571)
\curveto(621.27017357,536.19352081)(621.23892363,536.15966671)(621.20246538,536.13883342)
\curveto(621.17121544,536.11800013)(621.13475718,536.10758348)(621.0930906,536.10758348)
\lineto(616.24153792,536.10758348)
\curveto(616.1217465,536.10758348)(616.0071634,536.1466459)(615.89778862,536.22477074)
\curveto(615.79362216,536.30810391)(615.74153893,536.45133278)(615.74153893,536.65445737)
\lineto(615.74153893,545.67006412)
\curveto(615.74153893,545.8731887)(615.79362216,546.01381342)(615.89778862,546.09193826)
\curveto(616.0071634,546.17527143)(616.1217465,546.21693801)(616.24153792,546.21693801)
\lineto(621.03840321,546.21693801)
\curveto(621.08006979,546.21693801)(621.11652805,546.20652136)(621.14777799,546.18568807)
\curveto(621.18423625,546.16485478)(621.21288202,546.13100068)(621.23371531,546.08412578)
\curveto(621.2545486,546.03725087)(621.27017357,545.97735516)(621.28059022,545.90443864)
\curveto(621.29621519,545.83673044)(621.30402767,545.75079312)(621.30402767,545.64662666)
\curveto(621.30402767,545.55287685)(621.29621519,545.46954369)(621.28059022,545.39662717)
\curveto(621.27017357,545.32891897)(621.2545486,545.27162742)(621.23371531,545.22475252)
\curveto(621.21288202,545.18308594)(621.18423625,545.151836)(621.14777799,545.13100271)
\curveto(621.11652805,545.11016942)(621.08006979,545.09975277)(621.03840321,545.09975277)
\lineto(617.08528621,545.09975277)
\lineto(617.08528621,541.92788419)
\lineto(620.47590435,541.92788419)
\curveto(620.51757093,541.92788419)(620.55402919,541.91486339)(620.58527912,541.88882177)
\curveto(620.62173738,541.86798848)(620.65038316,541.83673854)(620.67121645,541.79507196)
\curveto(620.69725806,541.75340538)(620.71548719,541.69611383)(620.72590384,541.62319731)
\curveto(620.73632049,541.55028079)(620.74152881,541.46434347)(620.74152881,541.36538533)
\curveto(620.74152881,541.27163552)(620.73632049,541.19090652)(620.72590384,541.12319832)
\curveto(620.71548719,541.05549013)(620.69725806,541.00080274)(620.67121645,540.95913616)
\curveto(620.65038316,540.91746957)(620.62173738,540.88621964)(620.58527912,540.86538635)
\curveto(620.55402919,540.84976138)(620.51757093,540.84194889)(620.47590435,540.84194889)
\lineto(617.08528621,540.84194889)
\lineto(617.08528621,537.22476872)
\lineto(621.0930906,537.22476872)
\curveto(621.13475718,537.22476872)(621.17121544,537.21435207)(621.20246538,537.19351878)
\curveto(621.23892363,537.17268549)(621.27017357,537.13883139)(621.29621519,537.09195649)
\curveto(621.3222568,537.0502899)(621.34048593,536.99299835)(621.35090257,536.92008183)
\curveto(621.36131922,536.85237364)(621.36652754,536.76643631)(621.36652754,536.66226986)
\closepath
}
}
{
\newrgbcolor{curcolor}{0 0 0}
\pscustom[linestyle=none,fillstyle=solid,fillcolor=curcolor,opacity=0]
{
\newpath
\moveto(577.30184983,489.15680942)
\lineto(665.90009561,489.15680942)
\lineto(665.90009561,452.36948234)
\lineto(577.30184983,452.36948234)
\closepath
}
}
{
\newrgbcolor{curcolor}{1 1 1}
\pscustom[linestyle=none,fillstyle=solid,fillcolor=curcolor]
{
\newpath
\moveto(595.21589233,465.27341942)
\curveto(595.25755891,465.15883632)(595.2783922,465.06508651)(595.2783922,464.99216999)
\curveto(595.28360052,464.92446179)(595.26537139,464.87237856)(595.22370481,464.83592031)
\curveto(595.18203823,464.79946205)(595.11172587,464.77602459)(595.01276774,464.76560795)
\curveto(594.91901793,464.7551913)(594.79141402,464.74998298)(594.62995601,464.74998298)
\curveto(594.46849801,464.74998298)(594.33828994,464.7551913)(594.2393318,464.76560795)
\curveto(594.14558199,464.77081627)(594.07266548,464.78123292)(594.02058225,464.79685788)
\curveto(593.97370734,464.81769118)(593.93724908,464.84373279)(593.91120747,464.87498273)
\curveto(593.89037418,464.90623266)(593.86954089,464.94529508)(593.8487076,464.99216999)
\lineto(592.98152185,467.45310251)
\lineto(588.77840536,467.45310251)
\lineto(587.95028204,465.02341993)
\curveto(587.93465707,464.97654502)(587.91382378,464.93487844)(587.88778217,464.89842018)
\curveto(587.86174055,464.86717024)(587.82267813,464.83852447)(587.7705949,464.81248285)
\curveto(587.72372,464.79164956)(587.65340764,464.77602459)(587.55965783,464.76560795)
\curveto(587.47111634,464.7551913)(587.35392908,464.74998298)(587.20809604,464.74998298)
\curveto(587.05705468,464.74998298)(586.93205493,464.75779546)(586.8330968,464.77342043)
\curveto(586.73934699,464.78383708)(586.6716388,464.80727453)(586.62997221,464.84373279)
\curveto(586.59351395,464.88019105)(586.57528482,464.93227428)(586.57528482,464.99998247)
\curveto(586.58049315,465.07289899)(586.6039306,465.1666488)(586.64559718,465.2812319)
\lineto(590.03621532,474.67183789)
\curveto(590.05704861,474.72912944)(590.08309022,474.77600434)(590.11434016,474.8124626)
\curveto(590.15079842,474.84892086)(590.20027748,474.87756664)(590.26277736,474.89839993)
\curveto(590.33048555,474.91923322)(590.41381872,474.93225403)(590.51277685,474.93746235)
\curveto(590.61694331,474.94787899)(590.74715138,474.95308732)(590.90340106,474.95308732)
\curveto(591.07006739,474.95308732)(591.20808794,474.94787899)(591.31746272,474.93746235)
\curveto(591.4268375,474.93225403)(591.51537899,474.91923322)(591.58308718,474.89839993)
\curveto(591.65079538,474.87756664)(591.70287861,474.8463167)(591.73933687,474.80465012)
\curveto(591.77579513,474.76819186)(591.8044409,474.72131695)(591.82527419,474.6640254)
\closepath
\moveto(590.86433864,473.56246513)
\lineto(590.85652615,473.56246513)
\lineto(589.11434218,468.52341284)
\lineto(592.62996006,468.52341284)
\closepath
}
}
{
\newrgbcolor{curcolor}{1 1 1}
\pscustom[linestyle=none,fillstyle=solid,fillcolor=curcolor]
{
\newpath
\moveto(602.32325308,467.63278964)
\curveto(602.32325308,467.15883227)(602.23471159,466.73695812)(602.05762862,466.3671672)
\curveto(601.88575397,465.99737629)(601.64356696,465.68227276)(601.33106759,465.42185662)
\curveto(601.02377654,465.1666488)(600.65919395,464.97394086)(600.2373198,464.84373279)
\curveto(599.82065398,464.71352472)(599.37013406,464.64842068)(598.88576004,464.64842068)
\curveto(598.54721906,464.64842068)(598.23211553,464.67706646)(597.94044945,464.73435801)
\curveto(597.6539917,464.79164956)(597.39617972,464.86196192)(597.16701352,464.94529508)
\curveto(596.94305564,465.02862825)(596.75295186,465.11456557)(596.59670218,465.20310706)
\curveto(596.44566081,465.29164855)(596.3388902,465.36716923)(596.27639032,465.4296691)
\curveto(596.21909877,465.49216898)(596.17482803,465.57029382)(596.14357809,465.66404363)
\curveto(596.11753648,465.76300176)(596.10451567,465.89320983)(596.10451567,466.05466784)
\curveto(596.10451567,466.16925094)(596.10972399,466.26300075)(596.12014064,466.33591727)
\curveto(596.13055729,466.41404211)(596.14618225,466.47654198)(596.16701555,466.52341689)
\curveto(596.18784884,466.57029179)(596.21389045,466.60154173)(596.24514039,466.6171667)
\curveto(596.27639032,466.63799999)(596.31284858,466.64841663)(596.35451517,466.64841663)
\curveto(596.42743168,466.64841663)(596.52899398,466.60414589)(596.65920205,466.5156044)
\curveto(596.79461844,466.42706292)(596.96649309,466.33070894)(597.174826,466.22654249)
\curveto(597.38315892,466.12237603)(597.63315841,466.0234179)(597.92482449,465.92966809)
\curveto(598.22169888,465.8411266)(598.56284403,465.79685586)(598.94825991,465.79685586)
\curveto(599.23992599,465.79685586)(599.50555045,465.83591828)(599.7451333,465.91404312)
\curveto(599.98992447,465.99216796)(600.19825738,466.10154274)(600.37013203,466.24216746)
\curveto(600.54721501,466.3880005)(600.6826314,466.56508347)(600.77638121,466.77341638)
\curveto(600.87013102,466.98174929)(600.91700593,467.21872798)(600.91700593,467.48435244)
\curveto(600.91700593,467.7708102)(600.85190189,468.01560137)(600.72169382,468.21872595)
\curveto(600.59148575,468.42185054)(600.4196111,468.59893352)(600.20606987,468.74997488)
\curveto(599.99252863,468.90622456)(599.74773746,469.04684928)(599.47169635,469.17184902)
\curveto(599.20086357,469.30205709)(598.9222183,469.43226516)(598.63576055,469.56247323)
\curveto(598.34930279,469.69788963)(598.07065752,469.84632683)(597.79982474,470.00778483)
\curveto(597.52899195,470.16924284)(597.28680494,470.35934662)(597.07326371,470.57809618)
\curveto(596.85972248,470.79684573)(596.68524366,471.05205355)(596.54982727,471.34371963)
\curveto(596.4196192,471.64059403)(596.35451517,471.99475998)(596.35451517,472.40621748)
\curveto(596.35451517,472.82809162)(596.43003585,473.20309086)(596.58107721,473.5312152)
\curveto(596.73732689,473.86454786)(596.95086812,474.14319312)(597.22170091,474.367151)
\curveto(597.49774202,474.59110888)(597.82326219,474.76037937)(598.19826143,474.87496248)
\curveto(598.578469,474.9947539)(598.98732233,475.05464961)(599.42482145,475.05464961)
\curveto(599.64877933,475.05464961)(599.87273721,475.03381632)(600.09669509,474.99214974)
\curveto(600.32586129,474.95569148)(600.53940252,474.90360825)(600.73731879,474.83590006)
\curveto(600.94044338,474.77340018)(601.12013052,474.70048366)(601.2763802,474.6171505)
\curveto(601.43262988,474.53902566)(601.53419218,474.47392162)(601.58106708,474.42183839)
\curveto(601.63315031,474.37496349)(601.66700441,474.33590107)(601.68262938,474.30465113)
\curveto(601.69825434,474.27860952)(601.71127515,474.24215126)(601.7216918,474.19527635)
\curveto(601.73210844,474.15360977)(601.73992093,474.10152654)(601.74512925,474.03902667)
\curveto(601.75033757,473.9765268)(601.75294173,473.89579779)(601.75294173,473.79683966)
\curveto(601.75294173,473.70308985)(601.74773341,473.61975668)(601.73731677,473.54684017)
\curveto(601.73210844,473.47392365)(601.7216918,473.41142377)(601.70606683,473.35934055)
\curveto(601.69044186,473.31246564)(601.66700441,473.27600738)(601.63575447,473.24996577)
\curveto(601.60971286,473.22913248)(601.57846292,473.21871583)(601.54200466,473.21871583)
\curveto(601.48471311,473.21871583)(601.39356746,473.25517409)(601.26856771,473.32809061)
\curveto(601.14877629,473.40100713)(601.00033909,473.48173613)(600.82325612,473.57027762)
\curveto(600.64617314,473.66402743)(600.43523607,473.74736059)(600.1904449,473.82027711)
\curveto(599.95086205,473.89840195)(599.68002927,473.93746437)(599.37794654,473.93746437)
\curveto(599.09669711,473.93746437)(598.85190594,473.89840195)(598.64357303,473.82027711)
\curveto(598.43524012,473.74736059)(598.26336547,473.64840246)(598.12794907,473.52340271)
\curveto(597.99253268,473.39840297)(597.89097039,473.24996577)(597.82326219,473.07809111)
\curveto(597.755554,472.90621646)(597.7216999,472.72392517)(597.7216999,472.53121722)
\curveto(597.7216999,472.24996779)(597.78680393,472.00778078)(597.917012,471.80465619)
\curveto(598.04722007,471.6015316)(598.21909472,471.42184447)(598.43263596,471.26559478)
\curveto(598.65138551,471.1093451)(598.89878085,470.96611622)(599.17482195,470.83590816)
\curveto(599.45086306,470.70570009)(599.73211249,470.57288785)(600.01857025,470.43747146)
\curveto(600.305028,470.30726339)(600.58627743,470.16143035)(600.86231854,469.99997235)
\curveto(601.13835965,469.84372266)(601.38315082,469.65622304)(601.59669205,469.43747349)
\curveto(601.81544161,469.22393225)(601.98992042,468.96872444)(602.12012849,468.67185004)
\curveto(602.25554488,468.38018396)(602.32325308,468.0338305)(602.32325308,467.63278964)
\closepath
}
}
{
\newrgbcolor{curcolor}{1 1 1}
\pscustom[linestyle=none,fillstyle=solid,fillcolor=curcolor]
{
\newpath
\moveto(605.58130789,464.99998247)
\curveto(605.58130789,464.95831589)(605.57089125,464.92185763)(605.55005795,464.89060769)
\curveto(605.52922466,464.85935776)(605.4927664,464.83331614)(605.44068318,464.81248285)
\curveto(605.38859995,464.79685788)(605.31828759,464.78383708)(605.2297461,464.77342043)
\curveto(605.14641294,464.75779546)(605.03964232,464.74998298)(604.90943425,464.74998298)
\curveto(604.7844345,464.74998298)(604.67766389,464.75779546)(604.5891224,464.77342043)
\curveto(604.50058091,464.78383708)(604.43026855,464.79685788)(604.37818533,464.81248285)
\curveto(604.3261021,464.83331614)(604.28964384,464.85935776)(604.26881055,464.89060769)
\curveto(604.24797726,464.92185763)(604.23756061,464.95831589)(604.23756061,464.99998247)
\lineto(604.23756061,474.70308782)
\curveto(604.23756061,474.74475441)(604.24797726,474.78121267)(604.26881055,474.8124626)
\curveto(604.29485216,474.84371254)(604.33391458,474.86714999)(604.38599781,474.88277496)
\curveto(604.44328936,474.90360825)(604.51360172,474.91923322)(604.59693488,474.92964987)
\curveto(604.68547637,474.94527483)(604.78964283,474.95308732)(604.90943425,474.95308732)
\curveto(605.03964232,474.95308732)(605.14641294,474.94527483)(605.2297461,474.92964987)
\curveto(605.31828759,474.91923322)(605.38859995,474.90360825)(605.44068318,474.88277496)
\curveto(605.4927664,474.86714999)(605.52922466,474.84371254)(605.55005795,474.8124626)
\curveto(605.57089125,474.78121267)(605.58130789,474.74475441)(605.58130789,474.70308782)
\closepath
}
}
{
\newrgbcolor{curcolor}{1 1 1}
\pscustom[linestyle=none,fillstyle=solid,fillcolor=curcolor]
{
\newpath
\moveto(614.06604146,465.35935675)
\curveto(614.06604146,465.26039861)(614.06083313,465.17446129)(614.05041649,465.10154477)
\curveto(614.03999984,465.03383657)(614.02177071,464.97654502)(613.9957291,464.92967012)
\curveto(613.97489581,464.88279521)(613.94625003,464.84894111)(613.90979177,464.82810782)
\curveto(613.87854184,464.80727453)(613.83947942,464.79685788)(613.79260451,464.79685788)
\lineto(607.8238666,464.79685788)
\curveto(607.6884502,464.79685788)(607.57907543,464.83852447)(607.49574226,464.92185763)
\curveto(607.4124091,465.0051908)(607.37074251,465.13800303)(607.37074251,465.32029432)
\lineto(607.37074251,465.60935624)
\curveto(607.37074251,465.67706444)(607.37334668,465.73956431)(607.378555,465.79685586)
\curveto(607.38897164,465.85414741)(607.40720077,465.91404312)(607.43324239,465.976543)
\curveto(607.459284,466.04425119)(607.49574226,466.11716771)(607.54261717,466.19529255)
\curveto(607.58949207,466.27862572)(607.64678362,466.37497969)(607.71449182,466.48435447)
\lineto(612.37854487,473.78121469)
\lineto(607.76917921,473.78121469)
\curveto(607.71709598,473.78121469)(607.67282524,473.79163134)(607.63636698,473.81246463)
\curveto(607.59990872,473.83329792)(607.56865878,473.86454786)(607.54261717,473.90621444)
\curveto(607.52178388,473.95308934)(607.50615891,474.01038089)(607.49574226,474.07808909)
\curveto(607.48532562,474.15100561)(607.48011729,474.23433877)(607.48011729,474.32808858)
\curveto(607.48011729,474.43225504)(607.48532562,474.52079653)(607.49574226,474.59371305)
\curveto(607.50615891,474.66662956)(607.52178388,474.72652528)(607.54261717,474.77340018)
\curveto(607.56865878,474.82027509)(607.59990872,474.85412918)(607.63636698,474.87496248)
\curveto(607.67282524,474.89579577)(607.71709598,474.90621241)(607.76917921,474.90621241)
\lineto(613.39416782,474.90621241)
\curveto(613.53479253,474.90621241)(613.64416731,474.86454583)(613.72229215,474.78121267)
\curveto(613.80562532,474.70308782)(613.8472919,474.58069224)(613.8472919,474.41402591)
\lineto(613.8472919,474.10933903)
\curveto(613.8472919,474.02600586)(613.84208358,473.95048518)(613.83166693,473.88277699)
\curveto(613.82125029,473.82027711)(613.80302116,473.75517308)(613.77697954,473.68746488)
\curveto(613.75093793,473.61975668)(613.71447967,473.54684017)(613.66760476,473.46871532)
\curveto(613.62593818,473.39059048)(613.56864663,473.29944483)(613.49573011,473.19527838)
\lineto(608.85511451,465.92966809)
\lineto(613.79260451,465.92966809)
\curveto(613.881146,465.92966809)(613.94885419,465.88539735)(613.9957291,465.79685586)
\curveto(614.042604,465.70831437)(614.06604146,465.56248133)(614.06604146,465.35935675)
\closepath
}
}
{
\newrgbcolor{curcolor}{1 1 1}
\pscustom[linestyle=none,fillstyle=solid,fillcolor=curcolor]
{
\newpath
\moveto(621.36652754,465.35154426)
\curveto(621.36652754,465.25779445)(621.36131922,465.17446129)(621.35090257,465.10154477)
\curveto(621.34048593,465.03383657)(621.3222568,464.97654502)(621.29621519,464.92967012)
\curveto(621.27017357,464.88279521)(621.23892363,464.84894111)(621.20246538,464.82810782)
\curveto(621.17121544,464.80727453)(621.13475718,464.79685788)(621.0930906,464.79685788)
\lineto(616.24153792,464.79685788)
\curveto(616.1217465,464.79685788)(616.0071634,464.83592031)(615.89778862,464.91404515)
\curveto(615.79362216,464.99737831)(615.74153893,465.14060719)(615.74153893,465.34373178)
\lineto(615.74153893,474.35933852)
\curveto(615.74153893,474.56246311)(615.79362216,474.70308782)(615.89778862,474.78121267)
\curveto(616.0071634,474.86454583)(616.1217465,474.90621241)(616.24153792,474.90621241)
\lineto(621.03840321,474.90621241)
\curveto(621.08006979,474.90621241)(621.11652805,474.89579577)(621.14777799,474.87496248)
\curveto(621.18423625,474.85412918)(621.21288202,474.82027509)(621.23371531,474.77340018)
\curveto(621.2545486,474.72652528)(621.27017357,474.66662956)(621.28059022,474.59371305)
\curveto(621.29621519,474.52600485)(621.30402767,474.44006752)(621.30402767,474.33590107)
\curveto(621.30402767,474.24215126)(621.29621519,474.15881809)(621.28059022,474.08590157)
\curveto(621.27017357,474.01819338)(621.2545486,473.96090183)(621.23371531,473.91402692)
\curveto(621.21288202,473.87236034)(621.18423625,473.8411104)(621.14777799,473.82027711)
\curveto(621.11652805,473.79944382)(621.08006979,473.78902717)(621.03840321,473.78902717)
\lineto(617.08528621,473.78902717)
\lineto(617.08528621,470.6171586)
\lineto(620.47590435,470.6171586)
\curveto(620.51757093,470.6171586)(620.55402919,470.60413779)(620.58527912,470.57809618)
\curveto(620.62173738,470.55726289)(620.65038316,470.52601295)(620.67121645,470.48434637)
\curveto(620.69725806,470.44267978)(620.71548719,470.38538823)(620.72590384,470.31247172)
\curveto(620.73632049,470.2395552)(620.74152881,470.15361787)(620.74152881,470.05465974)
\curveto(620.74152881,469.96090993)(620.73632049,469.88018092)(620.72590384,469.81247273)
\curveto(620.71548719,469.74476453)(620.69725806,469.69007714)(620.67121645,469.64841056)
\curveto(620.65038316,469.60674398)(620.62173738,469.57549404)(620.58527912,469.55466075)
\curveto(620.55402919,469.53903578)(620.51757093,469.5312233)(620.47590435,469.5312233)
\lineto(617.08528621,469.5312233)
\lineto(617.08528621,465.91404312)
\lineto(621.0930906,465.91404312)
\curveto(621.13475718,465.91404312)(621.17121544,465.90362648)(621.20246538,465.88279319)
\curveto(621.23892363,465.86195989)(621.27017357,465.8281058)(621.29621519,465.78123089)
\curveto(621.3222568,465.73956431)(621.34048593,465.68227276)(621.35090257,465.60935624)
\curveto(621.36131922,465.54164804)(621.36652754,465.45571072)(621.36652754,465.35154426)
\closepath
}
}
{
\newrgbcolor{curcolor}{0 0 0}
\pscustom[linestyle=none,fillstyle=solid,fillcolor=curcolor,opacity=0]
{
\newpath
\moveto(577.30184983,421.34600186)
\lineto(665.90009561,421.34600186)
\lineto(665.90009561,384.55867478)
\lineto(577.30184983,384.55867478)
\closepath
}
}
{
\newrgbcolor{curcolor}{1 1 1}
\pscustom[linestyle=none,fillstyle=solid,fillcolor=curcolor]
{
\newpath
\moveto(595.21589233,397.46263674)
\curveto(595.25755891,397.34805363)(595.2783922,397.25430382)(595.2783922,397.18138731)
\curveto(595.28360052,397.11367911)(595.26537139,397.06159588)(595.22370481,397.02513762)
\curveto(595.18203823,396.98867936)(595.11172587,396.96524191)(595.01276774,396.95482526)
\curveto(594.91901793,396.94440862)(594.79141402,396.9392003)(594.62995601,396.9392003)
\curveto(594.46849801,396.9392003)(594.33828994,396.94440862)(594.2393318,396.95482526)
\curveto(594.14558199,396.96003359)(594.07266548,396.97045023)(594.02058225,396.9860752)
\curveto(593.97370734,397.00690849)(593.93724908,397.03295011)(593.91120747,397.06420004)
\curveto(593.89037418,397.09544998)(593.86954089,397.1345124)(593.8487076,397.18138731)
\lineto(592.98152185,399.64231982)
\lineto(588.77840536,399.64231982)
\lineto(587.95028204,397.21263724)
\curveto(587.93465707,397.16576234)(587.91382378,397.12409576)(587.88778217,397.0876375)
\curveto(587.86174055,397.05638756)(587.82267813,397.02774178)(587.7705949,397.00170017)
\curveto(587.72372,396.98086688)(587.65340764,396.96524191)(587.55965783,396.95482526)
\curveto(587.47111634,396.94440862)(587.35392908,396.9392003)(587.20809604,396.9392003)
\curveto(587.05705468,396.9392003)(586.93205493,396.94701278)(586.8330968,396.96263775)
\curveto(586.73934699,396.97305439)(586.6716388,396.99649185)(586.62997221,397.03295011)
\curveto(586.59351395,397.06940837)(586.57528482,397.12149159)(586.57528482,397.18919979)
\curveto(586.58049315,397.26211631)(586.6039306,397.35586612)(586.64559718,397.47044922)
\lineto(590.03621532,406.8610552)
\curveto(590.05704861,406.91834676)(590.08309022,406.96522166)(590.11434016,407.00167992)
\curveto(590.15079842,407.03813818)(590.20027748,407.06678395)(590.26277736,407.08761725)
\curveto(590.33048555,407.10845054)(590.41381872,407.12147134)(590.51277685,407.12667967)
\curveto(590.61694331,407.13709631)(590.74715138,407.14230463)(590.90340106,407.14230463)
\curveto(591.07006739,407.14230463)(591.20808794,407.13709631)(591.31746272,407.12667967)
\curveto(591.4268375,407.12147134)(591.51537899,407.10845054)(591.58308718,407.08761725)
\curveto(591.65079538,407.06678395)(591.70287861,407.03553402)(591.73933687,406.99386744)
\curveto(591.77579513,406.95740918)(591.8044409,406.91053427)(591.82527419,406.85324272)
\closepath
\moveto(590.86433864,405.75168245)
\lineto(590.85652615,405.75168245)
\lineto(589.11434218,400.71263015)
\lineto(592.62996006,400.71263015)
\closepath
}
}
{
\newrgbcolor{curcolor}{1 1 1}
\pscustom[linestyle=none,fillstyle=solid,fillcolor=curcolor]
{
\newpath
\moveto(602.32325308,399.82200696)
\curveto(602.32325308,399.34804958)(602.23471159,398.92617544)(602.05762862,398.55638452)
\curveto(601.88575397,398.1865936)(601.64356696,397.87149007)(601.33106759,397.61107394)
\curveto(601.02377654,397.35586612)(600.65919395,397.16315818)(600.2373198,397.03295011)
\curveto(599.82065398,396.90274204)(599.37013406,396.837638)(598.88576004,396.837638)
\curveto(598.54721906,396.837638)(598.23211553,396.86628378)(597.94044945,396.92357533)
\curveto(597.6539917,396.98086688)(597.39617972,397.05117924)(597.16701352,397.1345124)
\curveto(596.94305564,397.21784557)(596.75295186,397.30378289)(596.59670218,397.39232438)
\curveto(596.44566081,397.48086587)(596.3388902,397.55638655)(596.27639032,397.61888642)
\curveto(596.21909877,397.68138629)(596.17482803,397.75951114)(596.14357809,397.85326095)
\curveto(596.11753648,397.95221908)(596.10451567,398.08242715)(596.10451567,398.24388515)
\curveto(596.10451567,398.35846826)(596.10972399,398.45221807)(596.12014064,398.52513458)
\curveto(596.13055729,398.60325943)(596.14618225,398.6657593)(596.16701555,398.7126342)
\curveto(596.18784884,398.75950911)(596.21389045,398.79075905)(596.24514039,398.80638402)
\curveto(596.27639032,398.82721731)(596.31284858,398.83763395)(596.35451517,398.83763395)
\curveto(596.42743168,398.83763395)(596.52899398,398.79336321)(596.65920205,398.70482172)
\curveto(596.79461844,398.61628023)(596.96649309,398.51992626)(597.174826,398.41575981)
\curveto(597.38315892,398.31159335)(597.63315841,398.21263522)(597.92482449,398.11888541)
\curveto(598.22169888,398.03034392)(598.56284403,397.98607318)(598.94825991,397.98607318)
\curveto(599.23992599,397.98607318)(599.50555045,398.0251356)(599.7451333,398.10326044)
\curveto(599.98992447,398.18138528)(600.19825738,398.29076006)(600.37013203,398.43138477)
\curveto(600.54721501,398.57721781)(600.6826314,398.75430079)(600.77638121,398.9626337)
\curveto(600.87013102,399.17096661)(600.91700593,399.4079453)(600.91700593,399.67356976)
\curveto(600.91700593,399.96002751)(600.85190189,400.20481868)(600.72169382,400.40794327)
\curveto(600.59148575,400.61106786)(600.4196111,400.78815084)(600.20606987,400.9391922)
\curveto(599.99252863,401.09544188)(599.74773746,401.2360666)(599.47169635,401.36106634)
\curveto(599.20086357,401.49127441)(598.9222183,401.62148248)(598.63576055,401.75169055)
\curveto(598.34930279,401.88710694)(598.07065752,402.03554414)(597.79982474,402.19700215)
\curveto(597.52899195,402.35846016)(597.28680494,402.54856394)(597.07326371,402.76731349)
\curveto(596.85972248,402.98606305)(596.68524366,403.24127087)(596.54982727,403.53293694)
\curveto(596.4196192,403.82981134)(596.35451517,404.18397729)(596.35451517,404.59543479)
\curveto(596.35451517,405.01730894)(596.43003585,405.39230818)(596.58107721,405.72043251)
\curveto(596.73732689,406.05376517)(596.95086812,406.33241044)(597.22170091,406.55636832)
\curveto(597.49774202,406.7803262)(597.82326219,406.94959669)(598.19826143,407.06417979)
\curveto(598.578469,407.18397122)(598.98732233,407.24386693)(599.42482145,407.24386693)
\curveto(599.64877933,407.24386693)(599.87273721,407.22303364)(600.09669509,407.18136706)
\curveto(600.32586129,407.1449088)(600.53940252,407.09282557)(600.73731879,407.02511737)
\curveto(600.94044338,406.9626175)(601.12013052,406.88970098)(601.2763802,406.80636782)
\curveto(601.43262988,406.72824297)(601.53419218,406.66313894)(601.58106708,406.61105571)
\curveto(601.63315031,406.56418081)(601.66700441,406.52511838)(601.68262938,406.49386845)
\curveto(601.69825434,406.46782683)(601.71127515,406.43136857)(601.7216918,406.38449367)
\curveto(601.73210844,406.34282709)(601.73992093,406.29074386)(601.74512925,406.22824399)
\curveto(601.75033757,406.16574411)(601.75294173,406.08501511)(601.75294173,405.98605698)
\curveto(601.75294173,405.89230717)(601.74773341,405.808974)(601.73731677,405.73605748)
\curveto(601.73210844,405.66314096)(601.7216918,405.60064109)(601.70606683,405.54855786)
\curveto(601.69044186,405.50168296)(601.66700441,405.4652247)(601.63575447,405.43918308)
\curveto(601.60971286,405.41834979)(601.57846292,405.40793315)(601.54200466,405.40793315)
\curveto(601.48471311,405.40793315)(601.39356746,405.44439141)(601.26856771,405.51730793)
\curveto(601.14877629,405.59022444)(601.00033909,405.67095345)(600.82325612,405.75949494)
\curveto(600.64617314,405.85324475)(600.43523607,405.93657791)(600.1904449,406.00949443)
\curveto(599.95086205,406.08761927)(599.68002927,406.12668169)(599.37794654,406.12668169)
\curveto(599.09669711,406.12668169)(598.85190594,406.08761927)(598.64357303,406.00949443)
\curveto(598.43524012,405.93657791)(598.26336547,405.83761978)(598.12794907,405.71262003)
\curveto(597.99253268,405.58762028)(597.89097039,405.43918308)(597.82326219,405.26730843)
\curveto(597.755554,405.09543378)(597.7216999,404.91314248)(597.7216999,404.72043454)
\curveto(597.7216999,404.43918511)(597.78680393,404.1969981)(597.917012,403.99387351)
\curveto(598.04722007,403.79074892)(598.21909472,403.61106179)(598.43263596,403.4548121)
\curveto(598.65138551,403.29856242)(598.89878085,403.15533354)(599.17482195,403.02512547)
\curveto(599.45086306,402.8949174)(599.73211249,402.76210517)(600.01857025,402.62668878)
\curveto(600.305028,402.49648071)(600.58627743,402.35064767)(600.86231854,402.18918966)
\curveto(601.13835965,402.03293998)(601.38315082,401.84544036)(601.59669205,401.6266908)
\curveto(601.81544161,401.41314957)(601.98992042,401.15794175)(602.12012849,400.86106735)
\curveto(602.25554488,400.56940128)(602.32325308,400.22304781)(602.32325308,399.82200696)
\closepath
}
}
{
\newrgbcolor{curcolor}{1 1 1}
\pscustom[linestyle=none,fillstyle=solid,fillcolor=curcolor]
{
\newpath
\moveto(605.58130789,397.18919979)
\curveto(605.58130789,397.14753321)(605.57089125,397.11107495)(605.55005795,397.07982501)
\curveto(605.52922466,397.04857507)(605.4927664,397.02253346)(605.44068318,397.00170017)
\curveto(605.38859995,396.9860752)(605.31828759,396.97305439)(605.2297461,396.96263775)
\curveto(605.14641294,396.94701278)(605.03964232,396.9392003)(604.90943425,396.9392003)
\curveto(604.7844345,396.9392003)(604.67766389,396.94701278)(604.5891224,396.96263775)
\curveto(604.50058091,396.97305439)(604.43026855,396.9860752)(604.37818533,397.00170017)
\curveto(604.3261021,397.02253346)(604.28964384,397.04857507)(604.26881055,397.07982501)
\curveto(604.24797726,397.11107495)(604.23756061,397.14753321)(604.23756061,397.18919979)
\lineto(604.23756061,406.89230514)
\curveto(604.23756061,406.93397172)(604.24797726,406.97042998)(604.26881055,407.00167992)
\curveto(604.29485216,407.03292986)(604.33391458,407.05636731)(604.38599781,407.07199228)
\curveto(604.44328936,407.09282557)(604.51360172,407.10845054)(604.59693488,407.11886718)
\curveto(604.68547637,407.13449215)(604.78964283,407.14230463)(604.90943425,407.14230463)
\curveto(605.03964232,407.14230463)(605.14641294,407.13449215)(605.2297461,407.11886718)
\curveto(605.31828759,407.10845054)(605.38859995,407.09282557)(605.44068318,407.07199228)
\curveto(605.4927664,407.05636731)(605.52922466,407.03292986)(605.55005795,407.00167992)
\curveto(605.57089125,406.97042998)(605.58130789,406.93397172)(605.58130789,406.89230514)
\closepath
}
}
{
\newrgbcolor{curcolor}{1 1 1}
\pscustom[linestyle=none,fillstyle=solid,fillcolor=curcolor]
{
\newpath
\moveto(614.06604146,397.54857406)
\curveto(614.06604146,397.44961593)(614.06083313,397.3636786)(614.05041649,397.29076208)
\curveto(614.03999984,397.22305389)(614.02177071,397.16576234)(613.9957291,397.11888743)
\curveto(613.97489581,397.07201253)(613.94625003,397.03815843)(613.90979177,397.01732514)
\curveto(613.87854184,396.99649185)(613.83947942,396.9860752)(613.79260451,396.9860752)
\lineto(607.8238666,396.9860752)
\curveto(607.6884502,396.9860752)(607.57907543,397.02774178)(607.49574226,397.11107495)
\curveto(607.4124091,397.19440811)(607.37074251,397.32722034)(607.37074251,397.50951164)
\lineto(607.37074251,397.79857356)
\curveto(607.37074251,397.86628175)(607.37334668,397.92878163)(607.378555,397.98607318)
\curveto(607.38897164,398.04336473)(607.40720077,398.10326044)(607.43324239,398.16576031)
\curveto(607.459284,398.23346851)(607.49574226,398.30638503)(607.54261717,398.38450987)
\curveto(607.58949207,398.46784303)(607.64678362,398.56419701)(607.71449182,398.67357178)
\lineto(612.37854487,405.97043201)
\lineto(607.76917921,405.97043201)
\curveto(607.71709598,405.97043201)(607.67282524,405.98084865)(607.63636698,406.00168194)
\curveto(607.59990872,406.02251524)(607.56865878,406.05376517)(607.54261717,406.09543175)
\curveto(607.52178388,406.14230666)(607.50615891,406.19959821)(607.49574226,406.26730641)
\curveto(607.48532562,406.34022293)(607.48011729,406.42355609)(607.48011729,406.5173059)
\curveto(607.48011729,406.62147236)(607.48532562,406.71001384)(607.49574226,406.78293036)
\curveto(607.50615891,406.85584688)(607.52178388,406.91574259)(607.54261717,406.9626175)
\curveto(607.56865878,407.0094924)(607.59990872,407.0433465)(607.63636698,407.06417979)
\curveto(607.67282524,407.08501308)(607.71709598,407.09542973)(607.76917921,407.09542973)
\lineto(613.39416782,407.09542973)
\curveto(613.53479253,407.09542973)(613.64416731,407.05376315)(613.72229215,406.97042998)
\curveto(613.80562532,406.89230514)(613.8472919,406.76990956)(613.8472919,406.60324323)
\lineto(613.8472919,406.29855634)
\curveto(613.8472919,406.21522318)(613.84208358,406.1397025)(613.83166693,406.0719943)
\curveto(613.82125029,406.00949443)(613.80302116,405.94439039)(613.77697954,405.8766822)
\curveto(613.75093793,405.808974)(613.71447967,405.73605748)(613.66760476,405.65793264)
\curveto(613.62593818,405.5798078)(613.56864663,405.48866215)(613.49573011,405.38449569)
\lineto(608.85511451,398.11888541)
\lineto(613.79260451,398.11888541)
\curveto(613.881146,398.11888541)(613.94885419,398.07461466)(613.9957291,397.98607318)
\curveto(614.042604,397.89753169)(614.06604146,397.75169865)(614.06604146,397.54857406)
\closepath
}
}
{
\newrgbcolor{curcolor}{1 1 1}
\pscustom[linestyle=none,fillstyle=solid,fillcolor=curcolor]
{
\newpath
\moveto(621.36652754,397.54076158)
\curveto(621.36652754,397.44701177)(621.36131922,397.3636786)(621.35090257,397.29076208)
\curveto(621.34048593,397.22305389)(621.3222568,397.16576234)(621.29621519,397.11888743)
\curveto(621.27017357,397.07201253)(621.23892363,397.03815843)(621.20246538,397.01732514)
\curveto(621.17121544,396.99649185)(621.13475718,396.9860752)(621.0930906,396.9860752)
\lineto(616.24153792,396.9860752)
\curveto(616.1217465,396.9860752)(616.0071634,397.02513762)(615.89778862,397.10326246)
\curveto(615.79362216,397.18659563)(615.74153893,397.32982451)(615.74153893,397.53294909)
\lineto(615.74153893,406.54855584)
\curveto(615.74153893,406.75168043)(615.79362216,406.89230514)(615.89778862,406.97042998)
\curveto(616.0071634,407.05376315)(616.1217465,407.09542973)(616.24153792,407.09542973)
\lineto(621.03840321,407.09542973)
\curveto(621.08006979,407.09542973)(621.11652805,407.08501308)(621.14777799,407.06417979)
\curveto(621.18423625,407.0433465)(621.21288202,407.0094924)(621.23371531,406.9626175)
\curveto(621.2545486,406.91574259)(621.27017357,406.85584688)(621.28059022,406.78293036)
\curveto(621.29621519,406.71522217)(621.30402767,406.62928484)(621.30402767,406.52511838)
\curveto(621.30402767,406.43136857)(621.29621519,406.34803541)(621.28059022,406.27511889)
\curveto(621.27017357,406.20741069)(621.2545486,406.15011914)(621.23371531,406.10324424)
\curveto(621.21288202,406.06157766)(621.18423625,406.03032772)(621.14777799,406.00949443)
\curveto(621.11652805,405.98866114)(621.08006979,405.97824449)(621.03840321,405.97824449)
\lineto(617.08528621,405.97824449)
\lineto(617.08528621,402.80637592)
\lineto(620.47590435,402.80637592)
\curveto(620.51757093,402.80637592)(620.55402919,402.79335511)(620.58527912,402.76731349)
\curveto(620.62173738,402.7464802)(620.65038316,402.71523027)(620.67121645,402.67356368)
\curveto(620.69725806,402.6318971)(620.71548719,402.57460555)(620.72590384,402.50168903)
\curveto(620.73632049,402.42877251)(620.74152881,402.34283519)(620.74152881,402.24387705)
\curveto(620.74152881,402.15012724)(620.73632049,402.06939824)(620.72590384,402.00169004)
\curveto(620.71548719,401.93398185)(620.69725806,401.87929446)(620.67121645,401.83762788)
\curveto(620.65038316,401.79596129)(620.62173738,401.76471136)(620.58527912,401.74387807)
\curveto(620.55402919,401.7282531)(620.51757093,401.72044061)(620.47590435,401.72044061)
\lineto(617.08528621,401.72044061)
\lineto(617.08528621,398.10326044)
\lineto(621.0930906,398.10326044)
\curveto(621.13475718,398.10326044)(621.17121544,398.09284379)(621.20246538,398.0720105)
\curveto(621.23892363,398.05117721)(621.27017357,398.01732311)(621.29621519,397.97044821)
\curveto(621.3222568,397.92878163)(621.34048593,397.87149007)(621.35090257,397.79857356)
\curveto(621.36131922,397.73086536)(621.36652754,397.64492803)(621.36652754,397.54076158)
\closepath
}
}
{
\newrgbcolor{curcolor}{0 0 0}
\pscustom[linestyle=none,fillstyle=solid,fillcolor=curcolor,opacity=0]
{
\newpath
\moveto(335.54123703,348.91669971)
\lineto(399.10016344,348.91669971)
\lineto(399.10016344,312.12937263)
\lineto(335.54123703,312.12937263)
\closepath
}
}
{
\newrgbcolor{curcolor}{0.94509804 0.99215686 0.99607843}
\pscustom[linestyle=none,fillstyle=solid,fillcolor=curcolor]
{
\newpath
\moveto(349.74950707,328.17141939)
\curveto(349.74950707,328.097635)(349.74733694,328.03253113)(349.74299668,327.97610777)
\curveto(349.73865642,327.92402468)(349.72997591,327.87845197)(349.71695513,327.83938965)
\curveto(349.70827462,327.80466758)(349.69525384,327.77211565)(349.67789281,327.74173384)
\curveto(349.66487204,327.71569229)(349.63014997,327.67445984)(349.57372662,327.61803649)
\curveto(349.52164352,327.56595339)(349.4304981,327.49867939)(349.30029036,327.41621449)
\curveto(349.17008262,327.33808985)(349.02251385,327.26647559)(348.85758404,327.20137172)
\curveto(348.6969945,327.14060811)(348.52121405,327.09069514)(348.3302427,327.05163282)
\curveto(348.13927134,327.01257049)(347.9417896,326.99303933)(347.73779748,326.99303933)
\curveto(347.31679245,326.99303933)(346.94353026,327.06248346)(346.61801091,327.20137172)
\curveto(346.29249155,327.34025997)(346.0190553,327.54208197)(345.79770214,327.80683771)
\curveto(345.58068924,328.07593371)(345.41358931,328.40362319)(345.29640234,328.78990615)
\curveto(345.18355563,329.18052938)(345.12713228,329.62974608)(345.12713228,330.13755627)
\curveto(345.12713228,330.71481059)(345.1965764,331.2096)(345.33546466,331.62192451)
\curveto(345.47869318,332.03858928)(345.67183466,332.37929954)(345.91488911,332.64405528)
\curveto(346.16228381,332.90881101)(346.45091097,333.10412263)(346.78077058,333.22999011)
\curveto(347.11497045,333.36019785)(347.47521187,333.42530172)(347.86149483,333.42530172)
\curveto(348.04812592,333.42530172)(348.22824663,333.40794069)(348.40185695,333.37321862)
\curveto(348.57980753,333.33849656)(348.74256721,333.29292385)(348.89013598,333.23650049)
\curveto(349.03770475,333.18007714)(349.16791249,333.11497327)(349.2807592,333.04118888)
\curveto(349.39794617,332.9674045)(349.4825812,332.90447076)(349.5346643,332.85238766)
\curveto(349.58674739,332.80030456)(349.62146946,332.75907211)(349.63883049,332.72869031)
\curveto(349.66053178,332.6983085)(349.67789281,332.66141631)(349.69091358,332.61801373)
\curveto(349.70393436,332.5789514)(349.71261487,332.5333787)(349.71695513,332.4812956)
\curveto(349.72129539,332.4292125)(349.72346552,332.36410863)(349.72346552,332.28598399)
\curveto(349.72346552,332.11671393)(349.70393436,331.99735683)(349.66487204,331.9279127)
\curveto(349.62580971,331.86280883)(349.57806688,331.8302569)(349.52164352,331.8302569)
\curveto(349.45653965,331.8302569)(349.38058514,331.86497896)(349.29377998,331.93442309)
\curveto(349.21131507,332.00820748)(349.10497875,332.08850225)(348.97477101,332.17530741)
\curveto(348.84456327,332.26211257)(348.68614385,332.34023721)(348.49951276,332.40968134)
\curveto(348.31722192,332.48346573)(348.10020902,332.52035792)(347.84847406,332.52035792)
\curveto(347.33198335,332.52035792)(346.93484974,332.32070605)(346.65707323,331.92140231)
\curveto(346.38363697,331.52643884)(346.24691885,330.95135465)(346.24691885,330.19614975)
\curveto(346.24691885,329.8185473)(346.28164091,329.48651757)(346.35108504,329.20006054)
\curveto(346.42486942,328.91794377)(346.53120575,328.6813997)(346.670094,328.49042835)
\curveto(346.80898226,328.299457)(346.97825232,328.15622848)(347.17790419,328.06074281)
\curveto(347.38189632,327.96959739)(347.61410012,327.92402468)(347.8745156,327.92402468)
\curveto(348.12191031,327.92402468)(348.33892321,327.963087)(348.52555431,328.04121165)
\curveto(348.7121854,328.11933629)(348.87277495,328.20397132)(349.00732295,328.29511674)
\curveto(349.1462112,328.39060242)(349.26122804,328.47523745)(349.35237346,328.54902183)
\curveto(349.44785914,328.62714648)(349.52164352,328.6662088)(349.57372662,328.6662088)
\curveto(349.60410842,328.6662088)(349.63014997,328.65752828)(349.65185126,328.64016725)
\curveto(349.67355255,328.62280622)(349.69091358,328.59242441)(349.70393436,328.54902183)
\curveto(349.72129539,328.50995951)(349.73214604,328.45787642)(349.73648629,328.39277254)
\curveto(349.74516681,328.33200893)(349.74950707,328.25822455)(349.74950707,328.17141939)
\closepath
}
}
{
\newrgbcolor{curcolor}{0.94509804 0.99215686 0.99607843}
\pscustom[linestyle=none,fillstyle=solid,fillcolor=curcolor]
{
\newpath
\moveto(355.97511783,327.28600675)
\curveto(355.97511783,327.25128469)(355.96643731,327.22090288)(355.94907628,327.19486133)
\curveto(355.93171525,327.16881978)(355.90133344,327.14711849)(355.85793086,327.12975746)
\curveto(355.81886854,327.11239643)(355.76244518,327.09937565)(355.6886608,327.09069514)
\curveto(355.61487641,327.08201462)(355.52156086,327.07767436)(355.40871415,327.07767436)
\curveto(355.29152719,327.07767436)(355.19387138,327.08201462)(355.11574674,327.09069514)
\curveto(355.04196235,327.0950354)(354.97902861,327.10371591)(354.92694551,327.11673669)
\curveto(354.87486242,327.13409772)(354.83145984,327.15579901)(354.79673777,327.18184056)
\curveto(354.76635597,327.21222236)(354.73814429,327.24694443)(354.71210274,327.28600675)
\lineto(352.23815567,330.52817949)
\lineto(352.23815567,327.27949636)
\curveto(352.23815567,327.2447743)(352.22947516,327.21439249)(352.21211412,327.18835094)
\curveto(352.19475309,327.16664965)(352.16654141,327.14711849)(352.12747909,327.12975746)
\curveto(352.08841677,327.11239643)(352.03416354,327.09937565)(351.96471942,327.09069514)
\curveto(351.89527529,327.08201462)(351.8063,327.07767436)(351.69779355,327.07767436)
\curveto(351.59362736,327.07767436)(351.50682219,327.08201462)(351.43737807,327.09069514)
\curveto(351.36793394,327.09937565)(351.31151058,327.11239643)(351.268108,327.12975746)
\curveto(351.22904568,327.14711849)(351.200834,327.16664965)(351.18347297,327.18835094)
\curveto(351.1704522,327.21439249)(351.16394181,327.2447743)(351.16394181,327.27949636)
\lineto(351.16394181,335.97737343)
\curveto(351.16394181,336.0120955)(351.1704522,336.0424773)(351.18347297,336.06851885)
\curveto(351.200834,336.0945604)(351.22904568,336.11626169)(351.268108,336.13362272)
\curveto(351.31151058,336.15098376)(351.36793394,336.16400453)(351.43737807,336.17268505)
\curveto(351.50682219,336.18136556)(351.59362736,336.18570582)(351.69779355,336.18570582)
\curveto(351.8063,336.18570582)(351.89527529,336.18136556)(351.96471942,336.17268505)
\curveto(352.03416354,336.16400453)(352.08841677,336.15098376)(352.12747909,336.13362272)
\curveto(352.16654141,336.11626169)(352.19475309,336.0945604)(352.21211412,336.06851885)
\curveto(352.22947516,336.0424773)(352.23815567,336.0120955)(352.23815567,335.97737343)
\lineto(352.23815567,330.67791839)
\lineto(354.45168726,333.11280314)
\curveto(354.48640933,333.15620572)(354.52330152,333.19092779)(354.56236384,333.21696933)
\curveto(354.60142616,333.24735114)(354.64699887,333.27122256)(354.69908197,333.28858359)
\curveto(354.75550532,333.31028488)(354.82060919,333.32330566)(354.89439358,333.32764591)
\curveto(354.96817797,333.33632643)(355.05715326,333.34066669)(355.16131945,333.34066669)
\curveto(355.2698259,333.34066669)(355.36097132,333.33632643)(355.4347557,333.32764591)
\curveto(355.50854009,333.32330566)(355.56713357,333.31245501)(355.61053615,333.29509398)
\curveto(355.65827899,333.2820732)(355.69083093,333.26254204)(355.70819196,333.23650049)
\curveto(355.72989325,333.2147992)(355.74074389,333.1844174)(355.74074389,333.14535508)
\curveto(355.74074389,333.09327198)(355.72555299,333.04118888)(355.69517118,332.98910579)
\curveto(355.66912964,332.93702269)(355.62355693,332.87625908)(355.55845306,332.80681495)
\lineto(353.43606688,330.68442878)
\lineto(355.81886854,327.59199494)
\curveto(355.87529189,327.51821055)(355.91435421,327.45744694)(355.9360555,327.4097041)
\curveto(355.96209705,327.36630152)(355.97511783,327.32506907)(355.97511783,327.28600675)
\closepath
}
}
{
\newrgbcolor{curcolor}{0.94509804 0.99215686 0.99607843}
\pscustom[linestyle=none,fillstyle=solid,fillcolor=curcolor]
{
\newpath
\moveto(360.77563139,328.87454118)
\curveto(360.77563139,328.57506338)(360.71920803,328.30813751)(360.60636133,328.07376358)
\curveto(360.49785487,327.83938965)(360.34160559,327.64190791)(360.13761346,327.48131836)
\curveto(359.93362133,327.32072881)(359.69056688,327.19920159)(359.40845011,327.11673669)
\curveto(359.12633334,327.03427178)(358.81600489,326.99303933)(358.47746477,326.99303933)
\curveto(358.26913238,326.99303933)(358.06948051,327.01040037)(357.87850916,327.04512243)
\curveto(357.69187807,327.07550424)(357.522608,327.11456656)(357.37069897,327.1623094)
\curveto(357.2231302,327.21439249)(357.09726272,327.26647559)(356.99309652,327.31855868)
\curveto(356.88893033,327.37498204)(356.81297582,327.42489501)(356.76523298,327.46829759)
\curveto(356.71749014,327.51170017)(356.68276808,327.57246378)(356.66106679,327.65058842)
\curveto(356.6393655,327.72871307)(356.62851485,327.83504939)(356.62851485,327.96959739)
\curveto(356.62851485,328.05206229)(356.63285511,328.12150642)(356.64153562,328.17792977)
\curveto(356.65021614,328.23435313)(356.66106679,328.27992584)(356.67408756,328.3146479)
\curveto(356.68710833,328.34936996)(356.70446937,328.37324138)(356.72617066,328.38626216)
\curveto(356.7522122,328.40362319)(356.78042388,328.41230371)(356.81080569,328.41230371)
\curveto(356.85854853,328.41230371)(356.92799265,328.3819219)(357.01913807,328.32115829)
\curveto(357.11462375,328.26473493)(357.22964059,328.20180119)(357.36418858,328.13235706)
\curveto(357.50307684,328.06291294)(357.66583652,327.99780906)(357.85246761,327.93704545)
\curveto(358.03909871,327.8806221)(358.25394148,327.85241042)(358.49699593,327.85241042)
\curveto(358.67928676,327.85241042)(358.84421657,327.87194158)(358.99178534,327.9110039)
\curveto(359.13935411,327.95006623)(359.26739173,328.00648958)(359.37589818,328.08027397)
\curveto(359.48440463,328.15839861)(359.56686953,328.25605442)(359.62329288,328.37324138)
\curveto(359.6840565,328.49042835)(359.7144383,328.62931661)(359.7144383,328.78990615)
\curveto(359.7144383,328.95483596)(359.67103572,329.09372421)(359.58423056,329.20657092)
\curveto(359.50176566,329.31941763)(359.39108908,329.41924357)(359.25220082,329.50604873)
\curveto(359.11331257,329.59285389)(358.95706328,329.6688084)(358.78345296,329.73391227)
\curveto(358.60984264,329.8033564)(358.42972193,329.87497066)(358.24309083,329.94875504)
\curveto(358.0608,330.02253943)(357.88067929,330.10500433)(357.70272871,330.19614975)
\curveto(357.52911839,330.29163543)(357.3728691,330.40665227)(357.23398084,330.54120026)
\curveto(357.09509259,330.67574826)(356.98224588,330.83633781)(356.89544072,331.0229689)
\curveto(356.81297582,331.2096)(356.77174336,331.43312329)(356.77174336,331.69353877)
\curveto(356.77174336,331.92357244)(356.81514595,332.14275547)(356.90195111,332.35108786)
\curveto(356.99309652,332.5637605)(357.12764452,332.74822147)(357.3055951,332.90447076)
\curveto(357.48354568,333.0650603)(357.70489884,333.19309791)(357.96965458,333.28858359)
\curveto(358.23875058,333.38406927)(358.55124915,333.43181211)(358.90715031,333.43181211)
\curveto(359.0633996,333.43181211)(359.21964889,333.41879133)(359.37589818,333.39274978)
\curveto(359.53214747,333.36670824)(359.67320585,333.3341563)(359.79907333,333.29509398)
\curveto(359.92494082,333.25603166)(360.03127714,333.21262908)(360.1180823,333.16488624)
\curveto(360.20922772,333.12148366)(360.27650172,333.08242134)(360.3199043,333.04769927)
\curveto(360.36764713,333.01297721)(360.39802894,332.9825954)(360.41104971,332.95655385)
\curveto(360.42841075,332.9305123)(360.43926139,332.9001305)(360.44360165,332.86540843)
\curveto(360.45228217,332.83502663)(360.45879255,332.79596431)(360.46313281,332.74822147)
\curveto(360.47181333,332.70047863)(360.47615358,332.64188515)(360.47615358,332.57244102)
\curveto(360.47615358,332.49865663)(360.47181333,332.43355276)(360.46313281,332.37712941)
\curveto(360.45879255,332.32504631)(360.44794191,332.28164373)(360.43058088,332.24692167)
\curveto(360.4175601,332.2121996)(360.40019907,332.18615805)(360.37849778,332.16879702)
\curveto(360.35679649,332.15577625)(360.33292507,332.14926586)(360.30688352,332.14926586)
\curveto(360.2678212,332.14926586)(360.21139785,332.17313728)(360.13761346,332.22088012)
\curveto(360.06382907,332.26862296)(359.9683434,332.31853592)(359.85115643,332.37061902)
\curveto(359.73396946,332.42704237)(359.59508121,332.47912547)(359.43449166,332.52686831)
\curveto(359.27824237,332.57461115)(359.09812166,332.59848257)(358.89412954,332.59848257)
\curveto(358.7118387,332.59848257)(358.55124915,332.57678128)(358.4123609,332.5333787)
\curveto(358.27347264,332.49431637)(358.1584558,332.43572289)(358.06731038,332.35759825)
\curveto(357.98050522,332.28381386)(357.91323122,332.19483857)(357.86548839,332.09067238)
\curveto(357.82208581,331.98650619)(357.80038452,331.87365948)(357.80038452,331.75213225)
\curveto(357.80038452,331.58286219)(357.8437871,331.43963367)(357.93059226,331.32244671)
\curveto(358.01739742,331.2096)(358.13024413,331.10977407)(358.26913238,331.0229689)
\curveto(358.40802064,330.93616374)(358.56644006,330.8580391)(358.74439063,330.78859497)
\curveto(358.92234121,330.71915084)(359.10246192,330.64753659)(359.28475276,330.5737522)
\curveto(359.47138385,330.49996781)(359.65367469,330.41750291)(359.83162527,330.32635749)
\curveto(360.01391611,330.23521207)(360.17450565,330.12453549)(360.31339391,329.99432775)
\curveto(360.45228217,329.86412001)(360.56295875,329.70787072)(360.64542365,329.52557989)
\curveto(360.73222881,329.34328905)(360.77563139,329.12627615)(360.77563139,328.87454118)
\closepath
}
}
{
\newrgbcolor{curcolor}{0.94509804 0.99215686 0.99607843}
\pscustom[linestyle=none,fillstyle=solid,fillcolor=curcolor]
{
\newpath
\moveto(367.33009369,327.27949636)
\curveto(367.33009369,327.2447743)(367.32141317,327.21439249)(367.30405214,327.18835094)
\curveto(367.29103137,327.16664965)(367.26498982,327.14711849)(367.2259275,327.12975746)
\curveto(367.18686517,327.11239643)(367.13478208,327.09937565)(367.06967821,327.09069514)
\curveto(367.0089146,327.08201462)(366.93296008,327.07767436)(366.84181466,327.07767436)
\curveto(366.74198873,327.07767436)(366.65952383,327.08201462)(366.59441995,327.09069514)
\curveto(366.53365634,327.09937565)(366.48374338,327.11239643)(366.44468105,327.12975746)
\curveto(366.40995899,327.14711849)(366.38608757,327.16664965)(366.3730668,327.18835094)
\curveto(366.36004602,327.21439249)(366.35353563,327.2447743)(366.35353563,327.27949636)
\lineto(366.35353563,328.05423242)
\curveto(366.01933577,327.68531049)(365.68947616,327.41621449)(365.36395681,327.24694443)
\curveto(365.03843745,327.07767436)(364.70857785,326.99303933)(364.37437798,326.99303933)
\curveto(363.98375476,326.99303933)(363.65389515,327.0581432)(363.38479915,327.18835094)
\curveto(363.12004341,327.31855868)(362.90520064,327.49433913)(362.74027083,327.71569229)
\curveto(362.57534103,327.94138571)(362.45598393,328.20180119)(362.38219955,328.49693874)
\curveto(362.31275542,328.79641654)(362.27803335,329.15882809)(362.27803335,329.58417337)
\lineto(362.27803335,333.13884469)
\curveto(362.27803335,333.17356675)(362.28454374,333.20177843)(362.29756452,333.22347972)
\curveto(362.31492555,333.24952127)(362.34530735,333.27122256)(362.38870993,333.28858359)
\curveto(362.43211251,333.31028488)(362.48853587,333.32330566)(362.55798,333.32764591)
\curveto(362.62742412,333.33632643)(362.71422929,333.34066669)(362.81839548,333.34066669)
\curveto(362.92256167,333.34066669)(363.00936683,333.33632643)(363.07881096,333.32764591)
\curveto(363.14825509,333.32330566)(363.20250831,333.31028488)(363.24157063,333.28858359)
\curveto(363.28497321,333.27122256)(363.31535502,333.24952127)(363.33271605,333.22347972)
\curveto(363.35007709,333.20177843)(363.3587576,333.17356675)(363.3587576,333.13884469)
\lineto(363.3587576,329.72740189)
\curveto(363.3587576,329.3845215)(363.38262902,329.10891512)(363.43037186,328.90058273)
\curveto(363.48245495,328.69659061)(363.55840947,328.52081016)(363.6582354,328.37324138)
\curveto(363.7624016,328.23001287)(363.89260934,328.11716616)(364.04885863,328.03470126)
\curveto(364.20510791,327.95657661)(364.38739875,327.91751429)(364.59573114,327.91751429)
\curveto(364.86482713,327.91751429)(365.131753,328.01299997)(365.39650874,328.20397132)
\curveto(365.66560474,328.39494267)(365.94989164,328.67488932)(366.24936944,329.04381125)
\lineto(366.24936944,333.13884469)
\curveto(366.24936944,333.17356675)(366.25587983,333.20177843)(366.2689006,333.22347972)
\curveto(366.28626164,333.24952127)(366.31664344,333.27122256)(366.36004602,333.28858359)
\curveto(366.4034486,333.31028488)(366.45770183,333.32330566)(366.5228057,333.32764591)
\curveto(366.59224983,333.33632643)(366.68122512,333.34066669)(366.78973157,333.34066669)
\curveto(366.89389776,333.34066669)(366.98070292,333.33632643)(367.05014705,333.32764591)
\curveto(367.11959118,333.32330566)(367.1738444,333.31028488)(367.21290672,333.28858359)
\curveto(367.25196904,333.27122256)(367.28018072,333.24952127)(367.29754175,333.22347972)
\curveto(367.31924304,333.20177843)(367.33009369,333.17356675)(367.33009369,333.13884469)
\closepath
}
}
{
\newrgbcolor{curcolor}{0.94509804 0.99215686 0.99607843}
\pscustom[linestyle=none,fillstyle=solid,fillcolor=curcolor]
{
\newpath
\moveto(378.0019378,327.27949636)
\curveto(378.0019378,327.2447743)(377.99325728,327.21439249)(377.97589625,327.18835094)
\curveto(377.95853522,327.16664965)(377.93032354,327.14711849)(377.89126122,327.12975746)
\curveto(377.8521989,327.11239643)(377.79794567,327.09937565)(377.72850154,327.09069514)
\curveto(377.65905742,327.08201462)(377.57225226,327.07767436)(377.46808606,327.07767436)
\curveto(377.35957961,327.07767436)(377.27060432,327.08201462)(377.20116019,327.09069514)
\curveto(377.13171607,327.09937565)(377.07529271,327.11239643)(377.03189013,327.12975746)
\curveto(376.99282781,327.14711849)(376.96461613,327.16664965)(376.9472551,327.18835094)
\curveto(376.92989407,327.21439249)(376.92121355,327.2447743)(376.92121355,327.27949636)
\lineto(376.92121355,330.84067807)
\curveto(376.92121355,331.08807278)(376.89951226,331.31376619)(376.85610968,331.51775832)
\curveto(376.8127071,331.72175045)(376.74326297,331.8975309)(376.6477773,332.04509967)
\curveto(376.55229162,332.19266844)(376.4307644,332.30551515)(376.28319562,332.38363979)
\curveto(376.13562685,332.46176444)(375.96201653,332.50082676)(375.76236466,332.50082676)
\curveto(375.51496995,332.50082676)(375.26540512,332.40534108)(375.01367015,332.21436973)
\curveto(374.76627545,332.02339838)(374.49283919,331.74345174)(374.19336139,331.3745298)
\lineto(374.19336139,327.27949636)
\curveto(374.19336139,327.2447743)(374.18468087,327.21439249)(374.16731984,327.18835094)
\curveto(374.14995881,327.16664965)(374.119577,327.14711849)(374.07617442,327.12975746)
\curveto(374.0371121,327.11239643)(373.98285887,327.09937565)(373.91341474,327.09069514)
\curveto(373.84397062,327.08201462)(373.75716546,327.07767436)(373.65299926,327.07767436)
\curveto(373.55317333,327.07767436)(373.46636817,327.08201462)(373.39258378,327.09069514)
\curveto(373.32313965,327.09937565)(373.2667163,327.11239643)(373.22331372,327.12975746)
\curveto(373.1842514,327.14711849)(373.15603972,327.16664965)(373.13867869,327.18835094)
\curveto(373.12565791,327.21439249)(373.11914753,327.2447743)(373.11914753,327.27949636)
\lineto(373.11914753,330.84067807)
\curveto(373.11914753,331.08807278)(373.09527611,331.31376619)(373.04753327,331.51775832)
\curveto(372.99979043,331.72175045)(372.92817617,331.8975309)(372.8326905,332.04509967)
\curveto(372.73720482,332.19266844)(372.6156776,332.30551515)(372.46810882,332.38363979)
\curveto(372.32488031,332.46176444)(372.15344012,332.50082676)(371.95378825,332.50082676)
\curveto(371.70639354,332.50082676)(371.45682871,332.40534108)(371.20509374,332.21436973)
\curveto(370.95335877,332.02339838)(370.67992252,331.74345174)(370.38478497,331.3745298)
\lineto(370.38478497,327.27949636)
\curveto(370.38478497,327.2447743)(370.37610446,327.21439249)(370.35874343,327.18835094)
\curveto(370.34138239,327.16664965)(370.31317072,327.14711849)(370.27410839,327.12975746)
\curveto(370.23504607,327.11239643)(370.18079285,327.09937565)(370.11134872,327.09069514)
\curveto(370.04190459,327.08201462)(369.9529293,327.07767436)(369.84442285,327.07767436)
\curveto(369.74025666,327.07767436)(369.6534515,327.08201462)(369.58400737,327.09069514)
\curveto(369.51456324,327.09937565)(369.45813989,327.11239643)(369.41473731,327.12975746)
\curveto(369.37567498,327.14711849)(369.34746331,327.16664965)(369.33010228,327.18835094)
\curveto(369.3170815,327.21439249)(369.31057111,327.2447743)(369.31057111,327.27949636)
\lineto(369.31057111,333.13884469)
\curveto(369.31057111,333.17356675)(369.3170815,333.20177843)(369.33010228,333.22347972)
\curveto(369.34312305,333.24952127)(369.3691646,333.27122256)(369.40822692,333.28858359)
\curveto(369.44728924,333.31028488)(369.49720221,333.32330566)(369.55796582,333.32764591)
\curveto(369.61872943,333.33632643)(369.69902421,333.34066669)(369.79885014,333.34066669)
\curveto(369.89433582,333.34066669)(369.97246046,333.33632643)(370.03322407,333.32764591)
\curveto(370.09832794,333.32330566)(370.14824091,333.31028488)(370.18296298,333.28858359)
\curveto(370.21768504,333.27122256)(370.24155646,333.24952127)(370.25457723,333.22347972)
\curveto(370.27193827,333.20177843)(370.28061878,333.17356675)(370.28061878,333.13884469)
\lineto(370.28061878,332.36410863)
\curveto(370.61047839,332.73303056)(370.92948736,333.00212656)(371.23764568,333.17139662)
\curveto(371.55014425,333.34500695)(371.86481296,333.43181211)(372.18165179,333.43181211)
\curveto(372.42470624,333.43181211)(372.64171914,333.40360043)(372.8326905,333.34717707)
\curveto(373.02800211,333.29075372)(373.1994423,333.21045895)(373.34701107,333.10629275)
\curveto(373.49457985,333.00646682)(373.62044733,332.8849396)(373.72461352,332.74171108)
\curveto(373.82877971,332.60282282)(373.91558487,332.44657354)(373.985029,332.27296321)
\curveto(374.18034061,332.48563586)(374.36480158,332.66575657)(374.5384119,332.81332534)
\curveto(374.71636248,332.96089411)(374.88563254,333.08025121)(375.04622209,333.17139662)
\curveto(375.21115189,333.26254204)(375.36957131,333.32764591)(375.52148034,333.36670824)
\curveto(375.67772963,333.41011082)(375.83397892,333.43181211)(375.99022821,333.43181211)
\curveto(376.36783065,333.43181211)(376.68466949,333.36453811)(376.94074471,333.22999011)
\curveto(377.19681994,333.09978237)(377.40298219,332.92400192)(377.55923148,332.70264876)
\curveto(377.71982103,332.4812956)(377.83266774,332.22088012)(377.89777161,331.92140231)
\curveto(377.96721573,331.62626477)(378.0019378,331.31376619)(378.0019378,330.98390658)
\closepath
}
}
{
\newrgbcolor{curcolor}{0.90980393 0.90980393 0.90980393}
\pscustom[linestyle=none,fillstyle=solid,fillcolor=curcolor]
{
\newpath
\moveto(736.78197308,461.49111741)
\lineto(736.78197308,461.49111741)
\curveto(736.78197308,466.14429697)(740.55414392,469.91645468)(745.20731035,469.91645468)
\lineto(928.35623081,469.91645468)
\curveto(930.59076696,469.91645468)(932.73375475,469.02879244)(934.31380404,467.44874314)
\curveto(935.89390583,465.8686676)(936.78156808,463.72565357)(936.78156808,461.49111741)
\lineto(936.78156808,427.79071322)
\curveto(936.78156808,423.13753366)(933.00939724,419.36537595)(928.35623081,419.36537595)
\lineto(745.20731035,419.36537595)
\curveto(740.55414392,419.36537595)(736.78197308,423.13753366)(736.78197308,427.79071322)
\closepath
}
}
{
\newrgbcolor{curcolor}{0.10196079 0.8392157 0.96078432}
\pscustom[linewidth=0.99999798,linecolor=curcolor]
{
\newpath
\moveto(736.78197308,461.49111741)
\lineto(736.78197308,461.49111741)
\curveto(736.78197308,466.14429697)(740.55414392,469.91645468)(745.20731035,469.91645468)
\lineto(928.35623081,469.91645468)
\curveto(930.59076696,469.91645468)(932.73375475,469.02879244)(934.31380404,467.44874314)
\curveto(935.89390583,465.8686676)(936.78156808,463.72565357)(936.78156808,461.49111741)
\lineto(936.78156808,427.79071322)
\curveto(936.78156808,423.13753366)(933.00939724,419.36537595)(928.35623081,419.36537595)
\lineto(745.20731035,419.36537595)
\curveto(740.55414392,419.36537595)(736.78197308,423.13753366)(736.78197308,427.79071322)
\closepath
}
}
{
\newrgbcolor{curcolor}{0 0 0}
\pscustom[linestyle=none,fillstyle=solid,fillcolor=curcolor]
{
\newpath
\moveto(749.08820141,453.01387424)
\lineto(750.7561698,453.15970755)
\curveto(750.83516284,452.49130491)(751.01745446,451.94139184)(751.30304468,451.50996832)
\curveto(751.59471128,451.0846212)(752.04436396,450.73826711)(752.65200271,450.47090605)
\curveto(753.25964147,450.20962139)(753.94323507,450.07897906)(754.70278351,450.07897906)
\curveto(755.37726253,450.07897906)(755.97274851,450.17923945)(756.48924145,450.37976024)
\curveto(757.0057344,450.58028103)(757.38854681,450.85371847)(757.6376787,451.20007256)
\curveto(757.89288698,451.55250304)(758.02049112,451.93531545)(758.02049112,452.34850981)
\curveto(758.02049112,452.76778055)(757.89896337,453.1323638)(757.65590786,453.44225957)
\curveto(757.41285236,453.75823172)(757.01181078,454.02255458)(756.45278313,454.23522814)
\curveto(756.09427626,454.37498506)(755.30130769,454.59069681)(754.0738774,454.88236342)
\curveto(752.84644712,455.18010641)(751.98663828,455.45962023)(751.49445089,455.7209049)
\curveto(750.85643019,456.05510621)(750.37943377,456.46830057)(750.06346162,456.96048796)
\curveto(749.75356585,457.45875174)(749.59861797,458.0147412)(749.59861797,458.62845634)
\curveto(749.59861797,459.30293536)(749.79002418,459.93184147)(750.17283659,460.51517468)
\curveto(750.55564901,461.10458427)(751.11467666,461.55119876)(751.84991956,461.85501813)
\curveto(752.58516245,462.15883751)(753.40243658,462.3107472)(754.30174194,462.3107472)
\curveto(755.29219311,462.3107472)(756.16415472,462.14972293)(756.91762678,461.82767439)
\curveto(757.67717522,461.51170224)(758.26050843,461.0438204)(758.66762639,460.42402887)
\curveto(759.07474436,459.80423733)(759.29349431,459.10241457)(759.32387625,458.31856058)
\lineto(757.62856412,458.19095644)
\curveto(757.53741831,459.03557431)(757.22752254,459.673595)(756.69887683,460.10501852)
\curveto(756.1763075,460.53644203)(755.40156808,460.75215379)(754.37465859,460.75215379)
\curveto(753.30521438,460.75215379)(752.52439858,460.5546712)(752.03221118,460.15970601)
\curveto(751.54610018,459.7708172)(751.30304468,459.29989717)(751.30304468,458.7469459)
\curveto(751.30304468,458.26691128)(751.47622172,457.87194609)(751.82257581,457.56205033)
\curveto(752.16285352,457.25215456)(753.0500061,456.93314422)(754.48403356,456.60501929)
\curveto(755.92413741,456.28297075)(756.91155039,456.00041873)(757.44627249,455.75736322)
\curveto(758.2240501,455.39885636)(758.79826872,454.94312729)(759.16892837,454.39017603)
\curveto(759.53958801,453.84330115)(759.72491783,453.21135684)(759.72491783,452.49434311)
\curveto(759.72491783,451.78340576)(759.52135884,451.11196494)(759.11424088,450.48002063)
\curveto(758.70712291,449.85415272)(758.12075151,449.36500352)(757.35512668,449.01257304)
\curveto(756.59557824,448.66621895)(755.73880759,448.4930419)(754.78481475,448.4930419)
\curveto(753.57561362,448.4930419)(752.5608569,448.66925714)(751.74054458,449.02168762)
\curveto(750.92630865,449.3741181)(750.28524976,449.90276382)(749.81736792,450.60762477)
\curveto(749.35556247,451.31856212)(749.11250697,452.12064527)(749.08820141,453.01387424)
\closepath
}
}
{
\newrgbcolor{curcolor}{0 0 0}
\pscustom[linestyle=none,fillstyle=solid,fillcolor=curcolor]
{
\newpath
\moveto(768.53840307,451.83809325)
\lineto(770.2337152,451.62845788)
\curveto(769.96635415,450.63800671)(769.47112856,449.86934369)(768.74803844,449.32246881)
\curveto(768.02494832,448.77559393)(767.10133741,448.50215649)(765.97720572,448.50215649)
\curveto(764.56140742,448.50215649)(763.43727572,448.9366182)(762.60481062,449.80554162)
\curveto(761.77842192,450.68054142)(761.36522756,451.90493352)(761.36522756,453.47871789)
\curveto(761.36522756,455.10718976)(761.78449831,456.37107837)(762.62303979,457.27038373)
\curveto(763.46158127,458.16968908)(764.54925464,458.61934176)(765.8860599,458.61934176)
\curveto(767.18033045,458.61934176)(768.23762189,458.17880366)(769.05793421,457.29772747)
\curveto(769.87824653,456.41665127)(770.28840269,455.17706821)(770.28840269,453.57897829)
\curveto(770.28840269,453.48175609)(770.28536449,453.33592278)(770.2792881,453.14147838)
\lineto(763.06053969,453.14147838)
\curveto(763.12130357,452.07811056)(763.42208475,451.26387463)(763.96288324,450.69877059)
\curveto(764.50368174,450.13366654)(765.17816075,449.85111452)(765.9863203,449.85111452)
\curveto(766.58788267,449.85111452)(767.10133741,450.0091006)(767.52668454,450.32507275)
\curveto(767.95203167,450.6410449)(768.28927118,451.14538507)(768.53840307,451.83809325)
\closepath
\moveto(763.1516855,454.49043642)
\lineto(768.55663223,454.49043642)
\curveto(768.48371558,455.30467235)(768.27711841,455.9153493)(767.9368407,456.32246727)
\curveto(767.41427137,456.95441157)(766.73675416,457.27038373)(765.90428907,457.27038373)
\curveto(765.15081701,457.27038373)(764.51583451,457.01821364)(763.99934157,456.51387347)
\curveto(763.48892501,456.00953331)(763.20637299,455.33505429)(763.1516855,454.49043642)
\closepath
}
}
{
\newrgbcolor{curcolor}{0 0 0}
\pscustom[linestyle=none,fillstyle=solid,fillcolor=curcolor]
{
\newpath
\moveto(778.60715221,452.26647858)
\lineto(780.22043311,452.0568432)
\curveto(780.04421787,450.94486428)(779.591527,450.07290267)(778.86236049,449.44095836)
\curveto(778.13927037,448.81509045)(777.24907959,448.50215649)(776.19178816,448.50215649)
\curveto(774.86713567,448.50215649)(773.80072966,448.93358)(772.99257011,449.79642703)
\curveto(772.19048696,450.66535046)(771.78944538,451.90797171)(771.78944538,453.5242908)
\curveto(771.78944538,454.56942946)(771.96262242,455.48392578)(772.30897651,456.26777978)
\curveto(772.6553306,457.05163377)(773.18093813,457.63800517)(773.88579908,458.02689398)
\curveto(774.59673643,458.42185917)(775.36843765,458.61934176)(776.20090274,458.61934176)
\curveto(777.25211779,458.61934176)(778.11192663,458.35198071)(778.78032926,457.8172586)
\curveto(779.44873189,457.28861289)(779.87711721,456.53514083)(780.06548523,455.55684244)
\lineto(778.47043349,455.31074874)
\curveto(778.3185238,455.96092221)(778.04812456,456.45007141)(777.65923575,456.77819633)
\curveto(777.27642334,457.10632126)(776.81157969,457.27038373)(776.26470481,457.27038373)
\curveto(775.4383161,457.27038373)(774.76687528,456.97264074)(774.25038234,456.37715475)
\curveto(773.7338894,455.78774516)(773.47564292,454.85198148)(773.47564292,453.56986371)
\curveto(773.47564292,452.26951677)(773.72477481,451.3246385)(774.22303859,450.73522891)
\curveto(774.72130237,450.14581932)(775.37147584,449.85111452)(776.173559,449.85111452)
\curveto(776.81765608,449.85111452)(777.35541638,450.04859712)(777.78683989,450.44356231)
\curveto(778.21826341,450.8385275)(778.49170085,451.44616626)(778.60715221,452.26647858)
\closepath
}
}
{
\newrgbcolor{curcolor}{0 0 0}
\pscustom[linestyle=none,fillstyle=solid,fillcolor=curcolor]
{
\newpath
\moveto(781.01338356,453.56074912)
\curveto(781.01338356,455.35328345)(781.51164734,456.68097413)(782.5081749,457.54382117)
\curveto(783.34064,458.2608349)(784.35539672,458.61934176)(785.55244507,458.61934176)
\curveto(786.88317394,458.61934176)(787.97084731,458.18184186)(788.81546518,457.30684205)
\curveto(789.66008305,456.43791863)(790.08239199,455.23479389)(790.08239199,453.69746784)
\curveto(790.08239199,452.4518084)(789.89402397,451.47047181)(789.51728794,450.75345807)
\curveto(789.1466283,450.04252073)(788.60279162,449.48956946)(787.88577789,449.09460427)
\curveto(787.17484054,448.69963908)(786.39706294,448.50215649)(785.55244507,448.50215649)
\curveto(784.19741064,448.50215649)(783.10062269,448.9366182)(782.26208121,449.80554162)
\curveto(781.42961611,450.67446504)(781.01338356,451.92620087)(781.01338356,453.56074912)
\closepath
\moveto(782.69958111,453.56074912)
\curveto(782.69958111,452.32116606)(782.96998036,451.39147877)(783.51077885,450.77168724)
\curveto(784.05157734,450.15797209)(784.73213275,449.85111452)(785.55244507,449.85111452)
\curveto(786.366681,449.85111452)(787.04419821,450.16101029)(787.5849967,450.78080182)
\curveto(788.12579519,451.40059335)(788.39619444,452.34547161)(788.39619444,453.61543661)
\curveto(788.39619444,454.81248496)(788.122757,455.71786671)(787.57588212,456.33158185)
\curveto(787.03508363,456.95137338)(786.36060461,457.26126914)(785.55244507,457.26126914)
\curveto(784.73213275,457.26126914)(784.05157734,456.95441157)(783.51077885,456.34069643)
\curveto(782.96998036,455.72698129)(782.69958111,454.80033218)(782.69958111,453.56074912)
\closepath
}
}
{
\newrgbcolor{curcolor}{0 0 0}
\pscustom[linestyle=none,fillstyle=solid,fillcolor=curcolor]
{
\newpath
\moveto(792.00270923,448.72090644)
\lineto(792.00270923,458.40059181)
\lineto(793.47927141,458.40059181)
\lineto(793.47927141,457.02429003)
\curveto(794.19020875,458.08765785)(795.21711825,458.61934176)(796.5599999,458.61934176)
\curveto(797.1433331,458.61934176)(797.67805521,458.51300498)(798.16416621,458.30033142)
\curveto(798.65635361,458.09373424)(799.02397505,457.8202968)(799.26703055,457.4800191)
\curveto(799.51008606,457.13974139)(799.68022491,456.73566162)(799.77744711,456.26777978)
\curveto(799.83821098,455.9639604)(799.86859292,455.43227649)(799.86859292,454.67272805)
\lineto(799.86859292,448.72090644)
\lineto(798.22796828,448.72090644)
\lineto(798.22796828,454.60892598)
\curveto(798.22796828,455.27732861)(798.16416621,455.77559239)(798.03656208,456.10371732)
\curveto(797.90895794,456.43791863)(797.6810934,456.70224149)(797.35296848,456.89668589)
\curveto(797.03091994,457.09720668)(796.65114571,457.19746707)(796.21364581,457.19746707)
\curveto(795.51486124,457.19746707)(794.91026068,456.97567893)(794.39984412,456.53210264)
\curveto(793.89550396,456.08852635)(793.64333387,455.24694667)(793.64333387,454.00736361)
\lineto(793.64333387,448.72090644)
\closepath
}
}
{
\newrgbcolor{curcolor}{0 0 0}
\pscustom[linestyle=none,fillstyle=solid,fillcolor=curcolor]
{
\newpath
\moveto(808.66130068,448.72090644)
\lineto(808.66130068,449.94226034)
\curveto(808.04758554,448.9821911)(807.14524198,448.50215649)(805.95427002,448.50215649)
\curveto(805.1825688,448.50215649)(804.47163146,448.71483005)(803.82145799,449.14017718)
\curveto(803.17736091,449.56552431)(802.67605894,450.15797209)(802.31755207,450.91752054)
\curveto(801.96512159,451.68314537)(801.78890636,452.56118337)(801.78890636,453.55163454)
\curveto(801.78890636,454.51778016)(801.94993063,455.39277997)(802.27197917,456.17663397)
\curveto(802.59402771,456.96656435)(803.07710052,457.57116491)(803.7211976,457.99043565)
\curveto(804.36529468,458.40970639)(805.0853466,458.61934176)(805.88135337,458.61934176)
\curveto(806.46468658,458.61934176)(806.98421771,458.49477582)(807.43994678,458.24564393)
\curveto(807.89567585,458.00258843)(808.26633549,457.68357808)(808.5519257,457.28861289)
\lineto(808.5519257,462.08288267)
\lineto(810.18343576,462.08288267)
\lineto(810.18343576,448.72090644)
\closepath
\moveto(803.4751039,453.55163454)
\curveto(803.4751039,452.31205148)(803.73638857,451.38540238)(804.2589579,450.77168724)
\curveto(804.78152723,450.15797209)(805.39828056,449.85111452)(806.10921791,449.85111452)
\curveto(806.82623164,449.85111452)(807.43387039,450.14278113)(807.93213417,450.72611433)
\curveto(808.43647434,451.31552392)(808.68864442,452.21179109)(808.68864442,453.41491582)
\curveto(808.68864442,454.73956831)(808.43343615,455.71179032)(807.92301959,456.33158185)
\curveto(807.41260304,456.95137338)(806.78369692,457.26126914)(806.03630126,457.26126914)
\curveto(805.30713475,457.26126914)(804.6964578,456.96352615)(804.20427041,456.36804017)
\curveto(803.7181594,455.77255419)(803.4751039,454.83375232)(803.4751039,453.55163454)
\closepath
}
}
{
\newrgbcolor{curcolor}{0 0 0}
\pscustom[linestyle=none,fillstyle=solid,fillcolor=curcolor]
{
\newpath
\moveto(824.24705537,452.26647858)
\lineto(825.86033626,452.0568432)
\curveto(825.68412102,450.94486428)(825.23143015,450.07290267)(824.50226364,449.44095836)
\curveto(823.77917352,448.81509045)(822.88898275,448.50215649)(821.83169131,448.50215649)
\curveto(820.50703883,448.50215649)(819.44063281,448.93358)(818.63247327,449.79642703)
\curveto(817.83039011,450.66535046)(817.42934853,451.90797171)(817.42934853,453.5242908)
\curveto(817.42934853,454.56942946)(817.60252558,455.48392578)(817.94887967,456.26777978)
\curveto(818.29523376,457.05163377)(818.82084128,457.63800517)(819.52570224,458.02689398)
\curveto(820.23663958,458.42185917)(821.0083408,458.61934176)(821.8408059,458.61934176)
\curveto(822.89202094,458.61934176)(823.75182978,458.35198071)(824.42023241,457.8172586)
\curveto(825.08863504,457.28861289)(825.51702037,456.53514083)(825.70538838,455.55684244)
\lineto(824.11033665,455.31074874)
\curveto(823.95842696,455.96092221)(823.68802771,456.45007141)(823.29913891,456.77819633)
\curveto(822.91632649,457.10632126)(822.45148284,457.27038373)(821.90460796,457.27038373)
\curveto(821.07821926,457.27038373)(820.40677843,456.97264074)(819.89028549,456.37715475)
\curveto(819.37379255,455.78774516)(819.11554608,454.85198148)(819.11554608,453.56986371)
\curveto(819.11554608,452.26951677)(819.36467797,451.3246385)(819.86294175,450.73522891)
\curveto(820.36120553,450.14581932)(821.01137899,449.85111452)(821.81346215,449.85111452)
\curveto(822.45755923,449.85111452)(822.99531953,450.04859712)(823.42674305,450.44356231)
\curveto(823.85816656,450.8385275)(824.131604,451.44616626)(824.24705537,452.26647858)
\closepath
}
}
{
\newrgbcolor{curcolor}{0 0 0}
\pscustom[linestyle=none,fillstyle=solid,fillcolor=curcolor]
{
\newpath
\moveto(826.65329053,453.56074912)
\curveto(826.65329053,455.35328345)(827.15155431,456.68097413)(828.14808187,457.54382117)
\curveto(828.98054697,458.2608349)(829.99530369,458.61934176)(831.19235203,458.61934176)
\curveto(832.52308091,458.61934176)(833.61075428,458.18184186)(834.45537215,457.30684205)
\curveto(835.29999002,456.43791863)(835.72229896,455.23479389)(835.72229896,453.69746784)
\curveto(835.72229896,452.4518084)(835.53393094,451.47047181)(835.15719491,450.75345807)
\curveto(834.78653527,450.04252073)(834.24269859,449.48956946)(833.52568486,449.09460427)
\curveto(832.81474751,448.69963908)(832.0369699,448.50215649)(831.19235203,448.50215649)
\curveto(829.83731761,448.50215649)(828.74052966,448.9366182)(827.90198817,449.80554162)
\curveto(827.06952308,450.67446504)(826.65329053,451.92620087)(826.65329053,453.56074912)
\closepath
\moveto(828.33948808,453.56074912)
\curveto(828.33948808,452.32116606)(828.60988732,451.39147877)(829.15068582,450.77168724)
\curveto(829.69148431,450.15797209)(830.37203971,449.85111452)(831.19235203,449.85111452)
\curveto(832.00658797,449.85111452)(832.68410518,450.16101029)(833.22490367,450.78080182)
\curveto(833.76570216,451.40059335)(834.03610141,452.34547161)(834.03610141,453.61543661)
\curveto(834.03610141,454.81248496)(833.76266397,455.71786671)(833.21578909,456.33158185)
\curveto(832.6749906,456.95137338)(832.00051158,457.26126914)(831.19235203,457.26126914)
\curveto(830.37203971,457.26126914)(829.69148431,456.95441157)(829.15068582,456.34069643)
\curveto(828.60988732,455.72698129)(828.33948808,454.80033218)(828.33948808,453.56074912)
\closepath
}
}
{
\newrgbcolor{curcolor}{0 0 0}
\pscustom[linestyle=none,fillstyle=solid,fillcolor=curcolor]
{
\newpath
\moveto(837.6426162,445.01127184)
\lineto(837.6426162,458.40059181)
\lineto(839.13740754,458.40059181)
\lineto(839.13740754,457.14277959)
\curveto(839.48983802,457.63496698)(839.8878414,458.00258843)(840.3314177,458.24564393)
\curveto(840.77499399,458.49477582)(841.31275429,458.61934176)(841.94469859,458.61934176)
\curveto(842.7710873,458.61934176)(843.5002538,458.4066682)(844.13219811,457.98132107)
\curveto(844.76414242,457.55597394)(845.24113884,456.95441157)(845.56318738,456.17663397)
\curveto(845.88523592,455.40493275)(846.04626019,454.55727668)(846.04626019,453.63366577)
\curveto(846.04626019,452.6432146)(845.86700676,451.74998563)(845.50849989,450.95397886)
\curveto(845.15606941,450.16404848)(844.63957647,449.55640973)(843.95902106,449.1310626)
\curveto(843.28454205,448.71179186)(842.5736047,448.50215649)(841.82620903,448.50215649)
\curveto(841.27933415,448.50215649)(840.78714676,448.61760785)(840.34964686,448.84851058)
\curveto(839.91822334,449.0794133)(839.56275467,449.37107991)(839.28324084,449.72351038)
\lineto(839.28324084,445.01127184)
\closepath
\moveto(839.12829296,453.50606164)
\curveto(839.12829296,452.26040219)(839.38046304,451.33982947)(839.88480321,450.74434349)
\curveto(840.38914338,450.14885751)(840.99982033,449.85111452)(841.71683406,449.85111452)
\curveto(842.44600056,449.85111452)(843.06883029,450.15797209)(843.58532323,450.77168724)
\curveto(844.10789256,451.39147877)(844.36917722,452.34850981)(844.36917722,453.64278036)
\curveto(844.36917722,454.87628703)(844.11396895,455.79989794)(843.60355239,456.41361308)
\curveto(843.09921223,457.02732822)(842.49461166,457.33418579)(841.78975071,457.33418579)
\curveto(841.09096614,457.33418579)(840.47117461,457.00606087)(839.93037612,456.34981101)
\curveto(839.39565401,455.69963754)(839.12829296,454.75172108)(839.12829296,453.50606164)
\closepath
}
}
{
\newrgbcolor{curcolor}{0 0 0}
\pscustom[linestyle=none,fillstyle=solid,fillcolor=curcolor]
{
\newpath
\moveto(847.94834065,444.99304267)
\lineto(847.76604902,446.53340692)
\curveto(848.12455588,446.43618472)(848.43748984,446.38757362)(848.7048509,446.38757362)
\curveto(849.06943415,446.38757362)(849.36110075,446.44833749)(849.5798507,446.56986524)
\curveto(849.79860066,446.691393)(849.97785409,446.86153185)(850.117611,447.0802818)
\curveto(850.22090959,447.24434426)(850.38801025,447.65146223)(850.61891298,448.3016357)
\curveto(850.64929491,448.39278151)(850.69790601,448.52646204)(850.76474628,448.70267728)
\lineto(847.09157,458.40059181)
\lineto(848.85979878,458.40059181)
\lineto(850.87412125,452.79512429)
\curveto(851.13540592,452.08418695)(851.36934684,451.33679128)(851.57594401,450.55293729)
\curveto(851.76431203,451.30640934)(851.98913837,452.04165224)(852.25042303,452.75866597)
\lineto(854.31943299,458.40059181)
\lineto(855.96005763,458.40059181)
\lineto(852.27776678,448.55684397)
\curveto(851.88280159,447.49347615)(851.57594401,446.76127145)(851.35719406,446.36022987)
\curveto(851.06552746,445.81943138)(850.73132614,445.42446619)(850.35459012,445.1753343)
\curveto(849.97785409,444.92012602)(849.52820141,444.79252188)(849.00563208,444.79252188)
\curveto(848.68965993,444.79252188)(848.33722945,444.85936215)(847.94834065,444.99304267)
\closepath
}
}
{
\newrgbcolor{curcolor}{0 0 0}
\pscustom[linestyle=none,fillstyle=solid,fillcolor=curcolor]
{
\newpath
\moveto(861.91454878,453.56074912)
\curveto(861.91454878,455.35328345)(862.41281256,456.68097413)(863.40934012,457.54382117)
\curveto(864.24180521,458.2608349)(865.25656193,458.61934176)(866.45361028,458.61934176)
\curveto(867.78433916,458.61934176)(868.87201253,458.18184186)(869.7166304,457.30684205)
\curveto(870.56124827,456.43791863)(870.9835572,455.23479389)(870.9835572,453.69746784)
\curveto(870.9835572,452.4518084)(870.79518919,451.47047181)(870.41845316,450.75345807)
\curveto(870.04779352,450.04252073)(869.50395683,449.48956946)(868.7869431,449.09460427)
\curveto(868.07600576,448.69963908)(867.29822815,448.50215649)(866.45361028,448.50215649)
\curveto(865.09857586,448.50215649)(864.0017879,448.9366182)(863.16324642,449.80554162)
\curveto(862.33078133,450.67446504)(861.91454878,451.92620087)(861.91454878,453.56074912)
\closepath
\moveto(863.60074633,453.56074912)
\curveto(863.60074633,452.32116606)(863.87114557,451.39147877)(864.41194406,450.77168724)
\curveto(864.95274256,450.15797209)(865.63329796,449.85111452)(866.45361028,449.85111452)
\curveto(867.26784621,449.85111452)(867.94536343,450.16101029)(868.48616192,450.78080182)
\curveto(869.02696041,451.40059335)(869.29735966,452.34547161)(869.29735966,453.61543661)
\curveto(869.29735966,454.81248496)(869.02392222,455.71786671)(868.47704734,456.33158185)
\curveto(867.93624884,456.95137338)(867.26176983,457.26126914)(866.45361028,457.26126914)
\curveto(865.63329796,457.26126914)(864.95274256,456.95441157)(864.41194406,456.34069643)
\curveto(863.87114557,455.72698129)(863.60074633,454.80033218)(863.60074633,453.56074912)
\closepath
}
}
{
\newrgbcolor{curcolor}{0 0 0}
\pscustom[linestyle=none,fillstyle=solid,fillcolor=curcolor]
{
\newpath
\moveto(873.29580908,448.72090644)
\lineto(873.29580908,457.12455042)
\lineto(871.84659065,457.12455042)
\lineto(871.84659065,458.40059181)
\lineto(873.29580908,458.40059181)
\lineto(873.29580908,459.4305395)
\curveto(873.29580908,460.08071297)(873.35353476,460.56378578)(873.46898612,460.87975793)
\curveto(873.6269722,461.30510506)(873.90344783,461.64842096)(874.29841302,461.90970562)
\curveto(874.6994546,462.17706667)(875.25848226,462.3107472)(875.97549599,462.3107472)
\curveto(876.43730144,462.3107472)(876.947718,462.25605971)(877.50674565,462.14668474)
\lineto(877.26065196,460.71569547)
\curveto(876.92037425,460.77645934)(876.59832571,460.80684128)(876.29450633,460.80684128)
\curveto(875.79624255,460.80684128)(875.44381208,460.7005045)(875.2372149,460.48783093)
\curveto(875.03061772,460.27515737)(874.92731913,459.87715399)(874.92731913,459.29382078)
\lineto(874.92731913,458.40059181)
\lineto(876.81403747,458.40059181)
\lineto(876.81403747,457.12455042)
\lineto(874.92731913,457.12455042)
\lineto(874.92731913,448.72090644)
\closepath
}
}
{
\newrgbcolor{curcolor}{0 0 0}
\pscustom[linestyle=none,fillstyle=solid,fillcolor=curcolor]
{
\newpath
\moveto(889.52513916,448.72090644)
\lineto(889.52513916,449.94226034)
\curveto(888.91142401,448.9821911)(888.00908046,448.50215649)(886.8181085,448.50215649)
\curveto(886.04640728,448.50215649)(885.33546994,448.71483005)(884.68529647,449.14017718)
\curveto(884.04119939,449.56552431)(883.53989742,450.15797209)(883.18139055,450.91752054)
\curveto(882.82896007,451.68314537)(882.65274483,452.56118337)(882.65274483,453.55163454)
\curveto(882.65274483,454.51778016)(882.8137691,455.39277997)(883.13581764,456.17663397)
\curveto(883.45786618,456.96656435)(883.940939,457.57116491)(884.58503608,457.99043565)
\curveto(885.22913316,458.40970639)(885.94918508,458.61934176)(886.74519185,458.61934176)
\curveto(887.32852506,458.61934176)(887.84805619,458.49477582)(888.30378526,458.24564393)
\curveto(888.75951432,458.00258843)(889.13017397,457.68357808)(889.41576418,457.28861289)
\lineto(889.41576418,462.08288267)
\lineto(891.04727424,462.08288267)
\lineto(891.04727424,448.72090644)
\closepath
\moveto(884.33894238,453.55163454)
\curveto(884.33894238,452.31205148)(884.60022704,451.38540238)(885.12279637,450.77168724)
\curveto(885.6453657,450.15797209)(886.26211904,449.85111452)(886.97305638,449.85111452)
\curveto(887.69007012,449.85111452)(888.29770887,450.14278113)(888.79597265,450.72611433)
\curveto(889.30031282,451.31552392)(889.5524829,452.21179109)(889.5524829,453.41491582)
\curveto(889.5524829,454.73956831)(889.29727462,455.71179032)(888.78685807,456.33158185)
\curveto(888.27644151,456.95137338)(887.6475354,457.26126914)(886.90013973,457.26126914)
\curveto(886.17097323,457.26126914)(885.56029628,456.96352615)(885.06810889,456.36804017)
\curveto(884.58199788,455.77255419)(884.33894238,454.83375232)(884.33894238,453.55163454)
\closepath
}
}
{
\newrgbcolor{curcolor}{0 0 0}
\pscustom[linestyle=none,fillstyle=solid,fillcolor=curcolor]
{
\newpath
\moveto(899.94023857,449.91491659)
\curveto(899.33259982,449.39842365)(898.74622842,449.0338404)(898.18112438,448.82116683)
\curveto(897.62209672,448.60849327)(897.02053436,448.50215649)(896.37643727,448.50215649)
\curveto(895.31306945,448.50215649)(894.49579533,448.76040296)(893.9246149,449.2768959)
\curveto(893.35343447,449.79946523)(893.06784425,450.46482967)(893.06784425,451.27298921)
\curveto(893.06784425,451.74694744)(893.17418103,452.17837096)(893.3868546,452.56725976)
\curveto(893.60560455,452.96222495)(893.88815657,453.2781971)(894.23451066,453.51517622)
\curveto(894.58694114,453.75215533)(894.98190633,453.93140876)(895.41940623,454.05293652)
\curveto(895.74145478,454.13800594)(896.22756578,454.22003717)(896.87773925,454.29903021)
\curveto(898.20239173,454.45701629)(899.17765194,454.6453843)(899.80351985,454.86413425)
\curveto(899.80959624,455.08896059)(899.81263444,455.2317557)(899.81263444,455.29251958)
\curveto(899.81263444,455.96092221)(899.65768655,456.43184224)(899.34779079,456.70527968)
\curveto(898.92852005,457.07593932)(898.30569032,457.26126914)(897.47930162,457.26126914)
\curveto(896.7076004,457.26126914)(896.13641997,457.12455042)(895.76576033,456.85111298)
\curveto(895.40117707,456.58375193)(895.13077783,456.10675551)(894.95456259,455.42012372)
\lineto(893.35039627,455.63887367)
\curveto(893.49622957,456.32550546)(893.73624688,456.87845673)(894.0704482,457.29772747)
\curveto(894.40464951,457.7230746)(894.88772232,458.04816133)(895.51966663,458.27298767)
\curveto(896.15161094,458.5038904)(896.88381564,458.61934176)(897.71628073,458.61934176)
\curveto(898.54266944,458.61934176)(899.21411026,458.52211956)(899.7306032,458.32767516)
\curveto(900.24709615,458.13323076)(900.62687037,457.88713706)(900.86992587,457.58939407)
\curveto(901.11298137,457.29772747)(901.28312022,456.92706783)(901.38034242,456.47741515)
\curveto(901.43502991,456.19790132)(901.46237366,455.69356116)(901.46237366,454.96439465)
\lineto(901.46237366,452.77689513)
\curveto(901.46237366,451.25172185)(901.49579379,450.28557623)(901.56263405,449.87845827)
\curveto(901.6355507,449.47741669)(901.77530762,449.09156608)(901.98190479,448.72090644)
\lineto(900.2683635,448.72090644)
\curveto(900.09822465,449.06118414)(899.98884967,449.45918753)(899.94023857,449.91491659)
\closepath
\moveto(899.80351985,453.57897829)
\curveto(899.20803387,453.33592278)(898.3148049,453.12932561)(897.12383294,452.95918676)
\curveto(896.44935393,452.86196456)(895.9723575,452.75258958)(895.69284367,452.63106183)
\curveto(895.41332985,452.50953408)(895.19761809,452.33028064)(895.0457084,452.09330153)
\curveto(894.89379871,451.8623988)(894.81784387,451.60415233)(894.81784387,451.31856212)
\curveto(894.81784387,450.88106221)(894.98190633,450.51647896)(895.31003126,450.22481236)
\curveto(895.64423257,449.93314575)(896.13034358,449.78731245)(896.76836427,449.78731245)
\curveto(897.40030858,449.78731245)(897.96237443,449.92403117)(898.45456182,450.19746861)
\curveto(898.94674921,450.47698244)(899.30829427,450.85675666)(899.539197,451.33679128)
\curveto(899.71541223,451.70745092)(899.80351985,452.2543258)(899.80351985,452.97741592)
\closepath
}
}
{
\newrgbcolor{curcolor}{0 0 0}
\pscustom[linestyle=none,fillstyle=solid,fillcolor=curcolor]
{
\newpath
\moveto(907.58450527,450.18835403)
\lineto(907.82148438,448.7391356)
\curveto(907.35967893,448.6419134)(906.94648458,448.5933023)(906.58190132,448.5933023)
\curveto(905.98641534,448.5933023)(905.52460989,448.68748631)(905.19648496,448.87585432)
\curveto(904.86836003,449.06422233)(904.6374573,449.31031603)(904.50377678,449.61413541)
\curveto(904.37009625,449.92403117)(904.30325599,450.57116645)(904.30325599,451.55554123)
\lineto(904.30325599,457.12455042)
\lineto(903.10013125,457.12455042)
\lineto(903.10013125,458.40059181)
\lineto(904.30325599,458.40059181)
\lineto(904.30325599,460.7977267)
\lineto(905.93476605,461.78210148)
\lineto(905.93476605,458.40059181)
\lineto(907.58450527,458.40059181)
\lineto(907.58450527,457.12455042)
\lineto(905.93476605,457.12455042)
\lineto(905.93476605,451.46439542)
\curveto(905.93476605,450.99651358)(905.96210979,450.69573239)(906.01679728,450.56205187)
\curveto(906.07756115,450.42837134)(906.17174516,450.32203456)(906.2993493,450.24304152)
\curveto(906.43302983,450.16404848)(906.62139784,450.12455196)(906.86445334,450.12455196)
\curveto(907.04674497,450.12455196)(907.28676228,450.14581932)(907.58450527,450.18835403)
\closepath
}
}
{
\newrgbcolor{curcolor}{0 0 0}
\pscustom[linestyle=none,fillstyle=solid,fillcolor=curcolor]
{
\newpath
\moveto(915.48953875,449.91491659)
\curveto(914.8819,449.39842365)(914.2955286,449.0338404)(913.73042456,448.82116683)
\curveto(913.1713969,448.60849327)(912.56983453,448.50215649)(911.92573745,448.50215649)
\curveto(910.86236963,448.50215649)(910.0450955,448.76040296)(909.47391507,449.2768959)
\curveto(908.90273464,449.79946523)(908.61714443,450.46482967)(908.61714443,451.27298921)
\curveto(908.61714443,451.74694744)(908.72348121,452.17837096)(908.93615478,452.56725976)
\curveto(909.15490473,452.96222495)(909.43745675,453.2781971)(909.78381084,453.51517622)
\curveto(910.13624132,453.75215533)(910.53120651,453.93140876)(910.96870641,454.05293652)
\curveto(911.29075495,454.13800594)(911.77686596,454.22003717)(912.42703943,454.29903021)
\curveto(913.75169191,454.45701629)(914.72695211,454.6453843)(915.35282003,454.86413425)
\curveto(915.35889642,455.08896059)(915.36193461,455.2317557)(915.36193461,455.29251958)
\curveto(915.36193461,455.96092221)(915.20698673,456.43184224)(914.89709097,456.70527968)
\curveto(914.47782022,457.07593932)(913.8549905,457.26126914)(913.02860179,457.26126914)
\curveto(912.25690057,457.26126914)(911.68572014,457.12455042)(911.3150605,456.85111298)
\curveto(910.95047725,456.58375193)(910.680078,456.10675551)(910.50386276,455.42012372)
\lineto(908.89969645,455.63887367)
\curveto(909.04552975,456.32550546)(909.28554706,456.87845673)(909.61974838,457.29772747)
\curveto(909.95394969,457.7230746)(910.4370225,458.04816133)(911.06896681,458.27298767)
\curveto(911.70091111,458.5038904)(912.43311581,458.61934176)(913.26558091,458.61934176)
\curveto(914.09196961,458.61934176)(914.76341044,458.52211956)(915.27990338,458.32767516)
\curveto(915.79639632,458.13323076)(916.17617055,457.88713706)(916.41922605,457.58939407)
\curveto(916.66228155,457.29772747)(916.8324204,456.92706783)(916.9296426,456.47741515)
\curveto(916.98433009,456.19790132)(917.01167383,455.69356116)(917.01167383,454.96439465)
\lineto(917.01167383,452.77689513)
\curveto(917.01167383,451.25172185)(917.04509397,450.28557623)(917.11193423,449.87845827)
\curveto(917.18485088,449.47741669)(917.32460779,449.09156608)(917.53120497,448.72090644)
\lineto(915.81766368,448.72090644)
\curveto(915.64752483,449.06118414)(915.53814985,449.45918753)(915.48953875,449.91491659)
\closepath
\moveto(915.35282003,453.57897829)
\curveto(914.75733405,453.33592278)(913.86410508,453.12932561)(912.67313312,452.95918676)
\curveto(911.9986541,452.86196456)(911.52165768,452.75258958)(911.24214385,452.63106183)
\curveto(910.96263002,452.50953408)(910.74691827,452.33028064)(910.59500858,452.09330153)
\curveto(910.44309889,451.8623988)(910.36714404,451.60415233)(910.36714404,451.31856212)
\curveto(910.36714404,450.88106221)(910.53120651,450.51647896)(910.85933144,450.22481236)
\curveto(911.19353275,449.93314575)(911.67964376,449.78731245)(912.31766445,449.78731245)
\curveto(912.94960875,449.78731245)(913.5116746,449.92403117)(914.003862,450.19746861)
\curveto(914.49604939,450.47698244)(914.85759445,450.85675666)(915.08849717,451.33679128)
\curveto(915.26471241,451.70745092)(915.35282003,452.2543258)(915.35282003,452.97741592)
\closepath
}
}
{
\newrgbcolor{curcolor}{0 0 0}
\pscustom[linestyle=none,fillstyle=solid,fillcolor=curcolor]
{
\newpath
\moveto(752.61554439,422.79256843)
\curveto(751.71016264,423.93492929)(750.94453781,425.27173456)(750.31866989,426.80298422)
\curveto(749.69280198,428.33423388)(749.37986802,429.92017103)(749.37986802,431.56079567)
\curveto(749.37986802,433.00697591)(749.61380894,434.39239227)(750.08169078,435.71704476)
\curveto(750.62856566,437.25437081)(751.47318353,438.78562047)(752.61554439,440.31079375)
\lineto(753.79132538,440.31079375)
\curveto(753.05608249,439.04690514)(752.56997148,438.14456159)(752.33299237,437.6037631)
\curveto(751.96233273,436.76522161)(751.67066613,435.89022181)(751.45799256,434.97876367)
\curveto(751.1967079,433.8424792)(751.06606556,432.70011834)(751.06606556,431.55168109)
\curveto(751.06606556,428.62893868)(751.9744855,425.70923446)(753.79132538,422.79256843)
\closepath
}
}
{
\newrgbcolor{curcolor}{0 0 0}
\pscustom[linestyle=none,fillstyle=solid,fillcolor=curcolor]
{
\newpath
\moveto(756.08804243,426.72095299)
\lineto(756.08804243,435.12459697)
\lineto(754.63882399,435.12459697)
\lineto(754.63882399,436.40063836)
\lineto(756.08804243,436.40063836)
\lineto(756.08804243,437.43058605)
\curveto(756.08804243,438.08075952)(756.14576811,438.56383233)(756.26121947,438.87980448)
\curveto(756.41920555,439.30515161)(756.69568118,439.64846751)(757.09064637,439.90975217)
\curveto(757.49168795,440.17711322)(758.0507156,440.31079375)(758.76772934,440.31079375)
\curveto(759.22953479,440.31079375)(759.73995134,440.25610626)(760.298979,440.14673129)
\lineto(760.0528853,438.71574202)
\curveto(759.7126076,438.77650589)(759.39055906,438.80688783)(759.08673968,438.80688783)
\curveto(758.5884759,438.80688783)(758.23604543,438.70055105)(758.02944825,438.48787748)
\curveto(757.82285107,438.27520392)(757.71955248,437.87720054)(757.71955248,437.29386733)
\lineto(757.71955248,436.40063836)
\lineto(759.60627082,436.40063836)
\lineto(759.60627082,435.12459697)
\lineto(757.71955248,435.12459697)
\lineto(757.71955248,426.72095299)
\closepath
}
}
{
\newrgbcolor{curcolor}{0 0 0}
\pscustom[linestyle=none,fillstyle=solid,fillcolor=curcolor]
{
\newpath
\moveto(760.2560947,431.56079567)
\curveto(760.2560947,433.35333)(760.75435848,434.68102068)(761.75088604,435.54386772)
\curveto(762.58335114,436.26088145)(763.59810786,436.61938831)(764.79515621,436.61938831)
\curveto(766.12588508,436.61938831)(767.21355845,436.18188841)(768.05817632,435.3068886)
\curveto(768.90279419,434.43796518)(769.32510313,433.23484044)(769.32510313,431.69751439)
\curveto(769.32510313,430.45185495)(769.13673511,429.47051836)(768.75999908,428.75350462)
\curveto(768.38933944,428.04256728)(767.84550276,427.48961601)(767.12848903,427.09465082)
\curveto(766.41755168,426.69968563)(765.63977408,426.50220304)(764.79515621,426.50220304)
\curveto(763.44012178,426.50220304)(762.34333383,426.93666475)(761.50479235,427.80558817)
\curveto(760.67232725,428.67451159)(760.2560947,429.92624742)(760.2560947,431.56079567)
\closepath
\moveto(761.94229225,431.56079567)
\curveto(761.94229225,430.32121261)(762.2126915,429.39152532)(762.75348999,428.77173379)
\curveto(763.29428848,428.15801864)(763.97484389,427.85116107)(764.79515621,427.85116107)
\curveto(765.60939214,427.85116107)(766.28690935,428.16105684)(766.82770784,428.78084837)
\curveto(767.36850633,429.4006399)(767.63890558,430.34551816)(767.63890558,431.61548316)
\curveto(767.63890558,432.81253151)(767.36546814,433.71791326)(766.81859326,434.3316284)
\curveto(766.27779477,434.95141993)(765.60331575,435.26131569)(764.79515621,435.26131569)
\curveto(763.97484389,435.26131569)(763.29428848,434.95445812)(762.75348999,434.34074298)
\curveto(762.2126915,433.72702784)(761.94229225,432.80037873)(761.94229225,431.56079567)
\closepath
}
}
{
\newrgbcolor{curcolor}{0 0 0}
\pscustom[linestyle=none,fillstyle=solid,fillcolor=curcolor]
{
\newpath
\moveto(771.22718835,426.72095299)
\lineto(771.22718835,436.40063836)
\lineto(772.70375053,436.40063836)
\lineto(772.70375053,434.93319077)
\curveto(773.08048655,435.61982256)(773.42684064,436.07251343)(773.7428128,436.29126338)
\curveto(774.06486134,436.51001334)(774.41729182,436.61938831)(774.80010423,436.61938831)
\curveto(775.3530555,436.61938831)(775.91512135,436.44317307)(776.48630178,436.09074259)
\lineto(775.92119774,434.56860751)
\curveto(775.52015616,434.80558663)(775.11911458,434.92407618)(774.718073,434.92407618)
\curveto(774.35956613,434.92407618)(774.03751759,434.81470121)(773.75192738,434.59595126)
\curveto(773.46633716,434.38327769)(773.26277818,434.0855347)(773.14125043,433.70272229)
\curveto(772.9589588,433.11938908)(772.86781299,432.48136839)(772.86781299,431.78866021)
\lineto(772.86781299,426.72095299)
\closepath
}
}
{
\newrgbcolor{curcolor}{0 0 0}
\pscustom[linestyle=none,fillstyle=solid,fillcolor=curcolor]
{
\newpath
\moveto(782.63206171,426.72095299)
\lineto(782.63206171,436.40063836)
\lineto(784.0995093,436.40063836)
\lineto(784.0995093,435.04256574)
\curveto(784.40332868,435.51652397)(784.80740845,435.89629819)(785.31174862,436.18188841)
\curveto(785.81608878,436.47355501)(786.39030741,436.61938831)(787.03440449,436.61938831)
\curveto(787.75141822,436.61938831)(788.33778962,436.47051682)(788.79351869,436.17277383)
\curveto(789.25532414,435.87503084)(789.58041087,435.45879829)(789.76877889,434.92407618)
\curveto(790.53440372,436.05428427)(791.53093128,436.61938831)(792.75836156,436.61938831)
\curveto(793.7184308,436.61938831)(794.45671188,436.35202726)(794.97320483,435.81730515)
\curveto(795.48969777,435.28865944)(795.74794424,434.47138531)(795.74794424,433.36548278)
\lineto(795.74794424,426.72095299)
\lineto(794.11643418,426.72095299)
\lineto(794.11643418,432.8186079)
\curveto(794.11643418,433.47485775)(794.06174669,433.94577779)(793.95237172,434.231368)
\curveto(793.84907313,434.52303461)(793.65766692,434.75697553)(793.37815309,434.93319077)
\curveto(793.09863927,435.10940601)(792.77051434,435.19751362)(792.39377831,435.19751362)
\curveto(791.7132229,435.19751362)(791.14811886,434.96964909)(790.69846618,434.51392002)
\curveto(790.2488135,434.06426735)(790.02398716,433.34117723)(790.02398716,432.34464967)
\lineto(790.02398716,426.72095299)
\lineto(788.38336253,426.72095299)
\lineto(788.38336253,433.01001411)
\curveto(788.38336253,433.73918061)(788.249682,434.28605549)(787.98232095,434.65063874)
\curveto(787.71495989,435.015222)(787.27745999,435.19751362)(786.66982124,435.19751362)
\curveto(786.20801578,435.19751362)(785.77963046,435.07598587)(785.38466527,434.83293037)
\curveto(784.99577646,434.58987487)(784.71322444,434.2344062)(784.5370092,433.76652436)
\curveto(784.36079397,433.29864251)(784.27268635,432.6241635)(784.27268635,431.7430873)
\lineto(784.27268635,426.72095299)
\closepath
}
}
{
\newrgbcolor{curcolor}{0 0 0}
\pscustom[linestyle=none,fillstyle=solid,fillcolor=curcolor]
{
\newpath
\moveto(804.80766251,429.8381398)
\lineto(806.50297464,429.62850443)
\curveto(806.23561359,428.63805326)(805.740388,427.86939024)(805.01729788,427.32251536)
\curveto(804.29420776,426.77564048)(803.37059685,426.50220304)(802.24646516,426.50220304)
\curveto(800.83066686,426.50220304)(799.70653516,426.93666475)(798.87407007,427.80558817)
\curveto(798.04768136,428.68058797)(797.634487,429.90498007)(797.634487,431.47876444)
\curveto(797.634487,433.10723631)(798.05375775,434.37112492)(798.89229923,435.27043028)
\curveto(799.73084071,436.16973563)(800.81851408,436.61938831)(802.15531934,436.61938831)
\curveto(803.44958989,436.61938831)(804.50688133,436.17885021)(805.32719365,435.29777402)
\curveto(806.14750597,434.41669782)(806.55766213,433.17711476)(806.55766213,431.57902484)
\curveto(806.55766213,431.48180264)(806.55462393,431.33596933)(806.54854754,431.14152493)
\lineto(799.32979913,431.14152493)
\curveto(799.39056301,430.07815711)(799.69134419,429.26392118)(800.23214268,428.69881714)
\curveto(800.77294118,428.13371309)(801.44742019,427.85116107)(802.25557974,427.85116107)
\curveto(802.85714211,427.85116107)(803.37059685,428.00914715)(803.79594398,428.3251193)
\curveto(804.22129111,428.64109145)(804.55853062,429.14543162)(804.80766251,429.8381398)
\closepath
\moveto(799.42094495,432.49048297)
\lineto(804.82589167,432.49048297)
\curveto(804.75297502,433.3047189)(804.54637785,433.91539585)(804.20610014,434.32251382)
\curveto(803.68353081,434.95445812)(803.0060136,435.27043028)(802.17354851,435.27043028)
\curveto(801.42007645,435.27043028)(800.78509395,435.01826019)(800.26860101,434.51392002)
\curveto(799.75818445,434.00957986)(799.47563243,433.33510084)(799.42094495,432.49048297)
\closepath
}
}
{
\newrgbcolor{curcolor}{0 0 0}
\pscustom[linestyle=none,fillstyle=solid,fillcolor=curcolor]
{
\newpath
\moveto(812.14203725,428.18840058)
\lineto(812.37901637,426.73918215)
\curveto(811.91721091,426.64195995)(811.50401656,426.59334885)(811.13943331,426.59334885)
\curveto(810.54394733,426.59334885)(810.08214187,426.68753286)(809.75401695,426.87590087)
\curveto(809.42589202,427.06426888)(809.19498929,427.31036258)(809.06130876,427.61418196)
\curveto(808.92762824,427.92407772)(808.86078798,428.571213)(808.86078798,429.55558778)
\lineto(808.86078798,435.12459697)
\lineto(807.65766324,435.12459697)
\lineto(807.65766324,436.40063836)
\lineto(808.86078798,436.40063836)
\lineto(808.86078798,438.79777325)
\lineto(810.49229803,439.78214803)
\lineto(810.49229803,436.40063836)
\lineto(812.14203725,436.40063836)
\lineto(812.14203725,435.12459697)
\lineto(810.49229803,435.12459697)
\lineto(810.49229803,429.46444197)
\curveto(810.49229803,428.99656013)(810.51964178,428.69577894)(810.57432927,428.56209842)
\curveto(810.63509314,428.42841789)(810.72927715,428.32208111)(810.85688129,428.24308807)
\curveto(810.99056181,428.16409503)(811.17892983,428.12459851)(811.42198533,428.12459851)
\curveto(811.60427696,428.12459851)(811.84429426,428.14586587)(812.14203725,428.18840058)
\closepath
}
}
{
\newrgbcolor{curcolor}{0 0 0}
\pscustom[linestyle=none,fillstyle=solid,fillcolor=curcolor]
{
\newpath
\moveto(820.04706692,427.91496314)
\curveto(819.43942817,427.3984702)(818.85305677,427.03388695)(818.28795273,426.82121338)
\curveto(817.72892507,426.60853982)(817.1273627,426.50220304)(816.48326562,426.50220304)
\curveto(815.4198978,426.50220304)(814.60262368,426.76044951)(814.03144325,427.27694245)
\curveto(813.46026282,427.79951178)(813.1746726,428.46487622)(813.1746726,429.27303576)
\curveto(813.1746726,429.74699399)(813.28100938,430.17841751)(813.49368295,430.56730631)
\curveto(813.7124329,430.9622715)(813.99498492,431.27824365)(814.34133901,431.51522277)
\curveto(814.69376949,431.75220188)(815.08873468,431.93145531)(815.52623458,432.05298307)
\curveto(815.84828312,432.13805249)(816.33439413,432.22008372)(816.9845676,432.29907676)
\curveto(818.30922008,432.45706284)(819.28448029,432.64543085)(819.9103482,432.8641808)
\curveto(819.91642459,433.08900714)(819.91946278,433.23180225)(819.91946278,433.29256613)
\curveto(819.91946278,433.96096876)(819.7645149,434.43188879)(819.45461914,434.70532623)
\curveto(819.0353484,435.07598587)(818.41251867,435.26131569)(817.58612996,435.26131569)
\curveto(816.81442875,435.26131569)(816.24324832,435.12459697)(815.87258867,434.85115953)
\curveto(815.50800542,434.58379848)(815.23760618,434.10680206)(815.06139094,433.42017027)
\lineto(813.45722462,433.63892022)
\curveto(813.60305792,434.32555201)(813.84307523,434.87850328)(814.17727655,435.29777402)
\curveto(814.51147786,435.72312115)(814.99455067,436.04820788)(815.62649498,436.27303422)
\curveto(816.25843928,436.50393695)(816.99064398,436.61938831)(817.82310908,436.61938831)
\curveto(818.64949779,436.61938831)(819.32093861,436.52216611)(819.83743155,436.32772171)
\curveto(820.35392449,436.13327731)(820.73369872,435.88718361)(820.97675422,435.58944062)
\curveto(821.21980972,435.29777402)(821.38994857,434.92711438)(821.48717077,434.4774617)
\curveto(821.54185826,434.19794787)(821.56920201,433.69360771)(821.56920201,432.9644412)
\lineto(821.56920201,430.77694168)
\curveto(821.56920201,429.2517684)(821.60262214,428.28562278)(821.6694624,427.87850482)
\curveto(821.74237905,427.47746324)(821.88213596,427.09161263)(822.08873314,426.72095299)
\lineto(820.37519185,426.72095299)
\curveto(820.205053,427.06123069)(820.09567802,427.45923408)(820.04706692,427.91496314)
\closepath
\moveto(819.9103482,431.57902484)
\curveto(819.31486222,431.33596933)(818.42163325,431.12937216)(817.23066129,430.95923331)
\curveto(816.55618227,430.86201111)(816.07918585,430.75263613)(815.79967202,430.63110838)
\curveto(815.5201582,430.50958063)(815.30444644,430.33032719)(815.15253675,430.09334808)
\curveto(815.00062706,429.86244535)(814.92467222,429.60419888)(814.92467222,429.31860867)
\curveto(814.92467222,428.88110876)(815.08873468,428.51652551)(815.41685961,428.22485891)
\curveto(815.75106092,427.9331923)(816.23717193,427.787359)(816.87519262,427.787359)
\curveto(817.50713693,427.787359)(818.06920277,427.92407772)(818.56139017,428.19751516)
\curveto(819.05357756,428.47702899)(819.41512262,428.85680321)(819.64602534,429.33683783)
\curveto(819.82224058,429.70749747)(819.9103482,430.25437235)(819.9103482,430.97746247)
\closepath
}
}
{
\newrgbcolor{curcolor}{0 0 0}
\pscustom[linestyle=none,fillstyle=solid,fillcolor=curcolor]
{
\newpath
\moveto(830.38925732,426.72095299)
\lineto(830.38925732,427.94230689)
\curveto(829.77554218,426.98223765)(828.87319863,426.50220304)(827.68222667,426.50220304)
\curveto(826.91052545,426.50220304)(826.1995881,426.7148766)(825.54941463,427.14022373)
\curveto(824.90531755,427.56557086)(824.40401558,428.15801864)(824.04550871,428.91756709)
\curveto(823.69307824,429.68319192)(823.516863,430.56122992)(823.516863,431.55168109)
\curveto(823.516863,432.51782671)(823.67788727,433.39282652)(823.99993581,434.17668052)
\curveto(824.32198435,434.9666109)(824.80505716,435.57121146)(825.44915424,435.9904822)
\curveto(826.09325132,436.40975294)(826.81330324,436.61938831)(827.60931001,436.61938831)
\curveto(828.19264322,436.61938831)(828.71217436,436.49482237)(829.16790342,436.24569048)
\curveto(829.62363249,436.00263498)(829.99429213,435.68362463)(830.27988234,435.28865944)
\lineto(830.27988234,440.08292922)
\lineto(831.9113924,440.08292922)
\lineto(831.9113924,426.72095299)
\closepath
\moveto(825.20306054,431.55168109)
\curveto(825.20306054,430.31209803)(825.46434521,429.38544893)(825.98691454,428.77173379)
\curveto(826.50948387,428.15801864)(827.1262372,427.85116107)(827.83717455,427.85116107)
\curveto(828.55418828,427.85116107)(829.16182703,428.14282768)(829.66009081,428.72616088)
\curveto(830.16443098,429.31557047)(830.41660106,430.21183764)(830.41660106,431.41496237)
\curveto(830.41660106,432.73961486)(830.16139279,433.71183687)(829.65097623,434.3316284)
\curveto(829.14055968,434.95141993)(828.51165357,435.26131569)(827.7642579,435.26131569)
\curveto(827.03509139,435.26131569)(826.42441444,434.9635727)(825.93222705,434.36808672)
\curveto(825.44611605,433.77260074)(825.20306054,432.83379887)(825.20306054,431.55168109)
\closepath
}
}
{
\newrgbcolor{curcolor}{0 0 0}
\pscustom[linestyle=none,fillstyle=solid,fillcolor=curcolor]
{
\newpath
\moveto(840.80435674,427.91496314)
\curveto(840.19671798,427.3984702)(839.61034658,427.03388695)(839.04524254,426.82121338)
\curveto(838.48621489,426.60853982)(837.88465252,426.50220304)(837.24055544,426.50220304)
\curveto(836.17718762,426.50220304)(835.35991349,426.76044951)(834.78873306,427.27694245)
\curveto(834.21755263,427.79951178)(833.93196242,428.46487622)(833.93196242,429.27303576)
\curveto(833.93196242,429.74699399)(834.0382992,430.17841751)(834.25097276,430.56730631)
\curveto(834.46972271,430.9622715)(834.75227473,431.27824365)(835.09862883,431.51522277)
\curveto(835.4510593,431.75220188)(835.84602449,431.93145531)(836.2835244,432.05298307)
\curveto(836.60557294,432.13805249)(837.09168394,432.22008372)(837.74185741,432.29907676)
\curveto(839.0665099,432.45706284)(840.0417701,432.64543085)(840.66763802,432.8641808)
\curveto(840.67371441,433.08900714)(840.6767526,433.23180225)(840.6767526,433.29256613)
\curveto(840.6767526,433.96096876)(840.52180472,434.43188879)(840.21190895,434.70532623)
\curveto(839.79263821,435.07598587)(839.16980849,435.26131569)(838.34341978,435.26131569)
\curveto(837.57171856,435.26131569)(837.00053813,435.12459697)(836.62987849,434.85115953)
\curveto(836.26529524,434.58379848)(835.99489599,434.10680206)(835.81868075,433.42017027)
\lineto(834.21451444,433.63892022)
\curveto(834.36034774,434.32555201)(834.60036505,434.87850328)(834.93456636,435.29777402)
\curveto(835.26876768,435.72312115)(835.75184049,436.04820788)(836.38378479,436.27303422)
\curveto(837.0157291,436.50393695)(837.7479338,436.61938831)(838.58039889,436.61938831)
\curveto(839.4067876,436.61938831)(840.07822843,436.52216611)(840.59472137,436.32772171)
\curveto(841.11121431,436.13327731)(841.49098853,435.88718361)(841.73404403,435.58944062)
\curveto(841.97709954,435.29777402)(842.14723839,434.92711438)(842.24446059,434.4774617)
\curveto(842.29914808,434.19794787)(842.32649182,433.69360771)(842.32649182,432.9644412)
\lineto(842.32649182,430.77694168)
\curveto(842.32649182,429.2517684)(842.35991195,428.28562278)(842.42675221,427.87850482)
\curveto(842.49966887,427.47746324)(842.63942578,427.09161263)(842.84602296,426.72095299)
\lineto(841.13248167,426.72095299)
\curveto(840.96234281,427.06123069)(840.85296784,427.45923408)(840.80435674,427.91496314)
\closepath
\moveto(840.66763802,431.57902484)
\curveto(840.07215204,431.33596933)(839.17892307,431.12937216)(837.98795111,430.95923331)
\curveto(837.31347209,430.86201111)(836.83647567,430.75263613)(836.55696184,430.63110838)
\curveto(836.27744801,430.50958063)(836.06173625,430.33032719)(835.90982656,430.09334808)
\curveto(835.75791688,429.86244535)(835.68196203,429.60419888)(835.68196203,429.31860867)
\curveto(835.68196203,428.88110876)(835.84602449,428.51652551)(836.17414942,428.22485891)
\curveto(836.50835074,427.9331923)(836.99446174,427.787359)(837.63248244,427.787359)
\curveto(838.26442674,427.787359)(838.82649259,427.92407772)(839.31867998,428.19751516)
\curveto(839.81086737,428.47702899)(840.17241243,428.85680321)(840.40331516,429.33683783)
\curveto(840.5795304,429.70749747)(840.66763802,430.25437235)(840.66763802,430.97746247)
\closepath
}
}
{
\newrgbcolor{curcolor}{0 0 0}
\pscustom[linestyle=none,fillstyle=solid,fillcolor=curcolor]
{
\newpath
\moveto(848.44863106,428.18840058)
\lineto(848.68561018,426.73918215)
\curveto(848.22380472,426.64195995)(847.81061037,426.59334885)(847.44602711,426.59334885)
\curveto(846.85054113,426.59334885)(846.38873568,426.68753286)(846.06061075,426.87590087)
\curveto(845.73248582,427.06426888)(845.5015831,427.31036258)(845.36790257,427.61418196)
\curveto(845.23422205,427.92407772)(845.16738178,428.571213)(845.16738178,429.55558778)
\lineto(845.16738178,435.12459697)
\lineto(843.96425705,435.12459697)
\lineto(843.96425705,436.40063836)
\lineto(845.16738178,436.40063836)
\lineto(845.16738178,438.79777325)
\lineto(846.79889184,439.78214803)
\lineto(846.79889184,436.40063836)
\lineto(848.44863106,436.40063836)
\lineto(848.44863106,435.12459697)
\lineto(846.79889184,435.12459697)
\lineto(846.79889184,429.46444197)
\curveto(846.79889184,428.99656013)(846.82623558,428.69577894)(846.88092307,428.56209842)
\curveto(846.94168695,428.42841789)(847.03587095,428.32208111)(847.16347509,428.24308807)
\curveto(847.29715562,428.16409503)(847.48552363,428.12459851)(847.72857914,428.12459851)
\curveto(847.91087076,428.12459851)(848.15088807,428.14586587)(848.44863106,428.18840058)
\closepath
}
}
{
\newrgbcolor{curcolor}{0 0 0}
\pscustom[linestyle=none,fillstyle=solid,fillcolor=curcolor]
{
\newpath
\moveto(856.35365692,427.91496314)
\curveto(855.74601816,427.3984702)(855.15964676,427.03388695)(854.59454272,426.82121338)
\curveto(854.03551506,426.60853982)(853.4339527,426.50220304)(852.78985562,426.50220304)
\curveto(851.72648779,426.50220304)(850.90921367,426.76044951)(850.33803324,427.27694245)
\curveto(849.76685281,427.79951178)(849.48126259,428.46487622)(849.48126259,429.27303576)
\curveto(849.48126259,429.74699399)(849.58759938,430.17841751)(849.80027294,430.56730631)
\curveto(850.01902289,430.9622715)(850.30157491,431.27824365)(850.647929,431.51522277)
\curveto(851.00035948,431.75220188)(851.39532467,431.93145531)(851.83282458,432.05298307)
\curveto(852.15487312,432.13805249)(852.64098412,432.22008372)(853.29115759,432.29907676)
\curveto(854.61581008,432.45706284)(855.59107028,432.64543085)(856.2169382,432.8641808)
\curveto(856.22301458,433.08900714)(856.22605278,433.23180225)(856.22605278,433.29256613)
\curveto(856.22605278,433.96096876)(856.07110489,434.43188879)(855.76120913,434.70532623)
\curveto(855.34193839,435.07598587)(854.71910866,435.26131569)(853.89271996,435.26131569)
\curveto(853.12101874,435.26131569)(852.54983831,435.12459697)(852.17917867,434.85115953)
\curveto(851.81459541,434.58379848)(851.54419617,434.10680206)(851.36798093,433.42017027)
\lineto(849.76381461,433.63892022)
\curveto(849.90964792,434.32555201)(850.14966522,434.87850328)(850.48386654,435.29777402)
\curveto(850.81806785,435.72312115)(851.30114067,436.04820788)(851.93308497,436.27303422)
\curveto(852.56502928,436.50393695)(853.29723398,436.61938831)(854.12969907,436.61938831)
\curveto(854.95608778,436.61938831)(855.6275286,436.52216611)(856.14402154,436.32772171)
\curveto(856.66051449,436.13327731)(857.04028871,435.88718361)(857.28334421,435.58944062)
\curveto(857.52639971,435.29777402)(857.69653856,434.92711438)(857.79376077,434.4774617)
\curveto(857.84844825,434.19794787)(857.875792,433.69360771)(857.875792,432.9644412)
\lineto(857.875792,430.77694168)
\curveto(857.875792,429.2517684)(857.90921213,428.28562278)(857.97605239,427.87850482)
\curveto(858.04896904,427.47746324)(858.18872596,427.09161263)(858.39532313,426.72095299)
\lineto(856.68178184,426.72095299)
\curveto(856.51164299,427.06123069)(856.40226802,427.45923408)(856.35365692,427.91496314)
\closepath
\moveto(856.2169382,431.57902484)
\curveto(855.62145222,431.33596933)(854.72822325,431.12937216)(853.53725128,430.95923331)
\curveto(852.86277227,430.86201111)(852.38577584,430.75263613)(852.10626202,430.63110838)
\curveto(851.82674819,430.50958063)(851.61103643,430.33032719)(851.45912674,430.09334808)
\curveto(851.30721705,429.86244535)(851.23126221,429.60419888)(851.23126221,429.31860867)
\curveto(851.23126221,428.88110876)(851.39532467,428.51652551)(851.7234496,428.22485891)
\curveto(852.05765092,427.9331923)(852.54376192,427.787359)(853.18178261,427.787359)
\curveto(853.81372692,427.787359)(854.37579277,427.92407772)(854.86798016,428.19751516)
\curveto(855.36016755,428.47702899)(855.72171261,428.85680321)(855.95261534,429.33683783)
\curveto(856.12883058,429.70749747)(856.2169382,430.25437235)(856.2169382,430.97746247)
\closepath
}
}
{
\newrgbcolor{curcolor}{0 0 0}
\pscustom[linestyle=none,fillstyle=solid,fillcolor=curcolor]
{
\newpath
\moveto(861.49142137,422.79256843)
\lineto(860.31564038,422.79256843)
\curveto(862.13248026,425.70923446)(863.0409002,428.62893868)(863.0409002,431.55168109)
\curveto(863.0409002,432.69404195)(862.91025787,433.82728823)(862.6489732,434.95141993)
\curveto(862.44237602,435.86287806)(862.15374762,436.73787787)(861.78308798,437.57641935)
\curveto(861.54610886,438.12329423)(861.05695966,439.03475236)(860.31564038,440.31079375)
\lineto(861.49142137,440.31079375)
\curveto(862.63378223,438.78562047)(863.4784001,437.25437081)(864.02527498,435.71704476)
\curveto(864.49315682,434.39239227)(864.72709774,433.00697591)(864.72709774,431.56079567)
\curveto(864.72709774,429.92017103)(864.41112559,428.33423388)(863.77918129,426.80298422)
\curveto(863.15331337,425.27173456)(862.39072673,423.93492929)(861.49142137,422.79256843)
\closepath
}
}
{
\newrgbcolor{curcolor}{0 0 0}
\pscustom[linestyle=none,fillstyle=solid,fillcolor=curcolor,opacity=0]
{
\newpath
\moveto(689.69545425,466.40465077)
\curveto(701.46708396,466.40465077)(707.35289881,460.96371691)(713.23871366,455.52278304)
\curveto(719.12452852,450.08184918)(725.01034337,444.64091531)(736.78197308,444.64091531)
}
}
{
\newrgbcolor{curcolor}{0.49803922 0.49803922 0.49803922}
\pscustom[linewidth=2.99999393,linecolor=curcolor]
{
\newpath
\moveto(689.69545425,466.40465077)
\curveto(701.46708396,466.40465077)(707.35289881,460.96371691)(713.23871366,455.52278304)
\curveto(719.12452852,450.08184918)(725.01034337,444.64091531)(736.78197308,444.64091531)
}
}
{
\newrgbcolor{curcolor}{0 0 0}
\pscustom[linestyle=none,fillstyle=solid,fillcolor=curcolor,opacity=0]
{
\newpath
\moveto(689.69545425,432.01101963)
\curveto(701.46708396,432.01101963)(707.35289881,435.16849355)(713.23871366,438.32596747)
\curveto(719.12452852,441.48344139)(725.01034337,444.64091531)(736.78197308,444.64091531)
}
}
{
\newrgbcolor{curcolor}{0.49803922 0.49803922 0.49803922}
\pscustom[linewidth=2.99999393,linecolor=curcolor]
{
\newpath
\moveto(689.69545425,432.01101963)
\curveto(701.46708396,432.01101963)(707.35289881,435.16849355)(713.23871366,438.32596747)
\curveto(719.12452852,441.48344139)(725.01034337,444.64091531)(736.78197308,444.64091531)
}
}
{
\newrgbcolor{curcolor}{0.90980393 0.90980393 0.90980393}
\pscustom[linestyle=none,fillstyle=solid,fillcolor=curcolor]
{
\newpath
\moveto(736.78197308,391.17104984)
\lineto(736.78197308,391.17104984)
\curveto(736.78197308,394.89652786)(739.80207195,397.91660048)(743.52752372,397.91660048)
\lineto(930.03601743,397.91660048)
\curveto(931.82504268,397.91660048)(933.54083973,397.20591951)(934.80585029,395.9408827)
\curveto(936.07086085,394.67584589)(936.78156808,392.96010133)(936.78156808,391.17104984)
\lineto(936.78156808,364.18979214)
\curveto(936.78156808,360.46431412)(933.7614692,357.44424149)(930.03601743,357.44424149)
\lineto(743.52752372,357.44424149)
\lineto(743.52752372,357.44424149)
\curveto(739.80207195,357.44424149)(736.78197308,360.46431412)(736.78197308,364.18979214)
\closepath
}
}
{
\newrgbcolor{curcolor}{0.10196079 0.8392157 0.96078432}
\pscustom[linewidth=0.99999798,linecolor=curcolor]
{
\newpath
\moveto(736.78197308,391.17104984)
\lineto(736.78197308,391.17104984)
\curveto(736.78197308,394.89652786)(739.80207195,397.91660048)(743.52752372,397.91660048)
\lineto(930.03601743,397.91660048)
\curveto(931.82504268,397.91660048)(933.54083973,397.20591951)(934.80585029,395.9408827)
\curveto(936.07086085,394.67584589)(936.78156808,392.96010133)(936.78156808,391.17104984)
\lineto(936.78156808,364.18979214)
\curveto(936.78156808,360.46431412)(933.7614692,357.44424149)(930.03601743,357.44424149)
\lineto(743.52752372,357.44424149)
\lineto(743.52752372,357.44424149)
\curveto(739.80207195,357.44424149)(736.78197308,360.46431412)(736.78197308,364.18979214)
\closepath
}
}
{
\newrgbcolor{curcolor}{0 0 0}
\pscustom[linestyle=none,fillstyle=solid,fillcolor=curcolor]
{
\newpath
\moveto(752.59750361,381.76039903)
\lineto(752.59750361,393.54555269)
\lineto(748.19516083,393.54555269)
\lineto(748.19516083,395.12237526)
\lineto(758.78630434,395.12237526)
\lineto(758.78630434,393.54555269)
\lineto(754.36573239,393.54555269)
\lineto(754.36573239,381.76039903)
\closepath
}
}
{
\newrgbcolor{curcolor}{0 0 0}
\pscustom[linestyle=none,fillstyle=solid,fillcolor=curcolor]
{
\newpath
\moveto(760.37477265,381.76039903)
\lineto(760.37477265,395.12237526)
\lineto(762.01539729,395.12237526)
\lineto(762.01539729,390.32810548)
\curveto(762.78102212,391.21525807)(763.74716774,391.65883436)(764.91383415,391.65883436)
\curveto(765.63084788,391.65883436)(766.25367761,391.51603925)(766.78232332,391.23044903)
\curveto(767.31096904,390.95093521)(767.68770507,390.5620464)(767.91253141,390.06378262)
\curveto(768.14343413,389.56551885)(768.2588855,388.84242873)(768.2588855,387.89451227)
\lineto(768.2588855,381.76039903)
\lineto(766.61826086,381.76039903)
\lineto(766.61826086,387.89451227)
\curveto(766.61826086,388.71482459)(766.43900743,389.31031057)(766.08050056,389.68097021)
\curveto(765.72807008,390.05770624)(765.22676811,390.24607425)(764.57659464,390.24607425)
\curveto(764.09048364,390.24607425)(763.63171638,390.11847011)(763.20029286,389.86326184)
\curveto(762.77494573,389.61412995)(762.47112635,389.27385224)(762.28883473,388.84242873)
\curveto(762.1065431,388.41100521)(762.01539729,387.81551923)(762.01539729,387.05597079)
\lineto(762.01539729,381.76039903)
\closepath
}
}
{
\newrgbcolor{curcolor}{0 0 0}
\pscustom[linestyle=none,fillstyle=solid,fillcolor=curcolor]
{
\newpath
\moveto(770.76253309,393.23565693)
\lineto(770.76253309,395.12237526)
\lineto(772.40315773,395.12237526)
\lineto(772.40315773,393.23565693)
\closepath
\moveto(770.76253309,381.76039903)
\lineto(770.76253309,391.44008441)
\lineto(772.40315773,391.44008441)
\lineto(772.40315773,381.76039903)
\closepath
}
}
{
\newrgbcolor{curcolor}{0 0 0}
\pscustom[linestyle=none,fillstyle=solid,fillcolor=curcolor]
{
\newpath
\moveto(774.87918005,381.76039903)
\lineto(774.87918005,391.44008441)
\lineto(776.35574223,391.44008441)
\lineto(776.35574223,389.97263681)
\curveto(776.73247826,390.6592686)(777.07883235,391.11195948)(777.3948045,391.33070943)
\curveto(777.71685304,391.54945938)(778.06928352,391.65883436)(778.45209594,391.65883436)
\curveto(779.0050472,391.65883436)(779.56711305,391.48261912)(780.13829348,391.13018864)
\lineto(779.57318944,389.60805356)
\curveto(779.17214786,389.84503267)(778.77110628,389.96352223)(778.3700647,389.96352223)
\curveto(778.01155784,389.96352223)(777.6895093,389.85414725)(777.40391908,389.6353973)
\curveto(777.11832887,389.42272374)(776.91476989,389.12498075)(776.79324213,388.74216833)
\curveto(776.61095051,388.15883513)(776.51980469,387.52081443)(776.51980469,386.82810625)
\lineto(776.51980469,381.76039903)
\closepath
}
}
{
\newrgbcolor{curcolor}{0 0 0}
\pscustom[linestyle=none,fillstyle=solid,fillcolor=curcolor]
{
\newpath
\moveto(787.39334277,381.76039903)
\lineto(787.39334277,382.98175293)
\curveto(786.77962763,382.0216837)(785.87728407,381.54164908)(784.68631211,381.54164908)
\curveto(783.9146109,381.54164908)(783.20367355,381.75432265)(782.55350008,382.17966977)
\curveto(781.909403,382.6050169)(781.40810103,383.19746469)(781.04959416,383.95701313)
\curveto(780.69716369,384.72263797)(780.52094845,385.60067597)(780.52094845,386.59112714)
\curveto(780.52094845,387.55727276)(780.68197272,388.43227257)(781.00402126,389.21612656)
\curveto(781.3260698,390.00605694)(781.80914261,390.6106575)(782.45323969,391.02992825)
\curveto(783.09733677,391.44919899)(783.81738869,391.65883436)(784.61339546,391.65883436)
\curveto(785.19672867,391.65883436)(785.7162598,391.53426841)(786.17198887,391.28513652)
\curveto(786.62771794,391.04208102)(786.99837758,390.72307067)(787.28396779,390.32810548)
\lineto(787.28396779,395.12237526)
\lineto(788.91547785,395.12237526)
\lineto(788.91547785,381.76039903)
\closepath
\moveto(782.20714599,386.59112714)
\curveto(782.20714599,385.35154408)(782.46843066,384.42489498)(782.99099999,383.81117983)
\curveto(783.51356932,383.19746469)(784.13032265,382.89060712)(784.84126,382.89060712)
\curveto(785.55827373,382.89060712)(786.16591248,383.18227372)(786.66417626,383.76560693)
\curveto(787.16851643,384.35501652)(787.42068651,385.25128368)(787.42068651,386.45440842)
\curveto(787.42068651,387.7790609)(787.16547824,388.75128291)(786.65506168,389.37107444)
\curveto(786.14464513,389.99086597)(785.51573902,390.30076174)(784.76834335,390.30076174)
\curveto(784.03917684,390.30076174)(783.42849989,390.00301875)(782.9363125,389.40753277)
\curveto(782.45020149,388.81204679)(782.20714599,387.87324491)(782.20714599,386.59112714)
\closepath
}
}
{
\newrgbcolor{curcolor}{0 0 0}
\pscustom[linestyle=none,fillstyle=solid,fillcolor=curcolor]
{
\newpath
\moveto(802.97910127,385.30597117)
\lineto(804.59238217,385.0963358)
\curveto(804.41616693,383.98435688)(803.96347606,383.11239526)(803.23430955,382.48045096)
\curveto(802.51121943,381.85458304)(801.62102865,381.54164908)(800.56373722,381.54164908)
\curveto(799.23908473,381.54164908)(798.17267872,381.9730726)(797.36451917,382.83591963)
\curveto(796.56243602,383.70484305)(796.16139444,384.9474643)(796.16139444,386.56378339)
\curveto(796.16139444,387.60892205)(796.33457148,388.52341838)(796.68092557,389.30727237)
\curveto(797.02727966,390.09112637)(797.55288719,390.67749777)(798.25774814,391.06638657)
\curveto(798.96868549,391.46135176)(799.74038671,391.65883436)(800.5728518,391.65883436)
\curveto(801.62406685,391.65883436)(802.48387569,391.3914733)(803.15227832,390.8567512)
\curveto(803.82068095,390.32810548)(804.24906627,389.57463343)(804.43743429,388.59633503)
\lineto(802.84238255,388.35024133)
\curveto(802.69047286,389.0004148)(802.42007362,389.489564)(802.03118481,389.81768893)
\curveto(801.6483724,390.14581386)(801.18352875,390.30987632)(800.63665387,390.30987632)
\curveto(799.81026516,390.30987632)(799.13882434,390.01213333)(798.6223314,389.41664735)
\curveto(798.10583845,388.82723776)(797.84759198,387.89147407)(797.84759198,386.6093563)
\curveto(797.84759198,385.30900936)(798.09672387,384.3641311)(798.59498765,383.77472151)
\curveto(799.09325143,383.18531191)(799.7434249,382.89060712)(800.54550806,382.89060712)
\curveto(801.18960514,382.89060712)(801.72736544,383.08808971)(802.15878895,383.4830549)
\curveto(802.59021247,383.8780201)(802.86364991,384.48565885)(802.97910127,385.30597117)
\closepath
}
}
{
\newrgbcolor{curcolor}{0 0 0}
\pscustom[linestyle=none,fillstyle=solid,fillcolor=curcolor]
{
\newpath
\moveto(805.38533644,386.60024172)
\curveto(805.38533644,388.39277605)(805.88360022,389.72046673)(806.88012778,390.58331376)
\curveto(807.71259287,391.30032749)(808.72734959,391.65883436)(809.92439794,391.65883436)
\curveto(811.25512681,391.65883436)(812.34280019,391.22133445)(813.18741806,390.34633465)
\curveto(814.03203593,389.47741123)(814.45434486,388.27428649)(814.45434486,386.73696044)
\curveto(814.45434486,385.49130099)(814.26597685,384.5099644)(813.88924082,383.79295067)
\curveto(813.51858118,383.08201333)(812.97474449,382.52906206)(812.25773076,382.13409687)
\curveto(811.54679342,381.73913168)(810.76901581,381.54164908)(809.92439794,381.54164908)
\curveto(808.56936352,381.54164908)(807.47257556,381.97611079)(806.63403408,382.84503421)
\curveto(805.80156899,383.71395763)(805.38533644,384.96569347)(805.38533644,386.60024172)
\closepath
\moveto(807.07153398,386.60024172)
\curveto(807.07153398,385.36065866)(807.34193323,384.43097136)(807.88273172,383.81117983)
\curveto(808.42353021,383.19746469)(809.10408562,382.89060712)(809.92439794,382.89060712)
\curveto(810.73863387,382.89060712)(811.41615108,383.20050288)(811.95694958,383.82029441)
\curveto(812.49774807,384.44008594)(812.76814732,385.38496421)(812.76814732,386.65492921)
\curveto(812.76814732,387.85197756)(812.49470988,388.7573593)(811.947835,389.37107444)
\curveto(811.4070365,389.99086597)(810.73255749,390.30076174)(809.92439794,390.30076174)
\curveto(809.10408562,390.30076174)(808.42353021,389.99390417)(807.88273172,389.38018902)
\curveto(807.34193323,388.76647388)(807.07153398,387.83982478)(807.07153398,386.60024172)
\closepath
}
}
{
\newrgbcolor{curcolor}{0 0 0}
\pscustom[linestyle=none,fillstyle=solid,fillcolor=curcolor]
{
\newpath
\moveto(816.37465829,378.05076443)
\lineto(816.37465829,391.44008441)
\lineto(817.86944963,391.44008441)
\lineto(817.86944963,390.18227218)
\curveto(818.22188011,390.67445957)(818.6198835,391.04208102)(819.06345979,391.28513652)
\curveto(819.50703608,391.53426841)(820.04479638,391.65883436)(820.67674068,391.65883436)
\curveto(821.50312939,391.65883436)(822.2322959,391.44616079)(822.8642402,391.02081366)
\curveto(823.49618451,390.59546654)(823.97318093,389.99390417)(824.29522947,389.21612656)
\curveto(824.61727801,388.44442534)(824.77830228,387.59676928)(824.77830228,386.67315837)
\curveto(824.77830228,385.6827072)(824.59904885,384.78947823)(824.24054198,383.99347146)
\curveto(823.8881115,383.20354108)(823.37161856,382.59590232)(822.69106316,382.17055519)
\curveto(822.01658414,381.75128445)(821.30564679,381.54164908)(820.55825112,381.54164908)
\curveto(820.01137625,381.54164908)(819.51918885,381.65710044)(819.08168895,381.88800317)
\curveto(818.65026543,382.1189059)(818.29479676,382.4105725)(818.01528293,382.76300298)
\lineto(818.01528293,378.05076443)
\closepath
\moveto(817.86033505,386.54555423)
\curveto(817.86033505,385.29989478)(818.11250513,384.37932207)(818.6168453,383.78383609)
\curveto(819.12118547,383.18835011)(819.73186242,382.89060712)(820.44887615,382.89060712)
\curveto(821.17804266,382.89060712)(821.80087238,383.19746469)(822.31736532,383.81117983)
\curveto(822.83993465,384.43097136)(823.10121932,385.3880024)(823.10121932,386.68227295)
\curveto(823.10121932,387.91577962)(822.84601104,388.83939053)(822.33559448,389.45310568)
\curveto(821.83125432,390.06682082)(821.22665376,390.37367839)(820.5217928,390.37367839)
\curveto(819.82300823,390.37367839)(819.2032167,390.04555346)(818.66241821,389.38930361)
\curveto(818.1276961,388.73913014)(817.86033505,387.79121368)(817.86033505,386.54555423)
\closepath
}
}
{
\newrgbcolor{curcolor}{0 0 0}
\pscustom[linestyle=none,fillstyle=solid,fillcolor=curcolor]
{
\newpath
\moveto(826.68038274,378.03253527)
\lineto(826.49809111,379.57289951)
\curveto(826.85659798,379.47567731)(827.16953193,379.42706621)(827.43689299,379.42706621)
\curveto(827.80147624,379.42706621)(828.09314284,379.48783009)(828.31189279,379.60935784)
\curveto(828.53064275,379.73088559)(828.70989618,379.90102444)(828.84965309,380.11977439)
\curveto(828.95295168,380.28383686)(829.12005234,380.69095482)(829.35095507,381.34112829)
\curveto(829.381337,381.43227411)(829.4299481,381.56595463)(829.49678837,381.74216987)
\lineto(825.82361209,391.44008441)
\lineto(827.59184087,391.44008441)
\lineto(829.60616334,385.83461689)
\curveto(829.86744801,385.12367954)(830.10138893,384.37628387)(830.30798611,383.59242988)
\curveto(830.49635412,384.34590194)(830.72118046,385.08114483)(830.98246512,385.79815856)
\lineto(833.05147509,391.44008441)
\lineto(834.69209973,391.44008441)
\lineto(831.00980887,381.59633657)
\curveto(830.61484368,380.53296875)(830.30798611,379.80076405)(830.08923615,379.39972247)
\curveto(829.79756955,378.85892398)(829.46336824,378.46395879)(829.08663221,378.2148269)
\curveto(828.70989618,377.95961862)(828.2602435,377.83201448)(827.73767417,377.83201448)
\curveto(827.42170202,377.83201448)(827.06927154,377.89885474)(826.68038274,378.03253527)
\closepath
}
}
{
\newrgbcolor{curcolor}{0 0 0}
\pscustom[linestyle=none,fillstyle=solid,fillcolor=curcolor]
{
\newpath
\moveto(840.6465985,386.60024172)
\curveto(840.6465985,388.39277605)(841.14486228,389.72046673)(842.14138984,390.58331376)
\curveto(842.97385493,391.30032749)(843.98861165,391.65883436)(845.18566,391.65883436)
\curveto(846.51638888,391.65883436)(847.60406225,391.22133445)(848.44868012,390.34633465)
\curveto(849.29329799,389.47741123)(849.71560692,388.27428649)(849.71560692,386.73696044)
\curveto(849.71560692,385.49130099)(849.52723891,384.5099644)(849.15050288,383.79295067)
\curveto(848.77984324,383.08201333)(848.23600655,382.52906206)(847.51899282,382.13409687)
\curveto(846.80805548,381.73913168)(846.03027787,381.54164908)(845.18566,381.54164908)
\curveto(843.83062558,381.54164908)(842.73383762,381.97611079)(841.89529614,382.84503421)
\curveto(841.06283105,383.71395763)(840.6465985,384.96569347)(840.6465985,386.60024172)
\closepath
\moveto(842.33279605,386.60024172)
\curveto(842.33279605,385.36065866)(842.60319529,384.43097136)(843.14399378,383.81117983)
\curveto(843.68479228,383.19746469)(844.36534768,382.89060712)(845.18566,382.89060712)
\curveto(845.99989593,382.89060712)(846.67741315,383.20050288)(847.21821164,383.82029441)
\curveto(847.75901013,384.44008594)(848.02940938,385.38496421)(848.02940938,386.65492921)
\curveto(848.02940938,387.85197756)(847.75597194,388.7573593)(847.20909706,389.37107444)
\curveto(846.66829856,389.99086597)(845.99381955,390.30076174)(845.18566,390.30076174)
\curveto(844.36534768,390.30076174)(843.68479228,389.99390417)(843.14399378,389.38018902)
\curveto(842.60319529,388.76647388)(842.33279605,387.83982478)(842.33279605,386.60024172)
\closepath
}
}
{
\newrgbcolor{curcolor}{0 0 0}
\pscustom[linestyle=none,fillstyle=solid,fillcolor=curcolor]
{
\newpath
\moveto(852.02783591,381.76039903)
\lineto(852.02783591,390.16404302)
\lineto(850.57861748,390.16404302)
\lineto(850.57861748,391.44008441)
\lineto(852.02783591,391.44008441)
\lineto(852.02783591,392.4700321)
\curveto(852.02783591,393.12020556)(852.08556159,393.60327837)(852.20101295,393.91925053)
\curveto(852.35899903,394.34459766)(852.63547466,394.68791355)(853.03043986,394.94919822)
\curveto(853.43148143,395.21655927)(853.99050909,395.3502398)(854.70752282,395.3502398)
\curveto(855.16932827,395.3502398)(855.67974483,395.29555231)(856.23877248,395.18617733)
\lineto(855.99267879,393.75518806)
\curveto(855.65240108,393.81595194)(855.33035254,393.84633388)(855.02653317,393.84633388)
\curveto(854.52826939,393.84633388)(854.17583891,393.73999709)(853.96924173,393.52732353)
\curveto(853.76264456,393.31464997)(853.65934597,392.91664658)(853.65934597,392.33331338)
\lineto(853.65934597,391.44008441)
\lineto(855.5460643,391.44008441)
\lineto(855.5460643,390.16404302)
\lineto(853.65934597,390.16404302)
\lineto(853.65934597,381.76039903)
\closepath
}
}
{
\newrgbcolor{curcolor}{0 0 0}
\pscustom[linestyle=none,fillstyle=solid,fillcolor=curcolor]
{
\newpath
\moveto(868.25717362,381.76039903)
\lineto(868.25717362,382.98175293)
\curveto(867.64345848,382.0216837)(866.74111492,381.54164908)(865.55014296,381.54164908)
\curveto(864.77844174,381.54164908)(864.0675044,381.75432265)(863.41733093,382.17966977)
\curveto(862.77323385,382.6050169)(862.27193188,383.19746469)(861.91342501,383.95701313)
\curveto(861.56099453,384.72263797)(861.3847793,385.60067597)(861.3847793,386.59112714)
\curveto(861.3847793,387.55727276)(861.54580357,388.43227257)(861.86785211,389.21612656)
\curveto(862.18990065,390.00605694)(862.67297346,390.6106575)(863.31707054,391.02992825)
\curveto(863.96116762,391.44919899)(864.68121954,391.65883436)(865.47722631,391.65883436)
\curveto(866.06055952,391.65883436)(866.58009065,391.53426841)(867.03581972,391.28513652)
\curveto(867.49154879,391.04208102)(867.86220843,390.72307067)(868.14779864,390.32810548)
\lineto(868.14779864,395.12237526)
\lineto(869.7793087,395.12237526)
\lineto(869.7793087,381.76039903)
\closepath
\moveto(863.07097684,386.59112714)
\curveto(863.07097684,385.35154408)(863.33226151,384.42489498)(863.85483084,383.81117983)
\curveto(864.37740017,383.19746469)(864.9941535,382.89060712)(865.70509085,382.89060712)
\curveto(866.42210458,382.89060712)(867.02974333,383.18227372)(867.52800711,383.76560693)
\curveto(868.03234728,384.35501652)(868.28451736,385.25128368)(868.28451736,386.45440842)
\curveto(868.28451736,387.7790609)(868.02930908,388.75128291)(867.51889253,389.37107444)
\curveto(867.00847598,389.99086597)(866.37956986,390.30076174)(865.6321742,390.30076174)
\curveto(864.90300769,390.30076174)(864.29233074,390.00301875)(863.80014335,389.40753277)
\curveto(863.31403234,388.81204679)(863.07097684,387.87324491)(863.07097684,386.59112714)
\closepath
}
}
{
\newrgbcolor{curcolor}{0 0 0}
\pscustom[linestyle=none,fillstyle=solid,fillcolor=curcolor]
{
\newpath
\moveto(878.67228067,382.95440919)
\curveto(878.06464191,382.43791625)(877.47827051,382.07333299)(876.91316647,381.86065943)
\curveto(876.35413881,381.64798586)(875.75257645,381.54164908)(875.10847937,381.54164908)
\curveto(874.04511154,381.54164908)(873.22783742,381.79989555)(872.65665699,382.31638849)
\curveto(872.08547656,382.83895782)(871.79988634,383.50432226)(871.79988634,384.31248181)
\curveto(871.79988634,384.78644003)(871.90622313,385.21786355)(872.11889669,385.60675235)
\curveto(872.33764664,386.00171755)(872.62019866,386.3176897)(872.96655275,386.55466881)
\curveto(873.31898323,386.79164793)(873.71394842,386.97090136)(874.15144833,387.09242911)
\curveto(874.47349687,387.17749854)(874.95960787,387.25952977)(875.60978134,387.33852281)
\curveto(876.93443383,387.49650888)(877.90969403,387.6848769)(878.53556195,387.90362685)
\curveto(878.54163833,388.12845319)(878.54467653,388.2712483)(878.54467653,388.33201217)
\curveto(878.54467653,389.0004148)(878.38972864,389.47133484)(878.07983288,389.74477228)
\curveto(877.66056214,390.11543192)(877.03773241,390.30076174)(876.21134371,390.30076174)
\curveto(875.43964249,390.30076174)(874.86846206,390.16404302)(874.49780242,389.89060558)
\curveto(874.13321916,389.62324453)(873.86281992,389.1462481)(873.68660468,388.45961631)
\lineto(872.08243836,388.67836626)
\curveto(872.22827167,389.36499806)(872.46828897,389.91794932)(872.80249029,390.33722006)
\curveto(873.1366916,390.76256719)(873.61976442,391.08765393)(874.25170872,391.31248027)
\curveto(874.88365303,391.54338299)(875.61585773,391.65883436)(876.44832282,391.65883436)
\curveto(877.27471153,391.65883436)(877.94615235,391.56161216)(878.46264529,391.36716775)
\curveto(878.97913824,391.17272335)(879.35891246,390.92662966)(879.60196796,390.62888667)
\curveto(879.84502346,390.33722006)(880.01516231,389.96656042)(880.11238452,389.51690774)
\curveto(880.167072,389.23739392)(880.19441575,388.73305375)(880.19441575,388.00388724)
\lineto(880.19441575,385.81638773)
\curveto(880.19441575,384.29121445)(880.22783588,383.32506883)(880.29467614,382.91795086)
\curveto(880.36759279,382.51690928)(880.50734971,382.13105867)(880.71394688,381.76039903)
\lineto(879.00040559,381.76039903)
\curveto(878.83026674,382.10067674)(878.72089177,382.49868012)(878.67228067,382.95440919)
\closepath
\moveto(878.53556195,386.61847088)
\curveto(877.94007597,386.37541538)(877.046847,386.1688182)(875.85587503,385.99867935)
\curveto(875.18139602,385.90145715)(874.70439959,385.79208217)(874.42488577,385.67055442)
\curveto(874.14537194,385.54902667)(873.92966018,385.36977324)(873.77775049,385.13279413)
\curveto(873.6258408,384.9018914)(873.54988596,384.64364493)(873.54988596,384.35805471)
\curveto(873.54988596,383.92055481)(873.71394842,383.55597156)(874.04207335,383.26430495)
\curveto(874.37627467,382.97263835)(874.86238567,382.82680505)(875.50040636,382.82680505)
\curveto(876.13235067,382.82680505)(876.69441652,382.96352377)(877.18660391,383.23696121)
\curveto(877.6787913,383.51647504)(878.04033636,383.89624926)(878.27123909,384.37628387)
\curveto(878.44745433,384.74694352)(878.53556195,385.2938184)(878.53556195,386.01690851)
\closepath
}
}
{
\newrgbcolor{curcolor}{0 0 0}
\pscustom[linestyle=none,fillstyle=solid,fillcolor=curcolor]
{
\newpath
\moveto(886.31654736,383.22784663)
\lineto(886.55352647,381.7786282)
\curveto(886.09172102,381.681406)(885.67852667,381.63279489)(885.31394341,381.63279489)
\curveto(884.71845743,381.63279489)(884.25665198,381.7269789)(883.92852705,381.91534692)
\curveto(883.60040212,382.10371493)(883.3694994,382.34980863)(883.23581887,382.653628)
\curveto(883.10213834,382.96352377)(883.03529808,383.61065904)(883.03529808,384.59503383)
\lineto(883.03529808,390.16404302)
\lineto(881.83217335,390.16404302)
\lineto(881.83217335,391.44008441)
\lineto(883.03529808,391.44008441)
\lineto(883.03529808,393.8372193)
\lineto(884.66680814,394.82159408)
\lineto(884.66680814,391.44008441)
\lineto(886.31654736,391.44008441)
\lineto(886.31654736,390.16404302)
\lineto(884.66680814,390.16404302)
\lineto(884.66680814,384.50388801)
\curveto(884.66680814,384.03600617)(884.69415188,383.73522499)(884.74883937,383.60154446)
\curveto(884.80960325,383.46786394)(884.90378725,383.36152715)(885.03139139,383.28253412)
\curveto(885.16507192,383.20354108)(885.35343993,383.16404456)(885.59649543,383.16404456)
\curveto(885.77878706,383.16404456)(886.01880437,383.18531191)(886.31654736,383.22784663)
\closepath
}
}
{
\newrgbcolor{curcolor}{0 0 0}
\pscustom[linestyle=none,fillstyle=solid,fillcolor=curcolor]
{
\newpath
\moveto(894.22158084,382.95440919)
\curveto(893.61394209,382.43791625)(893.02757069,382.07333299)(892.46246665,381.86065943)
\curveto(891.90343899,381.64798586)(891.30187662,381.54164908)(890.65777954,381.54164908)
\curveto(889.59441172,381.54164908)(888.7771376,381.79989555)(888.20595717,382.31638849)
\curveto(887.63477674,382.83895782)(887.34918652,383.50432226)(887.34918652,384.31248181)
\curveto(887.34918652,384.78644003)(887.4555233,385.21786355)(887.66819687,385.60675235)
\curveto(887.88694682,386.00171755)(888.16949884,386.3176897)(888.51585293,386.55466881)
\curveto(888.86828341,386.79164793)(889.2632486,386.97090136)(889.7007485,387.09242911)
\curveto(890.02279704,387.17749854)(890.50890805,387.25952977)(891.15908152,387.33852281)
\curveto(892.483734,387.49650888)(893.45899421,387.6848769)(894.08486212,387.90362685)
\curveto(894.09093851,388.12845319)(894.0939767,388.2712483)(894.0939767,388.33201217)
\curveto(894.0939767,389.0004148)(893.93902882,389.47133484)(893.62913306,389.74477228)
\curveto(893.20986232,390.11543192)(892.58703259,390.30076174)(891.76064388,390.30076174)
\curveto(890.98894266,390.30076174)(890.41776223,390.16404302)(890.04710259,389.89060558)
\curveto(889.68251934,389.62324453)(889.41212009,389.1462481)(889.23590486,388.45961631)
\lineto(887.63173854,388.67836626)
\curveto(887.77757184,389.36499806)(888.01758915,389.91794932)(888.35179047,390.33722006)
\curveto(888.68599178,390.76256719)(889.16906459,391.08765393)(889.8010089,391.31248027)
\curveto(890.4329532,391.54338299)(891.1651579,391.65883436)(891.997623,391.65883436)
\curveto(892.82401171,391.65883436)(893.49545253,391.56161216)(894.01194547,391.36716775)
\curveto(894.52843841,391.17272335)(894.90821264,390.92662966)(895.15126814,390.62888667)
\curveto(895.39432364,390.33722006)(895.56446249,389.96656042)(895.66168469,389.51690774)
\curveto(895.71637218,389.23739392)(895.74371593,388.73305375)(895.74371593,388.00388724)
\lineto(895.74371593,385.81638773)
\curveto(895.74371593,384.29121445)(895.77713606,383.32506883)(895.84397632,382.91795086)
\curveto(895.91689297,382.51690928)(896.05664988,382.13105867)(896.26324706,381.76039903)
\lineto(894.54970577,381.76039903)
\curveto(894.37956692,382.10067674)(894.27019194,382.49868012)(894.22158084,382.95440919)
\closepath
\moveto(894.08486212,386.61847088)
\curveto(893.48937614,386.37541538)(892.59614717,386.1688182)(891.40517521,385.99867935)
\curveto(890.73069619,385.90145715)(890.25369977,385.79208217)(889.97418594,385.67055442)
\curveto(889.69467212,385.54902667)(889.47896036,385.36977324)(889.32705067,385.13279413)
\curveto(889.17514098,384.9018914)(889.09918614,384.64364493)(889.09918614,384.35805471)
\curveto(889.09918614,383.92055481)(889.2632486,383.55597156)(889.59137353,383.26430495)
\curveto(889.92557484,382.97263835)(890.41168585,382.82680505)(891.04970654,382.82680505)
\curveto(891.68165085,382.82680505)(892.24371669,382.96352377)(892.73590409,383.23696121)
\curveto(893.22809148,383.51647504)(893.58963654,383.89624926)(893.82053926,384.37628387)
\curveto(893.9967545,384.74694352)(894.08486212,385.2938184)(894.08486212,386.01690851)
\closepath
}
}
{
\newrgbcolor{curcolor}{0 0 0}
\pscustom[linestyle=none,fillstyle=solid,fillcolor=curcolor]
{
\newpath
\moveto(752.12354539,355.83205603)
\curveto(751.21816364,356.97441689)(750.45253881,358.31122215)(749.82667089,359.84247182)
\curveto(749.20080297,361.37372148)(748.88786901,362.95965863)(748.88786901,364.60028327)
\curveto(748.88786901,366.04646351)(749.12180993,367.43187987)(749.58969178,368.75653236)
\curveto(750.13656666,370.29385841)(750.98118453,371.82510807)(752.12354539,373.35028135)
\lineto(753.29932638,373.35028135)
\curveto(752.56408348,372.08639273)(752.07797248,371.18404918)(751.84099336,370.64325069)
\curveto(751.47033372,369.80470921)(751.17866712,368.9297094)(750.96599356,368.01825127)
\curveto(750.70470889,366.8819668)(750.57406656,365.73960594)(750.57406656,364.59116869)
\curveto(750.57406656,361.66842627)(751.4824865,358.74872206)(753.29932638,355.83205603)
\closepath
}
}
{
\newrgbcolor{curcolor}{0 0 0}
\pscustom[linestyle=none,fillstyle=solid,fillcolor=curcolor]
{
\newpath
\moveto(755.20411642,356.05080598)
\lineto(755.20411642,369.44012596)
\lineto(756.69890776,369.44012596)
\lineto(756.69890776,368.18231373)
\curveto(757.05133824,368.67450112)(757.44934163,369.04212257)(757.89291792,369.28517807)
\curveto(758.33649421,369.53430996)(758.87425451,369.65887591)(759.50619881,369.65887591)
\curveto(760.33258752,369.65887591)(761.06175403,369.44620234)(761.69369833,369.02085521)
\curveto(762.32564264,368.59550809)(762.80263906,367.99394572)(763.1246876,367.21616811)
\curveto(763.44673614,366.44446689)(763.60776041,365.59681083)(763.60776041,364.67319992)
\curveto(763.60776041,363.68274875)(763.42850698,362.78951978)(763.07000011,361.99351301)
\curveto(762.71756963,361.20358263)(762.20107669,360.59594387)(761.52052129,360.17059674)
\curveto(760.84604227,359.751326)(760.13510492,359.54169063)(759.38770925,359.54169063)
\curveto(758.84083437,359.54169063)(758.34864698,359.65714199)(757.91114708,359.88804472)
\curveto(757.47972356,360.11894745)(757.12425489,360.41061405)(756.84474106,360.76304453)
\lineto(756.84474106,356.05080598)
\closepath
\moveto(756.68979318,364.54559578)
\curveto(756.68979318,363.29993633)(756.94196326,362.37936362)(757.44630343,361.78387764)
\curveto(757.9506436,361.18839166)(758.56132055,360.89064867)(759.27833428,360.89064867)
\curveto(760.00750079,360.89064867)(760.63033051,361.19750624)(761.14682345,361.81122138)
\curveto(761.66939278,362.43101291)(761.93067745,363.38804395)(761.93067745,364.6823145)
\curveto(761.93067745,365.91582117)(761.67546917,366.83943208)(761.16505261,367.45314723)
\curveto(760.66071245,368.06686237)(760.05611189,368.37371994)(759.35125093,368.37371994)
\curveto(758.65246636,368.37371994)(758.03267483,368.04559501)(757.49187634,367.38934516)
\curveto(756.95715423,366.73917169)(756.68979318,365.79125523)(756.68979318,364.54559578)
\closepath
}
}
{
\newrgbcolor{curcolor}{0 0 0}
\pscustom[linestyle=none,fillstyle=solid,fillcolor=curcolor]
{
\newpath
\moveto(764.97208438,364.60028327)
\curveto(764.97208438,366.3928176)(765.47034816,367.72050828)(766.46687572,368.58335531)
\curveto(767.29934082,369.30036904)(768.31409754,369.65887591)(769.51114588,369.65887591)
\curveto(770.84187476,369.65887591)(771.92954813,369.221376)(772.774166,368.3463762)
\curveto(773.61878387,367.47745278)(774.04109281,366.27432804)(774.04109281,364.73700199)
\curveto(774.04109281,363.49134254)(773.85272479,362.51000595)(773.47598876,361.79299222)
\curveto(773.10532912,361.08205488)(772.56149244,360.52910361)(771.84447871,360.13413842)
\curveto(771.13354136,359.73917323)(770.35576375,359.54169063)(769.51114588,359.54169063)
\curveto(768.15611146,359.54169063)(767.05932351,359.97615234)(766.22078202,360.84507576)
\curveto(765.38831693,361.71399918)(764.97208438,362.96573502)(764.97208438,364.60028327)
\closepath
\moveto(766.65828193,364.60028327)
\curveto(766.65828193,363.36070021)(766.92868117,362.43101291)(767.46947967,361.81122138)
\curveto(768.01027816,361.19750624)(768.69083357,360.89064867)(769.51114588,360.89064867)
\curveto(770.32538182,360.89064867)(771.00289903,361.20054443)(771.54369752,361.82033596)
\curveto(772.08449601,362.44012749)(772.35489526,363.38500576)(772.35489526,364.65497076)
\curveto(772.35489526,365.85201911)(772.08145782,366.75740085)(771.53458294,367.37111599)
\curveto(770.99378445,367.99090752)(770.31930543,368.30080329)(769.51114588,368.30080329)
\curveto(768.69083357,368.30080329)(768.01027816,367.99394572)(767.46947967,367.38023057)
\curveto(766.92868117,366.76651543)(766.65828193,365.83986633)(766.65828193,364.60028327)
\closepath
}
}
{
\newrgbcolor{curcolor}{0 0 0}
\pscustom[linestyle=none,fillstyle=solid,fillcolor=curcolor]
{
\newpath
\moveto(775.35072929,364.60028327)
\curveto(775.35072929,366.3928176)(775.84899307,367.72050828)(776.84552063,368.58335531)
\curveto(777.67798572,369.30036904)(778.69274244,369.65887591)(779.88979079,369.65887591)
\curveto(781.22051967,369.65887591)(782.30819304,369.221376)(783.15281091,368.3463762)
\curveto(783.99742878,367.47745278)(784.41973771,366.27432804)(784.41973771,364.73700199)
\curveto(784.41973771,363.49134254)(784.2313697,362.51000595)(783.85463367,361.79299222)
\curveto(783.48397403,361.08205488)(782.94013734,360.52910361)(782.22312361,360.13413842)
\curveto(781.51218627,359.73917323)(780.73440866,359.54169063)(779.88979079,359.54169063)
\curveto(778.53475637,359.54169063)(777.43796841,359.97615234)(776.59942693,360.84507576)
\curveto(775.76696184,361.71399918)(775.35072929,362.96573502)(775.35072929,364.60028327)
\closepath
\moveto(777.03692684,364.60028327)
\curveto(777.03692684,363.36070021)(777.30732608,362.43101291)(777.84812457,361.81122138)
\curveto(778.38892307,361.19750624)(779.06947847,360.89064867)(779.88979079,360.89064867)
\curveto(780.70402672,360.89064867)(781.38154394,361.20054443)(781.92234243,361.82033596)
\curveto(782.46314092,362.44012749)(782.73354017,363.38500576)(782.73354017,364.65497076)
\curveto(782.73354017,365.85201911)(782.46010273,366.75740085)(781.91322785,367.37111599)
\curveto(781.37242935,367.99090752)(780.69795034,368.30080329)(779.88979079,368.30080329)
\curveto(779.06947847,368.30080329)(778.38892307,367.99394572)(777.84812457,367.38023057)
\curveto(777.30732608,366.76651543)(777.03692684,365.83986633)(777.03692684,364.60028327)
\closepath
}
}
{
\newrgbcolor{curcolor}{0 0 0}
\pscustom[linestyle=none,fillstyle=solid,fillcolor=curcolor]
{
\newpath
\moveto(786.30359282,359.76044058)
\lineto(786.30359282,373.12241681)
\lineto(787.94421746,373.12241681)
\lineto(787.94421746,359.76044058)
\closepath
}
}
{
\newrgbcolor{curcolor}{0 0 0}
\pscustom[linestyle=none,fillstyle=solid,fillcolor=curcolor]
{
\newpath
\moveto(789.84602307,363.77085637)
\lineto(789.84602307,365.42059559)
\lineto(794.88638654,365.42059559)
\lineto(794.88638654,363.77085637)
\closepath
}
}
{
\newrgbcolor{curcolor}{0 0 0}
\pscustom[linestyle=none,fillstyle=solid,fillcolor=curcolor]
{
\newpath
\moveto(798.48648872,359.76044058)
\lineto(795.52424979,369.44012596)
\lineto(797.21956191,369.44012596)
\lineto(798.75992616,363.8528876)
\lineto(799.33414478,361.77476306)
\curveto(799.35845033,361.87806165)(799.52555099,362.54342608)(799.83544675,363.77085637)
\lineto(801.375811,369.44012596)
\lineto(803.06200855,369.44012596)
\lineto(804.51122698,363.82554386)
\lineto(804.99429979,361.97528385)
\lineto(805.55028925,363.84377302)
\lineto(807.20914305,369.44012596)
\lineto(808.80419478,369.44012596)
\lineto(805.77815378,359.76044058)
\lineto(804.07372707,359.76044058)
\lineto(802.53336283,365.55731431)
\lineto(802.15966499,367.20705353)
\lineto(800.20003001,359.76044058)
\closepath
}
}
{
\newrgbcolor{curcolor}{0 0 0}
\pscustom[linestyle=none,fillstyle=solid,fillcolor=curcolor]
{
\newpath
\moveto(810.18645114,371.23569848)
\lineto(810.18645114,373.12241681)
\lineto(811.82707578,373.12241681)
\lineto(811.82707578,371.23569848)
\closepath
\moveto(810.18645114,359.76044058)
\lineto(810.18645114,369.44012596)
\lineto(811.82707578,369.44012596)
\lineto(811.82707578,359.76044058)
\closepath
}
}
{
\newrgbcolor{curcolor}{0 0 0}
\pscustom[linestyle=none,fillstyle=solid,fillcolor=curcolor]
{
\newpath
\moveto(820.60127189,359.76044058)
\lineto(820.60127189,360.98179448)
\curveto(819.98755675,360.02172525)(819.0852132,359.54169063)(817.89424124,359.54169063)
\curveto(817.12254002,359.54169063)(816.41160267,359.7543642)(815.76142921,360.17971132)
\curveto(815.11733212,360.60505845)(814.61603015,361.19750624)(814.25752329,361.95705468)
\curveto(813.90509281,362.72267952)(813.72887757,363.60071752)(813.72887757,364.59116869)
\curveto(813.72887757,365.55731431)(813.88990184,366.43231412)(814.21195038,367.21616811)
\curveto(814.53399892,368.00609849)(815.01707173,368.61069905)(815.66116881,369.0299698)
\curveto(816.30526589,369.44924054)(817.02531782,369.65887591)(817.82132459,369.65887591)
\curveto(818.40465779,369.65887591)(818.92418893,369.53430996)(819.37991799,369.28517807)
\curveto(819.83564706,369.04212257)(820.2063067,368.72311222)(820.49189692,368.32814703)
\lineto(820.49189692,373.12241681)
\lineto(822.12340697,373.12241681)
\lineto(822.12340697,359.76044058)
\closepath
\moveto(815.41507512,364.59116869)
\curveto(815.41507512,363.35158563)(815.67635978,362.42493653)(816.19892911,361.81122138)
\curveto(816.72149844,361.19750624)(817.33825178,360.89064867)(818.04918912,360.89064867)
\curveto(818.76620285,360.89064867)(819.37384161,361.18231527)(819.87210539,361.76564848)
\curveto(820.37644555,362.35505807)(820.62861564,363.25132523)(820.62861564,364.45444997)
\curveto(820.62861564,365.77910245)(820.37340736,366.75132446)(819.8629908,367.37111599)
\curveto(819.35257425,367.99090752)(818.72366814,368.30080329)(817.97627247,368.30080329)
\curveto(817.24710596,368.30080329)(816.63642901,368.0030603)(816.14424162,367.40757432)
\curveto(815.65813062,366.81208834)(815.41507512,365.87328646)(815.41507512,364.59116869)
\closepath
}
}
{
\newrgbcolor{curcolor}{0 0 0}
\pscustom[linestyle=none,fillstyle=solid,fillcolor=curcolor]
{
\newpath
\moveto(831.3262747,362.8776274)
\lineto(833.02158683,362.66799203)
\curveto(832.75422578,361.67754086)(832.25900019,360.90887783)(831.53591007,360.36200295)
\curveto(830.81281996,359.81512807)(829.88920905,359.54169063)(828.76507735,359.54169063)
\curveto(827.34927905,359.54169063)(826.22514735,359.97615234)(825.39268226,360.84507576)
\curveto(824.56629355,361.72007557)(824.1530992,362.94446766)(824.1530992,364.51825204)
\curveto(824.1530992,366.1467239)(824.57236994,367.41061251)(825.41091142,368.30991787)
\curveto(826.2494529,369.20922323)(827.33712628,369.65887591)(828.67393154,369.65887591)
\curveto(829.96820209,369.65887591)(831.02549352,369.21833781)(831.84580584,368.33726161)
\curveto(832.66611816,367.45618542)(833.07627432,366.21660236)(833.07627432,364.61851243)
\curveto(833.07627432,364.52129023)(833.07323613,364.37545693)(833.06715974,364.18101253)
\lineto(825.84841133,364.18101253)
\curveto(825.9091752,363.11764471)(826.20995638,362.30340877)(826.75075488,361.73830473)
\curveto(827.29155337,361.17320069)(827.96603239,360.89064867)(828.77419193,360.89064867)
\curveto(829.3757543,360.89064867)(829.88920905,361.04863474)(830.31455618,361.3646069)
\curveto(830.73990331,361.68057905)(831.07714281,362.18491922)(831.3262747,362.8776274)
\closepath
\moveto(825.93955714,365.52997056)
\lineto(831.34450387,365.52997056)
\curveto(831.27158722,366.3442065)(831.06499004,366.95488345)(830.72471234,367.36200141)
\curveto(830.20214301,367.99394572)(829.52462579,368.30991787)(828.6921607,368.30991787)
\curveto(827.93868864,368.30991787)(827.30370614,368.05774779)(826.7872132,367.55340762)
\curveto(826.27679665,367.04906745)(825.99424463,366.37458843)(825.93955714,365.52997056)
\closepath
}
}
{
\newrgbcolor{curcolor}{0 0 0}
\pscustom[linestyle=none,fillstyle=solid,fillcolor=curcolor]
{
\newpath
\moveto(840.24927426,359.76044058)
\lineto(840.24927426,369.44012596)
\lineto(841.71672185,369.44012596)
\lineto(841.71672185,368.08205334)
\curveto(842.02054123,368.55601157)(842.424621,368.93578579)(842.92896117,369.221376)
\curveto(843.43330133,369.51304261)(844.00751996,369.65887591)(844.65161704,369.65887591)
\curveto(845.36863077,369.65887591)(845.95500217,369.51000441)(846.41073123,369.21226142)
\curveto(846.87253669,368.91451843)(847.19762342,368.49828588)(847.38599144,367.96356378)
\curveto(848.15161627,369.09377186)(849.14814383,369.65887591)(850.37557411,369.65887591)
\curveto(851.33564335,369.65887591)(852.07392443,369.39151485)(852.59041738,368.85679275)
\curveto(853.10691032,368.32814703)(853.36515679,367.51087291)(853.36515679,366.40497037)
\lineto(853.36515679,359.76044058)
\lineto(851.73364673,359.76044058)
\lineto(851.73364673,365.85809549)
\curveto(851.73364673,366.51434535)(851.67895924,366.98526538)(851.56958427,367.2708556)
\curveto(851.46628568,367.5625222)(851.27487947,367.79646312)(850.99536564,367.97267836)
\curveto(850.71585182,368.1488936)(850.38772689,368.23700122)(850.01099086,368.23700122)
\curveto(849.33043545,368.23700122)(848.76533141,368.00913669)(848.31567873,367.55340762)
\curveto(847.86602605,367.10375494)(847.64119971,366.38066482)(847.64119971,365.38413726)
\lineto(847.64119971,359.76044058)
\lineto(846.00057507,359.76044058)
\lineto(846.00057507,366.0495017)
\curveto(846.00057507,366.77866821)(845.86689455,367.32554309)(845.5995335,367.69012634)
\curveto(845.33217244,368.05470959)(844.89467254,368.23700122)(844.28703378,368.23700122)
\curveto(843.82522833,368.23700122)(843.39684301,368.11547347)(843.00187782,367.87241797)
\curveto(842.61298901,367.62936246)(842.33043699,367.27389379)(842.15422175,366.80601195)
\curveto(841.97800651,366.33813011)(841.88989889,365.66365109)(841.88989889,364.7825749)
\lineto(841.88989889,359.76044058)
\closepath
}
}
{
\newrgbcolor{curcolor}{0 0 0}
\pscustom[linestyle=none,fillstyle=solid,fillcolor=curcolor]
{
\newpath
\moveto(862.42487506,362.8776274)
\lineto(864.12018719,362.66799203)
\curveto(863.85282613,361.67754086)(863.35760055,360.90887783)(862.63451043,360.36200295)
\curveto(861.91142031,359.81512807)(860.9878094,359.54169063)(859.86367771,359.54169063)
\curveto(858.44787941,359.54169063)(857.32374771,359.97615234)(856.49128261,360.84507576)
\curveto(855.66489391,361.72007557)(855.25169955,362.94446766)(855.25169955,364.51825204)
\curveto(855.25169955,366.1467239)(855.67097029,367.41061251)(856.50951178,368.30991787)
\curveto(857.34805326,369.20922323)(858.43572663,369.65887591)(859.77253189,369.65887591)
\curveto(861.06680244,369.65887591)(862.12409388,369.21833781)(862.9444062,368.33726161)
\curveto(863.76471851,367.45618542)(864.17487467,366.21660236)(864.17487467,364.61851243)
\curveto(864.17487467,364.52129023)(864.17183648,364.37545693)(864.16576009,364.18101253)
\lineto(856.94701168,364.18101253)
\curveto(857.00777556,363.11764471)(857.30855674,362.30340877)(857.84935523,361.73830473)
\curveto(858.39015372,361.17320069)(859.06463274,360.89064867)(859.87279229,360.89064867)
\curveto(860.47435465,360.89064867)(860.9878094,361.04863474)(861.41315653,361.3646069)
\curveto(861.83850366,361.68057905)(862.17574317,362.18491922)(862.42487506,362.8776274)
\closepath
\moveto(857.03815749,365.52997056)
\lineto(862.44310422,365.52997056)
\curveto(862.37018757,366.3442065)(862.16359039,366.95488345)(861.82331269,367.36200141)
\curveto(861.30074336,367.99394572)(860.62322615,368.30991787)(859.79076106,368.30991787)
\curveto(859.037289,368.30991787)(858.4023065,368.05774779)(857.88581356,367.55340762)
\curveto(857.375397,367.04906745)(857.09284498,366.37458843)(857.03815749,365.52997056)
\closepath
}
}
{
\newrgbcolor{curcolor}{0 0 0}
\pscustom[linestyle=none,fillstyle=solid,fillcolor=curcolor]
{
\newpath
\moveto(869.75925362,361.22788818)
\lineto(869.99623273,359.77866975)
\curveto(869.53442728,359.68144755)(869.12123292,359.63283644)(868.75664967,359.63283644)
\curveto(868.16116369,359.63283644)(867.69935824,359.72702045)(867.37123331,359.91538847)
\curveto(867.04310838,360.10375648)(866.81220565,360.34985018)(866.67852513,360.65366955)
\curveto(866.5448446,360.96356532)(866.47800434,361.61070059)(866.47800434,362.59507538)
\lineto(866.47800434,368.16408457)
\lineto(865.2748796,368.16408457)
\lineto(865.2748796,369.44012596)
\lineto(866.47800434,369.44012596)
\lineto(866.47800434,371.83726085)
\lineto(868.1095144,372.82163563)
\lineto(868.1095144,369.44012596)
\lineto(869.75925362,369.44012596)
\lineto(869.75925362,368.16408457)
\lineto(868.1095144,368.16408457)
\lineto(868.1095144,362.50392956)
\curveto(868.1095144,362.03604772)(868.13685814,361.73526654)(868.19154563,361.60158601)
\curveto(868.2523095,361.46790549)(868.34649351,361.3615687)(868.47409765,361.28257567)
\curveto(868.60777818,361.20358263)(868.79614619,361.16408611)(869.03920169,361.16408611)
\curveto(869.22149332,361.16408611)(869.46151063,361.18535346)(869.75925362,361.22788818)
\closepath
}
}
{
\newrgbcolor{curcolor}{0 0 0}
\pscustom[linestyle=none,fillstyle=solid,fillcolor=curcolor]
{
\newpath
\moveto(877.66427947,360.95445074)
\curveto(877.05664072,360.4379578)(876.47026932,360.07337454)(875.90516528,359.86070098)
\curveto(875.34613762,359.64802741)(874.74457525,359.54169063)(874.10047817,359.54169063)
\curveto(873.03711035,359.54169063)(872.21983622,359.7999371)(871.64865579,360.31643004)
\curveto(871.07747536,360.83899937)(870.79188515,361.50436381)(870.79188515,362.31252336)
\curveto(870.79188515,362.78648158)(870.89822193,363.2179051)(871.1108955,363.6067939)
\curveto(871.32964545,364.0017591)(871.61219747,364.31773125)(871.95855156,364.55471036)
\curveto(872.31098204,364.79168948)(872.70594723,364.97094291)(873.14344713,365.09247066)
\curveto(873.46549567,365.17754009)(873.95160668,365.25957132)(874.60178015,365.33856436)
\curveto(875.92643263,365.49655043)(876.90169283,365.68491845)(877.52756075,365.9036684)
\curveto(877.53363714,366.12849474)(877.53667533,366.27128985)(877.53667533,366.33205372)
\curveto(877.53667533,367.00045635)(877.38172745,367.47137639)(877.07183169,367.74481383)
\curveto(876.65256094,368.11547347)(876.02973122,368.30080329)(875.20334251,368.30080329)
\curveto(874.43164129,368.30080329)(873.86046086,368.16408457)(873.48980122,367.89064713)
\curveto(873.12521797,367.62328608)(872.85481872,367.14628965)(872.67860348,366.45965786)
\lineto(871.07443717,366.67840781)
\curveto(871.22027047,367.36503961)(871.46028778,367.91799087)(871.7944891,368.33726161)
\curveto(872.12869041,368.76260874)(872.61176322,369.08769548)(873.24370753,369.31252182)
\curveto(873.87565183,369.54342454)(874.60785653,369.65887591)(875.44032163,369.65887591)
\curveto(876.26671033,369.65887591)(876.93815116,369.56165371)(877.4546441,369.3672093)
\curveto(877.97113704,369.1727649)(878.35091127,368.92667121)(878.59396677,368.62892822)
\curveto(878.83702227,368.33726161)(879.00716112,367.96660197)(879.10438332,367.51694929)
\curveto(879.15907081,367.23743547)(879.18641455,366.7330953)(879.18641455,366.00392879)
\lineto(879.18641455,363.81642928)
\curveto(879.18641455,362.291256)(879.21983469,361.32511038)(879.28667495,360.91799241)
\curveto(879.3595916,360.51695083)(879.49934851,360.13110022)(879.70594569,359.76044058)
\lineto(877.9924044,359.76044058)
\curveto(877.82226555,360.10071829)(877.71289057,360.49872167)(877.66427947,360.95445074)
\closepath
\moveto(877.52756075,364.61851243)
\curveto(876.93207477,364.37545693)(876.0388458,364.16885975)(874.84787384,363.9987209)
\curveto(874.17339482,363.9014987)(873.6963984,363.79212372)(873.41688457,363.67059597)
\curveto(873.13737074,363.54906822)(872.92165899,363.36981479)(872.7697493,363.13283568)
\curveto(872.61783961,362.90193295)(872.54188476,362.64368648)(872.54188476,362.35809626)
\curveto(872.54188476,361.92059636)(872.70594723,361.55601311)(873.03407216,361.2643465)
\curveto(873.36827347,360.9726799)(873.85438448,360.8268466)(874.49240517,360.8268466)
\curveto(875.12434947,360.8268466)(875.68641532,360.96356532)(876.17860272,361.23700276)
\curveto(876.67079011,361.51651659)(877.03233517,361.89629081)(877.26323789,362.37632542)
\curveto(877.43945313,362.74698507)(877.52756075,363.29385995)(877.52756075,364.01695006)
\closepath
}
}
{
\newrgbcolor{curcolor}{0 0 0}
\pscustom[linestyle=none,fillstyle=solid,fillcolor=curcolor]
{
\newpath
\moveto(888.00646987,359.76044058)
\lineto(888.00646987,360.98179448)
\curveto(887.39275473,360.02172525)(886.49041117,359.54169063)(885.29943921,359.54169063)
\curveto(884.52773799,359.54169063)(883.81680065,359.7543642)(883.16662718,360.17971132)
\curveto(882.5225301,360.60505845)(882.02122813,361.19750624)(881.66272126,361.95705468)
\curveto(881.31029079,362.72267952)(881.13407555,363.60071752)(881.13407555,364.59116869)
\curveto(881.13407555,365.55731431)(881.29509982,366.43231412)(881.61714836,367.21616811)
\curveto(881.9391969,368.00609849)(882.42226971,368.61069905)(883.06636679,369.0299698)
\curveto(883.71046387,369.44924054)(884.43051579,369.65887591)(885.22652256,369.65887591)
\curveto(885.80985577,369.65887591)(886.3293869,369.53430996)(886.78511597,369.28517807)
\curveto(887.24084504,369.04212257)(887.61150468,368.72311222)(887.89709489,368.32814703)
\lineto(887.89709489,373.12241681)
\lineto(889.52860495,373.12241681)
\lineto(889.52860495,359.76044058)
\closepath
\moveto(882.82027309,364.59116869)
\curveto(882.82027309,363.35158563)(883.08155776,362.42493653)(883.60412709,361.81122138)
\curveto(884.12669642,361.19750624)(884.74344975,360.89064867)(885.4543871,360.89064867)
\curveto(886.17140083,360.89064867)(886.77903958,361.18231527)(887.27730336,361.76564848)
\curveto(887.78164353,362.35505807)(888.03381361,363.25132523)(888.03381361,364.45444997)
\curveto(888.03381361,365.77910245)(887.77860534,366.75132446)(887.26818878,367.37111599)
\curveto(886.75777223,367.99090752)(886.12886611,368.30080329)(885.38147045,368.30080329)
\curveto(884.65230394,368.30080329)(884.04162699,368.0030603)(883.5494396,367.40757432)
\curveto(883.06332859,366.81208834)(882.82027309,365.87328646)(882.82027309,364.59116869)
\closepath
}
}
{
\newrgbcolor{curcolor}{0 0 0}
\pscustom[linestyle=none,fillstyle=solid,fillcolor=curcolor]
{
\newpath
\moveto(898.42156929,360.95445074)
\curveto(897.81393053,360.4379578)(897.22755913,360.07337454)(896.66245509,359.86070098)
\curveto(896.10342743,359.64802741)(895.50186507,359.54169063)(894.85776799,359.54169063)
\curveto(893.79440016,359.54169063)(892.97712604,359.7999371)(892.40594561,360.31643004)
\curveto(891.83476518,360.83899937)(891.54917496,361.50436381)(891.54917496,362.31252336)
\curveto(891.54917496,362.78648158)(891.65551175,363.2179051)(891.86818531,363.6067939)
\curveto(892.08693526,364.0017591)(892.36948728,364.31773125)(892.71584137,364.55471036)
\curveto(893.06827185,364.79168948)(893.46323704,364.97094291)(893.90073695,365.09247066)
\curveto(894.22278549,365.17754009)(894.70889649,365.25957132)(895.35906996,365.33856436)
\curveto(896.68372245,365.49655043)(897.65898265,365.68491845)(898.28485057,365.9036684)
\curveto(898.29092695,366.12849474)(898.29396515,366.27128985)(898.29396515,366.33205372)
\curveto(898.29396515,367.00045635)(898.13901727,367.47137639)(897.8291215,367.74481383)
\curveto(897.40985076,368.11547347)(896.78702103,368.30080329)(895.96063233,368.30080329)
\curveto(895.18893111,368.30080329)(894.61775068,368.16408457)(894.24709104,367.89064713)
\curveto(893.88250778,367.62328608)(893.61210854,367.14628965)(893.4358933,366.45965786)
\lineto(891.83172699,366.67840781)
\curveto(891.97756029,367.36503961)(892.21757759,367.91799087)(892.55177891,368.33726161)
\curveto(892.88598023,368.76260874)(893.36905304,369.08769548)(894.00099734,369.31252182)
\curveto(894.63294165,369.54342454)(895.36514635,369.65887591)(896.19761144,369.65887591)
\curveto(897.02400015,369.65887591)(897.69544097,369.56165371)(898.21193392,369.3672093)
\curveto(898.72842686,369.1727649)(899.10820108,368.92667121)(899.35125658,368.62892822)
\curveto(899.59431208,368.33726161)(899.76445094,367.96660197)(899.86167314,367.51694929)
\curveto(899.91636062,367.23743547)(899.94370437,366.7330953)(899.94370437,366.00392879)
\lineto(899.94370437,363.81642928)
\curveto(899.94370437,362.291256)(899.9771245,361.32511038)(900.04396476,360.91799241)
\curveto(900.11688141,360.51695083)(900.25663833,360.13110022)(900.4632355,359.76044058)
\lineto(898.74969421,359.76044058)
\curveto(898.57955536,360.10071829)(898.47018039,360.49872167)(898.42156929,360.95445074)
\closepath
\moveto(898.28485057,364.61851243)
\curveto(897.68936459,364.37545693)(896.79613562,364.16885975)(895.60516366,363.9987209)
\curveto(894.93068464,363.9014987)(894.45368821,363.79212372)(894.17417439,363.67059597)
\curveto(893.89466056,363.54906822)(893.6789488,363.36981479)(893.52703911,363.13283568)
\curveto(893.37512942,362.90193295)(893.29917458,362.64368648)(893.29917458,362.35809626)
\curveto(893.29917458,361.92059636)(893.46323704,361.55601311)(893.79136197,361.2643465)
\curveto(894.12556329,360.9726799)(894.61167429,360.8268466)(895.24969498,360.8268466)
\curveto(895.88163929,360.8268466)(896.44370514,360.96356532)(896.93589253,361.23700276)
\curveto(897.42807992,361.51651659)(897.78962498,361.89629081)(898.02052771,362.37632542)
\curveto(898.19674295,362.74698507)(898.28485057,363.29385995)(898.28485057,364.01695006)
\closepath
}
}
{
\newrgbcolor{curcolor}{0 0 0}
\pscustom[linestyle=none,fillstyle=solid,fillcolor=curcolor]
{
\newpath
\moveto(906.06583598,361.22788818)
\lineto(906.30281509,359.77866975)
\curveto(905.84100964,359.68144755)(905.42781529,359.63283644)(905.06323203,359.63283644)
\curveto(904.46774605,359.63283644)(904.0059406,359.72702045)(903.67781567,359.91538847)
\curveto(903.34969074,360.10375648)(903.11878802,360.34985018)(902.98510749,360.65366955)
\curveto(902.85142696,360.96356532)(902.7845867,361.61070059)(902.7845867,362.59507538)
\lineto(902.7845867,368.16408457)
\lineto(901.58146197,368.16408457)
\lineto(901.58146197,369.44012596)
\lineto(902.7845867,369.44012596)
\lineto(902.7845867,371.83726085)
\lineto(904.41609676,372.82163563)
\lineto(904.41609676,369.44012596)
\lineto(906.06583598,369.44012596)
\lineto(906.06583598,368.16408457)
\lineto(904.41609676,368.16408457)
\lineto(904.41609676,362.50392956)
\curveto(904.41609676,362.03604772)(904.4434405,361.73526654)(904.49812799,361.60158601)
\curveto(904.55889187,361.46790549)(904.65307587,361.3615687)(904.78068001,361.28257567)
\curveto(904.91436054,361.20358263)(905.10272855,361.16408611)(905.34578406,361.16408611)
\curveto(905.52807568,361.16408611)(905.76809299,361.18535346)(906.06583598,361.22788818)
\closepath
}
}
{
\newrgbcolor{curcolor}{0 0 0}
\pscustom[linestyle=none,fillstyle=solid,fillcolor=curcolor]
{
\newpath
\moveto(913.97086946,360.95445074)
\curveto(913.36323071,360.4379578)(912.77685931,360.07337454)(912.21175527,359.86070098)
\curveto(911.65272761,359.64802741)(911.05116524,359.54169063)(910.40706816,359.54169063)
\curveto(909.34370034,359.54169063)(908.52642622,359.7999371)(907.95524579,360.31643004)
\curveto(907.38406536,360.83899937)(907.09847514,361.50436381)(907.09847514,362.31252336)
\curveto(907.09847514,362.78648158)(907.20481192,363.2179051)(907.41748549,363.6067939)
\curveto(907.63623544,364.0017591)(907.91878746,364.31773125)(908.26514155,364.55471036)
\curveto(908.61757203,364.79168948)(909.01253722,364.97094291)(909.45003712,365.09247066)
\curveto(909.77208566,365.17754009)(910.25819667,365.25957132)(910.90837014,365.33856436)
\curveto(912.23302262,365.49655043)(913.20828283,365.68491845)(913.83415074,365.9036684)
\curveto(913.84022713,366.12849474)(913.84326533,366.27128985)(913.84326533,366.33205372)
\curveto(913.84326533,367.00045635)(913.68831744,367.47137639)(913.37842168,367.74481383)
\curveto(912.95915094,368.11547347)(912.33632121,368.30080329)(911.50993251,368.30080329)
\curveto(910.73823129,368.30080329)(910.16705086,368.16408457)(909.79639122,367.89064713)
\curveto(909.43180796,367.62328608)(909.16140872,367.14628965)(908.98519348,366.45965786)
\lineto(907.38102716,366.67840781)
\curveto(907.52686046,367.36503961)(907.76687777,367.91799087)(908.10107909,368.33726161)
\curveto(908.4352804,368.76260874)(908.91835321,369.08769548)(909.55029752,369.31252182)
\curveto(910.18224182,369.54342454)(910.91444652,369.65887591)(911.74691162,369.65887591)
\curveto(912.57330033,369.65887591)(913.24474115,369.56165371)(913.76123409,369.3672093)
\curveto(914.27772704,369.1727649)(914.65750126,368.92667121)(914.90055676,368.62892822)
\curveto(915.14361226,368.33726161)(915.31375111,367.96660197)(915.41097331,367.51694929)
\curveto(915.4656608,367.23743547)(915.49300455,366.7330953)(915.49300455,366.00392879)
\lineto(915.49300455,363.81642928)
\curveto(915.49300455,362.291256)(915.52642468,361.32511038)(915.59326494,360.91799241)
\curveto(915.66618159,360.51695083)(915.80593851,360.13110022)(916.01253568,359.76044058)
\lineto(914.29899439,359.76044058)
\curveto(914.12885554,360.10071829)(914.01948056,360.49872167)(913.97086946,360.95445074)
\closepath
\moveto(913.83415074,364.61851243)
\curveto(913.23866476,364.37545693)(912.34543579,364.16885975)(911.15446383,363.9987209)
\curveto(910.47998481,363.9014987)(910.00298839,363.79212372)(909.72347456,363.67059597)
\curveto(909.44396074,363.54906822)(909.22824898,363.36981479)(909.07633929,363.13283568)
\curveto(908.9244296,362.90193295)(908.84847476,362.64368648)(908.84847476,362.35809626)
\curveto(908.84847476,361.92059636)(909.01253722,361.55601311)(909.34066215,361.2643465)
\curveto(909.67486346,360.9726799)(910.16097447,360.8268466)(910.79899516,360.8268466)
\curveto(911.43093947,360.8268466)(911.99300532,360.96356532)(912.48519271,361.23700276)
\curveto(912.9773801,361.51651659)(913.33892516,361.89629081)(913.56982789,362.37632542)
\curveto(913.74604312,362.74698507)(913.83415074,363.29385995)(913.83415074,364.01695006)
\closepath
}
}
{
\newrgbcolor{curcolor}{0 0 0}
\pscustom[linestyle=none,fillstyle=solid,fillcolor=curcolor]
{
\newpath
\moveto(919.10864155,355.83205603)
\lineto(917.93286056,355.83205603)
\curveto(919.74970044,358.74872206)(920.65812038,361.66842627)(920.65812038,364.59116869)
\curveto(920.65812038,365.73352955)(920.52747804,366.86677583)(920.26619338,367.99090752)
\curveto(920.0595962,368.90236566)(919.77096779,369.77736546)(919.40030815,370.61590695)
\curveto(919.16332904,371.16278183)(918.67417984,372.07423996)(917.93286056,373.35028135)
\lineto(919.10864155,373.35028135)
\curveto(920.25100241,371.82510807)(921.09562028,370.29385841)(921.64249516,368.75653236)
\curveto(922.110377,367.43187987)(922.34431792,366.04646351)(922.34431792,364.60028327)
\curveto(922.34431792,362.95965863)(922.02834577,361.37372148)(921.39640146,359.84247182)
\curveto(920.77053355,358.31122215)(920.00794691,356.97441689)(919.10864155,355.83205603)
\closepath
}
}
{
\newrgbcolor{curcolor}{0 0 0}
\pscustom[linestyle=none,fillstyle=solid,fillcolor=curcolor,opacity=0]
{
\newpath
\moveto(689.69545425,399.44415644)
\curveto(701.46708396,399.44415644)(707.35289881,394.00322258)(713.23871366,388.56228872)
\curveto(719.12452852,383.12135485)(725.01034337,377.68042099)(736.78197308,377.68042099)
}
}
{
\newrgbcolor{curcolor}{0.49803922 0.49803922 0.49803922}
\pscustom[linewidth=2.99999393,linecolor=curcolor]
{
\newpath
\moveto(689.69545425,399.44415644)
\curveto(701.46708396,399.44415644)(707.35289881,394.00322258)(713.23871366,388.56228872)
\curveto(719.12452852,383.12135485)(725.01034337,377.68042099)(736.78197308,377.68042099)
}
}
{
\newrgbcolor{curcolor}{0 0 0}
\pscustom[linestyle=none,fillstyle=solid,fillcolor=curcolor,opacity=0]
{
\newpath
\moveto(689.69545425,365.0505253)
\curveto(701.46708396,365.0505253)(707.35289881,368.20799923)(713.23871366,371.36547315)
\curveto(719.12452852,374.52294707)(725.01034337,377.68042099)(736.78197308,377.68042099)
}
}
{
\newrgbcolor{curcolor}{0.49803922 0.49803922 0.49803922}
\pscustom[linewidth=2.99999393,linecolor=curcolor]
{
\newpath
\moveto(689.69545425,365.0505253)
\curveto(701.46708396,365.0505253)(707.35289881,368.20799923)(713.23871366,371.36547315)
\curveto(719.12452852,374.52294707)(725.01034337,377.68042099)(736.78197308,377.68042099)
}
}
\end{pspicture}
}
    \captionsetup{width=0.75\linewidth}
    \caption{A ZFS block pointer is a complex structure powering many of ZFS's most useful features\cite{ahrens_openzfs_basics}.}
\label{fig:BlockPointer}
\end{figure}

The most important feature of block pointers is that they contain a 256-bit checksum of the data they point to.
These checksums protect against failures of the initial write and allow detecting any corruption that might occur later
when reading the data back from the disk.
They also allow for reparing data on one side of a mirror, if the other side is uncorrupted, 
as its correct data can then be written from the known-good side of the mirror and the corrupted data on the first side 
can be deleted.
Because block pointers are addresses, but also contain a checksum, they form a hash tree, also known as a Merkel tree, of addresses and checksums
(Figure \ref{fig:HashTree}).
Each checksum is also validating the checksum of the blocks below it in the tree as each block contains the checksum of
its child blocks.
Thus each layer of the tree can validate all of the data below it.

\begin{figure}[H]
    \centering
    \resizebox{0.75\linewidth}{!}{%LaTeX with PSTricks extensions
%%Creator: Inkscape 1.0.2-2 (e86c870879, 2021-01-15)
%%Please note this file requires PSTricks extensions
\psset{xunit=.5pt,yunit=.5pt,runit=.5pt}
\begin{pspicture}(1801.80041504,912.99963379)
{
\newrgbcolor{curcolor}{1 1 1}
\pscustom[linestyle=none,fillstyle=solid,fillcolor=curcolor]
{
\newpath
\moveto(1164.08950255,566.25774724)
\lineto(1570.94091247,566.25774724)
\lineto(1570.94091247,346.7418781)
\lineto(1164.08950255,346.7418781)
\closepath
}
}
{
\newrgbcolor{curcolor}{0 0 0}
\pscustom[linewidth=2,linecolor=curcolor]
{
\newpath
\moveto(1164.08950255,566.25774724)
\lineto(1570.94091247,566.25774724)
\lineto(1570.94091247,346.7418781)
\lineto(1164.08950255,346.7418781)
\closepath
}
}
{
\newrgbcolor{curcolor}{1 0.93333334 0.66666669}
\pscustom[linestyle=none,fillstyle=solid,fillcolor=curcolor]
{
\newpath
\moveto(1186.22475646,443.28289372)
\lineto(1544.80565856,443.28289372)
\lineto(1544.80565856,367.42924077)
\lineto(1186.22475646,367.42924077)
\closepath
}
}
{
\newrgbcolor{curcolor}{0 0 0}
\pscustom[linestyle=none,fillstyle=solid,fillcolor=curcolor]
{
\newpath
\moveto(1338.86281798,521.12853581)
\curveto(1338.86281798,519.68322331)(1338.58938048,518.40718164)(1338.04250548,517.30041081)
\curveto(1337.49563048,516.19363998)(1336.7599534,515.28218164)(1335.83547423,514.56603581)
\curveto(1334.74172423,513.70666081)(1333.53729715,513.09468164)(1332.22219298,512.73009831)
\curveto(1330.92010965,512.36551498)(1329.2599534,512.18322331)(1327.24172423,512.18322331)
\lineto(1316.92922423,512.18322331)
\lineto(1316.92922423,541.26525456)
\lineto(1325.54250548,541.26525456)
\curveto(1327.66490132,541.26525456)(1329.25344298,541.18712956)(1330.30813048,541.03087956)
\curveto(1331.36281798,540.87462956)(1332.37193257,540.54910873)(1333.33547423,540.05431706)
\curveto(1334.40318257,539.49442123)(1335.17792215,538.77176498)(1335.65969298,537.88634831)
\curveto(1336.14146382,537.01395248)(1336.38234923,535.96577539)(1336.38234923,534.74181706)
\curveto(1336.38234923,533.36160873)(1336.03078673,532.18322331)(1335.32766173,531.20666081)
\curveto(1334.62453673,530.24311914)(1333.68703673,529.46837956)(1332.51516173,528.88244206)
\lineto(1332.51516173,528.72619206)
\curveto(1334.48130757,528.32254623)(1336.03078673,527.45666081)(1337.16359923,526.12853581)
\curveto(1338.29641173,524.81343164)(1338.86281798,523.14676498)(1338.86281798,521.12853581)
\closepath
\moveto(1332.35891173,534.23400456)
\curveto(1332.35891173,534.93712956)(1332.24172423,535.52957748)(1332.00734923,536.01134831)
\curveto(1331.77297423,536.49311914)(1331.39537007,536.88374414)(1330.87453673,537.18322331)
\curveto(1330.26255757,537.53478581)(1329.52037007,537.74962956)(1328.64797423,537.82775456)
\curveto(1327.7755784,537.91890039)(1326.69484923,537.96447331)(1325.40578673,537.96447331)
\lineto(1320.79641173,537.96447331)
\lineto(1320.79641173,529.56603581)
\lineto(1325.79641173,529.56603581)
\curveto(1327.00734923,529.56603581)(1327.9708909,529.62462956)(1328.68703673,529.74181706)
\curveto(1329.40318257,529.87202539)(1330.06724507,530.13244206)(1330.67922423,530.52306706)
\curveto(1331.2912034,530.91369206)(1331.7208909,531.41499414)(1331.96828673,532.02697331)
\curveto(1332.2287034,532.65197331)(1332.35891173,533.38765039)(1332.35891173,534.23400456)
\closepath
\moveto(1334.83938048,520.97228581)
\curveto(1334.83938048,522.14416081)(1334.66359923,523.07515039)(1334.31203673,523.76525456)
\curveto(1333.96047423,524.45535873)(1333.3224534,525.04129623)(1332.39797423,525.52306706)
\curveto(1331.77297423,525.84858789)(1331.01125548,526.05692123)(1330.11281798,526.14806706)
\curveto(1329.22740132,526.25223373)(1328.14667215,526.30431706)(1326.87063048,526.30431706)
\lineto(1320.79641173,526.30431706)
\lineto(1320.79641173,515.48400456)
\lineto(1325.91359923,515.48400456)
\curveto(1327.60630757,515.48400456)(1328.99302632,515.56863998)(1330.07375548,515.73791081)
\curveto(1331.15448465,515.92020248)(1332.03990132,516.24572331)(1332.73000548,516.71447331)
\curveto(1333.45917215,517.22228581)(1333.99302632,517.80171289)(1334.33156798,518.45275456)
\curveto(1334.67010965,519.10379623)(1334.83938048,519.94363998)(1334.83938048,520.97228581)
\closepath
}
}
{
\newrgbcolor{curcolor}{0 0 0}
\pscustom[linestyle=none,fillstyle=solid,fillcolor=curcolor]
{
\newpath
\moveto(1347.76906798,512.18322331)
\lineto(1344.09719298,512.18322331)
\lineto(1344.09719298,542.57384831)
\lineto(1347.76906798,542.57384831)
\closepath
}
}
{
\newrgbcolor{curcolor}{0 0 0}
\pscustom[linestyle=none,fillstyle=solid,fillcolor=curcolor]
{
\newpath
\moveto(1373.62844298,523.08166081)
\curveto(1373.62844298,519.52697331)(1372.71698465,516.72098373)(1370.89406798,514.66369206)
\curveto(1369.07115132,512.60640039)(1366.62974507,511.57775456)(1363.56984923,511.57775456)
\curveto(1360.48391173,511.57775456)(1358.02948465,512.60640039)(1356.20656798,514.66369206)
\curveto(1354.39667215,516.72098373)(1353.49172423,519.52697331)(1353.49172423,523.08166081)
\curveto(1353.49172423,526.63634831)(1354.39667215,529.44233789)(1356.20656798,531.49962956)
\curveto(1358.02948465,533.56994206)(1360.48391173,534.60509831)(1363.56984923,534.60509831)
\curveto(1366.62974507,534.60509831)(1369.07115132,533.56994206)(1370.89406798,531.49962956)
\curveto(1372.71698465,529.44233789)(1373.62844298,526.63634831)(1373.62844298,523.08166081)
\closepath
\moveto(1369.83938048,523.08166081)
\curveto(1369.83938048,525.90718164)(1369.28599507,528.00353581)(1368.17922423,529.37072331)
\curveto(1367.0724534,530.75093164)(1365.53599507,531.44103581)(1363.56984923,531.44103581)
\curveto(1361.57766173,531.44103581)(1360.02818257,530.75093164)(1358.92141173,529.37072331)
\curveto(1357.82766173,528.00353581)(1357.28078673,525.90718164)(1357.28078673,523.08166081)
\curveto(1357.28078673,520.34728581)(1357.83417215,518.27046289)(1358.94094298,516.85119206)
\curveto(1360.04771382,515.44494206)(1361.59068257,514.74181706)(1363.56984923,514.74181706)
\curveto(1365.52297423,514.74181706)(1367.05292215,515.43843164)(1368.15969298,516.83166081)
\curveto(1369.27948465,518.23791081)(1369.83938048,520.32124414)(1369.83938048,523.08166081)
\closepath
}
}
{
\newrgbcolor{curcolor}{0 0 0}
\pscustom[linestyle=none,fillstyle=solid,fillcolor=curcolor]
{
\newpath
\moveto(1395.44484923,513.55041081)
\curveto(1394.2208909,512.96447331)(1393.05552632,512.50874414)(1391.94875548,512.18322331)
\curveto(1390.85500548,511.85770248)(1389.6896409,511.69494206)(1388.45266173,511.69494206)
\curveto(1386.8771409,511.69494206)(1385.4318284,511.92280664)(1384.11672423,512.37853581)
\curveto(1382.80162007,512.84728581)(1381.67531798,513.55041081)(1380.73781798,514.48791081)
\curveto(1379.78729715,515.42541081)(1379.05162007,516.61030664)(1378.53078673,518.04259831)
\curveto(1378.0099534,519.47488998)(1377.74953673,521.14806706)(1377.74953673,523.06212956)
\curveto(1377.74953673,526.62983789)(1378.72609923,529.42931706)(1380.67922423,531.46056706)
\curveto(1382.64537007,533.49181706)(1385.2365159,534.50744206)(1388.45266173,534.50744206)
\curveto(1389.70266173,534.50744206)(1390.92662007,534.33166081)(1392.12453673,533.98009831)
\curveto(1393.33547423,533.62853581)(1394.44224507,533.19884831)(1395.44484923,532.69103581)
\lineto(1395.44484923,528.60900456)
\lineto(1395.24953673,528.60900456)
\curveto(1394.12974507,529.48140039)(1392.9708909,530.15197331)(1391.77297423,530.62072331)
\curveto(1390.5880784,531.08947331)(1389.42922423,531.32384831)(1388.29641173,531.32384831)
\curveto(1386.2130784,531.32384831)(1384.56594298,530.62072331)(1383.35500548,529.21447331)
\curveto(1382.15708882,527.82124414)(1381.55813048,525.77046289)(1381.55813048,523.06212956)
\curveto(1381.55813048,520.43192123)(1382.14406798,518.40718164)(1383.31594298,516.98791081)
\curveto(1384.50083882,515.58166081)(1386.16099507,514.87853581)(1388.29641173,514.87853581)
\curveto(1389.03859923,514.87853581)(1389.79380757,514.97619206)(1390.56203673,515.17150456)
\curveto(1391.3302659,515.36681706)(1392.02037007,515.62072331)(1392.63234923,515.93322331)
\curveto(1393.1662034,516.20666081)(1393.66750548,516.49311914)(1394.13625548,516.79259831)
\curveto(1394.60500548,517.10509831)(1394.97609923,517.37202539)(1395.24953673,517.59337956)
\lineto(1395.44484923,517.59337956)
\closepath
}
}
{
\newrgbcolor{curcolor}{0 0 0}
\pscustom[linestyle=none,fillstyle=solid,fillcolor=curcolor]
{
\newpath
\moveto(1419.95656798,512.18322331)
\lineto(1415.11281798,512.18322331)
\lineto(1406.36281798,521.73400456)
\lineto(1403.98000548,519.46837956)
\lineto(1403.98000548,512.18322331)
\lineto(1400.30813048,512.18322331)
\lineto(1400.30813048,542.57384831)
\lineto(1403.98000548,542.57384831)
\lineto(1403.98000548,523.08166081)
\lineto(1414.58547423,533.99962956)
\lineto(1419.21438048,533.99962956)
\lineto(1409.07766173,523.92150456)
\closepath
}
}
{
\newrgbcolor{curcolor}{0 0 0}
\pscustom[linestyle=none,fillstyle=solid,fillcolor=curcolor]
{
\newpath
\moveto(1376.73391173,462.18322331)
\lineto(1357.04641173,462.18322331)
\lineto(1357.04641173,466.26525456)
\lineto(1361.14797423,469.78087956)
\curveto(1362.52818257,470.95275456)(1363.81073465,472.11811914)(1364.99563048,473.27697331)
\curveto(1367.49563048,475.69884831)(1369.20787007,477.61942123)(1370.13234923,479.03869206)
\curveto(1371.0568284,480.47098373)(1371.51906798,482.01395248)(1371.51906798,483.66759831)
\curveto(1371.51906798,485.17801498)(1371.0177659,486.35640039)(1370.01516173,487.20275456)
\curveto(1369.0255784,488.06212956)(1367.63885965,488.49181706)(1365.85500548,488.49181706)
\curveto(1364.67010965,488.49181706)(1363.38755757,488.28348373)(1362.00734923,487.86681706)
\curveto(1360.6271409,487.45015039)(1359.27948465,486.81212956)(1357.96438048,485.95275456)
\lineto(1357.76906798,485.95275456)
\lineto(1357.76906798,490.05431706)
\curveto(1358.69354715,490.51004623)(1359.9240159,490.92671289)(1361.46047423,491.30431706)
\curveto(1363.0099534,491.68192123)(1364.50734923,491.87072331)(1365.95266173,491.87072331)
\curveto(1368.93443257,491.87072331)(1371.27167215,491.14806706)(1372.96438048,489.70275456)
\curveto(1374.65708882,488.27046289)(1375.50344298,486.32384831)(1375.50344298,483.86291081)
\curveto(1375.50344298,482.75613998)(1375.36021382,481.72098373)(1375.07375548,480.75744206)
\curveto(1374.80031798,479.80692123)(1374.39016173,478.90197331)(1373.84328673,478.04259831)
\curveto(1373.33547423,477.23530664)(1372.7365159,476.44103581)(1372.04641173,475.65978581)
\curveto(1371.3693284,474.87853581)(1370.54250548,474.01265039)(1369.56594298,473.06212956)
\curveto(1368.17271382,471.69494206)(1366.73391173,470.36681706)(1365.24953673,469.07775456)
\curveto(1363.76516173,467.80171289)(1362.37844298,466.61681706)(1361.08938048,465.52306706)
\lineto(1376.73391173,465.52306706)
\closepath
}
}
{
\newrgbcolor{curcolor}{0 0 0}
\pscustom[linestyle=none,fillstyle=solid,fillcolor=curcolor]
{
\newpath
\moveto(1260.9214589,422.72611388)
\curveto(1260.9214589,421.26778054)(1260.71312557,419.95528054)(1260.2964589,418.78861388)
\curveto(1259.8902089,417.62194721)(1259.33812557,416.64278054)(1258.6402089,415.85111388)
\curveto(1257.90062557,415.02819721)(1257.08812557,414.40840554)(1256.2027089,413.99173888)
\curveto(1255.31729224,413.58548888)(1254.3433339,413.38236388)(1253.2808339,413.38236388)
\curveto(1252.29125057,413.38236388)(1251.42666724,413.50215554)(1250.6870839,413.74173888)
\curveto(1249.94750057,413.97090554)(1249.2183339,414.28340554)(1248.4995839,414.67923888)
\lineto(1248.3120839,413.86673888)
\lineto(1245.5620839,413.86673888)
\lineto(1245.5620839,438.17923888)
\lineto(1248.4995839,438.17923888)
\lineto(1248.4995839,429.49173888)
\curveto(1249.32250057,430.16882221)(1250.19750057,430.72090554)(1251.1245839,431.14798888)
\curveto(1252.05166724,431.58548888)(1253.0933339,431.80423888)(1254.2495839,431.80423888)
\curveto(1256.3120839,431.80423888)(1257.9370839,431.01257221)(1259.1245839,429.42923888)
\curveto(1260.32250057,427.84590554)(1260.9214589,425.61153054)(1260.9214589,422.72611388)
\closepath
\moveto(1257.8902089,422.64798888)
\curveto(1257.8902089,424.73132221)(1257.5464589,426.30944721)(1256.8589589,427.38236388)
\curveto(1256.1714589,428.46569721)(1255.0620839,429.00736388)(1253.5308339,429.00736388)
\curveto(1252.67666724,429.00736388)(1251.8120839,428.81986388)(1250.9370839,428.44486388)
\curveto(1250.0620839,428.08028054)(1249.2495839,427.60632221)(1248.4995839,427.02298888)
\lineto(1248.4995839,417.02298888)
\curveto(1249.33291724,416.64798888)(1250.0464589,416.38757221)(1250.6402089,416.24173888)
\curveto(1251.24437557,416.09590554)(1251.92666724,416.02298888)(1252.6870839,416.02298888)
\curveto(1254.3120839,416.02298888)(1255.58291724,416.55423888)(1256.4995839,417.61673888)
\curveto(1257.42666724,418.68965554)(1257.8902089,420.36673888)(1257.8902089,422.64798888)
\closepath
}
}
{
\newrgbcolor{curcolor}{0 0 0}
\pscustom[linestyle=none,fillstyle=solid,fillcolor=curcolor]
{
\newpath
\moveto(1268.4683339,413.86673888)
\lineto(1265.5308339,413.86673888)
\lineto(1265.5308339,438.17923888)
\lineto(1268.4683339,438.17923888)
\closepath
}
}
{
\newrgbcolor{curcolor}{0 0 0}
\pscustom[linestyle=none,fillstyle=solid,fillcolor=curcolor]
{
\newpath
\moveto(1289.1558339,422.58548888)
\curveto(1289.1558339,419.74173888)(1288.42666724,417.49694721)(1286.9683339,415.85111388)
\curveto(1285.51000057,414.20528054)(1283.55687557,413.38236388)(1281.1089589,413.38236388)
\curveto(1278.6402089,413.38236388)(1276.67666724,414.20528054)(1275.2183339,415.85111388)
\curveto(1273.77041724,417.49694721)(1273.0464589,419.74173888)(1273.0464589,422.58548888)
\curveto(1273.0464589,425.42923888)(1273.77041724,427.67403054)(1275.2183339,429.31986388)
\curveto(1276.67666724,430.97611388)(1278.6402089,431.80423888)(1281.1089589,431.80423888)
\curveto(1283.55687557,431.80423888)(1285.51000057,430.97611388)(1286.9683339,429.31986388)
\curveto(1288.42666724,427.67403054)(1289.1558339,425.42923888)(1289.1558339,422.58548888)
\closepath
\moveto(1286.1245839,422.58548888)
\curveto(1286.1245839,424.84590554)(1285.68187557,426.52298888)(1284.7964589,427.61673888)
\curveto(1283.91104224,428.72090554)(1282.68187557,429.27298888)(1281.1089589,429.27298888)
\curveto(1279.5152089,429.27298888)(1278.27562557,428.72090554)(1277.3902089,427.61673888)
\curveto(1276.5152089,426.52298888)(1276.0777089,424.84590554)(1276.0777089,422.58548888)
\curveto(1276.0777089,420.39798888)(1276.52041724,418.73653054)(1277.4058339,417.60111388)
\curveto(1278.29125057,416.47611388)(1279.52562557,415.91361388)(1281.1089589,415.91361388)
\curveto(1282.6714589,415.91361388)(1283.89541724,416.47090554)(1284.7808339,417.58548888)
\curveto(1285.67666724,418.71048888)(1286.1245839,420.37715554)(1286.1245839,422.58548888)
\closepath
}
}
{
\newrgbcolor{curcolor}{0 0 0}
\pscustom[linestyle=none,fillstyle=solid,fillcolor=curcolor]
{
\newpath
\moveto(1306.6089589,414.96048888)
\curveto(1305.62979224,414.49173888)(1304.69750057,414.12715554)(1303.8120839,413.86673888)
\curveto(1302.9370839,413.60632221)(1302.00479224,413.47611388)(1301.0152089,413.47611388)
\curveto(1299.75479224,413.47611388)(1298.59854224,413.65840554)(1297.5464589,414.02298888)
\curveto(1296.49437557,414.39798888)(1295.5933339,414.96048888)(1294.8433339,415.71048888)
\curveto(1294.08291724,416.46048888)(1293.49437557,417.40840554)(1293.0777089,418.55423888)
\curveto(1292.66104224,419.70007221)(1292.4527089,421.03861388)(1292.4527089,422.56986388)
\curveto(1292.4527089,425.42403054)(1293.2339589,427.66361388)(1294.7964589,429.28861388)
\curveto(1296.36937557,430.91361388)(1298.44229224,431.72611388)(1301.0152089,431.72611388)
\curveto(1302.0152089,431.72611388)(1302.99437557,431.58548888)(1303.9527089,431.30423888)
\curveto(1304.9214589,431.02298888)(1305.80687557,430.67923888)(1306.6089589,430.27298888)
\lineto(1306.6089589,427.00736388)
\lineto(1306.4527089,427.00736388)
\curveto(1305.55687557,427.70528054)(1304.62979224,428.24173888)(1303.6714589,428.61673888)
\curveto(1302.72354224,428.99173888)(1301.7964589,429.17923888)(1300.8902089,429.17923888)
\curveto(1299.22354224,429.17923888)(1297.9058339,428.61673888)(1296.9370839,427.49173888)
\curveto(1295.97875057,426.37715554)(1295.4995839,424.73653054)(1295.4995839,422.56986388)
\curveto(1295.4995839,420.46569721)(1295.9683339,418.84590554)(1296.9058339,417.71048888)
\curveto(1297.85375057,416.58548888)(1299.18187557,416.02298888)(1300.8902089,416.02298888)
\curveto(1301.4839589,416.02298888)(1302.08812557,416.10111388)(1302.7027089,416.25736388)
\curveto(1303.31729224,416.41361388)(1303.86937557,416.61673888)(1304.3589589,416.86673888)
\curveto(1304.78604224,417.08548888)(1305.1870839,417.31465554)(1305.5620839,417.55423888)
\curveto(1305.9370839,417.80423888)(1306.2339589,418.01778054)(1306.4527089,418.19486388)
\lineto(1306.6089589,418.19486388)
\closepath
}
}
{
\newrgbcolor{curcolor}{0 0 0}
\pscustom[linestyle=none,fillstyle=solid,fillcolor=curcolor]
{
\newpath
\moveto(1326.2183339,413.86673888)
\lineto(1322.3433339,413.86673888)
\lineto(1315.3433339,421.50736388)
\lineto(1313.4370839,419.69486388)
\lineto(1313.4370839,413.86673888)
\lineto(1310.4995839,413.86673888)
\lineto(1310.4995839,438.17923888)
\lineto(1313.4370839,438.17923888)
\lineto(1313.4370839,422.58548888)
\lineto(1321.9214589,431.31986388)
\lineto(1325.6245839,431.31986388)
\lineto(1317.5152089,423.25736388)
\closepath
}
}
{
\newrgbcolor{curcolor}{0 0 0}
\pscustom[linestyle=none,fillstyle=solid,fillcolor=curcolor]
{
\newpath
\moveto(1344.6714589,422.80423888)
\curveto(1344.6714589,421.38757221)(1344.4683339,420.09069721)(1344.0620839,418.91361388)
\curveto(1343.6558339,417.74694721)(1343.08291724,416.75736388)(1342.3433339,415.94486388)
\curveto(1341.6558339,415.17403054)(1340.8433339,414.57507221)(1339.9058339,414.14798888)
\curveto(1338.97875057,413.73132221)(1337.99437557,413.52298888)(1336.9527089,413.52298888)
\curveto(1336.0464589,413.52298888)(1335.22354224,413.62194721)(1334.4839589,413.81986388)
\curveto(1333.75479224,414.01778054)(1333.01000057,414.32507221)(1332.2495839,414.74173888)
\lineto(1332.2495839,407.42923888)
\lineto(1329.3120839,407.42923888)
\lineto(1329.3120839,431.31986388)
\lineto(1332.2495839,431.31986388)
\lineto(1332.2495839,429.49173888)
\curveto(1333.0308339,430.14798888)(1333.9058339,430.69486388)(1334.8745839,431.13236388)
\curveto(1335.85375057,431.58028054)(1336.89541724,431.80423888)(1337.9995839,431.80423888)
\curveto(1340.10375057,431.80423888)(1341.73916724,431.00736388)(1342.9058339,429.41361388)
\curveto(1344.08291724,427.83028054)(1344.6714589,425.62715554)(1344.6714589,422.80423888)
\closepath
\moveto(1341.6402089,422.72611388)
\curveto(1341.6402089,424.83028054)(1341.2808339,426.40319721)(1340.5620839,427.44486388)
\curveto(1339.8433339,428.48653054)(1338.73916724,429.00736388)(1337.2495839,429.00736388)
\curveto(1336.4058339,429.00736388)(1335.55687557,428.82507221)(1334.7027089,428.46048888)
\curveto(1333.84854224,428.09590554)(1333.0308339,427.61673888)(1332.2495839,427.02298888)
\lineto(1332.2495839,417.13236388)
\curveto(1333.08291724,416.75736388)(1333.7964589,416.50215554)(1334.3902089,416.36673888)
\curveto(1334.99437557,416.23132221)(1335.67666724,416.16361388)(1336.4370839,416.16361388)
\curveto(1338.07250057,416.16361388)(1339.34854224,416.71569721)(1340.2652089,417.81986388)
\curveto(1341.18187557,418.92403054)(1341.6402089,420.55944721)(1341.6402089,422.72611388)
\closepath
}
}
{
\newrgbcolor{curcolor}{0 0 0}
\pscustom[linestyle=none,fillstyle=solid,fillcolor=curcolor]
{
\newpath
\moveto(1364.1245839,422.58548888)
\curveto(1364.1245839,419.74173888)(1363.39541724,417.49694721)(1361.9370839,415.85111388)
\curveto(1360.47875057,414.20528054)(1358.52562557,413.38236388)(1356.0777089,413.38236388)
\curveto(1353.6089589,413.38236388)(1351.64541724,414.20528054)(1350.1870839,415.85111388)
\curveto(1348.73916724,417.49694721)(1348.0152089,419.74173888)(1348.0152089,422.58548888)
\curveto(1348.0152089,425.42923888)(1348.73916724,427.67403054)(1350.1870839,429.31986388)
\curveto(1351.64541724,430.97611388)(1353.6089589,431.80423888)(1356.0777089,431.80423888)
\curveto(1358.52562557,431.80423888)(1360.47875057,430.97611388)(1361.9370839,429.31986388)
\curveto(1363.39541724,427.67403054)(1364.1245839,425.42923888)(1364.1245839,422.58548888)
\closepath
\moveto(1361.0933339,422.58548888)
\curveto(1361.0933339,424.84590554)(1360.65062557,426.52298888)(1359.7652089,427.61673888)
\curveto(1358.87979224,428.72090554)(1357.65062557,429.27298888)(1356.0777089,429.27298888)
\curveto(1354.4839589,429.27298888)(1353.24437557,428.72090554)(1352.3589589,427.61673888)
\curveto(1351.4839589,426.52298888)(1351.0464589,424.84590554)(1351.0464589,422.58548888)
\curveto(1351.0464589,420.39798888)(1351.48916724,418.73653054)(1352.3745839,417.60111388)
\curveto(1353.26000057,416.47611388)(1354.49437557,415.91361388)(1356.0777089,415.91361388)
\curveto(1357.6402089,415.91361388)(1358.86416724,416.47090554)(1359.7495839,417.58548888)
\curveto(1360.64541724,418.71048888)(1361.0933339,420.37715554)(1361.0933339,422.58548888)
\closepath
}
}
{
\newrgbcolor{curcolor}{0 0 0}
\pscustom[linestyle=none,fillstyle=solid,fillcolor=curcolor]
{
\newpath
\moveto(1371.8277089,434.24173888)
\lineto(1368.5152089,434.24173888)
\lineto(1368.5152089,437.28861388)
\lineto(1371.8277089,437.28861388)
\closepath
\moveto(1371.6402089,413.86673888)
\lineto(1368.7027089,413.86673888)
\lineto(1368.7027089,431.31986388)
\lineto(1371.6402089,431.31986388)
\closepath
}
}
{
\newrgbcolor{curcolor}{0 0 0}
\pscustom[linestyle=none,fillstyle=solid,fillcolor=curcolor]
{
\newpath
\moveto(1392.0464589,413.86673888)
\lineto(1389.1089589,413.86673888)
\lineto(1389.1089589,423.80423888)
\curveto(1389.1089589,424.60632221)(1389.0620839,425.35632221)(1388.9683339,426.05423888)
\curveto(1388.8745839,426.76257221)(1388.7027089,427.31465554)(1388.4527089,427.71048888)
\curveto(1388.19229224,428.14798888)(1387.81729224,428.47090554)(1387.3277089,428.67923888)
\curveto(1386.83812557,428.89798888)(1386.2027089,429.00736388)(1385.4214589,429.00736388)
\curveto(1384.61937557,429.00736388)(1383.7808339,428.80944721)(1382.9058339,428.41361388)
\curveto(1382.0308339,428.01778054)(1381.19229224,427.51257221)(1380.3902089,426.89798888)
\lineto(1380.3902089,413.86673888)
\lineto(1377.4527089,413.86673888)
\lineto(1377.4527089,431.31986388)
\lineto(1380.3902089,431.31986388)
\lineto(1380.3902089,429.38236388)
\curveto(1381.30687557,430.14278054)(1382.25479224,430.73653054)(1383.2339589,431.16361388)
\curveto(1384.21312557,431.59069721)(1385.2183339,431.80423888)(1386.2495839,431.80423888)
\curveto(1388.13500057,431.80423888)(1389.57250057,431.23653054)(1390.5620839,430.10111388)
\curveto(1391.55166724,428.96569721)(1392.0464589,427.33028054)(1392.0464589,425.19486388)
\closepath
}
}
{
\newrgbcolor{curcolor}{0 0 0}
\pscustom[linestyle=none,fillstyle=solid,fillcolor=curcolor]
{
\newpath
\moveto(1406.7652089,414.02298888)
\curveto(1406.21312557,413.87715554)(1405.6089589,413.75736388)(1404.9527089,413.66361388)
\curveto(1404.30687557,413.56986388)(1403.72875057,413.52298888)(1403.2183339,413.52298888)
\curveto(1401.4370839,413.52298888)(1400.08291724,414.00215554)(1399.1558339,414.96048888)
\curveto(1398.22875057,415.91882221)(1397.7652089,417.45528054)(1397.7652089,419.56986388)
\lineto(1397.7652089,428.85111388)
\lineto(1395.7808339,428.85111388)
\lineto(1395.7808339,431.31986388)
\lineto(1397.7652089,431.31986388)
\lineto(1397.7652089,436.33548888)
\lineto(1400.7027089,436.33548888)
\lineto(1400.7027089,431.31986388)
\lineto(1406.7652089,431.31986388)
\lineto(1406.7652089,428.85111388)
\lineto(1400.7027089,428.85111388)
\lineto(1400.7027089,420.89798888)
\curveto(1400.7027089,419.98132221)(1400.72354224,419.26257221)(1400.7652089,418.74173888)
\curveto(1400.80687557,418.23132221)(1400.9527089,417.75215554)(1401.2027089,417.30423888)
\curveto(1401.43187557,416.88757221)(1401.74437557,416.58028054)(1402.1402089,416.38236388)
\curveto(1402.5464589,416.19486388)(1403.16104224,416.10111388)(1403.9839589,416.10111388)
\curveto(1404.46312557,416.10111388)(1404.96312557,416.16882221)(1405.4839589,416.30423888)
\curveto(1406.00479224,416.45007221)(1406.37979224,416.56986388)(1406.6089589,416.66361388)
\lineto(1406.7652089,416.66361388)
\closepath
}
}
{
\newrgbcolor{curcolor}{0 0 0}
\pscustom[linestyle=none,fillstyle=solid,fillcolor=curcolor]
{
\newpath
\moveto(1424.9214589,422.28861388)
\lineto(1412.0620839,422.28861388)
\curveto(1412.0620839,421.21569721)(1412.22354224,420.27819721)(1412.5464589,419.47611388)
\curveto(1412.86937557,418.68444721)(1413.3120839,418.03340554)(1413.8745839,417.52298888)
\curveto(1414.41625057,417.02298888)(1415.05687557,416.64798888)(1415.7964589,416.39798888)
\curveto(1416.5464589,416.14798888)(1417.36937557,416.02298888)(1418.2652089,416.02298888)
\curveto(1419.4527089,416.02298888)(1420.64541724,416.25736388)(1421.8433339,416.72611388)
\curveto(1423.05166724,417.20528054)(1423.91104224,417.67403054)(1424.4214589,418.13236388)
\lineto(1424.5777089,418.13236388)
\lineto(1424.5777089,414.92923888)
\curveto(1423.58812557,414.51257221)(1422.5777089,414.16361388)(1421.5464589,413.88236388)
\curveto(1420.5152089,413.60111388)(1419.43187557,413.46048888)(1418.2964589,413.46048888)
\curveto(1415.40062557,413.46048888)(1413.1402089,414.24173888)(1411.5152089,415.80423888)
\curveto(1409.8902089,417.37715554)(1409.0777089,419.60632221)(1409.0777089,422.49173888)
\curveto(1409.0777089,425.34590554)(1409.85375057,427.61153054)(1411.4058339,429.28861388)
\curveto(1412.9683339,430.96569721)(1415.02041724,431.80423888)(1417.5620839,431.80423888)
\curveto(1419.91625057,431.80423888)(1421.72875057,431.11673888)(1422.9995839,429.74173888)
\curveto(1424.2808339,428.36673888)(1424.9214589,426.41361388)(1424.9214589,423.88236388)
\closepath
\moveto(1422.0620839,424.53861388)
\curveto(1422.05166724,426.08028054)(1421.66104224,427.27298888)(1420.8902089,428.11673888)
\curveto(1420.12979224,428.96048888)(1418.9683339,429.38236388)(1417.4058339,429.38236388)
\curveto(1415.83291724,429.38236388)(1414.5777089,428.91882221)(1413.6402089,427.99173888)
\curveto(1412.71312557,427.06465554)(1412.1870839,425.91361388)(1412.0620839,424.53861388)
\closepath
}
}
{
\newrgbcolor{curcolor}{0 0 0}
\pscustom[linestyle=none,fillstyle=solid,fillcolor=curcolor]
{
\newpath
\moveto(1440.2652089,428.11673888)
\lineto(1440.1089589,428.11673888)
\curveto(1439.6714589,428.22090554)(1439.24437557,428.29382221)(1438.8277089,428.33548888)
\curveto(1438.4214589,428.38757221)(1437.9370839,428.41361388)(1437.3745839,428.41361388)
\curveto(1436.4683339,428.41361388)(1435.5933339,428.21048888)(1434.7495839,427.80423888)
\curveto(1433.9058339,427.40840554)(1433.0933339,426.89278054)(1432.3120839,426.25736388)
\lineto(1432.3120839,413.86673888)
\lineto(1429.3745839,413.86673888)
\lineto(1429.3745839,431.31986388)
\lineto(1432.3120839,431.31986388)
\lineto(1432.3120839,428.74173888)
\curveto(1433.47875057,429.67923888)(1434.50479224,430.34069721)(1435.3902089,430.72611388)
\curveto(1436.28604224,431.12194721)(1437.19750057,431.31986388)(1438.1245839,431.31986388)
\curveto(1438.63500057,431.31986388)(1439.00479224,431.30423888)(1439.2339589,431.27298888)
\curveto(1439.46312557,431.25215554)(1439.80687557,431.20528054)(1440.2652089,431.13236388)
\closepath
}
}
{
\newrgbcolor{curcolor}{0 0 0}
\pscustom[linestyle=none,fillstyle=solid,fillcolor=curcolor]
{
\newpath
\moveto(1452.3745839,407.42923888)
\lineto(1448.7964589,407.42923888)
\curveto(1446.9527089,409.54382221)(1445.52041724,411.85111388)(1444.4995839,414.35111388)
\curveto(1443.47875057,416.85111388)(1442.9683339,419.66882221)(1442.9683339,422.80423888)
\curveto(1442.9683339,425.93965554)(1443.47875057,428.75736388)(1444.4995839,431.25736388)
\curveto(1445.52041724,433.75736388)(1446.9527089,436.06465554)(1448.7964589,438.17923888)
\lineto(1452.3745839,438.17923888)
\lineto(1452.3745839,438.02298888)
\curveto(1451.5308339,437.26257221)(1450.72354224,436.38236388)(1449.9527089,435.38236388)
\curveto(1449.19229224,434.39278054)(1448.4839589,433.23653054)(1447.8277089,431.91361388)
\curveto(1447.2027089,430.63236388)(1446.69229224,429.22090554)(1446.2964589,427.67923888)
\curveto(1445.91104224,426.13757221)(1445.7183339,424.51257221)(1445.7183339,422.80423888)
\curveto(1445.7183339,421.02298888)(1445.9058339,419.39278054)(1446.2808339,417.91361388)
\curveto(1446.66625057,416.43444721)(1447.18187557,415.02819721)(1447.8277089,413.69486388)
\curveto(1448.4527089,412.41361388)(1449.16625057,411.25736388)(1449.9683339,410.22611388)
\curveto(1450.77041724,409.18444721)(1451.57250057,408.30423888)(1452.3745839,407.58548888)
\closepath
}
}
{
\newrgbcolor{curcolor}{0 0 0}
\pscustom[linestyle=none,fillstyle=solid,fillcolor=curcolor]
{
\newpath
\moveto(1472.7495839,421.25736388)
\curveto(1472.7495839,420.17403054)(1472.55166724,419.13757221)(1472.1558339,418.14798888)
\curveto(1471.76000057,417.15840554)(1471.2183339,416.32507221)(1470.5308339,415.64798888)
\curveto(1469.7808339,414.91882221)(1468.88500057,414.35632221)(1467.8433339,413.96048888)
\curveto(1466.8120839,413.57507221)(1465.61416724,413.38236388)(1464.2495839,413.38236388)
\curveto(1462.97875057,413.38236388)(1461.75479224,413.51778054)(1460.5777089,413.78861388)
\curveto(1459.40062557,414.04903054)(1458.4058339,414.36673888)(1457.5933339,414.74173888)
\lineto(1457.5933339,418.03861388)
\lineto(1457.8120839,418.03861388)
\curveto(1458.66625057,417.49694721)(1459.66625057,417.03340554)(1460.8120839,416.64798888)
\curveto(1461.95791724,416.27298888)(1463.08291724,416.08548888)(1464.1870839,416.08548888)
\curveto(1464.92666724,416.08548888)(1465.6402089,416.18965554)(1466.3277089,416.39798888)
\curveto(1467.02562557,416.60632221)(1467.64541724,416.97090554)(1468.1870839,417.49173888)
\curveto(1468.64541724,417.93965554)(1468.98916724,418.47611388)(1469.2183339,419.10111388)
\curveto(1469.45791724,419.72611388)(1469.5777089,420.45007221)(1469.5777089,421.27298888)
\curveto(1469.5777089,422.07507221)(1469.4370839,422.75215554)(1469.1558339,423.30423888)
\curveto(1468.88500057,423.85632221)(1468.50479224,424.29903054)(1468.0152089,424.63236388)
\curveto(1467.47354224,425.02819721)(1466.8120839,425.30423888)(1466.0308339,425.46048888)
\curveto(1465.26000057,425.62715554)(1464.39541724,425.71048888)(1463.4370839,425.71048888)
\curveto(1462.52041724,425.71048888)(1461.63500057,425.64798888)(1460.7808339,425.52298888)
\curveto(1459.9370839,425.39798888)(1459.20791724,425.27298888)(1458.5933339,425.14798888)
\lineto(1458.5933339,437.13236388)
\lineto(1472.5933339,437.13236388)
\lineto(1472.5933339,434.39798888)
\lineto(1461.6089589,434.39798888)
\lineto(1461.6089589,428.21048888)
\curveto(1462.05687557,428.25215554)(1462.5152089,428.28340554)(1462.9839589,428.30423888)
\curveto(1463.4527089,428.32507221)(1463.8589589,428.33548888)(1464.2027089,428.33548888)
\curveto(1465.46312557,428.33548888)(1466.56729224,428.22611388)(1467.5152089,428.00736388)
\curveto(1468.46312557,427.79903054)(1469.33291724,427.42403054)(1470.1245839,426.88236388)
\curveto(1470.95791724,426.30944721)(1471.60375057,425.56986388)(1472.0620839,424.66361388)
\curveto(1472.52041724,423.75736388)(1472.7495839,422.62194721)(1472.7495839,421.25736388)
\closepath
}
}
{
\newrgbcolor{curcolor}{0 0 0}
\pscustom[linestyle=none,fillstyle=solid,fillcolor=curcolor]
{
\newpath
\moveto(1486.7183339,422.80423888)
\curveto(1486.7183339,419.66882221)(1486.20791724,416.85111388)(1485.1870839,414.35111388)
\curveto(1484.16625057,411.85111388)(1482.7339589,409.54382221)(1480.8902089,407.42923888)
\lineto(1477.3120839,407.42923888)
\lineto(1477.3120839,407.58548888)
\curveto(1478.11416724,408.30423888)(1478.91625057,409.18444721)(1479.7183339,410.22611388)
\curveto(1480.5308339,411.25736388)(1481.24437557,412.41361388)(1481.8589589,413.69486388)
\curveto(1482.50479224,415.02819721)(1483.0152089,416.43444721)(1483.3902089,417.91361388)
\curveto(1483.77562557,419.39278054)(1483.9683339,421.02298888)(1483.9683339,422.80423888)
\curveto(1483.9683339,424.51257221)(1483.77562557,426.13757221)(1483.3902089,427.67923888)
\curveto(1483.00479224,429.22090554)(1482.49437557,430.63236388)(1481.8589589,431.91361388)
\curveto(1481.2027089,433.23653054)(1480.48916724,434.39278054)(1479.7183339,435.38236388)
\curveto(1478.95791724,436.38236388)(1478.1558339,437.26257221)(1477.3120839,438.02298888)
\lineto(1477.3120839,438.17923888)
\lineto(1480.8902089,438.17923888)
\curveto(1482.7339589,436.06465554)(1484.16625057,433.75736388)(1485.1870839,431.25736388)
\curveto(1486.20791724,428.75736388)(1486.7183339,425.93965554)(1486.7183339,422.80423888)
\closepath
}
}
{
\newrgbcolor{curcolor}{0 0 0}
\pscustom[linestyle=none,fillstyle=solid,fillcolor=curcolor]
{
\newpath
\moveto(1260.9214589,382.72611388)
\curveto(1260.9214589,381.26778054)(1260.71312557,379.95528054)(1260.2964589,378.78861388)
\curveto(1259.8902089,377.62194721)(1259.33812557,376.64278054)(1258.6402089,375.85111388)
\curveto(1257.90062557,375.02819721)(1257.08812557,374.40840554)(1256.2027089,373.99173888)
\curveto(1255.31729224,373.58548888)(1254.3433339,373.38236388)(1253.2808339,373.38236388)
\curveto(1252.29125057,373.38236388)(1251.42666724,373.50215554)(1250.6870839,373.74173888)
\curveto(1249.94750057,373.97090554)(1249.2183339,374.28340554)(1248.4995839,374.67923888)
\lineto(1248.3120839,373.86673888)
\lineto(1245.5620839,373.86673888)
\lineto(1245.5620839,398.17923888)
\lineto(1248.4995839,398.17923888)
\lineto(1248.4995839,389.49173888)
\curveto(1249.32250057,390.16882221)(1250.19750057,390.72090554)(1251.1245839,391.14798888)
\curveto(1252.05166724,391.58548888)(1253.0933339,391.80423888)(1254.2495839,391.80423888)
\curveto(1256.3120839,391.80423888)(1257.9370839,391.01257221)(1259.1245839,389.42923888)
\curveto(1260.32250057,387.84590554)(1260.9214589,385.61153054)(1260.9214589,382.72611388)
\closepath
\moveto(1257.8902089,382.64798888)
\curveto(1257.8902089,384.73132221)(1257.5464589,386.30944721)(1256.8589589,387.38236388)
\curveto(1256.1714589,388.46569721)(1255.0620839,389.00736388)(1253.5308339,389.00736388)
\curveto(1252.67666724,389.00736388)(1251.8120839,388.81986388)(1250.9370839,388.44486388)
\curveto(1250.0620839,388.08028054)(1249.2495839,387.60632221)(1248.4995839,387.02298888)
\lineto(1248.4995839,377.02298888)
\curveto(1249.33291724,376.64798888)(1250.0464589,376.38757221)(1250.6402089,376.24173888)
\curveto(1251.24437557,376.09590554)(1251.92666724,376.02298888)(1252.6870839,376.02298888)
\curveto(1254.3120839,376.02298888)(1255.58291724,376.55423888)(1256.4995839,377.61673888)
\curveto(1257.42666724,378.68965554)(1257.8902089,380.36673888)(1257.8902089,382.64798888)
\closepath
}
}
{
\newrgbcolor{curcolor}{0 0 0}
\pscustom[linestyle=none,fillstyle=solid,fillcolor=curcolor]
{
\newpath
\moveto(1268.4683339,373.86673888)
\lineto(1265.5308339,373.86673888)
\lineto(1265.5308339,398.17923888)
\lineto(1268.4683339,398.17923888)
\closepath
}
}
{
\newrgbcolor{curcolor}{0 0 0}
\pscustom[linestyle=none,fillstyle=solid,fillcolor=curcolor]
{
\newpath
\moveto(1289.1558339,382.58548888)
\curveto(1289.1558339,379.74173888)(1288.42666724,377.49694721)(1286.9683339,375.85111388)
\curveto(1285.51000057,374.20528054)(1283.55687557,373.38236388)(1281.1089589,373.38236388)
\curveto(1278.6402089,373.38236388)(1276.67666724,374.20528054)(1275.2183339,375.85111388)
\curveto(1273.77041724,377.49694721)(1273.0464589,379.74173888)(1273.0464589,382.58548888)
\curveto(1273.0464589,385.42923888)(1273.77041724,387.67403054)(1275.2183339,389.31986388)
\curveto(1276.67666724,390.97611388)(1278.6402089,391.80423888)(1281.1089589,391.80423888)
\curveto(1283.55687557,391.80423888)(1285.51000057,390.97611388)(1286.9683339,389.31986388)
\curveto(1288.42666724,387.67403054)(1289.1558339,385.42923888)(1289.1558339,382.58548888)
\closepath
\moveto(1286.1245839,382.58548888)
\curveto(1286.1245839,384.84590554)(1285.68187557,386.52298888)(1284.7964589,387.61673888)
\curveto(1283.91104224,388.72090554)(1282.68187557,389.27298888)(1281.1089589,389.27298888)
\curveto(1279.5152089,389.27298888)(1278.27562557,388.72090554)(1277.3902089,387.61673888)
\curveto(1276.5152089,386.52298888)(1276.0777089,384.84590554)(1276.0777089,382.58548888)
\curveto(1276.0777089,380.39798888)(1276.52041724,378.73653054)(1277.4058339,377.60111388)
\curveto(1278.29125057,376.47611388)(1279.52562557,375.91361388)(1281.1089589,375.91361388)
\curveto(1282.6714589,375.91361388)(1283.89541724,376.47090554)(1284.7808339,377.58548888)
\curveto(1285.67666724,378.71048888)(1286.1245839,380.37715554)(1286.1245839,382.58548888)
\closepath
}
}
{
\newrgbcolor{curcolor}{0 0 0}
\pscustom[linestyle=none,fillstyle=solid,fillcolor=curcolor]
{
\newpath
\moveto(1306.6089589,374.96048888)
\curveto(1305.62979224,374.49173888)(1304.69750057,374.12715554)(1303.8120839,373.86673888)
\curveto(1302.9370839,373.60632221)(1302.00479224,373.47611388)(1301.0152089,373.47611388)
\curveto(1299.75479224,373.47611388)(1298.59854224,373.65840554)(1297.5464589,374.02298888)
\curveto(1296.49437557,374.39798888)(1295.5933339,374.96048888)(1294.8433339,375.71048888)
\curveto(1294.08291724,376.46048888)(1293.49437557,377.40840554)(1293.0777089,378.55423888)
\curveto(1292.66104224,379.70007221)(1292.4527089,381.03861388)(1292.4527089,382.56986388)
\curveto(1292.4527089,385.42403054)(1293.2339589,387.66361388)(1294.7964589,389.28861388)
\curveto(1296.36937557,390.91361388)(1298.44229224,391.72611388)(1301.0152089,391.72611388)
\curveto(1302.0152089,391.72611388)(1302.99437557,391.58548888)(1303.9527089,391.30423888)
\curveto(1304.9214589,391.02298888)(1305.80687557,390.67923888)(1306.6089589,390.27298888)
\lineto(1306.6089589,387.00736388)
\lineto(1306.4527089,387.00736388)
\curveto(1305.55687557,387.70528054)(1304.62979224,388.24173888)(1303.6714589,388.61673888)
\curveto(1302.72354224,388.99173888)(1301.7964589,389.17923888)(1300.8902089,389.17923888)
\curveto(1299.22354224,389.17923888)(1297.9058339,388.61673888)(1296.9370839,387.49173888)
\curveto(1295.97875057,386.37715554)(1295.4995839,384.73653054)(1295.4995839,382.56986388)
\curveto(1295.4995839,380.46569721)(1295.9683339,378.84590554)(1296.9058339,377.71048888)
\curveto(1297.85375057,376.58548888)(1299.18187557,376.02298888)(1300.8902089,376.02298888)
\curveto(1301.4839589,376.02298888)(1302.08812557,376.10111388)(1302.7027089,376.25736388)
\curveto(1303.31729224,376.41361388)(1303.86937557,376.61673888)(1304.3589589,376.86673888)
\curveto(1304.78604224,377.08548888)(1305.1870839,377.31465554)(1305.5620839,377.55423888)
\curveto(1305.9370839,377.80423888)(1306.2339589,378.01778054)(1306.4527089,378.19486388)
\lineto(1306.6089589,378.19486388)
\closepath
}
}
{
\newrgbcolor{curcolor}{0 0 0}
\pscustom[linestyle=none,fillstyle=solid,fillcolor=curcolor]
{
\newpath
\moveto(1326.2183339,373.86673888)
\lineto(1322.3433339,373.86673888)
\lineto(1315.3433339,381.50736388)
\lineto(1313.4370839,379.69486388)
\lineto(1313.4370839,373.86673888)
\lineto(1310.4995839,373.86673888)
\lineto(1310.4995839,398.17923888)
\lineto(1313.4370839,398.17923888)
\lineto(1313.4370839,382.58548888)
\lineto(1321.9214589,391.31986388)
\lineto(1325.6245839,391.31986388)
\lineto(1317.5152089,383.25736388)
\closepath
}
}
{
\newrgbcolor{curcolor}{0 0 0}
\pscustom[linestyle=none,fillstyle=solid,fillcolor=curcolor]
{
\newpath
\moveto(1344.6714589,382.80423888)
\curveto(1344.6714589,381.38757221)(1344.4683339,380.09069721)(1344.0620839,378.91361388)
\curveto(1343.6558339,377.74694721)(1343.08291724,376.75736388)(1342.3433339,375.94486388)
\curveto(1341.6558339,375.17403054)(1340.8433339,374.57507221)(1339.9058339,374.14798888)
\curveto(1338.97875057,373.73132221)(1337.99437557,373.52298888)(1336.9527089,373.52298888)
\curveto(1336.0464589,373.52298888)(1335.22354224,373.62194721)(1334.4839589,373.81986388)
\curveto(1333.75479224,374.01778054)(1333.01000057,374.32507221)(1332.2495839,374.74173888)
\lineto(1332.2495839,367.42923888)
\lineto(1329.3120839,367.42923888)
\lineto(1329.3120839,391.31986388)
\lineto(1332.2495839,391.31986388)
\lineto(1332.2495839,389.49173888)
\curveto(1333.0308339,390.14798888)(1333.9058339,390.69486388)(1334.8745839,391.13236388)
\curveto(1335.85375057,391.58028054)(1336.89541724,391.80423888)(1337.9995839,391.80423888)
\curveto(1340.10375057,391.80423888)(1341.73916724,391.00736388)(1342.9058339,389.41361388)
\curveto(1344.08291724,387.83028054)(1344.6714589,385.62715554)(1344.6714589,382.80423888)
\closepath
\moveto(1341.6402089,382.72611388)
\curveto(1341.6402089,384.83028054)(1341.2808339,386.40319721)(1340.5620839,387.44486388)
\curveto(1339.8433339,388.48653054)(1338.73916724,389.00736388)(1337.2495839,389.00736388)
\curveto(1336.4058339,389.00736388)(1335.55687557,388.82507221)(1334.7027089,388.46048888)
\curveto(1333.84854224,388.09590554)(1333.0308339,387.61673888)(1332.2495839,387.02298888)
\lineto(1332.2495839,377.13236388)
\curveto(1333.08291724,376.75736388)(1333.7964589,376.50215554)(1334.3902089,376.36673888)
\curveto(1334.99437557,376.23132221)(1335.67666724,376.16361388)(1336.4370839,376.16361388)
\curveto(1338.07250057,376.16361388)(1339.34854224,376.71569721)(1340.2652089,377.81986388)
\curveto(1341.18187557,378.92403054)(1341.6402089,380.55944721)(1341.6402089,382.72611388)
\closepath
}
}
{
\newrgbcolor{curcolor}{0 0 0}
\pscustom[linestyle=none,fillstyle=solid,fillcolor=curcolor]
{
\newpath
\moveto(1364.1245839,382.58548888)
\curveto(1364.1245839,379.74173888)(1363.39541724,377.49694721)(1361.9370839,375.85111388)
\curveto(1360.47875057,374.20528054)(1358.52562557,373.38236388)(1356.0777089,373.38236388)
\curveto(1353.6089589,373.38236388)(1351.64541724,374.20528054)(1350.1870839,375.85111388)
\curveto(1348.73916724,377.49694721)(1348.0152089,379.74173888)(1348.0152089,382.58548888)
\curveto(1348.0152089,385.42923888)(1348.73916724,387.67403054)(1350.1870839,389.31986388)
\curveto(1351.64541724,390.97611388)(1353.6089589,391.80423888)(1356.0777089,391.80423888)
\curveto(1358.52562557,391.80423888)(1360.47875057,390.97611388)(1361.9370839,389.31986388)
\curveto(1363.39541724,387.67403054)(1364.1245839,385.42923888)(1364.1245839,382.58548888)
\closepath
\moveto(1361.0933339,382.58548888)
\curveto(1361.0933339,384.84590554)(1360.65062557,386.52298888)(1359.7652089,387.61673888)
\curveto(1358.87979224,388.72090554)(1357.65062557,389.27298888)(1356.0777089,389.27298888)
\curveto(1354.4839589,389.27298888)(1353.24437557,388.72090554)(1352.3589589,387.61673888)
\curveto(1351.4839589,386.52298888)(1351.0464589,384.84590554)(1351.0464589,382.58548888)
\curveto(1351.0464589,380.39798888)(1351.48916724,378.73653054)(1352.3745839,377.60111388)
\curveto(1353.26000057,376.47611388)(1354.49437557,375.91361388)(1356.0777089,375.91361388)
\curveto(1357.6402089,375.91361388)(1358.86416724,376.47090554)(1359.7495839,377.58548888)
\curveto(1360.64541724,378.71048888)(1361.0933339,380.37715554)(1361.0933339,382.58548888)
\closepath
}
}
{
\newrgbcolor{curcolor}{0 0 0}
\pscustom[linestyle=none,fillstyle=solid,fillcolor=curcolor]
{
\newpath
\moveto(1371.8277089,394.24173888)
\lineto(1368.5152089,394.24173888)
\lineto(1368.5152089,397.28861388)
\lineto(1371.8277089,397.28861388)
\closepath
\moveto(1371.6402089,373.86673888)
\lineto(1368.7027089,373.86673888)
\lineto(1368.7027089,391.31986388)
\lineto(1371.6402089,391.31986388)
\closepath
}
}
{
\newrgbcolor{curcolor}{0 0 0}
\pscustom[linestyle=none,fillstyle=solid,fillcolor=curcolor]
{
\newpath
\moveto(1392.0464589,373.86673888)
\lineto(1389.1089589,373.86673888)
\lineto(1389.1089589,383.80423888)
\curveto(1389.1089589,384.60632221)(1389.0620839,385.35632221)(1388.9683339,386.05423888)
\curveto(1388.8745839,386.76257221)(1388.7027089,387.31465554)(1388.4527089,387.71048888)
\curveto(1388.19229224,388.14798888)(1387.81729224,388.47090554)(1387.3277089,388.67923888)
\curveto(1386.83812557,388.89798888)(1386.2027089,389.00736388)(1385.4214589,389.00736388)
\curveto(1384.61937557,389.00736388)(1383.7808339,388.80944721)(1382.9058339,388.41361388)
\curveto(1382.0308339,388.01778054)(1381.19229224,387.51257221)(1380.3902089,386.89798888)
\lineto(1380.3902089,373.86673888)
\lineto(1377.4527089,373.86673888)
\lineto(1377.4527089,391.31986388)
\lineto(1380.3902089,391.31986388)
\lineto(1380.3902089,389.38236388)
\curveto(1381.30687557,390.14278054)(1382.25479224,390.73653054)(1383.2339589,391.16361388)
\curveto(1384.21312557,391.59069721)(1385.2183339,391.80423888)(1386.2495839,391.80423888)
\curveto(1388.13500057,391.80423888)(1389.57250057,391.23653054)(1390.5620839,390.10111388)
\curveto(1391.55166724,388.96569721)(1392.0464589,387.33028054)(1392.0464589,385.19486388)
\closepath
}
}
{
\newrgbcolor{curcolor}{0 0 0}
\pscustom[linestyle=none,fillstyle=solid,fillcolor=curcolor]
{
\newpath
\moveto(1406.7652089,374.02298888)
\curveto(1406.21312557,373.87715554)(1405.6089589,373.75736388)(1404.9527089,373.66361388)
\curveto(1404.30687557,373.56986388)(1403.72875057,373.52298888)(1403.2183339,373.52298888)
\curveto(1401.4370839,373.52298888)(1400.08291724,374.00215554)(1399.1558339,374.96048888)
\curveto(1398.22875057,375.91882221)(1397.7652089,377.45528054)(1397.7652089,379.56986388)
\lineto(1397.7652089,388.85111388)
\lineto(1395.7808339,388.85111388)
\lineto(1395.7808339,391.31986388)
\lineto(1397.7652089,391.31986388)
\lineto(1397.7652089,396.33548888)
\lineto(1400.7027089,396.33548888)
\lineto(1400.7027089,391.31986388)
\lineto(1406.7652089,391.31986388)
\lineto(1406.7652089,388.85111388)
\lineto(1400.7027089,388.85111388)
\lineto(1400.7027089,380.89798888)
\curveto(1400.7027089,379.98132221)(1400.72354224,379.26257221)(1400.7652089,378.74173888)
\curveto(1400.80687557,378.23132221)(1400.9527089,377.75215554)(1401.2027089,377.30423888)
\curveto(1401.43187557,376.88757221)(1401.74437557,376.58028054)(1402.1402089,376.38236388)
\curveto(1402.5464589,376.19486388)(1403.16104224,376.10111388)(1403.9839589,376.10111388)
\curveto(1404.46312557,376.10111388)(1404.96312557,376.16882221)(1405.4839589,376.30423888)
\curveto(1406.00479224,376.45007221)(1406.37979224,376.56986388)(1406.6089589,376.66361388)
\lineto(1406.7652089,376.66361388)
\closepath
}
}
{
\newrgbcolor{curcolor}{0 0 0}
\pscustom[linestyle=none,fillstyle=solid,fillcolor=curcolor]
{
\newpath
\moveto(1424.9214589,382.28861388)
\lineto(1412.0620839,382.28861388)
\curveto(1412.0620839,381.21569721)(1412.22354224,380.27819721)(1412.5464589,379.47611388)
\curveto(1412.86937557,378.68444721)(1413.3120839,378.03340554)(1413.8745839,377.52298888)
\curveto(1414.41625057,377.02298888)(1415.05687557,376.64798888)(1415.7964589,376.39798888)
\curveto(1416.5464589,376.14798888)(1417.36937557,376.02298888)(1418.2652089,376.02298888)
\curveto(1419.4527089,376.02298888)(1420.64541724,376.25736388)(1421.8433339,376.72611388)
\curveto(1423.05166724,377.20528054)(1423.91104224,377.67403054)(1424.4214589,378.13236388)
\lineto(1424.5777089,378.13236388)
\lineto(1424.5777089,374.92923888)
\curveto(1423.58812557,374.51257221)(1422.5777089,374.16361388)(1421.5464589,373.88236388)
\curveto(1420.5152089,373.60111388)(1419.43187557,373.46048888)(1418.2964589,373.46048888)
\curveto(1415.40062557,373.46048888)(1413.1402089,374.24173888)(1411.5152089,375.80423888)
\curveto(1409.8902089,377.37715554)(1409.0777089,379.60632221)(1409.0777089,382.49173888)
\curveto(1409.0777089,385.34590554)(1409.85375057,387.61153054)(1411.4058339,389.28861388)
\curveto(1412.9683339,390.96569721)(1415.02041724,391.80423888)(1417.5620839,391.80423888)
\curveto(1419.91625057,391.80423888)(1421.72875057,391.11673888)(1422.9995839,389.74173888)
\curveto(1424.2808339,388.36673888)(1424.9214589,386.41361388)(1424.9214589,383.88236388)
\closepath
\moveto(1422.0620839,384.53861388)
\curveto(1422.05166724,386.08028054)(1421.66104224,387.27298888)(1420.8902089,388.11673888)
\curveto(1420.12979224,388.96048888)(1418.9683339,389.38236388)(1417.4058339,389.38236388)
\curveto(1415.83291724,389.38236388)(1414.5777089,388.91882221)(1413.6402089,387.99173888)
\curveto(1412.71312557,387.06465554)(1412.1870839,385.91361388)(1412.0620839,384.53861388)
\closepath
}
}
{
\newrgbcolor{curcolor}{0 0 0}
\pscustom[linestyle=none,fillstyle=solid,fillcolor=curcolor]
{
\newpath
\moveto(1440.2652089,388.11673888)
\lineto(1440.1089589,388.11673888)
\curveto(1439.6714589,388.22090554)(1439.24437557,388.29382221)(1438.8277089,388.33548888)
\curveto(1438.4214589,388.38757221)(1437.9370839,388.41361388)(1437.3745839,388.41361388)
\curveto(1436.4683339,388.41361388)(1435.5933339,388.21048888)(1434.7495839,387.80423888)
\curveto(1433.9058339,387.40840554)(1433.0933339,386.89278054)(1432.3120839,386.25736388)
\lineto(1432.3120839,373.86673888)
\lineto(1429.3745839,373.86673888)
\lineto(1429.3745839,391.31986388)
\lineto(1432.3120839,391.31986388)
\lineto(1432.3120839,388.74173888)
\curveto(1433.47875057,389.67923888)(1434.50479224,390.34069721)(1435.3902089,390.72611388)
\curveto(1436.28604224,391.12194721)(1437.19750057,391.31986388)(1438.1245839,391.31986388)
\curveto(1438.63500057,391.31986388)(1439.00479224,391.30423888)(1439.2339589,391.27298888)
\curveto(1439.46312557,391.25215554)(1439.80687557,391.20528054)(1440.2652089,391.13236388)
\closepath
}
}
{
\newrgbcolor{curcolor}{0 0 0}
\pscustom[linestyle=none,fillstyle=solid,fillcolor=curcolor]
{
\newpath
\moveto(1452.3745839,367.42923888)
\lineto(1448.7964589,367.42923888)
\curveto(1446.9527089,369.54382221)(1445.52041724,371.85111388)(1444.4995839,374.35111388)
\curveto(1443.47875057,376.85111388)(1442.9683339,379.66882221)(1442.9683339,382.80423888)
\curveto(1442.9683339,385.93965554)(1443.47875057,388.75736388)(1444.4995839,391.25736388)
\curveto(1445.52041724,393.75736388)(1446.9527089,396.06465554)(1448.7964589,398.17923888)
\lineto(1452.3745839,398.17923888)
\lineto(1452.3745839,398.02298888)
\curveto(1451.5308339,397.26257221)(1450.72354224,396.38236388)(1449.9527089,395.38236388)
\curveto(1449.19229224,394.39278054)(1448.4839589,393.23653054)(1447.8277089,391.91361388)
\curveto(1447.2027089,390.63236388)(1446.69229224,389.22090554)(1446.2964589,387.67923888)
\curveto(1445.91104224,386.13757221)(1445.7183339,384.51257221)(1445.7183339,382.80423888)
\curveto(1445.7183339,381.02298888)(1445.9058339,379.39278054)(1446.2808339,377.91361388)
\curveto(1446.66625057,376.43444721)(1447.18187557,375.02819721)(1447.8277089,373.69486388)
\curveto(1448.4527089,372.41361388)(1449.16625057,371.25736388)(1449.9683339,370.22611388)
\curveto(1450.77041724,369.18444721)(1451.57250057,368.30423888)(1452.3745839,367.58548888)
\closepath
}
}
{
\newrgbcolor{curcolor}{0 0 0}
\pscustom[linestyle=none,fillstyle=solid,fillcolor=curcolor]
{
\newpath
\moveto(1473.2808339,381.41361388)
\curveto(1473.2808339,379.04903054)(1472.4995839,377.11673888)(1470.9370839,375.61673888)
\curveto(1469.38500057,374.12715554)(1467.47875057,373.38236388)(1465.2183339,373.38236388)
\curveto(1464.07250057,373.38236388)(1463.0308339,373.55944721)(1462.0933339,373.91361388)
\curveto(1461.1558339,374.26778054)(1460.3277089,374.79382221)(1459.6089589,375.49173888)
\curveto(1458.71312557,376.35632221)(1458.02041724,377.50215554)(1457.5308339,378.92923888)
\curveto(1457.05166724,380.35632221)(1456.8120839,382.07507221)(1456.8120839,384.08548888)
\curveto(1456.8120839,386.14798888)(1457.0308339,387.97611388)(1457.4683339,389.56986388)
\curveto(1457.91625057,391.16361388)(1458.6245839,392.58028054)(1459.5933339,393.81986388)
\curveto(1460.51000057,394.99694721)(1461.69229224,395.91361388)(1463.1402089,396.56986388)
\curveto(1464.58812557,397.23653054)(1466.27562557,397.56986388)(1468.2027089,397.56986388)
\curveto(1468.81729224,397.56986388)(1469.33291724,397.54382221)(1469.7495839,397.49173888)
\curveto(1470.16625057,397.43965554)(1470.58812557,397.34590554)(1471.0152089,397.21048888)
\lineto(1471.0152089,394.22611388)
\lineto(1470.8589589,394.22611388)
\curveto(1470.56729224,394.38236388)(1470.1245839,394.52819721)(1469.5308339,394.66361388)
\curveto(1468.94750057,394.80944721)(1468.34854224,394.88236388)(1467.7339589,394.88236388)
\curveto(1465.49437557,394.88236388)(1463.70791724,394.17923888)(1462.3745839,392.77298888)
\curveto(1461.04125057,391.37715554)(1460.2652089,389.48653054)(1460.0464589,387.10111388)
\curveto(1460.9214589,387.63236388)(1461.7808339,388.03340554)(1462.6245839,388.30423888)
\curveto(1463.47875057,388.58548888)(1464.46312557,388.72611388)(1465.5777089,388.72611388)
\curveto(1466.56729224,388.72611388)(1467.4370839,388.63236388)(1468.1870839,388.44486388)
\curveto(1468.94750057,388.26778054)(1469.72354224,387.90319721)(1470.5152089,387.35111388)
\curveto(1471.43187557,386.71569721)(1472.11937557,385.91361388)(1472.5777089,384.94486388)
\curveto(1473.0464589,383.97611388)(1473.2808339,382.79903054)(1473.2808339,381.41361388)
\closepath
\moveto(1470.1089589,381.28861388)
\curveto(1470.1089589,382.25736388)(1469.96312557,383.05944721)(1469.6714589,383.69486388)
\curveto(1469.3902089,384.33028054)(1468.9214589,384.88236388)(1468.2652089,385.35111388)
\curveto(1467.78604224,385.68444721)(1467.25479224,385.90319721)(1466.6714589,386.00736388)
\curveto(1466.08812557,386.11153054)(1465.47875057,386.16361388)(1464.8433339,386.16361388)
\curveto(1463.95791724,386.16361388)(1463.13500057,386.05944721)(1462.3745839,385.85111388)
\curveto(1461.61416724,385.64278054)(1460.83291724,385.31986388)(1460.0308339,384.88236388)
\curveto(1460.01000057,384.65319721)(1459.99437557,384.42923888)(1459.9839589,384.21048888)
\curveto(1459.97354224,384.00215554)(1459.9683339,383.73653054)(1459.9683339,383.41361388)
\curveto(1459.9683339,381.76778054)(1460.13500057,380.46569721)(1460.4683339,379.50736388)
\curveto(1460.8120839,378.55944721)(1461.2808339,377.80944721)(1461.8745839,377.25736388)
\curveto(1462.35375057,376.79903054)(1462.86937557,376.46048888)(1463.4214589,376.24173888)
\curveto(1463.9839589,376.03340554)(1464.5933339,375.92923888)(1465.2495839,375.92923888)
\curveto(1466.76000057,375.92923888)(1467.94750057,376.38757221)(1468.8120839,377.30423888)
\curveto(1469.67666724,378.23132221)(1470.1089589,379.55944721)(1470.1089589,381.28861388)
\closepath
}
}
{
\newrgbcolor{curcolor}{0 0 0}
\pscustom[linestyle=none,fillstyle=solid,fillcolor=curcolor]
{
\newpath
\moveto(1486.7183339,382.80423888)
\curveto(1486.7183339,379.66882221)(1486.20791724,376.85111388)(1485.1870839,374.35111388)
\curveto(1484.16625057,371.85111388)(1482.7339589,369.54382221)(1480.8902089,367.42923888)
\lineto(1477.3120839,367.42923888)
\lineto(1477.3120839,367.58548888)
\curveto(1478.11416724,368.30423888)(1478.91625057,369.18444721)(1479.7183339,370.22611388)
\curveto(1480.5308339,371.25736388)(1481.24437557,372.41361388)(1481.8589589,373.69486388)
\curveto(1482.50479224,375.02819721)(1483.0152089,376.43444721)(1483.3902089,377.91361388)
\curveto(1483.77562557,379.39278054)(1483.9683339,381.02298888)(1483.9683339,382.80423888)
\curveto(1483.9683339,384.51257221)(1483.77562557,386.13757221)(1483.3902089,387.67923888)
\curveto(1483.00479224,389.22090554)(1482.49437557,390.63236388)(1481.8589589,391.91361388)
\curveto(1481.2027089,393.23653054)(1480.48916724,394.39278054)(1479.7183339,395.38236388)
\curveto(1478.95791724,396.38236388)(1478.1558339,397.26257221)(1477.3120839,398.02298888)
\lineto(1477.3120839,398.17923888)
\lineto(1480.8902089,398.17923888)
\curveto(1482.7339589,396.06465554)(1484.16625057,393.75736388)(1485.1870839,391.25736388)
\curveto(1486.20791724,388.75736388)(1486.7183339,385.93965554)(1486.7183339,382.80423888)
\closepath
}
}
{
\newrgbcolor{curcolor}{1 1 1}
\pscustom[linestyle=none,fillstyle=solid,fillcolor=curcolor]
{
\newpath
\moveto(235.45682255,566.25774724)
\lineto(642.30823247,566.25774724)
\lineto(642.30823247,346.7418781)
\lineto(235.45682255,346.7418781)
\closepath
}
}
{
\newrgbcolor{curcolor}{0 0 0}
\pscustom[linewidth=2,linecolor=curcolor]
{
\newpath
\moveto(235.45682255,566.25774724)
\lineto(642.30823247,566.25774724)
\lineto(642.30823247,346.7418781)
\lineto(235.45682255,346.7418781)
\closepath
}
}
{
\newrgbcolor{curcolor}{1 0.93333334 0.66666669}
\pscustom[linestyle=none,fillstyle=solid,fillcolor=curcolor]
{
\newpath
\moveto(257.59207646,443.28289372)
\lineto(616.17297856,443.28289372)
\lineto(616.17297856,367.42924077)
\lineto(257.59207646,367.42924077)
\closepath
}
}
{
\newrgbcolor{curcolor}{0 0 0}
\pscustom[linestyle=none,fillstyle=solid,fillcolor=curcolor]
{
\newpath
\moveto(410.23020285,521.41173894)
\curveto(410.23020285,519.96642644)(409.95676535,518.69038477)(409.40989035,517.58361394)
\curveto(408.86301535,516.4768431)(408.12733826,515.56538477)(407.2028591,514.84923894)
\curveto(406.1091091,513.98986394)(404.90468201,513.37788477)(403.58957785,513.01330144)
\curveto(402.28749451,512.6487181)(400.62733826,512.46642644)(398.6091091,512.46642644)
\lineto(388.2966091,512.46642644)
\lineto(388.2966091,541.54845769)
\lineto(396.90989035,541.54845769)
\curveto(399.03228618,541.54845769)(400.62082785,541.47033269)(401.67551535,541.31408269)
\curveto(402.73020285,541.15783269)(403.73931743,540.83231185)(404.7028591,540.33752019)
\curveto(405.77056743,539.77762435)(406.54530701,539.0549681)(407.02707785,538.16955144)
\curveto(407.50884868,537.2971556)(407.7497341,536.24897852)(407.7497341,535.02502019)
\curveto(407.7497341,533.64481185)(407.3981716,532.46642644)(406.6950466,531.48986394)
\curveto(405.9919216,530.52632227)(405.0544216,529.75158269)(403.8825466,529.16564519)
\lineto(403.8825466,529.00939519)
\curveto(405.84869243,528.60574935)(407.3981716,527.73986394)(408.5309841,526.41173894)
\curveto(409.6637966,525.09663477)(410.23020285,523.4299681)(410.23020285,521.41173894)
\closepath
\moveto(403.7262966,534.51720769)
\curveto(403.7262966,535.22033269)(403.6091091,535.8127806)(403.3747341,536.29455144)
\curveto(403.1403591,536.77632227)(402.76275493,537.16694727)(402.2419216,537.46642644)
\curveto(401.62994243,537.81798894)(400.88775493,538.03283269)(400.0153591,538.11095769)
\curveto(399.14296326,538.20210352)(398.0622341,538.24767644)(396.7731716,538.24767644)
\lineto(392.1637966,538.24767644)
\lineto(392.1637966,529.84923894)
\lineto(397.1637966,529.84923894)
\curveto(398.3747341,529.84923894)(399.33827576,529.90783269)(400.0544216,530.02502019)
\curveto(400.77056743,530.15522852)(401.43462993,530.41564519)(402.0466091,530.80627019)
\curveto(402.65858826,531.19689519)(403.08827576,531.69819727)(403.3356716,532.31017644)
\curveto(403.59608826,532.93517644)(403.7262966,533.67085352)(403.7262966,534.51720769)
\closepath
\moveto(406.20676535,521.25548894)
\curveto(406.20676535,522.42736394)(406.0309841,523.35835352)(405.6794216,524.04845769)
\curveto(405.3278591,524.73856185)(404.68983826,525.32449935)(403.7653591,525.80627019)
\curveto(403.1403591,526.13179102)(402.37864035,526.34012435)(401.48020285,526.43127019)
\curveto(400.59478618,526.53543685)(399.51405701,526.58752019)(398.23801535,526.58752019)
\lineto(392.1637966,526.58752019)
\lineto(392.1637966,515.76720769)
\lineto(397.2809841,515.76720769)
\curveto(398.97369243,515.76720769)(400.36041118,515.8518431)(401.44114035,516.02111394)
\curveto(402.52186951,516.2034056)(403.40728618,516.52892644)(404.09739035,516.99767644)
\curveto(404.82655701,517.50548894)(405.36041118,518.08491602)(405.69895285,518.73595769)
\curveto(406.03749451,519.38699935)(406.20676535,520.2268431)(406.20676535,521.25548894)
\closepath
}
}
{
\newrgbcolor{curcolor}{0 0 0}
\pscustom[linestyle=none,fillstyle=solid,fillcolor=curcolor]
{
\newpath
\moveto(419.13645285,512.46642644)
\lineto(415.46457785,512.46642644)
\lineto(415.46457785,542.85705144)
\lineto(419.13645285,542.85705144)
\closepath
}
}
{
\newrgbcolor{curcolor}{0 0 0}
\pscustom[linestyle=none,fillstyle=solid,fillcolor=curcolor]
{
\newpath
\moveto(444.99582785,523.36486394)
\curveto(444.99582785,519.81017644)(444.08436951,517.00418685)(442.26145285,514.94689519)
\curveto(440.43853618,512.88960352)(437.99712993,511.86095769)(434.9372341,511.86095769)
\curveto(431.8512966,511.86095769)(429.39686951,512.88960352)(427.57395285,514.94689519)
\curveto(425.76405701,517.00418685)(424.8591091,519.81017644)(424.8591091,523.36486394)
\curveto(424.8591091,526.91955144)(425.76405701,529.72554102)(427.57395285,531.78283269)
\curveto(429.39686951,533.85314519)(431.8512966,534.88830144)(434.9372341,534.88830144)
\curveto(437.99712993,534.88830144)(440.43853618,533.85314519)(442.26145285,531.78283269)
\curveto(444.08436951,529.72554102)(444.99582785,526.91955144)(444.99582785,523.36486394)
\closepath
\moveto(441.20676535,523.36486394)
\curveto(441.20676535,526.19038477)(440.65337993,528.28673894)(439.5466091,529.65392644)
\curveto(438.43983826,531.03413477)(436.90337993,531.72423894)(434.9372341,531.72423894)
\curveto(432.9450466,531.72423894)(431.39556743,531.03413477)(430.2887966,529.65392644)
\curveto(429.1950466,528.28673894)(428.6481716,526.19038477)(428.6481716,523.36486394)
\curveto(428.6481716,520.63048894)(429.20155701,518.55366602)(430.30832785,517.13439519)
\curveto(431.41509868,515.72814519)(432.95806743,515.02502019)(434.9372341,515.02502019)
\curveto(436.8903591,515.02502019)(438.42030701,515.72163477)(439.52707785,517.11486394)
\curveto(440.64686951,518.52111394)(441.20676535,520.60444727)(441.20676535,523.36486394)
\closepath
}
}
{
\newrgbcolor{curcolor}{0 0 0}
\pscustom[linestyle=none,fillstyle=solid,fillcolor=curcolor]
{
\newpath
\moveto(466.8122341,513.83361394)
\curveto(465.58827576,513.24767644)(464.42291118,512.79194727)(463.31614035,512.46642644)
\curveto(462.22239035,512.1409056)(461.05702576,511.97814519)(459.8200466,511.97814519)
\curveto(458.24452576,511.97814519)(456.79921326,512.20600977)(455.4841091,512.66173894)
\curveto(454.16900493,513.13048894)(453.04270285,513.83361394)(452.10520285,514.77111394)
\curveto(451.15468201,515.70861394)(450.41900493,516.89350977)(449.8981716,518.32580144)
\curveto(449.37733826,519.7580931)(449.1169216,521.43127019)(449.1169216,523.34533269)
\curveto(449.1169216,526.91304102)(450.0934841,529.71252019)(452.0466091,531.74377019)
\curveto(454.01275493,533.77502019)(456.60390076,534.79064519)(459.8200466,534.79064519)
\curveto(461.0700466,534.79064519)(462.29400493,534.61486394)(463.4919216,534.26330144)
\curveto(464.7028591,533.91173894)(465.80962993,533.48205144)(466.8122341,532.97423894)
\lineto(466.8122341,528.89220769)
\lineto(466.6169216,528.89220769)
\curveto(465.49712993,529.76460352)(464.33827576,530.43517644)(463.1403591,530.90392644)
\curveto(461.95546326,531.37267644)(460.7966091,531.60705144)(459.6637966,531.60705144)
\curveto(457.58046326,531.60705144)(455.93332785,530.90392644)(454.72239035,529.49767644)
\curveto(453.52447368,528.10444727)(452.92551535,526.05366602)(452.92551535,523.34533269)
\curveto(452.92551535,520.71512435)(453.51145285,518.69038477)(454.68332785,517.27111394)
\curveto(455.86822368,515.86486394)(457.52837993,515.16173894)(459.6637966,515.16173894)
\curveto(460.4059841,515.16173894)(461.16119243,515.25939519)(461.9294216,515.45470769)
\curveto(462.69765076,515.65002019)(463.38775493,515.90392644)(463.9997341,516.21642644)
\curveto(464.53358826,516.48986394)(465.03489035,516.77632227)(465.50364035,517.07580144)
\curveto(465.97239035,517.38830144)(466.3434841,517.65522852)(466.6169216,517.87658269)
\lineto(466.8122341,517.87658269)
\closepath
}
}
{
\newrgbcolor{curcolor}{0 0 0}
\pscustom[linestyle=none,fillstyle=solid,fillcolor=curcolor]
{
\newpath
\moveto(491.32395285,512.46642644)
\lineto(486.48020285,512.46642644)
\lineto(477.73020285,522.01720769)
\lineto(475.34739035,519.75158269)
\lineto(475.34739035,512.46642644)
\lineto(471.67551535,512.46642644)
\lineto(471.67551535,542.85705144)
\lineto(475.34739035,542.85705144)
\lineto(475.34739035,523.36486394)
\lineto(485.9528591,534.28283269)
\lineto(490.58176535,534.28283269)
\lineto(480.4450466,524.20470769)
\closepath
}
}
{
\newrgbcolor{curcolor}{0 0 0}
\pscustom[linestyle=none,fillstyle=solid,fillcolor=curcolor]
{
\newpath
\moveto(446.44114035,462.46642644)
\lineto(430.69895285,462.46642644)
\lineto(430.69895285,465.43517644)
\lineto(436.75364035,465.43517644)
\lineto(436.75364035,484.92736394)
\lineto(430.69895285,484.92736394)
\lineto(430.69895285,487.58361394)
\curveto(431.51926535,487.58361394)(432.3981716,487.6487181)(433.3356716,487.77892644)
\curveto(434.2731716,487.9221556)(434.98280701,488.12397852)(435.46457785,488.38439519)
\curveto(436.06353618,488.70991602)(436.53228618,489.12007227)(436.87082785,489.61486394)
\curveto(437.22239035,490.12267644)(437.42421326,490.79975977)(437.4762966,491.64611394)
\lineto(440.50364035,491.64611394)
\lineto(440.50364035,465.43517644)
\lineto(446.44114035,465.43517644)
\closepath
}
}
{
\newrgbcolor{curcolor}{0 0 0}
\pscustom[linestyle=none,fillstyle=solid,fillcolor=curcolor]
{
\newpath
\moveto(332.2887781,422.72611564)
\curveto(332.2887781,421.26778231)(332.08044477,419.95528231)(331.6637781,418.78861564)
\curveto(331.2575281,417.62194897)(330.70544477,416.64278231)(330.0075281,415.85111564)
\curveto(329.26794477,415.02819897)(328.45544477,414.40840731)(327.5700281,413.99174064)
\curveto(326.68461143,413.58549064)(325.7106531,413.38236564)(324.6481531,413.38236564)
\curveto(323.65856977,413.38236564)(322.79398643,413.50215731)(322.0544031,413.74174064)
\curveto(321.31481977,413.97090731)(320.5856531,414.28340731)(319.8669031,414.67924064)
\lineto(319.6794031,413.86674064)
\lineto(316.9294031,413.86674064)
\lineto(316.9294031,438.17924064)
\lineto(319.8669031,438.17924064)
\lineto(319.8669031,429.49174064)
\curveto(320.68981977,430.16882397)(321.56481977,430.72090731)(322.4919031,431.14799064)
\curveto(323.41898643,431.58549064)(324.4606531,431.80424064)(325.6169031,431.80424064)
\curveto(327.6794031,431.80424064)(329.3044031,431.01257397)(330.4919031,429.42924064)
\curveto(331.68981977,427.84590731)(332.2887781,425.61153231)(332.2887781,422.72611564)
\closepath
\moveto(329.2575281,422.64799064)
\curveto(329.2575281,424.73132397)(328.9137781,426.30944897)(328.2262781,427.38236564)
\curveto(327.5387781,428.46569897)(326.4294031,429.00736564)(324.8981531,429.00736564)
\curveto(324.04398643,429.00736564)(323.1794031,428.81986564)(322.3044031,428.44486564)
\curveto(321.4294031,428.08028231)(320.6169031,427.60632397)(319.8669031,427.02299064)
\lineto(319.8669031,417.02299064)
\curveto(320.70023643,416.64799064)(321.4137781,416.38757397)(322.0075281,416.24174064)
\curveto(322.61169477,416.09590731)(323.29398643,416.02299064)(324.0544031,416.02299064)
\curveto(325.6794031,416.02299064)(326.95023643,416.55424064)(327.8669031,417.61674064)
\curveto(328.79398643,418.68965731)(329.2575281,420.36674064)(329.2575281,422.64799064)
\closepath
}
}
{
\newrgbcolor{curcolor}{0 0 0}
\pscustom[linestyle=none,fillstyle=solid,fillcolor=curcolor]
{
\newpath
\moveto(339.8356531,413.86674064)
\lineto(336.8981531,413.86674064)
\lineto(336.8981531,438.17924064)
\lineto(339.8356531,438.17924064)
\closepath
}
}
{
\newrgbcolor{curcolor}{0 0 0}
\pscustom[linestyle=none,fillstyle=solid,fillcolor=curcolor]
{
\newpath
\moveto(360.5231531,422.58549064)
\curveto(360.5231531,419.74174064)(359.79398643,417.49694897)(358.3356531,415.85111564)
\curveto(356.87731977,414.20528231)(354.92419477,413.38236564)(352.4762781,413.38236564)
\curveto(350.0075281,413.38236564)(348.04398643,414.20528231)(346.5856531,415.85111564)
\curveto(345.13773643,417.49694897)(344.4137781,419.74174064)(344.4137781,422.58549064)
\curveto(344.4137781,425.42924064)(345.13773643,427.67403231)(346.5856531,429.31986564)
\curveto(348.04398643,430.97611564)(350.0075281,431.80424064)(352.4762781,431.80424064)
\curveto(354.92419477,431.80424064)(356.87731977,430.97611564)(358.3356531,429.31986564)
\curveto(359.79398643,427.67403231)(360.5231531,425.42924064)(360.5231531,422.58549064)
\closepath
\moveto(357.4919031,422.58549064)
\curveto(357.4919031,424.84590731)(357.04919477,426.52299064)(356.1637781,427.61674064)
\curveto(355.27836143,428.72090731)(354.04919477,429.27299064)(352.4762781,429.27299064)
\curveto(350.8825281,429.27299064)(349.64294477,428.72090731)(348.7575281,427.61674064)
\curveto(347.8825281,426.52299064)(347.4450281,424.84590731)(347.4450281,422.58549064)
\curveto(347.4450281,420.39799064)(347.88773643,418.73653231)(348.7731531,417.60111564)
\curveto(349.65856977,416.47611564)(350.89294477,415.91361564)(352.4762781,415.91361564)
\curveto(354.0387781,415.91361564)(355.26273643,416.47090731)(356.1481531,417.58549064)
\curveto(357.04398643,418.71049064)(357.4919031,420.37715731)(357.4919031,422.58549064)
\closepath
}
}
{
\newrgbcolor{curcolor}{0 0 0}
\pscustom[linestyle=none,fillstyle=solid,fillcolor=curcolor]
{
\newpath
\moveto(377.9762781,414.96049064)
\curveto(376.99711143,414.49174064)(376.06481977,414.12715731)(375.1794031,413.86674064)
\curveto(374.3044031,413.60632397)(373.37211143,413.47611564)(372.3825281,413.47611564)
\curveto(371.12211143,413.47611564)(369.96586143,413.65840731)(368.9137781,414.02299064)
\curveto(367.86169477,414.39799064)(366.9606531,414.96049064)(366.2106531,415.71049064)
\curveto(365.45023643,416.46049064)(364.86169477,417.40840731)(364.4450281,418.55424064)
\curveto(364.02836143,419.70007397)(363.8200281,421.03861564)(363.8200281,422.56986564)
\curveto(363.8200281,425.42403231)(364.6012781,427.66361564)(366.1637781,429.28861564)
\curveto(367.73669477,430.91361564)(369.80961143,431.72611564)(372.3825281,431.72611564)
\curveto(373.3825281,431.72611564)(374.36169477,431.58549064)(375.3200281,431.30424064)
\curveto(376.2887781,431.02299064)(377.17419477,430.67924064)(377.9762781,430.27299064)
\lineto(377.9762781,427.00736564)
\lineto(377.8200281,427.00736564)
\curveto(376.92419477,427.70528231)(375.99711143,428.24174064)(375.0387781,428.61674064)
\curveto(374.09086143,428.99174064)(373.1637781,429.17924064)(372.2575281,429.17924064)
\curveto(370.59086143,429.17924064)(369.2731531,428.61674064)(368.3044031,427.49174064)
\curveto(367.34606977,426.37715731)(366.8669031,424.73653231)(366.8669031,422.56986564)
\curveto(366.8669031,420.46569897)(367.3356531,418.84590731)(368.2731531,417.71049064)
\curveto(369.22106977,416.58549064)(370.54919477,416.02299064)(372.2575281,416.02299064)
\curveto(372.8512781,416.02299064)(373.45544477,416.10111564)(374.0700281,416.25736564)
\curveto(374.68461143,416.41361564)(375.23669477,416.61674064)(375.7262781,416.86674064)
\curveto(376.15336143,417.08549064)(376.5544031,417.31465731)(376.9294031,417.55424064)
\curveto(377.3044031,417.80424064)(377.6012781,418.01778231)(377.8200281,418.19486564)
\lineto(377.9762781,418.19486564)
\closepath
}
}
{
\newrgbcolor{curcolor}{0 0 0}
\pscustom[linestyle=none,fillstyle=solid,fillcolor=curcolor]
{
\newpath
\moveto(397.5856531,413.86674064)
\lineto(393.7106531,413.86674064)
\lineto(386.7106531,421.50736564)
\lineto(384.8044031,419.69486564)
\lineto(384.8044031,413.86674064)
\lineto(381.8669031,413.86674064)
\lineto(381.8669031,438.17924064)
\lineto(384.8044031,438.17924064)
\lineto(384.8044031,422.58549064)
\lineto(393.2887781,431.31986564)
\lineto(396.9919031,431.31986564)
\lineto(388.8825281,423.25736564)
\closepath
}
}
{
\newrgbcolor{curcolor}{0 0 0}
\pscustom[linestyle=none,fillstyle=solid,fillcolor=curcolor]
{
\newpath
\moveto(416.0387781,422.80424064)
\curveto(416.0387781,421.38757397)(415.8356531,420.09069897)(415.4294031,418.91361564)
\curveto(415.0231531,417.74694897)(414.45023643,416.75736564)(413.7106531,415.94486564)
\curveto(413.0231531,415.17403231)(412.2106531,414.57507397)(411.2731531,414.14799064)
\curveto(410.34606977,413.73132397)(409.36169477,413.52299064)(408.3200281,413.52299064)
\curveto(407.4137781,413.52299064)(406.59086143,413.62194897)(405.8512781,413.81986564)
\curveto(405.12211143,414.01778231)(404.37731977,414.32507397)(403.6169031,414.74174064)
\lineto(403.6169031,407.42924064)
\lineto(400.6794031,407.42924064)
\lineto(400.6794031,431.31986564)
\lineto(403.6169031,431.31986564)
\lineto(403.6169031,429.49174064)
\curveto(404.3981531,430.14799064)(405.2731531,430.69486564)(406.2419031,431.13236564)
\curveto(407.22106977,431.58028231)(408.26273643,431.80424064)(409.3669031,431.80424064)
\curveto(411.47106977,431.80424064)(413.10648643,431.00736564)(414.2731531,429.41361564)
\curveto(415.45023643,427.83028231)(416.0387781,425.62715731)(416.0387781,422.80424064)
\closepath
\moveto(413.0075281,422.72611564)
\curveto(413.0075281,424.83028231)(412.6481531,426.40319897)(411.9294031,427.44486564)
\curveto(411.2106531,428.48653231)(410.10648643,429.00736564)(408.6169031,429.00736564)
\curveto(407.7731531,429.00736564)(406.92419477,428.82507397)(406.0700281,428.46049064)
\curveto(405.21586143,428.09590731)(404.3981531,427.61674064)(403.6169031,427.02299064)
\lineto(403.6169031,417.13236564)
\curveto(404.45023643,416.75736564)(405.1637781,416.50215731)(405.7575281,416.36674064)
\curveto(406.36169477,416.23132397)(407.04398643,416.16361564)(407.8044031,416.16361564)
\curveto(409.43981977,416.16361564)(410.71586143,416.71569897)(411.6325281,417.81986564)
\curveto(412.54919477,418.92403231)(413.0075281,420.55944897)(413.0075281,422.72611564)
\closepath
}
}
{
\newrgbcolor{curcolor}{0 0 0}
\pscustom[linestyle=none,fillstyle=solid,fillcolor=curcolor]
{
\newpath
\moveto(435.4919031,422.58549064)
\curveto(435.4919031,419.74174064)(434.76273643,417.49694897)(433.3044031,415.85111564)
\curveto(431.84606977,414.20528231)(429.89294477,413.38236564)(427.4450281,413.38236564)
\curveto(424.9762781,413.38236564)(423.01273643,414.20528231)(421.5544031,415.85111564)
\curveto(420.10648643,417.49694897)(419.3825281,419.74174064)(419.3825281,422.58549064)
\curveto(419.3825281,425.42924064)(420.10648643,427.67403231)(421.5544031,429.31986564)
\curveto(423.01273643,430.97611564)(424.9762781,431.80424064)(427.4450281,431.80424064)
\curveto(429.89294477,431.80424064)(431.84606977,430.97611564)(433.3044031,429.31986564)
\curveto(434.76273643,427.67403231)(435.4919031,425.42924064)(435.4919031,422.58549064)
\closepath
\moveto(432.4606531,422.58549064)
\curveto(432.4606531,424.84590731)(432.01794477,426.52299064)(431.1325281,427.61674064)
\curveto(430.24711143,428.72090731)(429.01794477,429.27299064)(427.4450281,429.27299064)
\curveto(425.8512781,429.27299064)(424.61169477,428.72090731)(423.7262781,427.61674064)
\curveto(422.8512781,426.52299064)(422.4137781,424.84590731)(422.4137781,422.58549064)
\curveto(422.4137781,420.39799064)(422.85648643,418.73653231)(423.7419031,417.60111564)
\curveto(424.62731977,416.47611564)(425.86169477,415.91361564)(427.4450281,415.91361564)
\curveto(429.0075281,415.91361564)(430.23148643,416.47090731)(431.1169031,417.58549064)
\curveto(432.01273643,418.71049064)(432.4606531,420.37715731)(432.4606531,422.58549064)
\closepath
}
}
{
\newrgbcolor{curcolor}{0 0 0}
\pscustom[linestyle=none,fillstyle=solid,fillcolor=curcolor]
{
\newpath
\moveto(443.1950281,434.24174064)
\lineto(439.8825281,434.24174064)
\lineto(439.8825281,437.28861564)
\lineto(443.1950281,437.28861564)
\closepath
\moveto(443.0075281,413.86674064)
\lineto(440.0700281,413.86674064)
\lineto(440.0700281,431.31986564)
\lineto(443.0075281,431.31986564)
\closepath
}
}
{
\newrgbcolor{curcolor}{0 0 0}
\pscustom[linestyle=none,fillstyle=solid,fillcolor=curcolor]
{
\newpath
\moveto(463.4137781,413.86674064)
\lineto(460.4762781,413.86674064)
\lineto(460.4762781,423.80424064)
\curveto(460.4762781,424.60632397)(460.4294031,425.35632397)(460.3356531,426.05424064)
\curveto(460.2419031,426.76257397)(460.0700281,427.31465731)(459.8200281,427.71049064)
\curveto(459.55961143,428.14799064)(459.18461143,428.47090731)(458.6950281,428.67924064)
\curveto(458.20544477,428.89799064)(457.5700281,429.00736564)(456.7887781,429.00736564)
\curveto(455.98669477,429.00736564)(455.1481531,428.80944897)(454.2731531,428.41361564)
\curveto(453.3981531,428.01778231)(452.55961143,427.51257397)(451.7575281,426.89799064)
\lineto(451.7575281,413.86674064)
\lineto(448.8200281,413.86674064)
\lineto(448.8200281,431.31986564)
\lineto(451.7575281,431.31986564)
\lineto(451.7575281,429.38236564)
\curveto(452.67419477,430.14278231)(453.62211143,430.73653231)(454.6012781,431.16361564)
\curveto(455.58044477,431.59069897)(456.5856531,431.80424064)(457.6169031,431.80424064)
\curveto(459.50231977,431.80424064)(460.93981977,431.23653231)(461.9294031,430.10111564)
\curveto(462.91898643,428.96569897)(463.4137781,427.33028231)(463.4137781,425.19486564)
\closepath
}
}
{
\newrgbcolor{curcolor}{0 0 0}
\pscustom[linestyle=none,fillstyle=solid,fillcolor=curcolor]
{
\newpath
\moveto(478.1325281,414.02299064)
\curveto(477.58044477,413.87715731)(476.9762781,413.75736564)(476.3200281,413.66361564)
\curveto(475.67419477,413.56986564)(475.09606977,413.52299064)(474.5856531,413.52299064)
\curveto(472.8044031,413.52299064)(471.45023643,414.00215731)(470.5231531,414.96049064)
\curveto(469.59606977,415.91882397)(469.1325281,417.45528231)(469.1325281,419.56986564)
\lineto(469.1325281,428.85111564)
\lineto(467.1481531,428.85111564)
\lineto(467.1481531,431.31986564)
\lineto(469.1325281,431.31986564)
\lineto(469.1325281,436.33549064)
\lineto(472.0700281,436.33549064)
\lineto(472.0700281,431.31986564)
\lineto(478.1325281,431.31986564)
\lineto(478.1325281,428.85111564)
\lineto(472.0700281,428.85111564)
\lineto(472.0700281,420.89799064)
\curveto(472.0700281,419.98132397)(472.09086143,419.26257397)(472.1325281,418.74174064)
\curveto(472.17419477,418.23132397)(472.3200281,417.75215731)(472.5700281,417.30424064)
\curveto(472.79919477,416.88757397)(473.11169477,416.58028231)(473.5075281,416.38236564)
\curveto(473.9137781,416.19486564)(474.52836143,416.10111564)(475.3512781,416.10111564)
\curveto(475.83044477,416.10111564)(476.33044477,416.16882397)(476.8512781,416.30424064)
\curveto(477.37211143,416.45007397)(477.74711143,416.56986564)(477.9762781,416.66361564)
\lineto(478.1325281,416.66361564)
\closepath
}
}
{
\newrgbcolor{curcolor}{0 0 0}
\pscustom[linestyle=none,fillstyle=solid,fillcolor=curcolor]
{
\newpath
\moveto(496.2887781,422.28861564)
\lineto(483.4294031,422.28861564)
\curveto(483.4294031,421.21569897)(483.59086143,420.27819897)(483.9137781,419.47611564)
\curveto(484.23669477,418.68444897)(484.6794031,418.03340731)(485.2419031,417.52299064)
\curveto(485.78356977,417.02299064)(486.42419477,416.64799064)(487.1637781,416.39799064)
\curveto(487.9137781,416.14799064)(488.73669477,416.02299064)(489.6325281,416.02299064)
\curveto(490.8200281,416.02299064)(492.01273643,416.25736564)(493.2106531,416.72611564)
\curveto(494.41898643,417.20528231)(495.27836143,417.67403231)(495.7887781,418.13236564)
\lineto(495.9450281,418.13236564)
\lineto(495.9450281,414.92924064)
\curveto(494.95544477,414.51257397)(493.9450281,414.16361564)(492.9137781,413.88236564)
\curveto(491.8825281,413.60111564)(490.79919477,413.46049064)(489.6637781,413.46049064)
\curveto(486.76794477,413.46049064)(484.5075281,414.24174064)(482.8825281,415.80424064)
\curveto(481.2575281,417.37715731)(480.4450281,419.60632397)(480.4450281,422.49174064)
\curveto(480.4450281,425.34590731)(481.22106977,427.61153231)(482.7731531,429.28861564)
\curveto(484.3356531,430.96569897)(486.38773643,431.80424064)(488.9294031,431.80424064)
\curveto(491.28356977,431.80424064)(493.09606977,431.11674064)(494.3669031,429.74174064)
\curveto(495.6481531,428.36674064)(496.2887781,426.41361564)(496.2887781,423.88236564)
\closepath
\moveto(493.4294031,424.53861564)
\curveto(493.41898643,426.08028231)(493.02836143,427.27299064)(492.2575281,428.11674064)
\curveto(491.49711143,428.96049064)(490.3356531,429.38236564)(488.7731531,429.38236564)
\curveto(487.20023643,429.38236564)(485.9450281,428.91882397)(485.0075281,427.99174064)
\curveto(484.08044477,427.06465731)(483.5544031,425.91361564)(483.4294031,424.53861564)
\closepath
}
}
{
\newrgbcolor{curcolor}{0 0 0}
\pscustom[linestyle=none,fillstyle=solid,fillcolor=curcolor]
{
\newpath
\moveto(511.6325281,428.11674064)
\lineto(511.4762781,428.11674064)
\curveto(511.0387781,428.22090731)(510.61169477,428.29382397)(510.1950281,428.33549064)
\curveto(509.7887781,428.38757397)(509.3044031,428.41361564)(508.7419031,428.41361564)
\curveto(507.8356531,428.41361564)(506.9606531,428.21049064)(506.1169031,427.80424064)
\curveto(505.2731531,427.40840731)(504.4606531,426.89278231)(503.6794031,426.25736564)
\lineto(503.6794031,413.86674064)
\lineto(500.7419031,413.86674064)
\lineto(500.7419031,431.31986564)
\lineto(503.6794031,431.31986564)
\lineto(503.6794031,428.74174064)
\curveto(504.84606977,429.67924064)(505.87211143,430.34069897)(506.7575281,430.72611564)
\curveto(507.65336143,431.12194897)(508.56481977,431.31986564)(509.4919031,431.31986564)
\curveto(510.00231977,431.31986564)(510.37211143,431.30424064)(510.6012781,431.27299064)
\curveto(510.83044477,431.25215731)(511.17419477,431.20528231)(511.6325281,431.13236564)
\closepath
}
}
{
\newrgbcolor{curcolor}{0 0 0}
\pscustom[linestyle=none,fillstyle=solid,fillcolor=curcolor]
{
\newpath
\moveto(523.7419031,407.42924064)
\lineto(520.1637781,407.42924064)
\curveto(518.3200281,409.54382397)(516.88773643,411.85111564)(515.8669031,414.35111564)
\curveto(514.84606977,416.85111564)(514.3356531,419.66882397)(514.3356531,422.80424064)
\curveto(514.3356531,425.93965731)(514.84606977,428.75736564)(515.8669031,431.25736564)
\curveto(516.88773643,433.75736564)(518.3200281,436.06465731)(520.1637781,438.17924064)
\lineto(523.7419031,438.17924064)
\lineto(523.7419031,438.02299064)
\curveto(522.8981531,437.26257397)(522.09086143,436.38236564)(521.3200281,435.38236564)
\curveto(520.55961143,434.39278231)(519.8512781,433.23653231)(519.1950281,431.91361564)
\curveto(518.5700281,430.63236564)(518.05961143,429.22090731)(517.6637781,427.67924064)
\curveto(517.27836143,426.13757397)(517.0856531,424.51257397)(517.0856531,422.80424064)
\curveto(517.0856531,421.02299064)(517.2731531,419.39278231)(517.6481531,417.91361564)
\curveto(518.03356977,416.43444897)(518.54919477,415.02819897)(519.1950281,413.69486564)
\curveto(519.8200281,412.41361564)(520.53356977,411.25736564)(521.3356531,410.22611564)
\curveto(522.13773643,409.18444897)(522.93981977,408.30424064)(523.7419031,407.58549064)
\closepath
}
}
{
\newrgbcolor{curcolor}{0 0 0}
\pscustom[linestyle=none,fillstyle=solid,fillcolor=curcolor]
{
\newpath
\moveto(542.2575281,425.06986564)
\curveto(542.7575281,424.62194897)(543.16898643,424.05944897)(543.4919031,423.38236564)
\curveto(543.81481977,422.70528231)(543.9762781,421.83028231)(543.9762781,420.75736564)
\curveto(543.9762781,419.69486564)(543.78356977,418.72090731)(543.3981531,417.83549064)
\curveto(543.01273643,416.95007397)(542.47106977,416.17924064)(541.7731531,415.52299064)
\curveto(540.9919031,414.79382397)(540.0700281,414.25215731)(539.0075281,413.89799064)
\curveto(537.95544477,413.55424064)(536.79919477,413.38236564)(535.5387781,413.38236564)
\curveto(534.24711143,413.38236564)(532.9762781,413.53861564)(531.7262781,413.85111564)
\curveto(530.4762781,414.15319897)(529.45023643,414.48653231)(528.6481531,414.85111564)
\lineto(528.6481531,418.11674064)
\lineto(528.8825281,418.11674064)
\curveto(529.76794477,417.53340731)(530.80961143,417.04903231)(532.0075281,416.66361564)
\curveto(533.20544477,416.27819897)(534.36169477,416.08549064)(535.4762781,416.08549064)
\curveto(536.1325281,416.08549064)(536.83044477,416.19486564)(537.5700281,416.41361564)
\curveto(538.30961143,416.63236564)(538.90856977,416.95528231)(539.3669031,417.38236564)
\curveto(539.84606977,417.84069897)(540.20023643,418.34590731)(540.4294031,418.89799064)
\curveto(540.66898643,419.45007397)(540.7887781,420.14799064)(540.7887781,420.99174064)
\curveto(540.7887781,421.82507397)(540.65336143,422.51257397)(540.3825281,423.05424064)
\curveto(540.12211143,423.60632397)(539.7575281,424.03861564)(539.2887781,424.35111564)
\curveto(538.8200281,424.67403231)(538.25231977,424.89278231)(537.5856531,425.00736564)
\curveto(536.91898643,425.13236564)(536.20023643,425.19486564)(535.4294031,425.19486564)
\lineto(534.0231531,425.19486564)
\lineto(534.0231531,427.78861564)
\lineto(535.1169031,427.78861564)
\curveto(536.70023643,427.78861564)(537.9606531,428.11674064)(538.8981531,428.77299064)
\curveto(539.84606977,429.43965731)(540.3200281,430.40840731)(540.3200281,431.67924064)
\curveto(540.3200281,432.24174064)(540.20023643,432.73132397)(539.9606531,433.14799064)
\curveto(539.72106977,433.57507397)(539.38773643,433.92403231)(538.9606531,434.19486564)
\curveto(538.51273643,434.46569897)(538.03356977,434.65319897)(537.5231531,434.75736564)
\curveto(537.01273643,434.86153231)(536.43461143,434.91361564)(535.7887781,434.91361564)
\curveto(534.79919477,434.91361564)(533.74711143,434.73653231)(532.6325281,434.38236564)
\curveto(531.51794477,434.02819897)(530.46586143,433.52819897)(529.4762781,432.88236564)
\lineto(529.3200281,432.88236564)
\lineto(529.3200281,436.14799064)
\curveto(530.05961143,436.51257397)(531.04398643,436.84590731)(532.2731531,437.14799064)
\curveto(533.51273643,437.46049064)(534.7106531,437.61674064)(535.8669031,437.61674064)
\curveto(537.00231977,437.61674064)(538.00231977,437.51257397)(538.8669031,437.30424064)
\curveto(539.73148643,437.09590731)(540.51273643,436.76257397)(541.2106531,436.30424064)
\curveto(541.9606531,435.80424064)(542.52836143,435.20007397)(542.9137781,434.49174064)
\curveto(543.29919477,433.78340731)(543.4919031,432.95528231)(543.4919031,432.00736564)
\curveto(543.4919031,430.71569897)(543.03356977,429.58549064)(542.1169031,428.61674064)
\curveto(541.2106531,427.65840731)(540.13773643,427.05424064)(538.8981531,426.80424064)
\lineto(538.8981531,426.58549064)
\curveto(539.3981531,426.50215731)(539.97106977,426.32507397)(540.6169031,426.05424064)
\curveto(541.26273643,425.79382397)(541.80961143,425.46569897)(542.2575281,425.06986564)
\closepath
}
}
{
\newrgbcolor{curcolor}{0 0 0}
\pscustom[linestyle=none,fillstyle=solid,fillcolor=curcolor]
{
\newpath
\moveto(558.0856531,422.80424064)
\curveto(558.0856531,419.66882397)(557.57523643,416.85111564)(556.5544031,414.35111564)
\curveto(555.53356977,411.85111564)(554.1012781,409.54382397)(552.2575281,407.42924064)
\lineto(548.6794031,407.42924064)
\lineto(548.6794031,407.58549064)
\curveto(549.48148643,408.30424064)(550.28356977,409.18444897)(551.0856531,410.22611564)
\curveto(551.8981531,411.25736564)(552.61169477,412.41361564)(553.2262781,413.69486564)
\curveto(553.87211143,415.02819897)(554.3825281,416.43444897)(554.7575281,417.91361564)
\curveto(555.14294477,419.39278231)(555.3356531,421.02299064)(555.3356531,422.80424064)
\curveto(555.3356531,424.51257397)(555.14294477,426.13757397)(554.7575281,427.67924064)
\curveto(554.37211143,429.22090731)(553.86169477,430.63236564)(553.2262781,431.91361564)
\curveto(552.5700281,433.23653231)(551.85648643,434.39278231)(551.0856531,435.38236564)
\curveto(550.32523643,436.38236564)(549.5231531,437.26257397)(548.6794031,438.02299064)
\lineto(548.6794031,438.17924064)
\lineto(552.2575281,438.17924064)
\curveto(554.1012781,436.06465731)(555.53356977,433.75736564)(556.5544031,431.25736564)
\curveto(557.57523643,428.75736564)(558.0856531,425.93965731)(558.0856531,422.80424064)
\closepath
}
}
{
\newrgbcolor{curcolor}{0 0 0}
\pscustom[linestyle=none,fillstyle=solid,fillcolor=curcolor]
{
\newpath
\moveto(332.2887781,382.72611564)
\curveto(332.2887781,381.26778231)(332.08044477,379.95528231)(331.6637781,378.78861564)
\curveto(331.2575281,377.62194897)(330.70544477,376.64278231)(330.0075281,375.85111564)
\curveto(329.26794477,375.02819897)(328.45544477,374.40840731)(327.5700281,373.99174064)
\curveto(326.68461143,373.58549064)(325.7106531,373.38236564)(324.6481531,373.38236564)
\curveto(323.65856977,373.38236564)(322.79398643,373.50215731)(322.0544031,373.74174064)
\curveto(321.31481977,373.97090731)(320.5856531,374.28340731)(319.8669031,374.67924064)
\lineto(319.6794031,373.86674064)
\lineto(316.9294031,373.86674064)
\lineto(316.9294031,398.17924064)
\lineto(319.8669031,398.17924064)
\lineto(319.8669031,389.49174064)
\curveto(320.68981977,390.16882397)(321.56481977,390.72090731)(322.4919031,391.14799064)
\curveto(323.41898643,391.58549064)(324.4606531,391.80424064)(325.6169031,391.80424064)
\curveto(327.6794031,391.80424064)(329.3044031,391.01257397)(330.4919031,389.42924064)
\curveto(331.68981977,387.84590731)(332.2887781,385.61153231)(332.2887781,382.72611564)
\closepath
\moveto(329.2575281,382.64799064)
\curveto(329.2575281,384.73132397)(328.9137781,386.30944897)(328.2262781,387.38236564)
\curveto(327.5387781,388.46569897)(326.4294031,389.00736564)(324.8981531,389.00736564)
\curveto(324.04398643,389.00736564)(323.1794031,388.81986564)(322.3044031,388.44486564)
\curveto(321.4294031,388.08028231)(320.6169031,387.60632397)(319.8669031,387.02299064)
\lineto(319.8669031,377.02299064)
\curveto(320.70023643,376.64799064)(321.4137781,376.38757397)(322.0075281,376.24174064)
\curveto(322.61169477,376.09590731)(323.29398643,376.02299064)(324.0544031,376.02299064)
\curveto(325.6794031,376.02299064)(326.95023643,376.55424064)(327.8669031,377.61674064)
\curveto(328.79398643,378.68965731)(329.2575281,380.36674064)(329.2575281,382.64799064)
\closepath
}
}
{
\newrgbcolor{curcolor}{0 0 0}
\pscustom[linestyle=none,fillstyle=solid,fillcolor=curcolor]
{
\newpath
\moveto(339.8356531,373.86674064)
\lineto(336.8981531,373.86674064)
\lineto(336.8981531,398.17924064)
\lineto(339.8356531,398.17924064)
\closepath
}
}
{
\newrgbcolor{curcolor}{0 0 0}
\pscustom[linestyle=none,fillstyle=solid,fillcolor=curcolor]
{
\newpath
\moveto(360.5231531,382.58549064)
\curveto(360.5231531,379.74174064)(359.79398643,377.49694897)(358.3356531,375.85111564)
\curveto(356.87731977,374.20528231)(354.92419477,373.38236564)(352.4762781,373.38236564)
\curveto(350.0075281,373.38236564)(348.04398643,374.20528231)(346.5856531,375.85111564)
\curveto(345.13773643,377.49694897)(344.4137781,379.74174064)(344.4137781,382.58549064)
\curveto(344.4137781,385.42924064)(345.13773643,387.67403231)(346.5856531,389.31986564)
\curveto(348.04398643,390.97611564)(350.0075281,391.80424064)(352.4762781,391.80424064)
\curveto(354.92419477,391.80424064)(356.87731977,390.97611564)(358.3356531,389.31986564)
\curveto(359.79398643,387.67403231)(360.5231531,385.42924064)(360.5231531,382.58549064)
\closepath
\moveto(357.4919031,382.58549064)
\curveto(357.4919031,384.84590731)(357.04919477,386.52299064)(356.1637781,387.61674064)
\curveto(355.27836143,388.72090731)(354.04919477,389.27299064)(352.4762781,389.27299064)
\curveto(350.8825281,389.27299064)(349.64294477,388.72090731)(348.7575281,387.61674064)
\curveto(347.8825281,386.52299064)(347.4450281,384.84590731)(347.4450281,382.58549064)
\curveto(347.4450281,380.39799064)(347.88773643,378.73653231)(348.7731531,377.60111564)
\curveto(349.65856977,376.47611564)(350.89294477,375.91361564)(352.4762781,375.91361564)
\curveto(354.0387781,375.91361564)(355.26273643,376.47090731)(356.1481531,377.58549064)
\curveto(357.04398643,378.71049064)(357.4919031,380.37715731)(357.4919031,382.58549064)
\closepath
}
}
{
\newrgbcolor{curcolor}{0 0 0}
\pscustom[linestyle=none,fillstyle=solid,fillcolor=curcolor]
{
\newpath
\moveto(377.9762781,374.96049064)
\curveto(376.99711143,374.49174064)(376.06481977,374.12715731)(375.1794031,373.86674064)
\curveto(374.3044031,373.60632397)(373.37211143,373.47611564)(372.3825281,373.47611564)
\curveto(371.12211143,373.47611564)(369.96586143,373.65840731)(368.9137781,374.02299064)
\curveto(367.86169477,374.39799064)(366.9606531,374.96049064)(366.2106531,375.71049064)
\curveto(365.45023643,376.46049064)(364.86169477,377.40840731)(364.4450281,378.55424064)
\curveto(364.02836143,379.70007397)(363.8200281,381.03861564)(363.8200281,382.56986564)
\curveto(363.8200281,385.42403231)(364.6012781,387.66361564)(366.1637781,389.28861564)
\curveto(367.73669477,390.91361564)(369.80961143,391.72611564)(372.3825281,391.72611564)
\curveto(373.3825281,391.72611564)(374.36169477,391.58549064)(375.3200281,391.30424064)
\curveto(376.2887781,391.02299064)(377.17419477,390.67924064)(377.9762781,390.27299064)
\lineto(377.9762781,387.00736564)
\lineto(377.8200281,387.00736564)
\curveto(376.92419477,387.70528231)(375.99711143,388.24174064)(375.0387781,388.61674064)
\curveto(374.09086143,388.99174064)(373.1637781,389.17924064)(372.2575281,389.17924064)
\curveto(370.59086143,389.17924064)(369.2731531,388.61674064)(368.3044031,387.49174064)
\curveto(367.34606977,386.37715731)(366.8669031,384.73653231)(366.8669031,382.56986564)
\curveto(366.8669031,380.46569897)(367.3356531,378.84590731)(368.2731531,377.71049064)
\curveto(369.22106977,376.58549064)(370.54919477,376.02299064)(372.2575281,376.02299064)
\curveto(372.8512781,376.02299064)(373.45544477,376.10111564)(374.0700281,376.25736564)
\curveto(374.68461143,376.41361564)(375.23669477,376.61674064)(375.7262781,376.86674064)
\curveto(376.15336143,377.08549064)(376.5544031,377.31465731)(376.9294031,377.55424064)
\curveto(377.3044031,377.80424064)(377.6012781,378.01778231)(377.8200281,378.19486564)
\lineto(377.9762781,378.19486564)
\closepath
}
}
{
\newrgbcolor{curcolor}{0 0 0}
\pscustom[linestyle=none,fillstyle=solid,fillcolor=curcolor]
{
\newpath
\moveto(397.5856531,373.86674064)
\lineto(393.7106531,373.86674064)
\lineto(386.7106531,381.50736564)
\lineto(384.8044031,379.69486564)
\lineto(384.8044031,373.86674064)
\lineto(381.8669031,373.86674064)
\lineto(381.8669031,398.17924064)
\lineto(384.8044031,398.17924064)
\lineto(384.8044031,382.58549064)
\lineto(393.2887781,391.31986564)
\lineto(396.9919031,391.31986564)
\lineto(388.8825281,383.25736564)
\closepath
}
}
{
\newrgbcolor{curcolor}{0 0 0}
\pscustom[linestyle=none,fillstyle=solid,fillcolor=curcolor]
{
\newpath
\moveto(416.0387781,382.80424064)
\curveto(416.0387781,381.38757397)(415.8356531,380.09069897)(415.4294031,378.91361564)
\curveto(415.0231531,377.74694897)(414.45023643,376.75736564)(413.7106531,375.94486564)
\curveto(413.0231531,375.17403231)(412.2106531,374.57507397)(411.2731531,374.14799064)
\curveto(410.34606977,373.73132397)(409.36169477,373.52299064)(408.3200281,373.52299064)
\curveto(407.4137781,373.52299064)(406.59086143,373.62194897)(405.8512781,373.81986564)
\curveto(405.12211143,374.01778231)(404.37731977,374.32507397)(403.6169031,374.74174064)
\lineto(403.6169031,367.42924064)
\lineto(400.6794031,367.42924064)
\lineto(400.6794031,391.31986564)
\lineto(403.6169031,391.31986564)
\lineto(403.6169031,389.49174064)
\curveto(404.3981531,390.14799064)(405.2731531,390.69486564)(406.2419031,391.13236564)
\curveto(407.22106977,391.58028231)(408.26273643,391.80424064)(409.3669031,391.80424064)
\curveto(411.47106977,391.80424064)(413.10648643,391.00736564)(414.2731531,389.41361564)
\curveto(415.45023643,387.83028231)(416.0387781,385.62715731)(416.0387781,382.80424064)
\closepath
\moveto(413.0075281,382.72611564)
\curveto(413.0075281,384.83028231)(412.6481531,386.40319897)(411.9294031,387.44486564)
\curveto(411.2106531,388.48653231)(410.10648643,389.00736564)(408.6169031,389.00736564)
\curveto(407.7731531,389.00736564)(406.92419477,388.82507397)(406.0700281,388.46049064)
\curveto(405.21586143,388.09590731)(404.3981531,387.61674064)(403.6169031,387.02299064)
\lineto(403.6169031,377.13236564)
\curveto(404.45023643,376.75736564)(405.1637781,376.50215731)(405.7575281,376.36674064)
\curveto(406.36169477,376.23132397)(407.04398643,376.16361564)(407.8044031,376.16361564)
\curveto(409.43981977,376.16361564)(410.71586143,376.71569897)(411.6325281,377.81986564)
\curveto(412.54919477,378.92403231)(413.0075281,380.55944897)(413.0075281,382.72611564)
\closepath
}
}
{
\newrgbcolor{curcolor}{0 0 0}
\pscustom[linestyle=none,fillstyle=solid,fillcolor=curcolor]
{
\newpath
\moveto(435.4919031,382.58549064)
\curveto(435.4919031,379.74174064)(434.76273643,377.49694897)(433.3044031,375.85111564)
\curveto(431.84606977,374.20528231)(429.89294477,373.38236564)(427.4450281,373.38236564)
\curveto(424.9762781,373.38236564)(423.01273643,374.20528231)(421.5544031,375.85111564)
\curveto(420.10648643,377.49694897)(419.3825281,379.74174064)(419.3825281,382.58549064)
\curveto(419.3825281,385.42924064)(420.10648643,387.67403231)(421.5544031,389.31986564)
\curveto(423.01273643,390.97611564)(424.9762781,391.80424064)(427.4450281,391.80424064)
\curveto(429.89294477,391.80424064)(431.84606977,390.97611564)(433.3044031,389.31986564)
\curveto(434.76273643,387.67403231)(435.4919031,385.42924064)(435.4919031,382.58549064)
\closepath
\moveto(432.4606531,382.58549064)
\curveto(432.4606531,384.84590731)(432.01794477,386.52299064)(431.1325281,387.61674064)
\curveto(430.24711143,388.72090731)(429.01794477,389.27299064)(427.4450281,389.27299064)
\curveto(425.8512781,389.27299064)(424.61169477,388.72090731)(423.7262781,387.61674064)
\curveto(422.8512781,386.52299064)(422.4137781,384.84590731)(422.4137781,382.58549064)
\curveto(422.4137781,380.39799064)(422.85648643,378.73653231)(423.7419031,377.60111564)
\curveto(424.62731977,376.47611564)(425.86169477,375.91361564)(427.4450281,375.91361564)
\curveto(429.0075281,375.91361564)(430.23148643,376.47090731)(431.1169031,377.58549064)
\curveto(432.01273643,378.71049064)(432.4606531,380.37715731)(432.4606531,382.58549064)
\closepath
}
}
{
\newrgbcolor{curcolor}{0 0 0}
\pscustom[linestyle=none,fillstyle=solid,fillcolor=curcolor]
{
\newpath
\moveto(443.1950281,394.24174064)
\lineto(439.8825281,394.24174064)
\lineto(439.8825281,397.28861564)
\lineto(443.1950281,397.28861564)
\closepath
\moveto(443.0075281,373.86674064)
\lineto(440.0700281,373.86674064)
\lineto(440.0700281,391.31986564)
\lineto(443.0075281,391.31986564)
\closepath
}
}
{
\newrgbcolor{curcolor}{0 0 0}
\pscustom[linestyle=none,fillstyle=solid,fillcolor=curcolor]
{
\newpath
\moveto(463.4137781,373.86674064)
\lineto(460.4762781,373.86674064)
\lineto(460.4762781,383.80424064)
\curveto(460.4762781,384.60632397)(460.4294031,385.35632397)(460.3356531,386.05424064)
\curveto(460.2419031,386.76257397)(460.0700281,387.31465731)(459.8200281,387.71049064)
\curveto(459.55961143,388.14799064)(459.18461143,388.47090731)(458.6950281,388.67924064)
\curveto(458.20544477,388.89799064)(457.5700281,389.00736564)(456.7887781,389.00736564)
\curveto(455.98669477,389.00736564)(455.1481531,388.80944897)(454.2731531,388.41361564)
\curveto(453.3981531,388.01778231)(452.55961143,387.51257397)(451.7575281,386.89799064)
\lineto(451.7575281,373.86674064)
\lineto(448.8200281,373.86674064)
\lineto(448.8200281,391.31986564)
\lineto(451.7575281,391.31986564)
\lineto(451.7575281,389.38236564)
\curveto(452.67419477,390.14278231)(453.62211143,390.73653231)(454.6012781,391.16361564)
\curveto(455.58044477,391.59069897)(456.5856531,391.80424064)(457.6169031,391.80424064)
\curveto(459.50231977,391.80424064)(460.93981977,391.23653231)(461.9294031,390.10111564)
\curveto(462.91898643,388.96569897)(463.4137781,387.33028231)(463.4137781,385.19486564)
\closepath
}
}
{
\newrgbcolor{curcolor}{0 0 0}
\pscustom[linestyle=none,fillstyle=solid,fillcolor=curcolor]
{
\newpath
\moveto(478.1325281,374.02299064)
\curveto(477.58044477,373.87715731)(476.9762781,373.75736564)(476.3200281,373.66361564)
\curveto(475.67419477,373.56986564)(475.09606977,373.52299064)(474.5856531,373.52299064)
\curveto(472.8044031,373.52299064)(471.45023643,374.00215731)(470.5231531,374.96049064)
\curveto(469.59606977,375.91882397)(469.1325281,377.45528231)(469.1325281,379.56986564)
\lineto(469.1325281,388.85111564)
\lineto(467.1481531,388.85111564)
\lineto(467.1481531,391.31986564)
\lineto(469.1325281,391.31986564)
\lineto(469.1325281,396.33549064)
\lineto(472.0700281,396.33549064)
\lineto(472.0700281,391.31986564)
\lineto(478.1325281,391.31986564)
\lineto(478.1325281,388.85111564)
\lineto(472.0700281,388.85111564)
\lineto(472.0700281,380.89799064)
\curveto(472.0700281,379.98132397)(472.09086143,379.26257397)(472.1325281,378.74174064)
\curveto(472.17419477,378.23132397)(472.3200281,377.75215731)(472.5700281,377.30424064)
\curveto(472.79919477,376.88757397)(473.11169477,376.58028231)(473.5075281,376.38236564)
\curveto(473.9137781,376.19486564)(474.52836143,376.10111564)(475.3512781,376.10111564)
\curveto(475.83044477,376.10111564)(476.33044477,376.16882397)(476.8512781,376.30424064)
\curveto(477.37211143,376.45007397)(477.74711143,376.56986564)(477.9762781,376.66361564)
\lineto(478.1325281,376.66361564)
\closepath
}
}
{
\newrgbcolor{curcolor}{0 0 0}
\pscustom[linestyle=none,fillstyle=solid,fillcolor=curcolor]
{
\newpath
\moveto(496.2887781,382.28861564)
\lineto(483.4294031,382.28861564)
\curveto(483.4294031,381.21569897)(483.59086143,380.27819897)(483.9137781,379.47611564)
\curveto(484.23669477,378.68444897)(484.6794031,378.03340731)(485.2419031,377.52299064)
\curveto(485.78356977,377.02299064)(486.42419477,376.64799064)(487.1637781,376.39799064)
\curveto(487.9137781,376.14799064)(488.73669477,376.02299064)(489.6325281,376.02299064)
\curveto(490.8200281,376.02299064)(492.01273643,376.25736564)(493.2106531,376.72611564)
\curveto(494.41898643,377.20528231)(495.27836143,377.67403231)(495.7887781,378.13236564)
\lineto(495.9450281,378.13236564)
\lineto(495.9450281,374.92924064)
\curveto(494.95544477,374.51257397)(493.9450281,374.16361564)(492.9137781,373.88236564)
\curveto(491.8825281,373.60111564)(490.79919477,373.46049064)(489.6637781,373.46049064)
\curveto(486.76794477,373.46049064)(484.5075281,374.24174064)(482.8825281,375.80424064)
\curveto(481.2575281,377.37715731)(480.4450281,379.60632397)(480.4450281,382.49174064)
\curveto(480.4450281,385.34590731)(481.22106977,387.61153231)(482.7731531,389.28861564)
\curveto(484.3356531,390.96569897)(486.38773643,391.80424064)(488.9294031,391.80424064)
\curveto(491.28356977,391.80424064)(493.09606977,391.11674064)(494.3669031,389.74174064)
\curveto(495.6481531,388.36674064)(496.2887781,386.41361564)(496.2887781,383.88236564)
\closepath
\moveto(493.4294031,384.53861564)
\curveto(493.41898643,386.08028231)(493.02836143,387.27299064)(492.2575281,388.11674064)
\curveto(491.49711143,388.96049064)(490.3356531,389.38236564)(488.7731531,389.38236564)
\curveto(487.20023643,389.38236564)(485.9450281,388.91882397)(485.0075281,387.99174064)
\curveto(484.08044477,387.06465731)(483.5544031,385.91361564)(483.4294031,384.53861564)
\closepath
}
}
{
\newrgbcolor{curcolor}{0 0 0}
\pscustom[linestyle=none,fillstyle=solid,fillcolor=curcolor]
{
\newpath
\moveto(511.6325281,388.11674064)
\lineto(511.4762781,388.11674064)
\curveto(511.0387781,388.22090731)(510.61169477,388.29382397)(510.1950281,388.33549064)
\curveto(509.7887781,388.38757397)(509.3044031,388.41361564)(508.7419031,388.41361564)
\curveto(507.8356531,388.41361564)(506.9606531,388.21049064)(506.1169031,387.80424064)
\curveto(505.2731531,387.40840731)(504.4606531,386.89278231)(503.6794031,386.25736564)
\lineto(503.6794031,373.86674064)
\lineto(500.7419031,373.86674064)
\lineto(500.7419031,391.31986564)
\lineto(503.6794031,391.31986564)
\lineto(503.6794031,388.74174064)
\curveto(504.84606977,389.67924064)(505.87211143,390.34069897)(506.7575281,390.72611564)
\curveto(507.65336143,391.12194897)(508.56481977,391.31986564)(509.4919031,391.31986564)
\curveto(510.00231977,391.31986564)(510.37211143,391.30424064)(510.6012781,391.27299064)
\curveto(510.83044477,391.25215731)(511.17419477,391.20528231)(511.6325281,391.13236564)
\closepath
}
}
{
\newrgbcolor{curcolor}{0 0 0}
\pscustom[linestyle=none,fillstyle=solid,fillcolor=curcolor]
{
\newpath
\moveto(523.7419031,367.42924064)
\lineto(520.1637781,367.42924064)
\curveto(518.3200281,369.54382397)(516.88773643,371.85111564)(515.8669031,374.35111564)
\curveto(514.84606977,376.85111564)(514.3356531,379.66882397)(514.3356531,382.80424064)
\curveto(514.3356531,385.93965731)(514.84606977,388.75736564)(515.8669031,391.25736564)
\curveto(516.88773643,393.75736564)(518.3200281,396.06465731)(520.1637781,398.17924064)
\lineto(523.7419031,398.17924064)
\lineto(523.7419031,398.02299064)
\curveto(522.8981531,397.26257397)(522.09086143,396.38236564)(521.3200281,395.38236564)
\curveto(520.55961143,394.39278231)(519.8512781,393.23653231)(519.1950281,391.91361564)
\curveto(518.5700281,390.63236564)(518.05961143,389.22090731)(517.6637781,387.67924064)
\curveto(517.27836143,386.13757397)(517.0856531,384.51257397)(517.0856531,382.80424064)
\curveto(517.0856531,381.02299064)(517.2731531,379.39278231)(517.6481531,377.91361564)
\curveto(518.03356977,376.43444897)(518.54919477,375.02819897)(519.1950281,373.69486564)
\curveto(519.8200281,372.41361564)(520.53356977,371.25736564)(521.3356531,370.22611564)
\curveto(522.13773643,369.18444897)(522.93981977,368.30424064)(523.7419031,367.58549064)
\closepath
}
}
{
\newrgbcolor{curcolor}{0 0 0}
\pscustom[linestyle=none,fillstyle=solid,fillcolor=curcolor]
{
\newpath
\moveto(544.8356531,380.41361564)
\lineto(541.3825281,380.41361564)
\lineto(541.3825281,373.86674064)
\lineto(538.3825281,373.86674064)
\lineto(538.3825281,380.41361564)
\lineto(527.2419031,380.41361564)
\lineto(527.2419031,384.00736564)
\lineto(538.5075281,397.13236564)
\lineto(541.3825281,397.13236564)
\lineto(541.3825281,382.91361564)
\lineto(544.8356531,382.91361564)
\closepath
\moveto(538.3825281,382.91361564)
\lineto(538.3825281,393.41361564)
\lineto(529.3669031,382.91361564)
\closepath
}
}
{
\newrgbcolor{curcolor}{0 0 0}
\pscustom[linestyle=none,fillstyle=solid,fillcolor=curcolor]
{
\newpath
\moveto(558.0856531,382.80424064)
\curveto(558.0856531,379.66882397)(557.57523643,376.85111564)(556.5544031,374.35111564)
\curveto(555.53356977,371.85111564)(554.1012781,369.54382397)(552.2575281,367.42924064)
\lineto(548.6794031,367.42924064)
\lineto(548.6794031,367.58549064)
\curveto(549.48148643,368.30424064)(550.28356977,369.18444897)(551.0856531,370.22611564)
\curveto(551.8981531,371.25736564)(552.61169477,372.41361564)(553.2262781,373.69486564)
\curveto(553.87211143,375.02819897)(554.3825281,376.43444897)(554.7575281,377.91361564)
\curveto(555.14294477,379.39278231)(555.3356531,381.02299064)(555.3356531,382.80424064)
\curveto(555.3356531,384.51257397)(555.14294477,386.13757397)(554.7575281,387.67924064)
\curveto(554.37211143,389.22090731)(553.86169477,390.63236564)(553.2262781,391.91361564)
\curveto(552.5700281,393.23653231)(551.85648643,394.39278231)(551.0856531,395.38236564)
\curveto(550.32523643,396.38236564)(549.5231531,397.26257397)(548.6794031,398.02299064)
\lineto(548.6794031,398.17924064)
\lineto(552.2575281,398.17924064)
\curveto(554.1012781,396.06465731)(555.53356977,393.75736564)(556.5544031,391.25736564)
\curveto(557.57523643,388.75736564)(558.0856531,385.93965731)(558.0856531,382.80424064)
\closepath
}
}
{
\newrgbcolor{curcolor}{1 1 1}
\pscustom[linestyle=none,fillstyle=solid,fillcolor=curcolor]
{
\newpath
\moveto(699.77320255,911.99963724)
\lineto(1106.62461247,911.99963724)
\lineto(1106.62461247,692.4837681)
\lineto(699.77320255,692.4837681)
\closepath
}
}
{
\newrgbcolor{curcolor}{0 0 0}
\pscustom[linewidth=2,linecolor=curcolor]
{
\newpath
\moveto(699.77320255,911.99963724)
\lineto(1106.62461247,911.99963724)
\lineto(1106.62461247,692.4837681)
\lineto(699.77320255,692.4837681)
\closepath
}
}
{
\newrgbcolor{curcolor}{1 0.93333334 0.66666669}
\pscustom[linestyle=none,fillstyle=solid,fillcolor=curcolor]
{
\newpath
\moveto(721.90845646,789.02478372)
\lineto(1080.48935856,789.02478372)
\lineto(1080.48935856,713.17113077)
\lineto(721.90845646,713.17113077)
\closepath
}
}
{
\newrgbcolor{curcolor}{0 0 0}
\pscustom[linestyle=none,fillstyle=solid,fillcolor=curcolor]
{
\newpath
\moveto(797.98015894,768.46800618)
\curveto(797.98015894,767.00967284)(797.77182561,765.69717284)(797.35515894,764.53050618)
\curveto(796.94890894,763.36383951)(796.39682561,762.38467284)(795.69890894,761.59300618)
\curveto(794.95932561,760.77008951)(794.14682561,760.15029784)(793.26140894,759.73363118)
\curveto(792.37599227,759.32738118)(791.40203394,759.12425618)(790.33953394,759.12425618)
\curveto(789.34995061,759.12425618)(788.48536727,759.24404784)(787.74578394,759.48363118)
\curveto(787.00620061,759.71279784)(786.27703394,760.02529784)(785.55828394,760.42113118)
\lineto(785.37078394,759.60863118)
\lineto(782.62078394,759.60863118)
\lineto(782.62078394,783.92113118)
\lineto(785.55828394,783.92113118)
\lineto(785.55828394,775.23363118)
\curveto(786.38120061,775.91071451)(787.25620061,776.46279784)(788.18328394,776.88988118)
\curveto(789.11036727,777.32738118)(790.15203394,777.54613118)(791.30828394,777.54613118)
\curveto(793.37078394,777.54613118)(794.99578394,776.75446451)(796.18328394,775.17113118)
\curveto(797.38120061,773.58779784)(797.98015894,771.35342284)(797.98015894,768.46800618)
\closepath
\moveto(794.94890894,768.38988118)
\curveto(794.94890894,770.47321451)(794.60515894,772.05133951)(793.91765894,773.12425618)
\curveto(793.23015894,774.20758951)(792.12078394,774.74925618)(790.58953394,774.74925618)
\curveto(789.73536727,774.74925618)(788.87078394,774.56175618)(787.99578394,774.18675618)
\curveto(787.12078394,773.82217284)(786.30828394,773.34821451)(785.55828394,772.76488118)
\lineto(785.55828394,762.76488118)
\curveto(786.39161727,762.38988118)(787.10515894,762.12946451)(787.69890894,761.98363118)
\curveto(788.30307561,761.83779784)(788.98536727,761.76488118)(789.74578394,761.76488118)
\curveto(791.37078394,761.76488118)(792.64161727,762.29613118)(793.55828394,763.35863118)
\curveto(794.48536727,764.43154784)(794.94890894,766.10863118)(794.94890894,768.38988118)
\closepath
}
}
{
\newrgbcolor{curcolor}{0 0 0}
\pscustom[linestyle=none,fillstyle=solid,fillcolor=curcolor]
{
\newpath
\moveto(805.52703394,759.60863118)
\lineto(802.58953394,759.60863118)
\lineto(802.58953394,783.92113118)
\lineto(805.52703394,783.92113118)
\closepath
}
}
{
\newrgbcolor{curcolor}{0 0 0}
\pscustom[linestyle=none,fillstyle=solid,fillcolor=curcolor]
{
\newpath
\moveto(826.21453394,768.32738118)
\curveto(826.21453394,765.48363118)(825.48536727,763.23883951)(824.02703394,761.59300618)
\curveto(822.56870061,759.94717284)(820.61557561,759.12425618)(818.16765894,759.12425618)
\curveto(815.69890894,759.12425618)(813.73536727,759.94717284)(812.27703394,761.59300618)
\curveto(810.82911727,763.23883951)(810.10515894,765.48363118)(810.10515894,768.32738118)
\curveto(810.10515894,771.17113118)(810.82911727,773.41592284)(812.27703394,775.06175618)
\curveto(813.73536727,776.71800618)(815.69890894,777.54613118)(818.16765894,777.54613118)
\curveto(820.61557561,777.54613118)(822.56870061,776.71800618)(824.02703394,775.06175618)
\curveto(825.48536727,773.41592284)(826.21453394,771.17113118)(826.21453394,768.32738118)
\closepath
\moveto(823.18328394,768.32738118)
\curveto(823.18328394,770.58779784)(822.74057561,772.26488118)(821.85515894,773.35863118)
\curveto(820.96974227,774.46279784)(819.74057561,775.01488118)(818.16765894,775.01488118)
\curveto(816.57390894,775.01488118)(815.33432561,774.46279784)(814.44890894,773.35863118)
\curveto(813.57390894,772.26488118)(813.13640894,770.58779784)(813.13640894,768.32738118)
\curveto(813.13640894,766.13988118)(813.57911727,764.47842284)(814.46453394,763.34300618)
\curveto(815.34995061,762.21800618)(816.58432561,761.65550618)(818.16765894,761.65550618)
\curveto(819.73015894,761.65550618)(820.95411727,762.21279784)(821.83953394,763.32738118)
\curveto(822.73536727,764.45238118)(823.18328394,766.11904784)(823.18328394,768.32738118)
\closepath
}
}
{
\newrgbcolor{curcolor}{0 0 0}
\pscustom[linestyle=none,fillstyle=solid,fillcolor=curcolor]
{
\newpath
\moveto(843.66765894,760.70238118)
\curveto(842.68849227,760.23363118)(841.75620061,759.86904784)(840.87078394,759.60863118)
\curveto(839.99578394,759.34821451)(839.06349227,759.21800618)(838.07390894,759.21800618)
\curveto(836.81349227,759.21800618)(835.65724227,759.40029784)(834.60515894,759.76488118)
\curveto(833.55307561,760.13988118)(832.65203394,760.70238118)(831.90203394,761.45238118)
\curveto(831.14161727,762.20238118)(830.55307561,763.15029784)(830.13640894,764.29613118)
\curveto(829.71974227,765.44196451)(829.51140894,766.78050618)(829.51140894,768.31175618)
\curveto(829.51140894,771.16592284)(830.29265894,773.40550618)(831.85515894,775.03050618)
\curveto(833.42807561,776.65550618)(835.50099227,777.46800618)(838.07390894,777.46800618)
\curveto(839.07390894,777.46800618)(840.05307561,777.32738118)(841.01140894,777.04613118)
\curveto(841.98015894,776.76488118)(842.86557561,776.42113118)(843.66765894,776.01488118)
\lineto(843.66765894,772.74925618)
\lineto(843.51140894,772.74925618)
\curveto(842.61557561,773.44717284)(841.68849227,773.98363118)(840.73015894,774.35863118)
\curveto(839.78224227,774.73363118)(838.85515894,774.92113118)(837.94890894,774.92113118)
\curveto(836.28224227,774.92113118)(834.96453394,774.35863118)(833.99578394,773.23363118)
\curveto(833.03745061,772.11904784)(832.55828394,770.47842284)(832.55828394,768.31175618)
\curveto(832.55828394,766.20758951)(833.02703394,764.58779784)(833.96453394,763.45238118)
\curveto(834.91245061,762.32738118)(836.24057561,761.76488118)(837.94890894,761.76488118)
\curveto(838.54265894,761.76488118)(839.14682561,761.84300618)(839.76140894,761.99925618)
\curveto(840.37599227,762.15550618)(840.92807561,762.35863118)(841.41765894,762.60863118)
\curveto(841.84474227,762.82738118)(842.24578394,763.05654784)(842.62078394,763.29613118)
\curveto(842.99578394,763.54613118)(843.29265894,763.75967284)(843.51140894,763.93675618)
\lineto(843.66765894,763.93675618)
\closepath
}
}
{
\newrgbcolor{curcolor}{0 0 0}
\pscustom[linestyle=none,fillstyle=solid,fillcolor=curcolor]
{
\newpath
\moveto(863.27703394,759.60863118)
\lineto(859.40203394,759.60863118)
\lineto(852.40203394,767.24925618)
\lineto(850.49578394,765.43675618)
\lineto(850.49578394,759.60863118)
\lineto(847.55828394,759.60863118)
\lineto(847.55828394,783.92113118)
\lineto(850.49578394,783.92113118)
\lineto(850.49578394,768.32738118)
\lineto(858.98015894,777.06175618)
\lineto(862.68328394,777.06175618)
\lineto(854.57390894,768.99925618)
\closepath
}
}
{
\newrgbcolor{curcolor}{0 0 0}
\pscustom[linestyle=none,fillstyle=solid,fillcolor=curcolor]
{
\newpath
\moveto(881.73015894,768.54613118)
\curveto(881.73015894,767.12946451)(881.52703394,765.83258951)(881.12078394,764.65550618)
\curveto(880.71453394,763.48883951)(880.14161727,762.49925618)(879.40203394,761.68675618)
\curveto(878.71453394,760.91592284)(877.90203394,760.31696451)(876.96453394,759.88988118)
\curveto(876.03745061,759.47321451)(875.05307561,759.26488118)(874.01140894,759.26488118)
\curveto(873.10515894,759.26488118)(872.28224227,759.36383951)(871.54265894,759.56175618)
\curveto(870.81349227,759.75967284)(870.06870061,760.06696451)(869.30828394,760.48363118)
\lineto(869.30828394,753.17113118)
\lineto(866.37078394,753.17113118)
\lineto(866.37078394,777.06175618)
\lineto(869.30828394,777.06175618)
\lineto(869.30828394,775.23363118)
\curveto(870.08953394,775.88988118)(870.96453394,776.43675618)(871.93328394,776.87425618)
\curveto(872.91245061,777.32217284)(873.95411727,777.54613118)(875.05828394,777.54613118)
\curveto(877.16245061,777.54613118)(878.79786727,776.74925618)(879.96453394,775.15550618)
\curveto(881.14161727,773.57217284)(881.73015894,771.36904784)(881.73015894,768.54613118)
\closepath
\moveto(878.69890894,768.46800618)
\curveto(878.69890894,770.57217284)(878.33953394,772.14508951)(877.62078394,773.18675618)
\curveto(876.90203394,774.22842284)(875.79786727,774.74925618)(874.30828394,774.74925618)
\curveto(873.46453394,774.74925618)(872.61557561,774.56696451)(871.76140894,774.20238118)
\curveto(870.90724227,773.83779784)(870.08953394,773.35863118)(869.30828394,772.76488118)
\lineto(869.30828394,762.87425618)
\curveto(870.14161727,762.49925618)(870.85515894,762.24404784)(871.44890894,762.10863118)
\curveto(872.05307561,761.97321451)(872.73536727,761.90550618)(873.49578394,761.90550618)
\curveto(875.13120061,761.90550618)(876.40724227,762.45758951)(877.32390894,763.56175618)
\curveto(878.24057561,764.66592284)(878.69890894,766.30133951)(878.69890894,768.46800618)
\closepath
}
}
{
\newrgbcolor{curcolor}{0 0 0}
\pscustom[linestyle=none,fillstyle=solid,fillcolor=curcolor]
{
\newpath
\moveto(901.18328394,768.32738118)
\curveto(901.18328394,765.48363118)(900.45411727,763.23883951)(898.99578394,761.59300618)
\curveto(897.53745061,759.94717284)(895.58432561,759.12425618)(893.13640894,759.12425618)
\curveto(890.66765894,759.12425618)(888.70411727,759.94717284)(887.24578394,761.59300618)
\curveto(885.79786727,763.23883951)(885.07390894,765.48363118)(885.07390894,768.32738118)
\curveto(885.07390894,771.17113118)(885.79786727,773.41592284)(887.24578394,775.06175618)
\curveto(888.70411727,776.71800618)(890.66765894,777.54613118)(893.13640894,777.54613118)
\curveto(895.58432561,777.54613118)(897.53745061,776.71800618)(898.99578394,775.06175618)
\curveto(900.45411727,773.41592284)(901.18328394,771.17113118)(901.18328394,768.32738118)
\closepath
\moveto(898.15203394,768.32738118)
\curveto(898.15203394,770.58779784)(897.70932561,772.26488118)(896.82390894,773.35863118)
\curveto(895.93849227,774.46279784)(894.70932561,775.01488118)(893.13640894,775.01488118)
\curveto(891.54265894,775.01488118)(890.30307561,774.46279784)(889.41765894,773.35863118)
\curveto(888.54265894,772.26488118)(888.10515894,770.58779784)(888.10515894,768.32738118)
\curveto(888.10515894,766.13988118)(888.54786727,764.47842284)(889.43328394,763.34300618)
\curveto(890.31870061,762.21800618)(891.55307561,761.65550618)(893.13640894,761.65550618)
\curveto(894.69890894,761.65550618)(895.92286727,762.21279784)(896.80828394,763.32738118)
\curveto(897.70411727,764.45238118)(898.15203394,766.11904784)(898.15203394,768.32738118)
\closepath
}
}
{
\newrgbcolor{curcolor}{0 0 0}
\pscustom[linestyle=none,fillstyle=solid,fillcolor=curcolor]
{
\newpath
\moveto(908.88640894,779.98363118)
\lineto(905.57390894,779.98363118)
\lineto(905.57390894,783.03050618)
\lineto(908.88640894,783.03050618)
\closepath
\moveto(908.69890894,759.60863118)
\lineto(905.76140894,759.60863118)
\lineto(905.76140894,777.06175618)
\lineto(908.69890894,777.06175618)
\closepath
}
}
{
\newrgbcolor{curcolor}{0 0 0}
\pscustom[linestyle=none,fillstyle=solid,fillcolor=curcolor]
{
\newpath
\moveto(929.10515894,759.60863118)
\lineto(926.16765894,759.60863118)
\lineto(926.16765894,769.54613118)
\curveto(926.16765894,770.34821451)(926.12078394,771.09821451)(926.02703394,771.79613118)
\curveto(925.93328394,772.50446451)(925.76140894,773.05654784)(925.51140894,773.45238118)
\curveto(925.25099227,773.88988118)(924.87599227,774.21279784)(924.38640894,774.42113118)
\curveto(923.89682561,774.63988118)(923.26140894,774.74925618)(922.48015894,774.74925618)
\curveto(921.67807561,774.74925618)(920.83953394,774.55133951)(919.96453394,774.15550618)
\curveto(919.08953394,773.75967284)(918.25099227,773.25446451)(917.44890894,772.63988118)
\lineto(917.44890894,759.60863118)
\lineto(914.51140894,759.60863118)
\lineto(914.51140894,777.06175618)
\lineto(917.44890894,777.06175618)
\lineto(917.44890894,775.12425618)
\curveto(918.36557561,775.88467284)(919.31349227,776.47842284)(920.29265894,776.90550618)
\curveto(921.27182561,777.33258951)(922.27703394,777.54613118)(923.30828394,777.54613118)
\curveto(925.19370061,777.54613118)(926.63120061,776.97842284)(927.62078394,775.84300618)
\curveto(928.61036727,774.70758951)(929.10515894,773.07217284)(929.10515894,770.93675618)
\closepath
}
}
{
\newrgbcolor{curcolor}{0 0 0}
\pscustom[linestyle=none,fillstyle=solid,fillcolor=curcolor]
{
\newpath
\moveto(943.82390894,759.76488118)
\curveto(943.27182561,759.61904784)(942.66765894,759.49925618)(942.01140894,759.40550618)
\curveto(941.36557561,759.31175618)(940.78745061,759.26488118)(940.27703394,759.26488118)
\curveto(938.49578394,759.26488118)(937.14161727,759.74404784)(936.21453394,760.70238118)
\curveto(935.28745061,761.66071451)(934.82390894,763.19717284)(934.82390894,765.31175618)
\lineto(934.82390894,774.59300618)
\lineto(932.83953394,774.59300618)
\lineto(932.83953394,777.06175618)
\lineto(934.82390894,777.06175618)
\lineto(934.82390894,782.07738118)
\lineto(937.76140894,782.07738118)
\lineto(937.76140894,777.06175618)
\lineto(943.82390894,777.06175618)
\lineto(943.82390894,774.59300618)
\lineto(937.76140894,774.59300618)
\lineto(937.76140894,766.63988118)
\curveto(937.76140894,765.72321451)(937.78224227,765.00446451)(937.82390894,764.48363118)
\curveto(937.86557561,763.97321451)(938.01140894,763.49404784)(938.26140894,763.04613118)
\curveto(938.49057561,762.62946451)(938.80307561,762.32217284)(939.19890894,762.12425618)
\curveto(939.60515894,761.93675618)(940.21974227,761.84300618)(941.04265894,761.84300618)
\curveto(941.52182561,761.84300618)(942.02182561,761.91071451)(942.54265894,762.04613118)
\curveto(943.06349227,762.19196451)(943.43849227,762.31175618)(943.66765894,762.40550618)
\lineto(943.82390894,762.40550618)
\closepath
}
}
{
\newrgbcolor{curcolor}{0 0 0}
\pscustom[linestyle=none,fillstyle=solid,fillcolor=curcolor]
{
\newpath
\moveto(961.98015894,768.03050618)
\lineto(949.12078394,768.03050618)
\curveto(949.12078394,766.95758951)(949.28224227,766.02008951)(949.60515894,765.21800618)
\curveto(949.92807561,764.42633951)(950.37078394,763.77529784)(950.93328394,763.26488118)
\curveto(951.47495061,762.76488118)(952.11557561,762.38988118)(952.85515894,762.13988118)
\curveto(953.60515894,761.88988118)(954.42807561,761.76488118)(955.32390894,761.76488118)
\curveto(956.51140894,761.76488118)(957.70411727,761.99925618)(958.90203394,762.46800618)
\curveto(960.11036727,762.94717284)(960.96974227,763.41592284)(961.48015894,763.87425618)
\lineto(961.63640894,763.87425618)
\lineto(961.63640894,760.67113118)
\curveto(960.64682561,760.25446451)(959.63640894,759.90550618)(958.60515894,759.62425618)
\curveto(957.57390894,759.34300618)(956.49057561,759.20238118)(955.35515894,759.20238118)
\curveto(952.45932561,759.20238118)(950.19890894,759.98363118)(948.57390894,761.54613118)
\curveto(946.94890894,763.11904784)(946.13640894,765.34821451)(946.13640894,768.23363118)
\curveto(946.13640894,771.08779784)(946.91245061,773.35342284)(948.46453394,775.03050618)
\curveto(950.02703394,776.70758951)(952.07911727,777.54613118)(954.62078394,777.54613118)
\curveto(956.97495061,777.54613118)(958.78745061,776.85863118)(960.05828394,775.48363118)
\curveto(961.33953394,774.10863118)(961.98015894,772.15550618)(961.98015894,769.62425618)
\closepath
\moveto(959.12078394,770.28050618)
\curveto(959.11036727,771.82217284)(958.71974227,773.01488118)(957.94890894,773.85863118)
\curveto(957.18849227,774.70238118)(956.02703394,775.12425618)(954.46453394,775.12425618)
\curveto(952.89161727,775.12425618)(951.63640894,774.66071451)(950.69890894,773.73363118)
\curveto(949.77182561,772.80654784)(949.24578394,771.65550618)(949.12078394,770.28050618)
\closepath
}
}
{
\newrgbcolor{curcolor}{0 0 0}
\pscustom[linestyle=none,fillstyle=solid,fillcolor=curcolor]
{
\newpath
\moveto(977.32390894,773.85863118)
\lineto(977.16765894,773.85863118)
\curveto(976.73015894,773.96279784)(976.30307561,774.03571451)(975.88640894,774.07738118)
\curveto(975.48015894,774.12946451)(974.99578394,774.15550618)(974.43328394,774.15550618)
\curveto(973.52703394,774.15550618)(972.65203394,773.95238118)(971.80828394,773.54613118)
\curveto(970.96453394,773.15029784)(970.15203394,772.63467284)(969.37078394,771.99925618)
\lineto(969.37078394,759.60863118)
\lineto(966.43328394,759.60863118)
\lineto(966.43328394,777.06175618)
\lineto(969.37078394,777.06175618)
\lineto(969.37078394,774.48363118)
\curveto(970.53745061,775.42113118)(971.56349227,776.08258951)(972.44890894,776.46800618)
\curveto(973.34474227,776.86383951)(974.25620061,777.06175618)(975.18328394,777.06175618)
\curveto(975.69370061,777.06175618)(976.06349227,777.04613118)(976.29265894,777.01488118)
\curveto(976.52182561,776.99404784)(976.86557561,776.94717284)(977.32390894,776.87425618)
\closepath
}
}
{
\newrgbcolor{curcolor}{0 0 0}
\pscustom[linestyle=none,fillstyle=solid,fillcolor=curcolor]
{
\newpath
\moveto(989.43328394,753.17113118)
\lineto(985.85515894,753.17113118)
\curveto(984.01140894,755.28571451)(982.57911727,757.59300618)(981.55828394,760.09300618)
\curveto(980.53745061,762.59300618)(980.02703394,765.41071451)(980.02703394,768.54613118)
\curveto(980.02703394,771.68154784)(980.53745061,774.49925618)(981.55828394,776.99925618)
\curveto(982.57911727,779.49925618)(984.01140894,781.80654784)(985.85515894,783.92113118)
\lineto(989.43328394,783.92113118)
\lineto(989.43328394,783.76488118)
\curveto(988.58953394,783.00446451)(987.78224227,782.12425618)(987.01140894,781.12425618)
\curveto(986.25099227,780.13467284)(985.54265894,778.97842284)(984.88640894,777.65550618)
\curveto(984.26140894,776.37425618)(983.75099227,774.96279784)(983.35515894,773.42113118)
\curveto(982.96974227,771.87946451)(982.77703394,770.25446451)(982.77703394,768.54613118)
\curveto(982.77703394,766.76488118)(982.96453394,765.13467284)(983.33953394,763.65550618)
\curveto(983.72495061,762.17633951)(984.24057561,760.77008951)(984.88640894,759.43675618)
\curveto(985.51140894,758.15550618)(986.22495061,756.99925618)(987.02703394,755.96800618)
\curveto(987.82911727,754.92633951)(988.63120061,754.04613118)(989.43328394,753.32738118)
\closepath
}
}
{
\newrgbcolor{curcolor}{0 0 0}
\pscustom[linestyle=none,fillstyle=solid,fillcolor=curcolor]
{
\newpath
\moveto(1008.66765894,759.60863118)
\lineto(996.07390894,759.60863118)
\lineto(996.07390894,761.98363118)
\lineto(1000.91765894,761.98363118)
\lineto(1000.91765894,777.57738118)
\lineto(996.07390894,777.57738118)
\lineto(996.07390894,779.70238118)
\curveto(996.73015894,779.70238118)(997.43328394,779.75446451)(998.18328394,779.85863118)
\curveto(998.93328394,779.97321451)(999.50099227,780.13467284)(999.88640894,780.34300618)
\curveto(1000.36557561,780.60342284)(1000.74057561,780.93154784)(1001.01140894,781.32738118)
\curveto(1001.29265894,781.73363118)(1001.45411727,782.27529784)(1001.49578394,782.95238118)
\lineto(1003.91765894,782.95238118)
\lineto(1003.91765894,761.98363118)
\lineto(1008.66765894,761.98363118)
\closepath
}
}
{
\newrgbcolor{curcolor}{0 0 0}
\pscustom[linestyle=none,fillstyle=solid,fillcolor=curcolor]
{
\newpath
\moveto(1023.77703394,768.54613118)
\curveto(1023.77703394,765.41071451)(1023.26661727,762.59300618)(1022.24578394,760.09300618)
\curveto(1021.22495061,757.59300618)(1019.79265894,755.28571451)(1017.94890894,753.17113118)
\lineto(1014.37078394,753.17113118)
\lineto(1014.37078394,753.32738118)
\curveto(1015.17286727,754.04613118)(1015.97495061,754.92633951)(1016.77703394,755.96800618)
\curveto(1017.58953394,756.99925618)(1018.30307561,758.15550618)(1018.91765894,759.43675618)
\curveto(1019.56349227,760.77008951)(1020.07390894,762.17633951)(1020.44890894,763.65550618)
\curveto(1020.83432561,765.13467284)(1021.02703394,766.76488118)(1021.02703394,768.54613118)
\curveto(1021.02703394,770.25446451)(1020.83432561,771.87946451)(1020.44890894,773.42113118)
\curveto(1020.06349227,774.96279784)(1019.55307561,776.37425618)(1018.91765894,777.65550618)
\curveto(1018.26140894,778.97842284)(1017.54786727,780.13467284)(1016.77703394,781.12425618)
\curveto(1016.01661727,782.12425618)(1015.21453394,783.00446451)(1014.37078394,783.76488118)
\lineto(1014.37078394,783.92113118)
\lineto(1017.94890894,783.92113118)
\curveto(1019.79265894,781.80654784)(1021.22495061,779.49925618)(1022.24578394,776.99925618)
\curveto(1023.26661727,774.49925618)(1023.77703394,771.68154784)(1023.77703394,768.54613118)
\closepath
}
}
{
\newrgbcolor{curcolor}{0 0 0}
\pscustom[linestyle=none,fillstyle=solid,fillcolor=curcolor]
{
\newpath
\moveto(797.98015894,728.46800618)
\curveto(797.98015894,727.00967284)(797.77182561,725.69717284)(797.35515894,724.53050618)
\curveto(796.94890894,723.36383951)(796.39682561,722.38467284)(795.69890894,721.59300618)
\curveto(794.95932561,720.77008951)(794.14682561,720.15029784)(793.26140894,719.73363118)
\curveto(792.37599227,719.32738118)(791.40203394,719.12425618)(790.33953394,719.12425618)
\curveto(789.34995061,719.12425618)(788.48536727,719.24404784)(787.74578394,719.48363118)
\curveto(787.00620061,719.71279784)(786.27703394,720.02529784)(785.55828394,720.42113118)
\lineto(785.37078394,719.60863118)
\lineto(782.62078394,719.60863118)
\lineto(782.62078394,743.92113118)
\lineto(785.55828394,743.92113118)
\lineto(785.55828394,735.23363118)
\curveto(786.38120061,735.91071451)(787.25620061,736.46279784)(788.18328394,736.88988118)
\curveto(789.11036727,737.32738118)(790.15203394,737.54613118)(791.30828394,737.54613118)
\curveto(793.37078394,737.54613118)(794.99578394,736.75446451)(796.18328394,735.17113118)
\curveto(797.38120061,733.58779784)(797.98015894,731.35342284)(797.98015894,728.46800618)
\closepath
\moveto(794.94890894,728.38988118)
\curveto(794.94890894,730.47321451)(794.60515894,732.05133951)(793.91765894,733.12425618)
\curveto(793.23015894,734.20758951)(792.12078394,734.74925618)(790.58953394,734.74925618)
\curveto(789.73536727,734.74925618)(788.87078394,734.56175618)(787.99578394,734.18675618)
\curveto(787.12078394,733.82217284)(786.30828394,733.34821451)(785.55828394,732.76488118)
\lineto(785.55828394,722.76488118)
\curveto(786.39161727,722.38988118)(787.10515894,722.12946451)(787.69890894,721.98363118)
\curveto(788.30307561,721.83779784)(788.98536727,721.76488118)(789.74578394,721.76488118)
\curveto(791.37078394,721.76488118)(792.64161727,722.29613118)(793.55828394,723.35863118)
\curveto(794.48536727,724.43154784)(794.94890894,726.10863118)(794.94890894,728.38988118)
\closepath
}
}
{
\newrgbcolor{curcolor}{0 0 0}
\pscustom[linestyle=none,fillstyle=solid,fillcolor=curcolor]
{
\newpath
\moveto(805.52703394,719.60863118)
\lineto(802.58953394,719.60863118)
\lineto(802.58953394,743.92113118)
\lineto(805.52703394,743.92113118)
\closepath
}
}
{
\newrgbcolor{curcolor}{0 0 0}
\pscustom[linestyle=none,fillstyle=solid,fillcolor=curcolor]
{
\newpath
\moveto(826.21453394,728.32738118)
\curveto(826.21453394,725.48363118)(825.48536727,723.23883951)(824.02703394,721.59300618)
\curveto(822.56870061,719.94717284)(820.61557561,719.12425618)(818.16765894,719.12425618)
\curveto(815.69890894,719.12425618)(813.73536727,719.94717284)(812.27703394,721.59300618)
\curveto(810.82911727,723.23883951)(810.10515894,725.48363118)(810.10515894,728.32738118)
\curveto(810.10515894,731.17113118)(810.82911727,733.41592284)(812.27703394,735.06175618)
\curveto(813.73536727,736.71800618)(815.69890894,737.54613118)(818.16765894,737.54613118)
\curveto(820.61557561,737.54613118)(822.56870061,736.71800618)(824.02703394,735.06175618)
\curveto(825.48536727,733.41592284)(826.21453394,731.17113118)(826.21453394,728.32738118)
\closepath
\moveto(823.18328394,728.32738118)
\curveto(823.18328394,730.58779784)(822.74057561,732.26488118)(821.85515894,733.35863118)
\curveto(820.96974227,734.46279784)(819.74057561,735.01488118)(818.16765894,735.01488118)
\curveto(816.57390894,735.01488118)(815.33432561,734.46279784)(814.44890894,733.35863118)
\curveto(813.57390894,732.26488118)(813.13640894,730.58779784)(813.13640894,728.32738118)
\curveto(813.13640894,726.13988118)(813.57911727,724.47842284)(814.46453394,723.34300618)
\curveto(815.34995061,722.21800618)(816.58432561,721.65550618)(818.16765894,721.65550618)
\curveto(819.73015894,721.65550618)(820.95411727,722.21279784)(821.83953394,723.32738118)
\curveto(822.73536727,724.45238118)(823.18328394,726.11904784)(823.18328394,728.32738118)
\closepath
}
}
{
\newrgbcolor{curcolor}{0 0 0}
\pscustom[linestyle=none,fillstyle=solid,fillcolor=curcolor]
{
\newpath
\moveto(843.66765894,720.70238118)
\curveto(842.68849227,720.23363118)(841.75620061,719.86904784)(840.87078394,719.60863118)
\curveto(839.99578394,719.34821451)(839.06349227,719.21800618)(838.07390894,719.21800618)
\curveto(836.81349227,719.21800618)(835.65724227,719.40029784)(834.60515894,719.76488118)
\curveto(833.55307561,720.13988118)(832.65203394,720.70238118)(831.90203394,721.45238118)
\curveto(831.14161727,722.20238118)(830.55307561,723.15029784)(830.13640894,724.29613118)
\curveto(829.71974227,725.44196451)(829.51140894,726.78050618)(829.51140894,728.31175618)
\curveto(829.51140894,731.16592284)(830.29265894,733.40550618)(831.85515894,735.03050618)
\curveto(833.42807561,736.65550618)(835.50099227,737.46800618)(838.07390894,737.46800618)
\curveto(839.07390894,737.46800618)(840.05307561,737.32738118)(841.01140894,737.04613118)
\curveto(841.98015894,736.76488118)(842.86557561,736.42113118)(843.66765894,736.01488118)
\lineto(843.66765894,732.74925618)
\lineto(843.51140894,732.74925618)
\curveto(842.61557561,733.44717284)(841.68849227,733.98363118)(840.73015894,734.35863118)
\curveto(839.78224227,734.73363118)(838.85515894,734.92113118)(837.94890894,734.92113118)
\curveto(836.28224227,734.92113118)(834.96453394,734.35863118)(833.99578394,733.23363118)
\curveto(833.03745061,732.11904784)(832.55828394,730.47842284)(832.55828394,728.31175618)
\curveto(832.55828394,726.20758951)(833.02703394,724.58779784)(833.96453394,723.45238118)
\curveto(834.91245061,722.32738118)(836.24057561,721.76488118)(837.94890894,721.76488118)
\curveto(838.54265894,721.76488118)(839.14682561,721.84300618)(839.76140894,721.99925618)
\curveto(840.37599227,722.15550618)(840.92807561,722.35863118)(841.41765894,722.60863118)
\curveto(841.84474227,722.82738118)(842.24578394,723.05654784)(842.62078394,723.29613118)
\curveto(842.99578394,723.54613118)(843.29265894,723.75967284)(843.51140894,723.93675618)
\lineto(843.66765894,723.93675618)
\closepath
}
}
{
\newrgbcolor{curcolor}{0 0 0}
\pscustom[linestyle=none,fillstyle=solid,fillcolor=curcolor]
{
\newpath
\moveto(863.27703394,719.60863118)
\lineto(859.40203394,719.60863118)
\lineto(852.40203394,727.24925618)
\lineto(850.49578394,725.43675618)
\lineto(850.49578394,719.60863118)
\lineto(847.55828394,719.60863118)
\lineto(847.55828394,743.92113118)
\lineto(850.49578394,743.92113118)
\lineto(850.49578394,728.32738118)
\lineto(858.98015894,737.06175618)
\lineto(862.68328394,737.06175618)
\lineto(854.57390894,728.99925618)
\closepath
}
}
{
\newrgbcolor{curcolor}{0 0 0}
\pscustom[linestyle=none,fillstyle=solid,fillcolor=curcolor]
{
\newpath
\moveto(881.73015894,728.54613118)
\curveto(881.73015894,727.12946451)(881.52703394,725.83258951)(881.12078394,724.65550618)
\curveto(880.71453394,723.48883951)(880.14161727,722.49925618)(879.40203394,721.68675618)
\curveto(878.71453394,720.91592284)(877.90203394,720.31696451)(876.96453394,719.88988118)
\curveto(876.03745061,719.47321451)(875.05307561,719.26488118)(874.01140894,719.26488118)
\curveto(873.10515894,719.26488118)(872.28224227,719.36383951)(871.54265894,719.56175618)
\curveto(870.81349227,719.75967284)(870.06870061,720.06696451)(869.30828394,720.48363118)
\lineto(869.30828394,713.17113118)
\lineto(866.37078394,713.17113118)
\lineto(866.37078394,737.06175618)
\lineto(869.30828394,737.06175618)
\lineto(869.30828394,735.23363118)
\curveto(870.08953394,735.88988118)(870.96453394,736.43675618)(871.93328394,736.87425618)
\curveto(872.91245061,737.32217284)(873.95411727,737.54613118)(875.05828394,737.54613118)
\curveto(877.16245061,737.54613118)(878.79786727,736.74925618)(879.96453394,735.15550618)
\curveto(881.14161727,733.57217284)(881.73015894,731.36904784)(881.73015894,728.54613118)
\closepath
\moveto(878.69890894,728.46800618)
\curveto(878.69890894,730.57217284)(878.33953394,732.14508951)(877.62078394,733.18675618)
\curveto(876.90203394,734.22842284)(875.79786727,734.74925618)(874.30828394,734.74925618)
\curveto(873.46453394,734.74925618)(872.61557561,734.56696451)(871.76140894,734.20238118)
\curveto(870.90724227,733.83779784)(870.08953394,733.35863118)(869.30828394,732.76488118)
\lineto(869.30828394,722.87425618)
\curveto(870.14161727,722.49925618)(870.85515894,722.24404784)(871.44890894,722.10863118)
\curveto(872.05307561,721.97321451)(872.73536727,721.90550618)(873.49578394,721.90550618)
\curveto(875.13120061,721.90550618)(876.40724227,722.45758951)(877.32390894,723.56175618)
\curveto(878.24057561,724.66592284)(878.69890894,726.30133951)(878.69890894,728.46800618)
\closepath
}
}
{
\newrgbcolor{curcolor}{0 0 0}
\pscustom[linestyle=none,fillstyle=solid,fillcolor=curcolor]
{
\newpath
\moveto(901.18328394,728.32738118)
\curveto(901.18328394,725.48363118)(900.45411727,723.23883951)(898.99578394,721.59300618)
\curveto(897.53745061,719.94717284)(895.58432561,719.12425618)(893.13640894,719.12425618)
\curveto(890.66765894,719.12425618)(888.70411727,719.94717284)(887.24578394,721.59300618)
\curveto(885.79786727,723.23883951)(885.07390894,725.48363118)(885.07390894,728.32738118)
\curveto(885.07390894,731.17113118)(885.79786727,733.41592284)(887.24578394,735.06175618)
\curveto(888.70411727,736.71800618)(890.66765894,737.54613118)(893.13640894,737.54613118)
\curveto(895.58432561,737.54613118)(897.53745061,736.71800618)(898.99578394,735.06175618)
\curveto(900.45411727,733.41592284)(901.18328394,731.17113118)(901.18328394,728.32738118)
\closepath
\moveto(898.15203394,728.32738118)
\curveto(898.15203394,730.58779784)(897.70932561,732.26488118)(896.82390894,733.35863118)
\curveto(895.93849227,734.46279784)(894.70932561,735.01488118)(893.13640894,735.01488118)
\curveto(891.54265894,735.01488118)(890.30307561,734.46279784)(889.41765894,733.35863118)
\curveto(888.54265894,732.26488118)(888.10515894,730.58779784)(888.10515894,728.32738118)
\curveto(888.10515894,726.13988118)(888.54786727,724.47842284)(889.43328394,723.34300618)
\curveto(890.31870061,722.21800618)(891.55307561,721.65550618)(893.13640894,721.65550618)
\curveto(894.69890894,721.65550618)(895.92286727,722.21279784)(896.80828394,723.32738118)
\curveto(897.70411727,724.45238118)(898.15203394,726.11904784)(898.15203394,728.32738118)
\closepath
}
}
{
\newrgbcolor{curcolor}{0 0 0}
\pscustom[linestyle=none,fillstyle=solid,fillcolor=curcolor]
{
\newpath
\moveto(908.88640894,739.98363118)
\lineto(905.57390894,739.98363118)
\lineto(905.57390894,743.03050618)
\lineto(908.88640894,743.03050618)
\closepath
\moveto(908.69890894,719.60863118)
\lineto(905.76140894,719.60863118)
\lineto(905.76140894,737.06175618)
\lineto(908.69890894,737.06175618)
\closepath
}
}
{
\newrgbcolor{curcolor}{0 0 0}
\pscustom[linestyle=none,fillstyle=solid,fillcolor=curcolor]
{
\newpath
\moveto(929.10515894,719.60863118)
\lineto(926.16765894,719.60863118)
\lineto(926.16765894,729.54613118)
\curveto(926.16765894,730.34821451)(926.12078394,731.09821451)(926.02703394,731.79613118)
\curveto(925.93328394,732.50446451)(925.76140894,733.05654784)(925.51140894,733.45238118)
\curveto(925.25099227,733.88988118)(924.87599227,734.21279784)(924.38640894,734.42113118)
\curveto(923.89682561,734.63988118)(923.26140894,734.74925618)(922.48015894,734.74925618)
\curveto(921.67807561,734.74925618)(920.83953394,734.55133951)(919.96453394,734.15550618)
\curveto(919.08953394,733.75967284)(918.25099227,733.25446451)(917.44890894,732.63988118)
\lineto(917.44890894,719.60863118)
\lineto(914.51140894,719.60863118)
\lineto(914.51140894,737.06175618)
\lineto(917.44890894,737.06175618)
\lineto(917.44890894,735.12425618)
\curveto(918.36557561,735.88467284)(919.31349227,736.47842284)(920.29265894,736.90550618)
\curveto(921.27182561,737.33258951)(922.27703394,737.54613118)(923.30828394,737.54613118)
\curveto(925.19370061,737.54613118)(926.63120061,736.97842284)(927.62078394,735.84300618)
\curveto(928.61036727,734.70758951)(929.10515894,733.07217284)(929.10515894,730.93675618)
\closepath
}
}
{
\newrgbcolor{curcolor}{0 0 0}
\pscustom[linestyle=none,fillstyle=solid,fillcolor=curcolor]
{
\newpath
\moveto(943.82390894,719.76488118)
\curveto(943.27182561,719.61904784)(942.66765894,719.49925618)(942.01140894,719.40550618)
\curveto(941.36557561,719.31175618)(940.78745061,719.26488118)(940.27703394,719.26488118)
\curveto(938.49578394,719.26488118)(937.14161727,719.74404784)(936.21453394,720.70238118)
\curveto(935.28745061,721.66071451)(934.82390894,723.19717284)(934.82390894,725.31175618)
\lineto(934.82390894,734.59300618)
\lineto(932.83953394,734.59300618)
\lineto(932.83953394,737.06175618)
\lineto(934.82390894,737.06175618)
\lineto(934.82390894,742.07738118)
\lineto(937.76140894,742.07738118)
\lineto(937.76140894,737.06175618)
\lineto(943.82390894,737.06175618)
\lineto(943.82390894,734.59300618)
\lineto(937.76140894,734.59300618)
\lineto(937.76140894,726.63988118)
\curveto(937.76140894,725.72321451)(937.78224227,725.00446451)(937.82390894,724.48363118)
\curveto(937.86557561,723.97321451)(938.01140894,723.49404784)(938.26140894,723.04613118)
\curveto(938.49057561,722.62946451)(938.80307561,722.32217284)(939.19890894,722.12425618)
\curveto(939.60515894,721.93675618)(940.21974227,721.84300618)(941.04265894,721.84300618)
\curveto(941.52182561,721.84300618)(942.02182561,721.91071451)(942.54265894,722.04613118)
\curveto(943.06349227,722.19196451)(943.43849227,722.31175618)(943.66765894,722.40550618)
\lineto(943.82390894,722.40550618)
\closepath
}
}
{
\newrgbcolor{curcolor}{0 0 0}
\pscustom[linestyle=none,fillstyle=solid,fillcolor=curcolor]
{
\newpath
\moveto(961.98015894,728.03050618)
\lineto(949.12078394,728.03050618)
\curveto(949.12078394,726.95758951)(949.28224227,726.02008951)(949.60515894,725.21800618)
\curveto(949.92807561,724.42633951)(950.37078394,723.77529784)(950.93328394,723.26488118)
\curveto(951.47495061,722.76488118)(952.11557561,722.38988118)(952.85515894,722.13988118)
\curveto(953.60515894,721.88988118)(954.42807561,721.76488118)(955.32390894,721.76488118)
\curveto(956.51140894,721.76488118)(957.70411727,721.99925618)(958.90203394,722.46800618)
\curveto(960.11036727,722.94717284)(960.96974227,723.41592284)(961.48015894,723.87425618)
\lineto(961.63640894,723.87425618)
\lineto(961.63640894,720.67113118)
\curveto(960.64682561,720.25446451)(959.63640894,719.90550618)(958.60515894,719.62425618)
\curveto(957.57390894,719.34300618)(956.49057561,719.20238118)(955.35515894,719.20238118)
\curveto(952.45932561,719.20238118)(950.19890894,719.98363118)(948.57390894,721.54613118)
\curveto(946.94890894,723.11904784)(946.13640894,725.34821451)(946.13640894,728.23363118)
\curveto(946.13640894,731.08779784)(946.91245061,733.35342284)(948.46453394,735.03050618)
\curveto(950.02703394,736.70758951)(952.07911727,737.54613118)(954.62078394,737.54613118)
\curveto(956.97495061,737.54613118)(958.78745061,736.85863118)(960.05828394,735.48363118)
\curveto(961.33953394,734.10863118)(961.98015894,732.15550618)(961.98015894,729.62425618)
\closepath
\moveto(959.12078394,730.28050618)
\curveto(959.11036727,731.82217284)(958.71974227,733.01488118)(957.94890894,733.85863118)
\curveto(957.18849227,734.70238118)(956.02703394,735.12425618)(954.46453394,735.12425618)
\curveto(952.89161727,735.12425618)(951.63640894,734.66071451)(950.69890894,733.73363118)
\curveto(949.77182561,732.80654784)(949.24578394,731.65550618)(949.12078394,730.28050618)
\closepath
}
}
{
\newrgbcolor{curcolor}{0 0 0}
\pscustom[linestyle=none,fillstyle=solid,fillcolor=curcolor]
{
\newpath
\moveto(977.32390894,733.85863118)
\lineto(977.16765894,733.85863118)
\curveto(976.73015894,733.96279784)(976.30307561,734.03571451)(975.88640894,734.07738118)
\curveto(975.48015894,734.12946451)(974.99578394,734.15550618)(974.43328394,734.15550618)
\curveto(973.52703394,734.15550618)(972.65203394,733.95238118)(971.80828394,733.54613118)
\curveto(970.96453394,733.15029784)(970.15203394,732.63467284)(969.37078394,731.99925618)
\lineto(969.37078394,719.60863118)
\lineto(966.43328394,719.60863118)
\lineto(966.43328394,737.06175618)
\lineto(969.37078394,737.06175618)
\lineto(969.37078394,734.48363118)
\curveto(970.53745061,735.42113118)(971.56349227,736.08258951)(972.44890894,736.46800618)
\curveto(973.34474227,736.86383951)(974.25620061,737.06175618)(975.18328394,737.06175618)
\curveto(975.69370061,737.06175618)(976.06349227,737.04613118)(976.29265894,737.01488118)
\curveto(976.52182561,736.99404784)(976.86557561,736.94717284)(977.32390894,736.87425618)
\closepath
}
}
{
\newrgbcolor{curcolor}{0 0 0}
\pscustom[linestyle=none,fillstyle=solid,fillcolor=curcolor]
{
\newpath
\moveto(989.43328394,713.17113118)
\lineto(985.85515894,713.17113118)
\curveto(984.01140894,715.28571451)(982.57911727,717.59300618)(981.55828394,720.09300618)
\curveto(980.53745061,722.59300618)(980.02703394,725.41071451)(980.02703394,728.54613118)
\curveto(980.02703394,731.68154784)(980.53745061,734.49925618)(981.55828394,736.99925618)
\curveto(982.57911727,739.49925618)(984.01140894,741.80654784)(985.85515894,743.92113118)
\lineto(989.43328394,743.92113118)
\lineto(989.43328394,743.76488118)
\curveto(988.58953394,743.00446451)(987.78224227,742.12425618)(987.01140894,741.12425618)
\curveto(986.25099227,740.13467284)(985.54265894,738.97842284)(984.88640894,737.65550618)
\curveto(984.26140894,736.37425618)(983.75099227,734.96279784)(983.35515894,733.42113118)
\curveto(982.96974227,731.87946451)(982.77703394,730.25446451)(982.77703394,728.54613118)
\curveto(982.77703394,726.76488118)(982.96453394,725.13467284)(983.33953394,723.65550618)
\curveto(983.72495061,722.17633951)(984.24057561,720.77008951)(984.88640894,719.43675618)
\curveto(985.51140894,718.15550618)(986.22495061,716.99925618)(987.02703394,715.96800618)
\curveto(987.82911727,714.92633951)(988.63120061,714.04613118)(989.43328394,713.32738118)
\closepath
}
}
{
\newrgbcolor{curcolor}{0 0 0}
\pscustom[linestyle=none,fillstyle=solid,fillcolor=curcolor]
{
\newpath
\moveto(1009.99578394,719.60863118)
\lineto(994.24578394,719.60863118)
\lineto(994.24578394,722.87425618)
\lineto(997.52703394,725.68675618)
\curveto(998.63120061,726.62425618)(999.65724227,727.55654784)(1000.60515894,728.48363118)
\curveto(1002.60515894,730.42113118)(1003.97495061,731.95758951)(1004.71453394,733.09300618)
\curveto(1005.45411727,734.23883951)(1005.82390894,735.47321451)(1005.82390894,736.79613118)
\curveto(1005.82390894,738.00446451)(1005.42286727,738.94717284)(1004.62078394,739.62425618)
\curveto(1003.82911727,740.31175618)(1002.71974227,740.65550618)(1001.29265894,740.65550618)
\curveto(1000.34474227,740.65550618)(999.31870061,740.48883951)(998.21453394,740.15550618)
\curveto(997.11036727,739.82217284)(996.03224227,739.31175618)(994.98015894,738.62425618)
\lineto(994.82390894,738.62425618)
\lineto(994.82390894,741.90550618)
\curveto(995.56349227,742.27008951)(996.54786727,742.60342284)(997.77703394,742.90550618)
\curveto(999.01661727,743.20758951)(1000.21453394,743.35863118)(1001.37078394,743.35863118)
\curveto(1003.75620061,743.35863118)(1005.62599227,742.78050618)(1006.98015894,741.62425618)
\curveto(1008.33432561,740.47842284)(1009.01140894,738.92113118)(1009.01140894,736.95238118)
\curveto(1009.01140894,736.06696451)(1008.89682561,735.23883951)(1008.66765894,734.46800618)
\curveto(1008.44890894,733.70758951)(1008.12078394,732.98363118)(1007.68328394,732.29613118)
\curveto(1007.27703394,731.65029784)(1006.79786727,731.01488118)(1006.24578394,730.38988118)
\curveto(1005.70411727,729.76488118)(1005.04265894,729.07217284)(1004.26140894,728.31175618)
\curveto(1003.14682561,727.21800618)(1001.99578394,726.15550618)(1000.80828394,725.12425618)
\curveto(999.62078394,724.10342284)(998.51140894,723.15550618)(997.48015894,722.28050618)
\lineto(1009.99578394,722.28050618)
\closepath
}
}
{
\newrgbcolor{curcolor}{0 0 0}
\pscustom[linestyle=none,fillstyle=solid,fillcolor=curcolor]
{
\newpath
\moveto(1023.77703394,728.54613118)
\curveto(1023.77703394,725.41071451)(1023.26661727,722.59300618)(1022.24578394,720.09300618)
\curveto(1021.22495061,717.59300618)(1019.79265894,715.28571451)(1017.94890894,713.17113118)
\lineto(1014.37078394,713.17113118)
\lineto(1014.37078394,713.32738118)
\curveto(1015.17286727,714.04613118)(1015.97495061,714.92633951)(1016.77703394,715.96800618)
\curveto(1017.58953394,716.99925618)(1018.30307561,718.15550618)(1018.91765894,719.43675618)
\curveto(1019.56349227,720.77008951)(1020.07390894,722.17633951)(1020.44890894,723.65550618)
\curveto(1020.83432561,725.13467284)(1021.02703394,726.76488118)(1021.02703394,728.54613118)
\curveto(1021.02703394,730.25446451)(1020.83432561,731.87946451)(1020.44890894,733.42113118)
\curveto(1020.06349227,734.96279784)(1019.55307561,736.37425618)(1018.91765894,737.65550618)
\curveto(1018.26140894,738.97842284)(1017.54786727,740.13467284)(1016.77703394,741.12425618)
\curveto(1016.01661727,742.12425618)(1015.21453394,743.00446451)(1014.37078394,743.76488118)
\lineto(1014.37078394,743.92113118)
\lineto(1017.94890894,743.92113118)
\curveto(1019.79265894,741.80654784)(1021.22495061,739.49925618)(1022.24578394,736.99925618)
\curveto(1023.26661727,734.49925618)(1023.77703394,731.68154784)(1023.77703394,728.54613118)
\closepath
}
}
{
\newrgbcolor{curcolor}{0 0 0}
\pscustom[linestyle=none,fillstyle=solid,fillcolor=curcolor]
{
\newpath
\moveto(873.61881998,865.81503595)
\curveto(873.61881998,864.36972345)(873.34538248,863.09368178)(872.79850748,861.98691095)
\curveto(872.25163248,860.88014012)(871.51595539,859.96868178)(870.59147623,859.25253595)
\curveto(869.49772623,858.39316095)(868.29329914,857.78118178)(866.97819498,857.41659845)
\curveto(865.67611164,857.05201512)(864.01595539,856.86972345)(861.99772623,856.86972345)
\lineto(851.68522623,856.86972345)
\lineto(851.68522623,885.9517547)
\lineto(860.29850748,885.9517547)
\curveto(862.42090331,885.9517547)(864.00944498,885.8736297)(865.06413248,885.7173797)
\curveto(866.11881998,885.5611297)(867.12793456,885.23560887)(868.09147623,884.7408172)
\curveto(869.15918456,884.18092137)(869.93392414,883.45826512)(870.41569498,882.57284845)
\curveto(870.89746581,881.70045262)(871.13835123,880.65227553)(871.13835123,879.4283172)
\curveto(871.13835123,878.04810887)(870.78678873,876.86972345)(870.08366373,875.89316095)
\curveto(869.38053873,874.92961928)(868.44303873,874.1548797)(867.27116373,873.5689422)
\lineto(867.27116373,873.4126922)
\curveto(869.23730956,873.00904637)(870.78678873,872.14316095)(871.91960123,870.81503595)
\curveto(873.05241373,869.49993178)(873.61881998,867.83326512)(873.61881998,865.81503595)
\closepath
\moveto(867.11491373,878.9205047)
\curveto(867.11491373,879.6236297)(866.99772623,880.21607762)(866.76335123,880.69784845)
\curveto(866.52897623,881.17961928)(866.15137206,881.57024428)(865.63053873,881.86972345)
\curveto(865.01855956,882.22128595)(864.27637206,882.4361297)(863.40397623,882.5142547)
\curveto(862.53158039,882.60540053)(861.45085123,882.65097345)(860.16178873,882.65097345)
\lineto(855.55241373,882.65097345)
\lineto(855.55241373,874.25253595)
\lineto(860.55241373,874.25253595)
\curveto(861.76335123,874.25253595)(862.72689289,874.3111297)(863.44303873,874.4283172)
\curveto(864.15918456,874.55852553)(864.82324706,874.8189422)(865.43522623,875.2095672)
\curveto(866.04720539,875.6001922)(866.47689289,876.10149428)(866.72428873,876.71347345)
\curveto(866.98470539,877.33847345)(867.11491373,878.07415053)(867.11491373,878.9205047)
\closepath
\moveto(869.59538248,865.65878595)
\curveto(869.59538248,866.83066095)(869.41960123,867.76165053)(869.06803873,868.4517547)
\curveto(868.71647623,869.14185887)(868.07845539,869.72779637)(867.15397623,870.2095672)
\curveto(866.52897623,870.53508803)(865.76725748,870.74342137)(864.86881998,870.8345672)
\curveto(863.98340331,870.93873387)(862.90267414,870.9908172)(861.62663248,870.9908172)
\lineto(855.55241373,870.9908172)
\lineto(855.55241373,860.1705047)
\lineto(860.66960123,860.1705047)
\curveto(862.36230956,860.1705047)(863.74902831,860.25514012)(864.82975748,860.42441095)
\curveto(865.91048664,860.60670262)(866.79590331,860.93222345)(867.48600748,861.40097345)
\curveto(868.21517414,861.90878595)(868.74902831,862.48821303)(869.08756998,863.1392547)
\curveto(869.42611164,863.79029637)(869.59538248,864.63014012)(869.59538248,865.65878595)
\closepath
}
}
{
\newrgbcolor{curcolor}{0 0 0}
\pscustom[linestyle=none,fillstyle=solid,fillcolor=curcolor]
{
\newpath
\moveto(882.52506998,856.86972345)
\lineto(878.85319498,856.86972345)
\lineto(878.85319498,887.26034845)
\lineto(882.52506998,887.26034845)
\closepath
}
}
{
\newrgbcolor{curcolor}{0 0 0}
\pscustom[linestyle=none,fillstyle=solid,fillcolor=curcolor]
{
\newpath
\moveto(908.38444498,867.76816095)
\curveto(908.38444498,864.21347345)(907.47298664,861.40748387)(905.65006998,859.3501922)
\curveto(903.82715331,857.29290053)(901.38574706,856.2642547)(898.32585123,856.2642547)
\curveto(895.23991373,856.2642547)(892.78548664,857.29290053)(890.96256998,859.3501922)
\curveto(889.15267414,861.40748387)(888.24772623,864.21347345)(888.24772623,867.76816095)
\curveto(888.24772623,871.32284845)(889.15267414,874.12883803)(890.96256998,876.1861297)
\curveto(892.78548664,878.2564422)(895.23991373,879.29159845)(898.32585123,879.29159845)
\curveto(901.38574706,879.29159845)(903.82715331,878.2564422)(905.65006998,876.1861297)
\curveto(907.47298664,874.12883803)(908.38444498,871.32284845)(908.38444498,867.76816095)
\closepath
\moveto(904.59538248,867.76816095)
\curveto(904.59538248,870.59368178)(904.04199706,872.69003595)(902.93522623,874.05722345)
\curveto(901.82845539,875.43743178)(900.29199706,876.12753595)(898.32585123,876.12753595)
\curveto(896.33366373,876.12753595)(894.78418456,875.43743178)(893.67741373,874.05722345)
\curveto(892.58366373,872.69003595)(892.03678873,870.59368178)(892.03678873,867.76816095)
\curveto(892.03678873,865.03378595)(892.59017414,862.95696303)(893.69694498,861.5376922)
\curveto(894.80371581,860.1314422)(896.34668456,859.4283172)(898.32585123,859.4283172)
\curveto(900.27897623,859.4283172)(901.80892414,860.12493178)(902.91569498,861.51816095)
\curveto(904.03548664,862.92441095)(904.59538248,865.00774428)(904.59538248,867.76816095)
\closepath
}
}
{
\newrgbcolor{curcolor}{0 0 0}
\pscustom[linestyle=none,fillstyle=solid,fillcolor=curcolor]
{
\newpath
\moveto(930.20085123,858.23691095)
\curveto(928.97689289,857.65097345)(927.81152831,857.19524428)(926.70475748,856.86972345)
\curveto(925.61100748,856.54420262)(924.44564289,856.3814422)(923.20866373,856.3814422)
\curveto(921.63314289,856.3814422)(920.18783039,856.60930678)(918.87272623,857.06503595)
\curveto(917.55762206,857.53378595)(916.43131998,858.23691095)(915.49381998,859.17441095)
\curveto(914.54329914,860.11191095)(913.80762206,861.29680678)(913.28678873,862.72909845)
\curveto(912.76595539,864.16139012)(912.50553873,865.8345672)(912.50553873,867.7486297)
\curveto(912.50553873,871.31633803)(913.48210123,874.1158172)(915.43522623,876.1470672)
\curveto(917.40137206,878.1783172)(919.99251789,879.1939422)(923.20866373,879.1939422)
\curveto(924.45866373,879.1939422)(925.68262206,879.01816095)(926.88053873,878.66659845)
\curveto(928.09147623,878.31503595)(929.19824706,877.88534845)(930.20085123,877.37753595)
\lineto(930.20085123,873.2955047)
\lineto(930.00553873,873.2955047)
\curveto(928.88574706,874.16790053)(927.72689289,874.83847345)(926.52897623,875.30722345)
\curveto(925.34408039,875.77597345)(924.18522623,876.01034845)(923.05241373,876.01034845)
\curveto(920.96908039,876.01034845)(919.32194498,875.30722345)(918.11100748,873.90097345)
\curveto(916.91309081,872.50774428)(916.31413248,870.45696303)(916.31413248,867.7486297)
\curveto(916.31413248,865.11842137)(916.90006998,863.09368178)(918.07194498,861.67441095)
\curveto(919.25684081,860.26816095)(920.91699706,859.56503595)(923.05241373,859.56503595)
\curveto(923.79460123,859.56503595)(924.54980956,859.6626922)(925.31803873,859.8580047)
\curveto(926.08626789,860.0533172)(926.77637206,860.30722345)(927.38835123,860.61972345)
\curveto(927.92220539,860.89316095)(928.42350748,861.17961928)(928.89225748,861.47909845)
\curveto(929.36100748,861.79159845)(929.73210123,862.05852553)(930.00553873,862.2798797)
\lineto(930.20085123,862.2798797)
\closepath
}
}
{
\newrgbcolor{curcolor}{0 0 0}
\pscustom[linestyle=none,fillstyle=solid,fillcolor=curcolor]
{
\newpath
\moveto(954.71256998,856.86972345)
\lineto(949.86881998,856.86972345)
\lineto(941.11881998,866.4205047)
\lineto(938.73600748,864.1548797)
\lineto(938.73600748,856.86972345)
\lineto(935.06413248,856.86972345)
\lineto(935.06413248,887.26034845)
\lineto(938.73600748,887.26034845)
\lineto(938.73600748,867.76816095)
\lineto(949.34147623,878.6861297)
\lineto(953.97038248,878.6861297)
\lineto(943.83366373,868.6080047)
\closepath
}
}
{
\newrgbcolor{curcolor}{0 0 0}
\pscustom[linestyle=none,fillstyle=solid,fillcolor=curcolor]
{
\newpath
\moveto(911.45085123,821.4205047)
\curveto(911.45085123,816.19915053)(910.63053873,812.36451512)(908.98991373,809.91659845)
\curveto(907.36230956,807.48170262)(904.82975748,806.2642547)(901.39225748,806.2642547)
\curveto(897.90267414,806.2642547)(895.35059081,807.50123387)(893.73600748,809.9751922)
\curveto(892.13444498,812.44915053)(891.33366373,816.25123387)(891.33366373,821.3814422)
\curveto(891.33366373,826.55071303)(892.14746581,830.3658172)(893.77506998,832.8267547)
\curveto(895.40267414,835.30071303)(897.94173664,836.5376922)(901.39225748,836.5376922)
\curveto(904.88184081,836.5376922)(907.42741373,835.28118178)(909.02897623,832.76816095)
\curveto(910.64355956,830.26816095)(911.45085123,826.48560887)(911.45085123,821.4205047)
\closepath
\moveto(906.31413248,812.5533172)
\curveto(906.76986164,813.6080047)(907.07585123,814.84498387)(907.23210123,816.2642547)
\curveto(907.40137206,817.69654637)(907.48600748,819.41529637)(907.48600748,821.4205047)
\curveto(907.48600748,823.39967137)(907.40137206,825.11842137)(907.23210123,826.5767547)
\curveto(907.07585123,828.03508803)(906.76335123,829.2720672)(906.29460123,830.2876922)
\curveto(905.83887206,831.29029637)(905.21387206,832.0455047)(904.41960123,832.5533172)
\curveto(903.63835123,833.0611297)(902.62923664,833.31503595)(901.39225748,833.31503595)
\curveto(900.16829914,833.31503595)(899.15267414,833.0611297)(898.34538248,832.5533172)
\curveto(897.55111164,832.0455047)(896.91960123,831.27727553)(896.45085123,830.2486297)
\curveto(896.00814289,829.28508803)(895.70215331,828.02857762)(895.53288248,826.47909845)
\curveto(895.37663248,824.92961928)(895.29850748,823.23040053)(895.29850748,821.3814422)
\curveto(895.29850748,819.3501922)(895.37012206,817.65097345)(895.51335123,816.28378595)
\curveto(895.65658039,814.91659845)(895.96256998,813.69264012)(896.43131998,812.61191095)
\curveto(896.86100748,811.59628595)(897.46647623,810.82154637)(898.24772623,810.2876922)
\curveto(899.04199706,809.75383803)(900.09017414,809.48691095)(901.39225748,809.48691095)
\curveto(902.61621581,809.48691095)(903.63184081,809.7408172)(904.43913248,810.2486297)
\curveto(905.24642414,810.7564422)(905.87142414,811.52467137)(906.31413248,812.5533172)
\closepath
}
}
{
\newrgbcolor{curcolor}{0 0 0}
\pscustom[linewidth=4,linecolor=curcolor]
{
\newpath
\moveto(206.724394,220.51590379)
\lineto(339.744294,346.74188379)
}
}
{
\newrgbcolor{curcolor}{0 0 0}
\pscustom[linestyle=none,fillstyle=solid,fillcolor=curcolor]
{
\newpath
\moveto(235.7399343,248.04949094)
\lineto(236.33271556,270.66914192)
\lineto(206.724394,220.51590379)
\lineto(258.35958528,247.45670968)
\closepath
}
}
{
\newrgbcolor{curcolor}{0 0 0}
\pscustom[linewidth=4.26666679,linecolor=curcolor]
{
\newpath
\moveto(235.7399343,248.04949094)
\lineto(236.33271556,270.66914192)
\lineto(206.724394,220.51590379)
\lineto(258.35958528,247.45670968)
\closepath
}
}
{
\newrgbcolor{curcolor}{0 0 0}
\pscustom[linewidth=4,linecolor=curcolor]
{
\newpath
\moveto(1135.356984,220.51590379)
\lineto(1268.376884,346.74188379)
}
}
{
\newrgbcolor{curcolor}{0 0 0}
\pscustom[linestyle=none,fillstyle=solid,fillcolor=curcolor]
{
\newpath
\moveto(1164.3725243,248.04949094)
\lineto(1164.96530556,270.66914192)
\lineto(1135.356984,220.51590379)
\lineto(1186.99217528,247.45670968)
\closepath
}
}
{
\newrgbcolor{curcolor}{0 0 0}
\pscustom[linewidth=4.26666679,linecolor=curcolor]
{
\newpath
\moveto(1164.3725243,248.04949094)
\lineto(1164.96530556,270.66914192)
\lineto(1135.356984,220.51590379)
\lineto(1186.99217528,247.45670968)
\closepath
}
}
{
\newrgbcolor{curcolor}{1 1 1}
\pscustom[linestyle=none,fillstyle=solid,fillcolor=curcolor]
{
\newpath
\moveto(0.99999966,220.5158905)
\lineto(407.85140957,220.5158905)
\lineto(407.85140957,1.00002136)
\lineto(0.99999966,1.00002136)
\closepath
}
}
{
\newrgbcolor{curcolor}{0 0 0}
\pscustom[linewidth=2,linecolor=curcolor]
{
\newpath
\moveto(0.99999966,220.5158905)
\lineto(407.85140957,220.5158905)
\lineto(407.85140957,1.00002136)
\lineto(0.99999966,1.00002136)
\closepath
}
}
{
\newrgbcolor{curcolor}{1 0.93333334 0.66666669}
\pscustom[linestyle=none,fillstyle=solid,fillcolor=curcolor]
{
\newpath
\moveto(23.13525357,97.5410675)
\lineto(381.71615567,97.5410675)
\lineto(381.71615567,21.68741455)
\lineto(23.13525357,21.68741455)
\closepath
}
}
{
\newrgbcolor{curcolor}{0 0 0}
\pscustom[linestyle=none,fillstyle=solid,fillcolor=curcolor]
{
\newpath
\moveto(175.77335324,175.20659484)
\curveto(175.77335324,173.76128234)(175.49991574,172.48524068)(174.95304074,171.37846984)
\curveto(174.40616574,170.27169901)(173.67048865,169.36024068)(172.74600949,168.64409484)
\curveto(171.65225949,167.78471984)(170.4478324,167.17274068)(169.13272824,166.80815734)
\curveto(167.8306449,166.44357401)(166.17048865,166.26128234)(164.15225949,166.26128234)
\lineto(153.83975949,166.26128234)
\lineto(153.83975949,195.34331359)
\lineto(162.45304074,195.34331359)
\curveto(164.57543657,195.34331359)(166.16397824,195.26518859)(167.21866574,195.10893859)
\curveto(168.27335324,194.95268859)(169.28246782,194.62716776)(170.24600949,194.13237609)
\curveto(171.31371782,193.57248026)(172.0884574,192.84982401)(172.57022824,191.96440734)
\curveto(173.05199907,191.09201151)(173.29288449,190.04383443)(173.29288449,188.81987609)
\curveto(173.29288449,187.43966776)(172.94132199,186.26128234)(172.23819699,185.28471984)
\curveto(171.53507199,184.32117818)(170.59757199,183.54643859)(169.42569699,182.96050109)
\lineto(169.42569699,182.80425109)
\curveto(171.39184282,182.40060526)(172.94132199,181.53471984)(174.07413449,180.20659484)
\curveto(175.20694699,178.89149068)(175.77335324,177.22482401)(175.77335324,175.20659484)
\closepath
\moveto(169.26944699,188.31206359)
\curveto(169.26944699,189.01518859)(169.15225949,189.60763651)(168.91788449,190.08940734)
\curveto(168.68350949,190.57117818)(168.30590532,190.96180318)(167.78507199,191.26128234)
\curveto(167.17309282,191.61284484)(166.43090532,191.82768859)(165.55850949,191.90581359)
\curveto(164.68611365,191.99695943)(163.60538449,192.04253234)(162.31632199,192.04253234)
\lineto(157.70694699,192.04253234)
\lineto(157.70694699,183.64409484)
\lineto(162.70694699,183.64409484)
\curveto(163.91788449,183.64409484)(164.88142615,183.70268859)(165.59757199,183.81987609)
\curveto(166.31371782,183.95008443)(166.97778032,184.21050109)(167.58975949,184.60112609)
\curveto(168.20173865,184.99175109)(168.63142615,185.49305318)(168.87882199,186.10503234)
\curveto(169.13923865,186.73003234)(169.26944699,187.46570943)(169.26944699,188.31206359)
\closepath
\moveto(171.74991574,175.05034484)
\curveto(171.74991574,176.22221984)(171.57413449,177.15320943)(171.22257199,177.84331359)
\curveto(170.87100949,178.53341776)(170.23298865,179.11935526)(169.30850949,179.60112609)
\curveto(168.68350949,179.92664693)(167.92179074,180.13498026)(167.02335324,180.22612609)
\curveto(166.13793657,180.33029276)(165.0572074,180.38237609)(163.78116574,180.38237609)
\lineto(157.70694699,180.38237609)
\lineto(157.70694699,169.56206359)
\lineto(162.82413449,169.56206359)
\curveto(164.51684282,169.56206359)(165.90356157,169.64669901)(166.98429074,169.81596984)
\curveto(168.0650199,169.99826151)(168.95043657,170.32378234)(169.64054074,170.79253234)
\curveto(170.3697074,171.30034484)(170.90356157,171.87977193)(171.24210324,172.53081359)
\curveto(171.5806449,173.18185526)(171.74991574,174.02169901)(171.74991574,175.05034484)
\closepath
}
}
{
\newrgbcolor{curcolor}{0 0 0}
\pscustom[linestyle=none,fillstyle=solid,fillcolor=curcolor]
{
\newpath
\moveto(184.67960324,166.26128234)
\lineto(181.00772824,166.26128234)
\lineto(181.00772824,196.65190734)
\lineto(184.67960324,196.65190734)
\closepath
}
}
{
\newrgbcolor{curcolor}{0 0 0}
\pscustom[linestyle=none,fillstyle=solid,fillcolor=curcolor]
{
\newpath
\moveto(210.53897824,177.15971984)
\curveto(210.53897824,173.60503234)(209.6275199,170.79904276)(207.80460324,168.74175109)
\curveto(205.98168657,166.68445943)(203.54028032,165.65581359)(200.48038449,165.65581359)
\curveto(197.39444699,165.65581359)(194.9400199,166.68445943)(193.11710324,168.74175109)
\curveto(191.3072074,170.79904276)(190.40225949,173.60503234)(190.40225949,177.15971984)
\curveto(190.40225949,180.71440734)(191.3072074,183.52039693)(193.11710324,185.57768859)
\curveto(194.9400199,187.64800109)(197.39444699,188.68315734)(200.48038449,188.68315734)
\curveto(203.54028032,188.68315734)(205.98168657,187.64800109)(207.80460324,185.57768859)
\curveto(209.6275199,183.52039693)(210.53897824,180.71440734)(210.53897824,177.15971984)
\closepath
\moveto(206.74991574,177.15971984)
\curveto(206.74991574,179.98524068)(206.19653032,182.08159484)(205.08975949,183.44878234)
\curveto(203.98298865,184.82899068)(202.44653032,185.51909484)(200.48038449,185.51909484)
\curveto(198.48819699,185.51909484)(196.93871782,184.82899068)(195.83194699,183.44878234)
\curveto(194.73819699,182.08159484)(194.19132199,179.98524068)(194.19132199,177.15971984)
\curveto(194.19132199,174.42534484)(194.7447074,172.34852193)(195.85147824,170.92925109)
\curveto(196.95824907,169.52300109)(198.50121782,168.81987609)(200.48038449,168.81987609)
\curveto(202.43350949,168.81987609)(203.9634574,169.51649068)(205.07022824,170.90971984)
\curveto(206.1900199,172.31596984)(206.74991574,174.39930318)(206.74991574,177.15971984)
\closepath
}
}
{
\newrgbcolor{curcolor}{0 0 0}
\pscustom[linestyle=none,fillstyle=solid,fillcolor=curcolor]
{
\newpath
\moveto(232.35538449,167.62846984)
\curveto(231.13142615,167.04253234)(229.96606157,166.58680318)(228.85929074,166.26128234)
\curveto(227.76554074,165.93576151)(226.60017615,165.77300109)(225.36319699,165.77300109)
\curveto(223.78767615,165.77300109)(222.34236365,166.00086568)(221.02725949,166.45659484)
\curveto(219.71215532,166.92534484)(218.58585324,167.62846984)(217.64835324,168.56596984)
\curveto(216.6978324,169.50346984)(215.96215532,170.68836568)(215.44132199,172.12065734)
\curveto(214.92048865,173.55294901)(214.66007199,175.22612609)(214.66007199,177.14018859)
\curveto(214.66007199,180.70789693)(215.63663449,183.50737609)(217.58975949,185.53862609)
\curveto(219.55590532,187.56987609)(222.14705115,188.58550109)(225.36319699,188.58550109)
\curveto(226.61319699,188.58550109)(227.83715532,188.40971984)(229.03507199,188.05815734)
\curveto(230.24600949,187.70659484)(231.35278032,187.27690734)(232.35538449,186.76909484)
\lineto(232.35538449,182.68706359)
\lineto(232.16007199,182.68706359)
\curveto(231.04028032,183.55945943)(229.88142615,184.23003234)(228.68350949,184.69878234)
\curveto(227.49861365,185.16753234)(226.33975949,185.40190734)(225.20694699,185.40190734)
\curveto(223.12361365,185.40190734)(221.47647824,184.69878234)(220.26554074,183.29253234)
\curveto(219.06762407,181.89930318)(218.46866574,179.84852193)(218.46866574,177.14018859)
\curveto(218.46866574,174.50998026)(219.05460324,172.48524068)(220.22647824,171.06596984)
\curveto(221.41137407,169.65971984)(223.07153032,168.95659484)(225.20694699,168.95659484)
\curveto(225.94913449,168.95659484)(226.70434282,169.05425109)(227.47257199,169.24956359)
\curveto(228.24080115,169.44487609)(228.93090532,169.69878234)(229.54288449,170.01128234)
\curveto(230.07673865,170.28471984)(230.57804074,170.57117818)(231.04679074,170.87065734)
\curveto(231.51554074,171.18315734)(231.88663449,171.45008443)(232.16007199,171.67143859)
\lineto(232.35538449,171.67143859)
\closepath
}
}
{
\newrgbcolor{curcolor}{0 0 0}
\pscustom[linestyle=none,fillstyle=solid,fillcolor=curcolor]
{
\newpath
\moveto(256.86710324,166.26128234)
\lineto(252.02335324,166.26128234)
\lineto(243.27335324,175.81206359)
\lineto(240.89054074,173.54643859)
\lineto(240.89054074,166.26128234)
\lineto(237.21866574,166.26128234)
\lineto(237.21866574,196.65190734)
\lineto(240.89054074,196.65190734)
\lineto(240.89054074,177.15971984)
\lineto(251.49600949,188.07768859)
\lineto(256.12491574,188.07768859)
\lineto(245.98819699,177.99956359)
\closepath
}
}
{
\newrgbcolor{curcolor}{0 0 0}
\pscustom[linestyle=none,fillstyle=solid,fillcolor=curcolor]
{
\newpath
\moveto(211.08585324,130.26518859)
\curveto(211.71085324,129.70529276)(212.22517615,129.00216776)(212.62882199,128.15581359)
\curveto(213.03246782,127.30945943)(213.23429074,126.21570943)(213.23429074,124.87456359)
\curveto(213.23429074,123.54643859)(212.99340532,122.32899068)(212.51163449,121.22221984)
\curveto(212.02986365,120.11544901)(211.35278032,119.15190734)(210.48038449,118.33159484)
\curveto(209.50382199,117.42013651)(208.35147824,116.74305318)(207.02335324,116.30034484)
\curveto(205.70824907,115.87065734)(204.26293657,115.65581359)(202.68741574,115.65581359)
\curveto(201.0728324,115.65581359)(199.48429074,115.85112609)(197.92179074,116.24175109)
\curveto(196.35929074,116.61935526)(195.07673865,117.03602193)(194.07413449,117.49175109)
\lineto(194.07413449,121.57378234)
\lineto(194.36710324,121.57378234)
\curveto(195.47387407,120.84461568)(196.7759574,120.23914693)(198.27335324,119.75737609)
\curveto(199.77074907,119.27560526)(201.21606157,119.03471984)(202.60929074,119.03471984)
\curveto(203.42960324,119.03471984)(204.30199907,119.17143859)(205.22647824,119.44487609)
\curveto(206.1509574,119.71831359)(206.89965532,120.12195943)(207.47257199,120.65581359)
\curveto(208.07153032,121.22873026)(208.51423865,121.86024068)(208.80069699,122.55034484)
\curveto(209.10017615,123.24044901)(209.24991574,124.11284484)(209.24991574,125.16753234)
\curveto(209.24991574,126.20919901)(209.0806449,127.06857401)(208.74210324,127.74565734)
\curveto(208.4165824,128.43576151)(207.96085324,128.97612609)(207.37491574,129.36675109)
\curveto(206.78897824,129.77039693)(206.07934282,130.04383443)(205.24600949,130.18706359)
\curveto(204.41267615,130.34331359)(203.51423865,130.42143859)(202.55069699,130.42143859)
\lineto(200.79288449,130.42143859)
\lineto(200.79288449,133.66362609)
\lineto(202.16007199,133.66362609)
\curveto(204.13923865,133.66362609)(205.71475949,134.07378234)(206.88663449,134.89409484)
\curveto(208.07153032,135.72742818)(208.66397824,136.93836568)(208.66397824,138.52690734)
\curveto(208.66397824,139.23003234)(208.51423865,139.84201151)(208.21475949,140.36284484)
\curveto(207.91528032,140.89669901)(207.49861365,141.33289693)(206.96475949,141.67143859)
\curveto(206.40486365,142.00998026)(205.80590532,142.24435526)(205.16788449,142.37456359)
\curveto(204.52986365,142.50477193)(203.8072074,142.56987609)(202.99991574,142.56987609)
\curveto(201.76293657,142.56987609)(200.4478324,142.34852193)(199.05460324,141.90581359)
\curveto(197.66137407,141.46310526)(196.3462699,140.83810526)(195.10929074,140.03081359)
\lineto(194.91397824,140.03081359)
\lineto(194.91397824,144.11284484)
\curveto(195.8384574,144.56857401)(197.06892615,144.98524068)(198.60538449,145.36284484)
\curveto(200.15486365,145.75346984)(201.65225949,145.94878234)(203.09757199,145.94878234)
\curveto(204.51684282,145.94878234)(205.76684282,145.81857401)(206.84757199,145.55815734)
\curveto(207.92830115,145.29774068)(208.90486365,144.88107401)(209.77725949,144.30815734)
\curveto(210.71475949,143.68315734)(211.4243949,142.92794901)(211.90616574,142.04253234)
\curveto(212.38793657,141.15711568)(212.62882199,140.12195943)(212.62882199,138.93706359)
\curveto(212.62882199,137.32248026)(212.05590532,135.90971984)(210.91007199,134.69878234)
\curveto(209.77725949,133.50086568)(208.43611365,132.74565734)(206.88663449,132.43315734)
\lineto(206.88663449,132.15971984)
\curveto(207.51163449,132.05555318)(208.22778032,131.83419901)(209.03507199,131.49565734)
\curveto(209.84236365,131.17013651)(210.5259574,130.75998026)(211.08585324,130.26518859)
\closepath
}
}
{
\newrgbcolor{curcolor}{0 0 0}
\pscustom[linestyle=none,fillstyle=solid,fillcolor=curcolor]
{
\newpath
\moveto(187.17569699,59.83299865)
\curveto(187.17569699,57.71841532)(186.71215532,55.80174865)(185.78507199,54.08299865)
\curveto(184.86840532,52.36424865)(183.64444699,51.03091532)(182.11319699,50.08299865)
\curveto(181.05069699,49.42674865)(179.86319699,48.95279032)(178.55069699,48.66112365)
\curveto(177.24861365,48.36945699)(175.52986365,48.22362365)(173.39444699,48.22362365)
\lineto(167.51944699,48.22362365)
\lineto(167.51944699,71.48924865)
\lineto(173.33194699,71.48924865)
\curveto(175.60278032,71.48924865)(177.40486365,71.32258199)(178.73819699,70.98924865)
\curveto(180.08194699,70.66633199)(181.21736365,70.21841532)(182.14444699,69.64549865)
\curveto(183.72778032,68.65591532)(184.96215532,67.33820699)(185.84757199,65.69237365)
\curveto(186.73298865,64.04654032)(187.17569699,62.09341532)(187.17569699,59.83299865)
\closepath
\moveto(183.94132199,59.87987365)
\curveto(183.94132199,61.70279032)(183.62361365,63.23924865)(182.98819699,64.48924865)
\curveto(182.35278032,65.73924865)(181.40486365,66.72362365)(180.14444699,67.44237365)
\curveto(179.22778032,67.96320699)(178.25382199,68.32258199)(177.22257199,68.52049865)
\curveto(176.19132199,68.72883199)(174.95694699,68.83299865)(173.51944699,68.83299865)
\lineto(170.61319699,68.83299865)
\lineto(170.61319699,50.87987365)
\lineto(173.51944699,50.87987365)
\curveto(175.00903032,50.87987365)(176.30590532,50.98924865)(177.41007199,51.20799865)
\curveto(178.52465532,51.42674865)(179.54548865,51.83299865)(180.47257199,52.42674865)
\curveto(181.62882199,53.16633199)(182.49340532,54.14029032)(183.06632199,55.34862365)
\curveto(183.64965532,56.55695699)(183.94132199,58.06737365)(183.94132199,59.87987365)
\closepath
}
}
{
\newrgbcolor{curcolor}{0 0 0}
\pscustom[linestyle=none,fillstyle=solid,fillcolor=curcolor]
{
\newpath
\moveto(205.50382199,48.22362365)
\lineto(202.58194699,48.22362365)
\lineto(202.58194699,50.08299865)
\curveto(202.32153032,49.90591532)(201.96736365,49.65591532)(201.51944699,49.33299865)
\curveto(201.08194699,49.02049865)(200.65486365,48.77049865)(200.23819699,48.58299865)
\curveto(199.74861365,48.34341532)(199.18611365,48.14549865)(198.55069699,47.98924865)
\curveto(197.91528032,47.82258199)(197.17048865,47.73924865)(196.31632199,47.73924865)
\curveto(194.74340532,47.73924865)(193.41007199,48.26008199)(192.31632199,49.30174865)
\curveto(191.22257199,50.34341532)(190.67569699,51.67154032)(190.67569699,53.28612365)
\curveto(190.67569699,54.60904032)(190.95694699,55.67674865)(191.51944699,56.48924865)
\curveto(192.09236365,57.31216532)(192.90486365,57.95799865)(193.95694699,58.42674865)
\curveto(195.01944699,58.89549865)(196.29548865,59.21320699)(197.78507199,59.37987365)
\curveto(199.27465532,59.54654032)(200.87361365,59.67154032)(202.58194699,59.75487365)
\lineto(202.58194699,60.20799865)
\curveto(202.58194699,60.87466532)(202.46215532,61.42674865)(202.22257199,61.86424865)
\curveto(201.99340532,62.30174865)(201.66007199,62.64549865)(201.22257199,62.89549865)
\curveto(200.80590532,63.13508199)(200.30590532,63.29654032)(199.72257199,63.37987365)
\curveto(199.13923865,63.46320699)(198.52986365,63.50487365)(197.89444699,63.50487365)
\curveto(197.12361365,63.50487365)(196.26423865,63.40070699)(195.31632199,63.19237365)
\curveto(194.36840532,62.99445699)(193.38923865,62.70279032)(192.37882199,62.31737365)
\lineto(192.22257199,62.31737365)
\lineto(192.22257199,65.30174865)
\curveto(192.79548865,65.45799865)(193.62361365,65.62987365)(194.70694699,65.81737365)
\curveto(195.79028032,66.00487365)(196.85798865,66.09862365)(197.91007199,66.09862365)
\curveto(199.13923865,66.09862365)(200.20694699,65.99445699)(201.11319699,65.78612365)
\curveto(202.02986365,65.58820699)(202.82153032,65.24445699)(203.48819699,64.75487365)
\curveto(204.14444699,64.27570699)(204.64444699,63.65591532)(204.98819699,62.89549865)
\curveto(205.33194699,62.13508199)(205.50382199,61.19237365)(205.50382199,60.06737365)
\closepath
\moveto(202.58194699,52.52049865)
\lineto(202.58194699,57.37987365)
\curveto(201.68611365,57.32779032)(200.62882199,57.24966532)(199.41007199,57.14549865)
\curveto(198.20173865,57.04133199)(197.24340532,56.89029032)(196.53507199,56.69237365)
\curveto(195.69132199,56.45279032)(195.00903032,56.07779032)(194.48819699,55.56737365)
\curveto(193.96736365,55.06737365)(193.70694699,54.37466532)(193.70694699,53.48924865)
\curveto(193.70694699,52.48924865)(194.00903032,51.73404032)(194.61319699,51.22362365)
\curveto(195.21736365,50.72362365)(196.13923865,50.47362365)(197.37882199,50.47362365)
\curveto(198.41007199,50.47362365)(199.35278032,50.67154032)(200.20694699,51.06737365)
\curveto(201.06111365,51.47362365)(201.85278032,51.95799865)(202.58194699,52.52049865)
\closepath
}
}
{
\newrgbcolor{curcolor}{0 0 0}
\pscustom[linestyle=none,fillstyle=solid,fillcolor=curcolor]
{
\newpath
\moveto(220.22257199,48.37987365)
\curveto(219.67048865,48.23404032)(219.06632199,48.11424865)(218.41007199,48.02049865)
\curveto(217.76423865,47.92674865)(217.18611365,47.87987365)(216.67569699,47.87987365)
\curveto(214.89444699,47.87987365)(213.54028032,48.35904032)(212.61319699,49.31737365)
\curveto(211.68611365,50.27570699)(211.22257199,51.81216532)(211.22257199,53.92674865)
\lineto(211.22257199,63.20799865)
\lineto(209.23819699,63.20799865)
\lineto(209.23819699,65.67674865)
\lineto(211.22257199,65.67674865)
\lineto(211.22257199,70.69237365)
\lineto(214.16007199,70.69237365)
\lineto(214.16007199,65.67674865)
\lineto(220.22257199,65.67674865)
\lineto(220.22257199,63.20799865)
\lineto(214.16007199,63.20799865)
\lineto(214.16007199,55.25487365)
\curveto(214.16007199,54.33820699)(214.18090532,53.61945699)(214.22257199,53.09862365)
\curveto(214.26423865,52.58820699)(214.41007199,52.10904032)(214.66007199,51.66112365)
\curveto(214.88923865,51.24445699)(215.20173865,50.93716532)(215.59757199,50.73924865)
\curveto(216.00382199,50.55174865)(216.61840532,50.45799865)(217.44132199,50.45799865)
\curveto(217.92048865,50.45799865)(218.42048865,50.52570699)(218.94132199,50.66112365)
\curveto(219.46215532,50.80695699)(219.83715532,50.92674865)(220.06632199,51.02049865)
\lineto(220.22257199,51.02049865)
\closepath
}
}
{
\newrgbcolor{curcolor}{0 0 0}
\pscustom[linestyle=none,fillstyle=solid,fillcolor=curcolor]
{
\newpath
\moveto(237.33194699,48.22362365)
\lineto(234.41007199,48.22362365)
\lineto(234.41007199,50.08299865)
\curveto(234.14965532,49.90591532)(233.79548865,49.65591532)(233.34757199,49.33299865)
\curveto(232.91007199,49.02049865)(232.48298865,48.77049865)(232.06632199,48.58299865)
\curveto(231.57673865,48.34341532)(231.01423865,48.14549865)(230.37882199,47.98924865)
\curveto(229.74340532,47.82258199)(228.99861365,47.73924865)(228.14444699,47.73924865)
\curveto(226.57153032,47.73924865)(225.23819699,48.26008199)(224.14444699,49.30174865)
\curveto(223.05069699,50.34341532)(222.50382199,51.67154032)(222.50382199,53.28612365)
\curveto(222.50382199,54.60904032)(222.78507199,55.67674865)(223.34757199,56.48924865)
\curveto(223.92048865,57.31216532)(224.73298865,57.95799865)(225.78507199,58.42674865)
\curveto(226.84757199,58.89549865)(228.12361365,59.21320699)(229.61319699,59.37987365)
\curveto(231.10278032,59.54654032)(232.70173865,59.67154032)(234.41007199,59.75487365)
\lineto(234.41007199,60.20799865)
\curveto(234.41007199,60.87466532)(234.29028032,61.42674865)(234.05069699,61.86424865)
\curveto(233.82153032,62.30174865)(233.48819699,62.64549865)(233.05069699,62.89549865)
\curveto(232.63403032,63.13508199)(232.13403032,63.29654032)(231.55069699,63.37987365)
\curveto(230.96736365,63.46320699)(230.35798865,63.50487365)(229.72257199,63.50487365)
\curveto(228.95173865,63.50487365)(228.09236365,63.40070699)(227.14444699,63.19237365)
\curveto(226.19653032,62.99445699)(225.21736365,62.70279032)(224.20694699,62.31737365)
\lineto(224.05069699,62.31737365)
\lineto(224.05069699,65.30174865)
\curveto(224.62361365,65.45799865)(225.45173865,65.62987365)(226.53507199,65.81737365)
\curveto(227.61840532,66.00487365)(228.68611365,66.09862365)(229.73819699,66.09862365)
\curveto(230.96736365,66.09862365)(232.03507199,65.99445699)(232.94132199,65.78612365)
\curveto(233.85798865,65.58820699)(234.64965532,65.24445699)(235.31632199,64.75487365)
\curveto(235.97257199,64.27570699)(236.47257199,63.65591532)(236.81632199,62.89549865)
\curveto(237.16007199,62.13508199)(237.33194699,61.19237365)(237.33194699,60.06737365)
\closepath
\moveto(234.41007199,52.52049865)
\lineto(234.41007199,57.37987365)
\curveto(233.51423865,57.32779032)(232.45694699,57.24966532)(231.23819699,57.14549865)
\curveto(230.02986365,57.04133199)(229.07153032,56.89029032)(228.36319699,56.69237365)
\curveto(227.51944699,56.45279032)(226.83715532,56.07779032)(226.31632199,55.56737365)
\curveto(225.79548865,55.06737365)(225.53507199,54.37466532)(225.53507199,53.48924865)
\curveto(225.53507199,52.48924865)(225.83715532,51.73404032)(226.44132199,51.22362365)
\curveto(227.04548865,50.72362365)(227.96736365,50.47362365)(229.20694699,50.47362365)
\curveto(230.23819699,50.47362365)(231.18090532,50.67154032)(232.03507199,51.06737365)
\curveto(232.88923865,51.47362365)(233.68090532,51.95799865)(234.41007199,52.52049865)
\closepath
}
}
{
\newrgbcolor{curcolor}{1 1 1}
\pscustom[linestyle=none,fillstyle=solid,fillcolor=curcolor]
{
\newpath
\moveto(465.31632199,220.5158905)
\lineto(872.1677319,220.5158905)
\lineto(872.1677319,1.00002136)
\lineto(465.31632199,1.00002136)
\closepath
}
}
{
\newrgbcolor{curcolor}{0 0 0}
\pscustom[linewidth=2,linecolor=curcolor]
{
\newpath
\moveto(465.31632199,220.5158905)
\lineto(872.1677319,220.5158905)
\lineto(872.1677319,1.00002136)
\lineto(465.31632199,1.00002136)
\closepath
}
}
{
\newrgbcolor{curcolor}{1 0.93333334 0.66666669}
\pscustom[linestyle=none,fillstyle=solid,fillcolor=curcolor]
{
\newpath
\moveto(487.45157589,97.5410675)
\lineto(846.03247799,97.5410675)
\lineto(846.03247799,21.68741455)
\lineto(487.45157589,21.68741455)
\closepath
}
}
{
\newrgbcolor{curcolor}{0 0 0}
\pscustom[linestyle=none,fillstyle=solid,fillcolor=curcolor]
{
\newpath
\moveto(640.08963742,175.20659484)
\curveto(640.08963742,173.76128234)(639.81619992,172.48524068)(639.26932492,171.37846984)
\curveto(638.72244992,170.27169901)(637.98677283,169.36024068)(637.06229367,168.64409484)
\curveto(635.96854367,167.78471984)(634.76411658,167.17274068)(633.44901242,166.80815734)
\curveto(632.14692908,166.44357401)(630.48677283,166.26128234)(628.46854367,166.26128234)
\lineto(618.15604367,166.26128234)
\lineto(618.15604367,195.34331359)
\lineto(626.76932492,195.34331359)
\curveto(628.89172075,195.34331359)(630.48026242,195.26518859)(631.53494992,195.10893859)
\curveto(632.58963742,194.95268859)(633.598752,194.62716776)(634.56229367,194.13237609)
\curveto(635.630002,193.57248026)(636.40474158,192.84982401)(636.88651242,191.96440734)
\curveto(637.36828325,191.09201151)(637.60916867,190.04383443)(637.60916867,188.81987609)
\curveto(637.60916867,187.43966776)(637.25760617,186.26128234)(636.55448117,185.28471984)
\curveto(635.85135617,184.32117818)(634.91385617,183.54643859)(633.74198117,182.96050109)
\lineto(633.74198117,182.80425109)
\curveto(635.708127,182.40060526)(637.25760617,181.53471984)(638.39041867,180.20659484)
\curveto(639.52323117,178.89149068)(640.08963742,177.22482401)(640.08963742,175.20659484)
\closepath
\moveto(633.58573117,188.31206359)
\curveto(633.58573117,189.01518859)(633.46854367,189.60763651)(633.23416867,190.08940734)
\curveto(632.99979367,190.57117818)(632.6221895,190.96180318)(632.10135617,191.26128234)
\curveto(631.489377,191.61284484)(630.7471895,191.82768859)(629.87479367,191.90581359)
\curveto(629.00239783,191.99695943)(627.92166867,192.04253234)(626.63260617,192.04253234)
\lineto(622.02323117,192.04253234)
\lineto(622.02323117,183.64409484)
\lineto(627.02323117,183.64409484)
\curveto(628.23416867,183.64409484)(629.19771033,183.70268859)(629.91385617,183.81987609)
\curveto(630.630002,183.95008443)(631.2940645,184.21050109)(631.90604367,184.60112609)
\curveto(632.51802283,184.99175109)(632.94771033,185.49305318)(633.19510617,186.10503234)
\curveto(633.45552283,186.73003234)(633.58573117,187.46570943)(633.58573117,188.31206359)
\closepath
\moveto(636.06619992,175.05034484)
\curveto(636.06619992,176.22221984)(635.89041867,177.15320943)(635.53885617,177.84331359)
\curveto(635.18729367,178.53341776)(634.54927283,179.11935526)(633.62479367,179.60112609)
\curveto(632.99979367,179.92664693)(632.23807492,180.13498026)(631.33963742,180.22612609)
\curveto(630.45422075,180.33029276)(629.37349158,180.38237609)(628.09744992,180.38237609)
\lineto(622.02323117,180.38237609)
\lineto(622.02323117,169.56206359)
\lineto(627.14041867,169.56206359)
\curveto(628.833127,169.56206359)(630.21984575,169.64669901)(631.30057492,169.81596984)
\curveto(632.38130408,169.99826151)(633.26672075,170.32378234)(633.95682492,170.79253234)
\curveto(634.68599158,171.30034484)(635.21984575,171.87977193)(635.55838742,172.53081359)
\curveto(635.89692908,173.18185526)(636.06619992,174.02169901)(636.06619992,175.05034484)
\closepath
}
}
{
\newrgbcolor{curcolor}{0 0 0}
\pscustom[linestyle=none,fillstyle=solid,fillcolor=curcolor]
{
\newpath
\moveto(648.99588742,166.26128234)
\lineto(645.32401242,166.26128234)
\lineto(645.32401242,196.65190734)
\lineto(648.99588742,196.65190734)
\closepath
}
}
{
\newrgbcolor{curcolor}{0 0 0}
\pscustom[linestyle=none,fillstyle=solid,fillcolor=curcolor]
{
\newpath
\moveto(674.85526242,177.15971984)
\curveto(674.85526242,173.60503234)(673.94380408,170.79904276)(672.12088742,168.74175109)
\curveto(670.29797075,166.68445943)(667.8565645,165.65581359)(664.79666867,165.65581359)
\curveto(661.71073117,165.65581359)(659.25630408,166.68445943)(657.43338742,168.74175109)
\curveto(655.62349158,170.79904276)(654.71854367,173.60503234)(654.71854367,177.15971984)
\curveto(654.71854367,180.71440734)(655.62349158,183.52039693)(657.43338742,185.57768859)
\curveto(659.25630408,187.64800109)(661.71073117,188.68315734)(664.79666867,188.68315734)
\curveto(667.8565645,188.68315734)(670.29797075,187.64800109)(672.12088742,185.57768859)
\curveto(673.94380408,183.52039693)(674.85526242,180.71440734)(674.85526242,177.15971984)
\closepath
\moveto(671.06619992,177.15971984)
\curveto(671.06619992,179.98524068)(670.5128145,182.08159484)(669.40604367,183.44878234)
\curveto(668.29927283,184.82899068)(666.7628145,185.51909484)(664.79666867,185.51909484)
\curveto(662.80448117,185.51909484)(661.255002,184.82899068)(660.14823117,183.44878234)
\curveto(659.05448117,182.08159484)(658.50760617,179.98524068)(658.50760617,177.15971984)
\curveto(658.50760617,174.42534484)(659.06099158,172.34852193)(660.16776242,170.92925109)
\curveto(661.27453325,169.52300109)(662.817502,168.81987609)(664.79666867,168.81987609)
\curveto(666.74979367,168.81987609)(668.27974158,169.51649068)(669.38651242,170.90971984)
\curveto(670.50630408,172.31596984)(671.06619992,174.39930318)(671.06619992,177.15971984)
\closepath
}
}
{
\newrgbcolor{curcolor}{0 0 0}
\pscustom[linestyle=none,fillstyle=solid,fillcolor=curcolor]
{
\newpath
\moveto(696.67166867,167.62846984)
\curveto(695.44771033,167.04253234)(694.28234575,166.58680318)(693.17557492,166.26128234)
\curveto(692.08182492,165.93576151)(690.91646033,165.77300109)(689.67948117,165.77300109)
\curveto(688.10396033,165.77300109)(686.65864783,166.00086568)(685.34354367,166.45659484)
\curveto(684.0284395,166.92534484)(682.90213742,167.62846984)(681.96463742,168.56596984)
\curveto(681.01411658,169.50346984)(680.2784395,170.68836568)(679.75760617,172.12065734)
\curveto(679.23677283,173.55294901)(678.97635617,175.22612609)(678.97635617,177.14018859)
\curveto(678.97635617,180.70789693)(679.95291867,183.50737609)(681.90604367,185.53862609)
\curveto(683.8721895,187.56987609)(686.46333533,188.58550109)(689.67948117,188.58550109)
\curveto(690.92948117,188.58550109)(692.1534395,188.40971984)(693.35135617,188.05815734)
\curveto(694.56229367,187.70659484)(695.6690645,187.27690734)(696.67166867,186.76909484)
\lineto(696.67166867,182.68706359)
\lineto(696.47635617,182.68706359)
\curveto(695.3565645,183.55945943)(694.19771033,184.23003234)(692.99979367,184.69878234)
\curveto(691.81489783,185.16753234)(690.65604367,185.40190734)(689.52323117,185.40190734)
\curveto(687.43989783,185.40190734)(685.79276242,184.69878234)(684.58182492,183.29253234)
\curveto(683.38390825,181.89930318)(682.78494992,179.84852193)(682.78494992,177.14018859)
\curveto(682.78494992,174.50998026)(683.37088742,172.48524068)(684.54276242,171.06596984)
\curveto(685.72765825,169.65971984)(687.3878145,168.95659484)(689.52323117,168.95659484)
\curveto(690.26541867,168.95659484)(691.020627,169.05425109)(691.78885617,169.24956359)
\curveto(692.55708533,169.44487609)(693.2471895,169.69878234)(693.85916867,170.01128234)
\curveto(694.39302283,170.28471984)(694.89432492,170.57117818)(695.36307492,170.87065734)
\curveto(695.83182492,171.18315734)(696.20291867,171.45008443)(696.47635617,171.67143859)
\lineto(696.67166867,171.67143859)
\closepath
}
}
{
\newrgbcolor{curcolor}{0 0 0}
\pscustom[linestyle=none,fillstyle=solid,fillcolor=curcolor]
{
\newpath
\moveto(721.18338742,166.26128234)
\lineto(716.33963742,166.26128234)
\lineto(707.58963742,175.81206359)
\lineto(705.20682492,173.54643859)
\lineto(705.20682492,166.26128234)
\lineto(701.53494992,166.26128234)
\lineto(701.53494992,196.65190734)
\lineto(705.20682492,196.65190734)
\lineto(705.20682492,177.15971984)
\lineto(715.81229367,188.07768859)
\lineto(720.44119992,188.07768859)
\lineto(710.30448117,177.99956359)
\closepath
}
}
{
\newrgbcolor{curcolor}{0 0 0}
\pscustom[linestyle=none,fillstyle=solid,fillcolor=curcolor]
{
\newpath
\moveto(678.62479367,124.44487609)
\lineto(674.30838742,124.44487609)
\lineto(674.30838742,116.26128234)
\lineto(670.55838742,116.26128234)
\lineto(670.55838742,124.44487609)
\lineto(656.63260617,124.44487609)
\lineto(656.63260617,128.93706359)
\lineto(670.71463742,145.34331359)
\lineto(674.30838742,145.34331359)
\lineto(674.30838742,127.56987609)
\lineto(678.62479367,127.56987609)
\closepath
\moveto(670.55838742,127.56987609)
\lineto(670.55838742,140.69487609)
\lineto(659.28885617,127.56987609)
\closepath
}
}
{
\newrgbcolor{curcolor}{0 0 0}
\pscustom[linestyle=none,fillstyle=solid,fillcolor=curcolor]
{
\newpath
\moveto(651.4920422,59.83299865)
\curveto(651.4920422,57.71841532)(651.02850054,55.80174865)(650.1014172,54.08299865)
\curveto(649.18475054,52.36424865)(647.9607922,51.03091532)(646.4295422,50.08299865)
\curveto(645.3670422,49.42674865)(644.1795422,48.95279032)(642.8670422,48.66112365)
\curveto(641.56495887,48.36945699)(639.84620887,48.22362365)(637.7107922,48.22362365)
\lineto(631.8357922,48.22362365)
\lineto(631.8357922,71.48924865)
\lineto(637.6482922,71.48924865)
\curveto(639.91912554,71.48924865)(641.72120887,71.32258199)(643.0545422,70.98924865)
\curveto(644.3982922,70.66633199)(645.53370887,70.21841532)(646.4607922,69.64549865)
\curveto(648.04412554,68.65591532)(649.27850054,67.33820699)(650.1639172,65.69237365)
\curveto(651.04933387,64.04654032)(651.4920422,62.09341532)(651.4920422,59.83299865)
\closepath
\moveto(648.2576672,59.87987365)
\curveto(648.2576672,61.70279032)(647.93995887,63.23924865)(647.3045422,64.48924865)
\curveto(646.66912554,65.73924865)(645.72120887,66.72362365)(644.4607922,67.44237365)
\curveto(643.54412554,67.96320699)(642.5701672,68.32258199)(641.5389172,68.52049865)
\curveto(640.5076672,68.72883199)(639.2732922,68.83299865)(637.8357922,68.83299865)
\lineto(634.9295422,68.83299865)
\lineto(634.9295422,50.87987365)
\lineto(637.8357922,50.87987365)
\curveto(639.32537554,50.87987365)(640.62225054,50.98924865)(641.7264172,51.20799865)
\curveto(642.84100054,51.42674865)(643.86183387,51.83299865)(644.7889172,52.42674865)
\curveto(645.9451672,53.16633199)(646.80975054,54.14029032)(647.3826672,55.34862365)
\curveto(647.96600054,56.55695699)(648.2576672,58.06737365)(648.2576672,59.87987365)
\closepath
}
}
{
\newrgbcolor{curcolor}{0 0 0}
\pscustom[linestyle=none,fillstyle=solid,fillcolor=curcolor]
{
\newpath
\moveto(669.8201672,48.22362365)
\lineto(666.8982922,48.22362365)
\lineto(666.8982922,50.08299865)
\curveto(666.63787554,49.90591532)(666.28370887,49.65591532)(665.8357922,49.33299865)
\curveto(665.3982922,49.02049865)(664.97120887,48.77049865)(664.5545422,48.58299865)
\curveto(664.06495887,48.34341532)(663.50245887,48.14549865)(662.8670422,47.98924865)
\curveto(662.23162554,47.82258199)(661.48683387,47.73924865)(660.6326672,47.73924865)
\curveto(659.05975054,47.73924865)(657.7264172,48.26008199)(656.6326672,49.30174865)
\curveto(655.5389172,50.34341532)(654.9920422,51.67154032)(654.9920422,53.28612365)
\curveto(654.9920422,54.60904032)(655.2732922,55.67674865)(655.8357922,56.48924865)
\curveto(656.40870887,57.31216532)(657.22120887,57.95799865)(658.2732922,58.42674865)
\curveto(659.3357922,58.89549865)(660.61183387,59.21320699)(662.1014172,59.37987365)
\curveto(663.59100054,59.54654032)(665.18995887,59.67154032)(666.8982922,59.75487365)
\lineto(666.8982922,60.20799865)
\curveto(666.8982922,60.87466532)(666.77850054,61.42674865)(666.5389172,61.86424865)
\curveto(666.30975054,62.30174865)(665.9764172,62.64549865)(665.5389172,62.89549865)
\curveto(665.12225054,63.13508199)(664.62225054,63.29654032)(664.0389172,63.37987365)
\curveto(663.45558387,63.46320699)(662.84620887,63.50487365)(662.2107922,63.50487365)
\curveto(661.43995887,63.50487365)(660.58058387,63.40070699)(659.6326672,63.19237365)
\curveto(658.68475054,62.99445699)(657.70558387,62.70279032)(656.6951672,62.31737365)
\lineto(656.5389172,62.31737365)
\lineto(656.5389172,65.30174865)
\curveto(657.11183387,65.45799865)(657.93995887,65.62987365)(659.0232922,65.81737365)
\curveto(660.10662554,66.00487365)(661.17433387,66.09862365)(662.2264172,66.09862365)
\curveto(663.45558387,66.09862365)(664.5232922,65.99445699)(665.4295422,65.78612365)
\curveto(666.34620887,65.58820699)(667.13787554,65.24445699)(667.8045422,64.75487365)
\curveto(668.4607922,64.27570699)(668.9607922,63.65591532)(669.3045422,62.89549865)
\curveto(669.6482922,62.13508199)(669.8201672,61.19237365)(669.8201672,60.06737365)
\closepath
\moveto(666.8982922,52.52049865)
\lineto(666.8982922,57.37987365)
\curveto(666.00245887,57.32779032)(664.9451672,57.24966532)(663.7264172,57.14549865)
\curveto(662.51808387,57.04133199)(661.55975054,56.89029032)(660.8514172,56.69237365)
\curveto(660.0076672,56.45279032)(659.32537554,56.07779032)(658.8045422,55.56737365)
\curveto(658.28370887,55.06737365)(658.0232922,54.37466532)(658.0232922,53.48924865)
\curveto(658.0232922,52.48924865)(658.32537554,51.73404032)(658.9295422,51.22362365)
\curveto(659.53370887,50.72362365)(660.45558387,50.47362365)(661.6951672,50.47362365)
\curveto(662.7264172,50.47362365)(663.66912554,50.67154032)(664.5232922,51.06737365)
\curveto(665.37745887,51.47362365)(666.16912554,51.95799865)(666.8982922,52.52049865)
\closepath
}
}
{
\newrgbcolor{curcolor}{0 0 0}
\pscustom[linestyle=none,fillstyle=solid,fillcolor=curcolor]
{
\newpath
\moveto(684.5389172,48.37987365)
\curveto(683.98683387,48.23404032)(683.3826672,48.11424865)(682.7264172,48.02049865)
\curveto(682.08058387,47.92674865)(681.50245887,47.87987365)(680.9920422,47.87987365)
\curveto(679.2107922,47.87987365)(677.85662554,48.35904032)(676.9295422,49.31737365)
\curveto(676.00245887,50.27570699)(675.5389172,51.81216532)(675.5389172,53.92674865)
\lineto(675.5389172,63.20799865)
\lineto(673.5545422,63.20799865)
\lineto(673.5545422,65.67674865)
\lineto(675.5389172,65.67674865)
\lineto(675.5389172,70.69237365)
\lineto(678.4764172,70.69237365)
\lineto(678.4764172,65.67674865)
\lineto(684.5389172,65.67674865)
\lineto(684.5389172,63.20799865)
\lineto(678.4764172,63.20799865)
\lineto(678.4764172,55.25487365)
\curveto(678.4764172,54.33820699)(678.49725054,53.61945699)(678.5389172,53.09862365)
\curveto(678.58058387,52.58820699)(678.7264172,52.10904032)(678.9764172,51.66112365)
\curveto(679.20558387,51.24445699)(679.51808387,50.93716532)(679.9139172,50.73924865)
\curveto(680.3201672,50.55174865)(680.93475054,50.45799865)(681.7576672,50.45799865)
\curveto(682.23683387,50.45799865)(682.73683387,50.52570699)(683.2576672,50.66112365)
\curveto(683.77850054,50.80695699)(684.15350054,50.92674865)(684.3826672,51.02049865)
\lineto(684.5389172,51.02049865)
\closepath
}
}
{
\newrgbcolor{curcolor}{0 0 0}
\pscustom[linestyle=none,fillstyle=solid,fillcolor=curcolor]
{
\newpath
\moveto(701.6482922,48.22362365)
\lineto(698.7264172,48.22362365)
\lineto(698.7264172,50.08299865)
\curveto(698.46600054,49.90591532)(698.11183387,49.65591532)(697.6639172,49.33299865)
\curveto(697.2264172,49.02049865)(696.79933387,48.77049865)(696.3826672,48.58299865)
\curveto(695.89308387,48.34341532)(695.33058387,48.14549865)(694.6951672,47.98924865)
\curveto(694.05975054,47.82258199)(693.31495887,47.73924865)(692.4607922,47.73924865)
\curveto(690.88787554,47.73924865)(689.5545422,48.26008199)(688.4607922,49.30174865)
\curveto(687.3670422,50.34341532)(686.8201672,51.67154032)(686.8201672,53.28612365)
\curveto(686.8201672,54.60904032)(687.1014172,55.67674865)(687.6639172,56.48924865)
\curveto(688.23683387,57.31216532)(689.04933387,57.95799865)(690.1014172,58.42674865)
\curveto(691.1639172,58.89549865)(692.43995887,59.21320699)(693.9295422,59.37987365)
\curveto(695.41912554,59.54654032)(697.01808387,59.67154032)(698.7264172,59.75487365)
\lineto(698.7264172,60.20799865)
\curveto(698.7264172,60.87466532)(698.60662554,61.42674865)(698.3670422,61.86424865)
\curveto(698.13787554,62.30174865)(697.8045422,62.64549865)(697.3670422,62.89549865)
\curveto(696.95037554,63.13508199)(696.45037554,63.29654032)(695.8670422,63.37987365)
\curveto(695.28370887,63.46320699)(694.67433387,63.50487365)(694.0389172,63.50487365)
\curveto(693.26808387,63.50487365)(692.40870887,63.40070699)(691.4607922,63.19237365)
\curveto(690.51287554,62.99445699)(689.53370887,62.70279032)(688.5232922,62.31737365)
\lineto(688.3670422,62.31737365)
\lineto(688.3670422,65.30174865)
\curveto(688.93995887,65.45799865)(689.76808387,65.62987365)(690.8514172,65.81737365)
\curveto(691.93475054,66.00487365)(693.00245887,66.09862365)(694.0545422,66.09862365)
\curveto(695.28370887,66.09862365)(696.3514172,65.99445699)(697.2576672,65.78612365)
\curveto(698.17433387,65.58820699)(698.96600054,65.24445699)(699.6326672,64.75487365)
\curveto(700.2889172,64.27570699)(700.7889172,63.65591532)(701.1326672,62.89549865)
\curveto(701.4764172,62.13508199)(701.6482922,61.19237365)(701.6482922,60.06737365)
\closepath
\moveto(698.7264172,52.52049865)
\lineto(698.7264172,57.37987365)
\curveto(697.83058387,57.32779032)(696.7732922,57.24966532)(695.5545422,57.14549865)
\curveto(694.34620887,57.04133199)(693.38787554,56.89029032)(692.6795422,56.69237365)
\curveto(691.8357922,56.45279032)(691.15350054,56.07779032)(690.6326672,55.56737365)
\curveto(690.11183387,55.06737365)(689.8514172,54.37466532)(689.8514172,53.48924865)
\curveto(689.8514172,52.48924865)(690.15350054,51.73404032)(690.7576672,51.22362365)
\curveto(691.36183387,50.72362365)(692.28370887,50.47362365)(693.5232922,50.47362365)
\curveto(694.5545422,50.47362365)(695.49725054,50.67154032)(696.3514172,51.06737365)
\curveto(697.20558387,51.47362365)(697.99725054,51.95799865)(698.7264172,52.52049865)
\closepath
}
}
{
\newrgbcolor{curcolor}{1 1 1}
\pscustom[linestyle=none,fillstyle=solid,fillcolor=curcolor]
{
\newpath
\moveto(929.6326672,220.5158905)
\lineto(1336.48407712,220.5158905)
\lineto(1336.48407712,1.00002136)
\lineto(929.6326672,1.00002136)
\closepath
}
}
{
\newrgbcolor{curcolor}{0 0 0}
\pscustom[linewidth=2,linecolor=curcolor]
{
\newpath
\moveto(929.6326672,220.5158905)
\lineto(1336.48407712,220.5158905)
\lineto(1336.48407712,1.00002136)
\lineto(929.6326672,1.00002136)
\closepath
}
}
{
\newrgbcolor{curcolor}{1 0.93333334 0.66666669}
\pscustom[linestyle=none,fillstyle=solid,fillcolor=curcolor]
{
\newpath
\moveto(951.76792111,97.5410675)
\lineto(1310.34882321,97.5410675)
\lineto(1310.34882321,21.68741455)
\lineto(951.76792111,21.68741455)
\closepath
}
}
{
\newrgbcolor{curcolor}{0 0 0}
\pscustom[linestyle=none,fillstyle=solid,fillcolor=curcolor]
{
\newpath
\moveto(1104.40598263,175.20659484)
\curveto(1104.40598263,173.76128234)(1104.13254513,172.48524068)(1103.58567013,171.37846984)
\curveto(1103.03879513,170.27169901)(1102.30311805,169.36024068)(1101.37863888,168.64409484)
\curveto(1100.28488888,167.78471984)(1099.0804618,167.17274068)(1097.76535763,166.80815734)
\curveto(1096.4632743,166.44357401)(1094.80311805,166.26128234)(1092.78488888,166.26128234)
\lineto(1082.47238888,166.26128234)
\lineto(1082.47238888,195.34331359)
\lineto(1091.08567013,195.34331359)
\curveto(1093.20806597,195.34331359)(1094.79660763,195.26518859)(1095.85129513,195.10893859)
\curveto(1096.90598263,194.95268859)(1097.91509722,194.62716776)(1098.87863888,194.13237609)
\curveto(1099.94634722,193.57248026)(1100.7210868,192.84982401)(1101.20285763,191.96440734)
\curveto(1101.68462847,191.09201151)(1101.92551388,190.04383443)(1101.92551388,188.81987609)
\curveto(1101.92551388,187.43966776)(1101.57395138,186.26128234)(1100.87082638,185.28471984)
\curveto(1100.16770138,184.32117818)(1099.23020138,183.54643859)(1098.05832638,182.96050109)
\lineto(1098.05832638,182.80425109)
\curveto(1100.02447222,182.40060526)(1101.57395138,181.53471984)(1102.70676388,180.20659484)
\curveto(1103.83957638,178.89149068)(1104.40598263,177.22482401)(1104.40598263,175.20659484)
\closepath
\moveto(1097.90207638,188.31206359)
\curveto(1097.90207638,189.01518859)(1097.78488888,189.60763651)(1097.55051388,190.08940734)
\curveto(1097.31613888,190.57117818)(1096.93853472,190.96180318)(1096.41770138,191.26128234)
\curveto(1095.80572222,191.61284484)(1095.06353472,191.82768859)(1094.19113888,191.90581359)
\curveto(1093.31874305,191.99695943)(1092.23801388,192.04253234)(1090.94895138,192.04253234)
\lineto(1086.33957638,192.04253234)
\lineto(1086.33957638,183.64409484)
\lineto(1091.33957638,183.64409484)
\curveto(1092.55051388,183.64409484)(1093.51405555,183.70268859)(1094.23020138,183.81987609)
\curveto(1094.94634722,183.95008443)(1095.61040972,184.21050109)(1096.22238888,184.60112609)
\curveto(1096.83436805,184.99175109)(1097.26405555,185.49305318)(1097.51145138,186.10503234)
\curveto(1097.77186805,186.73003234)(1097.90207638,187.46570943)(1097.90207638,188.31206359)
\closepath
\moveto(1100.38254513,175.05034484)
\curveto(1100.38254513,176.22221984)(1100.20676388,177.15320943)(1099.85520138,177.84331359)
\curveto(1099.50363888,178.53341776)(1098.86561805,179.11935526)(1097.94113888,179.60112609)
\curveto(1097.31613888,179.92664693)(1096.55442013,180.13498026)(1095.65598263,180.22612609)
\curveto(1094.77056597,180.33029276)(1093.6898368,180.38237609)(1092.41379513,180.38237609)
\lineto(1086.33957638,180.38237609)
\lineto(1086.33957638,169.56206359)
\lineto(1091.45676388,169.56206359)
\curveto(1093.14947222,169.56206359)(1094.53619097,169.64669901)(1095.61692013,169.81596984)
\curveto(1096.6976493,169.99826151)(1097.58306597,170.32378234)(1098.27317013,170.79253234)
\curveto(1099.0023368,171.30034484)(1099.53619097,171.87977193)(1099.87473263,172.53081359)
\curveto(1100.2132743,173.18185526)(1100.38254513,174.02169901)(1100.38254513,175.05034484)
\closepath
}
}
{
\newrgbcolor{curcolor}{0 0 0}
\pscustom[linestyle=none,fillstyle=solid,fillcolor=curcolor]
{
\newpath
\moveto(1113.31223263,166.26128234)
\lineto(1109.64035763,166.26128234)
\lineto(1109.64035763,196.65190734)
\lineto(1113.31223263,196.65190734)
\closepath
}
}
{
\newrgbcolor{curcolor}{0 0 0}
\pscustom[linestyle=none,fillstyle=solid,fillcolor=curcolor]
{
\newpath
\moveto(1139.17160763,177.15971984)
\curveto(1139.17160763,173.60503234)(1138.2601493,170.79904276)(1136.43723263,168.74175109)
\curveto(1134.61431597,166.68445943)(1132.17290972,165.65581359)(1129.11301388,165.65581359)
\curveto(1126.02707638,165.65581359)(1123.5726493,166.68445943)(1121.74973263,168.74175109)
\curveto(1119.9398368,170.79904276)(1119.03488888,173.60503234)(1119.03488888,177.15971984)
\curveto(1119.03488888,180.71440734)(1119.9398368,183.52039693)(1121.74973263,185.57768859)
\curveto(1123.5726493,187.64800109)(1126.02707638,188.68315734)(1129.11301388,188.68315734)
\curveto(1132.17290972,188.68315734)(1134.61431597,187.64800109)(1136.43723263,185.57768859)
\curveto(1138.2601493,183.52039693)(1139.17160763,180.71440734)(1139.17160763,177.15971984)
\closepath
\moveto(1135.38254513,177.15971984)
\curveto(1135.38254513,179.98524068)(1134.82915972,182.08159484)(1133.72238888,183.44878234)
\curveto(1132.61561805,184.82899068)(1131.07915972,185.51909484)(1129.11301388,185.51909484)
\curveto(1127.12082638,185.51909484)(1125.57134722,184.82899068)(1124.46457638,183.44878234)
\curveto(1123.37082638,182.08159484)(1122.82395138,179.98524068)(1122.82395138,177.15971984)
\curveto(1122.82395138,174.42534484)(1123.3773368,172.34852193)(1124.48410763,170.92925109)
\curveto(1125.59087847,169.52300109)(1127.13384722,168.81987609)(1129.11301388,168.81987609)
\curveto(1131.06613888,168.81987609)(1132.5960868,169.51649068)(1133.70285763,170.90971984)
\curveto(1134.8226493,172.31596984)(1135.38254513,174.39930318)(1135.38254513,177.15971984)
\closepath
}
}
{
\newrgbcolor{curcolor}{0 0 0}
\pscustom[linestyle=none,fillstyle=solid,fillcolor=curcolor]
{
\newpath
\moveto(1160.98801388,167.62846984)
\curveto(1159.76405555,167.04253234)(1158.59869097,166.58680318)(1157.49192013,166.26128234)
\curveto(1156.39817013,165.93576151)(1155.23280555,165.77300109)(1153.99582638,165.77300109)
\curveto(1152.42030555,165.77300109)(1150.97499305,166.00086568)(1149.65988888,166.45659484)
\curveto(1148.34478472,166.92534484)(1147.21848263,167.62846984)(1146.28098263,168.56596984)
\curveto(1145.3304618,169.50346984)(1144.59478472,170.68836568)(1144.07395138,172.12065734)
\curveto(1143.55311805,173.55294901)(1143.29270138,175.22612609)(1143.29270138,177.14018859)
\curveto(1143.29270138,180.70789693)(1144.26926388,183.50737609)(1146.22238888,185.53862609)
\curveto(1148.18853472,187.56987609)(1150.77968055,188.58550109)(1153.99582638,188.58550109)
\curveto(1155.24582638,188.58550109)(1156.46978472,188.40971984)(1157.66770138,188.05815734)
\curveto(1158.87863888,187.70659484)(1159.98540972,187.27690734)(1160.98801388,186.76909484)
\lineto(1160.98801388,182.68706359)
\lineto(1160.79270138,182.68706359)
\curveto(1159.67290972,183.55945943)(1158.51405555,184.23003234)(1157.31613888,184.69878234)
\curveto(1156.13124305,185.16753234)(1154.97238888,185.40190734)(1153.83957638,185.40190734)
\curveto(1151.75624305,185.40190734)(1150.10910763,184.69878234)(1148.89817013,183.29253234)
\curveto(1147.70025347,181.89930318)(1147.10129513,179.84852193)(1147.10129513,177.14018859)
\curveto(1147.10129513,174.50998026)(1147.68723263,172.48524068)(1148.85910763,171.06596984)
\curveto(1150.04400347,169.65971984)(1151.70415972,168.95659484)(1153.83957638,168.95659484)
\curveto(1154.58176388,168.95659484)(1155.33697222,169.05425109)(1156.10520138,169.24956359)
\curveto(1156.87343055,169.44487609)(1157.56353472,169.69878234)(1158.17551388,170.01128234)
\curveto(1158.70936805,170.28471984)(1159.21067013,170.57117818)(1159.67942013,170.87065734)
\curveto(1160.14817013,171.18315734)(1160.51926388,171.45008443)(1160.79270138,171.67143859)
\lineto(1160.98801388,171.67143859)
\closepath
}
}
{
\newrgbcolor{curcolor}{0 0 0}
\pscustom[linestyle=none,fillstyle=solid,fillcolor=curcolor]
{
\newpath
\moveto(1185.49973263,166.26128234)
\lineto(1180.65598263,166.26128234)
\lineto(1171.90598263,175.81206359)
\lineto(1169.52317013,173.54643859)
\lineto(1169.52317013,166.26128234)
\lineto(1165.85129513,166.26128234)
\lineto(1165.85129513,196.65190734)
\lineto(1169.52317013,196.65190734)
\lineto(1169.52317013,177.15971984)
\lineto(1180.12863888,188.07768859)
\lineto(1184.75754513,188.07768859)
\lineto(1174.62082638,177.99956359)
\closepath
}
}
{
\newrgbcolor{curcolor}{0 0 0}
\pscustom[linestyle=none,fillstyle=solid,fillcolor=curcolor]
{
\newpath
\moveto(1142.04270138,125.49956359)
\curveto(1142.04270138,124.14539693)(1141.79530555,122.84982401)(1141.30051388,121.61284484)
\curveto(1140.80572222,120.37586568)(1140.12863888,119.33419901)(1139.26926388,118.48784484)
\curveto(1138.33176388,117.57638651)(1137.21197222,116.87326151)(1135.90988888,116.37846984)
\curveto(1134.62082638,115.89669901)(1133.12343055,115.65581359)(1131.41770138,115.65581359)
\curveto(1129.82915972,115.65581359)(1128.2992118,115.82508443)(1126.82785763,116.16362609)
\curveto(1125.35650347,116.48914693)(1124.11301388,116.88628234)(1123.09738888,117.35503234)
\lineto(1123.09738888,121.47612609)
\lineto(1123.37082638,121.47612609)
\curveto(1124.43853472,120.79904276)(1125.68853472,120.21961568)(1127.12082638,119.73784484)
\curveto(1128.55311805,119.26909484)(1129.95936805,119.03471984)(1131.33957638,119.03471984)
\curveto(1132.26405555,119.03471984)(1133.15598263,119.16492818)(1134.01535763,119.42534484)
\curveto(1134.88775347,119.68576151)(1135.66249305,120.14149068)(1136.33957638,120.79253234)
\curveto(1136.91249305,121.35242818)(1137.34218055,122.02300109)(1137.62863888,122.80425109)
\curveto(1137.92811805,123.58550109)(1138.07785763,124.49044901)(1138.07785763,125.51909484)
\curveto(1138.07785763,126.52169901)(1137.90207638,127.36805318)(1137.55051388,128.05815734)
\curveto(1137.21197222,128.74826151)(1136.7367118,129.30164693)(1136.12473263,129.71831359)
\curveto(1135.4476493,130.21310526)(1134.62082638,130.55815734)(1133.64426388,130.75346984)
\curveto(1132.68072222,130.96180318)(1131.59999305,131.06596984)(1130.40207638,131.06596984)
\curveto(1129.25624305,131.06596984)(1128.14947222,130.98784484)(1127.08176388,130.83159484)
\curveto(1126.02707638,130.67534484)(1125.11561805,130.51909484)(1124.34738888,130.36284484)
\lineto(1124.34738888,145.34331359)
\lineto(1141.84738888,145.34331359)
\lineto(1141.84738888,141.92534484)
\lineto(1128.11692013,141.92534484)
\lineto(1128.11692013,134.19096984)
\curveto(1128.67681597,134.24305318)(1129.24973263,134.28211568)(1129.83567013,134.30815734)
\curveto(1130.42160763,134.33419901)(1130.92942013,134.34721984)(1131.35910763,134.34721984)
\curveto(1132.93462847,134.34721984)(1134.3148368,134.21050109)(1135.49973263,133.93706359)
\curveto(1136.68462847,133.67664693)(1137.77186805,133.20789693)(1138.76145138,132.53081359)
\curveto(1139.80311805,131.81466776)(1140.61040972,130.89018859)(1141.18332638,129.75737609)
\curveto(1141.75624305,128.62456359)(1142.04270138,127.20529276)(1142.04270138,125.49956359)
\closepath
}
}
{
\newrgbcolor{curcolor}{0 0 0}
\pscustom[linestyle=none,fillstyle=solid,fillcolor=curcolor]
{
\newpath
\moveto(1115.80832638,59.83299865)
\curveto(1115.80832638,57.71841532)(1115.34478472,55.80174865)(1114.41770138,54.08299865)
\curveto(1113.50103472,52.36424865)(1112.27707638,51.03091532)(1110.74582638,50.08299865)
\curveto(1109.68332638,49.42674865)(1108.49582638,48.95279032)(1107.18332638,48.66112365)
\curveto(1105.88124305,48.36945699)(1104.16249305,48.22362365)(1102.02707638,48.22362365)
\lineto(1096.15207638,48.22362365)
\lineto(1096.15207638,71.48924865)
\lineto(1101.96457638,71.48924865)
\curveto(1104.23540972,71.48924865)(1106.03749305,71.32258199)(1107.37082638,70.98924865)
\curveto(1108.71457638,70.66633199)(1109.84999305,70.21841532)(1110.77707638,69.64549865)
\curveto(1112.36040972,68.65591532)(1113.59478472,67.33820699)(1114.48020138,65.69237365)
\curveto(1115.36561805,64.04654032)(1115.80832638,62.09341532)(1115.80832638,59.83299865)
\closepath
\moveto(1112.57395138,59.87987365)
\curveto(1112.57395138,61.70279032)(1112.25624305,63.23924865)(1111.62082638,64.48924865)
\curveto(1110.98540972,65.73924865)(1110.03749305,66.72362365)(1108.77707638,67.44237365)
\curveto(1107.86040972,67.96320699)(1106.88645138,68.32258199)(1105.85520138,68.52049865)
\curveto(1104.82395138,68.72883199)(1103.58957638,68.83299865)(1102.15207638,68.83299865)
\lineto(1099.24582638,68.83299865)
\lineto(1099.24582638,50.87987365)
\lineto(1102.15207638,50.87987365)
\curveto(1103.64165972,50.87987365)(1104.93853472,50.98924865)(1106.04270138,51.20799865)
\curveto(1107.15728472,51.42674865)(1108.17811805,51.83299865)(1109.10520138,52.42674865)
\curveto(1110.26145138,53.16633199)(1111.12603472,54.14029032)(1111.69895138,55.34862365)
\curveto(1112.28228472,56.55695699)(1112.57395138,58.06737365)(1112.57395138,59.87987365)
\closepath
}
}
{
\newrgbcolor{curcolor}{0 0 0}
\pscustom[linestyle=none,fillstyle=solid,fillcolor=curcolor]
{
\newpath
\moveto(1134.13645138,48.22362365)
\lineto(1131.21457638,48.22362365)
\lineto(1131.21457638,50.08299865)
\curveto(1130.95415972,49.90591532)(1130.59999305,49.65591532)(1130.15207638,49.33299865)
\curveto(1129.71457638,49.02049865)(1129.28749305,48.77049865)(1128.87082638,48.58299865)
\curveto(1128.38124305,48.34341532)(1127.81874305,48.14549865)(1127.18332638,47.98924865)
\curveto(1126.54790972,47.82258199)(1125.80311805,47.73924865)(1124.94895138,47.73924865)
\curveto(1123.37603472,47.73924865)(1122.04270138,48.26008199)(1120.94895138,49.30174865)
\curveto(1119.85520138,50.34341532)(1119.30832638,51.67154032)(1119.30832638,53.28612365)
\curveto(1119.30832638,54.60904032)(1119.58957638,55.67674865)(1120.15207638,56.48924865)
\curveto(1120.72499305,57.31216532)(1121.53749305,57.95799865)(1122.58957638,58.42674865)
\curveto(1123.65207638,58.89549865)(1124.92811805,59.21320699)(1126.41770138,59.37987365)
\curveto(1127.90728472,59.54654032)(1129.50624305,59.67154032)(1131.21457638,59.75487365)
\lineto(1131.21457638,60.20799865)
\curveto(1131.21457638,60.87466532)(1131.09478472,61.42674865)(1130.85520138,61.86424865)
\curveto(1130.62603472,62.30174865)(1130.29270138,62.64549865)(1129.85520138,62.89549865)
\curveto(1129.43853472,63.13508199)(1128.93853472,63.29654032)(1128.35520138,63.37987365)
\curveto(1127.77186805,63.46320699)(1127.16249305,63.50487365)(1126.52707638,63.50487365)
\curveto(1125.75624305,63.50487365)(1124.89686805,63.40070699)(1123.94895138,63.19237365)
\curveto(1123.00103472,62.99445699)(1122.02186805,62.70279032)(1121.01145138,62.31737365)
\lineto(1120.85520138,62.31737365)
\lineto(1120.85520138,65.30174865)
\curveto(1121.42811805,65.45799865)(1122.25624305,65.62987365)(1123.33957638,65.81737365)
\curveto(1124.42290972,66.00487365)(1125.49061805,66.09862365)(1126.54270138,66.09862365)
\curveto(1127.77186805,66.09862365)(1128.83957638,65.99445699)(1129.74582638,65.78612365)
\curveto(1130.66249305,65.58820699)(1131.45415972,65.24445699)(1132.12082638,64.75487365)
\curveto(1132.77707638,64.27570699)(1133.27707638,63.65591532)(1133.62082638,62.89549865)
\curveto(1133.96457638,62.13508199)(1134.13645138,61.19237365)(1134.13645138,60.06737365)
\closepath
\moveto(1131.21457638,52.52049865)
\lineto(1131.21457638,57.37987365)
\curveto(1130.31874305,57.32779032)(1129.26145138,57.24966532)(1128.04270138,57.14549865)
\curveto(1126.83436805,57.04133199)(1125.87603472,56.89029032)(1125.16770138,56.69237365)
\curveto(1124.32395138,56.45279032)(1123.64165972,56.07779032)(1123.12082638,55.56737365)
\curveto(1122.59999305,55.06737365)(1122.33957638,54.37466532)(1122.33957638,53.48924865)
\curveto(1122.33957638,52.48924865)(1122.64165972,51.73404032)(1123.24582638,51.22362365)
\curveto(1123.84999305,50.72362365)(1124.77186805,50.47362365)(1126.01145138,50.47362365)
\curveto(1127.04270138,50.47362365)(1127.98540972,50.67154032)(1128.83957638,51.06737365)
\curveto(1129.69374305,51.47362365)(1130.48540972,51.95799865)(1131.21457638,52.52049865)
\closepath
}
}
{
\newrgbcolor{curcolor}{0 0 0}
\pscustom[linestyle=none,fillstyle=solid,fillcolor=curcolor]
{
\newpath
\moveto(1148.85520138,48.37987365)
\curveto(1148.30311805,48.23404032)(1147.69895138,48.11424865)(1147.04270138,48.02049865)
\curveto(1146.39686805,47.92674865)(1145.81874305,47.87987365)(1145.30832638,47.87987365)
\curveto(1143.52707638,47.87987365)(1142.17290972,48.35904032)(1141.24582638,49.31737365)
\curveto(1140.31874305,50.27570699)(1139.85520138,51.81216532)(1139.85520138,53.92674865)
\lineto(1139.85520138,63.20799865)
\lineto(1137.87082638,63.20799865)
\lineto(1137.87082638,65.67674865)
\lineto(1139.85520138,65.67674865)
\lineto(1139.85520138,70.69237365)
\lineto(1142.79270138,70.69237365)
\lineto(1142.79270138,65.67674865)
\lineto(1148.85520138,65.67674865)
\lineto(1148.85520138,63.20799865)
\lineto(1142.79270138,63.20799865)
\lineto(1142.79270138,55.25487365)
\curveto(1142.79270138,54.33820699)(1142.81353472,53.61945699)(1142.85520138,53.09862365)
\curveto(1142.89686805,52.58820699)(1143.04270138,52.10904032)(1143.29270138,51.66112365)
\curveto(1143.52186805,51.24445699)(1143.83436805,50.93716532)(1144.23020138,50.73924865)
\curveto(1144.63645138,50.55174865)(1145.25103472,50.45799865)(1146.07395138,50.45799865)
\curveto(1146.55311805,50.45799865)(1147.05311805,50.52570699)(1147.57395138,50.66112365)
\curveto(1148.09478472,50.80695699)(1148.46978472,50.92674865)(1148.69895138,51.02049865)
\lineto(1148.85520138,51.02049865)
\closepath
}
}
{
\newrgbcolor{curcolor}{0 0 0}
\pscustom[linestyle=none,fillstyle=solid,fillcolor=curcolor]
{
\newpath
\moveto(1165.96457638,48.22362365)
\lineto(1163.04270138,48.22362365)
\lineto(1163.04270138,50.08299865)
\curveto(1162.78228472,49.90591532)(1162.42811805,49.65591532)(1161.98020138,49.33299865)
\curveto(1161.54270138,49.02049865)(1161.11561805,48.77049865)(1160.69895138,48.58299865)
\curveto(1160.20936805,48.34341532)(1159.64686805,48.14549865)(1159.01145138,47.98924865)
\curveto(1158.37603472,47.82258199)(1157.63124305,47.73924865)(1156.77707638,47.73924865)
\curveto(1155.20415972,47.73924865)(1153.87082638,48.26008199)(1152.77707638,49.30174865)
\curveto(1151.68332638,50.34341532)(1151.13645138,51.67154032)(1151.13645138,53.28612365)
\curveto(1151.13645138,54.60904032)(1151.41770138,55.67674865)(1151.98020138,56.48924865)
\curveto(1152.55311805,57.31216532)(1153.36561805,57.95799865)(1154.41770138,58.42674865)
\curveto(1155.48020138,58.89549865)(1156.75624305,59.21320699)(1158.24582638,59.37987365)
\curveto(1159.73540972,59.54654032)(1161.33436805,59.67154032)(1163.04270138,59.75487365)
\lineto(1163.04270138,60.20799865)
\curveto(1163.04270138,60.87466532)(1162.92290972,61.42674865)(1162.68332638,61.86424865)
\curveto(1162.45415972,62.30174865)(1162.12082638,62.64549865)(1161.68332638,62.89549865)
\curveto(1161.26665972,63.13508199)(1160.76665972,63.29654032)(1160.18332638,63.37987365)
\curveto(1159.59999305,63.46320699)(1158.99061805,63.50487365)(1158.35520138,63.50487365)
\curveto(1157.58436805,63.50487365)(1156.72499305,63.40070699)(1155.77707638,63.19237365)
\curveto(1154.82915972,62.99445699)(1153.84999305,62.70279032)(1152.83957638,62.31737365)
\lineto(1152.68332638,62.31737365)
\lineto(1152.68332638,65.30174865)
\curveto(1153.25624305,65.45799865)(1154.08436805,65.62987365)(1155.16770138,65.81737365)
\curveto(1156.25103472,66.00487365)(1157.31874305,66.09862365)(1158.37082638,66.09862365)
\curveto(1159.59999305,66.09862365)(1160.66770138,65.99445699)(1161.57395138,65.78612365)
\curveto(1162.49061805,65.58820699)(1163.28228472,65.24445699)(1163.94895138,64.75487365)
\curveto(1164.60520138,64.27570699)(1165.10520138,63.65591532)(1165.44895138,62.89549865)
\curveto(1165.79270138,62.13508199)(1165.96457638,61.19237365)(1165.96457638,60.06737365)
\closepath
\moveto(1163.04270138,52.52049865)
\lineto(1163.04270138,57.37987365)
\curveto(1162.14686805,57.32779032)(1161.08957638,57.24966532)(1159.87082638,57.14549865)
\curveto(1158.66249305,57.04133199)(1157.70415972,56.89029032)(1156.99582638,56.69237365)
\curveto(1156.15207638,56.45279032)(1155.46978472,56.07779032)(1154.94895138,55.56737365)
\curveto(1154.42811805,55.06737365)(1154.16770138,54.37466532)(1154.16770138,53.48924865)
\curveto(1154.16770138,52.48924865)(1154.46978472,51.73404032)(1155.07395138,51.22362365)
\curveto(1155.67811805,50.72362365)(1156.59999305,50.47362365)(1157.83957638,50.47362365)
\curveto(1158.87082638,50.47362365)(1159.81353472,50.67154032)(1160.66770138,51.06737365)
\curveto(1161.52186805,51.47362365)(1162.31353472,51.95799865)(1163.04270138,52.52049865)
\closepath
}
}
{
\newrgbcolor{curcolor}{1 1 1}
\pscustom[linestyle=none,fillstyle=solid,fillcolor=curcolor]
{
\newpath
\moveto(1393.94895138,220.5158905)
\lineto(1800.80036129,220.5158905)
\lineto(1800.80036129,1.00002136)
\lineto(1393.94895138,1.00002136)
\closepath
}
}
{
\newrgbcolor{curcolor}{0 0 0}
\pscustom[linewidth=2,linecolor=curcolor]
{
\newpath
\moveto(1393.94895138,220.5158905)
\lineto(1800.80036129,220.5158905)
\lineto(1800.80036129,1.00002136)
\lineto(1393.94895138,1.00002136)
\closepath
}
}
{
\newrgbcolor{curcolor}{1 0.93333334 0.66666669}
\pscustom[linestyle=none,fillstyle=solid,fillcolor=curcolor]
{
\newpath
\moveto(1416.08420529,97.5410675)
\lineto(1774.66510739,97.5410675)
\lineto(1774.66510739,21.68741455)
\lineto(1416.08420529,21.68741455)
\closepath
}
}
{
\newrgbcolor{curcolor}{0 0 0}
\pscustom[linestyle=none,fillstyle=solid,fillcolor=curcolor]
{
\newpath
\moveto(1568.72238888,174.92339172)
\curveto(1568.72238888,173.47807922)(1568.44895138,172.20203755)(1567.90207638,171.09526672)
\curveto(1567.35520138,169.98849589)(1566.6195243,169.07703755)(1565.69504513,168.36089172)
\curveto(1564.60129513,167.50151672)(1563.39686805,166.88953755)(1562.08176388,166.52495422)
\curveto(1560.77968055,166.16037089)(1559.1195243,165.97807922)(1557.10129513,165.97807922)
\lineto(1546.78879513,165.97807922)
\lineto(1546.78879513,195.06011047)
\lineto(1555.40207638,195.06011047)
\curveto(1557.52447222,195.06011047)(1559.11301388,194.98198547)(1560.16770138,194.82573547)
\curveto(1561.22238888,194.66948547)(1562.23150347,194.34396464)(1563.19504513,193.84917297)
\curveto(1564.26275347,193.28927714)(1565.03749305,192.56662089)(1565.51926388,191.68120422)
\curveto(1566.00103472,190.80880839)(1566.24192013,189.7606313)(1566.24192013,188.53667297)
\curveto(1566.24192013,187.15646464)(1565.89035763,185.97807922)(1565.18723263,185.00151672)
\curveto(1564.48410763,184.03797505)(1563.54660763,183.26323547)(1562.37473263,182.67729797)
\lineto(1562.37473263,182.52104797)
\curveto(1564.34087847,182.11740214)(1565.89035763,181.25151672)(1567.02317013,179.92339172)
\curveto(1568.15598263,178.60828755)(1568.72238888,176.94162089)(1568.72238888,174.92339172)
\closepath
\moveto(1562.21848263,188.02886047)
\curveto(1562.21848263,188.73198547)(1562.10129513,189.32443339)(1561.86692013,189.80620422)
\curveto(1561.63254513,190.28797505)(1561.25494097,190.67860005)(1560.73410763,190.97807922)
\curveto(1560.12212847,191.32964172)(1559.37994097,191.54448547)(1558.50754513,191.62261047)
\curveto(1557.6351493,191.7137563)(1556.55442013,191.75932922)(1555.26535763,191.75932922)
\lineto(1550.65598263,191.75932922)
\lineto(1550.65598263,183.36089172)
\lineto(1555.65598263,183.36089172)
\curveto(1556.86692013,183.36089172)(1557.8304618,183.41948547)(1558.54660763,183.53667297)
\curveto(1559.26275347,183.6668813)(1559.92681597,183.92729797)(1560.53879513,184.31792297)
\curveto(1561.1507743,184.70854797)(1561.5804618,185.20985005)(1561.82785763,185.82182922)
\curveto(1562.0882743,186.44682922)(1562.21848263,187.1825063)(1562.21848263,188.02886047)
\closepath
\moveto(1564.69895138,174.76714172)
\curveto(1564.69895138,175.93901672)(1564.52317013,176.8700063)(1564.17160763,177.56011047)
\curveto(1563.82004513,178.25021464)(1563.1820243,178.83615214)(1562.25754513,179.31792297)
\curveto(1561.63254513,179.6434438)(1560.87082638,179.85177714)(1559.97238888,179.94292297)
\curveto(1559.08697222,180.04708964)(1558.00624305,180.09917297)(1556.73020138,180.09917297)
\lineto(1550.65598263,180.09917297)
\lineto(1550.65598263,169.27886047)
\lineto(1555.77317013,169.27886047)
\curveto(1557.46587847,169.27886047)(1558.85259722,169.36349589)(1559.93332638,169.53276672)
\curveto(1561.01405555,169.71505839)(1561.89947222,170.04057922)(1562.58957638,170.50932922)
\curveto(1563.31874305,171.01714172)(1563.85259722,171.5965688)(1564.19113888,172.24761047)
\curveto(1564.52968055,172.89865214)(1564.69895138,173.73849589)(1564.69895138,174.76714172)
\closepath
}
}
{
\newrgbcolor{curcolor}{0 0 0}
\pscustom[linestyle=none,fillstyle=solid,fillcolor=curcolor]
{
\newpath
\moveto(1577.62863888,165.97807922)
\lineto(1573.95676388,165.97807922)
\lineto(1573.95676388,196.36870422)
\lineto(1577.62863888,196.36870422)
\closepath
}
}
{
\newrgbcolor{curcolor}{0 0 0}
\pscustom[linestyle=none,fillstyle=solid,fillcolor=curcolor]
{
\newpath
\moveto(1603.48801388,176.87651672)
\curveto(1603.48801388,173.32182922)(1602.57655555,170.51583964)(1600.75363888,168.45854797)
\curveto(1598.93072222,166.4012563)(1596.48931597,165.37261047)(1593.42942013,165.37261047)
\curveto(1590.34348263,165.37261047)(1587.88905555,166.4012563)(1586.06613888,168.45854797)
\curveto(1584.25624305,170.51583964)(1583.35129513,173.32182922)(1583.35129513,176.87651672)
\curveto(1583.35129513,180.43120422)(1584.25624305,183.2371938)(1586.06613888,185.29448547)
\curveto(1587.88905555,187.36479797)(1590.34348263,188.39995422)(1593.42942013,188.39995422)
\curveto(1596.48931597,188.39995422)(1598.93072222,187.36479797)(1600.75363888,185.29448547)
\curveto(1602.57655555,183.2371938)(1603.48801388,180.43120422)(1603.48801388,176.87651672)
\closepath
\moveto(1599.69895138,176.87651672)
\curveto(1599.69895138,179.70203755)(1599.14556597,181.79839172)(1598.03879513,183.16557922)
\curveto(1596.9320243,184.54578755)(1595.39556597,185.23589172)(1593.42942013,185.23589172)
\curveto(1591.43723263,185.23589172)(1589.88775347,184.54578755)(1588.78098263,183.16557922)
\curveto(1587.68723263,181.79839172)(1587.14035763,179.70203755)(1587.14035763,176.87651672)
\curveto(1587.14035763,174.14214172)(1587.69374305,172.0653188)(1588.80051388,170.64604797)
\curveto(1589.90728472,169.23979797)(1591.45025347,168.53667297)(1593.42942013,168.53667297)
\curveto(1595.38254513,168.53667297)(1596.91249305,169.23328755)(1598.01926388,170.62651672)
\curveto(1599.13905555,172.03276672)(1599.69895138,174.11610005)(1599.69895138,176.87651672)
\closepath
}
}
{
\newrgbcolor{curcolor}{0 0 0}
\pscustom[linestyle=none,fillstyle=solid,fillcolor=curcolor]
{
\newpath
\moveto(1625.30442013,167.34526672)
\curveto(1624.0804618,166.75932922)(1622.91509722,166.30360005)(1621.80832638,165.97807922)
\curveto(1620.71457638,165.65255839)(1619.5492118,165.48979797)(1618.31223263,165.48979797)
\curveto(1616.7367118,165.48979797)(1615.2913993,165.71766255)(1613.97629513,166.17339172)
\curveto(1612.66119097,166.64214172)(1611.53488888,167.34526672)(1610.59738888,168.28276672)
\curveto(1609.64686805,169.22026672)(1608.91119097,170.40516255)(1608.39035763,171.83745422)
\curveto(1607.8695243,173.26974589)(1607.60910763,174.94292297)(1607.60910763,176.85698547)
\curveto(1607.60910763,180.4246938)(1608.58567013,183.22417297)(1610.53879513,185.25542297)
\curveto(1612.50494097,187.28667297)(1615.0960868,188.30229797)(1618.31223263,188.30229797)
\curveto(1619.56223263,188.30229797)(1620.78619097,188.12651672)(1621.98410763,187.77495422)
\curveto(1623.19504513,187.42339172)(1624.30181597,186.99370422)(1625.30442013,186.48589172)
\lineto(1625.30442013,182.40386047)
\lineto(1625.10910763,182.40386047)
\curveto(1623.98931597,183.2762563)(1622.8304618,183.94682922)(1621.63254513,184.41557922)
\curveto(1620.4476493,184.88432922)(1619.28879513,185.11870422)(1618.15598263,185.11870422)
\curveto(1616.0726493,185.11870422)(1614.42551388,184.41557922)(1613.21457638,183.00932922)
\curveto(1612.01665972,181.61610005)(1611.41770138,179.5653188)(1611.41770138,176.85698547)
\curveto(1611.41770138,174.22677714)(1612.00363888,172.20203755)(1613.17551388,170.78276672)
\curveto(1614.36040972,169.37651672)(1616.02056597,168.67339172)(1618.15598263,168.67339172)
\curveto(1618.89817013,168.67339172)(1619.65337847,168.77104797)(1620.42160763,168.96636047)
\curveto(1621.1898368,169.16167297)(1621.87994097,169.41557922)(1622.49192013,169.72807922)
\curveto(1623.0257743,170.00151672)(1623.52707638,170.28797505)(1623.99582638,170.58745422)
\curveto(1624.46457638,170.89995422)(1624.83567013,171.1668813)(1625.10910763,171.38823547)
\lineto(1625.30442013,171.38823547)
\closepath
}
}
{
\newrgbcolor{curcolor}{0 0 0}
\pscustom[linestyle=none,fillstyle=solid,fillcolor=curcolor]
{
\newpath
\moveto(1649.81613888,165.97807922)
\lineto(1644.97238888,165.97807922)
\lineto(1636.22238888,175.52886047)
\lineto(1633.83957638,173.26323547)
\lineto(1633.83957638,165.97807922)
\lineto(1630.16770138,165.97807922)
\lineto(1630.16770138,196.36870422)
\lineto(1633.83957638,196.36870422)
\lineto(1633.83957638,176.87651672)
\lineto(1644.44504513,187.79448547)
\lineto(1649.07395138,187.79448547)
\lineto(1638.93723263,177.71636047)
\closepath
}
}
{
\newrgbcolor{curcolor}{0 0 0}
\pscustom[linestyle=none,fillstyle=solid,fillcolor=curcolor]
{
\newpath
\moveto(1607.02317013,125.41167297)
\curveto(1607.02317013,122.4559438)(1606.04660763,120.04057922)(1604.09348263,118.16557922)
\curveto(1602.15337847,116.30360005)(1599.77056597,115.37261047)(1596.94504513,115.37261047)
\curveto(1595.51275347,115.37261047)(1594.21067013,115.59396464)(1593.03879513,116.03667297)
\curveto(1591.86692013,116.4793813)(1590.83176388,117.13693339)(1589.93332638,118.00932922)
\curveto(1588.81353472,119.09005839)(1587.9476493,120.52235005)(1587.33567013,122.30620422)
\curveto(1586.7367118,124.09005839)(1586.43723263,126.23849589)(1586.43723263,128.75151672)
\curveto(1586.43723263,131.32964172)(1586.71067013,133.61479797)(1587.25754513,135.60698547)
\curveto(1587.81744097,137.59917297)(1588.70285763,139.3700063)(1589.91379513,140.91948547)
\curveto(1591.05962847,142.39083964)(1592.53749305,143.53667297)(1594.34738888,144.35698547)
\curveto(1596.15728472,145.1903188)(1598.26665972,145.60698547)(1600.67551388,145.60698547)
\curveto(1601.44374305,145.60698547)(1602.0882743,145.57443339)(1602.60910763,145.50932922)
\curveto(1603.12994097,145.44422505)(1603.65728472,145.32703755)(1604.19113888,145.15776672)
\lineto(1604.19113888,141.42729797)
\lineto(1603.99582638,141.42729797)
\curveto(1603.63124305,141.62261047)(1603.07785763,141.80490214)(1602.33567013,141.97417297)
\curveto(1601.60650347,142.15646464)(1600.85780555,142.24761047)(1600.08957638,142.24761047)
\curveto(1597.29009722,142.24761047)(1595.0570243,141.36870422)(1593.39035763,139.61089172)
\curveto(1591.72369097,137.86610005)(1590.75363888,135.5028188)(1590.48020138,132.52104797)
\curveto(1591.57395138,133.18511047)(1592.64817013,133.68641255)(1593.70285763,134.02495422)
\curveto(1594.77056597,134.37651672)(1596.00103472,134.55229797)(1597.39426388,134.55229797)
\curveto(1598.63124305,134.55229797)(1599.71848263,134.43511047)(1600.65598263,134.20073547)
\curveto(1601.60650347,133.9793813)(1602.57655555,133.52365214)(1603.56613888,132.83354797)
\curveto(1604.71197222,132.03927714)(1605.57134722,131.03667297)(1606.14426388,129.82573547)
\curveto(1606.73020138,128.61479797)(1607.02317013,127.1434438)(1607.02317013,125.41167297)
\closepath
\moveto(1603.05832638,125.25542297)
\curveto(1603.05832638,126.46636047)(1602.87603472,127.46896464)(1602.51145138,128.26323547)
\curveto(1602.15988888,129.0575063)(1601.57395138,129.74761047)(1600.75363888,130.33354797)
\curveto(1600.15468055,130.75021464)(1599.49061805,131.02365214)(1598.76145138,131.15386047)
\curveto(1598.03228472,131.2840688)(1597.27056597,131.34917297)(1596.47629513,131.34917297)
\curveto(1595.3695243,131.34917297)(1594.34087847,131.21896464)(1593.39035763,130.95854797)
\curveto(1592.4398368,130.6981313)(1591.4632743,130.29448547)(1590.46067013,129.74761047)
\curveto(1590.43462847,129.46115214)(1590.41509722,129.18120422)(1590.40207638,128.90776672)
\curveto(1590.38905555,128.64735005)(1590.38254513,128.3153188)(1590.38254513,127.91167297)
\curveto(1590.38254513,125.8543813)(1590.59087847,124.22677714)(1591.00754513,123.02886047)
\curveto(1591.43723263,121.84396464)(1592.02317013,120.90646464)(1592.76535763,120.21636047)
\curveto(1593.36431597,119.6434438)(1594.00884722,119.22026672)(1594.69895138,118.94682922)
\curveto(1595.40207638,118.68641255)(1596.16379513,118.55620422)(1596.98410763,118.55620422)
\curveto(1598.87212847,118.55620422)(1600.35650347,119.12912089)(1601.43723263,120.27495422)
\curveto(1602.5179618,121.43380839)(1603.05832638,123.09396464)(1603.05832638,125.25542297)
\closepath
}
}
{
\newrgbcolor{curcolor}{0 0 0}
\pscustom[linestyle=none,fillstyle=solid,fillcolor=curcolor]
{
\newpath
\moveto(1580.12461056,59.83299865)
\curveto(1580.12461056,57.71841532)(1579.6610689,55.80174865)(1578.73398556,54.08299865)
\curveto(1577.8173189,52.36424865)(1576.59336056,51.03091532)(1575.06211056,50.08299865)
\curveto(1573.99961056,49.42674865)(1572.81211056,48.95279032)(1571.49961056,48.66112365)
\curveto(1570.19752723,48.36945699)(1568.47877723,48.22362365)(1566.34336056,48.22362365)
\lineto(1560.46836056,48.22362365)
\lineto(1560.46836056,71.48924865)
\lineto(1566.28086056,71.48924865)
\curveto(1568.5516939,71.48924865)(1570.35377723,71.32258199)(1571.68711056,70.98924865)
\curveto(1573.03086056,70.66633199)(1574.16627723,70.21841532)(1575.09336056,69.64549865)
\curveto(1576.6766939,68.65591532)(1577.9110689,67.33820699)(1578.79648556,65.69237365)
\curveto(1579.68190223,64.04654032)(1580.12461056,62.09341532)(1580.12461056,59.83299865)
\closepath
\moveto(1576.89023556,59.87987365)
\curveto(1576.89023556,61.70279032)(1576.57252723,63.23924865)(1575.93711056,64.48924865)
\curveto(1575.3016939,65.73924865)(1574.35377723,66.72362365)(1573.09336056,67.44237365)
\curveto(1572.1766939,67.96320699)(1571.20273556,68.32258199)(1570.17148556,68.52049865)
\curveto(1569.14023556,68.72883199)(1567.90586056,68.83299865)(1566.46836056,68.83299865)
\lineto(1563.56211056,68.83299865)
\lineto(1563.56211056,50.87987365)
\lineto(1566.46836056,50.87987365)
\curveto(1567.9579439,50.87987365)(1569.2548189,50.98924865)(1570.35898556,51.20799865)
\curveto(1571.4735689,51.42674865)(1572.49440223,51.83299865)(1573.42148556,52.42674865)
\curveto(1574.57773556,53.16633199)(1575.4423189,54.14029032)(1576.01523556,55.34862365)
\curveto(1576.5985689,56.55695699)(1576.89023556,58.06737365)(1576.89023556,59.87987365)
\closepath
}
}
{
\newrgbcolor{curcolor}{0 0 0}
\pscustom[linestyle=none,fillstyle=solid,fillcolor=curcolor]
{
\newpath
\moveto(1598.45273556,48.22362365)
\lineto(1595.53086056,48.22362365)
\lineto(1595.53086056,50.08299865)
\curveto(1595.2704439,49.90591532)(1594.91627723,49.65591532)(1594.46836056,49.33299865)
\curveto(1594.03086056,49.02049865)(1593.60377723,48.77049865)(1593.18711056,48.58299865)
\curveto(1592.69752723,48.34341532)(1592.13502723,48.14549865)(1591.49961056,47.98924865)
\curveto(1590.8641939,47.82258199)(1590.11940223,47.73924865)(1589.26523556,47.73924865)
\curveto(1587.6923189,47.73924865)(1586.35898556,48.26008199)(1585.26523556,49.30174865)
\curveto(1584.17148556,50.34341532)(1583.62461056,51.67154032)(1583.62461056,53.28612365)
\curveto(1583.62461056,54.60904032)(1583.90586056,55.67674865)(1584.46836056,56.48924865)
\curveto(1585.04127723,57.31216532)(1585.85377723,57.95799865)(1586.90586056,58.42674865)
\curveto(1587.96836056,58.89549865)(1589.24440223,59.21320699)(1590.73398556,59.37987365)
\curveto(1592.2235689,59.54654032)(1593.82252723,59.67154032)(1595.53086056,59.75487365)
\lineto(1595.53086056,60.20799865)
\curveto(1595.53086056,60.87466532)(1595.4110689,61.42674865)(1595.17148556,61.86424865)
\curveto(1594.9423189,62.30174865)(1594.60898556,62.64549865)(1594.17148556,62.89549865)
\curveto(1593.7548189,63.13508199)(1593.2548189,63.29654032)(1592.67148556,63.37987365)
\curveto(1592.08815223,63.46320699)(1591.47877723,63.50487365)(1590.84336056,63.50487365)
\curveto(1590.07252723,63.50487365)(1589.21315223,63.40070699)(1588.26523556,63.19237365)
\curveto(1587.3173189,62.99445699)(1586.33815223,62.70279032)(1585.32773556,62.31737365)
\lineto(1585.17148556,62.31737365)
\lineto(1585.17148556,65.30174865)
\curveto(1585.74440223,65.45799865)(1586.57252723,65.62987365)(1587.65586056,65.81737365)
\curveto(1588.7391939,66.00487365)(1589.80690223,66.09862365)(1590.85898556,66.09862365)
\curveto(1592.08815223,66.09862365)(1593.15586056,65.99445699)(1594.06211056,65.78612365)
\curveto(1594.97877723,65.58820699)(1595.7704439,65.24445699)(1596.43711056,64.75487365)
\curveto(1597.09336056,64.27570699)(1597.59336056,63.65591532)(1597.93711056,62.89549865)
\curveto(1598.28086056,62.13508199)(1598.45273556,61.19237365)(1598.45273556,60.06737365)
\closepath
\moveto(1595.53086056,52.52049865)
\lineto(1595.53086056,57.37987365)
\curveto(1594.63502723,57.32779032)(1593.57773556,57.24966532)(1592.35898556,57.14549865)
\curveto(1591.15065223,57.04133199)(1590.1923189,56.89029032)(1589.48398556,56.69237365)
\curveto(1588.64023556,56.45279032)(1587.9579439,56.07779032)(1587.43711056,55.56737365)
\curveto(1586.91627723,55.06737365)(1586.65586056,54.37466532)(1586.65586056,53.48924865)
\curveto(1586.65586056,52.48924865)(1586.9579439,51.73404032)(1587.56211056,51.22362365)
\curveto(1588.16627723,50.72362365)(1589.08815223,50.47362365)(1590.32773556,50.47362365)
\curveto(1591.35898556,50.47362365)(1592.3016939,50.67154032)(1593.15586056,51.06737365)
\curveto(1594.01002723,51.47362365)(1594.8016939,51.95799865)(1595.53086056,52.52049865)
\closepath
}
}
{
\newrgbcolor{curcolor}{0 0 0}
\pscustom[linestyle=none,fillstyle=solid,fillcolor=curcolor]
{
\newpath
\moveto(1613.17148556,48.37987365)
\curveto(1612.61940223,48.23404032)(1612.01523556,48.11424865)(1611.35898556,48.02049865)
\curveto(1610.71315223,47.92674865)(1610.13502723,47.87987365)(1609.62461056,47.87987365)
\curveto(1607.84336056,47.87987365)(1606.4891939,48.35904032)(1605.56211056,49.31737365)
\curveto(1604.63502723,50.27570699)(1604.17148556,51.81216532)(1604.17148556,53.92674865)
\lineto(1604.17148556,63.20799865)
\lineto(1602.18711056,63.20799865)
\lineto(1602.18711056,65.67674865)
\lineto(1604.17148556,65.67674865)
\lineto(1604.17148556,70.69237365)
\lineto(1607.10898556,70.69237365)
\lineto(1607.10898556,65.67674865)
\lineto(1613.17148556,65.67674865)
\lineto(1613.17148556,63.20799865)
\lineto(1607.10898556,63.20799865)
\lineto(1607.10898556,55.25487365)
\curveto(1607.10898556,54.33820699)(1607.1298189,53.61945699)(1607.17148556,53.09862365)
\curveto(1607.21315223,52.58820699)(1607.35898556,52.10904032)(1607.60898556,51.66112365)
\curveto(1607.83815223,51.24445699)(1608.15065223,50.93716532)(1608.54648556,50.73924865)
\curveto(1608.95273556,50.55174865)(1609.5673189,50.45799865)(1610.39023556,50.45799865)
\curveto(1610.86940223,50.45799865)(1611.36940223,50.52570699)(1611.89023556,50.66112365)
\curveto(1612.4110689,50.80695699)(1612.7860689,50.92674865)(1613.01523556,51.02049865)
\lineto(1613.17148556,51.02049865)
\closepath
}
}
{
\newrgbcolor{curcolor}{0 0 0}
\pscustom[linestyle=none,fillstyle=solid,fillcolor=curcolor]
{
\newpath
\moveto(1630.28086056,48.22362365)
\lineto(1627.35898556,48.22362365)
\lineto(1627.35898556,50.08299865)
\curveto(1627.0985689,49.90591532)(1626.74440223,49.65591532)(1626.29648556,49.33299865)
\curveto(1625.85898556,49.02049865)(1625.43190223,48.77049865)(1625.01523556,48.58299865)
\curveto(1624.52565223,48.34341532)(1623.96315223,48.14549865)(1623.32773556,47.98924865)
\curveto(1622.6923189,47.82258199)(1621.94752723,47.73924865)(1621.09336056,47.73924865)
\curveto(1619.5204439,47.73924865)(1618.18711056,48.26008199)(1617.09336056,49.30174865)
\curveto(1615.99961056,50.34341532)(1615.45273556,51.67154032)(1615.45273556,53.28612365)
\curveto(1615.45273556,54.60904032)(1615.73398556,55.67674865)(1616.29648556,56.48924865)
\curveto(1616.86940223,57.31216532)(1617.68190223,57.95799865)(1618.73398556,58.42674865)
\curveto(1619.79648556,58.89549865)(1621.07252723,59.21320699)(1622.56211056,59.37987365)
\curveto(1624.0516939,59.54654032)(1625.65065223,59.67154032)(1627.35898556,59.75487365)
\lineto(1627.35898556,60.20799865)
\curveto(1627.35898556,60.87466532)(1627.2391939,61.42674865)(1626.99961056,61.86424865)
\curveto(1626.7704439,62.30174865)(1626.43711056,62.64549865)(1625.99961056,62.89549865)
\curveto(1625.5829439,63.13508199)(1625.0829439,63.29654032)(1624.49961056,63.37987365)
\curveto(1623.91627723,63.46320699)(1623.30690223,63.50487365)(1622.67148556,63.50487365)
\curveto(1621.90065223,63.50487365)(1621.04127723,63.40070699)(1620.09336056,63.19237365)
\curveto(1619.1454439,62.99445699)(1618.16627723,62.70279032)(1617.15586056,62.31737365)
\lineto(1616.99961056,62.31737365)
\lineto(1616.99961056,65.30174865)
\curveto(1617.57252723,65.45799865)(1618.40065223,65.62987365)(1619.48398556,65.81737365)
\curveto(1620.5673189,66.00487365)(1621.63502723,66.09862365)(1622.68711056,66.09862365)
\curveto(1623.91627723,66.09862365)(1624.98398556,65.99445699)(1625.89023556,65.78612365)
\curveto(1626.80690223,65.58820699)(1627.5985689,65.24445699)(1628.26523556,64.75487365)
\curveto(1628.92148556,64.27570699)(1629.42148556,63.65591532)(1629.76523556,62.89549865)
\curveto(1630.10898556,62.13508199)(1630.28086056,61.19237365)(1630.28086056,60.06737365)
\closepath
\moveto(1627.35898556,52.52049865)
\lineto(1627.35898556,57.37987365)
\curveto(1626.46315223,57.32779032)(1625.40586056,57.24966532)(1624.18711056,57.14549865)
\curveto(1622.97877723,57.04133199)(1622.0204439,56.89029032)(1621.31211056,56.69237365)
\curveto(1620.46836056,56.45279032)(1619.7860689,56.07779032)(1619.26523556,55.56737365)
\curveto(1618.74440223,55.06737365)(1618.48398556,54.37466532)(1618.48398556,53.48924865)
\curveto(1618.48398556,52.48924865)(1618.7860689,51.73404032)(1619.39023556,51.22362365)
\curveto(1619.99440223,50.72362365)(1620.91627723,50.47362365)(1622.15586056,50.47362365)
\curveto(1623.18711056,50.47362365)(1624.1298189,50.67154032)(1624.98398556,51.06737365)
\curveto(1625.83815223,51.47362365)(1626.6298189,51.95799865)(1627.35898556,52.52049865)
\closepath
}
}
{
\newrgbcolor{curcolor}{0 0 0}
\pscustom[linewidth=4,linecolor=curcolor]
{
\newpath
\moveto(669.040694,223.51590379)
\lineto(536.020794,349.74188379)
}
}
{
\newrgbcolor{curcolor}{0 0 0}
\pscustom[linestyle=none,fillstyle=solid,fillcolor=curcolor]
{
\newpath
\moveto(640.0251537,251.04949094)
\lineto(617.40550272,250.45670968)
\lineto(669.040694,223.51590379)
\lineto(639.43237244,273.66914192)
\closepath
}
}
{
\newrgbcolor{curcolor}{0 0 0}
\pscustom[linewidth=4.26666679,linecolor=curcolor]
{
\newpath
\moveto(640.0251537,251.04949094)
\lineto(617.40550272,250.45670968)
\lineto(669.040694,223.51590379)
\lineto(639.43237244,273.66914192)
\closepath
}
}
{
\newrgbcolor{curcolor}{0 0 0}
\pscustom[linewidth=4,linecolor=curcolor]
{
\newpath
\moveto(1597.673184,223.51590379)
\lineto(1464.653284,349.74188379)
}
}
{
\newrgbcolor{curcolor}{0 0 0}
\pscustom[linestyle=none,fillstyle=solid,fillcolor=curcolor]
{
\newpath
\moveto(1568.6576437,251.04949094)
\lineto(1546.03799272,250.45670968)
\lineto(1597.673184,223.51590379)
\lineto(1568.06486244,273.66914192)
\closepath
}
}
{
\newrgbcolor{curcolor}{0 0 0}
\pscustom[linewidth=4.26666679,linecolor=curcolor]
{
\newpath
\moveto(1568.6576437,251.04949094)
\lineto(1546.03799272,250.45670968)
\lineto(1597.673184,223.51590379)
\lineto(1568.06486244,273.66914192)
\closepath
}
}
{
\newrgbcolor{curcolor}{0 0 0}
\pscustom[linewidth=4,linecolor=curcolor]
{
\newpath
\moveto(528.882494,569.25774779)
\lineto(752.043494,690.88653079)
}
}
{
\newrgbcolor{curcolor}{0 0 0}
\pscustom[linestyle=none,fillstyle=solid,fillcolor=curcolor]
{
\newpath
\moveto(564.00462676,588.40026043)
\lineto(570.39647481,610.10611859)
\lineto(528.882494,569.25774779)
\lineto(585.71048492,582.00841239)
\closepath
}
}
{
\newrgbcolor{curcolor}{0 0 0}
\pscustom[linewidth=4.26666679,linecolor=curcolor]
{
\newpath
\moveto(564.00462676,588.40026043)
\lineto(570.39647481,610.10611859)
\lineto(528.882494,569.25774779)
\lineto(585.71048492,582.00841239)
\closepath
}
}
{
\newrgbcolor{curcolor}{0 0 0}
\pscustom[linewidth=4,linecolor=curcolor]
{
\newpath
\moveto(1277.515184,569.25774779)
\lineto(1054.354184,690.88653079)
}
}
{
\newrgbcolor{curcolor}{0 0 0}
\pscustom[linestyle=none,fillstyle=solid,fillcolor=curcolor]
{
\newpath
\moveto(1242.39305124,588.40026043)
\lineto(1220.68719308,582.00841239)
\lineto(1277.515184,569.25774779)
\lineto(1236.00120319,610.10611859)
\closepath
}
}
{
\newrgbcolor{curcolor}{0 0 0}
\pscustom[linewidth=4.26666679,linecolor=curcolor]
{
\newpath
\moveto(1242.39305124,588.40026043)
\lineto(1220.68719308,582.00841239)
\lineto(1277.515184,569.25774779)
\lineto(1236.00120319,610.10611859)
\closepath
}
}
\end{pspicture}
}
    \captionsetup{width=0.75\linewidth}
    \caption{
        Block Pointers create a Merkel Tree allowing each layer to validate the blocks below it
        \cite{azaghal_diagram_2012}. 
        As long as Block 0 is known to be valid, the checksums within it ensure that blocks 1 and 2 can be proven to be valid, 
        and also every other block, as 1 and 2 contain checksums of 3, 4, 5, and 6, which themselves contain the user's data.
    }
\label{fig:HashTree}
\end{figure}

\section{ZIL}
The ZFS Intent Log is the ZFS's journal  \cite{ahrens_read_write}.
It functions like a journal in other filesystems, ensuring synchronous writes without flushing everything to disk and 
requiring a bunch of rewrites every time fsync is called. 
These fsync operations are intended to force all data to be written to disk,
but instead on ZFS only the ZIL is synced to disk.
This allows these operations to be fast without actually having to write all the new direct and indirect blocks to the disks,
but ensures that all data can be recovered by ZFS in the event of sudden power loss or crash before
the next checkpoint is written to disk.

The ZIL is a linked list of changes made to the filesystem.
It works around the problem of pointers in a copy-on-write filesystem, which would typically require
changing the previous block to have a pointer to the next block,
by adding a new empty block to the end of list any time it adds more data.
New data is then written to the first completely invalid block that it finds, 
under the assumption that it is the final empty block.
This allows it to not have to change the pointer address of previous blocks and keep ZFS's copy-on-write guarantees, 
never rewriting the data of already written blocks in place.

When ZFS restores from power loss, it gets the last good state of the current uberblock, the root block of the storage pool,
and then updates write by write to the last write that was synced to the ZIL \cite{mckusick_zfs_2015_presentation}.

\section{Checkpoints and Snapshots}
After a certain amount of writes to the pool accumulate, ZFS writes them out to the disks of the pool
all at once as a checkpoint, or transaction group
\cite{ahrens_read_write,mckusick_zfs_2015_presentation}.
It must first collect all updates that have happened since the last checkpoint from the ZIL
and then write them out to unused locations in the pool before finally writing a new uberblock for the entire pool.
Uberblocks are used in a rotation as checkpoints happen, with a set of either 128 or 256, depending on the disk's physical block size.

Because ZFS works on a system of checkpoints and copy-on-write storage, snapshots are trivial for the filesystem to keep around.
It must simply save the root block of the filesystem at a given point in time.
Deleting them, however, is much trickier, because it is difficult to find the blocks that only exist in that snapshot
when it is deleted.
This is critical to do, as a block can only be freed if everything referencing it no longer needs it.
Birth time is used to figure out if any snapshots need a particular block when ZFS is deleting a snapshot.
The original copy-on-write filesystems used garbage collection, but this was very slow.
ZFS instead uses a deadlist of what blocks the snapshot no longer cares about relative to the snapshot right before it.

\section{On-Disk Structure}
ZFS's on-disk structure starts with an uberblock, the root block for the entire pool 
\cite{ahrens_read_write,mckusick_zfs_2015_presentation}.
The uberblock points to the Meta-Object-Set, or MOS, which is a set that keeps track of everything in the pool such as
filesystems or snapshots, as well as clones, copies of a filesystem that are themselves filesystems, 
and ZVOLs, large files stored in ZFS outside of a filesystem that are intended to be used as disks for virtual machines.
ZFS treats all of these on-disk objects as sets of sets, and they are implemented through direct and indirect blocks,
which allows these sets to be arbitrarily large (Section \ref{chapter:blocks}).
The first object in the Meta-Object-Set is a master object set that contains information about the pool itself.
The last is the space map, which keeps track of which blocks onare free and which are being used over the entirety of
every disk in  the ZFS storage pool.
Each object in the MOS is itself an object set that describes the objects within it.
For example, filesystem objects are a set of files, directories, and symlinks, files that simply reference another file and are
treated by most programs as exactly the same as the files they link to.
Objects are also implemented in the same way as object sets, pointing to indirect blocks which eventually point to direct blocks which 
contain the data that makes up the object.

\begin{figure}[H]
    \centering
    \resizebox{!}{0.2\textheight}{%LaTeX with PSTricks extensions
%%Creator: Inkscape 1.0.2-2 (e86c870879, 2021-01-15)
%%Please note this file requires PSTricks extensions
\psset{xunit=.5pt,yunit=.5pt,runit=.5pt}
\begin{pspicture}(647.59562174,324.16853841)
{
\newrgbcolor{curcolor}{0 0 0}
\pscustom[linestyle=none,fillstyle=solid,fillcolor=curcolor]
{
\newpath
\moveto(11.87333375,133.05540986)
\curveto(13.46708448,133.05540986)(14.85770853,132.44082717)(16.04000057,131.22207646)
\curveto(17.21708461,129.99811775)(17.81083396,128.47728436)(17.81083396,126.6595763)
\curveto(17.81083396,124.78457623)(17.21187528,123.22207617)(16.01916723,121.97207613)
\curveto(14.83687519,120.73249342)(13.40458447,120.1179094)(11.72750042,120.1179094)
\curveto(10.01916702,120.1179094)(8.58687497,120.72207609)(7.4358336,121.93040946)
\curveto(6.27958422,123.14916017)(5.70666687,124.72207623)(5.70666687,126.63874296)
\curveto(5.70666687,128.59707637)(6.37333356,130.19082709)(7.70666694,131.4304098)
\curveto(8.85770831,132.51374317)(10.2483337,133.05540986)(11.87333375,133.05540986)
\closepath
\moveto(11.68583375,132.38874317)
\curveto(10.58687504,132.38874317)(9.70666701,131.97728448)(9.04000032,131.15957646)
\curveto(8.20666696,130.12832709)(7.79000028,128.63874303)(7.79000028,126.68040963)
\curveto(7.79000028,124.64916023)(8.21708429,123.09707617)(9.08166699,122.0137428)
\curveto(9.74833368,121.19082677)(10.62333371,120.78457609)(11.70666708,120.78457609)
\curveto(12.85770846,120.78457609)(13.81083382,121.23249344)(14.56083385,122.1387428)
\curveto(15.32125121,123.03978417)(15.70666722,124.46166022)(15.70666722,126.40957629)
\curveto(15.70666722,128.53457636)(15.29000054,130.10749375)(14.45666718,131.13874312)
\curveto(13.79000049,131.97207648)(12.86291779,132.38874317)(11.68583375,132.38874317)
\closepath
\moveto(11.68583375,132.38874317)
}
}
{
\newrgbcolor{curcolor}{0 0 0}
\pscustom[linestyle=none,fillstyle=solid,fillcolor=curcolor]
{
\newpath
\moveto(21.39233637,127.32624299)
\curveto(22.1944204,128.45124303)(23.0642111,129.01374305)(23.99650313,129.01374305)
\curveto(24.85587783,129.01374305)(25.60587785,128.64395103)(26.24650321,127.90957634)
\curveto(26.88192056,127.16999365)(27.20483658,126.16999361)(27.20483658,124.90957623)
\curveto(27.20483658,123.41999351)(26.71004456,122.22728414)(25.72566986,121.32624277)
\curveto(24.87671116,120.54499341)(23.93400313,120.15957607)(22.89233642,120.15957607)
\curveto(22.40275374,120.15957607)(21.91316972,120.2429094)(21.4131697,120.40957607)
\curveto(20.91316969,120.58666008)(20.40275367,120.85749342)(19.89233632,121.2220761)
\lineto(19.89233632,129.86790974)
\curveto(19.89233632,130.81061778)(19.86108698,131.38874313)(19.80900298,131.59707647)
\curveto(19.76733631,131.81582715)(19.69442031,131.97207648)(19.60066964,132.05540982)
\curveto(19.50171097,132.13874316)(19.3819203,132.18040983)(19.24650296,132.18040983)
\curveto(19.06421095,132.18040983)(18.85066961,132.12832716)(18.6006696,132.03457649)
\lineto(18.4756696,132.3470765)
\lineto(20.97566969,133.36790987)
\lineto(21.39233637,133.36790987)
\closepath
\moveto(21.39233637,126.74290963)
\lineto(21.39233637,121.74290946)
\curveto(21.69442038,121.43561744)(22.01733639,121.20645077)(22.35066974,121.05540943)
\curveto(22.68400308,120.89916009)(23.02254443,120.82624276)(23.37150311,120.82624276)
\curveto(23.92358713,120.82624276)(24.43921114,121.12311743)(24.91316983,121.72207612)
\curveto(25.38192051,122.33145081)(25.62150319,123.22207617)(25.62150319,124.38874288)
\curveto(25.62150319,125.45645092)(25.38192051,126.27416028)(24.91316983,126.8470763)
\curveto(24.43921114,127.41478432)(23.89754446,127.701243)(23.28816977,127.701243)
\curveto(22.96525376,127.701243)(22.64233641,127.62311766)(22.30900307,127.47207633)
\curveto(22.06942039,127.34707632)(21.76733638,127.10228431)(21.39233637,126.74290963)
\closepath
\moveto(21.39233637,126.74290963)
}
}
{
\newrgbcolor{curcolor}{0 0 0}
\pscustom[linestyle=none,fillstyle=solid,fillcolor=curcolor]
{
\newpath
\moveto(30.55900336,133.3887432)
\curveto(30.80900337,133.3887432)(31.02254471,133.28978453)(31.20483672,133.09707652)
\curveto(31.39754472,132.91478452)(31.49650339,132.70124318)(31.49650339,132.45124317)
\curveto(31.49650339,132.18561783)(31.39754472,131.96166048)(31.20483672,131.78457648)
\curveto(31.02254471,131.60228447)(30.80900337,131.51374314)(30.55900336,131.51374314)
\curveto(30.29337802,131.51374314)(30.06942068,131.60228447)(29.89233667,131.78457648)
\curveto(29.71004466,131.96166048)(29.62150333,132.18561783)(29.62150333,132.45124317)
\curveto(29.62150333,132.70124318)(29.71004466,132.91478452)(29.89233667,133.09707652)
\curveto(30.06942068,133.28978453)(30.29337802,133.3887432)(30.55900336,133.3887432)
\closepath
\moveto(31.32983672,129.01374305)
\lineto(31.32983672,120.59707608)
\curveto(31.32983672,119.16999203)(31.02254471,118.11270133)(30.41317002,117.4304093)
\curveto(29.81421133,116.73770128)(29.02775397,116.3887426)(28.05900327,116.3887426)
\curveto(27.50691925,116.3887426)(27.09546191,116.48770127)(26.82983656,116.68040928)
\curveto(26.55379522,116.87832528)(26.41316988,117.08665862)(26.41316988,117.3054093)
\curveto(26.41316988,117.51374264)(26.49129522,117.69603465)(26.64233656,117.84707598)
\curveto(26.79858589,118.00332532)(26.9704619,118.07624266)(27.16316991,118.07624266)
\curveto(27.32983658,118.07624266)(27.49129525,118.03457599)(27.64233659,117.95124265)
\curveto(27.75691993,117.90957599)(27.9548366,117.75853465)(28.24650328,117.4929093)
\curveto(28.53816996,117.23249196)(28.78816997,117.09707596)(28.99650331,117.09707596)
\curveto(29.13192064,117.09707596)(29.27254465,117.15436796)(29.41316999,117.26374263)
\curveto(29.54858733,117.3783253)(29.652754,117.56582531)(29.72567,117.82624265)
\curveto(29.793378,118.09186799)(29.82983667,118.66999201)(29.82983667,119.55540938)
\lineto(29.82983667,125.49290959)
\curveto(29.82983667,126.40957629)(29.79858733,126.99811764)(29.74650333,127.26374298)
\curveto(29.70483666,127.45645099)(29.63192066,127.58666033)(29.53816999,127.65957633)
\curveto(29.45483666,127.74290967)(29.32983665,127.78457634)(29.16316998,127.78457634)
\curveto(28.99650331,127.78457634)(28.79337797,127.74290967)(28.55900329,127.65957633)
\lineto(28.43400329,127.99290968)
\lineto(30.93400337,129.01374305)
\closepath
\moveto(31.32983672,129.01374305)
}
}
{
\newrgbcolor{curcolor}{0 0 0}
\pscustom[linestyle=none,fillstyle=solid,fillcolor=curcolor]
{
\newpath
\moveto(35.01916791,125.61790959)
\curveto(35.01916791,124.37832688)(35.32125192,123.40957618)(35.93583461,122.70124282)
\curveto(36.52958529,121.99290946)(37.24833465,121.63874279)(38.08166802,121.63874279)
\curveto(38.6233347,121.63874279)(39.09208539,121.78457612)(39.49833473,122.0762428)
\curveto(39.89937608,122.37832681)(40.23791876,122.89916016)(40.5191681,123.63874286)
\lineto(40.79000145,123.45124285)
\curveto(40.66500144,122.61790949)(40.29520943,121.85749346)(39.68583474,121.18040944)
\curveto(39.07125205,120.49811781)(38.30041869,120.15957607)(37.37333466,120.15957607)
\curveto(36.37333462,120.15957607)(35.50875192,120.54499341)(34.79000123,121.32624277)
\curveto(34.08166787,122.10228413)(33.72750119,123.14916017)(33.72750119,124.47207622)
\curveto(33.72750119,125.89916027)(34.09208521,127.01374298)(34.8316679,127.80540967)
\curveto(35.56604259,128.6074937)(36.48270929,129.01374305)(37.581668,129.01374305)
\curveto(38.52437603,129.01374305)(39.29520939,128.70124304)(39.89416808,128.07624301)
\curveto(40.48791877,127.46166032)(40.79000145,126.64395096)(40.79000145,125.61790959)
\closepath
\moveto(35.01916791,126.13874294)
\lineto(38.89416804,126.13874294)
\curveto(38.86291871,126.68040963)(38.80041871,127.05540964)(38.70666804,127.26374298)
\curveto(38.5504187,127.60749366)(38.32125202,127.87832701)(38.01916801,128.07624301)
\curveto(37.711876,128.26895102)(37.39937599,128.36790969)(37.08166798,128.36790969)
\curveto(36.58166796,128.36790969)(36.12854261,128.16999368)(35.72750127,127.78457634)
\curveto(35.32125192,127.39395099)(35.08687591,126.8470763)(35.01916791,126.13874294)
\closepath
\moveto(35.01916791,126.13874294)
}
}
{
\newrgbcolor{curcolor}{0 0 0}
\pscustom[linestyle=none,fillstyle=solid,fillcolor=curcolor]
{
\newpath
\moveto(48.99467296,123.57624285)
\curveto(48.77071429,122.47728415)(48.33321427,121.62832678)(47.68217291,121.0345761)
\curveto(47.02592356,120.45124274)(46.29675686,120.15957607)(45.49467284,120.15957607)
\curveto(44.5467568,120.15957607)(43.72383944,120.55540941)(43.01550608,121.34707611)
\curveto(42.31759006,122.1387428)(41.97383938,123.21166017)(41.97383938,124.57624289)
\curveto(41.97383938,125.87832694)(42.35925672,126.94082697)(43.14050609,127.763743)
\curveto(43.91654745,128.59707637)(44.85404748,129.01374305)(45.95300619,129.01374305)
\curveto(46.77071421,129.01374305)(47.44259024,128.78978437)(47.97383959,128.34707636)
\curveto(48.49988094,127.91478434)(48.76550629,127.46166032)(48.76550629,126.99290964)
\curveto(48.76550629,126.76895097)(48.68738095,126.58145096)(48.53633961,126.43040962)
\curveto(48.39571427,126.28978428)(48.18738093,126.22207628)(47.91133959,126.22207628)
\curveto(47.56238091,126.22207628)(47.29154757,126.33666028)(47.09883956,126.57624296)
\curveto(46.99988089,126.70124296)(46.93217289,126.94082697)(46.89050622,127.30540965)
\curveto(46.85925688,127.66478433)(46.74467288,127.93561767)(46.53633954,128.11790968)
\curveto(46.3280062,128.29499369)(46.02592352,128.38874302)(45.64050617,128.38874302)
\curveto(45.04154749,128.38874302)(44.5571728,128.16478435)(44.18217279,127.72207633)
\curveto(43.69259011,127.12311765)(43.4530061,126.33666028)(43.4530061,125.36790958)
\curveto(43.4530061,124.36790955)(43.69259011,123.48249352)(44.18217279,122.72207616)
\curveto(44.66654747,121.9564508)(45.32800616,121.57624278)(46.16133953,121.57624278)
\curveto(46.75509021,121.57624278)(47.29154757,121.77416012)(47.76550625,122.18040947)
\curveto(48.09883959,122.45645081)(48.42175694,122.96166017)(48.74467295,123.70124286)
\closepath
\moveto(48.99467296,123.57624285)
}
}
{
\newrgbcolor{curcolor}{0 0 0}
\pscustom[linestyle=none,fillstyle=solid,fillcolor=curcolor]
{
\newpath
\moveto(52.61600081,131.4929098)
\lineto(52.61600081,128.76374304)
\lineto(54.57433421,128.76374304)
\lineto(54.57433421,128.11790968)
\lineto(52.61600081,128.11790968)
\lineto(52.61600081,122.70124282)
\curveto(52.61600081,122.15957614)(52.68891814,121.78978412)(52.84516748,121.59707612)
\curveto(53.01183415,121.41478411)(53.20975149,121.32624277)(53.44933417,121.32624277)
\curveto(53.65766751,121.32624277)(53.85037552,121.38874278)(54.03266752,121.51374278)
\curveto(54.22537553,121.63874279)(54.37120887,121.82624279)(54.47016754,122.0762428)
\lineto(54.82433422,122.0762428)
\curveto(54.61600088,121.47728411)(54.3139182,121.0241601)(53.92850085,120.72207609)
\curveto(53.53787551,120.41478441)(53.13683416,120.26374274)(52.72016748,120.26374274)
\curveto(52.43891813,120.26374274)(52.16808479,120.34186767)(51.90766745,120.49290941)
\curveto(51.64204211,120.65957608)(51.4389181,120.87832676)(51.30350076,121.1595761)
\curveto(51.17850076,121.45124278)(51.11600075,121.89916013)(51.11600075,122.51374282)
\lineto(51.11600075,128.11790968)
\lineto(49.80350071,128.11790968)
\lineto(49.80350071,128.43040969)
\curveto(50.13683405,128.5554097)(50.4753754,128.77416037)(50.82433408,129.09707638)
\curveto(51.16808475,129.41478439)(51.48058477,129.79499374)(51.76183411,130.24290976)
\curveto(51.89725145,130.46166043)(52.09516745,130.87832711)(52.34516746,131.4929098)
\closepath
\moveto(52.61600081,131.4929098)
}
}
{
\newrgbcolor{curcolor}{0 0 0}
\pscustom[linestyle=none,fillstyle=solid,fillcolor=curcolor]
{
\newpath
\moveto(68.03449907,133.05540986)
\lineto(68.03449907,128.78457637)
\lineto(67.70116573,128.78457637)
\curveto(67.58658306,129.6022844)(67.38866572,130.25332709)(67.09699904,130.74290977)
\curveto(66.8157497,131.22728446)(66.41470702,131.60749381)(65.88866566,131.88874315)
\curveto(65.37304031,132.18040983)(64.83658296,132.32624316)(64.28449894,132.32624316)
\curveto(63.64387359,132.32624316)(63.11783223,132.12832716)(62.70116555,131.74290981)
\curveto(62.2949162,131.35228446)(62.09699886,130.91478445)(62.09699886,130.43040976)
\curveto(62.09699886,130.05540975)(62.22199887,129.70645107)(62.47199888,129.38874306)
\curveto(62.85741622,128.94082704)(63.75324959,128.33666036)(65.15949897,127.576243)
\curveto(66.32616568,126.95124297)(67.11783237,126.47207629)(67.53449906,126.13874294)
\curveto(67.95116574,125.81582693)(68.26887375,125.43040959)(68.49283242,124.97207624)
\curveto(68.7272071,124.52416022)(68.8469991,124.05540954)(68.8469991,123.55540952)
\curveto(68.8469991,122.62311749)(68.48241642,121.81582679)(67.76366573,121.13874277)
\curveto(67.03970704,120.45645114)(66.10220701,120.1179094)(64.95116563,120.1179094)
\curveto(64.60220695,120.1179094)(64.26887361,120.14915873)(63.9511656,120.20124273)
\curveto(63.75324959,120.2429094)(63.35741624,120.35749274)(62.76366555,120.53457608)
\curveto(62.1803322,120.72728409)(61.81054019,120.82624276)(61.65949885,120.82624276)
\curveto(61.50324951,120.82624276)(61.3782495,120.77416009)(61.28449883,120.68040942)
\curveto(61.2011655,120.59707608)(61.1386655,120.40957607)(61.09699883,120.1179094)
\lineto(60.76366548,120.1179094)
\lineto(60.76366548,124.36790955)
\lineto(61.09699883,124.36790955)
\curveto(61.24804017,123.47728418)(61.45637351,122.81061749)(61.72199885,122.36790948)
\curveto(61.99804019,121.93561746)(62.40949887,121.57624278)(62.95116556,121.28457611)
\curveto(63.50324958,120.99290943)(64.10741627,120.84707609)(64.76366562,120.84707609)
\curveto(65.52408298,120.84707609)(66.12304034,121.0397841)(66.55533235,121.43040944)
\curveto(66.99804037,121.83145079)(67.22199905,122.31061748)(67.22199905,122.8679095)
\curveto(67.22199905,123.16999351)(67.13866571,123.47728418)(66.97199904,123.78457619)
\curveto(66.80533236,124.08666021)(66.54491635,124.37311755)(66.20116568,124.63874289)
\curveto(65.961583,124.8314509)(65.31574964,125.22207625)(64.26366561,125.8054096)
\curveto(63.20637357,126.39916029)(62.45116554,126.87311764)(61.99283219,127.22207632)
\curveto(61.54491618,127.581451)(61.2063735,127.97207634)(60.97199882,128.38874302)
\curveto(60.74804015,128.81582704)(60.63866548,129.28978439)(60.63866548,129.80540974)
\curveto(60.63866548,130.69082711)(60.97720682,131.45645113)(61.65949885,132.09707649)
\curveto(62.33658287,132.73249384)(63.20637357,133.05540986)(64.26366561,133.05540986)
\curveto(64.91470696,133.05540986)(65.60741632,132.89395118)(66.34699901,132.57624317)
\curveto(66.68033236,132.41999383)(66.91470703,132.3470765)(67.05533237,132.3470765)
\curveto(67.22199905,132.3470765)(67.35220705,132.38874317)(67.45116572,132.4720765)
\curveto(67.54491639,132.56582717)(67.62824973,132.76374318)(67.70116573,133.05540986)
\closepath
\moveto(68.03449907,133.05540986)
}
}
{
\newrgbcolor{curcolor}{0 0 0}
\pscustom[linestyle=none,fillstyle=solid,fillcolor=curcolor]
{
\newpath
\moveto(71.82983816,125.61790959)
\curveto(71.82983816,124.37832688)(72.13192217,123.40957618)(72.74650486,122.70124282)
\curveto(73.34025554,121.99290946)(74.0590049,121.63874279)(74.89233827,121.63874279)
\curveto(75.43400495,121.63874279)(75.90275563,121.78457612)(76.30900498,122.0762428)
\curveto(76.71004633,122.37832681)(77.04858901,122.89916016)(77.32983835,123.63874286)
\lineto(77.60067169,123.45124285)
\curveto(77.47567169,122.61790949)(77.10587968,121.85749346)(76.49650499,121.18040944)
\curveto(75.8819223,120.49811781)(75.11108894,120.15957607)(74.18400491,120.15957607)
\curveto(73.18400487,120.15957607)(72.31942217,120.54499341)(71.60067148,121.32624277)
\curveto(70.89233812,122.10228413)(70.53817144,123.14916017)(70.53817144,124.47207622)
\curveto(70.53817144,125.89916027)(70.90275546,127.01374298)(71.64233815,127.80540967)
\curveto(72.37671284,128.6074937)(73.29337954,129.01374305)(74.39233825,129.01374305)
\curveto(75.33504628,129.01374305)(76.10587964,128.70124304)(76.70483833,128.07624301)
\curveto(77.29858902,127.46166032)(77.60067169,126.64395096)(77.60067169,125.61790959)
\closepath
\moveto(71.82983816,126.13874294)
\lineto(75.70483829,126.13874294)
\curveto(75.67358896,126.68040963)(75.61108896,127.05540964)(75.51733829,127.26374298)
\curveto(75.36108895,127.60749366)(75.13192227,127.87832701)(74.82983826,128.07624301)
\curveto(74.52254625,128.26895102)(74.21004624,128.36790969)(73.89233823,128.36790969)
\curveto(73.39233821,128.36790969)(72.93921286,128.16999368)(72.53817152,127.78457634)
\curveto(72.13192217,127.39395099)(71.89754616,126.8470763)(71.82983816,126.13874294)
\closepath
\moveto(71.82983816,126.13874294)
}
}
{
\newrgbcolor{curcolor}{0 0 0}
\pscustom[linestyle=none,fillstyle=solid,fillcolor=curcolor]
{
\newpath
\moveto(81.13867638,131.4929098)
\lineto(81.13867638,128.76374304)
\lineto(83.09700978,128.76374304)
\lineto(83.09700978,128.11790968)
\lineto(81.13867638,128.11790968)
\lineto(81.13867638,122.70124282)
\curveto(81.13867638,122.15957614)(81.21159371,121.78978412)(81.36784305,121.59707612)
\curveto(81.53450972,121.41478411)(81.73242707,121.32624277)(81.97200974,121.32624277)
\curveto(82.18034308,121.32624277)(82.37305109,121.38874278)(82.55534309,121.51374278)
\curveto(82.7480511,121.63874279)(82.89388444,121.82624279)(82.99284311,122.0762428)
\lineto(83.34700979,122.0762428)
\curveto(83.13867645,121.47728411)(82.83659377,121.0241601)(82.45117642,120.72207609)
\curveto(82.06055108,120.41478441)(81.65950973,120.26374274)(81.24284305,120.26374274)
\curveto(80.9615937,120.26374274)(80.69076036,120.34186767)(80.43034302,120.49290941)
\curveto(80.16471768,120.65957608)(79.96159367,120.87832676)(79.82617633,121.1595761)
\curveto(79.70117633,121.45124278)(79.63867632,121.89916013)(79.63867632,122.51374282)
\lineto(79.63867632,128.11790968)
\lineto(78.32617628,128.11790968)
\lineto(78.32617628,128.43040969)
\curveto(78.65950962,128.5554097)(78.99805097,128.77416037)(79.34700965,129.09707638)
\curveto(79.69076033,129.41478439)(80.00326034,129.79499374)(80.28450968,130.24290976)
\curveto(80.41992702,130.46166043)(80.61784303,130.87832711)(80.86784303,131.4929098)
\closepath
\moveto(81.13867638,131.4929098)
}
}
{
\newrgbcolor{curcolor}{0 0 0}
\pscustom[linestyle=none,fillstyle=solid,fillcolor=curcolor]
{
\newpath
\moveto(91.4530078,133.36790987)
\lineto(91.4530078,122.30540948)
\curveto(91.4530078,121.77416012)(91.48425847,121.41999344)(91.55717447,121.24290944)
\curveto(91.6405078,121.07624277)(91.75509181,120.95124276)(91.91134115,120.86790942)
\curveto(92.07800782,120.78457609)(92.3748825,120.74290942)(92.80717451,120.74290942)
\lineto(92.80717451,120.40957607)
\lineto(88.7030077,120.40957607)
\lineto(88.7030077,120.74290942)
\curveto(89.08842505,120.74290942)(89.35404906,120.77416009)(89.49467439,120.84707609)
\curveto(89.63009173,120.93040943)(89.7342584,121.06061743)(89.80717441,121.24290944)
\curveto(89.89050774,121.41999344)(89.93217441,121.77416012)(89.93217441,122.30540948)
\lineto(89.93217441,129.88874308)
\curveto(89.93217441,130.81582711)(89.91134108,131.38874313)(89.86967441,131.59707647)
\curveto(89.82800774,131.81582715)(89.75509174,131.97207648)(89.66134107,132.05540982)
\curveto(89.57800773,132.13874316)(89.45821573,132.18040983)(89.30717439,132.18040983)
\curveto(89.15092505,132.18040983)(88.95300771,132.12832716)(88.7030077,132.03457649)
\lineto(88.55717436,132.3470765)
\lineto(91.03634112,133.36790987)
\closepath
\moveto(91.4530078,133.36790987)
}
}
{
\newrgbcolor{curcolor}{0 0 0}
\pscustom[linestyle=none,fillstyle=solid,fillcolor=curcolor]
{
\newpath
\moveto(98.49650577,121.61790945)
\curveto(97.63192307,120.95124276)(97.10067239,120.56582675)(96.89233904,120.47207608)
\curveto(96.5590057,120.32103434)(96.20483902,120.2429094)(95.82983901,120.2429094)
\curveto(95.25692299,120.2429094)(94.7881723,120.43561781)(94.41317229,120.82624276)
\curveto(94.03817228,121.2272841)(93.85067227,121.74811746)(93.85067227,122.38874281)
\curveto(93.85067227,122.78978416)(93.93921361,123.14395084)(94.12150561,123.45124285)
\curveto(94.37150562,123.8522842)(94.79858964,124.23249354)(95.41317233,124.59707622)
\curveto(96.03817235,124.97207624)(97.06421372,125.41999359)(98.49650577,125.95124294)
\lineto(98.49650577,126.26374295)
\curveto(98.49650577,127.09707631)(98.36108976,127.66478433)(98.10067242,127.97207634)
\curveto(97.83504708,128.28978435)(97.45483906,128.45124303)(96.95483905,128.45124303)
\curveto(96.5642137,128.45124303)(96.25692302,128.34707636)(96.03817235,128.13874302)
\curveto(95.79858967,127.93040967)(95.68400567,127.690827)(95.68400567,127.43040966)
\lineto(95.704839,126.90957631)
\curveto(95.704839,126.61790963)(95.631923,126.39395095)(95.49650566,126.24290961)
\curveto(95.35588032,126.10228428)(95.16838032,126.03457627)(94.93400564,126.03457627)
\curveto(94.71004697,126.03457627)(94.52254696,126.10749361)(94.37150562,126.26374295)
\curveto(94.23088028,126.41478429)(94.16317228,126.62832696)(94.16317228,126.90957631)
\curveto(94.16317228,127.43561766)(94.42358962,127.91478434)(94.95483898,128.34707636)
\curveto(95.49650566,128.78978437)(96.25171369,129.01374305)(97.22567239,129.01374305)
\curveto(97.96004708,129.01374305)(98.56421377,128.88874304)(99.03817245,128.63874303)
\curveto(99.39754713,128.44082703)(99.66317248,128.14395102)(99.82983915,127.74290967)
\curveto(99.93921382,127.47728433)(99.99650582,126.94082697)(99.99650582,126.13874294)
\lineto(99.99650582,123.30540951)
\curveto(99.99650582,122.51374282)(100.00691916,122.02416013)(100.03817249,121.84707613)
\curveto(100.07983916,121.66478412)(100.13192316,121.53978411)(100.20483916,121.47207611)
\curveto(100.27254716,121.41478411)(100.3506725,121.38874278)(100.43400584,121.38874278)
\curveto(100.52775651,121.38874278)(100.62150584,121.40957611)(100.70483918,121.45124278)
\curveto(100.82983918,121.53457611)(101.06942319,121.75332679)(101.43400587,122.11790947)
\lineto(101.43400587,121.61790945)
\curveto(100.75171385,120.70124275)(100.10067249,120.2429094)(99.47567247,120.2429094)
\curveto(99.18400579,120.2429094)(98.94442312,120.34707607)(98.76733911,120.55540941)
\curveto(98.60067244,120.76374275)(98.5069231,121.11790943)(98.49650577,121.61790945)
\closepath
\moveto(98.49650577,122.20124281)
\lineto(98.49650577,125.38874292)
\curveto(97.57983907,125.02416024)(96.98608971,124.7689509)(96.72567237,124.61790956)
\curveto(96.23608969,124.33666021)(95.89233901,124.05540954)(95.68400567,123.76374286)
\curveto(95.47567233,123.47207618)(95.37150566,123.15957617)(95.37150566,122.82624283)
\curveto(95.37150566,122.37832681)(95.50171366,122.0137428)(95.76733901,121.72207612)
\curveto(96.02775635,121.43040944)(96.33504703,121.28457611)(96.6840057,121.28457611)
\curveto(97.14233905,121.28457611)(97.74650574,121.58666012)(98.49650577,122.20124281)
\closepath
\moveto(98.49650577,122.20124281)
}
}
{
\newrgbcolor{curcolor}{0 0 0}
\pscustom[linestyle=none,fillstyle=solid,fillcolor=curcolor]
{
\newpath
\moveto(101.5761771,128.76374304)
\lineto(105.47201057,128.76374304)
\lineto(105.47201057,128.43040969)
\lineto(105.28451057,128.43040969)
\curveto(105.00326122,128.43040969)(104.79492788,128.36790969)(104.65951054,128.24290969)
\curveto(104.53451054,128.12832702)(104.47201054,127.98249368)(104.47201054,127.80540967)
\curveto(104.47201054,127.565827)(104.56576121,127.24290965)(104.76367721,126.82624297)
\lineto(106.80534395,122.59707615)
\lineto(108.65951069,127.22207632)
\curveto(108.76888536,127.47207633)(108.82617736,127.71166033)(108.82617736,127.95124301)
\curveto(108.82617736,128.06061768)(108.80534402,128.14395102)(108.76367736,128.20124302)
\curveto(108.70638535,128.26895102)(108.62305202,128.32624302)(108.51367735,128.36790969)
\curveto(108.41471868,128.40957636)(108.232428,128.43040969)(107.97201066,128.43040969)
\lineto(107.97201066,128.76374304)
\lineto(110.70117742,128.76374304)
\lineto(110.70117742,128.43040969)
\curveto(110.47721875,128.39916036)(110.30534408,128.34707636)(110.18034407,128.26374302)
\curveto(110.05534407,128.19082702)(109.91992806,128.06061768)(109.78451073,127.86790967)
\curveto(109.72721872,127.78457634)(109.62305205,127.54499366)(109.47201071,127.15957631)
\lineto(106.07617726,118.82624269)
\curveto(105.74284392,118.02415866)(105.3053439,117.41999197)(104.76367721,117.01374262)
\curveto(104.23242786,116.59707594)(103.72721851,116.3887426)(103.24284383,116.3887426)
\curveto(102.87826115,116.3887426)(102.58138514,116.49290927)(102.34701046,116.70124261)
\curveto(102.12305179,116.89915862)(102.01367712,117.13353463)(102.01367712,117.40957597)
\curveto(102.01367712,117.65957598)(102.09701045,117.86270132)(102.26367713,118.01374266)
\curveto(102.4303438,118.169992)(102.65951047,118.24290933)(102.95117715,118.24290933)
\curveto(103.14388516,118.24290933)(103.4147185,118.18040933)(103.76367718,118.05540932)
\curveto(104.01367719,117.96165865)(104.16471853,117.90957599)(104.22201053,117.90957599)
\curveto(104.39909453,117.90957599)(104.59701054,118.00853466)(104.80534388,118.20124266)
\curveto(105.02409456,118.38353467)(105.2480519,118.74290935)(105.47201057,119.28457604)
\lineto(106.07617726,120.74290942)
\lineto(103.07617715,127.03457631)
\curveto(102.97721848,127.22728432)(102.83138515,127.46166032)(102.63867714,127.74290967)
\curveto(102.4824278,127.95124301)(102.3574278,128.08666035)(102.26367713,128.15957635)
\curveto(102.12305179,128.25332702)(101.89388511,128.34707636)(101.5761771,128.43040969)
\closepath
\moveto(101.5761771,128.76374304)
}
}
{
\newrgbcolor{curcolor}{0 0 0}
\pscustom[linestyle=none,fillstyle=solid,fillcolor=curcolor]
{
\newpath
\moveto(112.78451083,125.61790959)
\curveto(112.78451083,124.37832688)(113.08659484,123.40957618)(113.70117753,122.70124282)
\curveto(114.29492822,121.99290946)(115.01367758,121.63874279)(115.84701094,121.63874279)
\curveto(116.38867763,121.63874279)(116.85742831,121.78457612)(117.26367766,122.0762428)
\curveto(117.664719,122.37832681)(118.00326168,122.89916016)(118.28451103,123.63874286)
\lineto(118.55534437,123.45124285)
\curveto(118.43034437,122.61790949)(118.06055235,121.85749346)(117.45117766,121.18040944)
\curveto(116.83659498,120.49811781)(116.06576161,120.15957607)(115.13867758,120.15957607)
\curveto(114.13867755,120.15957607)(113.27409485,120.54499341)(112.55534416,121.32624277)
\curveto(111.8470108,122.10228413)(111.49284412,123.14916017)(111.49284412,124.47207622)
\curveto(111.49284412,125.89916027)(111.85742813,127.01374298)(112.59701083,127.80540967)
\curveto(113.33138552,128.6074937)(114.24805222,129.01374305)(115.34701092,129.01374305)
\curveto(116.28971896,129.01374305)(117.06055232,128.70124304)(117.659511,128.07624301)
\curveto(118.25326169,127.46166032)(118.55534437,126.64395096)(118.55534437,125.61790959)
\closepath
\moveto(112.78451083,126.13874294)
\lineto(116.65951097,126.13874294)
\curveto(116.62826163,126.68040963)(116.56576163,127.05540964)(116.47201096,127.26374298)
\curveto(116.31576162,127.60749366)(116.08659495,127.87832701)(115.78451094,128.07624301)
\curveto(115.47721893,128.26895102)(115.16471892,128.36790969)(114.84701091,128.36790969)
\curveto(114.34701089,128.36790969)(113.89388554,128.16999368)(113.49284419,127.78457634)
\curveto(113.08659484,127.39395099)(112.85221883,126.8470763)(112.78451083,126.13874294)
\closepath
\moveto(112.78451083,126.13874294)
}
}
{
\newrgbcolor{curcolor}{0 0 0}
\pscustom[linestyle=none,fillstyle=solid,fillcolor=curcolor]
{
\newpath
\moveto(122.11416204,129.01374305)
\lineto(122.11416204,127.11790965)
\curveto(122.8224954,128.37832702)(123.54124609,129.01374305)(124.28082878,129.01374305)
\curveto(124.61416213,129.01374305)(124.88499547,128.90957638)(125.09332881,128.70124304)
\curveto(125.31207949,128.49290969)(125.42666216,128.25332702)(125.42666216,127.99290968)
\curveto(125.42666216,127.753327)(125.34332882,127.55540966)(125.17666215,127.38874299)
\curveto(125.02041281,127.22207632)(124.8433288,127.13874298)(124.63499546,127.13874298)
\curveto(124.41103679,127.13874298)(124.16624611,127.24290965)(123.90582877,127.45124299)
\curveto(123.64020343,127.66999367)(123.44749542,127.78457634)(123.32249542,127.78457634)
\curveto(123.20791275,127.78457634)(123.09332874,127.72207633)(122.96832874,127.59707633)
\curveto(122.6870794,127.34707632)(122.40582872,126.93561764)(122.11416204,126.36790962)
\lineto(122.11416204,122.36790948)
\curveto(122.11416204,121.89395079)(122.17666204,121.53978411)(122.30166205,121.30540944)
\curveto(122.36937005,121.1491601)(122.49957939,121.01895076)(122.6974954,120.90957609)
\curveto(122.90582874,120.79499342)(123.19749541,120.74290942)(123.57249543,120.74290942)
\lineto(123.57249543,120.40957607)
\lineto(119.30166194,120.40957607)
\lineto(119.30166194,120.74290942)
\curveto(119.72874596,120.74290942)(120.05166197,120.80540942)(120.25999531,120.93040943)
\curveto(120.41103665,121.0241601)(120.51520332,121.18561744)(120.57249532,121.40957611)
\curveto(120.59853665,121.50332678)(120.61416199,121.79499346)(120.61416199,122.28457614)
\lineto(120.61416199,125.51374292)
\curveto(120.61416199,126.48249362)(120.59332865,127.06061764)(120.55166199,127.24290965)
\curveto(120.50999532,127.43561766)(120.43186998,127.576243)(120.32249531,127.65957633)
\curveto(120.22353664,127.74290967)(120.09853664,127.78457634)(119.9474953,127.78457634)
\curveto(119.74957929,127.78457634)(119.53603662,127.74290967)(119.30166194,127.65957633)
\lineto(119.21832861,127.99290968)
\lineto(121.73916203,129.01374305)
\closepath
\moveto(122.11416204,129.01374305)
}
}
{
\newrgbcolor{curcolor}{0 0 0}
\pscustom[linewidth=0.7466667,linecolor=curcolor]
{
\newpath
\moveto(92.67541662,83.02956987)
\lineto(92.67541662,55.29519555)
}
}
{
\newrgbcolor{curcolor}{0 0 0}
\pscustom[linewidth=0.7466667,linecolor=curcolor]
{
\newpath
\moveto(184.94104655,83.02956987)
\lineto(184.94104655,55.29519555)
}
}
{
\newrgbcolor{curcolor}{0 0 0}
\pscustom[linewidth=0.7466667,linecolor=curcolor]
{
\newpath
\moveto(277.34208983,83.02956987)
\lineto(277.34208983,55.29519555)
}
}
{
\newrgbcolor{curcolor}{0 0 0}
\pscustom[linewidth=0.7466667,linecolor=curcolor]
{
\newpath
\moveto(369.60771976,83.02956987)
\lineto(369.60771976,55.29519555)
}
}
{
\newrgbcolor{curcolor}{0 0 0}
\pscustom[linewidth=0.7466667,linecolor=curcolor]
{
\newpath
\moveto(462.00876304,83.02956987)
\lineto(462.00876304,55.29519555)
}
}
{
\newrgbcolor{curcolor}{0 0 0}
\pscustom[linewidth=0.7466667,linecolor=curcolor]
{
\newpath
\moveto(554.13897963,83.02956987)
\lineto(554.13897963,55.29519555)
}
}
{
\newrgbcolor{curcolor}{0 0 0}
\pscustom[linewidth=0.7466667,linecolor=curcolor]
{
\newpath
\moveto(0.37333335,176.14935983)
\lineto(647.22231626,176.14935983)
\lineto(647.22231626,204.18061416)
\lineto(0.37333335,204.18061416)
\closepath
\moveto(0.37333335,176.14935983)
}
}
{
\newrgbcolor{curcolor}{0 0 0}
\pscustom[linewidth=0.7466667,linecolor=curcolor]
{
\newpath
\moveto(92.67541662,203.02957412)
\lineto(92.67541662,175.43061314)
}
}
{
\newrgbcolor{curcolor}{0 0 0}
\pscustom[linewidth=0.7466667,linecolor=curcolor]
{
\newpath
\moveto(184.94104655,203.02957412)
\lineto(184.94104655,175.43061314)
}
}
{
\newrgbcolor{curcolor}{0 0 0}
\pscustom[linewidth=0.7466667,linecolor=curcolor]
{
\newpath
\moveto(277.34208983,203.02957412)
\lineto(277.34208983,175.43061314)
}
}
{
\newrgbcolor{curcolor}{0 0 0}
\pscustom[linewidth=0.7466667,linecolor=curcolor]
{
\newpath
\moveto(369.60771976,203.02957412)
\lineto(369.60771976,175.43061314)
}
}
{
\newrgbcolor{curcolor}{0 0 0}
\pscustom[linewidth=0.7466667,linecolor=curcolor]
{
\newpath
\moveto(462.00876304,203.02957412)
\lineto(462.00876304,175.43061314)
}
}
{
\newrgbcolor{curcolor}{0 0 0}
\pscustom[linewidth=0.7466667,linecolor=curcolor]
{
\newpath
\moveto(554.13897963,203.02957412)
\lineto(554.13897963,175.43061314)
}
}
{
\newrgbcolor{curcolor}{0 0 0}
\pscustom[linewidth=0.7466667,linecolor=curcolor]
{
\newpath
\moveto(0.40458268,83.02956987)
\lineto(323.47230479,119.96186451)
}
}
{
\newrgbcolor{curcolor}{0 0 0}
\pscustom[linewidth=0.7466667,linecolor=curcolor]
{
\newpath
\moveto(646.54002291,83.02956987)
\lineto(323.47230479,119.96186451)
}
}
{
\newrgbcolor{curcolor}{0 0 0}
\pscustom[linewidth=0.7466667,linecolor=curcolor]
{
\newpath
\moveto(55.73792197,27.69623457)
\lineto(231.07126152,55.29519555)
}
}
{
\newrgbcolor{curcolor}{0 0 0}
\pscustom[linewidth=0.7466667,linecolor=curcolor]
{
\newpath
\moveto(406.5400144,27.69623457)
\lineto(231.07126152,55.29519555)
}
}
{
\newrgbcolor{curcolor}{0 0 0}
\pscustom[linewidth=0.7466667,linecolor=curcolor]
{
\newpath
\moveto(54.99833528,0.37331361)
\lineto(406.55042774,0.37331361)
\lineto(406.55042774,28.40456793)
\lineto(54.99833528,28.40456793)
\closepath
\moveto(54.99833528,0.37331361)
}
}
{
\newrgbcolor{curcolor}{0 0 0}
\pscustom[linewidth=0.7466667,linecolor=curcolor]
{
\newpath
\moveto(267.63896948,119.98790451)
\lineto(378.42022674,119.98790451)
\lineto(378.42022674,148.01915884)
\lineto(267.63896948,148.01915884)
\closepath
\moveto(267.63896948,119.98790451)
}
}
{
\newrgbcolor{curcolor}{0 0 0}
\pscustom[linestyle=none,fillstyle=solid,fillcolor=curcolor]
{
\newpath
\moveto(323.50875813,147.73269216)
\lineto(321.10771804,157.3316525)
\lineto(325.90979821,157.3316525)
}
}
{
\newrgbcolor{curcolor}{0 0 0}
\pscustom[linewidth=0.13333334,linecolor=curcolor]
{
\newpath
\moveto(323.50875813,147.73269216)
\lineto(321.10771804,157.3316525)
\lineto(325.90979821,157.3316525)
\closepath
\moveto(323.50875813,147.73269216)
}
}
{
\newrgbcolor{curcolor}{0 0 0}
\pscustom[linewidth=0.7466667,linecolor=curcolor]
{
\newpath
\moveto(323.47230479,175.43061314)
\lineto(323.47230479,157.29519917)
}
}
{
\newrgbcolor{curcolor}{0 0 0}
\pscustom[linewidth=0.7466667,linecolor=curcolor]
{
\newpath
\moveto(0.40458268,203.02957412)
\lineto(323.47230479,239.96186876)
}
}
{
\newrgbcolor{curcolor}{0 0 0}
\pscustom[linewidth=0.7466667,linecolor=curcolor]
{
\newpath
\moveto(646.54002291,203.02957412)
\lineto(323.47230479,239.96186876)
}
}
{
\newrgbcolor{curcolor}{0 0 0}
\pscustom[linewidth=0.7466667,linecolor=curcolor]
{
\newpath
\moveto(267.63896948,239.60249542)
\lineto(378.42022674,239.60249542)
\lineto(378.42022674,267.63894974)
\lineto(267.63896948,267.63894974)
\closepath
\moveto(267.63896948,239.60249542)
}
}
{
\newrgbcolor{curcolor}{0 0 0}
\pscustom[linestyle=none,fillstyle=solid,fillcolor=curcolor]
{
\newpath
\moveto(323.50875813,267.63374974)
\lineto(321.10771804,277.23791008)
\lineto(325.90979821,277.23791008)
}
}
{
\newrgbcolor{curcolor}{0 0 0}
\pscustom[linewidth=0.13333334,linecolor=curcolor]
{
\newpath
\moveto(323.50875813,267.63373641)
\lineto(321.10771804,277.23791008)
\lineto(325.90979821,277.23791008)
\closepath
\moveto(323.50875813,267.63373641)
}
}
{
\newrgbcolor{curcolor}{0 0 0}
\pscustom[linewidth=0.7466667,linecolor=curcolor]
{
\newpath
\moveto(323.47230479,295.43061739)
\lineto(323.47230479,277.29520342)
}
}
{
\newrgbcolor{curcolor}{0 0 0}
\pscustom[linewidth=0.7466667,linecolor=curcolor]
{
\newpath
\moveto(267.63896948,295.76395074)
\lineto(378.42022674,295.76395074)
\lineto(378.42022674,323.79520506)
\lineto(267.63896948,323.79520506)
\closepath
\moveto(267.63896948,295.76395074)
}
}
{
\newrgbcolor{curcolor}{0 0 0}
\pscustom[linewidth=0.7466667,linecolor=curcolor]
{
\newpath
\moveto(0.40458268,166.09727948)
\lineto(7.73791627,166.09727948)
}
}
{
\newrgbcolor{curcolor}{0 0 0}
\pscustom[linewidth=0.7466667,linecolor=curcolor]
{
\newpath
\moveto(15.20666721,166.09727948)
\lineto(22.6754168,166.09727948)
}
}
{
\newrgbcolor{curcolor}{0 0 0}
\pscustom[linewidth=0.7466667,linecolor=curcolor]
{
\newpath
\moveto(30.0087504,166.09727948)
\lineto(37.47229333,166.09727948)
}
}
{
\newrgbcolor{curcolor}{0 0 0}
\pscustom[linewidth=0.7466667,linecolor=curcolor]
{
\newpath
\moveto(44.94104159,166.09727948)
\lineto(52.27437519,166.09727948)
}
}
{
\newrgbcolor{curcolor}{0 0 0}
\pscustom[linewidth=0.7466667,linecolor=curcolor]
{
\newpath
\moveto(59.73792212,166.09727948)
\lineto(67.20666905,166.09727948)
}
}
{
\newrgbcolor{curcolor}{0 0 0}
\pscustom[linewidth=0.7466667,linecolor=curcolor]
{
\newpath
\moveto(74.67541598,166.09727948)
\lineto(82.00874957,166.09727948)
}
}
{
\newrgbcolor{curcolor}{0 0 0}
\pscustom[linewidth=0.7466667,linecolor=curcolor]
{
\newpath
\moveto(89.4722965,166.09727948)
\lineto(96.8056301,166.09727948)
}
}
{
\newrgbcolor{curcolor}{0 0 0}
\pscustom[linewidth=0.7466667,linecolor=curcolor]
{
\newpath
\moveto(104.40459037,166.09727948)
\lineto(111.73792396,166.09727948)
}
}
{
\newrgbcolor{curcolor}{0 0 0}
\pscustom[linewidth=0.7466667,linecolor=curcolor]
{
\newpath
\moveto(119.20667089,166.09727948)
\lineto(126.67541782,166.09727948)
}
}
{
\newrgbcolor{curcolor}{0 0 0}
\pscustom[linewidth=0.7466667,linecolor=curcolor]
{
\newpath
\moveto(134.13896475,166.09727948)
\lineto(141.47229835,166.09727948)
}
}
{
\newrgbcolor{curcolor}{0 0 0}
\pscustom[linewidth=0.7466667,linecolor=curcolor]
{
\newpath
\moveto(148.94104528,166.09727948)
\lineto(156.27437887,166.09727948)
}
}
{
\newrgbcolor{curcolor}{0 0 0}
\pscustom[linewidth=0.7466667,linecolor=curcolor]
{
\newpath
\moveto(163.7379258,166.09727948)
\lineto(171.20667273,166.09727948)
}
}
{
\newrgbcolor{curcolor}{0 0 0}
\pscustom[linewidth=0.7466667,linecolor=curcolor]
{
\newpath
\moveto(178.67541966,166.09727948)
\lineto(186.00875326,166.09727948)
}
}
{
\newrgbcolor{curcolor}{0 0 0}
\pscustom[linewidth=0.7466667,linecolor=curcolor]
{
\newpath
\moveto(193.47230019,166.09727948)
\lineto(200.94104712,166.09727948)
}
}
{
\newrgbcolor{curcolor}{0 0 0}
\pscustom[linewidth=0.7466667,linecolor=curcolor]
{
\newpath
\moveto(208.27438071,166.09727948)
\lineto(215.73792764,166.09727948)
}
}
{
\newrgbcolor{curcolor}{0 0 0}
\pscustom[linewidth=0.7466667,linecolor=curcolor]
{
\newpath
\moveto(223.20667458,166.09727948)
\lineto(230.54000817,166.09727948)
}
}
{
\newrgbcolor{curcolor}{0 0 0}
\pscustom[linewidth=0.7466667,linecolor=curcolor]
{
\newpath
\moveto(238.0087551,166.09727948)
\lineto(245.47230203,166.09727948)
}
}
{
\newrgbcolor{curcolor}{0 0 0}
\pscustom[linewidth=0.7466667,linecolor=curcolor]
{
\newpath
\moveto(252.94104896,166.09727948)
\lineto(260.40459589,166.09727948)
}
}
{
\newrgbcolor{curcolor}{0 0 0}
\pscustom[linewidth=0.7466667,linecolor=curcolor]
{
\newpath
\moveto(267.73792949,166.09727948)
\lineto(275.07126308,166.09727948)
}
}
{
\newrgbcolor{curcolor}{0 0 0}
\pscustom[linewidth=0.7466667,linecolor=curcolor]
{
\newpath
\moveto(282.67542335,166.09727948)
\lineto(290.00875694,166.09727948)
}
}
{
\newrgbcolor{curcolor}{0 0 0}
\pscustom[linewidth=0.7466667,linecolor=curcolor]
{
\newpath
\moveto(297.47230387,166.09727948)
\lineto(304.9410508,166.09727948)
}
}
{
\newrgbcolor{curcolor}{0 0 0}
\pscustom[linewidth=0.7466667,linecolor=curcolor]
{
\newpath
\moveto(312.40459774,166.09727948)
\lineto(319.73793133,166.09727948)
}
}
{
\newrgbcolor{curcolor}{0 0 0}
\pscustom[linewidth=0.7466667,linecolor=curcolor]
{
\newpath
\moveto(327.20667826,166.09727948)
\lineto(334.54001185,166.09727948)
}
}
{
\newrgbcolor{curcolor}{0 0 0}
\pscustom[linewidth=0.7466667,linecolor=curcolor]
{
\newpath
\moveto(342.00875878,166.09727948)
\lineto(349.47230572,166.09727948)
}
}
{
\newrgbcolor{curcolor}{0 0 0}
\pscustom[linewidth=0.7466667,linecolor=curcolor]
{
\newpath
\moveto(356.94105265,166.09727948)
\lineto(364.27438624,166.09727948)
}
}
{
\newrgbcolor{curcolor}{0 0 0}
\pscustom[linewidth=0.7466667,linecolor=curcolor]
{
\newpath
\moveto(371.73793317,166.09727948)
\lineto(379.2066801,166.09727948)
}
}
{
\newrgbcolor{curcolor}{0 0 0}
\pscustom[linewidth=0.7466667,linecolor=curcolor]
{
\newpath
\moveto(386.5400137,166.09727948)
\lineto(394.00876063,166.09727948)
}
}
{
\newrgbcolor{curcolor}{0 0 0}
\pscustom[linewidth=0.7466667,linecolor=curcolor]
{
\newpath
\moveto(401.47230756,166.09727948)
\lineto(408.80564115,166.09727948)
}
}
{
\newrgbcolor{curcolor}{0 0 0}
\pscustom[linewidth=0.7466667,linecolor=curcolor]
{
\newpath
\moveto(416.40460142,166.09727948)
\lineto(423.73793501,166.09727948)
}
}
{
\newrgbcolor{curcolor}{0 0 0}
\pscustom[linewidth=0.7466667,linecolor=curcolor]
{
\newpath
\moveto(431.07126861,166.09727948)
\lineto(438.54001554,166.09727948)
}
}
{
\newrgbcolor{curcolor}{0 0 0}
\pscustom[linewidth=0.7466667,linecolor=curcolor]
{
\newpath
\moveto(446.00876247,166.09727948)
\lineto(453.34209606,166.09727948)
}
}
{
\newrgbcolor{curcolor}{0 0 0}
\pscustom[linewidth=0.7466667,linecolor=curcolor]
{
\newpath
\moveto(460.94105633,166.09727948)
\lineto(468.27438992,166.09727948)
}
}
{
\newrgbcolor{curcolor}{0 0 0}
\pscustom[linewidth=0.7466667,linecolor=curcolor]
{
\newpath
\moveto(475.73793686,166.09727948)
\lineto(483.20668379,166.09727948)
}
}
{
\newrgbcolor{curcolor}{0 0 0}
\pscustom[linewidth=0.7466667,linecolor=curcolor]
{
\newpath
\moveto(490.54001738,166.09727948)
\lineto(498.00876431,166.09727948)
}
}
{
\newrgbcolor{curcolor}{0 0 0}
\pscustom[linewidth=0.7466667,linecolor=curcolor]
{
\newpath
\moveto(505.47231124,166.09727948)
\lineto(512.80564484,166.09727948)
}
}
{
\newrgbcolor{curcolor}{0 0 0}
\pscustom[linewidth=0.7466667,linecolor=curcolor]
{
\newpath
\moveto(520.27439177,166.09727948)
\lineto(527.7379387,166.09727948)
}
}
{
\newrgbcolor{curcolor}{0 0 0}
\pscustom[linewidth=0.7466667,linecolor=curcolor]
{
\newpath
\moveto(535.20668563,166.09727948)
\lineto(542.54001922,166.09727948)
}
}
{
\newrgbcolor{curcolor}{0 0 0}
\pscustom[linewidth=0.7466667,linecolor=curcolor]
{
\newpath
\moveto(550.00876615,166.09727948)
\lineto(557.34209975,166.09727948)
}
}
{
\newrgbcolor{curcolor}{0 0 0}
\pscustom[linewidth=0.7466667,linecolor=curcolor]
{
\newpath
\moveto(564.80564668,166.09727948)
\lineto(572.27439361,166.09727948)
}
}
{
\newrgbcolor{curcolor}{0 0 0}
\pscustom[linewidth=0.7466667,linecolor=curcolor]
{
\newpath
\moveto(579.73794054,166.09727948)
\lineto(587.07127413,166.09727948)
}
}
{
\newrgbcolor{curcolor}{0 0 0}
\pscustom[linewidth=0.7466667,linecolor=curcolor]
{
\newpath
\moveto(594.6754344,166.09727948)
\lineto(602.008768,166.09727948)
}
}
{
\newrgbcolor{curcolor}{0 0 0}
\pscustom[linewidth=0.7466667,linecolor=curcolor]
{
\newpath
\moveto(609.34210159,166.09727948)
\lineto(616.80564852,166.09727948)
}
}
{
\newrgbcolor{curcolor}{0 0 0}
\pscustom[linewidth=0.7466667,linecolor=curcolor]
{
\newpath
\moveto(624.27439545,166.09727948)
\lineto(631.60772905,166.09727948)
}
}
{
\newrgbcolor{curcolor}{0 0 0}
\pscustom[linewidth=0.7466667,linecolor=curcolor]
{
\newpath
\moveto(639.20668931,166.09727948)
\lineto(646.54002291,166.09727948)
}
}
{
\newrgbcolor{curcolor}{0 0 0}
\pscustom[linewidth=0.7466667,linecolor=curcolor]
{
\newpath
\moveto(0.40458268,286.22749707)
\lineto(7.73791627,286.22749707)
}
}
{
\newrgbcolor{curcolor}{0 0 0}
\pscustom[linewidth=0.7466667,linecolor=curcolor]
{
\newpath
\moveto(15.20666721,286.22749707)
\lineto(22.6754168,286.22749707)
}
}
{
\newrgbcolor{curcolor}{0 0 0}
\pscustom[linewidth=0.7466667,linecolor=curcolor]
{
\newpath
\moveto(30.0087504,286.22749707)
\lineto(37.47229333,286.22749707)
}
}
{
\newrgbcolor{curcolor}{0 0 0}
\pscustom[linewidth=0.7466667,linecolor=curcolor]
{
\newpath
\moveto(44.94104159,286.22749707)
\lineto(52.27437519,286.22749707)
}
}
{
\newrgbcolor{curcolor}{0 0 0}
\pscustom[linewidth=0.7466667,linecolor=curcolor]
{
\newpath
\moveto(59.73792212,286.22749707)
\lineto(67.20666905,286.22749707)
}
}
{
\newrgbcolor{curcolor}{0 0 0}
\pscustom[linewidth=0.7466667,linecolor=curcolor]
{
\newpath
\moveto(74.67541598,286.22749707)
\lineto(82.00874957,286.22749707)
}
}
{
\newrgbcolor{curcolor}{0 0 0}
\pscustom[linewidth=0.7466667,linecolor=curcolor]
{
\newpath
\moveto(89.4722965,286.22749707)
\lineto(96.8056301,286.22749707)
}
}
{
\newrgbcolor{curcolor}{0 0 0}
\pscustom[linewidth=0.7466667,linecolor=curcolor]
{
\newpath
\moveto(104.40459037,286.22749707)
\lineto(111.73792396,286.22749707)
}
}
{
\newrgbcolor{curcolor}{0 0 0}
\pscustom[linewidth=0.7466667,linecolor=curcolor]
{
\newpath
\moveto(119.20667089,286.22749707)
\lineto(126.67541782,286.22749707)
}
}
{
\newrgbcolor{curcolor}{0 0 0}
\pscustom[linewidth=0.7466667,linecolor=curcolor]
{
\newpath
\moveto(134.13896475,286.22749707)
\lineto(141.47229835,286.22749707)
}
}
{
\newrgbcolor{curcolor}{0 0 0}
\pscustom[linewidth=0.7466667,linecolor=curcolor]
{
\newpath
\moveto(148.94104528,286.22749707)
\lineto(156.27437887,286.22749707)
}
}
{
\newrgbcolor{curcolor}{0 0 0}
\pscustom[linewidth=0.7466667,linecolor=curcolor]
{
\newpath
\moveto(163.7379258,286.22749707)
\lineto(171.20667273,286.22749707)
}
}
{
\newrgbcolor{curcolor}{0 0 0}
\pscustom[linewidth=0.7466667,linecolor=curcolor]
{
\newpath
\moveto(178.67541966,286.22749707)
\lineto(186.00875326,286.22749707)
}
}
{
\newrgbcolor{curcolor}{0 0 0}
\pscustom[linewidth=0.7466667,linecolor=curcolor]
{
\newpath
\moveto(193.47230019,286.22749707)
\lineto(200.94104712,286.22749707)
}
}
{
\newrgbcolor{curcolor}{0 0 0}
\pscustom[linewidth=0.7466667,linecolor=curcolor]
{
\newpath
\moveto(208.27438071,286.22749707)
\lineto(215.73792764,286.22749707)
}
}
{
\newrgbcolor{curcolor}{0 0 0}
\pscustom[linewidth=0.7466667,linecolor=curcolor]
{
\newpath
\moveto(223.20667458,286.22749707)
\lineto(230.54000817,286.22749707)
}
}
{
\newrgbcolor{curcolor}{0 0 0}
\pscustom[linewidth=0.7466667,linecolor=curcolor]
{
\newpath
\moveto(238.0087551,286.22749707)
\lineto(245.47230203,286.22749707)
}
}
{
\newrgbcolor{curcolor}{0 0 0}
\pscustom[linewidth=0.7466667,linecolor=curcolor]
{
\newpath
\moveto(252.94104896,286.22749707)
\lineto(260.40459589,286.22749707)
}
}
{
\newrgbcolor{curcolor}{0 0 0}
\pscustom[linewidth=0.7466667,linecolor=curcolor]
{
\newpath
\moveto(267.73792949,286.22749707)
\lineto(275.07126308,286.22749707)
}
}
{
\newrgbcolor{curcolor}{0 0 0}
\pscustom[linewidth=0.7466667,linecolor=curcolor]
{
\newpath
\moveto(282.67542335,286.22749707)
\lineto(290.00875694,286.22749707)
}
}
{
\newrgbcolor{curcolor}{0 0 0}
\pscustom[linewidth=0.7466667,linecolor=curcolor]
{
\newpath
\moveto(297.47230387,286.22749707)
\lineto(304.9410508,286.22749707)
}
}
{
\newrgbcolor{curcolor}{0 0 0}
\pscustom[linewidth=0.7466667,linecolor=curcolor]
{
\newpath
\moveto(312.40459774,286.22749707)
\lineto(319.73793133,286.22749707)
}
}
{
\newrgbcolor{curcolor}{0 0 0}
\pscustom[linewidth=0.7466667,linecolor=curcolor]
{
\newpath
\moveto(327.20667826,286.22749707)
\lineto(334.54001185,286.22749707)
}
}
{
\newrgbcolor{curcolor}{0 0 0}
\pscustom[linewidth=0.7466667,linecolor=curcolor]
{
\newpath
\moveto(342.00875878,286.22749707)
\lineto(349.47230572,286.22749707)
}
}
{
\newrgbcolor{curcolor}{0 0 0}
\pscustom[linewidth=0.7466667,linecolor=curcolor]
{
\newpath
\moveto(356.94105265,286.22749707)
\lineto(364.27438624,286.22749707)
}
}
{
\newrgbcolor{curcolor}{0 0 0}
\pscustom[linewidth=0.7466667,linecolor=curcolor]
{
\newpath
\moveto(371.73793317,286.22749707)
\lineto(379.2066801,286.22749707)
}
}
{
\newrgbcolor{curcolor}{0 0 0}
\pscustom[linewidth=0.7466667,linecolor=curcolor]
{
\newpath
\moveto(386.5400137,286.22749707)
\lineto(394.00876063,286.22749707)
}
}
{
\newrgbcolor{curcolor}{0 0 0}
\pscustom[linewidth=0.7466667,linecolor=curcolor]
{
\newpath
\moveto(401.47230756,286.22749707)
\lineto(408.80564115,286.22749707)
}
}
{
\newrgbcolor{curcolor}{0 0 0}
\pscustom[linewidth=0.7466667,linecolor=curcolor]
{
\newpath
\moveto(416.40460142,286.22749707)
\lineto(423.73793501,286.22749707)
}
}
{
\newrgbcolor{curcolor}{0 0 0}
\pscustom[linewidth=0.7466667,linecolor=curcolor]
{
\newpath
\moveto(431.07126861,286.22749707)
\lineto(438.54001554,286.22749707)
}
}
{
\newrgbcolor{curcolor}{0 0 0}
\pscustom[linewidth=0.7466667,linecolor=curcolor]
{
\newpath
\moveto(446.00876247,286.22749707)
\lineto(453.34209606,286.22749707)
}
}
{
\newrgbcolor{curcolor}{0 0 0}
\pscustom[linewidth=0.7466667,linecolor=curcolor]
{
\newpath
\moveto(460.94105633,286.22749707)
\lineto(468.27438992,286.22749707)
}
}
{
\newrgbcolor{curcolor}{0 0 0}
\pscustom[linewidth=0.7466667,linecolor=curcolor]
{
\newpath
\moveto(475.73793686,286.22749707)
\lineto(483.20668379,286.22749707)
}
}
{
\newrgbcolor{curcolor}{0 0 0}
\pscustom[linewidth=0.7466667,linecolor=curcolor]
{
\newpath
\moveto(490.54001738,286.22749707)
\lineto(498.00876431,286.22749707)
}
}
{
\newrgbcolor{curcolor}{0 0 0}
\pscustom[linewidth=0.7466667,linecolor=curcolor]
{
\newpath
\moveto(505.47231124,286.22749707)
\lineto(512.80564484,286.22749707)
}
}
{
\newrgbcolor{curcolor}{0 0 0}
\pscustom[linewidth=0.7466667,linecolor=curcolor]
{
\newpath
\moveto(520.27439177,286.22749707)
\lineto(527.7379387,286.22749707)
}
}
{
\newrgbcolor{curcolor}{0 0 0}
\pscustom[linewidth=0.7466667,linecolor=curcolor]
{
\newpath
\moveto(535.20668563,286.22749707)
\lineto(542.54001922,286.22749707)
}
}
{
\newrgbcolor{curcolor}{0 0 0}
\pscustom[linewidth=0.7466667,linecolor=curcolor]
{
\newpath
\moveto(550.00876615,286.22749707)
\lineto(557.34209975,286.22749707)
}
}
{
\newrgbcolor{curcolor}{0 0 0}
\pscustom[linewidth=0.7466667,linecolor=curcolor]
{
\newpath
\moveto(564.80564668,286.22749707)
\lineto(572.27439361,286.22749707)
}
}
{
\newrgbcolor{curcolor}{0 0 0}
\pscustom[linewidth=0.7466667,linecolor=curcolor]
{
\newpath
\moveto(579.73794054,286.22749707)
\lineto(587.07127413,286.22749707)
}
}
{
\newrgbcolor{curcolor}{0 0 0}
\pscustom[linewidth=0.7466667,linecolor=curcolor]
{
\newpath
\moveto(594.6754344,286.22749707)
\lineto(602.008768,286.22749707)
}
}
{
\newrgbcolor{curcolor}{0 0 0}
\pscustom[linewidth=0.7466667,linecolor=curcolor]
{
\newpath
\moveto(609.34210159,286.22749707)
\lineto(616.80564852,286.22749707)
}
}
{
\newrgbcolor{curcolor}{0 0 0}
\pscustom[linewidth=0.7466667,linecolor=curcolor]
{
\newpath
\moveto(624.27439545,286.22749707)
\lineto(631.60772905,286.22749707)
}
}
{
\newrgbcolor{curcolor}{0 0 0}
\pscustom[linewidth=0.7466667,linecolor=curcolor]
{
\newpath
\moveto(639.20668931,286.22749707)
\lineto(646.54002291,286.22749707)
}
}
{
\newrgbcolor{curcolor}{0 0 0}
\pscustom[linestyle=none,fillstyle=solid,fillcolor=curcolor]
{
\newpath
\moveto(114.54716552,192.74173196)
\lineto(114.54716552,189.88756519)
\lineto(114.25549884,189.88756519)
\curveto(114.01591616,190.77298255)(113.71904015,191.37714924)(113.35966547,191.70006525)
\curveto(112.9950828,192.0333986)(112.53674945,192.20006527)(111.98466543,192.20006527)
\curveto(111.55237341,192.20006527)(111.20341607,192.08027327)(110.94299872,191.84589859)
\curveto(110.67737338,191.62193992)(110.54716538,191.37714924)(110.54716538,191.1167319)
\curveto(110.54716538,190.76777322)(110.64091605,190.47610654)(110.83883205,190.24173187)
\curveto(111.03154006,189.99173186)(111.42216541,189.72610652)(112.00549876,189.45006517)
\lineto(113.31799881,188.80423182)
\curveto(114.55237352,188.18964913)(115.17216554,187.3927731)(115.17216554,186.4083984)
\curveto(115.17216554,185.65839837)(114.88570686,185.04381569)(114.31799884,184.575065)
\curveto(113.74508282,184.11673165)(113.09924947,183.88756498)(112.38049877,183.88756498)
\curveto(111.88049876,183.88756498)(111.2971654,183.98131431)(110.63049871,184.15839832)
\curveto(110.4325827,184.21048166)(110.27633203,184.24173166)(110.15133203,184.24173166)
\curveto(110.01070669,184.24173166)(109.90133202,184.16360672)(109.81799868,184.01256498)
\lineto(109.52633201,184.01256498)
\lineto(109.52633201,186.99173175)
\lineto(109.81799868,186.99173175)
\curveto(109.98466536,186.14277306)(110.30237337,185.50214903)(110.77633205,185.07506502)
\curveto(111.26070673,184.642773)(111.80237342,184.42923166)(112.40133211,184.42923166)
\curveto(112.81799879,184.42923166)(113.15654013,184.54381567)(113.42216548,184.78339834)
\curveto(113.68258282,185.03339835)(113.81799882,185.33027303)(113.81799882,185.67923171)
\curveto(113.81799882,186.09589839)(113.67216549,186.43964907)(113.38049881,186.72089841)
\curveto(113.08883213,187.01256509)(112.49508278,187.37193977)(111.60966541,187.80423178)
\curveto(110.73466538,188.2469398)(110.15654003,188.64798248)(109.88049868,189.01256516)
\curveto(109.61487334,189.35631584)(109.48466534,189.79381585)(109.48466534,190.32506521)
\curveto(109.48466534,191.00214923)(109.71904001,191.57506525)(110.1929987,192.0333986)
\curveto(110.66174938,192.50214928)(111.26591607,192.74173196)(112.00549876,192.74173196)
\curveto(112.33883211,192.74173196)(112.73987345,192.66881596)(113.21383214,192.53339862)
\curveto(113.51591615,192.43443995)(113.71904015,192.38756528)(113.81799882,192.38756528)
\curveto(113.91174949,192.38756528)(113.9898735,192.40839861)(114.0471655,192.45006528)
\curveto(114.0992495,192.49173195)(114.1721655,192.58548262)(114.25549884,192.74173196)
\closepath
\moveto(114.54716552,192.74173196)
}
}
{
\newrgbcolor{curcolor}{0 0 0}
\pscustom[linestyle=none,fillstyle=solid,fillcolor=curcolor]
{
\newpath
\moveto(118.85016453,190.97089856)
\curveto(119.81891523,192.1479826)(120.75120593,192.74173196)(121.6418313,192.74173196)
\curveto(122.10016465,192.74173196)(122.48558199,192.62193995)(122.808498,192.38756528)
\curveto(123.14183135,192.1636066)(123.40224869,191.78860659)(123.6001647,191.26256524)
\curveto(123.73558204,190.89798256)(123.80849804,190.34589854)(123.80849804,189.59589851)
\lineto(123.80849804,186.03339839)
\curveto(123.80849804,185.50214903)(123.85016471,185.14277302)(123.93349804,184.95006502)
\curveto(124.00120605,184.79381568)(124.10537272,184.67923167)(124.24599805,184.59589834)
\curveto(124.39703939,184.512565)(124.67308207,184.47089833)(125.07933142,184.47089833)
\lineto(125.07933142,184.13756499)
\lineto(120.95433127,184.13756499)
\lineto(120.95433127,184.47089833)
\lineto(121.12099794,184.47089833)
\curveto(121.50641529,184.47089833)(121.77724863,184.52298233)(121.93349797,184.637565)
\curveto(122.10016465,184.76256501)(122.20953932,184.93443968)(122.26683132,185.15839836)
\curveto(122.29287265,185.25214903)(122.30849799,185.5438157)(122.30849799,186.03339839)
\lineto(122.30849799,189.45006517)
\curveto(122.30849799,190.2000652)(122.20433132,190.74693989)(121.99599797,191.09589857)
\curveto(121.79808197,191.43964925)(121.47516462,191.61673192)(121.01683127,191.61673192)
\curveto(120.27724858,191.61673192)(119.55849789,191.22089857)(118.85016453,190.42923188)
\lineto(118.85016453,186.03339839)
\curveto(118.85016453,185.46048237)(118.8814152,185.10631569)(118.9543312,184.97089835)
\curveto(119.03766454,184.80423168)(119.14703921,184.67923167)(119.28766455,184.59589834)
\curveto(119.43870588,184.512565)(119.7459979,184.47089833)(120.20433124,184.47089833)
\lineto(120.20433124,184.13756499)
\lineto(116.0793311,184.13756499)
\lineto(116.0793311,184.47089833)
\lineto(116.26683111,184.47089833)
\curveto(116.68349779,184.47089833)(116.96474846,184.575065)(117.1209978,184.78339834)
\curveto(117.27203914,185.00214902)(117.35016448,185.4188157)(117.35016448,186.03339839)
\lineto(117.35016448,189.11673183)
\curveto(117.35016448,190.1271492)(117.31891514,190.74173189)(117.26683114,190.95006523)
\curveto(117.22516447,191.1688159)(117.15224847,191.31464924)(117.0584978,191.38756524)
\curveto(116.97516446,191.47089858)(116.85016446,191.51256525)(116.68349779,191.51256525)
\curveto(116.51683111,191.51256525)(116.31370577,191.47089858)(116.0793311,191.38756524)
\lineto(115.93349776,191.72089859)
\lineto(118.45433118,192.74173196)
\lineto(118.85016453,192.74173196)
\closepath
\moveto(118.85016453,190.97089856)
}
}
{
\newrgbcolor{curcolor}{0 0 0}
\pscustom[linestyle=none,fillstyle=solid,fillcolor=curcolor]
{
\newpath
\moveto(130.47516494,185.34589836)
\curveto(129.61058224,184.67923167)(129.07933156,184.29381566)(128.87099822,184.20006499)
\curveto(128.53766487,184.04902325)(128.18349819,183.97089831)(127.80849818,183.97089831)
\curveto(127.23558216,183.97089831)(126.76683148,184.16360672)(126.39183146,184.55423167)
\curveto(126.01683145,184.95527302)(125.82933144,185.47610637)(125.82933144,186.11673172)
\curveto(125.82933144,186.51777307)(125.91787278,186.87193975)(126.10016479,187.17923176)
\curveto(126.3501648,187.58027311)(126.77724881,187.96048246)(127.3918315,188.32506513)
\curveto(128.01683152,188.70006515)(129.04287289,189.1479825)(130.47516494,189.67923185)
\lineto(130.47516494,189.99173186)
\curveto(130.47516494,190.82506522)(130.33974894,191.39277324)(130.07933159,191.70006525)
\curveto(129.81370625,192.01777327)(129.43349824,192.17923194)(128.93349822,192.17923194)
\curveto(128.54287287,192.17923194)(128.2355822,192.07506527)(128.01683152,191.86673193)
\curveto(127.77724885,191.65839859)(127.66266484,191.41881591)(127.66266484,191.15839857)
\lineto(127.68349818,190.63756522)
\curveto(127.68349818,190.34589854)(127.61058217,190.12193987)(127.47516484,189.97089853)
\curveto(127.3345395,189.83027319)(127.14703949,189.76256519)(126.91266482,189.76256519)
\curveto(126.68870614,189.76256519)(126.50120613,189.83548252)(126.3501648,189.99173186)
\curveto(126.20953946,190.1427732)(126.14183145,190.35631587)(126.14183145,190.63756522)
\curveto(126.14183145,191.16360657)(126.4022488,191.64277325)(126.93349815,192.07506527)
\curveto(127.47516484,192.51777328)(128.23037286,192.74173196)(129.20433156,192.74173196)
\curveto(129.93870626,192.74173196)(130.54287294,192.61673195)(131.01683163,192.36673194)
\curveto(131.37620631,192.16881594)(131.64183165,191.87193993)(131.80849832,191.47089858)
\curveto(131.91787299,191.20527324)(131.97516499,190.66881588)(131.97516499,189.86673186)
\lineto(131.97516499,187.03339842)
\curveto(131.97516499,186.24173173)(131.98557833,185.75214904)(132.01683166,185.57506504)
\curveto(132.05849833,185.39277303)(132.11058233,185.26777303)(132.18349834,185.20006502)
\curveto(132.25120634,185.14277302)(132.32933167,185.11673169)(132.41266501,185.11673169)
\curveto(132.50641568,185.11673169)(132.60016502,185.13756502)(132.68349835,185.17923169)
\curveto(132.80849836,185.26256503)(133.04808237,185.4813157)(133.41266505,185.84589838)
\lineto(133.41266505,185.34589836)
\curveto(132.73037302,184.42923166)(132.07933167,183.97089831)(131.45433164,183.97089831)
\curveto(131.16266497,183.97089831)(130.92308229,184.07506498)(130.74599828,184.28339833)
\curveto(130.57933161,184.49173167)(130.48558228,184.84589834)(130.47516494,185.34589836)
\closepath
\moveto(130.47516494,185.92923172)
\lineto(130.47516494,189.11673183)
\curveto(129.55849824,188.75214915)(128.96474889,188.49693981)(128.70433155,188.34589847)
\curveto(128.21474886,188.06464913)(127.87099818,187.78339845)(127.66266484,187.49173177)
\curveto(127.4543315,187.2000651)(127.35016483,186.88756508)(127.35016483,186.55423174)
\curveto(127.35016483,186.10631572)(127.48037284,185.74173171)(127.74599818,185.45006503)
\curveto(128.00641552,185.15839836)(128.3137062,185.01256502)(128.66266488,185.01256502)
\curveto(129.12099823,185.01256502)(129.72516492,185.31464903)(130.47516494,185.92923172)
\closepath
\moveto(130.47516494,185.92923172)
}
}
{
\newrgbcolor{curcolor}{0 0 0}
\pscustom[linestyle=none,fillstyle=solid,fillcolor=curcolor]
{
\newpath
\moveto(133.42983627,191.65839859)
\lineto(136.0131697,192.70006529)
\lineto(136.34650304,192.70006529)
\lineto(136.34650304,190.74173189)
\curveto(136.77358706,191.47610658)(137.20587774,191.99173193)(137.63816975,192.28339861)
\curveto(138.08087777,192.58548262)(138.53921112,192.74173196)(139.0131698,192.74173196)
\curveto(139.8569205,192.74173196)(140.56004452,192.40839861)(141.11733654,191.74173192)
\curveto(141.79442057,190.95006523)(142.13816991,189.90839852)(142.13816991,188.61673181)
\curveto(142.13816991,187.16881576)(141.72671123,185.97610638)(140.9090032,185.03339835)
\curveto(140.22671118,184.26777339)(139.37254448,183.88756498)(138.34650311,183.88756498)
\curveto(137.88816976,183.88756498)(137.49754441,183.95006498)(137.1798364,184.07506498)
\curveto(136.92983639,184.16881499)(136.64858705,184.35110633)(136.34650304,184.61673167)
\lineto(136.34650304,182.07506491)
\curveto(136.34650304,181.50735689)(136.37775371,181.14798088)(136.45066971,180.99173154)
\curveto(136.51837771,180.8406902)(136.63816972,180.7208982)(136.80483639,180.63756486)
\curveto(136.9819204,180.54381419)(137.30483641,180.49173152)(137.76316976,180.49173152)
\lineto(137.76316976,180.13756484)
\lineto(133.3881696,180.13756484)
\lineto(133.3881696,180.49173152)
\lineto(133.61733628,180.49173152)
\curveto(133.95066962,180.48131419)(134.2319203,180.54381419)(134.47150297,180.67923153)
\curveto(134.58087764,180.75214753)(134.66942031,180.86152354)(134.74233632,181.01256488)
\curveto(134.81004432,181.15319021)(134.84650299,181.52819023)(134.84650299,182.13756492)
\lineto(134.84650299,190.03339853)
\curveto(134.84650299,190.57506521)(134.81525365,190.91360656)(134.76316965,191.0542319)
\curveto(134.72150298,191.20527324)(134.64337765,191.31464924)(134.53400298,191.38756524)
\curveto(134.41942031,191.45527325)(134.27358697,191.49173191)(134.09650296,191.49173191)
\curveto(133.95587762,191.49173191)(133.76837762,191.45006525)(133.53400294,191.36673191)
\closepath
\moveto(136.34650304,190.2000652)
\lineto(136.34650304,187.09589842)
\curveto(136.34650304,186.4136064)(136.37254437,185.97089838)(136.42983638,185.76256504)
\curveto(136.51316971,185.41360637)(136.72150305,185.10110635)(137.0548364,184.82506501)
\curveto(137.38816974,184.55943967)(137.80483643,184.42923166)(138.30483644,184.42923166)
\curveto(138.91421113,184.42923166)(139.40900315,184.66360634)(139.78400316,185.13756502)
\curveto(140.28400318,185.76256504)(140.53400319,186.63756508)(140.53400319,187.76256511)
\curveto(140.53400319,189.03860649)(140.24754451,190.01777319)(139.67983649,190.70006522)
\curveto(139.28921114,191.1688159)(138.8308778,191.40839858)(138.30483644,191.40839858)
\curveto(138.01316977,191.40839858)(137.72150309,191.33548257)(137.42983641,191.20006524)
\curveto(137.20587774,191.08548257)(136.84650306,190.75214922)(136.34650304,190.2000652)
\closepath
\moveto(136.34650304,190.2000652)
}
}
{
\newrgbcolor{curcolor}{0 0 0}
\pscustom[linestyle=none,fillstyle=solid,fillcolor=curcolor]
{
\newpath
\moveto(148.76317015,192.74173196)
\lineto(148.76317015,189.88756519)
\lineto(148.47150347,189.88756519)
\curveto(148.23192079,190.77298255)(147.93504478,191.37714924)(147.57567011,191.70006525)
\curveto(147.21108743,192.0333986)(146.75275408,192.20006527)(146.20067006,192.20006527)
\curveto(145.76837804,192.20006527)(145.4194207,192.08027327)(145.15900335,191.84589859)
\curveto(144.89337801,191.62193992)(144.76317001,191.37714924)(144.76317001,191.1167319)
\curveto(144.76317001,190.76777322)(144.85692068,190.47610654)(145.05483668,190.24173187)
\curveto(145.24754469,189.99173186)(145.63817004,189.72610652)(146.22150339,189.45006517)
\lineto(147.53400344,188.80423182)
\curveto(148.76837815,188.18964913)(149.38817017,187.3927731)(149.38817017,186.4083984)
\curveto(149.38817017,185.65839837)(149.10171149,185.04381569)(148.53400347,184.575065)
\curveto(147.96108745,184.11673165)(147.3152541,183.88756498)(146.5965034,183.88756498)
\curveto(146.09650339,183.88756498)(145.51317003,183.98131431)(144.84650334,184.15839832)
\curveto(144.64858733,184.21048166)(144.49233666,184.24173166)(144.36733666,184.24173166)
\curveto(144.22671132,184.24173166)(144.11733665,184.16360672)(144.03400331,184.01256498)
\lineto(143.74233664,184.01256498)
\lineto(143.74233664,186.99173175)
\lineto(144.03400331,186.99173175)
\curveto(144.20066999,186.14277306)(144.518378,185.50214903)(144.99233668,185.07506502)
\curveto(145.47671136,184.642773)(146.01837805,184.42923166)(146.61733674,184.42923166)
\curveto(147.03400342,184.42923166)(147.37254476,184.54381567)(147.63817011,184.78339834)
\curveto(147.89858745,185.03339835)(148.03400345,185.33027303)(148.03400345,185.67923171)
\curveto(148.03400345,186.09589839)(147.88817012,186.43964907)(147.59650344,186.72089841)
\curveto(147.30483676,187.01256509)(146.71108741,187.37193977)(145.82567004,187.80423178)
\curveto(144.95067001,188.2469398)(144.37254466,188.64798248)(144.09650332,189.01256516)
\curveto(143.83087797,189.35631584)(143.70066997,189.79381585)(143.70066997,190.32506521)
\curveto(143.70066997,191.00214923)(143.93504464,191.57506525)(144.40900333,192.0333986)
\curveto(144.87775401,192.50214928)(145.4819207,192.74173196)(146.22150339,192.74173196)
\curveto(146.55483674,192.74173196)(146.95587808,192.66881596)(147.42983677,192.53339862)
\curveto(147.73192078,192.43443995)(147.93504478,192.38756528)(148.03400345,192.38756528)
\curveto(148.12775412,192.38756528)(148.20587813,192.40839861)(148.26317013,192.45006528)
\curveto(148.31525413,192.49173195)(148.38817013,192.58548262)(148.47150347,192.74173196)
\closepath
\moveto(148.76317015,192.74173196)
}
}
{
\newrgbcolor{curcolor}{0 0 0}
\pscustom[linestyle=none,fillstyle=solid,fillcolor=curcolor]
{
\newpath
\moveto(153.08700249,197.09589878)
\lineto(153.08700249,190.9917319)
\curveto(153.75366918,191.72610659)(154.27971054,192.20006527)(154.67033588,192.40839861)
\curveto(155.07137723,192.62714929)(155.47241991,192.74173196)(155.87866926,192.74173196)
\curveto(156.34741994,192.74173196)(156.75366929,192.60631595)(157.08700264,192.34589861)
\curveto(157.43075331,192.08027327)(157.69116932,191.66881592)(157.857836,191.1167319)
\curveto(157.96721067,190.72610655)(158.02450267,190.01256519)(158.02450267,188.97089849)
\lineto(158.02450267,186.03339839)
\curveto(158.02450267,185.50214903)(158.06616934,185.14277302)(158.14950267,184.95006502)
\curveto(158.20158668,184.79381568)(158.30054401,184.67923167)(158.44116935,184.59589834)
\curveto(158.59221069,184.512565)(158.86304403,184.47089833)(159.25366938,184.47089833)
\lineto(159.25366938,184.13756499)
\lineto(155.14950257,184.13756499)
\lineto(155.14950257,184.47089833)
\lineto(155.35783591,184.47089833)
\curveto(155.74325326,184.47089833)(156.0140866,184.52298233)(156.17033594,184.637565)
\curveto(156.32137728,184.76256501)(156.42554395,184.93443968)(156.48283595,185.15839836)
\curveto(156.49324928,185.25214903)(156.50366928,185.5438157)(156.50366928,186.03339839)
\lineto(156.50366928,188.97089849)
\curveto(156.50366928,189.88756519)(156.45158661,190.48131588)(156.35783594,190.76256522)
\curveto(156.27450261,191.0542319)(156.12866927,191.26777324)(155.92033593,191.40839858)
\curveto(155.71200259,191.55943992)(155.46200258,191.63756525)(155.1703359,191.63756525)
\curveto(154.86304389,191.63756525)(154.55054388,191.55423192)(154.23283587,191.38756524)
\curveto(153.90991986,191.2313159)(153.52971051,190.91360656)(153.08700249,190.42923188)
\lineto(153.08700249,186.03339839)
\curveto(153.08700249,185.45006503)(153.11304383,185.08548235)(153.17033583,184.95006502)
\curveto(153.23804383,184.80943968)(153.35783584,184.68964901)(153.52450251,184.59589834)
\curveto(153.70158652,184.512565)(154.00366919,184.47089833)(154.42033587,184.47089833)
\lineto(154.42033587,184.13756499)
\lineto(150.29533573,184.13756499)
\lineto(150.29533573,184.47089833)
\curveto(150.67033574,184.47089833)(150.96200242,184.52298233)(151.17033576,184.637565)
\curveto(151.29533576,184.68964901)(151.38908643,184.80423168)(151.46200244,184.97089835)
\curveto(151.52971044,185.14798236)(151.56616911,185.50214903)(151.56616911,186.03339839)
\lineto(151.56616911,193.57506532)
\curveto(151.56616911,194.53339869)(151.54533577,195.11673204)(151.5036691,195.32506538)
\curveto(151.46200244,195.54381606)(151.38908643,195.7000654)(151.29533576,195.78339873)
\curveto(151.19637709,195.86673207)(151.07137709,195.90839874)(150.92033575,195.90839874)
\curveto(150.77971041,195.90839874)(150.57137707,195.85631607)(150.29533573,195.7625654)
\lineto(150.17033572,196.07506541)
\lineto(152.67033581,197.09589878)
\closepath
\moveto(153.08700249,197.09589878)
}
}
{
\newrgbcolor{curcolor}{0 0 0}
\pscustom[linestyle=none,fillstyle=solid,fillcolor=curcolor]
{
\newpath
\moveto(164.04533622,192.74173196)
\curveto(165.30575359,192.74173196)(166.32137763,192.25214927)(167.08700299,191.28339857)
\curveto(167.73804435,190.46048254)(168.06616969,189.52298251)(168.06616969,188.47089847)
\curveto(168.06616969,187.72089845)(167.88387769,186.96048242)(167.52450301,186.20006506)
\curveto(167.15992033,185.4344397)(166.67033631,184.85631568)(166.04533629,184.47089833)
\curveto(165.42033626,184.08548165)(164.71721091,183.88756498)(163.94116955,183.88756498)
\curveto(162.6911695,183.88756498)(161.69116947,184.387565)(160.94116944,185.38756503)
\curveto(160.31616942,186.23131573)(160.00366941,187.17923176)(160.00366941,188.22089846)
\curveto(160.00366941,188.99693983)(160.19116941,189.76256519)(160.56616943,190.51256521)
\curveto(160.95158677,191.27298257)(161.45158679,191.83548259)(162.06616948,192.20006527)
\curveto(162.6911695,192.55943995)(163.34742019,192.74173196)(164.04533622,192.74173196)
\closepath
\moveto(163.75366954,192.13756527)
\curveto(163.43075353,192.13756527)(163.10783618,192.0386066)(162.77450284,191.84589859)
\curveto(162.45158683,191.66360659)(162.19637748,191.33027324)(162.00366948,190.84589856)
\curveto(161.80575347,190.35631587)(161.7120028,189.74173185)(161.7120028,188.99173183)
\curveto(161.7120028,187.78339845)(161.94637747,186.73131575)(162.42033616,185.84589838)
\curveto(162.90471084,184.97089835)(163.5453362,184.53339833)(164.33700289,184.53339833)
\curveto(164.92033625,184.53339833)(165.40471093,184.77298234)(165.79533628,185.26256503)
\curveto(166.18075362,185.74693971)(166.37866963,186.58027307)(166.37866963,187.76256511)
\curveto(166.37866963,189.24693983)(166.05575362,190.41360654)(165.42033626,191.26256524)
\curveto(164.98804425,191.84589859)(164.43075356,192.13756527)(163.75366954,192.13756527)
\closepath
\moveto(163.75366954,192.13756527)
}
}
{
\newrgbcolor{curcolor}{0 0 0}
\pscustom[linestyle=none,fillstyle=solid,fillcolor=curcolor]
{
\newpath
\moveto(171.71200315,195.22089871)
\lineto(171.71200315,192.49173195)
\lineto(173.67033656,192.49173195)
\lineto(173.67033656,191.84589859)
\lineto(171.71200315,191.84589859)
\lineto(171.71200315,186.42923173)
\curveto(171.71200315,185.88756505)(171.78492049,185.51777304)(171.94116983,185.32506503)
\curveto(172.1078365,185.14277302)(172.30575384,185.05423169)(172.54533652,185.05423169)
\curveto(172.75366986,185.05423169)(172.94637786,185.11673169)(173.12866987,185.24173169)
\curveto(173.32137788,185.3667317)(173.46721122,185.5542317)(173.56616989,185.80423171)
\lineto(173.92033657,185.80423171)
\curveto(173.71200323,185.20527302)(173.40992055,184.75214901)(173.0245032,184.450065)
\curveto(172.63387785,184.14277332)(172.23283651,183.99173165)(171.81616982,183.99173165)
\curveto(171.53492048,183.99173165)(171.26408714,184.06985658)(171.0036698,184.22089832)
\curveto(170.73804445,184.387565)(170.53492045,184.60631567)(170.39950311,184.88756501)
\curveto(170.2745031,185.17923169)(170.2120031,185.62714904)(170.2120031,186.24173173)
\lineto(170.2120031,191.84589859)
\lineto(168.89950305,191.84589859)
\lineto(168.89950305,192.1583986)
\curveto(169.2328364,192.28339861)(169.57137775,192.50214928)(169.92033642,192.82506529)
\curveto(170.2640871,193.14277331)(170.57658711,193.52298265)(170.85783646,193.97089867)
\curveto(170.9932538,194.18964934)(171.1911698,194.60631602)(171.44116981,195.22089871)
\closepath
\moveto(171.71200315,195.22089871)
}
}
{
\newrgbcolor{curcolor}{0 0 0}
\pscustom[linestyle=none,fillstyle=solid,fillcolor=curcolor]
{
\newpath
\moveto(305.34095483,192.49173195)
\lineto(308.82012162,192.49173195)
\curveto(309.49720564,192.49173195)(309.94512166,192.50214528)(310.153455,192.53339862)
\lineto(310.54928835,192.53339862)
\lineto(310.54928835,186.07506506)
\curveto(310.54928835,185.39277303)(310.62220568,184.95527302)(310.77845502,184.76256501)
\curveto(310.94512169,184.580273)(311.30970571,184.48131567)(311.88262173,184.47089833)
\lineto(311.88262173,184.13756499)
\lineto(307.71595491,184.13756499)
\lineto(307.71595491,184.47089833)
\curveto(308.3097056,184.47089833)(308.69512161,184.60110634)(308.86178829,184.86673168)
\curveto(308.98678829,185.04381569)(309.04928829,185.45006503)(309.04928829,186.07506506)
\lineto(309.04928829,190.49173188)
\curveto(309.04928829,191.0594399)(309.01803896,191.39798258)(308.96595496,191.51256525)
\curveto(308.90866296,191.60631592)(308.80970562,191.67923192)(308.67428828,191.72089859)
\curveto(308.54928828,191.77298259)(308.11178826,191.80423192)(307.36178823,191.80423192)
\lineto(305.34095483,191.80423192)
\lineto(305.34095483,186.36673173)
\curveto(305.34095483,185.71048238)(305.3722055,185.28860636)(305.4451215,185.09589835)
\curveto(305.4972055,184.93964901)(305.61178817,184.81464901)(305.77845484,184.72089834)
\curveto(306.03887219,184.55423167)(306.26803886,184.47089833)(306.46595487,184.47089833)
\lineto(306.86178822,184.47089833)
\lineto(306.86178822,184.13756499)
\lineto(302.29928805,184.13756499)
\lineto(302.29928805,184.47089833)
\lineto(302.6326214,184.47089833)
\curveto(303.04928808,184.47089833)(303.34616276,184.580273)(303.52845476,184.80423168)
\curveto(303.72116277,185.03860635)(303.82012144,185.5594397)(303.82012144,186.36673173)
\lineto(303.82012144,191.80423192)
\lineto(302.21595472,191.80423192)
\lineto(302.21595472,192.49173195)
\lineto(303.82012144,192.49173195)
\curveto(303.82012144,193.60110666)(303.97116278,194.47089869)(304.27845479,195.09589871)
\curveto(304.5961628,195.72089873)(305.05449615,196.20527341)(305.65345484,196.55423209)
\curveto(306.24720553,196.91360677)(306.92949622,197.09589878)(307.69512158,197.09589878)
\curveto(308.22116293,197.09589878)(308.69512161,197.00214944)(309.1117883,196.82506544)
\curveto(309.53887231,196.65839876)(309.85137232,196.43444009)(310.04928833,196.15839875)
\curveto(310.25762167,195.8927734)(310.36178834,195.64277339)(310.36178834,195.40839872)
\curveto(310.36178834,195.20006538)(310.28887234,195.02298271)(310.153455,194.88756537)
\curveto(310.01282966,194.74694003)(309.85137232,194.67923203)(309.67428832,194.67923203)
\curveto(309.46595498,194.67923203)(309.28366297,194.73131603)(309.13262163,194.8458987)
\curveto(308.99199629,194.95527337)(308.74720562,195.25214938)(308.40345494,195.74173206)
\curveto(308.20553893,196.03339874)(307.99199626,196.24173208)(307.75762158,196.36673209)
\curveto(307.59095491,196.45006542)(307.38262157,196.49173209)(307.13262156,196.49173209)
\curveto(306.80970555,196.49173209)(306.49720554,196.36673209)(306.19512153,196.11673208)
\curveto(305.88782951,195.86673207)(305.67949617,195.58027339)(305.5701215,195.26256538)
\curveto(305.41387216,194.83027337)(305.34095483,194.06464934)(305.34095483,192.97089863)
\closepath
\moveto(305.34095483,192.49173195)
}
}
{
\newrgbcolor{curcolor}{0 0 0}
\pscustom[linestyle=none,fillstyle=solid,fillcolor=curcolor]
{
\newpath
\moveto(315.82378398,197.09589878)
\lineto(315.82378398,186.03339839)
\curveto(315.82378398,185.50214903)(315.85503464,185.14798236)(315.92795065,184.97089835)
\curveto(316.01128398,184.80423168)(316.12586799,184.67923167)(316.28211733,184.59589834)
\curveto(316.448784,184.512565)(316.74565868,184.47089833)(317.17795069,184.47089833)
\lineto(317.17795069,184.13756499)
\lineto(313.07378388,184.13756499)
\lineto(313.07378388,184.47089833)
\curveto(313.45920123,184.47089833)(313.72482524,184.502149)(313.86545057,184.575065)
\curveto(314.00086791,184.65839834)(314.10503458,184.78860634)(314.17795058,184.97089835)
\curveto(314.26128392,185.14798236)(314.30295059,185.50214903)(314.30295059,186.03339839)
\lineto(314.30295059,193.61673199)
\curveto(314.30295059,194.54381602)(314.28211726,195.11673204)(314.24045059,195.32506538)
\curveto(314.19878392,195.54381606)(314.12586792,195.7000654)(314.03211725,195.78339873)
\curveto(313.94878391,195.86673207)(313.82899191,195.90839874)(313.67795057,195.90839874)
\curveto(313.52170123,195.90839874)(313.32378389,195.85631607)(313.07378388,195.7625654)
\lineto(312.92795054,196.07506541)
\lineto(315.40711729,197.09589878)
\closepath
\moveto(315.82378398,197.09589878)
}
}
{
\newrgbcolor{curcolor}{0 0 0}
\pscustom[linestyle=none,fillstyle=solid,fillcolor=curcolor]
{
\newpath
\moveto(319.5339485,189.3458985)
\curveto(319.5339485,188.10631579)(319.83603251,187.13756509)(320.45061519,186.42923173)
\curveto(321.04436588,185.72089838)(321.76311524,185.3667317)(322.5964486,185.3667317)
\curveto(323.13811529,185.3667317)(323.60686597,185.51256504)(324.01311532,185.80423171)
\curveto(324.41415667,186.10631572)(324.75269935,186.62714907)(325.03394869,187.36673177)
\lineto(325.30478203,187.17923176)
\curveto(325.17978203,186.3458984)(324.80999002,185.58548237)(324.20061533,184.90839835)
\curveto(323.58603264,184.22610672)(322.81519928,183.88756498)(321.88811525,183.88756498)
\curveto(320.88811521,183.88756498)(320.02353251,184.27298232)(319.30478182,185.05423169)
\curveto(318.59644846,185.83027305)(318.24228178,186.87714908)(318.24228178,188.20006513)
\curveto(318.24228178,189.62714918)(318.6068658,190.74173189)(319.34644849,191.53339858)
\curveto(320.08082318,192.33548261)(320.99748988,192.74173196)(322.09644859,192.74173196)
\curveto(323.03915662,192.74173196)(323.80998998,192.42923195)(324.40894867,191.80423192)
\curveto(325.00269936,191.18964924)(325.30478203,190.37193987)(325.30478203,189.3458985)
\closepath
\moveto(319.5339485,189.86673186)
\lineto(323.40894863,189.86673186)
\curveto(323.3776993,190.40839854)(323.3151993,190.78339856)(323.22144863,190.9917319)
\curveto(323.06519929,191.33548257)(322.83603261,191.60631592)(322.5339486,191.80423192)
\curveto(322.22665659,191.99693993)(321.91415658,192.0958986)(321.59644857,192.0958986)
\curveto(321.09644855,192.0958986)(320.6433232,191.89798259)(320.24228185,191.51256525)
\curveto(319.83603251,191.1219399)(319.6016565,190.57506521)(319.5339485,189.86673186)
\closepath
\moveto(319.5339485,189.86673186)
}
}
{
\newrgbcolor{curcolor}{0 0 0}
\pscustom[linestyle=none,fillstyle=solid,fillcolor=curcolor]
{
\newpath
\moveto(331.82195349,192.74173196)
\lineto(331.82195349,189.88756519)
\lineto(331.53028681,189.88756519)
\curveto(331.29070414,190.77298255)(330.99382813,191.37714924)(330.63445345,191.70006525)
\curveto(330.26987077,192.0333986)(329.81153742,192.20006527)(329.2594534,192.20006527)
\curveto(328.82716138,192.20006527)(328.47820404,192.08027327)(328.21778669,191.84589859)
\curveto(327.95216135,191.62193992)(327.82195335,191.37714924)(327.82195335,191.1167319)
\curveto(327.82195335,190.76777322)(327.91570402,190.47610654)(328.11362002,190.24173187)
\curveto(328.30632803,189.99173186)(328.69695338,189.72610652)(329.28028673,189.45006517)
\lineto(330.59278678,188.80423182)
\curveto(331.82716149,188.18964913)(332.44695351,187.3927731)(332.44695351,186.4083984)
\curveto(332.44695351,185.65839837)(332.16049483,185.04381569)(331.59278681,184.575065)
\curveto(331.01987079,184.11673165)(330.37403744,183.88756498)(329.65528675,183.88756498)
\curveto(329.15528673,183.88756498)(328.57195337,183.98131431)(327.90528668,184.15839832)
\curveto(327.70737068,184.21048166)(327.55112,184.24173166)(327.42612,184.24173166)
\curveto(327.28549466,184.24173166)(327.17611999,184.16360672)(327.09278665,184.01256498)
\lineto(326.80111998,184.01256498)
\lineto(326.80111998,186.99173175)
\lineto(327.09278665,186.99173175)
\curveto(327.25945333,186.14277306)(327.57716134,185.50214903)(328.05112002,185.07506502)
\curveto(328.53549471,184.642773)(329.07716139,184.42923166)(329.67612008,184.42923166)
\curveto(330.09278676,184.42923166)(330.43132811,184.54381567)(330.69695345,184.78339834)
\curveto(330.95737079,185.03339835)(331.0927868,185.33027303)(331.0927868,185.67923171)
\curveto(331.0927868,186.09589839)(330.94695346,186.43964907)(330.65528678,186.72089841)
\curveto(330.3636201,187.01256509)(329.76987075,187.37193977)(328.88445338,187.80423178)
\curveto(328.00945335,188.2469398)(327.431328,188.64798248)(327.15528666,189.01256516)
\curveto(326.88966131,189.35631584)(326.75945331,189.79381585)(326.75945331,190.32506521)
\curveto(326.75945331,191.00214923)(326.99382798,191.57506525)(327.46778667,192.0333986)
\curveto(327.93653735,192.50214928)(328.54070404,192.74173196)(329.28028673,192.74173196)
\curveto(329.61362008,192.74173196)(330.01466142,192.66881596)(330.48862011,192.53339862)
\curveto(330.79070412,192.43443995)(330.99382813,192.38756528)(331.0927868,192.38756528)
\curveto(331.18653747,192.38756528)(331.26466147,192.40839861)(331.32195347,192.45006528)
\curveto(331.37403747,192.49173195)(331.44695348,192.58548262)(331.53028681,192.74173196)
\closepath
\moveto(331.82195349,192.74173196)
}
}
{
\newrgbcolor{curcolor}{0 0 0}
\pscustom[linestyle=none,fillstyle=solid,fillcolor=curcolor]
{
\newpath
\moveto(333.20826539,192.49173195)
\lineto(337.10409886,192.49173195)
\lineto(337.10409886,192.1583986)
\lineto(336.91659885,192.1583986)
\curveto(336.63534951,192.1583986)(336.42701617,192.0958986)(336.29159883,191.9708986)
\curveto(336.16659882,191.85631593)(336.10409882,191.71048259)(336.10409882,191.53339858)
\curveto(336.10409882,191.29381591)(336.19784949,190.97089856)(336.3957655,190.55423188)
\lineto(338.43743224,186.32506506)
\lineto(340.29159897,190.95006523)
\curveto(340.40097364,191.20006524)(340.45826564,191.43964925)(340.45826564,191.67923192)
\curveto(340.45826564,191.78860659)(340.43743231,191.87193993)(340.39576564,191.92923193)
\curveto(340.33847364,191.99693993)(340.2551403,192.05423193)(340.14576563,192.0958986)
\curveto(340.04680696,192.13756527)(339.86451629,192.1583986)(339.60409895,192.1583986)
\lineto(339.60409895,192.49173195)
\lineto(342.33326571,192.49173195)
\lineto(342.33326571,192.1583986)
\curveto(342.10930704,192.12714927)(341.93743236,192.07506527)(341.81243236,191.99173193)
\curveto(341.68743235,191.91881593)(341.55201635,191.78860659)(341.41659901,191.59589858)
\curveto(341.35930701,191.51256525)(341.25514034,191.27298257)(341.104099,190.88756523)
\lineto(337.70826555,182.5542316)
\curveto(337.3749322,181.75214757)(336.93743219,181.14798088)(336.3957655,180.74173153)
\curveto(335.86451615,180.32506485)(335.3593068,180.11673151)(334.87493211,180.11673151)
\curveto(334.51034943,180.11673151)(334.21347342,180.22089818)(333.97909875,180.42923152)
\curveto(333.75514007,180.62714753)(333.6457654,180.86152354)(333.6457654,181.13756488)
\curveto(333.6457654,181.38756489)(333.72909874,181.59069023)(333.89576541,181.74173157)
\curveto(334.06243208,181.89798091)(334.29159876,181.97089824)(334.58326544,181.97089824)
\curveto(334.77597344,181.97089824)(335.04680679,181.90839824)(335.39576546,181.78339824)
\curveto(335.64576547,181.68964757)(335.79680681,181.6375649)(335.85409881,181.6375649)
\curveto(336.03118282,181.6375649)(336.22909883,181.73652357)(336.43743217,181.92923157)
\curveto(336.65618284,182.11152358)(336.88014018,182.47089826)(337.10409886,183.01256495)
\lineto(337.70826555,184.47089833)
\lineto(334.70826544,190.76256522)
\curveto(334.60930677,190.95527323)(334.46347343,191.18964924)(334.27076542,191.47089858)
\curveto(334.11451609,191.67923192)(333.98951608,191.81464926)(333.89576541,191.88756526)
\curveto(333.75514007,191.98131593)(333.5259734,192.07506527)(333.20826539,192.1583986)
\closepath
\moveto(333.20826539,192.49173195)
}
}
{
\newrgbcolor{curcolor}{0 0 0}
\pscustom[linestyle=none,fillstyle=solid,fillcolor=curcolor]
{
\newpath
\moveto(348.41659926,192.74173196)
\lineto(348.41659926,189.88756519)
\lineto(348.12493258,189.88756519)
\curveto(347.88534991,190.77298255)(347.5884739,191.37714924)(347.22909922,191.70006525)
\curveto(346.86451654,192.0333986)(346.40618319,192.20006527)(345.85409917,192.20006527)
\curveto(345.42180715,192.20006527)(345.07284981,192.08027327)(344.81243246,191.84589859)
\curveto(344.54680712,191.62193992)(344.41659912,191.37714924)(344.41659912,191.1167319)
\curveto(344.41659912,190.76777322)(344.51034979,190.47610654)(344.70826579,190.24173187)
\curveto(344.9009738,189.99173186)(345.29159915,189.72610652)(345.8749325,189.45006517)
\lineto(347.18743255,188.80423182)
\curveto(348.42180726,188.18964913)(349.04159928,187.3927731)(349.04159928,186.4083984)
\curveto(349.04159928,185.65839837)(348.7551406,185.04381569)(348.18743258,184.575065)
\curveto(347.61451656,184.11673165)(346.96868321,183.88756498)(346.24993252,183.88756498)
\curveto(345.7499325,183.88756498)(345.16659914,183.98131431)(344.49993245,184.15839832)
\curveto(344.30201645,184.21048166)(344.14576577,184.24173166)(344.02076577,184.24173166)
\curveto(343.88014043,184.24173166)(343.77076576,184.16360672)(343.68743242,184.01256498)
\lineto(343.39576575,184.01256498)
\lineto(343.39576575,186.99173175)
\lineto(343.68743242,186.99173175)
\curveto(343.8540991,186.14277306)(344.17180711,185.50214903)(344.64576579,185.07506502)
\curveto(345.13014048,184.642773)(345.67180716,184.42923166)(346.27076585,184.42923166)
\curveto(346.68743253,184.42923166)(347.02597388,184.54381567)(347.29159922,184.78339834)
\curveto(347.55201656,185.03339835)(347.68743257,185.33027303)(347.68743257,185.67923171)
\curveto(347.68743257,186.09589839)(347.54159923,186.43964907)(347.24993255,186.72089841)
\curveto(346.95826587,187.01256509)(346.36451652,187.37193977)(345.47909915,187.80423178)
\curveto(344.60409912,188.2469398)(344.02597377,188.64798248)(343.74993243,189.01256516)
\curveto(343.48430708,189.35631584)(343.35409908,189.79381585)(343.35409908,190.32506521)
\curveto(343.35409908,191.00214923)(343.58847375,191.57506525)(344.06243244,192.0333986)
\curveto(344.53118312,192.50214928)(345.13534981,192.74173196)(345.8749325,192.74173196)
\curveto(346.20826585,192.74173196)(346.60930719,192.66881596)(347.08326588,192.53339862)
\curveto(347.38534989,192.43443995)(347.5884739,192.38756528)(347.68743257,192.38756528)
\curveto(347.78118324,192.38756528)(347.85930724,192.40839861)(347.91659924,192.45006528)
\curveto(347.96868324,192.49173195)(348.04159925,192.58548262)(348.12493258,192.74173196)
\closepath
\moveto(348.41659926,192.74173196)
}
}
{
\newrgbcolor{curcolor}{0 0 0}
\pscustom[linestyle=none,fillstyle=solid,fillcolor=curcolor]
{
\newpath
\moveto(569.45424544,192.74173196)
\lineto(569.45424544,189.88756519)
\lineto(569.16257876,189.88756519)
\curveto(568.92299608,190.77298255)(568.62612007,191.37714924)(568.26674539,191.70006525)
\curveto(567.90216271,192.0333986)(567.44382937,192.20006527)(566.89174535,192.20006527)
\curveto(566.45945333,192.20006527)(566.11049598,192.08027327)(565.85007864,191.84589859)
\curveto(565.5844533,191.62193992)(565.45424529,191.37714924)(565.45424529,191.1167319)
\curveto(565.45424529,190.76777322)(565.54799596,190.47610654)(565.74591197,190.24173187)
\curveto(565.93861998,189.99173186)(566.32924533,189.72610652)(566.91257868,189.45006517)
\lineto(568.22507873,188.80423182)
\curveto(569.45945344,188.18964913)(570.07924546,187.3927731)(570.07924546,186.4083984)
\curveto(570.07924546,185.65839837)(569.79278678,185.04381569)(569.22507876,184.575065)
\curveto(568.65216274,184.11673165)(568.00632939,183.88756498)(567.28757869,183.88756498)
\curveto(566.78757868,183.88756498)(566.20424532,183.98131431)(565.53757863,184.15839832)
\curveto(565.33966262,184.21048166)(565.18341195,184.24173166)(565.05841195,184.24173166)
\curveto(564.91778661,184.24173166)(564.80841194,184.16360672)(564.7250786,184.01256498)
\lineto(564.43341193,184.01256498)
\lineto(564.43341193,186.99173175)
\lineto(564.7250786,186.99173175)
\curveto(564.89174527,186.14277306)(565.20945329,185.50214903)(565.68341197,185.07506502)
\curveto(566.16778665,184.642773)(566.70945334,184.42923166)(567.30841203,184.42923166)
\curveto(567.72507871,184.42923166)(568.06362005,184.54381567)(568.3292454,184.78339834)
\curveto(568.58966274,185.03339835)(568.72507874,185.33027303)(568.72507874,185.67923171)
\curveto(568.72507874,186.09589839)(568.57924541,186.43964907)(568.28757873,186.72089841)
\curveto(567.99591205,187.01256509)(567.4021627,187.37193977)(566.51674533,187.80423178)
\curveto(565.6417453,188.2469398)(565.06361995,188.64798248)(564.7875786,189.01256516)
\curveto(564.52195326,189.35631584)(564.39174526,189.79381585)(564.39174526,190.32506521)
\curveto(564.39174526,191.00214923)(564.62611993,191.57506525)(565.10007862,192.0333986)
\curveto(565.5688293,192.50214928)(566.17299599,192.74173196)(566.91257868,192.74173196)
\curveto(567.24591203,192.74173196)(567.64695337,192.66881596)(568.12091206,192.53339862)
\curveto(568.42299607,192.43443995)(568.62612007,192.38756528)(568.72507874,192.38756528)
\curveto(568.81882941,192.38756528)(568.89695342,192.40839861)(568.95424542,192.45006528)
\curveto(569.00632942,192.49173195)(569.07924542,192.58548262)(569.16257876,192.74173196)
\closepath
\moveto(569.45424544,192.74173196)
}
}
{
\newrgbcolor{curcolor}{0 0 0}
\pscustom[linestyle=none,fillstyle=solid,fillcolor=curcolor]
{
\newpath
\moveto(570.71555733,191.65839859)
\lineto(573.29889076,192.70006529)
\lineto(573.6322241,192.70006529)
\lineto(573.6322241,190.74173189)
\curveto(574.05930812,191.47610658)(574.4915988,191.99173193)(574.92389081,192.28339861)
\curveto(575.36659883,192.58548262)(575.82493218,192.74173196)(576.29889086,192.74173196)
\curveto(577.14264156,192.74173196)(577.84576558,192.40839861)(578.4030576,191.74173192)
\curveto(579.08014163,190.95006523)(579.42389097,189.90839852)(579.42389097,188.61673181)
\curveto(579.42389097,187.16881576)(579.01243229,185.97610638)(578.19472426,185.03339835)
\curveto(577.51243224,184.26777339)(576.65826554,183.88756498)(575.63222417,183.88756498)
\curveto(575.17389082,183.88756498)(574.78326547,183.95006498)(574.46555746,184.07506498)
\curveto(574.21555745,184.16881499)(573.93430811,184.35110633)(573.6322241,184.61673167)
\lineto(573.6322241,182.07506491)
\curveto(573.6322241,181.50735689)(573.66347477,181.14798088)(573.73639077,180.99173154)
\curveto(573.80409877,180.8406902)(573.92389078,180.7208982)(574.09055745,180.63756486)
\curveto(574.26764146,180.54381419)(574.59055747,180.49173152)(575.04889082,180.49173152)
\lineto(575.04889082,180.13756484)
\lineto(570.67389066,180.13756484)
\lineto(570.67389066,180.49173152)
\lineto(570.90305734,180.49173152)
\curveto(571.23639068,180.48131419)(571.51764136,180.54381419)(571.75722403,180.67923153)
\curveto(571.8665987,180.75214753)(571.95514137,180.86152354)(572.02805738,181.01256488)
\curveto(572.09576538,181.15319021)(572.13222405,181.52819023)(572.13222405,182.13756492)
\lineto(572.13222405,190.03339853)
\curveto(572.13222405,190.57506521)(572.10097471,190.91360656)(572.04889071,191.0542319)
\curveto(572.00722404,191.20527324)(571.92909871,191.31464924)(571.81972404,191.38756524)
\curveto(571.70514137,191.45527325)(571.55930803,191.49173191)(571.38222402,191.49173191)
\curveto(571.24159868,191.49173191)(571.05409868,191.45006525)(570.819724,191.36673191)
\closepath
\moveto(573.6322241,190.2000652)
\lineto(573.6322241,187.09589842)
\curveto(573.6322241,186.4136064)(573.65826543,185.97089838)(573.71555744,185.76256504)
\curveto(573.79889077,185.41360637)(574.00722411,185.10110635)(574.34055746,184.82506501)
\curveto(574.6738908,184.55943967)(575.09055749,184.42923166)(575.5905575,184.42923166)
\curveto(576.19993219,184.42923166)(576.69472421,184.66360634)(577.06972422,185.13756502)
\curveto(577.56972424,185.76256504)(577.81972425,186.63756508)(577.81972425,187.76256511)
\curveto(577.81972425,189.03860649)(577.53326557,190.01777319)(576.96555755,190.70006522)
\curveto(576.5749322,191.1688159)(576.11659885,191.40839858)(575.5905575,191.40839858)
\curveto(575.29889083,191.40839858)(575.00722415,191.33548257)(574.71555747,191.20006524)
\curveto(574.4915988,191.08548257)(574.13222412,190.75214922)(573.6322241,190.2000652)
\closepath
\moveto(573.6322241,190.2000652)
}
}
{
\newrgbcolor{curcolor}{0 0 0}
\pscustom[linestyle=none,fillstyle=solid,fillcolor=curcolor]
{
\newpath
\moveto(585.38222452,185.34589836)
\curveto(584.51764182,184.67923167)(583.98639113,184.29381566)(583.77805779,184.20006499)
\curveto(583.44472445,184.04902325)(583.09055777,183.97089831)(582.71555776,183.97089831)
\curveto(582.14264174,183.97089831)(581.67389105,184.16360672)(581.29889104,184.55423167)
\curveto(580.92389103,184.95527302)(580.73639102,185.47610637)(580.73639102,186.11673172)
\curveto(580.73639102,186.51777307)(580.82493236,186.87193975)(581.00722436,187.17923176)
\curveto(581.25722437,187.58027311)(581.68430839,187.96048246)(582.29889107,188.32506513)
\curveto(582.9238911,188.70006515)(583.94993247,189.1479825)(585.38222452,189.67923185)
\lineto(585.38222452,189.99173186)
\curveto(585.38222452,190.82506522)(585.24680851,191.39277324)(584.98639117,191.70006525)
\curveto(584.72076583,192.01777327)(584.34055781,192.17923194)(583.8405578,192.17923194)
\curveto(583.44993245,192.17923194)(583.14264177,192.07506527)(582.9238911,191.86673193)
\curveto(582.68430842,191.65839859)(582.56972442,191.41881591)(582.56972442,191.15839857)
\lineto(582.59055775,190.63756522)
\curveto(582.59055775,190.34589854)(582.51764175,190.12193987)(582.38222441,189.97089853)
\curveto(582.24159907,189.83027319)(582.05409907,189.76256519)(581.81972439,189.76256519)
\curveto(581.59576572,189.76256519)(581.40826571,189.83548252)(581.25722437,189.99173186)
\curveto(581.11659903,190.1427732)(581.04889103,190.35631587)(581.04889103,190.63756522)
\curveto(581.04889103,191.16360657)(581.30930837,191.64277325)(581.84055772,192.07506527)
\curveto(582.38222441,192.51777328)(583.13743244,192.74173196)(584.11139114,192.74173196)
\curveto(584.84576583,192.74173196)(585.44993252,192.61673195)(585.9238912,192.36673194)
\curveto(586.28326588,192.16881594)(586.54889122,191.87193993)(586.7155579,191.47089858)
\curveto(586.82493257,191.20527324)(586.88222457,190.66881588)(586.88222457,189.86673186)
\lineto(586.88222457,187.03339842)
\curveto(586.88222457,186.24173173)(586.8926379,185.75214904)(586.92389124,185.57506504)
\curveto(586.96555791,185.39277303)(587.01764191,185.26777303)(587.09055791,185.20006502)
\curveto(587.15826591,185.14277302)(587.23639125,185.11673169)(587.31972459,185.11673169)
\curveto(587.41347526,185.11673169)(587.50722459,185.13756502)(587.59055793,185.17923169)
\curveto(587.71555793,185.26256503)(587.95514194,185.4813157)(588.31972462,185.84589838)
\lineto(588.31972462,185.34589836)
\curveto(587.6374326,184.42923166)(586.98639124,183.97089831)(586.36139122,183.97089831)
\curveto(586.06972454,183.97089831)(585.83014187,184.07506498)(585.65305786,184.28339833)
\curveto(585.48639119,184.49173167)(585.39264185,184.84589834)(585.38222452,185.34589836)
\closepath
\moveto(585.38222452,185.92923172)
\lineto(585.38222452,189.11673183)
\curveto(584.46555782,188.75214915)(583.87180846,188.49693981)(583.61139112,188.34589847)
\curveto(583.12180844,188.06464913)(582.77805776,187.78339845)(582.56972442,187.49173177)
\curveto(582.36139108,187.2000651)(582.25722441,186.88756508)(582.25722441,186.55423174)
\curveto(582.25722441,186.10631572)(582.38743241,185.74173171)(582.65305775,185.45006503)
\curveto(582.9134751,185.15839836)(583.22076577,185.01256502)(583.56972445,185.01256502)
\curveto(584.0280578,185.01256502)(584.63222449,185.31464903)(585.38222452,185.92923172)
\closepath
\moveto(585.38222452,185.92923172)
}
}
{
\newrgbcolor{curcolor}{0 0 0}
\pscustom[linestyle=none,fillstyle=solid,fillcolor=curcolor]
{
\newpath
\moveto(596.02439612,187.30423177)
\curveto(595.80043744,186.20527306)(595.36293743,185.3563157)(594.71189607,184.76256501)
\curveto(594.05564671,184.17923165)(593.32648002,183.88756498)(592.52439599,183.88756498)
\curveto(591.57647996,183.88756498)(590.7535626,184.28339833)(590.04522924,185.07506502)
\curveto(589.34731321,185.86673171)(589.00356254,186.93964909)(589.00356254,188.3042318)
\curveto(589.00356254,189.60631585)(589.38897988,190.66881588)(590.17022924,191.49173191)
\curveto(590.9462706,192.32506528)(591.88377064,192.74173196)(592.98272934,192.74173196)
\curveto(593.80043737,192.74173196)(594.4723134,192.51777328)(595.00356275,192.07506527)
\curveto(595.5296041,191.64277325)(595.79522944,191.18964924)(595.79522944,190.72089855)
\curveto(595.79522944,190.49693988)(595.71710411,190.30943987)(595.56606277,190.15839853)
\curveto(595.42543743,190.01777319)(595.21710409,189.95006519)(594.94106275,189.95006519)
\curveto(594.59210407,189.95006519)(594.32127072,190.0646492)(594.12856272,190.30423187)
\curveto(594.02960405,190.42923188)(593.96189604,190.66881588)(593.92022938,191.03339856)
\curveto(593.88898004,191.39277324)(593.77439604,191.66360659)(593.5660627,191.84589859)
\curveto(593.35772936,192.0229826)(593.05564668,192.11673194)(592.67022933,192.11673194)
\curveto(592.07127064,192.11673194)(591.58689596,191.89277326)(591.21189595,191.45006525)
\curveto(590.72231326,190.85110656)(590.48272925,190.0646492)(590.48272925,189.0958985)
\curveto(590.48272925,188.09589846)(590.72231326,187.21048243)(591.21189595,186.45006507)
\curveto(591.69627063,185.68443971)(592.35772932,185.30423169)(593.19106268,185.30423169)
\curveto(593.78481337,185.30423169)(594.32127072,185.50214903)(594.79522941,185.90839838)
\curveto(595.12856275,186.18443973)(595.4514801,186.68964908)(595.77439611,187.42923177)
\closepath
\moveto(596.02439612,187.30423177)
}
}
{
\newrgbcolor{curcolor}{0 0 0}
\pscustom[linestyle=none,fillstyle=solid,fillcolor=curcolor]
{
\newpath
\moveto(598.62490077,189.3458985)
\curveto(598.62490077,188.10631579)(598.92698478,187.13756509)(599.54156747,186.42923173)
\curveto(600.13531815,185.72089838)(600.85406751,185.3667317)(601.68740088,185.3667317)
\curveto(602.22906756,185.3667317)(602.69781824,185.51256504)(603.10406759,185.80423171)
\curveto(603.50510894,186.10631572)(603.84365162,186.62714907)(604.12490096,187.36673177)
\lineto(604.3957343,187.17923176)
\curveto(604.2707343,186.3458984)(603.90094229,185.58548237)(603.2915676,184.90839835)
\curveto(602.67698491,184.22610672)(601.90615155,183.88756498)(600.97906752,183.88756498)
\curveto(599.97906748,183.88756498)(599.11448478,184.27298232)(598.39573409,185.05423169)
\curveto(597.68740073,185.83027305)(597.33323405,186.87714908)(597.33323405,188.20006513)
\curveto(597.33323405,189.62714918)(597.69781807,190.74173189)(598.43740076,191.53339858)
\curveto(599.17177545,192.33548261)(600.08844215,192.74173196)(601.18740086,192.74173196)
\curveto(602.13010889,192.74173196)(602.90094225,192.42923195)(603.49990094,191.80423192)
\curveto(604.09365163,191.18964924)(604.3957343,190.37193987)(604.3957343,189.3458985)
\closepath
\moveto(598.62490077,189.86673186)
\lineto(602.4999009,189.86673186)
\curveto(602.46865157,190.40839854)(602.40615157,190.78339856)(602.3124009,190.9917319)
\curveto(602.15615156,191.33548257)(601.92698488,191.60631592)(601.62490087,191.80423192)
\curveto(601.31760886,191.99693993)(601.00510885,192.0958986)(600.68740084,192.0958986)
\curveto(600.18740082,192.0958986)(599.73427547,191.89798259)(599.33323413,191.51256525)
\curveto(598.92698478,191.1219399)(598.69260877,190.57506521)(598.62490077,189.86673186)
\closepath
\moveto(598.62490077,189.86673186)
}
}
{
\newrgbcolor{curcolor}{0 0 0}
\pscustom[linestyle=none,fillstyle=solid,fillcolor=curcolor]
{
\newpath
\moveto(612.66290582,190.95006523)
\curveto(613.27228051,191.55943992)(613.63165652,191.9136066)(613.74623919,192.01256527)
\curveto(614.00665654,192.23131594)(614.29832321,192.40839861)(614.62123922,192.53339862)
\curveto(614.93894724,192.66881596)(615.25144725,192.74173196)(615.55873926,192.74173196)
\curveto(616.08478061,192.74173196)(616.53790596,192.58548262)(616.91290597,192.28339861)
\curveto(617.28790599,191.9761066)(617.53790599,191.53339858)(617.662906,190.95006523)
\curveto(618.28790602,191.68443992)(618.81394737,192.1636066)(619.24623939,192.38756528)
\curveto(619.6733234,192.62193995)(620.12123942,192.74173196)(620.57957277,192.74173196)
\curveto(621.02228078,192.74173196)(621.41811413,192.62193995)(621.76707281,192.38756528)
\curveto(622.11082349,192.1636066)(622.38165683,191.79381592)(622.57957284,191.28339857)
\curveto(622.71499018,190.91881589)(622.78790618,190.36673187)(622.78790618,189.61673185)
\lineto(622.78790618,186.03339839)
\curveto(622.78790618,185.50214903)(622.81915685,185.14277302)(622.89207285,184.95006502)
\curveto(622.95978085,184.80943968)(623.06915686,184.68964901)(623.2254062,184.59589834)
\curveto(623.39207287,184.512565)(623.66290621,184.47089833)(624.03790622,184.47089833)
\lineto(624.03790622,184.13756499)
\lineto(619.91290608,184.13756499)
\lineto(619.91290608,184.47089833)
\lineto(620.10040609,184.47089833)
\curveto(620.44415676,184.47089833)(620.72540611,184.53860633)(620.93373945,184.67923167)
\curveto(621.06915679,184.77298234)(621.16811412,184.92923168)(621.22540613,185.13756502)
\curveto(621.25144746,185.24693969)(621.26707279,185.5438157)(621.26707279,186.03339839)
\lineto(621.26707279,189.61673185)
\curveto(621.26707279,190.29381587)(621.18373946,190.77298255)(621.01707278,191.0542319)
\curveto(620.77749011,191.43964925)(620.4024901,191.63756525)(619.89207274,191.63756525)
\curveto(619.5587394,191.63756525)(619.23061405,191.55423192)(618.91290604,191.38756524)
\curveto(618.58999003,191.2313159)(618.19415668,190.93443989)(617.725406,190.49173188)
\lineto(617.70457267,190.40839854)
\lineto(617.725406,190.01256519)
\lineto(617.725406,186.03339839)
\curveto(617.725406,185.45006503)(617.75144734,185.08548235)(617.80873934,184.95006502)
\curveto(617.87644734,184.80943968)(618.00144734,184.68964901)(618.18373935,184.59589834)
\curveto(618.36082336,184.512565)(618.66290603,184.47089833)(619.07957272,184.47089833)
\lineto(619.07957272,184.13756499)
\lineto(614.87123923,184.13756499)
\lineto(614.87123923,184.47089833)
\curveto(615.32957258,184.47089833)(615.64207259,184.52298233)(615.80873927,184.637565)
\curveto(615.98582327,184.74693967)(616.11082328,184.90839835)(616.18373928,185.11673169)
\curveto(616.20978061,185.22610636)(616.22540595,185.53339837)(616.22540595,186.03339839)
\lineto(616.22540595,189.61673185)
\curveto(616.22540595,190.29381587)(616.12644728,190.78339856)(615.93373927,191.07506523)
\curveto(615.65248993,191.46048258)(615.27748991,191.65839859)(614.80873923,191.65839859)
\curveto(614.47540589,191.65839859)(614.14728054,191.57506525)(613.82957253,191.40839858)
\curveto(613.32957251,191.12714923)(612.93894716,190.82506522)(612.66290582,190.49173188)
\lineto(612.66290582,186.03339839)
\curveto(612.66290582,185.47610637)(612.69415649,185.11673169)(612.76707249,184.95006502)
\curveto(612.85040583,184.79381568)(612.9597805,184.67923167)(613.10040584,184.59589834)
\curveto(613.25144718,184.512565)(613.55873919,184.47089833)(614.01707254,184.47089833)
\lineto(614.01707254,184.13756499)
\lineto(609.89207239,184.13756499)
\lineto(609.89207239,184.47089833)
\curveto(610.27748974,184.47089833)(610.54311375,184.512565)(610.68373908,184.59589834)
\curveto(610.83478042,184.67923167)(610.95457243,184.80423168)(611.03790576,184.97089835)
\curveto(611.1212391,185.14798236)(611.16290577,185.50214903)(611.16290577,186.03339839)
\lineto(611.16290577,189.2208985)
\curveto(611.16290577,190.1375652)(611.13165643,190.72610655)(611.07957243,190.9917319)
\curveto(611.03790576,191.1844399)(610.96498976,191.31464924)(610.87123909,191.38756524)
\curveto(610.78790576,191.47089858)(610.66290575,191.51256525)(610.49623908,191.51256525)
\curveto(610.32957241,191.51256525)(610.12644707,191.47089858)(609.89207239,191.38756524)
\lineto(609.74623905,191.72089859)
\lineto(612.26707247,192.74173196)
\lineto(612.66290582,192.74173196)
\closepath
\moveto(612.66290582,190.95006523)
}
}
{
\newrgbcolor{curcolor}{0 0 0}
\pscustom[linestyle=none,fillstyle=solid,fillcolor=curcolor]
{
\newpath
\moveto(629.43553011,185.34589836)
\curveto(628.57094742,184.67923167)(628.03969673,184.29381566)(627.83136339,184.20006499)
\curveto(627.49803005,184.04902325)(627.14386337,183.97089831)(626.76886335,183.97089831)
\curveto(626.19594733,183.97089831)(625.72719665,184.16360672)(625.35219664,184.55423167)
\curveto(624.97719662,184.95527302)(624.78969662,185.47610637)(624.78969662,186.11673172)
\curveto(624.78969662,186.51777307)(624.87823795,186.87193975)(625.06052996,187.17923176)
\curveto(625.31052997,187.58027311)(625.73761398,187.96048246)(626.35219667,188.32506513)
\curveto(626.97719669,188.70006515)(628.00323806,189.1479825)(629.43553011,189.67923185)
\lineto(629.43553011,189.99173186)
\curveto(629.43553011,190.82506522)(629.30011411,191.39277324)(629.03969677,191.70006525)
\curveto(628.77407142,192.01777327)(628.39386341,192.17923194)(627.89386339,192.17923194)
\curveto(627.50323805,192.17923194)(627.19594737,192.07506527)(626.97719669,191.86673193)
\curveto(626.73761402,191.65839859)(626.62303001,191.41881591)(626.62303001,191.15839857)
\lineto(626.64386335,190.63756522)
\curveto(626.64386335,190.34589854)(626.57094735,190.12193987)(626.43553001,189.97089853)
\curveto(626.29490467,189.83027319)(626.10740466,189.76256519)(625.87302999,189.76256519)
\curveto(625.64907131,189.76256519)(625.46157131,189.83548252)(625.31052997,189.99173186)
\curveto(625.16990463,190.1427732)(625.10219663,190.35631587)(625.10219663,190.63756522)
\curveto(625.10219663,191.16360657)(625.36261397,191.64277325)(625.89386332,192.07506527)
\curveto(626.43553001,192.51777328)(627.19073803,192.74173196)(628.16469674,192.74173196)
\curveto(628.89907143,192.74173196)(629.50323812,192.61673195)(629.9771968,192.36673194)
\curveto(630.33657148,192.16881594)(630.60219682,191.87193993)(630.76886349,191.47089858)
\curveto(630.87823816,191.20527324)(630.93553017,190.66881588)(630.93553017,189.86673186)
\lineto(630.93553017,187.03339842)
\curveto(630.93553017,186.24173173)(630.9459435,185.75214904)(630.97719684,185.57506504)
\curveto(631.0188635,185.39277303)(631.07094751,185.26777303)(631.14386351,185.20006502)
\curveto(631.21157151,185.14277302)(631.28969685,185.11673169)(631.37303018,185.11673169)
\curveto(631.46678085,185.11673169)(631.56053019,185.13756502)(631.64386353,185.17923169)
\curveto(631.76886353,185.26256503)(632.00844754,185.4813157)(632.37303022,185.84589838)
\lineto(632.37303022,185.34589836)
\curveto(631.69073819,184.42923166)(631.03969684,183.97089831)(630.41469682,183.97089831)
\curveto(630.12303014,183.97089831)(629.88344746,184.07506498)(629.70636346,184.28339833)
\curveto(629.53969678,184.49173167)(629.44594745,184.84589834)(629.43553011,185.34589836)
\closepath
\moveto(629.43553011,185.92923172)
\lineto(629.43553011,189.11673183)
\curveto(628.51886341,188.75214915)(627.92511406,188.49693981)(627.66469672,188.34589847)
\curveto(627.17511403,188.06464913)(626.83136335,187.78339845)(626.62303001,187.49173177)
\curveto(626.41469667,187.2000651)(626.31053,186.88756508)(626.31053,186.55423174)
\curveto(626.31053,186.10631572)(626.44073801,185.74173171)(626.70636335,185.45006503)
\curveto(626.96678069,185.15839836)(627.27407137,185.01256502)(627.62303005,185.01256502)
\curveto(628.0813634,185.01256502)(628.68553009,185.31464903)(629.43553011,185.92923172)
\closepath
\moveto(629.43553011,185.92923172)
}
}
{
\newrgbcolor{curcolor}{0 0 0}
\pscustom[linestyle=none,fillstyle=solid,fillcolor=curcolor]
{
\newpath
\moveto(632.39020144,191.65839859)
\lineto(634.97353487,192.70006529)
\lineto(635.30686821,192.70006529)
\lineto(635.30686821,190.74173189)
\curveto(635.73395223,191.47610658)(636.16624291,191.99173193)(636.59853492,192.28339861)
\curveto(637.04124294,192.58548262)(637.49957629,192.74173196)(637.97353497,192.74173196)
\curveto(638.81728567,192.74173196)(639.5204097,192.40839861)(640.07770171,191.74173192)
\curveto(640.75478574,190.95006523)(641.09853508,189.90839852)(641.09853508,188.61673181)
\curveto(641.09853508,187.16881576)(640.6870764,185.97610638)(639.86936837,185.03339835)
\curveto(639.18707635,184.26777339)(638.33290965,183.88756498)(637.30686828,183.88756498)
\curveto(636.84853493,183.88756498)(636.45790959,183.95006498)(636.14020158,184.07506498)
\curveto(635.89020157,184.16881499)(635.60895222,184.35110633)(635.30686821,184.61673167)
\lineto(635.30686821,182.07506491)
\curveto(635.30686821,181.50735689)(635.33811888,181.14798088)(635.41103488,180.99173154)
\curveto(635.47874289,180.8406902)(635.59853489,180.7208982)(635.76520156,180.63756486)
\curveto(635.94228557,180.54381419)(636.26520158,180.49173152)(636.72353493,180.49173152)
\lineto(636.72353493,180.13756484)
\lineto(632.34853477,180.13756484)
\lineto(632.34853477,180.49173152)
\lineto(632.57770145,180.49173152)
\curveto(632.91103479,180.48131419)(633.19228547,180.54381419)(633.43186815,180.67923153)
\curveto(633.54124282,180.75214753)(633.62978549,180.86152354)(633.70270149,181.01256488)
\curveto(633.77040949,181.15319021)(633.80686816,181.52819023)(633.80686816,182.13756492)
\lineto(633.80686816,190.03339853)
\curveto(633.80686816,190.57506521)(633.77561882,190.91360656)(633.72353482,191.0542319)
\curveto(633.68186815,191.20527324)(633.60374282,191.31464924)(633.49436815,191.38756524)
\curveto(633.37978548,191.45527325)(633.23395214,191.49173191)(633.05686813,191.49173191)
\curveto(632.91624279,191.49173191)(632.72874279,191.45006525)(632.49436811,191.36673191)
\closepath
\moveto(635.30686821,190.2000652)
\lineto(635.30686821,187.09589842)
\curveto(635.30686821,186.4136064)(635.33290955,185.97089838)(635.39020155,185.76256504)
\curveto(635.47353489,185.41360637)(635.68186823,185.10110635)(636.01520157,184.82506501)
\curveto(636.34853492,184.55943967)(636.7652016,184.42923166)(637.26520162,184.42923166)
\curveto(637.8745763,184.42923166)(638.36936832,184.66360634)(638.74436833,185.13756502)
\curveto(639.24436835,185.76256504)(639.49436836,186.63756508)(639.49436836,187.76256511)
\curveto(639.49436836,189.03860649)(639.20790968,190.01777319)(638.64020166,190.70006522)
\curveto(638.24957632,191.1688159)(637.79124297,191.40839858)(637.26520162,191.40839858)
\curveto(636.97353494,191.40839858)(636.68186826,191.33548257)(636.39020158,191.20006524)
\curveto(636.16624291,191.08548257)(635.80686823,190.75214922)(635.30686821,190.2000652)
\closepath
\moveto(635.30686821,190.2000652)
}
}
{
\newrgbcolor{curcolor}{0 0 0}
\pscustom[linestyle=none,fillstyle=solid,fillcolor=curcolor]
{
\newpath
\moveto(137.34450926,63.15534228)
\curveto(136.93825991,62.72305026)(136.54242657,62.41575958)(136.15700922,62.23867558)
\curveto(135.76638387,62.06159157)(135.34450919,61.96784223)(134.88617584,61.96784223)
\curveto(133.97992647,61.96784223)(133.18825978,62.34805065)(132.51117576,63.11367561)
\curveto(131.82888373,63.87409297)(131.49034239,64.85325967)(131.49034239,66.05117571)
\curveto(131.49034239,67.24388375)(131.8653424,68.33242646)(132.61534243,69.32200916)
\curveto(133.36534245,70.3220092)(134.33409315,70.82200921)(135.5320092,70.82200921)
\curveto(136.26638389,70.82200921)(136.87055058,70.58242654)(137.34450926,70.11367586)
\lineto(137.34450926,71.65534258)
\curveto(137.34450926,72.61367594)(137.32367593,73.1970093)(137.28200926,73.40534264)
\curveto(137.24034259,73.62409331)(137.16742659,73.78034265)(137.07367592,73.86367599)
\curveto(136.97471725,73.94700932)(136.84971724,73.98867599)(136.6986759,73.98867599)
\curveto(136.54242657,73.98867599)(136.33409322,73.93659332)(136.07367588,73.84284265)
\lineto(135.94867588,74.15534267)
\lineto(138.44867597,75.17617603)
\lineto(138.84450931,75.17617603)
\lineto(138.84450931,65.53034236)
\curveto(138.84450931,64.54075966)(138.86534265,63.93659297)(138.90700932,63.7178423)
\curveto(138.95909332,63.50950895)(139.03721732,63.36367562)(139.13617599,63.28034228)
\curveto(139.22992666,63.19700894)(139.34971733,63.15534228)(139.49034267,63.15534228)
\curveto(139.65700934,63.15534228)(139.87055068,63.20742628)(140.13617603,63.32200895)
\lineto(140.2403427,62.9886756)
\lineto(137.76117594,61.96784223)
\lineto(137.34450926,61.96784223)
\closepath
\moveto(137.34450926,63.80117563)
\lineto(137.34450926,68.09284245)
\curveto(137.30284259,68.50950913)(137.18825992,68.88450915)(137.01117592,69.21784249)
\curveto(136.82888391,69.56159317)(136.58409323,69.82200918)(136.28200922,69.98867585)
\curveto(135.99034255,70.16575986)(135.69867587,70.25950919)(135.40700919,70.25950919)
\curveto(134.87575984,70.25950919)(134.40700916,70.01471719)(133.99034248,69.5303425)
\curveto(133.43305046,68.90534248)(133.15700911,67.97825978)(133.15700911,66.75950907)
\curveto(133.15700911,65.53555036)(133.41742645,64.59805033)(133.94867581,63.94700897)
\curveto(134.49034249,63.29075961)(135.08409318,62.96784227)(135.74034254,62.96784227)
\curveto(136.29242656,62.96784227)(136.82888391,63.24388361)(137.34450926,63.80117563)
\closepath
\moveto(137.34450926,63.80117563)
}
}
{
\newrgbcolor{curcolor}{0 0 0}
\pscustom[linestyle=none,fillstyle=solid,fillcolor=curcolor]
{
\newpath
\moveto(142.90700946,75.17617603)
\curveto(143.15700947,75.17617603)(143.37055081,75.0824267)(143.55284281,74.90534269)
\curveto(143.72992682,74.72305069)(143.82367616,74.50950934)(143.82367616,74.25950934)
\curveto(143.82367616,74.00950933)(143.72992682,73.78555065)(143.55284281,73.59284265)
\curveto(143.37055081,73.41055064)(143.15700947,73.3220093)(142.90700946,73.3220093)
\curveto(142.65700945,73.3220093)(142.43305077,73.41055064)(142.24034277,73.59284265)
\curveto(142.05805076,73.78555065)(141.96950942,74.00950933)(141.96950942,74.25950934)
\curveto(141.96950942,74.50950934)(142.05805076,74.72305069)(142.24034277,74.90534269)
\curveto(142.41742677,75.0824267)(142.64138412,75.17617603)(142.90700946,75.17617603)
\closepath
\moveto(143.65700948,70.82200921)
\lineto(143.65700948,64.11367564)
\curveto(143.65700948,63.58242629)(143.68826015,63.22825961)(143.76117615,63.05117561)
\curveto(143.84450949,62.88450893)(143.9590935,62.75950893)(144.11534283,62.67617559)
\curveto(144.26638417,62.59284226)(144.53721752,62.55117559)(144.92784286,62.55117559)
\lineto(144.92784286,62.21784224)
\lineto(140.86534272,62.21784224)
\lineto(140.86534272,62.55117559)
\curveto(141.2820094,62.55117559)(141.55805074,62.58242626)(141.69867608,62.65534226)
\curveto(141.83409342,62.73867559)(141.94867609,62.8688836)(142.03200943,63.05117561)
\curveto(142.11534276,63.22825961)(142.15700943,63.58242629)(142.15700943,64.11367564)
\lineto(142.15700943,67.32200909)
\curveto(142.15700943,68.22305046)(142.1257601,68.80638381)(142.07367609,69.07200915)
\curveto(142.03200943,69.26471716)(141.95909342,69.3949265)(141.86534275,69.4678425)
\curveto(141.78200942,69.55117584)(141.65700941,69.5928425)(141.49034274,69.5928425)
\curveto(141.32367607,69.5928425)(141.11534273,69.55117584)(140.86534272,69.4678425)
\lineto(140.74034271,69.80117584)
\lineto(143.26117614,70.82200921)
\closepath
\moveto(143.65700948,70.82200921)
}
}
{
\newrgbcolor{curcolor}{0 0 0}
\pscustom[linestyle=none,fillstyle=solid,fillcolor=curcolor]
{
\newpath
\moveto(148.40884071,70.82200921)
\lineto(148.40884071,68.92617581)
\curveto(149.11717407,70.18659319)(149.83592476,70.82200921)(150.57550745,70.82200921)
\curveto(150.9088408,70.82200921)(151.17967414,70.71784254)(151.38800748,70.5095092)
\curveto(151.60675815,70.30117586)(151.72134082,70.06159319)(151.72134082,69.80117584)
\curveto(151.72134082,69.56159317)(151.63800749,69.36367583)(151.47134082,69.19700916)
\curveto(151.31509148,69.03034248)(151.13800747,68.94700915)(150.92967413,68.94700915)
\curveto(150.70571546,68.94700915)(150.46092478,69.05117582)(150.20050744,69.25950916)
\curveto(149.93488209,69.47825983)(149.74217409,69.5928425)(149.61717408,69.5928425)
\curveto(149.50259141,69.5928425)(149.38800741,69.5303425)(149.2630074,69.4053425)
\curveto(148.98175806,69.15534249)(148.70050738,68.74388381)(148.40884071,68.17617579)
\lineto(148.40884071,64.17617565)
\curveto(148.40884071,63.70221696)(148.47134071,63.34805028)(148.59634071,63.11367561)
\curveto(148.66404872,62.95742627)(148.79425805,62.82721693)(148.99217406,62.71784226)
\curveto(149.2005074,62.60325959)(149.49217408,62.55117559)(149.86717409,62.55117559)
\lineto(149.86717409,62.21784224)
\lineto(145.59634061,62.21784224)
\lineto(145.59634061,62.55117559)
\curveto(146.02342462,62.55117559)(146.34634063,62.61367559)(146.55467398,62.73867559)
\curveto(146.70571531,62.83242626)(146.80988198,62.9938836)(146.86717399,63.21784228)
\curveto(146.89321532,63.31159295)(146.90884065,63.60325962)(146.90884065,64.09284231)
\lineto(146.90884065,67.32200909)
\curveto(146.90884065,68.29075979)(146.88800732,68.86888381)(146.84634065,69.05117582)
\curveto(146.80467398,69.24388382)(146.72654865,69.38450916)(146.61717398,69.4678425)
\curveto(146.51821531,69.55117584)(146.3932153,69.5928425)(146.24217396,69.5928425)
\curveto(146.04425796,69.5928425)(145.83071528,69.55117584)(145.59634061,69.4678425)
\lineto(145.51300727,69.80117584)
\lineto(148.03384069,70.82200921)
\closepath
\moveto(148.40884071,70.82200921)
}
}
{
\newrgbcolor{curcolor}{0 0 0}
\pscustom[linestyle=none,fillstyle=solid,fillcolor=curcolor]
{
\newpath
\moveto(224.82060692,70.5720092)
\lineto(228.29977371,70.5720092)
\curveto(228.97685774,70.5720092)(229.42477375,70.58242254)(229.63310709,70.61367587)
\lineto(230.02894044,70.61367587)
\lineto(230.02894044,64.15534231)
\curveto(230.02894044,63.47305029)(230.10185778,63.03555027)(230.25810712,62.84284226)
\curveto(230.42477379,62.66055026)(230.7893578,62.56159292)(231.36227382,62.55117559)
\lineto(231.36227382,62.21784224)
\lineto(227.19560701,62.21784224)
\lineto(227.19560701,62.55117559)
\curveto(227.7893577,62.55117559)(228.17477371,62.68138359)(228.34144038,62.94700893)
\curveto(228.46644039,63.12409294)(228.52894039,63.53034229)(228.52894039,64.15534231)
\lineto(228.52894039,68.57200913)
\curveto(228.52894039,69.13971715)(228.49769105,69.47825983)(228.44560705,69.5928425)
\curveto(228.38831505,69.68659317)(228.28935771,69.75950918)(228.15394038,69.80117584)
\curveto(228.02894037,69.85325985)(227.59144036,69.88450918)(226.84144033,69.88450918)
\lineto(224.82060692,69.88450918)
\lineto(224.82060692,64.44700899)
\curveto(224.82060692,63.79075963)(224.85185759,63.36888362)(224.92477359,63.17617561)
\curveto(224.9768576,63.01992627)(225.09144027,62.89492627)(225.25810694,62.8011756)
\curveto(225.51852428,62.63450892)(225.74769096,62.55117559)(225.94560696,62.55117559)
\lineto(226.34144031,62.55117559)
\lineto(226.34144031,62.21784224)
\lineto(221.77894015,62.21784224)
\lineto(221.77894015,62.55117559)
\lineto(222.11227349,62.55117559)
\curveto(222.52894018,62.55117559)(222.82581485,62.66055026)(223.00810686,62.88450893)
\curveto(223.20081487,63.11888361)(223.29977354,63.63971696)(223.29977354,64.44700899)
\lineto(223.29977354,69.88450918)
\lineto(221.69560681,69.88450918)
\lineto(221.69560681,70.5720092)
\lineto(223.29977354,70.5720092)
\curveto(223.29977354,71.68138391)(223.45081488,72.55117594)(223.75810689,73.17617596)
\curveto(224.0758149,73.80117599)(224.53414825,74.28555067)(225.13310693,74.63450935)
\curveto(225.72685762,74.99388403)(226.40914831,75.17617603)(227.17477367,75.17617603)
\curveto(227.70081503,75.17617603)(228.17477371,75.0824267)(228.59144039,74.90534269)
\curveto(229.01852441,74.73867602)(229.33102442,74.51471734)(229.52894042,74.238676)
\curveto(229.73727376,73.97305066)(229.84144044,73.72305065)(229.84144044,73.48867597)
\curveto(229.84144044,73.28034263)(229.76852443,73.10325996)(229.63310709,72.96784262)
\curveto(229.49248176,72.82721728)(229.33102442,72.75950928)(229.15394041,72.75950928)
\curveto(228.94560707,72.75950928)(228.76331506,72.81159328)(228.61227372,72.92617596)
\curveto(228.47164839,73.03555063)(228.22685771,73.33242664)(227.88310703,73.82200932)
\curveto(227.68519103,74.113676)(227.47164835,74.32200934)(227.23727368,74.44700934)
\curveto(227.070607,74.53034268)(226.86227366,74.57200935)(226.61227365,74.57200935)
\curveto(226.28935764,74.57200935)(225.97685763,74.44700934)(225.67477362,74.19700933)
\curveto(225.36748161,73.94700932)(225.15914827,73.66055065)(225.0497736,73.34284264)
\curveto(224.89352426,72.91055062)(224.82060692,72.14492659)(224.82060692,71.05117589)
\closepath
\moveto(224.82060692,70.5720092)
}
}
{
\newrgbcolor{curcolor}{0 0 0}
\pscustom[linestyle=none,fillstyle=solid,fillcolor=curcolor]
{
\newpath
\moveto(235.30343607,75.17617603)
\lineto(235.30343607,64.11367564)
\curveto(235.30343607,63.58242629)(235.33468674,63.22825961)(235.40760274,63.05117561)
\curveto(235.49093608,62.88450893)(235.60552008,62.75950893)(235.76176942,62.67617559)
\curveto(235.92843609,62.59284226)(236.22531077,62.55117559)(236.65760279,62.55117559)
\lineto(236.65760279,62.21784224)
\lineto(232.55343597,62.21784224)
\lineto(232.55343597,62.55117559)
\curveto(232.93885332,62.55117559)(233.20447733,62.58242626)(233.34510267,62.65534226)
\curveto(233.48052001,62.73867559)(233.58468668,62.8688836)(233.65760268,63.05117561)
\curveto(233.74093602,63.22825961)(233.78260268,63.58242629)(233.78260268,64.11367564)
\lineto(233.78260268,71.69700924)
\curveto(233.78260268,72.62409328)(233.76176935,73.1970093)(233.72010268,73.40534264)
\curveto(233.67843601,73.62409331)(233.60552001,73.78034265)(233.51176934,73.86367599)
\curveto(233.428436,73.94700932)(233.308644,73.98867599)(233.15760266,73.98867599)
\curveto(233.00135332,73.98867599)(232.80343598,73.93659332)(232.55343597,73.84284265)
\lineto(232.40760264,74.15534267)
\lineto(234.88676939,75.17617603)
\closepath
\moveto(235.30343607,75.17617603)
}
}
{
\newrgbcolor{curcolor}{0 0 0}
\pscustom[linestyle=none,fillstyle=solid,fillcolor=curcolor]
{
\newpath
\moveto(239.01360059,67.42617576)
\curveto(239.01360059,66.18659305)(239.3156846,65.21784235)(239.93026729,64.50950899)
\curveto(240.52401798,63.80117563)(241.24276734,63.44700895)(242.0761007,63.44700895)
\curveto(242.61776739,63.44700895)(243.08651807,63.59284229)(243.49276742,63.88450897)
\curveto(243.89380876,64.18659298)(244.23235144,64.70742633)(244.51360079,65.44700902)
\lineto(244.78443413,65.25950902)
\curveto(244.65943412,64.42617565)(244.28964211,63.66575963)(243.68026742,62.9886756)
\curveto(243.06568473,62.30638398)(242.29485137,61.96784223)(241.36776734,61.96784223)
\curveto(240.36776731,61.96784223)(239.50318461,62.35325958)(238.78443392,63.13450894)
\curveto(238.07610056,63.9105503)(237.72193388,64.95742634)(237.72193388,66.28034239)
\curveto(237.72193388,67.70742644)(238.08651789,68.82200914)(238.82610058,69.61367584)
\curveto(239.56047528,70.41575987)(240.47714198,70.82200921)(241.57610068,70.82200921)
\curveto(242.51880872,70.82200921)(243.28964208,70.5095092)(243.88860076,69.88450918)
\curveto(244.48235145,69.26992649)(244.78443413,68.45221713)(244.78443413,67.42617576)
\closepath
\moveto(239.01360059,67.94700911)
\lineto(242.88860073,67.94700911)
\curveto(242.85735139,68.4886758)(242.79485139,68.86367581)(242.70110072,69.07200915)
\curveto(242.54485138,69.41575983)(242.31568471,69.68659317)(242.0136007,69.88450918)
\curveto(241.70630869,70.07721719)(241.39380868,70.17617586)(241.07610066,70.17617586)
\curveto(240.57610065,70.17617586)(240.1229753,69.97825985)(239.72193395,69.5928425)
\curveto(239.3156846,69.20221716)(239.08130859,68.65534247)(239.01360059,67.94700911)
\closepath
\moveto(239.01360059,67.94700911)
}
}
{
\newrgbcolor{curcolor}{0 0 0}
\pscustom[linestyle=none,fillstyle=solid,fillcolor=curcolor]
{
\newpath
\moveto(301.74045014,70.82200921)
\lineto(301.74045014,67.96784245)
\lineto(301.44878347,67.96784245)
\curveto(301.20920079,68.85325981)(300.91232478,69.4574265)(300.5529501,69.78034251)
\curveto(300.18836742,70.11367586)(299.73003407,70.28034253)(299.17795005,70.28034253)
\curveto(298.74565804,70.28034253)(298.39670069,70.16055052)(298.13628335,69.92617585)
\curveto(297.87065801,69.70221717)(297.74045,69.4574265)(297.74045,69.19700916)
\curveto(297.74045,68.84805048)(297.83420067,68.5563838)(298.03211668,68.32200913)
\curveto(298.22482469,68.07200912)(298.61545003,67.80638377)(299.19878339,67.53034243)
\lineto(300.51128343,66.88450907)
\curveto(301.74565814,66.26992639)(302.36545017,65.47305036)(302.36545017,64.48867566)
\curveto(302.36545017,63.73867563)(302.07899149,63.12409294)(301.51128347,62.65534226)
\curveto(300.93836745,62.19700891)(300.29253409,61.96784223)(299.5737834,61.96784223)
\curveto(299.07378338,61.96784223)(298.49045003,62.06159157)(297.82378334,62.23867558)
\curveto(297.62586733,62.29075891)(297.46961666,62.32200891)(297.34461665,62.32200891)
\curveto(297.20399132,62.32200891)(297.09461665,62.24388398)(297.01128331,62.09284224)
\lineto(296.71961663,62.09284224)
\lineto(296.71961663,65.07200901)
\lineto(297.01128331,65.07200901)
\curveto(297.17794998,64.22305031)(297.49565799,63.58242629)(297.96961668,63.15534228)
\curveto(298.45399136,62.72305026)(298.99565805,62.50950892)(299.59461673,62.50950892)
\curveto(300.01128342,62.50950892)(300.34982476,62.62409292)(300.6154501,62.8636756)
\curveto(300.87586745,63.11367561)(301.01128345,63.41055028)(301.01128345,63.75950896)
\curveto(301.01128345,64.17617565)(300.86545011,64.51992632)(300.57378344,64.80117567)
\curveto(300.28211676,65.09284234)(299.6883674,65.45221702)(298.80295004,65.88450904)
\curveto(297.92795001,66.32721705)(297.34982466,66.72825974)(297.07378331,67.09284242)
\curveto(296.80815797,67.43659309)(296.67794996,67.87409311)(296.67794996,68.40534246)
\curveto(296.67794996,69.08242649)(296.91232464,69.65534251)(297.38628332,70.11367586)
\curveto(297.85503401,70.58242654)(298.45920069,70.82200921)(299.19878339,70.82200921)
\curveto(299.53211673,70.82200921)(299.93315808,70.74909321)(300.40711676,70.61367587)
\curveto(300.70920077,70.5147172)(300.91232478,70.46784253)(301.01128345,70.46784253)
\curveto(301.10503412,70.46784253)(301.18315812,70.48867587)(301.24045013,70.53034254)
\curveto(301.29253413,70.5720092)(301.36545013,70.66575987)(301.44878347,70.82200921)
\closepath
\moveto(301.74045014,70.82200921)
}
}
{
\newrgbcolor{curcolor}{0 0 0}
\pscustom[linestyle=none,fillstyle=solid,fillcolor=curcolor]
{
\newpath
\moveto(303.12676204,70.5720092)
\lineto(307.02259551,70.5720092)
\lineto(307.02259551,70.23867586)
\lineto(306.83509551,70.23867586)
\curveto(306.55384616,70.23867586)(306.34551282,70.17617586)(306.21009548,70.05117585)
\curveto(306.08509548,69.93659318)(306.02259548,69.79075984)(306.02259548,69.61367584)
\curveto(306.02259548,69.37409316)(306.11634615,69.05117582)(306.31426216,68.63450914)
\lineto(308.35592889,64.40534232)
\lineto(310.21009563,69.03034248)
\curveto(310.3194703,69.28034249)(310.3767623,69.5199265)(310.3767623,69.75950918)
\curveto(310.3767623,69.86888385)(310.35592896,69.95221718)(310.3142623,70.00950919)
\curveto(310.25697029,70.07721719)(310.17363696,70.13450919)(310.06426229,70.17617586)
\curveto(309.96530362,70.21784253)(309.78301294,70.23867586)(309.5225956,70.23867586)
\lineto(309.5225956,70.5720092)
\lineto(312.25176237,70.5720092)
\lineto(312.25176237,70.23867586)
\curveto(312.02780369,70.20742653)(311.85592902,70.15534252)(311.73092901,70.07200919)
\curveto(311.60592901,69.99909318)(311.470513,69.86888385)(311.33509567,69.67617584)
\curveto(311.27780366,69.5928425)(311.17363699,69.35325983)(311.02259566,68.96784248)
\lineto(307.6267622,60.63450885)
\curveto(307.29342886,59.83242482)(306.85592884,59.22825814)(306.31426216,58.82200879)
\curveto(305.7830128,58.40534211)(305.27780345,58.19700877)(304.79342877,58.19700877)
\curveto(304.42884609,58.19700877)(304.13197008,58.30117544)(303.8975954,58.50950878)
\curveto(303.67363673,58.70742478)(303.56426206,58.94180079)(303.56426206,59.21784214)
\curveto(303.56426206,59.46784214)(303.64759539,59.67096749)(303.81426207,59.82200882)
\curveto(303.98092874,59.97825816)(304.21009541,60.0511755)(304.50176209,60.0511755)
\curveto(304.6944701,60.0511755)(304.96530344,59.9886755)(305.31426212,59.86367549)
\curveto(305.56426213,59.76992482)(305.71530347,59.71784215)(305.77259547,59.71784215)
\curveto(305.94967948,59.71784215)(306.14759548,59.81680082)(306.35592882,60.00950883)
\curveto(306.5746795,60.19180084)(306.79863684,60.55117552)(307.02259551,61.0928422)
\lineto(307.6267622,62.55117559)
\lineto(304.6267621,68.84284248)
\curveto(304.52780343,69.03555048)(304.38197009,69.26992649)(304.18926208,69.55117584)
\curveto(304.03301274,69.75950918)(303.90801274,69.89492651)(303.81426207,69.96784252)
\curveto(303.67363673,70.06159319)(303.44447005,70.15534252)(303.12676204,70.23867586)
\closepath
\moveto(303.12676204,70.5720092)
}
}
{
\newrgbcolor{curcolor}{0 0 0}
\pscustom[linestyle=none,fillstyle=solid,fillcolor=curcolor]
{
\newpath
\moveto(315.41842914,69.03034248)
\curveto(316.02780383,69.63971717)(316.38717985,69.99388385)(316.50176252,70.09284252)
\curveto(316.76217986,70.3115932)(317.05384654,70.48867587)(317.37676255,70.61367587)
\curveto(317.69447056,70.74909321)(318.00697057,70.82200921)(318.31426258,70.82200921)
\curveto(318.84030393,70.82200921)(319.29342928,70.66575987)(319.66842929,70.36367586)
\curveto(320.04342931,70.05638385)(320.29342932,69.61367584)(320.41842932,69.03034248)
\curveto(321.04342934,69.76471718)(321.5694707,70.24388386)(322.00176271,70.46784253)
\curveto(322.42884673,70.70221721)(322.87676274,70.82200921)(323.33509609,70.82200921)
\curveto(323.77780411,70.82200921)(324.17363745,70.70221721)(324.52259613,70.46784253)
\curveto(324.86634681,70.24388386)(325.13718016,69.87409318)(325.33509616,69.36367583)
\curveto(325.4705135,68.99909315)(325.5434295,68.44700913)(325.5434295,67.6970091)
\lineto(325.5434295,64.11367564)
\curveto(325.5434295,63.58242629)(325.57468017,63.22305028)(325.64759617,63.03034227)
\curveto(325.71530418,62.88971693)(325.82468018,62.76992626)(325.98092952,62.67617559)
\curveto(326.14759619,62.59284226)(326.41842953,62.55117559)(326.79342955,62.55117559)
\lineto(326.79342955,62.21784224)
\lineto(322.6684294,62.21784224)
\lineto(322.6684294,62.55117559)
\lineto(322.85592941,62.55117559)
\curveto(323.19968009,62.55117559)(323.48092943,62.61888359)(323.68926277,62.75950893)
\curveto(323.82468011,62.8532596)(323.92363745,63.00950894)(323.98092945,63.21784228)
\curveto(324.00697078,63.32721695)(324.02259612,63.62409296)(324.02259612,64.11367564)
\lineto(324.02259612,67.6970091)
\curveto(324.02259612,68.37409313)(323.93926278,68.85325981)(323.77259611,69.13450915)
\curveto(323.53301343,69.5199265)(323.15801342,69.71784251)(322.64759607,69.71784251)
\curveto(322.31426272,69.71784251)(321.98613738,69.63450917)(321.66842937,69.4678425)
\curveto(321.34551335,69.31159316)(320.94968001,69.01471715)(320.48092932,68.57200913)
\lineto(320.46009599,68.4886758)
\lineto(320.48092932,68.09284245)
\lineto(320.48092932,64.11367564)
\curveto(320.48092932,63.53034229)(320.50697066,63.16575961)(320.56426266,63.03034227)
\curveto(320.63197066,62.88971693)(320.75697067,62.76992626)(320.93926267,62.67617559)
\curveto(321.11634668,62.59284226)(321.41842936,62.55117559)(321.83509604,62.55117559)
\lineto(321.83509604,62.21784224)
\lineto(317.62676256,62.21784224)
\lineto(317.62676256,62.55117559)
\curveto(318.08509591,62.55117559)(318.39759592,62.60325959)(318.56426259,62.71784226)
\curveto(318.7413466,62.82721693)(318.8663466,62.9886756)(318.9392626,63.19700894)
\curveto(318.96530394,63.30638361)(318.98092927,63.61367563)(318.98092927,64.11367564)
\lineto(318.98092927,67.6970091)
\curveto(318.98092927,68.37409313)(318.8819706,68.86367581)(318.68926259,69.15534249)
\curveto(318.40801325,69.54075984)(318.03301324,69.73867584)(317.56426255,69.73867584)
\curveto(317.23092921,69.73867584)(316.90280386,69.65534251)(316.58509585,69.48867583)
\curveto(316.08509583,69.20742649)(315.69447049,68.90534248)(315.41842914,68.57200913)
\lineto(315.41842914,64.11367564)
\curveto(315.41842914,63.55638362)(315.44967981,63.19700894)(315.52259581,63.03034227)
\curveto(315.60592915,62.87409293)(315.71530382,62.75950893)(315.85592916,62.67617559)
\curveto(316.0069705,62.59284226)(316.31426251,62.55117559)(316.77259586,62.55117559)
\lineto(316.77259586,62.21784224)
\lineto(312.64759571,62.21784224)
\lineto(312.64759571,62.55117559)
\curveto(313.03301306,62.55117559)(313.29863707,62.59284226)(313.43926241,62.67617559)
\curveto(313.59030375,62.75950893)(313.71009575,62.88450893)(313.79342909,63.05117561)
\curveto(313.87676242,63.22825961)(313.91842909,63.58242629)(313.91842909,64.11367564)
\lineto(313.91842909,67.30117576)
\curveto(313.91842909,68.21784245)(313.88717976,68.80638381)(313.83509575,69.07200915)
\curveto(313.79342909,69.26471716)(313.72051308,69.3949265)(313.62676241,69.4678425)
\curveto(313.54342908,69.55117584)(313.41842907,69.5928425)(313.2517624,69.5928425)
\curveto(313.08509573,69.5928425)(312.88197039,69.55117584)(312.64759571,69.4678425)
\lineto(312.50176237,69.80117584)
\lineto(315.0225958,70.82200921)
\lineto(315.41842914,70.82200921)
\closepath
\moveto(315.41842914,69.03034248)
}
}
{
\newrgbcolor{curcolor}{0 0 0}
\pscustom[linestyle=none,fillstyle=solid,fillcolor=curcolor]
{
\newpath
\moveto(330.33692739,75.17617603)
\lineto(330.33692739,64.11367564)
\curveto(330.33692739,63.58242629)(330.36817806,63.22825961)(330.44109406,63.05117561)
\curveto(330.5244274,62.88450893)(330.6390114,62.75950893)(330.79526074,62.67617559)
\curveto(330.96192742,62.59284226)(331.25880209,62.55117559)(331.69109411,62.55117559)
\lineto(331.69109411,62.21784224)
\lineto(327.5869273,62.21784224)
\lineto(327.5869273,62.55117559)
\curveto(327.97234464,62.55117559)(328.23796865,62.58242626)(328.37859399,62.65534226)
\curveto(328.51401133,62.73867559)(328.618178,62.8688836)(328.691094,63.05117561)
\curveto(328.77442734,63.22825961)(328.81609401,63.58242629)(328.81609401,64.11367564)
\lineto(328.81609401,71.69700924)
\curveto(328.81609401,72.62409328)(328.79526067,73.1970093)(328.753594,73.40534264)
\curveto(328.71192734,73.62409331)(328.63901133,73.78034265)(328.54526066,73.86367599)
\curveto(328.46192733,73.94700932)(328.34213532,73.98867599)(328.19109398,73.98867599)
\curveto(328.03484465,73.98867599)(327.83692731,73.93659332)(327.5869273,73.84284265)
\lineto(327.44109396,74.15534267)
\lineto(329.92026071,75.17617603)
\closepath
\moveto(330.33692739,75.17617603)
}
}
{
\newrgbcolor{curcolor}{0 0 0}
\pscustom[linestyle=none,fillstyle=solid,fillcolor=curcolor]
{
\newpath
\moveto(334.77625861,75.17617603)
\curveto(335.02625862,75.17617603)(335.23979996,75.0824267)(335.42209196,74.90534269)
\curveto(335.59917597,74.72305069)(335.69292531,74.50950934)(335.69292531,74.25950934)
\curveto(335.69292531,74.00950933)(335.59917597,73.78555065)(335.42209196,73.59284265)
\curveto(335.23979996,73.41055064)(335.02625862,73.3220093)(334.77625861,73.3220093)
\curveto(334.5262586,73.3220093)(334.30229992,73.41055064)(334.10959192,73.59284265)
\curveto(333.92729991,73.78555065)(333.83875857,74.00950933)(333.83875857,74.25950934)
\curveto(333.83875857,74.50950934)(333.92729991,74.72305069)(334.10959192,74.90534269)
\curveto(334.28667592,75.0824267)(334.51063326,75.17617603)(334.77625861,75.17617603)
\closepath
\moveto(335.52625863,70.82200921)
\lineto(335.52625863,64.11367564)
\curveto(335.52625863,63.58242629)(335.5575093,63.22825961)(335.6304253,63.05117561)
\curveto(335.71375864,62.88450893)(335.82834264,62.75950893)(335.98459198,62.67617559)
\curveto(336.13563332,62.59284226)(336.40646666,62.55117559)(336.79709201,62.55117559)
\lineto(336.79709201,62.21784224)
\lineto(332.73459187,62.21784224)
\lineto(332.73459187,62.55117559)
\curveto(333.15125855,62.55117559)(333.42729989,62.58242626)(333.56792523,62.65534226)
\curveto(333.70334257,62.73867559)(333.81792524,62.8688836)(333.90125858,63.05117561)
\curveto(333.98459191,63.22825961)(334.02625858,63.58242629)(334.02625858,64.11367564)
\lineto(334.02625858,67.32200909)
\curveto(334.02625858,68.22305046)(333.99500925,68.80638381)(333.94292524,69.07200915)
\curveto(333.90125858,69.26471716)(333.82834257,69.3949265)(333.7345919,69.4678425)
\curveto(333.65125857,69.55117584)(333.52625856,69.5928425)(333.35959189,69.5928425)
\curveto(333.19292522,69.5928425)(332.98459188,69.55117584)(332.73459187,69.4678425)
\lineto(332.60959186,69.80117584)
\lineto(335.13042529,70.82200921)
\closepath
\moveto(335.52625863,70.82200921)
}
}
{
\newrgbcolor{curcolor}{0 0 0}
\pscustom[linestyle=none,fillstyle=solid,fillcolor=curcolor]
{
\newpath
\moveto(340.27808986,69.05117582)
\curveto(341.24684056,70.22825986)(342.17913126,70.82200921)(343.06975662,70.82200921)
\curveto(343.52808997,70.82200921)(343.91350732,70.70221721)(344.23642333,70.46784253)
\curveto(344.56975667,70.24388386)(344.83017402,69.86888385)(345.02809002,69.34284249)
\curveto(345.16350736,68.97825982)(345.23642336,68.4261758)(345.23642336,67.67617577)
\lineto(345.23642336,64.11367564)
\curveto(345.23642336,63.58242629)(345.27809003,63.22305028)(345.36142337,63.03034227)
\curveto(345.42913137,62.87409293)(345.53329804,62.75950893)(345.67392338,62.67617559)
\curveto(345.82496472,62.59284226)(346.1010074,62.55117559)(346.50725674,62.55117559)
\lineto(346.50725674,62.21784224)
\lineto(342.3822566,62.21784224)
\lineto(342.3822566,62.55117559)
\lineto(342.54892327,62.55117559)
\curveto(342.93434062,62.55117559)(343.20517396,62.60325959)(343.3614233,62.71784226)
\curveto(343.52808997,62.84284226)(343.63746464,63.01471694)(343.69475664,63.23867561)
\curveto(343.72079798,63.33242628)(343.73642331,63.62409296)(343.73642331,64.11367564)
\lineto(343.73642331,67.53034243)
\curveto(343.73642331,68.28034246)(343.63225664,68.82721714)(343.4239233,69.17617582)
\curveto(343.22600729,69.5199265)(342.90308995,69.69700917)(342.4447566,69.69700917)
\curveto(341.70517391,69.69700917)(340.98642321,69.30117583)(340.27808986,68.50950913)
\lineto(340.27808986,64.11367564)
\curveto(340.27808986,63.54075962)(340.30934052,63.18659294)(340.38225653,63.05117561)
\curveto(340.46558986,62.88450893)(340.57496453,62.75950893)(340.71558987,62.67617559)
\curveto(340.86663121,62.59284226)(341.17392322,62.55117559)(341.63225657,62.55117559)
\lineto(341.63225657,62.21784224)
\lineto(337.50725642,62.21784224)
\lineto(337.50725642,62.55117559)
\lineto(337.69475643,62.55117559)
\curveto(338.11142311,62.55117559)(338.39267379,62.65534226)(338.54892313,62.8636756)
\curveto(338.69996447,63.08242627)(338.7780898,63.49909295)(338.7780898,64.11367564)
\lineto(338.7780898,67.19700909)
\curveto(338.7780898,68.20742645)(338.74684047,68.82200914)(338.69475647,69.03034248)
\curveto(338.6530898,69.24909316)(338.5801738,69.3949265)(338.48642313,69.4678425)
\curveto(338.40308979,69.55117584)(338.27808978,69.5928425)(338.11142311,69.5928425)
\curveto(337.94475644,69.5928425)(337.7416311,69.55117584)(337.50725642,69.4678425)
\lineto(337.36142309,69.80117584)
\lineto(339.88225651,70.82200921)
\lineto(340.27808986,70.82200921)
\closepath
\moveto(340.27808986,69.05117582)
}
}
{
\newrgbcolor{curcolor}{0 0 0}
\pscustom[linestyle=none,fillstyle=solid,fillcolor=curcolor]
{
\newpath
\moveto(349.65309019,75.17617603)
\lineto(349.65309019,66.88450907)
\lineto(351.77809026,68.80117581)
\curveto(352.22079828,69.21784249)(352.47600762,69.47825983)(352.54892362,69.5928425)
\curveto(352.60100763,69.66055051)(352.63225696,69.72825984)(352.63225696,69.80117584)
\curveto(352.63225696,69.92617585)(352.57496496,70.03034252)(352.46559029,70.11367586)
\curveto(352.36663162,70.20742653)(352.20517428,70.26471719)(351.9864236,70.28034253)
\lineto(351.9864236,70.5720092)
\lineto(355.61142373,70.5720092)
\lineto(355.61142373,70.28034253)
\curveto(355.11142371,70.26471719)(354.69475703,70.18659319)(354.36142369,70.05117585)
\curveto(354.02809034,69.91055051)(353.66350766,69.66055051)(353.27809032,69.30117583)
\lineto(351.13225691,67.32200909)
\lineto(353.27809032,64.61367566)
\curveto(353.871841,63.86367563)(354.26767435,63.38971695)(354.46559036,63.19700894)
\curveto(354.75725704,62.90534227)(355.01246504,62.71784226)(355.23642372,62.63450892)
\curveto(355.38746506,62.57721692)(355.6530904,62.55117559)(356.02809041,62.55117559)
\lineto(356.02809041,62.21784224)
\lineto(351.9864236,62.21784224)
\lineto(351.9864236,62.55117559)
\curveto(352.20517428,62.55117559)(352.36142362,62.57721692)(352.44475695,62.63450892)
\curveto(352.52809029,62.70221693)(352.56975696,62.8011756)(352.56975696,62.9261756)
\curveto(352.56975696,63.06159294)(352.44996495,63.28555028)(352.21559028,63.59284229)
\lineto(349.65309019,66.86367574)
\lineto(349.65309019,64.09284231)
\curveto(349.65309019,63.55117562)(349.68434086,63.19700894)(349.75725686,63.03034227)
\curveto(349.84059019,62.8636756)(349.94996487,62.73867559)(350.0905902,62.65534226)
\curveto(350.22600754,62.58242626)(350.52809022,62.55117559)(350.98642357,62.55117559)
\lineto(350.98642357,62.21784224)
\lineto(346.73642342,62.21784224)
\lineto(346.73642342,62.55117559)
\curveto(347.16350743,62.55117559)(347.48642344,62.59805026)(347.69475679,62.69700893)
\curveto(347.83017412,62.76471693)(347.92913146,62.8688836)(347.98642346,63.00950894)
\curveto(348.08017413,63.21784228)(348.1322568,63.56159296)(348.1322568,64.05117564)
\lineto(348.1322568,71.65534258)
\curveto(348.1322568,72.61367594)(348.11142347,73.1970093)(348.0697568,73.40534264)
\curveto(348.02809013,73.62409331)(347.95517413,73.78034265)(347.86142346,73.86367599)
\curveto(347.76246479,73.94700932)(347.63225678,73.98867599)(347.46559011,73.98867599)
\curveto(347.34059011,73.98867599)(347.1530901,73.93659332)(346.90309009,73.84284265)
\lineto(346.73642342,74.15534267)
\lineto(349.21559017,75.17617603)
\closepath
\moveto(349.65309019,75.17617603)
}
}
{
\newrgbcolor{curcolor}{0 0 0}
\pscustom[linestyle=none,fillstyle=solid,fillcolor=curcolor]
{
\newpath
\moveto(594.03757964,70.5720092)
\lineto(597.51674643,70.5720092)
\curveto(598.19383045,70.5720092)(598.64174647,70.58242254)(598.85007981,70.61367587)
\lineto(599.24591316,70.61367587)
\lineto(599.24591316,64.15534231)
\curveto(599.24591316,63.47305029)(599.31883049,63.03555027)(599.47507983,62.84284226)
\curveto(599.64174651,62.66055026)(600.00633052,62.56159292)(600.57924654,62.55117559)
\lineto(600.57924654,62.21784224)
\lineto(596.41257973,62.21784224)
\lineto(596.41257973,62.55117559)
\curveto(597.00633041,62.55117559)(597.39174643,62.68138359)(597.5584131,62.94700893)
\curveto(597.6834131,63.12409294)(597.74591311,63.53034229)(597.74591311,64.15534231)
\lineto(597.74591311,68.57200913)
\curveto(597.74591311,69.13971715)(597.71466377,69.47825983)(597.66257977,69.5928425)
\curveto(597.60528777,69.68659317)(597.50633043,69.75950918)(597.37091309,69.80117584)
\curveto(597.24591309,69.85325985)(596.80841307,69.88450918)(596.05841305,69.88450918)
\lineto(594.03757964,69.88450918)
\lineto(594.03757964,64.44700899)
\curveto(594.03757964,63.79075963)(594.06883031,63.36888362)(594.14174631,63.17617561)
\curveto(594.19383031,63.01992627)(594.30841298,62.89492627)(594.47507966,62.8011756)
\curveto(594.735497,62.63450892)(594.96466367,62.55117559)(595.16257968,62.55117559)
\lineto(595.55841303,62.55117559)
\lineto(595.55841303,62.21784224)
\lineto(590.99591287,62.21784224)
\lineto(590.99591287,62.55117559)
\lineto(591.32924621,62.55117559)
\curveto(591.74591289,62.55117559)(592.04278757,62.66055026)(592.22507958,62.88450893)
\curveto(592.41778758,63.11888361)(592.51674625,63.63971696)(592.51674625,64.44700899)
\lineto(592.51674625,69.88450918)
\lineto(590.91257953,69.88450918)
\lineto(590.91257953,70.5720092)
\lineto(592.51674625,70.5720092)
\curveto(592.51674625,71.68138391)(592.66778759,72.55117594)(592.9750796,73.17617596)
\curveto(593.29278761,73.80117599)(593.75112096,74.28555067)(594.35007965,74.63450935)
\curveto(594.94383034,74.99388403)(595.62612103,75.17617603)(596.39174639,75.17617603)
\curveto(596.91778774,75.17617603)(597.39174643,75.0824267)(597.80841311,74.90534269)
\curveto(598.23549712,74.73867602)(598.54799713,74.51471734)(598.74591314,74.238676)
\curveto(598.95424648,73.97305066)(599.05841315,73.72305065)(599.05841315,73.48867597)
\curveto(599.05841315,73.28034263)(598.98549715,73.10325996)(598.85007981,72.96784262)
\curveto(598.70945447,72.82721728)(598.54799713,72.75950928)(598.37091313,72.75950928)
\curveto(598.16257979,72.75950928)(597.98028778,72.81159328)(597.82924644,72.92617596)
\curveto(597.6886211,73.03555063)(597.44383043,73.33242664)(597.10007975,73.82200932)
\curveto(596.90216374,74.113676)(596.68862107,74.32200934)(596.45424639,74.44700934)
\curveto(596.28757972,74.53034268)(596.07924638,74.57200935)(595.82924637,74.57200935)
\curveto(595.50633036,74.57200935)(595.19383035,74.44700934)(594.89174634,74.19700933)
\curveto(594.58445433,73.94700932)(594.37612099,73.66055065)(594.26674632,73.34284264)
\curveto(594.11049698,72.91055062)(594.03757964,72.14492659)(594.03757964,71.05117589)
\closepath
\moveto(594.03757964,70.5720092)
}
}
{
\newrgbcolor{curcolor}{0 0 0}
\pscustom[linestyle=none,fillstyle=solid,fillcolor=curcolor]
{
\newpath
\moveto(604.52040879,75.17617603)
\lineto(604.52040879,64.11367564)
\curveto(604.52040879,63.58242629)(604.55165946,63.22825961)(604.62457546,63.05117561)
\curveto(604.7079088,62.88450893)(604.8224928,62.75950893)(604.97874214,62.67617559)
\curveto(605.14540881,62.59284226)(605.44228349,62.55117559)(605.8745755,62.55117559)
\lineto(605.8745755,62.21784224)
\lineto(601.77040869,62.21784224)
\lineto(601.77040869,62.55117559)
\curveto(602.15582604,62.55117559)(602.42145005,62.58242626)(602.56207539,62.65534226)
\curveto(602.69749272,62.73867559)(602.80165939,62.8688836)(602.8745754,63.05117561)
\curveto(602.95790873,63.22825961)(602.9995754,63.58242629)(602.9995754,64.11367564)
\lineto(602.9995754,71.69700924)
\curveto(602.9995754,72.62409328)(602.97874207,73.1970093)(602.9370754,73.40534264)
\curveto(602.89540873,73.62409331)(602.82249273,73.78034265)(602.72874206,73.86367599)
\curveto(602.64540872,73.94700932)(602.52561672,73.98867599)(602.37457538,73.98867599)
\curveto(602.21832604,73.98867599)(602.0204087,73.93659332)(601.77040869,73.84284265)
\lineto(601.62457535,74.15534267)
\lineto(604.10374211,75.17617603)
\closepath
\moveto(604.52040879,75.17617603)
}
}
{
\newrgbcolor{curcolor}{0 0 0}
\pscustom[linestyle=none,fillstyle=solid,fillcolor=curcolor]
{
\newpath
\moveto(608.230614,67.42617576)
\curveto(608.230614,66.18659305)(608.53269801,65.21784235)(609.1472807,64.50950899)
\curveto(609.74103138,63.80117563)(610.45978074,63.44700895)(611.29311411,63.44700895)
\curveto(611.83478079,63.44700895)(612.30353148,63.59284229)(612.70978082,63.88450897)
\curveto(613.11082217,64.18659298)(613.44936485,64.70742633)(613.73061419,65.44700902)
\lineto(614.00144754,65.25950902)
\curveto(613.87644753,64.42617565)(613.50665552,63.66575963)(612.89728083,62.9886756)
\curveto(612.28269814,62.30638398)(611.51186478,61.96784223)(610.58478075,61.96784223)
\curveto(609.58478071,61.96784223)(608.72019802,62.35325958)(608.00144732,63.13450894)
\curveto(607.29311396,63.9105503)(606.93894729,64.95742634)(606.93894729,66.28034239)
\curveto(606.93894729,67.70742644)(607.3035313,68.82200914)(608.04311399,69.61367584)
\curveto(608.77748868,70.41575987)(609.69415538,70.82200921)(610.79311409,70.82200921)
\curveto(611.73582212,70.82200921)(612.50665548,70.5095092)(613.10561417,69.88450918)
\curveto(613.69936486,69.26992649)(614.00144754,68.45221713)(614.00144754,67.42617576)
\closepath
\moveto(608.230614,67.94700911)
\lineto(612.10561414,67.94700911)
\curveto(612.0743648,68.4886758)(612.0118648,68.86367581)(611.91811413,69.07200915)
\curveto(611.76186479,69.41575983)(611.53269811,69.68659317)(611.2306141,69.88450918)
\curveto(610.92332209,70.07721719)(610.61082208,70.17617586)(610.29311407,70.17617586)
\curveto(609.79311405,70.17617586)(609.3399887,69.97825985)(608.93894736,69.5928425)
\curveto(608.53269801,69.20221716)(608.298322,68.65534247)(608.230614,67.94700911)
\closepath
\moveto(608.230614,67.94700911)
}
}
{
\newrgbcolor{curcolor}{0 0 0}
\pscustom[linestyle=none,fillstyle=solid,fillcolor=curcolor]
{
\newpath
\moveto(296.41452222,310.57201771)
\lineto(296.41452222,305.50951753)
\curveto(296.41452222,304.53555883)(296.43535556,303.93660147)(296.47702222,303.7178508)
\curveto(296.52910622,303.50951746)(296.60723023,303.36368412)(296.7061889,303.28035078)
\curveto(296.79993957,303.19701745)(296.91973024,303.15535078)(297.06035558,303.15535078)
\curveto(297.25306358,303.15535078)(297.46660626,303.20743478)(297.70618893,303.32201745)
\lineto(297.83118894,302.98868411)
\lineto(295.33118885,301.96785074)
\lineto(294.91452217,301.96785074)
\lineto(294.91452217,303.73868413)
\curveto(294.20618881,302.95743477)(293.65410612,302.46785075)(293.26868878,302.25951741)
\curveto(292.89368876,302.06680941)(292.49785542,301.96785074)(292.08118873,301.96785074)
\curveto(291.60723005,301.96785074)(291.1957727,302.10326741)(290.85202202,302.36368408)
\curveto(290.51868868,302.63972543)(290.279106,302.98868411)(290.14368867,303.40535079)
\curveto(290.01868866,303.8324348)(289.95618866,304.43660149)(289.95618866,305.21785085)
\lineto(289.95618866,308.94701765)
\curveto(289.95618866,309.332435)(289.90410599,309.60326834)(289.81035532,309.75951768)
\curveto(289.72702198,309.91055902)(289.60202198,310.03035102)(289.43535531,310.11368436)
\curveto(289.26868863,310.1970177)(288.96660596,310.23868436)(288.53952194,310.23868436)
\lineto(288.53952194,310.57201771)
\lineto(291.45618871,310.57201771)
\lineto(291.45618871,304.98868418)
\curveto(291.45618871,304.20743482)(291.58639672,303.69701746)(291.85202206,303.44701746)
\curveto(292.1280634,303.20743478)(292.46139675,303.09285078)(292.85202209,303.09285078)
\curveto(293.11243944,303.09285078)(293.40410611,303.16576811)(293.72702213,303.32201745)
\curveto(294.06035547,303.48868412)(294.45618882,303.80118414)(294.91452217,304.25951748)
\lineto(294.91452217,309.00951765)
\curveto(294.91452217,309.47826834)(294.82077283,309.80118435)(294.64368882,309.96785102)
\curveto(294.47702215,310.13451769)(294.12285547,310.22305903)(293.58118879,310.23868436)
\lineto(293.58118879,310.57201771)
\closepath
\moveto(296.41452222,310.57201771)
}
}
{
\newrgbcolor{curcolor}{0 0 0}
\pscustom[linestyle=none,fillstyle=solid,fillcolor=curcolor]
{
\newpath
\moveto(300.72702237,309.13451766)
\curveto(301.5291064,310.2595177)(302.3988971,310.82201772)(303.33118913,310.82201772)
\curveto(304.19056383,310.82201772)(304.94056386,310.4522257)(305.58118921,309.71785101)
\curveto(306.21660657,308.97826832)(306.53952258,307.97826828)(306.53952258,306.71785091)
\curveto(306.53952258,305.22826819)(306.04473056,304.03555881)(305.06035586,303.13451744)
\curveto(304.21139716,302.35326808)(303.26868913,301.96785074)(302.22702243,301.96785074)
\curveto(301.73743974,301.96785074)(301.24785573,302.05118407)(300.74785571,302.21785075)
\curveto(300.24785569,302.39493475)(299.73743967,302.66576809)(299.22702232,303.03035077)
\lineto(299.22702232,311.67618441)
\curveto(299.22702232,312.61889245)(299.19577299,313.1970178)(299.14368898,313.40535114)
\curveto(299.10202232,313.62410182)(299.02910631,313.78035116)(298.93535564,313.86368449)
\curveto(298.83639697,313.94701783)(298.7166063,313.9886845)(298.58118896,313.9886845)
\curveto(298.39889696,313.9886845)(298.18535562,313.93660183)(297.93535561,313.84285116)
\lineto(297.8103556,314.15535117)
\lineto(300.31035569,315.17618454)
\lineto(300.72702237,315.17618454)
\closepath
\moveto(300.72702237,308.5511843)
\lineto(300.72702237,303.55118413)
\curveto(301.02910638,303.24389212)(301.3520224,303.01472544)(301.68535574,302.8636841)
\curveto(302.01868909,302.70743476)(302.35723043,302.63451743)(302.70618911,302.63451743)
\curveto(303.25827313,302.63451743)(303.77389715,302.9313921)(304.24785583,303.53035079)
\curveto(304.71660652,304.13972548)(304.95618919,305.03035085)(304.95618919,306.19701755)
\curveto(304.95618919,307.26472559)(304.71660652,308.08243495)(304.24785583,308.65535097)
\curveto(303.77389715,309.22305899)(303.23223046,309.50951767)(302.62285577,309.50951767)
\curveto(302.29993976,309.50951767)(301.97702242,309.43139233)(301.64368907,309.280351)
\curveto(301.4041064,309.15535099)(301.10202239,308.91055898)(300.72702237,308.5511843)
\closepath
\moveto(300.72702237,308.5511843)
}
}
{
\newrgbcolor{curcolor}{0 0 0}
\pscustom[linestyle=none,fillstyle=solid,fillcolor=curcolor]
{
\newpath
\moveto(309.16452267,307.42618426)
\curveto(309.16452267,306.18660155)(309.46660668,305.21785085)(310.08118937,304.50951749)
\curveto(310.67494006,303.80118414)(311.39368942,303.44701746)(312.22702278,303.44701746)
\curveto(312.76868947,303.44701746)(313.23744015,303.59285079)(313.6436895,303.88451747)
\curveto(314.04473085,304.18660148)(314.38327352,304.70743483)(314.66452287,305.44701753)
\lineto(314.93535621,305.25951752)
\curveto(314.81035621,304.42618416)(314.44056419,303.66576813)(313.8311895,302.98868411)
\curveto(313.21660682,302.30639248)(312.44577346,301.96785074)(311.51868942,301.96785074)
\curveto(310.51868939,301.96785074)(309.65410669,302.35326808)(308.935356,303.13451744)
\curveto(308.22702264,303.91055881)(307.87285596,304.95743484)(307.87285596,306.28035089)
\curveto(307.87285596,307.70743494)(308.23743997,308.82201765)(308.97702267,309.61368434)
\curveto(309.71139736,310.41576837)(310.62806406,310.82201772)(311.72702276,310.82201772)
\curveto(312.6697308,310.82201772)(313.44056416,310.50951771)(314.03952285,309.88451768)
\curveto(314.63327353,309.269935)(314.93535621,308.45222563)(314.93535621,307.42618426)
\closepath
\moveto(309.16452267,307.94701762)
\lineto(313.03952281,307.94701762)
\curveto(313.00827348,308.4886843)(312.94577347,308.86368431)(312.8520228,309.07201766)
\curveto(312.69577346,309.41576833)(312.46660679,309.68660168)(312.16452278,309.88451768)
\curveto(311.85723077,310.07722569)(311.54473076,310.17618436)(311.22702275,310.17618436)
\curveto(310.72702273,310.17618436)(310.27389738,309.97826835)(309.87285603,309.59285101)
\curveto(309.46660668,309.20222566)(309.23223068,308.65535097)(309.16452267,307.94701762)
\closepath
\moveto(309.16452267,307.94701762)
}
}
{
\newrgbcolor{curcolor}{0 0 0}
\pscustom[linestyle=none,fillstyle=solid,fillcolor=curcolor]
{
\newpath
\moveto(318.49419423,310.82201772)
\lineto(318.49419423,308.92618432)
\curveto(319.20252759,310.18660169)(319.92127828,310.82201772)(320.66086097,310.82201772)
\curveto(320.99419432,310.82201772)(321.26502766,310.71785105)(321.473361,310.50951771)
\curveto(321.69211167,310.30118437)(321.80669434,310.06160169)(321.80669434,309.80118435)
\curveto(321.80669434,309.56160167)(321.72336101,309.36368433)(321.55669434,309.19701766)
\curveto(321.400445,309.03035099)(321.22336099,308.94701765)(321.01502765,308.94701765)
\curveto(320.79106898,308.94701765)(320.5462783,309.05118432)(320.28586096,309.25951766)
\curveto(320.02023561,309.47826834)(319.82752761,309.59285101)(319.7025276,309.59285101)
\curveto(319.58794493,309.59285101)(319.47336093,309.530351)(319.34836092,309.405351)
\curveto(319.06711158,309.15535099)(318.7858609,308.74389231)(318.49419423,308.17618429)
\lineto(318.49419423,304.17618415)
\curveto(318.49419423,303.70222546)(318.55669423,303.34805879)(318.68169423,303.11368411)
\curveto(318.74940224,302.95743477)(318.87961157,302.82722543)(319.07752758,302.71785076)
\curveto(319.28586092,302.60326809)(319.5775276,302.55118409)(319.95252761,302.55118409)
\lineto(319.95252761,302.21785075)
\lineto(315.68169413,302.21785075)
\lineto(315.68169413,302.55118409)
\curveto(316.10877814,302.55118409)(316.43169415,302.61368409)(316.64002749,302.7386841)
\curveto(316.79106883,302.83243477)(316.8952355,302.99389211)(316.95252751,303.21785078)
\curveto(316.97856884,303.31160145)(316.99419417,303.60326813)(316.99419417,304.09285081)
\lineto(316.99419417,307.32201759)
\curveto(316.99419417,308.29076829)(316.97336084,308.86889231)(316.93169417,309.05118432)
\curveto(316.8900275,309.24389233)(316.81190217,309.38451767)(316.7025275,309.467851)
\curveto(316.60356883,309.55118434)(316.47856882,309.59285101)(316.32752748,309.59285101)
\curveto(316.12961148,309.59285101)(315.9160688,309.55118434)(315.68169413,309.467851)
\lineto(315.59836079,309.80118435)
\lineto(318.11919421,310.82201772)
\closepath
\moveto(318.49419423,310.82201772)
}
}
{
\newrgbcolor{curcolor}{0 0 0}
\pscustom[linestyle=none,fillstyle=solid,fillcolor=curcolor]
{
\newpath
\moveto(324.56434418,309.13451766)
\curveto(325.36642821,310.2595177)(326.23621891,310.82201772)(327.16851094,310.82201772)
\curveto(328.02788564,310.82201772)(328.77788566,310.4522257)(329.41851102,309.71785101)
\curveto(330.05392838,308.97826832)(330.37684439,307.97826828)(330.37684439,306.71785091)
\curveto(330.37684439,305.22826819)(329.88205237,304.03555881)(328.89767767,303.13451744)
\curveto(328.04871897,302.35326808)(327.10601094,301.96785074)(326.06434423,301.96785074)
\curveto(325.57476155,301.96785074)(325.08517753,302.05118407)(324.58517752,302.21785075)
\curveto(324.0851775,302.39493475)(323.57476148,302.66576809)(323.06434413,303.03035077)
\lineto(323.06434413,311.67618441)
\curveto(323.06434413,312.61889245)(323.03309479,313.1970178)(322.98101079,313.40535114)
\curveto(322.93934412,313.62410182)(322.86642812,313.78035116)(322.77267745,313.86368449)
\curveto(322.67371878,313.94701783)(322.55392811,313.9886845)(322.41851077,313.9886845)
\curveto(322.23621877,313.9886845)(322.02267743,313.93660183)(321.77267742,313.84285116)
\lineto(321.64767741,314.15535117)
\lineto(324.1476775,315.17618454)
\lineto(324.56434418,315.17618454)
\closepath
\moveto(324.56434418,308.5511843)
\lineto(324.56434418,303.55118413)
\curveto(324.86642819,303.24389212)(325.1893442,303.01472544)(325.52267755,302.8636841)
\curveto(325.85601089,302.70743476)(326.19455224,302.63451743)(326.54351092,302.63451743)
\curveto(327.09559494,302.63451743)(327.61121896,302.9313921)(328.08517764,303.53035079)
\curveto(328.55392832,304.13972548)(328.793511,305.03035085)(328.793511,306.19701755)
\curveto(328.793511,307.26472559)(328.55392832,308.08243495)(328.08517764,308.65535097)
\curveto(327.61121896,309.22305899)(327.06955227,309.50951767)(326.46017758,309.50951767)
\curveto(326.13726157,309.50951767)(325.81434423,309.43139233)(325.48101088,309.280351)
\curveto(325.24142821,309.15535099)(324.9393442,308.91055898)(324.56434418,308.5511843)
\closepath
\moveto(324.56434418,308.5511843)
}
}
{
\newrgbcolor{curcolor}{0 0 0}
\pscustom[linestyle=none,fillstyle=solid,fillcolor=curcolor]
{
\newpath
\moveto(334.4810112,315.17618454)
\lineto(334.4810112,304.11368415)
\curveto(334.4810112,303.58243479)(334.51226187,303.22826811)(334.58517787,303.05118411)
\curveto(334.66851121,302.88451744)(334.78309521,302.75951743)(334.93934455,302.6761841)
\curveto(335.10601122,302.59285076)(335.4028859,302.55118409)(335.83517791,302.55118409)
\lineto(335.83517791,302.21785075)
\lineto(331.7310111,302.21785075)
\lineto(331.7310111,302.55118409)
\curveto(332.11642845,302.55118409)(332.38205246,302.58243476)(332.5226778,302.65535076)
\curveto(332.65809514,302.7386841)(332.76226181,302.8688921)(332.83517781,303.05118411)
\curveto(332.91851114,303.22826811)(332.96017781,303.58243479)(332.96017781,304.11368415)
\lineto(332.96017781,311.69701775)
\curveto(332.96017781,312.62410178)(332.93934448,313.1970178)(332.89767781,313.40535114)
\curveto(332.85601114,313.62410182)(332.78309514,313.78035116)(332.68934447,313.86368449)
\curveto(332.60601113,313.94701783)(332.48621913,313.9886845)(332.33517779,313.9886845)
\curveto(332.17892845,313.9886845)(331.98101111,313.93660183)(331.7310111,313.84285116)
\lineto(331.58517776,314.15535117)
\lineto(334.06434452,315.17618454)
\closepath
\moveto(334.4810112,315.17618454)
}
}
{
\newrgbcolor{curcolor}{0 0 0}
\pscustom[linestyle=none,fillstyle=solid,fillcolor=curcolor]
{
\newpath
\moveto(340.87867581,310.82201772)
\curveto(342.13909319,310.82201772)(343.15471723,310.33243503)(343.92034259,309.36368433)
\curveto(344.57138395,308.5407683)(344.89950929,307.60326827)(344.89950929,306.55118423)
\curveto(344.89950929,305.80118421)(344.71721728,305.04076818)(344.3578426,304.28035082)
\curveto(343.99325992,303.51472546)(343.50367591,302.93660144)(342.87867589,302.55118409)
\curveto(342.25367586,302.16576741)(341.55055051,301.96785074)(340.77450914,301.96785074)
\curveto(339.5245091,301.96785074)(338.52450906,302.46785075)(337.77450904,303.46785079)
\curveto(337.14950902,304.31160149)(336.837009,305.25951752)(336.837009,306.30118422)
\curveto(336.837009,307.07722558)(337.02450901,307.84285095)(337.39950902,308.59285097)
\curveto(337.78492637,309.35326833)(338.28492639,309.91576835)(338.89950908,310.28035103)
\curveto(339.5245091,310.63972571)(340.18075979,310.82201772)(340.87867581,310.82201772)
\closepath
\moveto(340.58700914,310.21785103)
\curveto(340.26409313,310.21785103)(339.94117578,310.11889236)(339.60784244,309.92618435)
\curveto(339.28492642,309.74389235)(339.02971708,309.410559)(338.83700908,308.92618432)
\curveto(338.63909307,308.43660163)(338.5453424,307.82201761)(338.5453424,307.07201758)
\curveto(338.5453424,305.86368421)(338.77971707,304.8116015)(339.25367576,303.92618414)
\curveto(339.73805044,303.05118411)(340.3786758,302.61368409)(341.17034249,302.61368409)
\curveto(341.75367585,302.61368409)(342.23805053,302.8532681)(342.62867588,303.34285079)
\curveto(343.01409322,303.82722547)(343.21200923,304.66055883)(343.21200923,305.84285087)
\curveto(343.21200923,307.32722559)(342.88909322,308.4938923)(342.25367586,309.342851)
\curveto(341.82138385,309.92618435)(341.26409316,310.21785103)(340.58700914,310.21785103)
\closepath
\moveto(340.58700914,310.21785103)
}
}
{
\newrgbcolor{curcolor}{0 0 0}
\pscustom[linestyle=none,fillstyle=solid,fillcolor=curcolor]
{
\newpath
\moveto(353.21200958,305.38451752)
\curveto(352.98805091,304.28555882)(352.55055089,303.43660146)(351.89950954,302.84285077)
\curveto(351.24326018,302.25951741)(350.51409349,301.96785074)(349.71200946,301.96785074)
\curveto(348.76409343,301.96785074)(347.94117606,302.36368408)(347.23284271,303.15535078)
\curveto(346.53492668,303.94701747)(346.191176,305.01993484)(346.191176,306.38451756)
\curveto(346.191176,307.68660161)(346.57659335,308.74910164)(347.35784271,309.57201767)
\curveto(348.13388407,310.40535104)(349.0713841,310.82201772)(350.17034281,310.82201772)
\curveto(350.98805084,310.82201772)(351.65992686,310.59805904)(352.19117622,310.15535103)
\curveto(352.71721757,309.72305901)(352.98284291,309.269935)(352.98284291,308.80118431)
\curveto(352.98284291,308.57722564)(352.90471757,308.38972563)(352.75367624,308.23868429)
\curveto(352.6130509,308.09805895)(352.40471756,308.03035095)(352.12867621,308.03035095)
\curveto(351.77971753,308.03035095)(351.50888419,308.14493496)(351.31617618,308.38451763)
\curveto(351.21721751,308.50951764)(351.14950951,308.74910164)(351.10784284,309.11368432)
\curveto(351.07659351,309.473059)(350.96200951,309.74389235)(350.75367616,309.92618435)
\curveto(350.54534282,310.10326836)(350.24326015,310.1970177)(349.8578428,310.1970177)
\curveto(349.25888411,310.1970177)(348.77450943,309.97305902)(348.39950941,309.530351)
\curveto(347.90992673,308.93139232)(347.67034272,308.14493496)(347.67034272,307.17618425)
\curveto(347.67034272,306.17618422)(347.90992673,305.29076819)(348.39950941,304.53035083)
\curveto(348.8838841,303.76472547)(349.54534279,303.38451745)(350.37867615,303.38451745)
\curveto(350.97242684,303.38451745)(351.50888419,303.58243479)(351.98284287,303.98868414)
\curveto(352.31617622,304.26472548)(352.63909356,304.76993484)(352.96200958,305.50951753)
\closepath
\moveto(353.21200958,305.38451752)
}
}
{
\newrgbcolor{curcolor}{0 0 0}
\pscustom[linestyle=none,fillstyle=solid,fillcolor=curcolor]
{
\newpath
\moveto(356.89584761,315.17618454)
\lineto(356.89584761,306.88451758)
\lineto(359.02084768,308.80118431)
\curveto(359.4635557,309.21785099)(359.71876504,309.47826834)(359.79168104,309.59285101)
\curveto(359.84376504,309.66055901)(359.87501438,309.72826835)(359.87501438,309.80118435)
\curveto(359.87501438,309.92618435)(359.81772238,310.03035102)(359.70834771,310.11368436)
\curveto(359.60938904,310.20743503)(359.4479317,310.2647257)(359.22918102,310.28035103)
\lineto(359.22918102,310.57201771)
\lineto(362.85418115,310.57201771)
\lineto(362.85418115,310.28035103)
\curveto(362.35418113,310.2647257)(361.93751445,310.18660169)(361.60418111,310.05118436)
\curveto(361.27084776,309.91055902)(360.90626508,309.66055901)(360.52084773,309.30118433)
\lineto(358.37501433,307.32201759)
\lineto(360.52084773,304.61368416)
\curveto(361.11459842,303.86368414)(361.51043177,303.38972545)(361.70834778,303.19701745)
\curveto(362.00001445,302.90535077)(362.25522246,302.71785076)(362.47918114,302.63451743)
\curveto(362.63022248,302.57722543)(362.89584782,302.55118409)(363.27084783,302.55118409)
\lineto(363.27084783,302.21785075)
\lineto(359.22918102,302.21785075)
\lineto(359.22918102,302.55118409)
\curveto(359.4479317,302.55118409)(359.60418104,302.57722543)(359.68751437,302.63451743)
\curveto(359.77084771,302.70222543)(359.81251438,302.8011841)(359.81251438,302.9261841)
\curveto(359.81251438,303.06160144)(359.69272237,303.28555878)(359.4583477,303.59285079)
\lineto(356.89584761,306.86368424)
\lineto(356.89584761,304.09285081)
\curveto(356.89584761,303.55118413)(356.92709827,303.19701745)(357.00001428,303.03035077)
\curveto(357.08334761,302.8636841)(357.19272228,302.7386841)(357.33334762,302.65535076)
\curveto(357.46876496,302.58243476)(357.77084764,302.55118409)(358.22918099,302.55118409)
\lineto(358.22918099,302.21785075)
\lineto(353.97918084,302.21785075)
\lineto(353.97918084,302.55118409)
\curveto(354.40626485,302.55118409)(354.72918086,302.59805876)(354.9375142,302.69701743)
\curveto(355.07293154,302.76472543)(355.17188888,302.8688921)(355.22918088,303.00951744)
\curveto(355.32293155,303.21785078)(355.37501422,303.56160146)(355.37501422,304.05118414)
\lineto(355.37501422,311.65535108)
\curveto(355.37501422,312.61368445)(355.35418088,313.1970178)(355.31251422,313.40535114)
\curveto(355.27084755,313.62410182)(355.19793155,313.78035116)(355.10418088,313.86368449)
\curveto(355.00522221,313.94701783)(354.8750142,313.9886845)(354.70834753,313.9886845)
\curveto(354.58334752,313.9886845)(354.39584752,313.93660183)(354.14584751,313.84285116)
\lineto(353.97918084,314.15535117)
\lineto(356.45834759,315.17618454)
\closepath
\moveto(356.89584761,315.17618454)
}
}
{
\newrgbcolor{curcolor}{0 0 0}
\pscustom[linestyle=none,fillstyle=solid,fillcolor=curcolor]
{
\newpath
\moveto(294.99626859,255.52613767)
\curveto(296.25668597,255.52613767)(297.27231,255.03655498)(298.03793536,254.06780428)
\curveto(298.68897672,253.24488825)(299.01710207,252.30738822)(299.01710207,251.25530418)
\curveto(299.01710207,250.50530415)(298.83481006,249.74488813)(298.47543538,248.98447077)
\curveto(298.1108527,248.21884541)(297.62126868,247.64072139)(296.99626866,247.25530404)
\curveto(296.37126864,246.86988736)(295.66814328,246.67197069)(294.89210192,246.67197069)
\curveto(293.64210188,246.67197069)(292.64210184,247.1719707)(291.89210181,248.17197074)
\curveto(291.26710179,249.01572143)(290.95460178,249.96363747)(290.95460178,251.00530417)
\curveto(290.95460178,251.78134553)(291.14210179,252.54697089)(291.5171018,253.29697092)
\curveto(291.90251915,254.05738828)(292.40251916,254.6198883)(293.01710185,254.98447098)
\curveto(293.64210188,255.34384566)(294.29835257,255.52613767)(294.99626859,255.52613767)
\closepath
\moveto(294.70460191,254.92197098)
\curveto(294.3816859,254.92197098)(294.05876856,254.82301231)(293.72543521,254.6303043)
\curveto(293.4025192,254.44801229)(293.14730986,254.11467895)(292.95460185,253.63030427)
\curveto(292.75668584,253.14072158)(292.66293517,252.52613756)(292.66293517,251.77613753)
\curveto(292.66293517,250.56780416)(292.89730985,249.51572145)(293.37126853,248.63030409)
\curveto(293.85564322,247.75530406)(294.49626857,247.31780404)(295.28793527,247.31780404)
\curveto(295.87126862,247.31780404)(296.3556433,247.55738805)(296.74626865,248.04697073)
\curveto(297.131686,248.53134542)(297.32960201,249.36467878)(297.32960201,250.54697082)
\curveto(297.32960201,252.03134554)(297.00668599,253.19801225)(296.37126864,254.04697095)
\curveto(295.93897662,254.6303043)(295.38168594,254.92197098)(294.70460191,254.92197098)
\closepath
\moveto(294.70460191,254.92197098)
}
}
{
\newrgbcolor{curcolor}{0 0 0}
\pscustom[linestyle=none,fillstyle=solid,fillcolor=curcolor]
{
\newpath
\moveto(302.53793552,253.83863761)
\curveto(303.34001955,254.96363765)(304.20981025,255.52613767)(305.14210228,255.52613767)
\curveto(306.00147698,255.52613767)(306.75147701,255.15634565)(307.39210236,254.42197096)
\curveto(308.02751972,253.68238827)(308.35043573,252.68238823)(308.35043573,251.42197085)
\curveto(308.35043573,249.93238813)(307.85564371,248.73967876)(306.87126901,247.83863739)
\curveto(306.02231031,247.05738803)(305.07960228,246.67197069)(304.03793558,246.67197069)
\curveto(303.54835289,246.67197069)(303.05876888,246.75530402)(302.55876886,246.92197069)
\curveto(302.05876884,247.0990547)(301.54835282,247.36988804)(301.03793547,247.73447072)
\lineto(301.03793547,256.38030436)
\curveto(301.03793547,257.3230124)(301.00668614,257.90113775)(300.95460213,258.10947109)
\curveto(300.91293547,258.32822176)(300.84001946,258.4844711)(300.74626879,258.56780444)
\curveto(300.64731012,258.65113778)(300.52751945,258.69280444)(300.39210211,258.69280444)
\curveto(300.20981011,258.69280444)(299.99626877,258.64072178)(299.74626876,258.54697111)
\lineto(299.62126875,258.85947112)
\lineto(302.12126884,259.88030449)
\lineto(302.53793552,259.88030449)
\closepath
\moveto(302.53793552,253.25530425)
\lineto(302.53793552,248.25530407)
\curveto(302.84001953,247.94801206)(303.16293555,247.71884539)(303.49626889,247.56780405)
\curveto(303.82960224,247.41155471)(304.16814358,247.33863738)(304.51710226,247.33863738)
\curveto(305.06918628,247.33863738)(305.5848103,247.63551205)(306.05876898,248.23447074)
\curveto(306.52751967,248.84384543)(306.76710234,249.73447079)(306.76710234,250.9011375)
\curveto(306.76710234,251.96884554)(306.52751967,252.7865549)(306.05876898,253.35947092)
\curveto(305.5848103,253.92717894)(305.04314361,254.21363762)(304.43376892,254.21363762)
\curveto(304.11085291,254.21363762)(303.78793557,254.13551228)(303.45460222,253.98447094)
\curveto(303.21501955,253.85947094)(302.91293554,253.61467893)(302.53793552,253.25530425)
\closepath
\moveto(302.53793552,253.25530425)
}
}
{
\newrgbcolor{curcolor}{0 0 0}
\pscustom[linestyle=none,fillstyle=solid,fillcolor=curcolor]
{
\newpath
\moveto(311.70460252,259.90113782)
\curveto(311.95460252,259.90113782)(312.16814387,259.80217915)(312.35043587,259.60947114)
\curveto(312.54314388,259.42717914)(312.64210255,259.2136378)(312.64210255,258.96363779)
\curveto(312.64210255,258.69801244)(312.54314388,258.4740551)(312.35043587,258.2969711)
\curveto(312.16814387,258.11467909)(311.95460252,258.02613775)(311.70460252,258.02613775)
\curveto(311.43897717,258.02613775)(311.21501983,258.11467909)(311.03793582,258.2969711)
\curveto(310.85564382,258.4740551)(310.76710248,258.69801244)(310.76710248,258.96363779)
\curveto(310.76710248,259.2136378)(310.85564382,259.42717914)(311.03793582,259.60947114)
\curveto(311.21501983,259.80217915)(311.43897717,259.90113782)(311.70460252,259.90113782)
\closepath
\moveto(312.47543588,255.52613767)
\lineto(312.47543588,247.1094707)
\curveto(312.47543588,245.68238665)(312.16814387,244.62509595)(311.55876918,243.94280392)
\curveto(310.95981049,243.2500959)(310.17335313,242.90113722)(309.20460243,242.90113722)
\curveto(308.65251841,242.90113722)(308.24106106,243.00009589)(307.97543572,243.1928039)
\curveto(307.69939437,243.3907199)(307.55876904,243.59905324)(307.55876904,243.81780392)
\curveto(307.55876904,244.02613726)(307.63689437,244.20842926)(307.78793571,244.3594706)
\curveto(307.94418505,244.51571994)(308.11606105,244.58863728)(308.30876906,244.58863728)
\curveto(308.47543573,244.58863728)(308.63689441,244.54697061)(308.78793575,244.46363727)
\curveto(308.90251908,244.42197061)(309.10043576,244.27092927)(309.39210243,244.00530392)
\curveto(309.68376911,243.74488658)(309.93376912,243.60947058)(310.14210246,243.60947058)
\curveto(310.2775198,243.60947058)(310.4181438,243.66676258)(310.55876914,243.77613725)
\curveto(310.69418648,243.89071992)(310.79835315,244.07821993)(310.87126915,244.33863727)
\curveto(310.93897715,244.60426261)(310.97543582,245.18238663)(310.97543582,246.067804)
\lineto(310.97543582,252.00530421)
\curveto(310.97543582,252.92197091)(310.94418649,253.51051226)(310.89210249,253.7761376)
\curveto(310.85043582,253.96884561)(310.77751982,254.09905495)(310.68376915,254.17197095)
\curveto(310.60043581,254.25530429)(310.47543581,254.29697096)(310.30876913,254.29697096)
\curveto(310.14210246,254.29697096)(309.93897712,254.25530429)(309.70460244,254.17197095)
\lineto(309.57960244,254.5053043)
\lineto(312.07960253,255.52613767)
\closepath
\moveto(312.47543588,255.52613767)
}
}
{
\newrgbcolor{curcolor}{0 0 0}
\pscustom[linestyle=none,fillstyle=solid,fillcolor=curcolor]
{
\newpath
\moveto(316.16476706,252.13030421)
\curveto(316.16476706,250.8907215)(316.46685107,249.9219708)(317.08143376,249.21363744)
\curveto(317.67518445,248.50530408)(318.39393381,248.1511374)(319.22726717,248.1511374)
\curveto(319.76893386,248.1511374)(320.23768454,248.29697074)(320.64393389,248.58863742)
\curveto(321.04497523,248.89072143)(321.38351791,249.41155478)(321.66476726,250.15113748)
\lineto(321.9356006,249.96363747)
\curveto(321.81060059,249.13030411)(321.44080858,248.36988808)(320.83143389,247.69280405)
\curveto(320.2168512,247.01051243)(319.44601784,246.67197069)(318.51893381,246.67197069)
\curveto(317.51893378,246.67197069)(316.65435108,247.05738803)(315.93560039,247.83863739)
\curveto(315.22726703,248.61467875)(314.87310035,249.66155479)(314.87310035,250.98447084)
\curveto(314.87310035,252.41155489)(315.23768436,253.52613759)(315.97726705,254.31780429)
\curveto(316.71164175,255.11988832)(317.62830845,255.52613767)(318.72726715,255.52613767)
\curveto(319.66997519,255.52613767)(320.44080855,255.21363765)(321.03976723,254.58863763)
\curveto(321.63351792,253.97405494)(321.9356006,253.15634558)(321.9356006,252.13030421)
\closepath
\moveto(316.16476706,252.65113756)
\lineto(320.0397672,252.65113756)
\curveto(320.00851786,253.19280425)(319.94601786,253.56780426)(319.85226719,253.7761376)
\curveto(319.69601785,254.11988828)(319.46685118,254.39072163)(319.16476717,254.58863763)
\curveto(318.85747516,254.78134564)(318.54497515,254.88030431)(318.22726713,254.88030431)
\curveto(317.72726712,254.88030431)(317.27414177,254.6823883)(316.87310042,254.29697096)
\curveto(316.46685107,253.90634561)(316.23247506,253.35947092)(316.16476706,252.65113756)
\closepath
\moveto(316.16476706,252.65113756)
}
}
{
\newrgbcolor{curcolor}{0 0 0}
\pscustom[linestyle=none,fillstyle=solid,fillcolor=curcolor]
{
\newpath
\moveto(330.14027211,250.08863747)
\curveto(329.91631344,248.98967877)(329.47881342,248.1407214)(328.82777207,247.54697072)
\curveto(328.17152271,246.96363736)(327.44235602,246.67197069)(326.64027199,246.67197069)
\curveto(325.69235596,246.67197069)(324.86943859,247.06780403)(324.16110524,247.85947073)
\curveto(323.46318921,248.65113742)(323.11943853,249.72405479)(323.11943853,251.08863751)
\curveto(323.11943853,252.39072155)(323.50485588,253.45322159)(324.28610524,254.27613762)
\curveto(325.0621466,255.10947098)(325.99964663,255.52613767)(327.09860534,255.52613767)
\curveto(327.91631337,255.52613767)(328.58818939,255.30217899)(329.11943874,254.85947098)
\curveto(329.6454801,254.42717896)(329.91110544,253.97405494)(329.91110544,253.50530426)
\curveto(329.91110544,253.28134559)(329.8329801,253.09384558)(329.68193876,252.94280424)
\curveto(329.54131343,252.8021789)(329.33298009,252.7344709)(329.05693874,252.7344709)
\curveto(328.70798006,252.7344709)(328.43714672,252.8490549)(328.24443871,253.08863758)
\curveto(328.14548004,253.21363758)(328.07777204,253.45322159)(328.03610537,253.81780427)
\curveto(328.00485604,254.17717895)(327.89027203,254.44801229)(327.68193869,254.6303043)
\curveto(327.47360535,254.80738831)(327.17152268,254.90113764)(326.78610533,254.90113764)
\curveto(326.18714664,254.90113764)(325.70277196,254.67717897)(325.32777194,254.23447095)
\curveto(324.83818926,253.63551227)(324.59860525,252.8490549)(324.59860525,251.8803042)
\curveto(324.59860525,250.88030417)(324.83818926,249.99488814)(325.32777194,249.23447078)
\curveto(325.81214663,248.46884542)(326.47360532,248.0886374)(327.30693868,248.0886374)
\curveto(327.90068937,248.0886374)(328.43714672,248.28655474)(328.9111054,248.69280409)
\curveto(329.24443875,248.96884543)(329.56735609,249.47405478)(329.8902721,250.21363748)
\closepath
\moveto(330.14027211,250.08863747)
}
}
{
\newrgbcolor{curcolor}{0 0 0}
\pscustom[linestyle=none,fillstyle=solid,fillcolor=curcolor]
{
\newpath
\moveto(333.76161013,258.00530442)
\lineto(333.76161013,255.27613766)
\lineto(335.71994354,255.27613766)
\lineto(335.71994354,254.6303043)
\lineto(333.76161013,254.6303043)
\lineto(333.76161013,249.21363744)
\curveto(333.76161013,248.67197076)(333.83452747,248.30217874)(333.99077681,248.10947074)
\curveto(334.15744348,247.92717873)(334.35536082,247.83863739)(334.5949435,247.83863739)
\curveto(334.80327684,247.83863739)(334.99598484,247.9011374)(335.17827685,248.0261374)
\curveto(335.37098486,248.1511374)(335.51681819,248.33863741)(335.61577687,248.58863742)
\lineto(335.96994354,248.58863742)
\curveto(335.7616102,247.98967873)(335.45952753,247.53655472)(335.07411018,247.23447071)
\curveto(334.68348483,246.92717903)(334.28244348,246.77613736)(333.8657768,246.77613736)
\curveto(333.58452746,246.77613736)(333.31369412,246.85426229)(333.05327677,247.00530403)
\curveto(332.78765143,247.1719707)(332.58452742,247.39072138)(332.44911009,247.67197072)
\curveto(332.32411008,247.9636374)(332.26161008,248.41155475)(332.26161008,249.02613744)
\lineto(332.26161008,254.6303043)
\lineto(330.94911003,254.6303043)
\lineto(330.94911003,254.94280431)
\curveto(331.28244338,255.06780432)(331.62098472,255.28655499)(331.9699434,255.609471)
\curveto(332.31369408,255.92717901)(332.62619409,256.30738836)(332.90744344,256.75530438)
\curveto(333.04286077,256.97405505)(333.24077678,257.39072173)(333.49077679,258.00530442)
\closepath
\moveto(333.76161013,258.00530442)
}
}
{
\newrgbcolor{curcolor}{0 0 0}
\pscustom[linestyle=none,fillstyle=solid,fillcolor=curcolor]
{
\newpath
\moveto(346.59677498,255.52613767)
\lineto(346.59677498,252.6719709)
\lineto(346.3051083,252.6719709)
\curveto(346.06552562,253.55738826)(345.76864961,254.16155495)(345.40927493,254.48447096)
\curveto(345.04469225,254.81780431)(344.5863589,254.98447098)(344.03427488,254.98447098)
\curveto(343.60198287,254.98447098)(343.25302552,254.86467898)(342.99260818,254.6303043)
\curveto(342.72698284,254.40634563)(342.59677483,254.16155495)(342.59677483,253.90113761)
\curveto(342.59677483,253.55217893)(342.6905255,253.26051225)(342.88844151,253.02613758)
\curveto(343.08114952,252.77613757)(343.47177486,252.51051223)(344.05510822,252.23447088)
\lineto(345.36760827,251.58863753)
\curveto(346.60198298,250.97405484)(347.221775,250.17717881)(347.221775,249.19280411)
\curveto(347.221775,248.44280408)(346.93531632,247.82822139)(346.3676083,247.35947071)
\curveto(345.79469228,246.90113736)(345.14885892,246.67197069)(344.43010823,246.67197069)
\curveto(343.93010821,246.67197069)(343.34677486,246.76572002)(342.68010817,246.94280403)
\curveto(342.48219216,246.99488736)(342.32594149,247.02613736)(342.20094149,247.02613736)
\curveto(342.06031615,247.02613736)(341.95094148,246.94801243)(341.86760814,246.79697069)
\lineto(341.57594146,246.79697069)
\lineto(341.57594146,249.77613746)
\lineto(341.86760814,249.77613746)
\curveto(342.03427481,248.92717877)(342.35198283,248.28655474)(342.82594151,247.85947073)
\curveto(343.31031619,247.42717871)(343.85198288,247.21363737)(344.45094157,247.21363737)
\curveto(344.86760825,247.21363737)(345.20614959,247.32822138)(345.47177494,247.56780405)
\curveto(345.73219228,247.81780406)(345.86760828,248.11467874)(345.86760828,248.46363742)
\curveto(345.86760828,248.8803041)(345.72177494,249.22405478)(345.43010827,249.50530412)
\curveto(345.13844159,249.7969708)(344.54469224,250.15634548)(343.65927487,250.58863749)
\curveto(342.78427484,251.03134551)(342.20614949,251.43238819)(341.93010814,251.79697087)
\curveto(341.6644828,252.14072155)(341.5342748,252.57822156)(341.5342748,253.10947091)
\curveto(341.5342748,253.78655494)(341.76864947,254.35947096)(342.24260815,254.81780431)
\curveto(342.71135884,255.28655499)(343.31552553,255.52613767)(344.05510822,255.52613767)
\curveto(344.38844156,255.52613767)(344.78948291,255.45322166)(345.26344159,255.31780432)
\curveto(345.56552561,255.21884565)(345.76864961,255.17197099)(345.86760828,255.17197099)
\curveto(345.96135895,255.17197099)(346.03948296,255.19280432)(346.09677496,255.23447099)
\curveto(346.14885896,255.27613766)(346.22177496,255.36988833)(346.3051083,255.52613767)
\closepath
\moveto(346.59677498,255.52613767)
}
}
{
\newrgbcolor{curcolor}{0 0 0}
\pscustom[linestyle=none,fillstyle=solid,fillcolor=curcolor]
{
\newpath
\moveto(349.85812763,252.13030421)
\curveto(349.85812763,250.8907215)(350.16021164,249.9219708)(350.77479433,249.21363744)
\curveto(351.36854502,248.50530408)(352.08729438,248.1511374)(352.92062774,248.1511374)
\curveto(353.46229442,248.1511374)(353.93104511,248.29697074)(354.33729446,248.58863742)
\curveto(354.7383358,248.89072143)(355.07687848,249.41155478)(355.35812782,250.15113748)
\lineto(355.62896117,249.96363747)
\curveto(355.50396116,249.13030411)(355.13416915,248.36988808)(354.52479446,247.69280405)
\curveto(353.91021177,247.01051243)(353.13937841,246.67197069)(352.21229438,246.67197069)
\curveto(351.21229434,246.67197069)(350.34771165,247.05738803)(349.62896096,247.83863739)
\curveto(348.9206276,248.61467875)(348.56646092,249.66155479)(348.56646092,250.98447084)
\curveto(348.56646092,252.41155489)(348.93104493,253.52613759)(349.67062762,254.31780429)
\curveto(350.40500232,255.11988832)(351.32166902,255.52613767)(352.42062772,255.52613767)
\curveto(353.36333575,255.52613767)(354.13416911,255.21363765)(354.7331278,254.58863763)
\curveto(355.32687849,253.97405494)(355.62896117,253.15634558)(355.62896117,252.13030421)
\closepath
\moveto(349.85812763,252.65113756)
\lineto(353.73312777,252.65113756)
\curveto(353.70187843,253.19280425)(353.63937843,253.56780426)(353.54562776,253.7761376)
\curveto(353.38937842,254.11988828)(353.16021175,254.39072163)(352.85812774,254.58863763)
\curveto(352.55083573,254.78134564)(352.23833571,254.88030431)(351.9206277,254.88030431)
\curveto(351.42062769,254.88030431)(350.96750234,254.6823883)(350.56646099,254.29697096)
\curveto(350.16021164,253.90634561)(349.92583563,253.35947092)(349.85812763,252.65113756)
\closepath
\moveto(349.85812763,252.65113756)
}
}
{
\newrgbcolor{curcolor}{0 0 0}
\pscustom[linestyle=none,fillstyle=solid,fillcolor=curcolor]
{
\newpath
\moveto(359.16696585,258.00530442)
\lineto(359.16696585,255.27613766)
\lineto(361.12529925,255.27613766)
\lineto(361.12529925,254.6303043)
\lineto(359.16696585,254.6303043)
\lineto(359.16696585,249.21363744)
\curveto(359.16696585,248.67197076)(359.23988319,248.30217874)(359.39613253,248.10947074)
\curveto(359.5627992,247.92717873)(359.76071654,247.83863739)(360.00029921,247.83863739)
\curveto(360.20863255,247.83863739)(360.40134056,247.9011374)(360.58363257,248.0261374)
\curveto(360.77634057,248.1511374)(360.92217391,248.33863741)(361.02113258,248.58863742)
\lineto(361.37529926,248.58863742)
\curveto(361.16696592,247.98967873)(360.86488324,247.53655472)(360.4794659,247.23447071)
\curveto(360.08884055,246.92717903)(359.6877992,246.77613736)(359.27113252,246.77613736)
\curveto(358.98988318,246.77613736)(358.71904983,246.85426229)(358.45863249,247.00530403)
\curveto(358.19300715,247.1719707)(357.98988314,247.39072138)(357.8544658,247.67197072)
\curveto(357.7294658,247.9636374)(357.6669658,248.41155475)(357.6669658,249.02613744)
\lineto(357.6669658,254.6303043)
\lineto(356.35446575,254.6303043)
\lineto(356.35446575,254.94280431)
\curveto(356.6877991,255.06780432)(357.02634044,255.28655499)(357.37529912,255.609471)
\curveto(357.7190498,255.92717901)(358.03154981,256.30738836)(358.31279915,256.75530438)
\curveto(358.44821649,256.97405505)(358.6461325,257.39072173)(358.89613251,258.00530442)
\closepath
\moveto(359.16696585,258.00530442)
}
}
{
\newrgbcolor{curcolor}{0 0 0}
\pscustom[linestyle=none,fillstyle=solid,fillcolor=curcolor]
{
\newpath
\moveto(294.99626859,135.52615376)
\curveto(296.25668597,135.52615376)(297.27231,135.03657107)(298.03793536,134.06782037)
\curveto(298.68897672,133.24490434)(299.01710207,132.30740431)(299.01710207,131.25532027)
\curveto(299.01710207,130.50532025)(298.83481006,129.74490422)(298.47543538,128.98448686)
\curveto(298.1108527,128.2188615)(297.62126868,127.64073748)(296.99626866,127.25532013)
\curveto(296.37126864,126.86990345)(295.66814328,126.67198678)(294.89210192,126.67198678)
\curveto(293.64210188,126.67198678)(292.64210184,127.1719868)(291.89210181,128.17198683)
\curveto(291.26710179,129.01573753)(290.95460178,129.96365356)(290.95460178,131.00532027)
\curveto(290.95460178,131.78136163)(291.14210179,132.54698699)(291.5171018,133.29698701)
\curveto(291.90251915,134.05740437)(292.40251916,134.61990439)(293.01710185,134.98448707)
\curveto(293.64210188,135.34386175)(294.29835257,135.52615376)(294.99626859,135.52615376)
\closepath
\moveto(294.70460191,134.92198707)
\curveto(294.3816859,134.92198707)(294.05876856,134.8230284)(293.72543521,134.63032039)
\curveto(293.4025192,134.44802839)(293.14730986,134.11469504)(292.95460185,133.63032036)
\curveto(292.75668584,133.14073767)(292.66293517,132.52615365)(292.66293517,131.77615363)
\curveto(292.66293517,130.56782025)(292.89730985,129.51573755)(293.37126853,128.63032018)
\curveto(293.85564322,127.75532015)(294.49626857,127.31782013)(295.28793527,127.31782013)
\curveto(295.87126862,127.31782013)(296.3556433,127.55740414)(296.74626865,128.04698683)
\curveto(297.131686,128.53136151)(297.32960201,129.36469487)(297.32960201,130.54698692)
\curveto(297.32960201,132.03136164)(297.00668599,133.19802834)(296.37126864,134.04698704)
\curveto(295.93897662,134.63032039)(295.38168594,134.92198707)(294.70460191,134.92198707)
\closepath
\moveto(294.70460191,134.92198707)
}
}
{
\newrgbcolor{curcolor}{0 0 0}
\pscustom[linestyle=none,fillstyle=solid,fillcolor=curcolor]
{
\newpath
\moveto(302.53793552,133.8386537)
\curveto(303.34001955,134.96365374)(304.20981025,135.52615376)(305.14210228,135.52615376)
\curveto(306.00147698,135.52615376)(306.75147701,135.15636175)(307.39210236,134.42198705)
\curveto(308.02751972,133.68240436)(308.35043573,132.68240432)(308.35043573,131.42198695)
\curveto(308.35043573,129.93240423)(307.85564371,128.73969485)(306.87126901,127.83865349)
\curveto(306.02231031,127.05740413)(305.07960228,126.67198678)(304.03793558,126.67198678)
\curveto(303.54835289,126.67198678)(303.05876888,126.75532011)(302.55876886,126.92198679)
\curveto(302.05876884,127.09907079)(301.54835282,127.36990414)(301.03793547,127.73448682)
\lineto(301.03793547,136.38032046)
\curveto(301.03793547,137.32302849)(301.00668614,137.90115384)(300.95460213,138.10948718)
\curveto(300.91293547,138.32823786)(300.84001946,138.4844872)(300.74626879,138.56782053)
\curveto(300.64731012,138.65115387)(300.52751945,138.69282054)(300.39210211,138.69282054)
\curveto(300.20981011,138.69282054)(299.99626877,138.64073787)(299.74626876,138.5469872)
\lineto(299.62126875,138.85948721)
\lineto(302.12126884,139.88032058)
\lineto(302.53793552,139.88032058)
\closepath
\moveto(302.53793552,133.25532035)
\lineto(302.53793552,128.25532017)
\curveto(302.84001953,127.94802816)(303.16293555,127.71886148)(303.49626889,127.56782014)
\curveto(303.82960224,127.4115708)(304.16814358,127.33865347)(304.51710226,127.33865347)
\curveto(305.06918628,127.33865347)(305.5848103,127.63552815)(306.05876898,128.23448683)
\curveto(306.52751967,128.84386152)(306.76710234,129.73448689)(306.76710234,130.9011536)
\curveto(306.76710234,131.96886163)(306.52751967,132.786571)(306.05876898,133.35948702)
\curveto(305.5848103,133.92719504)(305.04314361,134.21365371)(304.43376892,134.21365371)
\curveto(304.11085291,134.21365371)(303.78793557,134.13552838)(303.45460222,133.98448704)
\curveto(303.21501955,133.85948703)(302.91293554,133.61469502)(302.53793552,133.25532035)
\closepath
\moveto(302.53793552,133.25532035)
}
}
{
\newrgbcolor{curcolor}{0 0 0}
\pscustom[linestyle=none,fillstyle=solid,fillcolor=curcolor]
{
\newpath
\moveto(311.70460252,139.90115391)
\curveto(311.95460252,139.90115391)(312.16814387,139.80219524)(312.35043587,139.60948724)
\curveto(312.54314388,139.42719523)(312.64210255,139.21365389)(312.64210255,138.96365388)
\curveto(312.64210255,138.69802854)(312.54314388,138.4740712)(312.35043587,138.29698719)
\curveto(312.16814387,138.11469518)(311.95460252,138.02615385)(311.70460252,138.02615385)
\curveto(311.43897717,138.02615385)(311.21501983,138.11469518)(311.03793582,138.29698719)
\curveto(310.85564382,138.4740712)(310.76710248,138.69802854)(310.76710248,138.96365388)
\curveto(310.76710248,139.21365389)(310.85564382,139.42719523)(311.03793582,139.60948724)
\curveto(311.21501983,139.80219524)(311.43897717,139.90115391)(311.70460252,139.90115391)
\closepath
\moveto(312.47543588,135.52615376)
\lineto(312.47543588,127.10948679)
\curveto(312.47543588,125.68240274)(312.16814387,124.62511204)(311.55876918,123.94282002)
\curveto(310.95981049,123.25011199)(310.17335313,122.90115331)(309.20460243,122.90115331)
\curveto(308.65251841,122.90115331)(308.24106106,123.00011198)(307.97543572,123.19281999)
\curveto(307.69939437,123.390736)(307.55876904,123.59906934)(307.55876904,123.81782001)
\curveto(307.55876904,124.02615335)(307.63689437,124.20844536)(307.78793571,124.3594867)
\curveto(307.94418505,124.51573604)(308.11606105,124.58865337)(308.30876906,124.58865337)
\curveto(308.47543573,124.58865337)(308.63689441,124.5469867)(308.78793575,124.46365337)
\curveto(308.90251908,124.4219867)(309.10043576,124.27094536)(309.39210243,124.00532002)
\curveto(309.68376911,123.74490267)(309.93376912,123.60948667)(310.14210246,123.60948667)
\curveto(310.2775198,123.60948667)(310.4181438,123.66677867)(310.55876914,123.77615334)
\curveto(310.69418648,123.89073601)(310.79835315,124.07823602)(310.87126915,124.33865336)
\curveto(310.93897715,124.60427871)(310.97543582,125.18240273)(310.97543582,126.06782009)
\lineto(310.97543582,132.0053203)
\curveto(310.97543582,132.921987)(310.94418649,133.51052835)(310.89210249,133.7761537)
\curveto(310.85043582,133.9688617)(310.77751982,134.09907104)(310.68376915,134.17198704)
\curveto(310.60043581,134.25532038)(310.47543581,134.29698705)(310.30876913,134.29698705)
\curveto(310.14210246,134.29698705)(309.93897712,134.25532038)(309.70460244,134.17198704)
\lineto(309.57960244,134.50532039)
\lineto(312.07960253,135.52615376)
\closepath
\moveto(312.47543588,135.52615376)
}
}
{
\newrgbcolor{curcolor}{0 0 0}
\pscustom[linestyle=none,fillstyle=solid,fillcolor=curcolor]
{
\newpath
\moveto(316.16476706,132.13032031)
\curveto(316.16476706,130.89073759)(316.46685107,129.92198689)(317.08143376,129.21365354)
\curveto(317.67518445,128.50532018)(318.39393381,128.1511535)(319.22726717,128.1511535)
\curveto(319.76893386,128.1511535)(320.23768454,128.29698684)(320.64393389,128.58865351)
\curveto(321.04497523,128.89073752)(321.38351791,129.41157088)(321.66476726,130.15115357)
\lineto(321.9356006,129.96365356)
\curveto(321.81060059,129.1303202)(321.44080858,128.36990417)(320.83143389,127.69282015)
\curveto(320.2168512,127.01052852)(319.44601784,126.67198678)(318.51893381,126.67198678)
\curveto(317.51893378,126.67198678)(316.65435108,127.05740413)(315.93560039,127.83865349)
\curveto(315.22726703,128.61469485)(314.87310035,129.66157088)(314.87310035,130.98448693)
\curveto(314.87310035,132.41157098)(315.23768436,133.52615369)(315.97726705,134.31782038)
\curveto(316.71164175,135.11990441)(317.62830845,135.52615376)(318.72726715,135.52615376)
\curveto(319.66997519,135.52615376)(320.44080855,135.21365375)(321.03976723,134.58865373)
\curveto(321.63351792,133.97407104)(321.9356006,133.15636168)(321.9356006,132.13032031)
\closepath
\moveto(316.16476706,132.65115366)
\lineto(320.0397672,132.65115366)
\curveto(320.00851786,133.19282034)(319.94601786,133.56782036)(319.85226719,133.7761537)
\curveto(319.69601785,134.11990438)(319.46685118,134.39073772)(319.16476717,134.58865373)
\curveto(318.85747516,134.78136173)(318.54497515,134.8803204)(318.22726713,134.8803204)
\curveto(317.72726712,134.8803204)(317.27414177,134.6824044)(316.87310042,134.29698705)
\curveto(316.46685107,133.9063617)(316.23247506,133.35948702)(316.16476706,132.65115366)
\closepath
\moveto(316.16476706,132.65115366)
}
}
{
\newrgbcolor{curcolor}{0 0 0}
\pscustom[linestyle=none,fillstyle=solid,fillcolor=curcolor]
{
\newpath
\moveto(330.14027211,130.08865357)
\curveto(329.91631344,128.98969486)(329.47881342,128.1407375)(328.82777207,127.54698681)
\curveto(328.17152271,126.96365346)(327.44235602,126.67198678)(326.64027199,126.67198678)
\curveto(325.69235596,126.67198678)(324.86943859,127.06782013)(324.16110524,127.85948682)
\curveto(323.46318921,128.65115352)(323.11943853,129.72407089)(323.11943853,131.0886536)
\curveto(323.11943853,132.39073765)(323.50485588,133.45323769)(324.28610524,134.27615371)
\curveto(325.0621466,135.10948708)(325.99964663,135.52615376)(327.09860534,135.52615376)
\curveto(327.91631337,135.52615376)(328.58818939,135.30219508)(329.11943874,134.85948707)
\curveto(329.6454801,134.42719505)(329.91110544,133.97407104)(329.91110544,133.50532035)
\curveto(329.91110544,133.28136168)(329.8329801,133.09386167)(329.68193876,132.94282033)
\curveto(329.54131343,132.802195)(329.33298009,132.73448699)(329.05693874,132.73448699)
\curveto(328.70798006,132.73448699)(328.43714672,132.849071)(328.24443871,133.08865367)
\curveto(328.14548004,133.21365368)(328.07777204,133.45323769)(328.03610537,133.81782037)
\curveto(328.00485604,134.17719504)(327.89027203,134.44802839)(327.68193869,134.63032039)
\curveto(327.47360535,134.8074044)(327.17152268,134.90115374)(326.78610533,134.90115374)
\curveto(326.18714664,134.90115374)(325.70277196,134.67719506)(325.32777194,134.23448705)
\curveto(324.83818926,133.63552836)(324.59860525,132.849071)(324.59860525,131.8803203)
\curveto(324.59860525,130.88032026)(324.83818926,129.99490423)(325.32777194,129.23448687)
\curveto(325.81214663,128.46886151)(326.47360532,128.0886535)(327.30693868,128.0886535)
\curveto(327.90068937,128.0886535)(328.43714672,128.28657084)(328.9111054,128.69282018)
\curveto(329.24443875,128.96886153)(329.56735609,129.47407088)(329.8902721,130.21365357)
\closepath
\moveto(330.14027211,130.08865357)
}
}
{
\newrgbcolor{curcolor}{0 0 0}
\pscustom[linestyle=none,fillstyle=solid,fillcolor=curcolor]
{
\newpath
\moveto(333.76161013,138.00532051)
\lineto(333.76161013,135.27615375)
\lineto(335.71994354,135.27615375)
\lineto(335.71994354,134.63032039)
\lineto(333.76161013,134.63032039)
\lineto(333.76161013,129.21365354)
\curveto(333.76161013,128.67198685)(333.83452747,128.30219484)(333.99077681,128.10948683)
\curveto(334.15744348,127.92719482)(334.35536082,127.83865349)(334.5949435,127.83865349)
\curveto(334.80327684,127.83865349)(334.99598484,127.90115349)(335.17827685,128.02615349)
\curveto(335.37098486,128.1511535)(335.51681819,128.3386535)(335.61577687,128.58865351)
\lineto(335.96994354,128.58865351)
\curveto(335.7616102,127.98969483)(335.45952753,127.53657081)(335.07411018,127.2344868)
\curveto(334.68348483,126.92719512)(334.28244348,126.77615345)(333.8657768,126.77615345)
\curveto(333.58452746,126.77615345)(333.31369412,126.85427839)(333.05327677,127.00532012)
\curveto(332.78765143,127.1719868)(332.58452742,127.39073747)(332.44911009,127.67198681)
\curveto(332.32411008,127.96365349)(332.26161008,128.41157084)(332.26161008,129.02615353)
\lineto(332.26161008,134.63032039)
\lineto(330.94911003,134.63032039)
\lineto(330.94911003,134.9428204)
\curveto(331.28244338,135.06782041)(331.62098472,135.28657108)(331.9699434,135.6094871)
\curveto(332.31369408,135.92719511)(332.62619409,136.30740445)(332.90744344,136.75532047)
\curveto(333.04286077,136.97407114)(333.24077678,137.39073783)(333.49077679,138.00532051)
\closepath
\moveto(333.76161013,138.00532051)
}
}
{
\newrgbcolor{curcolor}{0 0 0}
\pscustom[linestyle=none,fillstyle=solid,fillcolor=curcolor]
{
\newpath
\moveto(346.59677498,135.52615376)
\lineto(346.59677498,132.67198699)
\lineto(346.3051083,132.67198699)
\curveto(346.06552562,133.55740436)(345.76864961,134.16157104)(345.40927493,134.48448706)
\curveto(345.04469225,134.8178204)(344.5863589,134.98448707)(344.03427488,134.98448707)
\curveto(343.60198287,134.98448707)(343.25302552,134.86469507)(342.99260818,134.63032039)
\curveto(342.72698284,134.40636172)(342.59677483,134.16157104)(342.59677483,133.9011537)
\curveto(342.59677483,133.55219502)(342.6905255,133.26052835)(342.88844151,133.02615367)
\curveto(343.08114952,132.77615366)(343.47177486,132.51052832)(344.05510822,132.23448698)
\lineto(345.36760827,131.58865362)
\curveto(346.60198298,130.97407093)(347.221775,130.1771949)(347.221775,129.1928202)
\curveto(347.221775,128.44282017)(346.93531632,127.82823749)(346.3676083,127.3594868)
\curveto(345.79469228,126.90115345)(345.14885892,126.67198678)(344.43010823,126.67198678)
\curveto(343.93010821,126.67198678)(343.34677486,126.76573612)(342.68010817,126.94282012)
\curveto(342.48219216,126.99490346)(342.32594149,127.02615346)(342.20094149,127.02615346)
\curveto(342.06031615,127.02615346)(341.95094148,126.94802852)(341.86760814,126.79698678)
\lineto(341.57594146,126.79698678)
\lineto(341.57594146,129.77615356)
\lineto(341.86760814,129.77615356)
\curveto(342.03427481,128.92719486)(342.35198283,128.28657084)(342.82594151,127.85948682)
\curveto(343.31031619,127.42719481)(343.85198288,127.21365346)(344.45094157,127.21365346)
\curveto(344.86760825,127.21365346)(345.20614959,127.32823747)(345.47177494,127.56782014)
\curveto(345.73219228,127.81782015)(345.86760828,128.11469483)(345.86760828,128.46365351)
\curveto(345.86760828,128.88032019)(345.72177494,129.22407087)(345.43010827,129.50532021)
\curveto(345.13844159,129.79698689)(344.54469224,130.15636157)(343.65927487,130.58865358)
\curveto(342.78427484,131.0313616)(342.20614949,131.43240428)(341.93010814,131.79698696)
\curveto(341.6644828,132.14073764)(341.5342748,132.57823765)(341.5342748,133.10948701)
\curveto(341.5342748,133.78657103)(341.76864947,134.35948705)(342.24260815,134.8178204)
\curveto(342.71135884,135.28657108)(343.31552553,135.52615376)(344.05510822,135.52615376)
\curveto(344.38844156,135.52615376)(344.78948291,135.45323776)(345.26344159,135.31782042)
\curveto(345.56552561,135.21886175)(345.76864961,135.17198708)(345.86760828,135.17198708)
\curveto(345.96135895,135.17198708)(346.03948296,135.19282041)(346.09677496,135.23448708)
\curveto(346.14885896,135.27615375)(346.22177496,135.36990442)(346.3051083,135.52615376)
\closepath
\moveto(346.59677498,135.52615376)
}
}
{
\newrgbcolor{curcolor}{0 0 0}
\pscustom[linestyle=none,fillstyle=solid,fillcolor=curcolor]
{
\newpath
\moveto(349.85812763,132.13032031)
\curveto(349.85812763,130.89073759)(350.16021164,129.92198689)(350.77479433,129.21365354)
\curveto(351.36854502,128.50532018)(352.08729438,128.1511535)(352.92062774,128.1511535)
\curveto(353.46229442,128.1511535)(353.93104511,128.29698684)(354.33729446,128.58865351)
\curveto(354.7383358,128.89073752)(355.07687848,129.41157088)(355.35812782,130.15115357)
\lineto(355.62896117,129.96365356)
\curveto(355.50396116,129.1303202)(355.13416915,128.36990417)(354.52479446,127.69282015)
\curveto(353.91021177,127.01052852)(353.13937841,126.67198678)(352.21229438,126.67198678)
\curveto(351.21229434,126.67198678)(350.34771165,127.05740413)(349.62896096,127.83865349)
\curveto(348.9206276,128.61469485)(348.56646092,129.66157088)(348.56646092,130.98448693)
\curveto(348.56646092,132.41157098)(348.93104493,133.52615369)(349.67062762,134.31782038)
\curveto(350.40500232,135.11990441)(351.32166902,135.52615376)(352.42062772,135.52615376)
\curveto(353.36333575,135.52615376)(354.13416911,135.21365375)(354.7331278,134.58865373)
\curveto(355.32687849,133.97407104)(355.62896117,133.15636168)(355.62896117,132.13032031)
\closepath
\moveto(349.85812763,132.65115366)
\lineto(353.73312777,132.65115366)
\curveto(353.70187843,133.19282034)(353.63937843,133.56782036)(353.54562776,133.7761537)
\curveto(353.38937842,134.11990438)(353.16021175,134.39073772)(352.85812774,134.58865373)
\curveto(352.55083573,134.78136173)(352.23833571,134.8803204)(351.9206277,134.8803204)
\curveto(351.42062769,134.8803204)(350.96750234,134.6824044)(350.56646099,134.29698705)
\curveto(350.16021164,133.9063617)(349.92583563,133.35948702)(349.85812763,132.65115366)
\closepath
\moveto(349.85812763,132.65115366)
}
}
{
\newrgbcolor{curcolor}{0 0 0}
\pscustom[linestyle=none,fillstyle=solid,fillcolor=curcolor]
{
\newpath
\moveto(359.16696585,138.00532051)
\lineto(359.16696585,135.27615375)
\lineto(361.12529925,135.27615375)
\lineto(361.12529925,134.63032039)
\lineto(359.16696585,134.63032039)
\lineto(359.16696585,129.21365354)
\curveto(359.16696585,128.67198685)(359.23988319,128.30219484)(359.39613253,128.10948683)
\curveto(359.5627992,127.92719482)(359.76071654,127.83865349)(360.00029921,127.83865349)
\curveto(360.20863255,127.83865349)(360.40134056,127.90115349)(360.58363257,128.02615349)
\curveto(360.77634057,128.1511535)(360.92217391,128.3386535)(361.02113258,128.58865351)
\lineto(361.37529926,128.58865351)
\curveto(361.16696592,127.98969483)(360.86488324,127.53657081)(360.4794659,127.2344868)
\curveto(360.08884055,126.92719512)(359.6877992,126.77615345)(359.27113252,126.77615345)
\curveto(358.98988318,126.77615345)(358.71904983,126.85427839)(358.45863249,127.00532012)
\curveto(358.19300715,127.1719868)(357.98988314,127.39073747)(357.8544658,127.67198681)
\curveto(357.7294658,127.96365349)(357.6669658,128.41157084)(357.6669658,129.02615353)
\lineto(357.6669658,134.63032039)
\lineto(356.35446575,134.63032039)
\lineto(356.35446575,134.9428204)
\curveto(356.6877991,135.06782041)(357.02634044,135.28657108)(357.37529912,135.6094871)
\curveto(357.7190498,135.92719511)(358.03154981,136.30740445)(358.31279915,136.75532047)
\curveto(358.44821649,136.97407114)(358.6461325,137.39073783)(358.89613251,138.00532051)
\closepath
\moveto(359.16696585,138.00532051)
}
}
{
\newrgbcolor{curcolor}{0 0 0}
\pscustom[linestyle=none,fillstyle=solid,fillcolor=curcolor]
{
\newpath
\moveto(207.85877364,15.2761495)
\lineto(207.85877364,10.21364932)
\curveto(207.85877364,9.23969062)(207.87960697,8.64073326)(207.92127364,8.42198259)
\curveto(207.97335764,8.21364925)(208.05148165,8.06781591)(208.15044032,7.98448257)
\curveto(208.24419099,7.90114924)(208.36398166,7.85948257)(208.504607,7.85948257)
\curveto(208.697315,7.85948257)(208.91085768,7.91156657)(209.15044035,8.02614924)
\lineto(209.27544036,7.6928159)
\lineto(206.77544027,6.67198253)
\lineto(206.35877359,6.67198253)
\lineto(206.35877359,8.44281592)
\curveto(205.65044023,7.66156656)(205.09835754,7.17198254)(204.7129402,6.9636492)
\curveto(204.33794018,6.7709412)(203.94210684,6.67198253)(203.52544015,6.67198253)
\curveto(203.05148147,6.67198253)(202.64002412,6.8073992)(202.29627344,7.06781587)
\curveto(201.9629401,7.34385722)(201.72335742,7.6928159)(201.58794009,8.10948258)
\curveto(201.46294008,8.53656659)(201.40044008,9.14073328)(201.40044008,9.92198264)
\lineto(201.40044008,13.65114944)
\curveto(201.40044008,14.03656679)(201.34835741,14.30740013)(201.25460674,14.46364947)
\curveto(201.1712734,14.61469081)(201.0462734,14.73448281)(200.87960673,14.81781615)
\curveto(200.71294005,14.90114949)(200.41085738,14.94281615)(199.98377336,14.94281615)
\lineto(199.98377336,15.2761495)
\lineto(202.90044013,15.2761495)
\lineto(202.90044013,9.69281597)
\curveto(202.90044013,8.91156661)(203.03064814,8.40114925)(203.29627348,8.15114925)
\curveto(203.57231482,7.91156657)(203.90564817,7.79698257)(204.29627351,7.79698257)
\curveto(204.55669086,7.79698257)(204.84835753,7.8698999)(205.17127355,8.02614924)
\curveto(205.50460689,8.19281591)(205.90044024,8.50531593)(206.35877359,8.96364927)
\lineto(206.35877359,13.71364944)
\curveto(206.35877359,14.18240013)(206.26502425,14.50531614)(206.08794024,14.67198281)
\curveto(205.92127357,14.83864948)(205.56710689,14.92719082)(205.02544021,14.94281615)
\lineto(205.02544021,15.2761495)
\closepath
\moveto(207.85877364,15.2761495)
}
}
{
\newrgbcolor{curcolor}{0 0 0}
\pscustom[linestyle=none,fillstyle=solid,fillcolor=curcolor]
{
\newpath
\moveto(215.27544057,15.52614951)
\lineto(215.27544057,12.67198274)
\lineto(214.98377389,12.67198274)
\curveto(214.74419122,13.5574001)(214.44731521,14.16156679)(214.08794053,14.4844828)
\curveto(213.72335785,14.81781615)(213.2650245,14.98448282)(212.71294048,14.98448282)
\curveto(212.28064846,14.98448282)(211.93169112,14.86469082)(211.67127378,14.63031614)
\curveto(211.40564843,14.40635747)(211.27544043,14.16156679)(211.27544043,13.90114945)
\curveto(211.27544043,13.55219077)(211.3691911,13.26052409)(211.56710711,13.02614942)
\curveto(211.75981511,12.77614941)(212.15044046,12.51052407)(212.73377381,12.23448272)
\lineto(214.04627386,11.58864937)
\curveto(215.28064857,10.97406668)(215.90044059,10.17719065)(215.90044059,9.19281595)
\curveto(215.90044059,8.44281592)(215.61398192,7.82823323)(215.0462739,7.35948255)
\curveto(214.47335787,6.9011492)(213.82752452,6.67198253)(213.10877383,6.67198253)
\curveto(212.60877381,6.67198253)(212.02544045,6.76573186)(211.35877376,6.94281587)
\curveto(211.16085776,6.99489921)(211.00460709,7.02614921)(210.87960708,7.02614921)
\curveto(210.73898174,7.02614921)(210.62960707,6.94802427)(210.54627374,6.79698253)
\lineto(210.25460706,6.79698253)
\lineto(210.25460706,9.7761493)
\lineto(210.54627374,9.7761493)
\curveto(210.71294041,8.92719061)(211.03064842,8.28656658)(211.5046071,7.85948257)
\curveto(211.98898179,7.42719055)(212.53064847,7.21364921)(213.12960716,7.21364921)
\curveto(213.54627384,7.21364921)(213.88481519,7.32823322)(214.15044053,7.56781589)
\curveto(214.41085787,7.8178159)(214.54627388,8.11469058)(214.54627388,8.46364926)
\curveto(214.54627388,8.88031594)(214.40044054,9.22406662)(214.10877386,9.50531596)
\curveto(213.81710719,9.79698264)(213.22335783,10.15635732)(212.33794047,10.58864933)
\curveto(211.46294043,11.03135735)(210.88481508,11.43240003)(210.60877374,11.79698271)
\curveto(210.3431484,12.14073339)(210.21294039,12.5782334)(210.21294039,13.10948275)
\curveto(210.21294039,13.78656678)(210.44731507,14.3594828)(210.92127375,14.81781615)
\curveto(211.39002443,15.28656683)(211.99419112,15.52614951)(212.73377381,15.52614951)
\curveto(213.06710716,15.52614951)(213.46814851,15.4532335)(213.94210719,15.31781617)
\curveto(214.2441912,15.2188575)(214.44731521,15.17198283)(214.54627388,15.17198283)
\curveto(214.64002455,15.17198283)(214.71814855,15.19281616)(214.77544055,15.23448283)
\curveto(214.82752455,15.2761495)(214.90044056,15.36990017)(214.98377389,15.52614951)
\closepath
\moveto(215.27544057,15.52614951)
}
}
{
\newrgbcolor{curcolor}{0 0 0}
\pscustom[linestyle=none,fillstyle=solid,fillcolor=curcolor]
{
\newpath
\moveto(218.53677288,12.13031605)
\curveto(218.53677288,10.89073334)(218.83885689,9.92198264)(219.45343958,9.21364928)
\curveto(220.04719027,8.50531593)(220.76593963,8.15114925)(221.59927299,8.15114925)
\curveto(222.14093967,8.15114925)(222.60969036,8.29698258)(223.01593971,8.58864926)
\curveto(223.41698105,8.89073327)(223.75552373,9.41156662)(224.03677307,10.15114932)
\lineto(224.30760642,9.96364931)
\curveto(224.18260641,9.13031595)(223.8128144,8.36989992)(223.20343971,7.6928159)
\curveto(222.58885702,7.01052427)(221.81802366,6.67198253)(220.89093963,6.67198253)
\curveto(219.89093959,6.67198253)(219.0263569,7.05739987)(218.3076062,7.83864923)
\curveto(217.59927285,8.6146906)(217.24510617,9.66156663)(217.24510617,10.98448268)
\curveto(217.24510617,12.41156673)(217.60969018,13.52614944)(218.34927287,14.31781613)
\curveto(219.08364757,15.11990016)(220.00031426,15.52614951)(221.09927297,15.52614951)
\curveto(222.041981,15.52614951)(222.81281436,15.2136495)(223.41177305,14.58864947)
\curveto(224.00552374,13.97406679)(224.30760642,13.15635742)(224.30760642,12.13031605)
\closepath
\moveto(218.53677288,12.65114941)
\lineto(222.41177302,12.65114941)
\curveto(222.38052368,13.19281609)(222.31802368,13.5678161)(222.22427301,13.77614945)
\curveto(222.06802367,14.11990012)(221.838857,14.39073347)(221.53677299,14.58864947)
\curveto(221.22948098,14.78135748)(220.91698096,14.88031615)(220.59927295,14.88031615)
\curveto(220.09927294,14.88031615)(219.64614759,14.68240014)(219.24510624,14.2969828)
\curveto(218.83885689,13.90635745)(218.60448088,13.35948276)(218.53677288,12.65114941)
\closepath
\moveto(218.53677288,12.65114941)
}
}
{
\newrgbcolor{curcolor}{0 0 0}
\pscustom[linestyle=none,fillstyle=solid,fillcolor=curcolor]
{
\newpath
\moveto(227.86644443,15.52614951)
\lineto(227.86644443,13.63031611)
\curveto(228.57477779,14.89073348)(229.29352848,15.52614951)(230.03311118,15.52614951)
\curveto(230.36644452,15.52614951)(230.63727787,15.42198284)(230.84561121,15.2136495)
\curveto(231.06436188,15.00531616)(231.17894455,14.76573348)(231.17894455,14.50531614)
\curveto(231.17894455,14.26573346)(231.09561122,14.06781612)(230.92894454,13.90114945)
\curveto(230.7726952,13.73448278)(230.5956112,13.65114944)(230.38727786,13.65114944)
\curveto(230.16331918,13.65114944)(229.91852851,13.75531611)(229.65811116,13.96364945)
\curveto(229.39248582,14.18240013)(229.19977781,14.2969828)(229.07477781,14.2969828)
\curveto(228.96019514,14.2969828)(228.84561114,14.23448279)(228.72061113,14.10948279)
\curveto(228.43936179,13.85948278)(228.15811111,13.4480241)(227.86644443,12.88031608)
\lineto(227.86644443,8.88031594)
\curveto(227.86644443,8.40635726)(227.92894444,8.05219058)(228.05394444,7.8178159)
\curveto(228.12165244,7.66156656)(228.25186178,7.53135722)(228.44977779,7.42198255)
\curveto(228.65811113,7.30739988)(228.94977781,7.25531588)(229.32477782,7.25531588)
\lineto(229.32477782,6.92198254)
\lineto(225.05394433,6.92198254)
\lineto(225.05394433,7.25531588)
\curveto(225.48102835,7.25531588)(225.80394436,7.31781588)(226.0122777,7.44281589)
\curveto(226.16331904,7.53656656)(226.26748571,7.6980239)(226.32477771,7.92198257)
\curveto(226.35081905,8.01573324)(226.36644438,8.30739992)(226.36644438,8.7969826)
\lineto(226.36644438,12.02614938)
\curveto(226.36644438,12.99490008)(226.34561105,13.5730241)(226.30394438,13.75531611)
\curveto(226.26227771,13.94802412)(226.18415237,14.08864946)(226.0747777,14.17198279)
\curveto(225.97581903,14.25531613)(225.85081903,14.2969828)(225.69977769,14.2969828)
\curveto(225.50186168,14.2969828)(225.28831901,14.25531613)(225.05394433,14.17198279)
\lineto(224.970611,14.50531614)
\lineto(227.49144442,15.52614951)
\closepath
\moveto(227.86644443,15.52614951)
}
}
{
\newrgbcolor{curcolor}{0 0 0}
\pscustom[linestyle=none,fillstyle=solid,fillcolor=curcolor]
{
\newpath
\moveto(242.20744836,7.85948257)
\curveto(241.80119901,7.42719055)(241.40536567,7.11989988)(241.01994832,6.94281587)
\curveto(240.62932297,6.76573186)(240.20744829,6.67198253)(239.74911494,6.67198253)
\curveto(238.84286557,6.67198253)(238.05119888,7.05219094)(237.37411486,7.8178159)
\curveto(236.69182283,8.57823326)(236.35328149,9.55739996)(236.35328149,10.755316)
\curveto(236.35328149,11.94802405)(236.7282815,13.03656675)(237.47828153,14.02614945)
\curveto(238.22828155,15.02614949)(239.19703225,15.52614951)(240.3949483,15.52614951)
\curveto(241.12932299,15.52614951)(241.73348968,15.28656683)(242.20744836,14.81781615)
\lineto(242.20744836,16.35948287)
\curveto(242.20744836,17.31781624)(242.18661503,17.90114959)(242.14494836,18.10948293)
\curveto(242.10328169,18.32823361)(242.03036569,18.48448295)(241.93661502,18.56781628)
\curveto(241.83765635,18.65114962)(241.71265634,18.69281629)(241.561615,18.69281629)
\curveto(241.40536567,18.69281629)(241.19703232,18.64073362)(240.93661498,18.54698295)
\lineto(240.81161498,18.85948296)
\lineto(243.31161507,19.88031633)
\lineto(243.70744841,19.88031633)
\lineto(243.70744841,10.23448265)
\curveto(243.70744841,9.24489995)(243.72828175,8.64073326)(243.76994842,8.42198259)
\curveto(243.82203242,8.21364925)(243.90015642,8.06781591)(243.99911509,7.98448257)
\curveto(244.09286576,7.90114924)(244.21265643,7.85948257)(244.35328177,7.85948257)
\curveto(244.51994844,7.85948257)(244.73348978,7.91156657)(244.99911513,8.02614924)
\lineto(245.1032818,7.6928159)
\lineto(242.62411504,6.67198253)
\lineto(242.20744836,6.67198253)
\closepath
\moveto(242.20744836,8.50531593)
\lineto(242.20744836,12.79698274)
\curveto(242.16578169,13.21364943)(242.05119902,13.58864944)(241.87411502,13.92198278)
\curveto(241.69182301,14.26573346)(241.44703233,14.52614947)(241.14494832,14.69281614)
\curveto(240.85328165,14.86990015)(240.56161497,14.96364949)(240.26994829,14.96364949)
\curveto(239.73869894,14.96364949)(239.26994826,14.71885748)(238.85328157,14.23448279)
\curveto(238.29598955,13.60948277)(238.01994821,12.68240007)(238.01994821,11.46364936)
\curveto(238.01994821,10.23969065)(238.28036555,9.30219062)(238.81161491,8.65114926)
\curveto(239.35328159,7.99489991)(239.94703228,7.67198256)(240.60328164,7.67198256)
\curveto(241.15536566,7.67198256)(241.69182301,7.94802391)(242.20744836,8.50531593)
\closepath
\moveto(242.20744836,8.50531593)
}
}
{
\newrgbcolor{curcolor}{0 0 0}
\pscustom[linestyle=none,fillstyle=solid,fillcolor=curcolor]
{
\newpath
\moveto(250.37411532,8.13031591)
\curveto(249.50953262,7.46364922)(248.97828193,7.07823321)(248.76994859,6.98448254)
\curveto(248.43661525,6.8334408)(248.08244857,6.75531586)(247.70744856,6.75531586)
\curveto(247.13453253,6.75531586)(246.66578185,6.94802427)(246.29078184,7.33864922)
\curveto(245.91578182,7.73969056)(245.72828182,8.26052392)(245.72828182,8.90114927)
\curveto(245.72828182,9.30219062)(245.81682315,9.6563573)(245.99911516,9.96364931)
\curveto(246.24911517,10.36469066)(246.67619919,10.7449)(247.29078187,11.10948268)
\curveto(247.9157819,11.4844827)(248.94182327,11.93240005)(250.37411532,12.4636494)
\lineto(250.37411532,12.77614941)
\curveto(250.37411532,13.60948277)(250.23869931,14.17719079)(249.97828197,14.4844828)
\curveto(249.71265663,14.80219081)(249.33244861,14.96364949)(248.83244859,14.96364949)
\curveto(248.44182325,14.96364949)(248.13453257,14.85948282)(247.9157819,14.65114948)
\curveto(247.67619922,14.44281614)(247.56161522,14.20323346)(247.56161522,13.94281612)
\lineto(247.58244855,13.42198277)
\curveto(247.58244855,13.13031609)(247.50953255,12.90635741)(247.37411521,12.75531608)
\curveto(247.23348987,12.61469074)(247.04598987,12.54698274)(246.81161519,12.54698274)
\curveto(246.58765652,12.54698274)(246.40015651,12.61990007)(246.24911517,12.77614941)
\curveto(246.10848983,12.92719075)(246.04078183,13.14073342)(246.04078183,13.42198277)
\curveto(246.04078183,13.94802412)(246.30119917,14.4271908)(246.83244852,14.85948282)
\curveto(247.37411521,15.30219083)(248.12932324,15.52614951)(249.10328194,15.52614951)
\curveto(249.83765663,15.52614951)(250.44182332,15.4011495)(250.915782,15.15114949)
\curveto(251.27515668,14.95323349)(251.54078202,14.65635748)(251.7074487,14.25531613)
\curveto(251.81682337,13.98969079)(251.87411537,13.45323343)(251.87411537,12.65114941)
\lineto(251.87411537,9.81781597)
\curveto(251.87411537,9.02614928)(251.8845287,8.53656659)(251.91578204,8.35948259)
\curveto(251.95744871,8.17719058)(252.00953271,8.05219058)(252.08244871,7.98448257)
\curveto(252.15015671,7.92719057)(252.22828205,7.90114924)(252.31161538,7.90114924)
\curveto(252.40536605,7.90114924)(252.49911539,7.92198257)(252.58244873,7.96364924)
\curveto(252.70744873,8.04698258)(252.94703274,8.26573325)(253.31161542,8.63031593)
\lineto(253.31161542,8.13031591)
\curveto(252.6293234,7.21364921)(251.97828204,6.75531586)(251.35328202,6.75531586)
\curveto(251.06161534,6.75531586)(250.82203267,6.85948253)(250.64494866,7.06781587)
\curveto(250.47828199,7.27614921)(250.38453265,7.63031589)(250.37411532,8.13031591)
\closepath
\moveto(250.37411532,8.71364927)
\lineto(250.37411532,11.90114938)
\curveto(249.45744862,11.5365667)(248.86369926,11.28135736)(248.60328192,11.13031602)
\curveto(248.11369924,10.84906667)(247.76994856,10.567816)(247.56161522,10.27614932)
\curveto(247.35328188,9.98448264)(247.24911521,9.67198263)(247.24911521,9.33864929)
\curveto(247.24911521,8.89073327)(247.37932321,8.52614926)(247.64494855,8.23448258)
\curveto(247.9053659,7.94281591)(248.21265657,7.79698257)(248.56161525,7.79698257)
\curveto(249.0199486,7.79698257)(249.62411529,8.09906658)(250.37411532,8.71364927)
\closepath
\moveto(250.37411532,8.71364927)
}
}
{
\newrgbcolor{curcolor}{0 0 0}
\pscustom[linestyle=none,fillstyle=solid,fillcolor=curcolor]
{
\newpath
\moveto(256.34962009,18.00531626)
\lineto(256.34962009,15.2761495)
\lineto(258.30795349,15.2761495)
\lineto(258.30795349,14.63031614)
\lineto(256.34962009,14.63031614)
\lineto(256.34962009,9.21364928)
\curveto(256.34962009,8.6719826)(256.42253742,8.30219058)(256.57878676,8.10948258)
\curveto(256.74545343,7.92719057)(256.94337077,7.83864923)(257.18295345,7.83864923)
\curveto(257.39128679,7.83864923)(257.5839948,7.90114924)(257.7662868,8.02614924)
\curveto(257.95899481,8.15114925)(258.10482815,8.33864925)(258.20378682,8.58864926)
\lineto(258.5579535,8.58864926)
\curveto(258.34962016,7.98969057)(258.04753748,7.53656656)(257.66212013,7.23448255)
\curveto(257.27149478,6.92719087)(256.87045344,6.7761492)(256.45378676,6.7761492)
\curveto(256.17253741,6.7761492)(255.90170407,6.85427413)(255.64128673,7.00531587)
\curveto(255.37566138,7.17198254)(255.17253738,7.39073322)(255.03712004,7.67198256)
\curveto(254.91212003,7.96364924)(254.84962003,8.41156659)(254.84962003,9.02614928)
\lineto(254.84962003,14.63031614)
\lineto(253.53711999,14.63031614)
\lineto(253.53711999,14.94281615)
\curveto(253.87045333,15.06781616)(254.20899468,15.28656683)(254.55795336,15.60948284)
\curveto(254.90170403,15.92719085)(255.21420405,16.3074002)(255.49545339,16.75531622)
\curveto(255.63087073,16.97406689)(255.82878673,17.39073357)(256.07878674,18.00531626)
\closepath
\moveto(256.34962009,18.00531626)
}
}
{
\newrgbcolor{curcolor}{0 0 0}
\pscustom[linestyle=none,fillstyle=solid,fillcolor=curcolor]
{
\newpath
\moveto(263.85145141,8.13031591)
\curveto(262.98686871,7.46364922)(262.45561802,7.07823321)(262.24728468,6.98448254)
\curveto(261.91395134,6.8334408)(261.55978466,6.75531586)(261.18478464,6.75531586)
\curveto(260.61186862,6.75531586)(260.14311794,6.94802427)(259.76811793,7.33864922)
\curveto(259.39311791,7.73969056)(259.20561791,8.26052392)(259.20561791,8.90114927)
\curveto(259.20561791,9.30219062)(259.29415924,9.6563573)(259.47645125,9.96364931)
\curveto(259.72645126,10.36469066)(260.15353527,10.7449)(260.76811796,11.10948268)
\curveto(261.39311799,11.4844827)(262.41915935,11.93240005)(263.85145141,12.4636494)
\lineto(263.85145141,12.77614941)
\curveto(263.85145141,13.60948277)(263.7160354,14.17719079)(263.45561806,14.4844828)
\curveto(263.18999272,14.80219081)(262.8097847,14.96364949)(262.30978468,14.96364949)
\curveto(261.91915934,14.96364949)(261.61186866,14.85948282)(261.39311799,14.65114948)
\curveto(261.15353531,14.44281614)(261.03895131,14.20323346)(261.03895131,13.94281612)
\lineto(261.05978464,13.42198277)
\curveto(261.05978464,13.13031609)(260.98686864,12.90635741)(260.8514513,12.75531608)
\curveto(260.71082596,12.61469074)(260.52332595,12.54698274)(260.28895128,12.54698274)
\curveto(260.0649926,12.54698274)(259.8774926,12.61990007)(259.72645126,12.77614941)
\curveto(259.58582592,12.92719075)(259.51811792,13.14073342)(259.51811792,13.42198277)
\curveto(259.51811792,13.94802412)(259.77853526,14.4271908)(260.30978461,14.85948282)
\curveto(260.8514513,15.30219083)(261.60665933,15.52614951)(262.58061803,15.52614951)
\curveto(263.31499272,15.52614951)(263.91915941,15.4011495)(264.39311809,15.15114949)
\curveto(264.75249277,14.95323349)(265.01811811,14.65635748)(265.18478479,14.25531613)
\curveto(265.29415946,13.98969079)(265.35145146,13.45323343)(265.35145146,12.65114941)
\lineto(265.35145146,9.81781597)
\curveto(265.35145146,9.02614928)(265.36186479,8.53656659)(265.39311813,8.35948259)
\curveto(265.4347848,8.17719058)(265.4868688,8.05219058)(265.5597848,7.98448257)
\curveto(265.6274928,7.92719057)(265.70561814,7.90114924)(265.78895147,7.90114924)
\curveto(265.88270214,7.90114924)(265.97645148,7.92198257)(266.05978482,7.96364924)
\curveto(266.18478482,8.04698258)(266.42436883,8.26573325)(266.78895151,8.63031593)
\lineto(266.78895151,8.13031591)
\curveto(266.10665949,7.21364921)(265.45561813,6.75531586)(264.83061811,6.75531586)
\curveto(264.53895143,6.75531586)(264.29936875,6.85948253)(264.12228475,7.06781587)
\curveto(263.95561808,7.27614921)(263.86186874,7.63031589)(263.85145141,8.13031591)
\closepath
\moveto(263.85145141,8.71364927)
\lineto(263.85145141,11.90114938)
\curveto(262.93478471,11.5365667)(262.34103535,11.28135736)(262.08061801,11.13031602)
\curveto(261.59103533,10.84906667)(261.24728465,10.567816)(261.03895131,10.27614932)
\curveto(260.83061797,9.98448264)(260.7264513,9.67198263)(260.7264513,9.33864929)
\curveto(260.7264513,8.89073327)(260.8566593,8.52614926)(261.12228464,8.23448258)
\curveto(261.38270198,7.94281591)(261.68999266,7.79698257)(262.03895134,7.79698257)
\curveto(262.49728469,7.79698257)(263.10145138,8.09906658)(263.85145141,8.71364927)
\closepath
\moveto(263.85145141,8.71364927)
}
}
{
\newrgbcolor{curcolor}{0 0 0}
\pscustom[linestyle=none,fillstyle=solid,fillcolor=curcolor]
{
\newpath
\moveto(506.5607699,63.15534228)
\curveto(506.15452055,62.72305026)(505.75868721,62.41575958)(505.37326986,62.23867558)
\curveto(504.98264451,62.06159157)(504.56076983,61.96784223)(504.10243648,61.96784223)
\curveto(503.19618712,61.96784223)(502.40452042,62.34805065)(501.7274364,63.11367561)
\curveto(501.04514437,63.87409297)(500.70660303,64.85325967)(500.70660303,66.05117571)
\curveto(500.70660303,67.24388375)(501.08160304,68.33242646)(501.83160307,69.32200916)
\curveto(502.58160309,70.3220092)(503.55035379,70.82200921)(504.74826984,70.82200921)
\curveto(505.48264453,70.82200921)(506.08681122,70.58242654)(506.5607699,70.11367586)
\lineto(506.5607699,71.65534258)
\curveto(506.5607699,72.61367594)(506.53993657,73.1970093)(506.4982699,73.40534264)
\curveto(506.45660323,73.62409331)(506.38368723,73.78034265)(506.28993656,73.86367599)
\curveto(506.19097789,73.94700932)(506.06597788,73.98867599)(505.91493654,73.98867599)
\curveto(505.75868721,73.98867599)(505.55035387,73.93659332)(505.28993652,73.84284265)
\lineto(505.16493652,74.15534267)
\lineto(507.66493661,75.17617603)
\lineto(508.06076995,75.17617603)
\lineto(508.06076995,65.53034236)
\curveto(508.06076995,64.54075966)(508.08160329,63.93659297)(508.12326996,63.7178423)
\curveto(508.17535396,63.50950895)(508.25347796,63.36367562)(508.35243663,63.28034228)
\curveto(508.4461873,63.19700894)(508.56597797,63.15534228)(508.70660331,63.15534228)
\curveto(508.87326998,63.15534228)(509.08681132,63.20742628)(509.35243667,63.32200895)
\lineto(509.45660334,62.9886756)
\lineto(506.97743658,61.96784223)
\lineto(506.5607699,61.96784223)
\closepath
\moveto(506.5607699,63.80117563)
\lineto(506.5607699,68.09284245)
\curveto(506.51910323,68.50950913)(506.40452056,68.88450915)(506.22743656,69.21784249)
\curveto(506.04514455,69.56159317)(505.80035387,69.82200918)(505.49826986,69.98867585)
\curveto(505.20660319,70.16575986)(504.91493651,70.25950919)(504.62326983,70.25950919)
\curveto(504.09202048,70.25950919)(503.6232698,70.01471719)(503.20660312,69.5303425)
\curveto(502.6493111,68.90534248)(502.37326975,67.97825978)(502.37326975,66.75950907)
\curveto(502.37326975,65.53555036)(502.6336871,64.59805033)(503.16493645,63.94700897)
\curveto(503.70660313,63.29075961)(504.30035382,62.96784227)(504.95660318,62.96784227)
\curveto(505.5086872,62.96784227)(506.04514455,63.24388361)(506.5607699,63.80117563)
\closepath
\moveto(506.5607699,63.80117563)
}
}
{
\newrgbcolor{curcolor}{0 0 0}
\pscustom[linestyle=none,fillstyle=solid,fillcolor=curcolor]
{
\newpath
\moveto(512.1232701,75.17617603)
\curveto(512.37327011,75.17617603)(512.58681145,75.0824267)(512.76910345,74.90534269)
\curveto(512.94618746,74.72305069)(513.0399368,74.50950934)(513.0399368,74.25950934)
\curveto(513.0399368,74.00950933)(512.94618746,73.78555065)(512.76910345,73.59284265)
\curveto(512.58681145,73.41055064)(512.37327011,73.3220093)(512.1232701,73.3220093)
\curveto(511.87327009,73.3220093)(511.64931141,73.41055064)(511.45660341,73.59284265)
\curveto(511.2743114,73.78555065)(511.18577006,74.00950933)(511.18577006,74.25950934)
\curveto(511.18577006,74.50950934)(511.2743114,74.72305069)(511.45660341,74.90534269)
\curveto(511.63368741,75.0824267)(511.85764476,75.17617603)(512.1232701,75.17617603)
\closepath
\moveto(512.87327012,70.82200921)
\lineto(512.87327012,64.11367564)
\curveto(512.87327012,63.58242629)(512.90452079,63.22825961)(512.97743679,63.05117561)
\curveto(513.06077013,62.88450893)(513.17535414,62.75950893)(513.33160347,62.67617559)
\curveto(513.48264481,62.59284226)(513.75347816,62.55117559)(514.1441035,62.55117559)
\lineto(514.1441035,62.21784224)
\lineto(510.08160336,62.21784224)
\lineto(510.08160336,62.55117559)
\curveto(510.49827004,62.55117559)(510.77431138,62.58242626)(510.91493672,62.65534226)
\curveto(511.05035406,62.73867559)(511.16493673,62.8688836)(511.24827007,63.05117561)
\curveto(511.3316034,63.22825961)(511.37327007,63.58242629)(511.37327007,64.11367564)
\lineto(511.37327007,67.32200909)
\curveto(511.37327007,68.22305046)(511.34202074,68.80638381)(511.28993674,69.07200915)
\curveto(511.24827007,69.26471716)(511.17535406,69.3949265)(511.08160339,69.4678425)
\curveto(510.99827006,69.55117584)(510.87327005,69.5928425)(510.70660338,69.5928425)
\curveto(510.53993671,69.5928425)(510.33160337,69.55117584)(510.08160336,69.4678425)
\lineto(509.95660335,69.80117584)
\lineto(512.47743678,70.82200921)
\closepath
\moveto(512.87327012,70.82200921)
}
}
{
\newrgbcolor{curcolor}{0 0 0}
\pscustom[linestyle=none,fillstyle=solid,fillcolor=curcolor]
{
\newpath
\moveto(517.62510135,70.82200921)
\lineto(517.62510135,68.92617581)
\curveto(518.33343471,70.18659319)(519.0521854,70.82200921)(519.79176809,70.82200921)
\curveto(520.12510144,70.82200921)(520.39593478,70.71784254)(520.60426812,70.5095092)
\curveto(520.82301879,70.30117586)(520.93760147,70.06159319)(520.93760147,69.80117584)
\curveto(520.93760147,69.56159317)(520.85426813,69.36367583)(520.68760146,69.19700916)
\curveto(520.53135212,69.03034248)(520.35426811,68.94700915)(520.14593477,68.94700915)
\curveto(519.9219761,68.94700915)(519.67718542,69.05117582)(519.41676808,69.25950916)
\curveto(519.15114274,69.47825983)(518.95843473,69.5928425)(518.83343472,69.5928425)
\curveto(518.71885205,69.5928425)(518.60426805,69.5303425)(518.47926804,69.4053425)
\curveto(518.1980187,69.15534249)(517.91676802,68.74388381)(517.62510135,68.17617579)
\lineto(517.62510135,64.17617565)
\curveto(517.62510135,63.70221696)(517.68760135,63.34805028)(517.81260135,63.11367561)
\curveto(517.88030936,62.95742627)(518.01051869,62.82721693)(518.2084347,62.71784226)
\curveto(518.41676804,62.60325959)(518.70843472,62.55117559)(519.08343473,62.55117559)
\lineto(519.08343473,62.21784224)
\lineto(514.81260125,62.21784224)
\lineto(514.81260125,62.55117559)
\curveto(515.23968526,62.55117559)(515.56260127,62.61367559)(515.77093462,62.73867559)
\curveto(515.92197595,62.83242626)(516.02614262,62.9938836)(516.08343463,63.21784228)
\curveto(516.10947596,63.31159295)(516.12510129,63.60325962)(516.12510129,64.09284231)
\lineto(516.12510129,67.32200909)
\curveto(516.12510129,68.29075979)(516.10426796,68.86888381)(516.06260129,69.05117582)
\curveto(516.02093462,69.24388382)(515.94280929,69.38450916)(515.83343462,69.4678425)
\curveto(515.73447595,69.55117584)(515.60947594,69.5928425)(515.4584346,69.5928425)
\curveto(515.2605186,69.5928425)(515.04697592,69.55117584)(514.81260125,69.4678425)
\lineto(514.72926791,69.80117584)
\lineto(517.25010133,70.82200921)
\closepath
\moveto(517.62510135,70.82200921)
}
}
{
\newrgbcolor{curcolor}{0 0 0}
\pscustom[linestyle=none,fillstyle=solid,fillcolor=curcolor]
{
\newpath
\moveto(495.45713181,190.77204602)
\curveto(496.08213183,190.77204602)(496.61338252,190.54808734)(497.06129854,190.10537933)
\curveto(497.51963189,189.65746331)(497.74879856,189.11579662)(497.74879856,188.48037927)
\curveto(497.74879856,187.85537925)(497.51963189,187.31371256)(497.06129854,186.85537921)
\curveto(496.61338252,186.40746319)(496.08213183,186.18871252)(495.45713181,186.18871252)
\curveto(494.81650646,186.18871252)(494.27483977,186.40746319)(493.83213175,186.85537921)
\curveto(493.38421574,187.31371256)(493.16546506,187.85537925)(493.16546506,188.48037927)
\curveto(493.16546506,189.11579662)(493.38421574,189.65746331)(493.83213175,190.10537933)
\curveto(494.27483977,190.54808734)(494.81650646,190.77204602)(495.45713181,190.77204602)
\closepath
\moveto(495.45713181,190.77204602)
}
}
{
\newrgbcolor{curcolor}{0 0 0}
\pscustom[linestyle=none,fillstyle=solid,fillcolor=curcolor]
{
\newpath
\moveto(506.65716476,190.77204602)
\curveto(507.28216478,190.77204602)(507.81341547,190.54808734)(508.26133148,190.10537933)
\curveto(508.71966483,189.65746331)(508.94883151,189.11579662)(508.94883151,188.48037927)
\curveto(508.94883151,187.85537925)(508.71966483,187.31371256)(508.26133148,186.85537921)
\curveto(507.81341547,186.40746319)(507.28216478,186.18871252)(506.65716476,186.18871252)
\curveto(506.01653941,186.18871252)(505.47487272,186.40746319)(505.0321647,186.85537921)
\curveto(504.58424869,187.31371256)(504.36549801,187.85537925)(504.36549801,188.48037927)
\curveto(504.36549801,189.11579662)(504.58424869,189.65746331)(505.0321647,190.10537933)
\curveto(505.47487272,190.54808734)(506.01653941,190.77204602)(506.65716476,190.77204602)
\closepath
\moveto(506.65716476,190.77204602)
}
}
{
\newrgbcolor{curcolor}{0 0 0}
\pscustom[linestyle=none,fillstyle=solid,fillcolor=curcolor]
{
\newpath
\moveto(517.85715702,190.77204602)
\curveto(518.48215704,190.77204602)(519.01340773,190.54808734)(519.46132374,190.10537933)
\curveto(519.91965709,189.65746331)(520.14882377,189.11579662)(520.14882377,188.48037927)
\curveto(520.14882377,187.85537925)(519.91965709,187.31371256)(519.46132374,186.85537921)
\curveto(519.01340773,186.40746319)(518.48215704,186.18871252)(517.85715702,186.18871252)
\curveto(517.21653166,186.18871252)(516.67486498,186.40746319)(516.23215696,186.85537921)
\curveto(515.78424095,187.31371256)(515.56549027,187.85537925)(515.56549027,188.48037927)
\curveto(515.56549027,189.11579662)(515.78424095,189.65746331)(516.23215696,190.10537933)
\curveto(516.67486498,190.54808734)(517.21653166,190.77204602)(517.85715702,190.77204602)
\closepath
\moveto(517.85715702,190.77204602)
}
}
{
\newrgbcolor{curcolor}{0 0 0}
\pscustom[linestyle=none,fillstyle=solid,fillcolor=curcolor]
{
\newpath
\moveto(403.20086227,68.9480264)
\curveto(403.82586229,68.9480264)(404.35711298,68.72406773)(404.80502899,68.28135971)
\curveto(405.26336234,67.83344369)(405.49252902,67.29177701)(405.49252902,66.65635965)
\curveto(405.49252902,66.03135963)(405.26336234,65.48969294)(404.80502899,65.03135959)
\curveto(404.35711298,64.58344358)(403.82586229,64.3646929)(403.20086227,64.3646929)
\curveto(402.56023691,64.3646929)(402.01857023,64.58344358)(401.57586221,65.03135959)
\curveto(401.12794619,65.48969294)(400.90919552,66.03135963)(400.90919552,66.65635965)
\curveto(400.90919552,67.29177701)(401.12794619,67.83344369)(401.57586221,68.28135971)
\curveto(402.01857023,68.72406773)(402.56023691,68.9480264)(403.20086227,68.9480264)
\closepath
\moveto(403.20086227,68.9480264)
}
}
{
\newrgbcolor{curcolor}{0 0 0}
\pscustom[linestyle=none,fillstyle=solid,fillcolor=curcolor]
{
\newpath
\moveto(414.40085453,68.9480264)
\curveto(415.02585455,68.9480264)(415.55710523,68.72406773)(416.00502125,68.28135971)
\curveto(416.4633546,67.83344369)(416.69252127,67.29177701)(416.69252127,66.65635965)
\curveto(416.69252127,66.03135963)(416.4633546,65.48969294)(416.00502125,65.03135959)
\curveto(415.55710523,64.58344358)(415.02585455,64.3646929)(414.40085453,64.3646929)
\curveto(413.76022917,64.3646929)(413.21856248,64.58344358)(412.77585447,65.03135959)
\curveto(412.32793845,65.48969294)(412.10918778,66.03135963)(412.10918778,66.65635965)
\curveto(412.10918778,67.29177701)(412.32793845,67.83344369)(412.77585447,68.28135971)
\curveto(413.21856248,68.72406773)(413.76022917,68.9480264)(414.40085453,68.9480264)
\closepath
\moveto(414.40085453,68.9480264)
}
}
{
\newrgbcolor{curcolor}{0 0 0}
\pscustom[linestyle=none,fillstyle=solid,fillcolor=curcolor]
{
\newpath
\moveto(425.60088748,68.9480264)
\curveto(426.2258875,68.9480264)(426.75713818,68.72406773)(427.2050542,68.28135971)
\curveto(427.66338755,67.83344369)(427.89255422,67.29177701)(427.89255422,66.65635965)
\curveto(427.89255422,66.03135963)(427.66338755,65.48969294)(427.2050542,65.03135959)
\curveto(426.75713818,64.58344358)(426.2258875,64.3646929)(425.60088748,64.3646929)
\curveto(424.96026212,64.3646929)(424.41859543,64.58344358)(423.97588742,65.03135959)
\curveto(423.5279714,65.48969294)(423.30922073,66.03135963)(423.30922073,66.65635965)
\curveto(423.30922073,67.29177701)(423.5279714,67.83344369)(423.97588742,68.28135971)
\curveto(424.41859543,68.72406773)(424.96026212,68.9480264)(425.60088748,68.9480264)
\closepath
\moveto(425.60088748,68.9480264)
}
}
{
\newrgbcolor{curcolor}{0 0 0}
\pscustom[linestyle=none,fillstyle=solid,fillcolor=curcolor]
{
\newpath
\moveto(219.87104213,186.70006508)
\lineto(219.76687546,184.13756499)
\lineto(212.41270854,184.13756499)
\lineto(212.41270854,184.47089833)
\lineto(217.9543754,191.84589859)
\lineto(215.22520863,191.84589859)
\curveto(214.62624995,191.84589859)(214.23562593,191.80423192)(214.05854193,191.72089859)
\curveto(213.87624992,191.64798259)(213.73041658,191.51256525)(213.62104191,191.30423191)
\curveto(213.45437524,190.9969399)(213.35541657,190.62193988)(213.32937523,190.17923187)
\lineto(212.97520855,190.17923187)
\lineto(213.01687522,192.49173195)
\lineto(220.01687547,192.49173195)
\lineto(220.01687547,192.1583986)
\lineto(214.43354194,184.74173167)
\lineto(217.47520871,184.74173167)
\curveto(218.11062607,184.74173167)(218.54291675,184.79381568)(218.76687543,184.90839835)
\curveto(219.0012501,185.01777302)(219.18875011,185.20527302)(219.32937545,185.47089837)
\curveto(219.42312612,185.66360637)(219.50125012,186.07506506)(219.55854212,186.70006508)
\closepath
\moveto(219.87104213,186.70006508)
}
}
{
\newrgbcolor{curcolor}{0 0 0}
\pscustom[linestyle=none,fillstyle=solid,fillcolor=curcolor]
{
\newpath
\moveto(220.47152637,192.49173195)
\lineto(224.40902651,192.49173195)
\lineto(224.40902651,192.1583986)
\lineto(224.1590265,192.1583986)
\curveto(223.91944382,192.1583986)(223.74235982,192.0958986)(223.61735981,191.9708986)
\curveto(223.49235981,191.85631593)(223.4298598,191.70527325)(223.4298598,191.51256525)
\curveto(223.4298598,191.30423191)(223.49235981,191.0542319)(223.61735981,190.76256522)
\lineto(225.55485988,186.13756506)
\lineto(227.51319328,190.92923189)
\curveto(227.64861062,191.26256524)(227.72152662,191.51777325)(227.72152662,191.70006525)
\curveto(227.72152662,191.78339859)(227.69027729,191.85110659)(227.63819329,191.9083986)
\curveto(227.58090128,192.00214927)(227.49756795,192.06464927)(227.38819328,192.0958986)
\curveto(227.28923461,192.13756527)(227.08090127,192.1583986)(226.76319326,192.1583986)
\lineto(226.76319326,192.49173195)
\lineto(229.49236002,192.49173195)
\lineto(229.49236002,192.1583986)
\curveto(229.16944401,192.12714927)(228.95069333,192.05943993)(228.82569333,191.95006526)
\curveto(228.61735999,191.76777326)(228.42985998,191.47089858)(228.26319331,191.0542319)
\lineto(225.28402654,183.88756498)
\lineto(224.92985986,183.88756498)
\lineto(221.92985975,190.92923189)
\curveto(221.80485975,191.26256524)(221.67985974,191.49693991)(221.55485974,191.63756525)
\curveto(221.42985973,191.77298259)(221.27361039,191.89277326)(221.09652639,191.99173193)
\curveto(220.98194372,192.04381593)(220.77361038,192.1011066)(220.47152637,192.1583986)
\closepath
\moveto(220.47152637,192.49173195)
}
}
{
\newrgbcolor{curcolor}{0 0 0}
\pscustom[linestyle=none,fillstyle=solid,fillcolor=curcolor]
{
\newpath
\moveto(233.95230077,192.74173196)
\curveto(235.21271815,192.74173196)(236.22834218,192.25214927)(236.99396754,191.28339857)
\curveto(237.6450089,190.46048254)(237.97313425,189.52298251)(237.97313425,188.47089847)
\curveto(237.97313425,187.72089845)(237.79084224,186.96048242)(237.43146756,186.20006506)
\curveto(237.06688488,185.4344397)(236.57730086,184.85631568)(235.95230084,184.47089833)
\curveto(235.32730082,184.08548165)(234.62417546,183.88756498)(233.8481341,183.88756498)
\curveto(232.59813405,183.88756498)(231.59813402,184.387565)(230.84813399,185.38756503)
\curveto(230.22313397,186.23131573)(229.91063396,187.17923176)(229.91063396,188.22089846)
\curveto(229.91063396,188.99693983)(230.09813397,189.76256519)(230.47313398,190.51256521)
\curveto(230.85855133,191.27298257)(231.35855134,191.83548259)(231.97313403,192.20006527)
\curveto(232.59813405,192.55943995)(233.25438474,192.74173196)(233.95230077,192.74173196)
\closepath
\moveto(233.66063409,192.13756527)
\curveto(233.33771808,192.13756527)(233.01480074,192.0386066)(232.68146739,191.84589859)
\curveto(232.35855138,191.66360659)(232.10334204,191.33027324)(231.91063403,190.84589856)
\curveto(231.71271802,190.35631587)(231.61896735,189.74173185)(231.61896735,188.99173183)
\curveto(231.61896735,187.78339845)(231.85334203,186.73131575)(232.32730071,185.84589838)
\curveto(232.8116754,184.97089835)(233.45230075,184.53339833)(234.24396745,184.53339833)
\curveto(234.8273008,184.53339833)(235.31167548,184.77298234)(235.70230083,185.26256503)
\curveto(236.08771818,185.74693971)(236.28563419,186.58027307)(236.28563419,187.76256511)
\curveto(236.28563419,189.24693983)(235.96271817,190.41360654)(235.32730082,191.26256524)
\curveto(234.8950088,191.84589859)(234.33771812,192.13756527)(233.66063409,192.13756527)
\closepath
\moveto(233.66063409,192.13756527)
}
}
{
\newrgbcolor{curcolor}{0 0 0}
\pscustom[linestyle=none,fillstyle=solid,fillcolor=curcolor]
{
\newpath
\moveto(242.07730106,197.09589878)
\lineto(242.07730106,186.03339839)
\curveto(242.07730106,185.50214903)(242.10855173,185.14798236)(242.18146773,184.97089835)
\curveto(242.26480106,184.80423168)(242.37938507,184.67923167)(242.53563441,184.59589834)
\curveto(242.70230108,184.512565)(242.99917576,184.47089833)(243.43146777,184.47089833)
\lineto(243.43146777,184.13756499)
\lineto(239.32730096,184.13756499)
\lineto(239.32730096,184.47089833)
\curveto(239.71271831,184.47089833)(239.97834232,184.502149)(240.11896765,184.575065)
\curveto(240.25438499,184.65839834)(240.35855166,184.78860634)(240.43146767,184.97089835)
\curveto(240.514801,185.14798236)(240.55646767,185.50214903)(240.55646767,186.03339839)
\lineto(240.55646767,193.61673199)
\curveto(240.55646767,194.54381602)(240.53563434,195.11673204)(240.49396767,195.32506538)
\curveto(240.452301,195.54381606)(240.379385,195.7000654)(240.28563433,195.78339873)
\curveto(240.20230099,195.86673207)(240.08250899,195.90839874)(239.93146765,195.90839874)
\curveto(239.77521831,195.90839874)(239.57730097,195.85631607)(239.32730096,195.7625654)
\lineto(239.18146762,196.07506541)
\lineto(241.66063438,197.09589878)
\closepath
\moveto(242.07730106,197.09589878)
}
}
{
\newrgbcolor{curcolor}{0 0 0}
\pscustom[linestyle=none,fillstyle=solid,fillcolor=curcolor]
{
\newpath
\moveto(401.80547648,187.30423177)
\curveto(401.5815178,186.20527306)(401.14401779,185.3563157)(400.49297643,184.76256501)
\curveto(399.83672707,184.17923165)(399.10756038,183.88756498)(398.30547635,183.88756498)
\curveto(397.35756032,183.88756498)(396.53464296,184.28339833)(395.8263096,185.07506502)
\curveto(395.12839357,185.86673171)(394.78464289,186.93964909)(394.78464289,188.3042318)
\curveto(394.78464289,189.60631585)(395.17006024,190.66881588)(395.9513096,191.49173191)
\curveto(396.72735096,192.32506528)(397.664851,192.74173196)(398.7638097,192.74173196)
\curveto(399.58151773,192.74173196)(400.25339375,192.51777328)(400.78464311,192.07506527)
\curveto(401.31068446,191.64277325)(401.5763098,191.18964924)(401.5763098,190.72089855)
\curveto(401.5763098,190.49693988)(401.49818446,190.30943987)(401.34714313,190.15839853)
\curveto(401.20651779,190.01777319)(400.99818445,189.95006519)(400.7221431,189.95006519)
\curveto(400.37318443,189.95006519)(400.10235108,190.0646492)(399.90964308,190.30423187)
\curveto(399.81068441,190.42923188)(399.7429764,190.66881588)(399.70130973,191.03339856)
\curveto(399.6700604,191.39277324)(399.5554764,191.66360659)(399.34714306,191.84589859)
\curveto(399.13880971,192.0229826)(398.83672704,192.11673194)(398.45130969,192.11673194)
\curveto(397.852351,192.11673194)(397.36797632,191.89277326)(396.99297631,191.45006525)
\curveto(396.50339362,190.85110656)(396.26380961,190.0646492)(396.26380961,189.0958985)
\curveto(396.26380961,188.09589846)(396.50339362,187.21048243)(396.99297631,186.45006507)
\curveto(397.47735099,185.68443971)(398.13880968,185.30423169)(398.97214304,185.30423169)
\curveto(399.56589373,185.30423169)(400.10235108,185.50214903)(400.57630977,185.90839838)
\curveto(400.90964311,186.18443973)(401.23256046,186.68964908)(401.55547647,187.42923177)
\closepath
\moveto(401.80547648,187.30423177)
}
}
{
\newrgbcolor{curcolor}{0 0 0}
\pscustom[linestyle=none,fillstyle=solid,fillcolor=curcolor]
{
\newpath
\moveto(405.88514784,197.09589878)
\lineto(405.88514784,186.03339839)
\curveto(405.88514784,185.50214903)(405.91639851,185.14798236)(405.98931451,184.97089835)
\curveto(406.07264785,184.80423168)(406.18723186,184.67923167)(406.34348119,184.59589834)
\curveto(406.51014787,184.512565)(406.80702254,184.47089833)(407.23931456,184.47089833)
\lineto(407.23931456,184.13756499)
\lineto(403.13514775,184.13756499)
\lineto(403.13514775,184.47089833)
\curveto(403.52056509,184.47089833)(403.7861891,184.502149)(403.92681444,184.575065)
\curveto(404.06223178,184.65839834)(404.16639845,184.78860634)(404.23931445,184.97089835)
\curveto(404.32264779,185.14798236)(404.36431446,185.50214903)(404.36431446,186.03339839)
\lineto(404.36431446,193.61673199)
\curveto(404.36431446,194.54381602)(404.34348112,195.11673204)(404.30181445,195.32506538)
\curveto(404.26014779,195.54381606)(404.18723178,195.7000654)(404.09348111,195.78339873)
\curveto(404.01014778,195.86673207)(403.89035577,195.90839874)(403.73931443,195.90839874)
\curveto(403.5830651,195.90839874)(403.38514776,195.85631607)(403.13514775,195.7625654)
\lineto(402.98931441,196.07506541)
\lineto(405.46848116,197.09589878)
\closepath
\moveto(405.88514784,197.09589878)
}
}
{
\newrgbcolor{curcolor}{0 0 0}
\pscustom[linestyle=none,fillstyle=solid,fillcolor=curcolor]
{
\newpath
\moveto(412.28281246,192.74173196)
\curveto(413.54322984,192.74173196)(414.55885387,192.25214927)(415.32447923,191.28339857)
\curveto(415.97552059,190.46048254)(416.30364593,189.52298251)(416.30364593,188.47089847)
\curveto(416.30364593,187.72089845)(416.12135393,186.96048242)(415.76197925,186.20006506)
\curveto(415.39739657,185.4344397)(414.90781255,184.85631568)(414.28281253,184.47089833)
\curveto(413.65781251,184.08548165)(412.95468715,183.88756498)(412.17864579,183.88756498)
\curveto(410.92864574,183.88756498)(409.92864571,184.387565)(409.17864568,185.38756503)
\curveto(408.55364566,186.23131573)(408.24114565,187.17923176)(408.24114565,188.22089846)
\curveto(408.24114565,188.99693983)(408.42864566,189.76256519)(408.80364567,190.51256521)
\curveto(409.18906302,191.27298257)(409.68906303,191.83548259)(410.30364572,192.20006527)
\curveto(410.92864574,192.55943995)(411.58489643,192.74173196)(412.28281246,192.74173196)
\closepath
\moveto(411.99114578,192.13756527)
\curveto(411.66822977,192.13756527)(411.34531243,192.0386066)(411.01197908,191.84589859)
\curveto(410.68906307,191.66360659)(410.43385373,191.33027324)(410.24114572,190.84589856)
\curveto(410.04322971,190.35631587)(409.94947904,189.74173185)(409.94947904,188.99173183)
\curveto(409.94947904,187.78339845)(410.18385372,186.73131575)(410.6578124,185.84589838)
\curveto(411.14218709,184.97089835)(411.78281244,184.53339833)(412.57447914,184.53339833)
\curveto(413.15781249,184.53339833)(413.64218717,184.77298234)(414.03281252,185.26256503)
\curveto(414.41822987,185.74693971)(414.61614588,186.58027307)(414.61614588,187.76256511)
\curveto(414.61614588,189.24693983)(414.29322986,190.41360654)(413.65781251,191.26256524)
\curveto(413.22552049,191.84589859)(412.66822981,192.13756527)(411.99114578,192.13756527)
\closepath
\moveto(411.99114578,192.13756527)
}
}
{
\newrgbcolor{curcolor}{0 0 0}
\pscustom[linestyle=none,fillstyle=solid,fillcolor=curcolor]
{
\newpath
\moveto(419.97031273,190.97089856)
\curveto(420.93906343,192.1479826)(421.87135413,192.74173196)(422.7619795,192.74173196)
\curveto(423.22031285,192.74173196)(423.60573019,192.62193995)(423.92864621,192.38756528)
\curveto(424.26197955,192.1636066)(424.52239689,191.78860659)(424.7203129,191.26256524)
\curveto(424.85573024,190.89798256)(424.92864624,190.34589854)(424.92864624,189.59589851)
\lineto(424.92864624,186.03339839)
\curveto(424.92864624,185.50214903)(424.97031291,185.14277302)(425.05364624,184.95006502)
\curveto(425.12135425,184.79381568)(425.22552092,184.67923167)(425.36614626,184.59589834)
\curveto(425.51718759,184.512565)(425.79323027,184.47089833)(426.19947962,184.47089833)
\lineto(426.19947962,184.13756499)
\lineto(422.07447947,184.13756499)
\lineto(422.07447947,184.47089833)
\lineto(422.24114615,184.47089833)
\curveto(422.62656349,184.47089833)(422.89739684,184.52298233)(423.05364617,184.637565)
\curveto(423.22031285,184.76256501)(423.32968752,184.93443968)(423.38697952,185.15839836)
\curveto(423.41302085,185.25214903)(423.42864619,185.5438157)(423.42864619,186.03339839)
\lineto(423.42864619,189.45006517)
\curveto(423.42864619,190.2000652)(423.32447952,190.74693989)(423.11614618,191.09589857)
\curveto(422.91823017,191.43964925)(422.59531282,191.61673192)(422.13697947,191.61673192)
\curveto(421.39739678,191.61673192)(420.67864609,191.22089857)(419.97031273,190.42923188)
\lineto(419.97031273,186.03339839)
\curveto(419.97031273,185.46048237)(420.0015634,185.10631569)(420.0744794,184.97089835)
\curveto(420.15781274,184.80423168)(420.26718741,184.67923167)(420.40781275,184.59589834)
\curveto(420.55885409,184.512565)(420.8661461,184.47089833)(421.32447945,184.47089833)
\lineto(421.32447945,184.13756499)
\lineto(417.1994793,184.13756499)
\lineto(417.1994793,184.47089833)
\lineto(417.38697931,184.47089833)
\curveto(417.80364599,184.47089833)(418.08489666,184.575065)(418.241146,184.78339834)
\curveto(418.39218734,185.00214902)(418.47031268,185.4188157)(418.47031268,186.03339839)
\lineto(418.47031268,189.11673183)
\curveto(418.47031268,190.1271492)(418.43906334,190.74173189)(418.38697934,190.95006523)
\curveto(418.34531267,191.1688159)(418.27239667,191.31464924)(418.178646,191.38756524)
\curveto(418.09531266,191.47089858)(417.97031266,191.51256525)(417.80364599,191.51256525)
\curveto(417.63697932,191.51256525)(417.43385397,191.47089858)(417.1994793,191.38756524)
\lineto(417.05364596,191.72089859)
\lineto(419.57447938,192.74173196)
\lineto(419.97031273,192.74173196)
\closepath
\moveto(419.97031273,190.97089856)
}
}
{
\newrgbcolor{curcolor}{0 0 0}
\pscustom[linestyle=none,fillstyle=solid,fillcolor=curcolor]
{
\newpath
\moveto(428.26197969,189.3458985)
\curveto(428.26197969,188.10631579)(428.5640637,187.13756509)(429.17864639,186.42923173)
\curveto(429.77239708,185.72089838)(430.49114644,185.3667317)(431.3244798,185.3667317)
\curveto(431.86614649,185.3667317)(432.33489717,185.51256504)(432.74114652,185.80423171)
\curveto(433.14218786,186.10631572)(433.48073054,186.62714907)(433.76197989,187.36673177)
\lineto(434.03281323,187.17923176)
\curveto(433.90781323,186.3458984)(433.53802121,185.58548237)(432.92864652,184.90839835)
\curveto(432.31406384,184.22610672)(431.54323047,183.88756498)(430.61614644,183.88756498)
\curveto(429.61614641,183.88756498)(428.75156371,184.27298232)(428.03281302,185.05423169)
\curveto(427.32447966,185.83027305)(426.97031298,186.87714908)(426.97031298,188.20006513)
\curveto(426.97031298,189.62714918)(427.33489699,190.74173189)(428.07447969,191.53339858)
\curveto(428.80885438,192.33548261)(429.72552108,192.74173196)(430.82447978,192.74173196)
\curveto(431.76718782,192.74173196)(432.53802118,192.42923195)(433.13697986,191.80423192)
\curveto(433.73073055,191.18964924)(434.03281323,190.37193987)(434.03281323,189.3458985)
\closepath
\moveto(428.26197969,189.86673186)
\lineto(432.13697983,189.86673186)
\curveto(432.10573049,190.40839854)(432.04323049,190.78339856)(431.94947982,190.9917319)
\curveto(431.79323048,191.33548257)(431.56406381,191.60631592)(431.2619798,191.80423192)
\curveto(430.95468779,191.99693993)(430.64218778,192.0958986)(430.32447976,192.0958986)
\curveto(429.82447975,192.0958986)(429.3713544,191.89798259)(428.97031305,191.51256525)
\curveto(428.5640637,191.1219399)(428.32968769,190.57506521)(428.26197969,189.86673186)
\closepath
\moveto(428.26197969,189.86673186)
}
}
{
\newrgbcolor{curcolor}{0 0 0}
\pscustom[linestyle=none,fillstyle=solid,fillcolor=curcolor]
{
\newpath
\moveto(26.96686049,69.03034248)
\curveto(27.57623518,69.63971717)(27.93561119,69.99388385)(28.05019386,70.09284252)
\curveto(28.31061121,70.3115932)(28.60227788,70.48867587)(28.9251939,70.61367587)
\curveto(29.24290191,70.74909321)(29.55540192,70.82200921)(29.86269393,70.82200921)
\curveto(30.38873528,70.82200921)(30.84186063,70.66575987)(31.21686064,70.36367586)
\curveto(31.59186066,70.05638385)(31.84186067,69.61367584)(31.96686067,69.03034248)
\curveto(32.59186069,69.76471718)(33.11790204,70.24388386)(33.55019406,70.46784253)
\curveto(33.97727807,70.70221721)(34.42519409,70.82200921)(34.88352744,70.82200921)
\curveto(35.32623546,70.82200921)(35.7220688,70.70221721)(36.07102748,70.46784253)
\curveto(36.41477816,70.24388386)(36.6856115,69.87409318)(36.88352751,69.36367583)
\curveto(37.01894485,68.99909315)(37.09186085,68.44700913)(37.09186085,67.6970091)
\lineto(37.09186085,64.11367564)
\curveto(37.09186085,63.58242629)(37.12311152,63.22305028)(37.19602752,63.03034227)
\curveto(37.26373552,62.88971693)(37.37311153,62.76992626)(37.52936087,62.67617559)
\curveto(37.69602754,62.59284226)(37.96686088,62.55117559)(38.3418609,62.55117559)
\lineto(38.3418609,62.21784224)
\lineto(34.21686075,62.21784224)
\lineto(34.21686075,62.55117559)
\lineto(34.40436076,62.55117559)
\curveto(34.74811144,62.55117559)(35.02936078,62.61888359)(35.23769412,62.75950893)
\curveto(35.37311146,62.8532596)(35.47206879,63.00950894)(35.5293608,63.21784228)
\curveto(35.55540213,63.32721695)(35.57102746,63.62409296)(35.57102746,64.11367564)
\lineto(35.57102746,67.6970091)
\curveto(35.57102746,68.37409313)(35.48769413,68.85325981)(35.32102746,69.13450915)
\curveto(35.08144478,69.5199265)(34.70644477,69.71784251)(34.19602742,69.71784251)
\curveto(33.86269407,69.71784251)(33.53456873,69.63450917)(33.21686071,69.4678425)
\curveto(32.8939447,69.31159316)(32.49811136,69.01471715)(32.02936067,68.57200913)
\lineto(32.00852734,68.4886758)
\lineto(32.02936067,68.09284245)
\lineto(32.02936067,64.11367564)
\curveto(32.02936067,63.53034229)(32.05540201,63.16575961)(32.11269401,63.03034227)
\curveto(32.18040201,62.88971693)(32.30540202,62.76992626)(32.48769402,62.67617559)
\curveto(32.66477803,62.59284226)(32.96686071,62.55117559)(33.38352739,62.55117559)
\lineto(33.38352739,62.21784224)
\lineto(29.1751939,62.21784224)
\lineto(29.1751939,62.55117559)
\curveto(29.63352725,62.55117559)(29.94602727,62.60325959)(30.11269394,62.71784226)
\curveto(30.28977794,62.82721693)(30.41477795,62.9886756)(30.48769395,63.19700894)
\curveto(30.51373529,63.30638361)(30.52936062,63.61367563)(30.52936062,64.11367564)
\lineto(30.52936062,67.6970091)
\curveto(30.52936062,68.37409313)(30.43040195,68.86367581)(30.23769394,69.15534249)
\curveto(29.9564446,69.54075984)(29.58144459,69.73867584)(29.1126939,69.73867584)
\curveto(28.77936056,69.73867584)(28.45123521,69.65534251)(28.1335272,69.48867583)
\curveto(27.63352718,69.20742649)(27.24290184,68.90534248)(26.96686049,68.57200913)
\lineto(26.96686049,64.11367564)
\curveto(26.96686049,63.55638362)(26.99811116,63.19700894)(27.07102716,63.03034227)
\curveto(27.1543605,62.87409293)(27.26373517,62.75950893)(27.40436051,62.67617559)
\curveto(27.55540185,62.59284226)(27.86269386,62.55117559)(28.32102721,62.55117559)
\lineto(28.32102721,62.21784224)
\lineto(24.19602706,62.21784224)
\lineto(24.19602706,62.55117559)
\curveto(24.58144441,62.55117559)(24.84706842,62.59284226)(24.98769376,62.67617559)
\curveto(25.1387351,62.75950893)(25.2585271,62.88450893)(25.34186044,63.05117561)
\curveto(25.42519377,63.22825961)(25.46686044,63.58242629)(25.46686044,64.11367564)
\lineto(25.46686044,67.30117576)
\curveto(25.46686044,68.21784245)(25.43561111,68.80638381)(25.3835271,69.07200915)
\curveto(25.34186044,69.26471716)(25.26894443,69.3949265)(25.17519376,69.4678425)
\curveto(25.09186043,69.55117584)(24.96686042,69.5928425)(24.80019375,69.5928425)
\curveto(24.63352708,69.5928425)(24.43040174,69.55117584)(24.19602706,69.4678425)
\lineto(24.05019372,69.80117584)
\lineto(26.57102715,70.82200921)
\lineto(26.96686049,70.82200921)
\closepath
\moveto(26.96686049,69.03034248)
}
}
{
\newrgbcolor{curcolor}{0 0 0}
\pscustom[linestyle=none,fillstyle=solid,fillcolor=curcolor]
{
\newpath
\moveto(43.73952548,63.42617562)
\curveto(42.87494278,62.75950893)(42.34369209,62.37409291)(42.13535875,62.28034224)
\curveto(41.80202541,62.12930051)(41.44785873,62.05117557)(41.07285871,62.05117557)
\curveto(40.49994269,62.05117557)(40.03119201,62.24388398)(39.656192,62.63450892)
\curveto(39.28119198,63.03555027)(39.09369198,63.55638362)(39.09369198,64.19700898)
\curveto(39.09369198,64.59805033)(39.18223331,64.95221701)(39.36452532,65.25950902)
\curveto(39.61452533,65.66055036)(40.04160934,66.04075971)(40.65619203,66.40534239)
\curveto(41.28119206,66.7803424)(42.30723342,67.22825975)(43.73952548,67.75950911)
\lineto(43.73952548,68.07200912)
\curveto(43.73952548,68.90534248)(43.60410947,69.4730505)(43.34369213,69.78034251)
\curveto(43.07806679,70.09805052)(42.69785877,70.25950919)(42.19785875,70.25950919)
\curveto(41.80723341,70.25950919)(41.49994273,70.15534252)(41.28119206,69.94700918)
\curveto(41.04160938,69.73867584)(40.92702538,69.49909317)(40.92702538,69.23867582)
\lineto(40.94785871,68.71784247)
\curveto(40.94785871,68.4261758)(40.87494271,68.20221712)(40.73952537,68.05117578)
\curveto(40.59890003,67.91055044)(40.41140002,67.84284244)(40.17702535,67.84284244)
\curveto(39.95306667,67.84284244)(39.76556667,67.91575978)(39.61452533,68.07200912)
\curveto(39.47389999,68.22305046)(39.40619199,68.43659313)(39.40619199,68.71784247)
\curveto(39.40619199,69.24388382)(39.66660933,69.72305051)(40.19785868,70.15534252)
\curveto(40.73952537,70.59805054)(41.4947334,70.82200921)(42.4686921,70.82200921)
\curveto(43.20306679,70.82200921)(43.80723348,70.69700921)(44.28119216,70.4470092)
\curveto(44.64056684,70.24909319)(44.90619218,69.95221718)(45.07285886,69.55117584)
\curveto(45.18223353,69.28555049)(45.23952553,68.74909314)(45.23952553,67.94700911)
\lineto(45.23952553,65.11367568)
\curveto(45.23952553,64.32200898)(45.24993886,63.8324263)(45.2811922,63.65534229)
\curveto(45.32285886,63.47305029)(45.37494287,63.34805028)(45.44785887,63.28034228)
\curveto(45.51556687,63.22305028)(45.59369221,63.19700894)(45.67702554,63.19700894)
\curveto(45.77077621,63.19700894)(45.86452555,63.21784228)(45.94785889,63.25950895)
\curveto(46.07285889,63.34284228)(46.3124429,63.56159296)(46.67702558,63.92617564)
\lineto(46.67702558,63.42617562)
\curveto(45.99473356,62.50950892)(45.3436922,62.05117557)(44.71869218,62.05117557)
\curveto(44.4270255,62.05117557)(44.18744282,62.15534224)(44.01035882,62.36367558)
\curveto(43.84369215,62.57200892)(43.74994281,62.9261756)(43.73952548,63.42617562)
\closepath
\moveto(43.73952548,64.00950897)
\lineto(43.73952548,67.19700909)
\curveto(42.82285878,66.83242641)(42.22910942,66.57721706)(41.96869208,66.42617572)
\curveto(41.4791094,66.14492638)(41.13535872,65.8636757)(40.92702538,65.57200903)
\curveto(40.71869204,65.28034235)(40.61452536,64.96784234)(40.61452536,64.63450899)
\curveto(40.61452536,64.18659298)(40.74473337,63.82200897)(41.01035871,63.53034229)
\curveto(41.27077605,63.23867561)(41.57806673,63.09284227)(41.92702541,63.09284227)
\curveto(42.38535876,63.09284227)(42.98952545,63.39492628)(43.73952548,64.00950897)
\closepath
\moveto(43.73952548,64.00950897)
}
}
{
\newrgbcolor{curcolor}{0 0 0}
\pscustom[linestyle=none,fillstyle=solid,fillcolor=curcolor]
{
\newpath
\moveto(52.69419702,70.82200921)
\lineto(52.69419702,67.96784245)
\lineto(52.40253034,67.96784245)
\curveto(52.16294766,68.85325981)(51.86607165,69.4574265)(51.50669697,69.78034251)
\curveto(51.1421143,70.11367586)(50.68378095,70.28034253)(50.13169693,70.28034253)
\curveto(49.69940491,70.28034253)(49.35044756,70.16055052)(49.09003022,69.92617585)
\curveto(48.82440488,69.70221717)(48.69419688,69.4574265)(48.69419688,69.19700916)
\curveto(48.69419688,68.84805048)(48.78794754,68.5563838)(48.98586355,68.32200913)
\curveto(49.17857156,68.07200912)(49.56919691,67.80638377)(50.15253026,67.53034243)
\lineto(51.46503031,66.88450907)
\curveto(52.69940502,66.26992639)(53.31919704,65.47305036)(53.31919704,64.48867566)
\curveto(53.31919704,63.73867563)(53.03273836,63.12409294)(52.46503034,62.65534226)
\curveto(51.89211432,62.19700891)(51.24628097,61.96784223)(50.52753027,61.96784223)
\curveto(50.02753026,61.96784223)(49.4441969,62.06159157)(48.77753021,62.23867558)
\curveto(48.5796142,62.29075891)(48.42336353,62.32200891)(48.29836353,62.32200891)
\curveto(48.15773819,62.32200891)(48.04836352,62.24388398)(47.96503018,62.09284224)
\lineto(47.67336351,62.09284224)
\lineto(47.67336351,65.07200901)
\lineto(47.96503018,65.07200901)
\curveto(48.13169686,64.22305031)(48.44940487,63.58242629)(48.92336355,63.15534228)
\curveto(49.40773823,62.72305026)(49.94940492,62.50950892)(50.54836361,62.50950892)
\curveto(50.96503029,62.50950892)(51.30357163,62.62409292)(51.56919698,62.8636756)
\curveto(51.82961432,63.11367561)(51.96503032,63.41055028)(51.96503032,63.75950896)
\curveto(51.96503032,64.17617565)(51.81919699,64.51992632)(51.52753031,64.80117567)
\curveto(51.23586363,65.09284234)(50.64211428,65.45221702)(49.75669691,65.88450904)
\curveto(48.88169688,66.32721705)(48.30357153,66.72825974)(48.02753018,67.09284242)
\curveto(47.76190484,67.43659309)(47.63169684,67.87409311)(47.63169684,68.40534246)
\curveto(47.63169684,69.08242649)(47.86607151,69.65534251)(48.3400302,70.11367586)
\curveto(48.80878088,70.58242654)(49.41294757,70.82200921)(50.15253026,70.82200921)
\curveto(50.48586361,70.82200921)(50.88690495,70.74909321)(51.36086364,70.61367587)
\curveto(51.66294765,70.5147172)(51.86607165,70.46784253)(51.96503032,70.46784253)
\curveto(52.05878099,70.46784253)(52.136905,70.48867587)(52.194197,70.53034254)
\curveto(52.246281,70.5720092)(52.319197,70.66575987)(52.40253034,70.82200921)
\closepath
\moveto(52.69419702,70.82200921)
}
}
{
\newrgbcolor{curcolor}{0 0 0}
\pscustom[linestyle=none,fillstyle=solid,fillcolor=curcolor]
{
\newpath
\moveto(56.9763627,73.30117597)
\lineto(56.9763627,70.5720092)
\lineto(58.9346961,70.5720092)
\lineto(58.9346961,69.92617585)
\lineto(56.9763627,69.92617585)
\lineto(56.9763627,64.50950899)
\curveto(56.9763627,63.9678423)(57.04928003,63.59805029)(57.20552937,63.40534228)
\curveto(57.37219604,63.22305028)(57.57011338,63.13450894)(57.80969606,63.13450894)
\curveto(58.0180294,63.13450894)(58.21073741,63.19700894)(58.39302941,63.32200895)
\curveto(58.58573742,63.44700895)(58.73157076,63.63450896)(58.83052943,63.88450897)
\lineto(59.18469611,63.88450897)
\curveto(58.97636277,63.28555028)(58.67428009,62.83242626)(58.28886274,62.53034225)
\curveto(57.8982374,62.22305058)(57.49719605,62.0720089)(57.08052937,62.0720089)
\curveto(56.79928002,62.0720089)(56.52844668,62.15013384)(56.26802934,62.30117558)
\curveto(56.00240399,62.46784225)(55.79927999,62.68659293)(55.66386265,62.96784227)
\curveto(55.53886264,63.25950895)(55.47636264,63.7074263)(55.47636264,64.32200898)
\lineto(55.47636264,69.92617585)
\lineto(54.1638626,69.92617585)
\lineto(54.1638626,70.23867586)
\curveto(54.49719594,70.36367586)(54.83573729,70.58242654)(55.18469597,70.90534255)
\curveto(55.52844664,71.22305056)(55.84094666,71.60325991)(56.122196,72.05117592)
\curveto(56.25761334,72.2699266)(56.45552934,72.68659328)(56.70552935,73.30117597)
\closepath
\moveto(56.9763627,73.30117597)
}
}
{
\newrgbcolor{curcolor}{0 0 0}
\pscustom[linestyle=none,fillstyle=solid,fillcolor=curcolor]
{
\newpath
\moveto(61.14486056,67.42617576)
\curveto(61.14486056,66.18659305)(61.44694458,65.21784235)(62.06152726,64.50950899)
\curveto(62.65527795,63.80117563)(63.37402731,63.44700895)(64.20736067,63.44700895)
\curveto(64.74902736,63.44700895)(65.21777804,63.59284229)(65.62402739,63.88450897)
\curveto(66.02506874,64.18659298)(66.36361142,64.70742633)(66.64486076,65.44700902)
\lineto(66.9156941,65.25950902)
\curveto(66.7906941,64.42617565)(66.42090209,63.66575963)(65.8115274,62.9886756)
\curveto(65.19694471,62.30638398)(64.42611135,61.96784223)(63.49902731,61.96784223)
\curveto(62.49902728,61.96784223)(61.63444458,62.35325958)(60.91569389,63.13450894)
\curveto(60.20736053,63.9105503)(59.85319385,64.95742634)(59.85319385,66.28034239)
\curveto(59.85319385,67.70742644)(60.21777787,68.82200914)(60.95736056,69.61367584)
\curveto(61.69173525,70.41575987)(62.60840195,70.82200921)(63.70736066,70.82200921)
\curveto(64.65006869,70.82200921)(65.42090205,70.5095092)(66.01986074,69.88450918)
\curveto(66.61361143,69.26992649)(66.9156941,68.45221713)(66.9156941,67.42617576)
\closepath
\moveto(61.14486056,67.94700911)
\lineto(65.0198607,67.94700911)
\curveto(64.98861137,68.4886758)(64.92611137,68.86367581)(64.8323607,69.07200915)
\curveto(64.67611136,69.41575983)(64.44694468,69.68659317)(64.14486067,69.88450918)
\curveto(63.83756866,70.07721719)(63.52506865,70.17617586)(63.20736064,70.17617586)
\curveto(62.70736062,70.17617586)(62.25423527,69.97825985)(61.85319392,69.5928425)
\curveto(61.44694458,69.20221716)(61.21256857,68.65534247)(61.14486056,67.94700911)
\closepath
\moveto(61.14486056,67.94700911)
}
}
{
\newrgbcolor{curcolor}{0 0 0}
\pscustom[linestyle=none,fillstyle=solid,fillcolor=curcolor]
{
\newpath
\moveto(70.47453212,70.82200921)
\lineto(70.47453212,68.92617581)
\curveto(71.18286548,70.18659319)(71.90161617,70.82200921)(72.64119886,70.82200921)
\curveto(72.97453221,70.82200921)(73.24536555,70.71784254)(73.45369889,70.5095092)
\curveto(73.67244957,70.30117586)(73.78703224,70.06159319)(73.78703224,69.80117584)
\curveto(73.78703224,69.56159317)(73.7036989,69.36367583)(73.53703223,69.19700916)
\curveto(73.38078289,69.03034248)(73.20369888,68.94700915)(72.99536554,68.94700915)
\curveto(72.77140687,68.94700915)(72.52661619,69.05117582)(72.26619885,69.25950916)
\curveto(72.00057351,69.47825983)(71.8078655,69.5928425)(71.6828655,69.5928425)
\curveto(71.56828282,69.5928425)(71.45369882,69.5303425)(71.32869882,69.4053425)
\curveto(71.04744947,69.15534249)(70.7661988,68.74388381)(70.47453212,68.17617579)
\lineto(70.47453212,64.17617565)
\curveto(70.47453212,63.70221696)(70.53703212,63.34805028)(70.66203213,63.11367561)
\curveto(70.72974013,62.95742627)(70.85994947,62.82721693)(71.05786547,62.71784226)
\curveto(71.26619881,62.60325959)(71.55786549,62.55117559)(71.9328655,62.55117559)
\lineto(71.9328655,62.21784224)
\lineto(67.66203202,62.21784224)
\lineto(67.66203202,62.55117559)
\curveto(68.08911603,62.55117559)(68.41203205,62.61367559)(68.62036539,62.73867559)
\curveto(68.77140673,62.83242626)(68.8755734,62.9938836)(68.9328654,63.21784228)
\curveto(68.95890673,63.31159295)(68.97453207,63.60325962)(68.97453207,64.09284231)
\lineto(68.97453207,67.32200909)
\curveto(68.97453207,68.29075979)(68.95369873,68.86888381)(68.91203206,69.05117582)
\curveto(68.8703654,69.24388382)(68.79224006,69.38450916)(68.68286539,69.4678425)
\curveto(68.58390672,69.55117584)(68.45890671,69.5928425)(68.30786538,69.5928425)
\curveto(68.10994937,69.5928425)(67.89640669,69.55117584)(67.66203202,69.4678425)
\lineto(67.57869868,69.80117584)
\lineto(70.09953211,70.82200921)
\closepath
\moveto(70.47453212,70.82200921)
}
}
{
\newrgbcolor{curcolor}{0 0 0}
\pscustom[linestyle=none,fillstyle=solid,fillcolor=curcolor]
{
\newpath
\moveto(26.96686049,190.95006523)
\curveto(27.57623518,191.55943992)(27.93561119,191.9136066)(28.05019386,192.01256527)
\curveto(28.31061121,192.23131594)(28.60227788,192.40839861)(28.9251939,192.53339862)
\curveto(29.24290191,192.66881596)(29.55540192,192.74173196)(29.86269393,192.74173196)
\curveto(30.38873528,192.74173196)(30.84186063,192.58548262)(31.21686064,192.28339861)
\curveto(31.59186066,191.9761066)(31.84186067,191.53339858)(31.96686067,190.95006523)
\curveto(32.59186069,191.68443992)(33.11790204,192.1636066)(33.55019406,192.38756528)
\curveto(33.97727807,192.62193995)(34.42519409,192.74173196)(34.88352744,192.74173196)
\curveto(35.32623546,192.74173196)(35.7220688,192.62193995)(36.07102748,192.38756528)
\curveto(36.41477816,192.1636066)(36.6856115,191.79381592)(36.88352751,191.28339857)
\curveto(37.01894485,190.91881589)(37.09186085,190.36673187)(37.09186085,189.61673185)
\lineto(37.09186085,186.03339839)
\curveto(37.09186085,185.50214903)(37.12311152,185.14277302)(37.19602752,184.95006502)
\curveto(37.26373552,184.80943968)(37.37311153,184.68964901)(37.52936087,184.59589834)
\curveto(37.69602754,184.512565)(37.96686088,184.47089833)(38.3418609,184.47089833)
\lineto(38.3418609,184.13756499)
\lineto(34.21686075,184.13756499)
\lineto(34.21686075,184.47089833)
\lineto(34.40436076,184.47089833)
\curveto(34.74811144,184.47089833)(35.02936078,184.53860633)(35.23769412,184.67923167)
\curveto(35.37311146,184.77298234)(35.47206879,184.92923168)(35.5293608,185.13756502)
\curveto(35.55540213,185.24693969)(35.57102746,185.5438157)(35.57102746,186.03339839)
\lineto(35.57102746,189.61673185)
\curveto(35.57102746,190.29381587)(35.48769413,190.77298255)(35.32102746,191.0542319)
\curveto(35.08144478,191.43964925)(34.70644477,191.63756525)(34.19602742,191.63756525)
\curveto(33.86269407,191.63756525)(33.53456873,191.55423192)(33.21686071,191.38756524)
\curveto(32.8939447,191.2313159)(32.49811136,190.93443989)(32.02936067,190.49173188)
\lineto(32.00852734,190.40839854)
\lineto(32.02936067,190.01256519)
\lineto(32.02936067,186.03339839)
\curveto(32.02936067,185.45006503)(32.05540201,185.08548235)(32.11269401,184.95006502)
\curveto(32.18040201,184.80943968)(32.30540202,184.68964901)(32.48769402,184.59589834)
\curveto(32.66477803,184.512565)(32.96686071,184.47089833)(33.38352739,184.47089833)
\lineto(33.38352739,184.13756499)
\lineto(29.1751939,184.13756499)
\lineto(29.1751939,184.47089833)
\curveto(29.63352725,184.47089833)(29.94602727,184.52298233)(30.11269394,184.637565)
\curveto(30.28977794,184.74693967)(30.41477795,184.90839835)(30.48769395,185.11673169)
\curveto(30.51373529,185.22610636)(30.52936062,185.53339837)(30.52936062,186.03339839)
\lineto(30.52936062,189.61673185)
\curveto(30.52936062,190.29381587)(30.43040195,190.78339856)(30.23769394,191.07506523)
\curveto(29.9564446,191.46048258)(29.58144459,191.65839859)(29.1126939,191.65839859)
\curveto(28.77936056,191.65839859)(28.45123521,191.57506525)(28.1335272,191.40839858)
\curveto(27.63352718,191.12714923)(27.24290184,190.82506522)(26.96686049,190.49173188)
\lineto(26.96686049,186.03339839)
\curveto(26.96686049,185.47610637)(26.99811116,185.11673169)(27.07102716,184.95006502)
\curveto(27.1543605,184.79381568)(27.26373517,184.67923167)(27.40436051,184.59589834)
\curveto(27.55540185,184.512565)(27.86269386,184.47089833)(28.32102721,184.47089833)
\lineto(28.32102721,184.13756499)
\lineto(24.19602706,184.13756499)
\lineto(24.19602706,184.47089833)
\curveto(24.58144441,184.47089833)(24.84706842,184.512565)(24.98769376,184.59589834)
\curveto(25.1387351,184.67923167)(25.2585271,184.80423168)(25.34186044,184.97089835)
\curveto(25.42519377,185.14798236)(25.46686044,185.50214903)(25.46686044,186.03339839)
\lineto(25.46686044,189.2208985)
\curveto(25.46686044,190.1375652)(25.43561111,190.72610655)(25.3835271,190.9917319)
\curveto(25.34186044,191.1844399)(25.26894443,191.31464924)(25.17519376,191.38756524)
\curveto(25.09186043,191.47089858)(24.96686042,191.51256525)(24.80019375,191.51256525)
\curveto(24.63352708,191.51256525)(24.43040174,191.47089858)(24.19602706,191.38756524)
\lineto(24.05019372,191.72089859)
\lineto(26.57102715,192.74173196)
\lineto(26.96686049,192.74173196)
\closepath
\moveto(26.96686049,190.95006523)
}
}
{
\newrgbcolor{curcolor}{0 0 0}
\pscustom[linestyle=none,fillstyle=solid,fillcolor=curcolor]
{
\newpath
\moveto(43.73952548,185.34589836)
\curveto(42.87494278,184.67923167)(42.34369209,184.29381566)(42.13535875,184.20006499)
\curveto(41.80202541,184.04902325)(41.44785873,183.97089831)(41.07285871,183.97089831)
\curveto(40.49994269,183.97089831)(40.03119201,184.16360672)(39.656192,184.55423167)
\curveto(39.28119198,184.95527302)(39.09369198,185.47610637)(39.09369198,186.11673172)
\curveto(39.09369198,186.51777307)(39.18223331,186.87193975)(39.36452532,187.17923176)
\curveto(39.61452533,187.58027311)(40.04160934,187.96048246)(40.65619203,188.32506513)
\curveto(41.28119206,188.70006515)(42.30723342,189.1479825)(43.73952548,189.67923185)
\lineto(43.73952548,189.99173186)
\curveto(43.73952548,190.82506522)(43.60410947,191.39277324)(43.34369213,191.70006525)
\curveto(43.07806679,192.01777327)(42.69785877,192.17923194)(42.19785875,192.17923194)
\curveto(41.80723341,192.17923194)(41.49994273,192.07506527)(41.28119206,191.86673193)
\curveto(41.04160938,191.65839859)(40.92702538,191.41881591)(40.92702538,191.15839857)
\lineto(40.94785871,190.63756522)
\curveto(40.94785871,190.34589854)(40.87494271,190.12193987)(40.73952537,189.97089853)
\curveto(40.59890003,189.83027319)(40.41140002,189.76256519)(40.17702535,189.76256519)
\curveto(39.95306667,189.76256519)(39.76556667,189.83548252)(39.61452533,189.99173186)
\curveto(39.47389999,190.1427732)(39.40619199,190.35631587)(39.40619199,190.63756522)
\curveto(39.40619199,191.16360657)(39.66660933,191.64277325)(40.19785868,192.07506527)
\curveto(40.73952537,192.51777328)(41.4947334,192.74173196)(42.4686921,192.74173196)
\curveto(43.20306679,192.74173196)(43.80723348,192.61673195)(44.28119216,192.36673194)
\curveto(44.64056684,192.16881594)(44.90619218,191.87193993)(45.07285886,191.47089858)
\curveto(45.18223353,191.20527324)(45.23952553,190.66881588)(45.23952553,189.86673186)
\lineto(45.23952553,187.03339842)
\curveto(45.23952553,186.24173173)(45.24993886,185.75214904)(45.2811922,185.57506504)
\curveto(45.32285886,185.39277303)(45.37494287,185.26777303)(45.44785887,185.20006502)
\curveto(45.51556687,185.14277302)(45.59369221,185.11673169)(45.67702554,185.11673169)
\curveto(45.77077621,185.11673169)(45.86452555,185.13756502)(45.94785889,185.17923169)
\curveto(46.07285889,185.26256503)(46.3124429,185.4813157)(46.67702558,185.84589838)
\lineto(46.67702558,185.34589836)
\curveto(45.99473356,184.42923166)(45.3436922,183.97089831)(44.71869218,183.97089831)
\curveto(44.4270255,183.97089831)(44.18744282,184.07506498)(44.01035882,184.28339833)
\curveto(43.84369215,184.49173167)(43.74994281,184.84589834)(43.73952548,185.34589836)
\closepath
\moveto(43.73952548,185.92923172)
\lineto(43.73952548,189.11673183)
\curveto(42.82285878,188.75214915)(42.22910942,188.49693981)(41.96869208,188.34589847)
\curveto(41.4791094,188.06464913)(41.13535872,187.78339845)(40.92702538,187.49173177)
\curveto(40.71869204,187.2000651)(40.61452536,186.88756508)(40.61452536,186.55423174)
\curveto(40.61452536,186.10631572)(40.74473337,185.74173171)(41.01035871,185.45006503)
\curveto(41.27077605,185.15839836)(41.57806673,185.01256502)(41.92702541,185.01256502)
\curveto(42.38535876,185.01256502)(42.98952545,185.31464903)(43.73952548,185.92923172)
\closepath
\moveto(43.73952548,185.92923172)
}
}
{
\newrgbcolor{curcolor}{0 0 0}
\pscustom[linestyle=none,fillstyle=solid,fillcolor=curcolor]
{
\newpath
\moveto(52.69419702,192.74173196)
\lineto(52.69419702,189.88756519)
\lineto(52.40253034,189.88756519)
\curveto(52.16294766,190.77298255)(51.86607165,191.37714924)(51.50669697,191.70006525)
\curveto(51.1421143,192.0333986)(50.68378095,192.20006527)(50.13169693,192.20006527)
\curveto(49.69940491,192.20006527)(49.35044756,192.08027327)(49.09003022,191.84589859)
\curveto(48.82440488,191.62193992)(48.69419688,191.37714924)(48.69419688,191.1167319)
\curveto(48.69419688,190.76777322)(48.78794754,190.47610654)(48.98586355,190.24173187)
\curveto(49.17857156,189.99173186)(49.56919691,189.72610652)(50.15253026,189.45006517)
\lineto(51.46503031,188.80423182)
\curveto(52.69940502,188.18964913)(53.31919704,187.3927731)(53.31919704,186.4083984)
\curveto(53.31919704,185.65839837)(53.03273836,185.04381569)(52.46503034,184.575065)
\curveto(51.89211432,184.11673165)(51.24628097,183.88756498)(50.52753027,183.88756498)
\curveto(50.02753026,183.88756498)(49.4441969,183.98131431)(48.77753021,184.15839832)
\curveto(48.5796142,184.21048166)(48.42336353,184.24173166)(48.29836353,184.24173166)
\curveto(48.15773819,184.24173166)(48.04836352,184.16360672)(47.96503018,184.01256498)
\lineto(47.67336351,184.01256498)
\lineto(47.67336351,186.99173175)
\lineto(47.96503018,186.99173175)
\curveto(48.13169686,186.14277306)(48.44940487,185.50214903)(48.92336355,185.07506502)
\curveto(49.40773823,184.642773)(49.94940492,184.42923166)(50.54836361,184.42923166)
\curveto(50.96503029,184.42923166)(51.30357163,184.54381567)(51.56919698,184.78339834)
\curveto(51.82961432,185.03339835)(51.96503032,185.33027303)(51.96503032,185.67923171)
\curveto(51.96503032,186.09589839)(51.81919699,186.43964907)(51.52753031,186.72089841)
\curveto(51.23586363,187.01256509)(50.64211428,187.37193977)(49.75669691,187.80423178)
\curveto(48.88169688,188.2469398)(48.30357153,188.64798248)(48.02753018,189.01256516)
\curveto(47.76190484,189.35631584)(47.63169684,189.79381585)(47.63169684,190.32506521)
\curveto(47.63169684,191.00214923)(47.86607151,191.57506525)(48.3400302,192.0333986)
\curveto(48.80878088,192.50214928)(49.41294757,192.74173196)(50.15253026,192.74173196)
\curveto(50.48586361,192.74173196)(50.88690495,192.66881596)(51.36086364,192.53339862)
\curveto(51.66294765,192.43443995)(51.86607165,192.38756528)(51.96503032,192.38756528)
\curveto(52.05878099,192.38756528)(52.136905,192.40839861)(52.194197,192.45006528)
\curveto(52.246281,192.49173195)(52.319197,192.58548262)(52.40253034,192.74173196)
\closepath
\moveto(52.69419702,192.74173196)
}
}
{
\newrgbcolor{curcolor}{0 0 0}
\pscustom[linestyle=none,fillstyle=solid,fillcolor=curcolor]
{
\newpath
\moveto(56.9763627,195.22089871)
\lineto(56.9763627,192.49173195)
\lineto(58.9346961,192.49173195)
\lineto(58.9346961,191.84589859)
\lineto(56.9763627,191.84589859)
\lineto(56.9763627,186.42923173)
\curveto(56.9763627,185.88756505)(57.04928003,185.51777304)(57.20552937,185.32506503)
\curveto(57.37219604,185.14277302)(57.57011338,185.05423169)(57.80969606,185.05423169)
\curveto(58.0180294,185.05423169)(58.21073741,185.11673169)(58.39302941,185.24173169)
\curveto(58.58573742,185.3667317)(58.73157076,185.5542317)(58.83052943,185.80423171)
\lineto(59.18469611,185.80423171)
\curveto(58.97636277,185.20527302)(58.67428009,184.75214901)(58.28886274,184.450065)
\curveto(57.8982374,184.14277332)(57.49719605,183.99173165)(57.08052937,183.99173165)
\curveto(56.79928002,183.99173165)(56.52844668,184.06985658)(56.26802934,184.22089832)
\curveto(56.00240399,184.387565)(55.79927999,184.60631567)(55.66386265,184.88756501)
\curveto(55.53886264,185.17923169)(55.47636264,185.62714904)(55.47636264,186.24173173)
\lineto(55.47636264,191.84589859)
\lineto(54.1638626,191.84589859)
\lineto(54.1638626,192.1583986)
\curveto(54.49719594,192.28339861)(54.83573729,192.50214928)(55.18469597,192.82506529)
\curveto(55.52844664,193.14277331)(55.84094666,193.52298265)(56.122196,193.97089867)
\curveto(56.25761334,194.18964934)(56.45552934,194.60631602)(56.70552935,195.22089871)
\closepath
\moveto(56.9763627,195.22089871)
}
}
{
\newrgbcolor{curcolor}{0 0 0}
\pscustom[linestyle=none,fillstyle=solid,fillcolor=curcolor]
{
\newpath
\moveto(61.14486056,189.3458985)
\curveto(61.14486056,188.10631579)(61.44694458,187.13756509)(62.06152726,186.42923173)
\curveto(62.65527795,185.72089838)(63.37402731,185.3667317)(64.20736067,185.3667317)
\curveto(64.74902736,185.3667317)(65.21777804,185.51256504)(65.62402739,185.80423171)
\curveto(66.02506874,186.10631572)(66.36361142,186.62714907)(66.64486076,187.36673177)
\lineto(66.9156941,187.17923176)
\curveto(66.7906941,186.3458984)(66.42090209,185.58548237)(65.8115274,184.90839835)
\curveto(65.19694471,184.22610672)(64.42611135,183.88756498)(63.49902731,183.88756498)
\curveto(62.49902728,183.88756498)(61.63444458,184.27298232)(60.91569389,185.05423169)
\curveto(60.20736053,185.83027305)(59.85319385,186.87714908)(59.85319385,188.20006513)
\curveto(59.85319385,189.62714918)(60.21777787,190.74173189)(60.95736056,191.53339858)
\curveto(61.69173525,192.33548261)(62.60840195,192.74173196)(63.70736066,192.74173196)
\curveto(64.65006869,192.74173196)(65.42090205,192.42923195)(66.01986074,191.80423192)
\curveto(66.61361143,191.18964924)(66.9156941,190.37193987)(66.9156941,189.3458985)
\closepath
\moveto(61.14486056,189.86673186)
\lineto(65.0198607,189.86673186)
\curveto(64.98861137,190.40839854)(64.92611137,190.78339856)(64.8323607,190.9917319)
\curveto(64.67611136,191.33548257)(64.44694468,191.60631592)(64.14486067,191.80423192)
\curveto(63.83756866,191.99693993)(63.52506865,192.0958986)(63.20736064,192.0958986)
\curveto(62.70736062,192.0958986)(62.25423527,191.89798259)(61.85319392,191.51256525)
\curveto(61.44694458,191.1219399)(61.21256857,190.57506521)(61.14486056,189.86673186)
\closepath
\moveto(61.14486056,189.86673186)
}
}
{
\newrgbcolor{curcolor}{0 0 0}
\pscustom[linestyle=none,fillstyle=solid,fillcolor=curcolor]
{
\newpath
\moveto(70.47453212,192.74173196)
\lineto(70.47453212,190.84589856)
\curveto(71.18286548,192.10631594)(71.90161617,192.74173196)(72.64119886,192.74173196)
\curveto(72.97453221,192.74173196)(73.24536555,192.63756529)(73.45369889,192.42923195)
\curveto(73.67244957,192.22089861)(73.78703224,191.98131593)(73.78703224,191.72089859)
\curveto(73.78703224,191.48131591)(73.7036989,191.28339857)(73.53703223,191.1167319)
\curveto(73.38078289,190.95006523)(73.20369888,190.86673189)(72.99536554,190.86673189)
\curveto(72.77140687,190.86673189)(72.52661619,190.97089856)(72.26619885,191.1792319)
\curveto(72.00057351,191.39798258)(71.8078655,191.51256525)(71.6828655,191.51256525)
\curveto(71.56828282,191.51256525)(71.45369882,191.45006525)(71.32869882,191.32506524)
\curveto(71.04744947,191.07506523)(70.7661988,190.66360655)(70.47453212,190.09589853)
\lineto(70.47453212,186.09589839)
\curveto(70.47453212,185.62193971)(70.53703212,185.26777303)(70.66203213,185.03339835)
\curveto(70.72974013,184.87714901)(70.85994947,184.74693967)(71.05786547,184.637565)
\curveto(71.26619881,184.52298233)(71.55786549,184.47089833)(71.9328655,184.47089833)
\lineto(71.9328655,184.13756499)
\lineto(67.66203202,184.13756499)
\lineto(67.66203202,184.47089833)
\curveto(68.08911603,184.47089833)(68.41203205,184.53339833)(68.62036539,184.65839834)
\curveto(68.77140673,184.75214901)(68.8755734,184.91360635)(68.9328654,185.13756502)
\curveto(68.95890673,185.23131569)(68.97453207,185.52298237)(68.97453207,186.01256505)
\lineto(68.97453207,189.24173183)
\curveto(68.97453207,190.21048254)(68.95369873,190.78860656)(68.91203206,190.97089856)
\curveto(68.8703654,191.16360657)(68.79224006,191.30423191)(68.68286539,191.38756524)
\curveto(68.58390672,191.47089858)(68.45890671,191.51256525)(68.30786538,191.51256525)
\curveto(68.10994937,191.51256525)(67.89640669,191.47089858)(67.66203202,191.38756524)
\lineto(67.57869868,191.72089859)
\lineto(70.09953211,192.74173196)
\closepath
\moveto(70.47453212,192.74173196)
}
}
{
\newrgbcolor{curcolor}{0 0 0}
\pscustom[linestyle=none,fillstyle=solid,fillcolor=curcolor]
{
\newpath
\moveto(12.68618982,248.84170953)
\lineto(7.89452298,259.23754323)
\lineto(7.89452298,250.98754294)
\curveto(7.89452298,250.22191758)(7.97785632,249.74796023)(8.14452299,249.57087622)
\curveto(8.36327367,249.30525088)(8.71744035,249.17504288)(9.20702303,249.17504288)
\lineto(9.64452305,249.17504288)
\lineto(9.64452305,248.84170953)
\lineto(5.35285623,248.84170953)
\lineto(5.35285623,249.17504288)
\lineto(5.79035624,249.17504288)
\curveto(6.3163976,249.17504288)(6.68618961,249.33129355)(6.89452295,249.65420956)
\curveto(7.02994029,249.84691757)(7.10285629,250.28962692)(7.10285629,250.98754294)
\lineto(7.10285629,259.05004323)
\curveto(7.10285629,259.60212725)(7.04035629,259.99796059)(6.91535628,260.23754327)
\curveto(6.83202295,260.41462728)(6.67056427,260.56046061)(6.4361896,260.67504328)
\curveto(6.21223093,260.80004329)(5.85285625,260.86254329)(5.35285623,260.86254329)
\lineto(5.35285623,261.19587664)
\lineto(8.85285635,261.19587664)
\lineto(13.33202318,251.52920963)
\lineto(17.74869,261.19587664)
\lineto(21.24869012,261.19587664)
\lineto(21.24869012,260.86254329)
\lineto(20.81119011,260.86254329)
\curveto(20.27994076,260.86254329)(19.91535674,260.70108462)(19.7070234,260.38337661)
\curveto(19.56639806,260.1854606)(19.49869006,259.74275125)(19.49869006,259.05004323)
\lineto(19.49869006,250.98754294)
\curveto(19.49869006,250.22191758)(19.5820234,249.74796023)(19.74869007,249.57087622)
\curveto(19.96744075,249.30525088)(20.32160743,249.17504288)(20.81119011,249.17504288)
\lineto(21.24869012,249.17504288)
\lineto(21.24869012,248.84170953)
\lineto(15.99868994,248.84170953)
\lineto(15.99868994,249.17504288)
\lineto(16.43618995,249.17504288)
\curveto(16.96223131,249.17504288)(17.33202332,249.33129355)(17.54035666,249.65420956)
\curveto(17.675774,249.84691757)(17.74869,250.28962692)(17.74869,250.98754294)
\lineto(17.74869,259.23754323)
\lineto(12.9778565,248.84170953)
\closepath
\moveto(12.68618982,248.84170953)
}
}
{
\newrgbcolor{curcolor}{0 0 0}
\pscustom[linestyle=none,fillstyle=solid,fillcolor=curcolor]
{
\newpath
\moveto(23.61419924,254.05004305)
\curveto(23.61419924,252.81046034)(23.91628325,251.84170964)(24.53086594,251.13337628)
\curveto(25.12461663,250.42504292)(25.84336599,250.07087624)(26.67669935,250.07087624)
\curveto(27.21836604,250.07087624)(27.68711672,250.21670958)(28.09336607,250.50837626)
\curveto(28.49440741,250.81046027)(28.83295009,251.33129362)(29.11419944,252.07087631)
\lineto(29.38503278,251.88337631)
\curveto(29.26003277,251.05004294)(28.89024076,250.28962692)(28.28086607,249.61254289)
\curveto(27.66628339,248.93025127)(26.89545002,248.59170952)(25.96836599,248.59170952)
\curveto(24.96836596,248.59170952)(24.10378326,248.97712687)(23.38503257,249.75837623)
\curveto(22.67669921,250.53441759)(22.32253253,251.58129363)(22.32253253,252.90420968)
\curveto(22.32253253,254.33129373)(22.68711654,255.44587643)(23.42669923,256.23754313)
\curveto(24.16107393,257.03962716)(25.07774063,257.4458765)(26.17669933,257.4458765)
\curveto(27.11940737,257.4458765)(27.89024073,257.13337649)(28.48919941,256.50837647)
\curveto(29.0829501,255.89379378)(29.38503278,255.07608442)(29.38503278,254.05004305)
\closepath
\moveto(23.61419924,254.5708764)
\lineto(27.48919938,254.5708764)
\curveto(27.45795004,255.11254309)(27.39545004,255.4875431)(27.30169937,255.69587644)
\curveto(27.14545003,256.03962712)(26.91628336,256.31046046)(26.61419935,256.50837647)
\curveto(26.30690734,256.70108448)(25.99440733,256.80004315)(25.67669931,256.80004315)
\curveto(25.1766993,256.80004315)(24.72357395,256.60212714)(24.3225326,256.21670979)
\curveto(23.91628325,255.82608445)(23.68190724,255.27920976)(23.61419924,254.5708764)
\closepath
\moveto(23.61419924,254.5708764)
}
}
{
\newrgbcolor{curcolor}{0 0 0}
\pscustom[linestyle=none,fillstyle=solid,fillcolor=curcolor]
{
\newpath
\moveto(32.92302729,259.92504326)
\lineto(32.92302729,257.19587649)
\lineto(34.88136069,257.19587649)
\lineto(34.88136069,256.55004314)
\lineto(32.92302729,256.55004314)
\lineto(32.92302729,251.13337628)
\curveto(32.92302729,250.59170959)(32.99594463,250.22191758)(33.15219396,250.02920957)
\curveto(33.31886064,249.84691757)(33.51677798,249.75837623)(33.75636065,249.75837623)
\curveto(33.96469399,249.75837623)(34.157402,249.82087623)(34.33969401,249.94587624)
\curveto(34.53240201,250.07087624)(34.67823535,250.25837625)(34.77719402,250.50837626)
\lineto(35.1313607,250.50837626)
\curveto(34.92302736,249.90941757)(34.62094468,249.45629355)(34.23552734,249.15420954)
\curveto(33.84490199,248.84691787)(33.44386064,248.69587619)(33.02719396,248.69587619)
\curveto(32.74594462,248.69587619)(32.47511127,248.77400113)(32.21469393,248.92504287)
\curveto(31.94906859,249.09170954)(31.74594458,249.31046021)(31.61052724,249.59170956)
\curveto(31.48552724,249.88337624)(31.42302724,250.33129358)(31.42302724,250.94587627)
\lineto(31.42302724,256.55004314)
\lineto(30.11052719,256.55004314)
\lineto(30.11052719,256.86254315)
\curveto(30.44386054,256.98754315)(30.78240188,257.20629383)(31.13136056,257.52920984)
\curveto(31.47511124,257.84691785)(31.78761125,258.2271272)(32.06886059,258.67504321)
\curveto(32.20427793,258.89379389)(32.40219394,259.31046057)(32.65219395,259.92504326)
\closepath
\moveto(32.92302729,259.92504326)
}
}
{
\newrgbcolor{curcolor}{0 0 0}
\pscustom[linestyle=none,fillstyle=solid,fillcolor=curcolor]
{
\newpath
\moveto(40.42485861,250.05004291)
\curveto(39.56027591,249.38337622)(39.02902523,248.9979602)(38.82069189,248.90420953)
\curveto(38.48735854,248.7531678)(38.13319186,248.67504286)(37.75819185,248.67504286)
\curveto(37.18527583,248.67504286)(36.71652515,248.86775127)(36.34152513,249.25837621)
\curveto(35.96652512,249.65941756)(35.77902511,250.18025091)(35.77902511,250.82087627)
\curveto(35.77902511,251.22191762)(35.86756645,251.5760843)(36.04985846,251.88337631)
\curveto(36.29985846,252.28441765)(36.72694248,252.664627)(37.34152517,253.02920968)
\curveto(37.96652519,253.40420969)(38.99256656,253.85212704)(40.42485861,254.38337639)
\lineto(40.42485861,254.69587641)
\curveto(40.42485861,255.52920977)(40.28944261,256.09691779)(40.02902526,256.4042098)
\curveto(39.76339992,256.72191781)(39.38319191,256.88337648)(38.88319189,256.88337648)
\curveto(38.49256654,256.88337648)(38.18527586,256.77920981)(37.96652519,256.57087647)
\curveto(37.72694251,256.36254313)(37.61235851,256.12296046)(37.61235851,255.86254311)
\lineto(37.63319184,255.34170976)
\curveto(37.63319184,255.05004308)(37.56027584,254.82608441)(37.4248585,254.67504307)
\curveto(37.28423317,254.53441773)(37.09673316,254.46670973)(36.86235848,254.46670973)
\curveto(36.63839981,254.46670973)(36.4508998,254.53962707)(36.29985846,254.69587641)
\curveto(36.15923313,254.84691774)(36.09152512,255.06046042)(36.09152512,255.34170976)
\curveto(36.09152512,255.86775111)(36.35194247,256.3469178)(36.88319182,256.77920981)
\curveto(37.4248585,257.22191783)(38.18006653,257.4458765)(39.15402523,257.4458765)
\curveto(39.88839992,257.4458765)(40.49256661,257.3208765)(40.9665253,257.07087649)
\curveto(41.32589998,256.87296048)(41.59152532,256.57608447)(41.75819199,256.17504312)
\curveto(41.86756666,255.90941778)(41.92485866,255.37296043)(41.92485866,254.5708764)
\lineto(41.92485866,251.73754297)
\curveto(41.92485866,250.94587627)(41.935272,250.45629359)(41.96652533,250.27920958)
\curveto(42.008192,250.09691758)(42.060276,249.97191757)(42.133192,249.90420957)
\curveto(42.20090001,249.84691757)(42.27902534,249.82087623)(42.36235868,249.82087623)
\curveto(42.45610935,249.82087623)(42.54985869,249.84170957)(42.63319202,249.88337624)
\curveto(42.75819203,249.96670957)(42.99777603,250.18546025)(43.36235871,250.55004293)
\lineto(43.36235871,250.05004291)
\curveto(42.68006669,249.13337621)(42.02902533,248.67504286)(41.40402531,248.67504286)
\curveto(41.11235863,248.67504286)(40.87277596,248.77920953)(40.69569195,248.98754287)
\curveto(40.52902528,249.19587621)(40.43527594,249.55004289)(40.42485861,250.05004291)
\closepath
\moveto(40.42485861,250.63337626)
\lineto(40.42485861,253.82087637)
\curveto(39.50819191,253.4562937)(38.91444256,253.20108435)(38.65402521,253.05004301)
\curveto(38.16444253,252.76879367)(37.82069185,252.48754299)(37.61235851,252.19587632)
\curveto(37.40402517,251.90420964)(37.2998585,251.59170963)(37.2998585,251.25837628)
\curveto(37.2998585,250.81046027)(37.4300665,250.44587626)(37.69569185,250.15420958)
\curveto(37.95610919,249.8625429)(38.26339987,249.71670956)(38.61235855,249.71670956)
\curveto(39.0706919,249.71670956)(39.67485858,250.01879357)(40.42485861,250.63337626)
\closepath
\moveto(40.42485861,250.63337626)
}
}
{
\newrgbcolor{curcolor}{0 0 0}
\pscustom[linestyle=none,fillstyle=solid,fillcolor=curcolor]
{
\newpath
\moveto(44.1503633,253.7167097)
\lineto(48.87953013,253.7167097)
\lineto(48.87953013,252.34170966)
\lineto(44.1503633,252.34170966)
\closepath
\moveto(44.1503633,253.7167097)
}
}
{
\newrgbcolor{curcolor}{0 0 0}
\pscustom[linestyle=none,fillstyle=solid,fillcolor=curcolor]
{
\newpath
\moveto(56.44969031,261.48754331)
\curveto(58.04344104,261.48754331)(59.43406509,260.87296062)(60.61635713,259.65420991)
\curveto(61.79344117,258.4302512)(62.38719052,256.90941782)(62.38719052,255.09170975)
\curveto(62.38719052,253.21670969)(61.78823184,251.65420963)(60.59552379,250.40420959)
\curveto(59.41323175,249.16462688)(57.98094103,248.55004285)(56.30385698,248.55004285)
\curveto(54.59552358,248.55004285)(53.16323153,249.15420954)(52.01219016,250.36254292)
\curveto(50.85594078,251.58129363)(50.28302343,253.15420968)(50.28302343,255.07087642)
\curveto(50.28302343,257.02920982)(50.94969012,258.62296054)(52.2830235,259.86254326)
\curveto(53.43406487,260.94587663)(54.82469026,261.48754331)(56.44969031,261.48754331)
\closepath
\moveto(56.26219031,260.82087662)
\curveto(55.1632316,260.82087662)(54.28302357,260.40941794)(53.61635688,259.59170991)
\curveto(52.78302352,258.56046054)(52.36635684,257.07087649)(52.36635684,255.11254309)
\curveto(52.36635684,253.08129368)(52.79344085,251.52920963)(53.65802355,250.44587626)
\curveto(54.32469024,249.62296023)(55.19969027,249.21670954)(56.28302364,249.21670954)
\curveto(57.43406502,249.21670954)(58.38719038,249.66462689)(59.13719041,250.57087626)
\curveto(59.89760777,251.47191762)(60.28302378,252.89379368)(60.28302378,254.84170974)
\curveto(60.28302378,256.96670982)(59.8663571,258.53962721)(59.03302374,259.57087658)
\curveto(58.36635705,260.40420994)(57.43927435,260.82087662)(56.26219031,260.82087662)
\closepath
\moveto(56.26219031,260.82087662)
}
}
{
\newrgbcolor{curcolor}{0 0 0}
\pscustom[linestyle=none,fillstyle=solid,fillcolor=curcolor]
{
\newpath
\moveto(65.96869293,255.75837644)
\curveto(66.77077696,256.88337648)(67.64056766,257.4458765)(68.57285969,257.4458765)
\curveto(69.43223439,257.4458765)(70.18223441,257.07608449)(70.82285977,256.3417098)
\curveto(71.45827712,255.6021271)(71.78119314,254.60212707)(71.78119314,253.34170969)
\curveto(71.78119314,251.85212697)(71.28640112,250.6594176)(70.30202642,249.75837623)
\curveto(69.45306772,248.97712687)(68.51035969,248.59170952)(67.46869298,248.59170952)
\curveto(66.9791103,248.59170952)(66.48952628,248.67504286)(65.98952626,248.84170953)
\curveto(65.48952625,249.01879354)(64.97911023,249.28962688)(64.46869288,249.65420956)
\lineto(64.46869288,258.3000432)
\curveto(64.46869288,259.24275123)(64.43744354,259.82087659)(64.38535954,260.02920993)
\curveto(64.34369287,260.2479606)(64.27077687,260.40420994)(64.1770262,260.48754328)
\curveto(64.07806753,260.57087661)(63.95827686,260.61254328)(63.82285952,260.61254328)
\curveto(63.64056751,260.61254328)(63.42702617,260.56046061)(63.17702616,260.46670994)
\lineto(63.05202616,260.77920995)
\lineto(65.55202625,261.80004332)
\lineto(65.96869293,261.80004332)
\closepath
\moveto(65.96869293,255.17504309)
\lineto(65.96869293,250.17504291)
\curveto(66.27077694,249.8677509)(66.59369295,249.63858423)(66.9270263,249.48754289)
\curveto(67.26035964,249.33129355)(67.59890099,249.25837621)(67.94785967,249.25837621)
\curveto(68.49994369,249.25837621)(69.0155677,249.55525089)(69.48952639,250.15420958)
\curveto(69.95827707,250.76358427)(70.19785975,251.65420963)(70.19785975,252.82087634)
\curveto(70.19785975,253.88858438)(69.95827707,254.70629374)(69.48952639,255.27920976)
\curveto(69.0155677,255.84691778)(68.47390102,256.13337646)(67.86452633,256.13337646)
\curveto(67.54161032,256.13337646)(67.21869297,256.05525112)(66.88535963,255.90420978)
\curveto(66.64577695,255.77920978)(66.34369294,255.53441777)(65.96869293,255.17504309)
\closepath
\moveto(65.96869293,255.17504309)
}
}
{
\newrgbcolor{curcolor}{0 0 0}
\pscustom[linestyle=none,fillstyle=solid,fillcolor=curcolor]
{
\newpath
\moveto(75.13535992,261.82087666)
\curveto(75.38535993,261.82087666)(75.59890127,261.72191799)(75.78119328,261.52920998)
\curveto(75.97390128,261.34691797)(76.07285995,261.13337663)(76.07285995,260.88337662)
\curveto(76.07285995,260.61775128)(75.97390128,260.39379394)(75.78119328,260.21670993)
\curveto(75.59890127,260.03441793)(75.38535993,259.94587659)(75.13535992,259.94587659)
\curveto(74.86973458,259.94587659)(74.64577724,260.03441793)(74.46869323,260.21670993)
\curveto(74.28640122,260.39379394)(74.19785989,260.61775128)(74.19785989,260.88337662)
\curveto(74.19785989,261.13337663)(74.28640122,261.34691797)(74.46869323,261.52920998)
\curveto(74.64577724,261.72191799)(74.86973458,261.82087666)(75.13535992,261.82087666)
\closepath
\moveto(75.90619328,257.4458765)
\lineto(75.90619328,249.02920954)
\curveto(75.90619328,247.60212549)(75.59890127,246.54483478)(74.98952658,245.86254276)
\curveto(74.39056789,245.16983473)(73.60411053,244.82087606)(72.63535983,244.82087606)
\curveto(72.08327581,244.82087606)(71.67181847,244.91983473)(71.40619312,245.11254273)
\curveto(71.13015178,245.31045874)(70.98952644,245.51879208)(70.98952644,245.73754275)
\curveto(70.98952644,245.9458761)(71.06765178,246.1281681)(71.21869312,246.27920944)
\curveto(71.37494245,246.43545878)(71.54681846,246.50837612)(71.73952647,246.50837612)
\curveto(71.90619314,246.50837612)(72.06765181,246.46670945)(72.21869315,246.38337611)
\curveto(72.33327649,246.34170944)(72.53119316,246.1906681)(72.82285984,245.92504276)
\curveto(73.11452652,245.66462542)(73.36452653,245.52920941)(73.57285987,245.52920941)
\curveto(73.7082772,245.52920941)(73.84890121,245.58650142)(73.98952655,245.69587609)
\curveto(74.12494389,245.81045876)(74.22911056,245.99795876)(74.30202656,246.25837611)
\curveto(74.36973456,246.52400145)(74.40619323,247.10212547)(74.40619323,247.98754283)
\lineto(74.40619323,253.92504305)
\curveto(74.40619323,254.84170974)(74.37494389,255.4302511)(74.32285989,255.69587644)
\curveto(74.28119322,255.88858445)(74.20827722,256.01879379)(74.11452655,256.09170979)
\curveto(74.03119322,256.17504312)(73.90619321,256.21670979)(73.73952654,256.21670979)
\curveto(73.57285987,256.21670979)(73.36973453,256.17504312)(73.13535985,256.09170979)
\lineto(73.01035985,256.42504313)
\lineto(75.51035993,257.4458765)
\closepath
\moveto(75.90619328,257.4458765)
}
}
{
\newrgbcolor{curcolor}{0 0 0}
\pscustom[linestyle=none,fillstyle=solid,fillcolor=curcolor]
{
\newpath
\moveto(79.59552447,254.05004305)
\curveto(79.59552447,252.81046034)(79.89760848,251.84170964)(80.51219117,251.13337628)
\curveto(81.10594185,250.42504292)(81.82469121,250.07087624)(82.65802458,250.07087624)
\curveto(83.19969126,250.07087624)(83.66844194,250.21670958)(84.07469129,250.50837626)
\curveto(84.47573264,250.81046027)(84.81427532,251.33129362)(85.09552466,252.07087631)
\lineto(85.36635801,251.88337631)
\curveto(85.241358,251.05004294)(84.87156599,250.28962692)(84.2621913,249.61254289)
\curveto(83.64760861,248.93025127)(82.87677525,248.59170952)(81.94969122,248.59170952)
\curveto(80.94969118,248.59170952)(80.08510848,248.97712687)(79.36635779,249.75837623)
\curveto(78.65802443,250.53441759)(78.30385775,251.58129363)(78.30385775,252.90420968)
\curveto(78.30385775,254.33129373)(78.66844177,255.44587643)(79.40802446,256.23754313)
\curveto(80.14239915,257.03962716)(81.05906585,257.4458765)(82.15802456,257.4458765)
\curveto(83.10073259,257.4458765)(83.87156595,257.13337649)(84.47052464,256.50837647)
\curveto(85.06427533,255.89379378)(85.36635801,255.07608442)(85.36635801,254.05004305)
\closepath
\moveto(79.59552447,254.5708764)
\lineto(83.4705246,254.5708764)
\curveto(83.43927527,255.11254309)(83.37677527,255.4875431)(83.2830246,255.69587644)
\curveto(83.12677526,256.03962712)(82.89760858,256.31046046)(82.59552457,256.50837647)
\curveto(82.28823256,256.70108448)(81.97573255,256.80004315)(81.65802454,256.80004315)
\curveto(81.15802452,256.80004315)(80.70489917,256.60212714)(80.30385783,256.21670979)
\curveto(79.89760848,255.82608445)(79.66323247,255.27920976)(79.59552447,254.5708764)
\closepath
\moveto(79.59552447,254.5708764)
}
}
{
\newrgbcolor{curcolor}{0 0 0}
\pscustom[linestyle=none,fillstyle=solid,fillcolor=curcolor]
{
\newpath
\moveto(93.57102952,252.00837631)
\curveto(93.34707085,250.9094176)(92.90957083,250.06046024)(92.25852947,249.46670955)
\curveto(91.60228012,248.8833762)(90.87311342,248.59170952)(90.0710294,248.59170952)
\curveto(89.12311336,248.59170952)(88.300196,248.98754287)(87.59186264,249.77920956)
\curveto(86.89394662,250.57087626)(86.55019594,251.64379363)(86.55019594,253.00837635)
\curveto(86.55019594,254.31046039)(86.93561328,255.37296043)(87.71686265,256.19587646)
\curveto(88.49290401,257.02920982)(89.43040404,257.4458765)(90.52936275,257.4458765)
\curveto(91.34707077,257.4458765)(92.0189468,257.22191783)(92.55019615,256.77920981)
\curveto(93.0762375,256.3469178)(93.34186284,255.89379378)(93.34186284,255.4250431)
\curveto(93.34186284,255.20108442)(93.26373751,255.01358442)(93.11269617,254.86254308)
\curveto(92.97207083,254.72191774)(92.76373749,254.65420974)(92.48769615,254.65420974)
\curveto(92.13873747,254.65420974)(91.86790413,254.76879374)(91.67519612,255.00837642)
\curveto(91.57623745,255.13337642)(91.50852945,255.37296043)(91.46686278,255.73754311)
\curveto(91.43561344,256.09691779)(91.32102944,256.36775113)(91.1126961,256.55004314)
\curveto(90.90436276,256.72712714)(90.60228008,256.82087648)(90.21686273,256.82087648)
\curveto(89.61790405,256.82087648)(89.13352936,256.59691781)(88.75852935,256.15420979)
\curveto(88.26894667,255.5552511)(88.02936266,254.76879374)(88.02936266,253.80004304)
\curveto(88.02936266,252.80004301)(88.26894667,251.91462697)(88.75852935,251.15420961)
\curveto(89.24290403,250.38858425)(89.90436272,250.00837624)(90.73769609,250.00837624)
\curveto(91.33144677,250.00837624)(91.86790413,250.20629358)(92.34186281,250.61254293)
\curveto(92.67519615,250.88858427)(92.9981135,251.39379362)(93.32102951,252.13337631)
\closepath
\moveto(93.57102952,252.00837631)
}
}
{
\newrgbcolor{curcolor}{0 0 0}
\pscustom[linestyle=none,fillstyle=solid,fillcolor=curcolor]
{
\newpath
\moveto(97.19236754,259.92504326)
\lineto(97.19236754,257.19587649)
\lineto(99.15070094,257.19587649)
\lineto(99.15070094,256.55004314)
\lineto(97.19236754,256.55004314)
\lineto(97.19236754,251.13337628)
\curveto(97.19236754,250.59170959)(97.26528487,250.22191758)(97.42153421,250.02920957)
\curveto(97.58820089,249.84691757)(97.78611823,249.75837623)(98.0257009,249.75837623)
\curveto(98.23403424,249.75837623)(98.42674225,249.82087623)(98.60903426,249.94587624)
\curveto(98.80174226,250.07087624)(98.9475756,250.25837625)(99.04653427,250.50837626)
\lineto(99.40070095,250.50837626)
\curveto(99.19236761,249.90941757)(98.89028493,249.45629355)(98.50486759,249.15420954)
\curveto(98.11424224,248.84691787)(97.71320089,248.69587619)(97.29653421,248.69587619)
\curveto(97.01528487,248.69587619)(96.74445152,248.77400113)(96.48403418,248.92504287)
\curveto(96.21840884,249.09170954)(96.01528483,249.31046021)(95.87986749,249.59170956)
\curveto(95.75486749,249.88337624)(95.69236749,250.33129358)(95.69236749,250.94587627)
\lineto(95.69236749,256.55004314)
\lineto(94.37986744,256.55004314)
\lineto(94.37986744,256.86254315)
\curveto(94.71320078,256.98754315)(95.05174213,257.20629383)(95.40070081,257.52920984)
\curveto(95.74445149,257.84691785)(96.0569515,258.2271272)(96.33820084,258.67504321)
\curveto(96.47361818,258.89379389)(96.67153419,259.31046057)(96.9215342,259.92504326)
\closepath
\moveto(97.19236754,259.92504326)
}
}
{
\newrgbcolor{curcolor}{0 0 0}
\pscustom[linestyle=none,fillstyle=solid,fillcolor=curcolor]
{
\newpath
\moveto(112.61086581,261.48754331)
\lineto(112.61086581,257.21670983)
\lineto(112.27753246,257.21670983)
\curveto(112.16294979,258.03441786)(111.96503245,258.68546055)(111.67336577,259.17504323)
\curveto(111.39211643,259.65941791)(110.99107375,260.03962726)(110.4650324,260.32087661)
\curveto(109.94940705,260.61254328)(109.41294969,260.75837662)(108.86086567,260.75837662)
\curveto(108.22024032,260.75837662)(107.69419897,260.56046061)(107.27753228,260.17504327)
\curveto(106.87128294,259.78441792)(106.6733656,259.3469179)(106.6733656,258.86254322)
\curveto(106.6733656,258.48754321)(106.7983656,258.13858453)(107.04836561,257.82087652)
\curveto(107.43378296,257.3729605)(108.32961632,256.76879381)(109.7358657,256.00837645)
\curveto(110.90253241,255.38337643)(111.69419911,254.90420975)(112.11086579,254.5708764)
\curveto(112.52753247,254.24796039)(112.84524048,253.86254304)(113.06919916,253.40420969)
\curveto(113.30357383,252.95629368)(113.42336584,252.48754299)(113.42336584,251.98754298)
\curveto(113.42336584,251.05525094)(113.05878316,250.24796025)(112.34003246,249.57087622)
\curveto(111.61607377,248.8885846)(110.67857374,248.55004285)(109.52753236,248.55004285)
\curveto(109.17857368,248.55004285)(108.84524034,248.58129219)(108.52753233,248.63337619)
\curveto(108.32961632,248.67504286)(107.93378297,248.7896262)(107.34003229,248.96670954)
\curveto(106.75669893,249.15941754)(106.38690692,249.25837621)(106.23586558,249.25837621)
\curveto(106.07961624,249.25837621)(105.95461624,249.20629354)(105.86086557,249.11254287)
\curveto(105.77753223,249.02920954)(105.71503223,248.84170953)(105.67336556,248.55004285)
\lineto(105.34003222,248.55004285)
\lineto(105.34003222,252.80004301)
\lineto(105.67336556,252.80004301)
\curveto(105.8244069,251.90941764)(106.03274024,251.24275095)(106.29836558,250.80004293)
\curveto(106.57440693,250.36775092)(106.98586561,250.00837624)(107.52753229,249.71670956)
\curveto(108.07961631,249.42504289)(108.683783,249.27920955)(109.34003236,249.27920955)
\curveto(110.10044972,249.27920955)(110.69940707,249.47191755)(111.13169909,249.8625429)
\curveto(111.5744071,250.26358425)(111.79836578,250.74275093)(111.79836578,251.30004295)
\curveto(111.79836578,251.60212696)(111.71503244,251.90941764)(111.54836577,252.21670965)
\curveto(111.3816991,252.51879366)(111.12128309,252.80525101)(110.77753241,253.07087635)
\curveto(110.53794973,253.26358435)(109.89211638,253.6542097)(108.84003234,254.23754306)
\curveto(107.7827403,254.83129374)(107.02753228,255.30525109)(106.56919893,255.65420977)
\curveto(106.12128291,256.01358445)(105.78274023,256.4042098)(105.54836556,256.82087648)
\curveto(105.32440688,257.2479605)(105.21503221,257.72191785)(105.21503221,258.2375432)
\curveto(105.21503221,259.12296056)(105.55357356,259.88858459)(106.23586558,260.52920995)
\curveto(106.9129496,261.1646273)(107.7827403,261.48754331)(108.84003234,261.48754331)
\curveto(109.4910737,261.48754331)(110.18378305,261.32608464)(110.92336575,261.00837663)
\curveto(111.25669909,260.85212729)(111.49107377,260.77920995)(111.6316991,260.77920995)
\curveto(111.79836578,260.77920995)(111.92857378,260.82087662)(112.02753245,260.90420996)
\curveto(112.12128312,260.99796063)(112.20461646,261.19587664)(112.27753246,261.48754331)
\closepath
\moveto(112.61086581,261.48754331)
}
}
{
\newrgbcolor{curcolor}{0 0 0}
\pscustom[linestyle=none,fillstyle=solid,fillcolor=curcolor]
{
\newpath
\moveto(116.40619472,254.05004305)
\curveto(116.40619472,252.81046034)(116.70827873,251.84170964)(117.32286142,251.13337628)
\curveto(117.9166121,250.42504292)(118.63536146,250.07087624)(119.46869483,250.07087624)
\curveto(120.01036151,250.07087624)(120.47911219,250.21670958)(120.88536154,250.50837626)
\curveto(121.28640289,250.81046027)(121.62494557,251.33129362)(121.90619491,252.07087631)
\lineto(122.17702825,251.88337631)
\curveto(122.05202825,251.05004294)(121.68223624,250.28962692)(121.07286155,249.61254289)
\curveto(120.45827886,248.93025127)(119.6874455,248.59170952)(118.76036147,248.59170952)
\curveto(117.76036143,248.59170952)(116.89577873,248.97712687)(116.17702804,249.75837623)
\curveto(115.46869468,250.53441759)(115.114528,251.58129363)(115.114528,252.90420968)
\curveto(115.114528,254.33129373)(115.47911202,255.44587643)(116.21869471,256.23754313)
\curveto(116.9530694,257.03962716)(117.8697361,257.4458765)(118.96869481,257.4458765)
\curveto(119.91140284,257.4458765)(120.6822362,257.13337649)(121.28119489,256.50837647)
\curveto(121.87494558,255.89379378)(122.17702825,255.07608442)(122.17702825,254.05004305)
\closepath
\moveto(116.40619472,254.5708764)
\lineto(120.28119485,254.5708764)
\curveto(120.24994552,255.11254309)(120.18744552,255.4875431)(120.09369485,255.69587644)
\curveto(119.93744551,256.03962712)(119.70827883,256.31046046)(119.40619482,256.50837647)
\curveto(119.09890281,256.70108448)(118.7864028,256.80004315)(118.46869479,256.80004315)
\curveto(117.96869477,256.80004315)(117.51556942,256.60212714)(117.11452808,256.21670979)
\curveto(116.70827873,255.82608445)(116.47390272,255.27920976)(116.40619472,254.5708764)
\closepath
\moveto(116.40619472,254.5708764)
}
}
{
\newrgbcolor{curcolor}{0 0 0}
\pscustom[linestyle=none,fillstyle=solid,fillcolor=curcolor]
{
\newpath
\moveto(125.71503294,259.92504326)
\lineto(125.71503294,257.19587649)
\lineto(127.67336634,257.19587649)
\lineto(127.67336634,256.55004314)
\lineto(125.71503294,256.55004314)
\lineto(125.71503294,251.13337628)
\curveto(125.71503294,250.59170959)(125.78795027,250.22191758)(125.94419961,250.02920957)
\curveto(126.11086628,249.84691757)(126.30878362,249.75837623)(126.5483663,249.75837623)
\curveto(126.75669964,249.75837623)(126.94940765,249.82087623)(127.13169965,249.94587624)
\curveto(127.32440766,250.07087624)(127.470241,250.25837625)(127.56919967,250.50837626)
\lineto(127.92336635,250.50837626)
\curveto(127.71503301,249.90941757)(127.41295033,249.45629355)(127.02753298,249.15420954)
\curveto(126.63690764,248.84691787)(126.23586629,248.69587619)(125.81919961,248.69587619)
\curveto(125.53795026,248.69587619)(125.26711692,248.77400113)(125.00669958,248.92504287)
\curveto(124.74107424,249.09170954)(124.53795023,249.31046021)(124.40253289,249.59170956)
\curveto(124.27753289,249.88337624)(124.21503288,250.33129358)(124.21503288,250.94587627)
\lineto(124.21503288,256.55004314)
\lineto(122.90253284,256.55004314)
\lineto(122.90253284,256.86254315)
\curveto(123.23586618,256.98754315)(123.57440753,257.20629383)(123.92336621,257.52920984)
\curveto(124.26711689,257.84691785)(124.5796169,258.2271272)(124.86086624,258.67504321)
\curveto(124.99628358,258.89379389)(125.19419959,259.31046057)(125.44419959,259.92504326)
\closepath
\moveto(125.71503294,259.92504326)
}
}
{
\newrgbcolor{curcolor}{0 0 0}
\pscustom[linestyle=none,fillstyle=solid,fillcolor=curcolor]
{
\newpath
\moveto(136.02936436,261.80004332)
\lineto(136.02936436,250.73754293)
\curveto(136.02936436,250.20629358)(136.06061503,249.8521269)(136.13353103,249.67504289)
\curveto(136.21686436,249.50837622)(136.33144837,249.38337622)(136.48769771,249.30004288)
\curveto(136.65436438,249.21670954)(136.95123906,249.17504288)(137.38353107,249.17504288)
\lineto(137.38353107,248.84170953)
\lineto(133.27936426,248.84170953)
\lineto(133.27936426,249.17504288)
\curveto(133.66478161,249.17504288)(133.93040562,249.20629354)(134.07103095,249.27920955)
\curveto(134.20644829,249.36254288)(134.31061496,249.49275089)(134.38353097,249.67504289)
\curveto(134.4668643,249.8521269)(134.50853097,250.20629358)(134.50853097,250.73754293)
\lineto(134.50853097,258.32087653)
\curveto(134.50853097,259.24796057)(134.48769764,259.82087659)(134.44603097,260.02920993)
\curveto(134.4043643,260.2479606)(134.3314483,260.40420994)(134.23769763,260.48754328)
\curveto(134.15436429,260.57087661)(134.03457229,260.61254328)(133.88353095,260.61254328)
\curveto(133.72728161,260.61254328)(133.52936427,260.56046061)(133.27936426,260.46670994)
\lineto(133.13353092,260.77920995)
\lineto(135.61269768,261.80004332)
\closepath
\moveto(136.02936436,261.80004332)
}
}
{
\newrgbcolor{curcolor}{0 0 0}
\pscustom[linestyle=none,fillstyle=solid,fillcolor=curcolor]
{
\newpath
\moveto(143.07286233,250.05004291)
\curveto(142.20827963,249.38337622)(141.67702895,248.9979602)(141.4686956,248.90420953)
\curveto(141.13536226,248.7531678)(140.78119558,248.67504286)(140.40619557,248.67504286)
\curveto(139.83327955,248.67504286)(139.36452886,248.86775127)(138.98952885,249.25837621)
\curveto(138.61452884,249.65941756)(138.42702883,250.18025091)(138.42702883,250.82087627)
\curveto(138.42702883,251.22191762)(138.51557017,251.5760843)(138.69786217,251.88337631)
\curveto(138.94786218,252.28441765)(139.3749462,252.664627)(139.98952889,253.02920968)
\curveto(140.61452891,253.40420969)(141.64057028,253.85212704)(143.07286233,254.38337639)
\lineto(143.07286233,254.69587641)
\curveto(143.07286233,255.52920977)(142.93744632,256.09691779)(142.67702898,256.4042098)
\curveto(142.41140364,256.72191781)(142.03119562,256.88337648)(141.53119561,256.88337648)
\curveto(141.14057026,256.88337648)(140.83327958,256.77920981)(140.61452891,256.57087647)
\curveto(140.37494623,256.36254313)(140.26036223,256.12296046)(140.26036223,255.86254311)
\lineto(140.28119556,255.34170976)
\curveto(140.28119556,255.05004308)(140.20827956,254.82608441)(140.07286222,254.67504307)
\curveto(139.93223688,254.53441773)(139.74473688,254.46670973)(139.5103622,254.46670973)
\curveto(139.28640353,254.46670973)(139.09890352,254.53962707)(138.94786218,254.69587641)
\curveto(138.80723684,254.84691774)(138.73952884,255.06046042)(138.73952884,255.34170976)
\curveto(138.73952884,255.86775111)(138.99994618,256.3469178)(139.53119554,256.77920981)
\curveto(140.07286222,257.22191783)(140.82807025,257.4458765)(141.80202895,257.4458765)
\curveto(142.53640364,257.4458765)(143.14057033,257.3208765)(143.61452901,257.07087649)
\curveto(143.97390369,256.87296048)(144.23952904,256.57608447)(144.40619571,256.17504312)
\curveto(144.51557038,255.90941778)(144.57286238,255.37296043)(144.57286238,254.5708764)
\lineto(144.57286238,251.73754297)
\curveto(144.57286238,250.94587627)(144.58327572,250.45629359)(144.61452905,250.27920958)
\curveto(144.65619572,250.09691758)(144.70827972,249.97191757)(144.78119572,249.90420957)
\curveto(144.84890372,249.84691757)(144.92702906,249.82087623)(145.0103624,249.82087623)
\curveto(145.10411307,249.82087623)(145.1978624,249.84170957)(145.28119574,249.88337624)
\curveto(145.40619574,249.96670957)(145.64577975,250.18546025)(146.01036243,250.55004293)
\lineto(146.01036243,250.05004291)
\curveto(145.32807041,249.13337621)(144.67702905,248.67504286)(144.05202903,248.67504286)
\curveto(143.76036235,248.67504286)(143.52077968,248.77920953)(143.34369567,248.98754287)
\curveto(143.177029,249.19587621)(143.08327966,249.55004289)(143.07286233,250.05004291)
\closepath
\moveto(143.07286233,250.63337626)
\lineto(143.07286233,253.82087637)
\curveto(142.15619563,253.4562937)(141.56244627,253.20108435)(141.30202893,253.05004301)
\curveto(140.81244625,252.76879367)(140.46869557,252.48754299)(140.26036223,252.19587632)
\curveto(140.05202889,251.90420964)(139.94786222,251.59170963)(139.94786222,251.25837628)
\curveto(139.94786222,250.81046027)(140.07807022,250.44587626)(140.34369556,250.15420958)
\curveto(140.60411291,249.8625429)(140.91140359,249.71670956)(141.26036226,249.71670956)
\curveto(141.71869561,249.71670956)(142.3228623,250.01879357)(143.07286233,250.63337626)
\closepath
\moveto(143.07286233,250.63337626)
}
}
{
\newrgbcolor{curcolor}{0 0 0}
\pscustom[linestyle=none,fillstyle=solid,fillcolor=curcolor]
{
\newpath
\moveto(146.15253366,257.19587649)
\lineto(150.04836713,257.19587649)
\lineto(150.04836713,256.86254315)
\lineto(149.86086713,256.86254315)
\curveto(149.57961778,256.86254315)(149.37128444,256.80004315)(149.2358671,256.67504314)
\curveto(149.1108671,256.56046047)(149.0483671,256.41462713)(149.0483671,256.23754313)
\curveto(149.0483671,255.99796045)(149.14211777,255.67504311)(149.34003377,255.25837643)
\lineto(151.38170051,251.02920961)
\lineto(153.23586725,255.65420977)
\curveto(153.34524192,255.90420978)(153.40253392,256.14379379)(153.40253392,256.38337647)
\curveto(153.40253392,256.49275114)(153.38170058,256.57608447)(153.34003392,256.63337647)
\curveto(153.28274191,256.70108448)(153.19940858,256.75837648)(153.09003391,256.80004315)
\curveto(152.99107524,256.84170982)(152.80878456,256.86254315)(152.54836722,256.86254315)
\lineto(152.54836722,257.19587649)
\lineto(155.27753398,257.19587649)
\lineto(155.27753398,256.86254315)
\curveto(155.05357531,256.83129381)(154.88170064,256.77920981)(154.75670063,256.69587648)
\curveto(154.63170063,256.62296047)(154.49628462,256.49275114)(154.36086729,256.30004313)
\curveto(154.30357528,256.21670979)(154.19940861,255.97712712)(154.04836727,255.59170977)
\lineto(150.65253382,247.25837614)
\curveto(150.31920048,246.45629211)(149.88170046,245.85212543)(149.34003377,245.44587608)
\curveto(148.80878442,245.0292094)(148.30357507,244.82087606)(147.81920039,244.82087606)
\curveto(147.45461771,244.82087606)(147.1577417,244.92504273)(146.92336702,245.13337607)
\curveto(146.69940835,245.33129207)(146.59003368,245.56566808)(146.59003368,245.84170943)
\curveto(146.59003368,246.09170943)(146.67336701,246.29483477)(146.84003369,246.44587611)
\curveto(147.00670036,246.60212545)(147.23586703,246.67504279)(147.52753371,246.67504279)
\curveto(147.72024172,246.67504279)(147.99107506,246.61254279)(148.34003374,246.48754278)
\curveto(148.59003375,246.39379211)(148.74107509,246.34170944)(148.79836709,246.34170944)
\curveto(148.97545109,246.34170944)(149.1733671,246.44066811)(149.38170044,246.63337612)
\curveto(149.60045112,246.81566813)(149.82440846,247.17504281)(150.04836713,247.71670949)
\lineto(150.65253382,249.17504288)
\lineto(147.65253371,255.46670977)
\curveto(147.55357504,255.65941777)(147.40774171,255.89379378)(147.2150337,256.17504312)
\curveto(147.05878436,256.38337647)(146.93378436,256.5187938)(146.84003369,256.59170981)
\curveto(146.69940835,256.68546048)(146.47024167,256.77920981)(146.15253366,256.86254315)
\closepath
\moveto(146.15253366,257.19587649)
}
}
{
\newrgbcolor{curcolor}{0 0 0}
\pscustom[linestyle=none,fillstyle=solid,fillcolor=curcolor]
{
\newpath
\moveto(157.36086739,254.05004305)
\curveto(157.36086739,252.81046034)(157.6629514,251.84170964)(158.27753409,251.13337628)
\curveto(158.87128478,250.42504292)(159.59003414,250.07087624)(160.4233675,250.07087624)
\curveto(160.96503419,250.07087624)(161.43378487,250.21670958)(161.84003422,250.50837626)
\curveto(162.24107556,250.81046027)(162.57961824,251.33129362)(162.86086759,252.07087631)
\lineto(163.13170093,251.88337631)
\curveto(163.00670093,251.05004294)(162.63690891,250.28962692)(162.02753422,249.61254289)
\curveto(161.41295154,248.93025127)(160.64211817,248.59170952)(159.71503414,248.59170952)
\curveto(158.71503411,248.59170952)(157.85045141,248.97712687)(157.13170072,249.75837623)
\curveto(156.42336736,250.53441759)(156.06920068,251.58129363)(156.06920068,252.90420968)
\curveto(156.06920068,254.33129373)(156.43378469,255.44587643)(157.17336739,256.23754313)
\curveto(157.90774208,257.03962716)(158.82440878,257.4458765)(159.92336748,257.4458765)
\curveto(160.86607552,257.4458765)(161.63690888,257.13337649)(162.23586756,256.50837647)
\curveto(162.82961825,255.89379378)(163.13170093,255.07608442)(163.13170093,254.05004305)
\closepath
\moveto(157.36086739,254.5708764)
\lineto(161.23586753,254.5708764)
\curveto(161.20461819,255.11254309)(161.14211819,255.4875431)(161.04836752,255.69587644)
\curveto(160.89211818,256.03962712)(160.66295151,256.31046046)(160.3608675,256.50837647)
\curveto(160.05357549,256.70108448)(159.74107548,256.80004315)(159.42336746,256.80004315)
\curveto(158.92336745,256.80004315)(158.4702421,256.60212714)(158.06920075,256.21670979)
\curveto(157.6629514,255.82608445)(157.42857539,255.27920976)(157.36086739,254.5708764)
\closepath
\moveto(157.36086739,254.5708764)
}
}
{
\newrgbcolor{curcolor}{0 0 0}
\pscustom[linestyle=none,fillstyle=solid,fillcolor=curcolor]
{
\newpath
\moveto(166.69053895,257.4458765)
\lineto(166.69053895,255.5500431)
\curveto(167.3988723,256.81046048)(168.117623,257.4458765)(168.85720569,257.4458765)
\curveto(169.19053903,257.4458765)(169.46137238,257.34170983)(169.66970572,257.13337649)
\curveto(169.88845639,256.92504315)(170.00303906,256.68546048)(170.00303906,256.42504313)
\curveto(170.00303906,256.18546046)(169.91970573,255.98754312)(169.75303905,255.82087645)
\curveto(169.59678972,255.65420977)(169.41970571,255.57087644)(169.21137237,255.57087644)
\curveto(168.98741369,255.57087644)(168.74262302,255.67504311)(168.48220568,255.88337645)
\curveto(168.21658033,256.10212712)(168.02387233,256.21670979)(167.89887232,256.21670979)
\curveto(167.78428965,256.21670979)(167.66970565,256.15420979)(167.54470564,256.02920979)
\curveto(167.2634563,255.77920978)(166.98220562,255.3677511)(166.69053895,254.80004308)
\lineto(166.69053895,250.80004293)
\curveto(166.69053895,250.32608425)(166.75303895,249.97191757)(166.87803895,249.7375429)
\curveto(166.94574696,249.58129356)(167.07595629,249.45108422)(167.2738723,249.34170955)
\curveto(167.48220564,249.22712688)(167.77387232,249.17504288)(168.14887233,249.17504288)
\lineto(168.14887233,248.84170953)
\lineto(163.87803885,248.84170953)
\lineto(163.87803885,249.17504288)
\curveto(164.30512286,249.17504288)(164.62803887,249.23754288)(164.83637221,249.36254288)
\curveto(164.98741355,249.45629355)(165.09158022,249.61775089)(165.14887223,249.84170957)
\curveto(165.17491356,249.93546024)(165.19053889,250.22712691)(165.19053889,250.7167096)
\lineto(165.19053889,253.94587638)
\curveto(165.19053889,254.91462708)(165.16970556,255.4927511)(165.12803889,255.67504311)
\curveto(165.08637222,255.86775111)(165.00824689,256.00837645)(164.89887222,256.09170979)
\curveto(164.79991355,256.17504312)(164.67491354,256.21670979)(164.5238722,256.21670979)
\curveto(164.3259562,256.21670979)(164.11241352,256.17504312)(163.87803885,256.09170979)
\lineto(163.79470551,256.42504313)
\lineto(166.31553893,257.4458765)
\closepath
\moveto(166.69053895,257.4458765)
}
}
{
\newrgbcolor{curcolor}{0 0 0}
\pscustom[linewidth=0.7466667,linecolor=curcolor]
{
\newpath
\moveto(0.37333335,54.99831554)
\lineto(647.22231626,54.99831554)
\lineto(647.22231626,83.02956987)
\lineto(0.37333335,83.02956987)
\closepath
\moveto(0.37333335,54.99831554)
}
}
\end{pspicture}
}
    %\captionsetup{width=0.75\linewidth}
    \caption{On disk a ZFS pool consists of a set of object sets\cite{mckusick_zfs_2015_presentation}.
    The first layer is the Meta-Object Set, consisting of the filesystems, snapshots, and other components
    that make up the storage pool, along with essential object sets like 
    the master object set for metadata and the space map for storing what parts of the disks have
    and have not been already filled with data.}
    \label{fig:ZFSOnDisk}
\end{figure}

\begin{figure}[H]
    \centering
    \resizebox{!}{0.2\textheight}{%LaTeX with PSTricks extensions
%%Creator: Inkscape 1.0.2-2 (e86c870879, 2021-01-15)
%%Please note this file requires PSTricks extensions
\psset{xunit=.5pt,yunit=.5pt,runit=.5pt}
\begin{pspicture}(647.59466553,324.84732056)
{
\newrgbcolor{curcolor}{0 0 0}
\pscustom[linewidth=0.74666665,linecolor=curcolor]
{
\newpath
\moveto(92.67333129,83.55759167)
\lineto(92.67333129,55.82425903)
}
}
{
\newrgbcolor{curcolor}{0 0 0}
\pscustom[linewidth=0.74666665,linecolor=curcolor]
{
\newpath
\moveto(184.93999565,83.55759167)
\lineto(184.93999565,55.82425903)
}
}
{
\newrgbcolor{curcolor}{0 0 0}
\pscustom[linewidth=0.74666665,linecolor=curcolor]
{
\newpath
\moveto(277.33999334,83.55759167)
\lineto(277.33999334,55.82425903)
}
}
{
\newrgbcolor{curcolor}{0 0 0}
\pscustom[linewidth=0.74666665,linecolor=curcolor]
{
\newpath
\moveto(369.6066577,83.55759167)
\lineto(369.6066577,55.82425903)
}
}
{
\newrgbcolor{curcolor}{0 0 0}
\pscustom[linewidth=0.74666665,linecolor=curcolor]
{
\newpath
\moveto(462.00665539,83.55759167)
\lineto(462.00665539,55.82425903)
}
}
{
\newrgbcolor{curcolor}{0 0 0}
\pscustom[linewidth=0.74666665,linecolor=curcolor]
{
\newpath
\moveto(554.13998642,83.55759167)
\lineto(554.13998642,55.82425903)
}
}
{
\newrgbcolor{curcolor}{0 0 0}
\pscustom[linewidth=0.74666665,linecolor=curcolor]
{
\newpath
\moveto(0.3733336,176.46692268)
\lineto(647.22131743,176.46692268)
\lineto(647.22131743,204.49865532)
\lineto(0.3733336,204.49865532)
\closepath
}
}
{
\newrgbcolor{curcolor}{0 0 0}
\pscustom[linewidth=0.74666665,linecolor=curcolor]
{
\newpath
\moveto(92.67333129,204.28278865)
\lineto(92.67333129,176.68278934)
}
}
{
\newrgbcolor{curcolor}{0 0 0}
\pscustom[linewidth=0.74666665,linecolor=curcolor]
{
\newpath
\moveto(184.93999565,204.28278865)
\lineto(184.93999565,176.68278934)
}
}
{
\newrgbcolor{curcolor}{0 0 0}
\pscustom[linewidth=0.74666665,linecolor=curcolor]
{
\newpath
\moveto(277.33999334,204.28278865)
\lineto(277.33999334,176.68278934)
}
}
{
\newrgbcolor{curcolor}{0 0 0}
\pscustom[linewidth=0.74666665,linecolor=curcolor]
{
\newpath
\moveto(369.6066577,204.28278865)
\lineto(369.6066577,176.68278934)
}
}
{
\newrgbcolor{curcolor}{0 0 0}
\pscustom[linewidth=0.74666665,linecolor=curcolor]
{
\newpath
\moveto(462.00665539,204.28278865)
\lineto(462.00665539,176.68278934)
}
}
{
\newrgbcolor{curcolor}{0 0 0}
\pscustom[linewidth=0.74666665,linecolor=curcolor]
{
\newpath
\moveto(554.13998642,204.28278865)
\lineto(554.13998642,176.68278934)
}
}
{
\newrgbcolor{curcolor}{0 0 0}
\pscustom[linewidth=0.74666665,linecolor=curcolor]
{
\newpath
\moveto(0.40666693,83.70759167)
\lineto(323.47332552,120.64092408)
}
}
{
\newrgbcolor{curcolor}{0 0 0}
\pscustom[linewidth=0.74666665,linecolor=curcolor]
{
\newpath
\moveto(646.53998411,83.70759167)
\lineto(323.47332552,120.64092408)
}
}
{
\newrgbcolor{curcolor}{0 0 0}
\pscustom[linewidth=0.74666665,linecolor=curcolor]
{
\newpath
\moveto(237.83231433,27.69733974)
\lineto(413.16564328,55.29733905)
}
}
{
\newrgbcolor{curcolor}{0 0 0}
\pscustom[linewidth=0.74666665,linecolor=curcolor]
{
\newpath
\moveto(588.63230556,27.69733974)
\lineto(413.16564328,55.29733905)
}
}
{
\newrgbcolor{curcolor}{0 0 0}
\pscustom[linewidth=0.74666665,linecolor=curcolor]
{
\newpath
\moveto(237.08964768,0.37334042)
\lineto(588.64163889,0.37334042)
\lineto(588.64163889,28.40507305)
\lineto(237.08964768,28.40507305)
\closepath
}
}
{
\newrgbcolor{curcolor}{0 0 0}
\pscustom[linewidth=0.74666665,linecolor=curcolor]
{
\newpath
\moveto(0.40666693,203.70758867)
\lineto(323.47332552,240.64092108)
}
}
{
\newrgbcolor{curcolor}{0 0 0}
\pscustom[linewidth=0.74666665,linecolor=curcolor]
{
\newpath
\moveto(646.53998411,203.70758867)
\lineto(323.47332552,240.64092108)
}
}
{
\newrgbcolor{curcolor}{0 0 0}
\pscustom[linewidth=0.74666665,linecolor=curcolor]
{
\newpath
\moveto(267.63732692,240.28225442)
\lineto(378.42119082,240.28225442)
\lineto(378.42119082,268.31452039)
\lineto(267.63732692,268.31452039)
\closepath
}
}
{
\newrgbcolor{curcolor}{0 0 0}
\pscustom[linewidth=0.74666665,linecolor=curcolor]
{
\newpath
\moveto(267.63732692,296.44225302)
\lineto(378.42119082,296.44225302)
\lineto(378.42119082,324.47398565)
\lineto(267.63732692,324.47398565)
\closepath
}
}
{
\newrgbcolor{curcolor}{0 0 0}
\pscustom[linewidth=0.74666665,linecolor=curcolor]
{
\newpath
\moveto(0.40666693,166.77425626)
\lineto(7.74000008,166.77425626)
}
}
{
\newrgbcolor{curcolor}{0 0 0}
\pscustom[linewidth=0.74666665,linecolor=curcolor]
{
\newpath
\moveto(15.20666656,166.77425626)
\lineto(22.67333304,166.77425626)
}
}
{
\newrgbcolor{curcolor}{0 0 0}
\pscustom[linewidth=0.74666665,linecolor=curcolor]
{
\newpath
\moveto(30.00666619,166.77425626)
\lineto(37.47333267,166.77425626)
}
}
{
\newrgbcolor{curcolor}{0 0 0}
\pscustom[linewidth=0.74666665,linecolor=curcolor]
{
\newpath
\moveto(44.93999915,166.77425626)
\lineto(52.2733323,166.77425626)
}
}
{
\newrgbcolor{curcolor}{0 0 0}
\pscustom[linewidth=0.74666665,linecolor=curcolor]
{
\newpath
\moveto(59.73999878,166.77425626)
\lineto(67.20666526,166.77425626)
}
}
{
\newrgbcolor{curcolor}{0 0 0}
\pscustom[linewidth=0.74666665,linecolor=curcolor]
{
\newpath
\moveto(74.67333174,166.77425626)
\lineto(82.00666489,166.77425626)
}
}
{
\newrgbcolor{curcolor}{0 0 0}
\pscustom[linewidth=0.74666665,linecolor=curcolor]
{
\newpath
\moveto(89.47333137,166.77425626)
\lineto(96.80666452,166.77425626)
}
}
{
\newrgbcolor{curcolor}{0 0 0}
\pscustom[linewidth=0.74666665,linecolor=curcolor]
{
\newpath
\moveto(104.40666433,166.77425626)
\lineto(111.73999748,166.77425626)
}
}
{
\newrgbcolor{curcolor}{0 0 0}
\pscustom[linewidth=0.74666665,linecolor=curcolor]
{
\newpath
\moveto(119.20666396,166.77425626)
\lineto(126.67333044,166.77425626)
}
}
{
\newrgbcolor{curcolor}{0 0 0}
\pscustom[linewidth=0.74666665,linecolor=curcolor]
{
\newpath
\moveto(134.13999692,166.77425626)
\lineto(141.47333007,166.77425626)
}
}
{
\newrgbcolor{curcolor}{0 0 0}
\pscustom[linewidth=0.74666665,linecolor=curcolor]
{
\newpath
\moveto(148.93999655,166.77425626)
\lineto(156.2733297,166.77425626)
}
}
{
\newrgbcolor{curcolor}{0 0 0}
\pscustom[linewidth=0.74666665,linecolor=curcolor]
{
\newpath
\moveto(163.73999618,166.77425626)
\lineto(171.20666266,166.77425626)
}
}
{
\newrgbcolor{curcolor}{0 0 0}
\pscustom[linewidth=0.74666665,linecolor=curcolor]
{
\newpath
\moveto(178.67332914,166.77425626)
\lineto(186.00666229,166.77425626)
}
}
{
\newrgbcolor{curcolor}{0 0 0}
\pscustom[linewidth=0.74666665,linecolor=curcolor]
{
\newpath
\moveto(193.47332877,166.77425626)
\lineto(200.93999525,166.77425626)
}
}
{
\newrgbcolor{curcolor}{0 0 0}
\pscustom[linewidth=0.74666665,linecolor=curcolor]
{
\newpath
\moveto(208.2733284,166.77425626)
\lineto(215.73999488,166.77425626)
}
}
{
\newrgbcolor{curcolor}{0 0 0}
\pscustom[linewidth=0.74666665,linecolor=curcolor]
{
\newpath
\moveto(223.20666136,166.77425626)
\lineto(230.53999451,166.77425626)
}
}
{
\newrgbcolor{curcolor}{0 0 0}
\pscustom[linewidth=0.74666665,linecolor=curcolor]
{
\newpath
\moveto(238.00666099,166.77425626)
\lineto(245.47332747,166.77425626)
}
}
{
\newrgbcolor{curcolor}{0 0 0}
\pscustom[linewidth=0.74666665,linecolor=curcolor]
{
\newpath
\moveto(252.93999395,166.77425626)
\lineto(260.40666043,166.77425626)
}
}
{
\newrgbcolor{curcolor}{0 0 0}
\pscustom[linewidth=0.74666665,linecolor=curcolor]
{
\newpath
\moveto(267.73999358,166.77425626)
\lineto(275.07332673,166.77425626)
}
}
{
\newrgbcolor{curcolor}{0 0 0}
\pscustom[linewidth=0.74666665,linecolor=curcolor]
{
\newpath
\moveto(282.67332654,166.77425626)
\lineto(290.00665969,166.77425626)
}
}
{
\newrgbcolor{curcolor}{0 0 0}
\pscustom[linewidth=0.74666665,linecolor=curcolor]
{
\newpath
\moveto(297.47332617,166.77425626)
\lineto(304.93999265,166.77425626)
}
}
{
\newrgbcolor{curcolor}{0 0 0}
\pscustom[linewidth=0.74666665,linecolor=curcolor]
{
\newpath
\moveto(312.40665913,166.77425626)
\lineto(319.73999228,166.77425626)
}
}
{
\newrgbcolor{curcolor}{0 0 0}
\pscustom[linewidth=0.74666665,linecolor=curcolor]
{
\newpath
\moveto(327.20665876,166.77425626)
\lineto(334.53999191,166.77425626)
}
}
{
\newrgbcolor{curcolor}{0 0 0}
\pscustom[linewidth=0.74666665,linecolor=curcolor]
{
\newpath
\moveto(342.00665839,166.77425626)
\lineto(349.47332487,166.77425626)
}
}
{
\newrgbcolor{curcolor}{0 0 0}
\pscustom[linewidth=0.74666665,linecolor=curcolor]
{
\newpath
\moveto(356.93999135,166.77425626)
\lineto(364.2733245,166.77425626)
}
}
{
\newrgbcolor{curcolor}{0 0 0}
\pscustom[linewidth=0.74666665,linecolor=curcolor]
{
\newpath
\moveto(371.73999098,166.77425626)
\lineto(379.20665746,166.77425626)
}
}
{
\newrgbcolor{curcolor}{0 0 0}
\pscustom[linewidth=0.74666665,linecolor=curcolor]
{
\newpath
\moveto(386.53999061,166.77425626)
\lineto(394.00665709,166.77425626)
}
}
{
\newrgbcolor{curcolor}{0 0 0}
\pscustom[linewidth=0.74666665,linecolor=curcolor]
{
\newpath
\moveto(401.47332357,166.77425626)
\lineto(408.80665672,166.77425626)
}
}
{
\newrgbcolor{curcolor}{0 0 0}
\pscustom[linewidth=0.74666665,linecolor=curcolor]
{
\newpath
\moveto(416.40665653,166.77425626)
\lineto(423.73998968,166.77425626)
}
}
{
\newrgbcolor{curcolor}{0 0 0}
\pscustom[linewidth=0.74666665,linecolor=curcolor]
{
\newpath
\moveto(431.07332283,166.77425626)
\lineto(438.53998931,166.77425626)
}
}
{
\newrgbcolor{curcolor}{0 0 0}
\pscustom[linewidth=0.74666665,linecolor=curcolor]
{
\newpath
\moveto(446.00665579,166.77425626)
\lineto(453.33998894,166.77425626)
}
}
{
\newrgbcolor{curcolor}{0 0 0}
\pscustom[linewidth=0.74666665,linecolor=curcolor]
{
\newpath
\moveto(460.93998875,166.77425626)
\lineto(468.2733219,166.77425626)
}
}
{
\newrgbcolor{curcolor}{0 0 0}
\pscustom[linewidth=0.74666665,linecolor=curcolor]
{
\newpath
\moveto(475.73998838,166.77425626)
\lineto(483.20665486,166.77425626)
}
}
{
\newrgbcolor{curcolor}{0 0 0}
\pscustom[linewidth=0.74666665,linecolor=curcolor]
{
\newpath
\moveto(490.53998801,166.77425626)
\lineto(498.00665449,166.77425626)
}
}
{
\newrgbcolor{curcolor}{0 0 0}
\pscustom[linewidth=0.74666665,linecolor=curcolor]
{
\newpath
\moveto(505.47332097,166.77425626)
\lineto(512.80665412,166.77425626)
}
}
{
\newrgbcolor{curcolor}{0 0 0}
\pscustom[linewidth=0.74666665,linecolor=curcolor]
{
\newpath
\moveto(520.2733206,166.77425626)
\lineto(527.73998708,166.77425626)
}
}
{
\newrgbcolor{curcolor}{0 0 0}
\pscustom[linewidth=0.74666665,linecolor=curcolor]
{
\newpath
\moveto(535.20665356,166.77425626)
\lineto(542.53998671,166.77425626)
}
}
{
\newrgbcolor{curcolor}{0 0 0}
\pscustom[linewidth=0.74666665,linecolor=curcolor]
{
\newpath
\moveto(550.00665319,166.77425626)
\lineto(557.33998634,166.77425626)
}
}
{
\newrgbcolor{curcolor}{0 0 0}
\pscustom[linewidth=0.74666665,linecolor=curcolor]
{
\newpath
\moveto(564.80665282,166.77425626)
\lineto(572.2733193,166.77425626)
}
}
{
\newrgbcolor{curcolor}{0 0 0}
\pscustom[linewidth=0.74666665,linecolor=curcolor]
{
\newpath
\moveto(579.73998578,166.77425626)
\lineto(587.07331893,166.77425626)
}
}
{
\newrgbcolor{curcolor}{0 0 0}
\pscustom[linewidth=0.74666665,linecolor=curcolor]
{
\newpath
\moveto(594.67331874,166.77425626)
\lineto(602.00665189,166.77425626)
}
}
{
\newrgbcolor{curcolor}{0 0 0}
\pscustom[linewidth=0.74666665,linecolor=curcolor]
{
\newpath
\moveto(609.33998504,166.77425626)
\lineto(616.80665152,166.77425626)
}
}
{
\newrgbcolor{curcolor}{0 0 0}
\pscustom[linewidth=0.74666665,linecolor=curcolor]
{
\newpath
\moveto(624.273318,166.77425626)
\lineto(631.60665115,166.77425626)
}
}
{
\newrgbcolor{curcolor}{0 0 0}
\pscustom[linewidth=0.74666665,linecolor=curcolor]
{
\newpath
\moveto(639.20665096,166.77425626)
\lineto(646.53998411,166.77425626)
}
}
{
\newrgbcolor{curcolor}{0 0 0}
\pscustom[linewidth=0.74666665,linecolor=curcolor]
{
\newpath
\moveto(0.40666693,286.90758659)
\lineto(7.74000008,286.90758659)
}
}
{
\newrgbcolor{curcolor}{0 0 0}
\pscustom[linewidth=0.74666665,linecolor=curcolor]
{
\newpath
\moveto(15.20666656,286.90758659)
\lineto(22.67333304,286.90758659)
}
}
{
\newrgbcolor{curcolor}{0 0 0}
\pscustom[linewidth=0.74666665,linecolor=curcolor]
{
\newpath
\moveto(30.00666619,286.90758659)
\lineto(37.47333267,286.90758659)
}
}
{
\newrgbcolor{curcolor}{0 0 0}
\pscustom[linewidth=0.74666665,linecolor=curcolor]
{
\newpath
\moveto(44.93999915,286.90758659)
\lineto(52.2733323,286.90758659)
}
}
{
\newrgbcolor{curcolor}{0 0 0}
\pscustom[linewidth=0.74666665,linecolor=curcolor]
{
\newpath
\moveto(59.73999878,286.90758659)
\lineto(67.20666526,286.90758659)
}
}
{
\newrgbcolor{curcolor}{0 0 0}
\pscustom[linewidth=0.74666665,linecolor=curcolor]
{
\newpath
\moveto(74.67333174,286.90758659)
\lineto(82.00666489,286.90758659)
}
}
{
\newrgbcolor{curcolor}{0 0 0}
\pscustom[linewidth=0.74666665,linecolor=curcolor]
{
\newpath
\moveto(89.47333137,286.90758659)
\lineto(96.80666452,286.90758659)
}
}
{
\newrgbcolor{curcolor}{0 0 0}
\pscustom[linewidth=0.74666665,linecolor=curcolor]
{
\newpath
\moveto(104.40666433,286.90758659)
\lineto(111.73999748,286.90758659)
}
}
{
\newrgbcolor{curcolor}{0 0 0}
\pscustom[linewidth=0.74666665,linecolor=curcolor]
{
\newpath
\moveto(119.20666396,286.90758659)
\lineto(126.67333044,286.90758659)
}
}
{
\newrgbcolor{curcolor}{0 0 0}
\pscustom[linewidth=0.74666665,linecolor=curcolor]
{
\newpath
\moveto(134.13999692,286.90758659)
\lineto(141.47333007,286.90758659)
}
}
{
\newrgbcolor{curcolor}{0 0 0}
\pscustom[linewidth=0.74666665,linecolor=curcolor]
{
\newpath
\moveto(148.93999655,286.90758659)
\lineto(156.2733297,286.90758659)
}
}
{
\newrgbcolor{curcolor}{0 0 0}
\pscustom[linewidth=0.74666665,linecolor=curcolor]
{
\newpath
\moveto(163.73999618,286.90758659)
\lineto(171.20666266,286.90758659)
}
}
{
\newrgbcolor{curcolor}{0 0 0}
\pscustom[linewidth=0.74666665,linecolor=curcolor]
{
\newpath
\moveto(178.67332914,286.90758659)
\lineto(186.00666229,286.90758659)
}
}
{
\newrgbcolor{curcolor}{0 0 0}
\pscustom[linewidth=0.74666665,linecolor=curcolor]
{
\newpath
\moveto(193.47332877,286.90758659)
\lineto(200.93999525,286.90758659)
}
}
{
\newrgbcolor{curcolor}{0 0 0}
\pscustom[linewidth=0.74666665,linecolor=curcolor]
{
\newpath
\moveto(208.2733284,286.90758659)
\lineto(215.73999488,286.90758659)
}
}
{
\newrgbcolor{curcolor}{0 0 0}
\pscustom[linewidth=0.74666665,linecolor=curcolor]
{
\newpath
\moveto(223.20666136,286.90758659)
\lineto(230.53999451,286.90758659)
}
}
{
\newrgbcolor{curcolor}{0 0 0}
\pscustom[linewidth=0.74666665,linecolor=curcolor]
{
\newpath
\moveto(238.00666099,286.90758659)
\lineto(245.47332747,286.90758659)
}
}
{
\newrgbcolor{curcolor}{0 0 0}
\pscustom[linewidth=0.74666665,linecolor=curcolor]
{
\newpath
\moveto(252.93999395,286.90758659)
\lineto(260.40666043,286.90758659)
}
}
{
\newrgbcolor{curcolor}{0 0 0}
\pscustom[linewidth=0.74666665,linecolor=curcolor]
{
\newpath
\moveto(267.73999358,286.90758659)
\lineto(275.07332673,286.90758659)
}
}
{
\newrgbcolor{curcolor}{0 0 0}
\pscustom[linewidth=0.74666665,linecolor=curcolor]
{
\newpath
\moveto(282.67332654,286.90758659)
\lineto(290.00665969,286.90758659)
}
}
{
\newrgbcolor{curcolor}{0 0 0}
\pscustom[linewidth=0.74666665,linecolor=curcolor]
{
\newpath
\moveto(297.47332617,286.90758659)
\lineto(304.93999265,286.90758659)
}
}
{
\newrgbcolor{curcolor}{0 0 0}
\pscustom[linewidth=0.74666665,linecolor=curcolor]
{
\newpath
\moveto(312.40665913,286.90758659)
\lineto(319.73999228,286.90758659)
}
}
{
\newrgbcolor{curcolor}{0 0 0}
\pscustom[linewidth=0.74666665,linecolor=curcolor]
{
\newpath
\moveto(327.20665876,286.90758659)
\lineto(334.53999191,286.90758659)
}
}
{
\newrgbcolor{curcolor}{0 0 0}
\pscustom[linewidth=0.74666665,linecolor=curcolor]
{
\newpath
\moveto(342.00665839,286.90758659)
\lineto(349.47332487,286.90758659)
}
}
{
\newrgbcolor{curcolor}{0 0 0}
\pscustom[linewidth=0.74666665,linecolor=curcolor]
{
\newpath
\moveto(356.93999135,286.90758659)
\lineto(364.2733245,286.90758659)
}
}
{
\newrgbcolor{curcolor}{0 0 0}
\pscustom[linewidth=0.74666665,linecolor=curcolor]
{
\newpath
\moveto(371.73999098,286.90758659)
\lineto(379.20665746,286.90758659)
}
}
{
\newrgbcolor{curcolor}{0 0 0}
\pscustom[linewidth=0.74666665,linecolor=curcolor]
{
\newpath
\moveto(386.53999061,286.90758659)
\lineto(394.00665709,286.90758659)
}
}
{
\newrgbcolor{curcolor}{0 0 0}
\pscustom[linewidth=0.74666665,linecolor=curcolor]
{
\newpath
\moveto(401.47332357,286.90758659)
\lineto(408.80665672,286.90758659)
}
}
{
\newrgbcolor{curcolor}{0 0 0}
\pscustom[linewidth=0.74666665,linecolor=curcolor]
{
\newpath
\moveto(416.40665653,286.90758659)
\lineto(423.73998968,286.90758659)
}
}
{
\newrgbcolor{curcolor}{0 0 0}
\pscustom[linewidth=0.74666665,linecolor=curcolor]
{
\newpath
\moveto(431.07332283,286.90758659)
\lineto(438.53998931,286.90758659)
}
}
{
\newrgbcolor{curcolor}{0 0 0}
\pscustom[linewidth=0.74666665,linecolor=curcolor]
{
\newpath
\moveto(446.00665579,286.90758659)
\lineto(453.33998894,286.90758659)
}
}
{
\newrgbcolor{curcolor}{0 0 0}
\pscustom[linewidth=0.74666665,linecolor=curcolor]
{
\newpath
\moveto(460.93998875,286.90758659)
\lineto(468.2733219,286.90758659)
}
}
{
\newrgbcolor{curcolor}{0 0 0}
\pscustom[linewidth=0.74666665,linecolor=curcolor]
{
\newpath
\moveto(475.73998838,286.90758659)
\lineto(483.20665486,286.90758659)
}
}
{
\newrgbcolor{curcolor}{0 0 0}
\pscustom[linewidth=0.74666665,linecolor=curcolor]
{
\newpath
\moveto(490.53998801,286.90758659)
\lineto(498.00665449,286.90758659)
}
}
{
\newrgbcolor{curcolor}{0 0 0}
\pscustom[linewidth=0.74666665,linecolor=curcolor]
{
\newpath
\moveto(505.47332097,286.90758659)
\lineto(512.80665412,286.90758659)
}
}
{
\newrgbcolor{curcolor}{0 0 0}
\pscustom[linewidth=0.74666665,linecolor=curcolor]
{
\newpath
\moveto(520.2733206,286.90758659)
\lineto(527.73998708,286.90758659)
}
}
{
\newrgbcolor{curcolor}{0 0 0}
\pscustom[linewidth=0.74666665,linecolor=curcolor]
{
\newpath
\moveto(535.20665356,286.90758659)
\lineto(542.53998671,286.90758659)
}
}
{
\newrgbcolor{curcolor}{0 0 0}
\pscustom[linewidth=0.74666665,linecolor=curcolor]
{
\newpath
\moveto(550.00665319,286.90758659)
\lineto(557.33998634,286.90758659)
}
}
{
\newrgbcolor{curcolor}{0 0 0}
\pscustom[linewidth=0.74666665,linecolor=curcolor]
{
\newpath
\moveto(564.80665282,286.90758659)
\lineto(572.2733193,286.90758659)
}
}
{
\newrgbcolor{curcolor}{0 0 0}
\pscustom[linewidth=0.74666665,linecolor=curcolor]
{
\newpath
\moveto(579.73998578,286.90758659)
\lineto(587.07331893,286.90758659)
}
}
{
\newrgbcolor{curcolor}{0 0 0}
\pscustom[linewidth=0.74666665,linecolor=curcolor]
{
\newpath
\moveto(594.67331874,286.90758659)
\lineto(602.00665189,286.90758659)
}
}
{
\newrgbcolor{curcolor}{0 0 0}
\pscustom[linewidth=0.74666665,linecolor=curcolor]
{
\newpath
\moveto(609.33998504,286.90758659)
\lineto(616.80665152,286.90758659)
}
}
{
\newrgbcolor{curcolor}{0 0 0}
\pscustom[linewidth=0.74666665,linecolor=curcolor]
{
\newpath
\moveto(624.273318,286.90758659)
\lineto(631.60665115,286.90758659)
}
}
{
\newrgbcolor{curcolor}{0 0 0}
\pscustom[linewidth=0.74666665,linecolor=curcolor]
{
\newpath
\moveto(639.20665096,286.90758659)
\lineto(646.53998411,286.90758659)
}
}
{
\newrgbcolor{curcolor}{0 0 0}
\pscustom[linewidth=0.74666665,linecolor=curcolor]
{
\newpath
\moveto(0.3733336,55.67505904)
\lineto(647.22131743,55.67505904)
\lineto(647.22131743,83.70679167)
\lineto(0.3733336,83.70679167)
\closepath
}
}
{
\newrgbcolor{curcolor}{0 0 0}
\pscustom[linestyle=none,fillstyle=solid,fillcolor=curcolor]
{
\newpath
\moveto(293.54813096,312.45424936)
\lineto(293.54813096,307.39565574)
\curveto(293.54813096,306.42950993)(293.56939832,305.83706203)(293.61193304,305.61831203)
\curveto(293.66054415,305.40563843)(293.73346081,305.2567669)(293.83068303,305.17169746)
\curveto(293.93398164,305.08662802)(294.05247122,305.0440933)(294.18615178,305.0440933)
\curveto(294.37451983,305.0440933)(294.58719343,305.0957426)(294.82417259,305.19904121)
\lineto(294.95177676,304.8800308)
\lineto(292.45438099,303.85008291)
\lineto(292.04422475,303.85008291)
\lineto(292.04422475,305.61831203)
\curveto(291.32721087,304.84053427)(290.78033589,304.35138498)(290.40359979,304.15086415)
\curveto(290.02686369,303.95034332)(289.62886022,303.85008291)(289.2095894,303.85008291)
\curveto(288.74170747,303.85008291)(288.33458942,303.98376346)(287.98823526,304.25112457)
\curveto(287.64795749,304.52456206)(287.41097833,304.87395441)(287.27729778,305.29930162)
\curveto(287.14361723,305.72464884)(287.07677695,306.32621132)(287.07677695,307.10398908)
\lineto(287.07677695,310.83185357)
\curveto(287.07677695,311.22681884)(287.03424223,311.50025633)(286.94917279,311.65216605)
\curveto(286.86410335,311.80407577)(286.73649918,311.91952715)(286.5663603,311.99852021)
\curveto(286.4022978,312.08358965)(286.10151656,312.12308618)(285.66401657,312.11700979)
\lineto(285.66401657,312.45424936)
\lineto(288.58979775,312.45424936)
\lineto(288.58979775,306.86700992)
\curveto(288.58979775,306.08923216)(288.7234783,305.57881551)(288.99083941,305.33575996)
\curveto(289.2642769,305.09270441)(289.59240189,304.97117663)(289.97521438,304.97117663)
\curveto(290.2364991,304.97117663)(290.53120395,305.05320788)(290.85932894,305.21727038)
\curveto(291.19353032,305.38133287)(291.58849559,305.69426689)(292.04422475,306.15607244)
\lineto(292.04422475,310.88654107)
\curveto(292.04422475,311.36049939)(291.95611711,311.6795098)(291.77990184,311.84357229)
\curveto(291.60976295,312.01371118)(291.25125602,312.10485701)(290.70438103,312.11700979)
\lineto(290.70438103,312.45424936)
\closepath
}
}
{
\newrgbcolor{curcolor}{0 0 0}
\pscustom[linestyle=none,fillstyle=solid,fillcolor=curcolor]
{
\newpath
\moveto(297.85021418,311.01414523)
\curveto(298.65837389,312.13827715)(299.53033567,312.70034311)(300.46609954,312.70034311)
\curveto(301.32287035,312.70034311)(302.07026616,312.33272159)(302.70828698,311.59747855)
\curveto(303.3463078,310.8683119)(303.66531821,309.86874595)(303.66531821,308.59878071)
\curveto(303.66531821,307.11614186)(303.17313072,305.92213147)(302.18875574,305.01674955)
\curveto(301.34413771,304.23897179)(300.40229745,303.85008291)(299.36323498,303.85008291)
\curveto(298.87712388,303.85008291)(298.3818982,303.93819055)(297.87755793,304.11440582)
\curveto(297.37929406,304.29062109)(296.8688774,304.554944)(296.34630797,304.90737455)
\lineto(296.34630797,313.55711392)
\curveto(296.34630797,314.50503056)(296.32200242,315.08836388)(296.27339131,315.30711387)
\curveto(296.23085659,315.52586387)(296.16097812,315.67473539)(296.0637559,315.75372845)
\curveto(295.96653368,315.8327215)(295.8450059,315.87221803)(295.69917257,315.87221803)
\curveto(295.52903369,315.87221803)(295.31636008,315.82360692)(295.06115175,315.7263847)
\lineto(294.93354759,316.04539511)
\lineto(297.44005794,317.06622841)
\lineto(297.85021418,317.06622841)
\closepath
\moveto(297.85021418,310.43081191)
\lineto(297.85021418,305.43602037)
\curveto(298.16011001,305.13220093)(298.47912042,304.90129816)(298.80724541,304.74331205)
\curveto(299.14144679,304.59140234)(299.48172456,304.51544748)(299.82807872,304.51544748)
\curveto(300.38103009,304.51544748)(300.89448494,304.81926691)(301.36844326,305.42690579)
\curveto(301.84847797,306.03454466)(302.08849533,306.91865922)(302.08849533,308.07924947)
\curveto(302.08849533,309.14869389)(301.84847797,309.96900637)(301.36844326,310.54018691)
\curveto(300.89448494,311.11744384)(300.35368634,311.4060723)(299.74604747,311.4060723)
\curveto(299.42399887,311.4060723)(299.10195026,311.32404106)(298.77990166,311.15997856)
\curveto(298.53684611,311.03845079)(298.22695029,310.79539524)(297.85021418,310.43081191)
\closepath
}
}
{
\newrgbcolor{curcolor}{0 0 0}
\pscustom[linestyle=none,fillstyle=solid,fillcolor=curcolor]
{
\newpath
\moveto(306.29943272,309.30971819)
\curveto(306.29335633,308.07013489)(306.59413758,307.09791269)(307.20177645,306.3930516)
\curveto(307.80941532,305.6881905)(308.523391,305.33575996)(309.34370348,305.33575996)
\curveto(309.89057847,305.33575996)(310.36453679,305.48463148)(310.76557844,305.78237453)
\curveto(311.17269649,306.08619397)(311.51297426,306.60268701)(311.78641175,307.33185366)
\lineto(312.06896383,307.14956199)
\curveto(311.94135966,306.31709674)(311.57069995,305.55754815)(310.95698469,304.87091622)
\curveto(310.34326943,304.19036068)(309.57460625,303.85008291)(308.65099516,303.85008291)
\curveto(307.64839102,303.85008291)(306.78858202,304.23897179)(306.07156815,305.01674955)
\curveto(305.36063066,305.80060369)(305.00516192,306.85181895)(305.00516192,308.1703953)
\curveto(305.00516192,309.59834666)(305.36974525,310.71032579)(306.09891189,311.50633272)
\curveto(306.83415493,312.30841603)(307.75472783,312.70945769)(308.86063058,312.70945769)
\curveto(309.79639444,312.70945769)(310.56505762,312.39956186)(311.1666201,311.77977021)
\curveto(311.76818259,311.16605495)(312.06896383,310.34270428)(312.06896383,309.30971819)
\closepath
\moveto(306.29943272,309.83836401)
\lineto(310.16401596,309.83836401)
\curveto(310.13363402,310.37308622)(310.06983193,310.74982232)(309.97260971,310.96857232)
\curveto(309.8207,311.30885009)(309.59283542,311.57621119)(309.28901598,311.77065563)
\curveto(308.99127293,311.96510007)(308.67833891,312.06232229)(308.35021392,312.06232229)
\curveto(307.84587366,312.06232229)(307.3931827,311.86483965)(306.99214104,311.46987439)
\curveto(306.59717577,311.08098551)(306.366273,310.53714872)(306.29943272,309.83836401)
\closepath
}
}
{
\newrgbcolor{curcolor}{0 0 0}
\pscustom[linestyle=none,fillstyle=solid,fillcolor=curcolor]
{
\newpath
\moveto(315.62365124,312.70034311)
\lineto(315.62365124,310.82273899)
\curveto(316.32243594,312.07447507)(317.03944982,312.70034311)(317.77469285,312.70034311)
\curveto(318.10889423,312.70034311)(318.38536992,312.5970445)(318.60411992,312.39044728)
\curveto(318.82286991,312.18992645)(318.93224491,311.95598549)(318.93224491,311.68862438)
\curveto(318.93224491,311.45164522)(318.85325185,311.25112439)(318.69526575,311.0870619)
\curveto(318.53727964,310.9229994)(318.34891159,310.84096815)(318.13016159,310.84096815)
\curveto(317.91748799,310.84096815)(317.67747063,310.94426676)(317.41010953,311.15086398)
\curveto(317.14882481,311.36353758)(316.95438037,311.46987439)(316.82677621,311.46987439)
\curveto(316.71740121,311.46987439)(316.59891163,311.4091105)(316.47130747,311.28758272)
\curveto(316.19786998,311.03845079)(315.9153179,310.62829455)(315.62365124,310.05711401)
\lineto(315.62365124,306.05581202)
\curveto(315.62365124,305.59400648)(315.68137693,305.24461413)(315.79682832,305.00763496)
\curveto(315.87582137,304.84357247)(316.01557831,304.70685372)(316.21609914,304.59747873)
\curveto(316.41661997,304.48810373)(316.70524843,304.43341623)(317.08198454,304.43341623)
\lineto(317.08198454,304.10529124)
\lineto(312.80724506,304.10529124)
\lineto(312.80724506,304.43341623)
\curveto(313.23259227,304.43341623)(313.54856449,304.50025651)(313.7551617,304.63393706)
\curveto(313.90707142,304.73115928)(314.01340822,304.88610719)(314.07417211,305.0987808)
\curveto(314.10455406,305.2020794)(314.11974503,305.49678426)(314.11974503,305.98289536)
\lineto(314.11974503,309.21857236)
\curveto(314.11974503,310.19079456)(314.09847767,310.76805149)(314.05594295,310.95034315)
\curveto(314.01948461,311.1387112)(313.94656795,311.27542995)(313.83719295,311.36049939)
\curveto(313.73389434,311.44556883)(313.60325198,311.48810355)(313.44526588,311.48810355)
\curveto(313.25689783,311.48810355)(313.04422422,311.44253064)(312.80724506,311.35138481)
\lineto(312.71609923,311.6795098)
\lineto(315.24083875,312.70034311)
\closepath
}
}
{
\newrgbcolor{curcolor}{0 0 0}
\pscustom[linestyle=none,fillstyle=solid,fillcolor=curcolor]
{
\newpath
\moveto(321.684849,311.01414523)
\curveto(322.49300871,312.13827715)(323.36497049,312.70034311)(324.30073436,312.70034311)
\curveto(325.15750517,312.70034311)(325.90490098,312.33272159)(326.5429218,311.59747855)
\curveto(327.18094262,310.8683119)(327.49995303,309.86874595)(327.49995303,308.59878071)
\curveto(327.49995303,307.11614186)(327.00776554,305.92213147)(326.02339056,305.01674955)
\curveto(325.17877253,304.23897179)(324.23693227,303.85008291)(323.1978698,303.85008291)
\curveto(322.7117587,303.85008291)(322.21653302,303.93819055)(321.71219275,304.11440582)
\curveto(321.21392888,304.29062109)(320.70351222,304.554944)(320.18094279,304.90737455)
\lineto(320.18094279,313.55711392)
\curveto(320.18094279,314.50503056)(320.15663724,315.08836388)(320.10802613,315.30711387)
\curveto(320.06549141,315.52586387)(319.99561294,315.67473539)(319.89839072,315.75372845)
\curveto(319.8011685,315.8327215)(319.67964072,315.87221803)(319.53380739,315.87221803)
\curveto(319.36366851,315.87221803)(319.1509949,315.82360692)(318.89578657,315.7263847)
\lineto(318.76818241,316.04539511)
\lineto(321.27469276,317.06622841)
\lineto(321.684849,317.06622841)
\closepath
\moveto(321.684849,310.43081191)
\lineto(321.684849,305.43602037)
\curveto(321.99474483,305.13220093)(322.31375524,304.90129816)(322.64188023,304.74331205)
\curveto(322.97608161,304.59140234)(323.31635938,304.51544748)(323.66271354,304.51544748)
\curveto(324.21566491,304.51544748)(324.72911976,304.81926691)(325.20307808,305.42690579)
\curveto(325.68311279,306.03454466)(325.92313015,306.91865922)(325.92313015,308.07924947)
\curveto(325.92313015,309.14869389)(325.68311279,309.96900637)(325.20307808,310.54018691)
\curveto(324.72911976,311.11744384)(324.18832116,311.4060723)(323.58068229,311.4060723)
\curveto(323.25863369,311.4060723)(322.93658508,311.32404106)(322.61453648,311.15997856)
\curveto(322.37148093,311.03845079)(322.06158511,310.79539524)(321.684849,310.43081191)
\closepath
}
}
{
\newrgbcolor{curcolor}{0 0 0}
\pscustom[linestyle=none,fillstyle=solid,fillcolor=curcolor]
{
\newpath
\moveto(331.60151542,317.06622841)
\lineto(331.60151542,305.99200994)
\curveto(331.60151542,305.46944051)(331.63797376,305.12308635)(331.71089042,304.95294747)
\curveto(331.78988347,304.78280858)(331.90837305,304.65216622)(332.06635916,304.56102039)
\curveto(332.22434527,304.47595095)(332.51905012,304.43341623)(332.95047372,304.43341623)
\lineto(332.95047372,304.10529124)
\lineto(328.85802591,304.10529124)
\lineto(328.85802591,304.43341623)
\curveto(329.2408384,304.43341623)(329.50212311,304.47291276)(329.64188006,304.55190581)
\curveto(329.781637,304.63089886)(329.89101199,304.76154122)(329.97000505,304.94383288)
\curveto(330.0489981,305.12612455)(330.08849463,305.4755169)(330.08849463,305.99200994)
\lineto(330.08849463,313.57534308)
\curveto(330.08849463,314.51718334)(330.06722727,315.09444027)(330.02469255,315.30711387)
\curveto(329.98215783,315.52586387)(329.91227935,315.67473539)(329.81505713,315.75372845)
\curveto(329.7239113,315.8327215)(329.60542172,315.87221803)(329.45958839,315.87221803)
\curveto(329.30160229,315.87221803)(329.10108146,315.82360692)(328.85802591,315.7263847)
\lineto(328.703078,316.04539511)
\lineto(331.19135918,317.06622841)
\closepath
}
}
{
\newrgbcolor{curcolor}{0 0 0}
\pscustom[linestyle=none,fillstyle=solid,fillcolor=curcolor]
{
\newpath
\moveto(337.99995276,312.70034311)
\curveto(339.26384162,312.70034311)(340.27859854,312.2203084)(341.04422352,311.26023898)
\curveto(341.69439712,310.4399265)(342.01948391,309.49808624)(342.01948391,308.43471821)
\curveto(342.01948391,307.6873224)(341.84023045,306.930812)(341.48172351,306.16518702)
\curveto(341.12321657,305.39956204)(340.62799089,304.82230511)(339.99604646,304.43341623)
\curveto(339.37017842,304.04452735)(338.67139372,303.85008291)(337.89969235,303.85008291)
\curveto(336.64187988,303.85008291)(335.64231393,304.35138498)(334.90099451,305.35398912)
\curveto(334.27512647,306.19860716)(333.96219245,307.1465238)(333.96219245,308.19773905)
\curveto(333.96219245,308.96336403)(334.1505605,309.72291262)(334.5272966,310.47638483)
\curveto(334.91010909,311.23593342)(335.41141116,311.79496118)(336.03120281,312.15346812)
\curveto(336.65099446,312.51805144)(337.30724445,312.70034311)(337.99995276,312.70034311)
\closepath
\moveto(337.71740069,312.1078952)
\curveto(337.39535208,312.1078952)(337.07026529,312.01067298)(336.74214029,311.81622854)
\curveto(336.42009169,311.62786049)(336.15880698,311.29365911)(335.95828615,310.8136244)
\curveto(335.75776532,310.33358969)(335.65750491,309.71683624)(335.65750491,308.96336403)
\curveto(335.65750491,307.74808629)(335.89752226,306.69990923)(336.37755697,305.81883286)
\curveto(336.86366807,304.93775649)(337.50168889,304.49721831)(338.29161942,304.49721831)
\curveto(338.88102913,304.49721831)(339.36714023,304.74027386)(339.74995272,305.22638496)
\curveto(340.13276521,305.71249606)(340.32417146,306.54799951)(340.32417146,307.73289531)
\curveto(340.32417146,309.21553417)(340.00516105,310.3822008)(339.36714023,311.23289523)
\curveto(338.93571663,311.81622854)(338.38580345,312.1078952)(337.71740069,312.1078952)
\closepath
}
}
{
\newrgbcolor{curcolor}{0 0 0}
\pscustom[linestyle=none,fillstyle=solid,fillcolor=curcolor]
{
\newpath
\moveto(350.34109829,307.27716616)
\curveto(350.11627191,306.1773398)(349.67573372,305.32968357)(349.01948374,304.73419747)
\curveto(348.36323375,304.14478776)(347.6371053,303.85008291)(346.84109838,303.85008291)
\curveto(345.89318173,303.85008291)(345.06679286,304.24808637)(344.36193177,305.0440933)
\curveto(343.65707068,305.84010022)(343.30464013,306.91562103)(343.30464013,308.27065572)
\curveto(343.30464013,309.58315568)(343.69352901,310.64956191)(344.47130677,311.46987439)
\curveto(345.25516092,312.29018687)(346.19396298,312.70034311)(347.28771295,312.70034311)
\curveto(348.10802543,312.70034311)(348.78250458,312.48159311)(349.3111504,312.04409312)
\curveto(349.83979622,311.61266952)(350.10411913,311.16301676)(350.10411913,310.69513482)
\curveto(350.10411913,310.46423205)(350.02816427,310.275864)(349.87625455,310.13003067)
\curveto(349.73042122,309.99027373)(349.523824,309.92039526)(349.2564629,309.92039526)
\curveto(348.89795596,309.92039526)(348.62755666,310.03584664)(348.445265,310.26674942)
\curveto(348.34196639,310.39435358)(348.27208792,310.63740913)(348.23562959,310.99591607)
\curveto(348.20524765,311.354423)(348.08371987,311.62786049)(347.87104627,311.81622854)
\curveto(347.65837266,311.99852021)(347.36366781,312.08966604)(346.98693171,312.08966604)
\curveto(346.37929283,312.08966604)(345.89014354,311.86483965)(345.51948383,311.41518689)
\curveto(345.02729634,310.81970079)(344.78120259,310.03280845)(344.78120259,309.05450986)
\curveto(344.78120259,308.05798211)(345.02425814,307.17690574)(345.51036924,306.41128076)
\curveto(346.00255673,305.65173217)(346.6648831,305.27195787)(347.49734836,305.27195787)
\curveto(348.09283446,305.27195787)(348.62755666,305.4755169)(349.10151499,305.88263494)
\curveto(349.43571637,306.16214882)(349.76080316,306.66952728)(350.07677538,307.40477032)
\closepath
}
}
{
\newrgbcolor{curcolor}{0 0 0}
\pscustom[linestyle=none,fillstyle=solid,fillcolor=curcolor]
{
\newpath
\moveto(354.0051607,317.06622841)
\lineto(354.0051607,308.7628432)
\lineto(356.12885856,310.69513482)
\curveto(356.57851133,311.10832926)(356.83979604,311.36961397)(356.91271271,311.47898897)
\curveto(356.96132382,311.55190563)(356.98562937,311.6248223)(356.98562937,311.69773896)
\curveto(356.98562937,311.81926674)(356.93398007,311.92256535)(356.83068146,312.00763479)
\curveto(356.73345924,312.09878062)(356.56939674,312.15042993)(356.33849397,312.1625827)
\lineto(356.33849397,312.45424936)
\lineto(359.96609805,312.45424936)
\lineto(359.96609805,312.1625827)
\curveto(359.46783417,312.15042993)(359.05160154,312.07447507)(358.71740016,311.93471812)
\curveto(358.38927517,311.79496118)(358.02773004,311.54582925)(357.63276477,311.18732231)
\lineto(355.49083774,309.20945778)
\lineto(357.63276477,306.50242659)
\curveto(358.22825087,305.75503078)(358.62929253,305.28107246)(358.83588974,305.08055163)
\curveto(359.1275564,304.79496136)(359.38276473,304.6096315)(359.60151472,304.52456206)
\curveto(359.75342444,304.46379817)(360.01774735,304.43341623)(360.39448345,304.43341623)
\lineto(360.39448345,304.10529124)
\lineto(356.33849397,304.10529124)
\lineto(356.33849397,304.43341623)
\curveto(356.56939674,304.43949262)(356.72434466,304.47291276)(356.80333771,304.53367664)
\curveto(356.88840715,304.60051692)(356.93094187,304.69166275)(356.93094187,304.80711414)
\curveto(356.93094187,304.94687108)(356.8094141,305.17169746)(356.56635855,305.48159329)
\lineto(354.0051607,308.75372862)
\lineto(354.0051607,305.98289536)
\curveto(354.0051607,305.44209676)(354.04161903,305.08662802)(354.11453569,304.91648913)
\curveto(354.19352875,304.74635025)(354.30290374,304.62482247)(354.44266069,304.55190581)
\curveto(354.58241763,304.47898914)(354.88623706,304.43949262)(355.354119,304.43341623)
\lineto(355.354119,304.10529124)
\lineto(351.10672327,304.10529124)
\lineto(351.10672327,304.43341623)
\curveto(351.53207048,304.43341623)(351.85108089,304.48506553)(352.0637545,304.58836414)
\curveto(352.19135866,304.65520442)(352.28858088,304.75850303)(352.35542115,304.89825997)
\curveto(352.44656699,305.0987808)(352.4921399,305.44513495)(352.4921399,305.93732244)
\lineto(352.4921399,313.53888475)
\curveto(352.4921399,314.50503056)(352.47087254,315.09444027)(352.42833782,315.30711387)
\curveto(352.3858031,315.52586387)(352.31592463,315.67473539)(352.21870241,315.75372845)
\curveto(352.12148019,315.83879789)(351.99387602,315.88133261)(351.83588992,315.88133261)
\curveto(351.70828575,315.88133261)(351.51687951,315.82968331)(351.26167118,315.7263847)
\lineto(351.10672327,316.04539511)
\lineto(353.58588987,317.06622841)
\closepath
}
}
{
\newrgbcolor{curcolor}{0 0 0}
\pscustom[linestyle=none,fillstyle=solid,fillcolor=curcolor]
{
\newpath
\moveto(291.86193308,258.42275657)
\curveto(293.12582194,258.42275657)(294.14057886,257.94272186)(294.90620384,256.98265244)
\curveto(295.55637744,256.16233996)(295.88146423,255.22049971)(295.88146423,254.15713168)
\curveto(295.88146423,253.40973586)(295.70221077,252.65322547)(295.34370383,251.88760049)
\curveto(294.98519689,251.1219755)(294.48997121,250.54471857)(293.85802678,250.1558297)
\curveto(293.23215874,249.76694082)(292.53337404,249.57249638)(291.76167267,249.57249638)
\curveto(290.5038602,249.57249638)(289.50429425,250.07379845)(288.76297483,251.07640259)
\curveto(288.13710679,251.92102062)(287.82417277,252.86893727)(287.82417277,253.92015252)
\curveto(287.82417277,254.6857775)(288.01254082,255.44532609)(288.38927692,256.19879829)
\curveto(288.77208941,256.95834689)(289.27339148,257.51737465)(289.89318313,257.87588159)
\curveto(290.51297478,258.24046491)(291.16922477,258.42275657)(291.86193308,258.42275657)
\closepath
\moveto(291.57938101,257.83030867)
\curveto(291.2573324,257.83030867)(290.93224561,257.73308645)(290.60412062,257.53864201)
\curveto(290.28207201,257.35027396)(290.0207873,257.01607258)(289.82026647,256.53603787)
\curveto(289.61974564,256.05600316)(289.51948523,255.4392497)(289.51948523,254.6857775)
\curveto(289.51948523,253.47049975)(289.75950258,252.42232269)(290.23953729,251.54124633)
\curveto(290.72564839,250.66016996)(291.36366921,250.21963178)(292.15359974,250.21963178)
\curveto(292.74300945,250.21963178)(293.22912055,250.46268733)(293.61193304,250.94879843)
\curveto(293.99474553,251.43490952)(294.18615178,252.27041298)(294.18615178,253.45530878)
\curveto(294.18615178,254.93794763)(293.86714137,256.10461427)(293.22912055,256.95530869)
\curveto(292.79769695,257.53864201)(292.24778377,257.83030867)(291.57938101,257.83030867)
\closepath
}
}
{
\newrgbcolor{curcolor}{0 0 0}
\pscustom[linestyle=none,fillstyle=solid,fillcolor=curcolor]
{
\newpath
\moveto(299.39969331,256.7365587)
\curveto(300.20785301,257.86069061)(301.0798148,258.42275657)(302.01557866,258.42275657)
\curveto(302.87234948,258.42275657)(303.61974529,258.05513505)(304.25776611,257.31989202)
\curveto(304.89578692,256.59072537)(305.21479733,255.59115942)(305.21479733,254.32119417)
\curveto(305.21479733,252.83855532)(304.72260985,251.64454494)(303.73823487,250.73916301)
\curveto(302.89361684,249.96138526)(301.95177658,249.57249638)(300.91271411,249.57249638)
\curveto(300.42660301,249.57249638)(299.93137733,249.66060401)(299.42703706,249.83681929)
\curveto(298.92877318,250.01303456)(298.41835653,250.27735747)(297.8957871,250.62978802)
\lineto(297.8957871,259.27952738)
\curveto(297.8957871,260.22744403)(297.87148154,260.81077735)(297.82287043,261.02952734)
\curveto(297.78033571,261.24827733)(297.71045724,261.39714886)(297.61323502,261.47614191)
\curveto(297.5160128,261.55513497)(297.39448503,261.59463149)(297.2486517,261.59463149)
\curveto(297.07851281,261.59463149)(296.86583921,261.54602038)(296.61063088,261.44879816)
\lineto(296.48302672,261.76780857)
\lineto(298.98953707,262.78864188)
\lineto(299.39969331,262.78864188)
\closepath
\moveto(299.39969331,256.15322538)
\lineto(299.39969331,251.15843384)
\curveto(299.70958914,250.8546144)(300.02859955,250.62371163)(300.35672454,250.46572552)
\curveto(300.69092592,250.3138158)(301.03120369,250.23786094)(301.37755785,250.23786094)
\curveto(301.93050922,250.23786094)(302.44396407,250.54168038)(302.91792239,251.14931925)
\curveto(303.3979571,251.75695813)(303.63797446,252.64107269)(303.63797446,253.80166294)
\curveto(303.63797446,254.87110736)(303.3979571,255.69141983)(302.91792239,256.26260038)
\curveto(302.44396407,256.83985731)(301.90316547,257.12848577)(301.2955266,257.12848577)
\curveto(300.97347799,257.12848577)(300.65142939,257.04645452)(300.32938079,256.88239203)
\curveto(300.08632524,256.76086425)(299.77642941,256.5178087)(299.39969331,256.15322538)
\closepath
}
}
{
\newrgbcolor{curcolor}{0 0 0}
\pscustom[linestyle=none,fillstyle=solid,fillcolor=curcolor]
{
\newpath
\moveto(308.55984933,262.79775646)
\curveto(308.82113405,262.79775646)(309.04292224,262.70661063)(309.2252139,262.52431897)
\curveto(309.40750556,262.34202731)(309.49865139,262.12023912)(309.49865139,261.8589544)
\curveto(309.49865139,261.60374608)(309.40750556,261.38499608)(309.2252139,261.20270442)
\curveto(309.04292224,261.02041276)(308.82113405,260.92926693)(308.55984933,260.92926693)
\curveto(308.30464101,260.92926693)(308.08589101,261.02041276)(307.90359935,261.20270442)
\curveto(307.72130769,261.38499608)(307.63016186,261.60374608)(307.63016186,261.8589544)
\curveto(307.63016186,262.12023912)(307.72130769,262.34202731)(307.90359935,262.52431897)
\curveto(308.08589101,262.70661063)(308.30464101,262.79775646)(308.55984933,262.79775646)
\closepath
\moveto(309.34370348,258.42275657)
\lineto(309.34370348,250.01911095)
\curveto(309.34370348,248.5911596)(309.03988404,247.53082976)(308.43224517,246.83812145)
\curveto(307.8246063,246.14541313)(307.03467576,245.79905897)(306.06245356,245.79905897)
\curveto(305.50950219,245.79905897)(305.09934595,245.89931939)(304.83198484,246.09984021)
\curveto(304.56462374,246.30036104)(304.43094319,246.50695826)(304.43094319,246.71963186)
\curveto(304.43094319,246.93230547)(304.50689805,247.11459713)(304.65880776,247.26650685)
\curveto(304.80464109,247.41841657)(304.97781817,247.49437143)(305.178339,247.49437143)
\curveto(305.33632511,247.49437143)(305.49734941,247.4548749)(305.66141191,247.37588185)
\curveto(305.76471051,247.33334713)(305.96219315,247.18143741)(306.25385981,246.92015269)
\curveto(306.55160286,246.65279159)(306.80073479,246.51911104)(307.00125562,246.51911104)
\curveto(307.14708895,246.51911104)(307.28988409,246.57683673)(307.42964103,246.69228812)
\curveto(307.56939797,246.80166311)(307.67269658,246.99003116)(307.73953685,247.25739227)
\curveto(307.80637713,247.51867698)(307.83979727,248.08985752)(307.83979727,248.97093389)
\lineto(307.83979727,254.91364208)
\curveto(307.83979727,255.83117678)(307.81245352,256.42058648)(307.75776602,256.6818712)
\curveto(307.7152313,256.88239203)(307.64839102,257.01911077)(307.55724519,257.09202744)
\curveto(307.46609936,257.17102049)(307.34153339,257.21051702)(307.18354728,257.21051702)
\curveto(307.0134084,257.21051702)(306.80681118,257.1649441)(306.56375563,257.07379827)
\lineto(306.43615147,257.40192326)
\lineto(308.95177641,258.42275657)
\closepath
}
}
{
\newrgbcolor{curcolor}{0 0 0}
\pscustom[linestyle=none,fillstyle=solid,fillcolor=curcolor]
{
\newpath
\moveto(313.03510964,255.03213166)
\curveto(313.02903325,253.79254835)(313.32981449,252.82032616)(313.93745337,252.11546506)
\curveto(314.54509224,251.41060397)(315.25906792,251.05817342)(316.07938039,251.05817342)
\curveto(316.62625538,251.05817342)(317.1002137,251.20704495)(317.50125536,251.504788)
\curveto(317.9083734,251.80860743)(318.24865117,252.32510047)(318.52208867,253.05426712)
\lineto(318.80464074,252.87197546)
\curveto(318.67703658,252.0395102)(318.30637687,251.27996161)(317.6926616,250.59332968)
\curveto(317.07894634,249.91277415)(316.31028317,249.57249638)(315.38667208,249.57249638)
\curveto(314.38406794,249.57249638)(313.52425893,249.96138526)(312.80724506,250.73916301)
\curveto(312.09630758,251.52301716)(311.74083884,252.57423241)(311.74083884,253.89280877)
\curveto(311.74083884,255.32076012)(312.10542216,256.43273926)(312.83458881,257.22874619)
\curveto(313.56983185,258.0308295)(314.49040474,258.43187116)(315.59630749,258.43187116)
\curveto(316.53207136,258.43187116)(317.30073453,258.12197533)(317.90229702,257.50218368)
\curveto(318.5038595,256.88846842)(318.80464074,256.06511774)(318.80464074,255.03213166)
\closepath
\moveto(313.03510964,255.56077748)
\lineto(316.89969287,255.56077748)
\curveto(316.86931093,256.09549969)(316.80550885,256.47223579)(316.70828663,256.69098578)
\curveto(316.55637691,257.03126355)(316.32851233,257.29862466)(316.0246929,257.4930691)
\curveto(315.72694985,257.68751353)(315.41401583,257.78473575)(315.08589084,257.78473575)
\curveto(314.58155057,257.78473575)(314.12885961,257.58725312)(313.72781795,257.19228785)
\curveto(313.33285269,256.80339897)(313.10194991,256.25956218)(313.03510964,255.56077748)
\closepath
}
}
{
\newrgbcolor{curcolor}{0 0 0}
\pscustom[linestyle=none,fillstyle=solid,fillcolor=curcolor]
{
\newpath
\moveto(327.00776554,252.99957962)
\curveto(326.78293916,251.89975326)(326.34240097,251.05209703)(325.68615099,250.45661094)
\curveto(325.029901,249.86720123)(324.30377255,249.57249638)(323.50776563,249.57249638)
\curveto(322.55984898,249.57249638)(321.73346011,249.97049984)(321.02859902,250.76650676)
\curveto(320.32373793,251.56251369)(319.97130738,252.63803449)(319.97130738,253.99306918)
\curveto(319.97130738,255.30556915)(320.36019626,256.37197537)(321.13797402,257.19228785)
\curveto(321.92182817,258.01260033)(322.86063023,258.42275657)(323.9543802,258.42275657)
\curveto(324.77469268,258.42275657)(325.44917183,258.20400658)(325.97781765,257.76650659)
\curveto(326.50646347,257.33508299)(326.77078638,256.88543022)(326.77078638,256.41754829)
\curveto(326.77078638,256.18664552)(326.69483152,255.99827747)(326.5429218,255.85244414)
\curveto(326.39708847,255.7126872)(326.19049125,255.64280872)(325.92313015,255.64280872)
\curveto(325.56462321,255.64280872)(325.29422391,255.75826011)(325.11193225,255.98916288)
\curveto(325.00863364,256.11676705)(324.93875517,256.3598226)(324.90229684,256.71832953)
\curveto(324.8719149,257.07683647)(324.75038712,257.35027396)(324.53771352,257.53864201)
\curveto(324.32503991,257.72093367)(324.03033506,257.8120795)(323.65359896,257.8120795)
\curveto(323.04596008,257.8120795)(322.55681079,257.58725312)(322.18615108,257.13760035)
\curveto(321.69396359,256.54211426)(321.44786984,255.75522192)(321.44786984,254.77692333)
\curveto(321.44786984,253.78039558)(321.69092539,252.89931921)(322.17703649,252.13369423)
\curveto(322.66922398,251.37414564)(323.33155035,250.99437134)(324.16401561,250.99437134)
\curveto(324.75950171,250.99437134)(325.29422391,251.19793036)(325.76818224,251.60504841)
\curveto(326.10238362,251.88456229)(326.42747041,252.39194075)(326.74344263,253.12718379)
\closepath
}
}
{
\newrgbcolor{curcolor}{0 0 0}
\pscustom[linestyle=none,fillstyle=solid,fillcolor=curcolor]
{
\newpath
\moveto(330.62625503,260.92015234)
\lineto(330.62625503,258.17666283)
\lineto(332.57677582,258.17666283)
\lineto(332.57677582,257.53864201)
\lineto(330.62625503,257.53864201)
\lineto(330.62625503,252.12457965)
\curveto(330.62625503,251.58378105)(330.70220989,251.21919772)(330.85411961,251.03082967)
\curveto(331.01210572,250.84246162)(331.21262654,250.7482776)(331.45568209,250.7482776)
\curveto(331.65620292,250.7482776)(331.85064736,250.80904148)(332.03901541,250.93056926)
\curveto(332.22738346,251.05817342)(332.37321679,251.24350328)(332.4765154,251.48655883)
\lineto(332.83198414,251.48655883)
\curveto(332.61931054,250.89107273)(332.31852929,250.44141997)(331.92964042,250.13760053)
\curveto(331.54075154,249.83985748)(331.13970988,249.69098596)(330.72651545,249.69098596)
\curveto(330.44700156,249.69098596)(330.17356407,249.76694082)(329.90620297,249.91885053)
\curveto(329.63884186,250.07683664)(329.44135923,250.29862483)(329.31375506,250.5842151)
\curveto(329.1861509,250.87588176)(329.12234882,251.32249633)(329.12234882,251.92405882)
\lineto(329.12234882,257.53864201)
\lineto(327.80073427,257.53864201)
\lineto(327.80073427,257.83942325)
\curveto(328.13493565,257.97310381)(328.47521342,258.19793019)(328.82156758,258.5139024)
\curveto(329.17399812,258.83595101)(329.48693214,259.2157253)(329.76036964,259.65322529)
\curveto(329.90012658,259.88412806)(330.09457102,260.30643708)(330.34370295,260.92015234)
\closepath
}
}
{
\newrgbcolor{curcolor}{0 0 0}
\pscustom[linestyle=none,fillstyle=solid,fillcolor=curcolor]
{
\newpath
\moveto(343.45047346,258.42275657)
\lineto(343.45047346,255.57900664)
\lineto(343.14969222,255.57900664)
\curveto(342.91878945,256.47223579)(342.6210464,257.07987466)(342.25646307,257.40192326)
\curveto(341.89795614,257.72397187)(341.43918879,257.88499617)(340.88016102,257.88499617)
\curveto(340.45481381,257.88499617)(340.11149785,257.77258298)(339.85021313,257.54775659)
\curveto(339.58892842,257.32293021)(339.45828606,257.07379827)(339.45828606,256.80036078)
\curveto(339.45828606,256.46008301)(339.55550828,256.16841635)(339.74995272,255.9253608)
\curveto(339.93832077,255.67622886)(340.32113326,255.41190595)(340.89839019,255.13239207)
\lineto(342.22911932,254.48525667)
\curveto(343.46262624,253.88369419)(344.07937969,253.09072546)(344.07937969,252.10635048)
\curveto(344.07937969,251.34680189)(343.79075123,250.73308663)(343.2134943,250.26520469)
\curveto(342.64231376,249.80339915)(342.00125475,249.57249638)(341.29031726,249.57249638)
\curveto(340.77990061,249.57249638)(340.19656729,249.66364221)(339.54031731,249.84593387)
\curveto(339.33979648,249.90669776)(339.17573398,249.9370797)(339.04812982,249.9370797)
\curveto(338.90837288,249.9370797)(338.79899788,249.85808665)(338.72000483,249.70010054)
\lineto(338.41922359,249.70010054)
\lineto(338.41922359,252.68056922)
\lineto(338.72000483,252.68056922)
\curveto(338.89014371,251.82987479)(339.21523051,251.18881578)(339.69526522,250.75739218)
\curveto(340.17529993,250.32596858)(340.71306033,250.11025678)(341.30854643,250.11025678)
\curveto(341.72781725,250.11025678)(342.06809502,250.23178455)(342.32937974,250.4748401)
\curveto(342.59674084,250.72397204)(342.7304214,251.02171509)(342.7304214,251.36806925)
\curveto(342.7304214,251.78734007)(342.58154987,252.13977062)(342.28380682,252.42536089)
\curveto(341.99214016,252.71095116)(341.40576865,253.07249629)(340.52469228,253.50999628)
\curveto(339.64361592,253.94749627)(339.06635899,254.34246154)(338.79292149,254.69489208)
\curveto(338.519484,255.04124624)(338.38276525,255.47874623)(338.38276525,256.00739205)
\curveto(338.38276525,256.69402398)(338.61670622,257.26824271)(339.08458815,257.73004826)
\curveto(339.55854647,258.1918538)(340.16922354,258.42275657)(340.91661936,258.42275657)
\curveto(341.24474435,258.42275657)(341.64274781,258.3528781)(342.11062974,258.21312116)
\curveto(342.42052557,258.12197533)(342.62712279,258.07640241)(342.7304214,258.07640241)
\curveto(342.82764362,258.07640241)(342.90359847,258.09766977)(342.95828597,258.1402045)
\curveto(343.01297347,258.18273922)(343.07677555,258.27692324)(343.14969222,258.42275657)
\closepath
}
}
{
\newrgbcolor{curcolor}{0 0 0}
\pscustom[linestyle=none,fillstyle=solid,fillcolor=curcolor]
{
\newpath
\moveto(346.7226088,255.03213166)
\curveto(346.71653241,253.79254835)(347.01731365,252.82032616)(347.62495252,252.11546506)
\curveto(348.2325914,251.41060397)(348.94656707,251.05817342)(349.76687955,251.05817342)
\curveto(350.31375454,251.05817342)(350.78771286,251.20704495)(351.18875452,251.504788)
\curveto(351.59587256,251.80860743)(351.93615033,252.32510047)(352.20958782,253.05426712)
\lineto(352.4921399,252.87197546)
\curveto(352.36453574,252.0395102)(351.99387602,251.27996161)(351.38016076,250.59332968)
\curveto(350.7664455,249.91277415)(349.99778232,249.57249638)(349.07417124,249.57249638)
\curveto(348.0715671,249.57249638)(347.21175809,249.96138526)(346.49474422,250.73916301)
\curveto(345.78380674,251.52301716)(345.42833799,252.57423241)(345.42833799,253.89280877)
\curveto(345.42833799,255.32076012)(345.79292132,256.43273926)(346.52208797,257.22874619)
\curveto(347.257331,258.0308295)(348.1779039,258.43187116)(349.28380665,258.43187116)
\curveto(350.21957051,258.43187116)(350.98823369,258.12197533)(351.58979617,257.50218368)
\curveto(352.19135866,256.88846842)(352.4921399,256.06511774)(352.4921399,255.03213166)
\closepath
\moveto(346.7226088,255.56077748)
\lineto(350.58719203,255.56077748)
\curveto(350.55681009,256.09549969)(350.49300801,256.47223579)(350.39578579,256.69098578)
\curveto(350.24387607,257.03126355)(350.01601149,257.29862466)(349.71219205,257.4930691)
\curveto(349.41444901,257.68751353)(349.10151499,257.78473575)(348.77338999,257.78473575)
\curveto(348.26904973,257.78473575)(347.81635877,257.58725312)(347.41531711,257.19228785)
\curveto(347.02035184,256.80339897)(346.78944907,256.25956218)(346.7226088,255.56077748)
\closepath
}
}
{
\newrgbcolor{curcolor}{0 0 0}
\pscustom[linestyle=none,fillstyle=solid,fillcolor=curcolor]
{
\newpath
\moveto(356.02859815,260.92015234)
\lineto(356.02859815,258.17666283)
\lineto(357.97911893,258.17666283)
\lineto(357.97911893,257.53864201)
\lineto(356.02859815,257.53864201)
\lineto(356.02859815,252.12457965)
\curveto(356.02859815,251.58378105)(356.10455301,251.21919772)(356.25646272,251.03082967)
\curveto(356.41444883,250.84246162)(356.61496966,250.7482776)(356.85802521,250.7482776)
\curveto(357.05854604,250.7482776)(357.25299048,250.80904148)(357.44135853,250.93056926)
\curveto(357.62972658,251.05817342)(357.77555991,251.24350328)(357.87885852,251.48655883)
\lineto(358.23432726,251.48655883)
\curveto(358.02165365,250.89107273)(357.72087241,250.44141997)(357.33198353,250.13760053)
\curveto(356.94309465,249.83985748)(356.54205299,249.69098596)(356.12885856,249.69098596)
\curveto(355.84934468,249.69098596)(355.57590719,249.76694082)(355.30854608,249.91885053)
\curveto(355.04118498,250.07683664)(354.84370234,250.29862483)(354.71609818,250.5842151)
\curveto(354.58849402,250.87588176)(354.52469193,251.32249633)(354.52469193,251.92405882)
\lineto(354.52469193,257.53864201)
\lineto(353.20307738,257.53864201)
\lineto(353.20307738,257.83942325)
\curveto(353.53727876,257.97310381)(353.87755653,258.19793019)(354.22391069,258.5139024)
\curveto(354.57634124,258.83595101)(354.88927526,259.2157253)(355.16271275,259.65322529)
\curveto(355.30246969,259.88412806)(355.49691413,260.30643708)(355.74604607,260.92015234)
\closepath
}
}
{
\newrgbcolor{curcolor}{0 0 0}
\pscustom[linestyle=none,fillstyle=solid,fillcolor=curcolor]
{
\newpath
\moveto(24.89071783,191.88190537)
\curveto(25.49835671,192.48954424)(25.85686364,192.8389366)(25.96623864,192.93008243)
\curveto(26.23967613,193.1609852)(26.53438099,193.34023867)(26.8503532,193.46784283)
\curveto(27.16632542,193.59544699)(27.47925944,193.65924908)(27.78915526,193.65924908)
\curveto(28.31172469,193.65924908)(28.76137746,193.50733936)(29.13811356,193.20351992)
\curveto(29.51484966,192.89970048)(29.7670198,192.4591623)(29.89462396,191.88190537)
\curveto(30.520492,192.61107202)(31.04913782,193.08806854)(31.48056142,193.31289492)
\curveto(31.91198502,193.54379769)(32.3555614,193.65924908)(32.81129055,193.65924908)
\curveto(33.25486693,193.65924908)(33.646794,193.54379769)(33.98707177,193.31289492)
\curveto(34.33342593,193.08806854)(34.60686342,192.71740882)(34.80738425,192.20091578)
\curveto(34.94106481,191.84848523)(35.00790508,191.29553386)(35.00790508,190.54206165)
\lineto(35.00790508,186.95091591)
\curveto(35.00790508,186.42834648)(35.04740161,186.06983954)(35.12639466,185.8753951)
\curveto(35.18715855,185.74171455)(35.29957174,185.62626317)(35.46363424,185.52904095)
\curveto(35.62769673,185.43789512)(35.89505784,185.3923222)(36.26571755,185.3923222)
\lineto(36.26571755,185.06419721)
\lineto(32.14592599,185.06419721)
\lineto(32.14592599,185.3923222)
\lineto(32.31910307,185.3923222)
\curveto(32.67761,185.3923222)(32.95712388,185.46220067)(33.15764471,185.60195761)
\curveto(33.29740165,185.69917983)(33.39766207,185.85412774)(33.45842595,186.06680135)
\curveto(33.48273151,186.17009996)(33.49488429,186.46480481)(33.49488429,186.95091591)
\lineto(33.49488429,190.54206165)
\curveto(33.49488429,191.22261719)(33.41285304,191.7026519)(33.24879054,191.98216579)
\curveto(33.01181138,192.37105466)(32.63203708,192.5654991)(32.10946765,192.5654991)
\curveto(31.78741905,192.5654991)(31.46233225,192.48346786)(31.13420726,192.31940536)
\curveto(30.81215866,192.16141925)(30.42023158,191.8636762)(29.95842604,191.42617622)
\lineto(29.94019687,191.3259158)
\lineto(29.95842604,190.93398873)
\lineto(29.95842604,186.95091591)
\curveto(29.95842604,186.37973537)(29.98880798,186.02426663)(30.04957187,185.88450969)
\curveto(30.11641215,185.74475275)(30.23793992,185.62626317)(30.4141552,185.52904095)
\curveto(30.59037047,185.43789512)(30.89115171,185.3923222)(31.31649892,185.3923222)
\lineto(31.31649892,185.06419721)
\lineto(27.09644695,185.06419721)
\lineto(27.09644695,185.3923222)
\curveto(27.55825249,185.3923222)(27.8742247,185.4470097)(28.04436359,185.5563847)
\curveto(28.22057886,185.66575969)(28.34210664,185.82982219)(28.40894691,186.04857218)
\curveto(28.43932886,186.15187079)(28.45451983,186.45265203)(28.45451983,186.95091591)
\lineto(28.45451983,190.54206165)
\curveto(28.45451983,191.22261719)(28.35425941,191.71176649)(28.15373859,192.00950953)
\curveto(27.88637748,192.39839841)(27.51267957,192.59284285)(27.03264486,192.59284285)
\curveto(26.70451987,192.59284285)(26.37943307,192.50473522)(26.05738447,192.32851994)
\curveto(25.55304421,192.06115884)(25.16415533,191.7603776)(24.89071783,191.42617622)
\lineto(24.89071783,186.95091591)
\curveto(24.89071783,186.40404092)(24.92717617,186.04857218)(25.00009283,185.88450969)
\curveto(25.07908588,185.72044719)(25.19149908,185.59588122)(25.33733241,185.51081178)
\curveto(25.48924212,185.43181873)(25.79306156,185.3923222)(26.24879072,185.3923222)
\lineto(26.24879072,185.06419721)
\lineto(22.11988457,185.06419721)
\lineto(22.11988457,185.3923222)
\curveto(22.50269706,185.3923222)(22.77005816,185.43181873)(22.92196788,185.51081178)
\curveto(23.0738776,185.59588122)(23.18932899,185.72652358)(23.26832204,185.90273885)
\curveto(23.34731509,186.08503052)(23.38681162,186.43442287)(23.38681162,186.95091591)
\lineto(23.38681162,190.14102)
\curveto(23.38681162,191.0585547)(23.35946787,191.6510026)(23.30478037,191.9183637)
\curveto(23.26224565,192.11888453)(23.19540538,192.25560328)(23.10425955,192.32851994)
\curveto(23.01311371,192.407513)(22.88854775,192.44700952)(22.73056164,192.44700952)
\curveto(22.56042275,192.44700952)(22.35686373,192.40143661)(22.11988457,192.31029078)
\lineto(21.98316582,192.63841577)
\lineto(24.49879076,193.65924908)
\lineto(24.89071783,193.65924908)
\closepath
}
}
{
\newrgbcolor{curcolor}{0 0 0}
\pscustom[linestyle=none,fillstyle=solid,fillcolor=curcolor]
{
\newpath
\moveto(41.66155075,186.26732218)
\curveto(40.80477994,185.60499581)(40.26701953,185.22218332)(40.04826954,185.11888471)
\curveto(39.72014455,184.96697499)(39.37075219,184.89102013)(39.00009248,184.89102013)
\curveto(38.42283555,184.89102013)(37.94583904,185.08850276)(37.56910293,185.48346803)
\curveto(37.19844322,185.8784333)(37.01311336,186.39796454)(37.01311336,187.04206174)
\curveto(37.01311336,187.44917979)(37.1042592,187.80161033)(37.28655086,188.09935338)
\curveto(37.5356828,188.51254782)(37.9671064,188.9014367)(38.58082166,189.26602002)
\curveto(39.20061331,189.63060334)(40.22752301,190.07417972)(41.66155075,190.59674915)
\lineto(41.66155075,190.92487414)
\curveto(41.66155075,191.7573394)(41.5278702,192.32851994)(41.26050909,192.63841577)
\curveto(40.99922438,192.94831159)(40.61641189,193.10325951)(40.11207162,193.10325951)
\curveto(39.72925913,193.10325951)(39.42543969,192.9999609)(39.20061331,192.79336368)
\curveto(38.96971054,192.58676646)(38.85425915,192.3497873)(38.85425915,192.0824262)
\lineto(38.87248832,191.55378038)
\curveto(38.87248832,191.2742665)(38.79957165,191.0585547)(38.65373832,190.90664498)
\curveto(38.51398138,190.75473526)(38.32865153,190.6787804)(38.09774875,190.6787804)
\curveto(37.87292237,190.6787804)(37.68759251,190.75777345)(37.54175918,190.91575956)
\curveto(37.40200224,191.07374567)(37.33212377,191.28945747)(37.33212377,191.56289496)
\curveto(37.33212377,192.08546439)(37.59948488,192.5654991)(38.13420709,193.00299909)
\curveto(38.6689293,193.44049908)(39.4193633,193.65924908)(40.38550911,193.65924908)
\curveto(41.12682854,193.65924908)(41.73446741,193.53468311)(42.20842573,193.28555117)
\curveto(42.56693267,193.09718312)(42.83125558,192.80247826)(43.00139446,192.40143661)
\curveto(43.11076946,192.14015189)(43.16545696,191.60542968)(43.16545696,190.79726998)
\lineto(43.16545696,187.96263464)
\curveto(43.16545696,187.16662771)(43.18064793,186.67747842)(43.21102988,186.49518676)
\curveto(43.24141182,186.31897148)(43.29002293,186.2004819)(43.35686321,186.13971801)
\curveto(43.42977987,186.07895413)(43.51181112,186.04857218)(43.60295695,186.04857218)
\curveto(43.70017917,186.04857218)(43.78524861,186.06983954)(43.85816528,186.11237427)
\curveto(43.98576944,186.19136732)(44.23186318,186.41315551)(44.59644651,186.77773883)
\lineto(44.59644651,186.26732218)
\curveto(43.91589097,185.35586387)(43.26571737,184.90013471)(42.64592572,184.90013471)
\curveto(42.34818268,184.90013471)(42.11120351,185.00343332)(41.93498824,185.21003054)
\curveto(41.75877297,185.41662775)(41.66762714,185.7690583)(41.66155075,186.26732218)
\closepath
\moveto(41.66155075,186.85977008)
\lineto(41.66155075,190.04075958)
\curveto(40.74401605,189.67617626)(40.15156815,189.41792974)(39.88420704,189.26602002)
\curveto(39.40417233,188.99865892)(39.06085637,188.71914503)(38.85425915,188.42747837)
\curveto(38.64766193,188.13581171)(38.54436333,187.81680131)(38.54436333,187.47044715)
\curveto(38.54436333,187.03294716)(38.67500568,186.66836383)(38.9362904,186.37669718)
\curveto(39.19757512,186.0911069)(39.49835636,185.94831177)(39.83863413,185.94831177)
\curveto(40.30043967,185.94831177)(40.90807854,186.25213121)(41.66155075,186.85977008)
\closepath
}
}
{
\newrgbcolor{curcolor}{0 0 0}
\pscustom[linestyle=none,fillstyle=solid,fillcolor=curcolor]
{
\newpath
\moveto(50.61207136,193.65924908)
\lineto(50.61207136,190.81549915)
\lineto(50.31129012,190.81549915)
\curveto(50.08038734,191.70872829)(49.7826443,192.31636717)(49.41806097,192.63841577)
\curveto(49.05955404,192.96046437)(48.60078669,193.12148867)(48.04175892,193.12148867)
\curveto(47.61641171,193.12148867)(47.27309575,193.00907548)(47.01181103,192.7842491)
\curveto(46.75052632,192.55942272)(46.61988396,192.31029078)(46.61988396,192.03685328)
\curveto(46.61988396,191.69657551)(46.71710618,191.40490886)(46.91155062,191.16185331)
\curveto(47.09991867,190.91272137)(47.48273116,190.64839846)(48.05998809,190.36888458)
\lineto(49.39071722,189.72174918)
\curveto(50.62422414,189.12018669)(51.24097759,188.32721796)(51.24097759,187.34284298)
\curveto(51.24097759,186.58329439)(50.95234913,185.96957913)(50.3750922,185.5016972)
\curveto(49.80391166,185.03989165)(49.16285264,184.80898888)(48.45191516,184.80898888)
\curveto(47.94149851,184.80898888)(47.35816519,184.90013471)(46.70191521,185.08242637)
\curveto(46.50139438,185.14319026)(46.33733188,185.17357221)(46.20972772,185.17357221)
\curveto(46.06997078,185.17357221)(45.96059578,185.09457915)(45.88160273,184.93659304)
\lineto(45.58082148,184.93659304)
\lineto(45.58082148,187.91706172)
\lineto(45.88160273,187.91706172)
\curveto(46.05174161,187.0663673)(46.37682841,186.42530829)(46.85686312,185.99388468)
\curveto(47.33689783,185.56246108)(47.87465823,185.34674928)(48.47014433,185.34674928)
\curveto(48.88941515,185.34674928)(49.22969292,185.46827706)(49.49097764,185.71133261)
\curveto(49.75833874,185.96046455)(49.89201929,186.25820759)(49.89201929,186.60456175)
\curveto(49.89201929,187.02383258)(49.74314777,187.37626312)(49.44540472,187.66185339)
\curveto(49.15373806,187.94744366)(48.56736655,188.30898879)(47.68629018,188.74648878)
\curveto(46.80521381,189.18398877)(46.22795688,189.57895404)(45.95451939,189.93138459)
\curveto(45.6810819,190.27773874)(45.54436315,190.71523873)(45.54436315,191.24388455)
\curveto(45.54436315,191.93051648)(45.77830412,192.50473522)(46.24618605,192.96654076)
\curveto(46.72014437,193.4283463)(47.33082144,193.65924908)(48.07821725,193.65924908)
\curveto(48.40634225,193.65924908)(48.80434571,193.58937061)(49.27222764,193.44961367)
\curveto(49.58212347,193.35846783)(49.78872068,193.31289492)(49.89201929,193.31289492)
\curveto(49.98924151,193.31289492)(50.06519637,193.33416228)(50.11988387,193.376697)
\curveto(50.17457137,193.41923172)(50.23837345,193.51341575)(50.31129012,193.65924908)
\closepath
}
}
{
\newrgbcolor{curcolor}{0 0 0}
\pscustom[linestyle=none,fillstyle=solid,fillcolor=curcolor]
{
\newpath
\moveto(54.90504,196.15664485)
\lineto(54.90504,193.41315533)
\lineto(56.85556078,193.41315533)
\lineto(56.85556078,192.77513452)
\lineto(54.90504,192.77513452)
\lineto(54.90504,187.36107215)
\curveto(54.90504,186.82027355)(54.98099486,186.45569023)(55.13290458,186.26732218)
\curveto(55.29089069,186.07895413)(55.49141151,185.9847701)(55.73446706,185.9847701)
\curveto(55.93498789,185.9847701)(56.12943233,186.04553399)(56.31780038,186.16706176)
\curveto(56.50616843,186.29466593)(56.65200176,186.47999578)(56.75530037,186.72305133)
\lineto(57.11076911,186.72305133)
\curveto(56.89809551,186.12756524)(56.59731426,185.67791247)(56.20842538,185.37409303)
\curveto(55.81953651,185.07634999)(55.41849485,184.92747846)(55.00530041,184.92747846)
\curveto(54.72578653,184.92747846)(54.45234904,185.00343332)(54.18498794,185.15534304)
\curveto(53.91762683,185.31332915)(53.7201442,185.53511734)(53.59254003,185.82070761)
\curveto(53.46493587,186.11237427)(53.40113379,186.55898884)(53.40113379,187.16055132)
\lineto(53.40113379,192.77513452)
\lineto(52.07951924,192.77513452)
\lineto(52.07951924,193.07591576)
\curveto(52.41372062,193.20959631)(52.75399839,193.43442269)(53.10035255,193.75039491)
\curveto(53.45278309,194.07244351)(53.76571711,194.45221781)(54.03915461,194.8897178)
\curveto(54.17891155,195.12062057)(54.37335599,195.54292959)(54.62248792,196.15664485)
\closepath
}
}
{
\newrgbcolor{curcolor}{0 0 0}
\pscustom[linestyle=none,fillstyle=solid,fillcolor=curcolor]
{
\newpath
\moveto(59.07040448,190.26862416)
\curveto(59.06432809,189.02904086)(59.36510933,188.05681866)(59.97274821,187.35195757)
\curveto(60.58038708,186.64709647)(61.29436276,186.29466593)(62.11467524,186.29466593)
\curveto(62.66155022,186.29466593)(63.13550854,186.44353745)(63.5365502,186.7412805)
\curveto(63.94366825,187.04509994)(64.28394602,187.56159298)(64.55738351,188.29075963)
\lineto(64.83993559,188.10846797)
\curveto(64.71233142,187.27600271)(64.34167171,186.51645412)(63.72795645,185.82982219)
\curveto(63.11424118,185.14926665)(62.34557801,184.80898888)(61.42196692,184.80898888)
\curveto(60.41936278,184.80898888)(59.55955377,185.19787776)(58.8425399,185.97565552)
\curveto(58.13160242,186.75950967)(57.77613368,187.81072492)(57.77613368,189.12930127)
\curveto(57.77613368,190.55725263)(58.140717,191.66923177)(58.86988365,192.46523869)
\curveto(59.60512669,193.267322)(60.52569958,193.66836366)(61.63160233,193.66836366)
\curveto(62.5673662,193.66836366)(63.33602937,193.35846783)(63.93759186,192.73867618)
\curveto(64.53915434,192.12496092)(64.83993559,191.30161025)(64.83993559,190.26862416)
\closepath
\moveto(59.07040448,190.79726998)
\lineto(62.93498772,190.79726998)
\curveto(62.90460577,191.33199219)(62.84080369,191.70872829)(62.74358147,191.92747829)
\curveto(62.59167175,192.26775606)(62.36380718,192.53511716)(62.05998774,192.7295616)
\curveto(61.76224469,192.92400604)(61.44931067,193.02122826)(61.12118568,193.02122826)
\curveto(60.61684541,193.02122826)(60.16415445,192.82374563)(59.7631128,192.42878036)
\curveto(59.36814753,192.03989148)(59.13724476,191.49605469)(59.07040448,190.79726998)
\closepath
}
}
{
\newrgbcolor{curcolor}{0 0 0}
\pscustom[linestyle=none,fillstyle=solid,fillcolor=curcolor]
{
\newpath
\moveto(68.394623,193.65924908)
\lineto(68.394623,191.78164496)
\curveto(69.0934077,193.03338104)(69.81042157,193.65924908)(70.54566461,193.65924908)
\curveto(70.87986599,193.65924908)(71.15634168,193.55595047)(71.37509167,193.34935325)
\curveto(71.59384167,193.14883242)(71.70321666,192.91489146)(71.70321666,192.64753035)
\curveto(71.70321666,192.41055119)(71.62422361,192.21003036)(71.4662375,192.04596787)
\curveto(71.3082514,191.88190537)(71.11988335,191.79987412)(70.90113335,191.79987412)
\curveto(70.68845974,191.79987412)(70.44844239,191.90317273)(70.18108129,192.10976995)
\curveto(69.91979657,192.32244355)(69.72535213,192.42878036)(69.59774797,192.42878036)
\curveto(69.48837297,192.42878036)(69.36988339,192.36801647)(69.24227923,192.2464887)
\curveto(68.96884173,191.99735676)(68.68628966,191.58720052)(68.394623,191.01601998)
\lineto(68.394623,187.01471799)
\curveto(68.394623,186.55291245)(68.45234869,186.2035201)(68.56780008,185.96654094)
\curveto(68.64679313,185.80247844)(68.78655007,185.66575969)(68.9870709,185.5563847)
\curveto(69.18759173,185.4470097)(69.47622019,185.3923222)(69.85295629,185.3923222)
\lineto(69.85295629,185.06419721)
\lineto(65.57821682,185.06419721)
\lineto(65.57821682,185.3923222)
\curveto(66.00356403,185.3923222)(66.31953624,185.45916248)(66.52613346,185.59284303)
\curveto(66.67804318,185.69006525)(66.78437998,185.84501316)(66.84514387,186.05768677)
\curveto(66.87552581,186.16098538)(66.89071678,186.45569023)(66.89071678,186.94180133)
\lineto(66.89071678,190.17747833)
\curveto(66.89071678,191.14970053)(66.86944942,191.72695746)(66.8269147,191.90924912)
\curveto(66.79045637,192.09761717)(66.71753971,192.23433592)(66.60816471,192.31940536)
\curveto(66.5048661,192.4044748)(66.37422374,192.44700952)(66.21623763,192.44700952)
\curveto(66.02786958,192.44700952)(65.81519598,192.40143661)(65.57821682,192.31029078)
\lineto(65.48707099,192.63841577)
\lineto(68.01181051,193.65924908)
\closepath
}
}
{
\newrgbcolor{curcolor}{0 0 0}
\pscustom[linestyle=none,fillstyle=solid,fillcolor=curcolor]
{
\newpath
\moveto(25.36937586,71.08999758)
\curveto(25.97701474,71.69763646)(26.33552167,72.04702881)(26.44489667,72.13817464)
\curveto(26.71833416,72.36907741)(27.01303902,72.54833088)(27.32901123,72.67593504)
\curveto(27.64498344,72.80353921)(27.95791746,72.86734129)(28.26781329,72.86734129)
\curveto(28.79038272,72.86734129)(29.24003549,72.71543157)(29.61677159,72.41161213)
\curveto(29.99350769,72.1077927)(30.24567782,71.66725451)(30.37328199,71.08999758)
\curveto(30.99915003,71.81916423)(31.52779585,72.29616075)(31.95921945,72.52098713)
\curveto(32.39064305,72.7518899)(32.83421943,72.86734129)(33.28994858,72.86734129)
\curveto(33.73352496,72.86734129)(34.12545203,72.7518899)(34.4657298,72.52098713)
\curveto(34.81208396,72.29616075)(35.08552145,71.92550103)(35.28604228,71.40900799)
\curveto(35.41972283,71.05657745)(35.48656311,70.50362607)(35.48656311,69.75015387)
\lineto(35.48656311,66.15900812)
\curveto(35.48656311,65.63643869)(35.52605964,65.27793176)(35.60505269,65.08348732)
\curveto(35.66581658,64.94980676)(35.77822977,64.83435538)(35.94229226,64.73713316)
\curveto(36.10635476,64.64598733)(36.37371587,64.60041441)(36.74437558,64.60041441)
\lineto(36.74437558,64.27228942)
\lineto(32.62458401,64.27228942)
\lineto(32.62458401,64.60041441)
\lineto(32.79776109,64.60041441)
\curveto(33.15626803,64.60041441)(33.43578191,64.67029288)(33.63630274,64.81004982)
\curveto(33.77605968,64.90727204)(33.87632009,65.06221996)(33.93708398,65.27489356)
\curveto(33.96138954,65.37819217)(33.97354231,65.67289702)(33.97354231,66.15900812)
\lineto(33.97354231,69.75015387)
\curveto(33.97354231,70.43070941)(33.89151107,70.91074412)(33.72744857,71.190258)
\curveto(33.49046941,71.57914688)(33.11069511,71.77359132)(32.58812568,71.77359132)
\curveto(32.26607708,71.77359132)(31.94099028,71.69156007)(31.61286529,71.52749757)
\curveto(31.29081669,71.36951147)(30.89888961,71.07176842)(30.43708407,70.63426843)
\lineto(30.4188549,70.53400801)
\lineto(30.43708407,70.14208094)
\lineto(30.43708407,66.15900812)
\curveto(30.43708407,65.58782758)(30.46746601,65.23235884)(30.5282299,65.0926019)
\curveto(30.59507018,64.95284496)(30.71659795,64.83435538)(30.89281322,64.73713316)
\curveto(31.0690285,64.64598733)(31.36980974,64.60041441)(31.79515695,64.60041441)
\lineto(31.79515695,64.27228942)
\lineto(27.57510497,64.27228942)
\lineto(27.57510497,64.60041441)
\curveto(28.03691052,64.60041441)(28.35288273,64.65510191)(28.52302162,64.76447691)
\curveto(28.69923689,64.87385191)(28.82076467,65.0379144)(28.88760494,65.2566644)
\curveto(28.91798688,65.359963)(28.93317786,65.66074425)(28.93317786,66.15900812)
\lineto(28.93317786,69.75015387)
\curveto(28.93317786,70.43070941)(28.83291744,70.9198587)(28.63239661,71.21760175)
\curveto(28.36503551,71.60649063)(27.9913376,71.80093507)(27.51130289,71.80093507)
\curveto(27.1831779,71.80093507)(26.8580911,71.71282743)(26.5360425,71.53661216)
\curveto(26.03170223,71.26925105)(25.64281336,70.96846981)(25.36937586,70.63426843)
\lineto(25.36937586,66.15900812)
\curveto(25.36937586,65.61213314)(25.40583419,65.2566644)(25.47875086,65.0926019)
\curveto(25.55774391,64.9285394)(25.6701571,64.80397344)(25.81599043,64.71890399)
\curveto(25.96790015,64.63991094)(26.27171959,64.60041441)(26.72744875,64.60041441)
\lineto(26.72744875,64.27228942)
\lineto(22.5985426,64.27228942)
\lineto(22.5985426,64.60041441)
\curveto(22.98135509,64.60041441)(23.24871619,64.63991094)(23.40062591,64.71890399)
\curveto(23.55253563,64.80397344)(23.66798702,64.93461579)(23.74698007,65.11083107)
\curveto(23.82597312,65.29312273)(23.86546965,65.64251508)(23.86546965,66.15900812)
\lineto(23.86546965,69.34911221)
\curveto(23.86546965,70.26664691)(23.8381259,70.85909481)(23.7834384,71.12645592)
\curveto(23.74090368,71.32697674)(23.6740634,71.46369549)(23.58291757,71.53661216)
\curveto(23.49177174,71.61560521)(23.36720577,71.65510174)(23.20921967,71.65510174)
\curveto(23.03908078,71.65510174)(22.83552176,71.60952882)(22.5985426,71.51838299)
\lineto(22.46182385,71.84650798)
\lineto(24.97744879,72.86734129)
\lineto(25.36937586,72.86734129)
\closepath
}
}
{
\newrgbcolor{curcolor}{0 0 0}
\pscustom[linestyle=none,fillstyle=solid,fillcolor=curcolor]
{
\newpath
\moveto(42.14020878,65.47541439)
\curveto(41.28343796,64.81308802)(40.74567756,64.43027553)(40.52692757,64.32697692)
\curveto(40.19880258,64.1750672)(39.84941022,64.09911234)(39.47875051,64.09911234)
\curveto(38.90149358,64.09911234)(38.42449706,64.29659498)(38.04776096,64.69156024)
\curveto(37.67710125,65.08652551)(37.49177139,65.60605675)(37.49177139,66.25015395)
\curveto(37.49177139,66.657272)(37.58291722,67.00970255)(37.76520889,67.30744559)
\curveto(38.01434082,67.72064003)(38.44576442,68.10952891)(39.05947969,68.47411223)
\curveto(39.67927134,68.83869556)(40.70618103,69.28227193)(42.14020878,69.80484137)
\lineto(42.14020878,70.13296636)
\curveto(42.14020878,70.96543161)(42.00652822,71.53661216)(41.73916712,71.84650798)
\curveto(41.4778824,72.15640381)(41.09506991,72.31135172)(40.59072965,72.31135172)
\curveto(40.20791716,72.31135172)(39.90409772,72.20805311)(39.67927134,72.00145589)
\curveto(39.44836857,71.79485868)(39.33291718,71.55787952)(39.33291718,71.29051841)
\lineto(39.35114635,70.76187259)
\curveto(39.35114635,70.48235871)(39.27822968,70.26664691)(39.13239635,70.11473719)
\curveto(38.99263941,69.96282747)(38.80730955,69.88687261)(38.57640678,69.88687261)
\curveto(38.3515804,69.88687261)(38.16625054,69.96586567)(38.02041721,70.12385177)
\curveto(37.88066027,70.28183788)(37.8107818,70.49754968)(37.8107818,70.77098717)
\curveto(37.8107818,71.29355661)(38.07814291,71.77359132)(38.61286511,72.21109131)
\curveto(39.14758732,72.64859129)(39.89802133,72.86734129)(40.86416714,72.86734129)
\curveto(41.60548657,72.86734129)(42.21312544,72.74277532)(42.68708376,72.49364338)
\curveto(43.0455907,72.30527533)(43.30991361,72.01057048)(43.48005249,71.60952882)
\curveto(43.58942749,71.3482441)(43.64411499,70.8135219)(43.64411499,70.00536219)
\lineto(43.64411499,67.17072685)
\curveto(43.64411499,66.37471992)(43.65930596,65.88557063)(43.6896879,65.70327897)
\curveto(43.72006985,65.52706369)(43.76868096,65.40857411)(43.83552123,65.34781023)
\curveto(43.9084379,65.28704634)(43.99046915,65.2566644)(44.08161498,65.2566644)
\curveto(44.1788372,65.2566644)(44.26390664,65.27793176)(44.3368233,65.32046648)
\curveto(44.46442747,65.39945953)(44.71052121,65.62124772)(45.07510454,65.98583104)
\lineto(45.07510454,65.47541439)
\curveto(44.394549,64.56395608)(43.7443754,64.10822692)(43.12458375,64.10822692)
\curveto(42.8268407,64.10822692)(42.58986154,64.21152553)(42.41364627,64.41812275)
\curveto(42.237431,64.62471997)(42.14628517,64.97715051)(42.14020878,65.47541439)
\closepath
\moveto(42.14020878,66.06786229)
\lineto(42.14020878,69.2488518)
\curveto(41.22267408,68.88426847)(40.63022618,68.62602195)(40.36286507,68.47411223)
\curveto(39.88283036,68.20675113)(39.5395144,67.92723725)(39.33291718,67.63557059)
\curveto(39.12631996,67.34390393)(39.02302135,67.02489352)(39.02302135,66.67853936)
\curveto(39.02302135,66.24103937)(39.15366371,65.87645605)(39.41494843,65.58478939)
\curveto(39.67623314,65.29919912)(39.97701439,65.15640398)(40.31729216,65.15640398)
\curveto(40.7790977,65.15640398)(41.38673657,65.46022342)(42.14020878,66.06786229)
\closepath
}
}
{
\newrgbcolor{curcolor}{0 0 0}
\pscustom[linestyle=none,fillstyle=solid,fillcolor=curcolor]
{
\newpath
\moveto(51.09072939,72.86734129)
\lineto(51.09072939,70.02359136)
\lineto(50.78994814,70.02359136)
\curveto(50.55904537,70.9168205)(50.26130232,71.52445938)(49.896719,71.84650798)
\curveto(49.53821206,72.16855658)(49.07944471,72.32958089)(48.52041695,72.32958089)
\curveto(48.09506974,72.32958089)(47.75175378,72.21716769)(47.49046906,71.99234131)
\curveto(47.22918434,71.76751493)(47.09854199,71.51838299)(47.09854199,71.2449455)
\curveto(47.09854199,70.90466773)(47.19576421,70.61300107)(47.39020865,70.36994552)
\curveto(47.5785767,70.12081358)(47.96138919,69.85649067)(48.53864612,69.57697679)
\lineto(49.86937525,68.92984139)
\curveto(51.10288216,68.3282789)(51.71963562,67.53531017)(51.71963562,66.5509352)
\curveto(51.71963562,65.7913866)(51.43100716,65.17767134)(50.85375023,64.70978941)
\curveto(50.28256968,64.24798387)(49.64151067,64.01708109)(48.93057319,64.01708109)
\curveto(48.42015654,64.01708109)(47.83682322,64.10822692)(47.18057323,64.29051859)
\curveto(46.98005241,64.35128247)(46.81598991,64.38166442)(46.68838575,64.38166442)
\curveto(46.54862881,64.38166442)(46.43925381,64.30267136)(46.36026075,64.14468526)
\lineto(46.05947951,64.14468526)
\lineto(46.05947951,67.12515393)
\lineto(46.36026075,67.12515393)
\curveto(46.53039964,66.27445951)(46.85548644,65.6334005)(47.33552115,65.2019769)
\curveto(47.81555586,64.7705533)(48.35331626,64.5548415)(48.94880236,64.5548415)
\curveto(49.36807318,64.5548415)(49.70835095,64.67636927)(49.96963566,64.91942482)
\curveto(50.23699677,65.16855676)(50.37067732,65.46629981)(50.37067732,65.81265397)
\curveto(50.37067732,66.23192479)(50.2218058,66.58435534)(49.92406275,66.86994561)
\curveto(49.63239609,67.15553588)(49.04602458,67.51708101)(48.16494821,67.954581)
\curveto(47.28387184,68.39208098)(46.70661491,68.78704625)(46.43317742,69.1394768)
\curveto(46.15973993,69.48583096)(46.02302118,69.92333095)(46.02302118,70.45197677)
\curveto(46.02302118,71.13860869)(46.25696215,71.71282743)(46.72484408,72.17463297)
\curveto(47.1988024,72.63643852)(47.80947947,72.86734129)(48.55687528,72.86734129)
\curveto(48.88500027,72.86734129)(49.28300374,72.79746282)(49.75088567,72.65770588)
\curveto(50.0607815,72.56656005)(50.26737871,72.52098713)(50.37067732,72.52098713)
\curveto(50.46789954,72.52098713)(50.5438544,72.54225449)(50.5985419,72.58478921)
\curveto(50.6532294,72.62732393)(50.71703148,72.72150796)(50.78994814,72.86734129)
\closepath
}
}
{
\newrgbcolor{curcolor}{0 0 0}
\pscustom[linestyle=none,fillstyle=solid,fillcolor=curcolor]
{
\newpath
\moveto(55.38369803,75.36473706)
\lineto(55.38369803,72.62124755)
\lineto(57.33421881,72.62124755)
\lineto(57.33421881,71.98322673)
\lineto(55.38369803,71.98322673)
\lineto(55.38369803,66.56916436)
\curveto(55.38369803,66.02836577)(55.45965289,65.66378244)(55.61156261,65.47541439)
\curveto(55.76954871,65.28704634)(55.97006954,65.19286231)(56.21312509,65.19286231)
\curveto(56.41364592,65.19286231)(56.60809036,65.2536262)(56.79645841,65.37515398)
\curveto(56.98482646,65.50275814)(57.13065979,65.688088)(57.2339584,65.93114355)
\lineto(57.58942714,65.93114355)
\curveto(57.37675353,65.33565745)(57.07597229,64.88600468)(56.68708341,64.58218525)
\curveto(56.29819453,64.2844422)(55.89715288,64.13557067)(55.48395844,64.13557067)
\curveto(55.20444456,64.13557067)(54.93100707,64.21152553)(54.66364596,64.36343525)
\curveto(54.39628486,64.52142136)(54.19880223,64.74320955)(54.07119806,65.02879982)
\curveto(53.9435939,65.32046648)(53.87979182,65.76708105)(53.87979182,66.36864353)
\lineto(53.87979182,71.98322673)
\lineto(52.55817727,71.98322673)
\lineto(52.55817727,72.28400797)
\curveto(52.89237865,72.41768852)(53.23265642,72.64251491)(53.57901057,72.95848712)
\curveto(53.93144112,73.28053572)(54.24437514,73.66031002)(54.51781263,74.09781001)
\curveto(54.65756957,74.32871278)(54.85201401,74.7510218)(55.10114595,75.36473706)
\closepath
}
}
{
\newrgbcolor{curcolor}{0 0 0}
\pscustom[linestyle=none,fillstyle=solid,fillcolor=curcolor]
{
\newpath
\moveto(59.54906251,69.47671637)
\curveto(59.54298612,68.23713307)(59.84376736,67.26491087)(60.45140624,66.56004978)
\curveto(61.05904511,65.85518869)(61.77302079,65.50275814)(62.59333327,65.50275814)
\curveto(63.14020825,65.50275814)(63.61416657,65.65162966)(64.01520823,65.94937271)
\curveto(64.42232628,66.25319215)(64.76260404,66.76968519)(65.03604154,67.49885184)
\lineto(65.31859361,67.31656018)
\curveto(65.19098945,66.48409492)(64.82032974,65.72454633)(64.20661447,65.0379144)
\curveto(63.59289921,64.35735886)(62.82423604,64.01708109)(61.90062495,64.01708109)
\curveto(60.89802081,64.01708109)(60.0382118,64.40596997)(59.32119793,65.18374773)
\curveto(58.61026045,65.96760188)(58.25479171,67.01881713)(58.25479171,68.33739349)
\curveto(58.25479171,69.76534484)(58.61937503,70.87732398)(59.34854168,71.6733309)
\curveto(60.08378472,72.47541422)(61.00435761,72.87645587)(62.11026036,72.87645587)
\curveto(63.04602423,72.87645587)(63.8146874,72.56656005)(64.41624989,71.9467684)
\curveto(65.01781237,71.33305313)(65.31859361,70.50970246)(65.31859361,69.47671637)
\closepath
\moveto(59.54906251,70.00536219)
\lineto(63.41364574,70.00536219)
\curveto(63.3832638,70.5400844)(63.31946172,70.9168205)(63.2222395,71.1355705)
\curveto(63.07032978,71.47584827)(62.8424652,71.74320937)(62.53864577,71.93765381)
\curveto(62.24090272,72.13209825)(61.9279687,72.22932047)(61.59984371,72.22932047)
\curveto(61.09550344,72.22932047)(60.64281248,72.03183784)(60.24177082,71.63687257)
\curveto(59.84680556,71.24798369)(59.61590278,70.7041469)(59.54906251,70.00536219)
\closepath
}
}
{
\newrgbcolor{curcolor}{0 0 0}
\pscustom[linestyle=none,fillstyle=solid,fillcolor=curcolor]
{
\newpath
\moveto(68.87328102,72.86734129)
\lineto(68.87328102,70.98973717)
\curveto(69.57206573,72.24147325)(70.2890796,72.86734129)(71.02432264,72.86734129)
\curveto(71.35852402,72.86734129)(71.63499971,72.76404268)(71.8537497,72.55744546)
\curveto(72.07249969,72.35692464)(72.18187469,72.12298367)(72.18187469,71.85562256)
\curveto(72.18187469,71.6186434)(72.10288164,71.41812258)(71.94489553,71.25406008)
\curveto(71.78690942,71.08999758)(71.59854137,71.00796634)(71.37979138,71.00796634)
\curveto(71.16711777,71.00796634)(70.92710042,71.11126494)(70.65973931,71.31786216)
\curveto(70.3984546,71.53053577)(70.20401016,71.63687257)(70.07640599,71.63687257)
\curveto(69.967031,71.63687257)(69.84854142,71.57610868)(69.72093725,71.45458091)
\curveto(69.44749976,71.20544897)(69.16494768,70.79529273)(68.87328102,70.22411219)
\lineto(68.87328102,66.22281021)
\curveto(68.87328102,65.76100466)(68.93100672,65.41161231)(69.0464581,65.17463315)
\curveto(69.12545116,65.01057065)(69.2652081,64.87385191)(69.46572893,64.76447691)
\curveto(69.66624976,64.65510191)(69.95487822,64.60041441)(70.33161432,64.60041441)
\lineto(70.33161432,64.27228942)
\lineto(66.05687485,64.27228942)
\lineto(66.05687485,64.60041441)
\curveto(66.48222206,64.60041441)(66.79819427,64.66725469)(67.00479149,64.80093524)
\curveto(67.15670121,64.89815746)(67.26303801,65.05310537)(67.3238019,65.26577898)
\curveto(67.35418384,65.36907759)(67.36937481,65.66378244)(67.36937481,66.14989354)
\lineto(67.36937481,69.38557054)
\curveto(67.36937481,70.35779274)(67.34810745,70.93504967)(67.30557273,71.11734133)
\curveto(67.2691144,71.30570938)(67.19619773,71.44242813)(67.08682274,71.52749757)
\curveto(66.98352413,71.61256701)(66.85288177,71.65510174)(66.69489566,71.65510174)
\curveto(66.50652761,71.65510174)(66.29385401,71.60952882)(66.05687485,71.51838299)
\lineto(65.96572901,71.84650798)
\lineto(68.49046853,72.86734129)
\closepath
}
}
{
\newrgbcolor{curcolor}{0 0 0}
\pscustom[linewidth=0.74666665,linecolor=curcolor]
{
\newpath
\moveto(267.63732692,120.66625741)
\lineto(378.42119082,120.66625741)
\lineto(378.42119082,148.69799004)
\lineto(267.63732692,148.69799004)
\closepath
}
}
{
\newrgbcolor{curcolor}{0 0 0}
\pscustom[linestyle=none,fillstyle=solid,fillcolor=curcolor]
{
\newpath
\moveto(291.86193308,138.80646724)
\curveto(293.12582194,138.80646724)(294.14057886,138.32643253)(294.90620384,137.36636311)
\curveto(295.55637744,136.54605063)(295.88146423,135.60421038)(295.88146423,134.54084235)
\curveto(295.88146423,133.79344654)(295.70221077,133.03693614)(295.34370383,132.27131116)
\curveto(294.98519689,131.50568618)(294.48997121,130.92842925)(293.85802678,130.53954037)
\curveto(293.23215874,130.15065149)(292.53337404,129.95620705)(291.76167267,129.95620705)
\curveto(290.5038602,129.95620705)(289.50429425,130.45750912)(288.76297483,131.46011326)
\curveto(288.13710679,132.3047313)(287.82417277,133.25264794)(287.82417277,134.30386319)
\curveto(287.82417277,135.06948817)(288.01254082,135.82903676)(288.38927692,136.58250897)
\curveto(288.77208941,137.34205756)(289.27339148,137.90108532)(289.89318313,138.25959226)
\curveto(290.51297478,138.62417558)(291.16922477,138.80646724)(291.86193308,138.80646724)
\closepath
\moveto(291.57938101,138.21401934)
\curveto(291.2573324,138.21401934)(290.93224561,138.11679712)(290.60412062,137.92235268)
\curveto(290.28207201,137.73398463)(290.0207873,137.39978325)(289.82026647,136.91974854)
\curveto(289.61974564,136.43971383)(289.51948523,135.82296037)(289.51948523,135.06948817)
\curveto(289.51948523,133.85421042)(289.75950258,132.80603337)(290.23953729,131.924957)
\curveto(290.72564839,131.04388063)(291.36366921,130.60334245)(292.15359974,130.60334245)
\curveto(292.74300945,130.60334245)(293.22912055,130.846398)(293.61193304,131.3325091)
\curveto(293.99474553,131.8186202)(294.18615178,132.65412365)(294.18615178,133.83901945)
\curveto(294.18615178,135.3216583)(293.86714137,136.48832494)(293.22912055,137.33901936)
\curveto(292.79769695,137.92235268)(292.24778377,138.21401934)(291.57938101,138.21401934)
\closepath
}
}
{
\newrgbcolor{curcolor}{0 0 0}
\pscustom[linestyle=none,fillstyle=solid,fillcolor=curcolor]
{
\newpath
\moveto(299.39969331,137.12026937)
\curveto(300.20785301,138.24440129)(301.0798148,138.80646724)(302.01557866,138.80646724)
\curveto(302.87234948,138.80646724)(303.61974529,138.43884573)(304.25776611,137.70360269)
\curveto(304.89578692,136.97443604)(305.21479733,135.97487009)(305.21479733,134.70490485)
\curveto(305.21479733,133.222266)(304.72260985,132.02825561)(303.73823487,131.12287369)
\curveto(302.89361684,130.34509593)(301.95177658,129.95620705)(300.91271411,129.95620705)
\curveto(300.42660301,129.95620705)(299.93137733,130.04431469)(299.42703706,130.22052996)
\curveto(298.92877318,130.39674523)(298.41835653,130.66106814)(297.8957871,131.01349869)
\lineto(297.8957871,139.66323806)
\curveto(297.8957871,140.6111547)(297.87148154,141.19448802)(297.82287043,141.41323801)
\curveto(297.78033571,141.63198801)(297.71045724,141.78085953)(297.61323502,141.85985259)
\curveto(297.5160128,141.93884564)(297.39448503,141.97834217)(297.2486517,141.97834217)
\curveto(297.07851281,141.97834217)(296.86583921,141.92973106)(296.61063088,141.83250884)
\lineto(296.48302672,142.15151924)
\lineto(298.98953707,143.17235255)
\lineto(299.39969331,143.17235255)
\closepath
\moveto(299.39969331,136.53693605)
\lineto(299.39969331,131.54214451)
\curveto(299.70958914,131.23832507)(300.02859955,131.0074223)(300.35672454,130.84943619)
\curveto(300.69092592,130.69752648)(301.03120369,130.62157162)(301.37755785,130.62157162)
\curveto(301.93050922,130.62157162)(302.44396407,130.92539105)(302.91792239,131.53302993)
\curveto(303.3979571,132.1406688)(303.63797446,133.02478336)(303.63797446,134.18537361)
\curveto(303.63797446,135.25481803)(303.3979571,136.07513051)(302.91792239,136.64631105)
\curveto(302.44396407,137.22356798)(301.90316547,137.51219644)(301.2955266,137.51219644)
\curveto(300.97347799,137.51219644)(300.65142939,137.4301652)(300.32938079,137.2661027)
\curveto(300.08632524,137.14457493)(299.77642941,136.90151938)(299.39969331,136.53693605)
\closepath
}
}
{
\newrgbcolor{curcolor}{0 0 0}
\pscustom[linestyle=none,fillstyle=solid,fillcolor=curcolor]
{
\newpath
\moveto(308.55984933,143.18146714)
\curveto(308.82113405,143.18146714)(309.04292224,143.0903213)(309.2252139,142.90802964)
\curveto(309.40750556,142.72573798)(309.49865139,142.50394979)(309.49865139,142.24266508)
\curveto(309.49865139,141.98745675)(309.40750556,141.76870675)(309.2252139,141.58641509)
\curveto(309.04292224,141.40412343)(308.82113405,141.3129776)(308.55984933,141.3129776)
\curveto(308.30464101,141.3129776)(308.08589101,141.40412343)(307.90359935,141.58641509)
\curveto(307.72130769,141.76870675)(307.63016186,141.98745675)(307.63016186,142.24266508)
\curveto(307.63016186,142.50394979)(307.72130769,142.72573798)(307.90359935,142.90802964)
\curveto(308.08589101,143.0903213)(308.30464101,143.18146714)(308.55984933,143.18146714)
\closepath
\moveto(309.34370348,138.80646724)
\lineto(309.34370348,130.40282162)
\curveto(309.34370348,128.97487027)(309.03988404,127.91454043)(308.43224517,127.22183212)
\curveto(307.8246063,126.5291238)(307.03467576,126.18276964)(306.06245356,126.18276964)
\curveto(305.50950219,126.18276964)(305.09934595,126.28303006)(304.83198484,126.48355089)
\curveto(304.56462374,126.68407171)(304.43094319,126.89066893)(304.43094319,127.10334254)
\curveto(304.43094319,127.31601614)(304.50689805,127.49830781)(304.65880776,127.65021752)
\curveto(304.80464109,127.80212724)(304.97781817,127.8780821)(305.178339,127.8780821)
\curveto(305.33632511,127.8780821)(305.49734941,127.83858557)(305.66141191,127.75959252)
\curveto(305.76471051,127.7170578)(305.96219315,127.56514808)(306.25385981,127.30386337)
\curveto(306.55160286,127.03650226)(306.80073479,126.90282171)(307.00125562,126.90282171)
\curveto(307.14708895,126.90282171)(307.28988409,126.9605474)(307.42964103,127.07599879)
\curveto(307.56939797,127.18537379)(307.67269658,127.37374184)(307.73953685,127.64110294)
\curveto(307.80637713,127.90238766)(307.83979727,128.4735682)(307.83979727,129.35464456)
\lineto(307.83979727,135.29735275)
\curveto(307.83979727,136.21488745)(307.81245352,136.80429716)(307.75776602,137.06558187)
\curveto(307.7152313,137.2661027)(307.64839102,137.40282145)(307.55724519,137.47573811)
\curveto(307.46609936,137.55473116)(307.34153339,137.59422769)(307.18354728,137.59422769)
\curveto(307.0134084,137.59422769)(306.80681118,137.54865478)(306.56375563,137.45750895)
\lineto(306.43615147,137.78563394)
\lineto(308.95177641,138.80646724)
\closepath
}
}
{
\newrgbcolor{curcolor}{0 0 0}
\pscustom[linestyle=none,fillstyle=solid,fillcolor=curcolor]
{
\newpath
\moveto(313.03510964,135.41584233)
\curveto(313.02903325,134.17625903)(313.32981449,133.20403683)(313.93745337,132.49917574)
\curveto(314.54509224,131.79431464)(315.25906792,131.4418841)(316.07938039,131.4418841)
\curveto(316.62625538,131.4418841)(317.1002137,131.59075562)(317.50125536,131.88849867)
\curveto(317.9083734,132.1923181)(318.24865117,132.70881115)(318.52208867,133.4379778)
\lineto(318.80464074,133.25568613)
\curveto(318.67703658,132.42322088)(318.30637687,131.66367228)(317.6926616,130.97704036)
\curveto(317.07894634,130.29648482)(316.31028317,129.95620705)(315.38667208,129.95620705)
\curveto(314.38406794,129.95620705)(313.52425893,130.34509593)(312.80724506,131.12287369)
\curveto(312.09630758,131.90672783)(311.74083884,132.95794309)(311.74083884,134.27651944)
\curveto(311.74083884,135.70447079)(312.10542216,136.81644993)(312.83458881,137.61245686)
\curveto(313.56983185,138.41454017)(314.49040474,138.81558183)(315.59630749,138.81558183)
\curveto(316.53207136,138.81558183)(317.30073453,138.505686)(317.90229702,137.88589435)
\curveto(318.5038595,137.27217909)(318.80464074,136.44882841)(318.80464074,135.41584233)
\closepath
\moveto(313.03510964,135.94448815)
\lineto(316.89969287,135.94448815)
\curveto(316.86931093,136.47921036)(316.80550885,136.85594646)(316.70828663,137.07469645)
\curveto(316.55637691,137.41497422)(316.32851233,137.68233533)(316.0246929,137.87677977)
\curveto(315.72694985,138.07122421)(315.41401583,138.16844643)(315.08589084,138.16844643)
\curveto(314.58155057,138.16844643)(314.12885961,137.97096379)(313.72781795,137.57599853)
\curveto(313.33285269,137.18710965)(313.10194991,136.64327285)(313.03510964,135.94448815)
\closepath
}
}
{
\newrgbcolor{curcolor}{0 0 0}
\pscustom[linestyle=none,fillstyle=solid,fillcolor=curcolor]
{
\newpath
\moveto(327.00776554,133.3832903)
\curveto(326.78293916,132.28346394)(326.34240097,131.43580771)(325.68615099,130.84032161)
\curveto(325.029901,130.2509119)(324.30377255,129.95620705)(323.50776563,129.95620705)
\curveto(322.55984898,129.95620705)(321.73346011,130.35421051)(321.02859902,131.15021744)
\curveto(320.32373793,131.94622436)(319.97130738,133.02174517)(319.97130738,134.37677986)
\curveto(319.97130738,135.68927982)(320.36019626,136.75568605)(321.13797402,137.57599853)
\curveto(321.92182817,138.39631101)(322.86063023,138.80646724)(323.9543802,138.80646724)
\curveto(324.77469268,138.80646724)(325.44917183,138.58771725)(325.97781765,138.15021726)
\curveto(326.50646347,137.71879366)(326.77078638,137.26914089)(326.77078638,136.80125896)
\curveto(326.77078638,136.57035619)(326.69483152,136.38198814)(326.5429218,136.23615481)
\curveto(326.39708847,136.09639787)(326.19049125,136.0265194)(325.92313015,136.0265194)
\curveto(325.56462321,136.0265194)(325.29422391,136.14197078)(325.11193225,136.37287356)
\curveto(325.00863364,136.50047772)(324.93875517,136.74353327)(324.90229684,137.1020402)
\curveto(324.8719149,137.46054714)(324.75038712,137.73398463)(324.53771352,137.92235268)
\curveto(324.32503991,138.10464435)(324.03033506,138.19579018)(323.65359896,138.19579018)
\curveto(323.04596008,138.19579018)(322.55681079,137.97096379)(322.18615108,137.52131103)
\curveto(321.69396359,136.92582493)(321.44786984,136.13893259)(321.44786984,135.160634)
\curveto(321.44786984,134.16410625)(321.69092539,133.28302988)(322.17703649,132.5174049)
\curveto(322.66922398,131.75785631)(323.33155035,131.37808201)(324.16401561,131.37808201)
\curveto(324.75950171,131.37808201)(325.29422391,131.58164104)(325.76818224,131.98875908)
\curveto(326.10238362,132.26827296)(326.42747041,132.77565142)(326.74344263,133.51089446)
\closepath
}
}
{
\newrgbcolor{curcolor}{0 0 0}
\pscustom[linestyle=none,fillstyle=solid,fillcolor=curcolor]
{
\newpath
\moveto(330.62625503,141.30386302)
\lineto(330.62625503,138.5603735)
\lineto(332.57677582,138.5603735)
\lineto(332.57677582,137.92235268)
\lineto(330.62625503,137.92235268)
\lineto(330.62625503,132.50829032)
\curveto(330.62625503,131.96749172)(330.70220989,131.6029084)(330.85411961,131.41454035)
\curveto(331.01210572,131.2261723)(331.21262654,131.13198827)(331.45568209,131.13198827)
\curveto(331.65620292,131.13198827)(331.85064736,131.19275216)(332.03901541,131.31427993)
\curveto(332.22738346,131.4418841)(332.37321679,131.62721395)(332.4765154,131.8702695)
\lineto(332.83198414,131.8702695)
\curveto(332.61931054,131.27478341)(332.31852929,130.82513064)(331.92964042,130.5213112)
\curveto(331.54075154,130.22356815)(331.13970988,130.07469663)(330.72651545,130.07469663)
\curveto(330.44700156,130.07469663)(330.17356407,130.15065149)(329.90620297,130.30256121)
\curveto(329.63884186,130.46054731)(329.44135923,130.6823355)(329.31375506,130.96792577)
\curveto(329.1861509,131.25959243)(329.12234882,131.70620701)(329.12234882,132.30776949)
\lineto(329.12234882,137.92235268)
\lineto(327.80073427,137.92235268)
\lineto(327.80073427,138.22313393)
\curveto(328.13493565,138.35681448)(328.47521342,138.58164086)(328.82156758,138.89761308)
\curveto(329.17399812,139.21966168)(329.48693214,139.59943597)(329.76036964,140.03693596)
\curveto(329.90012658,140.26783874)(330.09457102,140.69014775)(330.34370295,141.30386302)
\closepath
}
}
{
\newrgbcolor{curcolor}{0 0 0}
\pscustom[linestyle=none,fillstyle=solid,fillcolor=curcolor]
{
\newpath
\moveto(343.45047346,138.80646724)
\lineto(343.45047346,135.96271732)
\lineto(343.14969222,135.96271732)
\curveto(342.91878945,136.85594646)(342.6210464,137.46358533)(342.25646307,137.78563394)
\curveto(341.89795614,138.10768254)(341.43918879,138.26870684)(340.88016102,138.26870684)
\curveto(340.45481381,138.26870684)(340.11149785,138.15629365)(339.85021313,137.93146727)
\curveto(339.58892842,137.70664088)(339.45828606,137.45750895)(339.45828606,137.18407145)
\curveto(339.45828606,136.84379368)(339.55550828,136.55212702)(339.74995272,136.30907147)
\curveto(339.93832077,136.05993954)(340.32113326,135.79561663)(340.89839019,135.51610274)
\lineto(342.22911932,134.86896734)
\curveto(343.46262624,134.26740486)(344.07937969,133.47443613)(344.07937969,132.49006115)
\curveto(344.07937969,131.73051256)(343.79075123,131.1167973)(343.2134943,130.64891537)
\curveto(342.64231376,130.18710982)(342.00125475,129.95620705)(341.29031726,129.95620705)
\curveto(340.77990061,129.95620705)(340.19656729,130.04735288)(339.54031731,130.22964454)
\curveto(339.33979648,130.29040843)(339.17573398,130.32079037)(339.04812982,130.32079037)
\curveto(338.90837288,130.32079037)(338.79899788,130.24179732)(338.72000483,130.08381121)
\lineto(338.41922359,130.08381121)
\lineto(338.41922359,133.06427989)
\lineto(338.72000483,133.06427989)
\curveto(338.89014371,132.21358547)(339.21523051,131.57252645)(339.69526522,131.14110285)
\curveto(340.17529993,130.70967925)(340.71306033,130.49396745)(341.30854643,130.49396745)
\curveto(341.72781725,130.49396745)(342.06809502,130.61549523)(342.32937974,130.85855078)
\curveto(342.59674084,131.10768272)(342.7304214,131.40542576)(342.7304214,131.75177992)
\curveto(342.7304214,132.17105074)(342.58154987,132.52348129)(342.28380682,132.80907156)
\curveto(341.99214016,133.09466183)(341.40576865,133.45620696)(340.52469228,133.89370695)
\curveto(339.64361592,134.33120694)(339.06635899,134.72617221)(338.79292149,135.07860275)
\curveto(338.519484,135.42495691)(338.38276525,135.8624569)(338.38276525,136.39110272)
\curveto(338.38276525,137.07773465)(338.61670622,137.65195338)(339.08458815,138.11375893)
\curveto(339.55854647,138.57556447)(340.16922354,138.80646724)(340.91661936,138.80646724)
\curveto(341.24474435,138.80646724)(341.64274781,138.73658877)(342.11062974,138.59683183)
\curveto(342.42052557,138.505686)(342.62712279,138.46011309)(342.7304214,138.46011309)
\curveto(342.82764362,138.46011309)(342.90359847,138.48138045)(342.95828597,138.52391517)
\curveto(343.01297347,138.56644989)(343.07677555,138.66063392)(343.14969222,138.80646724)
\closepath
}
}
{
\newrgbcolor{curcolor}{0 0 0}
\pscustom[linestyle=none,fillstyle=solid,fillcolor=curcolor]
{
\newpath
\moveto(346.7226088,135.41584233)
\curveto(346.71653241,134.17625903)(347.01731365,133.20403683)(347.62495252,132.49917574)
\curveto(348.2325914,131.79431464)(348.94656707,131.4418841)(349.76687955,131.4418841)
\curveto(350.31375454,131.4418841)(350.78771286,131.59075562)(351.18875452,131.88849867)
\curveto(351.59587256,132.1923181)(351.93615033,132.70881115)(352.20958782,133.4379778)
\lineto(352.4921399,133.25568613)
\curveto(352.36453574,132.42322088)(351.99387602,131.66367228)(351.38016076,130.97704036)
\curveto(350.7664455,130.29648482)(349.99778232,129.95620705)(349.07417124,129.95620705)
\curveto(348.0715671,129.95620705)(347.21175809,130.34509593)(346.49474422,131.12287369)
\curveto(345.78380674,131.90672783)(345.42833799,132.95794309)(345.42833799,134.27651944)
\curveto(345.42833799,135.70447079)(345.79292132,136.81644993)(346.52208797,137.61245686)
\curveto(347.257331,138.41454017)(348.1779039,138.81558183)(349.28380665,138.81558183)
\curveto(350.21957051,138.81558183)(350.98823369,138.505686)(351.58979617,137.88589435)
\curveto(352.19135866,137.27217909)(352.4921399,136.44882841)(352.4921399,135.41584233)
\closepath
\moveto(346.7226088,135.94448815)
\lineto(350.58719203,135.94448815)
\curveto(350.55681009,136.47921036)(350.49300801,136.85594646)(350.39578579,137.07469645)
\curveto(350.24387607,137.41497422)(350.01601149,137.68233533)(349.71219205,137.87677977)
\curveto(349.41444901,138.07122421)(349.10151499,138.16844643)(348.77338999,138.16844643)
\curveto(348.26904973,138.16844643)(347.81635877,137.97096379)(347.41531711,137.57599853)
\curveto(347.02035184,137.18710965)(346.78944907,136.64327285)(346.7226088,135.94448815)
\closepath
}
}
{
\newrgbcolor{curcolor}{0 0 0}
\pscustom[linestyle=none,fillstyle=solid,fillcolor=curcolor]
{
\newpath
\moveto(356.02859815,141.30386302)
\lineto(356.02859815,138.5603735)
\lineto(357.97911893,138.5603735)
\lineto(357.97911893,137.92235268)
\lineto(356.02859815,137.92235268)
\lineto(356.02859815,132.50829032)
\curveto(356.02859815,131.96749172)(356.10455301,131.6029084)(356.25646272,131.41454035)
\curveto(356.41444883,131.2261723)(356.61496966,131.13198827)(356.85802521,131.13198827)
\curveto(357.05854604,131.13198827)(357.25299048,131.19275216)(357.44135853,131.31427993)
\curveto(357.62972658,131.4418841)(357.77555991,131.62721395)(357.87885852,131.8702695)
\lineto(358.23432726,131.8702695)
\curveto(358.02165365,131.27478341)(357.72087241,130.82513064)(357.33198353,130.5213112)
\curveto(356.94309465,130.22356815)(356.54205299,130.07469663)(356.12885856,130.07469663)
\curveto(355.84934468,130.07469663)(355.57590719,130.15065149)(355.30854608,130.30256121)
\curveto(355.04118498,130.46054731)(354.84370234,130.6823355)(354.71609818,130.96792577)
\curveto(354.58849402,131.25959243)(354.52469193,131.70620701)(354.52469193,132.30776949)
\lineto(354.52469193,137.92235268)
\lineto(353.20307738,137.92235268)
\lineto(353.20307738,138.22313393)
\curveto(353.53727876,138.35681448)(353.87755653,138.58164086)(354.22391069,138.89761308)
\curveto(354.57634124,139.21966168)(354.88927526,139.59943597)(355.16271275,140.03693596)
\curveto(355.30246969,140.26783874)(355.49691413,140.69014775)(355.74604607,141.30386302)
\closepath
}
}
{
\newrgbcolor{curcolor}{0 0 0}
\pscustom[linestyle=none,fillstyle=solid,fillcolor=curcolor]
{
\newpath
\moveto(568.5710733,196.77187921)
\lineto(568.5710733,193.92812928)
\lineto(568.27029206,193.92812928)
\curveto(568.03938929,194.82135842)(567.74164624,195.4289973)(567.37706291,195.7510459)
\curveto(567.01855598,196.0730945)(566.55978863,196.2341188)(566.00076087,196.2341188)
\curveto(565.57541365,196.2341188)(565.23209769,196.12170561)(564.97081297,195.89687923)
\curveto(564.70952826,195.67205285)(564.5788859,195.42292091)(564.5788859,195.14948341)
\curveto(564.5788859,194.80920565)(564.67610812,194.51753899)(564.87055256,194.27448344)
\curveto(565.05892061,194.0253515)(565.4417331,193.76102859)(566.01899003,193.48151471)
\lineto(567.34971917,192.83437931)
\curveto(568.58322608,192.23281682)(569.19997954,191.43984809)(569.19997954,190.45547312)
\curveto(569.19997954,189.69592452)(568.91135107,189.08220926)(568.33409414,188.61432733)
\curveto(567.7629136,188.15252178)(567.12185459,187.92161901)(566.41091711,187.92161901)
\curveto(565.90050045,187.92161901)(565.31716713,188.01276484)(564.66091715,188.1950565)
\curveto(564.46039632,188.25582039)(564.29633382,188.28620234)(564.16872966,188.28620234)
\curveto(564.02897272,188.28620234)(563.91959772,188.20720928)(563.84060467,188.04922318)
\lineto(563.53982343,188.04922318)
\lineto(563.53982343,191.02969185)
\lineto(563.84060467,191.02969185)
\curveto(564.01074355,190.17899743)(564.33583035,189.53793842)(564.81586506,189.10651482)
\curveto(565.29589977,188.67509122)(565.83366018,188.45937941)(566.42914627,188.45937941)
\curveto(566.84841709,188.45937941)(567.18869486,188.58090719)(567.44997958,188.82396274)
\curveto(567.71734068,189.07309468)(567.85102124,189.37083773)(567.85102124,189.71719188)
\curveto(567.85102124,190.13646271)(567.70214971,190.48889325)(567.40440666,190.77448352)
\curveto(567.11274,191.06007379)(566.52636849,191.42161892)(565.64529212,191.85911891)
\curveto(564.76421576,192.2966189)(564.18695883,192.69158417)(563.91352133,193.04401472)
\curveto(563.64008384,193.39036887)(563.50336509,193.82786886)(563.50336509,194.35651468)
\curveto(563.50336509,195.04314661)(563.73730606,195.61736535)(564.20518799,196.07917089)
\curveto(564.67914632,196.54097644)(565.28982338,196.77187921)(566.0372192,196.77187921)
\curveto(566.36534419,196.77187921)(566.76334765,196.70200074)(567.23122958,196.5622438)
\curveto(567.54112541,196.47109796)(567.74772263,196.42552505)(567.85102124,196.42552505)
\curveto(567.94824346,196.42552505)(568.02419831,196.44679241)(568.07888581,196.48932713)
\curveto(568.13357331,196.53186185)(568.19737539,196.62604588)(568.27029206,196.77187921)
\closepath
}
}
{
\newrgbcolor{curcolor}{0 0 0}
\pscustom[linestyle=none,fillstyle=solid,fillcolor=curcolor]
{
\newpath
\moveto(569.83800035,195.6963584)
\lineto(572.40831279,196.73542087)
\lineto(572.75466695,196.73542087)
\lineto(572.75466695,194.78490009)
\curveto(573.18609055,195.52014313)(573.61751415,196.03359798)(574.04893775,196.32526463)
\curveto(574.48643774,196.62300768)(574.94520509,196.77187921)(575.4252398,196.77187921)
\curveto(576.26378144,196.77187921)(576.96256615,196.44375422)(577.52159391,195.78750423)
\curveto(578.20822584,194.98542092)(578.5515418,193.94028206)(578.5515418,192.65208764)
\curveto(578.5515418,191.21198351)(578.13834737,190.02101132)(577.3119585,189.07917107)
\curveto(576.63140296,188.3074697)(575.77463215,187.92161901)(574.74164606,187.92161901)
\curveto(574.2919933,187.92161901)(573.90310442,187.98542109)(573.57497943,188.11302526)
\curveto(573.33192388,188.20417109)(573.05848638,188.38646275)(572.75466695,188.65990024)
\lineto(572.75466695,186.11693156)
\curveto(572.75466695,185.54575102)(572.78808708,185.18420589)(572.85492736,185.03229617)
\curveto(572.92784403,184.87431006)(573.0493718,184.74974409)(573.21951069,184.65859826)
\curveto(573.39572596,184.56745243)(573.71169817,184.52187951)(574.16742733,184.52187951)
\lineto(574.16742733,184.18463994)
\lineto(569.79242744,184.18463994)
\lineto(569.79242744,184.52187951)
\lineto(570.02029202,184.52187951)
\curveto(570.3544934,184.51580312)(570.64008367,184.57960521)(570.87706283,184.71328576)
\curveto(570.99251421,184.78012603)(571.08062185,184.88950103)(571.14138574,185.04141075)
\curveto(571.20822601,185.18724408)(571.24164615,185.56398018)(571.24164615,186.17161906)
\lineto(571.24164615,194.06484802)
\curveto(571.24164615,194.60564662)(571.2173406,194.94896259)(571.16872949,195.09479592)
\curveto(571.12011838,195.24062925)(571.04112532,195.35000424)(570.93175033,195.42292091)
\curveto(570.82845172,195.49583757)(570.68565658,195.5322959)(570.50336492,195.5322959)
\curveto(570.35753159,195.5322959)(570.17220173,195.48976118)(569.94737535,195.40469174)
\closepath
\moveto(572.75466695,194.24713969)
\lineto(572.75466695,191.12995226)
\curveto(572.75466695,190.45547312)(572.7820107,190.01189674)(572.83669819,189.79922313)
\curveto(572.92176764,189.44679258)(573.12836485,189.13689676)(573.45648985,188.86953565)
\curveto(573.79069123,188.60217455)(574.20996205,188.468494)(574.71430231,188.468494)
\curveto(575.32194119,188.468494)(575.81412868,188.70547316)(576.19086478,189.17943148)
\curveto(576.68305227,189.79922313)(576.92914601,190.67118492)(576.92914601,191.79531683)
\curveto(576.92914601,193.07135847)(576.64963213,194.05269525)(576.09060436,194.73932717)
\curveto(575.70171548,195.2132855)(575.23990994,195.45026466)(574.70518773,195.45026466)
\curveto(574.41352107,195.45026466)(574.12489261,195.37734799)(573.83930234,195.23151466)
\curveto(573.62055234,195.12213967)(573.25900721,194.79401467)(572.75466695,194.24713969)
\closepath
}
}
{
\newrgbcolor{curcolor}{0 0 0}
\pscustom[linestyle=none,fillstyle=solid,fillcolor=curcolor]
{
\newpath
\moveto(584.50336457,189.37995231)
\curveto(583.64659376,188.71762594)(583.10883335,188.33481345)(582.89008336,188.23151484)
\curveto(582.56195837,188.07960512)(582.21256602,188.00365026)(581.8419063,188.00365026)
\curveto(581.26464937,188.00365026)(580.78765286,188.20113289)(580.41091676,188.59609816)
\curveto(580.04025704,188.99106343)(579.85492719,189.51059467)(579.85492719,190.15469187)
\curveto(579.85492719,190.56180992)(579.94607302,190.91424046)(580.12836468,191.21198351)
\curveto(580.37749662,191.62517795)(580.80892022,192.01406683)(581.42263548,192.37865015)
\curveto(582.04242713,192.74323347)(583.06933683,193.18680985)(584.50336457,193.70937928)
\lineto(584.50336457,194.03750428)
\curveto(584.50336457,194.86996953)(584.36968402,195.44115007)(584.10232291,195.7510459)
\curveto(583.8410382,196.06094172)(583.45822571,196.21588964)(582.95388544,196.21588964)
\curveto(582.57107295,196.21588964)(582.26725351,196.11259103)(582.04242713,195.90599381)
\curveto(581.81152436,195.6993966)(581.69607297,195.46241743)(581.69607297,195.19505633)
\lineto(581.71430214,194.66641051)
\curveto(581.71430214,194.38689663)(581.64138547,194.17118483)(581.49555214,194.01927511)
\curveto(581.3557952,193.86736539)(581.17046535,193.79141053)(580.93956258,193.79141053)
\curveto(580.71473619,193.79141053)(580.52940634,193.87040359)(580.38357301,194.02838969)
\curveto(580.24381607,194.1863758)(580.17393759,194.4020876)(580.17393759,194.67552509)
\curveto(580.17393759,195.19809452)(580.4412987,195.67812923)(580.97602091,196.11562922)
\curveto(581.51074312,196.55312921)(582.26117713,196.77187921)(583.22732293,196.77187921)
\curveto(583.96864236,196.77187921)(584.57628123,196.64731324)(585.05023956,196.3981813)
\curveto(585.40874649,196.20981325)(585.6730694,195.9151084)(585.84320829,195.51406674)
\curveto(585.95258328,195.25278202)(586.00727078,194.71805981)(586.00727078,193.90990011)
\lineto(586.00727078,191.07526477)
\curveto(586.00727078,190.27925784)(586.02246175,189.79010855)(586.0528437,189.60781689)
\curveto(586.08322564,189.43160161)(586.13183675,189.31311203)(586.19867703,189.25234815)
\curveto(586.27159369,189.19158426)(586.35362494,189.16120231)(586.44477077,189.16120231)
\curveto(586.54199299,189.16120231)(586.62706243,189.18246967)(586.6999791,189.2250044)
\curveto(586.82758326,189.30399745)(587.07367701,189.52578564)(587.43826033,189.89036896)
\lineto(587.43826033,189.37995231)
\curveto(586.75770479,188.468494)(586.1075312,188.01276484)(585.48773955,188.01276484)
\curveto(585.1899965,188.01276484)(584.95301734,188.11606345)(584.77680206,188.32266067)
\curveto(584.60058679,188.52925789)(584.50944096,188.88168843)(584.50336457,189.37995231)
\closepath
\moveto(584.50336457,189.97240021)
\lineto(584.50336457,193.15338971)
\curveto(583.58582987,192.78880639)(582.99338197,192.53055987)(582.72602086,192.37865015)
\curveto(582.24598615,192.11128905)(581.90267019,191.83177516)(581.69607297,191.5401085)
\curveto(581.48947576,191.24844185)(581.38617715,190.92943144)(581.38617715,190.58307728)
\curveto(581.38617715,190.14557729)(581.51681951,189.78099397)(581.77810422,189.48932731)
\curveto(582.03938894,189.20373704)(582.34017018,189.0609419)(582.68044795,189.0609419)
\curveto(583.14225349,189.0609419)(583.74989237,189.36476134)(584.50336457,189.97240021)
\closepath
}
}
{
\newrgbcolor{curcolor}{0 0 0}
\pscustom[linestyle=none,fillstyle=solid,fillcolor=curcolor]
{
\newpath
\moveto(595.14919764,191.34870226)
\curveto(594.92437125,190.2488759)(594.48383307,189.40121967)(593.82758309,188.80573357)
\curveto(593.1713331,188.21632387)(592.44520465,187.92161901)(591.64919772,187.92161901)
\curveto(590.70128108,187.92161901)(589.87489221,188.31962247)(589.17003112,189.1156294)
\curveto(588.46517003,189.91163632)(588.11273948,190.98715713)(588.11273948,192.34219182)
\curveto(588.11273948,193.65469179)(588.50162836,194.72109801)(589.27940612,195.54141049)
\curveto(590.06326026,196.36172297)(591.00206232,196.77187921)(592.0958123,196.77187921)
\curveto(592.91612478,196.77187921)(593.59060393,196.55312921)(594.11924975,196.11562922)
\curveto(594.64789557,195.68420562)(594.91221848,195.23455286)(594.91221848,194.76667092)
\curveto(594.91221848,194.53576815)(594.83626362,194.3474001)(594.6843539,194.20156677)
\curveto(594.53852057,194.06180983)(594.33192335,193.99193136)(594.06456225,193.99193136)
\curveto(593.70605531,193.99193136)(593.43565601,194.10738275)(593.25336435,194.33828552)
\curveto(593.15006574,194.46588968)(593.08018727,194.70894523)(593.04372894,195.06745217)
\curveto(593.013347,195.4259591)(592.89181922,195.6993966)(592.67914562,195.88776465)
\curveto(592.46647201,196.07005631)(592.17176716,196.16120214)(591.79503105,196.16120214)
\curveto(591.18739218,196.16120214)(590.69824289,195.93637576)(590.32758317,195.48672299)
\curveto(589.83539569,194.89123689)(589.58930194,194.10434455)(589.58930194,193.12604596)
\curveto(589.58930194,192.12951821)(589.83235749,191.24844185)(590.31846859,190.48281686)
\curveto(590.81065608,189.72326827)(591.47298245,189.34349398)(592.30544771,189.34349398)
\curveto(592.9009338,189.34349398)(593.43565601,189.547053)(593.90961433,189.95417104)
\curveto(594.24381572,190.23368493)(594.56890251,190.74106339)(594.88487473,191.47630642)
\closepath
}
}
{
\newrgbcolor{curcolor}{0 0 0}
\pscustom[linestyle=none,fillstyle=solid,fillcolor=curcolor]
{
\newpath
\moveto(597.74685382,193.38125429)
\curveto(597.74077743,192.14167099)(598.04155868,191.16944879)(598.64919755,190.4645877)
\curveto(599.25683642,189.7597266)(599.9708121,189.40729606)(600.79112458,189.40729606)
\curveto(601.33799957,189.40729606)(601.81195789,189.55616758)(602.21299954,189.85391063)
\curveto(602.62011759,190.15773007)(602.96039536,190.67422311)(603.23383285,191.40338976)
\lineto(603.51638493,191.2210981)
\curveto(603.38878076,190.38863284)(603.01812105,189.62908425)(602.40440579,188.94245232)
\curveto(601.79069053,188.26189678)(601.02202735,187.92161901)(600.09841626,187.92161901)
\curveto(599.09581212,187.92161901)(598.23600312,188.31050789)(597.51898924,189.08828565)
\curveto(596.80805176,189.8721398)(596.45258302,190.92335505)(596.45258302,192.2419314)
\curveto(596.45258302,193.66988276)(596.81716635,194.7818619)(597.54633299,195.57786882)
\curveto(598.28157603,196.37995213)(599.20214892,196.78099379)(600.30805167,196.78099379)
\curveto(601.24381554,196.78099379)(602.01247872,196.47109796)(602.6140412,195.85130631)
\curveto(603.21560369,195.23759105)(603.51638493,194.41424038)(603.51638493,193.38125429)
\closepath
\moveto(597.74685382,193.90990011)
\lineto(601.61143706,193.90990011)
\curveto(601.58105511,194.44462232)(601.51725303,194.82135842)(601.42003081,195.04010842)
\curveto(601.26812109,195.38038619)(601.04025652,195.64774729)(600.73643708,195.84219173)
\curveto(600.43869403,196.03663617)(600.12576001,196.13385839)(599.79763502,196.13385839)
\curveto(599.29329476,196.13385839)(598.84060379,195.93637576)(598.43956214,195.54141049)
\curveto(598.04459687,195.15252161)(597.8136941,194.60868482)(597.74685382,193.90990011)
\closepath
}
}
{
\newrgbcolor{curcolor}{0 0 0}
\pscustom[linestyle=none,fillstyle=solid,fillcolor=curcolor]
{
\newpath
\moveto(611.77419722,194.9945355)
\curveto(612.38183609,195.60217438)(612.74034303,195.95156673)(612.84971803,196.04271256)
\curveto(613.12315552,196.27361533)(613.41786037,196.4528688)(613.73383259,196.58047296)
\curveto(614.0498048,196.70807713)(614.36273882,196.77187921)(614.67263465,196.77187921)
\curveto(615.19520408,196.77187921)(615.64485685,196.61996949)(616.02159295,196.31615005)
\curveto(616.39832905,196.01233061)(616.65049918,195.57179243)(616.77810335,194.9945355)
\curveto(617.40397139,195.72370215)(617.93261721,196.20069867)(618.36404081,196.42552505)
\curveto(618.79546441,196.65642782)(619.23904078,196.77187921)(619.69476994,196.77187921)
\curveto(620.13834632,196.77187921)(620.53027339,196.65642782)(620.87055116,196.42552505)
\curveto(621.21690532,196.20069867)(621.49034281,195.83003895)(621.69086364,195.31354591)
\curveto(621.82454419,194.96111536)(621.89138447,194.40816399)(621.89138447,193.65469179)
\lineto(621.89138447,190.06354604)
\curveto(621.89138447,189.54097661)(621.930881,189.18246967)(622.00987405,188.98802524)
\curveto(622.07063794,188.85434468)(622.18305113,188.7388933)(622.34711362,188.64167108)
\curveto(622.51117612,188.55052525)(622.77853722,188.50495233)(623.14919694,188.50495233)
\lineto(623.14919694,188.17682734)
\lineto(619.02940537,188.17682734)
\lineto(619.02940537,188.50495233)
\lineto(619.20258245,188.50495233)
\curveto(619.56108939,188.50495233)(619.84060327,188.5748308)(620.0411241,188.71458774)
\curveto(620.18088104,188.81180996)(620.28114145,188.96675787)(620.34190534,189.17943148)
\curveto(620.3662109,189.28273009)(620.37836367,189.57743494)(620.37836367,190.06354604)
\lineto(620.37836367,193.65469179)
\curveto(620.37836367,194.33524732)(620.29633242,194.81528203)(620.13226993,195.09479592)
\curveto(619.89529077,195.48368479)(619.51551647,195.67812923)(618.99294704,195.67812923)
\curveto(618.67089844,195.67812923)(618.34581164,195.59609799)(618.01768665,195.43203549)
\curveto(617.69563805,195.27404938)(617.30371097,194.97630634)(616.84190543,194.53880635)
\lineto(616.82367626,194.43854593)
\lineto(616.84190543,194.04661886)
\lineto(616.84190543,190.06354604)
\curveto(616.84190543,189.4923655)(616.87228737,189.13689676)(616.93305126,188.99713982)
\curveto(616.99989154,188.85738288)(617.12141931,188.7388933)(617.29763458,188.64167108)
\curveto(617.47384986,188.55052525)(617.7746311,188.50495233)(618.19997831,188.50495233)
\lineto(618.19997831,188.17682734)
\lineto(613.97992633,188.17682734)
\lineto(613.97992633,188.50495233)
\curveto(614.44173188,188.50495233)(614.75770409,188.55963983)(614.92784298,188.66901483)
\curveto(615.10405825,188.77838982)(615.22558602,188.94245232)(615.2924263,189.16120231)
\curveto(615.32280824,189.26450092)(615.33799922,189.56528217)(615.33799922,190.06354604)
\lineto(615.33799922,193.65469179)
\curveto(615.33799922,194.33524732)(615.2377388,194.82439662)(615.03721797,195.12213967)
\curveto(614.76985687,195.51102854)(614.39615896,195.70547298)(613.91612425,195.70547298)
\curveto(613.58799926,195.70547298)(613.26291246,195.61736535)(612.94086386,195.44115007)
\curveto(612.43652359,195.17378897)(612.04763471,194.87300773)(611.77419722,194.53880635)
\lineto(611.77419722,190.06354604)
\curveto(611.77419722,189.51667106)(611.81065555,189.16120231)(611.88357222,188.99713982)
\curveto(611.96256527,188.83307732)(612.07497846,188.70851135)(612.22081179,188.62344191)
\curveto(612.37272151,188.54444886)(612.67654095,188.50495233)(613.1322701,188.50495233)
\lineto(613.1322701,188.17682734)
\lineto(609.00336396,188.17682734)
\lineto(609.00336396,188.50495233)
\curveto(609.38617645,188.50495233)(609.65353755,188.54444886)(609.80544727,188.62344191)
\curveto(609.95735699,188.70851135)(610.07280837,188.83915371)(610.15180143,189.01536898)
\curveto(610.23079448,189.19766065)(610.27029101,189.547053)(610.27029101,190.06354604)
\lineto(610.27029101,193.25365013)
\curveto(610.27029101,194.17118483)(610.24294726,194.76363273)(610.18825976,195.03099383)
\curveto(610.14572504,195.23151466)(610.07888476,195.36823341)(609.98773893,195.44115007)
\curveto(609.8965931,195.52014313)(609.77202713,195.55963965)(609.61404103,195.55963965)
\curveto(609.44390214,195.55963965)(609.24034312,195.51406674)(609.00336396,195.42292091)
\lineto(608.86664521,195.7510459)
\lineto(611.38227015,196.77187921)
\lineto(611.77419722,196.77187921)
\closepath
}
}
{
\newrgbcolor{curcolor}{0 0 0}
\pscustom[linestyle=none,fillstyle=solid,fillcolor=curcolor]
{
\newpath
\moveto(628.54503014,189.37995231)
\curveto(627.68825932,188.71762594)(627.15049892,188.33481345)(626.93174893,188.23151484)
\curveto(626.60362393,188.07960512)(626.25423158,188.00365026)(625.88357187,188.00365026)
\curveto(625.30631494,188.00365026)(624.82931842,188.20113289)(624.45258232,188.59609816)
\curveto(624.08192261,188.99106343)(623.89659275,189.51059467)(623.89659275,190.15469187)
\curveto(623.89659275,190.56180992)(623.98773858,190.91424046)(624.17003024,191.21198351)
\curveto(624.41916218,191.62517795)(624.85058578,192.01406683)(625.46430105,192.37865015)
\curveto(626.0840927,192.74323347)(627.11100239,193.18680985)(628.54503014,193.70937928)
\lineto(628.54503014,194.03750428)
\curveto(628.54503014,194.86996953)(628.41134958,195.44115007)(628.14398848,195.7510459)
\curveto(627.88270376,196.06094172)(627.49989127,196.21588964)(626.99555101,196.21588964)
\curveto(626.61273852,196.21588964)(626.30891908,196.11259103)(626.0840927,195.90599381)
\curveto(625.85318992,195.6993966)(625.73773854,195.46241743)(625.73773854,195.19505633)
\lineto(625.75596771,194.66641051)
\curveto(625.75596771,194.38689663)(625.68305104,194.17118483)(625.53721771,194.01927511)
\curveto(625.39746077,193.86736539)(625.21213091,193.79141053)(624.98122814,193.79141053)
\curveto(624.75640176,193.79141053)(624.5710719,193.87040359)(624.42523857,194.02838969)
\curveto(624.28548163,194.1863758)(624.21560316,194.4020876)(624.21560316,194.67552509)
\curveto(624.21560316,195.19809452)(624.48296426,195.67812923)(625.01768647,196.11562922)
\curveto(625.55240868,196.55312921)(626.30284269,196.77187921)(627.2689885,196.77187921)
\curveto(628.01030793,196.77187921)(628.6179468,196.64731324)(629.09190512,196.3981813)
\curveto(629.45041206,196.20981325)(629.71473497,195.9151084)(629.88487385,195.51406674)
\curveto(629.99424885,195.25278202)(630.04893635,194.71805981)(630.04893635,193.90990011)
\lineto(630.04893635,191.07526477)
\curveto(630.04893635,190.27925784)(630.06412732,189.79010855)(630.09450926,189.60781689)
\curveto(630.12489121,189.43160161)(630.17350232,189.31311203)(630.24034259,189.25234815)
\curveto(630.31325926,189.19158426)(630.39529051,189.16120231)(630.48643634,189.16120231)
\curveto(630.58365856,189.16120231)(630.668728,189.18246967)(630.74164466,189.2250044)
\curveto(630.86924883,189.30399745)(631.11534257,189.52578564)(631.4799259,189.89036896)
\lineto(631.4799259,189.37995231)
\curveto(630.79937036,188.468494)(630.14919676,188.01276484)(629.52940511,188.01276484)
\curveto(629.23166206,188.01276484)(628.9946829,188.11606345)(628.81846763,188.32266067)
\curveto(628.64225236,188.52925789)(628.55110652,188.88168843)(628.54503014,189.37995231)
\closepath
\moveto(628.54503014,189.97240021)
\lineto(628.54503014,193.15338971)
\curveto(627.62749544,192.78880639)(627.03504753,192.53055987)(626.76768643,192.37865015)
\curveto(626.28765172,192.11128905)(625.94433576,191.83177516)(625.73773854,191.5401085)
\curveto(625.53114132,191.24844185)(625.42784271,190.92943144)(625.42784271,190.58307728)
\curveto(625.42784271,190.14557729)(625.55848507,189.78099397)(625.81976979,189.48932731)
\curveto(626.0810545,189.20373704)(626.38183574,189.0609419)(626.72211351,189.0609419)
\curveto(627.18391906,189.0609419)(627.79155793,189.36476134)(628.54503014,189.97240021)
\closepath
}
}
{
\newrgbcolor{curcolor}{0 0 0}
\pscustom[linestyle=none,fillstyle=solid,fillcolor=curcolor]
{
\newpath
\moveto(631.49815506,195.6963584)
\lineto(634.0684675,196.73542087)
\lineto(634.41482166,196.73542087)
\lineto(634.41482166,194.78490009)
\curveto(634.84624526,195.52014313)(635.27766886,196.03359798)(635.70909246,196.32526463)
\curveto(636.14659245,196.62300768)(636.60535979,196.77187921)(637.08539451,196.77187921)
\curveto(637.92393615,196.77187921)(638.62272086,196.44375422)(639.18174862,195.78750423)
\curveto(639.86838055,194.98542092)(640.21169651,193.94028206)(640.21169651,192.65208764)
\curveto(640.21169651,191.21198351)(639.79850208,190.02101132)(638.97211321,189.07917107)
\curveto(638.29155767,188.3074697)(637.43478686,187.92161901)(636.40180077,187.92161901)
\curveto(635.95214801,187.92161901)(635.56325913,187.98542109)(635.23513413,188.11302526)
\curveto(634.99207859,188.20417109)(634.71864109,188.38646275)(634.41482166,188.65990024)
\lineto(634.41482166,186.11693156)
\curveto(634.41482166,185.54575102)(634.44824179,185.18420589)(634.51508207,185.03229617)
\curveto(634.58799873,184.87431006)(634.70952651,184.74974409)(634.87966539,184.65859826)
\curveto(635.05588067,184.56745243)(635.37185288,184.52187951)(635.82758204,184.52187951)
\lineto(635.82758204,184.18463994)
\lineto(631.45258215,184.18463994)
\lineto(631.45258215,184.52187951)
\lineto(631.68044672,184.52187951)
\curveto(632.0146481,184.51580312)(632.30023837,184.57960521)(632.53721754,184.71328576)
\curveto(632.65266892,184.78012603)(632.74077656,184.88950103)(632.80154045,185.04141075)
\curveto(632.86838072,185.18724408)(632.90180086,185.56398018)(632.90180086,186.17161906)
\lineto(632.90180086,194.06484802)
\curveto(632.90180086,194.60564662)(632.8774953,194.94896259)(632.82888419,195.09479592)
\curveto(632.78027308,195.24062925)(632.70128003,195.35000424)(632.59190503,195.42292091)
\curveto(632.48860643,195.49583757)(632.34581129,195.5322959)(632.16351963,195.5322959)
\curveto(632.0176863,195.5322959)(631.83235644,195.48976118)(631.60753006,195.40469174)
\closepath
\moveto(634.41482166,194.24713969)
\lineto(634.41482166,191.12995226)
\curveto(634.41482166,190.45547312)(634.4421654,190.01189674)(634.4968529,189.79922313)
\curveto(634.58192235,189.44679258)(634.78851956,189.13689676)(635.11664455,188.86953565)
\curveto(635.45084593,188.60217455)(635.87011676,188.468494)(636.37445702,188.468494)
\curveto(636.9820959,188.468494)(637.47428338,188.70547316)(637.85101949,189.17943148)
\curveto(638.34320697,189.79922313)(638.58930072,190.67118492)(638.58930072,191.79531683)
\curveto(638.58930072,193.07135847)(638.30978684,194.05269525)(637.75075907,194.73932717)
\curveto(637.36187019,195.2132855)(636.90006465,195.45026466)(636.36534244,195.45026466)
\curveto(636.07367578,195.45026466)(635.78504732,195.37734799)(635.49945704,195.23151466)
\curveto(635.28070705,195.12213967)(634.91916192,194.79401467)(634.41482166,194.24713969)
\closepath
}
}
{
\newrgbcolor{curcolor}{0 0 0}
\pscustom[linestyle=none,fillstyle=solid,fillcolor=curcolor]
{
\newpath
\moveto(387.34935745,16.38528704)
\lineto(387.34935745,11.32669341)
\curveto(387.34935745,10.3605476)(387.37062481,9.7680997)(387.41315954,9.54934971)
\curveto(387.46177065,9.3366761)(387.53468731,9.18780458)(387.63190953,9.10273514)
\curveto(387.73520814,9.01766569)(387.85369772,8.97513097)(387.98737827,8.97513097)
\curveto(388.17574632,8.97513097)(388.38841993,9.02678028)(388.62539909,9.13007889)
\lineto(388.75300325,8.81106848)
\lineto(386.25560748,7.78112059)
\lineto(385.84545124,7.78112059)
\lineto(385.84545124,9.54934971)
\curveto(385.12843737,8.77157195)(384.58156238,8.28242266)(384.20482628,8.08190183)
\curveto(383.82809018,7.881381)(383.43008672,7.78112059)(383.0108159,7.78112059)
\curveto(382.54293396,7.78112059)(382.13581592,7.91480114)(381.78946176,8.18216224)
\curveto(381.44918399,8.45559974)(381.21220483,8.80499209)(381.07852428,9.2303393)
\curveto(380.94484373,9.65568651)(380.87800345,10.257249)(380.87800345,11.03502675)
\lineto(380.87800345,14.76289124)
\curveto(380.87800345,15.15785651)(380.83546873,15.43129401)(380.75039929,15.58320372)
\curveto(380.66532984,15.73511344)(380.53772568,15.85056483)(380.3675868,15.92955788)
\curveto(380.2035243,16.01462732)(379.90274306,16.05412385)(379.46524307,16.04804746)
\lineto(379.46524307,16.38528704)
\lineto(382.39102424,16.38528704)
\lineto(382.39102424,10.79804759)
\curveto(382.39102424,10.02026984)(382.5247048,9.50985318)(382.7920659,9.26679763)
\curveto(383.06550339,9.02374208)(383.39362839,8.90221431)(383.77644088,8.90221431)
\curveto(384.03772559,8.90221431)(384.33243045,8.98424556)(384.66055544,9.14830805)
\curveto(384.99475682,9.31237055)(385.38972209,9.62530457)(385.84545124,10.08711011)
\lineto(385.84545124,14.81757874)
\curveto(385.84545124,15.29153706)(385.75734361,15.61054747)(385.58112833,15.77460997)
\curveto(385.41098945,15.94474885)(385.05248251,16.03589468)(384.50560753,16.04804746)
\lineto(384.50560753,16.38528704)
\closepath
}
}
{
\newrgbcolor{curcolor}{0 0 0}
\pscustom[linestyle=none,fillstyle=solid,fillcolor=curcolor]
{
\newpath
\moveto(394.75951352,16.63138078)
\lineto(394.75951352,13.78763085)
\lineto(394.45873228,13.78763085)
\curveto(394.2278295,14.68086)(393.93008646,15.28849887)(393.56550313,15.61054747)
\curveto(393.2069962,15.93259608)(392.74822885,16.09362038)(392.18920108,16.09362038)
\curveto(391.76385387,16.09362038)(391.42053791,15.98120719)(391.15925319,15.7563808)
\curveto(390.89796848,15.53155442)(390.76732612,15.28242248)(390.76732612,15.00898499)
\curveto(390.76732612,14.66870722)(390.86454834,14.37704056)(391.05899278,14.13398501)
\curveto(391.24736083,13.88485307)(391.63017332,13.62053016)(392.20743025,13.34101628)
\lineto(393.53815938,12.69388088)
\curveto(394.7716663,12.09231839)(395.38841975,11.29934966)(395.38841975,10.31497469)
\curveto(395.38841975,9.5554261)(395.09979129,8.94171083)(394.52253436,8.4738289)
\curveto(393.95135382,8.01202336)(393.31029481,7.78112059)(392.59935732,7.78112059)
\curveto(392.08894067,7.78112059)(391.50560735,7.87226642)(390.84935737,8.05455808)
\curveto(390.64883654,8.11532197)(390.48477404,8.14570391)(390.35716988,8.14570391)
\curveto(390.21741294,8.14570391)(390.10803794,8.06671086)(390.02904489,7.90872475)
\lineto(389.72826364,7.90872475)
\lineto(389.72826364,10.88919342)
\lineto(390.02904489,10.88919342)
\curveto(390.19918377,10.038499)(390.52427057,9.39743999)(391.00430528,8.96601639)
\curveto(391.48433999,8.53459279)(392.02210039,8.31888099)(392.61758649,8.31888099)
\curveto(393.03685731,8.31888099)(393.37713508,8.44040876)(393.6384198,8.68346431)
\curveto(393.9057809,8.93259625)(394.03946145,9.2303393)(394.03946145,9.57669346)
\curveto(394.03946145,9.99596428)(393.89058993,10.34839483)(393.59284688,10.6339851)
\curveto(393.30118022,10.91957537)(392.71480871,11.2811205)(391.83373234,11.71862049)
\curveto(390.95265598,12.15612048)(390.37539905,12.55108574)(390.10196155,12.90351629)
\curveto(389.82852406,13.24987045)(389.69180531,13.68737044)(389.69180531,14.21601626)
\curveto(389.69180531,14.90264819)(389.92574628,15.47686692)(390.39362821,15.93867246)
\curveto(390.86758653,16.40047801)(391.4782636,16.63138078)(392.22565942,16.63138078)
\curveto(392.55378441,16.63138078)(392.95178787,16.56150231)(393.4196698,16.42174537)
\curveto(393.72956563,16.33059954)(393.93616285,16.28502662)(394.03946145,16.28502662)
\curveto(394.13668367,16.28502662)(394.21263853,16.30629398)(394.26732603,16.3488287)
\curveto(394.32201353,16.39136343)(394.38581561,16.48554745)(394.45873228,16.63138078)
\closepath
}
}
{
\newrgbcolor{curcolor}{0 0 0}
\pscustom[linestyle=none,fillstyle=solid,fillcolor=curcolor]
{
\newpath
\moveto(398.03164885,13.24075587)
\curveto(398.02557247,12.00117256)(398.32635371,11.02895037)(398.93399258,10.32408927)
\curveto(399.54163145,9.61922818)(400.25560713,9.26679763)(401.07591961,9.26679763)
\curveto(401.6227946,9.26679763)(402.09675292,9.41566916)(402.49779458,9.7134122)
\curveto(402.90491262,10.01723164)(403.24519039,10.53372468)(403.51862788,11.26289133)
\lineto(403.80117996,11.08059967)
\curveto(403.6735758,10.24813441)(403.30291608,9.48858582)(402.68920082,8.80195389)
\curveto(402.07548556,8.12139835)(401.30682238,7.78112059)(400.3832113,7.78112059)
\curveto(399.38060715,7.78112059)(398.52079815,8.17000946)(397.80378428,8.94778722)
\curveto(397.09284679,9.73164137)(396.73737805,10.78285662)(396.73737805,12.10143298)
\curveto(396.73737805,13.52938433)(397.10196138,14.64136347)(397.83112803,15.43737039)
\curveto(398.56637106,16.23945371)(399.48694396,16.64049536)(400.59284671,16.64049536)
\curveto(401.52861057,16.64049536)(402.29727375,16.33059954)(402.89883623,15.71080789)
\curveto(403.50039872,15.09709262)(403.80117996,14.27374195)(403.80117996,13.24075587)
\closepath
\moveto(398.03164885,13.76940169)
\lineto(401.89623209,13.76940169)
\curveto(401.86585015,14.30412389)(401.80204807,14.68086)(401.70482585,14.89960999)
\curveto(401.55291613,15.23988776)(401.32505155,15.50724886)(401.02123211,15.7016933)
\curveto(400.72348906,15.89613774)(400.41055504,15.99335996)(400.08243005,15.99335996)
\curveto(399.57808979,15.99335996)(399.12539883,15.79587733)(398.72435717,15.40091206)
\curveto(398.3293919,15.01202318)(398.09848913,14.46818639)(398.03164885,13.76940169)
\closepath
}
}
{
\newrgbcolor{curcolor}{0 0 0}
\pscustom[linestyle=none,fillstyle=solid,fillcolor=curcolor]
{
\newpath
\moveto(407.35586737,16.63138078)
\lineto(407.35586737,14.75377666)
\curveto(408.05465208,16.00551274)(408.77166595,16.63138078)(409.50690898,16.63138078)
\curveto(409.84111036,16.63138078)(410.11758605,16.52808217)(410.33633605,16.32148496)
\curveto(410.55508604,16.12096413)(410.66446104,15.88702316)(410.66446104,15.61966206)
\curveto(410.66446104,15.3826829)(410.58546798,15.18216207)(410.42748188,15.01809957)
\curveto(410.26949577,14.85403708)(410.08112772,14.77200583)(409.86237772,14.77200583)
\curveto(409.64970412,14.77200583)(409.40968676,14.87530444)(409.14232566,15.08190165)
\curveto(408.88104094,15.29457526)(408.6865965,15.40091206)(408.55899234,15.40091206)
\curveto(408.44961734,15.40091206)(408.33112776,15.34014817)(408.2035236,15.2186204)
\curveto(407.93008611,14.96948846)(407.64753403,14.55933222)(407.35586737,13.98815168)
\lineto(407.35586737,9.9868497)
\curveto(407.35586737,9.52504415)(407.41359306,9.1756518)(407.52904445,8.93867264)
\curveto(407.6080375,8.77461014)(407.74779444,8.6378914)(407.94831527,8.5285164)
\curveto(408.1488361,8.4191414)(408.43746457,8.3644539)(408.81420067,8.3644539)
\lineto(408.81420067,8.03632891)
\lineto(404.53946119,8.03632891)
\lineto(404.53946119,8.3644539)
\curveto(404.9648084,8.3644539)(405.28078062,8.43129418)(405.48737783,8.56497473)
\curveto(405.63928755,8.66219695)(405.74562436,8.81714487)(405.80638824,9.02981847)
\curveto(405.83677019,9.13311708)(405.85196116,9.42782193)(405.85196116,9.91393303)
\lineto(405.85196116,13.14961003)
\curveto(405.85196116,14.12183223)(405.8306938,14.69908916)(405.78815908,14.88138082)
\curveto(405.75170074,15.06974888)(405.67878408,15.20646762)(405.56940908,15.29153706)
\curveto(405.46611047,15.37660651)(405.33546812,15.41914123)(405.17748201,15.41914123)
\curveto(404.98911396,15.41914123)(404.77644035,15.37356831)(404.53946119,15.28242248)
\lineto(404.44831536,15.61054747)
\lineto(406.97305488,16.63138078)
\closepath
}
}
{
\newrgbcolor{curcolor}{0 0 0}
\pscustom[linestyle=none,fillstyle=solid,fillcolor=curcolor]
{
\newpath
\moveto(421.6931066,8.97513097)
\curveto(421.28598855,8.54978376)(420.88798509,8.24292613)(420.49909621,8.05455808)
\curveto(420.11020733,7.87226642)(419.69093651,7.78112059)(419.24128374,7.78112059)
\curveto(418.32982543,7.78112059)(417.53381851,8.16089488)(416.85326297,8.92044347)
\curveto(416.17270743,9.68606845)(415.83242966,10.66740524)(415.83242966,11.86445382)
\curveto(415.83242966,13.0615024)(416.20916576,14.15525237)(416.96263796,15.14570373)
\curveto(417.71611017,16.14223149)(418.68529417,16.64049536)(419.87018997,16.64049536)
\curveto(420.60543301,16.64049536)(421.21307189,16.4065544)(421.6931066,15.93867246)
\lineto(421.6931066,17.47903701)
\curveto(421.6931066,18.43303004)(421.66880104,19.01940155)(421.62018993,19.23815155)
\curveto(421.57765521,19.45690154)(421.50777674,19.60577307)(421.41055452,19.68476612)
\curveto(421.3133323,19.76375917)(421.19180452,19.8032557)(421.0459712,19.8032557)
\curveto(420.88798509,19.8032557)(420.67834968,19.75464459)(420.41706496,19.65742237)
\lineto(420.29857538,19.97643278)
\lineto(422.78685657,20.99726609)
\lineto(423.19701281,20.99726609)
\lineto(423.19701281,11.34492258)
\curveto(423.19701281,10.36662399)(423.21828017,9.7680997)(423.26081489,9.54934971)
\curveto(423.309426,9.3366761)(423.38234266,9.18780458)(423.47956488,9.10273514)
\curveto(423.58286349,9.01766569)(423.70135307,8.97513097)(423.83503363,8.97513097)
\curveto(423.99909612,8.97513097)(424.21784612,9.02678028)(424.49128361,9.13007889)
\lineto(424.59154402,8.81106848)
\lineto(422.11237742,7.78112059)
\lineto(421.6931066,7.78112059)
\closepath
\moveto(421.6931066,9.61315179)
\lineto(421.6931066,13.91523502)
\curveto(421.65664826,14.32842945)(421.54727327,14.70516555)(421.3649816,15.04544332)
\curveto(421.18268994,15.38572109)(420.93963439,15.64092942)(420.63581496,15.8110683)
\curveto(420.33807191,15.98728357)(420.04640525,16.07539121)(419.76081498,16.07539121)
\curveto(419.22609277,16.07539121)(418.74909625,15.83537386)(418.32982543,15.35533915)
\curveto(417.77687405,14.72339472)(417.50039837,13.79978363)(417.50039837,12.58450588)
\curveto(417.50039837,11.35707536)(417.76775947,10.4152351)(418.30248168,9.75898512)
\curveto(418.83720389,9.10881152)(419.43268999,8.78372473)(420.08893997,8.78372473)
\curveto(420.64189134,8.78372473)(421.17661355,9.06020041)(421.6931066,9.61315179)
\closepath
}
}
{
\newrgbcolor{curcolor}{0 0 0}
\pscustom[linestyle=none,fillstyle=solid,fillcolor=curcolor]
{
\newpath
\moveto(429.85977306,9.23945388)
\curveto(429.00300225,8.57712751)(428.46524184,8.19431502)(428.24649185,8.09101641)
\curveto(427.91836686,7.93910669)(427.5689745,7.86315183)(427.19831479,7.86315183)
\curveto(426.62105786,7.86315183)(426.14406135,8.06063447)(425.76732524,8.45559974)
\curveto(425.39666553,8.850565)(425.21133567,9.37009624)(425.21133567,10.01419345)
\curveto(425.21133567,10.42131149)(425.30248151,10.77374204)(425.48477317,11.07148509)
\curveto(425.73390511,11.48467952)(426.16532871,11.8735684)(426.77904397,12.23815172)
\curveto(427.39883562,12.60273505)(428.42574532,13.04631143)(429.85977306,13.56888086)
\lineto(429.85977306,13.89700585)
\curveto(429.85977306,14.72947111)(429.72609251,15.30065165)(429.4587314,15.61054747)
\curveto(429.19744669,15.9204433)(428.8146342,16.07539121)(428.31029393,16.07539121)
\curveto(427.92748144,16.07539121)(427.623662,15.9720926)(427.39883562,15.76549539)
\curveto(427.16793285,15.55889817)(427.05248146,15.32191901)(427.05248146,15.0545579)
\lineto(427.07071063,14.52591208)
\curveto(427.07071063,14.2463982)(426.99779396,14.0306864)(426.85196063,13.87877668)
\curveto(426.71220369,13.72686696)(426.52687384,13.65091211)(426.29597106,13.65091211)
\curveto(426.07114468,13.65091211)(425.88581482,13.72990516)(425.73998149,13.88789127)
\curveto(425.60022455,14.04587737)(425.53034608,14.26158917)(425.53034608,14.53502667)
\curveto(425.53034608,15.0575961)(425.79770719,15.53763081)(426.3324294,15.9751308)
\curveto(426.86715161,16.41263079)(427.61758561,16.63138078)(428.58373142,16.63138078)
\curveto(429.32505085,16.63138078)(429.93268972,16.50681481)(430.40664804,16.25768287)
\curveto(430.76515498,16.06931482)(431.02947789,15.77460997)(431.19961677,15.37356831)
\curveto(431.30899177,15.1122836)(431.36367927,14.57756139)(431.36367927,13.76940169)
\lineto(431.36367927,10.93476634)
\curveto(431.36367927,10.13875942)(431.37887024,9.64961012)(431.40925219,9.46731846)
\curveto(431.43963413,9.29110319)(431.48824524,9.17261361)(431.55508552,9.11184972)
\curveto(431.62800218,9.05108583)(431.71003343,9.02070389)(431.80117926,9.02070389)
\curveto(431.89840148,9.02070389)(431.98347092,9.04197125)(432.05638759,9.08450597)
\curveto(432.18399175,9.16349902)(432.43008549,9.38528721)(432.79466882,9.74987054)
\lineto(432.79466882,9.23945388)
\curveto(432.11411328,8.32799557)(431.46393968,7.87226642)(430.84414803,7.87226642)
\curveto(430.54640499,7.87226642)(430.30942582,7.97556503)(430.13321055,8.18216224)
\curveto(429.95699528,8.38875946)(429.86584945,8.74119001)(429.85977306,9.23945388)
\closepath
\moveto(429.85977306,9.83190178)
\lineto(429.85977306,13.01289129)
\curveto(428.94223836,12.64830796)(428.34979046,12.39006144)(428.08242935,12.23815172)
\curveto(427.60239464,11.97079062)(427.25907868,11.69127674)(427.05248146,11.39961008)
\curveto(426.84588424,11.10794342)(426.74258564,10.78893301)(426.74258564,10.44257885)
\curveto(426.74258564,10.00507886)(426.87322799,9.64049554)(427.13451271,9.34882888)
\curveto(427.39579743,9.06323861)(427.69657867,8.92044347)(428.03685644,8.92044347)
\curveto(428.49866198,8.92044347)(429.10630085,9.22426291)(429.85977306,9.83190178)
\closepath
}
}
{
\newrgbcolor{curcolor}{0 0 0}
\pscustom[linestyle=none,fillstyle=solid,fillcolor=curcolor]
{
\newpath
\moveto(435.83893958,19.12877655)
\lineto(435.83893958,16.38528704)
\lineto(437.78946036,16.38528704)
\lineto(437.78946036,15.74726622)
\lineto(435.83893958,15.74726622)
\lineto(435.83893958,10.33320386)
\curveto(435.83893958,9.79240526)(435.91489443,9.42782193)(436.06680415,9.23945388)
\curveto(436.22479026,9.05108583)(436.42531109,8.95690181)(436.66836664,8.95690181)
\curveto(436.86888747,8.95690181)(437.06333191,9.01766569)(437.25169996,9.13919347)
\curveto(437.44006801,9.26679763)(437.58590134,9.45212749)(437.68919995,9.69518304)
\lineto(438.04466869,9.69518304)
\curveto(437.83199508,9.09969694)(437.53121384,8.65004417)(437.14232496,8.34622474)
\curveto(436.75343608,8.04848169)(436.35239442,7.89961017)(435.93919999,7.89961017)
\curveto(435.65968611,7.89961017)(435.38624861,7.97556503)(435.11888751,8.12747474)
\curveto(434.85152641,8.28546085)(434.65404377,8.50724904)(434.52643961,8.79283931)
\curveto(434.39883544,9.08450597)(434.33503336,9.53112054)(434.33503336,10.13268303)
\lineto(434.33503336,15.74726622)
\lineto(433.01341881,15.74726622)
\lineto(433.01341881,16.04804746)
\curveto(433.34762019,16.18172801)(433.68789796,16.4065544)(434.03425212,16.72252661)
\curveto(434.38668267,17.04457522)(434.69961669,17.42434951)(434.97305418,17.8618495)
\curveto(435.11281112,18.09275227)(435.30725556,18.51506129)(435.5563875,19.12877655)
\closepath
}
}
{
\newrgbcolor{curcolor}{0 0 0}
\pscustom[linestyle=none,fillstyle=solid,fillcolor=curcolor]
{
\newpath
\moveto(443.33112689,9.23945388)
\curveto(442.47435608,8.57712751)(441.93659567,8.19431502)(441.71784568,8.09101641)
\curveto(441.38972069,7.93910669)(441.04032833,7.86315183)(440.66966862,7.86315183)
\curveto(440.09241169,7.86315183)(439.61541518,8.06063447)(439.23867907,8.45559974)
\curveto(438.86801936,8.850565)(438.6826895,9.37009624)(438.6826895,10.01419345)
\curveto(438.6826895,10.42131149)(438.77383534,10.77374204)(438.956127,11.07148509)
\curveto(439.20525894,11.48467952)(439.63668254,11.8735684)(440.2503978,12.23815172)
\curveto(440.87018945,12.60273505)(441.89709915,13.04631143)(443.33112689,13.56888086)
\lineto(443.33112689,13.89700585)
\curveto(443.33112689,14.72947111)(443.19744634,15.30065165)(442.93008523,15.61054747)
\curveto(442.66880052,15.9204433)(442.28598803,16.07539121)(441.78164776,16.07539121)
\curveto(441.39883527,16.07539121)(441.09501583,15.9720926)(440.87018945,15.76549539)
\curveto(440.63928668,15.55889817)(440.52383529,15.32191901)(440.52383529,15.0545579)
\lineto(440.54206446,14.52591208)
\curveto(440.54206446,14.2463982)(440.46914779,14.0306864)(440.32331446,13.87877668)
\curveto(440.18355752,13.72686696)(439.99822767,13.65091211)(439.76732489,13.65091211)
\curveto(439.54249851,13.65091211)(439.35716865,13.72990516)(439.21133532,13.88789127)
\curveto(439.07157838,14.04587737)(439.00169991,14.26158917)(439.00169991,14.53502667)
\curveto(439.00169991,15.0575961)(439.26906102,15.53763081)(439.80378323,15.9751308)
\curveto(440.33850544,16.41263079)(441.08893944,16.63138078)(442.05508525,16.63138078)
\curveto(442.79640468,16.63138078)(443.40404355,16.50681481)(443.87800187,16.25768287)
\curveto(444.23650881,16.06931482)(444.50083172,15.77460997)(444.6709706,15.37356831)
\curveto(444.7803456,15.1122836)(444.8350331,14.57756139)(444.8350331,13.76940169)
\lineto(444.8350331,10.93476634)
\curveto(444.8350331,10.13875942)(444.85022407,9.64961012)(444.88060602,9.46731846)
\curveto(444.91098796,9.29110319)(444.95959907,9.17261361)(445.02643935,9.11184972)
\curveto(445.09935601,9.05108583)(445.18138726,9.02070389)(445.27253309,9.02070389)
\curveto(445.36975531,9.02070389)(445.45482475,9.04197125)(445.52774142,9.08450597)
\curveto(445.65534558,9.16349902)(445.90143932,9.38528721)(446.26602265,9.74987054)
\lineto(446.26602265,9.23945388)
\curveto(445.58546711,8.32799557)(444.93529351,7.87226642)(444.31550186,7.87226642)
\curveto(444.01775882,7.87226642)(443.78077965,7.97556503)(443.60456438,8.18216224)
\curveto(443.42834911,8.38875946)(443.33720328,8.74119001)(443.33112689,9.23945388)
\closepath
\moveto(443.33112689,9.83190178)
\lineto(443.33112689,13.01289129)
\curveto(442.41359219,12.64830796)(441.82114429,12.39006144)(441.55378318,12.23815172)
\curveto(441.07374847,11.97079062)(440.73043251,11.69127674)(440.52383529,11.39961008)
\curveto(440.31723807,11.10794342)(440.21393947,10.78893301)(440.21393947,10.44257885)
\curveto(440.21393947,10.00507886)(440.34458182,9.64049554)(440.60586654,9.34882888)
\curveto(440.86715126,9.06323861)(441.1679325,8.92044347)(441.50821027,8.92044347)
\curveto(441.97001581,8.92044347)(442.57765468,9.22426291)(443.33112689,9.83190178)
\closepath
}
}
{
\newrgbcolor{curcolor}{0 0 0}
\pscustom[linestyle=none,fillstyle=solid,fillcolor=curcolor]
{
\newpath
\moveto(121.61071664,195.98802506)
\lineto(121.61071664,191.12995226)
\lineto(123.86201866,191.12995226)
\curveto(124.37851171,191.12995226)(124.75524781,191.24236546)(124.99222697,191.46719184)
\curveto(125.23528252,191.69809461)(125.39630682,192.15078557)(125.47529987,192.82526472)
\lineto(125.81253945,192.82526472)
\lineto(125.81253945,188.64167108)
\lineto(125.47529987,188.64167108)
\curveto(125.46922348,189.12170579)(125.4054214,189.47413633)(125.28389363,189.69896272)
\curveto(125.16844224,189.9237891)(125.00437975,190.09088979)(124.79170614,190.20026479)
\curveto(124.58510892,190.31571617)(124.2752131,190.37344187)(123.86201866,190.37344187)
\lineto(121.61071664,190.37344187)
\lineto(121.61071664,186.49062946)
\curveto(121.61071664,185.86476142)(121.65021316,185.45156699)(121.72920622,185.25104616)
\curveto(121.7899701,185.09913644)(121.91757427,184.96849409)(122.11201871,184.85911909)
\curveto(122.37937981,184.71328576)(122.65889369,184.64036909)(122.95056035,184.64036909)
\lineto(123.39717493,184.64036909)
\lineto(123.39717493,184.30312952)
\lineto(118.09248756,184.30312952)
\lineto(118.09248756,184.64036909)
\lineto(118.52998755,184.64036909)
\curveto(119.0404042,184.64036909)(119.41106391,184.78924062)(119.64196669,185.08698367)
\curveto(119.78780002,185.28142811)(119.86071668,185.74931004)(119.86071668,186.49062946)
\lineto(119.86071668,194.47500426)
\curveto(119.86071668,195.1008723)(119.82122015,195.51406674)(119.7422271,195.71458757)
\curveto(119.68146321,195.86649729)(119.55689724,195.99713964)(119.36852919,196.10651464)
\curveto(119.10724448,196.25234797)(118.8277306,196.32526463)(118.52998755,196.32526463)
\lineto(118.09248756,196.32526463)
\lineto(118.09248756,196.66250421)
\lineto(127.28910191,196.66250421)
\lineto(127.40759149,193.94635844)
\lineto(127.08858108,193.94635844)
\curveto(126.93059498,194.52361537)(126.74526512,194.94592439)(126.53259151,195.2132855)
\curveto(126.3259943,195.48672299)(126.06774778,195.68420562)(125.75785195,195.8057334)
\curveto(125.45403251,195.92726117)(124.98007419,195.98802506)(124.33597699,195.98802506)
\closepath
}
}
{
\newrgbcolor{curcolor}{0 0 0}
\pscustom[linestyle=none,fillstyle=solid,fillcolor=curcolor]
{
\newpath
\moveto(136.73181001,196.94505629)
\lineto(136.73181001,192.67031681)
\lineto(136.39457043,192.67031681)
\curveto(136.28519544,193.49062929)(136.0877128,194.14384108)(135.80212253,194.62995218)
\curveto(135.52260865,195.11606328)(135.12156699,195.50191396)(134.59899756,195.78750423)
\curveto(134.07642813,196.0730945)(133.53562953,196.21588964)(132.97660177,196.21588964)
\curveto(132.34465734,196.21588964)(131.82208791,196.0214452)(131.40889348,195.63255632)
\curveto(130.99569904,195.24974383)(130.78910182,194.81224384)(130.78910182,194.32005635)
\curveto(130.78910182,193.94332025)(130.91974418,193.60000429)(131.1810289,193.29010846)
\curveto(131.557765,192.83437931)(132.45403234,192.22674043)(133.86983091,191.46719184)
\curveto(135.02434477,190.84740019)(135.81123712,190.37040367)(136.23050794,190.03620229)
\curveto(136.65585515,189.7080773)(136.98094195,189.31918842)(137.20576833,188.86953565)
\curveto(137.4366711,188.41988289)(137.55212249,187.94896276)(137.55212249,187.45677527)
\curveto(137.55212249,186.52101141)(137.18753916,185.71285171)(136.45837252,185.03229617)
\curveto(135.73528226,184.35781702)(134.80255658,184.02057744)(133.6601955,184.02057744)
\curveto(133.30168857,184.02057744)(132.96444899,184.04792119)(132.64847678,184.10260869)
\curveto(132.46010873,184.13299063)(132.06818165,184.24236563)(131.47269556,184.43073368)
\curveto(130.88328585,184.62517812)(130.50958794,184.72240034)(130.35160183,184.72240034)
\curveto(130.19969212,184.72240034)(130.07816434,184.67682743)(129.98701851,184.5856816)
\curveto(129.90194907,184.49453576)(129.83814699,184.30616771)(129.79561227,184.02057744)
\lineto(129.45837269,184.02057744)
\lineto(129.45837269,188.25885859)
\lineto(129.79561227,188.25885859)
\curveto(129.95359837,187.37170583)(130.16627198,186.70634126)(130.43363308,186.26276489)
\curveto(130.70099419,185.8252649)(131.10811223,185.46068157)(131.65498722,185.16901491)
\curveto(132.20793859,184.87734825)(132.81253927,184.73151492)(133.46878926,184.73151492)
\curveto(134.22833785,184.73151492)(134.82686214,184.93203575)(135.26436213,185.33307741)
\curveto(135.70793851,185.73411907)(135.9297267,186.20807739)(135.9297267,186.75495237)
\curveto(135.9297267,187.05877181)(135.84465725,187.36562944)(135.67451837,187.67552527)
\curveto(135.51045587,187.98542109)(135.25220935,188.27404956)(134.8997788,188.54141066)
\curveto(134.66279964,188.72370232)(134.01566424,189.10955301)(132.9583726,189.69896272)
\curveto(131.90108096,190.29444881)(131.14760876,190.76840713)(130.69795599,191.12083768)
\curveto(130.25437962,191.47326823)(129.91714004,191.86215711)(129.68623727,192.28750432)
\curveto(129.4553345,192.71285153)(129.33988311,193.18073346)(129.33988311,193.69115012)
\curveto(129.33988311,194.57830287)(129.68016088,195.34088966)(130.36071642,195.97891048)
\curveto(131.04127196,196.62300768)(131.90715735,196.94505629)(132.9583726,196.94505629)
\curveto(133.61462259,196.94505629)(134.3103691,196.78403198)(135.04561213,196.46198338)
\curveto(135.3858899,196.31007366)(135.62590726,196.2341188)(135.7656642,196.2341188)
\curveto(135.92365031,196.2341188)(136.05125447,196.27969172)(136.14847669,196.37083755)
\curveto(136.2517753,196.46805977)(136.33380655,196.65946602)(136.39457043,196.94505629)
\closepath
}
}
{
\newrgbcolor{curcolor}{0 0 0}
\pscustom[linestyle=none,fillstyle=solid,fillcolor=curcolor]
{
\newpath
\moveto(150.73180966,188.44115025)
\lineto(145.94665353,188.44115025)
\lineto(145.10811188,186.49062946)
\curveto(144.90151467,186.01059475)(144.79821606,185.65208782)(144.79821606,185.41510866)
\curveto(144.79821606,185.22674061)(144.88632369,185.05963992)(145.06253897,184.91380659)
\curveto(145.24483063,184.77404965)(145.63371951,184.68290381)(146.2292056,184.64036909)
\lineto(146.2292056,184.30312952)
\lineto(142.33727862,184.30312952)
\lineto(142.33727862,184.64036909)
\curveto(142.85377166,184.73151492)(143.18797304,184.85000451)(143.33988276,184.99583783)
\curveto(143.64977859,185.28750449)(143.99309455,185.8799524)(144.36983065,186.77318154)
\lineto(148.71748679,196.94505629)
\lineto(149.0364972,196.94505629)
\lineto(153.33858043,186.66380654)
\curveto(153.68493458,185.83741767)(153.9978686,185.29965727)(154.27738249,185.05052533)
\curveto(154.56297276,184.80746978)(154.95793803,184.67075104)(155.46227829,184.64036909)
\lineto(155.46227829,184.30312952)
\lineto(150.58597633,184.30312952)
\lineto(150.58597633,184.64036909)
\curveto(151.07816382,184.66467465)(151.409327,184.7467059)(151.57946589,184.88646284)
\curveto(151.75568116,185.02621978)(151.8437888,185.19635866)(151.8437888,185.39687949)
\curveto(151.8437888,185.6642406)(151.72226102,186.08654961)(151.47920547,186.66380654)
\closepath
\moveto(150.47660133,189.1156294)
\lineto(148.38024722,194.11042094)
\lineto(146.2292056,189.1156294)
\closepath
}
}
{
\newrgbcolor{curcolor}{0 0 0}
\pscustom[linestyle=none,fillstyle=solid,fillcolor=curcolor]
{
\newpath
\moveto(398.32986662,195.98802506)
\lineto(398.32986662,191.12995226)
\lineto(400.58116865,191.12995226)
\curveto(401.09766169,191.12995226)(401.47439779,191.24236546)(401.71137695,191.46719184)
\curveto(401.9544325,191.69809461)(402.1154568,192.15078557)(402.19444986,192.82526472)
\lineto(402.53168943,192.82526472)
\lineto(402.53168943,188.64167108)
\lineto(402.19444986,188.64167108)
\curveto(402.18837347,189.12170579)(402.12457139,189.47413633)(402.00304361,189.69896272)
\curveto(401.88759223,189.9237891)(401.72352973,190.09088979)(401.51085612,190.20026479)
\curveto(401.30425891,190.31571617)(400.99436308,190.37344187)(400.58116865,190.37344187)
\lineto(398.32986662,190.37344187)
\lineto(398.32986662,186.49062946)
\curveto(398.32986662,185.86476142)(398.36936315,185.45156699)(398.4483562,185.25104616)
\curveto(398.50912009,185.09913644)(398.63672425,184.96849409)(398.83116869,184.85911909)
\curveto(399.0985298,184.71328576)(399.37804368,184.64036909)(399.66971034,184.64036909)
\lineto(400.11632491,184.64036909)
\lineto(400.11632491,184.30312952)
\lineto(394.81163754,184.30312952)
\lineto(394.81163754,184.64036909)
\lineto(395.24913753,184.64036909)
\curveto(395.75955418,184.64036909)(396.1302139,184.78924062)(396.36111667,185.08698367)
\curveto(396.50695,185.28142811)(396.57986666,185.74931004)(396.57986666,186.49062946)
\lineto(396.57986666,194.47500426)
\curveto(396.57986666,195.1008723)(396.54037014,195.51406674)(396.46137708,195.71458757)
\curveto(396.4006132,195.86649729)(396.27604723,195.99713964)(396.08767918,196.10651464)
\curveto(395.82639446,196.25234797)(395.54688058,196.32526463)(395.24913753,196.32526463)
\lineto(394.81163754,196.32526463)
\lineto(394.81163754,196.66250421)
\lineto(404.00825189,196.66250421)
\lineto(404.12674147,193.94635844)
\lineto(403.80773107,193.94635844)
\curveto(403.64974496,194.52361537)(403.4644151,194.94592439)(403.2517415,195.2132855)
\curveto(403.04514428,195.48672299)(402.78689776,195.68420562)(402.47700193,195.8057334)
\curveto(402.1731825,195.92726117)(401.69922417,195.98802506)(401.05512697,195.98802506)
\closepath
}
}
{
\newrgbcolor{curcolor}{0 0 0}
\pscustom[linestyle=none,fillstyle=solid,fillcolor=curcolor]
{
\newpath
\moveto(413.45095999,196.94505629)
\lineto(413.45095999,192.67031681)
\lineto(413.11372042,192.67031681)
\curveto(413.00434542,193.49062929)(412.80686279,194.14384108)(412.52127252,194.62995218)
\curveto(412.24175863,195.11606328)(411.84071698,195.50191396)(411.31814755,195.78750423)
\curveto(410.79557811,196.0730945)(410.25477952,196.21588964)(409.69575175,196.21588964)
\curveto(409.06380732,196.21588964)(408.54123789,196.0214452)(408.12804346,195.63255632)
\curveto(407.71484902,195.24974383)(407.50825181,194.81224384)(407.50825181,194.32005635)
\curveto(407.50825181,193.94332025)(407.63889416,193.60000429)(407.90017888,193.29010846)
\curveto(408.27691498,192.83437931)(409.17318232,192.22674043)(410.5889809,191.46719184)
\curveto(411.74349476,190.84740019)(412.5303871,190.37040367)(412.94965792,190.03620229)
\curveto(413.37500513,189.7080773)(413.70009193,189.31918842)(413.92491831,188.86953565)
\curveto(414.15582109,188.41988289)(414.27127247,187.94896276)(414.27127247,187.45677527)
\curveto(414.27127247,186.52101141)(413.90668915,185.71285171)(413.1775225,185.03229617)
\curveto(412.45443224,184.35781702)(411.52170657,184.02057744)(410.37934549,184.02057744)
\curveto(410.02083855,184.02057744)(409.68359897,184.04792119)(409.36762676,184.10260869)
\curveto(409.17925871,184.13299063)(408.78733164,184.24236563)(408.19184554,184.43073368)
\curveto(407.60243583,184.62517812)(407.22873793,184.72240034)(407.07075182,184.72240034)
\curveto(406.9188421,184.72240034)(406.79731432,184.67682743)(406.70616849,184.5856816)
\curveto(406.62109905,184.49453576)(406.55729697,184.30616771)(406.51476225,184.02057744)
\lineto(406.17752267,184.02057744)
\lineto(406.17752267,188.25885859)
\lineto(406.51476225,188.25885859)
\curveto(406.67274836,187.37170583)(406.88542196,186.70634126)(407.15278307,186.26276489)
\curveto(407.42014417,185.8252649)(407.82726222,185.46068157)(408.3741372,185.16901491)
\curveto(408.92708858,184.87734825)(409.53168926,184.73151492)(410.18793924,184.73151492)
\curveto(410.94748783,184.73151492)(411.54601212,184.93203575)(411.98351211,185.33307741)
\curveto(412.42708849,185.73411907)(412.64887668,186.20807739)(412.64887668,186.75495237)
\curveto(412.64887668,187.05877181)(412.56380724,187.36562944)(412.39366835,187.67552527)
\curveto(412.22960586,187.98542109)(411.97135933,188.27404956)(411.61892879,188.54141066)
\curveto(411.38194963,188.72370232)(410.73481423,189.10955301)(409.67752259,189.69896272)
\curveto(408.62023095,190.29444881)(407.86675874,190.76840713)(407.41710598,191.12083768)
\curveto(406.9735296,191.47326823)(406.63629002,191.86215711)(406.40538725,192.28750432)
\curveto(406.17448448,192.71285153)(406.05903309,193.18073346)(406.05903309,193.69115012)
\curveto(406.05903309,194.57830287)(406.39931086,195.34088966)(407.0798664,195.97891048)
\curveto(407.76042194,196.62300768)(408.62630733,196.94505629)(409.67752259,196.94505629)
\curveto(410.33377257,196.94505629)(411.02951908,196.78403198)(411.76476212,196.46198338)
\curveto(412.10503989,196.31007366)(412.34505724,196.2341188)(412.48481418,196.2341188)
\curveto(412.64280029,196.2341188)(412.77040445,196.27969172)(412.86762667,196.37083755)
\curveto(412.97092528,196.46805977)(413.05295653,196.65946602)(413.11372042,196.94505629)
\closepath
}
}
{
\newrgbcolor{curcolor}{0 0 0}
\pscustom[linestyle=none,fillstyle=solid,fillcolor=curcolor]
{
\newpath
\moveto(428.56293878,190.61042103)
\curveto(429.41970959,190.42812937)(430.0607686,190.13646271)(430.48611582,189.73542105)
\curveto(431.07552552,189.17639329)(431.37023038,188.49279955)(431.37023038,187.68463985)
\curveto(431.37023038,187.07092459)(431.17578594,186.48151488)(430.78689706,185.91641073)
\curveto(430.39800818,185.35738296)(429.86328597,184.94722672)(429.18273043,184.68594201)
\curveto(428.50825128,184.43073368)(427.4752652,184.30312952)(426.08377218,184.30312952)
\lineto(420.25043899,184.30312952)
\lineto(420.25043899,184.64036909)
\lineto(420.71528273,184.64036909)
\curveto(421.23177577,184.64036909)(421.60243548,184.80443159)(421.82726187,185.13255658)
\curveto(421.96701881,185.34523019)(422.03689728,185.79792115)(422.03689728,186.49062946)
\lineto(422.03689728,194.47500426)
\curveto(422.03689728,195.24062925)(421.94878964,195.72370215)(421.77257437,195.92422298)
\curveto(421.53559521,196.19158408)(421.18316466,196.32526463)(420.71528273,196.32526463)
\lineto(420.25043899,196.32526463)
\lineto(420.25043899,196.66250421)
\lineto(425.59158469,196.66250421)
\curveto(426.58811244,196.66250421)(427.38715756,196.58958755)(427.98872005,196.44375422)
\curveto(428.90017836,196.22500422)(429.59592487,195.83611534)(430.07595958,195.27708758)
\curveto(430.55599429,194.7241362)(430.79601164,194.08611539)(430.79601164,193.36302513)
\curveto(430.79601164,192.74323347)(430.60764359,192.18724391)(430.23090749,191.69505642)
\curveto(429.85417139,191.20894532)(429.29818182,190.84740019)(428.56293878,190.61042103)
\closepath
\moveto(423.78689723,191.10260852)
\curveto(424.01172362,191.06007379)(424.26693194,191.02665366)(424.55252221,191.0023481)
\curveto(424.84418887,190.98411894)(425.16319928,190.97500435)(425.50955344,190.97500435)
\curveto(426.3967062,190.97500435)(427.06207076,191.06918838)(427.50564714,191.25755643)
\curveto(427.95529991,191.45200087)(428.29861587,191.74670572)(428.53559503,192.14167099)
\curveto(428.77257419,192.53663626)(428.89106377,192.96805986)(428.89106377,193.43594179)
\curveto(428.89106377,194.15903205)(428.59635892,194.77578551)(428.00694921,195.28620216)
\curveto(427.4175395,195.79661881)(426.5577305,196.05182714)(425.42752219,196.05182714)
\curveto(424.81988332,196.05182714)(424.27300833,195.98498687)(423.78689723,195.85130631)
\closepath
\moveto(423.78689723,185.19635866)
\curveto(424.49175833,185.03229617)(425.18750484,184.95026492)(425.87413676,184.95026492)
\curveto(426.97396313,184.95026492)(427.81250477,185.19635866)(428.3897617,185.68854615)
\curveto(428.96701863,186.18681003)(429.2556471,186.80052529)(429.2556471,187.52969194)
\curveto(429.2556471,188.00972665)(429.12500474,188.47153219)(428.86372002,188.91510857)
\curveto(428.60243531,189.35868495)(428.1770881,189.7080773)(427.58767839,189.96328563)
\curveto(426.99826868,190.21849395)(426.26910203,190.34609812)(425.40017844,190.34609812)
\curveto(425.02344234,190.34609812)(424.70139374,190.34002173)(424.43403263,190.32786895)
\curveto(424.16667153,190.31571617)(423.95095973,190.29444881)(423.78689723,190.26406687)
\closepath
}
}
{
\newrgbcolor{curcolor}{0 0 0}
\pscustom[linestyle=none,fillstyle=solid,fillcolor=curcolor]
{
\newpath
\moveto(485.27969355,198.79987395)
\lineto(485.27969355,194.52513447)
\lineto(484.94245397,194.52513447)
\curveto(484.83307898,195.34544695)(484.63559634,195.99865874)(484.35000607,196.48476984)
\curveto(484.07049219,196.97088094)(483.66945053,197.35673162)(483.1468811,197.64232189)
\curveto(482.62431167,197.92791216)(482.08351307,198.0707073)(481.52448531,198.0707073)
\curveto(480.89254088,198.0707073)(480.36997145,197.87626286)(479.95677701,197.48737398)
\curveto(479.54358258,197.10456149)(479.33698536,196.6670615)(479.33698536,196.17487401)
\curveto(479.33698536,195.79813791)(479.46762772,195.45482195)(479.72891244,195.14492612)
\curveto(480.10564854,194.68919697)(481.00191588,194.08155809)(482.41771445,193.3220095)
\curveto(483.57222831,192.70221785)(484.35912065,192.22522133)(484.77839148,191.89101995)
\curveto(485.20373869,191.56289496)(485.52882549,191.17400608)(485.75365187,190.72435332)
\curveto(485.98455464,190.27470055)(486.10000603,189.80378042)(486.10000603,189.31159294)
\curveto(486.10000603,188.37582907)(485.7354227,187.56766937)(485.00625605,186.88711383)
\curveto(484.28316579,186.21263468)(483.35044012,185.8753951)(482.20807904,185.8753951)
\curveto(481.84957211,185.8753951)(481.51233253,185.90273885)(481.19636032,185.95742635)
\curveto(481.00799227,185.9878083)(480.61606519,186.09718329)(480.0205791,186.28555134)
\curveto(479.43116939,186.47999578)(479.05747148,186.577218)(478.89948537,186.577218)
\curveto(478.74757566,186.577218)(478.62604788,186.53164509)(478.53490205,186.44049926)
\curveto(478.44983261,186.34935343)(478.38603053,186.16098538)(478.3434958,185.8753951)
\lineto(478.00625623,185.8753951)
\lineto(478.00625623,190.11367625)
\lineto(478.3434958,190.11367625)
\curveto(478.50148191,189.22652349)(478.71415552,188.56115893)(478.98151662,188.11758255)
\curveto(479.24887773,187.68008256)(479.65599577,187.31549924)(480.20287076,187.02383258)
\curveto(480.75582213,186.73216592)(481.36042281,186.58633259)(482.0166728,186.58633259)
\curveto(482.77622139,186.58633259)(483.37474568,186.78685342)(483.81224567,187.18789507)
\curveto(484.25582205,187.58893673)(484.47761023,188.06289505)(484.47761023,188.60977004)
\curveto(484.47761023,188.91358947)(484.39254079,189.2204471)(484.22240191,189.53034293)
\curveto(484.05833941,189.84023876)(483.80009289,190.12886722)(483.44766234,190.39622832)
\curveto(483.21068318,190.57851999)(482.56354778,190.96437067)(481.50625614,191.55378038)
\curveto(480.4489645,192.14926648)(479.6954923,192.6232248)(479.24583953,192.97565534)
\curveto(478.80226315,193.32808589)(478.46502358,193.71697477)(478.23412081,194.14232198)
\curveto(478.00321804,194.56766919)(477.88776665,195.03555113)(477.88776665,195.54596778)
\curveto(477.88776665,196.43312054)(478.22804442,197.19570732)(478.90859996,197.83372814)
\curveto(479.5891555,198.47782535)(480.45504089,198.79987395)(481.50625614,198.79987395)
\curveto(482.16250613,198.79987395)(482.85825264,198.63884965)(483.59349567,198.31680104)
\curveto(483.93377344,198.16489133)(484.1737908,198.08893647)(484.31354774,198.08893647)
\curveto(484.47153385,198.08893647)(484.59913801,198.13450938)(484.69636023,198.22565521)
\curveto(484.79965884,198.32287743)(484.88169009,198.51428368)(484.94245397,198.79987395)
\closepath
}
}
{
\newrgbcolor{curcolor}{0 0 0}
\pscustom[linestyle=none,fillstyle=solid,fillcolor=curcolor]
{
\newpath
\moveto(490.11953718,192.98476993)
\curveto(491.09783576,194.16358934)(492.03056143,194.75299905)(492.91771419,194.75299905)
\curveto(493.37344335,194.75299905)(493.76537042,194.63754766)(494.09349541,194.40664489)
\curveto(494.4216204,194.18181851)(494.68290512,193.8081206)(494.87734956,193.28555117)
\curveto(495.01103011,192.92096785)(495.07787039,192.36194008)(495.07787039,191.60846788)
\lineto(495.07787039,188.04466588)
\curveto(495.07787039,187.51602006)(495.12040511,187.15751313)(495.20547455,186.96914508)
\curveto(495.27231483,186.81723536)(495.37865163,186.69874578)(495.52448496,186.61367634)
\curveto(495.67639468,186.52860689)(495.95287036,186.48607217)(496.35391202,186.48607217)
\lineto(496.35391202,186.15794718)
\lineto(492.22500587,186.15794718)
\lineto(492.22500587,186.48607217)
\lineto(492.39818295,186.48607217)
\curveto(492.78707183,186.48607217)(493.05747113,186.54379787)(493.20938085,186.65924925)
\curveto(493.36736696,186.78077703)(493.47674195,186.9569923)(493.53750584,187.18789507)
\curveto(493.5618114,187.2790409)(493.57396417,187.56463117)(493.57396417,188.04466588)
\lineto(493.57396417,191.46263455)
\curveto(493.57396417,192.22218314)(493.47370376,192.77209632)(493.27318293,193.11237409)
\curveto(493.07873849,193.45872825)(492.74757531,193.63190533)(492.27969337,193.63190533)
\curveto(491.55660311,193.63190533)(490.83655105,193.23694006)(490.11953718,192.44700952)
\lineto(490.11953718,188.04466588)
\curveto(490.11953718,187.47956173)(490.15295731,187.13016938)(490.21979759,186.99648883)
\curveto(490.30486703,186.82027355)(490.42031842,186.6896312)(490.56615175,186.60456175)
\curveto(490.71806147,186.5255687)(491.0218809,186.48607217)(491.47761006,186.48607217)
\lineto(491.47761006,186.15794718)
\lineto(487.34870391,186.15794718)
\lineto(487.34870391,186.48607217)
\lineto(487.53099557,186.48607217)
\curveto(487.95634279,186.48607217)(488.24193306,186.59240898)(488.38776639,186.80508258)
\curveto(488.53967611,187.02383258)(488.61563096,187.43702701)(488.61563096,188.04466588)
\lineto(488.61563096,191.14362414)
\curveto(488.61563096,192.14622828)(488.59132541,192.75690535)(488.5427143,192.97565534)
\curveto(488.50017958,193.19440534)(488.43030111,193.34327686)(488.33307889,193.42226992)
\curveto(488.24193306,193.50126297)(488.11736709,193.5407595)(487.95938098,193.5407595)
\curveto(487.7892421,193.5407595)(487.58568307,193.49518658)(487.34870391,193.40404075)
\lineto(487.21198517,193.73216574)
\lineto(489.7276101,194.75299905)
\lineto(490.11953718,194.75299905)
\closepath
}
}
{
\newrgbcolor{curcolor}{0 0 0}
\pscustom[linestyle=none,fillstyle=solid,fillcolor=curcolor]
{
\newpath
\moveto(501.74974522,187.36107215)
\curveto(500.89297441,186.69874578)(500.355214,186.31593329)(500.13646401,186.21263468)
\curveto(499.80833902,186.06072496)(499.45894667,185.9847701)(499.08828695,185.9847701)
\curveto(498.51103002,185.9847701)(498.03403351,186.18225274)(497.6572974,186.577218)
\curveto(497.28663769,186.97218327)(497.10130784,187.49171451)(497.10130784,188.13581171)
\curveto(497.10130784,188.54292976)(497.19245367,188.89536031)(497.37474533,189.19310335)
\curveto(497.62387727,189.60629779)(498.05530087,189.99518667)(498.66901613,190.35976999)
\curveto(499.28880778,190.72435332)(500.31571748,191.16792969)(501.74974522,191.69049913)
\lineto(501.74974522,192.01862412)
\curveto(501.74974522,192.85108937)(501.61606467,193.42226992)(501.34870356,193.73216574)
\curveto(501.08741885,194.04206157)(500.70460636,194.19700948)(500.20026609,194.19700948)
\curveto(499.8174536,194.19700948)(499.51363416,194.09371087)(499.28880778,193.88711365)
\curveto(499.05790501,193.68051644)(498.94245362,193.44353728)(498.94245362,193.17617617)
\lineto(498.96068279,192.64753035)
\curveto(498.96068279,192.36801647)(498.88776612,192.15230467)(498.74193279,192.00039495)
\curveto(498.60217585,191.84848523)(498.416846,191.77253037)(498.18594322,191.77253037)
\curveto(497.96111684,191.77253037)(497.77578699,191.85152343)(497.62995366,192.00950953)
\curveto(497.49019671,192.16749564)(497.42031824,192.38320744)(497.42031824,192.65664493)
\curveto(497.42031824,193.17921437)(497.68767935,193.65924908)(498.22240156,194.09674907)
\curveto(498.75712377,194.53424905)(499.50755778,194.75299905)(500.47370358,194.75299905)
\curveto(501.21502301,194.75299905)(501.82266188,194.62843308)(502.29662021,194.37930114)
\curveto(502.65512714,194.19093309)(502.91945005,193.89622824)(503.08958894,193.49518658)
\curveto(503.19896393,193.23390187)(503.25365143,192.69917966)(503.25365143,191.89101995)
\lineto(503.25365143,189.05638461)
\curveto(503.25365143,188.26037768)(503.2688424,187.77122839)(503.29922435,187.58893673)
\curveto(503.32960629,187.41272145)(503.3782174,187.29423187)(503.44505768,187.23346799)
\curveto(503.51797434,187.1727041)(503.60000559,187.14232216)(503.69115142,187.14232216)
\curveto(503.78837364,187.14232216)(503.87344308,187.16358952)(503.94635975,187.20612424)
\curveto(504.07396391,187.28511729)(504.32005765,187.50690548)(504.68464098,187.8714888)
\lineto(504.68464098,187.36107215)
\curveto(504.00408544,186.44961384)(503.35391185,185.99388468)(502.73412019,185.99388468)
\curveto(502.43637715,185.99388468)(502.19939799,186.09718329)(502.02318271,186.30378051)
\curveto(501.84696744,186.51037773)(501.75582161,186.86280827)(501.74974522,187.36107215)
\closepath
\moveto(501.74974522,187.95352005)
\lineto(501.74974522,191.13450956)
\curveto(500.83221052,190.76992623)(500.23976262,190.51167971)(499.97240151,190.35976999)
\curveto(499.4923668,190.09240889)(499.14905084,189.81289501)(498.94245362,189.52122835)
\curveto(498.73585641,189.22956169)(498.6325578,188.91055128)(498.6325578,188.56419712)
\curveto(498.6325578,188.12669713)(498.76320016,187.76211381)(499.02448487,187.47044715)
\curveto(499.28576959,187.18485688)(499.58655083,187.04206174)(499.9268286,187.04206174)
\curveto(500.38863414,187.04206174)(500.99627302,187.34588118)(501.74974522,187.95352005)
\closepath
}
}
{
\newrgbcolor{curcolor}{0 0 0}
\pscustom[linestyle=none,fillstyle=solid,fillcolor=curcolor]
{
\newpath
\moveto(504.70287015,193.67747824)
\lineto(507.27318258,194.71654072)
\lineto(507.61953674,194.71654072)
\lineto(507.61953674,192.76601993)
\curveto(508.05096034,193.50126297)(508.48238394,194.01471782)(508.91380754,194.30638448)
\curveto(509.35130753,194.60412753)(509.81007488,194.75299905)(510.29010959,194.75299905)
\curveto(511.12865123,194.75299905)(511.82743594,194.42487406)(512.3864637,193.76862407)
\curveto(513.07309563,192.96654076)(513.41641159,191.9214019)(513.41641159,190.63320749)
\curveto(513.41641159,189.19310335)(513.00321716,188.00213116)(512.17682829,187.06029091)
\curveto(511.49627275,186.28858954)(510.63950194,185.90273885)(509.60651586,185.90273885)
\curveto(509.15686309,185.90273885)(508.76797421,185.96654094)(508.43984922,186.0941451)
\curveto(508.19679367,186.18529093)(507.92335618,186.36758259)(507.61953674,186.64102009)
\lineto(507.61953674,184.0980514)
\curveto(507.61953674,183.52687086)(507.65295688,183.16532573)(507.71979715,183.01341601)
\curveto(507.79271382,182.8554299)(507.91424159,182.73086393)(508.08438048,182.6397181)
\curveto(508.26059575,182.54857227)(508.57656797,182.50299936)(509.03229712,182.50299936)
\lineto(509.03229712,182.16575978)
\lineto(504.65729723,182.16575978)
\lineto(504.65729723,182.50299936)
\lineto(504.88516181,182.50299936)
\curveto(505.21936319,182.49692297)(505.50495346,182.56072505)(505.74193262,182.6944056)
\curveto(505.85738401,182.76124588)(505.94549164,182.87062087)(506.00625553,183.02253059)
\curveto(506.07309581,183.16836392)(506.10651594,183.54510002)(506.10651594,184.1527389)
\lineto(506.10651594,192.04596787)
\curveto(506.10651594,192.58676646)(506.08221039,192.93008243)(506.03359928,193.07591576)
\curveto(505.98498817,193.22174909)(505.90599512,193.33112408)(505.79662012,193.40404075)
\curveto(505.69332151,193.47695741)(505.55052637,193.51341575)(505.36823471,193.51341575)
\curveto(505.22240138,193.51341575)(505.03707153,193.47088103)(504.81224514,193.38581158)
\closepath
\moveto(507.61953674,192.22825953)
\lineto(507.61953674,189.11107211)
\curveto(507.61953674,188.43659296)(507.64688049,187.99301658)(507.70156799,187.78034297)
\curveto(507.78663743,187.42791243)(507.99323465,187.1180166)(508.32135964,186.8506555)
\curveto(508.65556102,186.58329439)(509.07483184,186.44961384)(509.57917211,186.44961384)
\curveto(510.18681098,186.44961384)(510.67899847,186.686593)(511.05573457,187.16055132)
\curveto(511.54792206,187.78034297)(511.7940158,188.65230476)(511.7940158,189.77643667)
\curveto(511.7940158,191.05247831)(511.51450192,192.03381509)(510.95547416,192.72044702)
\curveto(510.56658528,193.19440534)(510.10477973,193.4313845)(509.57005752,193.4313845)
\curveto(509.27839086,193.4313845)(508.9897624,193.35846783)(508.70417213,193.2126345)
\curveto(508.48542213,193.10325951)(508.123877,192.77513452)(507.61953674,192.22825953)
\closepath
}
}
{
\newrgbcolor{curcolor}{0 0 0}
\pscustom[linestyle=none,fillstyle=solid,fillcolor=curcolor]
{
\newpath
\moveto(520.90859891,197.30508232)
\lineto(523.91641133,198.7725302)
\lineto(524.21719257,198.7725302)
\lineto(524.21719257,188.33633254)
\curveto(524.21719257,187.64362423)(524.24453632,187.21220063)(524.29922382,187.04206174)
\curveto(524.35998771,186.87192286)(524.48151548,186.7412805)(524.66380715,186.65013467)
\curveto(524.84609881,186.55898884)(525.21675852,186.50733953)(525.77578629,186.49518676)
\lineto(525.77578629,186.15794718)
\lineto(521.1273489,186.15794718)
\lineto(521.1273489,186.49518676)
\curveto(521.71068222,186.50733953)(522.08741832,186.55595064)(522.25755721,186.64102009)
\curveto(522.42769609,186.73216592)(522.54618567,186.8506555)(522.61302595,186.99648883)
\curveto(522.67986622,187.14839854)(522.71328636,187.59501312)(522.71328636,188.33633254)
\lineto(522.71328636,195.00820738)
\curveto(522.71328636,195.90751291)(522.68290442,196.48476984)(522.62214053,196.73997817)
\curveto(522.57960581,196.93442261)(522.50061276,197.07721774)(522.38516137,197.16836357)
\curveto(522.27578637,197.2595094)(522.14210582,197.30508232)(521.98411971,197.30508232)
\curveto(521.75929333,197.30508232)(521.44635931,197.21089829)(521.04531765,197.02253024)
\closepath
}
}
{
\newrgbcolor{curcolor}{0 0 0}
\pscustom[linestyle=none,fillstyle=solid,fillcolor=curcolor]
{
\newpath
\moveto(536.67682768,192.46523869)
\curveto(537.53359849,192.28294703)(538.1746575,191.99128037)(538.60000471,191.59023871)
\curveto(539.18941442,191.03121095)(539.48411928,190.34761721)(539.48411928,189.53945751)
\curveto(539.48411928,188.92574225)(539.28967484,188.33633254)(538.90078596,187.77122839)
\curveto(538.51189708,187.21220063)(537.97717487,186.80204439)(537.29661933,186.54075967)
\curveto(536.62214018,186.28555134)(535.5891541,186.15794718)(534.19766107,186.15794718)
\lineto(528.36432789,186.15794718)
\lineto(528.36432789,186.49518676)
\lineto(528.82917163,186.49518676)
\curveto(529.34566467,186.49518676)(529.71632438,186.65924925)(529.94115076,186.98737424)
\curveto(530.08090771,187.20004785)(530.15078618,187.65273881)(530.15078618,188.34544713)
\lineto(530.15078618,196.32982193)
\curveto(530.15078618,197.09544691)(530.06267854,197.57851981)(529.88646327,197.77904064)
\curveto(529.64948411,198.04640174)(529.29705356,198.1800823)(528.82917163,198.1800823)
\lineto(528.36432789,198.1800823)
\lineto(528.36432789,198.51732187)
\lineto(533.70547359,198.51732187)
\curveto(534.70200134,198.51732187)(535.50104646,198.44440521)(536.10260894,198.29857188)
\curveto(537.01406725,198.07982188)(537.70981376,197.690933)(538.18984847,197.13190524)
\curveto(538.66988319,196.57895386)(538.90990054,195.94093305)(538.90990054,195.21784279)
\curveto(538.90990054,194.59805114)(538.72153249,194.04206157)(538.34479639,193.54987408)
\curveto(537.96806029,193.06376298)(537.41207072,192.70221785)(536.67682768,192.46523869)
\closepath
\moveto(531.90078613,192.95742618)
\curveto(532.12561252,192.91489146)(532.38082084,192.88147132)(532.66641111,192.85716576)
\curveto(532.95807777,192.8389366)(533.27708818,192.82982201)(533.62344234,192.82982201)
\curveto(534.51059509,192.82982201)(535.17595966,192.92400604)(535.61953604,193.11237409)
\curveto(536.06918881,193.30681853)(536.41250477,193.60152338)(536.64948393,193.99648865)
\curveto(536.88646309,194.39145392)(537.00495267,194.82287752)(537.00495267,195.29075945)
\curveto(537.00495267,196.01384971)(536.71024782,196.63060317)(536.12083811,197.14101982)
\curveto(535.5314284,197.65143648)(534.6716194,197.9066448)(533.54141109,197.9066448)
\curveto(532.93377222,197.9066448)(532.38689723,197.83980453)(531.90078613,197.70612398)
\closepath
\moveto(531.90078613,187.05117633)
\curveto(532.60564723,186.88711383)(533.30139374,186.80508258)(533.98802566,186.80508258)
\curveto(535.08785202,186.80508258)(535.92639367,187.05117633)(536.5036506,187.54336381)
\curveto(537.08090753,188.04162769)(537.369536,188.65534295)(537.369536,189.3845096)
\curveto(537.369536,189.86454431)(537.23889364,190.32634985)(536.97760892,190.76992623)
\curveto(536.71632421,191.21350261)(536.29097699,191.56289496)(535.70156729,191.81810329)
\curveto(535.11215758,192.07331162)(534.38299093,192.20091578)(533.51406734,192.20091578)
\curveto(533.13733124,192.20091578)(532.81528264,192.19483939)(532.54792153,192.18268661)
\curveto(532.28056043,192.17053384)(532.06484863,192.14926648)(531.90078613,192.11888453)
\closepath
}
}
{
\newrgbcolor{curcolor}{0 0 0}
\pscustom[linestyle=none,fillstyle=solid,fillcolor=curcolor]
{
\newpath
\moveto(206.20739421,198.79987395)
\lineto(206.20739421,194.52513447)
\lineto(205.87015463,194.52513447)
\curveto(205.76077964,195.34544695)(205.563297,195.99865874)(205.27770673,196.48476984)
\curveto(204.99819285,196.97088094)(204.59715119,197.35673162)(204.07458176,197.64232189)
\curveto(203.55201233,197.92791216)(203.01121373,198.0707073)(202.45218597,198.0707073)
\curveto(201.82024154,198.0707073)(201.29767211,197.87626286)(200.88447768,197.48737398)
\curveto(200.47128324,197.10456149)(200.26468602,196.6670615)(200.26468602,196.17487401)
\curveto(200.26468602,195.79813791)(200.39532838,195.45482195)(200.6566131,195.14492612)
\curveto(201.0333492,194.68919697)(201.92961654,194.08155809)(203.34541511,193.3220095)
\curveto(204.49992897,192.70221785)(205.28682132,192.22522133)(205.70609214,191.89101995)
\curveto(206.13143935,191.56289496)(206.45652615,191.17400608)(206.68135253,190.72435332)
\curveto(206.9122553,190.27470055)(207.02770669,189.80378042)(207.02770669,189.31159294)
\curveto(207.02770669,188.37582907)(206.66312336,187.56766937)(205.93395672,186.88711383)
\curveto(205.21086646,186.21263468)(204.27814079,185.8753951)(203.1357797,185.8753951)
\curveto(202.77727277,185.8753951)(202.44003319,185.90273885)(202.12406098,185.95742635)
\curveto(201.93569293,185.9878083)(201.54376585,186.09718329)(200.94827976,186.28555134)
\curveto(200.35887005,186.47999578)(199.98517214,186.577218)(199.82718604,186.577218)
\curveto(199.67527632,186.577218)(199.55374854,186.53164509)(199.46260271,186.44049926)
\curveto(199.37753327,186.34935343)(199.31373119,186.16098538)(199.27119647,185.8753951)
\lineto(198.93395689,185.8753951)
\lineto(198.93395689,190.11367625)
\lineto(199.27119647,190.11367625)
\curveto(199.42918257,189.22652349)(199.64185618,188.56115893)(199.90921728,188.11758255)
\curveto(200.17657839,187.68008256)(200.58369643,187.31549924)(201.13057142,187.02383258)
\curveto(201.68352279,186.73216592)(202.28812347,186.58633259)(202.94437346,186.58633259)
\curveto(203.70392205,186.58633259)(204.30244634,186.78685342)(204.73994633,187.18789507)
\curveto(205.18352271,187.58893673)(205.4053109,188.06289505)(205.4053109,188.60977004)
\curveto(205.4053109,188.91358947)(205.32024145,189.2204471)(205.15010257,189.53034293)
\curveto(204.98604007,189.84023876)(204.72779355,190.12886722)(204.37536301,190.39622832)
\curveto(204.13838384,190.57851999)(203.49124844,190.96437067)(202.4339568,191.55378038)
\curveto(201.37666516,192.14926648)(200.62319296,192.6232248)(200.17354019,192.97565534)
\curveto(199.72996382,193.32808589)(199.39272424,193.71697477)(199.16182147,194.14232198)
\curveto(198.9309187,194.56766919)(198.81546731,195.03555113)(198.81546731,195.54596778)
\curveto(198.81546731,196.43312054)(199.15574508,197.19570732)(199.83630062,197.83372814)
\curveto(200.51685616,198.47782535)(201.38274155,198.79987395)(202.4339568,198.79987395)
\curveto(203.09020679,198.79987395)(203.7859533,198.63884965)(204.52119633,198.31680104)
\curveto(204.8614741,198.16489133)(205.10149146,198.08893647)(205.2412484,198.08893647)
\curveto(205.39923451,198.08893647)(205.52683867,198.13450938)(205.62406089,198.22565521)
\curveto(205.7273595,198.32287743)(205.80939075,198.51428368)(205.87015463,198.79987395)
\closepath
}
}
{
\newrgbcolor{curcolor}{0 0 0}
\pscustom[linestyle=none,fillstyle=solid,fillcolor=curcolor]
{
\newpath
\moveto(211.04723784,192.98476993)
\curveto(212.02553643,194.16358934)(212.9582621,194.75299905)(213.84541485,194.75299905)
\curveto(214.30114401,194.75299905)(214.69307108,194.63754766)(215.02119607,194.40664489)
\curveto(215.34932106,194.18181851)(215.61060578,193.8081206)(215.80505022,193.28555117)
\curveto(215.93873077,192.92096785)(216.00557105,192.36194008)(216.00557105,191.60846788)
\lineto(216.00557105,188.04466588)
\curveto(216.00557105,187.51602006)(216.04810577,187.15751313)(216.13317521,186.96914508)
\curveto(216.20001549,186.81723536)(216.30635229,186.69874578)(216.45218562,186.61367634)
\curveto(216.60409534,186.52860689)(216.88057103,186.48607217)(217.28161268,186.48607217)
\lineto(217.28161268,186.15794718)
\lineto(213.15270654,186.15794718)
\lineto(213.15270654,186.48607217)
\lineto(213.32588361,186.48607217)
\curveto(213.71477249,186.48607217)(213.98517179,186.54379787)(214.13708151,186.65924925)
\curveto(214.29506762,186.78077703)(214.40444262,186.9569923)(214.4652065,187.18789507)
\curveto(214.48951206,187.2790409)(214.50166484,187.56463117)(214.50166484,188.04466588)
\lineto(214.50166484,191.46263455)
\curveto(214.50166484,192.22218314)(214.40140442,192.77209632)(214.20088359,193.11237409)
\curveto(214.00643915,193.45872825)(213.67527597,193.63190533)(213.20739403,193.63190533)
\curveto(212.48430377,193.63190533)(211.76425171,193.23694006)(211.04723784,192.44700952)
\lineto(211.04723784,188.04466588)
\curveto(211.04723784,187.47956173)(211.08065798,187.13016938)(211.14749825,186.99648883)
\curveto(211.23256769,186.82027355)(211.34801908,186.6896312)(211.49385241,186.60456175)
\curveto(211.64576213,186.5255687)(211.94958157,186.48607217)(212.40531072,186.48607217)
\lineto(212.40531072,186.15794718)
\lineto(208.27640457,186.15794718)
\lineto(208.27640457,186.48607217)
\lineto(208.45869624,186.48607217)
\curveto(208.88404345,186.48607217)(209.16963372,186.59240898)(209.31546705,186.80508258)
\curveto(209.46737677,187.02383258)(209.54333163,187.43702701)(209.54333163,188.04466588)
\lineto(209.54333163,191.14362414)
\curveto(209.54333163,192.14622828)(209.51902607,192.75690535)(209.47041496,192.97565534)
\curveto(209.42788024,193.19440534)(209.35800177,193.34327686)(209.26077955,193.42226992)
\curveto(209.16963372,193.50126297)(209.04506775,193.5407595)(208.88708164,193.5407595)
\curveto(208.71694276,193.5407595)(208.51338374,193.49518658)(208.27640457,193.40404075)
\lineto(208.13968583,193.73216574)
\lineto(210.65531076,194.75299905)
\lineto(211.04723784,194.75299905)
\closepath
}
}
{
\newrgbcolor{curcolor}{0 0 0}
\pscustom[linestyle=none,fillstyle=solid,fillcolor=curcolor]
{
\newpath
\moveto(222.67744588,187.36107215)
\curveto(221.82067507,186.69874578)(221.28291467,186.31593329)(221.06416467,186.21263468)
\curveto(220.73603968,186.06072496)(220.38664733,185.9847701)(220.01598761,185.9847701)
\curveto(219.43873068,185.9847701)(218.96173417,186.18225274)(218.58499807,186.577218)
\curveto(218.21433835,186.97218327)(218.0290085,187.49171451)(218.0290085,188.13581171)
\curveto(218.0290085,188.54292976)(218.12015433,188.89536031)(218.30244599,189.19310335)
\curveto(218.55157793,189.60629779)(218.98300153,189.99518667)(219.59671679,190.35976999)
\curveto(220.21650844,190.72435332)(221.24341814,191.16792969)(222.67744588,191.69049913)
\lineto(222.67744588,192.01862412)
\curveto(222.67744588,192.85108937)(222.54376533,193.42226992)(222.27640422,193.73216574)
\curveto(222.01511951,194.04206157)(221.63230702,194.19700948)(221.12796675,194.19700948)
\curveto(220.74515426,194.19700948)(220.44133483,194.09371087)(220.21650844,193.88711365)
\curveto(219.98560567,193.68051644)(219.87015428,193.44353728)(219.87015428,193.17617617)
\lineto(219.88838345,192.64753035)
\curveto(219.88838345,192.36801647)(219.81546679,192.15230467)(219.66963346,192.00039495)
\curveto(219.52987652,191.84848523)(219.34454666,191.77253037)(219.11364389,191.77253037)
\curveto(218.8888175,191.77253037)(218.70348765,191.85152343)(218.55765432,192.00950953)
\curveto(218.41789738,192.16749564)(218.34801891,192.38320744)(218.34801891,192.65664493)
\curveto(218.34801891,193.17921437)(218.61538001,193.65924908)(219.15010222,194.09674907)
\curveto(219.68482443,194.53424905)(220.43525844,194.75299905)(221.40140425,194.75299905)
\curveto(222.14272367,194.75299905)(222.75036255,194.62843308)(223.22432087,194.37930114)
\curveto(223.5828278,194.19093309)(223.84715071,193.89622824)(224.0172896,193.49518658)
\curveto(224.12666459,193.23390187)(224.18135209,192.69917966)(224.18135209,191.89101995)
\lineto(224.18135209,189.05638461)
\curveto(224.18135209,188.26037768)(224.19654307,187.77122839)(224.22692501,187.58893673)
\curveto(224.25730695,187.41272145)(224.30591806,187.29423187)(224.37275834,187.23346799)
\curveto(224.445675,187.1727041)(224.52770625,187.14232216)(224.61885208,187.14232216)
\curveto(224.7160743,187.14232216)(224.80114374,187.16358952)(224.87406041,187.20612424)
\curveto(225.00166457,187.28511729)(225.24775832,187.50690548)(225.61234164,187.8714888)
\lineto(225.61234164,187.36107215)
\curveto(224.9317861,186.44961384)(224.28161251,185.99388468)(223.66182086,185.99388468)
\curveto(223.36407781,185.99388468)(223.12709865,186.09718329)(222.95088337,186.30378051)
\curveto(222.7746681,186.51037773)(222.68352227,186.86280827)(222.67744588,187.36107215)
\closepath
\moveto(222.67744588,187.95352005)
\lineto(222.67744588,191.13450956)
\curveto(221.75991118,190.76992623)(221.16746328,190.51167971)(220.90010218,190.35976999)
\curveto(220.42006747,190.09240889)(220.0767515,189.81289501)(219.87015428,189.52122835)
\curveto(219.66355707,189.22956169)(219.56025846,188.91055128)(219.56025846,188.56419712)
\curveto(219.56025846,188.12669713)(219.69090082,187.76211381)(219.95218553,187.47044715)
\curveto(220.21347025,187.18485688)(220.51425149,187.04206174)(220.85452926,187.04206174)
\curveto(221.3163348,187.04206174)(221.92397368,187.34588118)(222.67744588,187.95352005)
\closepath
}
}
{
\newrgbcolor{curcolor}{0 0 0}
\pscustom[linestyle=none,fillstyle=solid,fillcolor=curcolor]
{
\newpath
\moveto(225.63057081,193.67747824)
\lineto(228.20088324,194.71654072)
\lineto(228.5472374,194.71654072)
\lineto(228.5472374,192.76601993)
\curveto(228.978661,193.50126297)(229.4100846,194.01471782)(229.8415082,194.30638448)
\curveto(230.27900819,194.60412753)(230.73777554,194.75299905)(231.21781025,194.75299905)
\curveto(232.0563519,194.75299905)(232.7551366,194.42487406)(233.31416437,193.76862407)
\curveto(234.00079629,192.96654076)(234.34411226,191.9214019)(234.34411226,190.63320749)
\curveto(234.34411226,189.19310335)(233.93091782,188.00213116)(233.10452895,187.06029091)
\curveto(232.42397342,186.28858954)(231.5672026,185.90273885)(230.53421652,185.90273885)
\curveto(230.08456375,185.90273885)(229.69567487,185.96654094)(229.36754988,186.0941451)
\curveto(229.12449433,186.18529093)(228.85105684,186.36758259)(228.5472374,186.64102009)
\lineto(228.5472374,184.0980514)
\curveto(228.5472374,183.52687086)(228.58065754,183.16532573)(228.64749782,183.01341601)
\curveto(228.72041448,182.8554299)(228.84194225,182.73086393)(229.01208114,182.6397181)
\curveto(229.18829641,182.54857227)(229.50426863,182.50299936)(229.95999778,182.50299936)
\lineto(229.95999778,182.16575978)
\lineto(225.58499789,182.16575978)
\lineto(225.58499789,182.50299936)
\lineto(225.81286247,182.50299936)
\curveto(226.14706385,182.49692297)(226.43265412,182.56072505)(226.66963328,182.6944056)
\curveto(226.78508467,182.76124588)(226.8731923,182.87062087)(226.93395619,183.02253059)
\curveto(227.00079647,183.16836392)(227.03421661,183.54510002)(227.03421661,184.1527389)
\lineto(227.03421661,192.04596787)
\curveto(227.03421661,192.58676646)(227.00991105,192.93008243)(226.96129994,193.07591576)
\curveto(226.91268883,193.22174909)(226.83369578,193.33112408)(226.72432078,193.40404075)
\curveto(226.62102217,193.47695741)(226.47822704,193.51341575)(226.29593537,193.51341575)
\curveto(226.15010204,193.51341575)(225.96477219,193.47088103)(225.7399458,193.38581158)
\closepath
\moveto(228.5472374,192.22825953)
\lineto(228.5472374,189.11107211)
\curveto(228.5472374,188.43659296)(228.57458115,187.99301658)(228.62926865,187.78034297)
\curveto(228.71433809,187.42791243)(228.92093531,187.1180166)(229.2490603,186.8506555)
\curveto(229.58326168,186.58329439)(230.0025325,186.44961384)(230.50687277,186.44961384)
\curveto(231.11451164,186.44961384)(231.60669913,186.686593)(231.98343523,187.16055132)
\curveto(232.47562272,187.78034297)(232.72171646,188.65230476)(232.72171646,189.77643667)
\curveto(232.72171646,191.05247831)(232.44220258,192.03381509)(231.88317482,192.72044702)
\curveto(231.49428594,193.19440534)(231.03248039,193.4313845)(230.49775819,193.4313845)
\curveto(230.20609153,193.4313845)(229.91746306,193.35846783)(229.63187279,193.2126345)
\curveto(229.4131228,193.10325951)(229.05157767,192.77513452)(228.5472374,192.22825953)
\closepath
}
}
{
\newrgbcolor{curcolor}{0 0 0}
\pscustom[linestyle=none,fillstyle=solid,fillcolor=curcolor]
{
\newpath
\moveto(241.83629957,197.30508232)
\lineto(244.84411199,198.7725302)
\lineto(245.14489324,198.7725302)
\lineto(245.14489324,188.33633254)
\curveto(245.14489324,187.64362423)(245.17223699,187.21220063)(245.22692448,187.04206174)
\curveto(245.28768837,186.87192286)(245.40921615,186.7412805)(245.59150781,186.65013467)
\curveto(245.77379947,186.55898884)(246.14445918,186.50733953)(246.70348695,186.49518676)
\lineto(246.70348695,186.15794718)
\lineto(242.05504956,186.15794718)
\lineto(242.05504956,186.49518676)
\curveto(242.63838288,186.50733953)(243.01511898,186.55595064)(243.18525787,186.64102009)
\curveto(243.35539675,186.73216592)(243.47388633,186.8506555)(243.54072661,186.99648883)
\curveto(243.60756689,187.14839854)(243.64098702,187.59501312)(243.64098702,188.33633254)
\lineto(243.64098702,195.00820738)
\curveto(243.64098702,195.90751291)(243.61060508,196.48476984)(243.54984119,196.73997817)
\curveto(243.50730647,196.93442261)(243.42831342,197.07721774)(243.31286203,197.16836357)
\curveto(243.20348703,197.2595094)(243.06980648,197.30508232)(242.91182038,197.30508232)
\curveto(242.68699399,197.30508232)(242.37405997,197.21089829)(241.97301832,197.02253024)
\closepath
}
}
{
\newrgbcolor{curcolor}{0 0 0}
\pscustom[linestyle=none,fillstyle=solid,fillcolor=curcolor]
{
\newpath
\moveto(257.52249709,190.29596791)
\lineto(252.73734096,190.29596791)
\lineto(251.89879932,188.34544713)
\curveto(251.6922021,187.86541242)(251.58890349,187.50690548)(251.58890349,187.26992632)
\curveto(251.58890349,187.08155827)(251.67701113,186.91445758)(251.8532264,186.76862425)
\curveto(252.03551806,186.62886731)(252.42440694,186.53772148)(253.01989304,186.49518676)
\lineto(253.01989304,186.15794718)
\lineto(249.12796605,186.15794718)
\lineto(249.12796605,186.49518676)
\curveto(249.6444591,186.58633259)(249.97866048,186.70482217)(250.13057019,186.8506555)
\curveto(250.44046602,187.14232216)(250.78378198,187.73477006)(251.16051809,188.6279992)
\lineto(255.50817423,198.79987395)
\lineto(255.82718464,198.79987395)
\lineto(260.12926786,188.51862421)
\curveto(260.47562202,187.69223534)(260.78855604,187.15447493)(261.06806992,186.905343)
\curveto(261.35366019,186.66228745)(261.74862546,186.5255687)(262.25296572,186.49518676)
\lineto(262.25296572,186.15794718)
\lineto(257.37666376,186.15794718)
\lineto(257.37666376,186.49518676)
\curveto(257.86885125,186.51949231)(258.20001444,186.60152356)(258.37015332,186.7412805)
\curveto(258.5463686,186.88103744)(258.63447623,187.05117633)(258.63447623,187.25169715)
\curveto(258.63447623,187.51905826)(258.51294846,187.94136728)(258.26989291,188.51862421)
\closepath
\moveto(257.26728877,190.97044706)
\lineto(255.17093465,195.9652386)
\lineto(253.01989304,190.97044706)
\closepath
}
}
{
\newrgbcolor{curcolor}{0 0 0}
\pscustom[linestyle=none,fillstyle=solid,fillcolor=curcolor]
{
\newpath
\moveto(298.03014582,198.79987395)
\lineto(298.03014582,194.52513447)
\lineto(297.69290625,194.52513447)
\curveto(297.58353125,195.34544695)(297.38604861,195.99865874)(297.10045834,196.48476984)
\curveto(296.82094446,196.97088094)(296.4199028,197.35673162)(295.89733337,197.64232189)
\curveto(295.37476394,197.92791216)(294.83396534,198.0707073)(294.27493758,198.0707073)
\curveto(293.64299315,198.0707073)(293.12042372,197.87626286)(292.70722929,197.48737398)
\curveto(292.29403485,197.10456149)(292.08743764,196.6670615)(292.08743764,196.17487401)
\curveto(292.08743764,195.79813791)(292.21807999,195.45482195)(292.47936471,195.14492612)
\curveto(292.85610081,194.68919697)(293.75236815,194.08155809)(295.16816672,193.3220095)
\curveto(296.32268058,192.70221785)(297.10957293,192.22522133)(297.52884375,191.89101995)
\curveto(297.95419096,191.56289496)(298.27927776,191.17400608)(298.50410414,190.72435332)
\curveto(298.73500691,190.27470055)(298.8504583,189.80378042)(298.8504583,189.31159294)
\curveto(298.8504583,188.37582907)(298.48587498,187.56766937)(297.75670833,186.88711383)
\curveto(297.03361807,186.21263468)(296.1008924,185.8753951)(294.95853131,185.8753951)
\curveto(294.60002438,185.8753951)(294.2627848,185.90273885)(293.94681259,185.95742635)
\curveto(293.75844454,185.9878083)(293.36651746,186.09718329)(292.77103137,186.28555134)
\curveto(292.18162166,186.47999578)(291.80792375,186.577218)(291.64993765,186.577218)
\curveto(291.49802793,186.577218)(291.37650015,186.53164509)(291.28535432,186.44049926)
\curveto(291.20028488,186.34935343)(291.1364828,186.16098538)(291.09394808,185.8753951)
\lineto(290.7567085,185.8753951)
\lineto(290.7567085,190.11367625)
\lineto(291.09394808,190.11367625)
\curveto(291.25193418,189.22652349)(291.46460779,188.56115893)(291.73196889,188.11758255)
\curveto(291.99933,187.68008256)(292.40644804,187.31549924)(292.95332303,187.02383258)
\curveto(293.50627441,186.73216592)(294.11087508,186.58633259)(294.76712507,186.58633259)
\curveto(295.52667366,186.58633259)(296.12519795,186.78685342)(296.56269794,187.18789507)
\curveto(297.00627432,187.58893673)(297.22806251,188.06289505)(297.22806251,188.60977004)
\curveto(297.22806251,188.91358947)(297.14299306,189.2204471)(296.97285418,189.53034293)
\curveto(296.80879168,189.84023876)(296.55054516,190.12886722)(296.19811462,190.39622832)
\curveto(295.96113546,190.57851999)(295.31400005,190.96437067)(294.25670841,191.55378038)
\curveto(293.19941677,192.14926648)(292.44594457,192.6232248)(291.9962918,192.97565534)
\curveto(291.55271543,193.32808589)(291.21547585,193.71697477)(290.98457308,194.14232198)
\curveto(290.75367031,194.56766919)(290.63821892,195.03555113)(290.63821892,195.54596778)
\curveto(290.63821892,196.43312054)(290.97849669,197.19570732)(291.65905223,197.83372814)
\curveto(292.33960777,198.47782535)(293.20549316,198.79987395)(294.25670841,198.79987395)
\curveto(294.9129584,198.79987395)(295.60870491,198.63884965)(296.34394795,198.31680104)
\curveto(296.68422571,198.16489133)(296.92424307,198.08893647)(297.06400001,198.08893647)
\curveto(297.22198612,198.08893647)(297.34959028,198.13450938)(297.4468125,198.22565521)
\curveto(297.55011111,198.32287743)(297.63214236,198.51428368)(297.69290625,198.79987395)
\closepath
}
}
{
\newrgbcolor{curcolor}{0 0 0}
\pscustom[linestyle=none,fillstyle=solid,fillcolor=curcolor]
{
\newpath
\moveto(302.86998945,192.98476993)
\curveto(303.84828804,194.16358934)(304.78101371,194.75299905)(305.66816646,194.75299905)
\curveto(306.12389562,194.75299905)(306.51582269,194.63754766)(306.84394768,194.40664489)
\curveto(307.17207267,194.18181851)(307.43335739,193.8081206)(307.62780183,193.28555117)
\curveto(307.76148238,192.92096785)(307.82832266,192.36194008)(307.82832266,191.60846788)
\lineto(307.82832266,188.04466588)
\curveto(307.82832266,187.51602006)(307.87085738,187.15751313)(307.95592682,186.96914508)
\curveto(308.0227671,186.81723536)(308.1291039,186.69874578)(308.27493723,186.61367634)
\curveto(308.42684695,186.52860689)(308.70332264,186.48607217)(309.10436429,186.48607217)
\lineto(309.10436429,186.15794718)
\lineto(304.97545815,186.15794718)
\lineto(304.97545815,186.48607217)
\lineto(305.14863523,186.48607217)
\curveto(305.5375241,186.48607217)(305.8079234,186.54379787)(305.95983312,186.65924925)
\curveto(306.11781923,186.78077703)(306.22719423,186.9569923)(306.28795811,187.18789507)
\curveto(306.31226367,187.2790409)(306.32441645,187.56463117)(306.32441645,188.04466588)
\lineto(306.32441645,191.46263455)
\curveto(306.32441645,192.22218314)(306.22415603,192.77209632)(306.0236352,193.11237409)
\curveto(305.82919076,193.45872825)(305.49802758,193.63190533)(305.03014565,193.63190533)
\curveto(304.30705539,193.63190533)(303.58700332,193.23694006)(302.86998945,192.44700952)
\lineto(302.86998945,188.04466588)
\curveto(302.86998945,187.47956173)(302.90340959,187.13016938)(302.97024986,186.99648883)
\curveto(303.05531931,186.82027355)(303.17077069,186.6896312)(303.31660402,186.60456175)
\curveto(303.46851374,186.5255687)(303.77233318,186.48607217)(304.22806233,186.48607217)
\lineto(304.22806233,186.15794718)
\lineto(300.09915618,186.15794718)
\lineto(300.09915618,186.48607217)
\lineto(300.28144785,186.48607217)
\curveto(300.70679506,186.48607217)(300.99238533,186.59240898)(301.13821866,186.80508258)
\curveto(301.29012838,187.02383258)(301.36608324,187.43702701)(301.36608324,188.04466588)
\lineto(301.36608324,191.14362414)
\curveto(301.36608324,192.14622828)(301.34177768,192.75690535)(301.29316657,192.97565534)
\curveto(301.25063185,193.19440534)(301.18075338,193.34327686)(301.08353116,193.42226992)
\curveto(300.99238533,193.50126297)(300.86781936,193.5407595)(300.70983325,193.5407595)
\curveto(300.53969437,193.5407595)(300.33613535,193.49518658)(300.09915618,193.40404075)
\lineto(299.96243744,193.73216574)
\lineto(302.47806238,194.75299905)
\lineto(302.86998945,194.75299905)
\closepath
}
}
{
\newrgbcolor{curcolor}{0 0 0}
\pscustom[linestyle=none,fillstyle=solid,fillcolor=curcolor]
{
\newpath
\moveto(314.50019749,187.36107215)
\curveto(313.64342668,186.69874578)(313.10566628,186.31593329)(312.88691628,186.21263468)
\curveto(312.55879129,186.06072496)(312.20939894,185.9847701)(311.83873922,185.9847701)
\curveto(311.26148229,185.9847701)(310.78448578,186.18225274)(310.40774968,186.577218)
\curveto(310.03708996,186.97218327)(309.85176011,187.49171451)(309.85176011,188.13581171)
\curveto(309.85176011,188.54292976)(309.94290594,188.89536031)(310.1251976,189.19310335)
\curveto(310.37432954,189.60629779)(310.80575314,189.99518667)(311.4194684,190.35976999)
\curveto(312.03926005,190.72435332)(313.06616975,191.16792969)(314.50019749,191.69049913)
\lineto(314.50019749,192.01862412)
\curveto(314.50019749,192.85108937)(314.36651694,193.42226992)(314.09915583,193.73216574)
\curveto(313.83787112,194.04206157)(313.45505863,194.19700948)(312.95071836,194.19700948)
\curveto(312.56790587,194.19700948)(312.26408644,194.09371087)(312.03926005,193.88711365)
\curveto(311.80835728,193.68051644)(311.6929059,193.44353728)(311.6929059,193.17617617)
\lineto(311.71113506,192.64753035)
\curveto(311.71113506,192.36801647)(311.6382184,192.15230467)(311.49238507,192.00039495)
\curveto(311.35262813,191.84848523)(311.16729827,191.77253037)(310.9363955,191.77253037)
\curveto(310.71156911,191.77253037)(310.52623926,191.85152343)(310.38040593,192.00950953)
\curveto(310.24064899,192.16749564)(310.17077052,192.38320744)(310.17077052,192.65664493)
\curveto(310.17077052,193.17921437)(310.43813162,193.65924908)(310.97285383,194.09674907)
\curveto(311.50757604,194.53424905)(312.25801005,194.75299905)(313.22415586,194.75299905)
\curveto(313.96547528,194.75299905)(314.57311416,194.62843308)(315.04707248,194.37930114)
\curveto(315.40557941,194.19093309)(315.66990232,193.89622824)(315.84004121,193.49518658)
\curveto(315.94941621,193.23390187)(316.0041037,192.69917966)(316.0041037,191.89101995)
\lineto(316.0041037,189.05638461)
\curveto(316.0041037,188.26037768)(316.01929468,187.77122839)(316.04967662,187.58893673)
\curveto(316.08005856,187.41272145)(316.12866967,187.29423187)(316.19550995,187.23346799)
\curveto(316.26842661,187.1727041)(316.35045786,187.14232216)(316.44160369,187.14232216)
\curveto(316.53882591,187.14232216)(316.62389536,187.16358952)(316.69681202,187.20612424)
\curveto(316.82441618,187.28511729)(317.07050993,187.50690548)(317.43509325,187.8714888)
\lineto(317.43509325,187.36107215)
\curveto(316.75453771,186.44961384)(316.10436412,185.99388468)(315.48457247,185.99388468)
\curveto(315.18682942,185.99388468)(314.94985026,186.09718329)(314.77363498,186.30378051)
\curveto(314.59741971,186.51037773)(314.50627388,186.86280827)(314.50019749,187.36107215)
\closepath
\moveto(314.50019749,187.95352005)
\lineto(314.50019749,191.13450956)
\curveto(313.58266279,190.76992623)(312.99021489,190.51167971)(312.72285379,190.35976999)
\curveto(312.24281908,190.09240889)(311.89950311,189.81289501)(311.6929059,189.52122835)
\curveto(311.48630868,189.22956169)(311.38301007,188.91055128)(311.38301007,188.56419712)
\curveto(311.38301007,188.12669713)(311.51365243,187.76211381)(311.77493714,187.47044715)
\curveto(312.03622186,187.18485688)(312.3370031,187.04206174)(312.67728087,187.04206174)
\curveto(313.13908641,187.04206174)(313.74672529,187.34588118)(314.50019749,187.95352005)
\closepath
}
}
{
\newrgbcolor{curcolor}{0 0 0}
\pscustom[linestyle=none,fillstyle=solid,fillcolor=curcolor]
{
\newpath
\moveto(317.45332242,193.67747824)
\lineto(320.02363485,194.71654072)
\lineto(320.36998901,194.71654072)
\lineto(320.36998901,192.76601993)
\curveto(320.80141261,193.50126297)(321.23283621,194.01471782)(321.66425981,194.30638448)
\curveto(322.1017598,194.60412753)(322.56052715,194.75299905)(323.04056186,194.75299905)
\curveto(323.87910351,194.75299905)(324.57788821,194.42487406)(325.13691598,193.76862407)
\curveto(325.8235479,192.96654076)(326.16686387,191.9214019)(326.16686387,190.63320749)
\curveto(326.16686387,189.19310335)(325.75366943,188.00213116)(324.92728056,187.06029091)
\curveto(324.24672503,186.28858954)(323.38995421,185.90273885)(322.35696813,185.90273885)
\curveto(321.90731536,185.90273885)(321.51842648,185.96654094)(321.19030149,186.0941451)
\curveto(320.94724594,186.18529093)(320.67380845,186.36758259)(320.36998901,186.64102009)
\lineto(320.36998901,184.0980514)
\curveto(320.36998901,183.52687086)(320.40340915,183.16532573)(320.47024943,183.01341601)
\curveto(320.54316609,182.8554299)(320.66469387,182.73086393)(320.83483275,182.6397181)
\curveto(321.01104802,182.54857227)(321.32702024,182.50299936)(321.78274939,182.50299936)
\lineto(321.78274939,182.16575978)
\lineto(317.4077495,182.16575978)
\lineto(317.4077495,182.50299936)
\lineto(317.63561408,182.50299936)
\curveto(317.96981546,182.49692297)(318.25540573,182.56072505)(318.49238489,182.6944056)
\curveto(318.60783628,182.76124588)(318.69594391,182.87062087)(318.7567078,183.02253059)
\curveto(318.82354808,183.16836392)(318.85696822,183.54510002)(318.85696822,184.1527389)
\lineto(318.85696822,192.04596787)
\curveto(318.85696822,192.58676646)(318.83266266,192.93008243)(318.78405155,193.07591576)
\curveto(318.73544044,193.22174909)(318.65644739,193.33112408)(318.54707239,193.40404075)
\curveto(318.44377378,193.47695741)(318.30097865,193.51341575)(318.11868698,193.51341575)
\curveto(317.97285365,193.51341575)(317.7875238,193.47088103)(317.56269742,193.38581158)
\closepath
\moveto(320.36998901,192.22825953)
\lineto(320.36998901,189.11107211)
\curveto(320.36998901,188.43659296)(320.39733276,187.99301658)(320.45202026,187.78034297)
\curveto(320.5370897,187.42791243)(320.74368692,187.1180166)(321.07181191,186.8506555)
\curveto(321.40601329,186.58329439)(321.82528411,186.44961384)(322.32962438,186.44961384)
\curveto(322.93726325,186.44961384)(323.42945074,186.686593)(323.80618684,187.16055132)
\curveto(324.29837433,187.78034297)(324.54446807,188.65230476)(324.54446807,189.77643667)
\curveto(324.54446807,191.05247831)(324.26495419,192.03381509)(323.70592643,192.72044702)
\curveto(323.31703755,193.19440534)(322.855232,193.4313845)(322.3205098,193.4313845)
\curveto(322.02884314,193.4313845)(321.74021467,193.35846783)(321.4546244,193.2126345)
\curveto(321.23587441,193.10325951)(320.87432928,192.77513452)(320.36998901,192.22825953)
\closepath
}
}
{
\newrgbcolor{curcolor}{0 0 0}
\pscustom[linestyle=none,fillstyle=solid,fillcolor=curcolor]
{
\newpath
\moveto(340.03014477,188.53685337)
\lineto(339.16425938,186.15794718)
\lineto(331.87259289,186.15794718)
\lineto(331.87259289,186.49518676)
\curveto(334.01755811,188.45178393)(335.52754072,190.04987417)(336.40254069,191.28945747)
\curveto(337.27754067,192.52904077)(337.71504066,193.66228727)(337.71504066,194.68919697)
\curveto(337.71504066,195.47305111)(337.47502331,196.11714832)(336.9949886,196.62148859)
\curveto(336.51495389,197.12582885)(335.94073515,197.37799898)(335.27233239,197.37799898)
\curveto(334.66469352,197.37799898)(334.11781853,197.19874552)(333.63170743,196.84023858)
\curveto(333.15167272,196.48780803)(332.79620398,195.9682768)(332.56530121,195.28164487)
\lineto(332.22806163,195.28164487)
\curveto(332.37997135,196.40577679)(332.76886023,197.26862399)(333.39472827,197.87018647)
\curveto(334.0266727,198.47174896)(334.81356504,198.7725302)(335.75540529,198.7725302)
\curveto(336.75800944,198.7725302)(337.59351289,198.4504816)(338.26191565,197.80638439)
\curveto(338.9363948,197.16228718)(339.27363437,196.40273859)(339.27363437,195.52773861)
\curveto(339.27363437,194.90187057)(339.12780104,194.27600253)(338.83613438,193.65013449)
\curveto(338.38648162,192.66575952)(337.65731497,191.62365885)(336.64863444,190.52383249)
\curveto(335.13561364,188.87105475)(334.19073519,187.874527)(333.81399909,187.53424923)
\lineto(337.04056151,187.53424923)
\curveto(337.6968115,187.53424923)(338.15557884,187.55855478)(338.41686356,187.60716589)
\curveto(338.68422466,187.655777)(338.92424202,187.75299922)(339.13691563,187.89883255)
\curveto(339.34958923,188.05074227)(339.53491909,188.26341588)(339.6929052,188.53685337)
\closepath
}
}
{
\newrgbcolor{curcolor}{0 0 0}
\pscustom[linestyle=none,fillstyle=solid,fillcolor=curcolor]
{
\newpath
\moveto(349.3452487,190.29596791)
\lineto(344.56009257,190.29596791)
\lineto(343.72155093,188.34544713)
\curveto(343.51495371,187.86541242)(343.4116551,187.50690548)(343.4116551,187.26992632)
\curveto(343.4116551,187.08155827)(343.49976274,186.91445758)(343.67597801,186.76862425)
\curveto(343.85826967,186.62886731)(344.24715855,186.53772148)(344.84264465,186.49518676)
\lineto(344.84264465,186.15794718)
\lineto(340.95071766,186.15794718)
\lineto(340.95071766,186.49518676)
\curveto(341.46721071,186.58633259)(341.80141209,186.70482217)(341.95332181,186.8506555)
\curveto(342.26321763,187.14232216)(342.60653359,187.73477006)(342.9832697,188.6279992)
\lineto(347.33092584,198.79987395)
\lineto(347.64993625,198.79987395)
\lineto(351.95201947,188.51862421)
\curveto(352.29837363,187.69223534)(352.61130765,187.15447493)(352.89082153,186.905343)
\curveto(353.1764118,186.66228745)(353.57137707,186.5255687)(354.07571734,186.49518676)
\lineto(354.07571734,186.15794718)
\lineto(349.19941537,186.15794718)
\lineto(349.19941537,186.49518676)
\curveto(349.69160286,186.51949231)(350.02276605,186.60152356)(350.19290493,186.7412805)
\curveto(350.36912021,186.88103744)(350.45722784,187.05117633)(350.45722784,187.25169715)
\curveto(350.45722784,187.51905826)(350.33570007,187.94136728)(350.09264452,188.51862421)
\closepath
\moveto(349.09004038,190.97044706)
\lineto(346.99368626,195.9652386)
\lineto(344.84264465,190.97044706)
\closepath
}
}
{
\newrgbcolor{curcolor}{0 0 0}
\pscustom[linewidth=0.6913773,linecolor=curcolor]
{
\newpath
\moveto(137.49785017,176.78200268)
\lineto(61.10605208,148.08113673)
}
}
{
\newrgbcolor{curcolor}{0 0 0}
\pscustom[linestyle=none,fillstyle=solid,fillcolor=curcolor]
{
\newpath
\moveto(67.57811579,150.51273066)
\lineto(71.13957885,148.89654275)
\lineto(61.10605208,148.08113673)
\lineto(69.1943037,154.07419372)
\closepath
}
}
{
\newrgbcolor{curcolor}{0 0 0}
\pscustom[linewidth=0.73746915,linecolor=curcolor]
{
\newpath
\moveto(67.57811579,150.51273066)
\lineto(71.13957885,148.89654275)
\lineto(61.10605208,148.08113673)
\lineto(69.1943037,154.07419372)
\closepath
}
}
{
\newrgbcolor{curcolor}{0 0 0}
\pscustom[linewidth=0.95949735,linecolor=curcolor]
{
\newpath
\moveto(326.54872545,295.9227997)
\lineto(326.54872545,268.16069372)
}
}
{
\newrgbcolor{curcolor}{0 0 0}
\pscustom[linestyle=none,fillstyle=solid,fillcolor=curcolor]
{
\newpath
\moveto(326.54872545,277.75566721)
\lineto(330.38671484,281.5936566)
\lineto(326.54872545,268.16069372)
\lineto(322.71073605,281.5936566)
\closepath
}
}
{
\newrgbcolor{curcolor}{0 0 0}
\pscustom[linewidth=1.02346387,linecolor=curcolor]
{
\newpath
\moveto(326.54872545,277.75566721)
\lineto(330.38671484,281.5936566)
\lineto(326.54872545,268.16069372)
\lineto(322.71073605,281.5936566)
\closepath
}
}
{
\newrgbcolor{curcolor}{0 0 0}
\pscustom[linewidth=0.95949735,linecolor=curcolor]
{
\newpath
\moveto(325.59140547,176.71128268)
\lineto(325.59140547,148.9491767)
}
}
{
\newrgbcolor{curcolor}{0 0 0}
\pscustom[linestyle=none,fillstyle=solid,fillcolor=curcolor]
{
\newpath
\moveto(325.59140547,158.54415019)
\lineto(329.42939486,162.38213958)
\lineto(325.59140547,148.9491767)
\lineto(321.75341608,162.38213958)
\closepath
}
}
{
\newrgbcolor{curcolor}{0 0 0}
\pscustom[linewidth=1.02346387,linecolor=curcolor]
{
\newpath
\moveto(325.59140547,158.54415019)
\lineto(329.42939486,162.38213958)
\lineto(325.59140547,148.9491767)
\lineto(321.75341608,162.38213958)
\closepath
}
}
{
\newrgbcolor{curcolor}{0 0 0}
\pscustom[linewidth=0.74666665,linecolor=curcolor]
{
\newpath
\moveto(4.06626684,120.59955113)
\lineto(114.85013074,120.59955113)
\lineto(114.85013074,148.63128377)
\lineto(4.06626684,148.63128377)
\closepath
}
}
{
\newrgbcolor{curcolor}{0 0 0}
\pscustom[linestyle=none,fillstyle=solid,fillcolor=curcolor]
{
\newpath
\moveto(28.29087301,138.73976097)
\curveto(29.55476186,138.73976097)(30.56951878,138.25972626)(31.33514376,137.29965684)
\curveto(31.98531736,136.47934436)(32.31040416,135.5375041)(32.31040416,134.47413607)
\curveto(32.31040416,133.72674026)(32.13115069,132.97022986)(31.77264375,132.20460488)
\curveto(31.41413682,131.4389799)(30.91891113,130.86172297)(30.28696671,130.47283409)
\curveto(29.66109867,130.08394521)(28.96231396,129.88950077)(28.19061259,129.88950077)
\curveto(26.93280012,129.88950077)(25.93323418,130.39080284)(25.19191475,131.39340698)
\curveto(24.56604671,132.23802502)(24.25311269,133.18594166)(24.25311269,134.23715691)
\curveto(24.25311269,135.00278189)(24.44148074,135.76233049)(24.81821684,136.51580269)
\curveto(25.20102933,137.27535128)(25.7023314,137.83437904)(26.32212306,138.19288598)
\curveto(26.94191471,138.5574693)(27.59816469,138.73976097)(28.29087301,138.73976097)
\closepath
\moveto(28.00832093,138.14731306)
\curveto(27.68627233,138.14731306)(27.36118553,138.05009084)(27.03306054,137.85564641)
\curveto(26.71101193,137.66727835)(26.44972722,137.33307697)(26.24920639,136.85304226)
\curveto(26.04868556,136.37300755)(25.94842515,135.7562541)(25.94842515,135.00278189)
\curveto(25.94842515,133.78750415)(26.1884425,132.73932709)(26.66847721,131.85825072)
\curveto(27.15458831,130.97717435)(27.79260913,130.53663617)(28.58253967,130.53663617)
\curveto(29.17194937,130.53663617)(29.65806047,130.77969172)(30.04087296,131.26580282)
\curveto(30.42368545,131.75191392)(30.6150917,132.58741737)(30.6150917,133.77231317)
\curveto(30.6150917,135.25495203)(30.29608129,136.42161866)(29.65806047,137.27231309)
\curveto(29.22663687,137.85564641)(28.67672369,138.14731306)(28.00832093,138.14731306)
\closepath
}
}
{
\newrgbcolor{curcolor}{0 0 0}
\pscustom[linestyle=none,fillstyle=solid,fillcolor=curcolor]
{
\newpath
\moveto(35.82863323,137.05356309)
\curveto(36.63679294,138.17769501)(37.50875472,138.73976097)(38.44451859,138.73976097)
\curveto(39.3012894,138.73976097)(40.04868521,138.37213945)(40.68670603,137.63689641)
\curveto(41.32472685,136.90772976)(41.64373726,135.90816382)(41.64373726,134.63819857)
\curveto(41.64373726,133.15555972)(41.15154977,131.96154933)(40.16717479,131.05616741)
\curveto(39.32255676,130.27838965)(38.3807165,129.88950077)(37.34165403,129.88950077)
\curveto(36.85554293,129.88950077)(36.36031725,129.97760841)(35.85597698,130.15382368)
\curveto(35.35771311,130.33003895)(34.84729645,130.59436186)(34.32472702,130.94679241)
\lineto(34.32472702,139.59653178)
\curveto(34.32472702,140.54444842)(34.30042147,141.12778174)(34.25181036,141.34653173)
\curveto(34.20927564,141.56528173)(34.13939717,141.71415325)(34.04217495,141.79314631)
\curveto(33.94495273,141.87213936)(33.82342495,141.91163589)(33.67759162,141.91163589)
\curveto(33.50745274,141.91163589)(33.29477913,141.86302478)(33.0395708,141.76580256)
\lineto(32.91196664,142.08481297)
\lineto(35.41847699,143.10564627)
\lineto(35.82863323,143.10564627)
\closepath
\moveto(35.82863323,136.47022977)
\lineto(35.82863323,131.47543823)
\curveto(36.13852906,131.17161879)(36.45753947,130.94071602)(36.78566446,130.78272992)
\curveto(37.11986584,130.6308202)(37.46014361,130.55486534)(37.80649777,130.55486534)
\curveto(38.35944914,130.55486534)(38.87290399,130.85868477)(39.34686231,131.46632365)
\curveto(39.82689702,132.07396252)(40.06691438,132.95807708)(40.06691438,134.11866733)
\curveto(40.06691438,135.18811175)(39.82689702,136.00842423)(39.34686231,136.57960477)
\curveto(38.87290399,137.1568617)(38.33210539,137.44549017)(37.72446652,137.44549017)
\curveto(37.40241792,137.44549017)(37.08036931,137.36345892)(36.75832071,137.19939642)
\curveto(36.51526516,137.07786865)(36.20536934,136.8348131)(35.82863323,136.47022977)
\closepath
}
}
{
\newrgbcolor{curcolor}{0 0 0}
\pscustom[linestyle=none,fillstyle=solid,fillcolor=curcolor]
{
\newpath
\moveto(44.98878926,143.11476086)
\curveto(45.25007397,143.11476086)(45.47186216,143.02361503)(45.65415382,142.84132336)
\curveto(45.83644548,142.6590317)(45.92759132,142.43724351)(45.92759132,142.1759588)
\curveto(45.92759132,141.92075047)(45.83644548,141.70200048)(45.65415382,141.51970881)
\curveto(45.47186216,141.33741715)(45.25007397,141.24627132)(44.98878926,141.24627132)
\curveto(44.73358093,141.24627132)(44.51483093,141.33741715)(44.33253927,141.51970881)
\curveto(44.15024761,141.70200048)(44.05910178,141.92075047)(44.05910178,142.1759588)
\curveto(44.05910178,142.43724351)(44.15024761,142.6590317)(44.33253927,142.84132336)
\curveto(44.51483093,143.02361503)(44.73358093,143.11476086)(44.98878926,143.11476086)
\closepath
\moveto(45.7726434,138.73976097)
\lineto(45.7726434,130.33611534)
\curveto(45.7726434,128.90816399)(45.46882397,127.84783416)(44.86118509,127.15512584)
\curveto(44.25354622,126.46241752)(43.46361568,126.11606337)(42.49139348,126.11606337)
\curveto(41.93844211,126.11606337)(41.52828587,126.21632378)(41.26092477,126.41684461)
\curveto(40.99356366,126.61736544)(40.85988311,126.82396265)(40.85988311,127.03663626)
\curveto(40.85988311,127.24930986)(40.93583797,127.43160153)(41.08774769,127.58351125)
\curveto(41.23358102,127.73542096)(41.40675809,127.81137582)(41.60727892,127.81137582)
\curveto(41.76526503,127.81137582)(41.92628933,127.7718793)(42.09035183,127.69288624)
\curveto(42.19365044,127.65035152)(42.39113307,127.4984418)(42.68279973,127.23715709)
\curveto(42.98054278,126.96979598)(43.22967472,126.83611543)(43.43019554,126.83611543)
\curveto(43.57602887,126.83611543)(43.71882401,126.89384112)(43.85858095,127.00929251)
\curveto(43.99833789,127.11866751)(44.1016365,127.30703556)(44.16847678,127.57439666)
\curveto(44.23531705,127.83568138)(44.26873719,128.40686192)(44.26873719,129.28793829)
\lineto(44.26873719,135.23064647)
\curveto(44.26873719,136.14818117)(44.24139344,136.73759088)(44.18670594,136.99887559)
\curveto(44.14417122,137.19939642)(44.07733094,137.33611517)(43.98618511,137.40903183)
\curveto(43.89503928,137.48802489)(43.77047331,137.52752141)(43.61248721,137.52752141)
\curveto(43.44234832,137.52752141)(43.2357511,137.4819485)(42.99269556,137.39080267)
\lineto(42.86509139,137.71892766)
\lineto(45.38071633,138.73976097)
\closepath
}
}
{
\newrgbcolor{curcolor}{0 0 0}
\pscustom[linestyle=none,fillstyle=solid,fillcolor=curcolor]
{
\newpath
\moveto(49.46404956,135.34913605)
\curveto(49.45797317,134.10955275)(49.75875441,133.13733055)(50.36639329,132.43246946)
\curveto(50.97403216,131.72760836)(51.68800784,131.37517782)(52.50832032,131.37517782)
\curveto(53.0551953,131.37517782)(53.52915363,131.52404934)(53.93019528,131.82179239)
\curveto(54.33731333,132.12561183)(54.6775911,132.64210487)(54.95102859,133.37127152)
\lineto(55.23358067,133.18897986)
\curveto(55.1059765,132.3565146)(54.73531679,131.59696601)(54.12160153,130.91033408)
\curveto(53.50788626,130.22977854)(52.73922309,129.88950077)(51.815612,129.88950077)
\curveto(50.81300786,129.88950077)(49.95319885,130.27838965)(49.23618498,131.05616741)
\curveto(48.5252475,131.84002156)(48.16977876,132.89123681)(48.16977876,134.20981316)
\curveto(48.16977876,135.63776452)(48.53436208,136.74974366)(49.26352873,137.54575058)
\curveto(49.99877177,138.34783389)(50.91934466,138.74887555)(52.02524741,138.74887555)
\curveto(52.96101128,138.74887555)(53.72967445,138.43897972)(54.33123694,137.81918807)
\curveto(54.93279942,137.20547281)(55.23358067,136.38212214)(55.23358067,135.34913605)
\closepath
\moveto(49.46404956,135.87778187)
\lineto(53.3286328,135.87778187)
\curveto(53.29825085,136.41250408)(53.23444877,136.78924018)(53.13722655,137.00799018)
\curveto(52.98531683,137.34826795)(52.75745226,137.61562905)(52.45363282,137.81007349)
\curveto(52.15588977,138.00451793)(51.84295575,138.10174015)(51.51483076,138.10174015)
\curveto(51.01049049,138.10174015)(50.55779953,137.90425752)(50.15675788,137.50929225)
\curveto(49.76179261,137.12040337)(49.53088984,136.57656658)(49.46404956,135.87778187)
\closepath
}
}
{
\newrgbcolor{curcolor}{0 0 0}
\pscustom[linestyle=none,fillstyle=solid,fillcolor=curcolor]
{
\newpath
\moveto(63.43670546,133.31658402)
\curveto(63.21187908,132.21675766)(62.77134089,131.36910143)(62.11509091,130.77361533)
\curveto(61.45884093,130.18420562)(60.73271247,129.88950077)(59.93670555,129.88950077)
\curveto(58.98878891,129.88950077)(58.16240004,130.28750423)(57.45753894,131.08351116)
\curveto(56.75267785,131.87951808)(56.4002473,132.95503889)(56.4002473,134.31007358)
\curveto(56.4002473,135.62257354)(56.78913618,136.68897977)(57.56691394,137.50929225)
\curveto(58.35076809,138.32960473)(59.28957015,138.73976097)(60.38332012,138.73976097)
\curveto(61.2036326,138.73976097)(61.87811175,138.52101097)(62.40675757,138.08351098)
\curveto(62.93540339,137.65208738)(63.1997263,137.20243462)(63.1997263,136.73455268)
\curveto(63.1997263,136.50364991)(63.12377144,136.31528186)(62.97186172,136.16944853)
\curveto(62.82602839,136.02969159)(62.61943118,135.95981312)(62.35207007,135.95981312)
\curveto(61.99356314,135.95981312)(61.72316384,136.07526451)(61.54087217,136.30616728)
\curveto(61.43757357,136.43377144)(61.3676951,136.67682699)(61.33123676,137.03533393)
\curveto(61.30085482,137.39384086)(61.17932705,137.66727835)(60.96665344,137.85564641)
\curveto(60.75397983,138.03793807)(60.45927498,138.1290839)(60.08253888,138.1290839)
\curveto(59.4749,138.1290839)(58.98575071,137.90425752)(58.615091,137.45460475)
\curveto(58.12290351,136.85911865)(57.87680977,136.07222631)(57.87680977,135.09392772)
\curveto(57.87680977,134.09739997)(58.11986532,133.2163236)(58.60597642,132.45069862)
\curveto(59.0981639,131.69115003)(59.76049028,131.31137574)(60.59295553,131.31137574)
\curveto(61.18844163,131.31137574)(61.72316384,131.51493476)(62.19712216,131.9220528)
\curveto(62.53132354,132.20156669)(62.85641034,132.70894515)(63.17238255,133.44418818)
\closepath
}
}
{
\newrgbcolor{curcolor}{0 0 0}
\pscustom[linestyle=none,fillstyle=solid,fillcolor=curcolor]
{
\newpath
\moveto(67.05519495,141.23715674)
\lineto(67.05519495,138.49366722)
\lineto(69.00571574,138.49366722)
\lineto(69.00571574,137.85564641)
\lineto(67.05519495,137.85564641)
\lineto(67.05519495,132.44158404)
\curveto(67.05519495,131.90078544)(67.13114981,131.53620212)(67.28305953,131.34783407)
\curveto(67.44104564,131.15946602)(67.64156647,131.06528199)(67.88462202,131.06528199)
\curveto(68.08514284,131.06528199)(68.27958728,131.12604588)(68.46795534,131.24757365)
\curveto(68.65632339,131.37517782)(68.80215672,131.56050767)(68.90545532,131.80356322)
\lineto(69.26092407,131.80356322)
\curveto(69.04825046,131.20807713)(68.74746922,130.75842436)(68.35858034,130.45460492)
\curveto(67.96969146,130.15686188)(67.5686498,130.00799035)(67.15545537,130.00799035)
\curveto(66.87594149,130.00799035)(66.60250399,130.08394521)(66.33514289,130.23585493)
\curveto(66.06778178,130.39384104)(65.87029915,130.61562923)(65.74269499,130.9012195)
\curveto(65.61509082,131.19288616)(65.55128874,131.63950073)(65.55128874,132.24106321)
\lineto(65.55128874,137.85564641)
\lineto(64.22967419,137.85564641)
\lineto(64.22967419,138.15642765)
\curveto(64.56387557,138.2901082)(64.90415334,138.51493458)(65.2505075,138.8309068)
\curveto(65.60293805,139.1529554)(65.91587207,139.5327297)(66.18930956,139.97022969)
\curveto(66.3290665,140.20113246)(66.52351094,140.62344147)(66.77264288,141.23715674)
\closepath
}
}
{
\newrgbcolor{curcolor}{0 0 0}
\pscustom[linestyle=none,fillstyle=solid,fillcolor=curcolor]
{
\newpath
\moveto(79.87941338,138.73976097)
\lineto(79.87941338,135.89601104)
\lineto(79.57863214,135.89601104)
\curveto(79.34772937,136.78924018)(79.04998632,137.39687906)(78.685403,137.71892766)
\curveto(78.32689606,138.04097626)(77.86812871,138.20200056)(77.30910095,138.20200056)
\curveto(76.88375374,138.20200056)(76.54043777,138.08958737)(76.27915306,137.86476099)
\curveto(76.01786834,137.63993461)(75.88722598,137.39080267)(75.88722598,137.11736517)
\curveto(75.88722598,136.7770874)(75.9844482,136.48542075)(76.17889264,136.2423652)
\curveto(76.36726069,135.99323326)(76.75007318,135.72891035)(77.32733011,135.44939647)
\lineto(78.65805925,134.80226106)
\curveto(79.89156616,134.20069858)(80.50831962,133.40772985)(80.50831962,132.42335487)
\curveto(80.50831962,131.66380628)(80.21969115,131.05009102)(79.64243422,130.58220909)
\curveto(79.07125368,130.12040354)(78.43019467,129.88950077)(77.71925719,129.88950077)
\curveto(77.20884053,129.88950077)(76.62550721,129.9806466)(75.96925723,130.16293826)
\curveto(75.7687364,130.22370215)(75.60467391,130.2540841)(75.47706974,130.2540841)
\curveto(75.3373128,130.2540841)(75.22793781,130.17509104)(75.14894475,130.01710493)
\lineto(74.84816351,130.01710493)
\lineto(74.84816351,132.99757361)
\lineto(75.14894475,132.99757361)
\curveto(75.31908364,132.14687919)(75.64417043,131.50582018)(76.12420514,131.07439657)
\curveto(76.60423985,130.64297297)(77.14200026,130.42726117)(77.73748635,130.42726117)
\curveto(78.15675718,130.42726117)(78.49703495,130.54878895)(78.75831966,130.7918445)
\curveto(79.02568077,131.04097644)(79.15936132,131.33871948)(79.15936132,131.68507364)
\curveto(79.15936132,132.10434447)(79.01048979,132.45677501)(78.71274675,132.74236528)
\curveto(78.42108009,133.02795555)(77.83470857,133.38950068)(76.95363221,133.82700067)
\curveto(76.07255584,134.26450066)(75.49529891,134.65946593)(75.22186142,135.01189648)
\curveto(74.94842392,135.35825063)(74.81170518,135.79575062)(74.81170518,136.32439644)
\curveto(74.81170518,137.01102837)(75.04564614,137.58524711)(75.51352808,138.04705265)
\curveto(75.9874864,138.50885819)(76.59816347,138.73976097)(77.34555928,138.73976097)
\curveto(77.67368427,138.73976097)(78.07168773,138.6698825)(78.53956967,138.53012556)
\curveto(78.84946549,138.43897972)(79.05606271,138.39340681)(79.15936132,138.39340681)
\curveto(79.25658354,138.39340681)(79.3325384,138.41467417)(79.3872259,138.45720889)
\curveto(79.44191339,138.49974361)(79.50571548,138.59392764)(79.57863214,138.73976097)
\closepath
}
}
{
\newrgbcolor{curcolor}{0 0 0}
\pscustom[linestyle=none,fillstyle=solid,fillcolor=curcolor]
{
\newpath
\moveto(83.15154872,135.34913605)
\curveto(83.14547233,134.10955275)(83.44625357,133.13733055)(84.05389245,132.43246946)
\curveto(84.66153132,131.72760836)(85.375507,131.37517782)(86.19581948,131.37517782)
\curveto(86.74269446,131.37517782)(87.21665278,131.52404934)(87.61769444,131.82179239)
\curveto(88.02481249,132.12561183)(88.36509025,132.64210487)(88.63852775,133.37127152)
\lineto(88.92107982,133.18897986)
\curveto(88.79347566,132.3565146)(88.42281595,131.59696601)(87.80910068,130.91033408)
\curveto(87.19538542,130.22977854)(86.42672225,129.88950077)(85.50311116,129.88950077)
\curveto(84.50050702,129.88950077)(83.64069801,130.27838965)(82.92368414,131.05616741)
\curveto(82.21274666,131.84002156)(81.85727792,132.89123681)(81.85727792,134.20981316)
\curveto(81.85727792,135.63776452)(82.22186124,136.74974366)(82.95102789,137.54575058)
\curveto(83.68627093,138.34783389)(84.60684382,138.74887555)(85.71274657,138.74887555)
\curveto(86.64851044,138.74887555)(87.41717361,138.43897972)(88.0187361,137.81918807)
\curveto(88.62029858,137.20547281)(88.92107982,136.38212214)(88.92107982,135.34913605)
\closepath
\moveto(83.15154872,135.87778187)
\lineto(87.01613195,135.87778187)
\curveto(86.98575001,136.41250408)(86.92194793,136.78924018)(86.82472571,137.00799018)
\curveto(86.67281599,137.34826795)(86.44495141,137.61562905)(86.14113198,137.81007349)
\curveto(85.84338893,138.00451793)(85.53045491,138.10174015)(85.20232992,138.10174015)
\curveto(84.69798965,138.10174015)(84.24529869,137.90425752)(83.84425703,137.50929225)
\curveto(83.44929177,137.12040337)(83.21838899,136.57656658)(83.15154872,135.87778187)
\closepath
}
}
{
\newrgbcolor{curcolor}{0 0 0}
\pscustom[linestyle=none,fillstyle=solid,fillcolor=curcolor]
{
\newpath
\moveto(92.45753807,141.23715674)
\lineto(92.45753807,138.49366722)
\lineto(94.40805885,138.49366722)
\lineto(94.40805885,137.85564641)
\lineto(92.45753807,137.85564641)
\lineto(92.45753807,132.44158404)
\curveto(92.45753807,131.90078544)(92.53349293,131.53620212)(92.68540265,131.34783407)
\curveto(92.84338875,131.15946602)(93.04390958,131.06528199)(93.28696513,131.06528199)
\curveto(93.48748596,131.06528199)(93.6819304,131.12604588)(93.87029845,131.24757365)
\curveto(94.0586665,131.37517782)(94.20449983,131.56050767)(94.30779844,131.80356322)
\lineto(94.66326718,131.80356322)
\curveto(94.45059357,131.20807713)(94.14981233,130.75842436)(93.76092345,130.45460492)
\curveto(93.37203457,130.15686188)(92.97099292,130.00799035)(92.55779848,130.00799035)
\curveto(92.2782846,130.00799035)(92.00484711,130.08394521)(91.737486,130.23585493)
\curveto(91.4701249,130.39384104)(91.27264227,130.61562923)(91.1450381,130.9012195)
\curveto(91.01743394,131.19288616)(90.95363186,131.63950073)(90.95363186,132.24106321)
\lineto(90.95363186,137.85564641)
\lineto(89.63201731,137.85564641)
\lineto(89.63201731,138.15642765)
\curveto(89.96621869,138.2901082)(90.30649646,138.51493458)(90.65285061,138.8309068)
\curveto(91.00528116,139.1529554)(91.31821518,139.5327297)(91.59165267,139.97022969)
\curveto(91.73140961,140.20113246)(91.92585405,140.62344147)(92.17498599,141.23715674)
\closepath
}
}
{
\newrgbcolor{curcolor}{0 0 0}
\pscustom[linewidth=0.76542264,linecolor=curcolor]
{
\newpath
\moveto(241.7737409,177.86952265)
\lineto(187.52827559,148.62091005)
}
}
{
\newrgbcolor{curcolor}{0 0 0}
\pscustom[linestyle=none,fillstyle=solid,fillcolor=curcolor]
{
\newpath
\moveto(194.26555098,152.25358159)
\lineto(198.41352975,151.01174004)
\lineto(187.52827559,148.62091005)
\lineto(195.50739252,156.40156036)
\closepath
}
}
{
\newrgbcolor{curcolor}{0 0 0}
\pscustom[linewidth=0.81645084,linecolor=curcolor]
{
\newpath
\moveto(194.26555098,152.25358159)
\lineto(198.41352975,151.01174004)
\lineto(187.52827559,148.62091005)
\lineto(195.50739252,156.40156036)
\closepath
}
}
{
\newrgbcolor{curcolor}{0 0 0}
\pscustom[linewidth=0.74666665,linecolor=curcolor]
{
\newpath
\moveto(416.66849653,121.55686484)
\lineto(527.45236042,121.55686484)
\lineto(527.45236042,149.58859748)
\lineto(416.66849653,149.58859748)
\closepath
}
}
{
\newrgbcolor{curcolor}{0 0 0}
\pscustom[linestyle=none,fillstyle=solid,fillcolor=curcolor]
{
\newpath
\moveto(440.89310269,139.69707468)
\curveto(442.15699155,139.69707468)(443.17174847,139.21703997)(443.93737345,138.25697055)
\curveto(444.58754704,137.43665807)(444.91263384,136.49481781)(444.91263384,135.43144978)
\curveto(444.91263384,134.68405397)(444.73338037,133.92754357)(444.37487344,133.16191859)
\curveto(444.0163665,132.39629361)(443.52114082,131.81903668)(442.88919639,131.4301478)
\curveto(442.26332835,131.04125892)(441.56454365,130.84681448)(440.79284228,130.84681448)
\curveto(439.53502981,130.84681448)(438.53546386,131.34811655)(437.79414444,132.35072069)
\curveto(437.1682764,133.19533873)(436.85534238,134.14325537)(436.85534238,135.19447062)
\curveto(436.85534238,135.9600956)(437.04371043,136.71964419)(437.42044653,137.4731164)
\curveto(437.80325902,138.23266499)(438.30456109,138.79169275)(438.92435274,139.15019969)
\curveto(439.54414439,139.51478301)(440.20039438,139.69707468)(440.89310269,139.69707468)
\closepath
\moveto(440.61055061,139.10462677)
\curveto(440.28850201,139.10462677)(439.96341521,139.00740455)(439.63529022,138.81296011)
\curveto(439.31324162,138.62459206)(439.0519569,138.29039068)(438.85143608,137.81035597)
\curveto(438.65091525,137.33032126)(438.55065483,136.71356781)(438.55065483,135.9600956)
\curveto(438.55065483,134.74481786)(438.79067219,133.6966408)(439.2707069,132.81556443)
\curveto(439.756818,131.93448806)(440.39483881,131.49394988)(441.18476935,131.49394988)
\curveto(441.77417906,131.49394988)(442.26029016,131.73700543)(442.64310265,132.22311653)
\curveto(443.02591514,132.70922763)(443.21732138,133.54473108)(443.21732138,134.72962688)
\curveto(443.21732138,136.21226574)(442.89831097,137.37893237)(442.26029016,138.2296268)
\curveto(441.82886656,138.81296011)(441.27895338,139.10462677)(440.61055061,139.10462677)
\closepath
}
}
{
\newrgbcolor{curcolor}{0 0 0}
\pscustom[linestyle=none,fillstyle=solid,fillcolor=curcolor]
{
\newpath
\moveto(448.43086292,138.0108768)
\curveto(449.23902262,139.13500872)(450.11098441,139.69707468)(451.04674827,139.69707468)
\curveto(451.90351908,139.69707468)(452.6509149,139.32945316)(453.28893571,138.59421012)
\curveto(453.92695653,137.86504347)(454.24596694,136.86547752)(454.24596694,135.59551228)
\curveto(454.24596694,134.11287343)(453.75377945,132.91886304)(452.76940448,132.01348112)
\curveto(451.92478644,131.23570336)(450.98294619,130.84681448)(449.94388371,130.84681448)
\curveto(449.45777262,130.84681448)(448.96254693,130.93492212)(448.45820667,131.11113739)
\curveto(447.95994279,131.28735266)(447.44952614,131.55167557)(446.92695671,131.90410612)
\lineto(446.92695671,140.55384549)
\curveto(446.92695671,141.50176213)(446.90265115,142.08509545)(446.85404004,142.30384544)
\curveto(446.81150532,142.52259544)(446.74162685,142.67146696)(446.64440463,142.75046002)
\curveto(446.54718241,142.82945307)(446.42565464,142.8689496)(446.27982131,142.8689496)
\curveto(446.10968242,142.8689496)(445.89700882,142.82033849)(445.64180049,142.72311627)
\lineto(445.51419633,143.04212668)
\lineto(448.02070668,144.06295998)
\lineto(448.43086292,144.06295998)
\closepath
\moveto(448.43086292,137.42754348)
\lineto(448.43086292,132.43275194)
\curveto(448.74075874,132.1289325)(449.05976915,131.89802973)(449.38789415,131.74004362)
\curveto(449.72209553,131.58813391)(450.0623733,131.51217905)(450.40872745,131.51217905)
\curveto(450.96167883,131.51217905)(451.47513368,131.81599848)(451.949092,132.42363736)
\curveto(452.42912671,133.03127623)(452.66914406,133.91539079)(452.66914406,135.07598104)
\curveto(452.66914406,136.14542546)(452.42912671,136.96573794)(451.949092,137.53691848)
\curveto(451.47513368,138.11417541)(450.93433508,138.40280387)(450.32669621,138.40280387)
\curveto(450.0046476,138.40280387)(449.682599,138.32077263)(449.3605504,138.15671013)
\curveto(449.11749485,138.03518236)(448.80759902,137.79212681)(448.43086292,137.42754348)
\closepath
}
}
{
\newrgbcolor{curcolor}{0 0 0}
\pscustom[linestyle=none,fillstyle=solid,fillcolor=curcolor]
{
\newpath
\moveto(457.59101894,144.07207457)
\curveto(457.85230366,144.07207457)(458.07409184,143.98092874)(458.25638351,143.79863707)
\curveto(458.43867517,143.61634541)(458.529821,143.39455722)(458.529821,143.13327251)
\curveto(458.529821,142.87806418)(458.43867517,142.65931419)(458.25638351,142.47702252)
\curveto(458.07409184,142.29473086)(457.85230366,142.20358503)(457.59101894,142.20358503)
\curveto(457.33581061,142.20358503)(457.11706062,142.29473086)(456.93476896,142.47702252)
\curveto(456.75247729,142.65931419)(456.66133146,142.87806418)(456.66133146,143.13327251)
\curveto(456.66133146,143.39455722)(456.75247729,143.61634541)(456.93476896,143.79863707)
\curveto(457.11706062,143.98092874)(457.33581061,144.07207457)(457.59101894,144.07207457)
\closepath
\moveto(458.37487309,139.69707468)
\lineto(458.37487309,131.29342905)
\curveto(458.37487309,129.8654777)(458.07105365,128.80514786)(457.46341478,128.11243955)
\curveto(456.8557759,127.41973123)(456.06584537,127.07337707)(455.09362317,127.07337707)
\curveto(454.54067179,127.07337707)(454.13051555,127.17363749)(453.86315445,127.37415832)
\curveto(453.59579335,127.57467915)(453.46211279,127.78127636)(453.46211279,127.99394997)
\curveto(453.46211279,128.20662357)(453.53806765,128.38891524)(453.68997737,128.54082495)
\curveto(453.8358107,128.69273467)(454.00898778,128.76868953)(454.20950861,128.76868953)
\curveto(454.36749472,128.76868953)(454.52851902,128.72919301)(454.69258151,128.65019995)
\curveto(454.79588012,128.60766523)(454.99336276,128.45575551)(455.28502941,128.1944708)
\curveto(455.58277246,127.92710969)(455.8319044,127.79342914)(456.03242523,127.79342914)
\curveto(456.17825856,127.79342914)(456.32105369,127.85115483)(456.46081064,127.96660622)
\curveto(456.60056758,128.07598122)(456.70386618,128.26434927)(456.77070646,128.53171037)
\curveto(456.83754674,128.79299509)(456.87096688,129.36417563)(456.87096688,130.245252)
\lineto(456.87096688,136.18796018)
\curveto(456.87096688,137.10549488)(456.84362313,137.69490459)(456.78893563,137.9561893)
\curveto(456.74640091,138.15671013)(456.67956063,138.29342888)(456.5884148,138.36634554)
\curveto(456.49726897,138.4453386)(456.372703,138.48483512)(456.21471689,138.48483512)
\curveto(456.04457801,138.48483512)(455.83798079,138.43926221)(455.59492524,138.34811638)
\lineto(455.46732108,138.67624137)
\lineto(457.98294601,139.69707468)
\closepath
}
}
{
\newrgbcolor{curcolor}{0 0 0}
\pscustom[linestyle=none,fillstyle=solid,fillcolor=curcolor]
{
\newpath
\moveto(462.06627925,136.30644976)
\curveto(462.06020286,135.06686646)(462.3609841,134.09464426)(462.96862297,133.38978317)
\curveto(463.57626185,132.68492207)(464.29023752,132.33249153)(465.11055,132.33249153)
\curveto(465.65742499,132.33249153)(466.13138331,132.48136305)(466.53242497,132.7791061)
\curveto(466.93954301,133.08292554)(467.27982078,133.59941858)(467.55325827,134.32858523)
\lineto(467.83581035,134.14629356)
\curveto(467.70820619,133.31382831)(467.33754647,132.55427972)(466.72383121,131.86764779)
\curveto(466.11011595,131.18709225)(465.34145277,130.84681448)(464.41784169,130.84681448)
\curveto(463.41523754,130.84681448)(462.55542854,131.23570336)(461.83841467,132.01348112)
\curveto(461.12747719,132.79733527)(460.77200844,133.84855052)(460.77200844,135.16712687)
\curveto(460.77200844,136.59507823)(461.13659177,137.70705736)(461.86575842,138.50306429)
\curveto(462.60100145,139.3051476)(463.52157435,139.70618926)(464.6274771,139.70618926)
\curveto(465.56324096,139.70618926)(466.33190414,139.39629343)(466.93346662,138.77650178)
\curveto(467.53502911,138.16278652)(467.83581035,137.33943585)(467.83581035,136.30644976)
\closepath
\moveto(462.06627925,136.83509558)
\lineto(465.93086248,136.83509558)
\curveto(465.90048054,137.36981779)(465.83667846,137.74655389)(465.73945624,137.96530389)
\curveto(465.58754652,138.30558166)(465.35968194,138.57294276)(465.0558625,138.7673872)
\curveto(464.75811946,138.96183164)(464.44518544,139.05905386)(464.11706044,139.05905386)
\curveto(463.61272018,139.05905386)(463.16002922,138.86157122)(462.75898756,138.46660596)
\curveto(462.36402229,138.07771708)(462.13311952,137.53388029)(462.06627925,136.83509558)
\closepath
}
}
{
\newrgbcolor{curcolor}{0 0 0}
\pscustom[linestyle=none,fillstyle=solid,fillcolor=curcolor]
{
\newpath
\moveto(476.03893515,134.27389773)
\curveto(475.81410876,133.17407137)(475.37357058,132.32641514)(474.7173206,131.73092904)
\curveto(474.06107061,131.14151933)(473.33494216,130.84681448)(472.53893523,130.84681448)
\curveto(471.59101859,130.84681448)(470.76462972,131.24481794)(470.05976863,132.04082487)
\curveto(469.35490754,132.83683179)(469.00247699,133.9123526)(469.00247699,135.26738729)
\curveto(469.00247699,136.57988725)(469.39136587,137.64629348)(470.16914363,138.46660596)
\curveto(470.95299777,139.28691844)(471.89179983,139.69707468)(472.98554981,139.69707468)
\curveto(473.80586229,139.69707468)(474.48034143,139.47832468)(475.00898725,139.04082469)
\curveto(475.53763308,138.60940109)(475.80195599,138.15974833)(475.80195599,137.69186639)
\curveto(475.80195599,137.46096362)(475.72600113,137.27259557)(475.57409141,137.12676224)
\curveto(475.42825808,136.9870053)(475.22166086,136.91712683)(474.95429976,136.91712683)
\curveto(474.59579282,136.91712683)(474.32539352,137.03257821)(474.14310186,137.26348099)
\curveto(474.03980325,137.39108515)(473.96992478,137.6341407)(473.93346645,137.99264764)
\curveto(473.9030845,138.35115457)(473.78155673,138.62459206)(473.56888312,138.81296011)
\curveto(473.35620952,138.99525178)(473.06150466,139.08639761)(472.68476856,139.08639761)
\curveto(472.07712969,139.08639761)(471.5879804,138.86157122)(471.21732068,138.41191846)
\curveto(470.7251332,137.81643236)(470.47903945,137.02954002)(470.47903945,136.05124143)
\curveto(470.47903945,135.05471368)(470.722095,134.17363731)(471.2082061,133.40801233)
\curveto(471.70039359,132.64846374)(472.36271996,132.26868944)(473.19518522,132.26868944)
\curveto(473.79067131,132.26868944)(474.32539352,132.47224847)(474.79935184,132.87936651)
\curveto(475.13355322,133.15888039)(475.45864002,133.66625885)(475.77461224,134.40150189)
\closepath
}
}
{
\newrgbcolor{curcolor}{0 0 0}
\pscustom[linestyle=none,fillstyle=solid,fillcolor=curcolor]
{
\newpath
\moveto(479.65742464,142.19447045)
\lineto(479.65742464,139.45098093)
\lineto(481.60794542,139.45098093)
\lineto(481.60794542,138.81296011)
\lineto(479.65742464,138.81296011)
\lineto(479.65742464,133.39889775)
\curveto(479.65742464,132.85809915)(479.7333795,132.49351583)(479.88528922,132.30514778)
\curveto(480.04327532,132.11677973)(480.24379615,132.0225957)(480.4868517,132.0225957)
\curveto(480.68737253,132.0225957)(480.88181697,132.08335959)(481.07018502,132.20488736)
\curveto(481.25855307,132.33249153)(481.4043864,132.51782138)(481.50768501,132.76087693)
\lineto(481.86315375,132.76087693)
\curveto(481.65048014,132.16539084)(481.3496989,131.71573807)(480.96081002,131.41191863)
\curveto(480.57192114,131.11417558)(480.17087949,130.96530406)(479.75768505,130.96530406)
\curveto(479.47817117,130.96530406)(479.20473368,131.04125892)(478.93737257,131.19316864)
\curveto(478.67001147,131.35115475)(478.47252884,131.57294293)(478.34492467,131.85853321)
\curveto(478.21732051,132.15019986)(478.15351843,132.59681444)(478.15351843,133.19837692)
\lineto(478.15351843,138.81296011)
\lineto(476.83190388,138.81296011)
\lineto(476.83190388,139.11374136)
\curveto(477.16610526,139.24742191)(477.50638303,139.47224829)(477.85273718,139.78822051)
\curveto(478.20516773,140.11026911)(478.51810175,140.49004341)(478.79153924,140.9275434)
\curveto(478.93129618,141.15844617)(479.12574062,141.58075518)(479.37487256,142.19447045)
\closepath
}
}
{
\newrgbcolor{curcolor}{0 0 0}
\pscustom[linestyle=none,fillstyle=solid,fillcolor=curcolor]
{
\newpath
\moveto(492.48164307,139.69707468)
\lineto(492.48164307,136.85332475)
\lineto(492.18086183,136.85332475)
\curveto(491.94995905,137.74655389)(491.65221601,138.35419276)(491.28763268,138.67624137)
\curveto(490.92912575,138.99828997)(490.4703584,139.15931427)(489.91133063,139.15931427)
\curveto(489.48598342,139.15931427)(489.14266746,139.04690108)(488.88138274,138.8220747)
\curveto(488.62009803,138.59724831)(488.48945567,138.34811638)(488.48945567,138.07467888)
\curveto(488.48945567,137.73440111)(488.58667789,137.44273445)(488.78112233,137.1996789)
\curveto(488.96949038,136.95054697)(489.35230287,136.68622406)(489.9295598,136.40671017)
\lineto(491.26028893,135.75957477)
\curveto(492.49379585,135.15801229)(493.1105493,134.36504356)(493.1105493,133.38066858)
\curveto(493.1105493,132.62111999)(492.82192084,132.00740473)(492.24466391,131.5395228)
\curveto(491.67348337,131.07771725)(491.03242435,130.84681448)(490.32148687,130.84681448)
\curveto(489.81107022,130.84681448)(489.2277369,130.93796031)(488.57148692,131.12025197)
\curveto(488.37096609,131.18101586)(488.20690359,131.2113978)(488.07929943,131.2113978)
\curveto(487.93954249,131.2113978)(487.83016749,131.13240475)(487.75117444,130.97441864)
\lineto(487.45039319,130.97441864)
\lineto(487.45039319,133.95488732)
\lineto(487.75117444,133.95488732)
\curveto(487.92131332,133.1041929)(488.24640012,132.46313388)(488.72643483,132.03171028)
\curveto(489.20646954,131.60028668)(489.74422994,131.38457488)(490.33971604,131.38457488)
\curveto(490.75898686,131.38457488)(491.09926463,131.50610266)(491.36054935,131.74915821)
\curveto(491.62791045,131.99829015)(491.761591,132.29603319)(491.761591,132.64238735)
\curveto(491.761591,133.06165818)(491.61271948,133.41408872)(491.31497643,133.69967899)
\curveto(491.02330977,133.98526926)(490.43693826,134.34681439)(489.55586189,134.78431438)
\curveto(488.67478552,135.22181437)(488.09752859,135.61677964)(487.8240911,135.96921019)
\curveto(487.55065361,136.31556434)(487.41393486,136.75306433)(487.41393486,137.28171015)
\curveto(487.41393486,137.96834208)(487.64787583,138.54256082)(488.11575776,139.00436636)
\curveto(488.58971608,139.4661719)(489.20039315,139.69707468)(489.94778896,139.69707468)
\curveto(490.27591396,139.69707468)(490.67391742,139.62719621)(491.14179935,139.48743926)
\curveto(491.45169518,139.39629343)(491.65829239,139.35072052)(491.761591,139.35072052)
\curveto(491.85881322,139.35072052)(491.93476808,139.37198788)(491.98945558,139.4145226)
\curveto(492.04414308,139.45705732)(492.10794516,139.55124135)(492.18086183,139.69707468)
\closepath
}
}
{
\newrgbcolor{curcolor}{0 0 0}
\pscustom[linestyle=none,fillstyle=solid,fillcolor=curcolor]
{
\newpath
\moveto(495.7537784,136.30644976)
\curveto(495.74770201,135.06686646)(496.04848326,134.09464426)(496.65612213,133.38978317)
\curveto(497.263761,132.68492207)(497.97773668,132.33249153)(498.79804916,132.33249153)
\curveto(499.34492415,132.33249153)(499.81888247,132.48136305)(500.21992412,132.7791061)
\curveto(500.62704217,133.08292554)(500.96731994,133.59941858)(501.24075743,134.32858523)
\lineto(501.52330951,134.14629356)
\curveto(501.39570535,133.31382831)(501.02504563,132.55427972)(500.41133037,131.86764779)
\curveto(499.79761511,131.18709225)(499.02895193,130.84681448)(498.10534084,130.84681448)
\curveto(497.1027367,130.84681448)(496.2429277,131.23570336)(495.52591383,132.01348112)
\curveto(494.81497634,132.79733527)(494.4595076,133.84855052)(494.4595076,135.16712687)
\curveto(494.4595076,136.59507823)(494.82409093,137.70705736)(495.55325757,138.50306429)
\curveto(496.28850061,139.3051476)(497.20907351,139.70618926)(498.31497626,139.70618926)
\curveto(499.25074012,139.70618926)(500.0194033,139.39629343)(500.62096578,138.77650178)
\curveto(501.22252827,138.16278652)(501.52330951,137.33943585)(501.52330951,136.30644976)
\closepath
\moveto(495.7537784,136.83509558)
\lineto(499.61836164,136.83509558)
\curveto(499.5879797,137.36981779)(499.52417761,137.74655389)(499.42695539,137.96530389)
\curveto(499.27504568,138.30558166)(499.0471811,138.57294276)(498.74336166,138.7673872)
\curveto(498.44561861,138.96183164)(498.13268459,139.05905386)(497.8045596,139.05905386)
\curveto(497.30021934,139.05905386)(496.84752838,138.86157122)(496.44648672,138.46660596)
\curveto(496.05152145,138.07771708)(495.82061868,137.53388029)(495.7537784,136.83509558)
\closepath
}
}
{
\newrgbcolor{curcolor}{0 0 0}
\pscustom[linestyle=none,fillstyle=solid,fillcolor=curcolor]
{
\newpath
\moveto(505.05976775,142.19447045)
\lineto(505.05976775,139.45098093)
\lineto(507.01028854,139.45098093)
\lineto(507.01028854,138.81296011)
\lineto(505.05976775,138.81296011)
\lineto(505.05976775,133.39889775)
\curveto(505.05976775,132.85809915)(505.13572261,132.49351583)(505.28763233,132.30514778)
\curveto(505.44561844,132.11677973)(505.64613927,132.0225957)(505.88919482,132.0225957)
\curveto(506.08971564,132.0225957)(506.28416008,132.08335959)(506.47252814,132.20488736)
\curveto(506.66089619,132.33249153)(506.80672952,132.51782138)(506.91002812,132.76087693)
\lineto(507.26549687,132.76087693)
\curveto(507.05282326,132.16539084)(506.75204202,131.71573807)(506.36315314,131.41191863)
\curveto(505.97426426,131.11417558)(505.5732226,130.96530406)(505.16002817,130.96530406)
\curveto(504.88051429,130.96530406)(504.60707679,131.04125892)(504.33971569,131.19316864)
\curveto(504.07235458,131.35115475)(503.87487195,131.57294293)(503.74726779,131.85853321)
\curveto(503.61966362,132.15019986)(503.55586154,132.59681444)(503.55586154,133.19837692)
\lineto(503.55586154,138.81296011)
\lineto(502.23424699,138.81296011)
\lineto(502.23424699,139.11374136)
\curveto(502.56844837,139.24742191)(502.90872614,139.47224829)(503.2550803,139.78822051)
\curveto(503.60751085,140.11026911)(503.92044487,140.49004341)(504.19388236,140.9275434)
\curveto(504.3336393,141.15844617)(504.52808374,141.58075518)(504.77721568,142.19447045)
\closepath
}
}
{
\newrgbcolor{curcolor}{0 0 0}
\pscustom[linewidth=0.74666665,linecolor=curcolor]
{
\newpath
\moveto(133.30359694,120.59955136)
\lineto(244.08746084,120.59955136)
\lineto(244.08746084,148.63128399)
\lineto(133.30359694,148.63128399)
\closepath
}
}
{
\newrgbcolor{curcolor}{0 0 0}
\pscustom[linestyle=none,fillstyle=solid,fillcolor=curcolor]
{
\newpath
\moveto(157.52820311,138.73976119)
\curveto(158.79209197,138.73976119)(159.80684888,138.25972648)(160.57247387,137.29965706)
\curveto(161.22264746,136.47934458)(161.54773426,135.53750433)(161.54773426,134.4741363)
\curveto(161.54773426,133.72674049)(161.36848079,132.97023009)(161.00997385,132.20460511)
\curveto(160.65146692,131.43898013)(160.15624124,130.8617232)(159.52429681,130.47283432)
\curveto(158.89842877,130.08394544)(158.19964406,129.889501)(157.42794269,129.889501)
\curveto(156.17013023,129.889501)(155.17056428,130.39080307)(154.42924485,131.39340721)
\curveto(153.80337681,132.23802524)(153.49044279,133.18594189)(153.49044279,134.23715714)
\curveto(153.49044279,135.00278212)(153.67881084,135.76233071)(154.05554695,136.51580292)
\curveto(154.43835944,137.27535151)(154.93966151,137.83437927)(155.55945316,138.19288621)
\curveto(156.17924481,138.55746953)(156.83549479,138.73976119)(157.52820311,138.73976119)
\closepath
\moveto(157.24565103,138.14731329)
\curveto(156.92360243,138.14731329)(156.59851563,138.05009107)(156.27039064,137.85564663)
\curveto(155.94834204,137.66727858)(155.68705732,137.3330772)(155.48653649,136.85304249)
\curveto(155.28601566,136.37300778)(155.18575525,135.75625432)(155.18575525,135.00278212)
\curveto(155.18575525,133.78750437)(155.42577261,132.73932732)(155.90580732,131.85825095)
\curveto(156.39191841,130.97717458)(157.02993923,130.5366364)(157.81986977,130.5366364)
\curveto(158.40927948,130.5366364)(158.89539057,130.77969195)(159.27820306,131.26580305)
\curveto(159.66101556,131.75191415)(159.8524218,132.5874176)(159.8524218,133.7723134)
\curveto(159.8524218,135.25495225)(159.53341139,136.42161889)(158.89539057,137.27231331)
\curveto(158.46396697,137.85564663)(157.91405379,138.14731329)(157.24565103,138.14731329)
\closepath
}
}
{
\newrgbcolor{curcolor}{0 0 0}
\pscustom[linestyle=none,fillstyle=solid,fillcolor=curcolor]
{
\newpath
\moveto(165.06596334,137.05356332)
\curveto(165.87412304,138.17769523)(166.74608482,138.73976119)(167.68184869,138.73976119)
\curveto(168.5386195,138.73976119)(169.28601531,138.37213967)(169.92403613,137.63689664)
\curveto(170.56205695,136.90772999)(170.88106736,135.90816404)(170.88106736,134.6381988)
\curveto(170.88106736,133.15555994)(170.38887987,131.96154956)(169.4045049,131.05616764)
\curveto(168.55988686,130.27838988)(167.61804661,129.889501)(166.57898413,129.889501)
\curveto(166.09287303,129.889501)(165.59764735,129.97760863)(165.09330709,130.15382391)
\curveto(164.59504321,130.33003918)(164.08462656,130.59436209)(163.56205712,130.94679264)
\lineto(163.56205712,139.59653201)
\curveto(163.56205712,140.54444865)(163.53775157,141.12778197)(163.48914046,141.34653196)
\curveto(163.44660574,141.56528196)(163.37672727,141.71415348)(163.27950505,141.79314653)
\curveto(163.18228283,141.87213959)(163.06075505,141.91163611)(162.91492172,141.91163611)
\curveto(162.74478284,141.91163611)(162.53210923,141.863025)(162.27690091,141.76580278)
\lineto(162.14929674,142.08481319)
\lineto(164.6558071,143.1056465)
\lineto(165.06596334,143.1056465)
\closepath
\moveto(165.06596334,136.47023)
\lineto(165.06596334,131.47543846)
\curveto(165.37585916,131.17161902)(165.69486957,130.94071625)(166.02299456,130.78273014)
\curveto(166.35719594,130.63082042)(166.69747371,130.55486556)(167.04382787,130.55486556)
\curveto(167.59677925,130.55486556)(168.11023409,130.858685)(168.58419242,131.46632387)
\curveto(169.06422713,132.07396275)(169.30424448,132.95807731)(169.30424448,134.11866756)
\curveto(169.30424448,135.18811198)(169.06422713,136.00842446)(168.58419242,136.579605)
\curveto(168.11023409,137.15686193)(167.5694355,137.44549039)(166.96179662,137.44549039)
\curveto(166.63974802,137.44549039)(166.31769942,137.36345914)(165.99565081,137.19939665)
\curveto(165.75259526,137.07786887)(165.44269944,136.83481332)(165.06596334,136.47023)
\closepath
}
}
{
\newrgbcolor{curcolor}{0 0 0}
\pscustom[linestyle=none,fillstyle=solid,fillcolor=curcolor]
{
\newpath
\moveto(174.22611936,143.11476108)
\curveto(174.48740407,143.11476108)(174.70919226,143.02361525)(174.89148392,142.84132359)
\curveto(175.07377559,142.65903193)(175.16492142,142.43724374)(175.16492142,142.17595902)
\curveto(175.16492142,141.9207507)(175.07377559,141.7020007)(174.89148392,141.51970904)
\curveto(174.70919226,141.33741738)(174.48740407,141.24627155)(174.22611936,141.24627155)
\curveto(173.97091103,141.24627155)(173.75216104,141.33741738)(173.56986937,141.51970904)
\curveto(173.38757771,141.7020007)(173.29643188,141.9207507)(173.29643188,142.17595902)
\curveto(173.29643188,142.43724374)(173.38757771,142.65903193)(173.56986937,142.84132359)
\curveto(173.75216104,143.02361525)(173.97091103,143.11476108)(174.22611936,143.11476108)
\closepath
\moveto(175.0099735,138.73976119)
\lineto(175.0099735,130.33611557)
\curveto(175.0099735,128.90816422)(174.70615407,127.84783438)(174.09851519,127.15512607)
\curveto(173.49087632,126.46241775)(172.70094578,126.11606359)(171.72872359,126.11606359)
\curveto(171.17577221,126.11606359)(170.76561597,126.21632401)(170.49825487,126.41684483)
\curveto(170.23089376,126.61736566)(170.09721321,126.82396288)(170.09721321,127.03663649)
\curveto(170.09721321,127.24931009)(170.17316807,127.43160175)(170.32507779,127.58351147)
\curveto(170.47091112,127.73542119)(170.6440882,127.81137605)(170.84460903,127.81137605)
\curveto(171.00259513,127.81137605)(171.16361943,127.77187952)(171.32768193,127.69288647)
\curveto(171.43098054,127.65035175)(171.62846317,127.49844203)(171.92012983,127.23715731)
\curveto(172.21787288,126.96979621)(172.46700482,126.83611566)(172.66752565,126.83611566)
\curveto(172.81335898,126.83611566)(172.95615411,126.89384135)(173.09591105,127.00929274)
\curveto(173.23566799,127.11866773)(173.3389666,127.30703578)(173.40580688,127.57439689)
\curveto(173.47264715,127.8356816)(173.50606729,128.40686215)(173.50606729,129.28793851)
\lineto(173.50606729,135.2306467)
\curveto(173.50606729,136.1481814)(173.47872354,136.7375911)(173.42403604,136.99887582)
\curveto(173.38150132,137.19939665)(173.31466105,137.33611539)(173.22351522,137.40903206)
\curveto(173.13236939,137.48802511)(173.00780342,137.52752164)(172.84981731,137.52752164)
\curveto(172.67967842,137.52752164)(172.47308121,137.48194872)(172.23002566,137.39080289)
\lineto(172.10242149,137.71892789)
\lineto(174.61804643,138.73976119)
\closepath
}
}
{
\newrgbcolor{curcolor}{0 0 0}
\pscustom[linestyle=none,fillstyle=solid,fillcolor=curcolor]
{
\newpath
\moveto(178.70137966,135.34913628)
\curveto(178.69530327,134.10955298)(178.99608452,133.13733078)(179.60372339,132.43246968)
\curveto(180.21136226,131.72760859)(180.92533794,131.37517804)(181.74565042,131.37517804)
\curveto(182.29252541,131.37517804)(182.76648373,131.52404957)(183.16752538,131.82179262)
\curveto(183.57464343,132.12561205)(183.9149212,132.6421051)(184.18835869,133.37127174)
\lineto(184.47091077,133.18898008)
\curveto(184.3433066,132.35651482)(183.97264689,131.59696623)(183.35893163,130.91033431)
\curveto(182.74521637,130.22977877)(181.97655319,129.889501)(181.0529421,129.889501)
\curveto(180.05033796,129.889501)(179.19052896,130.27838988)(178.47351508,131.05616764)
\curveto(177.7625776,131.84002178)(177.40710886,132.89123703)(177.40710886,134.20981339)
\curveto(177.40710886,135.63776474)(177.77169219,136.74974388)(178.50085883,137.54575081)
\curveto(179.23610187,138.34783412)(180.15667477,138.74887578)(181.26257752,138.74887578)
\curveto(182.19834138,138.74887578)(182.96700456,138.43897995)(183.56856704,137.8191883)
\curveto(184.17012953,137.20547304)(184.47091077,136.38212236)(184.47091077,135.34913628)
\closepath
\moveto(178.70137966,135.8777821)
\lineto(182.5659629,135.8777821)
\curveto(182.53558096,136.41250431)(182.47177887,136.78924041)(182.37455665,137.0079904)
\curveto(182.22264694,137.34826817)(181.99478236,137.61562928)(181.69096292,137.81007372)
\curveto(181.39321987,138.00451816)(181.08028585,138.10174038)(180.75216086,138.10174038)
\curveto(180.2478206,138.10174038)(179.79512964,137.90425774)(179.39408798,137.50929247)
\curveto(178.99912271,137.12040359)(178.76821994,136.5765668)(178.70137966,135.8777821)
\closepath
}
}
{
\newrgbcolor{curcolor}{0 0 0}
\pscustom[linestyle=none,fillstyle=solid,fillcolor=curcolor]
{
\newpath
\moveto(192.67403556,133.31658425)
\curveto(192.44920918,132.21675788)(192.008671,131.36910166)(191.35242101,130.77361556)
\curveto(190.69617103,130.18420585)(189.97004258,129.889501)(189.17403565,129.889501)
\curveto(188.22611901,129.889501)(187.39973014,130.28750446)(186.69486905,131.08351138)
\curveto(185.99000795,131.87951831)(185.63757741,132.95503912)(185.63757741,134.3100738)
\curveto(185.63757741,135.62257377)(186.02646629,136.68897999)(186.80424404,137.50929247)
\curveto(187.58809819,138.32960495)(188.52690025,138.73976119)(189.62065022,138.73976119)
\curveto(190.4409627,138.73976119)(191.11544185,138.5210112)(191.64408767,138.08351121)
\curveto(192.17273349,137.65208761)(192.4370564,137.20243484)(192.4370564,136.73455291)
\curveto(192.4370564,136.50365014)(192.36110154,136.31528209)(192.20919182,136.16944876)
\curveto(192.0633585,136.02969182)(191.85676128,135.95981335)(191.58940017,135.95981335)
\curveto(191.23089324,135.95981335)(190.96049394,136.07526473)(190.77820228,136.3061675)
\curveto(190.67490367,136.43377167)(190.6050252,136.67682722)(190.56856687,137.03533415)
\curveto(190.53818492,137.39384109)(190.41665715,137.66727858)(190.20398354,137.85564663)
\curveto(189.99130994,138.03793829)(189.69660508,138.12908413)(189.31986898,138.12908413)
\curveto(188.71223011,138.12908413)(188.22308081,137.90425774)(187.8524211,137.45460498)
\curveto(187.36023361,136.85911888)(187.11413987,136.07222654)(187.11413987,135.09392795)
\curveto(187.11413987,134.0974002)(187.35719542,133.21632383)(187.84330652,132.45069885)
\curveto(188.33549401,131.69115026)(188.99782038,131.31137596)(189.83028563,131.31137596)
\curveto(190.42577173,131.31137596)(190.96049394,131.51493498)(191.43445226,131.92205303)
\curveto(191.76865364,132.20156691)(192.09374044,132.70894537)(192.40971265,133.44418841)
\closepath
}
}
{
\newrgbcolor{curcolor}{0 0 0}
\pscustom[linestyle=none,fillstyle=solid,fillcolor=curcolor]
{
\newpath
\moveto(196.29252506,141.23715696)
\lineto(196.29252506,138.49366745)
\lineto(198.24304584,138.49366745)
\lineto(198.24304584,137.85564663)
\lineto(196.29252506,137.85564663)
\lineto(196.29252506,132.44158427)
\curveto(196.29252506,131.90078567)(196.36847992,131.53620235)(196.52038963,131.34783429)
\curveto(196.67837574,131.15946624)(196.87889657,131.06528222)(197.12195212,131.06528222)
\curveto(197.32247295,131.06528222)(197.51691739,131.12604611)(197.70528544,131.24757388)
\curveto(197.89365349,131.37517804)(198.03948682,131.5605079)(198.14278543,131.80356345)
\lineto(198.49825417,131.80356345)
\curveto(198.28558056,131.20807735)(197.98479932,130.75842459)(197.59591044,130.45460515)
\curveto(197.20702156,130.1568621)(196.8059799,130.00799058)(196.39278547,130.00799058)
\curveto(196.11327159,130.00799058)(195.8398341,130.08394544)(195.57247299,130.23585516)
\curveto(195.30511189,130.39384126)(195.10762925,130.61562945)(194.98002509,130.90121972)
\curveto(194.85242093,131.19288638)(194.78861884,131.63950095)(194.78861884,132.24106344)
\lineto(194.78861884,137.85564663)
\lineto(193.46700429,137.85564663)
\lineto(193.46700429,138.15642787)
\curveto(193.80120567,138.29010843)(194.14148344,138.51493481)(194.4878376,138.83090702)
\curveto(194.84026815,139.15295563)(195.15320217,139.53272992)(195.42663966,139.97022991)
\curveto(195.5663966,140.20113268)(195.76084104,140.6234417)(196.00997298,141.23715696)
\closepath
}
}
{
\newrgbcolor{curcolor}{0 0 0}
\pscustom[linestyle=none,fillstyle=solid,fillcolor=curcolor]
{
\newpath
\moveto(209.11674349,138.73976119)
\lineto(209.11674349,135.89601126)
\lineto(208.81596224,135.89601126)
\curveto(208.58505947,136.78924041)(208.28731642,137.39687928)(207.9227331,137.71892789)
\curveto(207.56422616,138.04097649)(207.10545881,138.20200079)(206.54643105,138.20200079)
\curveto(206.12108384,138.20200079)(205.77776787,138.0895876)(205.51648316,137.86476122)
\curveto(205.25519844,137.63993483)(205.12455609,137.39080289)(205.12455609,137.1173654)
\curveto(205.12455609,136.77708763)(205.22177831,136.48542097)(205.41622274,136.24236542)
\curveto(205.6045908,135.99323348)(205.98740329,135.72891057)(206.56466022,135.44939669)
\lineto(207.89538935,134.80226129)
\curveto(209.12889626,134.20069881)(209.74564972,133.40773008)(209.74564972,132.4233551)
\curveto(209.74564972,131.66380651)(209.45702125,131.05009125)(208.87976432,130.58220931)
\curveto(208.30858378,130.12040377)(207.66752477,129.889501)(206.95658729,129.889501)
\curveto(206.44617064,129.889501)(205.86283732,129.98064683)(205.20658733,130.16293849)
\curveto(205.00606651,130.22370238)(204.84200401,130.25408432)(204.71439985,130.25408432)
\curveto(204.5746429,130.25408432)(204.46526791,130.17509127)(204.38627485,130.01710516)
\lineto(204.08549361,130.01710516)
\lineto(204.08549361,132.99757384)
\lineto(204.38627485,132.99757384)
\curveto(204.55641374,132.14687941)(204.88150054,131.5058204)(205.36153525,131.0743968)
\curveto(205.84156996,130.6429732)(206.37933036,130.4272614)(206.97481646,130.4272614)
\curveto(207.39408728,130.4272614)(207.73436505,130.54878918)(207.99564976,130.79184473)
\curveto(208.26301087,131.04097666)(208.39669142,131.33871971)(208.39669142,131.68507387)
\curveto(208.39669142,132.10434469)(208.2478199,132.45677524)(207.95007685,132.74236551)
\curveto(207.65841019,133.02795578)(207.07203868,133.38950091)(206.19096231,133.8270009)
\curveto(205.30988594,134.26450089)(204.73262901,134.65946616)(204.45919152,135.0118967)
\curveto(204.18575403,135.35825086)(204.04903528,135.79575085)(204.04903528,136.32439667)
\curveto(204.04903528,137.0110286)(204.28297625,137.58524733)(204.75085818,138.04705288)
\curveto(205.2248165,138.50885842)(205.83549357,138.73976119)(206.58288938,138.73976119)
\curveto(206.91101437,138.73976119)(207.30901784,138.66988272)(207.77689977,138.53012578)
\curveto(208.08679559,138.43897995)(208.29339281,138.39340704)(208.39669142,138.39340704)
\curveto(208.49391364,138.39340704)(208.5698685,138.4146744)(208.624556,138.45720912)
\curveto(208.6792435,138.49974384)(208.74304558,138.59392786)(208.81596224,138.73976119)
\closepath
}
}
{
\newrgbcolor{curcolor}{0 0 0}
\pscustom[linestyle=none,fillstyle=solid,fillcolor=curcolor]
{
\newpath
\moveto(212.38887882,135.34913628)
\curveto(212.38280243,134.10955298)(212.68358367,133.13733078)(213.29122255,132.43246968)
\curveto(213.89886142,131.72760859)(214.6128371,131.37517804)(215.43314958,131.37517804)
\curveto(215.98002456,131.37517804)(216.45398289,131.52404957)(216.85502454,131.82179262)
\curveto(217.26214259,132.12561205)(217.60242036,132.6421051)(217.87585785,133.37127174)
\lineto(218.15840993,133.18898008)
\curveto(218.03080576,132.35651482)(217.66014605,131.59696623)(217.04643079,130.91033431)
\curveto(216.43271552,130.22977877)(215.66405235,129.889501)(214.74044126,129.889501)
\curveto(213.73783712,129.889501)(212.87802811,130.27838988)(212.16101424,131.05616764)
\curveto(211.45007676,131.84002178)(211.09460802,132.89123703)(211.09460802,134.20981339)
\curveto(211.09460802,135.63776474)(211.45919134,136.74974388)(212.18835799,137.54575081)
\curveto(212.92360103,138.34783412)(213.84417392,138.74887578)(214.95007667,138.74887578)
\curveto(215.88584054,138.74887578)(216.65450371,138.43897995)(217.2560662,137.8191883)
\curveto(217.85762868,137.20547304)(218.15840993,136.38212236)(218.15840993,135.34913628)
\closepath
\moveto(212.38887882,135.8777821)
\lineto(216.25346206,135.8777821)
\curveto(216.22308011,136.41250431)(216.15927803,136.78924041)(216.06205581,137.0079904)
\curveto(215.91014609,137.34826817)(215.68228152,137.61562928)(215.37846208,137.81007372)
\curveto(215.08071903,138.00451816)(214.76778501,138.10174038)(214.43966002,138.10174038)
\curveto(213.93531975,138.10174038)(213.48262879,137.90425774)(213.08158714,137.50929247)
\curveto(212.68662187,137.12040359)(212.4557191,136.5765668)(212.38887882,135.8777821)
\closepath
}
}
{
\newrgbcolor{curcolor}{0 0 0}
\pscustom[linestyle=none,fillstyle=solid,fillcolor=curcolor]
{
\newpath
\moveto(221.69486817,141.23715696)
\lineto(221.69486817,138.49366745)
\lineto(223.64538896,138.49366745)
\lineto(223.64538896,137.85564663)
\lineto(221.69486817,137.85564663)
\lineto(221.69486817,132.44158427)
\curveto(221.69486817,131.90078567)(221.77082303,131.53620235)(221.92273275,131.34783429)
\curveto(222.08071886,131.15946624)(222.28123968,131.06528222)(222.52429523,131.06528222)
\curveto(222.72481606,131.06528222)(222.9192605,131.12604611)(223.10762855,131.24757388)
\curveto(223.2959966,131.37517804)(223.44182993,131.5605079)(223.54512854,131.80356345)
\lineto(223.90059728,131.80356345)
\curveto(223.68792368,131.20807735)(223.38714243,130.75842459)(222.99825356,130.45460515)
\curveto(222.60936468,130.1568621)(222.20832302,130.00799058)(221.79512859,130.00799058)
\curveto(221.5156147,130.00799058)(221.24217721,130.08394544)(220.97481611,130.23585516)
\curveto(220.707455,130.39384126)(220.50997237,130.61562945)(220.3823682,130.90121972)
\curveto(220.25476404,131.19288638)(220.19096196,131.63950095)(220.19096196,132.24106344)
\lineto(220.19096196,137.85564663)
\lineto(218.86934741,137.85564663)
\lineto(218.86934741,138.15642787)
\curveto(219.20354879,138.29010843)(219.54382656,138.51493481)(219.89018072,138.83090702)
\curveto(220.24261126,139.15295563)(220.55554528,139.53272992)(220.82898278,139.97022991)
\curveto(220.96873972,140.20113268)(221.16318416,140.6234417)(221.41231609,141.23715696)
\closepath
}
}
{
\newrgbcolor{curcolor}{0 0 0}
\pscustom[linewidth=1.04055195,linecolor=curcolor]
{
\newpath
\moveto(414.11429659,175.26041605)
\lineto(467.62240192,150.44317667)
}
}
{
\newrgbcolor{curcolor}{0 0 0}
\pscustom[linestyle=none,fillstyle=solid,fillcolor=curcolor]
{
\newpath
\moveto(458.18276402,154.8213125)
\lineto(456.1581632,160.34842199)
\lineto(467.62240192,150.44317667)
\lineto(452.65565453,152.79671168)
\closepath
}
}
{
\newrgbcolor{curcolor}{0 0 0}
\pscustom[linewidth=1.10992212,linecolor=curcolor]
{
\newpath
\moveto(458.18276402,154.8213125)
\lineto(456.1581632,160.34842199)
\lineto(467.62240192,150.44317667)
\lineto(452.65565453,152.79671168)
\closepath
}
}
{
\newrgbcolor{curcolor}{0 0 0}
\pscustom[linewidth=1.04939331,linecolor=curcolor]
{
\newpath
\moveto(512.74626746,176.25857602)
\lineto(467.62240192,150.44317667)
}
}
{
\newrgbcolor{curcolor}{0 0 0}
\pscustom[linestyle=none,fillstyle=solid,fillcolor=curcolor]
{
\newpath
\moveto(476.73104857,155.65424066)
\lineto(482.45893283,154.0952076)
\lineto(467.62240192,150.44317667)
\lineto(478.29008164,161.38212492)
\closepath
}
}
{
\newrgbcolor{curcolor}{0 0 0}
\pscustom[linewidth=1.1193529,linecolor=curcolor]
{
\newpath
\moveto(476.73104857,155.65424066)
\lineto(482.45893283,154.0952076)
\lineto(467.62240192,150.44317667)
\lineto(478.29008164,161.38212492)
\closepath
}
}
{
\newrgbcolor{curcolor}{0 0 0}
\pscustom[linestyle=none,fillstyle=solid,fillcolor=curcolor]
{
\newpath
\moveto(132.91574972,64.27684671)
\curveto(132.50863167,63.8514995)(132.11062821,63.54464187)(131.72173933,63.35627382)
\curveto(131.33285045,63.17398216)(130.91357963,63.08283633)(130.46392686,63.08283633)
\curveto(129.55246855,63.08283633)(128.75646163,63.46261062)(128.07590609,64.22215921)
\curveto(127.39535055,64.98778419)(127.05507278,65.96912098)(127.05507278,67.16616956)
\curveto(127.05507278,68.36321814)(127.43180888,69.45696811)(128.18528109,70.44741947)
\curveto(128.93875329,71.44394723)(129.9079373,71.9422111)(131.0928331,71.9422111)
\curveto(131.82807614,71.9422111)(132.43571501,71.70827014)(132.91574972,71.2403882)
\lineto(132.91574972,72.78075275)
\curveto(132.91574972,73.73474578)(132.89144417,74.32111729)(132.84283306,74.53986729)
\curveto(132.80029833,74.75861728)(132.73041986,74.90748881)(132.63319764,74.98648186)
\curveto(132.53597542,75.06547491)(132.41444765,75.10497144)(132.26861432,75.10497144)
\curveto(132.11062821,75.10497144)(131.9009928,75.05636033)(131.63970809,74.95913811)
\lineto(131.5212185,75.27814852)
\lineto(134.00949969,76.29898183)
\lineto(134.41965593,76.29898183)
\lineto(134.41965593,66.64663832)
\curveto(134.41965593,65.66833973)(134.44092329,65.06981544)(134.48345801,64.85106545)
\curveto(134.53206912,64.63839184)(134.60498579,64.48952032)(134.70220801,64.40445088)
\curveto(134.80550662,64.31938143)(134.9239962,64.27684671)(135.05767675,64.27684671)
\curveto(135.22173925,64.27684671)(135.44048924,64.32849602)(135.71392673,64.43179462)
\lineto(135.81418715,64.11278422)
\lineto(133.33502054,63.08283633)
\lineto(132.91574972,63.08283633)
\closepath
\moveto(132.91574972,64.91486753)
\lineto(132.91574972,69.21695076)
\curveto(132.87929139,69.63014519)(132.76991639,70.00688129)(132.58762473,70.34715906)
\curveto(132.40533307,70.68743683)(132.16227752,70.94264516)(131.85845808,71.11278404)
\curveto(131.56071503,71.28899931)(131.26904837,71.37710695)(130.9834581,71.37710695)
\curveto(130.44873589,71.37710695)(129.97173938,71.1370896)(129.55246855,70.65705489)
\curveto(128.99951718,70.02511046)(128.72304149,69.10149937)(128.72304149,67.88622162)
\curveto(128.72304149,66.6587911)(128.9904026,65.71695084)(129.5251248,65.06070086)
\curveto(130.05984701,64.41052726)(130.65533311,64.08544047)(131.31158309,64.08544047)
\curveto(131.86453447,64.08544047)(132.39925668,64.36191615)(132.91574972,64.91486753)
\closepath
}
}
{
\newrgbcolor{curcolor}{0 0 0}
\pscustom[linestyle=none,fillstyle=solid,fillcolor=curcolor]
{
\newpath
\moveto(138.47564541,76.29898183)
\curveto(138.73085374,76.29898183)(138.94656554,76.207836)(139.12278081,76.02554434)
\curveto(139.30507248,75.84932906)(139.39621831,75.63361726)(139.39621831,75.37840893)
\curveto(139.39621831,75.12320061)(139.30507248,74.90445061)(139.12278081,74.72215895)
\curveto(138.94656554,74.53986729)(138.73085374,74.44872146)(138.47564541,74.44872146)
\curveto(138.22043709,74.44872146)(138.00168709,74.53986729)(137.81939543,74.72215895)
\curveto(137.63710377,74.90445061)(137.54595794,75.12320061)(137.54595794,75.37840893)
\curveto(137.54595794,75.63361726)(137.63406557,75.84932906)(137.81028085,76.02554434)
\curveto(137.99257251,76.207836)(138.2143607,76.29898183)(138.47564541,76.29898183)
\closepath
\moveto(139.23215581,71.93309652)
\lineto(139.23215581,65.22476336)
\curveto(139.23215581,64.70219392)(139.26861414,64.35280157)(139.34153081,64.1765863)
\curveto(139.42052386,64.00644741)(139.53293705,63.87884325)(139.67877038,63.79377381)
\curveto(139.8306801,63.70870437)(140.1041176,63.66616964)(140.49908286,63.66616964)
\lineto(140.49908286,63.33804465)
\lineto(136.44309338,63.33804465)
\lineto(136.44309338,63.66616964)
\curveto(136.85021143,63.66616964)(137.12364892,63.70566617)(137.26340586,63.78465922)
\curveto(137.4031628,63.86365228)(137.5125378,63.99429464)(137.59153085,64.1765863)
\curveto(137.6766003,64.35887796)(137.71913502,64.70827031)(137.71913502,65.22476336)
\lineto(137.71913502,68.44221119)
\curveto(137.71913502,69.34759311)(137.69179127,69.93396463)(137.63710377,70.20132573)
\curveto(137.59456905,70.39577017)(137.52772877,70.52945072)(137.43658294,70.60236739)
\curveto(137.34543711,70.68136044)(137.22087114,70.72085697)(137.06288503,70.72085697)
\curveto(136.89274615,70.72085697)(136.68614893,70.67528405)(136.44309338,70.58413822)
\lineto(136.31548922,70.91226321)
\lineto(138.83111416,71.93309652)
\closepath
}
}
{
\newrgbcolor{curcolor}{0 0 0}
\pscustom[linestyle=none,fillstyle=solid,fillcolor=curcolor]
{
\newpath
\moveto(143.98085361,71.93309652)
\lineto(143.98085361,70.0554924)
\curveto(144.67963831,71.30722848)(145.39665219,71.93309652)(146.13189522,71.93309652)
\curveto(146.4660966,71.93309652)(146.74257229,71.82979791)(146.96132229,71.6232007)
\curveto(147.18007228,71.42267987)(147.28944728,71.1887389)(147.28944728,70.9213778)
\curveto(147.28944728,70.68439864)(147.21045422,70.48387781)(147.05246812,70.31981531)
\curveto(146.89448201,70.15575282)(146.70611396,70.07372157)(146.48736396,70.07372157)
\curveto(146.27469036,70.07372157)(146.034673,70.17702018)(145.7673119,70.38361739)
\curveto(145.50602718,70.596291)(145.31158274,70.7026278)(145.18397858,70.7026278)
\curveto(145.07460358,70.7026278)(144.956114,70.64186391)(144.82850984,70.52033614)
\curveto(144.55507235,70.2712042)(144.27252027,69.86104796)(143.98085361,69.28986742)
\lineto(143.98085361,65.28856544)
\curveto(143.98085361,64.82675989)(144.0385793,64.47736754)(144.15403069,64.24038838)
\curveto(144.23302374,64.07632588)(144.37278068,63.93960714)(144.57330151,63.83023214)
\curveto(144.77382234,63.72085714)(145.06245081,63.66616964)(145.43918691,63.66616964)
\lineto(145.43918691,63.33804465)
\lineto(141.16444743,63.33804465)
\lineto(141.16444743,63.66616964)
\curveto(141.58979464,63.66616964)(141.90576686,63.73300992)(142.11236407,63.86669047)
\curveto(142.26427379,63.96391269)(142.37061059,64.11886061)(142.43137448,64.33153421)
\curveto(142.46175643,64.43483282)(142.4769474,64.72953767)(142.4769474,65.21564877)
\lineto(142.4769474,68.45132577)
\curveto(142.4769474,69.42354797)(142.45568004,70.0008049)(142.41314532,70.18309656)
\curveto(142.37668698,70.37146462)(142.30377032,70.50818336)(142.19439532,70.5932528)
\curveto(142.09109671,70.67832225)(141.96045436,70.72085697)(141.80246825,70.72085697)
\curveto(141.6141002,70.72085697)(141.40142659,70.67528405)(141.16444743,70.58413822)
\lineto(141.0733016,70.91226321)
\lineto(143.59804112,71.93309652)
\closepath
}
}
{
\newrgbcolor{curcolor}{0 0 0}
\pscustom[linestyle=none,fillstyle=solid,fillcolor=curcolor]
{
\newpath
\moveto(220.37503708,71.03075279)
\lineto(220.37503708,65.54377376)
\curveto(220.37503708,64.76599601)(220.46010653,64.27380852)(220.63024541,64.0672113)
\curveto(220.85507179,63.7998502)(221.15585304,63.66616964)(221.53258914,63.66616964)
\lineto(222.28909954,63.66616964)
\lineto(222.28909954,63.33804465)
\lineto(217.30342258,63.33804465)
\lineto(217.30342258,63.66616964)
\lineto(217.67712049,63.66616964)
\curveto(217.92017603,63.66616964)(218.14196422,63.72693353)(218.34248505,63.84846131)
\curveto(218.54300588,63.96998908)(218.67972463,64.13405158)(218.75264129,64.34064879)
\curveto(218.83163435,64.54724601)(218.87113087,64.94828767)(218.87113087,65.54377376)
\lineto(218.87113087,71.03075279)
\lineto(217.24873508,71.03075279)
\lineto(217.24873508,71.68700278)
\lineto(218.87113087,71.68700278)
\lineto(218.87113087,72.23387776)
\curveto(218.87113087,73.06634302)(219.00481142,73.77120411)(219.27217253,74.34846104)
\curveto(219.53953363,74.92571797)(219.94665168,75.39056171)(220.49352666,75.74299226)
\curveto(221.04647804,76.10149919)(221.66626969,76.28075266)(222.35290162,76.28075266)
\curveto(222.99092244,76.28075266)(223.57729395,76.07415545)(224.11201616,75.66096101)
\curveto(224.4644467,75.38752352)(224.64066198,75.08066589)(224.64066198,74.74038812)
\curveto(224.64066198,74.55809646)(224.56166892,74.38491938)(224.40368282,74.22085688)
\curveto(224.24569671,74.06287077)(224.07555783,73.98387772)(223.89326616,73.98387772)
\curveto(223.75350922,73.98387772)(223.6046377,74.03248883)(223.44665159,74.12971105)
\curveto(223.29474187,74.23300966)(223.10637382,74.44872146)(222.88154744,74.77684645)
\curveto(222.65672106,75.11104783)(222.45012384,75.33587421)(222.26175579,75.4513256)
\curveto(222.07338774,75.56677699)(221.86375232,75.62450268)(221.63284955,75.62450268)
\curveto(221.35333567,75.62450268)(221.11635651,75.54854782)(220.92191207,75.3966381)
\curveto(220.72746763,75.25080477)(220.58771069,75.019902)(220.50264125,74.70392978)
\curveto(220.41757181,74.39403396)(220.37503708,73.58891245)(220.37503708,72.28856526)
\lineto(220.37503708,71.68700278)
\lineto(222.5260787,71.68700278)
\lineto(222.5260787,71.03075279)
\closepath
}
}
{
\newrgbcolor{curcolor}{0 0 0}
\pscustom[linestyle=none,fillstyle=solid,fillcolor=curcolor]
{
\newpath
\moveto(225.45185987,76.29898183)
\curveto(225.7070682,76.29898183)(225.92278,76.207836)(226.09899527,76.02554434)
\curveto(226.28128694,75.84932906)(226.37243277,75.63361726)(226.37243277,75.37840893)
\curveto(226.37243277,75.12320061)(226.28128694,74.90445061)(226.09899527,74.72215895)
\curveto(225.92278,74.53986729)(225.7070682,74.44872146)(225.45185987,74.44872146)
\curveto(225.19665155,74.44872146)(224.97790155,74.53986729)(224.79560989,74.72215895)
\curveto(224.61331823,74.90445061)(224.5221724,75.12320061)(224.5221724,75.37840893)
\curveto(224.5221724,75.63361726)(224.61028003,75.84932906)(224.78649531,76.02554434)
\curveto(224.96878697,76.207836)(225.19057516,76.29898183)(225.45185987,76.29898183)
\closepath
\moveto(226.20837027,71.93309652)
\lineto(226.20837027,65.22476336)
\curveto(226.20837027,64.70219392)(226.2448286,64.35280157)(226.31774527,64.1765863)
\curveto(226.39673832,64.00644741)(226.50915151,63.87884325)(226.65498484,63.79377381)
\curveto(226.80689456,63.70870437)(227.08033206,63.66616964)(227.47529732,63.66616964)
\lineto(227.47529732,63.33804465)
\lineto(223.41930784,63.33804465)
\lineto(223.41930784,63.66616964)
\curveto(223.82642589,63.66616964)(224.09986338,63.70566617)(224.23962032,63.78465922)
\curveto(224.37937726,63.86365228)(224.48875226,63.99429464)(224.56774531,64.1765863)
\curveto(224.65281476,64.35887796)(224.69534948,64.70827031)(224.69534948,65.22476336)
\lineto(224.69534948,68.44221119)
\curveto(224.69534948,69.34759311)(224.66800573,69.93396463)(224.61331823,70.20132573)
\curveto(224.57078351,70.39577017)(224.50394323,70.52945072)(224.4127974,70.60236739)
\curveto(224.32165157,70.68136044)(224.1970856,70.72085697)(224.03909949,70.72085697)
\curveto(223.86896061,70.72085697)(223.66236339,70.67528405)(223.41930784,70.58413822)
\lineto(223.29170368,70.91226321)
\lineto(225.80732862,71.93309652)
\closepath
}
}
{
\newrgbcolor{curcolor}{0 0 0}
\pscustom[linestyle=none,fillstyle=solid,fillcolor=curcolor]
{
\newpath
\moveto(231.38545348,76.29898183)
\lineto(231.38545348,65.22476336)
\curveto(231.38545348,64.70219392)(231.42191181,64.35583977)(231.49482847,64.18570088)
\curveto(231.57382153,64.015562)(231.69231111,63.88491964)(231.85029721,63.79377381)
\curveto(232.00828332,63.70870437)(232.30298818,63.66616964)(232.73441178,63.66616964)
\lineto(232.73441178,63.33804465)
\lineto(228.64196396,63.33804465)
\lineto(228.64196396,63.66616964)
\curveto(229.02477645,63.66616964)(229.28606117,63.70566617)(229.42581811,63.78465922)
\curveto(229.56557505,63.86365228)(229.67495005,63.99429464)(229.7539431,64.1765863)
\curveto(229.83293615,64.35887796)(229.87243268,64.70827031)(229.87243268,65.22476336)
\lineto(229.87243268,72.8080965)
\curveto(229.87243268,73.74993675)(229.85116532,74.32719368)(229.8086306,74.53986729)
\curveto(229.76609588,74.75861728)(229.69621741,74.90748881)(229.59899519,74.98648186)
\curveto(229.50784936,75.06547491)(229.38935978,75.10497144)(229.24352645,75.10497144)
\curveto(229.08554034,75.10497144)(228.88501951,75.05636033)(228.64196396,74.95913811)
\lineto(228.48701605,75.27814852)
\lineto(230.97529724,76.29898183)
\closepath
}
}
{
\newrgbcolor{curcolor}{0 0 0}
\pscustom[linestyle=none,fillstyle=solid,fillcolor=curcolor]
{
\newpath
\moveto(235.10420338,68.54247161)
\curveto(235.09812699,67.3028883)(235.39890824,66.33066611)(236.00654711,65.62580501)
\curveto(236.61418598,64.92094392)(237.32816166,64.56851337)(238.14847414,64.56851337)
\curveto(238.69534913,64.56851337)(239.16930745,64.7173849)(239.5703491,65.01512794)
\curveto(239.97746715,65.31894738)(240.31774492,65.83544042)(240.59118241,66.56460707)
\lineto(240.87373449,66.38231541)
\curveto(240.74613033,65.54985015)(240.37547061,64.79030156)(239.76175535,64.10366963)
\curveto(239.14804009,63.42311409)(238.37937691,63.08283633)(237.45576582,63.08283633)
\curveto(236.45316168,63.08283633)(235.59335268,63.4717252)(234.87633881,64.24950296)
\curveto(234.16540132,65.03335711)(233.80993258,66.08457236)(233.80993258,67.40314872)
\curveto(233.80993258,68.83110007)(234.17451591,69.94307921)(234.90368255,70.73908613)
\curveto(235.63892559,71.54116945)(236.55949849,71.9422111)(237.66540124,71.9422111)
\curveto(238.6011651,71.9422111)(239.36982828,71.63231528)(239.97139076,71.01252363)
\curveto(240.57295325,70.39880836)(240.87373449,69.57545769)(240.87373449,68.54247161)
\closepath
\moveto(235.10420338,69.07111743)
\lineto(238.96878662,69.07111743)
\curveto(238.93840468,69.60583963)(238.87460259,69.98257574)(238.77738037,70.20132573)
\curveto(238.62547066,70.5416035)(238.39760608,70.8089646)(238.09378664,71.00340904)
\curveto(237.79604359,71.19785348)(237.48310957,71.2950757)(237.15498458,71.2950757)
\curveto(236.65064432,71.2950757)(236.19795336,71.09759307)(235.7969117,70.7026278)
\curveto(235.40194643,70.31373892)(235.17104366,69.76990213)(235.10420338,69.07111743)
\closepath
}
}
{
\newrgbcolor{curcolor}{0 0 0}
\pscustom[linestyle=none,fillstyle=solid,fillcolor=curcolor]
{
\newpath
\moveto(316.42451385,64.27684671)
\curveto(316.0173958,63.8514995)(315.61939234,63.54464187)(315.23050346,63.35627382)
\curveto(314.84161458,63.17398216)(314.42234376,63.08283633)(313.97269099,63.08283633)
\curveto(313.06123268,63.08283633)(312.26522576,63.46261062)(311.58467022,64.22215921)
\curveto(310.90411468,64.98778419)(310.56383691,65.96912098)(310.56383691,67.16616956)
\curveto(310.56383691,68.36321814)(310.94057301,69.45696811)(311.69404522,70.44741947)
\curveto(312.44751742,71.44394723)(313.41670142,71.9422111)(314.60159723,71.9422111)
\curveto(315.33684027,71.9422111)(315.94447914,71.70827014)(316.42451385,71.2403882)
\lineto(316.42451385,72.78075275)
\curveto(316.42451385,73.73474578)(316.40020829,74.32111729)(316.35159718,74.53986729)
\curveto(316.30906246,74.75861728)(316.23918399,74.90748881)(316.14196177,74.98648186)
\curveto(316.04473955,75.06547491)(315.92321178,75.10497144)(315.77737845,75.10497144)
\curveto(315.61939234,75.10497144)(315.40975693,75.05636033)(315.14847221,74.95913811)
\lineto(315.02998263,75.27814852)
\lineto(317.51826382,76.29898183)
\lineto(317.92842006,76.29898183)
\lineto(317.92842006,66.64663832)
\curveto(317.92842006,65.66833973)(317.94968742,65.06981544)(317.99222214,64.85106545)
\curveto(318.04083325,64.63839184)(318.11374992,64.48952032)(318.21097214,64.40445088)
\curveto(318.31427075,64.31938143)(318.43276033,64.27684671)(318.56644088,64.27684671)
\curveto(318.73050338,64.27684671)(318.94925337,64.32849602)(319.22269086,64.43179462)
\lineto(319.32295128,64.11278422)
\lineto(316.84378467,63.08283633)
\lineto(316.42451385,63.08283633)
\closepath
\moveto(316.42451385,64.91486753)
\lineto(316.42451385,69.21695076)
\curveto(316.38805552,69.63014519)(316.27868052,70.00688129)(316.09638886,70.34715906)
\curveto(315.9140972,70.68743683)(315.67104165,70.94264516)(315.36722221,71.11278404)
\curveto(315.06947916,71.28899931)(314.7778125,71.37710695)(314.49222223,71.37710695)
\curveto(313.95750002,71.37710695)(313.48050351,71.1370896)(313.06123268,70.65705489)
\curveto(312.50828131,70.02511046)(312.23180562,69.10149937)(312.23180562,67.88622162)
\curveto(312.23180562,66.6587911)(312.49916673,65.71695084)(313.03388893,65.06070086)
\curveto(313.56861114,64.41052726)(314.16409724,64.08544047)(314.82034722,64.08544047)
\curveto(315.3732986,64.08544047)(315.90802081,64.36191615)(316.42451385,64.91486753)
\closepath
}
}
{
\newrgbcolor{curcolor}{0 0 0}
\pscustom[linestyle=none,fillstyle=solid,fillcolor=curcolor]
{
\newpath
\moveto(321.98440954,76.29898183)
\curveto(322.23961787,76.29898183)(322.45532967,76.207836)(322.63154494,76.02554434)
\curveto(322.81383661,75.84932906)(322.90498244,75.63361726)(322.90498244,75.37840893)
\curveto(322.90498244,75.12320061)(322.81383661,74.90445061)(322.63154494,74.72215895)
\curveto(322.45532967,74.53986729)(322.23961787,74.44872146)(321.98440954,74.44872146)
\curveto(321.72920122,74.44872146)(321.51045122,74.53986729)(321.32815956,74.72215895)
\curveto(321.1458679,74.90445061)(321.05472207,75.12320061)(321.05472207,75.37840893)
\curveto(321.05472207,75.63361726)(321.1428297,75.84932906)(321.31904498,76.02554434)
\curveto(321.50133664,76.207836)(321.72312483,76.29898183)(321.98440954,76.29898183)
\closepath
\moveto(322.74091994,71.93309652)
\lineto(322.74091994,65.22476336)
\curveto(322.74091994,64.70219392)(322.77737827,64.35280157)(322.85029494,64.1765863)
\curveto(322.92928799,64.00644741)(323.04170118,63.87884325)(323.18753451,63.79377381)
\curveto(323.33944423,63.70870437)(323.61288173,63.66616964)(324.00784699,63.66616964)
\lineto(324.00784699,63.33804465)
\lineto(319.95185751,63.33804465)
\lineto(319.95185751,63.66616964)
\curveto(320.35897556,63.66616964)(320.63241305,63.70566617)(320.77216999,63.78465922)
\curveto(320.91192693,63.86365228)(321.02130193,63.99429464)(321.10029498,64.1765863)
\curveto(321.18536443,64.35887796)(321.22789915,64.70827031)(321.22789915,65.22476336)
\lineto(321.22789915,68.44221119)
\curveto(321.22789915,69.34759311)(321.2005554,69.93396463)(321.1458679,70.20132573)
\curveto(321.10333318,70.39577017)(321.0364929,70.52945072)(320.94534707,70.60236739)
\curveto(320.85420124,70.68136044)(320.72963527,70.72085697)(320.57164916,70.72085697)
\curveto(320.40151028,70.72085697)(320.19491306,70.67528405)(319.95185751,70.58413822)
\lineto(319.82425335,70.91226321)
\lineto(322.33987829,71.93309652)
\closepath
}
}
{
\newrgbcolor{curcolor}{0 0 0}
\pscustom[linestyle=none,fillstyle=solid,fillcolor=curcolor]
{
\newpath
\moveto(327.48961774,71.93309652)
\lineto(327.48961774,70.0554924)
\curveto(328.18840244,71.30722848)(328.90541632,71.93309652)(329.64065935,71.93309652)
\curveto(329.97486073,71.93309652)(330.25133642,71.82979791)(330.47008642,71.6232007)
\curveto(330.68883641,71.42267987)(330.79821141,71.1887389)(330.79821141,70.9213778)
\curveto(330.79821141,70.68439864)(330.71921835,70.48387781)(330.56123225,70.31981531)
\curveto(330.40324614,70.15575282)(330.21487809,70.07372157)(329.99612809,70.07372157)
\curveto(329.78345449,70.07372157)(329.54343713,70.17702018)(329.27607603,70.38361739)
\curveto(329.01479131,70.596291)(328.82034687,70.7026278)(328.69274271,70.7026278)
\curveto(328.58336771,70.7026278)(328.46487813,70.64186391)(328.33727397,70.52033614)
\curveto(328.06383648,70.2712042)(327.7812844,69.86104796)(327.48961774,69.28986742)
\lineto(327.48961774,65.28856544)
\curveto(327.48961774,64.82675989)(327.54734343,64.47736754)(327.66279482,64.24038838)
\curveto(327.74178787,64.07632588)(327.88154481,63.93960714)(328.08206564,63.83023214)
\curveto(328.28258647,63.72085714)(328.57121493,63.66616964)(328.94795104,63.66616964)
\lineto(328.94795104,63.33804465)
\lineto(324.67321156,63.33804465)
\lineto(324.67321156,63.66616964)
\curveto(325.09855877,63.66616964)(325.41453099,63.73300992)(325.6211282,63.86669047)
\curveto(325.77303792,63.96391269)(325.87937472,64.11886061)(325.94013861,64.33153421)
\curveto(325.97052056,64.43483282)(325.98571153,64.72953767)(325.98571153,65.21564877)
\lineto(325.98571153,68.45132577)
\curveto(325.98571153,69.42354797)(325.96444417,70.0008049)(325.92190945,70.18309656)
\curveto(325.88545111,70.37146462)(325.81253445,70.50818336)(325.70315945,70.5932528)
\curveto(325.59986084,70.67832225)(325.46921848,70.72085697)(325.31123238,70.72085697)
\curveto(325.12286433,70.72085697)(324.91019072,70.67528405)(324.67321156,70.58413822)
\lineto(324.58206573,70.91226321)
\lineto(327.10680525,71.93309652)
\closepath
}
}
{
\newrgbcolor{curcolor}{0 0 0}
\pscustom[linestyle=none,fillstyle=solid,fillcolor=curcolor]
{
\newpath
\moveto(406.08714995,71.03075279)
\lineto(406.08714995,65.54377376)
\curveto(406.08714995,64.76599601)(406.1722194,64.27380852)(406.34235828,64.0672113)
\curveto(406.56718466,63.7998502)(406.86796591,63.66616964)(407.24470201,63.66616964)
\lineto(408.00121241,63.66616964)
\lineto(408.00121241,63.33804465)
\lineto(403.01553545,63.33804465)
\lineto(403.01553545,63.66616964)
\lineto(403.38923336,63.66616964)
\curveto(403.6322889,63.66616964)(403.85407709,63.72693353)(404.05459792,63.84846131)
\curveto(404.25511875,63.96998908)(404.3918375,64.13405158)(404.46475416,64.34064879)
\curveto(404.54374722,64.54724601)(404.58324374,64.94828767)(404.58324374,65.54377376)
\lineto(404.58324374,71.03075279)
\lineto(402.96084795,71.03075279)
\lineto(402.96084795,71.68700278)
\lineto(404.58324374,71.68700278)
\lineto(404.58324374,72.23387776)
\curveto(404.58324374,73.06634302)(404.71692429,73.77120411)(404.9842854,74.34846104)
\curveto(405.2516465,74.92571797)(405.65876455,75.39056171)(406.20563953,75.74299226)
\curveto(406.75859091,76.10149919)(407.37838256,76.28075266)(408.06501449,76.28075266)
\curveto(408.70303531,76.28075266)(409.28940682,76.07415545)(409.82412903,75.66096101)
\curveto(410.17655957,75.38752352)(410.35277485,75.08066589)(410.35277485,74.74038812)
\curveto(410.35277485,74.55809646)(410.27378179,74.38491938)(410.11579569,74.22085688)
\curveto(409.95780958,74.06287077)(409.7876707,73.98387772)(409.60537903,73.98387772)
\curveto(409.46562209,73.98387772)(409.31675057,74.03248883)(409.15876446,74.12971105)
\curveto(409.00685474,74.23300966)(408.81848669,74.44872146)(408.59366031,74.77684645)
\curveto(408.36883393,75.11104783)(408.16223671,75.33587421)(407.97386866,75.4513256)
\curveto(407.78550061,75.56677699)(407.5758652,75.62450268)(407.34496242,75.62450268)
\curveto(407.06544854,75.62450268)(406.82846938,75.54854782)(406.63402494,75.3966381)
\curveto(406.4395805,75.25080477)(406.29982356,75.019902)(406.21475412,74.70392978)
\curveto(406.12968468,74.39403396)(406.08714995,73.58891245)(406.08714995,72.28856526)
\lineto(406.08714995,71.68700278)
\lineto(408.23819157,71.68700278)
\lineto(408.23819157,71.03075279)
\closepath
}
}
{
\newrgbcolor{curcolor}{0 0 0}
\pscustom[linestyle=none,fillstyle=solid,fillcolor=curcolor]
{
\newpath
\moveto(411.16397274,76.29898183)
\curveto(411.41918107,76.29898183)(411.63489287,76.207836)(411.81110814,76.02554434)
\curveto(411.99339981,75.84932906)(412.08454564,75.63361726)(412.08454564,75.37840893)
\curveto(412.08454564,75.12320061)(411.99339981,74.90445061)(411.81110814,74.72215895)
\curveto(411.63489287,74.53986729)(411.41918107,74.44872146)(411.16397274,74.44872146)
\curveto(410.90876442,74.44872146)(410.69001442,74.53986729)(410.50772276,74.72215895)
\curveto(410.3254311,74.90445061)(410.23428527,75.12320061)(410.23428527,75.37840893)
\curveto(410.23428527,75.63361726)(410.3223929,75.84932906)(410.49860818,76.02554434)
\curveto(410.68089984,76.207836)(410.90268803,76.29898183)(411.16397274,76.29898183)
\closepath
\moveto(411.92048314,71.93309652)
\lineto(411.92048314,65.22476336)
\curveto(411.92048314,64.70219392)(411.95694147,64.35280157)(412.02985814,64.1765863)
\curveto(412.10885119,64.00644741)(412.22126438,63.87884325)(412.36709771,63.79377381)
\curveto(412.51900743,63.70870437)(412.79244493,63.66616964)(413.18741019,63.66616964)
\lineto(413.18741019,63.33804465)
\lineto(409.13142071,63.33804465)
\lineto(409.13142071,63.66616964)
\curveto(409.53853876,63.66616964)(409.81197625,63.70566617)(409.95173319,63.78465922)
\curveto(410.09149013,63.86365228)(410.20086513,63.99429464)(410.27985818,64.1765863)
\curveto(410.36492763,64.35887796)(410.40746235,64.70827031)(410.40746235,65.22476336)
\lineto(410.40746235,68.44221119)
\curveto(410.40746235,69.34759311)(410.3801186,69.93396463)(410.3254311,70.20132573)
\curveto(410.28289638,70.39577017)(410.2160561,70.52945072)(410.12491027,70.60236739)
\curveto(410.03376444,70.68136044)(409.90919847,70.72085697)(409.75121236,70.72085697)
\curveto(409.58107348,70.72085697)(409.37447626,70.67528405)(409.13142071,70.58413822)
\lineto(409.00381655,70.91226321)
\lineto(411.51944149,71.93309652)
\closepath
}
}
{
\newrgbcolor{curcolor}{0 0 0}
\pscustom[linestyle=none,fillstyle=solid,fillcolor=curcolor]
{
\newpath
\moveto(417.09756635,76.29898183)
\lineto(417.09756635,65.22476336)
\curveto(417.09756635,64.70219392)(417.13402468,64.35583977)(417.20694134,64.18570088)
\curveto(417.2859344,64.015562)(417.40442398,63.88491964)(417.56241008,63.79377381)
\curveto(417.72039619,63.70870437)(418.01510105,63.66616964)(418.44652465,63.66616964)
\lineto(418.44652465,63.33804465)
\lineto(414.35407683,63.33804465)
\lineto(414.35407683,63.66616964)
\curveto(414.73688932,63.66616964)(414.99817404,63.70566617)(415.13793098,63.78465922)
\curveto(415.27768792,63.86365228)(415.38706292,63.99429464)(415.46605597,64.1765863)
\curveto(415.54504902,64.35887796)(415.58454555,64.70827031)(415.58454555,65.22476336)
\lineto(415.58454555,72.8080965)
\curveto(415.58454555,73.74993675)(415.56327819,74.32719368)(415.52074347,74.53986729)
\curveto(415.47820875,74.75861728)(415.40833028,74.90748881)(415.31110806,74.98648186)
\curveto(415.21996223,75.06547491)(415.10147265,75.10497144)(414.95563932,75.10497144)
\curveto(414.79765321,75.10497144)(414.59713238,75.05636033)(414.35407683,74.95913811)
\lineto(414.19912892,75.27814852)
\lineto(416.68741011,76.29898183)
\closepath
}
}
{
\newrgbcolor{curcolor}{0 0 0}
\pscustom[linestyle=none,fillstyle=solid,fillcolor=curcolor]
{
\newpath
\moveto(420.81631625,68.54247161)
\curveto(420.81023986,67.3028883)(421.11102111,66.33066611)(421.71865998,65.62580501)
\curveto(422.32629885,64.92094392)(423.04027453,64.56851337)(423.86058701,64.56851337)
\curveto(424.407462,64.56851337)(424.88142032,64.7173849)(425.28246197,65.01512794)
\curveto(425.68958002,65.31894738)(426.02985779,65.83544042)(426.30329528,66.56460707)
\lineto(426.58584736,66.38231541)
\curveto(426.4582432,65.54985015)(426.08758348,64.79030156)(425.47386822,64.10366963)
\curveto(424.86015296,63.42311409)(424.09148978,63.08283633)(423.16787869,63.08283633)
\curveto(422.16527455,63.08283633)(421.30546555,63.4717252)(420.58845168,64.24950296)
\curveto(419.87751419,65.03335711)(419.52204545,66.08457236)(419.52204545,67.40314872)
\curveto(419.52204545,68.83110007)(419.88662878,69.94307921)(420.61579542,70.73908613)
\curveto(421.35103846,71.54116945)(422.27161136,71.9422111)(423.37751411,71.9422111)
\curveto(424.31327797,71.9422111)(425.08194115,71.63231528)(425.68350363,71.01252363)
\curveto(426.28506612,70.39880836)(426.58584736,69.57545769)(426.58584736,68.54247161)
\closepath
\moveto(420.81631625,69.07111743)
\lineto(424.68089949,69.07111743)
\curveto(424.65051755,69.60583963)(424.58671546,69.98257574)(424.48949324,70.20132573)
\curveto(424.33758353,70.5416035)(424.10971895,70.8089646)(423.80589951,71.00340904)
\curveto(423.50815646,71.19785348)(423.19522244,71.2950757)(422.86709745,71.2950757)
\curveto(422.36275719,71.2950757)(421.91006623,71.09759307)(421.50902457,70.7026278)
\curveto(421.1140593,70.31373892)(420.88315653,69.76990213)(420.81631625,69.07111743)
\closepath
}
}
{
\newrgbcolor{curcolor}{0 0 0}
\pscustom[linestyle=none,fillstyle=solid,fillcolor=curcolor]
{
\newpath
\moveto(590.48049658,64.27684671)
\curveto(590.07337854,63.8514995)(589.67537507,63.54464187)(589.28648619,63.35627382)
\curveto(588.89759732,63.17398216)(588.47832649,63.08283633)(588.02867373,63.08283633)
\curveto(587.11721542,63.08283633)(586.32120849,63.46261062)(585.64065295,64.22215921)
\curveto(584.96009741,64.98778419)(584.61981964,65.96912098)(584.61981964,67.16616956)
\curveto(584.61981964,68.36321814)(584.99655575,69.45696811)(585.75002795,70.44741947)
\curveto(586.50350015,71.44394723)(587.47268416,71.9422111)(588.65757996,71.9422111)
\curveto(589.392823,71.9422111)(590.00046187,71.70827014)(590.48049658,71.2403882)
\lineto(590.48049658,72.78075275)
\curveto(590.48049658,73.73474578)(590.45619103,74.32111729)(590.40757992,74.53986729)
\curveto(590.3650452,74.75861728)(590.29516673,74.90748881)(590.19794451,74.98648186)
\curveto(590.10072229,75.06547491)(589.97919451,75.10497144)(589.83336118,75.10497144)
\curveto(589.67537507,75.10497144)(589.46573966,75.05636033)(589.20445495,74.95913811)
\lineto(589.08596537,75.27814852)
\lineto(591.57424655,76.29898183)
\lineto(591.98440279,76.29898183)
\lineto(591.98440279,66.64663832)
\curveto(591.98440279,65.66833973)(592.00567015,65.06981544)(592.04820488,64.85106545)
\curveto(592.09681599,64.63839184)(592.16973265,64.48952032)(592.26695487,64.40445088)
\curveto(592.37025348,64.31938143)(592.48874306,64.27684671)(592.62242361,64.27684671)
\curveto(592.78648611,64.27684671)(593.0052361,64.32849602)(593.2786736,64.43179462)
\lineto(593.37893401,64.11278422)
\lineto(590.8997674,63.08283633)
\lineto(590.48049658,63.08283633)
\closepath
\moveto(590.48049658,64.91486753)
\lineto(590.48049658,69.21695076)
\curveto(590.44403825,69.63014519)(590.33466325,70.00688129)(590.15237159,70.34715906)
\curveto(589.97007993,70.68743683)(589.72702438,70.94264516)(589.42320494,71.11278404)
\curveto(589.12546189,71.28899931)(588.83379523,71.37710695)(588.54820496,71.37710695)
\curveto(588.01348275,71.37710695)(587.53648624,71.1370896)(587.11721542,70.65705489)
\curveto(586.56426404,70.02511046)(586.28778835,69.10149937)(586.28778835,67.88622162)
\curveto(586.28778835,66.6587911)(586.55514946,65.71695084)(587.08987167,65.06070086)
\curveto(587.62459388,64.41052726)(588.22007997,64.08544047)(588.87632996,64.08544047)
\curveto(589.42928133,64.08544047)(589.96400354,64.36191615)(590.48049658,64.91486753)
\closepath
}
}
{
\newrgbcolor{curcolor}{0 0 0}
\pscustom[linestyle=none,fillstyle=solid,fillcolor=curcolor]
{
\newpath
\moveto(596.04039228,76.29898183)
\curveto(596.2956006,76.29898183)(596.5113124,76.207836)(596.68752768,76.02554434)
\curveto(596.86981934,75.84932906)(596.96096517,75.63361726)(596.96096517,75.37840893)
\curveto(596.96096517,75.12320061)(596.86981934,74.90445061)(596.68752768,74.72215895)
\curveto(596.5113124,74.53986729)(596.2956006,74.44872146)(596.04039228,74.44872146)
\curveto(595.78518395,74.44872146)(595.56643395,74.53986729)(595.38414229,74.72215895)
\curveto(595.20185063,74.90445061)(595.1107048,75.12320061)(595.1107048,75.37840893)
\curveto(595.1107048,75.63361726)(595.19881244,75.84932906)(595.37502771,76.02554434)
\curveto(595.55731937,76.207836)(595.77910756,76.29898183)(596.04039228,76.29898183)
\closepath
\moveto(596.79690267,71.93309652)
\lineto(596.79690267,65.22476336)
\curveto(596.79690267,64.70219392)(596.83336101,64.35280157)(596.90627767,64.1765863)
\curveto(596.98527072,64.00644741)(597.09768392,63.87884325)(597.24351725,63.79377381)
\curveto(597.39542696,63.70870437)(597.66886446,63.66616964)(598.06382973,63.66616964)
\lineto(598.06382973,63.33804465)
\lineto(594.00784024,63.33804465)
\lineto(594.00784024,63.66616964)
\curveto(594.41495829,63.66616964)(594.68839578,63.70566617)(594.82815272,63.78465922)
\curveto(594.96790966,63.86365228)(595.07728466,63.99429464)(595.15627771,64.1765863)
\curveto(595.24134716,64.35887796)(595.28388188,64.70827031)(595.28388188,65.22476336)
\lineto(595.28388188,68.44221119)
\curveto(595.28388188,69.34759311)(595.25653813,69.93396463)(595.20185063,70.20132573)
\curveto(595.15931591,70.39577017)(595.09247563,70.52945072)(595.0013298,70.60236739)
\curveto(594.91018397,70.68136044)(594.785618,70.72085697)(594.62763189,70.72085697)
\curveto(594.45749301,70.72085697)(594.25089579,70.67528405)(594.00784024,70.58413822)
\lineto(593.88023608,70.91226321)
\lineto(596.39586102,71.93309652)
\closepath
}
}
{
\newrgbcolor{curcolor}{0 0 0}
\pscustom[linestyle=none,fillstyle=solid,fillcolor=curcolor]
{
\newpath
\moveto(601.54560047,71.93309652)
\lineto(601.54560047,70.0554924)
\curveto(602.24438518,71.30722848)(602.96139905,71.93309652)(603.69664208,71.93309652)
\curveto(604.03084347,71.93309652)(604.30731915,71.82979791)(604.52606915,71.6232007)
\curveto(604.74481914,71.42267987)(604.85419414,71.1887389)(604.85419414,70.9213778)
\curveto(604.85419414,70.68439864)(604.77520109,70.48387781)(604.61721498,70.31981531)
\curveto(604.45922887,70.15575282)(604.27086082,70.07372157)(604.05211083,70.07372157)
\curveto(603.83943722,70.07372157)(603.59941986,70.17702018)(603.33205876,70.38361739)
\curveto(603.07077404,70.596291)(602.87632961,70.7026278)(602.74872544,70.7026278)
\curveto(602.63935044,70.7026278)(602.52086086,70.64186391)(602.3932567,70.52033614)
\curveto(602.11981921,70.2712042)(601.83726713,69.86104796)(601.54560047,69.28986742)
\lineto(601.54560047,65.28856544)
\curveto(601.54560047,64.82675989)(601.60332616,64.47736754)(601.71877755,64.24038838)
\curveto(601.7977706,64.07632588)(601.93752755,63.93960714)(602.13804837,63.83023214)
\curveto(602.3385692,63.72085714)(602.62719767,63.66616964)(603.00393377,63.66616964)
\lineto(603.00393377,63.33804465)
\lineto(598.72919429,63.33804465)
\lineto(598.72919429,63.66616964)
\curveto(599.1545415,63.66616964)(599.47051372,63.73300992)(599.67711094,63.86669047)
\curveto(599.82902065,63.96391269)(599.93535746,64.11886061)(599.99612134,64.33153421)
\curveto(600.02650329,64.43483282)(600.04169426,64.72953767)(600.04169426,65.21564877)
\lineto(600.04169426,68.45132577)
\curveto(600.04169426,69.42354797)(600.0204269,70.0008049)(599.97789218,70.18309656)
\curveto(599.94143385,70.37146462)(599.86851718,70.50818336)(599.75914218,70.5932528)
\curveto(599.65584357,70.67832225)(599.52520122,70.72085697)(599.36721511,70.72085697)
\curveto(599.17884706,70.72085697)(598.96617345,70.67528405)(598.72919429,70.58413822)
\lineto(598.63804846,70.91226321)
\lineto(601.16278798,71.93309652)
\closepath
}
}
{
\newrgbcolor{curcolor}{0 0 0}
\pscustom[linestyle=none,fillstyle=solid,fillcolor=curcolor]
{
\newpath
\moveto(491.07070903,64.27684671)
\curveto(490.66359098,63.8514995)(490.26558752,63.54464187)(489.87669864,63.35627382)
\curveto(489.48780976,63.17398216)(489.06853894,63.08283633)(488.61888617,63.08283633)
\curveto(487.70742786,63.08283633)(486.91142094,63.46261062)(486.2308654,64.22215921)
\curveto(485.55030986,64.98778419)(485.21003209,65.96912098)(485.21003209,67.16616956)
\curveto(485.21003209,68.36321814)(485.58676819,69.45696811)(486.3402404,70.44741947)
\curveto(487.0937126,71.44394723)(488.0628966,71.9422111)(489.24779241,71.9422111)
\curveto(489.98303544,71.9422111)(490.59067432,71.70827014)(491.07070903,71.2403882)
\lineto(491.07070903,72.78075275)
\curveto(491.07070903,73.73474578)(491.04640347,74.32111729)(490.99779236,74.53986729)
\curveto(490.95525764,74.75861728)(490.88537917,74.90748881)(490.78815695,74.98648186)
\curveto(490.69093473,75.06547491)(490.56940696,75.10497144)(490.42357363,75.10497144)
\curveto(490.26558752,75.10497144)(490.05595211,75.05636033)(489.79466739,74.95913811)
\lineto(489.67617781,75.27814852)
\lineto(492.164459,76.29898183)
\lineto(492.57461524,76.29898183)
\lineto(492.57461524,66.64663832)
\curveto(492.57461524,65.66833973)(492.5958826,65.06981544)(492.63841732,64.85106545)
\curveto(492.68702843,64.63839184)(492.7599451,64.48952032)(492.85716732,64.40445088)
\curveto(492.96046593,64.31938143)(493.07895551,64.27684671)(493.21263606,64.27684671)
\curveto(493.37669855,64.27684671)(493.59544855,64.32849602)(493.86888604,64.43179462)
\lineto(493.96914646,64.11278422)
\lineto(491.48997985,63.08283633)
\lineto(491.07070903,63.08283633)
\closepath
\moveto(491.07070903,64.91486753)
\lineto(491.07070903,69.21695076)
\curveto(491.0342507,69.63014519)(490.9248757,70.00688129)(490.74258404,70.34715906)
\curveto(490.56029237,70.68743683)(490.31723682,70.94264516)(490.01341739,71.11278404)
\curveto(489.71567434,71.28899931)(489.42400768,71.37710695)(489.13841741,71.37710695)
\curveto(488.6036952,71.37710695)(488.12669868,71.1370896)(487.70742786,70.65705489)
\curveto(487.15447649,70.02511046)(486.8780008,69.10149937)(486.8780008,67.88622162)
\curveto(486.8780008,66.6587911)(487.1453619,65.71695084)(487.68008411,65.06070086)
\curveto(488.21480632,64.41052726)(488.81029242,64.08544047)(489.4665424,64.08544047)
\curveto(490.01949378,64.08544047)(490.55421599,64.36191615)(491.07070903,64.91486753)
\closepath
}
}
{
\newrgbcolor{curcolor}{0 0 0}
\pscustom[linestyle=none,fillstyle=solid,fillcolor=curcolor]
{
\newpath
\moveto(496.63060472,76.29898183)
\curveto(496.88581305,76.29898183)(497.10152485,76.207836)(497.27774012,76.02554434)
\curveto(497.46003178,75.84932906)(497.55117762,75.63361726)(497.55117762,75.37840893)
\curveto(497.55117762,75.12320061)(497.46003178,74.90445061)(497.27774012,74.72215895)
\curveto(497.10152485,74.53986729)(496.88581305,74.44872146)(496.63060472,74.44872146)
\curveto(496.3753964,74.44872146)(496.1566464,74.53986729)(495.97435474,74.72215895)
\curveto(495.79206308,74.90445061)(495.70091725,75.12320061)(495.70091725,75.37840893)
\curveto(495.70091725,75.63361726)(495.78902488,75.84932906)(495.96524016,76.02554434)
\curveto(496.14753182,76.207836)(496.36932001,76.29898183)(496.63060472,76.29898183)
\closepath
\moveto(497.38711512,71.93309652)
\lineto(497.38711512,65.22476336)
\curveto(497.38711512,64.70219392)(497.42357345,64.35280157)(497.49649012,64.1765863)
\curveto(497.57548317,64.00644741)(497.68789636,63.87884325)(497.83372969,63.79377381)
\curveto(497.98563941,63.70870437)(498.2590769,63.66616964)(498.65404217,63.66616964)
\lineto(498.65404217,63.33804465)
\lineto(494.59805269,63.33804465)
\lineto(494.59805269,63.66616964)
\curveto(495.00517074,63.66616964)(495.27860823,63.70566617)(495.41836517,63.78465922)
\curveto(495.55812211,63.86365228)(495.66749711,63.99429464)(495.74649016,64.1765863)
\curveto(495.8315596,64.35887796)(495.87409432,64.70827031)(495.87409432,65.22476336)
\lineto(495.87409432,68.44221119)
\curveto(495.87409432,69.34759311)(495.84675058,69.93396463)(495.79206308,70.20132573)
\curveto(495.74952836,70.39577017)(495.68268808,70.52945072)(495.59154225,70.60236739)
\curveto(495.50039642,70.68136044)(495.37583045,70.72085697)(495.21784434,70.72085697)
\curveto(495.04770546,70.72085697)(494.84110824,70.67528405)(494.59805269,70.58413822)
\lineto(494.47044853,70.91226321)
\lineto(496.98607346,71.93309652)
\closepath
}
}
{
\newrgbcolor{curcolor}{0 0 0}
\pscustom[linestyle=none,fillstyle=solid,fillcolor=curcolor]
{
\newpath
\moveto(502.13581292,71.93309652)
\lineto(502.13581292,70.0554924)
\curveto(502.83459762,71.30722848)(503.55161149,71.93309652)(504.28685453,71.93309652)
\curveto(504.62105591,71.93309652)(504.8975316,71.82979791)(505.11628159,71.6232007)
\curveto(505.33503159,71.42267987)(505.44440659,71.1887389)(505.44440659,70.9213778)
\curveto(505.44440659,70.68439864)(505.36541353,70.48387781)(505.20742742,70.31981531)
\curveto(505.04944132,70.15575282)(504.86107327,70.07372157)(504.64232327,70.07372157)
\curveto(504.42964967,70.07372157)(504.18963231,70.17702018)(503.92227121,70.38361739)
\curveto(503.66098649,70.596291)(503.46654205,70.7026278)(503.33893789,70.7026278)
\curveto(503.22956289,70.7026278)(503.11107331,70.64186391)(502.98346915,70.52033614)
\curveto(502.71003165,70.2712042)(502.42747958,69.86104796)(502.13581292,69.28986742)
\lineto(502.13581292,65.28856544)
\curveto(502.13581292,64.82675989)(502.19353861,64.47736754)(502.30899,64.24038838)
\curveto(502.38798305,64.07632588)(502.52773999,63.93960714)(502.72826082,63.83023214)
\curveto(502.92878165,63.72085714)(503.21741011,63.66616964)(503.59414621,63.66616964)
\lineto(503.59414621,63.33804465)
\lineto(499.31940674,63.33804465)
\lineto(499.31940674,63.66616964)
\curveto(499.74475395,63.66616964)(500.06072616,63.73300992)(500.26732338,63.86669047)
\curveto(500.4192331,63.96391269)(500.5255699,64.11886061)(500.58633379,64.33153421)
\curveto(500.61671573,64.43483282)(500.63190671,64.72953767)(500.63190671,65.21564877)
\lineto(500.63190671,68.45132577)
\curveto(500.63190671,69.42354797)(500.61063934,70.0008049)(500.56810462,70.18309656)
\curveto(500.53164629,70.37146462)(500.45872963,70.50818336)(500.34935463,70.5932528)
\curveto(500.24605602,70.67832225)(500.11541366,70.72085697)(499.95742756,70.72085697)
\curveto(499.7690595,70.72085697)(499.5563859,70.67528405)(499.31940674,70.58413822)
\lineto(499.22826091,70.91226321)
\lineto(501.75300043,71.93309652)
\closepath
}
}
\end{pspicture}
}
    %\captionsetup{width=0.75\linewidth}
    \caption{A sample ZFS pool, with two filesystems and three snapshots \cite{mckusick_zfs_2015_presentation}.
        When the snapshot and the filesystem are identical they point to the exact same object set,
        while once they diverge the snapshot continues pointing to the same object set, which might still point to many of the same
        objects if those files are unchanged between the snapshot and the filesystem.
        In this example filesystem A and its two snapshots are all different from each other, so each points to its own object set.
        Filesystem B, on the other hand, is the same as its one snapshot, so both point to the same object set.
    }
\label{fig:ZFSOnDiskExample}
\end{figure}

\section{VDEVs}
ZFS implements mirrors and RAID-like configurations, known as RAIDZ, through Virtual Devices, or VDEVs.
Storage in ZFS is implemented as a set of disks organized into a tree, with one root whose leaves are either
physical devices, mirrors, or RAIDZ devices of various levels of redundancy.
These mirror or RAIDZ devices have child devices which are the actual physical disks.
Physical disks, on the other hand, are always the leaf nodes of the tree as they are what actually stores the data.

\begin{figure}[H]
    \centering
    \resizebox{0.75\linewidth}{!}{%LaTeX with PSTricks extensions
%%Creator: Inkscape 1.0.2-2 (e86c870879, 2021-01-15)
%%Please note this file requires PSTricks extensions
\psset{xunit=.5pt,yunit=.5pt,runit=.5pt}
\begin{pspicture}(947.66272838,393.37698388)
{
\newrgbcolor{curcolor}{0.7019608 0.7019608 0.7019608}
\pscustom[linestyle=none,fillstyle=solid,fillcolor=curcolor,opacity=0.92623001]
{
\newpath
\moveto(442.07651963,392.72906146)
\lineto(555.71654581,392.72906146)
\lineto(555.71654581,333.37473532)
\lineto(442.07651963,333.37473532)
\closepath
}
}
{
\newrgbcolor{curcolor}{0 0 0}
\pscustom[linewidth=1.29584504,linecolor=curcolor]
{
\newpath
\moveto(442.07651963,392.72906146)
\lineto(555.71654581,392.72906146)
\lineto(555.71654581,333.37473532)
\lineto(442.07651963,333.37473532)
\closepath
}
}
{
\newrgbcolor{curcolor}{0 0 0}
\pscustom[linestyle=none,fillstyle=solid,fillcolor=curcolor]
{
\newpath
\moveto(481.15091654,347.3376145)
\lineto(476.13140127,347.3376145)
\lineto(466.40486974,358.90007768)
\lineto(460.95566834,358.90007768)
\lineto(460.95566834,347.3376145)
\lineto(457.08849315,347.3376145)
\lineto(457.08849315,376.41955315)
\lineto(465.23299847,376.41955315)
\curveto(466.99080537,376.41955315)(468.45564446,376.30236603)(469.62751573,376.06799177)
\curveto(470.799387,375.84663831)(471.85407114,375.44299376)(472.79156816,374.85705813)
\curveto(473.8462523,374.19299774)(474.66656219,373.35315667)(475.25249782,372.3375349)
\curveto(475.85145425,371.33493393)(476.15093246,370.05889632)(476.15093246,368.50942209)
\curveto(476.15093246,366.4130746)(475.62359039,364.65526769)(474.56890625,363.23600138)
\curveto(473.5142221,361.82975586)(472.06240381,360.76856132)(470.21345136,360.05241776)
\closepath
\moveto(472.10797658,368.23598546)
\curveto(472.10797658,369.06931614)(471.95823748,369.80499088)(471.65875926,370.44300968)
\curveto(471.37230184,371.09404928)(470.89053254,371.64092253)(470.21345136,372.08362946)
\curveto(469.65355731,372.46123242)(468.98949693,372.72164826)(468.22127021,372.86487697)
\curveto(467.45304349,373.02112647)(466.54809845,373.09925122)(465.5064351,373.09925122)
\lineto(460.95566834,373.09925122)
\lineto(460.95566834,362.12272367)
\lineto(464.8619059,362.12272367)
\curveto(466.08586034,362.12272367)(467.15356527,362.22689001)(468.0650207,362.43522268)
\curveto(468.97647614,362.65657614)(469.75121325,363.06022069)(470.38923205,363.64615632)
\curveto(470.97516769,364.19302958)(471.40485382,364.81802759)(471.67829045,365.52115035)
\curveto(471.96474787,366.23729391)(472.10797658,367.14223894)(472.10797658,368.23598546)
\closepath
}
}
{
\newrgbcolor{curcolor}{0 0 0}
\pscustom[linestyle=none,fillstyle=solid,fillcolor=curcolor]
{
\newpath
\moveto(502.22506893,358.2360173)
\curveto(502.22506893,354.68134112)(501.3136135,351.87536047)(499.49070264,349.81807535)
\curveto(497.66779177,347.76079023)(495.2263933,346.73214768)(492.16650721,346.73214768)
\curveto(489.08057953,346.73214768)(486.62616026,347.76079023)(484.8032494,349.81807535)
\curveto(482.99335933,351.87536047)(482.08841429,354.68134112)(482.08841429,358.2360173)
\curveto(482.08841429,361.79069348)(482.99335933,364.59667413)(484.8032494,366.65395925)
\curveto(486.62616026,368.72426515)(489.08057953,369.75941811)(492.16650721,369.75941811)
\curveto(495.2263933,369.75941811)(497.66779177,368.72426515)(499.49070264,366.65395925)
\curveto(501.3136135,364.59667413)(502.22506893,361.79069348)(502.22506893,358.2360173)
\closepath
\moveto(498.43601849,358.2360173)
\curveto(498.43601849,361.06152914)(497.88263484,363.15787663)(496.77586753,364.52505977)
\curveto(495.66910022,365.90526371)(494.13264678,366.59536568)(492.16650721,366.59536568)
\curveto(490.17432605,366.59536568)(488.62485182,365.90526371)(487.51808451,364.52505977)
\curveto(486.42433799,363.15787663)(485.87746473,361.06152914)(485.87746473,358.2360173)
\curveto(485.87746473,355.501651)(486.43084838,353.4248347)(487.53761569,352.00556839)
\curveto(488.644383,350.59932286)(490.18734684,349.8962001)(492.16650721,349.8962001)
\curveto(494.11962599,349.8962001)(495.64956903,350.59281247)(496.75633634,351.9860372)
\curveto(497.87612444,353.39228272)(498.43601849,355.47560942)(498.43601849,358.2360173)
\closepath
}
}
{
\newrgbcolor{curcolor}{0 0 0}
\pscustom[linestyle=none,fillstyle=solid,fillcolor=curcolor]
{
\newpath
\moveto(526.50233639,358.2360173)
\curveto(526.50233639,354.68134112)(525.59088096,351.87536047)(523.76797009,349.81807535)
\curveto(521.94505923,347.76079023)(519.50366075,346.73214768)(516.44377466,346.73214768)
\curveto(513.35784699,346.73214768)(510.90342772,347.76079023)(509.08051686,349.81807535)
\curveto(507.27062679,351.87536047)(506.36568175,354.68134112)(506.36568175,358.2360173)
\curveto(506.36568175,361.79069348)(507.27062679,364.59667413)(509.08051686,366.65395925)
\curveto(510.90342772,368.72426515)(513.35784699,369.75941811)(516.44377466,369.75941811)
\curveto(519.50366075,369.75941811)(521.94505923,368.72426515)(523.76797009,366.65395925)
\curveto(525.59088096,364.59667413)(526.50233639,361.79069348)(526.50233639,358.2360173)
\closepath
\moveto(522.71328595,358.2360173)
\curveto(522.71328595,361.06152914)(522.1599023,363.15787663)(521.05313499,364.52505977)
\curveto(519.94636768,365.90526371)(518.40991424,366.59536568)(516.44377466,366.59536568)
\curveto(514.45159351,366.59536568)(512.90211927,365.90526371)(511.79535196,364.52505977)
\curveto(510.70160545,363.15787663)(510.15473219,361.06152914)(510.15473219,358.2360173)
\curveto(510.15473219,355.501651)(510.70811584,353.4248347)(511.81488315,352.00556839)
\curveto(512.92165046,350.59932286)(514.4646143,349.8962001)(516.44377466,349.8962001)
\curveto(518.39689344,349.8962001)(519.92683649,350.59281247)(521.0336038,351.9860372)
\curveto(522.1533919,353.39228272)(522.71328595,355.47560942)(522.71328595,358.2360173)
\closepath
}
}
{
\newrgbcolor{curcolor}{0 0 0}
\pscustom[linestyle=none,fillstyle=solid,fillcolor=curcolor]
{
\newpath
\moveto(543.51399657,347.53292638)
\curveto(542.8238946,347.35063529)(542.06868867,347.20089618)(541.24837878,347.08370906)
\curveto(540.44108969,346.96652193)(539.71843574,346.90792837)(539.08041694,346.90792837)
\curveto(536.85386152,346.90792837)(535.16115858,347.50688479)(534.0023081,348.70479764)
\curveto(532.84345763,349.9027105)(532.26403239,351.8232773)(532.26403239,354.46649805)
\lineto(532.26403239,366.06802361)
\lineto(529.78357154,366.06802361)
\lineto(529.78357154,369.15395129)
\lineto(532.26403239,369.15395129)
\lineto(532.26403239,375.42346257)
\lineto(535.9358957,375.42346257)
\lineto(535.9358957,369.15395129)
\lineto(543.51399657,369.15395129)
\lineto(543.51399657,366.06802361)
\lineto(535.9358957,366.06802361)
\lineto(535.9358957,356.12664901)
\curveto(535.9358957,354.98081933)(535.96193728,354.08238469)(536.01402045,353.4313451)
\curveto(536.06610362,352.79332629)(536.2483947,352.19436987)(536.56089371,351.63447582)
\curveto(536.84735113,351.11364414)(537.23797488,350.72953078)(537.73276498,350.48213574)
\curveto(538.24057586,350.24776148)(539.00880258,350.13057436)(540.03744514,350.13057436)
\curveto(540.63640156,350.13057436)(541.26139957,350.2152095)(541.91243917,350.3844798)
\curveto(542.56347876,350.56677088)(543.03222727,350.71650999)(543.31868469,350.83369712)
\lineto(543.51399657,350.83369712)
\closepath
}
}
{
\newrgbcolor{curcolor}{0.7019608 0.7019608 0.7019608}
\pscustom[linestyle=none,fillstyle=solid,fillcolor=curcolor,opacity=0.92623001]
{
\newpath
\moveto(145.09835953,242.56439502)
\lineto(324.83762887,242.56439502)
\lineto(324.83762887,183.5393992)
\lineto(145.09835953,183.5393992)
\closepath
}
}
{
\newrgbcolor{curcolor}{0 0 0}
\pscustom[linewidth=1.62517795,linecolor=curcolor]
{
\newpath
\moveto(145.09835953,242.56439502)
\lineto(324.83762887,242.56439502)
\lineto(324.83762887,183.5393992)
\lineto(145.09835953,183.5393992)
\closepath
}
}
{
\newrgbcolor{curcolor}{0 0 0}
\pscustom[linestyle=none,fillstyle=solid,fillcolor=curcolor]
{
\newpath
\moveto(185.84400851,197.33761412)
\lineto(181.97683332,197.33761412)
\lineto(181.97683332,222.39612809)
\lineto(173.89092157,205.34540112)
\lineto(171.58624141,205.34540112)
\lineto(163.55892321,222.39612809)
\lineto(163.55892321,197.33761412)
\lineto(159.94565347,197.33761412)
\lineto(159.94565347,226.41955277)
\lineto(165.21907418,226.41955277)
\lineto(172.97295574,210.22819808)
\lineto(180.47293186,226.41955277)
\lineto(185.84400851,226.41955277)
\closepath
}
}
{
\newrgbcolor{curcolor}{0 0 0}
\pscustom[linestyle=none,fillstyle=solid,fillcolor=curcolor]
{
\newpath
\moveto(197.30881616,222.80628303)
\lineto(193.16820434,222.80628303)
\lineto(193.16820434,226.61486465)
\lineto(197.30881616,226.61486465)
\closepath
\moveto(197.0744419,197.33761412)
\lineto(193.40257859,197.33761412)
\lineto(193.40257859,219.15395091)
\lineto(197.0744419,219.15395091)
\closepath
}
}
{
\newrgbcolor{curcolor}{0 0 0}
\pscustom[linestyle=none,fillstyle=solid,fillcolor=curcolor]
{
\newpath
\moveto(217.95328178,215.15005741)
\lineto(217.7579699,215.15005741)
\curveto(217.21109665,215.28026532)(216.67724418,215.37141087)(216.1564125,215.42349404)
\curveto(215.64860162,215.48859799)(215.0431348,215.52114997)(214.34001204,215.52114997)
\curveto(213.20720314,215.52114997)(212.11345663,215.26724453)(211.05877248,214.75943365)
\curveto(210.00408834,214.26464356)(208.98846658,213.62011436)(208.01190719,212.82584606)
\lineto(208.01190719,197.33761412)
\lineto(204.34004388,197.33761412)
\lineto(204.34004388,219.15395091)
\lineto(208.01190719,219.15395091)
\lineto(208.01190719,215.93130492)
\curveto(209.47023588,217.10317619)(210.75278388,217.92999647)(211.85955118,218.41176577)
\curveto(212.97933929,218.90655586)(214.11865858,219.15395091)(215.27750905,219.15395091)
\curveto(215.91552785,219.15395091)(216.37776597,219.13441972)(216.66422339,219.09535734)
\curveto(216.95068081,219.06931576)(217.38036694,219.0107222)(217.95328178,218.91957665)
\closepath
}
}
{
\newrgbcolor{curcolor}{0 0 0}
\pscustom[linestyle=none,fillstyle=solid,fillcolor=curcolor]
{
\newpath
\moveto(235.02354142,215.15005741)
\lineto(234.82822954,215.15005741)
\curveto(234.28135628,215.28026532)(233.74750381,215.37141087)(233.22667214,215.42349404)
\curveto(232.71886126,215.48859799)(232.11339443,215.52114997)(231.41027167,215.52114997)
\curveto(230.27746278,215.52114997)(229.18371626,215.26724453)(228.12903212,214.75943365)
\curveto(227.07434798,214.26464356)(226.05872621,213.62011436)(225.08216682,212.82584606)
\lineto(225.08216682,197.33761412)
\lineto(221.41030351,197.33761412)
\lineto(221.41030351,219.15395091)
\lineto(225.08216682,219.15395091)
\lineto(225.08216682,215.93130492)
\curveto(226.54049551,217.10317619)(227.82304351,217.92999647)(228.92981082,218.41176577)
\curveto(230.04959892,218.90655586)(231.18891821,219.15395091)(232.34776869,219.15395091)
\curveto(232.98578749,219.15395091)(233.4480256,219.13441972)(233.73448302,219.09535734)
\curveto(234.02094044,219.06931576)(234.45062658,219.0107222)(235.02354142,218.91957665)
\closepath
}
}
{
\newrgbcolor{curcolor}{0 0 0}
\pscustom[linestyle=none,fillstyle=solid,fillcolor=curcolor]
{
\newpath
\moveto(257.07424674,208.23601692)
\curveto(257.07424674,204.68134074)(256.1627913,201.87536009)(254.33988044,199.81807497)
\curveto(252.51696958,197.76078986)(250.0755711,196.7321473)(247.01568501,196.7321473)
\curveto(243.92975734,196.7321473)(241.47533807,197.76078986)(239.65242721,199.81807497)
\curveto(237.84253714,201.87536009)(236.9375921,204.68134074)(236.9375921,208.23601692)
\curveto(236.9375921,211.7906931)(237.84253714,214.59667375)(239.65242721,216.65395887)
\curveto(241.47533807,218.72426478)(243.92975734,219.75941773)(247.01568501,219.75941773)
\curveto(250.0755711,219.75941773)(252.51696958,218.72426478)(254.33988044,216.65395887)
\curveto(256.1627913,214.59667375)(257.07424674,211.7906931)(257.07424674,208.23601692)
\closepath
\moveto(253.2851963,208.23601692)
\curveto(253.2851963,211.06152876)(252.73181265,213.15787625)(251.62504534,214.5250594)
\curveto(250.51827803,215.90526333)(248.98182459,216.5953653)(247.01568501,216.5953653)
\curveto(245.02350385,216.5953653)(243.47402962,215.90526333)(242.36726231,214.5250594)
\curveto(241.27351579,213.15787625)(240.72664254,211.06152876)(240.72664254,208.23601692)
\curveto(240.72664254,205.50165063)(241.28002619,203.42483432)(242.3867935,202.00556801)
\curveto(243.49356081,200.59932248)(245.03652465,199.89619972)(247.01568501,199.89619972)
\curveto(248.96880379,199.89619972)(250.49874684,200.59281209)(251.60551415,201.98603682)
\curveto(252.72530225,203.39228234)(253.2851963,205.47560904)(253.2851963,208.23601692)
\closepath
}
}
{
\newrgbcolor{curcolor}{0 0 0}
\pscustom[linestyle=none,fillstyle=solid,fillcolor=curcolor]
{
\newpath
\moveto(276.3710631,215.15005741)
\lineto(276.17575122,215.15005741)
\curveto(275.62887796,215.28026532)(275.0950255,215.37141087)(274.57419382,215.42349404)
\curveto(274.06638294,215.48859799)(273.46091612,215.52114997)(272.75779336,215.52114997)
\curveto(271.62498446,215.52114997)(270.53123795,215.26724453)(269.4765538,214.75943365)
\curveto(268.42186966,214.26464356)(267.4062479,213.62011436)(266.4296885,212.82584606)
\lineto(266.4296885,197.33761412)
\lineto(262.7578252,197.33761412)
\lineto(262.7578252,219.15395091)
\lineto(266.4296885,219.15395091)
\lineto(266.4296885,215.93130492)
\curveto(267.88801719,217.10317619)(269.17056519,217.92999647)(270.2773325,218.41176577)
\curveto(271.39712061,218.90655586)(272.53643989,219.15395091)(273.69529037,219.15395091)
\curveto(274.33330917,219.15395091)(274.79554728,219.13441972)(275.08200471,219.09535734)
\curveto(275.36846213,219.06931576)(275.79814826,219.0107222)(276.3710631,218.91957665)
\closepath
}
}
{
\newrgbcolor{curcolor}{0 0 0}
\pscustom[linestyle=none,fillstyle=solid,fillcolor=curcolor]
{
\newpath
\moveto(313.0701608,211.88834904)
\curveto(313.0701608,206.6670115)(312.24985091,202.83238829)(310.60923114,200.38447942)
\curveto(308.98163215,197.94959134)(306.44908813,196.7321473)(303.01159908,196.7321473)
\curveto(299.52202685,196.7321473)(296.96995165,197.96912253)(295.35537345,200.44307298)
\curveto(293.75381605,202.91702344)(292.95303735,206.71909467)(292.95303735,211.84928667)
\curveto(292.95303735,217.01854104)(293.76683685,220.83363306)(295.39443583,223.29456272)
\curveto(297.02203481,225.76851318)(299.56108923,227.00548841)(303.01159908,227.00548841)
\curveto(306.5011713,227.00548841)(309.04673611,225.74898199)(310.64829351,223.23596916)
\curveto(312.26287171,220.73597712)(313.0701608,216.95343708)(313.0701608,211.88834904)
\closepath
\moveto(307.93345841,203.02118977)
\curveto(308.38918612,204.07587392)(308.69517473,205.31284914)(308.85142423,206.73211546)
\curveto(309.02069453,208.16440256)(309.10532968,209.88314709)(309.10532968,211.88834904)
\curveto(309.10532968,213.86750941)(309.02069453,215.58625393)(308.85142423,217.04458262)
\curveto(308.69517473,218.50291131)(308.38267573,219.73988654)(307.91392722,220.75550831)
\curveto(307.4581995,221.75810928)(306.83320149,222.51331521)(306.03893319,223.0211261)
\curveto(305.25768568,223.52893698)(304.24857431,223.78284242)(303.01159908,223.78284242)
\curveto(301.78764464,223.78284242)(300.77202287,223.52893698)(299.96473378,223.0211261)
\curveto(299.17046547,222.51331521)(298.53895707,221.74508849)(298.07020856,220.71644593)
\curveto(297.62750164,219.75290733)(297.32151303,218.49640092)(297.15224273,216.94692669)
\curveto(296.99599323,215.39745245)(296.91786848,213.69823911)(296.91786848,211.84928667)
\curveto(296.91786848,209.81804313)(296.98948283,208.11882979)(297.13271155,206.75164665)
\curveto(297.27594026,205.3844635)(297.58192887,204.16050906)(298.05067737,203.07978334)
\curveto(298.4803635,202.06416157)(299.08583033,201.28942445)(299.86707784,200.75557199)
\curveto(300.66134614,200.22171952)(301.70951989,199.95479329)(303.01159908,199.95479329)
\curveto(304.23555351,199.95479329)(305.25117528,200.20869873)(306.05846438,200.71650961)
\curveto(306.86575347,201.22432049)(307.49075148,201.99254722)(307.93345841,203.02118977)
\closepath
}
}
{
\newrgbcolor{curcolor}{0.7019608 0.7019608 0.7019608}
\pscustom[linestyle=none,fillstyle=solid,fillcolor=curcolor,opacity=0.92623001]
{
\newpath
\moveto(203.89651216,37.69475901)
\curveto(203.89651216,50.55086656)(179.18409728,62.14116877)(141.28355824,67.06070968)
\curveto(103.38309649,71.98024055)(59.75896581,69.26011318)(30.75075397,60.16951151)
\curveto(1.74254213,51.07890983)(-6.93741479,37.40796735)(8.76086337,25.53070986)
\curveto(24.45917355,13.65342816)(61.44395778,5.90904455)(102.46794107,5.90904455)
\curveto(143.49192437,5.90904455)(180.4767086,13.65342816)(196.17501878,25.53070986)
\curveto(211.87329694,37.40796735)(203.19334002,51.07890983)(174.18512818,60.16951151)
\curveto(145.17691634,69.26011318)(101.55278566,71.98024055)(63.65232391,67.06070968)
\curveto(25.75178487,62.14116877)(1.03936999,50.55086656)(1.03936999,37.69475901)
\curveto(1.03936999,24.83865146)(25.75178487,13.24834926)(63.65232391,8.32880835)
\curveto(101.55278566,3.40927747)(145.17691634,6.12940484)(174.18512818,15.22000652)
\curveto(203.19334002,24.31060819)(211.87329694,37.98155068)(196.17501878,49.85880816)
\curveto(180.4767086,61.73608986)(143.49192437,69.48047347)(102.46794107,69.48047347)
\curveto(61.44395778,69.48047347)(24.45917355,61.73608986)(8.76086337,49.85880816)
\curveto(-6.93741479,37.98155068)(1.74254213,24.31060819)(30.75075397,15.22000652)
\curveto(59.75896581,6.12940484)(103.38309649,3.40927747)(141.28355824,8.32880835)
\curveto(179.18409728,13.24834926)(203.89651216,24.83865146)(203.89651216,37.69475901)
\closepath
}
}
{
\newrgbcolor{curcolor}{0 0 0}
\pscustom[linewidth=2.0787402,linecolor=curcolor]
{
\newpath
\moveto(203.89651216,37.69475901)
\curveto(203.89651216,50.55086656)(179.18409728,62.14116877)(141.28355824,67.06070968)
\curveto(103.38309649,71.98024055)(59.75896581,69.26011318)(30.75075397,60.16951151)
\curveto(1.74254213,51.07890983)(-6.93741479,37.40796735)(8.76086337,25.53070986)
\curveto(24.45917355,13.65342816)(61.44395778,5.90904455)(102.46794107,5.90904455)
\curveto(143.49192437,5.90904455)(180.4767086,13.65342816)(196.17501878,25.53070986)
\curveto(211.87329694,37.40796735)(203.19334002,51.07890983)(174.18512818,60.16951151)
\curveto(145.17691634,69.26011318)(101.55278566,71.98024055)(63.65232391,67.06070968)
\curveto(25.75178487,62.14116877)(1.03936999,50.55086656)(1.03936999,37.69475901)
\curveto(1.03936999,24.83865146)(25.75178487,13.24834926)(63.65232391,8.32880835)
\curveto(101.55278566,3.40927747)(145.17691634,6.12940484)(174.18512818,15.22000652)
\curveto(203.19334002,24.31060819)(211.87329694,37.98155068)(196.17501878,49.85880816)
\curveto(180.4767086,61.73608986)(143.49192437,69.48047347)(102.46794107,69.48047347)
\curveto(61.44395778,69.48047347)(24.45917355,61.73608986)(8.76086337,49.85880816)
\curveto(-6.93741479,37.98155068)(1.74254213,24.31060819)(30.75075397,15.22000652)
\curveto(59.75896581,6.12940484)(103.38309649,3.40927747)(141.28355824,8.32880835)
\curveto(179.18409728,13.24834926)(203.89651216,24.83865146)(203.89651216,37.69475901)
\closepath
}
}
{
\newrgbcolor{curcolor}{0 0 0}
\pscustom[linestyle=none,fillstyle=solid,fillcolor=curcolor]
{
\newpath
\moveto(64.89969648,22.80222895)
\lineto(61.22783317,22.80222895)
\lineto(61.22783317,25.08737792)
\curveto(60.17314903,24.17592249)(59.07289211,23.46628933)(57.92706243,22.95847845)
\curveto(56.78123274,22.45066757)(55.53774712,22.19676212)(54.19660556,22.19676212)
\curveto(51.59244718,22.19676212)(49.52214127,23.1993631)(47.98568783,25.20456505)
\curveto(46.46225518,27.209767)(45.70053886,29.98970606)(45.70053886,33.54438224)
\curveto(45.70053886,35.39333469)(45.9609547,37.04046486)(46.48178637,38.48577276)
\curveto(47.01563884,39.93108066)(47.73178239,41.16154549)(48.63021703,42.17716726)
\curveto(49.51563088,43.16674744)(50.54427344,43.92195337)(51.7161447,44.44278504)
\curveto(52.90103677,44.96361672)(54.1249912,45.22403256)(55.38800801,45.22403256)
\curveto(56.5338377,45.22403256)(57.54945947,45.10033503)(58.43487331,44.85293999)
\curveto(59.32028716,44.61856573)(60.25127378,44.24747317)(61.22783317,43.73966228)
\lineto(61.22783317,53.19275718)
\lineto(64.89969648,53.19275718)
\closepath
\moveto(61.22783317,28.17330559)
\lineto(61.22783317,40.69279698)
\curveto(60.23825299,41.13550391)(59.35283914,41.44149252)(58.57159163,41.61076281)
\curveto(57.79034412,41.78003311)(56.93748225,41.86466825)(56.01300602,41.86466825)
\curveto(53.95572091,41.86466825)(52.35416351,41.1485247)(51.20833382,39.71623759)
\curveto(50.06250414,38.28395049)(49.48958929,36.25270695)(49.48958929,33.62250699)
\curveto(49.48958929,31.03136941)(49.93229622,29.05871944)(50.81771007,27.70455709)
\curveto(51.70312391,26.36341552)(53.12239023,25.69284474)(55.07550901,25.69284474)
\curveto(56.11717236,25.69284474)(57.1718565,25.9207086)(58.23956143,26.37643632)
\curveto(59.30726637,26.84518482)(60.30335695,27.44414125)(61.22783317,28.17330559)
\closepath
}
}
{
\newrgbcolor{curcolor}{0 0 0}
\pscustom[linestyle=none,fillstyle=solid,fillcolor=curcolor]
{
\newpath
\moveto(76.07153536,48.27089786)
\lineto(71.93092354,48.27089786)
\lineto(71.93092354,52.07947948)
\lineto(76.07153536,52.07947948)
\closepath
\moveto(75.83716111,22.80222895)
\lineto(72.1652978,22.80222895)
\lineto(72.1652978,44.61856573)
\lineto(75.83716111,44.61856573)
\closepath
}
}
{
\newrgbcolor{curcolor}{0 0 0}
\pscustom[linestyle=none,fillstyle=solid,fillcolor=curcolor]
{
\newpath
\moveto(98.68865276,29.09127142)
\curveto(98.68865276,27.09909026)(97.86183247,25.46498088)(96.2081919,24.18894328)
\curveto(94.56757213,22.91290568)(92.32148553,22.27488687)(89.46993211,22.27488687)
\curveto(87.85535392,22.27488687)(86.37098364,22.46368836)(85.01682129,22.84129132)
\curveto(83.67567972,23.23191508)(82.54938123,23.65509081)(81.63792579,24.11081853)
\lineto(81.63792579,28.23189916)
\lineto(81.83323767,28.23189916)
\curveto(82.99208815,27.3595061)(84.28114655,26.66289374)(85.70041286,26.14206206)
\curveto(87.11967917,25.63425118)(88.48035193,25.38034574)(89.78243111,25.38034574)
\curveto(91.39700931,25.38034574)(92.66002612,25.64076157)(93.57148155,26.16159325)
\curveto(94.48293698,26.68242492)(94.9386647,27.50273481)(94.9386647,28.62252291)
\curveto(94.9386647,29.48189518)(94.69126965,30.13293477)(94.19647956,30.5756417)
\curveto(93.70168947,31.01834862)(92.75117166,31.39595158)(91.34492614,31.70845059)
\curveto(90.82409446,31.82563772)(90.14050289,31.96235603)(89.29415142,32.11860553)
\curveto(88.46082074,32.27485504)(87.69910441,32.44412533)(87.00900244,32.62641642)
\curveto(85.09494604,33.1342273)(83.73427329,33.87641244)(82.92698419,34.85297183)
\curveto(82.13271589,35.84255201)(81.73558173,37.05348565)(81.73558173,38.48577276)
\curveto(81.73558173,39.3842074)(81.91787282,40.23055887)(82.28245499,41.02482718)
\curveto(82.66005796,41.81909548)(83.2264624,42.52872864)(83.98166833,43.15372665)
\curveto(84.71083268,43.76570387)(85.6353089,44.24747317)(86.755097,44.59903455)
\curveto(87.8879059,44.96361672)(89.15092271,45.14590781)(90.54414744,45.14590781)
\curveto(91.84622663,45.14590781)(93.16132661,44.98314791)(94.48944738,44.65762811)
\curveto(95.83058894,44.3451291)(96.94386665,43.96101574)(97.82928049,43.50528803)
\lineto(97.82928049,39.57951928)
\lineto(97.63396861,39.57951928)
\curveto(96.6964716,40.26962125)(95.55715231,40.84904649)(94.21601075,41.31779499)
\curveto(92.87486918,41.79956429)(91.5597692,42.04044894)(90.27071081,42.04044894)
\curveto(88.92956925,42.04044894)(87.79676035,41.78003311)(86.87228413,41.25920143)
\curveto(85.94780791,40.75139055)(85.48556979,39.98967422)(85.48556979,38.97405246)
\curveto(85.48556979,38.07561782)(85.76551682,37.39853664)(86.32541087,36.94280892)
\curveto(86.87228413,36.48708121)(87.75769798,36.11598864)(88.98165241,35.82953122)
\curveto(89.65873359,35.67328172)(90.41393952,35.51703221)(91.2472702,35.36078271)
\curveto(92.09362167,35.20453321)(92.79674443,35.0613045)(93.35663848,34.93109658)
\curveto(95.06236222,34.54047282)(96.3774622,33.86990204)(97.30193842,32.91938423)
\curveto(98.22641464,31.95584563)(98.68865276,30.67980803)(98.68865276,29.09127142)
\closepath
}
}
{
\newrgbcolor{curcolor}{0 0 0}
\pscustom[linestyle=none,fillstyle=solid,fillcolor=curcolor]
{
\newpath
\moveto(123.74716339,22.80222895)
\lineto(118.90342881,22.80222895)
\lineto(110.15345667,32.35297979)
\lineto(107.77065176,30.087362)
\lineto(107.77065176,22.80222895)
\lineto(104.09878845,22.80222895)
\lineto(104.09878845,53.19275718)
\lineto(107.77065176,53.19275718)
\lineto(107.77065176,33.70063175)
\lineto(118.37608674,44.61856573)
\lineto(123.00497825,44.61856573)
\lineto(112.86829178,34.54047282)
\closepath
}
}
{
\newrgbcolor{curcolor}{0 0 0}
\pscustom[linestyle=none,fillstyle=solid,fillcolor=curcolor]
{
\newpath
\moveto(159.23533645,22.80222895)
\lineto(143.49319907,22.80222895)
\lineto(143.49319907,25.77096949)
\lineto(149.54786729,25.77096949)
\lineto(149.54786729,45.26309493)
\lineto(143.49319907,45.26309493)
\lineto(143.49319907,47.91933647)
\curveto(144.31350896,47.91933647)(145.19241241,47.98444043)(146.12990942,48.11464835)
\curveto(147.06740644,48.25787706)(147.7770396,48.45969934)(148.2588089,48.72011517)
\curveto(148.85776532,49.04563497)(149.32651383,49.45578992)(149.66505442,49.95058001)
\curveto(150.0166158,50.45839089)(150.21843807,51.13547207)(150.27052124,51.98182354)
\lineto(153.29785535,51.98182354)
\lineto(153.29785535,25.77096949)
\lineto(159.23533645,25.77096949)
\closepath
}
}
{
\newrgbcolor{curcolor}{0.7019608 0.7019608 0.7019608}
\pscustom[linestyle=none,fillstyle=solid,fillcolor=curcolor,opacity=0.92623001]
{
\newpath
\moveto(446.03940602,35.19475649)
\curveto(446.03940602,48.05086404)(421.32699114,59.64116625)(383.4264521,64.56070716)
\curveto(345.52599035,69.48023803)(301.90185966,66.76011066)(272.89364783,57.66950899)
\curveto(243.88543599,48.57890731)(235.20547907,34.90796483)(250.90375723,23.03070734)
\curveto(266.60206741,11.15342564)(303.58685164,3.40904203)(344.61083493,3.40904203)
\curveto(385.63481822,3.40904203)(422.61960246,11.15342564)(438.31791264,23.03070734)
\curveto(454.0161908,34.90796483)(445.33623388,48.57890731)(416.32802204,57.66950899)
\curveto(387.3198102,66.76011066)(343.69567951,69.48023803)(305.79521777,64.56070716)
\curveto(267.89467873,59.64116625)(243.18226385,48.05086404)(243.18226385,35.19475649)
\curveto(243.18226385,22.33864894)(267.89467873,10.74834674)(305.79521777,5.82880583)
\curveto(343.69567951,0.90927495)(387.3198102,3.62940232)(416.32802204,12.720004)
\curveto(445.33623388,21.81060567)(454.0161908,35.48154816)(438.31791264,47.35880564)
\curveto(422.61960246,59.23608735)(385.63481822,66.98047095)(344.61083493,66.98047095)
\curveto(303.58685164,66.98047095)(266.60206741,59.23608735)(250.90375723,47.35880564)
\curveto(235.20547907,35.48154816)(243.88543599,21.81060567)(272.89364783,12.720004)
\curveto(301.90185966,3.62940232)(345.52599035,0.90927495)(383.4264521,5.82880583)
\curveto(421.32699114,10.74834674)(446.03940602,22.33864894)(446.03940602,35.19475649)
\closepath
}
}
{
\newrgbcolor{curcolor}{0 0 0}
\pscustom[linewidth=2.0787402,linecolor=curcolor]
{
\newpath
\moveto(446.03940602,35.19475649)
\curveto(446.03940602,48.05086404)(421.32699114,59.64116625)(383.4264521,64.56070716)
\curveto(345.52599035,69.48023803)(301.90185966,66.76011066)(272.89364783,57.66950899)
\curveto(243.88543599,48.57890731)(235.20547907,34.90796483)(250.90375723,23.03070734)
\curveto(266.60206741,11.15342564)(303.58685164,3.40904203)(344.61083493,3.40904203)
\curveto(385.63481822,3.40904203)(422.61960246,11.15342564)(438.31791264,23.03070734)
\curveto(454.0161908,34.90796483)(445.33623388,48.57890731)(416.32802204,57.66950899)
\curveto(387.3198102,66.76011066)(343.69567951,69.48023803)(305.79521777,64.56070716)
\curveto(267.89467873,59.64116625)(243.18226385,48.05086404)(243.18226385,35.19475649)
\curveto(243.18226385,22.33864894)(267.89467873,10.74834674)(305.79521777,5.82880583)
\curveto(343.69567951,0.90927495)(387.3198102,3.62940232)(416.32802204,12.720004)
\curveto(445.33623388,21.81060567)(454.0161908,35.48154816)(438.31791264,47.35880564)
\curveto(422.61960246,59.23608735)(385.63481822,66.98047095)(344.61083493,66.98047095)
\curveto(303.58685164,66.98047095)(266.60206741,59.23608735)(250.90375723,47.35880564)
\curveto(235.20547907,35.48154816)(243.88543599,21.81060567)(272.89364783,12.720004)
\curveto(301.90185966,3.62940232)(345.52599035,0.90927495)(383.4264521,5.82880583)
\curveto(421.32699114,10.74834674)(446.03940602,22.33864894)(446.03940602,35.19475649)
\closepath
}
}
{
\newrgbcolor{curcolor}{0 0 0}
\pscustom[linestyle=none,fillstyle=solid,fillcolor=curcolor]
{
\newpath
\moveto(307.04259034,20.30222643)
\lineto(303.37072703,20.30222643)
\lineto(303.37072703,22.5873754)
\curveto(302.31604289,21.67591997)(301.21578597,20.96628681)(300.06995629,20.45847593)
\curveto(298.9241266,19.95066505)(297.68064098,19.6967596)(296.33949942,19.6967596)
\curveto(293.73534104,19.6967596)(291.66503513,20.69936058)(290.12858169,22.70456253)
\curveto(288.60514904,24.70976448)(287.84343272,27.48970354)(287.84343272,31.04437972)
\curveto(287.84343272,32.89333217)(288.10384855,34.54046234)(288.62468023,35.98577024)
\curveto(289.1585327,37.43107814)(289.87467625,38.66154297)(290.77311089,39.67716474)
\curveto(291.65852474,40.66674492)(292.68716729,41.42195085)(293.85903856,41.94278252)
\curveto(295.04393062,42.4636142)(296.26788506,42.72403004)(297.53090187,42.72403004)
\curveto(298.67673156,42.72403004)(299.69235332,42.60033251)(300.57776717,42.35293747)
\curveto(301.46318102,42.11856321)(302.39416764,41.74747065)(303.37072703,41.23965976)
\lineto(303.37072703,50.69275466)
\lineto(307.04259034,50.69275466)
\closepath
\moveto(303.37072703,25.67330307)
\lineto(303.37072703,38.19279446)
\curveto(302.38114685,38.63550139)(301.495733,38.94149)(300.71448549,39.11076029)
\curveto(299.93323797,39.28003059)(299.08037611,39.36466573)(298.15589988,39.36466573)
\curveto(296.09861477,39.36466573)(294.49705737,38.64852218)(293.35122768,37.21623507)
\curveto(292.20539799,35.78394797)(291.63248315,33.75270443)(291.63248315,31.12250448)
\curveto(291.63248315,28.53136689)(292.07519008,26.55871692)(292.96060392,25.20455457)
\curveto(293.84601777,23.863413)(295.26528409,23.19284222)(297.21840287,23.19284222)
\curveto(298.26006622,23.19284222)(299.31475036,23.42070608)(300.38245529,23.8764338)
\curveto(301.45016023,24.3451823)(302.44625081,24.94413873)(303.37072703,25.67330307)
\closepath
}
}
{
\newrgbcolor{curcolor}{0 0 0}
\pscustom[linestyle=none,fillstyle=solid,fillcolor=curcolor]
{
\newpath
\moveto(318.21442922,45.77089534)
\lineto(314.0738174,45.77089534)
\lineto(314.0738174,49.57947696)
\lineto(318.21442922,49.57947696)
\closepath
\moveto(317.98005496,20.30222643)
\lineto(314.30819166,20.30222643)
\lineto(314.30819166,42.11856321)
\lineto(317.98005496,42.11856321)
\closepath
}
}
{
\newrgbcolor{curcolor}{0 0 0}
\pscustom[linestyle=none,fillstyle=solid,fillcolor=curcolor]
{
\newpath
\moveto(340.83154661,26.5912689)
\curveto(340.83154661,24.59908775)(340.00472633,22.96497836)(338.35108576,21.68894076)
\curveto(336.71046599,20.41290316)(334.46437939,19.77488436)(331.61282597,19.77488436)
\curveto(329.99824777,19.77488436)(328.5138775,19.96368584)(327.15971514,20.3412888)
\curveto(325.81857358,20.73191256)(324.69227508,21.15508829)(323.78081965,21.61081601)
\lineto(323.78081965,25.73189664)
\lineto(323.97613153,25.73189664)
\curveto(325.13498201,24.85950358)(326.4240404,24.16289122)(327.84330672,23.64205954)
\curveto(329.26257303,23.13424866)(330.62324578,22.88034322)(331.92532497,22.88034322)
\curveto(333.53990316,22.88034322)(334.80291998,23.14075906)(335.71437541,23.66159073)
\curveto(336.62583084,24.18242241)(337.08155855,25.00273229)(337.08155855,26.12252039)
\curveto(337.08155855,26.98189266)(336.83416351,27.63293225)(336.33937342,28.07563918)
\curveto(335.84458333,28.5183461)(334.89406552,28.89594906)(333.48782,29.20844807)
\curveto(332.96698832,29.3256352)(332.28339675,29.46235351)(331.43704528,29.61860301)
\curveto(330.6037146,29.77485252)(329.84199827,29.94412281)(329.1518963,30.1264139)
\curveto(327.2378399,30.63422478)(325.87716715,31.37640992)(325.06987805,32.35296931)
\curveto(324.27560974,33.34254949)(323.87847559,34.55348313)(323.87847559,35.98577024)
\curveto(323.87847559,36.88420488)(324.06076668,37.73055635)(324.42534885,38.52482466)
\curveto(324.80295182,39.31909296)(325.36935626,40.02872612)(326.12456219,40.65372413)
\curveto(326.85372654,41.26570135)(327.77820276,41.74747065)(328.89799086,42.09903203)
\curveto(330.03079975,42.4636142)(331.29381657,42.64590529)(332.6870413,42.64590529)
\curveto(333.98912048,42.64590529)(335.30422046,42.48314539)(336.63234123,42.15762559)
\curveto(337.9734828,41.84512658)(339.0867605,41.46101322)(339.97217435,41.00528551)
\lineto(339.97217435,37.07951676)
\lineto(339.77686247,37.07951676)
\curveto(338.83936546,37.76961873)(337.70004617,38.34904397)(336.35890461,38.81779247)
\curveto(335.01776304,39.29956177)(333.70266306,39.54044642)(332.41360467,39.54044642)
\curveto(331.0724631,39.54044642)(329.93965421,39.28003059)(329.01517799,38.75919891)
\curveto(328.09070176,38.25138803)(327.62846365,37.4896717)(327.62846365,36.47404994)
\curveto(327.62846365,35.5756153)(327.90841068,34.89853412)(328.46830473,34.4428064)
\curveto(329.01517799,33.98707869)(329.90059183,33.61598612)(331.12454627,33.3295287)
\curveto(331.80162745,33.1732792)(332.55683338,33.01702969)(333.39016406,32.86078019)
\curveto(334.23651553,32.70453069)(334.93963829,32.56130198)(335.49953234,32.43109406)
\curveto(337.20525608,32.0404703)(338.52035606,31.36989952)(339.44483228,30.41938171)
\curveto(340.3693085,29.45584311)(340.83154661,28.17980551)(340.83154661,26.5912689)
\closepath
}
}
{
\newrgbcolor{curcolor}{0 0 0}
\pscustom[linestyle=none,fillstyle=solid,fillcolor=curcolor]
{
\newpath
\moveto(365.89005725,20.30222643)
\lineto(361.04632267,20.30222643)
\lineto(352.29635053,29.85297727)
\lineto(349.91354561,27.58735948)
\lineto(349.91354561,20.30222643)
\lineto(346.24168231,20.30222643)
\lineto(346.24168231,50.69275466)
\lineto(349.91354561,50.69275466)
\lineto(349.91354561,31.20062923)
\lineto(360.5189806,42.11856321)
\lineto(365.14787211,42.11856321)
\lineto(355.01118563,32.0404703)
\closepath
}
}
{
\newrgbcolor{curcolor}{0 0 0}
\pscustom[linestyle=none,fillstyle=solid,fillcolor=curcolor]
{
\newpath
\moveto(403.03838127,20.30222643)
\lineto(383.35094395,20.30222643)
\lineto(383.35094395,24.38424468)
\lineto(387.45249339,27.89985849)
\curveto(388.83269733,29.07172975)(390.11524533,30.23709063)(391.30013739,31.3959411)
\curveto(393.80012943,33.81780839)(395.51236357,35.7383752)(396.43683979,37.15764151)
\curveto(397.36131601,38.58992862)(397.82355412,40.13289245)(397.82355412,41.78653302)
\curveto(397.82355412,43.29694488)(397.32225364,44.47532654)(396.31965266,45.32167802)
\curveto(395.33007248,46.18105028)(393.94335814,46.61073641)(392.15950966,46.61073641)
\curveto(390.9746176,46.61073641)(389.6920696,46.40240374)(388.31186566,45.9857384)
\curveto(386.93166172,45.56907306)(385.58400976,44.93105426)(384.26890978,44.071682)
\lineto(384.0735979,44.071682)
\lineto(384.0735979,48.17323144)
\curveto(384.99807413,48.62895915)(386.22853896,49.04562449)(387.7649924,49.42322746)
\curveto(389.31446663,49.80083042)(390.8118577,49.9896319)(392.2571656,49.9896319)
\curveto(395.23892694,49.9896319)(397.57615908,49.26697795)(399.26886202,47.82167006)
\curveto(400.96156497,46.38938295)(401.80791644,44.44277456)(401.80791644,41.9818449)
\curveto(401.80791644,40.87507759)(401.66468773,39.83992464)(401.37823031,38.87638604)
\curveto(401.10479368,37.92586823)(400.69463873,37.02092319)(400.14776547,36.16155093)
\curveto(399.63995459,35.35426183)(399.04099816,34.55999353)(398.35089619,33.77874602)
\curveto(397.67381502,32.99749851)(396.84699473,32.13161585)(395.87043534,31.18109804)
\curveto(394.47721061,29.81391489)(393.03841311,28.48579412)(391.55404284,27.19673572)
\curveto(390.06967256,25.92069812)(388.68295823,24.73580606)(387.39389983,23.64205954)
\lineto(403.03838127,23.64205954)
\closepath
}
}
{
\newrgbcolor{curcolor}{0 0 0}
\pscustom[linewidth=1.00157475,linecolor=curcolor]
{
\newpath
\moveto(106.75369096,69.48047793)
\lineto(235.32510009,183.05190596)
\lineto(346.03940561,66.62332895)
}
}
{
\newrgbcolor{curcolor}{0 0 0}
\pscustom[linestyle=none,fillstyle=solid,fillcolor=curcolor]
{
\newpath
\moveto(114.26022492,76.11125056)
\lineto(114.61052945,81.7661732)
\lineto(106.75369096,69.48047793)
\lineto(119.91514756,75.76094603)
\closepath
}
}
{
\newrgbcolor{curcolor}{0 0 0}
\pscustom[linewidth=1.06834643,linecolor=curcolor]
{
\newpath
\moveto(114.26022492,76.11125056)
\lineto(114.61052945,81.7661732)
\lineto(106.75369096,69.48047793)
\lineto(119.91514756,75.76094603)
\closepath
}
}
{
\newrgbcolor{curcolor}{0 0 0}
\pscustom[linestyle=none,fillstyle=solid,fillcolor=curcolor]
{
\newpath
\moveto(339.13755449,73.8814037)
\lineto(333.47358415,74.02389314)
\lineto(346.03940561,66.62332895)
\lineto(339.28004394,79.54537404)
\closepath
}
}
{
\newrgbcolor{curcolor}{0 0 0}
\pscustom[linewidth=1.06834643,linecolor=curcolor]
{
\newpath
\moveto(339.13755449,73.8814037)
\lineto(333.47358415,74.02389314)
\lineto(346.03940561,66.62332895)
\lineto(339.28004394,79.54537404)
\closepath
}
}
{
\newrgbcolor{curcolor}{0.7019608 0.7019608 0.7019608}
\pscustom[linestyle=none,fillstyle=solid,fillcolor=curcolor,opacity=0.92623001]
{
\newpath
\moveto(645.68227512,240.19475218)
\lineto(825.42154446,240.19475218)
\lineto(825.42154446,181.16975637)
\lineto(645.68227512,181.16975637)
\closepath
}
}
{
\newrgbcolor{curcolor}{0 0 0}
\pscustom[linewidth=1.62517795,linecolor=curcolor]
{
\newpath
\moveto(645.68227512,240.19475218)
\lineto(825.42154446,240.19475218)
\lineto(825.42154446,181.16975637)
\lineto(645.68227512,181.16975637)
\closepath
}
}
{
\newrgbcolor{curcolor}{0 0 0}
\pscustom[linestyle=none,fillstyle=solid,fillcolor=curcolor]
{
\newpath
\moveto(686.4279241,194.96797129)
\lineto(682.56074891,194.96797129)
\lineto(682.56074891,220.02648525)
\lineto(674.47483716,202.97575829)
\lineto(672.170157,202.97575829)
\lineto(664.1428388,220.02648525)
\lineto(664.1428388,194.96797129)
\lineto(660.52956906,194.96797129)
\lineto(660.52956906,224.04990994)
\lineto(665.80298977,224.04990994)
\lineto(673.55687133,207.85855524)
\lineto(681.05684745,224.04990994)
\lineto(686.4279241,224.04990994)
\closepath
}
}
{
\newrgbcolor{curcolor}{0 0 0}
\pscustom[linestyle=none,fillstyle=solid,fillcolor=curcolor]
{
\newpath
\moveto(697.89273175,220.43664019)
\lineto(693.75211993,220.43664019)
\lineto(693.75211993,224.24522182)
\lineto(697.89273175,224.24522182)
\closepath
\moveto(697.65835749,194.96797129)
\lineto(693.98649419,194.96797129)
\lineto(693.98649419,216.78430807)
\lineto(697.65835749,216.78430807)
\closepath
}
}
{
\newrgbcolor{curcolor}{0 0 0}
\pscustom[linestyle=none,fillstyle=solid,fillcolor=curcolor]
{
\newpath
\moveto(718.53719737,212.78041457)
\lineto(718.3418855,212.78041457)
\curveto(717.79501224,212.91062249)(717.26115977,213.00176803)(716.74032809,213.0538512)
\curveto(716.23251721,213.11895516)(715.62705039,213.15150714)(714.92392763,213.15150714)
\curveto(713.79111873,213.15150714)(712.69737222,212.8976017)(711.64268807,212.38979082)
\curveto(710.58800393,211.89500072)(709.57238217,211.25047153)(708.59582278,210.45620322)
\lineto(708.59582278,194.96797129)
\lineto(704.92395947,194.96797129)
\lineto(704.92395947,216.78430807)
\lineto(708.59582278,216.78430807)
\lineto(708.59582278,213.56166208)
\curveto(710.05415147,214.73353335)(711.33669947,215.56035364)(712.44346678,216.04212294)
\curveto(713.56325488,216.53691303)(714.70257417,216.78430807)(715.86142464,216.78430807)
\curveto(716.49944344,216.78430807)(716.96168156,216.76477689)(717.24813898,216.72571451)
\curveto(717.5345964,216.69967293)(717.96428253,216.64107936)(718.53719737,216.54993382)
\closepath
}
}
{
\newrgbcolor{curcolor}{0 0 0}
\pscustom[linestyle=none,fillstyle=solid,fillcolor=curcolor]
{
\newpath
\moveto(735.60745701,212.78041457)
\lineto(735.41214513,212.78041457)
\curveto(734.86527187,212.91062249)(734.3314194,213.00176803)(733.81058773,213.0538512)
\curveto(733.30277685,213.11895516)(732.69731002,213.15150714)(731.99418726,213.15150714)
\curveto(730.86137837,213.15150714)(729.76763185,212.8976017)(728.71294771,212.38979082)
\curveto(727.65826357,211.89500072)(726.6426418,211.25047153)(725.66608241,210.45620322)
\lineto(725.66608241,194.96797129)
\lineto(721.9942191,194.96797129)
\lineto(721.9942191,216.78430807)
\lineto(725.66608241,216.78430807)
\lineto(725.66608241,213.56166208)
\curveto(727.1244111,214.73353335)(728.4069591,215.56035364)(729.51372641,216.04212294)
\curveto(730.63351451,216.53691303)(731.7728338,216.78430807)(732.93168428,216.78430807)
\curveto(733.56970308,216.78430807)(734.03194119,216.76477689)(734.31839861,216.72571451)
\curveto(734.60485603,216.69967293)(735.03454217,216.64107936)(735.60745701,216.54993382)
\closepath
}
}
{
\newrgbcolor{curcolor}{0 0 0}
\pscustom[linestyle=none,fillstyle=solid,fillcolor=curcolor]
{
\newpath
\moveto(757.65816233,205.86637409)
\curveto(757.65816233,202.3116979)(756.7467069,199.50571725)(754.92379603,197.44843214)
\curveto(753.10088517,195.39114702)(750.65948669,194.36250446)(747.5996006,194.36250446)
\curveto(744.51367293,194.36250446)(742.05925366,195.39114702)(740.2363428,197.44843214)
\curveto(738.42645273,199.50571725)(737.52150769,202.3116979)(737.52150769,205.86637409)
\curveto(737.52150769,209.42105027)(738.42645273,212.22703092)(740.2363428,214.28431603)
\curveto(742.05925366,216.35462194)(744.51367293,217.3897749)(747.5996006,217.3897749)
\curveto(750.65948669,217.3897749)(753.10088517,216.35462194)(754.92379603,214.28431603)
\curveto(756.7467069,212.22703092)(757.65816233,209.42105027)(757.65816233,205.86637409)
\closepath
\moveto(753.86911189,205.86637409)
\curveto(753.86911189,208.69188592)(753.31572824,210.78823341)(752.20896093,212.15541656)
\curveto(751.10219362,213.5356205)(749.56574018,214.22572247)(747.5996006,214.22572247)
\curveto(745.60741945,214.22572247)(744.05794521,213.5356205)(742.9511779,212.15541656)
\curveto(741.85743139,210.78823341)(741.31055813,208.69188592)(741.31055813,205.86637409)
\curveto(741.31055813,203.13200779)(741.86394178,201.05519149)(742.97070909,199.63592517)
\curveto(744.0774764,198.22967965)(745.62044024,197.52655689)(747.5996006,197.52655689)
\curveto(749.55271938,197.52655689)(751.08266243,198.22316925)(752.18942974,199.61639398)
\curveto(753.30921784,201.02263951)(753.86911189,203.10596621)(753.86911189,205.86637409)
\closepath
}
}
{
\newrgbcolor{curcolor}{0 0 0}
\pscustom[linestyle=none,fillstyle=solid,fillcolor=curcolor]
{
\newpath
\moveto(776.95497869,212.78041457)
\lineto(776.75966681,212.78041457)
\curveto(776.21279356,212.91062249)(775.67894109,213.00176803)(775.15810941,213.0538512)
\curveto(774.65029853,213.11895516)(774.04483171,213.15150714)(773.34170895,213.15150714)
\curveto(772.20890005,213.15150714)(771.11515354,212.8976017)(770.06046939,212.38979082)
\curveto(769.00578525,211.89500072)(767.99016349,211.25047153)(767.0136041,210.45620322)
\lineto(767.0136041,194.96797129)
\lineto(763.34174079,194.96797129)
\lineto(763.34174079,216.78430807)
\lineto(767.0136041,216.78430807)
\lineto(767.0136041,213.56166208)
\curveto(768.47193279,214.73353335)(769.75448078,215.56035364)(770.86124809,216.04212294)
\curveto(771.9810362,216.53691303)(773.12035548,216.78430807)(774.27920596,216.78430807)
\curveto(774.91722476,216.78430807)(775.37946288,216.76477689)(775.6659203,216.72571451)
\curveto(775.95237772,216.69967293)(776.38206385,216.64107936)(776.95497869,216.54993382)
\closepath
}
}
{
\newrgbcolor{curcolor}{0 0 0}
\pscustom[linestyle=none,fillstyle=solid,fillcolor=curcolor]
{
\newpath
\moveto(812.0329878,194.96797129)
\lineto(796.29085043,194.96797129)
\lineto(796.29085043,197.93671183)
\lineto(802.34551865,197.93671183)
\lineto(802.34551865,217.42883727)
\lineto(796.29085043,217.42883727)
\lineto(796.29085043,220.08507881)
\curveto(797.11116031,220.08507881)(797.99006377,220.15018277)(798.92756078,220.28039069)
\curveto(799.8650578,220.4236194)(800.57469095,220.62544168)(801.05646025,220.88585751)
\curveto(801.65541668,221.21137731)(802.12416519,221.62153226)(802.46270577,222.11632235)
\curveto(802.81426716,222.62413323)(803.01608943,223.30121441)(803.0681726,224.14756588)
\lineto(806.09550671,224.14756588)
\lineto(806.09550671,197.93671183)
\lineto(812.0329878,197.93671183)
\closepath
}
}
{
\newrgbcolor{curcolor}{0.7019608 0.7019608 0.7019608}
\pscustom[linestyle=none,fillstyle=solid,fillcolor=curcolor,opacity=0.92623001]
{
\newpath
\moveto(704.48045799,35.32511618)
\curveto(704.48045799,48.18122373)(679.76804311,59.77152593)(641.86750407,64.69106684)
\curveto(603.96704232,69.61059772)(560.34291163,66.89047035)(531.33469979,57.79986867)
\curveto(502.32648795,48.70926699)(493.64653104,35.03832451)(509.3448092,23.16106703)
\curveto(525.04311937,11.28378532)(562.02790361,3.53940172)(603.0518869,3.53940172)
\curveto(644.07587019,3.53940172)(681.06065443,11.28378532)(696.75896461,23.16106703)
\curveto(712.45724277,35.03832451)(703.77728585,48.70926699)(674.76907401,57.79986867)
\curveto(645.76086217,66.89047035)(602.13673148,69.61059772)(564.23626974,64.69106684)
\curveto(526.33573069,59.77152593)(501.62331582,48.18122373)(501.62331582,35.32511618)
\curveto(501.62331582,22.46900862)(526.33573069,10.87870642)(564.23626974,5.95916551)
\curveto(602.13673148,1.03963464)(645.76086217,3.759762)(674.76907401,12.85036368)
\curveto(703.77728585,21.94096536)(712.45724277,35.61190784)(696.75896461,47.48916532)
\curveto(681.06065443,59.36644703)(644.07587019,67.11083063)(603.0518869,67.11083063)
\curveto(562.02790361,67.11083063)(525.04311937,59.36644703)(509.3448092,47.48916532)
\curveto(493.64653104,35.61190784)(502.32648795,21.94096536)(531.33469979,12.85036368)
\curveto(560.34291163,3.759762)(603.96704232,1.03963464)(641.86750407,5.95916551)
\curveto(679.76804311,10.87870642)(704.48045799,22.46900862)(704.48045799,35.32511618)
\closepath
}
}
{
\newrgbcolor{curcolor}{0 0 0}
\pscustom[linewidth=2.0787402,linecolor=curcolor]
{
\newpath
\moveto(704.48045799,35.32511618)
\curveto(704.48045799,48.18122373)(679.76804311,59.77152593)(641.86750407,64.69106684)
\curveto(603.96704232,69.61059772)(560.34291163,66.89047035)(531.33469979,57.79986867)
\curveto(502.32648795,48.70926699)(493.64653104,35.03832451)(509.3448092,23.16106703)
\curveto(525.04311937,11.28378532)(562.02790361,3.53940172)(603.0518869,3.53940172)
\curveto(644.07587019,3.53940172)(681.06065443,11.28378532)(696.75896461,23.16106703)
\curveto(712.45724277,35.03832451)(703.77728585,48.70926699)(674.76907401,57.79986867)
\curveto(645.76086217,66.89047035)(602.13673148,69.61059772)(564.23626974,64.69106684)
\curveto(526.33573069,59.77152593)(501.62331582,48.18122373)(501.62331582,35.32511618)
\curveto(501.62331582,22.46900862)(526.33573069,10.87870642)(564.23626974,5.95916551)
\curveto(602.13673148,1.03963464)(645.76086217,3.759762)(674.76907401,12.85036368)
\curveto(703.77728585,21.94096536)(712.45724277,35.61190784)(696.75896461,47.48916532)
\curveto(681.06065443,59.36644703)(644.07587019,67.11083063)(603.0518869,67.11083063)
\curveto(562.02790361,67.11083063)(525.04311937,59.36644703)(509.3448092,47.48916532)
\curveto(493.64653104,35.61190784)(502.32648795,21.94096536)(531.33469979,12.85036368)
\curveto(560.34291163,3.759762)(603.96704232,1.03963464)(641.86750407,5.95916551)
\curveto(679.76804311,10.87870642)(704.48045799,22.46900862)(704.48045799,35.32511618)
\closepath
}
}
{
\newrgbcolor{curcolor}{0 0 0}
\pscustom[linestyle=none,fillstyle=solid,fillcolor=curcolor]
{
\newpath
\moveto(565.48364231,20.43258611)
\lineto(561.811779,20.43258611)
\lineto(561.811779,22.71773509)
\curveto(560.75709486,21.80627965)(559.65683794,21.0966465)(558.51100826,20.58883561)
\curveto(557.36517857,20.08102473)(556.12169295,19.82711929)(554.78055138,19.82711929)
\curveto(552.17639301,19.82711929)(550.1060871,20.82972026)(548.56963366,22.83492221)
\curveto(547.04620101,24.84012416)(546.28448469,27.62006323)(546.28448469,31.17473941)
\curveto(546.28448469,33.02369186)(546.54490052,34.67082203)(547.0657322,36.11612993)
\curveto(547.59958466,37.56143782)(548.31572822,38.79190266)(549.21416286,39.80752442)
\curveto(550.0995767,40.79710461)(551.12821926,41.55231053)(552.30009053,42.07314221)
\curveto(553.48498259,42.59397388)(554.70893703,42.85438972)(555.97195384,42.85438972)
\curveto(557.11778353,42.85438972)(558.13340529,42.7306922)(559.01881914,42.48329715)
\curveto(559.90423299,42.2489229)(560.83521961,41.87783033)(561.811779,41.37001945)
\lineto(561.811779,50.82311435)
\lineto(565.48364231,50.82311435)
\closepath
\moveto(561.811779,25.80366276)
\lineto(561.811779,38.32315415)
\curveto(560.82219881,38.76586107)(559.93678497,39.07184968)(559.15553745,39.24111998)
\curveto(558.37428994,39.41039027)(557.52142807,39.49502542)(556.59695185,39.49502542)
\curveto(554.53966673,39.49502542)(552.93810933,38.77888186)(551.79227965,37.34659476)
\curveto(550.64644996,35.91430765)(550.07353512,33.88306412)(550.07353512,31.25286416)
\curveto(550.07353512,28.66172658)(550.51624204,26.68907661)(551.40165589,25.33491425)
\curveto(552.28706974,23.99377269)(553.70633605,23.32320191)(555.65945484,23.32320191)
\curveto(556.70111819,23.32320191)(557.75580233,23.55106577)(558.82350726,24.00679348)
\curveto(559.8912122,24.47554199)(560.88730277,25.07449841)(561.811779,25.80366276)
\closepath
}
}
{
\newrgbcolor{curcolor}{0 0 0}
\pscustom[linestyle=none,fillstyle=solid,fillcolor=curcolor]
{
\newpath
\moveto(576.65548119,45.90125502)
\lineto(572.51486937,45.90125502)
\lineto(572.51486937,49.70983664)
\lineto(576.65548119,49.70983664)
\closepath
\moveto(576.42110693,20.43258611)
\lineto(572.74924362,20.43258611)
\lineto(572.74924362,42.2489229)
\lineto(576.42110693,42.2489229)
\closepath
}
}
{
\newrgbcolor{curcolor}{0 0 0}
\pscustom[linestyle=none,fillstyle=solid,fillcolor=curcolor]
{
\newpath
\moveto(599.27259858,26.72162859)
\curveto(599.27259858,24.72944743)(598.4457783,23.09533805)(596.79213773,21.81930045)
\curveto(595.15151795,20.54326284)(592.90543136,19.90524404)(590.05387793,19.90524404)
\curveto(588.43929974,19.90524404)(586.95492947,20.09404552)(585.60076711,20.47164849)
\curveto(584.25962555,20.86227224)(583.13332705,21.28544798)(582.22187162,21.74117569)
\lineto(582.22187162,25.86225632)
\lineto(582.4171835,25.86225632)
\curveto(583.57603398,24.98986327)(584.86509237,24.2932509)(586.28435869,23.77241923)
\curveto(587.703625,23.26460834)(589.06429775,23.0107029)(590.36637694,23.0107029)
\curveto(591.98095513,23.0107029)(593.24397194,23.27111874)(594.15542738,23.79195042)
\curveto(595.06688281,24.31278209)(595.52261052,25.13309198)(595.52261052,26.25288008)
\curveto(595.52261052,27.11225234)(595.27521548,27.76329194)(594.78042539,28.20599886)
\curveto(594.28563529,28.64870578)(593.33511749,29.02630875)(591.92887197,29.33880775)
\curveto(591.40804029,29.45599488)(590.72444872,29.5927132)(589.87809724,29.7489627)
\curveto(589.04476656,29.9052122)(588.28305024,30.0744825)(587.59294827,30.25677358)
\curveto(585.67889186,30.76458446)(584.31821911,31.5067696)(583.51093002,32.48332899)
\curveto(582.71666171,33.47290917)(582.31952756,34.68384282)(582.31952756,36.11612993)
\curveto(582.31952756,37.01456457)(582.50181865,37.86091604)(582.86640082,38.65518434)
\curveto(583.24400378,39.44945265)(583.81040823,40.1590858)(584.56561416,40.78408381)
\curveto(585.2947785,41.39606103)(586.21925473,41.87783033)(587.33904283,42.22939171)
\curveto(588.47185172,42.59397388)(589.73486853,42.77626497)(591.12809326,42.77626497)
\curveto(592.43017245,42.77626497)(593.74527243,42.61350507)(595.0733932,42.28798528)
\curveto(596.41453477,41.97548627)(597.52781247,41.59137291)(598.41322632,41.13564519)
\lineto(598.41322632,37.20987644)
\lineto(598.21791444,37.20987644)
\curveto(597.28041743,37.89997841)(596.14109814,38.47940365)(594.79995657,38.94815216)
\curveto(593.45881501,39.42992146)(592.14371503,39.67080611)(590.85465664,39.67080611)
\curveto(589.51351507,39.67080611)(588.38070618,39.41039027)(587.45622996,38.8895586)
\curveto(586.53175373,38.38174771)(586.06951562,37.62003139)(586.06951562,36.60440962)
\curveto(586.06951562,35.70597498)(586.34946265,35.0288938)(586.9093567,34.57316609)
\curveto(587.45622996,34.11743837)(588.3416438,33.7463458)(589.56559824,33.45988838)
\curveto(590.24267942,33.30363888)(590.99788535,33.14738938)(591.83121603,32.99113988)
\curveto(592.6775675,32.83489037)(593.38069026,32.69166166)(593.94058431,32.56145374)
\curveto(595.64630805,32.17082999)(596.96140803,31.50025921)(597.88588425,30.5497414)
\curveto(598.81036047,29.5862028)(599.27259858,28.3101652)(599.27259858,26.72162859)
\closepath
}
}
{
\newrgbcolor{curcolor}{0 0 0}
\pscustom[linestyle=none,fillstyle=solid,fillcolor=curcolor]
{
\newpath
\moveto(624.33110922,20.43258611)
\lineto(619.48737464,20.43258611)
\lineto(610.7374025,29.98333695)
\lineto(608.35459758,27.71771917)
\lineto(608.35459758,20.43258611)
\lineto(604.68273427,20.43258611)
\lineto(604.68273427,50.82311435)
\lineto(608.35459758,50.82311435)
\lineto(608.35459758,31.33098891)
\lineto(618.96003257,42.2489229)
\lineto(623.58892408,42.2489229)
\lineto(613.4522376,32.17082999)
\closepath
}
}
{
\newrgbcolor{curcolor}{0 0 0}
\pscustom[linestyle=none,fillstyle=solid,fillcolor=curcolor]
{
\newpath
\moveto(658.92084764,34.43644777)
\curveto(659.54584565,33.87655372)(660.06016692,33.17343096)(660.46381147,32.32707949)
\curveto(660.86745602,31.48072802)(661.06927829,30.3869815)(661.06927829,29.04583994)
\curveto(661.06927829,27.71771917)(660.82839365,26.50027513)(660.34662435,25.39350782)
\curveto(659.86485505,24.28674051)(659.18777387,23.32320191)(658.31538081,22.50289202)
\curveto(657.33882142,21.59143659)(656.18648134,20.91435541)(654.85836057,20.47164849)
\curveto(653.54326059,20.04196235)(652.09795269,19.82711929)(650.52243688,19.82711929)
\curveto(648.90785868,19.82711929)(647.31932207,20.02243117)(645.75682705,20.41305492)
\curveto(644.19433202,20.79065789)(642.91178402,21.20732323)(641.90918305,21.66305094)
\lineto(641.90918305,25.7450692)
\lineto(642.20215087,25.7450692)
\curveto(643.30891818,25.01590485)(644.61099736,24.41043803)(646.10838843,23.92866873)
\curveto(647.60577949,23.44689943)(649.05108739,23.20601478)(650.44431212,23.20601478)
\curveto(651.26462201,23.20601478)(652.13701507,23.3427331)(653.06149129,23.61616972)
\curveto(653.98596751,23.88960635)(654.73466305,24.2932509)(655.30757789,24.82710337)
\curveto(655.90653432,25.40001821)(656.34924124,26.03152662)(656.63569866,26.72162859)
\curveto(656.93517687,27.41173056)(657.08491598,28.28412361)(657.08491598,29.33880775)
\curveto(657.08491598,30.3804711)(656.91564569,31.23984337)(656.5771051,31.91692455)
\curveto(656.2515853,32.60702652)(655.79585758,33.14738938)(655.20992195,33.53801313)
\curveto(654.62398632,33.94165768)(653.91435316,34.21509431)(653.08102248,34.35832302)
\curveto(652.2476918,34.51457253)(651.34925716,34.59269728)(650.38571856,34.59269728)
\lineto(648.62791166,34.59269728)
\lineto(648.62791166,37.83487445)
\lineto(649.9950948,37.83487445)
\curveto(651.97425517,37.83487445)(653.54977099,38.2450294)(654.72164226,39.06533929)
\curveto(655.90653432,39.89866997)(656.49898035,41.10960361)(656.49898035,42.69814022)
\curveto(656.49898035,43.40126298)(656.34924124,44.0132402)(656.04976303,44.53407187)
\curveto(655.75028481,45.06792434)(655.33361947,45.50412087)(654.79976701,45.84266146)
\curveto(654.23987296,46.18120205)(653.64091653,46.4155763)(653.00289773,46.54578422)
\curveto(652.36487893,46.67599214)(651.64222498,46.7410961)(650.83493588,46.7410961)
\curveto(649.59796065,46.7410961)(648.28286067,46.51974263)(646.88963594,46.07703571)
\curveto(645.49641121,45.63432879)(644.18131123,45.00933078)(642.944336,44.20204168)
\lineto(642.74902413,44.20204168)
\lineto(642.74902413,48.28405993)
\curveto(643.67350035,48.73978765)(644.90396518,49.15645299)(646.44041862,49.53405595)
\curveto(647.98989286,49.92467971)(649.48728392,50.11999159)(650.93259182,50.11999159)
\curveto(652.35185813,50.11999159)(653.60185415,49.98978367)(654.68257988,49.72936783)
\curveto(655.76330561,49.46895199)(656.739865,49.05228665)(657.61225805,48.47937181)
\curveto(658.54975507,47.8543738)(659.25938822,47.09916787)(659.74115752,46.21375403)
\curveto(660.22292682,45.32834018)(660.46381147,44.29318722)(660.46381147,43.10829516)
\curveto(660.46381147,41.49371697)(659.89089663,40.08096105)(658.74506694,38.87002741)
\curveto(657.61225805,37.67211455)(656.27111649,36.91690863)(654.72164226,36.60440962)
\lineto(654.72164226,36.33097299)
\curveto(655.34664027,36.22680666)(656.06278382,36.00545319)(656.87007291,35.66691261)
\curveto(657.67736201,35.34139281)(658.36095358,34.93123787)(658.92084764,34.43644777)
\closepath
}
}
{
\newrgbcolor{curcolor}{0.7019608 0.7019608 0.7019608}
\pscustom[linestyle=none,fillstyle=solid,fillcolor=curcolor,opacity=0.92623001]
{
\newpath
\moveto(946.62334051,32.82511366)
\curveto(946.62334051,45.68122121)(921.91092563,57.27152341)(884.01038659,62.19106432)
\curveto(846.10992484,67.1105952)(802.48579415,64.39046783)(773.47758231,55.29986615)
\curveto(744.46937047,46.20926448)(735.78941355,32.53832199)(751.48769172,20.66106451)
\curveto(767.18600189,8.7837828)(804.17078613,1.0393992)(845.19476942,1.0393992)
\curveto(886.21875271,1.0393992)(923.20353695,8.7837828)(938.90184712,20.66106451)
\curveto(954.60012529,32.53832199)(945.92016837,46.20926448)(916.91195653,55.29986615)
\curveto(887.90374469,64.39046783)(844.279614,67.1105952)(806.37915226,62.19106432)
\curveto(768.47861321,57.27152341)(743.76619834,45.68122121)(743.76619834,32.82511366)
\curveto(743.76619834,19.9690061)(768.47861321,8.3787039)(806.37915226,3.45916299)
\curveto(844.279614,-1.46036788)(887.90374469,1.25975948)(916.91195653,10.35036116)
\curveto(945.92016837,19.44096284)(954.60012529,33.11190532)(938.90184712,44.98916281)
\curveto(923.20353695,56.86644451)(886.21875271,64.61082811)(845.19476942,64.61082811)
\curveto(804.17078613,64.61082811)(767.18600189,56.86644451)(751.48769172,44.98916281)
\curveto(735.78941355,33.11190532)(744.46937047,19.44096284)(773.47758231,10.35036116)
\curveto(802.48579415,1.25975948)(846.10992484,-1.46036788)(884.01038659,3.45916299)
\curveto(921.91092563,8.3787039)(946.62334051,19.9690061)(946.62334051,32.82511366)
\closepath
}
}
{
\newrgbcolor{curcolor}{0 0 0}
\pscustom[linewidth=2.0787402,linecolor=curcolor]
{
\newpath
\moveto(946.62334051,32.82511366)
\curveto(946.62334051,45.68122121)(921.91092563,57.27152341)(884.01038659,62.19106432)
\curveto(846.10992484,67.1105952)(802.48579415,64.39046783)(773.47758231,55.29986615)
\curveto(744.46937047,46.20926448)(735.78941355,32.53832199)(751.48769172,20.66106451)
\curveto(767.18600189,8.7837828)(804.17078613,1.0393992)(845.19476942,1.0393992)
\curveto(886.21875271,1.0393992)(923.20353695,8.7837828)(938.90184712,20.66106451)
\curveto(954.60012529,32.53832199)(945.92016837,46.20926448)(916.91195653,55.29986615)
\curveto(887.90374469,64.39046783)(844.279614,67.1105952)(806.37915226,62.19106432)
\curveto(768.47861321,57.27152341)(743.76619834,45.68122121)(743.76619834,32.82511366)
\curveto(743.76619834,19.9690061)(768.47861321,8.3787039)(806.37915226,3.45916299)
\curveto(844.279614,-1.46036788)(887.90374469,1.25975948)(916.91195653,10.35036116)
\curveto(945.92016837,19.44096284)(954.60012529,33.11190532)(938.90184712,44.98916281)
\curveto(923.20353695,56.86644451)(886.21875271,64.61082811)(845.19476942,64.61082811)
\curveto(804.17078613,64.61082811)(767.18600189,56.86644451)(751.48769172,44.98916281)
\curveto(735.78941355,33.11190532)(744.46937047,19.44096284)(773.47758231,10.35036116)
\curveto(802.48579415,1.25975948)(846.10992484,-1.46036788)(884.01038659,3.45916299)
\curveto(921.91092563,8.3787039)(946.62334051,19.9690061)(946.62334051,32.82511366)
\closepath
}
}
{
\newrgbcolor{curcolor}{0 0 0}
\pscustom[linestyle=none,fillstyle=solid,fillcolor=curcolor]
{
\newpath
\moveto(807.62652483,17.93258359)
\lineto(803.95466152,17.93258359)
\lineto(803.95466152,20.21773257)
\curveto(802.89997737,19.30627713)(801.79972046,18.59664398)(800.65389078,18.08883309)
\curveto(799.50806109,17.58102221)(798.26457547,17.32711677)(796.9234339,17.32711677)
\curveto(794.31927553,17.32711677)(792.24896962,18.32971774)(790.71251618,20.33491969)
\curveto(789.18908353,22.34012164)(788.4273672,25.12006071)(788.4273672,28.67473689)
\curveto(788.4273672,30.52368934)(788.68778304,32.17081951)(789.20861472,33.61612741)
\curveto(789.74246718,35.0614353)(790.45861074,36.29190014)(791.35704538,37.3075219)
\curveto(792.24245922,38.29710209)(793.27110178,39.05230801)(794.44297305,39.57313969)
\curveto(795.62786511,40.09397136)(796.85181955,40.3543872)(798.11483636,40.3543872)
\curveto(799.26066605,40.3543872)(800.27628781,40.23068968)(801.16170166,39.98329463)
\curveto(802.04711551,39.74892038)(802.97810213,39.37782781)(803.95466152,38.87001693)
\lineto(803.95466152,48.32311183)
\lineto(807.62652483,48.32311183)
\closepath
\moveto(803.95466152,23.30366024)
\lineto(803.95466152,35.82315163)
\curveto(802.96508133,36.26585855)(802.07966749,36.57184716)(801.29841997,36.74111746)
\curveto(800.51717246,36.91038775)(799.66431059,36.9950229)(798.73983437,36.9950229)
\curveto(796.68254925,36.9950229)(795.08099185,36.27887934)(793.93516217,34.84659224)
\curveto(792.78933248,33.41430513)(792.21641764,31.3830616)(792.21641764,28.75286164)
\curveto(792.21641764,26.16172406)(792.65912456,24.18907409)(793.54453841,22.83491173)
\curveto(794.42995226,21.49377017)(795.84921857,20.82319939)(797.80233736,20.82319939)
\curveto(798.84400071,20.82319939)(799.89868485,21.05106325)(800.96638978,21.50679096)
\curveto(802.03409472,21.97553947)(803.03018529,22.5744959)(803.95466152,23.30366024)
\closepath
}
}
{
\newrgbcolor{curcolor}{0 0 0}
\pscustom[linestyle=none,fillstyle=solid,fillcolor=curcolor]
{
\newpath
\moveto(818.79836371,43.4012525)
\lineto(814.65775189,43.4012525)
\lineto(814.65775189,47.20983412)
\lineto(818.79836371,47.20983412)
\closepath
\moveto(818.56398945,17.93258359)
\lineto(814.89212614,17.93258359)
\lineto(814.89212614,39.74892038)
\lineto(818.56398945,39.74892038)
\closepath
}
}
{
\newrgbcolor{curcolor}{0 0 0}
\pscustom[linestyle=none,fillstyle=solid,fillcolor=curcolor]
{
\newpath
\moveto(841.4154811,24.22162607)
\curveto(841.4154811,22.22944491)(840.58866082,20.59533553)(838.93502025,19.31929793)
\curveto(837.29440047,18.04326032)(835.04831388,17.40524152)(832.19676045,17.40524152)
\curveto(830.58218226,17.40524152)(829.09781199,17.594043)(827.74364963,17.97164597)
\curveto(826.40250807,18.36226972)(825.27620957,18.78544546)(824.36475414,19.24117317)
\lineto(824.36475414,23.3622538)
\lineto(824.56006602,23.3622538)
\curveto(825.7189165,22.48986075)(827.00797489,21.79324838)(828.42724121,21.27241671)
\curveto(829.84650752,20.76460582)(831.20718027,20.51070038)(832.50925946,20.51070038)
\curveto(834.12383765,20.51070038)(835.38685446,20.77111622)(836.2983099,21.2919479)
\curveto(837.20976533,21.81277957)(837.66549304,22.63308946)(837.66549304,23.75287756)
\curveto(837.66549304,24.61224982)(837.418098,25.26328942)(836.92330791,25.70599634)
\curveto(836.42851781,26.14870327)(835.47800001,26.52630623)(834.07175448,26.83880523)
\curveto(833.55092281,26.95599236)(832.86733124,27.09271068)(832.02097976,27.24896018)
\curveto(831.18764908,27.40520968)(830.42593276,27.57447998)(829.73583079,27.75677106)
\curveto(827.82177438,28.26458195)(826.46110163,29.00676708)(825.65381254,29.98332647)
\curveto(824.85954423,30.97290666)(824.46241008,32.1838403)(824.46241008,33.61612741)
\curveto(824.46241008,34.51456205)(824.64470117,35.36091352)(825.00928334,36.15518182)
\curveto(825.3868863,36.94945013)(825.95329075,37.65908328)(826.70849668,38.28408129)
\curveto(827.43766102,38.89605851)(828.36213725,39.37782781)(829.48192535,39.72938919)
\curveto(830.61473424,40.09397136)(831.87775105,40.27626245)(833.27097578,40.27626245)
\curveto(834.57305497,40.27626245)(835.88815495,40.11350255)(837.21627572,39.78798276)
\curveto(838.55741729,39.47548375)(839.67069499,39.09137039)(840.55610884,38.63564267)
\lineto(840.55610884,34.70987392)
\lineto(840.36079696,34.70987392)
\curveto(839.42329995,35.39997589)(838.28398066,35.97940113)(836.94283909,36.44814964)
\curveto(835.60169753,36.92991894)(834.28659755,37.17080359)(832.99753916,37.17080359)
\curveto(831.65639759,37.17080359)(830.5235887,36.91038775)(829.59911248,36.38955608)
\curveto(828.67463625,35.88174519)(828.21239814,35.12002887)(828.21239814,34.1044071)
\curveto(828.21239814,33.20597246)(828.49234517,32.52889128)(829.05223922,32.07316357)
\curveto(829.59911248,31.61743585)(830.48452632,31.24634328)(831.70848076,30.95988586)
\curveto(832.38556194,30.80363636)(833.14076787,30.64738686)(833.97409855,30.49113736)
\curveto(834.82045002,30.33488785)(835.52357278,30.19165914)(836.08346683,30.06145122)
\curveto(837.78919057,29.67082747)(839.10429054,29.00025669)(840.02876677,28.04973888)
\curveto(840.95324299,27.08620028)(841.4154811,25.81016268)(841.4154811,24.22162607)
\closepath
}
}
{
\newrgbcolor{curcolor}{0 0 0}
\pscustom[linestyle=none,fillstyle=solid,fillcolor=curcolor]
{
\newpath
\moveto(866.47399173,17.93258359)
\lineto(861.63025716,17.93258359)
\lineto(852.88028502,27.48333443)
\lineto(850.4974801,25.21771665)
\lineto(850.4974801,17.93258359)
\lineto(846.82561679,17.93258359)
\lineto(846.82561679,48.32311183)
\lineto(850.4974801,48.32311183)
\lineto(850.4974801,28.83098639)
\lineto(861.10291509,39.74892038)
\lineto(865.7318066,39.74892038)
\lineto(855.59512012,29.67082747)
\closepath
}
}
{
\newrgbcolor{curcolor}{0 0 0}
\pscustom[linestyle=none,fillstyle=solid,fillcolor=curcolor]
{
\newpath
\moveto(904.28637614,26.11615129)
\lineto(899.96998364,26.11615129)
\lineto(899.96998364,17.93258359)
\lineto(896.21999558,17.93258359)
\lineto(896.21999558,26.11615129)
\lineto(882.29425867,26.11615129)
\lineto(882.29425867,30.60832448)
\lineto(896.37624508,47.01452225)
\lineto(899.96998364,47.01452225)
\lineto(899.96998364,29.24114134)
\lineto(904.28637614,29.24114134)
\closepath
\moveto(896.21999558,29.24114134)
\lineto(896.21999558,42.36609955)
\lineto(884.95050021,29.24114134)
\closepath
}
}
{
\newrgbcolor{curcolor}{0 0 0}
\pscustom[linewidth=1.00157475,linecolor=curcolor]
{
\newpath
\moveto(607.33762167,67.1108351)
\lineto(735.90902324,180.68226313)
\lineto(846.62332876,64.25368612)
}
}
{
\newrgbcolor{curcolor}{0 0 0}
\pscustom[linestyle=none,fillstyle=solid,fillcolor=curcolor]
{
\newpath
\moveto(614.84415543,73.74160795)
\lineto(615.1944598,79.3965306)
\lineto(607.33762167,67.1108351)
\lineto(620.49907808,73.39130358)
\closepath
}
}
{
\newrgbcolor{curcolor}{0 0 0}
\pscustom[linewidth=1.06834643,linecolor=curcolor]
{
\newpath
\moveto(614.84415543,73.74160795)
\lineto(615.1944598,79.3965306)
\lineto(607.33762167,67.1108351)
\lineto(620.49907808,73.39130358)
\closepath
}
}
{
\newrgbcolor{curcolor}{0 0 0}
\pscustom[linestyle=none,fillstyle=solid,fillcolor=curcolor]
{
\newpath
\moveto(839.72147764,71.51176086)
\lineto(834.05750729,71.65425031)
\lineto(846.62332876,64.25368612)
\lineto(839.86396709,77.1757312)
\closepath
}
}
{
\newrgbcolor{curcolor}{0 0 0}
\pscustom[linewidth=1.06834643,linecolor=curcolor]
{
\newpath
\moveto(839.72147764,71.51176086)
\lineto(834.05750729,71.65425031)
\lineto(846.62332876,64.25368612)
\lineto(839.86396709,77.1757312)
\closepath
}
}
{
\newrgbcolor{curcolor}{0 0 0}
\pscustom[linewidth=1.27017074,linecolor=curcolor]
{
\newpath
\moveto(229.29361285,244.0620036)
\lineto(498.78354293,331.20388328)
\lineto(730.84434293,241.86974911)
}
}
{
\newrgbcolor{curcolor}{0 0 0}
\pscustom[linestyle=none,fillstyle=solid,fillcolor=curcolor]
{
\newpath
\moveto(241.37918725,247.96997771)
\lineto(244.65022736,254.36739712)
\lineto(229.29361285,244.0620036)
\lineto(247.77660665,244.6989376)
\closepath
}
}
{
\newrgbcolor{curcolor}{0 0 0}
\pscustom[linewidth=1.35484883,linecolor=curcolor]
{
\newpath
\moveto(241.37918725,247.96997771)
\lineto(244.65022736,254.36739712)
\lineto(229.29361285,244.0620036)
\lineto(247.77660665,244.6989376)
\closepath
}
}
{
\newrgbcolor{curcolor}{0 0 0}
\pscustom[linestyle=none,fillstyle=solid,fillcolor=curcolor]
{
\newpath
\moveto(718.99062952,246.43295507)
\lineto(712.42386177,243.51675208)
\lineto(730.84434293,241.86974911)
\lineto(716.07442653,252.99972281)
\closepath
}
}
{
\newrgbcolor{curcolor}{0 0 0}
\pscustom[linewidth=1.35484883,linecolor=curcolor]
{
\newpath
\moveto(718.99062952,246.43295507)
\lineto(712.42386177,243.51675208)
\lineto(730.84434293,241.86974911)
\lineto(716.07442653,252.99972281)
\closepath
}
}
\end{pspicture}
}
    \captionsetup{width=0.75\linewidth}
    \caption{A very simple VDEV configuration with four disks and two mirrors, each mirror with two disks.}
    \label{fig:VDEVExample}
\end{figure}

\section{SPL}
The SPL, or Solaris Portability Layer, is a major component of ZFS that allows it to work in Linux, 
by wrapping Linux kernel interfaces to be more Solaris-like \cite{zfs}.
This even includes redefining Linux kernel functions with entirely different interfaces in its headers.
Most important low-level operations performed by ZFS are handled through the SPL, such as allocating memory or creating new threads.
There are two major exceptions, which are disk operations and allocating ARC Data Buffers (Section \ref{chapter:ABD}).

It was written by Brian Behlendorf at the Lawrence Livermore National Laboratory as part of the Laboratory's port of ZFS to Linux,
which eventually became OpenZFS.
It is a GPL-licensed kernel module, under the same license as the Linux kernel. 
This is in order to use Linux functions that are only exported to modules licensed under the GPL.
These functions cannot be used from ZFS directly as it is licensed under the CDDL.

\section{The Many Components of ZFS}

\begin{figure}[H]
    \centering
    \resizebox{!}{0.35\textheight}{%LaTeX with PSTricks extensions
%%Creator: Inkscape 1.0.2-2 (e86c870879, 2021-01-15)
%%Please note this file requires PSTricks extensions
\psset{xunit=.5pt,yunit=.5pt,runit=.5pt}
\begin{pspicture}(531.16082139,938.6281027)
{
\newrgbcolor{curcolor}{0.7019608 0.7019608 0.7019608}
\pscustom[linestyle=none,fillstyle=solid,fillcolor=curcolor,opacity=0.92623001]
{
\newpath
\moveto(127.53264469,937.58873271)
\lineto(403.96120376,937.58873271)
\lineto(403.96120376,853.30302186)
\lineto(127.53264469,853.30302186)
\closepath
}
}
{
\newrgbcolor{curcolor}{0 0 0}
\pscustom[linewidth=2.0787402,linecolor=curcolor]
{
\newpath
\moveto(127.53264469,937.58873271)
\lineto(403.96120376,937.58873271)
\lineto(403.96120376,853.30302186)
\lineto(127.53264469,853.30302186)
\closepath
}
}
{
\newrgbcolor{curcolor}{0 0 0}
\pscustom[linestyle=none,fillstyle=solid,fillcolor=curcolor]
{
\newpath
\moveto(176.29409345,913.35597639)
\lineto(165.70818965,884.27403773)
\lineto(160.55195607,884.27403773)
\lineto(149.96605227,913.35597639)
\lineto(154.10666409,913.35597639)
\lineto(163.2277288,887.77012035)
\lineto(172.34879351,913.35597639)
\closepath
}
}
{
\newrgbcolor{curcolor}{0 0 0}
\pscustom[linestyle=none,fillstyle=solid,fillcolor=curcolor]
{
\newpath
\moveto(199.28230086,909.91848733)
\lineto(184.57531643,909.91848733)
\lineto(184.57531643,901.71538845)
\lineto(197.21199495,901.71538845)
\lineto(197.21199495,898.27789939)
\lineto(184.57531643,898.27789939)
\lineto(184.57531643,884.27403773)
\lineto(180.70814125,884.27403773)
\lineto(180.70814125,913.35597639)
\lineto(199.28230086,913.35597639)
\closepath
}
}
{
\newrgbcolor{curcolor}{0 0 0}
\pscustom[linestyle=none,fillstyle=solid,fillcolor=curcolor]
{
\newpath
\moveto(224.82909682,892.57479255)
\curveto(224.82909682,891.44198366)(224.56217059,890.32219556)(224.02831812,889.21542825)
\curveto(223.50748645,888.10866094)(222.77181171,887.17116392)(221.8212939,886.4029372)
\curveto(220.77963055,885.56960652)(219.56218651,884.91856693)(218.16896178,884.44981842)
\curveto(216.78875784,883.98106991)(215.12209648,883.74669566)(213.1689777,883.74669566)
\curveto(211.07263021,883.74669566)(209.18461538,883.94200754)(207.50493323,884.33263129)
\curveto(205.83827187,884.72325505)(204.13905853,885.30268029)(202.40729321,886.07090701)
\lineto(202.40729321,890.91464159)
\lineto(202.68072984,890.91464159)
\curveto(204.15207932,889.69068715)(205.85129266,888.74667974)(207.77836986,888.08261935)
\curveto(209.70544706,887.41855897)(211.51533713,887.08652878)(213.20804007,887.08652878)
\curveto(215.60386578,887.08652878)(217.46583902,887.5357461)(218.79395979,888.43418073)
\curveto(220.13510135,889.33261537)(220.80567213,890.53052823)(220.80567213,892.02791929)
\curveto(220.80567213,893.31697769)(220.48666273,894.2674955)(219.84864393,894.87947271)
\curveto(219.22364592,895.49144993)(218.26661772,895.96670884)(216.97755932,896.30524942)
\curveto(216.00099993,896.56566526)(214.93980539,896.78050833)(213.79397571,896.94977862)
\curveto(212.66116681,897.11904892)(211.45674357,897.33389198)(210.18070596,897.59430782)
\curveto(207.60258917,898.14118108)(205.68853277,899.0721677)(204.43853675,900.38726768)
\curveto(203.20156152,901.71538845)(202.5830739,903.44064337)(202.5830739,905.56303245)
\curveto(202.5830739,907.99792053)(203.61171646,909.99010169)(205.66900158,911.53957592)
\curveto(207.72628669,913.10207094)(210.33695546,913.88331846)(213.50100789,913.88331846)
\curveto(215.54527222,913.88331846)(217.42026625,913.68800658)(219.12598998,913.29738282)
\curveto(220.83171372,912.90675907)(222.34212557,912.42498977)(223.65722555,911.85207492)
\lineto(223.65722555,907.28177698)
\lineto(223.38378892,907.28177698)
\curveto(222.27702162,908.21927399)(220.81869293,908.99401111)(219.00880285,909.60598833)
\curveto(217.21193358,910.23098634)(215.36949153,910.54348534)(213.4814767,910.54348534)
\curveto(211.41117079,910.54348534)(209.74450943,910.11379921)(208.48149262,909.25442694)
\curveto(207.2314966,908.39505468)(206.60649859,907.28828737)(206.60649859,905.93412502)
\curveto(206.60649859,904.72319137)(206.9189976,903.77267356)(207.54399561,903.0825716)
\curveto(208.16899362,902.39246963)(209.26925053,901.86512756)(210.84476635,901.50054538)
\curveto(211.67809703,901.3182543)(212.86298909,901.09690083)(214.39944253,900.836485)
\curveto(215.93589597,900.57606916)(217.23797516,900.30914293)(218.30568009,900.0357063)
\curveto(220.46713154,899.46279145)(222.09473053,898.59690879)(223.18847705,897.43805832)
\curveto(224.28222356,896.27920784)(224.82909682,894.65811925)(224.82909682,892.57479255)
\closepath
}
}
{
\newrgbcolor{curcolor}{0 0 0}
\pscustom[linestyle=none,fillstyle=solid,fillcolor=curcolor]
{
\newpath
\moveto(256.48914954,876.22718835)
\lineto(252.01650753,876.22718835)
\curveto(249.71182737,878.8704091)(247.92146848,881.7545145)(246.64543088,884.87950455)
\curveto(245.36939328,888.0044946)(244.73137447,891.52661881)(244.73137447,895.44587716)
\curveto(244.73137447,899.36513551)(245.36939328,902.88725972)(246.64543088,906.01224977)
\curveto(247.92146848,909.13723982)(249.71182737,912.02134522)(252.01650753,914.66456597)
\lineto(256.48914954,914.66456597)
\lineto(256.48914954,914.46925409)
\curveto(255.4344654,913.51873628)(254.42535403,912.41847937)(253.46181543,911.16848335)
\curveto(252.51129762,909.93150812)(251.62588377,908.48620022)(250.80557388,906.83255966)
\curveto(250.02432637,905.23100225)(249.38630757,903.46668496)(248.89151748,901.53960776)
\curveto(248.40974818,899.61253056)(248.16886353,897.58128703)(248.16886353,895.44587716)
\curveto(248.16886353,893.21932175)(248.40323778,891.18156782)(248.87198629,889.33261537)
\curveto(249.35375559,887.48366293)(249.99828479,885.72585602)(250.80557388,884.05919466)
\curveto(251.5868214,882.45763726)(252.47874564,881.01232937)(253.48134661,879.72327097)
\curveto(254.48394759,878.42119178)(255.48654856,877.32093487)(256.48914954,876.42250023)
\closepath
}
}
{
\newrgbcolor{curcolor}{0 0 0}
\pscustom[linestyle=none,fillstyle=solid,fillcolor=curcolor]
{
\newpath
\moveto(281.66485079,884.27403773)
\lineto(263.26647187,884.27403773)
\lineto(263.26647187,913.35597639)
\lineto(267.13364706,913.35597639)
\lineto(267.13364706,887.71152679)
\lineto(281.66485079,887.71152679)
\closepath
}
}
{
\newrgbcolor{curcolor}{0 0 0}
\pscustom[linestyle=none,fillstyle=solid,fillcolor=curcolor]
{
\newpath
\moveto(289.1843569,909.74270664)
\lineto(285.04374509,909.74270664)
\lineto(285.04374509,913.55128826)
\lineto(289.1843569,913.55128826)
\closepath
\moveto(288.94998265,884.27403773)
\lineto(285.27811934,884.27403773)
\lineto(285.27811934,906.09037452)
\lineto(288.94998265,906.09037452)
\closepath
}
}
{
\newrgbcolor{curcolor}{0 0 0}
\pscustom[linestyle=none,fillstyle=solid,fillcolor=curcolor]
{
\newpath
\moveto(314.45771044,884.27403773)
\lineto(310.78584713,884.27403773)
\lineto(310.78584713,896.69587318)
\curveto(310.78584713,897.69847415)(310.72725356,898.63597117)(310.61006644,899.50836423)
\curveto(310.49287931,900.39377807)(310.27803624,901.08388004)(309.96553724,901.57867013)
\curveto(309.64001744,902.12554339)(309.17126893,902.52918794)(308.55929172,902.78960378)
\curveto(307.9473145,903.06304041)(307.15304619,903.19975872)(306.1764868,903.19975872)
\curveto(305.17388583,903.19975872)(304.12571208,902.95236368)(303.03196557,902.45757359)
\curveto(301.93821905,901.96278349)(300.8900453,901.33127509)(299.88744433,900.56304837)
\lineto(299.88744433,884.27403773)
\lineto(296.21558102,884.27403773)
\lineto(296.21558102,906.09037452)
\lineto(299.88744433,906.09037452)
\lineto(299.88744433,903.66850723)
\curveto(301.03327401,904.61902504)(302.21816607,905.36121017)(303.44212051,905.89506264)
\curveto(304.66607495,906.42891511)(305.92258136,906.69584134)(307.21163976,906.69584134)
\curveto(309.56840309,906.69584134)(311.36527237,905.98620818)(312.60224759,904.56694187)
\curveto(313.83922282,903.14767555)(314.45771044,901.10341123)(314.45771044,898.4341489)
\closepath
}
}
{
\newrgbcolor{curcolor}{0 0 0}
\pscustom[linestyle=none,fillstyle=solid,fillcolor=curcolor]
{
\newpath
\moveto(339.61388885,884.27403773)
\lineto(335.94202554,884.27403773)
\lineto(335.94202554,886.69590502)
\curveto(334.70505031,885.71934563)(333.52015825,884.9706501)(332.38734935,884.44981842)
\curveto(331.25454046,883.92898675)(330.00454444,883.66857091)(328.63736129,883.66857091)
\curveto(326.34570192,883.66857091)(324.56185344,884.36518327)(323.28581583,885.758408)
\curveto(322.00977823,887.16465353)(321.37175943,889.22193864)(321.37175943,891.93026335)
\lineto(321.37175943,906.09037452)
\lineto(325.04362274,906.09037452)
\lineto(325.04362274,893.66853907)
\curveto(325.04362274,892.56177176)(325.0957059,891.61125395)(325.19987224,890.81698565)
\curveto(325.30403857,890.03573814)(325.52539204,889.36516735)(325.86393262,888.8052733)
\curveto(326.215494,888.23235846)(326.67122172,887.81569312)(327.23111577,887.55527728)
\curveto(327.79100982,887.29486145)(328.60480931,887.16465353)(329.67251425,887.16465353)
\curveto(330.62303205,887.16465353)(331.65818501,887.41204857)(332.77797311,887.90683866)
\curveto(333.910782,888.40162876)(334.96546615,889.03313716)(335.94202554,889.80136388)
\lineto(335.94202554,906.09037452)
\lineto(339.61388885,906.09037452)
\closepath
}
}
{
\newrgbcolor{curcolor}{0 0 0}
\pscustom[linestyle=none,fillstyle=solid,fillcolor=curcolor]
{
\newpath
\moveto(365.72708104,884.27403773)
\lineto(361.09818953,884.27403773)
\lineto(354.90680299,892.6529173)
\lineto(348.67635408,884.27403773)
\lineto(344.39902395,884.27403773)
\lineto(352.91462183,895.15290934)
\lineto(344.4771487,906.09037452)
\lineto(349.10604021,906.09037452)
\lineto(355.25836437,897.84821326)
\lineto(361.43021972,906.09037452)
\lineto(365.72708104,906.09037452)
\lineto(357.15288959,895.34822122)
\closepath
}
}
{
\newrgbcolor{curcolor}{0 0 0}
\pscustom[linestyle=none,fillstyle=solid,fillcolor=curcolor]
{
\newpath
\moveto(381.52781526,895.44587716)
\curveto(381.52781526,891.52661881)(380.88979646,888.0044946)(379.61375886,884.87950455)
\curveto(378.33772125,881.7545145)(376.54736237,878.8704091)(374.24268221,876.22718835)
\lineto(369.7700402,876.22718835)
\lineto(369.7700402,876.42250023)
\curveto(370.77264117,877.32093487)(371.77524215,878.42119178)(372.77784312,879.72327097)
\curveto(373.79346489,881.01232937)(374.68538913,882.45763726)(375.45361585,884.05919466)
\curveto(376.26090495,885.72585602)(376.89892375,887.48366293)(377.36767226,889.33261537)
\curveto(377.84944156,891.18156782)(378.09032621,893.21932175)(378.09032621,895.44587716)
\curveto(378.09032621,897.58128703)(377.84944156,899.61253056)(377.36767226,901.53960776)
\curveto(376.88590296,903.46668496)(376.24788416,905.23100225)(375.45361585,906.83255966)
\curveto(374.63330596,908.48620022)(373.74138172,909.93150812)(372.77784312,911.16848335)
\curveto(371.82732531,912.41847937)(370.82472434,913.51873628)(369.7700402,914.46925409)
\lineto(369.7700402,914.66456597)
\lineto(374.24268221,914.66456597)
\curveto(376.54736237,912.02134522)(378.33772125,909.13723982)(379.61375886,906.01224977)
\curveto(380.88979646,902.88725972)(381.52781526,899.36513551)(381.52781526,895.44587716)
\closepath
}
}
{
\newrgbcolor{curcolor}{0.7019608 0.7019608 0.7019608}
\pscustom[linestyle=none,fillstyle=solid,fillcolor=curcolor,opacity=0.92623001]
{
\newpath
\moveto(206.46121199,799.34236782)
\lineto(325.0326361,799.34236782)
\lineto(325.0326361,706.48523045)
\lineto(206.46121199,706.48523045)
\closepath
}
}
{
\newrgbcolor{curcolor}{0 0 0}
\pscustom[linewidth=2.0787402,linecolor=curcolor]
{
\newpath
\moveto(206.46121199,799.34236782)
\lineto(325.0326361,799.34236782)
\lineto(325.0326361,706.48523045)
\lineto(206.46121199,706.48523045)
\closepath
}
}
{
\newrgbcolor{curcolor}{0 0 0}
\pscustom[linestyle=none,fillstyle=solid,fillcolor=curcolor]
{
\newpath
\moveto(252.75868165,738.37283157)
\lineto(230.06344141,738.37283157)
\lineto(230.06344141,741.96657013)
\lineto(247.89541588,764.01728117)
\lineto(230.72750179,764.01728117)
\lineto(230.72750179,767.45477023)
\lineto(252.32899551,767.45477023)
\lineto(252.32899551,763.95868761)
\lineto(234.32124035,741.81032063)
\lineto(252.75868165,741.81032063)
\closepath
}
}
{
\newrgbcolor{curcolor}{0 0 0}
\pscustom[linestyle=none,fillstyle=solid,fillcolor=curcolor]
{
\newpath
\moveto(277.93438228,758.66573571)
\curveto(277.93438228,757.37667731)(277.70651842,756.17876446)(277.2507907,755.07199715)
\curveto(276.80808378,753.97825063)(276.18308577,753.02773283)(275.37579667,752.22044373)
\curveto(274.3731957,751.21784276)(273.18830364,750.46263683)(271.82112049,749.95482594)
\curveto(270.45393734,749.46003585)(268.72868242,749.21264081)(266.64535572,749.21264081)
\lineto(262.77818053,749.21264081)
\lineto(262.77818053,738.37283157)
\lineto(258.91100535,738.37283157)
\lineto(258.91100535,767.45477023)
\lineto(266.80160522,767.45477023)
\curveto(268.54639133,767.45477023)(270.02425121,767.30503112)(271.23518486,767.00555291)
\curveto(272.4461185,766.71909548)(273.52033383,766.26336777)(274.45783085,765.63836976)
\curveto(275.56459815,764.89618462)(276.41746002,763.9717084)(277.01641645,762.86494109)
\curveto(277.62839367,761.75817378)(277.93438228,760.35843865)(277.93438228,758.66573571)
\closepath
\moveto(273.91095759,758.56807977)
\curveto(273.91095759,759.57068074)(273.7351769,760.4430738)(273.38361552,761.18525894)
\curveto(273.03205413,761.92744407)(272.49820167,762.5329109)(271.78205811,763.0016594)
\curveto(271.1570601,763.40530395)(270.44091655,763.69176137)(269.63362746,763.86103167)
\curveto(268.83935915,764.04332275)(267.83024778,764.1344683)(266.60629334,764.1344683)
\lineto(262.77818053,764.1344683)
\lineto(262.77818053,752.51341155)
\lineto(266.0398889,752.51341155)
\curveto(267.60238392,752.51341155)(268.87191113,752.65012986)(269.84847052,752.92356649)
\curveto(270.82502991,753.21002391)(271.61929822,753.65924123)(272.23127543,754.27121845)
\curveto(272.84325265,754.89621646)(273.27293878,755.55376645)(273.52033383,756.24386842)
\curveto(273.78074967,756.93397039)(273.91095759,757.70870751)(273.91095759,758.56807977)
\closepath
}
}
{
\newrgbcolor{curcolor}{0 0 0}
\pscustom[linestyle=none,fillstyle=solid,fillcolor=curcolor]
{
\newpath
\moveto(301.43040112,738.37283157)
\lineto(283.0320222,738.37283157)
\lineto(283.0320222,767.45477023)
\lineto(286.89919739,767.45477023)
\lineto(286.89919739,741.81032063)
\lineto(301.43040112,741.81032063)
\closepath
}
}
{
\newrgbcolor{curcolor}{0.7019608 0.7019608 0.7019608}
\pscustom[linestyle=none,fillstyle=solid,fillcolor=curcolor,opacity=0.92623001]
{
\newpath
\moveto(1.03936951,710.77096534)
\lineto(119.61079362,710.77096534)
\lineto(119.61079362,617.91382797)
\lineto(1.03936951,617.91382797)
\closepath
}
}
{
\newrgbcolor{curcolor}{0 0 0}
\pscustom[linewidth=2.0787402,linecolor=curcolor]
{
\newpath
\moveto(1.03936951,710.77096534)
\lineto(119.61079362,710.77096534)
\lineto(119.61079362,617.91382797)
\lineto(1.03936951,617.91382797)
\closepath
}
}
{
\newrgbcolor{curcolor}{0 0 0}
\pscustom[linestyle=none,fillstyle=solid,fillcolor=curcolor]
{
\newpath
\moveto(50.97940888,649.80142909)
\lineto(28.28416864,649.80142909)
\lineto(28.28416864,653.39516765)
\lineto(46.11614312,675.44587869)
\lineto(28.94822903,675.44587869)
\lineto(28.94822903,678.88336774)
\lineto(50.54972275,678.88336774)
\lineto(50.54972275,675.38728512)
\lineto(32.54196759,653.23891814)
\lineto(50.97940888,653.23891814)
\closepath
}
}
{
\newrgbcolor{curcolor}{0 0 0}
\pscustom[linestyle=none,fillstyle=solid,fillcolor=curcolor]
{
\newpath
\moveto(67.38560619,649.80142909)
\lineto(55.90126775,649.80142909)
\lineto(55.90126775,652.77016964)
\lineto(59.70984938,652.77016964)
\lineto(59.70984938,675.9146272)
\lineto(55.90126775,675.9146272)
\lineto(55.90126775,678.88336774)
\lineto(67.38560619,678.88336774)
\lineto(67.38560619,675.9146272)
\lineto(63.57702456,675.9146272)
\lineto(63.57702456,652.77016964)
\lineto(67.38560619,652.77016964)
\closepath
}
}
{
\newrgbcolor{curcolor}{0 0 0}
\pscustom[linestyle=none,fillstyle=solid,fillcolor=curcolor]
{
\newpath
\moveto(92.36599614,649.80142909)
\lineto(73.96761722,649.80142909)
\lineto(73.96761722,678.88336774)
\lineto(77.83479241,678.88336774)
\lineto(77.83479241,653.23891814)
\lineto(92.36599614,653.23891814)
\closepath
}
}
{
\newrgbcolor{curcolor}{0.7019608 0.7019608 0.7019608}
\pscustom[linestyle=none,fillstyle=solid,fillcolor=curcolor,opacity=0.92623001]
{
\newpath
\moveto(411.55002717,711.48525242)
\lineto(530.12145128,711.48525242)
\lineto(530.12145128,618.62811506)
\lineto(411.55002717,618.62811506)
\closepath
}
}
{
\newrgbcolor{curcolor}{0 0 0}
\pscustom[linewidth=2.0787402,linecolor=curcolor]
{
\newpath
\moveto(411.55002717,711.48525242)
\lineto(530.12145128,711.48525242)
\lineto(530.12145128,618.62811506)
\lineto(411.55002717,618.62811506)
\closepath
}
}
{
\newrgbcolor{curcolor}{0 0 0}
\pscustom[linestyle=none,fillstyle=solid,fillcolor=curcolor]
{
\newpath
\moveto(455.92367831,650.51571617)
\lineto(433.22843807,650.51571617)
\lineto(433.22843807,654.10945473)
\lineto(451.06041255,676.16016577)
\lineto(433.89249846,676.16016577)
\lineto(433.89249846,679.59765483)
\lineto(455.49399218,679.59765483)
\lineto(455.49399218,676.10157221)
\lineto(437.48623702,653.95320523)
\lineto(455.92367831,653.95320523)
\closepath
}
}
{
\newrgbcolor{curcolor}{0 0 0}
\pscustom[linestyle=none,fillstyle=solid,fillcolor=curcolor]
{
\newpath
\moveto(485.00561651,650.51571617)
\lineto(480.88453588,650.51571617)
\lineto(478.03298246,658.62115912)
\lineto(465.4548975,658.62115912)
\lineto(462.60334408,650.51571617)
\lineto(458.67757533,650.51571617)
\lineto(469.26347913,679.59765483)
\lineto(474.41971271,679.59765483)
\closepath
\moveto(476.84158,661.94146104)
\lineto(471.74393998,676.21875934)
\lineto(466.62676877,661.94146104)
\closepath
}
}
{
\newrgbcolor{curcolor}{0 0 0}
\pscustom[linestyle=none,fillstyle=solid,fillcolor=curcolor]
{
\newpath
\moveto(508.44304123,670.80862031)
\curveto(508.44304123,669.51956192)(508.21517738,668.32164906)(507.75944966,667.21488175)
\curveto(507.31674274,666.12113524)(506.69174473,665.17061743)(505.88445563,664.36332833)
\curveto(504.88185466,663.36072736)(503.6969626,662.60552143)(502.32977945,662.09771055)
\curveto(500.9625963,661.60292046)(499.23734138,661.35552541)(497.15401468,661.35552541)
\lineto(493.28683949,661.35552541)
\lineto(493.28683949,650.51571617)
\lineto(489.4196643,650.51571617)
\lineto(489.4196643,679.59765483)
\lineto(497.31026418,679.59765483)
\curveto(499.05505029,679.59765483)(500.53291017,679.44791572)(501.74384381,679.14843751)
\curveto(502.95477746,678.86198009)(504.02899279,678.40625237)(504.9664898,677.78125436)
\curveto(506.07325711,677.03906922)(506.92611898,676.114593)(507.52507541,675.00782569)
\curveto(508.13705263,673.90105838)(508.44304123,672.50132326)(508.44304123,670.80862031)
\closepath
\moveto(504.41961654,670.71096437)
\curveto(504.41961654,671.71356535)(504.24383585,672.5859584)(503.89227447,673.32814354)
\curveto(503.54071309,674.07032868)(503.00686063,674.6757955)(502.29071707,675.14454401)
\curveto(501.66571906,675.54818856)(500.94957551,675.83464598)(500.14228641,676.00391627)
\curveto(499.34801811,676.18620736)(498.33890674,676.2773529)(497.1149523,676.2773529)
\lineto(493.28683949,676.2773529)
\lineto(493.28683949,664.65629615)
\lineto(496.54854786,664.65629615)
\curveto(498.11104288,664.65629615)(499.38057009,664.79301447)(500.35712948,665.06645109)
\curveto(501.33368887,665.35290852)(502.12795717,665.80212584)(502.73993439,666.41410305)
\curveto(503.35191161,667.03910106)(503.78159774,667.69665105)(504.02899279,668.38675302)
\curveto(504.28940863,669.07685499)(504.41961654,669.85159211)(504.41961654,670.71096437)
\closepath
}
}
{
\newrgbcolor{curcolor}{0.7019608 0.7019608 0.7019608}
\pscustom[linestyle=none,fillstyle=solid,fillcolor=curcolor,opacity=0.92623001]
{
\newpath
\moveto(206.46121209,626.12807098)
\lineto(325.0326362,626.12807098)
\lineto(325.0326362,533.27093362)
\lineto(206.46121209,533.27093362)
\closepath
}
}
{
\newrgbcolor{curcolor}{0 0 0}
\pscustom[linewidth=2.0787402,linecolor=curcolor]
{
\newpath
\moveto(206.46121209,626.12807098)
\lineto(325.0326362,626.12807098)
\lineto(325.0326362,533.27093362)
\lineto(206.46121209,533.27093362)
\closepath
}
}
{
\newrgbcolor{curcolor}{0 0 0}
\pscustom[linestyle=none,fillstyle=solid,fillcolor=curcolor]
{
\newpath
\moveto(247.10442236,579.97295127)
\curveto(247.10442236,577.32973052)(246.52499712,574.93390481)(245.36614664,572.78547415)
\curveto(244.22031696,570.63704349)(242.69037391,568.97038213)(240.7763175,567.78549007)
\curveto(239.44819673,566.96518018)(237.96382646,566.37273415)(236.32320668,566.00815198)
\curveto(234.6956077,565.64356981)(232.54717704,565.46127872)(229.8779147,565.46127872)
\lineto(222.53418809,565.46127872)
\lineto(222.53418809,594.54321738)
\lineto(229.79978995,594.54321738)
\curveto(232.63832258,594.54321738)(234.89091958,594.33488471)(236.55758094,593.91821937)
\curveto(238.23726309,593.51457482)(239.6565294,592.95468077)(240.81537988,592.23853721)
\curveto(242.79454025,591.00156198)(244.33750408,589.35443181)(245.44427139,587.2971467)
\curveto(246.5510387,585.23986158)(247.10442236,582.7984631)(247.10442236,579.97295127)
\closepath
\moveto(243.06146648,580.03154483)
\curveto(243.06146648,582.31018341)(242.66433233,584.23075021)(241.87006402,585.79324523)
\curveto(241.07579572,587.35574026)(239.89090366,588.58620509)(238.31538784,589.48463973)
\curveto(237.16955815,590.13567933)(235.95211411,590.58489664)(234.66305572,590.83229169)
\curveto(233.37399732,591.09270753)(231.83103349,591.22291545)(230.03416421,591.22291545)
\lineto(226.40136327,591.22291545)
\lineto(226.40136327,568.78158065)
\lineto(230.03416421,568.78158065)
\curveto(231.89613744,568.78158065)(233.51722603,568.91829896)(234.89742997,569.19173559)
\curveto(236.2906547,569.46517222)(237.56669231,569.97298311)(238.72554278,570.71516824)
\curveto(240.17085068,571.63964447)(241.25157641,572.85708851)(241.96771996,574.36750036)
\curveto(242.69688431,575.87791222)(243.06146648,577.76592704)(243.06146648,580.03154483)
\closepath
}
}
{
\newrgbcolor{curcolor}{0 0 0}
\pscustom[linestyle=none,fillstyle=solid,fillcolor=curcolor]
{
\newpath
\moveto(279.25275831,565.46127872)
\lineto(275.38558312,565.46127872)
\lineto(275.38558312,590.51979269)
\lineto(267.29967137,573.46906572)
\lineto(264.9949912,573.46906572)
\lineto(256.96767301,590.51979269)
\lineto(256.96767301,565.46127872)
\lineto(253.35440327,565.46127872)
\lineto(253.35440327,594.54321738)
\lineto(258.62782398,594.54321738)
\lineto(266.38170554,578.35186268)
\lineto(273.88168166,594.54321738)
\lineto(279.25275831,594.54321738)
\closepath
}
}
{
\newrgbcolor{curcolor}{0 0 0}
\pscustom[linestyle=none,fillstyle=solid,fillcolor=curcolor]
{
\newpath
\moveto(308.95969537,577.14092903)
\curveto(308.95969537,575.03156075)(308.72532112,573.1891187)(308.25657261,571.61360288)
\curveto(307.8008449,570.05110786)(307.04563897,568.74902867)(305.99095483,567.70736532)
\curveto(304.98835385,566.71778514)(303.81648258,565.99513119)(302.47534102,565.53940347)
\curveto(301.13419946,565.08367576)(299.57170443,564.8558119)(297.78785594,564.8558119)
\curveto(295.96494508,564.8558119)(294.37640847,565.09669655)(293.02224612,565.57846585)
\curveto(291.66808376,566.06023515)(290.52876447,566.7698683)(289.60428825,567.70736532)
\curveto(288.54960411,568.77507025)(287.78788778,570.06412865)(287.31913928,571.57454051)
\curveto(286.86341156,573.08495236)(286.6355477,574.94041521)(286.6355477,577.14092903)
\lineto(286.6355477,594.54321738)
\lineto(290.50272289,594.54321738)
\lineto(290.50272289,576.94561715)
\curveto(290.50272289,575.37010134)(290.60688922,574.12661571)(290.81522189,573.21516028)
\curveto(291.03657536,572.30370485)(291.40115753,571.47688457)(291.90896841,570.73469943)
\curveto(292.48188325,569.88834796)(293.25662037,569.25032916)(294.23317976,568.82064302)
\curveto(295.22275994,568.39095689)(296.40765201,568.17611383)(297.78785594,568.17611383)
\curveto(299.18108068,568.17611383)(300.36597274,568.3844465)(301.34253213,568.80111184)
\curveto(302.31909152,569.23079797)(303.10033903,569.87532717)(303.68627466,570.73469943)
\curveto(304.19408555,571.47688457)(304.55215732,572.32323604)(304.76048999,573.27375385)
\curveto(304.98184346,574.23729244)(305.09252019,575.4286949)(305.09252019,576.84796122)
\lineto(305.09252019,594.54321738)
\lineto(308.95969537,594.54321738)
\closepath
}
}
{
\newrgbcolor{curcolor}{0.7019608 0.7019608 0.7019608}
\pscustom[linestyle=none,fillstyle=solid,fillcolor=curcolor,opacity=0.92623001]
{
\newpath
\moveto(206.46121204,452.91382797)
\lineto(325.03263615,452.91382797)
\lineto(325.03263615,360.05669061)
\lineto(206.46121204,360.05669061)
\closepath
}
}
{
\newrgbcolor{curcolor}{0 0 0}
\pscustom[linewidth=2.0787402,linecolor=curcolor]
{
\newpath
\moveto(206.46121204,452.91382797)
\lineto(325.03263615,452.91382797)
\lineto(325.03263615,360.05669061)
\lineto(206.46121204,360.05669061)
\closepath
}
}
{
\newrgbcolor{curcolor}{0 0 0}
\pscustom[linestyle=none,fillstyle=solid,fillcolor=curcolor]
{
\newpath
\moveto(251.5672701,391.94426289)
\lineto(247.44618947,391.94426289)
\lineto(244.59463605,400.04970583)
\lineto(232.0165511,400.04970583)
\lineto(229.16499768,391.94426289)
\lineto(225.23922893,391.94426289)
\lineto(235.82513272,421.02620154)
\lineto(240.98136631,421.02620154)
\closepath
\moveto(243.40323359,403.37000776)
\lineto(238.30559357,417.64730605)
\lineto(233.18842237,403.37000776)
\closepath
}
}
{
\newrgbcolor{curcolor}{0 0 0}
\pscustom[linestyle=none,fillstyle=solid,fillcolor=curcolor]
{
\newpath
\moveto(280.04374129,391.94426289)
\lineto(275.02422602,391.94426289)
\lineto(265.29769449,403.50672607)
\lineto(259.84849309,403.50672607)
\lineto(259.84849309,391.94426289)
\lineto(255.9813179,391.94426289)
\lineto(255.9813179,421.02620154)
\lineto(264.12582322,421.02620154)
\curveto(265.88363012,421.02620154)(267.34846921,420.90901442)(268.52034048,420.67464016)
\curveto(269.69221174,420.4532867)(270.74689589,420.04964215)(271.6843929,419.46370652)
\curveto(272.73907704,418.79964613)(273.55938693,417.95980506)(274.14532257,416.94418329)
\curveto(274.74427899,415.94158232)(275.04375721,414.66554471)(275.04375721,413.11607048)
\curveto(275.04375721,411.01972299)(274.51641513,409.26191608)(273.46173099,407.84264977)
\curveto(272.40704685,406.43640425)(270.95522856,405.37520971)(269.10627611,404.65906615)
\closepath
\moveto(271.00080133,412.84263385)
\curveto(271.00080133,413.67596453)(270.85106222,414.41163927)(270.55158401,415.04965807)
\curveto(270.26512659,415.70069767)(269.78335729,416.24757092)(269.10627611,416.69027785)
\curveto(268.54638206,417.06788081)(267.88232167,417.32829665)(267.11409495,417.47152536)
\curveto(266.34586823,417.62777486)(265.4409232,417.70589961)(264.39925985,417.70589961)
\lineto(259.84849309,417.70589961)
\lineto(259.84849309,406.72937206)
\lineto(263.75473065,406.72937206)
\curveto(264.97868509,406.72937206)(266.04639002,406.8335384)(266.95784545,407.04187107)
\curveto(267.86930088,407.26322453)(268.644038,407.66686908)(269.2820568,408.25280471)
\curveto(269.86799243,408.79967797)(270.29767857,409.42467598)(270.5711152,410.12779874)
\curveto(270.85757262,410.8439423)(271.00080133,411.74888733)(271.00080133,412.84263385)
\closepath
}
}
{
\newrgbcolor{curcolor}{0 0 0}
\pscustom[linestyle=none,fillstyle=solid,fillcolor=curcolor]
{
\newpath
\moveto(306.25459618,394.05363117)
\curveto(305.53845262,393.74113217)(304.88741303,393.44816435)(304.3014774,393.17472772)
\curveto(303.72856255,392.90129109)(302.97335662,392.61483367)(302.03585961,392.31535546)
\curveto(301.2415913,392.06796041)(300.37570865,391.85962774)(299.43821163,391.69035745)
\curveto(298.51373541,391.50806636)(297.49160324,391.41692082)(296.37181514,391.41692082)
\curveto(294.26244686,391.41692082)(292.34188006,391.70988863)(290.61011474,392.29582427)
\curveto(288.89137021,392.89478069)(287.39397914,393.82576731)(286.11794154,395.08878413)
\curveto(284.86794552,396.32575935)(283.89138613,397.89476477)(283.18826337,399.79580039)
\curveto(282.48514061,401.70985679)(282.13357923,403.92990181)(282.13357923,406.45593543)
\curveto(282.13357923,408.85176114)(282.47211982,410.9936814)(283.14920099,412.88169622)
\curveto(283.82628217,414.76971105)(284.80284156,416.36475805)(286.07887917,417.66683724)
\curveto(287.31585439,418.92985405)(288.80673506,419.89339265)(290.55152117,420.55745304)
\curveto(292.30932808,421.22151342)(294.25593646,421.55354361)(296.39134633,421.55354361)
\curveto(297.95384136,421.55354361)(299.50982599,421.36474213)(301.05930022,420.98713917)
\curveto(302.62179524,420.6095362)(304.35356056,419.94547582)(306.25459618,418.99495801)
\lineto(306.25459618,414.40512887)
\lineto(305.96162836,414.40512887)
\curveto(304.36007096,415.74627044)(302.77153435,416.72282983)(301.19601853,417.33480705)
\curveto(299.62050272,417.94678426)(297.93431017,418.25277287)(296.13744089,418.25277287)
\curveto(294.66609141,418.25277287)(293.33797064,418.01188822)(292.15307858,417.53011892)
\curveto(290.98120731,417.06137042)(289.93303356,416.32569568)(289.00855734,415.3230947)
\curveto(288.1101227,414.34653531)(287.40699994,413.10956008)(286.89918905,411.61216902)
\curveto(286.40439896,410.12779874)(286.15700392,408.40905421)(286.15700392,406.45593543)
\curveto(286.15700392,404.41167111)(286.43044055,402.65386421)(286.9773138,401.18251472)
\curveto(287.53720786,399.71116524)(288.25335141,398.51325239)(289.12574446,397.58877617)
\curveto(290.0371999,396.62523757)(291.09839443,395.90909401)(292.30932808,395.44034551)
\curveto(293.53328251,394.98461779)(294.82234091,394.75675393)(296.17650327,394.75675393)
\curveto(298.0384765,394.75675393)(299.78326261,395.07576333)(301.4108616,395.71378214)
\curveto(303.03846058,396.35180094)(304.56189323,397.30882914)(305.98115955,398.58486674)
\lineto(306.25459618,398.58486674)
\closepath
}
}
{
\newrgbcolor{curcolor}{0.7019608 0.7019608 0.7019608}
\pscustom[linestyle=none,fillstyle=solid,fillcolor=curcolor,opacity=0.92623001]
{
\newpath
\moveto(1.37238536,280.50276341)
\lineto(530.12146263,280.50276341)
\lineto(530.12146263,1.03938258)
\lineto(1.37238536,1.03938258)
\closepath
}
}
{
\newrgbcolor{curcolor}{0 0 0}
\pscustom[linewidth=2.0787402,linecolor=curcolor]
{
\newpath
\moveto(1.37238536,280.50276341)
\lineto(530.12146263,280.50276341)
\lineto(530.12146263,1.03938258)
\lineto(1.37238536,1.03938258)
\closepath
}
}
{
\newrgbcolor{curcolor}{0 0 0}
\pscustom[linestyle=none,fillstyle=solid,fillcolor=curcolor]
{
\newpath
\moveto(29.38675992,251.31655188)
\curveto(29.38675992,250.18374299)(29.11983369,249.06395489)(28.58598122,247.95718758)
\curveto(28.06514954,246.85042027)(27.3294748,245.91292325)(26.378957,245.14469653)
\curveto(25.33729365,244.31136585)(24.11984961,243.66032626)(22.72662487,243.19157775)
\curveto(21.34642094,242.72282924)(19.67975958,242.48845499)(17.72664079,242.48845499)
\curveto(15.6302933,242.48845499)(13.74227848,242.68376687)(12.06259633,243.07439062)
\curveto(10.39593497,243.46501438)(8.69672163,244.04443962)(6.96495631,244.81266634)
\lineto(6.96495631,249.65640092)
\lineto(7.23839294,249.65640092)
\curveto(8.70974242,248.43244648)(10.40895576,247.48843907)(12.33603296,246.82437868)
\curveto(14.26311016,246.1603183)(16.07300023,245.82828811)(17.76570317,245.82828811)
\curveto(20.16152888,245.82828811)(22.02350211,246.27750543)(23.35162288,247.17594006)
\curveto(24.69276445,248.0743747)(25.36333523,249.27228756)(25.36333523,250.76967862)
\curveto(25.36333523,252.05873702)(25.04432583,253.00925483)(24.40630703,253.62123204)
\curveto(23.78130902,254.23320926)(22.82428081,254.70846816)(21.53522242,255.04700875)
\curveto(20.55866303,255.30742459)(19.49746849,255.52226766)(18.3516388,255.69153795)
\curveto(17.21882991,255.86080825)(16.01440666,256.07565131)(14.73836906,256.33606715)
\curveto(12.16025227,256.88294041)(10.24619586,257.81392703)(8.99619984,259.12902701)
\curveto(7.75922461,260.45714778)(7.140737,262.1824027)(7.140737,264.30479178)
\curveto(7.140737,266.73967986)(8.16937956,268.73186102)(10.22666467,270.28133525)
\curveto(12.28394979,271.84383027)(14.89461856,272.62507779)(18.05867099,272.62507779)
\curveto(20.10293531,272.62507779)(21.97792934,272.42976591)(23.68365308,272.03914215)
\curveto(25.38937681,271.6485184)(26.89978867,271.1667491)(28.21488865,270.59383425)
\lineto(28.21488865,266.02353631)
\lineto(27.94145202,266.02353631)
\curveto(26.83468471,266.96103332)(25.37635602,267.73577044)(23.56646595,268.34774766)
\curveto(21.76959667,268.97274567)(19.92715462,269.28524467)(18.0391398,269.28524467)
\curveto(15.96883389,269.28524467)(14.30217253,268.85555854)(13.03915572,267.99618627)
\curveto(11.7891597,267.13681401)(11.16416169,266.0300467)(11.16416169,264.67588435)
\curveto(11.16416169,263.4649507)(11.47666069,262.51443289)(12.1016587,261.82433093)
\curveto(12.72665671,261.13422896)(13.82691363,260.60688688)(15.40242944,260.24230471)
\curveto(16.23576012,260.06001363)(17.42065219,259.83866016)(18.95710563,259.57824433)
\curveto(20.49355907,259.31782849)(21.79563826,259.05090226)(22.86334319,258.77746563)
\curveto(25.02479464,258.20455078)(26.65239363,257.33866812)(27.74614014,256.17981765)
\curveto(28.83988666,255.02096717)(29.38675992,253.39987858)(29.38675992,251.31655188)
\closepath
}
}
{
\newrgbcolor{curcolor}{0 0 0}
\pscustom[linestyle=none,fillstyle=solid,fillcolor=curcolor]
{
\newpath
\moveto(54.62105393,263.3087012)
\curveto(54.62105393,262.0196428)(54.39319007,260.82172995)(53.93746235,259.71496264)
\curveto(53.49475543,258.62121612)(52.86975742,257.67069832)(52.06246832,256.86340922)
\curveto(51.05986735,255.86080825)(49.87497529,255.10560232)(48.50779214,254.59779143)
\curveto(47.14060899,254.10300134)(45.41535407,253.8556063)(43.33202737,253.8556063)
\lineto(39.46485218,253.8556063)
\lineto(39.46485218,243.01579706)
\lineto(35.597677,243.01579706)
\lineto(35.597677,272.09773572)
\lineto(43.48827687,272.09773572)
\curveto(45.23306298,272.09773572)(46.71092286,271.94799661)(47.92185651,271.6485184)
\curveto(49.13279015,271.36206097)(50.20700548,270.90633326)(51.1445025,270.28133525)
\curveto(52.25126981,269.53915011)(53.10413167,268.61467389)(53.7030881,267.50790658)
\curveto(54.31506532,266.40113927)(54.62105393,265.00140414)(54.62105393,263.3087012)
\closepath
\moveto(50.59762924,263.21104526)
\curveto(50.59762924,264.21364623)(50.42184855,265.08603929)(50.07028717,265.82822443)
\curveto(49.71872579,266.57040956)(49.18487332,267.17587639)(48.46872977,267.64462489)
\curveto(47.84373176,268.04826944)(47.1275882,268.33472686)(46.32029911,268.50399716)
\curveto(45.5260308,268.68628824)(44.51691943,268.77743379)(43.292965,268.77743379)
\lineto(39.46485218,268.77743379)
\lineto(39.46485218,257.15637704)
\lineto(42.72656055,257.15637704)
\curveto(44.28905557,257.15637704)(45.55858278,257.29309535)(46.53514217,257.56653198)
\curveto(47.51170156,257.8529894)(48.30596987,258.30220672)(48.91794709,258.91418394)
\curveto(49.5299243,259.53918195)(49.95961044,260.19673194)(50.20700548,260.88683391)
\curveto(50.46742132,261.57693588)(50.59762924,262.351673)(50.59762924,263.21104526)
\closepath
}
}
{
\newrgbcolor{curcolor}{0 0 0}
\pscustom[linestyle=none,fillstyle=solid,fillcolor=curcolor]
{
\newpath
\moveto(81.67174705,243.01579706)
\lineto(77.55066642,243.01579706)
\lineto(74.699113,251.12124)
\lineto(62.12102805,251.12124)
\lineto(59.26947462,243.01579706)
\lineto(55.34370587,243.01579706)
\lineto(65.92960967,272.09773572)
\lineto(71.08584325,272.09773572)
\closepath
\moveto(73.50771054,254.44154193)
\lineto(68.41007052,268.71884022)
\lineto(63.29289931,254.44154193)
\closepath
}
}
{
\newrgbcolor{curcolor}{0.7019608 0.7019608 0.7019608}
\pscustom[linestyle=none,fillstyle=solid,fillcolor=curcolor,opacity=0.92623001]
{
\newpath
\moveto(206.46121209,280.50276343)
\lineto(325.0326362,280.50276343)
\lineto(325.0326362,187.64562606)
\lineto(206.46121209,187.64562606)
\closepath
}
}
{
\newrgbcolor{curcolor}{0 0 0}
\pscustom[linewidth=2.0787402,linecolor=curcolor]
{
\newpath
\moveto(206.46121209,280.50276343)
\lineto(325.0326362,280.50276343)
\lineto(325.0326362,187.64562606)
\lineto(206.46121209,187.64562606)
\closepath
}
}
{
\newrgbcolor{curcolor}{0 0 0}
\pscustom[linestyle=none,fillstyle=solid,fillcolor=curcolor]
{
\newpath
\moveto(252.92471659,219.53319834)
\lineto(230.22947635,219.53319834)
\lineto(230.22947635,223.1269369)
\lineto(248.06145083,245.17764794)
\lineto(230.89353674,245.17764794)
\lineto(230.89353674,248.615137)
\lineto(252.49503046,248.615137)
\lineto(252.49503046,245.11905438)
\lineto(234.4872753,222.9706874)
\lineto(252.92471659,222.9706874)
\closepath
}
}
{
\newrgbcolor{curcolor}{0 0 0}
\pscustom[linestyle=none,fillstyle=solid,fillcolor=curcolor]
{
\newpath
\moveto(269.33091389,219.53319834)
\lineto(257.84657546,219.53319834)
\lineto(257.84657546,222.50193889)
\lineto(261.65515708,222.50193889)
\lineto(261.65515708,245.64639645)
\lineto(257.84657546,245.64639645)
\lineto(257.84657546,248.615137)
\lineto(269.33091389,248.615137)
\lineto(269.33091389,245.64639645)
\lineto(265.52233227,245.64639645)
\lineto(265.52233227,222.50193889)
\lineto(269.33091389,222.50193889)
\closepath
}
}
{
\newrgbcolor{curcolor}{0 0 0}
\pscustom[linestyle=none,fillstyle=solid,fillcolor=curcolor]
{
\newpath
\moveto(297.5925434,245.27530388)
\curveto(298.77743546,243.9732247)(299.6823805,242.37817769)(300.30737851,240.49016287)
\curveto(300.94539731,238.60214805)(301.26440671,236.46022778)(301.26440671,234.06440208)
\curveto(301.26440671,231.66857637)(300.93888691,229.52014571)(300.28784732,227.6191101)
\curveto(299.64982852,225.73109528)(298.75139388,224.15557946)(297.5925434,222.89256265)
\curveto(296.39463055,221.57746267)(294.97536423,220.58788249)(293.33474446,219.9238221)
\curveto(291.70714547,219.25976171)(289.84517223,218.92773152)(287.74882474,218.92773152)
\curveto(285.70456042,218.92773152)(283.84258718,219.26627211)(282.16290503,219.94335329)
\curveto(280.49624367,220.62043447)(279.07697735,221.60350425)(277.90510608,222.89256265)
\curveto(276.73323482,224.18162104)(275.82828978,225.76364726)(275.19027098,227.63864129)
\curveto(274.56527297,229.51363532)(274.25277396,231.65555558)(274.25277396,234.06440208)
\curveto(274.25277396,236.4341862)(274.56527297,238.55657527)(275.19027098,240.4315693)
\curveto(275.81526899,242.31958413)(276.72672442,243.93416232)(277.92463727,245.27530388)
\curveto(279.07046696,246.55134149)(280.48973327,247.52790088)(282.18243622,248.20498205)
\curveto(283.88815995,248.88206323)(285.74362279,249.22060382)(287.74882474,249.22060382)
\curveto(289.83215144,249.22060382)(291.70063508,248.87555284)(293.35427565,248.18545087)
\curveto(295.02093701,247.50836969)(296.43369292,246.53832069)(297.5925434,245.27530388)
\closepath
\moveto(297.24098202,234.06440208)
\curveto(297.24098202,237.84043172)(296.39463055,240.75057871)(294.7019276,242.79484303)
\curveto(293.00922466,244.85212815)(290.6980341,245.8807707)(287.76835593,245.8807707)
\curveto(284.81263617,245.8807707)(282.48842482,244.85212815)(280.79572188,242.79484303)
\curveto(279.11603973,240.75057871)(278.27619865,237.84043172)(278.27619865,234.06440208)
\curveto(278.27619865,230.24931006)(279.13557092,227.32614228)(280.85431544,225.29489875)
\curveto(282.57305997,223.27667601)(284.87774013,222.26756464)(287.76835593,222.26756464)
\curveto(290.65897173,222.26756464)(292.95714149,223.27667601)(294.66286523,225.29489875)
\curveto(296.38160976,227.32614228)(297.24098202,230.24931006)(297.24098202,234.06440208)
\closepath
}
}
{
\newrgbcolor{curcolor}{0.7019608 0.7019608 0.7019608}
\pscustom[linestyle=none,fillstyle=solid,fillcolor=curcolor,opacity=0.92623001]
{
\newpath
\moveto(206.46121209,93.89651977)
\lineto(325.0326362,93.89651977)
\lineto(325.0326362,1.03938241)
\lineto(206.46121209,1.03938241)
\closepath
}
}
{
\newrgbcolor{curcolor}{0 0 0}
\pscustom[linewidth=2.0787402,linecolor=curcolor]
{
\newpath
\moveto(206.46121209,93.89651977)
\lineto(325.0326362,93.89651977)
\lineto(325.0326362,1.03938241)
\lineto(206.46121209,1.03938241)
\closepath
}
}
{
\newrgbcolor{curcolor}{0 0 0}
\pscustom[linestyle=none,fillstyle=solid,fillcolor=curcolor]
{
\newpath
\moveto(237.18256219,62.00889334)
\lineto(226.5966584,32.92695469)
\lineto(221.44042481,32.92695469)
\lineto(210.85452102,62.00889334)
\lineto(214.99513283,62.00889334)
\lineto(224.11619754,36.42303731)
\lineto(233.23726225,62.00889334)
\closepath
}
}
{
\newrgbcolor{curcolor}{0 0 0}
\pscustom[linestyle=none,fillstyle=solid,fillcolor=curcolor]
{
\newpath
\moveto(266.16684426,47.43862723)
\curveto(266.16684426,44.79540648)(265.58741902,42.39958078)(264.42856854,40.25115012)
\curveto(263.28273886,38.10271946)(261.75279581,36.4360581)(259.83873941,35.25116604)
\curveto(258.51061864,34.43085615)(257.02624836,33.83841012)(255.38562859,33.47382795)
\curveto(253.7580296,33.10924577)(251.60959894,32.92695469)(248.94033661,32.92695469)
\lineto(241.59660999,32.92695469)
\lineto(241.59660999,62.00889334)
\lineto(248.86221186,62.00889334)
\curveto(251.70074449,62.00889334)(253.95334148,61.80056067)(255.62000284,61.38389533)
\curveto(257.29968499,60.98025079)(258.71895131,60.42035673)(259.87780178,59.70421318)
\curveto(261.85696215,58.46723795)(263.39992599,56.82010778)(264.5066933,54.76282266)
\curveto(265.61346061,52.70553755)(266.16684426,50.26413907)(266.16684426,47.43862723)
\closepath
\moveto(262.12388838,47.4972208)
\curveto(262.12388838,49.77585938)(261.72675423,51.69642618)(260.93248593,53.2589212)
\curveto(260.13821762,54.82141623)(258.95332556,56.05188106)(257.37780974,56.9503157)
\curveto(256.23198006,57.60135529)(255.01453602,58.05057261)(253.72547762,58.29796766)
\curveto(252.43641923,58.5583835)(250.89345539,58.68859141)(249.09658611,58.68859141)
\lineto(245.46378518,58.68859141)
\lineto(245.46378518,36.24725662)
\lineto(249.09658611,36.24725662)
\curveto(250.95855935,36.24725662)(252.57964794,36.38397493)(253.95985188,36.65741156)
\curveto(255.35307661,36.93084819)(256.62911421,37.43865907)(257.78796469,38.18084421)
\curveto(259.23327259,39.10532043)(260.31399831,40.32276447)(261.03014186,41.83317633)
\curveto(261.75930621,43.34358819)(262.12388838,45.23160301)(262.12388838,47.4972208)
\closepath
}
}
{
\newrgbcolor{curcolor}{0 0 0}
\pscustom[linestyle=none,fillstyle=solid,fillcolor=curcolor]
{
\newpath
\moveto(291.57692042,32.92695469)
\lineto(272.41682517,32.92695469)
\lineto(272.41682517,62.00889334)
\lineto(291.57692042,62.00889334)
\lineto(291.57692042,58.57140429)
\lineto(276.28400036,58.57140429)
\lineto(276.28400036,50.60267966)
\lineto(291.57692042,50.60267966)
\lineto(291.57692042,47.1651906)
\lineto(276.28400036,47.1651906)
\lineto(276.28400036,36.36444374)
\lineto(291.57692042,36.36444374)
\closepath
}
}
{
\newrgbcolor{curcolor}{0 0 0}
\pscustom[linestyle=none,fillstyle=solid,fillcolor=curcolor]
{
\newpath
\moveto(320.63932431,62.00889334)
\lineto(310.05342051,32.92695469)
\lineto(304.89718693,32.92695469)
\lineto(294.31128314,62.00889334)
\lineto(298.45189495,62.00889334)
\lineto(307.57295966,36.42303731)
\lineto(316.69402437,62.00889334)
\closepath
}
}
{
\newrgbcolor{curcolor}{0 0 0}
\pscustom[linewidth=1.9457763,linecolor=curcolor]
{
\newpath
\moveto(265.74692409,854.34238931)
\lineto(265.74692409,804.27592506)
}
}
{
\newrgbcolor{curcolor}{0 0 0}
\pscustom[linestyle=none,fillstyle=solid,fillcolor=curcolor]
{
\newpath
\moveto(265.74692409,823.73368807)
\lineto(257.96381889,831.51679328)
\lineto(265.74692409,804.27592506)
\lineto(273.5300293,831.51679328)
\closepath
}
}
{
\newrgbcolor{curcolor}{0 0 0}
\pscustom[linewidth=2.07549478,linecolor=curcolor]
{
\newpath
\moveto(265.74692409,823.73368807)
\lineto(257.96381889,831.51679328)
\lineto(265.74692409,804.27592506)
\lineto(273.5300293,831.51679328)
\closepath
}
}
{
\newrgbcolor{curcolor}{0 0 0}
\pscustom[linewidth=2.03376763,linecolor=curcolor]
{
\newpath
\moveto(265.74692409,704.34239687)
\lineto(265.74692409,629.13377829)
}
}
{
\newrgbcolor{curcolor}{0 0 0}
\pscustom[linestyle=none,fillstyle=solid,fillcolor=curcolor]
{
\newpath
\moveto(265.74692409,649.47145454)
\lineto(257.61185359,657.60652504)
\lineto(265.74692409,629.13377829)
\lineto(273.88199459,657.60652504)
\closepath
}
}
{
\newrgbcolor{curcolor}{0 0 0}
\pscustom[linewidth=2.1693522,linecolor=curcolor]
{
\newpath
\moveto(265.74692409,649.47145454)
\lineto(257.61185359,657.60652504)
\lineto(265.74692409,629.13377829)
\lineto(273.88199459,657.60652504)
\closepath
}
}
{
\newrgbcolor{curcolor}{0 0 0}
\pscustom[linewidth=1.99659212,linecolor=curcolor]
{
\newpath
\moveto(264.67550362,531.48526176)
\lineto(264.67550362,457.02433577)
}
}
{
\newrgbcolor{curcolor}{0 0 0}
\pscustom[linestyle=none,fillstyle=solid,fillcolor=curcolor]
{
\newpath
\moveto(264.67550362,476.99025701)
\lineto(256.68913513,484.9766255)
\lineto(264.67550362,457.02433577)
\lineto(272.66187212,484.9766255)
\closepath
}
}
{
\newrgbcolor{curcolor}{0 0 0}
\pscustom[linewidth=2.12969833,linecolor=curcolor]
{
\newpath
\moveto(264.67550362,476.99025701)
\lineto(256.68913513,484.9766255)
\lineto(264.67550362,457.02433577)
\lineto(272.66187212,484.9766255)
\closepath
}
}
{
\newrgbcolor{curcolor}{0 0 0}
\pscustom[linewidth=1.99659212,linecolor=curcolor]
{
\newpath
\moveto(265.74692409,360.65379089)
\lineto(265.74692409,286.19286491)
}
}
{
\newrgbcolor{curcolor}{0 0 0}
\pscustom[linestyle=none,fillstyle=solid,fillcolor=curcolor]
{
\newpath
\moveto(265.74692409,306.15878614)
\lineto(257.7605556,314.14515464)
\lineto(265.74692409,286.19286491)
\lineto(273.73329259,314.14515464)
\closepath
}
}
{
\newrgbcolor{curcolor}{0 0 0}
\pscustom[linewidth=2.12969833,linecolor=curcolor]
{
\newpath
\moveto(265.74692409,306.15878614)
\lineto(257.7605556,314.14515464)
\lineto(265.74692409,286.19286491)
\lineto(273.73329259,314.14515464)
\closepath
}
}
{
\newrgbcolor{curcolor}{0 0 0}
\pscustom[linewidth=2.18703871,linecolor=curcolor]
{
\newpath
\moveto(264.87630992,188.36240191)
\lineto(264.87630992,99.01883026)
}
}
{
\newrgbcolor{curcolor}{0 0 0}
\pscustom[linestyle=none,fillstyle=solid,fillcolor=curcolor]
{
\newpath
\moveto(264.87630992,120.88921733)
\lineto(256.12815509,129.63737216)
\lineto(264.87630992,99.01883026)
\lineto(273.62446475,129.63737216)
\closepath
}
}
{
\newrgbcolor{curcolor}{0 0 0}
\pscustom[linewidth=2.33284136,linecolor=curcolor]
{
\newpath
\moveto(264.87630992,120.88921733)
\lineto(256.12815509,129.63737216)
\lineto(264.87630992,99.01883026)
\lineto(273.62446475,129.63737216)
\closepath
}
}
{
\newrgbcolor{curcolor}{0 0 0}
\pscustom[linewidth=2.52864755,linecolor=curcolor]
{
\newpath
\moveto(206.88957732,776.67115152)
\lineto(59.20807559,713.33549924)
}
}
{
\newrgbcolor{curcolor}{0 0 0}
\pscustom[linestyle=none,fillstyle=solid,fillcolor=curcolor]
{
\newpath
\moveto(82.44753014,723.30212332)
\lineto(87.75666232,736.58455478)
\lineto(59.20807559,713.33549924)
\lineto(95.72996159,717.99299114)
\closepath
}
}
{
\newrgbcolor{curcolor}{0 0 0}
\pscustom[linewidth=2.69722413,linecolor=curcolor]
{
\newpath
\moveto(82.44753014,723.30212332)
\lineto(87.75666232,736.58455478)
\lineto(59.20807559,713.33549924)
\lineto(95.72996159,717.99299114)
\closepath
}
}
{
\newrgbcolor{curcolor}{0 0 0}
\pscustom[linewidth=2.01889504,linecolor=curcolor]
{
\newpath
\moveto(56.03149606,617.95587467)
\lineto(240.73290709,282.88997357)
}
}
{
\newrgbcolor{curcolor}{0 0 0}
\pscustom[linestyle=none,fillstyle=solid,fillcolor=curcolor]
{
\newpath
\moveto(230.98665836,300.57059439)
\lineto(220.01591054,303.74434322)
\lineto(240.73290709,282.88997357)
\lineto(234.16040719,311.54134221)
\closepath
}
}
{
\newrgbcolor{curcolor}{0 0 0}
\pscustom[linewidth=2.1534881,linecolor=curcolor]
{
\newpath
\moveto(230.98665836,300.57059439)
\lineto(220.01591054,303.74434322)
\lineto(240.73290709,282.88997357)
\lineto(234.16040719,311.54134221)
\closepath
}
}
{
\newrgbcolor{curcolor}{0 0 0}
\pscustom[linewidth=2.10693547,linecolor=curcolor]
{
\newpath
\moveto(473.06070047,618.67491467)
\lineto(329.18916661,572.96345262)
}
}
{
\newrgbcolor{curcolor}{0 0 0}
\pscustom[linestyle=none,fillstyle=solid,fillcolor=curcolor]
{
\newpath
\moveto(349.26935139,579.34341244)
\lineto(354.74944138,589.92747027)
\lineto(329.18916661,572.96345262)
\lineto(359.85340923,573.86332245)
\closepath
}
}
{
\newrgbcolor{curcolor}{0 0 0}
\pscustom[linewidth=2.2473979,linecolor=curcolor]
{
\newpath
\moveto(349.26935139,579.34341244)
\lineto(354.74944138,589.92747027)
\lineto(329.18916661,572.96345262)
\lineto(359.85340923,573.86332245)
\closepath
}
}
{
\newrgbcolor{curcolor}{0 0 0}
\pscustom[linewidth=2.41853489,linecolor=curcolor]
{
\newpath
\moveto(325.88756031,778.3434753)
\lineto(466.08898394,713.21688254)
}
}
{
\newrgbcolor{curcolor}{0 0 0}
\pscustom[linestyle=none,fillstyle=solid,fillcolor=curcolor]
{
\newpath
\moveto(444.15463244,723.40586308)
\lineto(431.30529963,718.7077147)
\lineto(466.08898394,713.21688254)
\lineto(439.45648406,736.2551959)
\closepath
}
}
{
\newrgbcolor{curcolor}{0 0 0}
\pscustom[linewidth=2.57977062,linecolor=curcolor]
{
\newpath
\moveto(444.15463244,723.40586308)
\lineto(431.30529963,718.7077147)
\lineto(466.08898394,713.21688254)
\lineto(439.45648406,736.2551959)
\closepath
}
}
\end{pspicture}
}
    %\captionsetup{width=0.75\linewidth}
    \caption{ZFS consists of many components working together to write and read data from disks.
    This diagram depicts ZFS from the perspective of a file read operation,
    including the VFS layer above it that actually calls into ZFS to start a read.}
    \label{fig:ZFSOrganization}
\end{figure}

\subsection{ZPL}
The ZPL is the ZFS Posix Layer \cite{dawidek_closer_2008}. 
This component interfaces with the VFS, or Virtual Filesystem, layer above it, which handles filesystem operations to local or network
filesystems.
within the Linux kernel.
All file operations within ZFS are actually organized as operations on objects, of which files are just one special type.
Thus the ZPL must translate requests for file reads and writes into operations on specific objects in specific object sets.

\subsection{ZAP}
The ZAP is the ZFS Attribute Processor \cite{dawidek_closer_2008}. 
It is a key-value store that is used primarily for directories, and it stores a mapping of names to objects in the object set.

\subsection{DMU}
The DMU is the Data Management Unit \cite{ahrens_read_write}. 
Its primary job is to bundle operations into transactions, and hold the first layer of caching for the filesystem as a set of dbuf objects. 
These dbuf objects are just metadata that points to data stored in the ARC, to avoid even having to ask the ARC for frequently accessed data.

\subsection{ARC}
The Adaptive Replacement Cache is ZFS's large in-memory cache \cite{ahrens_openzfs_basics, ahrens_read_write}.
This cache consists of two internal dynamically-sized caches with different policies, a least recently used cache
for data accessed only once and a least frequently used cache for data accessed multiple times.
The ARC is the most important part of ZFS for the purposes of this work and is covered in more detail in Section \ref{chapter:ARC}.

\subsection{ZIO}
ZIO is the ZFS I/O Pipeline and it is the primary component of the SPA, or Storage Pool Allocator \cite{ahrens_read_write}.
All operations performed on disks go through it. 
It also handles compression, checksumming (and checksum verification), and encryption.

There are many different ZIO queues, contained within a two-dimensional array in the per-pool spa object 
\cite[{module/zfs/spa.c}]{zfs}.
There are six  types of ZIO: read, write, free, claim, ioctl, trim, each with 4 queues of different priorities,
issue, issue high, interrupt, and interrupt high.
The ZIO scheduler considers five classes of I/O, prioritized in this order: synchronous read, synchronous write, asynchronous read, asynchronous write, and scrub or resilver.

\section{A Read Operation in ZFS}
A read operation to a filesystem in ZFS goes through almost every part of the filesystem and gives a good overview of how
ZFS works.
It is also the operation that this work focuses on improving.
Each snapshot, filesystem, or clone has its own ZPL and DMU, but the ARC and SPA are shared per storage pool 
\cite{ahrens_read_write}.

\subsection{ZPL}
The ZFS POSIX Layer is entrypoint from the Linux VFS, the Virtual FileSystem layer into ZFS \cite{ahrens_read_write,zfs}.
The VFS is the common interface between Linux and all of its filesystems.
It asks the ZPL for some amount of data from a specific file at a specific offset and gives it a buffer to fill.
The ZPL then converts this information into a ZFS object via the ZAP and requests that from the DMU.

\subsection{DMU}
The Data Management Unit takes the object request from the ZPL and first looks to see if that object is cached in its internal cache
\cite{ahrens_read_write,zfs}.
This cache consists of DBUFs, or DMU buffers, which represent one logical block, one block of one file of one filesystem.
Their actual content is just a reference to an ARC buffer (Section \ref{chapter:ARC}).
If the DMU has all the DBUFs that are needed to satisfy this particular read, then it sends the data back to the ZPL.
This is the quickest path through the read code available within ZFS, and is the best of best cases.

Otherwise, the DMU will make a new DBUF and and locate the block pointers for the data requested.
To do this it has to look for the indirect block parents that point to the blocks or the parents of those blocks,
or the parents of the parents of those blocks, 
continuing up the tree of blocks until it locates something in the DMU cache.
The dnode of the file, which contains the root indirect block pointer of the file,
will always be cached because the file reference passed by the VFS is a vnode that points to it.
Once an indirect block is located, the chain of block pointers must then be fetched to eventually 
find the block pointer for the actual data.
These block pointers are then passed to the ARC to retrieve the data they point to.
The ARC will eventually gives back a pointer to an ABD, which is then associated with the new DBUF.
The prefetch engine is informed about all reads as they happen, in order to detect patterns and preemptively
request data that it believes will be needed soon (Section \ref{chapter:prefetch}).

The DMU also creates an empty ZIO root and passes it along to the ARC.
This allows it to wait on this root which will be made the root of any disk operations that the ARC
needs to perform in order to retrieve the data.
By waiting on the root it will wait on all of its children, which requires waiting on their children, 
which ensures that all required I/O has finished.
This guarantees that once the wait is over all the data the DMU requested has been populated into the correct buffers
and it can simply return that data to the user.

\subsection{Prefetch}
\label{chapter:prefetch}
The prefetch engine looks for patterns in the requests that the DMU receives and preemptively requests the data that it thinks will be needed\cite{ahrens_read_write}.
It tracks many streams of reads at once  to find those that are sequential in order to fetch the next blocks in those sequences.
When it detects a read that is sequential to a previous read, it will request that the next
few blocks in the sequence be loaded into the ARC.
How many blocks it asks for depends on how successful prefetching has been previously, fetching
more and more data up to 8 MB at a time if prefetching continues to correctly guess the data that will be
accessed next\cite[{module/zfs/dmu.c} {module/zfs/dbuf.c}]{zfs}.
This ensures that the data the process will be asking for is already in the ARC before it even needs it,
which can speed up file accesses.

\subsection{ARC}
The Adaptive Replacement Cache is primarily a hash map of blocks indexed by block pointer, 
so the first thing it does is check this hash map for the passed block pointer
\cite{ahrens_read_write,zfs}.
If it is already in the ARC, then it simply returns that cached data to the DMU.
Otherwise, a new entry in the ARC, called an ARC header, is created.
The ARC then invokes the ZIO pipeline to retrieve the data from the disks.

\subsection{The ZIO Pipeline}
ZIOs form a dependency graph, which lets lots of I/O be pending at once \cite{ahrens_read_write}. 
Due to the format of ZFS on disk, one small read might depend on reading a large number of blocks, 
depending on the size of the file.
The indirect blocks that allow files to grow as large as needed also mean that more reads are required 
in order to retrieve the data stored in the direct block from a disk.

This system is especially beneficial to writes, as each indirect block can be added to the tree with a dependency on 
all the data blocks below it, and once those are complete
that indirect block can be written, regardless of how many other data blocks might still need to written.

The goal of this complex dependency system is to keep all disks busy at all times whenever ZFS has to read or write data.
As disks are the typically the bottleneck, being the slowest component of a typical modern system,
it is important to maximize their throughput for the best performance.
It also allows for easy representation of more complex data layouts, like mirroring or RAIDZ, 
where the logical operation of writing a single block depends on multiple physical writes to disks.

ZIOs also verify the checksums of the blocks they read, 
which allows them to try and recover when those checksums show that data to be invalid.
It can use the block pointer to find the locations of other copies of the data, if others were created, 
and create new children ZIOs to get that data instead.
Thanks to the checksum in the block pointer, it can then create another child ZIO to transparently write a new block
containing the correct data to the corrupted side of the mirror, once it has found a valid copy.

If the data being read is compressed or encrypted on disk, then a secondary buffer will be allocated for the on-disk data,
and a decompression or decryption step will be added to the process via pushing a the required transformation 
to the ZIO's transformation stack.
Compression is almost always used in real deployments of ZFS, as it improves performance by allowing for fewer disk operations.

This process begins with a logical ZIO associated with a block pointer, but these are transformed into
child physical ZIOs for a specific VDEV and offset, based on the DVAs present in the block pointer,
typically just picking the first DVA found in the block pointer.
These physical ZIOs pass their data on to the logical ZIO, 
which figures out if the read succeeded or if it needs to try a different DVA.

A DVA might point to a non-leaf VDEV, that is a mirror or RAIDZ VDEV instead of a physical disk, 
in which case more physical child ZIOs will be created to retrieve the data for the original physical ZIO.

Eventually, there will be a ZIO associated with a leaf VDEV, at which point it is, finally, queued for being read from disk.
Since disks can perform far fewer operations at once, ZIOs are prioritized and sorted by disk offset to avoid having to 
unnecessarily seek the disk head when possible.
The queue also aggregates small reads into larger reads when it finds that smaller reads currently queued are physically sequential on disk.

The pipeline is given the next ZIO to execute and runs the relevant operation on disk, 
which eventually calls Linux's \texttt{submit\_bio} function.
In the case of a synchronous operation, it will call \texttt{zio\_wait} and execute ZIOs from the queue, in parallel with zio worker threads,
until the passed ZIO, the root ZIO of all the required operations, is complete.

\subsection{Completion}
The ZIO pipeline will, after completing many, many ZIOs, populate an ARC buffer and call the ARC read done callback passed by the DMU, 
which associates the passed ARC buffer with the relevant DBUF \cite{ahrens_read_write,zfs}.
Once all the necessary DBUFs are in the DMU, then the data is copied into the buffer passed by the ZPL, which ultimately came from VFS layer and completes the read operation.

\section{Adaptive Replacement Cache}
\label{chapter:ARC}
ZFS relies heavily on memory to ensure performance while retaining its always-consistent property \cite{ahrens_read_write}.
The primary way in which this is done is the ARC, or the Adaptive Replacement Cache. 
This cache works like a page cache in other filesystems, storing the contents of files that have been accessed
in order to ensure quicker access to those files in the future and storing the contents of writes that have yet to be
propagated to disk.

The ARC consists of two dynamically-sized caches, one that follows a least recently used eviction policy 
and another which follows a least frequently used eviction policy.
The first contains data that has been accessed only once and the second contains data accessed twice or more \cite{megiddo_dharmendra_ARC}.
A least recently used cache policy will discard that data which has been accessed least recently
when filled and needing to add more data, as the name would suggest, 
A least frequency used cache policy will discard instead that data which has been accessed least frequently, 
that is the fewest number of times.
These caches are hash tables of block pointers, so the ARC has no understanding of files, 
delegating that to the abstraction layers above it.

The relative size of these two caches is determined by a ghost cache. 
The ghost cache contains the headers, but not the data, of all the recently evicted blocks
along with which part of the cache they were stored in.
When a cache miss occurs in the ARC, it checks the ghost cache to see if the data requested was recently stored.
If it was, then the ARC will make the component of the cache that data was previous stored in larger and the other smaller,
under the assumption that this would have helped the ARC perform better and have fewer cache misses.

This combination of multiple cache types gives the ARC the important advantage of being scan-resistant,
that is reading a very large file one time sequentially will not substantially degrade the performance of the cache.
In simpler caching systems, such as a single LRU, reading data that is larger than the size of the cache will replace the entire
contents of the cache.
This is a problem, as one large file read can then destroy the performance of the cache on most systems,
where often certain files are accessed quite frequently and should always been cached if memory allows.

The ARC has also been extended in ZFS with fast disk storage, known as the L2ARC,
while the in-memory ARC is now referred to within ZFS as the L1ARC.
When configured, the ARC will use a faster disk, such as an SSD, as a secondary larger cache for speeding up access to your data.
L2ARC is of course still slower than L1ARC, but it can be much larger than the L1ARC,
because of the vast differences in size that generally exist between disks and RAM.

\begin{figure}
    \centering
    \resizebox{0.75\linewidth}{!}{%LaTeX with PSTricks extensions
%%Creator: Inkscape 1.0.2-2 (e86c870879, 2021-01-15)
%%Please note this file requires PSTricks extensions
\psset{xunit=.5pt,yunit=.5pt,runit=.5pt}
\begin{pspicture}(757.55496048,389.66862235)
{
\newrgbcolor{curcolor}{0 0 0}
\pscustom[linewidth=1.88976378,linecolor=curcolor]
{
\newpath
\moveto(0.9457848,372.66979592)
\lineto(143.45492566,372.66979592)
\lineto(143.45492566,290.69652691)
\lineto(0.9457848,290.69652691)
\closepath
}
}
{
\newrgbcolor{curcolor}{0 0 0}
\pscustom[linewidth=1.88976378,linecolor=curcolor]
{
\newpath
\moveto(0.94488189,290.56258598)
\lineto(143.4558538,290.56258598)
\lineto(143.4558538,253.65979817)
\lineto(0.94488189,253.65979817)
\closepath
}
}
{
\newrgbcolor{curcolor}{0 0 0}
\pscustom[linewidth=1.88976378,linecolor=curcolor]
{
\newpath
\moveto(0.94701211,253.4228496)
\lineto(143.45605204,253.4228496)
\lineto(143.45605204,171.65560308)
\lineto(0.94701211,171.65560308)
\closepath
}
}
{
\newrgbcolor{curcolor}{0 0 0}
\pscustom[linestyle=none,fillstyle=solid,fillcolor=curcolor]
{
\newpath
\moveto(6.20418244,379.8639431)
\lineto(6.20418244,380.80925481)
\curveto(5.25105825,380.00717215)(4.23282994,379.60613082)(3.14949751,379.60613082)
\curveto(2.36303984,379.60613082)(1.74845703,379.80404732)(1.30574906,380.19988032)
\curveto(0.8630411,380.60092165)(0.64168712,381.09050457)(0.64168712,381.66862908)
\curveto(0.64168712,382.30404522)(0.93335354,382.85873225)(1.51668639,383.33269018)
\curveto(2.10001923,383.80664812)(2.95158101,384.04362709)(4.07137174,384.04362709)
\curveto(4.37345482,384.04362709)(4.70157954,384.02279377)(5.05574591,383.98112714)
\curveto(5.40991228,383.94466884)(5.79272446,383.88477305)(6.20418244,383.80143979)
\lineto(6.20418244,384.8639389)
\curveto(6.20418244,385.22331359)(6.03751592,385.53581333)(5.70418287,385.80143811)
\curveto(5.37084981,386.06706288)(4.87085023,386.19987527)(4.20418413,386.19987527)
\curveto(3.69376789,386.19987527)(2.97762266,386.0514379)(2.05574843,385.75456315)
\curveto(1.88908191,385.70247986)(1.78231116,385.67643821)(1.7354362,385.67643821)
\curveto(1.65210294,385.67643821)(1.57918633,385.70768819)(1.51668639,385.77018813)
\curveto(1.45939477,385.83268808)(1.43074896,385.91081302)(1.43074896,386.00456294)
\curveto(1.43074896,386.09310453)(1.4567906,386.16341697)(1.50887389,386.21550026)
\curveto(1.5817905,386.29362519)(1.87606108,386.40039594)(2.39168565,386.53581249)
\curveto(3.20418497,386.75456231)(3.81876778,386.86393721)(4.2354341,386.86393721)
\curveto(5.0635584,386.86393721)(5.70939119,386.65820822)(6.17293247,386.24675023)
\curveto(6.63647375,385.84050057)(6.86824439,385.37956346)(6.86824439,384.8639389)
\lineto(6.86824439,380.52019255)
\lineto(7.74324365,380.52019255)
\curveto(7.90470185,380.52019255)(8.01928508,380.48894258)(8.08699336,380.42644263)
\curveto(8.15470164,380.36915101)(8.18855578,380.29102608)(8.18855578,380.19206782)
\curveto(8.18855578,380.0983179)(8.15470164,380.02019297)(8.08699336,379.95769302)
\curveto(8.01928508,379.89519307)(7.90470185,379.8639431)(7.74324365,379.8639431)
\closepath
\moveto(6.20418244,383.12956535)
\curveto(5.89689104,383.21810695)(5.57137048,383.28321106)(5.22762077,383.32487769)
\curveto(4.88387106,383.36654432)(4.52189219,383.38737764)(4.14168418,383.38737764)
\curveto(3.18855998,383.38737764)(2.44376894,383.18164864)(1.90731106,382.77019066)
\curveto(1.5010614,382.46289925)(1.29793657,382.09571206)(1.29793657,381.66862908)
\curveto(1.29793657,381.27279608)(1.45158228,380.93946303)(1.75887368,380.66862992)
\curveto(2.07137342,380.39779682)(2.52449804,380.26238027)(3.11824754,380.26238027)
\curveto(3.6859554,380.26238027)(4.21199662,380.37435934)(4.69637121,380.59831748)
\curveto(5.18595413,380.82748396)(5.68855788,381.18946282)(6.20418244,381.68425407)
\closepath
}
}
{
\newrgbcolor{curcolor}{0 0 0}
\pscustom[linestyle=none,fillstyle=solid,fillcolor=curcolor]
{
\newpath
\moveto(13.13386396,386.62956241)
\lineto(13.13386396,384.9733138)
\curveto(13.98802991,385.74414649)(14.62605021,386.23893774)(15.04792485,386.45768756)
\curveto(15.47500783,386.6816457)(15.86823666,386.79362477)(16.22761136,386.79362477)
\curveto(16.61823603,386.79362477)(16.98021489,386.66081238)(17.31354795,386.39518761)
\curveto(17.65208933,386.13477116)(17.82136002,385.93685466)(17.82136002,385.80143811)
\curveto(17.82136002,385.70247986)(17.78750588,385.61914659)(17.7197976,385.55143832)
\curveto(17.65729766,385.48893837)(17.57656856,385.4576884)(17.47761031,385.4576884)
\curveto(17.42552702,385.4576884)(17.38125622,385.46550089)(17.34479792,385.48112588)
\curveto(17.30833962,385.50195919)(17.24063134,385.56185498)(17.14167309,385.66081323)
\curveto(16.95938158,385.84310474)(16.80052754,385.96810463)(16.66511099,386.03581291)
\curveto(16.52969444,386.10352119)(16.39688205,386.13737532)(16.26667383,386.13737532)
\curveto(15.98021573,386.13737532)(15.63386186,386.02279209)(15.2276122,385.79362561)
\curveto(14.82657087,385.56445914)(14.12865479,385.00456378)(13.13386396,384.11393953)
\lineto(13.13386396,380.52019255)
\lineto(16.04011152,380.52019255)
\curveto(16.20156971,380.52019255)(16.31615295,380.48894258)(16.38386123,380.42644263)
\curveto(16.4515695,380.36915101)(16.48542364,380.29102608)(16.48542364,380.19206782)
\curveto(16.48542364,380.0983179)(16.4515695,380.02019297)(16.38386123,379.95769302)
\curveto(16.31615295,379.89519307)(16.20156971,379.8639431)(16.04011152,379.8639431)
\lineto(10.89167835,379.8639431)
\curveto(10.73542848,379.8639431)(10.62344941,379.89258891)(10.55574113,379.94988053)
\curveto(10.48803285,380.01238048)(10.45417872,380.09050541)(10.45417872,380.18425533)
\curveto(10.45417872,380.27279692)(10.48542869,380.34571353)(10.54792864,380.40300515)
\curveto(10.61563691,380.46550509)(10.73022015,380.49675507)(10.89167835,380.49675507)
\lineto(12.47761451,380.49675507)
\lineto(12.47761451,385.96550047)
\lineto(11.26667803,385.96550047)
\curveto(11.11042816,385.96550047)(10.99844909,385.99675044)(10.93074081,386.05925039)
\curveto(10.86303254,386.12175034)(10.8291784,386.20247944)(10.8291784,386.30143769)
\curveto(10.8291784,386.39518761)(10.86042837,386.47331254)(10.92292832,386.53581249)
\curveto(10.9906366,386.59831244)(11.10521983,386.62956241)(11.26667803,386.62956241)
\closepath
}
}
{
\newrgbcolor{curcolor}{0 0 0}
\pscustom[linestyle=none,fillstyle=solid,fillcolor=curcolor]
{
\newpath
\moveto(26.25885277,385.96550047)
\lineto(26.25885277,386.18425029)
\curveto(26.25885277,386.34570848)(26.29010274,386.46029172)(26.35260269,386.528)
\curveto(26.41510264,386.59570827)(26.49322757,386.62956241)(26.58697749,386.62956241)
\curveto(26.68593574,386.62956241)(26.76666484,386.59570827)(26.82916479,386.528)
\curveto(26.89166474,386.46029172)(26.92291471,386.34570848)(26.92291471,386.18425029)
\lineto(26.92291471,384.69987653)
\curveto(26.91770638,384.53841834)(26.88385224,384.4238351)(26.8213523,384.35612682)
\curveto(26.76406068,384.28841855)(26.68593574,384.25456441)(26.58697749,384.25456441)
\curveto(26.4984359,384.25456441)(26.42291513,384.28321022)(26.36041518,384.34050184)
\curveto(26.30312356,384.40300178)(26.26926943,384.5045642)(26.25885277,384.64518908)
\curveto(26.22760279,385.01498044)(25.98281133,385.36654264)(25.52447839,385.69987569)
\curveto(25.07135377,386.03320875)(24.45937511,386.19987527)(23.68854243,386.19987527)
\curveto(22.71458492,386.19987527)(21.9750022,385.89518803)(21.4697943,385.28581354)
\curveto(20.96458639,384.67643905)(20.71198243,383.97852297)(20.71198243,383.1920653)
\curveto(20.71198243,382.34310768)(20.99062803,381.64258744)(21.54791923,381.09050457)
\curveto(22.10521043,380.5384217)(22.82656399,380.26238027)(23.71197991,380.26238027)
\curveto(24.22239615,380.26238027)(24.74062488,380.35613019)(25.2666661,380.54363003)
\curveto(25.79791566,380.73112987)(26.27708192,381.03321295)(26.70416489,381.44987927)
\curveto(26.8135398,381.55404585)(26.90989389,381.60612914)(26.99322715,381.60612914)
\curveto(27.08176874,381.60612914)(27.15468535,381.57487916)(27.21197697,381.51237921)
\curveto(27.27447691,381.4550876)(27.30572689,381.38217099)(27.30572689,381.2936294)
\curveto(27.30572689,381.06967125)(27.04270628,380.78581733)(26.51666505,380.44206761)
\curveto(25.66770743,379.88477642)(24.72239573,379.60613082)(23.68072994,379.60613082)
\curveto(22.62343916,379.60613082)(21.75364822,379.94206804)(21.07135713,380.61394247)
\curveto(20.39427437,381.29102523)(20.05573299,382.14779535)(20.05573299,383.18425281)
\curveto(20.05573299,384.24154359)(20.40208686,385.11914701)(21.09479461,385.81706309)
\curveto(21.79271069,386.51497917)(22.67031412,386.86393721)(23.7276049,386.86393721)
\curveto(24.73281238,386.86393721)(25.57656168,386.5644583)(26.25885277,385.96550047)
\closepath
}
}
{
\newrgbcolor{curcolor}{0 0 0}
\pscustom[linestyle=none,fillstyle=solid,fillcolor=curcolor]
{
\newpath
\moveto(37.48540657,375.48113429)
\lineto(28.75885141,375.48113429)
\curveto(28.60260154,375.48113429)(28.49062247,375.51238426)(28.42291419,375.57488421)
\curveto(28.35520592,375.63217583)(28.32135178,375.71030076)(28.32135178,375.80925901)
\curveto(28.32135178,375.90821726)(28.35520592,375.98894636)(28.42291419,376.05144631)
\curveto(28.49062247,376.10873793)(28.60260154,376.13738374)(28.75885141,376.13738374)
\lineto(37.48540657,376.13738374)
\curveto(37.64686477,376.13738374)(37.75884384,376.10873793)(37.82134379,376.05144631)
\curveto(37.88905207,375.98894636)(37.9229062,375.90821726)(37.9229062,375.80925901)
\curveto(37.9229062,375.71030076)(37.88905207,375.63217583)(37.82134379,375.57488421)
\curveto(37.75884384,375.51238426)(37.64686477,375.48113429)(37.48540657,375.48113429)
\closepath
}
}
{
\newrgbcolor{curcolor}{0 0 0}
\pscustom[linestyle=none,fillstyle=solid,fillcolor=curcolor]
{
\newpath
\moveto(40.25102819,389.66862235)
\lineto(40.25102819,385.31706351)
\curveto(41.04269419,386.34831265)(41.99842255,386.86393721)(43.11821328,386.86393721)
\curveto(44.07654581,386.86393721)(44.89685762,386.51497917)(45.57914871,385.81706309)
\curveto(46.2614398,385.12435534)(46.60258535,384.27279356)(46.60258535,383.26237774)
\curveto(46.60258535,382.24154527)(46.25623147,381.37696266)(45.56352372,380.66862992)
\curveto(44.8760243,379.96029719)(44.06092082,379.60613082)(43.11821328,379.60613082)
\curveto(41.97238091,379.60613082)(41.01665255,380.12175538)(40.25102819,381.15300452)
\lineto(40.25102819,379.8639431)
\lineto(38.71196698,379.8639431)
\curveto(38.55571712,379.8639431)(38.44373804,379.89519307)(38.37602977,379.95769302)
\curveto(38.30832149,380.02019297)(38.27446735,380.0983179)(38.27446735,380.19206782)
\curveto(38.27446735,380.29102608)(38.30832149,380.36915101)(38.37602977,380.42644263)
\curveto(38.44373804,380.48894258)(38.55571712,380.52019255)(38.71196698,380.52019255)
\lineto(39.59477874,380.52019255)
\lineto(39.59477874,389.00456041)
\lineto(38.71196698,389.00456041)
\curveto(38.55571712,389.00456041)(38.44373804,389.03581039)(38.37602977,389.09831033)
\curveto(38.30832149,389.16081028)(38.27446735,389.24153938)(38.27446735,389.34049763)
\curveto(38.27446735,389.43424755)(38.30832149,389.51237249)(38.37602977,389.57487243)
\curveto(38.44373804,389.63737238)(38.55571712,389.66862235)(38.71196698,389.66862235)
\closepath
\moveto(45.9463359,383.23112777)
\curveto(45.9463359,384.05925207)(45.66248197,384.75977232)(45.09477412,385.3326885)
\curveto(44.52706626,385.91081302)(43.86300432,386.19987527)(43.10258829,386.19987527)
\curveto(42.34217227,386.19987527)(41.67811032,385.91081302)(41.11040247,385.3326885)
\curveto(40.54269461,384.75977232)(40.25884068,384.05925207)(40.25884068,383.23112777)
\curveto(40.25884068,382.40300347)(40.54269461,381.69987906)(41.11040247,381.12175454)
\curveto(41.67811032,380.54883836)(42.34217227,380.26238027)(43.10258829,380.26238027)
\curveto(43.86300432,380.26238027)(44.52706626,380.54883836)(45.09477412,381.12175454)
\curveto(45.66248197,381.69987906)(45.9463359,382.40300347)(45.9463359,383.23112777)
\closepath
}
}
{
\newrgbcolor{curcolor}{0 0 0}
\pscustom[linestyle=none,fillstyle=solid,fillcolor=curcolor]
{
\newpath
\moveto(54.57132759,379.8639431)
\lineto(54.57132759,380.82487979)
\curveto(53.67549501,380.01238048)(52.70674583,379.60613082)(51.66508003,379.60613082)
\curveto(51.02445557,379.60613082)(50.53747682,379.78060984)(50.20414376,380.12956788)
\curveto(49.77185246,380.58790083)(49.55570681,381.12175454)(49.55570681,381.73112903)
\lineto(49.55570681,385.96550047)
\lineto(48.67289505,385.96550047)
\curveto(48.51664518,385.96550047)(48.40466611,385.99675044)(48.33695783,386.05925039)
\curveto(48.26924956,386.12175034)(48.23539542,386.20247944)(48.23539542,386.30143769)
\curveto(48.23539542,386.39518761)(48.26924956,386.47331254)(48.33695783,386.53581249)
\curveto(48.40466611,386.59831244)(48.51664518,386.62956241)(48.67289505,386.62956241)
\lineto(50.21195626,386.62956241)
\lineto(50.21195626,381.73112903)
\curveto(50.21195626,381.30404606)(50.34737281,380.95248385)(50.61820592,380.67644242)
\curveto(50.88903902,380.40040098)(51.2275804,380.26238027)(51.63383006,380.26238027)
\curveto(52.7015375,380.26238027)(53.68070334,380.75196319)(54.57132759,381.73112903)
\lineto(54.57132759,385.96550047)
\lineto(53.36039111,385.96550047)
\curveto(53.20414124,385.96550047)(53.09216217,385.99675044)(53.02445389,386.05925039)
\curveto(52.95674562,386.12175034)(52.92289148,386.20247944)(52.92289148,386.30143769)
\curveto(52.92289148,386.39518761)(52.95674562,386.47331254)(53.02445389,386.53581249)
\curveto(53.09216217,386.59831244)(53.20414124,386.62956241)(53.36039111,386.62956241)
\lineto(55.22757704,386.62956241)
\lineto(55.22757704,380.52019255)
\lineto(55.78226407,380.52019255)
\curveto(55.93851394,380.52019255)(56.05049301,380.48894258)(56.11820129,380.42644263)
\curveto(56.18590957,380.36915101)(56.2197637,380.29102608)(56.2197637,380.19206782)
\curveto(56.2197637,380.0983179)(56.18590957,380.02019297)(56.11820129,379.95769302)
\curveto(56.05049301,379.89519307)(55.93851394,379.8639431)(55.78226407,379.8639431)
\closepath
}
}
{
\newrgbcolor{curcolor}{0 0 0}
\pscustom[linestyle=none,fillstyle=solid,fillcolor=curcolor]
{
\newpath
\moveto(61.49319932,385.96550047)
\lineto(61.49319932,380.52019255)
\lineto(64.37600939,380.52019255)
\curveto(64.53225926,380.52019255)(64.64423833,380.48894258)(64.71194661,380.42644263)
\curveto(64.77965489,380.36915101)(64.81350902,380.29102608)(64.81350902,380.19206782)
\curveto(64.81350902,380.0983179)(64.77965489,380.02019297)(64.71194661,379.95769302)
\curveto(64.64423833,379.89519307)(64.53225926,379.8639431)(64.37600939,379.8639431)
\lineto(59.24320121,379.8639431)
\curveto(59.08695134,379.8639431)(58.97497227,379.89519307)(58.90726399,379.95769302)
\curveto(58.83955572,380.02019297)(58.80570158,380.0983179)(58.80570158,380.19206782)
\curveto(58.80570158,380.29102608)(58.83955572,380.36915101)(58.90726399,380.42644263)
\curveto(58.97497227,380.48894258)(59.08695134,380.52019255)(59.24320121,380.52019255)
\lineto(60.82913738,380.52019255)
\lineto(60.82913738,385.96550047)
\lineto(59.40726357,385.96550047)
\curveto(59.2510137,385.96550047)(59.13903463,385.99675044)(59.07132635,386.05925039)
\curveto(59.00361808,386.12175034)(58.96976394,386.20247944)(58.96976394,386.30143769)
\curveto(58.96976394,386.39518761)(59.00361808,386.47331254)(59.07132635,386.53581249)
\curveto(59.13903463,386.59831244)(59.2510137,386.62956241)(59.40726357,386.62956241)
\lineto(60.82913738,386.62956241)
\lineto(60.82913738,387.62174908)
\curveto(60.82913738,388.17383195)(61.05309552,388.65299821)(61.50101181,389.05924787)
\curveto(61.9489281,389.46549753)(62.5426776,389.66862235)(63.28226031,389.66862235)
\curveto(63.90205146,389.66862235)(64.56350924,389.61133074)(65.26663364,389.4967475)
\curveto(65.53225842,389.45508087)(65.69111245,389.40560174)(65.74319574,389.34831012)
\curveto(65.80048736,389.29101851)(65.82913317,389.21549774)(65.82913317,389.12174781)
\curveto(65.82913317,389.02799789)(65.7978832,388.94987296)(65.73538325,388.88737301)
\curveto(65.6728833,388.83008139)(65.58955004,388.80143558)(65.48538346,388.80143558)
\curveto(65.44371683,388.80143558)(65.37340439,388.80924808)(65.27444614,388.82487306)
\curveto(64.48798847,388.94466463)(63.82392652,389.00456041)(63.28226031,389.00456041)
\curveto(62.70934413,389.00456041)(62.26663617,388.86393553)(61.95413643,388.58268577)
\curveto(61.64684502,388.301436)(61.49319932,387.98112377)(61.49319932,387.62174908)
\lineto(61.49319932,386.62956241)
\lineto(64.56350924,386.62956241)
\curveto(64.7197591,386.62956241)(64.83173818,386.59831244)(64.89944645,386.53581249)
\curveto(64.96715473,386.47331254)(65.00100887,386.39258344)(65.00100887,386.29362519)
\curveto(65.00100887,386.19987527)(64.96715473,386.12175034)(64.89944645,386.05925039)
\curveto(64.83173818,385.99675044)(64.7197591,385.96550047)(64.56350924,385.96550047)
\closepath
}
}
{
\newrgbcolor{curcolor}{0 0 0}
\pscustom[linestyle=none,fillstyle=solid,fillcolor=curcolor]
{
\newpath
\moveto(75.89162365,375.48113429)
\lineto(67.16506849,375.48113429)
\curveto(67.00881862,375.48113429)(66.89683955,375.51238426)(66.82913127,375.57488421)
\curveto(66.761423,375.63217583)(66.72756886,375.71030076)(66.72756886,375.80925901)
\curveto(66.72756886,375.90821726)(66.761423,375.98894636)(66.82913127,376.05144631)
\curveto(66.89683955,376.10873793)(67.00881862,376.13738374)(67.16506849,376.13738374)
\lineto(75.89162365,376.13738374)
\curveto(76.05308185,376.13738374)(76.16506092,376.10873793)(76.22756087,376.05144631)
\curveto(76.29526915,375.98894636)(76.32912328,375.90821726)(76.32912328,375.80925901)
\curveto(76.32912328,375.71030076)(76.29526915,375.63217583)(76.22756087,375.57488421)
\curveto(76.16506092,375.51238426)(76.05308185,375.48113429)(75.89162365,375.48113429)
\closepath
}
}
{
\newrgbcolor{curcolor}{0 0 0}
\pscustom[linestyle=none,fillstyle=solid,fillcolor=curcolor]
{
\newpath
\moveto(78.96974501,389.66862235)
\lineto(78.96974501,385.62956325)
\curveto(79.38641132,386.08268787)(79.78484849,386.40039594)(80.1650565,386.58268745)
\curveto(80.55047284,386.77018729)(80.98015998,386.86393721)(81.45411792,386.86393721)
\curveto(81.96453416,386.86393721)(82.39682546,386.77279146)(82.75099183,386.59049994)
\curveto(83.11036652,386.41341676)(83.40984544,386.13737532)(83.64942857,385.76237564)
\curveto(83.8890117,385.39258428)(84.00880327,385.00195961)(84.00880327,384.59050163)
\lineto(84.00880327,380.52019255)
\lineto(84.74317765,380.52019255)
\curveto(84.90463585,380.52019255)(85.01661492,380.48894258)(85.07911487,380.42644263)
\curveto(85.14682315,380.36915101)(85.18067728,380.29102608)(85.18067728,380.19206782)
\curveto(85.18067728,380.0983179)(85.14682315,380.02019297)(85.07911487,379.95769302)
\curveto(85.01661492,379.89519307)(84.90463585,379.8639431)(84.74317765,379.8639431)
\lineto(82.61036695,379.8639431)
\curveto(82.44890875,379.8639431)(82.33432551,379.89519307)(82.26661723,379.95769302)
\curveto(82.19890896,380.02019297)(82.16505482,380.0983179)(82.16505482,380.19206782)
\curveto(82.16505482,380.29102608)(82.19890896,380.36915101)(82.26661723,380.42644263)
\curveto(82.33432551,380.48894258)(82.44890875,380.52019255)(82.61036695,380.52019255)
\lineto(83.34474133,380.52019255)
\lineto(83.34474133,384.54362667)
\curveto(83.34474133,385.0175846)(83.17286647,385.4134176)(82.82911676,385.73112567)
\curveto(82.49057538,386.04883373)(82.01661744,386.20768777)(81.40724296,386.20768777)
\curveto(80.92807669,386.20768777)(80.51922287,386.09050036)(80.18068149,385.85612556)
\curveto(79.93589003,385.68945903)(79.53224453,385.2962302)(78.96974501,384.67643905)
\lineto(78.96974501,380.52019255)
\lineto(79.71193188,380.52019255)
\curveto(79.86818175,380.52019255)(79.98016082,380.48894258)(80.0478691,380.42644263)
\curveto(80.11557738,380.36915101)(80.14943151,380.29102608)(80.14943151,380.19206782)
\curveto(80.14943151,380.0983179)(80.11557738,380.02019297)(80.0478691,379.95769302)
\curveto(79.98016082,379.89519307)(79.86818175,379.8639431)(79.71193188,379.8639431)
\lineto(77.57130868,379.8639431)
\curveto(77.41505881,379.8639431)(77.30307974,379.89519307)(77.23537147,379.95769302)
\curveto(77.16766319,380.02019297)(77.13380905,380.0983179)(77.13380905,380.19206782)
\curveto(77.13380905,380.29102608)(77.16766319,380.36915101)(77.23537147,380.42644263)
\curveto(77.30307974,380.48894258)(77.41505881,380.52019255)(77.57130868,380.52019255)
\lineto(78.31349556,380.52019255)
\lineto(78.31349556,389.00456041)
\lineto(77.4306838,389.00456041)
\curveto(77.27443393,389.00456041)(77.16245486,389.03581039)(77.09474658,389.09831033)
\curveto(77.02703831,389.16081028)(76.99318417,389.24153938)(76.99318417,389.34049763)
\curveto(76.99318417,389.43424755)(77.02703831,389.51237249)(77.09474658,389.57487243)
\curveto(77.16245486,389.63737238)(77.27443393,389.66862235)(77.4306838,389.66862235)
\closepath
}
}
{
\newrgbcolor{curcolor}{0 0 0}
\pscustom[linestyle=none,fillstyle=solid,fillcolor=curcolor]
{
\newpath
\moveto(93.94629386,389.66862235)
\lineto(93.94629386,380.52019255)
\lineto(94.82129312,380.52019255)
\curveto(94.98275132,380.52019255)(95.09733455,380.48894258)(95.16504283,380.42644263)
\curveto(95.23275111,380.36915101)(95.26660524,380.29102608)(95.26660524,380.19206782)
\curveto(95.26660524,380.0983179)(95.23275111,380.02019297)(95.16504283,379.95769302)
\curveto(95.09733455,379.89519307)(94.98275132,379.8639431)(94.82129312,379.8639431)
\lineto(93.28223191,379.8639431)
\lineto(93.28223191,381.1686295)
\curveto(92.52181589,380.12696371)(91.55567087,379.60613082)(90.38379685,379.60613082)
\curveto(89.79004735,379.60613082)(89.21973533,379.76238069)(88.67286079,380.07488042)
\curveto(88.13119458,380.39258849)(87.70150744,380.84310894)(87.38379937,381.42644179)
\curveto(87.07129964,382.00977463)(86.91504977,382.61133662)(86.91504977,383.23112777)
\curveto(86.91504977,383.85612724)(87.07129964,384.45768924)(87.38379937,385.03581375)
\curveto(87.70150744,385.61914659)(88.13119458,386.06966705)(88.67286079,386.38737511)
\curveto(89.21973533,386.70508318)(89.79265152,386.86393721)(90.39160934,386.86393721)
\curveto(91.53744171,386.86393721)(92.50098257,386.34310432)(93.28223191,385.30143853)
\lineto(93.28223191,389.00456041)
\lineto(92.40723265,389.00456041)
\curveto(92.24577445,389.00456041)(92.13119122,389.03581039)(92.06348294,389.09831033)
\curveto(91.99577466,389.16081028)(91.96192052,389.24153938)(91.96192052,389.34049763)
\curveto(91.96192052,389.43424755)(91.99577466,389.51237249)(92.06348294,389.57487243)
\curveto(92.13119122,389.63737238)(92.24577445,389.66862235)(92.40723265,389.66862235)
\closepath
\moveto(93.28223191,383.23112777)
\curveto(93.28223191,384.0644604)(93.00098215,384.76758481)(92.43848262,385.34050099)
\curveto(91.8759831,385.91341718)(91.20671283,386.19987527)(90.43067181,386.19987527)
\curveto(89.64942247,386.19987527)(88.97754803,385.91341718)(88.41504851,385.34050099)
\curveto(87.85254898,384.76758481)(87.57129922,384.0644604)(87.57129922,383.23112777)
\curveto(87.57129922,382.40300347)(87.85254898,381.69987906)(88.41504851,381.12175454)
\curveto(88.97754803,380.54883836)(89.64942247,380.26238027)(90.43067181,380.26238027)
\curveto(91.20671283,380.26238027)(91.8759831,380.54883836)(92.43848262,381.12175454)
\curveto(93.00098215,381.69987906)(93.28223191,382.40300347)(93.28223191,383.23112777)
\closepath
}
}
{
\newrgbcolor{curcolor}{0 0 0}
\pscustom[linestyle=none,fillstyle=solid,fillcolor=curcolor]
{
\newpath
\moveto(99.54785059,386.62956241)
\lineto(99.54785059,384.9733138)
\curveto(100.40201654,385.74414649)(101.04003683,386.23893774)(101.46191148,386.45768756)
\curveto(101.88899445,386.6816457)(102.28222329,386.79362477)(102.64159799,386.79362477)
\curveto(103.03222266,386.79362477)(103.39420152,386.66081238)(103.72753457,386.39518761)
\curveto(104.06607595,386.13477116)(104.23534665,385.93685466)(104.23534665,385.80143811)
\curveto(104.23534665,385.70247986)(104.20149251,385.61914659)(104.13378423,385.55143832)
\curveto(104.07128428,385.48893837)(103.99055519,385.4576884)(103.89159693,385.4576884)
\curveto(103.83951365,385.4576884)(103.79524285,385.46550089)(103.75878455,385.48112588)
\curveto(103.72232624,385.50195919)(103.65461797,385.56185498)(103.55565972,385.66081323)
\curveto(103.3733682,385.84310474)(103.21451417,385.96810463)(103.07909762,386.03581291)
\curveto(102.94368107,386.10352119)(102.81086868,386.13737532)(102.68066045,386.13737532)
\curveto(102.39420236,386.13737532)(102.04784849,386.02279209)(101.64159883,385.79362561)
\curveto(101.2405575,385.56445914)(100.54264142,385.00456378)(99.54785059,384.11393953)
\lineto(99.54785059,380.52019255)
\lineto(102.45409814,380.52019255)
\curveto(102.61555634,380.52019255)(102.73013958,380.48894258)(102.79784785,380.42644263)
\curveto(102.86555613,380.36915101)(102.89941027,380.29102608)(102.89941027,380.19206782)
\curveto(102.89941027,380.0983179)(102.86555613,380.02019297)(102.79784785,379.95769302)
\curveto(102.73013958,379.89519307)(102.61555634,379.8639431)(102.45409814,379.8639431)
\lineto(97.30566497,379.8639431)
\curveto(97.14941511,379.8639431)(97.03743603,379.89258891)(96.96972776,379.94988053)
\curveto(96.90201948,380.01238048)(96.86816534,380.09050541)(96.86816534,380.18425533)
\curveto(96.86816534,380.27279692)(96.89941532,380.34571353)(96.96191526,380.40300515)
\curveto(97.02962354,380.46550509)(97.14420678,380.49675507)(97.30566497,380.49675507)
\lineto(98.89160114,380.49675507)
\lineto(98.89160114,385.96550047)
\lineto(97.68066466,385.96550047)
\curveto(97.52441479,385.96550047)(97.41243572,385.99675044)(97.34472744,386.05925039)
\curveto(97.27701916,386.12175034)(97.24316503,386.20247944)(97.24316503,386.30143769)
\curveto(97.24316503,386.39518761)(97.274415,386.47331254)(97.33691495,386.53581249)
\curveto(97.40462322,386.59831244)(97.51920646,386.62956241)(97.68066466,386.62956241)
\closepath
}
}
{
\newrgbcolor{curcolor}{0 0 0}
\pscustom[linestyle=none,fillstyle=solid,fillcolor=curcolor]
{
\newpath
\moveto(114.29783713,375.48113429)
\lineto(105.57128197,375.48113429)
\curveto(105.4150321,375.48113429)(105.30305303,375.51238426)(105.23534475,375.57488421)
\curveto(105.16763647,375.63217583)(105.13378233,375.71030076)(105.13378233,375.80925901)
\curveto(105.13378233,375.90821726)(105.16763647,375.98894636)(105.23534475,376.05144631)
\curveto(105.30305303,376.10873793)(105.4150321,376.13738374)(105.57128197,376.13738374)
\lineto(114.29783713,376.13738374)
\curveto(114.45929532,376.13738374)(114.5712744,376.10873793)(114.63377434,376.05144631)
\curveto(114.70148262,375.98894636)(114.73533676,375.90821726)(114.73533676,375.80925901)
\curveto(114.73533676,375.71030076)(114.70148262,375.63217583)(114.63377434,375.57488421)
\curveto(114.5712744,375.51238426)(114.45929532,375.48113429)(114.29783713,375.48113429)
\closepath
}
}
{
\newrgbcolor{curcolor}{0 0 0}
\pscustom[linestyle=none,fillstyle=solid,fillcolor=curcolor]
{
\newpath
\moveto(118.18846501,386.62956241)
\lineto(121.75096201,386.62956241)
\curveto(121.90721188,386.62956241)(122.01919095,386.59831244)(122.08689923,386.53581249)
\curveto(122.15460751,386.47331254)(122.18846164,386.39258344)(122.18846164,386.29362519)
\curveto(122.18846164,386.19987527)(122.15460751,386.12175034)(122.08689923,386.05925039)
\curveto(122.01919095,385.99675044)(121.90721188,385.96550047)(121.75096201,385.96550047)
\lineto(118.18846501,385.96550047)
\lineto(118.18846501,381.59831664)
\curveto(118.18846501,381.21810863)(118.33950655,380.90040056)(118.64158963,380.64519244)
\curveto(118.94888103,380.38998433)(119.39679732,380.26238027)(119.9853385,380.26238027)
\curveto(120.42804646,380.26238027)(120.90721272,380.32748438)(121.42283729,380.4576926)
\curveto(121.93846185,380.59310915)(122.33950318,380.74415069)(122.62596128,380.91081722)
\curveto(122.73012785,380.9785255)(122.81606528,381.01237964)(122.88377356,381.01237964)
\curveto(122.96710682,381.01237964)(123.04002343,380.9785255)(123.10252337,380.91081722)
\curveto(123.16502332,380.84831727)(123.1962733,380.7727965)(123.1962733,380.68425491)
\curveto(123.1962733,380.60612998)(123.16241916,380.53321337)(123.09471088,380.46550509)
\curveto(122.92804435,380.29363024)(122.5217947,380.1061304)(121.87596191,379.90300557)
\curveto(121.23533745,379.70508907)(120.62075463,379.60613082)(120.03221346,379.60613082)
\curveto(119.2665891,379.60613082)(118.65721461,379.78581817)(118.20408999,380.14519286)
\curveto(117.75096538,380.50456756)(117.52440307,380.98894215)(117.52440307,381.59831664)
\lineto(117.52440307,385.96550047)
\lineto(116.31346658,385.96550047)
\curveto(116.15721672,385.96550047)(116.04523764,385.99675044)(115.97752937,386.05925039)
\curveto(115.90982109,386.12175034)(115.87596695,386.20247944)(115.87596695,386.30143769)
\curveto(115.87596695,386.39518761)(115.90982109,386.47331254)(115.97752937,386.53581249)
\curveto(116.04523764,386.59831244)(116.15721672,386.62956241)(116.31346658,386.62956241)
\lineto(117.52440307,386.62956241)
\lineto(117.52440307,388.56706078)
\curveto(117.52440307,388.72331065)(117.55565304,388.83528972)(117.61815299,388.902998)
\curveto(117.68065293,388.97070628)(117.75877787,389.00456041)(117.85252779,389.00456041)
\curveto(117.95148604,389.00456041)(118.03221514,388.97070628)(118.09471509,388.902998)
\curveto(118.15721503,388.83528972)(118.18846501,388.72331065)(118.18846501,388.56706078)
\closepath
}
}
{
\newrgbcolor{curcolor}{0 0 0}
\pscustom[linestyle=none,fillstyle=solid,fillcolor=curcolor]
{
\newpath
\moveto(4.89218298,360.85278083)
\lineto(4.89218298,357.09496863)
\lineto(7.05718232,357.09496863)
\curveto(7.2075295,357.09496863)(7.31527831,357.06371863)(7.38042875,357.00121864)
\curveto(7.44557919,356.94392698)(7.47815441,356.86580198)(7.47815441,356.76684366)
\curveto(7.47815441,356.67309366)(7.44557919,356.59496867)(7.38042875,356.53246868)
\curveto(7.31527831,356.46996868)(7.2075295,356.43871868)(7.05718232,356.43871868)
\lineto(3.4112633,356.43871868)
\curveto(3.26091612,356.43871868)(3.15316731,356.46996868)(3.08801687,356.53246868)
\curveto(3.02286642,356.59496867)(2.9902912,356.67309366)(2.9902912,356.76684366)
\curveto(2.9902912,356.86580198)(3.02286642,356.94392698)(3.08801687,357.00121864)
\curveto(3.15316731,357.06371863)(3.26091612,357.09496863)(3.4112633,357.09496863)
\lineto(4.26072484,357.09496863)
\lineto(4.26072484,364.92309301)
\lineto(3.4112633,364.92309301)
\curveto(3.26091612,364.92309301)(3.15316731,364.95173884)(3.08801687,365.0090305)
\curveto(3.02286642,365.0715305)(2.9902912,365.15225966)(2.9902912,365.25121798)
\curveto(2.9902912,365.35017631)(3.027878,365.43350963)(3.10305159,365.50121796)
\curveto(3.15817888,365.55330129)(3.26091612,365.57934295)(3.4112633,365.57934295)
\lineto(10.3648202,365.57934295)
\lineto(10.3648202,363.32153063)
\curveto(10.3648202,363.16528065)(10.33475076,363.05330149)(10.27461189,362.98559316)
\curveto(10.21447302,362.91788483)(10.13929943,362.88403067)(10.04909113,362.88403067)
\curveto(9.95387125,362.88403067)(9.87619188,362.91788483)(9.81605301,362.98559316)
\curveto(9.75591413,363.05330149)(9.7258447,363.16528065)(9.7258447,363.32153063)
\lineto(9.7258447,364.92309301)
\lineto(4.89218298,364.92309301)
\lineto(4.89218298,361.50903078)
\lineto(7.15490799,361.50903078)
\lineto(7.15490799,362.25121822)
\curveto(7.15490799,362.40746821)(7.18497742,362.51944737)(7.24511629,362.58715569)
\curveto(7.30525516,362.65486402)(7.38293454,362.68871819)(7.47815441,362.68871819)
\curveto(7.56836272,362.68871819)(7.64353631,362.65486402)(7.70367518,362.58715569)
\curveto(7.76381405,362.51944737)(7.79388348,362.40746821)(7.79388348,362.25121822)
\lineto(7.79388348,360.11059339)
\curveto(7.79388348,359.9543434)(7.76381405,359.84236425)(7.70367518,359.77465592)
\curveto(7.64353631,359.70694759)(7.56836272,359.67309343)(7.47815441,359.67309343)
\curveto(7.38293454,359.67309343)(7.30525516,359.70694759)(7.24511629,359.77465592)
\curveto(7.18497742,359.84236425)(7.15490799,359.9543434)(7.15490799,360.11059339)
\lineto(7.15490799,360.85278083)
\closepath
}
}
{
\newrgbcolor{curcolor}{0 0 0}
\pscustom[linestyle=none,fillstyle=solid,fillcolor=curcolor]
{
\newpath
\moveto(15.96525295,366.57153038)
\lineto(15.96525295,364.87621801)
\lineto(15.03310046,364.87621801)
\lineto(15.03310046,366.57153038)
\closepath
\moveto(15.98780503,363.20434314)
\lineto(15.98780503,357.09496863)
\lineto(18.46853344,357.09496863)
\curveto(18.62389218,357.09496863)(18.73414678,357.06371863)(18.79929722,357.00121864)
\curveto(18.86444767,356.94392698)(18.89702289,356.86580198)(18.89702289,356.76684366)
\curveto(18.89702289,356.67309366)(18.86444767,356.59496867)(18.79929722,356.53246868)
\curveto(18.73414678,356.46996868)(18.62389218,356.43871868)(18.46853344,356.43871868)
\lineto(12.87561848,356.43871868)
\curveto(12.7252713,356.43871868)(12.61752249,356.46996868)(12.55237205,356.53246868)
\curveto(12.48722161,356.59496867)(12.45464639,356.67309366)(12.45464639,356.76684366)
\curveto(12.45464639,356.86580198)(12.48722161,356.94392698)(12.55237205,357.00121864)
\curveto(12.61752249,357.06371863)(12.7252713,357.09496863)(12.87561848,357.09496863)
\lineto(15.35634689,357.09496863)
\lineto(15.35634689,362.5402807)
\lineto(13.51459398,362.5402807)
\curveto(13.3642468,362.5402807)(13.25399221,362.57153069)(13.18383019,362.63403069)
\curveto(13.11867975,362.69653068)(13.08610453,362.77465568)(13.08610453,362.86840567)
\curveto(13.08610453,362.967364)(13.11867975,363.04809316)(13.18383019,363.11059315)
\curveto(13.24898063,363.17309315)(13.35923523,363.20434314)(13.51459398,363.20434314)
\closepath
}
}
{
\newrgbcolor{curcolor}{0 0 0}
\pscustom[linestyle=none,fillstyle=solid,fillcolor=curcolor]
{
\newpath
\moveto(28.30875656,359.66528093)
\lineto(21.87389742,359.66528093)
\curveto(21.98415201,358.81632266)(22.32493894,358.13142688)(22.89625821,357.61059359)
\curveto(23.47258906,357.09496863)(24.18423236,356.83715615)(25.03118812,356.83715615)
\curveto(25.50227594,356.83715615)(25.99591583,356.91788531)(26.5121078,357.07934363)
\curveto(27.02829978,357.24080195)(27.44927187,357.4543436)(27.77502408,357.71996858)
\curveto(27.87024396,357.79809357)(27.95293491,357.83715607)(28.02309692,357.83715607)
\curveto(28.10328209,357.83715607)(28.1734441,357.80330191)(28.23358297,357.73559358)
\curveto(28.29372184,357.67309358)(28.32379128,357.59757276)(28.32379128,357.5090311)
\curveto(28.32379128,357.42048944)(28.2836987,357.33455195)(28.20351354,357.25121862)
\curveto(27.96295805,356.99080197)(27.5344686,356.74601033)(26.91804518,356.51684368)
\curveto(26.30663333,356.29288536)(25.67768098,356.1809062)(25.03118812,356.1809062)
\curveto(23.94868845,356.1809062)(23.0440996,356.54809367)(22.31742159,357.28246862)
\curveto(21.59575514,358.02205189)(21.23492192,358.91528099)(21.23492192,359.9621559)
\curveto(21.23492192,360.91528083)(21.57320306,361.73298909)(22.24976536,362.41528071)
\curveto(22.93133922,363.09757232)(23.77328341,363.43871813)(24.77559792,363.43871813)
\curveto(25.80798186,363.43871813)(26.65744341,363.08715565)(27.32398255,362.38403071)
\curveto(27.9905217,361.6861141)(28.3187797,360.77986417)(28.30875656,359.66528093)
\closepath
\moveto(27.66978106,360.32934337)
\curveto(27.54449175,361.05330165)(27.21372796,361.64184327)(26.6774897,362.09496823)
\curveto(26.14626301,362.5480932)(25.51229908,362.77465568)(24.77559792,362.77465568)
\curveto(24.03889675,362.77465568)(23.40493283,362.55069736)(22.87370614,362.10278073)
\curveto(22.34247945,361.6548641)(22.01171566,361.06371831)(21.88141477,360.32934337)
\closepath
}
}
{
\newrgbcolor{curcolor}{0 0 0}
\pscustom[linestyle=none,fillstyle=solid,fillcolor=curcolor]
{
\newpath
\moveto(34.45044003,366.2434054)
\lineto(34.45044003,357.09496863)
\lineto(36.93116844,357.09496863)
\curveto(37.08652719,357.09496863)(37.19678178,357.06371863)(37.26193223,357.00121864)
\curveto(37.32708267,356.94392698)(37.35965789,356.86580198)(37.35965789,356.76684366)
\curveto(37.35965789,356.67309366)(37.32708267,356.59496867)(37.26193223,356.53246868)
\curveto(37.19678178,356.46996868)(37.08652719,356.43871868)(36.93116844,356.43871868)
\lineto(31.33825348,356.43871868)
\curveto(31.18790631,356.43871868)(31.0801575,356.46996868)(31.01500705,356.53246868)
\curveto(30.94985661,356.59496867)(30.91728139,356.67309366)(30.91728139,356.76684366)
\curveto(30.91728139,356.86580198)(30.94985661,356.94392698)(31.01500705,357.00121864)
\curveto(31.0801575,357.06371863)(31.18790631,357.09496863)(31.33825348,357.09496863)
\lineto(33.81898189,357.09496863)
\lineto(33.81898189,365.57934295)
\lineto(31.99978106,365.57934295)
\curveto(31.84943388,365.57934295)(31.73917929,365.61059295)(31.66901727,365.67309295)
\curveto(31.60386683,365.73559294)(31.57129161,365.8163221)(31.57129161,365.91528043)
\curveto(31.57129161,366.00903042)(31.60386683,366.08715541)(31.66901727,366.14965541)
\curveto(31.73416771,366.2121554)(31.84442231,366.2434054)(31.99978106,366.2434054)
\closepath
}
}
{
\newrgbcolor{curcolor}{0 0 0}
\pscustom[linestyle=none,fillstyle=solid,fillcolor=curcolor]
{
\newpath
\moveto(46.46317897,366.2434054)
\lineto(46.46317897,357.09496863)
\lineto(47.30512315,357.09496863)
\curveto(47.4604819,357.09496863)(47.5707365,357.06371863)(47.63588694,357.00121864)
\curveto(47.70103738,356.94392698)(47.73361261,356.86580198)(47.73361261,356.76684366)
\curveto(47.73361261,356.67309366)(47.70103738,356.59496867)(47.63588694,356.53246868)
\curveto(47.5707365,356.46996868)(47.4604819,356.43871868)(47.30512315,356.43871868)
\lineto(45.82420347,356.43871868)
\lineto(45.82420347,357.74340608)
\curveto(45.09251388,356.7017395)(44.16286717,356.1809062)(43.03526335,356.1809062)
\curveto(42.46394408,356.1809062)(41.91517689,356.33715619)(41.38896177,356.64965617)
\curveto(40.86775822,356.96736447)(40.45430349,357.41788527)(40.14859756,358.00121856)
\curveto(39.84790321,358.58455185)(39.69755604,359.1861143)(39.69755604,359.80590591)
\curveto(39.69755604,360.43090587)(39.84790321,361.03246832)(40.14859756,361.61059327)
\curveto(40.45430349,362.19392656)(40.86775822,362.64444736)(41.38896177,362.96215566)
\curveto(41.91517689,363.27986397)(42.46644987,363.43871813)(43.04278071,363.43871813)
\curveto(44.14532667,363.43871813)(45.07246759,362.91788483)(45.82420347,361.87621825)
\lineto(45.82420347,365.57934295)
\lineto(44.98225928,365.57934295)
\curveto(44.82690053,365.57934295)(44.71664594,365.61059295)(44.65149549,365.67309295)
\curveto(44.58634505,365.73559294)(44.55376983,365.8163221)(44.55376983,365.91528043)
\curveto(44.55376983,366.00903042)(44.58634505,366.08715541)(44.65149549,366.14965541)
\curveto(44.71664594,366.2121554)(44.82690053,366.2434054)(44.98225928,366.2434054)
\closepath
\moveto(45.82420347,359.80590591)
\curveto(45.82420347,360.63923918)(45.55357855,361.34236413)(45.01232872,361.91528075)
\curveto(44.47107888,362.48819737)(43.82709181,362.77465568)(43.0803675,362.77465568)
\curveto(42.32863162,362.77465568)(41.68213876,362.48819737)(41.14088893,361.91528075)
\curveto(40.59963909,361.34236413)(40.32901418,360.63923918)(40.32901418,359.80590591)
\curveto(40.32901418,358.97778098)(40.59963909,358.27465604)(41.14088893,357.69653108)
\curveto(41.68213876,357.12361446)(42.32863162,356.83715615)(43.0803675,356.83715615)
\curveto(43.82709181,356.83715615)(44.47107888,357.12361446)(45.01232872,357.69653108)
\curveto(45.55357855,358.27465604)(45.82420347,358.97778098)(45.82420347,359.80590591)
\closepath
}
}
{
\newrgbcolor{curcolor}{0 0 0}
\pscustom[linestyle=none,fillstyle=solid,fillcolor=curcolor]
{
\newpath
\moveto(54.69468642,362.77465568)
\curveto(54.69468642,362.92569733)(54.72475586,363.03507232)(54.78489473,363.10278065)
\curveto(54.8450336,363.17048898)(54.92020718,363.20434314)(55.01041549,363.20434314)
\curveto(55.10563537,363.20434314)(55.18331474,363.17048898)(55.24345361,363.10278065)
\curveto(55.30359248,363.03507232)(55.33366192,362.920489)(55.33366192,362.75903068)
\lineto(55.33366192,361.63403077)
\curveto(55.33366192,361.47778078)(55.30359248,361.36580162)(55.24345361,361.2980933)
\curveto(55.18331474,361.23038497)(55.10563537,361.1965308)(55.01041549,361.1965308)
\curveto(54.92521876,361.1965308)(54.85255096,361.22517664)(54.79241209,361.2824683)
\curveto(54.73728479,361.33975996)(54.70470957,361.43350995)(54.69468642,361.56371827)
\curveto(54.66461699,361.87621825)(54.50925824,362.13403073)(54.22861017,362.33715571)
\curveto(53.81766123,362.62882236)(53.27390561,362.77465568)(52.59734331,362.77465568)
\curveto(51.89071158,362.77465568)(51.34194439,362.62621819)(50.95104173,362.32934321)
\curveto(50.65535895,362.1053849)(50.50751756,361.85538492)(50.50751756,361.57934327)
\curveto(50.50751756,361.2668433)(50.6829226,361.00642665)(51.03373268,360.79809334)
\curveto(51.27428816,360.65226001)(51.73034126,360.54028086)(52.40189198,360.46215586)
\curveto(53.27891718,360.36319754)(53.88782324,360.25121838)(54.22861017,360.12621839)
\curveto(54.71473271,359.94392674)(55.07556593,359.69132259)(55.31110984,359.36840595)
\curveto(55.55166533,359.04548931)(55.67194307,358.696531)(55.67194307,358.32153103)
\curveto(55.67194307,357.76423941)(55.41384708,357.26684362)(54.89765511,356.82934365)
\curveto(54.38146314,356.39705202)(53.62471568,356.1809062)(52.62741275,356.1809062)
\curveto(51.63010981,356.1809062)(50.81322349,356.44392702)(50.17675377,356.96996864)
\curveto(50.17675377,356.79288532)(50.16673063,356.678302)(50.14668434,356.62621867)
\curveto(50.12663805,356.57413534)(50.08905125,356.52986451)(50.03392396,356.49340618)
\curveto(49.98380823,356.45694785)(49.92617515,356.43871868)(49.8610247,356.43871868)
\curveto(49.7708164,356.43871868)(49.69564281,356.47257285)(49.63550394,356.54028118)
\curveto(49.57536507,356.6079895)(49.54529563,356.71996866)(49.54529563,356.87621865)
\lineto(49.54529563,358.22778104)
\curveto(49.54529563,358.38403103)(49.57285928,358.49601019)(49.62798658,358.56371851)
\curveto(49.68812545,358.63142684)(49.76580483,358.66528101)(49.8610247,358.66528101)
\curveto(49.95123301,358.66528101)(50.0264066,358.63142684)(50.08654547,358.56371851)
\curveto(50.15169591,358.50121852)(50.18427113,358.41528103)(50.18427113,358.30590603)
\curveto(50.18427113,358.06632272)(50.24190422,357.8658019)(50.35717039,357.70434358)
\curveto(50.53257542,357.4543436)(50.8107177,357.24601029)(51.19159721,357.07934363)
\curveto(51.5774883,356.91788531)(52.04857612,356.83715615)(52.60486067,356.83715615)
\curveto(53.42675857,356.83715615)(54.03817042,356.99601031)(54.43909622,357.31371861)
\curveto(54.84002202,357.63142692)(55.04048493,357.96736439)(55.04048493,358.32153103)
\curveto(55.04048493,358.727781)(54.83751624,359.05330181)(54.43157886,359.29809346)
\curveto(54.02062991,359.5428851)(53.421747,359.70694759)(52.63493011,359.79028092)
\curveto(51.85312479,359.87361424)(51.29182866,359.98298923)(50.95104173,360.11840589)
\curveto(50.6102548,360.25382255)(50.34464145,360.45694753)(50.1542017,360.72778084)
\curveto(49.96376194,360.99861415)(49.86854206,361.2902808)(49.86854206,361.60278077)
\curveto(49.86854206,362.16528073)(50.13415541,362.61059319)(50.6653821,362.93871817)
\curveto(51.19660879,363.27205147)(51.83057271,363.43871813)(52.56727388,363.43871813)
\curveto(53.4392875,363.43871813)(54.14842501,363.21736398)(54.69468642,362.77465568)
\closepath
}
}
{
\newrgbcolor{curcolor}{0 0 0}
\pscustom[linestyle=none,fillstyle=solid,fillcolor=curcolor]
{
\newpath
\moveto(73.72112068,362.5402807)
\lineto(73.72112068,362.75903068)
\curveto(73.72112068,362.920489)(73.75119012,363.03507232)(73.81132899,363.10278065)
\curveto(73.87146786,363.17048898)(73.94664145,363.20434314)(74.03684975,363.20434314)
\curveto(74.13206963,363.20434314)(74.20974901,363.17048898)(74.26988788,363.10278065)
\curveto(74.33002675,363.03507232)(74.36009618,362.920489)(74.36009618,362.75903068)
\lineto(74.36009618,361.2746558)
\curveto(74.35508461,361.11319748)(74.32250939,360.99861415)(74.26237052,360.93090583)
\curveto(74.20724322,360.8631975)(74.13206963,360.82934333)(74.03684975,360.82934333)
\curveto(73.95165302,360.82934333)(73.87898522,360.85798916)(73.81884635,360.91528083)
\curveto(73.76371905,360.97778082)(73.73114383,361.07934331)(73.72112068,361.2199683)
\curveto(73.69105125,361.58975994)(73.45550734,361.94132241)(73.01448895,362.27465572)
\curveto(72.57848214,362.60798902)(71.98962237,362.77465568)(71.24790963,362.77465568)
\curveto(70.31074557,362.77465568)(69.59910227,362.4699682)(69.11297973,361.86059325)
\curveto(68.62685719,361.2512183)(68.38379593,360.55330169)(68.38379593,359.76684342)
\curveto(68.38379593,358.91788515)(68.65191506,358.21736437)(69.18815332,357.66528109)
\curveto(69.72439158,357.1131978)(70.41849438,356.83715615)(71.27046171,356.83715615)
\curveto(71.76159582,356.83715615)(72.26024729,356.93090614)(72.76641611,357.11840613)
\curveto(73.27759651,357.30590611)(73.73866119,357.60798942)(74.14961014,358.02465606)
\curveto(74.25485316,358.12882272)(74.34756725,358.18090604)(74.42775241,358.18090604)
\curveto(74.51294914,358.18090604)(74.58311116,358.14965605)(74.63823846,358.08715605)
\curveto(74.69837733,358.02986439)(74.72844676,357.95694773)(74.72844676,357.86840607)
\curveto(74.72844676,357.64444775)(74.47536235,357.36059361)(73.96919352,357.01684364)
\curveto(73.1523072,356.45955201)(72.24270678,356.1809062)(71.24039227,356.1809062)
\curveto(70.22304305,356.1809062)(69.38611043,356.51684368)(68.72959443,357.18871862)
\curveto(68.07809,357.8658019)(67.75233779,358.72257267)(67.75233779,359.75903092)
\curveto(67.75233779,360.8163225)(68.08560736,361.6939266)(68.75214651,362.39184321)
\curveto(69.42369723,363.08975982)(70.2681472,363.43871813)(71.28549643,363.43871813)
\curveto(72.25272993,363.43871813)(73.06460468,363.13923898)(73.72112068,362.5402807)
\closepath
}
}
{
\newrgbcolor{curcolor}{0 0 0}
\pscustom[linestyle=none,fillstyle=solid,fillcolor=curcolor]
{
\newpath
\moveto(83.85452345,359.80590591)
\curveto(83.85452345,358.80590599)(83.50872495,357.9517394)(82.81712794,357.24340612)
\curveto(82.1305425,356.53507284)(81.30112724,356.1809062)(80.32888217,356.1809062)
\curveto(79.34661395,356.1809062)(78.51218712,356.53507284)(77.82560168,357.24340612)
\curveto(77.13901625,357.95694773)(76.79572353,358.81111433)(76.79572353,359.80590591)
\curveto(76.79572353,360.80590584)(77.13901625,361.66007243)(77.82560168,362.36840571)
\curveto(78.51218712,363.08194732)(79.34661395,363.43871813)(80.32888217,363.43871813)
\curveto(81.30112724,363.43871813)(82.1305425,363.08455149)(82.81712794,362.37621821)
\curveto(83.50872495,361.66788493)(83.85452345,360.81111417)(83.85452345,359.80590591)
\closepath
\moveto(83.21554795,359.80590591)
\curveto(83.21554795,360.62882252)(82.9323941,361.32934329)(82.36608641,361.90746825)
\curveto(81.80479028,362.4855932)(81.12321642,362.77465568)(80.32136481,362.77465568)
\curveto(79.5195132,362.77465568)(78.83543355,362.48298903)(78.26912585,361.89965575)
\curveto(77.70782973,361.32153079)(77.42718167,360.62361418)(77.42718167,359.80590591)
\curveto(77.42718167,358.99340598)(77.70782973,358.29548937)(78.26912585,357.71215608)
\curveto(78.83543355,357.12882279)(79.5195132,356.83715615)(80.32136481,356.83715615)
\curveto(81.12321642,356.83715615)(81.80479028,357.12621863)(82.36608641,357.70434358)
\curveto(82.9323941,358.28767687)(83.21554795,358.98819765)(83.21554795,359.80590591)
\closepath
}
}
{
\newrgbcolor{curcolor}{0 0 0}
\pscustom[linestyle=none,fillstyle=solid,fillcolor=curcolor]
{
\newpath
\moveto(86.68857081,363.20434314)
\lineto(86.68857081,362.5402807)
\curveto(87.22480908,363.13923898)(87.76355312,363.43871813)(88.30480296,363.43871813)
\curveto(88.63055517,363.43871813)(88.91621481,363.3475723)(89.16178186,363.16528065)
\curveto(89.40734892,362.98819733)(89.61282339,362.71736402)(89.77820529,362.35278071)
\curveto(90.05885335,362.71736402)(90.3420072,362.98819733)(90.62766683,363.16528065)
\curveto(90.91833804,363.3475723)(91.20900925,363.43871813)(91.49968045,363.43871813)
\curveto(91.95573355,363.43871813)(92.31907256,363.2850723)(92.58969748,362.97778066)
\curveto(92.94551913,362.58194736)(93.12342996,362.14965573)(93.12342996,361.68090577)
\lineto(93.12342996,357.09496863)
\lineto(93.65716243,357.09496863)
\curveto(93.80750961,357.09496863)(93.91525842,357.06371863)(93.98040886,357.00121864)
\curveto(94.0455593,356.94392698)(94.07813453,356.86580198)(94.07813453,356.76684366)
\curveto(94.07813453,356.67309366)(94.0455593,356.59496867)(93.98040886,356.53246868)
\curveto(93.91525842,356.46996868)(93.80750961,356.43871868)(93.65716243,356.43871868)
\lineto(92.49197182,356.43871868)
\lineto(92.49197182,361.61840577)
\curveto(92.49197182,361.95173908)(92.39424615,362.22778072)(92.19879482,362.4465307)
\curveto(92.00334349,362.66528069)(91.77782273,362.77465568)(91.52223253,362.77465568)
\curveto(91.29170019,362.77465568)(91.04863892,362.68350985)(90.79304872,362.5012182)
\curveto(90.53745853,362.32413488)(90.24678732,361.97257241)(89.9210351,361.44653078)
\lineto(89.9210351,357.09496863)
\lineto(90.44725022,357.09496863)
\curveto(90.5975974,357.09496863)(90.70534621,357.06371863)(90.77049665,357.00121864)
\curveto(90.83564709,356.94392698)(90.86822231,356.86580198)(90.86822231,356.76684366)
\curveto(90.86822231,356.67309366)(90.83564709,356.59496867)(90.77049665,356.53246868)
\curveto(90.70534621,356.46996868)(90.5975974,356.43871868)(90.44725022,356.43871868)
\lineto(89.2820596,356.43871868)
\lineto(89.2820596,361.57153077)
\curveto(89.2820596,361.92048908)(89.18182815,362.20694739)(88.98136525,362.43090571)
\curveto(88.78591392,362.66007235)(88.56540473,362.77465568)(88.31983768,362.77465568)
\curveto(88.09431691,362.77465568)(87.87130193,362.69913485)(87.65079274,362.5480932)
\curveto(87.34508682,362.33455155)(87.02434617,361.96736408)(86.68857081,361.44653078)
\lineto(86.68857081,357.09496863)
\lineto(87.22230329,357.09496863)
\curveto(87.37265047,357.09496863)(87.48039927,357.06371863)(87.54554972,357.00121864)
\curveto(87.61070016,356.94392698)(87.64327538,356.86580198)(87.64327538,356.76684366)
\curveto(87.64327538,356.67309366)(87.61070016,356.59496867)(87.54554972,356.53246868)
\curveto(87.48039927,356.46996868)(87.37265047,356.43871868)(87.22230329,356.43871868)
\lineto(85.5233802,356.43871868)
\curveto(85.37303302,356.43871868)(85.26528421,356.46996868)(85.20013377,356.53246868)
\curveto(85.13498333,356.59496867)(85.1024081,356.67309366)(85.1024081,356.76684366)
\curveto(85.1024081,356.86580198)(85.13498333,356.94392698)(85.20013377,357.00121864)
\curveto(85.26528421,357.06371863)(85.37303302,357.09496863)(85.5233802,357.09496863)
\lineto(86.05711267,357.09496863)
\lineto(86.05711267,362.5402807)
\lineto(85.5233802,362.5402807)
\curveto(85.37303302,362.5402807)(85.26528421,362.57153069)(85.20013377,362.63403069)
\curveto(85.13498333,362.69653068)(85.1024081,362.77725984)(85.1024081,362.87621817)
\curveto(85.1024081,362.96996816)(85.13498333,363.04809316)(85.20013377,363.11059315)
\curveto(85.26528421,363.17309315)(85.37303302,363.20434314)(85.5233802,363.20434314)
\closepath
}
}
{
\newrgbcolor{curcolor}{0 0 0}
\pscustom[linestyle=none,fillstyle=solid,fillcolor=curcolor]
{
\newpath
\moveto(95.92740081,363.20434314)
\lineto(95.92740081,362.5402807)
\curveto(96.46363907,363.13923898)(97.00238312,363.43871813)(97.54363296,363.43871813)
\curveto(97.86938517,363.43871813)(98.15504481,363.3475723)(98.40061186,363.16528065)
\curveto(98.64617892,362.98819733)(98.85165339,362.71736402)(99.01703528,362.35278071)
\curveto(99.29768335,362.71736402)(99.58083719,362.98819733)(99.86649683,363.16528065)
\curveto(100.15716804,363.3475723)(100.44783924,363.43871813)(100.73851045,363.43871813)
\curveto(101.19456355,363.43871813)(101.55790256,363.2850723)(101.82852748,362.97778066)
\curveto(102.18434913,362.58194736)(102.36225996,362.14965573)(102.36225996,361.68090577)
\lineto(102.36225996,357.09496863)
\lineto(102.89599243,357.09496863)
\curveto(103.04633961,357.09496863)(103.15408842,357.06371863)(103.21923886,357.00121864)
\curveto(103.2843893,356.94392698)(103.31696452,356.86580198)(103.31696452,356.76684366)
\curveto(103.31696452,356.67309366)(103.2843893,356.59496867)(103.21923886,356.53246868)
\curveto(103.15408842,356.46996868)(103.04633961,356.43871868)(102.89599243,356.43871868)
\lineto(101.73080182,356.43871868)
\lineto(101.73080182,361.61840577)
\curveto(101.73080182,361.95173908)(101.63307615,362.22778072)(101.43762482,362.4465307)
\curveto(101.24217349,362.66528069)(101.01665273,362.77465568)(100.76106253,362.77465568)
\curveto(100.53053019,362.77465568)(100.28746892,362.68350985)(100.03187872,362.5012182)
\curveto(99.77628852,362.32413488)(99.48561732,361.97257241)(99.1598651,361.44653078)
\lineto(99.1598651,357.09496863)
\lineto(99.68608022,357.09496863)
\curveto(99.83642739,357.09496863)(99.9441762,357.06371863)(100.00932665,357.00121864)
\curveto(100.07447709,356.94392698)(100.10705231,356.86580198)(100.10705231,356.76684366)
\curveto(100.10705231,356.67309366)(100.07447709,356.59496867)(100.00932665,356.53246868)
\curveto(99.9441762,356.46996868)(99.83642739,356.43871868)(99.68608022,356.43871868)
\lineto(98.5208896,356.43871868)
\lineto(98.5208896,361.57153077)
\curveto(98.5208896,361.92048908)(98.42065815,362.20694739)(98.22019525,362.43090571)
\curveto(98.02474392,362.66007235)(97.80423473,362.77465568)(97.55866767,362.77465568)
\curveto(97.33314691,362.77465568)(97.11013193,362.69913485)(96.88962274,362.5480932)
\curveto(96.58391681,362.33455155)(96.26317617,361.96736408)(95.92740081,361.44653078)
\lineto(95.92740081,357.09496863)
\lineto(96.46113329,357.09496863)
\curveto(96.61148046,357.09496863)(96.71922927,357.06371863)(96.78437972,357.00121864)
\curveto(96.84953016,356.94392698)(96.88210538,356.86580198)(96.88210538,356.76684366)
\curveto(96.88210538,356.67309366)(96.84953016,356.59496867)(96.78437972,356.53246868)
\curveto(96.71922927,356.46996868)(96.61148046,356.43871868)(96.46113329,356.43871868)
\lineto(94.7622102,356.43871868)
\curveto(94.61186302,356.43871868)(94.50411421,356.46996868)(94.43896377,356.53246868)
\curveto(94.37381332,356.59496867)(94.3412381,356.67309366)(94.3412381,356.76684366)
\curveto(94.3412381,356.86580198)(94.37381332,356.94392698)(94.43896377,357.00121864)
\curveto(94.50411421,357.06371863)(94.61186302,357.09496863)(94.7622102,357.09496863)
\lineto(95.29594267,357.09496863)
\lineto(95.29594267,362.5402807)
\lineto(94.7622102,362.5402807)
\curveto(94.61186302,362.5402807)(94.50411421,362.57153069)(94.43896377,362.63403069)
\curveto(94.37381332,362.69653068)(94.3412381,362.77725984)(94.3412381,362.87621817)
\curveto(94.3412381,362.96996816)(94.37381332,363.04809316)(94.43896377,363.11059315)
\curveto(94.50411421,363.17309315)(94.61186302,363.20434314)(94.7622102,363.20434314)
\closepath
}
}
{
\newrgbcolor{curcolor}{0 0 0}
\pscustom[linestyle=none,fillstyle=solid,fillcolor=curcolor]
{
\newpath
\moveto(111.57102759,359.80590591)
\curveto(111.57102759,358.80590599)(111.22522908,357.9517394)(110.53363207,357.24340612)
\curveto(109.84704664,356.53507284)(109.01763138,356.1809062)(108.04538631,356.1809062)
\curveto(107.06311809,356.1809062)(106.22869126,356.53507284)(105.54210582,357.24340612)
\curveto(104.85552038,357.95694773)(104.51222767,358.81111433)(104.51222767,359.80590591)
\curveto(104.51222767,360.80590584)(104.85552038,361.66007243)(105.54210582,362.36840571)
\curveto(106.22869126,363.08194732)(107.06311809,363.43871813)(108.04538631,363.43871813)
\curveto(109.01763138,363.43871813)(109.84704664,363.08455149)(110.53363207,362.37621821)
\curveto(111.22522908,361.66788493)(111.57102759,360.81111417)(111.57102759,359.80590591)
\closepath
\moveto(110.93205209,359.80590591)
\curveto(110.93205209,360.62882252)(110.64889824,361.32934329)(110.08259055,361.90746825)
\curveto(109.52129442,362.4855932)(108.83972056,362.77465568)(108.03786895,362.77465568)
\curveto(107.23601734,362.77465568)(106.55193769,362.48298903)(105.98562999,361.89965575)
\curveto(105.42433387,361.32153079)(105.14368581,360.62361418)(105.14368581,359.80590591)
\curveto(105.14368581,358.99340598)(105.42433387,358.29548937)(105.98562999,357.71215608)
\curveto(106.55193769,357.12882279)(107.23601734,356.83715615)(108.03786895,356.83715615)
\curveto(108.83972056,356.83715615)(109.52129442,357.12621863)(110.08259055,357.70434358)
\curveto(110.64889824,358.28767687)(110.93205209,358.98819765)(110.93205209,359.80590591)
\closepath
}
}
{
\newrgbcolor{curcolor}{0 0 0}
\pscustom[linestyle=none,fillstyle=solid,fillcolor=curcolor]
{
\newpath
\moveto(115.23950178,363.20434314)
\lineto(115.23950178,362.21215572)
\curveto(115.68052016,362.67569735)(116.07894018,362.99600983)(116.43476183,363.17309315)
\curveto(116.79058348,363.35017647)(117.19150929,363.43871813)(117.63753924,363.43871813)
\curveto(118.11865021,363.43871813)(118.5571628,363.3319473)(118.95307703,363.11840565)
\curveto(119.2337251,362.96215566)(119.48680951,362.70173902)(119.71233027,362.33715571)
\curveto(119.94286261,361.97778074)(120.05812878,361.6079891)(120.05812878,361.2277808)
\lineto(120.05812878,357.09496863)
\lineto(120.59186125,357.09496863)
\curveto(120.74220843,357.09496863)(120.84995724,357.06371863)(120.91510768,357.00121864)
\curveto(120.98025813,356.94392698)(121.01283335,356.86580198)(121.01283335,356.76684366)
\curveto(121.01283335,356.67309366)(120.98025813,356.59496867)(120.91510768,356.53246868)
\curveto(120.84995724,356.46996868)(120.74220843,356.43871868)(120.59186125,356.43871868)
\lineto(118.90045552,356.43871868)
\curveto(118.74509677,356.43871868)(118.63484218,356.46996868)(118.56969173,356.53246868)
\curveto(118.50454129,356.59496867)(118.47196607,356.67309366)(118.47196607,356.76684366)
\curveto(118.47196607,356.86580198)(118.50454129,356.94392698)(118.56969173,357.00121864)
\curveto(118.63484218,357.06371863)(118.74509677,357.09496863)(118.90045552,357.09496863)
\lineto(119.42667064,357.09496863)
\lineto(119.42667064,361.11840581)
\curveto(119.42667064,361.58194744)(119.26379453,361.97257241)(118.93804232,362.29028072)
\curveto(118.6122901,362.61319736)(118.17628329,362.77465568)(117.63002188,362.77465568)
\curveto(117.21406136,362.77465568)(116.85322814,362.68611402)(116.54752221,362.5090307)
\curveto(116.24181629,362.33715571)(115.80580948,361.90486408)(115.23950178,361.2121558)
\lineto(115.23950178,357.09496863)
\lineto(115.95365087,357.09496863)
\curveto(116.10399804,357.09496863)(116.21174685,357.06371863)(116.2768973,357.00121864)
\curveto(116.34204774,356.94392698)(116.37462296,356.86580198)(116.37462296,356.76684366)
\curveto(116.37462296,356.67309366)(116.34204774,356.59496867)(116.2768973,356.53246868)
\curveto(116.21174685,356.46996868)(116.10399804,356.43871868)(115.95365087,356.43871868)
\lineto(113.89389455,356.43871868)
\curveto(113.74354738,356.43871868)(113.63579857,356.46996868)(113.57064812,356.53246868)
\curveto(113.50549768,356.59496867)(113.47292246,356.67309366)(113.47292246,356.76684366)
\curveto(113.47292246,356.86580198)(113.50549768,356.94392698)(113.57064812,357.00121864)
\curveto(113.63579857,357.06371863)(113.74354738,357.09496863)(113.89389455,357.09496863)
\lineto(114.60804364,357.09496863)
\lineto(114.60804364,362.5402807)
\lineto(114.07431116,362.5402807)
\curveto(113.92396399,362.5402807)(113.81621518,362.57153069)(113.75106474,362.63403069)
\curveto(113.68591429,362.69653068)(113.65333907,362.77725984)(113.65333907,362.87621817)
\curveto(113.65333907,362.96996816)(113.68591429,363.04809316)(113.75106474,363.11059315)
\curveto(113.81621518,363.17309315)(113.92396399,363.20434314)(114.07431116,363.20434314)
\closepath
}
}
{
\newrgbcolor{curcolor}{0 0 0}
\pscustom[linestyle=none,fillstyle=solid,fillcolor=curcolor]
{
\newpath
\moveto(5.11770375,343.20434474)
\lineto(8.54561937,343.20434474)
\curveto(8.69596654,343.20434474)(8.80371535,343.17309474)(8.86886579,343.11059474)
\curveto(8.93401624,343.04809475)(8.96659146,342.96736559)(8.96659146,342.86840726)
\curveto(8.96659146,342.77465727)(8.93401624,342.69653228)(8.86886579,342.63403228)
\curveto(8.80371535,342.57153229)(8.69596654,342.54028229)(8.54561937,342.54028229)
\lineto(5.11770375,342.54028229)
\lineto(5.11770375,338.17309514)
\curveto(5.11770375,337.79288684)(5.26303935,337.47517853)(5.55371056,337.21997021)
\curveto(5.84939334,336.9647619)(6.28038858,336.83715774)(6.84669627,336.83715774)
\curveto(7.27267994,336.83715774)(7.73374461,336.90226191)(8.2298903,337.03247023)
\curveto(8.72603598,337.16788689)(9.11192706,337.31892854)(9.38756355,337.48559519)
\curveto(9.487795,337.55330352)(9.57048595,337.58715769)(9.63563639,337.58715769)
\curveto(9.71582155,337.58715769)(9.78598357,337.55330352)(9.84612244,337.48559519)
\curveto(9.90626131,337.4230952)(9.93633075,337.34757437)(9.93633075,337.25903271)
\curveto(9.93633075,337.18090772)(9.90375552,337.10799106)(9.83860508,337.04028273)
\curveto(9.67823476,336.86840774)(9.2873321,336.68090776)(8.66589711,336.47778277)
\curveto(8.04947368,336.27986612)(7.45810812,336.1809078)(6.89180043,336.1809078)
\curveto(6.15509926,336.1809078)(5.56874528,336.36059528)(5.13273846,336.71997025)
\curveto(4.69673165,337.07934523)(4.47872825,337.56372019)(4.47872825,338.17309514)
\lineto(4.47872825,342.54028229)
\lineto(3.31353763,342.54028229)
\curveto(3.16319046,342.54028229)(3.05544165,342.57153229)(2.9902912,342.63403228)
\curveto(2.92514076,342.69653228)(2.89256554,342.77726144)(2.89256554,342.87621976)
\curveto(2.89256554,342.96996976)(2.92514076,343.04809475)(2.9902912,343.11059474)
\curveto(3.05544165,343.17309474)(3.16319046,343.20434474)(3.31353763,343.20434474)
\lineto(4.47872825,343.20434474)
\lineto(4.47872825,345.14184458)
\curveto(4.47872825,345.29809457)(4.50879768,345.41007373)(4.56893655,345.47778206)
\curveto(4.62907542,345.54549038)(4.70424901,345.57934455)(4.79445732,345.57934455)
\curveto(4.8896772,345.57934455)(4.96735657,345.54549038)(5.02749544,345.47778206)
\curveto(5.08763431,345.41007373)(5.11770375,345.29809457)(5.11770375,345.14184458)
\closepath
}
}
{
\newrgbcolor{curcolor}{0 0 0}
\pscustom[linestyle=none,fillstyle=solid,fillcolor=curcolor]
{
\newpath
\moveto(19.18268252,339.80590751)
\curveto(19.18268252,338.80590759)(18.83688402,337.95174099)(18.14528701,337.24340771)
\curveto(17.45870157,336.53507444)(16.62928631,336.1809078)(15.65704124,336.1809078)
\curveto(14.67477302,336.1809078)(13.84034619,336.53507444)(13.15376076,337.24340771)
\curveto(12.46717532,337.95694932)(12.1238826,338.81111592)(12.1238826,339.80590751)
\curveto(12.1238826,340.80590743)(12.46717532,341.66007403)(13.15376076,342.3684073)
\curveto(13.84034619,343.08194891)(14.67477302,343.43871972)(15.65704124,343.43871972)
\curveto(16.62928631,343.43871972)(17.45870157,343.08455308)(18.14528701,342.3762198)
\curveto(18.83688402,341.66788653)(19.18268252,340.81111576)(19.18268252,339.80590751)
\closepath
\moveto(18.54370702,339.80590751)
\curveto(18.54370702,340.62882411)(18.26055318,341.32934489)(17.69424548,341.90746984)
\curveto(17.13294935,342.48559479)(16.45137549,342.77465727)(15.64952388,342.77465727)
\curveto(14.84767227,342.77465727)(14.16359262,342.48299063)(13.59728493,341.89965734)
\curveto(13.0359888,341.32153239)(12.75534074,340.62361578)(12.75534074,339.80590751)
\curveto(12.75534074,338.99340757)(13.0359888,338.29549096)(13.59728493,337.71215768)
\curveto(14.16359262,337.12882439)(14.84767227,336.83715774)(15.64952388,336.83715774)
\curveto(16.45137549,336.83715774)(17.13294935,337.12622022)(17.69424548,337.70434518)
\curveto(18.26055318,338.28767846)(18.54370702,338.98819924)(18.54370702,339.80590751)
\closepath
}
}
{
\newrgbcolor{curcolor}{0 0 0}
\pscustom[linestyle=none,fillstyle=solid,fillcolor=curcolor]
{
\newpath
\moveto(33.03717657,344.9230946)
\lineto(33.03717657,337.09497022)
\lineto(37.23186279,337.09497022)
\lineto(37.23186279,339.70434502)
\curveto(37.23186279,339.860595)(37.26193223,339.97257416)(37.3220711,340.04028249)
\curveto(37.38220997,340.10799082)(37.45988934,340.14184498)(37.55510922,340.14184498)
\curveto(37.64531753,340.14184498)(37.72049111,340.10799082)(37.78062998,340.04028249)
\curveto(37.84076886,339.97778249)(37.87083829,339.86580334)(37.87083829,339.70434502)
\lineto(37.87083829,336.43872028)
\lineto(30.89472931,336.43872028)
\curveto(30.74438214,336.43872028)(30.63663333,336.46997027)(30.57148288,336.53247027)
\curveto(30.50633244,336.59497026)(30.47375722,336.67309526)(30.47375722,336.76684525)
\curveto(30.47375722,336.86580358)(30.50633244,336.94392857)(30.57148288,337.00122023)
\curveto(30.63663333,337.06372023)(30.74438214,337.09497022)(30.89472931,337.09497022)
\lineto(32.39820108,337.09497022)
\lineto(32.39820108,344.9230946)
\lineto(30.89472931,344.9230946)
\curveto(30.74438214,344.9230946)(30.63663333,344.95174043)(30.57148288,345.00903209)
\curveto(30.50633244,345.07153209)(30.47375722,345.15226125)(30.47375722,345.25121957)
\curveto(30.47375722,345.3501779)(30.50633244,345.42830289)(30.57148288,345.48559456)
\curveto(30.63663333,345.54809455)(30.74438214,345.57934455)(30.89472931,345.57934455)
\lineto(34.54064834,345.57934455)
\curveto(34.69099551,345.57934455)(34.79874432,345.54809455)(34.86389477,345.48559456)
\curveto(34.92904521,345.42830289)(34.96162043,345.3501779)(34.96162043,345.25121957)
\curveto(34.96162043,345.15226125)(34.92904521,345.07153209)(34.86389477,345.00903209)
\curveto(34.79874432,344.95174043)(34.69099551,344.9230946)(34.54064834,344.9230946)
\closepath
}
}
{
\newrgbcolor{curcolor}{0 0 0}
\pscustom[linestyle=none,fillstyle=solid,fillcolor=curcolor]
{
\newpath
\moveto(43.68175621,346.37621948)
\lineto(43.68175621,337.09497022)
\lineto(45.84675554,337.09497022)
\curveto(45.99710272,337.09497022)(46.10485153,337.06372023)(46.17000197,337.00122023)
\curveto(46.23515242,336.94392857)(46.26772764,336.86580358)(46.26772764,336.76684525)
\curveto(46.26772764,336.67309526)(46.23515242,336.59497026)(46.17000197,336.53247027)
\curveto(46.10485153,336.46997027)(45.99710272,336.43872028)(45.84675554,336.43872028)
\lineto(40.88529873,336.43872028)
\curveto(40.73495155,336.43872028)(40.62720274,336.46997027)(40.5620523,336.53247027)
\curveto(40.49690186,336.59497026)(40.46432663,336.67309526)(40.46432663,336.76684525)
\curveto(40.46432663,336.86580358)(40.49690186,336.94392857)(40.5620523,337.00122023)
\curveto(40.62720274,337.06372023)(40.73495155,337.09497022)(40.88529873,337.09497022)
\lineto(43.05029807,337.09497022)
\lineto(43.05029807,345.46996956)
\lineto(40.99054175,344.79809461)
\curveto(40.8903103,344.76684461)(40.8176425,344.75121961)(40.77253835,344.75121961)
\curveto(40.69736476,344.75121961)(40.62720274,344.78246961)(40.5620523,344.84496961)
\curveto(40.50191343,344.91267793)(40.47184399,344.99080293)(40.47184399,345.07934459)
\curveto(40.47184399,345.15746958)(40.49690186,345.23038624)(40.54701758,345.29809457)
\curveto(40.58711016,345.33976123)(40.66980111,345.38403206)(40.79509042,345.43090706)
\closepath
}
}
{
\newrgbcolor{curcolor}{0 0 0}
\pscustom[linestyle=none,fillstyle=solid,fillcolor=curcolor]
{
\newpath
\moveto(55.8298076,341.43090738)
\lineto(55.8298076,340.49340745)
\lineto(49.3874311,340.49340745)
\lineto(49.3874311,341.43090738)
\closepath
}
}
{
\newrgbcolor{curcolor}{0 0 0}
\pscustom[linestyle=none,fillstyle=solid,fillcolor=curcolor]
{
\newpath
\moveto(72.90172857,336.43872028)
\lineto(72.90172857,337.3840327)
\curveto(71.9846108,336.58194943)(71.00484837,336.1809078)(69.96244128,336.1809078)
\curveto(69.20569382,336.1809078)(68.61432826,336.37882445)(68.1883446,336.77465775)
\curveto(67.76236093,337.17569938)(67.5493691,337.66528268)(67.5493691,338.24340763)
\curveto(67.5493691,338.87882425)(67.83001716,339.4335117)(68.39131328,339.90747)
\curveto(68.95260941,340.3814283)(69.77200152,340.61840744)(70.84948962,340.61840744)
\curveto(71.14016082,340.61840744)(71.45588989,340.59757411)(71.79667683,340.55590745)
\curveto(72.13746376,340.51944912)(72.50581434,340.45955329)(72.90172857,340.37621996)
\lineto(72.90172857,341.43871988)
\curveto(72.90172857,341.79809485)(72.74135825,342.11059482)(72.42061761,342.3762198)
\curveto(72.09987697,342.64184478)(71.618766,342.77465727)(70.97728472,342.77465727)
\curveto(70.48615061,342.77465727)(69.79705938,342.62621978)(68.91001104,342.32934481)
\curveto(68.74964072,342.27726148)(68.64690348,342.25121981)(68.60179933,342.25121981)
\curveto(68.52161417,342.25121981)(68.45145216,342.28246981)(68.39131328,342.34496981)
\curveto(68.33618599,342.4074698)(68.30862234,342.48559479)(68.30862234,342.57934479)
\curveto(68.30862234,342.66788645)(68.3336802,342.73819894)(68.38379593,342.79028227)
\curveto(68.45395794,342.86840726)(68.73711179,342.97517809)(69.23325747,343.11059474)
\curveto(70.01506279,343.32934473)(70.60642835,343.43871972)(71.00735415,343.43871972)
\curveto(71.80419419,343.43871972)(72.42562918,343.23299057)(72.87165914,342.82153227)
\curveto(73.31768909,342.4152823)(73.54070407,341.95434484)(73.54070407,341.43871988)
\lineto(73.54070407,337.09497022)
\lineto(74.38264826,337.09497022)
\curveto(74.53800701,337.09497022)(74.6482616,337.06372023)(74.71341205,337.00122023)
\curveto(74.77856249,336.94392857)(74.81113771,336.86580358)(74.81113771,336.76684525)
\curveto(74.81113771,336.67309526)(74.77856249,336.59497026)(74.71341205,336.53247027)
\curveto(74.6482616,336.46997027)(74.53800701,336.43872028)(74.38264826,336.43872028)
\closepath
\moveto(72.90172857,339.70434502)
\curveto(72.60604579,339.79288668)(72.29282251,339.85799084)(71.96205872,339.8996575)
\curveto(71.63129493,339.94132416)(71.28299064,339.9621575)(70.91714585,339.9621575)
\curveto(70.00002807,339.9621575)(69.2833732,339.75642835)(68.76718123,339.34497005)
\curveto(68.37627857,339.0376784)(68.18082724,338.67049093)(68.18082724,338.24340763)
\curveto(68.18082724,337.84757433)(68.32866863,337.51424102)(68.62435141,337.24340771)
\curveto(68.92504576,336.9725744)(69.36105257,336.83715774)(69.93237184,336.83715774)
\curveto(70.47863325,336.83715774)(70.98480207,336.9491369)(71.45087832,337.17309522)
\curveto(71.92196614,337.40226187)(72.40558289,337.764241)(72.90172857,338.25903263)
\closepath
}
}
{
\newrgbcolor{curcolor}{0 0 0}
\pscustom[linestyle=none,fillstyle=solid,fillcolor=curcolor]
{
\newpath
\moveto(78.28416057,343.20434474)
\lineto(78.28416057,342.21215732)
\curveto(78.72517896,342.67569895)(79.12359897,342.99601142)(79.47942062,343.17309474)
\curveto(79.83524227,343.35017806)(80.23616808,343.43871972)(80.68219803,343.43871972)
\curveto(81.163309,343.43871972)(81.60182159,343.33194889)(81.99773582,343.11840724)
\curveto(82.27838389,342.96215726)(82.5314683,342.70174061)(82.75698906,342.33715731)
\curveto(82.9875214,341.97778234)(83.10278757,341.6079907)(83.10278757,341.22778239)
\lineto(83.10278757,337.09497022)
\lineto(83.63652005,337.09497022)
\curveto(83.78686722,337.09497022)(83.89461603,337.06372023)(83.95976647,337.00122023)
\curveto(84.02491692,336.94392857)(84.05749214,336.86580358)(84.05749214,336.76684525)
\curveto(84.05749214,336.67309526)(84.02491692,336.59497026)(83.95976647,336.53247027)
\curveto(83.89461603,336.46997027)(83.78686722,336.43872028)(83.63652005,336.43872028)
\lineto(81.94511431,336.43872028)
\curveto(81.78975556,336.43872028)(81.67950097,336.46997027)(81.61435053,336.53247027)
\curveto(81.54920008,336.59497026)(81.51662486,336.67309526)(81.51662486,336.76684525)
\curveto(81.51662486,336.86580358)(81.54920008,336.94392857)(81.61435053,337.00122023)
\curveto(81.67950097,337.06372023)(81.78975556,337.09497022)(81.94511431,337.09497022)
\lineto(82.47132943,337.09497022)
\lineto(82.47132943,341.1184074)
\curveto(82.47132943,341.58194903)(82.30845332,341.972574)(81.98270111,342.29028231)
\curveto(81.65694889,342.61319895)(81.22094208,342.77465727)(80.67468067,342.77465727)
\curveto(80.25872015,342.77465727)(79.89788693,342.68611561)(79.592181,342.50903229)
\curveto(79.28647508,342.33715731)(78.85046827,341.90486567)(78.28416057,341.2121574)
\lineto(78.28416057,337.09497022)
\lineto(78.99830966,337.09497022)
\curveto(79.14865683,337.09497022)(79.25640564,337.06372023)(79.32155609,337.00122023)
\curveto(79.38670653,336.94392857)(79.41928175,336.86580358)(79.41928175,336.76684525)
\curveto(79.41928175,336.67309526)(79.38670653,336.59497026)(79.32155609,336.53247027)
\curveto(79.25640564,336.46997027)(79.14865683,336.43872028)(78.99830966,336.43872028)
\lineto(76.93855334,336.43872028)
\curveto(76.78820617,336.43872028)(76.68045736,336.46997027)(76.61530691,336.53247027)
\curveto(76.55015647,336.59497026)(76.51758125,336.67309526)(76.51758125,336.76684525)
\curveto(76.51758125,336.86580358)(76.55015647,336.94392857)(76.61530691,337.00122023)
\curveto(76.68045736,337.06372023)(76.78820617,337.09497022)(76.93855334,337.09497022)
\lineto(77.65270243,337.09497022)
\lineto(77.65270243,342.54028229)
\lineto(77.11896996,342.54028229)
\curveto(76.96862278,342.54028229)(76.86087397,342.57153229)(76.79572353,342.63403228)
\curveto(76.73057308,342.69653228)(76.69799786,342.77726144)(76.69799786,342.87621976)
\curveto(76.69799786,342.96996976)(76.73057308,343.04809475)(76.79572353,343.11059474)
\curveto(76.86087397,343.17309474)(76.96862278,343.20434474)(77.11896996,343.20434474)
\closepath
}
}
{
\newrgbcolor{curcolor}{0 0 0}
\pscustom[linestyle=none,fillstyle=solid,fillcolor=curcolor]
{
\newpath
\moveto(92.65735371,346.243407)
\lineto(92.65735371,337.09497022)
\lineto(93.4992979,337.09497022)
\curveto(93.65465665,337.09497022)(93.76491124,337.06372023)(93.83006169,337.00122023)
\curveto(93.89521213,336.94392857)(93.92778735,336.86580358)(93.92778735,336.76684525)
\curveto(93.92778735,336.67309526)(93.89521213,336.59497026)(93.83006169,336.53247027)
\curveto(93.76491124,336.46997027)(93.65465665,336.43872028)(93.4992979,336.43872028)
\lineto(92.01837821,336.43872028)
\lineto(92.01837821,337.74340767)
\curveto(91.28668862,336.70174109)(90.35704191,336.1809078)(89.22943809,336.1809078)
\curveto(88.65811882,336.1809078)(88.10935163,336.33715778)(87.58313651,336.64965776)
\curveto(87.06193297,336.96736607)(86.64847823,337.41788687)(86.34277231,338.00122015)
\curveto(86.04207796,338.58455344)(85.89173078,339.18611589)(85.89173078,339.80590751)
\curveto(85.89173078,340.43090746)(86.04207796,341.03246991)(86.34277231,341.61059486)
\curveto(86.64847823,342.19392815)(87.06193297,342.64444895)(87.58313651,342.96215726)
\curveto(88.10935163,343.27986556)(88.66062461,343.43871972)(89.23695545,343.43871972)
\curveto(90.33950141,343.43871972)(91.26664233,342.91788643)(92.01837821,341.87621984)
\lineto(92.01837821,345.57934455)
\lineto(91.17643402,345.57934455)
\curveto(91.02107528,345.57934455)(90.91082068,345.61059455)(90.84567024,345.67309454)
\curveto(90.78051979,345.73559454)(90.74794457,345.8163237)(90.74794457,345.91528202)
\curveto(90.74794457,346.00903201)(90.78051979,346.08715701)(90.84567024,346.149657)
\curveto(90.91082068,346.212157)(91.02107528,346.243407)(91.17643402,346.243407)
\closepath
\moveto(92.01837821,339.80590751)
\curveto(92.01837821,340.63924078)(91.74775329,341.34236572)(91.20650346,341.91528234)
\curveto(90.66525363,342.48819896)(90.02126655,342.77465727)(89.27454224,342.77465727)
\curveto(88.52280636,342.77465727)(87.87631351,342.48819896)(87.33506367,341.91528234)
\curveto(86.79381384,341.34236572)(86.52318892,340.63924078)(86.52318892,339.80590751)
\curveto(86.52318892,338.97778257)(86.79381384,338.27465763)(87.33506367,337.69653268)
\curveto(87.87631351,337.12361606)(88.52280636,336.83715774)(89.27454224,336.83715774)
\curveto(90.02126655,336.83715774)(90.66525363,337.12361606)(91.20650346,337.69653268)
\curveto(91.74775329,338.27465763)(92.01837821,338.97778257)(92.01837821,339.80590751)
\closepath
}
}
{
\newrgbcolor{curcolor}{0 0 0}
\pscustom[linestyle=none,fillstyle=solid,fillcolor=curcolor]
{
\newpath
\moveto(5.32067244,324.92309619)
\lineto(5.32067244,317.09497182)
\lineto(9.51535865,317.09497182)
\lineto(9.51535865,319.70434661)
\curveto(9.51535865,319.8605966)(9.54542809,319.97257576)(9.60556696,320.04028408)
\curveto(9.66570583,320.10799241)(9.7433852,320.14184657)(9.83860508,320.14184657)
\curveto(9.92881339,320.14184657)(10.00398698,320.10799241)(10.06412585,320.04028408)
\curveto(10.12426472,319.97778409)(10.15433415,319.86580493)(10.15433415,319.70434661)
\lineto(10.15433415,316.43872187)
\lineto(3.17822517,316.43872187)
\curveto(3.027878,316.43872187)(2.92012919,316.46997187)(2.85497874,316.53247186)
\curveto(2.7898283,316.59497186)(2.75725308,316.67309685)(2.75725308,316.76684684)
\curveto(2.75725308,316.86580517)(2.7898283,316.94393016)(2.85497874,317.00122183)
\curveto(2.92012919,317.06372182)(3.027878,317.09497182)(3.17822517,317.09497182)
\lineto(4.68169694,317.09497182)
\lineto(4.68169694,324.92309619)
\lineto(3.17822517,324.92309619)
\curveto(3.027878,324.92309619)(2.92012919,324.95174203)(2.85497874,325.00903369)
\curveto(2.7898283,325.07153368)(2.75725308,325.15226284)(2.75725308,325.25122117)
\curveto(2.75725308,325.35017949)(2.7898283,325.42830449)(2.85497874,325.48559615)
\curveto(2.92012919,325.54809614)(3.027878,325.57934614)(3.17822517,325.57934614)
\lineto(6.8241442,325.57934614)
\curveto(6.97449137,325.57934614)(7.08224018,325.54809614)(7.14739063,325.48559615)
\curveto(7.21254107,325.42830449)(7.24511629,325.35017949)(7.24511629,325.25122117)
\curveto(7.24511629,325.15226284)(7.21254107,325.07153368)(7.14739063,325.00903369)
\curveto(7.08224018,324.95174203)(6.97449137,324.92309619)(6.8241442,324.92309619)
\closepath
}
}
{
\newrgbcolor{curcolor}{0 0 0}
\pscustom[linestyle=none,fillstyle=solid,fillcolor=curcolor]
{
\newpath
\moveto(12.93575735,317.09497182)
\lineto(17.78445378,317.09497182)
\lineto(17.78445378,317.3527843)
\curveto(17.78445378,317.50903428)(17.81452322,317.62101344)(17.87466209,317.68872177)
\curveto(17.93480096,317.7564301)(18.01248033,317.79028426)(18.10770021,317.79028426)
\curveto(18.19790852,317.79028426)(18.27308211,317.7564301)(18.33322098,317.68872177)
\curveto(18.39335985,317.62101344)(18.42342928,317.50903428)(18.42342928,317.3527843)
\lineto(18.42342928,316.43872187)
\lineto(12.31933393,316.43872187)
\lineto(12.31933393,317.40747179)
\curveto(13.6022965,318.60538836)(14.85017806,319.82413827)(16.06297862,321.0637215)
\curveto(16.62928631,321.64184646)(17.01768319,322.06632559)(17.22816923,322.3371589)
\curveto(17.43865528,322.60799221)(17.5814851,322.84497136)(17.65665868,323.04809634)
\curveto(17.73684385,323.25642966)(17.77693643,323.46476298)(17.77693643,323.67309629)
\curveto(17.77693643,324.23559625)(17.55893302,324.73299204)(17.12292621,325.16528367)
\curveto(16.6869194,325.59757531)(16.16070428,325.81372112)(15.54428086,325.81372112)
\curveto(14.99300788,325.81372112)(14.50437956,325.64965864)(14.07839589,325.32153366)
\curveto(13.65241222,324.99340869)(13.37677573,324.58715872)(13.25148642,324.10278376)
\curveto(13.22141698,323.97778377)(13.18383019,323.89705461)(13.13872604,323.86059628)
\curveto(13.07858717,323.80851295)(13.00842515,323.78247128)(12.92823999,323.78247128)
\curveto(12.84304326,323.78247128)(12.77037546,323.81111712)(12.71023659,323.86840878)
\curveto(12.65510929,323.93090877)(12.62754564,324.00382543)(12.62754564,324.08715876)
\curveto(12.62754564,324.33715874)(12.75784652,324.67570038)(13.0184483,325.10278368)
\curveto(13.28406164,325.52986698)(13.64740065,325.86580445)(14.10846532,326.1105961)
\curveto(14.57454157,326.35538775)(15.05064096,326.47778357)(15.5367635,326.47778357)
\curveto(16.32358039,326.47778357)(17.00014268,326.19392943)(17.56645038,325.62622114)
\curveto(18.13275808,325.05851285)(18.41591192,324.41528373)(18.41591192,323.69653379)
\curveto(18.41591192,323.39445048)(18.36830199,323.11580467)(18.27308211,322.86059636)
\curveto(18.17786223,322.61059638)(18.01498612,322.33455473)(17.78445378,322.03247142)
\curveto(17.55893302,321.73559645)(17.15048986,321.28507565)(16.5591243,320.68090903)
\curveto(15.07569883,319.16007582)(13.86790984,317.98038841)(12.93575735,317.14184681)
\closepath
}
}
{
\newrgbcolor{curcolor}{0 0 0}
\pscustom[linestyle=none,fillstyle=solid,fillcolor=curcolor]
{
\newpath
\moveto(26.97066669,319.49340913)
\lineto(22.82108463,319.49340913)
\lineto(21.97162308,317.09497182)
\lineto(23.18191785,317.09497182)
\curveto(23.33226503,317.09497182)(23.44001384,317.06372182)(23.50516428,317.00122183)
\curveto(23.57031472,316.94393016)(23.60288994,316.86580517)(23.60288994,316.76684684)
\curveto(23.60288994,316.67309685)(23.57031472,316.59497186)(23.50516428,316.53247186)
\curveto(23.44001384,316.46997187)(23.33226503,316.43872187)(23.18191785,316.43872187)
\lineto(20.82146718,316.43872187)
\curveto(20.67112001,316.43872187)(20.5633712,316.46997187)(20.49822075,316.53247186)
\curveto(20.43307031,316.59497186)(20.40049509,316.67309685)(20.40049509,316.76684684)
\curveto(20.40049509,316.86580517)(20.43307031,316.94393016)(20.49822075,317.00122183)
\curveto(20.5633712,317.06372182)(20.67112001,317.09497182)(20.82146718,317.09497182)
\lineto(21.3100955,317.09497182)
\lineto(24.09903562,324.92309619)
\lineto(22.23473064,324.92309619)
\curveto(22.08438346,324.92309619)(21.97663465,324.95174203)(21.91148421,325.00903369)
\curveto(21.84633377,325.07153368)(21.81375855,325.15226284)(21.81375855,325.25122117)
\curveto(21.81375855,325.35017949)(21.84633377,325.42830449)(21.91148421,325.48559615)
\curveto(21.97663465,325.54809614)(22.08438346,325.57934614)(22.23473064,325.57934614)
\lineto(25.41457342,325.57934614)
\lineto(28.49669053,317.09497182)
\lineto(28.98531885,317.09497182)
\curveto(29.13566603,317.09497182)(29.24341484,317.06372182)(29.30856528,317.00122183)
\curveto(29.37371572,316.94393016)(29.40629095,316.86580517)(29.40629095,316.76684684)
\curveto(29.40629095,316.67309685)(29.37371572,316.59497186)(29.30856528,316.53247186)
\curveto(29.24341484,316.46997187)(29.13566603,316.43872187)(28.98531885,316.43872187)
\lineto(26.63238554,316.43872187)
\curveto(26.4770268,316.43872187)(26.3667722,316.46997187)(26.30162176,316.53247186)
\curveto(26.23647131,316.59497186)(26.20389609,316.67309685)(26.20389609,316.76684684)
\curveto(26.20389609,316.86580517)(26.23647131,316.94393016)(26.30162176,317.00122183)
\curveto(26.3667722,317.06372182)(26.4770268,317.09497182)(26.63238554,317.09497182)
\lineto(27.83516295,317.09497182)
\closepath
\moveto(26.73011121,320.14965907)
\lineto(24.98608396,324.92309619)
\lineto(24.75304584,324.92309619)
\lineto(23.06164011,320.14965907)
\closepath
}
}
{
\newrgbcolor{curcolor}{0 0 0}
\pscustom[linestyle=none,fillstyle=solid,fillcolor=curcolor]
{
\newpath
\moveto(32.07495465,320.51684655)
\lineto(32.07495465,317.09497182)
\lineto(33.24014526,317.09497182)
\curveto(33.39550401,317.09497182)(33.50575861,317.06372182)(33.57090905,317.00122183)
\curveto(33.63605949,316.94393016)(33.66863471,316.86580517)(33.66863471,316.76684684)
\curveto(33.66863471,316.67309685)(33.63605949,316.59497186)(33.57090905,316.53247186)
\curveto(33.50575861,316.46997187)(33.39550401,316.43872187)(33.24014526,316.43872187)
\lineto(30.59403496,316.43872187)
\curveto(30.44368778,316.43872187)(30.33593897,316.46997187)(30.27078853,316.53247186)
\curveto(30.20563809,316.59497186)(30.17306287,316.67309685)(30.17306287,316.76684684)
\curveto(30.17306287,316.86580517)(30.20563809,316.94393016)(30.27078853,317.00122183)
\curveto(30.33593897,317.06372182)(30.44368778,317.09497182)(30.59403496,317.09497182)
\lineto(31.44349651,317.09497182)
\lineto(31.44349651,324.92309619)
\lineto(30.59403496,324.92309619)
\curveto(30.44368778,324.92309619)(30.33593897,324.95174203)(30.27078853,325.00903369)
\curveto(30.20563809,325.07153368)(30.17306287,325.15226284)(30.17306287,325.25122117)
\curveto(30.17306287,325.35017949)(30.20563809,325.42830449)(30.27078853,325.48559615)
\curveto(30.33593897,325.54809614)(30.44368778,325.57934614)(30.59403496,325.57934614)
\lineto(34.41285324,325.57934614)
\curveto(35.19967013,325.57934614)(35.8611977,325.3189295)(36.39743596,324.7980962)
\curveto(36.9386858,324.28247125)(37.20931072,323.69913796)(37.20931072,323.04809634)
\curveto(37.20931072,322.57934638)(37.04142303,322.13663808)(36.70564767,321.71997145)
\curveto(36.37488389,321.30851315)(35.81859933,320.96476318)(35.03679402,320.68872153)
\curveto(35.48783555,320.36580489)(35.87372663,320.00122159)(36.19446728,319.59497162)
\curveto(36.51520792,319.18872165)(37.0288941,318.35538838)(37.73552583,317.09497182)
\lineto(38.2166368,317.09497182)
\curveto(38.36698397,317.09497182)(38.47473278,317.06372182)(38.53988322,317.00122183)
\curveto(38.60503367,316.94393016)(38.63760889,316.86580517)(38.63760889,316.76684684)
\curveto(38.63760889,316.67309685)(38.60503367,316.59497186)(38.53988322,316.53247186)
\curveto(38.47473278,316.46997187)(38.36698397,316.43872187)(38.2166368,316.43872187)
\lineto(37.37469261,316.43872187)
\curveto(36.59288729,317.89184675)(35.98648701,318.86580501)(35.55549178,319.36059664)
\curveto(35.12950811,319.85538826)(34.638374,320.2408049)(34.08208945,320.51684655)
\closepath
\moveto(32.07495465,321.17309649)
\lineto(33.79642981,321.17309649)
\curveto(34.34770279,321.17309649)(34.84635426,321.27726315)(35.29238422,321.48559647)
\curveto(35.74342575,321.69392978)(36.06917796,321.93872143)(36.26964086,322.21997141)
\curveto(36.47511534,322.50122139)(36.57785257,322.7876797)(36.57785257,323.07934634)
\curveto(36.57785257,323.51684631)(36.36486074,323.93351294)(35.93887708,324.32934624)
\curveto(35.51790498,324.72517954)(35.0042188,324.92309619)(34.39781852,324.92309619)
\lineto(32.07495465,324.92309619)
\closepath
}
}
{
\newrgbcolor{curcolor}{0 0 0}
\pscustom[linestyle=none,fillstyle=solid,fillcolor=curcolor]
{
\newpath
\moveto(46.1474499,324.68090871)
\lineto(46.1474499,325.14184618)
\curveto(46.1474499,325.29809616)(46.17501355,325.41007532)(46.23014084,325.47778365)
\curveto(46.29027971,325.54549198)(46.36795909,325.57934614)(46.46317897,325.57934614)
\curveto(46.55839885,325.57934614)(46.63357243,325.54549198)(46.68869973,325.47778365)
\curveto(46.7488386,325.41007532)(46.77890804,325.29809616)(46.77890804,325.14184618)
\lineto(46.77890804,323.32153382)
\curveto(46.77890804,323.1600755)(46.7488386,323.04549218)(46.68869973,322.97778385)
\curveto(46.63357243,322.91007552)(46.55839885,322.87622136)(46.46317897,322.87622136)
\curveto(46.37798223,322.87622136)(46.30531443,322.90747135)(46.24517556,322.96997135)
\curveto(46.19004826,323.03247134)(46.15747304,323.13403384)(46.1474499,323.27465883)
\curveto(46.12239203,323.71215879)(45.89436548,324.10017959)(45.46337025,324.43872123)
\curveto(44.88202783,324.90226286)(44.22050025,325.13403368)(43.47878752,325.13403368)
\curveto(42.99266498,325.13403368)(42.53410609,325.02205452)(42.10311086,324.7980962)
\curveto(41.78237021,324.63663788)(41.52678001,324.43872123)(41.33634026,324.20434625)
\curveto(41.00557647,323.79809628)(40.74246891,323.34757549)(40.54701758,322.85278386)
\curveto(40.40669355,322.48820055)(40.33653153,322.07674225)(40.33653153,321.61840896)
\lineto(40.33653153,320.46997155)
\curveto(40.33653153,319.49080496)(40.67731847,318.63924253)(41.35889233,317.91528425)
\curveto(42.0404662,317.19653431)(42.83229466,316.83715934)(43.73437772,316.83715934)
\curveto(44.27562755,316.83715934)(44.7592443,316.95955516)(45.18522797,317.20434681)
\curveto(45.61622321,317.44913846)(46.03218373,317.81893009)(46.43310953,318.31372172)
\curveto(46.51830627,318.42309671)(46.61352614,318.47778421)(46.71876917,318.47778421)
\curveto(46.80897747,318.47778421)(46.88164527,318.44913838)(46.93677257,318.39184671)
\curveto(46.99189987,318.33455505)(47.01946352,318.26163839)(47.01946352,318.17309673)
\curveto(47.01946352,318.05330507)(46.91171471,317.87101342)(46.69621709,317.62622178)
\curveto(46.28526814,317.14705515)(45.82169768,316.78507601)(45.30550571,316.54028436)
\curveto(44.79432531,316.30070105)(44.27562755,316.18090939)(43.74941244,316.18090939)
\curveto(43.29335933,316.18090939)(42.82978887,316.27205522)(42.35870106,316.45434687)
\curveto(41.99786783,316.59497186)(41.69967927,316.75903434)(41.46413536,316.94653433)
\curveto(41.22859145,317.13403431)(40.93792024,317.44913846)(40.59212173,317.89184675)
\curveto(40.2513348,318.33976339)(40.01829668,318.75122169)(39.89300736,319.12622166)
\curveto(39.76771805,319.50642996)(39.70507339,319.92309659)(39.70507339,320.37622156)
\lineto(39.70507339,321.71215895)
\curveto(39.70507339,322.35799223)(39.87045529,323.01684635)(40.20121908,323.68872129)
\curveto(40.53699444,324.36580457)(40.99304754,324.8840337)(41.56937838,325.24340867)
\curveto(42.15072079,325.60799197)(42.77967315,325.79028362)(43.45623544,325.79028362)
\curveto(44.49864253,325.79028362)(45.39571402,325.42049199)(46.1474499,324.68090871)
\closepath
}
}
{
\newrgbcolor{curcolor}{0 0 0}
\pscustom[linestyle=none,fillstyle=solid,fillcolor=curcolor]
{
\newpath
\moveto(6.1512925,286.51684479)
\lineto(6.1512925,277.36840799)
\lineto(8.63202096,277.36840799)
\curveto(8.78737972,277.36840799)(8.89763431,277.33715799)(8.96278476,277.27465799)
\curveto(9.0279352,277.21736633)(9.06051043,277.13924134)(9.06051043,277.04028301)
\curveto(9.06051043,276.94653302)(9.0279352,276.86840802)(8.96278476,276.80590803)
\curveto(8.89763431,276.74340803)(8.78737972,276.71215804)(8.63202096,276.71215804)
\lineto(3.03910587,276.71215804)
\curveto(2.88875869,276.71215804)(2.78100988,276.74340803)(2.71585944,276.80590803)
\curveto(2.65070899,276.86840802)(2.61813377,276.94653302)(2.61813377,277.04028301)
\curveto(2.61813377,277.13924134)(2.65070899,277.21736633)(2.71585944,277.27465799)
\curveto(2.78100988,277.33715799)(2.88875869,277.36840799)(3.03910587,277.36840799)
\lineto(5.51983434,277.36840799)
\lineto(5.51983434,285.85278234)
\lineto(3.70063346,285.85278234)
\curveto(3.55028628,285.85278234)(3.44003168,285.88403234)(3.36986967,285.94653234)
\curveto(3.30471922,286.00903233)(3.272144,286.08976149)(3.272144,286.18871982)
\curveto(3.272144,286.28246981)(3.30471922,286.3605948)(3.36986967,286.4230948)
\curveto(3.43502011,286.48559479)(3.54527471,286.51684479)(3.70063346,286.51684479)
\closepath
}
}
{
\newrgbcolor{curcolor}{0 0 0}
\pscustom[linestyle=none,fillstyle=solid,fillcolor=curcolor]
{
\newpath
\moveto(12.3531141,277.36840799)
\lineto(17.20181066,277.36840799)
\lineto(17.20181066,277.62622047)
\curveto(17.20181066,277.78247045)(17.23188009,277.89444961)(17.29201896,277.96215794)
\curveto(17.35215784,278.02986627)(17.42983721,278.06372043)(17.52505709,278.06372043)
\curveto(17.6152654,278.06372043)(17.69043899,278.02986627)(17.75057786,277.96215794)
\curveto(17.81071673,277.89444961)(17.84078617,277.78247045)(17.84078617,277.62622047)
\lineto(17.84078617,276.71215804)
\lineto(11.73669067,276.71215804)
\lineto(11.73669067,277.68090796)
\curveto(13.01965327,278.87882454)(14.26753486,280.09757445)(15.48033545,281.33715768)
\curveto(16.04664316,281.91528264)(16.43504004,282.33976178)(16.64552609,282.61059509)
\curveto(16.85601214,282.8814284)(16.99884196,283.11840755)(17.07401555,283.32153253)
\curveto(17.15420071,283.52986585)(17.1942933,283.73819917)(17.1942933,283.94653249)
\curveto(17.1942933,284.50903244)(16.97628989,285.00642824)(16.54028306,285.43871987)
\curveto(16.10427624,285.87101151)(15.57806111,286.08715732)(14.96163767,286.08715732)
\curveto(14.41036468,286.08715732)(13.92173635,285.92309484)(13.49575267,285.59496986)
\curveto(13.06976899,285.26684489)(12.7941325,284.86059492)(12.66884318,284.37621995)
\curveto(12.63877375,284.25121996)(12.60118695,284.1704908)(12.5560828,284.13403247)
\curveto(12.49594392,284.08194914)(12.42578191,284.05590748)(12.34559674,284.05590748)
\curveto(12.26040001,284.05590748)(12.18773221,284.08455331)(12.12759333,284.14184497)
\curveto(12.07246603,284.20434497)(12.04490238,284.27726163)(12.04490238,284.36059496)
\curveto(12.04490238,284.61059494)(12.17520327,284.94913658)(12.43580505,285.37621988)
\curveto(12.7014184,285.80330318)(13.06475742,286.13924065)(13.52582211,286.3840323)
\curveto(13.99189836,286.62882395)(14.46799777,286.75121977)(14.95412032,286.75121977)
\curveto(15.74093722,286.75121977)(16.41749953,286.46736563)(16.98380724,285.89965734)
\curveto(17.55011496,285.33194905)(17.83326881,284.68871993)(17.83326881,283.96996999)
\curveto(17.83326881,283.66788667)(17.78565887,283.38924086)(17.69043899,283.13403255)
\curveto(17.59521911,282.88403257)(17.432343,282.60799092)(17.20181066,282.30590761)
\curveto(16.97628989,282.00903263)(16.56784671,281.55851183)(15.97648114,280.95434521)
\curveto(14.49305563,279.433512)(13.28526662,278.25382459)(12.3531141,277.41528298)
\closepath
}
}
{
\newrgbcolor{curcolor}{0 0 0}
\pscustom[linestyle=none,fillstyle=solid,fillcolor=curcolor]
{
\newpath
\moveto(26.12491622,276.71215804)
\lineto(26.12491622,277.65747046)
\curveto(25.20779842,276.85538719)(24.22803597,276.45434556)(23.18562885,276.45434556)
\curveto(22.42888138,276.45434556)(21.83751581,276.65226221)(21.41153213,277.04809551)
\curveto(20.98554845,277.44913715)(20.77255662,277.93872044)(20.77255662,278.5168454)
\curveto(20.77255662,279.15226202)(21.05320468,279.70694948)(21.61450082,280.18090777)
\curveto(22.17579696,280.65486607)(22.99518909,280.89184522)(24.07267721,280.89184522)
\curveto(24.36334843,280.89184522)(24.67907751,280.87101189)(25.01986445,280.82934522)
\curveto(25.36065139,280.79288689)(25.72900198,280.73299106)(26.12491622,280.64965774)
\lineto(26.12491622,281.71215766)
\curveto(26.12491622,282.07153263)(25.96454589,282.38403261)(25.64380524,282.64965759)
\curveto(25.32306459,282.91528257)(24.84195362,283.04809506)(24.20047232,283.04809506)
\curveto(23.7093382,283.04809506)(23.02024695,282.89965757)(22.13319859,282.60278259)
\curveto(21.97282827,282.55069926)(21.87009103,282.52465759)(21.82498687,282.52465759)
\curveto(21.74480171,282.52465759)(21.67463969,282.55590759)(21.61450082,282.61840759)
\curveto(21.55937352,282.68090758)(21.53180987,282.75903258)(21.53180987,282.85278257)
\curveto(21.53180987,282.94132423)(21.55686774,283.01163672)(21.60698346,283.06372005)
\curveto(21.67714548,283.14184505)(21.96029934,283.24861587)(22.45644503,283.38403253)
\curveto(23.23825037,283.60278251)(23.82961594,283.7121575)(24.23054175,283.7121575)
\curveto(25.02738181,283.7121575)(25.64881682,283.50642835)(26.09484678,283.09497005)
\curveto(26.54087675,282.68872008)(26.76389173,282.22778262)(26.76389173,281.71215766)
\lineto(26.76389173,277.36840799)
\lineto(27.60583594,277.36840799)
\curveto(27.76119469,277.36840799)(27.87144929,277.33715799)(27.93659974,277.27465799)
\curveto(28.00175018,277.21736633)(28.0343254,277.13924134)(28.0343254,277.04028301)
\curveto(28.0343254,276.94653302)(28.00175018,276.86840802)(27.93659974,276.80590803)
\curveto(27.87144929,276.74340803)(27.76119469,276.71215804)(27.60583594,276.71215804)
\closepath
\moveto(26.12491622,279.97778279)
\curveto(25.82923343,280.06632445)(25.51601014,280.13142861)(25.18524635,280.17309527)
\curveto(24.85448255,280.21476194)(24.50617825,280.23559527)(24.14033344,280.23559527)
\curveto(23.22321565,280.23559527)(22.50656076,280.02986612)(21.99036877,279.61840782)
\curveto(21.5994661,279.31111617)(21.40401477,278.9439287)(21.40401477,278.5168454)
\curveto(21.40401477,278.1210121)(21.55185616,277.78767879)(21.84753895,277.51684547)
\curveto(22.14823331,277.24601216)(22.58424013,277.11059551)(23.15555942,277.11059551)
\curveto(23.70182084,277.11059551)(24.20798968,277.22257466)(24.67406593,277.44653298)
\curveto(25.14515376,277.67569963)(25.62877053,278.03767877)(26.12491622,278.5324704)
\closepath
}
}
{
\newrgbcolor{curcolor}{0 0 0}
\pscustom[linestyle=none,fillstyle=solid,fillcolor=curcolor]
{
\newpath
\moveto(32.79281497,283.47778252)
\lineto(32.79281497,281.82153265)
\curveto(33.61471289,282.59236592)(34.22863054,283.08715755)(34.63456792,283.30590754)
\curveto(35.04551688,283.52986585)(35.42389062,283.64184501)(35.76968913,283.64184501)
\curveto(36.14555708,283.64184501)(36.49386138,283.50903252)(36.81460203,283.24340754)
\curveto(37.14035425,282.98299089)(37.30323037,282.78507424)(37.30323037,282.64965759)
\curveto(37.30323037,282.55069926)(37.27065514,282.46736593)(37.2055047,282.3996576)
\curveto(37.14536583,282.33715761)(37.06768645,282.30590761)(36.97246657,282.30590761)
\curveto(36.92235084,282.30590761)(36.87975248,282.31372011)(36.84467147,282.32934511)
\curveto(36.80959046,282.35017844)(36.74444001,282.41007427)(36.64922013,282.5090326)
\curveto(36.47381509,282.69132425)(36.32096212,282.81632424)(36.19066123,282.88403257)
\curveto(36.06036035,282.9517409)(35.93256524,282.98559506)(35.80727593,282.98559506)
\curveto(35.53163943,282.98559506)(35.19836985,282.87101174)(34.80746718,282.64184509)
\curveto(34.42157608,282.41267844)(33.75002535,281.85278265)(32.79281497,280.96215771)
\lineto(32.79281497,277.36840799)
\lineto(35.58927252,277.36840799)
\curveto(35.74463127,277.36840799)(35.85488587,277.33715799)(35.92003631,277.27465799)
\curveto(35.98518676,277.21736633)(36.01776198,277.13924134)(36.01776198,277.04028301)
\curveto(36.01776198,276.94653302)(35.98518676,276.86840802)(35.92003631,276.80590803)
\curveto(35.85488587,276.74340803)(35.74463127,276.71215804)(35.58927252,276.71215804)
\lineto(30.63533294,276.71215804)
\curveto(30.48498576,276.71215804)(30.37723695,276.74080387)(30.3120865,276.79809553)
\curveto(30.24693606,276.86059552)(30.21436083,276.93872052)(30.21436083,277.03247051)
\curveto(30.21436083,277.12101217)(30.24443027,277.19392883)(30.30456914,277.25122049)
\curveto(30.36971959,277.31372049)(30.47997419,277.34497049)(30.63533294,277.34497049)
\lineto(32.16135681,277.34497049)
\lineto(32.16135681,282.81372007)
\lineto(30.99616617,282.81372007)
\curveto(30.84581899,282.81372007)(30.73807018,282.84497007)(30.67291973,282.90747007)
\curveto(30.60776929,282.96997006)(30.57519407,283.05069922)(30.57519407,283.14965755)
\curveto(30.57519407,283.24340754)(30.6052635,283.32153253)(30.66540237,283.38403253)
\curveto(30.73055282,283.44653252)(30.84080742,283.47778252)(30.99616617,283.47778252)
\closepath
}
}
{
\newrgbcolor{curcolor}{0 0 0}
\pscustom[linestyle=none,fillstyle=solid,fillcolor=curcolor]
{
\newpath
\moveto(45.42197763,282.81372007)
\lineto(45.42197763,283.03247006)
\curveto(45.42197763,283.19392838)(45.45204707,283.3085117)(45.51218594,283.37622003)
\curveto(45.57232481,283.44392836)(45.6474984,283.47778252)(45.73770671,283.47778252)
\curveto(45.83292659,283.47778252)(45.91060597,283.44392836)(45.97074484,283.37622003)
\curveto(46.03088371,283.3085117)(46.06095315,283.19392838)(46.06095315,283.03247006)
\lineto(46.06095315,281.54809517)
\curveto(46.05594158,281.38663685)(46.02336635,281.27205352)(45.96322748,281.20434519)
\curveto(45.90810018,281.13663687)(45.83292659,281.1027827)(45.73770671,281.1027827)
\curveto(45.65250998,281.1027827)(45.57984217,281.13142853)(45.5197033,281.1887202)
\curveto(45.464576,281.25122019)(45.43200078,281.35278268)(45.42197763,281.49340767)
\curveto(45.3919082,281.86319931)(45.15636428,282.21476179)(44.71534589,282.54809509)
\curveto(44.27933907,282.8814284)(43.69047928,283.04809506)(42.94876652,283.04809506)
\curveto(42.01160244,283.04809506)(41.29995912,282.74340758)(40.81383657,282.13403262)
\curveto(40.32771402,281.52465767)(40.08465275,280.82674106)(40.08465275,280.04028278)
\curveto(40.08465275,279.19132451)(40.35277189,278.49080373)(40.88901016,277.93872044)
\curveto(41.42524844,277.38663715)(42.11935125,277.11059551)(42.9713186,277.11059551)
\curveto(43.46245272,277.11059551)(43.9611042,277.2043455)(44.46727304,277.39184548)
\curveto(44.97845345,277.57934547)(45.43951814,277.88142878)(45.8504671,278.29809542)
\curveto(45.95571012,278.40226207)(46.04842422,278.4543454)(46.12860938,278.4543454)
\curveto(46.21380611,278.4543454)(46.28396813,278.42309541)(46.33909543,278.36059541)
\curveto(46.3992343,278.30330375)(46.42930374,278.23038709)(46.42930374,278.14184543)
\curveto(46.42930374,277.91788711)(46.17621932,277.63403297)(45.67005048,277.29028299)
\curveto(44.85316414,276.73299137)(43.9435637,276.45434556)(42.94124917,276.45434556)
\curveto(41.92389991,276.45434556)(41.08696728,276.79028303)(40.43045126,277.46215798)
\curveto(39.77894682,278.13924126)(39.45319459,278.99601203)(39.45319459,280.03247028)
\curveto(39.45319459,281.08976187)(39.78646417,281.96736597)(40.45300334,282.66528258)
\curveto(41.12455408,283.3631992)(41.96900407,283.7121575)(42.98635332,283.7121575)
\curveto(43.95358684,283.7121575)(44.76546161,283.41267836)(45.42197763,282.81372007)
\closepath
}
}
{
\newrgbcolor{curcolor}{0 0 0}
\pscustom[linestyle=none,fillstyle=solid,fillcolor=curcolor]
{
\newpath
\moveto(56.22442206,272.32934587)
\lineto(47.82753207,272.32934587)
\curveto(47.67718489,272.32934587)(47.56943607,272.36059587)(47.50428563,272.42309586)
\curveto(47.43913518,272.48038752)(47.40655996,272.55851252)(47.40655996,272.65747084)
\curveto(47.40655996,272.75642917)(47.43913518,272.83715833)(47.50428563,272.89965832)
\curveto(47.56943607,272.95694999)(47.67718489,272.98559582)(47.82753207,272.98559582)
\lineto(56.22442206,272.98559582)
\curveto(56.37978082,272.98559582)(56.48752963,272.95694999)(56.5476685,272.89965832)
\curveto(56.61281894,272.83715833)(56.64539417,272.75642917)(56.64539417,272.65747084)
\curveto(56.64539417,272.55851252)(56.61281894,272.48038752)(56.5476685,272.42309586)
\curveto(56.48752963,272.36059587)(56.37978082,272.32934587)(56.22442206,272.32934587)
\closepath
}
}
{
\newrgbcolor{curcolor}{0 0 0}
\pscustom[linestyle=none,fillstyle=solid,fillcolor=curcolor]
{
\newpath
\moveto(58.88557024,286.51684479)
\lineto(58.88557024,282.16528262)
\curveto(59.64732928,283.19653254)(60.56695286,283.7121575)(61.64444099,283.7121575)
\curveto(62.56657036,283.7121575)(63.35589305,283.3631992)(64.01240907,282.66528258)
\curveto(64.66892509,281.9725743)(64.9971831,281.12101187)(64.9971831,280.11059528)
\curveto(64.9971831,279.08976202)(64.66391352,278.22517875)(63.99737435,277.51684547)
\curveto(63.33584676,276.8085122)(62.55153564,276.45434556)(61.64444099,276.45434556)
\curveto(60.541895,276.45434556)(59.62227142,276.96997052)(58.88557024,278.00122044)
\lineto(58.88557024,276.71215804)
\lineto(57.40465051,276.71215804)
\curveto(57.25430333,276.71215804)(57.14655452,276.74340803)(57.08140408,276.80590803)
\curveto(57.01625363,276.86840802)(56.98367841,276.94653302)(56.98367841,277.04028301)
\curveto(56.98367841,277.13924134)(57.01625363,277.21736633)(57.08140408,277.27465799)
\curveto(57.14655452,277.33715799)(57.25430333,277.36840799)(57.40465051,277.36840799)
\lineto(58.25411208,277.36840799)
\lineto(58.25411208,285.85278234)
\lineto(57.40465051,285.85278234)
\curveto(57.25430333,285.85278234)(57.14655452,285.88403234)(57.08140408,285.94653234)
\curveto(57.01625363,286.00903233)(56.98367841,286.08976149)(56.98367841,286.18871982)
\curveto(56.98367841,286.28246981)(57.01625363,286.3605948)(57.08140408,286.4230948)
\curveto(57.14655452,286.48559479)(57.25430333,286.51684479)(57.40465051,286.51684479)
\closepath
\moveto(64.36572494,280.07934528)
\curveto(64.36572494,280.90747022)(64.09259423,281.607991)(63.54633281,282.18090762)
\curveto(63.00007139,282.75903258)(62.36109588,283.04809506)(61.62940627,283.04809506)
\curveto(60.89771666,283.04809506)(60.25874115,282.75903258)(59.71247973,282.18090762)
\curveto(59.16621831,281.607991)(58.8930876,280.90747022)(58.8930876,280.07934528)
\curveto(58.8930876,279.25122034)(59.16621831,278.5480954)(59.71247973,277.96997044)
\curveto(60.25874115,277.39705382)(60.89771666,277.11059551)(61.62940627,277.11059551)
\curveto(62.36109588,277.11059551)(63.00007139,277.39705382)(63.54633281,277.96997044)
\curveto(64.09259423,278.5480954)(64.36572494,279.25122034)(64.36572494,280.07934528)
\closepath
}
}
{
\newrgbcolor{curcolor}{0 0 0}
\pscustom[linestyle=none,fillstyle=solid,fillcolor=curcolor]
{
\newpath
\moveto(72.66488529,276.71215804)
\lineto(72.66488529,277.67309546)
\curveto(71.80289479,276.86059552)(70.87074228,276.45434556)(69.86842775,276.45434556)
\curveto(69.25200431,276.45434556)(68.78342226,276.62882471)(68.46268161,276.97778302)
\curveto(68.04672108,277.43611631)(67.83874082,277.96997044)(67.83874082,278.57934539)
\lineto(67.83874082,282.81372007)
\lineto(66.98927925,282.81372007)
\curveto(66.83893207,282.81372007)(66.73118326,282.84497007)(66.66603281,282.90747007)
\curveto(66.60088237,282.96997006)(66.56830715,283.05069922)(66.56830715,283.14965755)
\curveto(66.56830715,283.24340754)(66.60088237,283.32153253)(66.66603281,283.38403253)
\curveto(66.73118326,283.44653252)(66.83893207,283.47778252)(66.98927925,283.47778252)
\lineto(68.47019897,283.47778252)
\lineto(68.47019897,278.57934539)
\curveto(68.47019897,278.15226209)(68.60049986,277.80069962)(68.86110164,277.52465797)
\curveto(69.12170342,277.24861633)(69.44745564,277.11059551)(69.83835831,277.11059551)
\curveto(70.86573071,277.11059551)(71.80790637,277.6001788)(72.66488529,278.57934539)
\lineto(72.66488529,282.81372007)
\lineto(71.49969465,282.81372007)
\curveto(71.34934747,282.81372007)(71.24159866,282.84497007)(71.17644821,282.90747007)
\curveto(71.11129777,282.96997006)(71.07872254,283.05069922)(71.07872254,283.14965755)
\curveto(71.07872254,283.24340754)(71.11129777,283.32153253)(71.17644821,283.38403253)
\curveto(71.24159866,283.44653252)(71.34934747,283.47778252)(71.49969465,283.47778252)
\lineto(73.29634345,283.47778252)
\lineto(73.29634345,277.36840799)
\lineto(73.83007594,277.36840799)
\curveto(73.98042312,277.36840799)(74.08817193,277.33715799)(74.15332237,277.27465799)
\curveto(74.21847282,277.21736633)(74.25104804,277.13924134)(74.25104804,277.04028301)
\curveto(74.25104804,276.94653302)(74.21847282,276.86840802)(74.15332237,276.80590803)
\curveto(74.08817193,276.74340803)(73.98042312,276.71215804)(73.83007594,276.71215804)
\closepath
}
}
{
\newrgbcolor{curcolor}{0 0 0}
\pscustom[linestyle=none,fillstyle=solid,fillcolor=curcolor]
{
\newpath
\moveto(79.32526845,282.81372007)
\lineto(79.32526845,277.36840799)
\lineto(82.09917392,277.36840799)
\curveto(82.2495211,277.36840799)(82.35726991,277.33715799)(82.42242036,277.27465799)
\curveto(82.4875708,277.21736633)(82.52014602,277.13924134)(82.52014602,277.04028301)
\curveto(82.52014602,276.94653302)(82.4875708,276.86840802)(82.42242036,276.80590803)
\curveto(82.35726991,276.74340803)(82.2495211,276.71215804)(82.09917392,276.71215804)
\lineto(77.16026906,276.71215804)
\curveto(77.00992188,276.71215804)(76.90217307,276.74340803)(76.83702262,276.80590803)
\curveto(76.77187218,276.86840802)(76.73929696,276.94653302)(76.73929696,277.04028301)
\curveto(76.73929696,277.13924134)(76.77187218,277.21736633)(76.83702262,277.27465799)
\curveto(76.90217307,277.33715799)(77.00992188,277.36840799)(77.16026906,277.36840799)
\lineto(78.68629294,277.36840799)
\lineto(78.68629294,282.81372007)
\lineto(77.3181336,282.81372007)
\curveto(77.16778642,282.81372007)(77.06003761,282.84497007)(76.99488716,282.90747007)
\curveto(76.92973672,282.96997006)(76.89716149,283.05069922)(76.89716149,283.14965755)
\curveto(76.89716149,283.24340754)(76.92973672,283.32153253)(76.99488716,283.38403253)
\curveto(77.06003761,283.44653252)(77.16778642,283.47778252)(77.3181336,283.47778252)
\lineto(78.68629294,283.47778252)
\lineto(78.68629294,284.46996995)
\curveto(78.68629294,285.02205324)(78.90179056,285.50121987)(79.33278581,285.90746984)
\curveto(79.76378106,286.31371981)(80.33510034,286.51684479)(81.04674366,286.51684479)
\curveto(81.64312081,286.51684479)(82.27959053,286.45955313)(82.95615284,286.34496981)
\curveto(83.21174305,286.30330314)(83.36459602,286.25382398)(83.41471174,286.19653232)
\curveto(83.46983904,286.13924065)(83.49740269,286.06371983)(83.49740269,285.96996983)
\curveto(83.49740269,285.87621984)(83.46733326,285.79809485)(83.40719438,285.73559485)
\curveto(83.34705551,285.67830319)(83.26687035,285.64965736)(83.1666389,285.64965736)
\curveto(83.12654631,285.64965736)(83.05889008,285.65746986)(82.9636702,285.67309486)
\curveto(82.20692273,285.79288651)(81.56794722,285.85278234)(81.04674366,285.85278234)
\curveto(80.49547067,285.85278234)(80.06948699,285.71215735)(79.76879263,285.43090737)
\curveto(79.47310984,285.1496574)(79.32526845,284.82934492)(79.32526845,284.46996995)
\lineto(79.32526845,283.47778252)
\lineto(82.27959053,283.47778252)
\curveto(82.42993771,283.47778252)(82.53768653,283.44653252)(82.60283697,283.38403253)
\curveto(82.66798742,283.32153253)(82.70056264,283.24080337)(82.70056264,283.14184505)
\curveto(82.70056264,283.04809506)(82.66798742,282.96997006)(82.60283697,282.90747007)
\curveto(82.53768653,282.84497007)(82.42993771,282.81372007)(82.27959053,282.81372007)
\closepath
}
}
{
\newrgbcolor{curcolor}{0 0 0}
\pscustom[linestyle=none,fillstyle=solid,fillcolor=curcolor]
{
\newpath
\moveto(93.17976417,272.32934587)
\lineto(84.78287417,272.32934587)
\curveto(84.63252699,272.32934587)(84.52477818,272.36059587)(84.45962773,272.42309586)
\curveto(84.39447729,272.48038752)(84.36190207,272.55851252)(84.36190207,272.65747084)
\curveto(84.36190207,272.75642917)(84.39447729,272.83715833)(84.45962773,272.89965832)
\curveto(84.52477818,272.95694999)(84.63252699,272.98559582)(84.78287417,272.98559582)
\lineto(93.17976417,272.98559582)
\curveto(93.33512292,272.98559582)(93.44287173,272.95694999)(93.5030106,272.89965832)
\curveto(93.56816105,272.83715833)(93.60073627,272.75642917)(93.60073627,272.65747084)
\curveto(93.60073627,272.55851252)(93.56816105,272.48038752)(93.5030106,272.42309586)
\curveto(93.44287173,272.36059587)(93.33512292,272.32934587)(93.17976417,272.32934587)
\closepath
}
}
{
\newrgbcolor{curcolor}{0 0 0}
\pscustom[linestyle=none,fillstyle=solid,fillcolor=curcolor]
{
\newpath
\moveto(96.14159963,286.51684479)
\lineto(96.14159963,282.4777826)
\curveto(96.54252544,282.93090756)(96.92591075,283.24861587)(97.29175555,283.43090753)
\curveto(97.66261193,283.61840751)(98.07606668,283.7121575)(98.53211979,283.7121575)
\curveto(99.02325391,283.7121575)(99.43921444,283.62101168)(99.78000138,283.43872003)
\curveto(100.1257999,283.26163671)(100.41396532,282.98559506)(100.64449767,282.61059509)
\curveto(100.87503001,282.24080345)(100.99029618,281.85017848)(100.99029618,281.43872018)
\lineto(100.99029618,277.36840799)
\lineto(101.69692793,277.36840799)
\curveto(101.85228668,277.36840799)(101.96003549,277.33715799)(102.02017436,277.27465799)
\curveto(102.08532481,277.21736633)(102.11790003,277.13924134)(102.11790003,277.04028301)
\curveto(102.11790003,276.94653302)(102.08532481,276.86840802)(102.02017436,276.80590803)
\curveto(101.96003549,276.74340803)(101.85228668,276.71215804)(101.69692793,276.71215804)
\lineto(99.64468892,276.71215804)
\curveto(99.48933017,276.71215804)(99.37907557,276.74340803)(99.31392512,276.80590803)
\curveto(99.24877468,276.86840802)(99.21619946,276.94653302)(99.21619946,277.04028301)
\curveto(99.21619946,277.13924134)(99.24877468,277.21736633)(99.31392512,277.27465799)
\curveto(99.37907557,277.33715799)(99.48933017,277.36840799)(99.64468892,277.36840799)
\lineto(100.35132067,277.36840799)
\lineto(100.35132067,281.39184518)
\curveto(100.35132067,281.86580348)(100.18593877,282.26163678)(99.85517497,282.57934509)
\curveto(99.52942275,282.8970534)(99.07336964,283.05590755)(98.48701563,283.05590755)
\curveto(98.02595095,283.05590755)(97.6325425,282.93872006)(97.30679027,282.70434508)
\curveto(97.07124636,282.53767843)(96.68284948,282.14444929)(96.14159963,281.52465767)
\lineto(96.14159963,277.36840799)
\lineto(96.85574873,277.36840799)
\curveto(97.00609591,277.36840799)(97.11384472,277.33715799)(97.17899517,277.27465799)
\curveto(97.24414561,277.21736633)(97.27672084,277.13924134)(97.27672084,277.04028301)
\curveto(97.27672084,276.94653302)(97.24414561,276.86840802)(97.17899517,276.80590803)
\curveto(97.11384472,276.74340803)(97.00609591,276.71215804)(96.85574873,276.71215804)
\lineto(94.79599237,276.71215804)
\curveto(94.64564519,276.71215804)(94.53789638,276.74340803)(94.47274593,276.80590803)
\curveto(94.40759549,276.86840802)(94.37502026,276.94653302)(94.37502026,277.04028301)
\curveto(94.37502026,277.13924134)(94.40759549,277.21736633)(94.47274593,277.27465799)
\curveto(94.53789638,277.33715799)(94.64564519,277.36840799)(94.79599237,277.36840799)
\lineto(95.51014147,277.36840799)
\lineto(95.51014147,285.85278234)
\lineto(94.66067991,285.85278234)
\curveto(94.51033273,285.85278234)(94.40258391,285.88403234)(94.33743347,285.94653234)
\curveto(94.27228302,286.00903233)(94.2397078,286.08976149)(94.2397078,286.18871982)
\curveto(94.2397078,286.28246981)(94.27228302,286.3605948)(94.33743347,286.4230948)
\curveto(94.40258391,286.48559479)(94.51033273,286.51684479)(94.66067991,286.51684479)
\closepath
}
}
{
\newrgbcolor{curcolor}{0 0 0}
\pscustom[linestyle=none,fillstyle=solid,fillcolor=curcolor]
{
\newpath
\moveto(110.55237991,286.51684479)
\lineto(110.55237991,277.36840799)
\lineto(111.39432412,277.36840799)
\curveto(111.54968287,277.36840799)(111.65993747,277.33715799)(111.72508791,277.27465799)
\curveto(111.79023836,277.21736633)(111.82281358,277.13924134)(111.82281358,277.04028301)
\curveto(111.82281358,276.94653302)(111.79023836,276.86840802)(111.72508791,276.80590803)
\curveto(111.65993747,276.74340803)(111.54968287,276.71215804)(111.39432412,276.71215804)
\lineto(109.9134044,276.71215804)
\lineto(109.9134044,278.01684544)
\curveto(109.18171479,276.97517885)(108.25206806,276.45434556)(107.12446421,276.45434556)
\curveto(106.55314493,276.45434556)(106.00437772,276.61059554)(105.47816259,276.92309552)
\curveto(104.95695903,277.24080383)(104.54350429,277.69132463)(104.23779835,278.27465792)
\curveto(103.937104,278.85799121)(103.78675682,279.45955366)(103.78675682,280.07934528)
\curveto(103.78675682,280.70434523)(103.937104,281.30590769)(104.23779835,281.88403264)
\curveto(104.54350429,282.46736593)(104.95695903,282.91788673)(105.47816259,283.23559504)
\curveto(106.00437772,283.55330335)(106.55565071,283.7121575)(107.13198157,283.7121575)
\curveto(108.23452755,283.7121575)(109.1616685,283.19132421)(109.9134044,282.14965762)
\lineto(109.9134044,285.85278234)
\lineto(109.07146019,285.85278234)
\curveto(108.91610144,285.85278234)(108.80584684,285.88403234)(108.74069639,285.94653234)
\curveto(108.67554595,286.00903233)(108.64297073,286.08976149)(108.64297073,286.18871982)
\curveto(108.64297073,286.28246981)(108.67554595,286.3605948)(108.74069639,286.4230948)
\curveto(108.80584684,286.48559479)(108.91610144,286.51684479)(109.07146019,286.51684479)
\closepath
\moveto(109.9134044,280.07934528)
\curveto(109.9134044,280.91267855)(109.64277947,281.6158035)(109.10152962,282.18872012)
\curveto(108.56027978,282.76163674)(107.91629269,283.04809506)(107.16956836,283.04809506)
\curveto(106.41783246,283.04809506)(105.77133959,282.76163674)(105.23008974,282.18872012)
\curveto(104.68883989,281.6158035)(104.41821497,280.91267855)(104.41821497,280.07934528)
\curveto(104.41821497,279.25122034)(104.68883989,278.5480954)(105.23008974,277.96997044)
\curveto(105.77133959,277.39705382)(106.41783246,277.11059551)(107.16956836,277.11059551)
\curveto(107.91629269,277.11059551)(108.56027978,277.39705382)(109.10152962,277.96997044)
\curveto(109.64277947,278.5480954)(109.9134044,279.25122034)(109.9134044,280.07934528)
\closepath
}
}
{
\newrgbcolor{curcolor}{0 0 0}
\pscustom[linestyle=none,fillstyle=solid,fillcolor=curcolor]
{
\newpath
\moveto(115.9423294,283.47778252)
\lineto(115.9423294,281.82153265)
\curveto(116.76422732,282.59236592)(117.37814497,283.08715755)(117.78408235,283.30590754)
\curveto(118.19503131,283.52986585)(118.57340505,283.64184501)(118.91920356,283.64184501)
\curveto(119.29507151,283.64184501)(119.64337581,283.50903252)(119.96411646,283.24340754)
\curveto(120.28986868,282.98299089)(120.4527448,282.78507424)(120.4527448,282.64965759)
\curveto(120.4527448,282.55069926)(120.42016957,282.46736593)(120.35501913,282.3996576)
\curveto(120.29488026,282.33715761)(120.21720088,282.30590761)(120.121981,282.30590761)
\curveto(120.07186527,282.30590761)(120.02926691,282.31372011)(119.9941859,282.32934511)
\curveto(119.95910489,282.35017844)(119.89395444,282.41007427)(119.79873456,282.5090326)
\curveto(119.62332952,282.69132425)(119.47047655,282.81632424)(119.34017566,282.88403257)
\curveto(119.20987478,282.9517409)(119.08207967,282.98559506)(118.95679036,282.98559506)
\curveto(118.68115386,282.98559506)(118.34788428,282.87101174)(117.95698161,282.64184509)
\curveto(117.57109051,282.41267844)(116.89953978,281.85278265)(115.9423294,280.96215771)
\lineto(115.9423294,277.36840799)
\lineto(118.73878694,277.36840799)
\curveto(118.8941457,277.36840799)(119.0044003,277.33715799)(119.06955074,277.27465799)
\curveto(119.13470119,277.21736633)(119.16727641,277.13924134)(119.16727641,277.04028301)
\curveto(119.16727641,276.94653302)(119.13470119,276.86840802)(119.06955074,276.80590803)
\curveto(119.0044003,276.74340803)(118.8941457,276.71215804)(118.73878694,276.71215804)
\lineto(113.78484737,276.71215804)
\curveto(113.63450019,276.71215804)(113.52675138,276.74080387)(113.46160093,276.79809553)
\curveto(113.39645049,276.86059552)(113.36387526,276.93872052)(113.36387526,277.03247051)
\curveto(113.36387526,277.12101217)(113.3939447,277.19392883)(113.45408357,277.25122049)
\curveto(113.51923402,277.31372049)(113.62948861,277.34497049)(113.78484737,277.34497049)
\lineto(115.31087124,277.34497049)
\lineto(115.31087124,282.81372007)
\lineto(114.1456806,282.81372007)
\curveto(113.99533342,282.81372007)(113.88758461,282.84497007)(113.82243416,282.90747007)
\curveto(113.75728372,282.96997006)(113.7247085,283.05069922)(113.7247085,283.14965755)
\curveto(113.7247085,283.24340754)(113.75477793,283.32153253)(113.8149168,283.38403253)
\curveto(113.88006725,283.44653252)(113.99032185,283.47778252)(114.1456806,283.47778252)
\closepath
}
}
{
\newrgbcolor{curcolor}{0 0 0}
\pscustom[linestyle=none,fillstyle=solid,fillcolor=curcolor]
{
\newpath
\moveto(130.13510627,272.32934587)
\lineto(121.73821627,272.32934587)
\curveto(121.58786909,272.32934587)(121.48012028,272.36059587)(121.41496984,272.42309586)
\curveto(121.34981939,272.48038752)(121.31724417,272.55851252)(121.31724417,272.65747084)
\curveto(121.31724417,272.75642917)(121.34981939,272.83715833)(121.41496984,272.89965832)
\curveto(121.48012028,272.95694999)(121.58786909,272.98559582)(121.73821627,272.98559582)
\lineto(130.13510627,272.98559582)
\curveto(130.29046502,272.98559582)(130.39821383,272.95694999)(130.45835271,272.89965832)
\curveto(130.52350315,272.83715833)(130.55607837,272.75642917)(130.55607837,272.65747084)
\curveto(130.55607837,272.55851252)(130.52350315,272.48038752)(130.45835271,272.42309586)
\curveto(130.39821383,272.36059587)(130.29046502,272.32934587)(130.13510627,272.32934587)
\closepath
}
}
{
\newrgbcolor{curcolor}{0 0 0}
\pscustom[linestyle=none,fillstyle=solid,fillcolor=curcolor]
{
\newpath
\moveto(133.87874,283.47778252)
\lineto(137.3066557,283.47778252)
\curveto(137.45700288,283.47778252)(137.56475169,283.44653252)(137.62990213,283.38403253)
\curveto(137.69505258,283.32153253)(137.7276278,283.24080337)(137.7276278,283.14184505)
\curveto(137.7276278,283.04809506)(137.69505258,282.96997006)(137.62990213,282.90747007)
\curveto(137.56475169,282.84497007)(137.45700288,282.81372007)(137.3066557,282.81372007)
\lineto(133.87874,282.81372007)
\lineto(133.87874,278.4465329)
\curveto(133.87874,278.0663246)(134.0240756,277.74861629)(134.31474682,277.49340798)
\curveto(134.6104296,277.23819966)(135.04142485,277.11059551)(135.60773256,277.11059551)
\curveto(136.03371624,277.11059551)(136.49478093,277.17569967)(136.99092662,277.30590799)
\curveto(137.48707231,277.44132465)(137.87296341,277.5923663)(138.1485999,277.75903296)
\curveto(138.24883136,277.82674128)(138.33152231,277.86059545)(138.39667275,277.86059545)
\curveto(138.47685791,277.86059545)(138.54701993,277.82674128)(138.6071588,277.75903296)
\curveto(138.66729768,277.69653296)(138.69736711,277.62101213)(138.69736711,277.53247047)
\curveto(138.69736711,277.45434548)(138.66479189,277.38142882)(138.59964144,277.31372049)
\curveto(138.43927112,277.1418455)(138.04836845,276.95434552)(137.42693344,276.75122053)
\curveto(136.81051,276.55330388)(136.21914443,276.45434556)(135.65283672,276.45434556)
\curveto(134.91613554,276.45434556)(134.32978154,276.63403304)(133.89377471,276.99340801)
\curveto(133.45776789,277.35278299)(133.23976448,277.83715795)(133.23976448,278.4465329)
\lineto(133.23976448,282.81372007)
\lineto(132.07457384,282.81372007)
\curveto(131.92422666,282.81372007)(131.81647785,282.84497007)(131.7513274,282.90747007)
\curveto(131.68617696,282.96997006)(131.65360173,283.05069922)(131.65360173,283.14965755)
\curveto(131.65360173,283.24340754)(131.68617696,283.32153253)(131.7513274,283.38403253)
\curveto(131.81647785,283.44653252)(131.92422666,283.47778252)(132.07457384,283.47778252)
\lineto(133.23976448,283.47778252)
\lineto(133.23976448,285.41528238)
\curveto(133.23976448,285.57153236)(133.26983392,285.68351152)(133.32997279,285.75121985)
\curveto(133.39011166,285.81892818)(133.46528525,285.85278234)(133.55549356,285.85278234)
\curveto(133.65071344,285.85278234)(133.72839282,285.81892818)(133.78853169,285.75121985)
\curveto(133.84867056,285.68351152)(133.87874,285.57153236)(133.87874,285.41528238)
\closepath
}
}
{
\newrgbcolor{curcolor}{0 0 0}
\pscustom[linestyle=none,fillstyle=solid,fillcolor=curcolor]
{
\newpath
\moveto(7.86737215,248.26907402)
\lineto(7.86737215,239.12063721)
\lineto(10.34810062,239.12063721)
\curveto(10.50345937,239.12063721)(10.61371397,239.08938721)(10.67886441,239.02688722)
\curveto(10.74401486,238.96959556)(10.77659008,238.89147056)(10.77659008,238.79251224)
\curveto(10.77659008,238.69876224)(10.74401486,238.62063725)(10.67886441,238.55813726)
\curveto(10.61371397,238.49563726)(10.50345937,238.46438726)(10.34810062,238.46438726)
\lineto(4.75518553,238.46438726)
\curveto(4.60483835,238.46438726)(4.49708953,238.49563726)(4.43193909,238.55813726)
\curveto(4.36678864,238.62063725)(4.33421342,238.69876224)(4.33421342,238.79251224)
\curveto(4.33421342,238.89147056)(4.36678864,238.96959556)(4.43193909,239.02688722)
\curveto(4.49708953,239.08938721)(4.60483835,239.12063721)(4.75518553,239.12063721)
\lineto(7.23591399,239.12063721)
\lineto(7.23591399,247.60501157)
\lineto(5.41671312,247.60501157)
\curveto(5.26636594,247.60501157)(5.15611134,247.63626157)(5.08594932,247.69876156)
\curveto(5.02079888,247.76126156)(4.98822365,247.84199072)(4.98822365,247.94094904)
\curveto(4.98822365,248.03469904)(5.02079888,248.11282403)(5.08594932,248.17532403)
\curveto(5.15109977,248.23782402)(5.26135436,248.26907402)(5.41671312,248.26907402)
\closepath
}
}
{
\newrgbcolor{curcolor}{0 0 0}
\pscustom[linestyle=none,fillstyle=solid,fillcolor=curcolor]
{
\newpath
\moveto(17.09868943,248.40188651)
\lineto(17.09868943,239.12063721)
\lineto(19.26368882,239.12063721)
\curveto(19.414036,239.12063721)(19.52178481,239.08938721)(19.58693526,239.02688722)
\curveto(19.6520857,238.96959556)(19.68466093,238.89147056)(19.68466093,238.79251224)
\curveto(19.68466093,238.69876224)(19.6520857,238.62063725)(19.58693526,238.55813726)
\curveto(19.52178481,238.49563726)(19.414036,238.46438726)(19.26368882,238.46438726)
\lineto(14.30223189,238.46438726)
\curveto(14.15188471,238.46438726)(14.04413589,238.49563726)(13.97898545,238.55813726)
\curveto(13.913835,238.62063725)(13.88125978,238.69876224)(13.88125978,238.79251224)
\curveto(13.88125978,238.89147056)(13.913835,238.96959556)(13.97898545,239.02688722)
\curveto(14.04413589,239.08938721)(14.15188471,239.12063721)(14.30223189,239.12063721)
\lineto(16.46723128,239.12063721)
\lineto(16.46723128,247.49563658)
\lineto(14.40747491,246.82376163)
\curveto(14.30724346,246.79251163)(14.23457565,246.77688663)(14.1894715,246.77688663)
\curveto(14.11429791,246.77688663)(14.04413589,246.80813663)(13.97898545,246.87063662)
\curveto(13.91884658,246.93834495)(13.88877714,247.01646995)(13.88877714,247.10501161)
\curveto(13.88877714,247.1831366)(13.913835,247.25605326)(13.96395073,247.32376159)
\curveto(14.00404331,247.36542825)(14.08673426,247.40969908)(14.21202358,247.45657408)
\closepath
}
}
{
\newrgbcolor{curcolor}{0 0 0}
\pscustom[linestyle=none,fillstyle=solid,fillcolor=curcolor]
{
\newpath
\moveto(27.84099587,238.46438726)
\lineto(27.84099587,239.40969969)
\curveto(26.92387808,238.60761642)(25.94411562,238.20657478)(24.90170851,238.20657478)
\curveto(24.14496103,238.20657478)(23.55359546,238.40449143)(23.12761178,238.80032474)
\curveto(22.70162811,239.20136637)(22.48863627,239.69094967)(22.48863627,240.26907463)
\curveto(22.48863627,240.90449124)(22.76928434,241.4591787)(23.33058048,241.933137)
\curveto(23.89187661,242.4070953)(24.71126874,242.64407445)(25.78875687,242.64407445)
\curveto(26.07942808,242.64407445)(26.39515716,242.62324111)(26.7359441,242.58157445)
\curveto(27.07673104,242.54511612)(27.44508163,242.48522029)(27.84099587,242.40188696)
\lineto(27.84099587,243.46438688)
\curveto(27.84099587,243.82376186)(27.68062555,244.13626183)(27.3598849,244.40188681)
\curveto(27.03914425,244.66751179)(26.55803327,244.80032428)(25.91655197,244.80032428)
\curveto(25.42541785,244.80032428)(24.73632661,244.65188679)(23.84927825,244.35501182)
\curveto(23.68890792,244.30292849)(23.58617068,244.27688682)(23.54106653,244.27688682)
\curveto(23.46088137,244.27688682)(23.39071935,244.30813682)(23.33058048,244.37063681)
\curveto(23.27545318,244.43313681)(23.24788953,244.5112618)(23.24788953,244.6050118)
\curveto(23.24788953,244.69355346)(23.27294739,244.76386595)(23.32306312,244.81594928)
\curveto(23.39322513,244.89407427)(23.67637899,245.0008451)(24.17252468,245.13626176)
\curveto(24.95433002,245.35501174)(25.54569559,245.46438673)(25.94662141,245.46438673)
\curveto(26.74346146,245.46438673)(27.36489647,245.25865758)(27.81092644,244.84719928)
\curveto(28.2569564,244.44094931)(28.47997139,243.98001184)(28.47997139,243.46438688)
\lineto(28.47997139,239.12063721)
\lineto(29.32191559,239.12063721)
\curveto(29.47727435,239.12063721)(29.58752895,239.08938721)(29.65267939,239.02688722)
\curveto(29.71782983,238.96959556)(29.75040506,238.89147056)(29.75040506,238.79251224)
\curveto(29.75040506,238.69876224)(29.71782983,238.62063725)(29.65267939,238.55813726)
\curveto(29.58752895,238.49563726)(29.47727435,238.46438726)(29.32191559,238.46438726)
\closepath
\moveto(27.84099587,241.73001201)
\curveto(27.54531309,241.81855367)(27.23208979,241.88365784)(26.901326,241.9253245)
\curveto(26.5705622,241.96699116)(26.2222579,241.98782449)(25.8564131,241.98782449)
\curveto(24.9392953,241.98782449)(24.22264041,241.78209534)(23.70644843,241.37063704)
\curveto(23.31554576,241.0633454)(23.12009442,240.69615793)(23.12009442,240.26907463)
\curveto(23.12009442,239.87324132)(23.26793582,239.53990801)(23.5636186,239.2690747)
\curveto(23.86431296,238.99824139)(24.30031979,238.86282473)(24.87163907,238.86282473)
\curveto(25.41790049,238.86282473)(25.92406933,238.97480389)(26.39014559,239.19876221)
\curveto(26.86123342,239.42792886)(27.34485018,239.78990799)(27.84099587,240.28469962)
\closepath
}
}
{
\newrgbcolor{curcolor}{0 0 0}
\pscustom[linestyle=none,fillstyle=solid,fillcolor=curcolor]
{
\newpath
\moveto(34.50889462,245.23001175)
\lineto(34.50889462,243.57376187)
\curveto(35.33079254,244.34459515)(35.94471019,244.83938678)(36.35064758,245.05813676)
\curveto(36.76159653,245.28209508)(37.13997027,245.39407424)(37.48576878,245.39407424)
\curveto(37.86163673,245.39407424)(38.20994103,245.26126175)(38.53068168,244.99563677)
\curveto(38.85643391,244.73522012)(39.01931002,244.53730347)(39.01931002,244.40188681)
\curveto(39.01931002,244.30292849)(38.9867348,244.21959516)(38.92158435,244.15188683)
\curveto(38.86144548,244.08938684)(38.7837661,244.05813684)(38.68854622,244.05813684)
\curveto(38.6384305,244.05813684)(38.59583213,244.06594934)(38.56075112,244.08157434)
\curveto(38.52567011,244.10240767)(38.46051967,244.1623035)(38.36529979,244.26126182)
\curveto(38.18989474,244.44355348)(38.03704178,244.56855347)(37.90674089,244.63626179)
\curveto(37.77644,244.70397012)(37.6486449,244.73782429)(37.52335558,244.73782429)
\curveto(37.24771908,244.73782429)(36.9144495,244.62324096)(36.52354683,244.39407431)
\curveto(36.13765574,244.16490766)(35.466105,243.60501187)(34.50889462,242.71438694)
\lineto(34.50889462,239.12063721)
\lineto(37.30535217,239.12063721)
\curveto(37.46071092,239.12063721)(37.57096552,239.08938721)(37.63611596,239.02688722)
\curveto(37.70126641,238.96959556)(37.73384163,238.89147056)(37.73384163,238.79251224)
\curveto(37.73384163,238.69876224)(37.70126641,238.62063725)(37.63611596,238.55813726)
\curveto(37.57096552,238.49563726)(37.46071092,238.46438726)(37.30535217,238.46438726)
\lineto(32.35141259,238.46438726)
\curveto(32.20106541,238.46438726)(32.0933166,238.49303309)(32.02816615,238.55032476)
\curveto(31.96301571,238.61282475)(31.93044049,238.69094974)(31.93044049,238.78469974)
\curveto(31.93044049,238.8732414)(31.96050992,238.94615806)(32.0206488,239.00344972)
\curveto(32.08579924,239.06594972)(32.19605384,239.09719971)(32.35141259,239.09719971)
\lineto(33.87743647,239.09719971)
\lineto(33.87743647,244.5659493)
\lineto(32.71224582,244.5659493)
\curveto(32.56189864,244.5659493)(32.45414983,244.5971993)(32.38899939,244.65969929)
\curveto(32.32384894,244.72219929)(32.29127372,244.80292845)(32.29127372,244.90188677)
\curveto(32.29127372,244.99563677)(32.32134316,245.07376176)(32.38148203,245.13626176)
\curveto(32.44663247,245.19876175)(32.55688707,245.23001175)(32.71224582,245.23001175)
\closepath
}
}
{
\newrgbcolor{curcolor}{0 0 0}
\pscustom[linestyle=none,fillstyle=solid,fillcolor=curcolor]
{
\newpath
\moveto(47.13805729,244.5659493)
\lineto(47.13805729,244.78469928)
\curveto(47.13805729,244.9461576)(47.16812672,245.06074093)(47.22826559,245.12844926)
\curveto(47.28840447,245.19615758)(47.36357806,245.23001175)(47.45378636,245.23001175)
\curveto(47.54900625,245.23001175)(47.62668562,245.19615758)(47.68682449,245.12844926)
\curveto(47.74696337,245.06074093)(47.7770328,244.9461576)(47.7770328,244.78469928)
\lineto(47.7770328,243.3003244)
\curveto(47.77202123,243.13886607)(47.73944601,243.02428275)(47.67930713,242.95657442)
\curveto(47.62417984,242.88886609)(47.54900625,242.85501193)(47.45378636,242.85501193)
\curveto(47.36858963,242.85501193)(47.29592183,242.88365776)(47.23578295,242.94094942)
\curveto(47.18065565,243.00344942)(47.14808043,243.10501191)(47.13805729,243.2456369)
\curveto(47.10798785,243.61542854)(46.87244394,243.96699101)(46.43142554,244.30032432)
\curveto(45.99541872,244.63365763)(45.40655893,244.80032428)(44.66484618,244.80032428)
\curveto(43.72768209,244.80032428)(43.01603877,244.4956368)(42.52991622,243.88626185)
\curveto(42.04379368,243.2768869)(41.8007324,242.57897028)(41.8007324,241.79251201)
\curveto(41.8007324,240.94355374)(42.06885154,240.24303296)(42.60508981,239.69094967)
\curveto(43.14132809,239.13886638)(43.8354309,238.86282473)(44.68739825,238.86282473)
\curveto(45.17853238,238.86282473)(45.67718386,238.95657472)(46.18335269,239.14407471)
\curveto(46.69453311,239.3315747)(47.15559779,239.63365801)(47.56654675,240.05032464)
\curveto(47.67178978,240.1544913)(47.76450387,240.20657463)(47.84468903,240.20657463)
\curveto(47.92988577,240.20657463)(48.00004779,240.17532463)(48.05517508,240.11282464)
\curveto(48.11531396,240.05553297)(48.14538339,239.98261631)(48.14538339,239.89407465)
\curveto(48.14538339,239.67011634)(47.89229897,239.38626219)(47.38613013,239.04251222)
\curveto(46.56924379,238.48522059)(45.65964335,238.20657478)(44.65732882,238.20657478)
\curveto(43.63997957,238.20657478)(42.80304693,238.54251226)(42.14653091,239.21438721)
\curveto(41.49502647,239.89147049)(41.16927425,240.74824126)(41.16927425,241.78469951)
\curveto(41.16927425,242.8419911)(41.50254383,243.7195952)(42.16908299,244.41751181)
\curveto(42.84063373,245.11542842)(43.68508372,245.46438673)(44.70243297,245.46438673)
\curveto(45.6696665,245.46438673)(46.48154127,245.16490759)(47.13805729,244.5659493)
\closepath
}
}
{
\newrgbcolor{curcolor}{0 0 0}
\pscustom[linestyle=none,fillstyle=solid,fillcolor=curcolor]
{
\newpath
\moveto(57.94050172,234.08157509)
\lineto(49.54361172,234.08157509)
\curveto(49.39326454,234.08157509)(49.28551573,234.11282509)(49.22036528,234.17532509)
\curveto(49.15521484,234.23261675)(49.12263962,234.31074174)(49.12263962,234.40970007)
\curveto(49.12263962,234.5086584)(49.15521484,234.58938756)(49.22036528,234.65188755)
\curveto(49.28551573,234.70917921)(49.39326454,234.73782504)(49.54361172,234.73782504)
\lineto(57.94050172,234.73782504)
\curveto(58.09586047,234.73782504)(58.20360928,234.70917921)(58.26374815,234.65188755)
\curveto(58.3288986,234.58938756)(58.36147382,234.5086584)(58.36147382,234.40970007)
\curveto(58.36147382,234.31074174)(58.3288986,234.23261675)(58.26374815,234.17532509)
\curveto(58.20360928,234.11282509)(58.09586047,234.08157509)(57.94050172,234.08157509)
\closepath
}
}
{
\newrgbcolor{curcolor}{0 0 0}
\pscustom[linestyle=none,fillstyle=solid,fillcolor=curcolor]
{
\newpath
\moveto(60.60164989,248.26907402)
\lineto(60.60164989,243.91751185)
\curveto(61.36340893,244.94876177)(62.28303252,245.46438673)(63.36052064,245.46438673)
\curveto(64.28265001,245.46438673)(65.0719727,245.11542842)(65.72848872,244.41751181)
\curveto(66.38500474,243.72480353)(66.71326275,242.87324109)(66.71326275,241.8628245)
\curveto(66.71326275,240.84199125)(66.37999317,239.97740798)(65.71345401,239.2690747)
\curveto(65.05192641,238.56074142)(64.26761529,238.20657478)(63.36052064,238.20657478)
\curveto(62.25797465,238.20657478)(61.33835107,238.72219974)(60.60164989,239.75344966)
\lineto(60.60164989,238.46438726)
\lineto(59.12073017,238.46438726)
\curveto(58.97038299,238.46438726)(58.86263418,238.49563726)(58.79748373,238.55813726)
\curveto(58.73233329,238.62063725)(58.69975806,238.69876224)(58.69975806,238.79251224)
\curveto(58.69975806,238.89147056)(58.73233329,238.96959556)(58.79748373,239.02688722)
\curveto(58.86263418,239.08938721)(58.97038299,239.12063721)(59.12073017,239.12063721)
\lineto(59.97019173,239.12063721)
\lineto(59.97019173,247.60501157)
\lineto(59.12073017,247.60501157)
\curveto(58.97038299,247.60501157)(58.86263418,247.63626157)(58.79748373,247.69876156)
\curveto(58.73233329,247.76126156)(58.69975806,247.84199072)(58.69975806,247.94094904)
\curveto(58.69975806,248.03469904)(58.73233329,248.11282403)(58.79748373,248.17532403)
\curveto(58.86263418,248.23782402)(58.97038299,248.26907402)(59.12073017,248.26907402)
\closepath
\moveto(66.0818046,241.83157451)
\curveto(66.0818046,242.65969944)(65.80867389,243.36022022)(65.26241247,243.93313685)
\curveto(64.71615105,244.5112618)(64.07717553,244.80032428)(63.34548592,244.80032428)
\curveto(62.61379631,244.80032428)(61.9748208,244.5112618)(61.42855938,243.93313685)
\curveto(60.88229796,243.36022022)(60.60916725,242.65969944)(60.60916725,241.83157451)
\curveto(60.60916725,241.00344957)(60.88229796,240.30032462)(61.42855938,239.72219967)
\curveto(61.9748208,239.14928304)(62.61379631,238.86282473)(63.34548592,238.86282473)
\curveto(64.07717553,238.86282473)(64.71615105,239.14928304)(65.26241247,239.72219967)
\curveto(65.80867389,240.30032462)(66.0818046,241.00344957)(66.0818046,241.83157451)
\closepath
}
}
{
\newrgbcolor{curcolor}{0 0 0}
\pscustom[linestyle=none,fillstyle=solid,fillcolor=curcolor]
{
\newpath
\moveto(74.38096494,238.46438726)
\lineto(74.38096494,239.42532469)
\curveto(73.51897445,238.61282475)(72.58682193,238.20657478)(71.5845074,238.20657478)
\curveto(70.96808396,238.20657478)(70.49950192,238.38105394)(70.17876127,238.73001224)
\curveto(69.76280074,239.18834554)(69.55482047,239.72219967)(69.55482047,240.33157462)
\lineto(69.55482047,244.5659493)
\lineto(68.7053589,244.5659493)
\curveto(68.55501172,244.5659493)(68.44726291,244.5971993)(68.38211247,244.65969929)
\curveto(68.31696202,244.72219929)(68.2843868,244.80292845)(68.2843868,244.90188677)
\curveto(68.2843868,244.99563677)(68.31696202,245.07376176)(68.38211247,245.13626176)
\curveto(68.44726291,245.19876175)(68.55501172,245.23001175)(68.7053589,245.23001175)
\lineto(70.18627863,245.23001175)
\lineto(70.18627863,240.33157462)
\curveto(70.18627863,239.90449132)(70.31657951,239.55292885)(70.57718129,239.2768872)
\curveto(70.83778307,239.00084555)(71.16353529,238.86282473)(71.55443796,238.86282473)
\curveto(72.58181036,238.86282473)(73.52398602,239.35240803)(74.38096494,240.33157462)
\lineto(74.38096494,244.5659493)
\lineto(73.2157743,244.5659493)
\curveto(73.06542712,244.5659493)(72.95767831,244.5971993)(72.89252786,244.65969929)
\curveto(72.82737742,244.72219929)(72.7948022,244.80292845)(72.7948022,244.90188677)
\curveto(72.7948022,244.99563677)(72.82737742,245.07376176)(72.89252786,245.13626176)
\curveto(72.95767831,245.19876175)(73.06542712,245.23001175)(73.2157743,245.23001175)
\lineto(75.0124231,245.23001175)
\lineto(75.0124231,239.12063721)
\lineto(75.54615559,239.12063721)
\curveto(75.69650277,239.12063721)(75.80425158,239.08938721)(75.86940203,239.02688722)
\curveto(75.93455247,238.96959556)(75.96712769,238.89147056)(75.96712769,238.79251224)
\curveto(75.96712769,238.69876224)(75.93455247,238.62063725)(75.86940203,238.55813726)
\curveto(75.80425158,238.49563726)(75.69650277,238.46438726)(75.54615559,238.46438726)
\closepath
}
}
{
\newrgbcolor{curcolor}{0 0 0}
\pscustom[linestyle=none,fillstyle=solid,fillcolor=curcolor]
{
\newpath
\moveto(81.0413481,244.5659493)
\lineto(81.0413481,239.12063721)
\lineto(83.81525357,239.12063721)
\curveto(83.96560075,239.12063721)(84.07334956,239.08938721)(84.13850001,239.02688722)
\curveto(84.20365045,238.96959556)(84.23622568,238.89147056)(84.23622568,238.79251224)
\curveto(84.23622568,238.69876224)(84.20365045,238.62063725)(84.13850001,238.55813726)
\curveto(84.07334956,238.49563726)(83.96560075,238.46438726)(83.81525357,238.46438726)
\lineto(78.87634871,238.46438726)
\curveto(78.72600153,238.46438726)(78.61825272,238.49563726)(78.55310228,238.55813726)
\curveto(78.48795183,238.62063725)(78.45537661,238.69876224)(78.45537661,238.79251224)
\curveto(78.45537661,238.89147056)(78.48795183,238.96959556)(78.55310228,239.02688722)
\curveto(78.61825272,239.08938721)(78.72600153,239.12063721)(78.87634871,239.12063721)
\lineto(80.40237259,239.12063721)
\lineto(80.40237259,244.5659493)
\lineto(79.03421325,244.5659493)
\curveto(78.88386607,244.5659493)(78.77611726,244.5971993)(78.71096681,244.65969929)
\curveto(78.64581637,244.72219929)(78.61324115,244.80292845)(78.61324115,244.90188677)
\curveto(78.61324115,244.99563677)(78.64581637,245.07376176)(78.71096681,245.13626176)
\curveto(78.77611726,245.19876175)(78.88386607,245.23001175)(79.03421325,245.23001175)
\lineto(80.40237259,245.23001175)
\lineto(80.40237259,246.22219917)
\curveto(80.40237259,246.77428247)(80.61787021,247.2534491)(81.04886546,247.65969906)
\curveto(81.47986071,248.06594903)(82.05117999,248.26907402)(82.76282331,248.26907402)
\curveto(83.35920046,248.26907402)(83.99567019,248.21178236)(84.6722325,248.09719903)
\curveto(84.9278227,248.05553237)(85.08067567,248.00605321)(85.1307914,247.94876154)
\curveto(85.1859187,247.89146988)(85.21348235,247.81594905)(85.21348235,247.72219906)
\curveto(85.21348235,247.62844907)(85.18341291,247.55032407)(85.12327404,247.48782408)
\curveto(85.06313517,247.43053242)(84.98295,247.40188658)(84.88271855,247.40188658)
\curveto(84.84262597,247.40188658)(84.77496974,247.40969908)(84.67974986,247.42532408)
\curveto(83.92300238,247.54511574)(83.28402687,247.60501157)(82.76282331,247.60501157)
\curveto(82.21155032,247.60501157)(81.78556664,247.46438658)(81.48487228,247.1831366)
\curveto(81.1891895,246.90188662)(81.0413481,246.58157415)(81.0413481,246.22219917)
\lineto(81.0413481,245.23001175)
\lineto(83.99567019,245.23001175)
\curveto(84.14601737,245.23001175)(84.25376618,245.19876175)(84.31891662,245.13626176)
\curveto(84.38406707,245.07376176)(84.41664229,244.9930326)(84.41664229,244.89407427)
\curveto(84.41664229,244.80032428)(84.38406707,244.72219929)(84.31891662,244.65969929)
\curveto(84.25376618,244.5971993)(84.14601737,244.5659493)(83.99567019,244.5659493)
\closepath
}
}
{
\newrgbcolor{curcolor}{0 0 0}
\pscustom[linestyle=none,fillstyle=solid,fillcolor=curcolor]
{
\newpath
\moveto(94.89584382,234.08157509)
\lineto(86.49895382,234.08157509)
\curveto(86.34860664,234.08157509)(86.24085783,234.11282509)(86.17570739,234.17532509)
\curveto(86.11055694,234.23261675)(86.07798172,234.31074174)(86.07798172,234.40970007)
\curveto(86.07798172,234.5086584)(86.11055694,234.58938756)(86.17570739,234.65188755)
\curveto(86.24085783,234.70917921)(86.34860664,234.73782504)(86.49895382,234.73782504)
\lineto(94.89584382,234.73782504)
\curveto(95.05120257,234.73782504)(95.15895138,234.70917921)(95.21909026,234.65188755)
\curveto(95.2842407,234.58938756)(95.31681592,234.5086584)(95.31681592,234.40970007)
\curveto(95.31681592,234.31074174)(95.2842407,234.23261675)(95.21909026,234.17532509)
\curveto(95.15895138,234.11282509)(95.05120257,234.08157509)(94.89584382,234.08157509)
\closepath
}
}
{
\newrgbcolor{curcolor}{0 0 0}
\pscustom[linestyle=none,fillstyle=solid,fillcolor=curcolor]
{
\newpath
\moveto(97.85767928,248.26907402)
\lineto(97.85767928,244.23001182)
\curveto(98.25860509,244.68313679)(98.6419904,245.0008451)(99.00783521,245.18313675)
\curveto(99.37869158,245.37063674)(99.79214633,245.46438673)(100.24819944,245.46438673)
\curveto(100.73933356,245.46438673)(101.15529409,245.3732409)(101.49608103,245.19094925)
\curveto(101.84187955,245.01386593)(102.13004498,244.73782429)(102.36057732,244.36282431)
\curveto(102.59110966,243.99303268)(102.70637583,243.60240771)(102.70637583,243.1909494)
\lineto(102.70637583,239.12063721)
\lineto(103.41300758,239.12063721)
\curveto(103.56836633,239.12063721)(103.67611514,239.08938721)(103.73625402,239.02688722)
\curveto(103.80140446,238.96959556)(103.83397968,238.89147056)(103.83397968,238.79251224)
\curveto(103.83397968,238.69876224)(103.80140446,238.62063725)(103.73625402,238.55813726)
\curveto(103.67611514,238.49563726)(103.56836633,238.46438726)(103.41300758,238.46438726)
\lineto(101.36076857,238.46438726)
\curveto(101.20540982,238.46438726)(101.09515522,238.49563726)(101.03000478,238.55813726)
\curveto(100.96485433,238.62063725)(100.93227911,238.69876224)(100.93227911,238.79251224)
\curveto(100.93227911,238.89147056)(100.96485433,238.96959556)(101.03000478,239.02688722)
\curveto(101.09515522,239.08938721)(101.20540982,239.12063721)(101.36076857,239.12063721)
\lineto(102.06740032,239.12063721)
\lineto(102.06740032,243.14407441)
\curveto(102.06740032,243.6180327)(101.90201842,244.01386601)(101.57125462,244.33157432)
\curveto(101.2455024,244.64928263)(100.78944929,244.80813678)(100.20309529,244.80813678)
\curveto(99.7420306,244.80813678)(99.34862215,244.69094929)(99.02286993,244.45657431)
\curveto(98.78732601,244.28990765)(98.39892913,243.89667852)(97.85767928,243.2768869)
\lineto(97.85767928,239.12063721)
\lineto(98.57182839,239.12063721)
\curveto(98.72217557,239.12063721)(98.82992438,239.08938721)(98.89507482,239.02688722)
\curveto(98.96022527,238.96959556)(98.99280049,238.89147056)(98.99280049,238.79251224)
\curveto(98.99280049,238.69876224)(98.96022527,238.62063725)(98.89507482,238.55813726)
\curveto(98.82992438,238.49563726)(98.72217557,238.46438726)(98.57182839,238.46438726)
\lineto(96.51207202,238.46438726)
\curveto(96.36172484,238.46438726)(96.25397603,238.49563726)(96.18882558,238.55813726)
\curveto(96.12367514,238.62063725)(96.09109992,238.69876224)(96.09109992,238.79251224)
\curveto(96.09109992,238.89147056)(96.12367514,238.96959556)(96.18882558,239.02688722)
\curveto(96.25397603,239.08938721)(96.36172484,239.12063721)(96.51207202,239.12063721)
\lineto(97.22622113,239.12063721)
\lineto(97.22622113,247.60501157)
\lineto(96.37675956,247.60501157)
\curveto(96.22641238,247.60501157)(96.11866357,247.63626157)(96.05351312,247.69876156)
\curveto(95.98836268,247.76126156)(95.95578746,247.84199072)(95.95578746,247.94094904)
\curveto(95.95578746,248.03469904)(95.98836268,248.11282403)(96.05351312,248.17532403)
\curveto(96.11866357,248.23782402)(96.22641238,248.26907402)(96.37675956,248.26907402)
\closepath
}
}
{
\newrgbcolor{curcolor}{0 0 0}
\pscustom[linestyle=none,fillstyle=solid,fillcolor=curcolor]
{
\newpath
\moveto(112.26845956,248.26907402)
\lineto(112.26845956,239.12063721)
\lineto(113.11040377,239.12063721)
\curveto(113.26576252,239.12063721)(113.37601712,239.08938721)(113.44116757,239.02688722)
\curveto(113.50631801,238.96959556)(113.53889323,238.89147056)(113.53889323,238.79251224)
\curveto(113.53889323,238.69876224)(113.50631801,238.62063725)(113.44116757,238.55813726)
\curveto(113.37601712,238.49563726)(113.26576252,238.46438726)(113.11040377,238.46438726)
\lineto(111.62948405,238.46438726)
\lineto(111.62948405,239.76907466)
\curveto(110.89779444,238.72740808)(109.96814771,238.20657478)(108.84054386,238.20657478)
\curveto(108.26922458,238.20657478)(107.72045737,238.36282477)(107.19424224,238.67532475)
\curveto(106.67303869,238.99303306)(106.25958394,239.44355385)(105.95387801,240.02688714)
\curveto(105.65318365,240.61022043)(105.50283647,241.21178289)(105.50283647,241.83157451)
\curveto(105.50283647,242.45657446)(105.65318365,243.05813691)(105.95387801,243.63626187)
\curveto(106.25958394,244.21959516)(106.67303869,244.67011596)(107.19424224,244.98782427)
\curveto(107.72045737,245.30553258)(108.27173036,245.46438673)(108.84806122,245.46438673)
\curveto(109.95060721,245.46438673)(110.87774815,244.94355344)(111.62948405,243.90188685)
\lineto(111.62948405,247.60501157)
\lineto(110.78753984,247.60501157)
\curveto(110.63218109,247.60501157)(110.52192649,247.63626157)(110.45677605,247.69876156)
\curveto(110.3916256,247.76126156)(110.35905038,247.84199072)(110.35905038,247.94094904)
\curveto(110.35905038,248.03469904)(110.3916256,248.11282403)(110.45677605,248.17532403)
\curveto(110.52192649,248.23782402)(110.63218109,248.26907402)(110.78753984,248.26907402)
\closepath
\moveto(111.62948405,241.83157451)
\curveto(111.62948405,242.66490778)(111.35885913,243.36803272)(110.81760928,243.94094935)
\curveto(110.27635943,244.51386597)(109.63237234,244.80032428)(108.88564802,244.80032428)
\curveto(108.13391212,244.80032428)(107.48741924,244.51386597)(106.9461694,243.94094935)
\curveto(106.40491955,243.36803272)(106.13429462,242.66490778)(106.13429462,241.83157451)
\curveto(106.13429462,241.00344957)(106.40491955,240.30032462)(106.9461694,239.72219967)
\curveto(107.48741924,239.14928304)(108.13391212,238.86282473)(108.88564802,238.86282473)
\curveto(109.63237234,238.86282473)(110.27635943,239.14928304)(110.81760928,239.72219967)
\curveto(111.35885913,240.30032462)(111.62948405,241.00344957)(111.62948405,241.83157451)
\closepath
}
}
{
\newrgbcolor{curcolor}{0 0 0}
\pscustom[linestyle=none,fillstyle=solid,fillcolor=curcolor]
{
\newpath
\moveto(117.65840905,245.23001175)
\lineto(117.65840905,243.57376187)
\curveto(118.48030697,244.34459515)(119.09422462,244.83938678)(119.50016201,245.05813676)
\curveto(119.91111096,245.28209508)(120.2894847,245.39407424)(120.63528321,245.39407424)
\curveto(121.01115116,245.39407424)(121.35945546,245.26126175)(121.68019611,244.99563677)
\curveto(122.00594834,244.73522012)(122.16882445,244.53730347)(122.16882445,244.40188681)
\curveto(122.16882445,244.30292849)(122.13624923,244.21959516)(122.07109878,244.15188683)
\curveto(122.01095991,244.08938684)(121.93328053,244.05813684)(121.83806065,244.05813684)
\curveto(121.78794493,244.05813684)(121.74534656,244.06594934)(121.71026555,244.08157434)
\curveto(121.67518454,244.10240767)(121.6100341,244.1623035)(121.51481422,244.26126182)
\curveto(121.33940917,244.44355348)(121.18655621,244.56855347)(121.05625532,244.63626179)
\curveto(120.92595443,244.70397012)(120.79815933,244.73782429)(120.67287001,244.73782429)
\curveto(120.39723351,244.73782429)(120.06396393,244.62324096)(119.67306126,244.39407431)
\curveto(119.28717017,244.16490766)(118.61561943,243.60501187)(117.65840905,242.71438694)
\lineto(117.65840905,239.12063721)
\lineto(120.4548666,239.12063721)
\curveto(120.61022535,239.12063721)(120.72047995,239.08938721)(120.78563039,239.02688722)
\curveto(120.85078084,238.96959556)(120.88335606,238.89147056)(120.88335606,238.79251224)
\curveto(120.88335606,238.69876224)(120.85078084,238.62063725)(120.78563039,238.55813726)
\curveto(120.72047995,238.49563726)(120.61022535,238.46438726)(120.4548666,238.46438726)
\lineto(115.50092702,238.46438726)
\curveto(115.35057984,238.46438726)(115.24283103,238.49303309)(115.17768058,238.55032476)
\curveto(115.11253014,238.61282475)(115.07995492,238.69094974)(115.07995492,238.78469974)
\curveto(115.07995492,238.8732414)(115.11002435,238.94615806)(115.17016322,239.00344972)
\curveto(115.23531367,239.06594972)(115.34556827,239.09719971)(115.50092702,239.09719971)
\lineto(117.0269509,239.09719971)
\lineto(117.0269509,244.5659493)
\lineto(115.86176025,244.5659493)
\curveto(115.71141307,244.5659493)(115.60366426,244.5971993)(115.53851382,244.65969929)
\curveto(115.47336337,244.72219929)(115.44078815,244.80292845)(115.44078815,244.90188677)
\curveto(115.44078815,244.99563677)(115.47085758,245.07376176)(115.53099646,245.13626176)
\curveto(115.5961469,245.19876175)(115.7064015,245.23001175)(115.86176025,245.23001175)
\closepath
}
}
{
\newrgbcolor{curcolor}{0 0 0}
\pscustom[linestyle=none,fillstyle=solid,fillcolor=curcolor]
{
\newpath
\moveto(131.85118592,234.08157509)
\lineto(123.45429593,234.08157509)
\curveto(123.30394875,234.08157509)(123.19619993,234.11282509)(123.13104949,234.17532509)
\curveto(123.06589904,234.23261675)(123.03332382,234.31074174)(123.03332382,234.40970007)
\curveto(123.03332382,234.5086584)(123.06589904,234.58938756)(123.13104949,234.65188755)
\curveto(123.19619993,234.70917921)(123.30394875,234.73782504)(123.45429593,234.73782504)
\lineto(131.85118592,234.73782504)
\curveto(132.00654468,234.73782504)(132.11429349,234.70917921)(132.17443236,234.65188755)
\curveto(132.2395828,234.58938756)(132.27215803,234.5086584)(132.27215803,234.40970007)
\curveto(132.27215803,234.31074174)(132.2395828,234.23261675)(132.17443236,234.17532509)
\curveto(132.11429349,234.11282509)(132.00654468,234.08157509)(131.85118592,234.08157509)
\closepath
}
}
{
\newrgbcolor{curcolor}{0 0 0}
\pscustom[linestyle=none,fillstyle=solid,fillcolor=curcolor]
{
\newpath
\moveto(135.59481965,245.23001175)
\lineto(139.02273535,245.23001175)
\curveto(139.17308253,245.23001175)(139.28083134,245.19876175)(139.34598179,245.13626176)
\curveto(139.41113223,245.07376176)(139.44370745,244.9930326)(139.44370745,244.89407427)
\curveto(139.44370745,244.80032428)(139.41113223,244.72219929)(139.34598179,244.65969929)
\curveto(139.28083134,244.5971993)(139.17308253,244.5659493)(139.02273535,244.5659493)
\lineto(135.59481965,244.5659493)
\lineto(135.59481965,240.19876213)
\curveto(135.59481965,239.81855383)(135.74015526,239.50084552)(136.03082647,239.2456372)
\curveto(136.32650926,238.99042889)(136.75750451,238.86282473)(137.32381222,238.86282473)
\curveto(137.74979589,238.86282473)(138.21086058,238.92792889)(138.70700627,239.05813722)
\curveto(139.20315197,239.19355387)(139.58904306,239.34459553)(139.86467956,239.51126218)
\curveto(139.96491101,239.57897051)(140.04760196,239.61282468)(140.1127524,239.61282468)
\curveto(140.19293757,239.61282468)(140.26309958,239.57897051)(140.32323846,239.51126218)
\curveto(140.38337733,239.44876219)(140.41344676,239.37324136)(140.41344676,239.2846997)
\curveto(140.41344676,239.20657471)(140.38087154,239.13365804)(140.3157211,239.06594972)
\curveto(140.15535077,238.89407473)(139.7644481,238.70657474)(139.14301309,238.50344976)
\curveto(138.52658966,238.30553311)(137.93522408,238.20657478)(137.36891637,238.20657478)
\curveto(136.63221519,238.20657478)(136.04586119,238.38626227)(135.60985437,238.74563724)
\curveto(135.17384755,239.10501221)(134.95584413,239.58938718)(134.95584413,240.19876213)
\lineto(134.95584413,244.5659493)
\lineto(133.79065349,244.5659493)
\curveto(133.64030631,244.5659493)(133.5325575,244.5971993)(133.46740705,244.65969929)
\curveto(133.40225661,244.72219929)(133.36968139,244.80292845)(133.36968139,244.90188677)
\curveto(133.36968139,244.99563677)(133.40225661,245.07376176)(133.46740705,245.13626176)
\curveto(133.5325575,245.19876175)(133.64030631,245.23001175)(133.79065349,245.23001175)
\lineto(134.95584413,245.23001175)
\lineto(134.95584413,247.1675116)
\curveto(134.95584413,247.32376159)(134.98591357,247.43574075)(135.04605244,247.50344908)
\curveto(135.10619131,247.5711574)(135.1813649,247.60501157)(135.27157321,247.60501157)
\curveto(135.36679309,247.60501157)(135.44447247,247.5711574)(135.50461134,247.50344908)
\curveto(135.56475021,247.43574075)(135.59481965,247.32376159)(135.59481965,247.1675116)
\closepath
}
}
{
\newrgbcolor{curcolor}{0 0 0}
\pscustom[linestyle=none,fillstyle=solid,fillcolor=curcolor]
{
\newpath
\moveto(5.16864027,208.26907705)
\lineto(5.16864027,203.91751488)
\curveto(5.93039931,204.9487648)(6.8500229,205.46438977)(7.92751102,205.46438977)
\curveto(8.84964039,205.46438977)(9.63896309,205.11543146)(10.2954791,204.41751484)
\curveto(10.95199512,203.72480656)(11.28025313,202.87324413)(11.28025313,201.86282754)
\curveto(11.28025313,200.84199428)(10.94698355,199.97741101)(10.28044439,199.26907774)
\curveto(9.61891679,198.56074446)(8.83460567,198.20657782)(7.92751102,198.20657782)
\curveto(6.82496503,198.20657782)(5.90534145,198.72220278)(5.16864027,199.7534527)
\lineto(5.16864027,198.4643903)
\lineto(3.68772055,198.4643903)
\curveto(3.53737337,198.4643903)(3.42962456,198.49564029)(3.36447411,198.55814029)
\curveto(3.29932367,198.62064028)(3.26674844,198.69876528)(3.26674844,198.79251527)
\curveto(3.26674844,198.8914736)(3.29932367,198.96959859)(3.36447411,199.02689025)
\curveto(3.42962456,199.08939025)(3.53737337,199.12064025)(3.68772055,199.12064025)
\lineto(4.53718211,199.12064025)
\lineto(4.53718211,207.6050146)
\lineto(3.68772055,207.6050146)
\curveto(3.53737337,207.6050146)(3.42962456,207.6362646)(3.36447411,207.6987646)
\curveto(3.29932367,207.76126459)(3.26674844,207.84199375)(3.26674844,207.94095208)
\curveto(3.26674844,208.03470207)(3.29932367,208.11282706)(3.36447411,208.17532706)
\curveto(3.42962456,208.23782705)(3.53737337,208.26907705)(3.68772055,208.26907705)
\closepath
\moveto(10.64879498,201.83157754)
\curveto(10.64879498,202.65970248)(10.37566427,203.36022326)(9.82940285,203.93313988)
\curveto(9.28314143,204.51126484)(8.64416591,204.80032732)(7.9124763,204.80032732)
\curveto(7.18078669,204.80032732)(6.54181118,204.51126484)(5.99554976,203.93313988)
\curveto(5.44928834,203.36022326)(5.17615763,202.65970248)(5.17615763,201.83157754)
\curveto(5.17615763,201.0034526)(5.44928834,200.30032766)(5.99554976,199.7222027)
\curveto(6.54181118,199.14928608)(7.18078669,198.86282777)(7.9124763,198.86282777)
\curveto(8.64416591,198.86282777)(9.28314143,199.14928608)(9.82940285,199.7222027)
\curveto(10.37566427,200.30032766)(10.64879498,201.0034526)(10.64879498,201.83157754)
\closepath
}
}
{
\newrgbcolor{curcolor}{0 0 0}
\pscustom[linestyle=none,fillstyle=solid,fillcolor=curcolor]
{
\newpath
\moveto(20.98516403,194.08157813)
\lineto(12.58827404,194.08157813)
\curveto(12.43792686,194.08157813)(12.33017804,194.11282813)(12.2650276,194.17532812)
\curveto(12.19987715,194.23261978)(12.16730193,194.31074478)(12.16730193,194.4097031)
\curveto(12.16730193,194.50866143)(12.19987715,194.58939059)(12.2650276,194.65189059)
\curveto(12.33017804,194.70918225)(12.43792686,194.73782808)(12.58827404,194.73782808)
\lineto(20.98516403,194.73782808)
\curveto(21.14052278,194.73782808)(21.2482716,194.70918225)(21.30841047,194.65189059)
\curveto(21.37356091,194.58939059)(21.40613614,194.50866143)(21.40613614,194.4097031)
\curveto(21.40613614,194.31074478)(21.37356091,194.23261978)(21.30841047,194.17532812)
\curveto(21.2482716,194.11282813)(21.14052278,194.08157813)(20.98516403,194.08157813)
\closepath
}
}
{
\newrgbcolor{curcolor}{0 0 0}
\pscustom[linestyle=none,fillstyle=solid,fillcolor=curcolor]
{
\newpath
\moveto(23.64630955,208.26907705)
\lineto(23.64630955,203.91751488)
\curveto(24.4080686,204.9487648)(25.32769218,205.46438977)(26.4051803,205.46438977)
\curveto(27.32730967,205.46438977)(28.11663237,205.11543146)(28.77314839,204.41751484)
\curveto(29.42966441,203.72480656)(29.75792242,202.87324413)(29.75792242,201.86282754)
\curveto(29.75792242,200.84199428)(29.42465283,199.97741101)(28.75811367,199.26907774)
\curveto(28.09658608,198.56074446)(27.31227496,198.20657782)(26.4051803,198.20657782)
\curveto(25.30263432,198.20657782)(24.38301074,198.72220278)(23.64630955,199.7534527)
\lineto(23.64630955,198.4643903)
\lineto(22.16538983,198.4643903)
\curveto(22.01504265,198.4643903)(21.90729384,198.49564029)(21.84214339,198.55814029)
\curveto(21.77699295,198.62064028)(21.74441773,198.69876528)(21.74441773,198.79251527)
\curveto(21.74441773,198.8914736)(21.77699295,198.96959859)(21.84214339,199.02689025)
\curveto(21.90729384,199.08939025)(22.01504265,199.12064025)(22.16538983,199.12064025)
\lineto(23.0148514,199.12064025)
\lineto(23.0148514,207.6050146)
\lineto(22.16538983,207.6050146)
\curveto(22.01504265,207.6050146)(21.90729384,207.6362646)(21.84214339,207.6987646)
\curveto(21.77699295,207.76126459)(21.74441773,207.84199375)(21.74441773,207.94095208)
\curveto(21.74441773,208.03470207)(21.77699295,208.11282706)(21.84214339,208.17532706)
\curveto(21.90729384,208.23782705)(22.01504265,208.26907705)(22.16538983,208.26907705)
\closepath
\moveto(29.12646426,201.83157754)
\curveto(29.12646426,202.65970248)(28.85333355,203.36022326)(28.30707213,203.93313988)
\curveto(27.76081071,204.51126484)(27.1218352,204.80032732)(26.39014559,204.80032732)
\curveto(25.65845598,204.80032732)(25.01948046,204.51126484)(24.47321904,203.93313988)
\curveto(23.92695762,203.36022326)(23.65382691,202.65970248)(23.65382691,201.83157754)
\curveto(23.65382691,201.0034526)(23.92695762,200.30032766)(24.47321904,199.7222027)
\curveto(25.01948046,199.14928608)(25.65845598,198.86282777)(26.39014559,198.86282777)
\curveto(27.1218352,198.86282777)(27.76081071,199.14928608)(28.30707213,199.7222027)
\curveto(28.85333355,200.30032766)(29.12646426,201.0034526)(29.12646426,201.83157754)
\closepath
}
}
{
\newrgbcolor{curcolor}{0 0 0}
\pscustom[linestyle=none,fillstyle=solid,fillcolor=curcolor]
{
\newpath
\moveto(37.42562991,198.4643903)
\lineto(37.42562991,199.42532772)
\curveto(36.56363941,198.61282778)(35.6314869,198.20657782)(34.62917237,198.20657782)
\curveto(34.01274893,198.20657782)(33.54416688,198.38105697)(33.22342623,198.73001528)
\curveto(32.8074657,199.18834857)(32.59948544,199.7222027)(32.59948544,200.33157765)
\lineto(32.59948544,204.56595233)
\lineto(31.75002387,204.56595233)
\curveto(31.59967669,204.56595233)(31.49192788,204.59720233)(31.42677743,204.65970233)
\curveto(31.36162699,204.72220232)(31.32905177,204.80293148)(31.32905177,204.90188981)
\curveto(31.32905177,204.9956398)(31.36162699,205.07376479)(31.42677743,205.13626479)
\curveto(31.49192788,205.19876479)(31.59967669,205.23001478)(31.75002387,205.23001478)
\lineto(33.23094359,205.23001478)
\lineto(33.23094359,200.33157765)
\curveto(33.23094359,199.90449435)(33.36124448,199.55293188)(33.62184626,199.27689023)
\curveto(33.88244804,199.00084859)(34.20820026,198.86282777)(34.59910293,198.86282777)
\curveto(35.62647533,198.86282777)(36.56865099,199.35241106)(37.42562991,200.33157765)
\lineto(37.42562991,204.56595233)
\lineto(36.26043927,204.56595233)
\curveto(36.11009209,204.56595233)(36.00234328,204.59720233)(35.93719283,204.65970233)
\curveto(35.87204239,204.72220232)(35.83946716,204.80293148)(35.83946716,204.90188981)
\curveto(35.83946716,204.9956398)(35.87204239,205.07376479)(35.93719283,205.13626479)
\curveto(36.00234328,205.19876479)(36.11009209,205.23001478)(36.26043927,205.23001478)
\lineto(38.05708807,205.23001478)
\lineto(38.05708807,199.12064025)
\lineto(38.59082056,199.12064025)
\curveto(38.74116774,199.12064025)(38.84891655,199.08939025)(38.91406699,199.02689025)
\curveto(38.97921744,198.96959859)(39.01179266,198.8914736)(39.01179266,198.79251527)
\curveto(39.01179266,198.69876528)(38.97921744,198.62064028)(38.91406699,198.55814029)
\curveto(38.84891655,198.49564029)(38.74116774,198.4643903)(38.59082056,198.4643903)
\closepath
}
}
{
\newrgbcolor{curcolor}{0 0 0}
\pscustom[linestyle=none,fillstyle=solid,fillcolor=curcolor]
{
\newpath
\moveto(44.08600954,204.56595233)
\lineto(44.08600954,199.12064025)
\lineto(46.859915,199.12064025)
\curveto(47.01026218,199.12064025)(47.118011,199.08939025)(47.18316144,199.02689025)
\curveto(47.24831189,198.96959859)(47.28088711,198.8914736)(47.28088711,198.79251527)
\curveto(47.28088711,198.69876528)(47.24831189,198.62064028)(47.18316144,198.55814029)
\curveto(47.118011,198.49564029)(47.01026218,198.4643903)(46.859915,198.4643903)
\lineto(41.92101014,198.4643903)
\curveto(41.77066296,198.4643903)(41.66291415,198.49564029)(41.59776371,198.55814029)
\curveto(41.53261326,198.62064028)(41.50003804,198.69876528)(41.50003804,198.79251527)
\curveto(41.50003804,198.8914736)(41.53261326,198.96959859)(41.59776371,199.02689025)
\curveto(41.66291415,199.08939025)(41.77066296,199.12064025)(41.92101014,199.12064025)
\lineto(43.44703402,199.12064025)
\lineto(43.44703402,204.56595233)
\lineto(42.07887468,204.56595233)
\curveto(41.9285275,204.56595233)(41.82077869,204.59720233)(41.75562825,204.65970233)
\curveto(41.6904778,204.72220232)(41.65790258,204.80293148)(41.65790258,204.90188981)
\curveto(41.65790258,204.9956398)(41.6904778,205.07376479)(41.75562825,205.13626479)
\curveto(41.82077869,205.19876479)(41.9285275,205.23001478)(42.07887468,205.23001478)
\lineto(43.44703402,205.23001478)
\lineto(43.44703402,206.22220221)
\curveto(43.44703402,206.7742855)(43.66253165,207.25345213)(44.09352689,207.6597021)
\curveto(44.52452214,208.06595207)(45.09584143,208.26907705)(45.80748475,208.26907705)
\curveto(46.40386189,208.26907705)(47.04033162,208.21178539)(47.71689393,208.09720207)
\curveto(47.97248414,208.0555354)(48.1253371,208.00605624)(48.17545283,207.94876458)
\curveto(48.23058013,207.89147291)(48.25814378,207.81595209)(48.25814378,207.72220209)
\curveto(48.25814378,207.6284521)(48.22807434,207.55032711)(48.16793547,207.48782711)
\curveto(48.1077966,207.43053545)(48.02761143,207.40188962)(47.92737998,207.40188962)
\curveto(47.8872874,207.40188962)(47.81963117,207.40970212)(47.72441129,207.42532712)
\curveto(46.96766382,207.54511877)(46.3286883,207.6050146)(45.80748475,207.6050146)
\curveto(45.25621175,207.6050146)(44.83022808,207.46438961)(44.52953372,207.18313963)
\curveto(44.23385093,206.90188966)(44.08600954,206.58157718)(44.08600954,206.22220221)
\lineto(44.08600954,205.23001478)
\lineto(47.04033162,205.23001478)
\curveto(47.1906788,205.23001478)(47.29842761,205.19876479)(47.36357806,205.13626479)
\curveto(47.4287285,205.07376479)(47.46130372,204.99303563)(47.46130372,204.89407731)
\curveto(47.46130372,204.80032732)(47.4287285,204.72220232)(47.36357806,204.65970233)
\curveto(47.29842761,204.59720233)(47.1906788,204.56595233)(47.04033162,204.56595233)
\closepath
}
}
{
\newrgbcolor{curcolor}{0 0 0}
\pscustom[linestyle=none,fillstyle=solid,fillcolor=curcolor]
{
\newpath
\moveto(5.16864027,188.26907857)
\lineto(5.16864027,183.9175164)
\curveto(5.93039931,184.94876632)(6.8500229,185.46439128)(7.92751102,185.46439128)
\curveto(8.84964039,185.46439128)(9.63896309,185.11543298)(10.2954791,184.41751636)
\curveto(10.95199512,183.72480808)(11.28025313,182.87324565)(11.28025313,181.86282906)
\curveto(11.28025313,180.8419958)(10.94698355,179.97741253)(10.28044439,179.26907925)
\curveto(9.61891679,178.56074597)(8.83460567,178.20657933)(7.92751102,178.20657933)
\curveto(6.82496503,178.20657933)(5.90534145,178.72220429)(5.16864027,179.75345422)
\lineto(5.16864027,178.46439181)
\lineto(3.68772055,178.46439181)
\curveto(3.53737337,178.46439181)(3.42962456,178.49564181)(3.36447411,178.55814181)
\curveto(3.29932367,178.6206418)(3.26674844,178.6987668)(3.26674844,178.79251679)
\curveto(3.26674844,178.89147511)(3.29932367,178.96960011)(3.36447411,179.02689177)
\curveto(3.42962456,179.08939177)(3.53737337,179.12064176)(3.68772055,179.12064176)
\lineto(4.53718211,179.12064176)
\lineto(4.53718211,187.60501612)
\lineto(3.68772055,187.60501612)
\curveto(3.53737337,187.60501612)(3.42962456,187.63626612)(3.36447411,187.69876611)
\curveto(3.29932367,187.76126611)(3.26674844,187.84199527)(3.26674844,187.94095359)
\curveto(3.26674844,188.03470359)(3.29932367,188.11282858)(3.36447411,188.17532858)
\curveto(3.42962456,188.23782857)(3.53737337,188.26907857)(3.68772055,188.26907857)
\closepath
\moveto(10.64879498,181.83157906)
\curveto(10.64879498,182.659704)(10.37566427,183.36022478)(9.82940285,183.9331414)
\curveto(9.28314143,184.51126635)(8.64416591,184.80032883)(7.9124763,184.80032883)
\curveto(7.18078669,184.80032883)(6.54181118,184.51126635)(5.99554976,183.9331414)
\curveto(5.44928834,183.36022478)(5.17615763,182.659704)(5.17615763,181.83157906)
\curveto(5.17615763,181.00345412)(5.44928834,180.30032917)(5.99554976,179.72220422)
\curveto(6.54181118,179.14928759)(7.18078669,178.86282928)(7.9124763,178.86282928)
\curveto(8.64416591,178.86282928)(9.28314143,179.14928759)(9.82940285,179.72220422)
\curveto(10.37566427,180.30032917)(10.64879498,181.00345412)(10.64879498,181.83157906)
\closepath
}
}
{
\newrgbcolor{curcolor}{0 0 0}
\pscustom[linestyle=none,fillstyle=solid,fillcolor=curcolor]
{
\newpath
\moveto(20.98516403,174.08157965)
\lineto(12.58827404,174.08157965)
\curveto(12.43792686,174.08157965)(12.33017804,174.11282964)(12.2650276,174.17532964)
\curveto(12.19987715,174.2326213)(12.16730193,174.3107463)(12.16730193,174.40970462)
\curveto(12.16730193,174.50866295)(12.19987715,174.58939211)(12.2650276,174.6518921)
\curveto(12.33017804,174.70918376)(12.43792686,174.7378296)(12.58827404,174.7378296)
\lineto(20.98516403,174.7378296)
\curveto(21.14052278,174.7378296)(21.2482716,174.70918376)(21.30841047,174.6518921)
\curveto(21.37356091,174.58939211)(21.40613614,174.50866295)(21.40613614,174.40970462)
\curveto(21.40613614,174.3107463)(21.37356091,174.2326213)(21.30841047,174.17532964)
\curveto(21.2482716,174.11282964)(21.14052278,174.08157965)(20.98516403,174.08157965)
\closepath
}
}
{
\newrgbcolor{curcolor}{0 0 0}
\pscustom[linestyle=none,fillstyle=solid,fillcolor=curcolor]
{
\newpath
\moveto(23.64630955,185.2300163)
\lineto(23.64630955,184.03470389)
\curveto(24.0221775,184.50866219)(24.4256091,184.86543299)(24.85660435,185.10501631)
\curveto(25.2875996,185.34459962)(25.79627423,185.46439128)(26.38262823,185.46439128)
\curveto(27.00406324,185.46439128)(27.57788831,185.31334963)(28.10410344,185.01126632)
\curveto(28.63031857,184.70918301)(29.03625595,184.28730804)(29.32191559,183.74564141)
\curveto(29.61258681,183.20918312)(29.75792242,182.644079)(29.75792242,182.05032904)
\curveto(29.75792242,181.10762078)(29.43217019,180.30032917)(28.78066575,179.62845422)
\curveto(28.13417287,178.96178761)(27.33733282,178.6284543)(26.39014559,178.6284543)
\curveto(25.26254174,178.6284543)(24.34792973,179.10501676)(23.64630955,180.05814169)
\lineto(23.64630955,176.10501699)
\lineto(25.17985079,176.10501699)
\curveto(25.33019797,176.10501699)(25.43794678,176.07637116)(25.50309723,176.0190795)
\curveto(25.56824767,175.9565795)(25.60082289,175.87585034)(25.60082289,175.77689202)
\curveto(25.60082289,175.68314202)(25.56824767,175.60501703)(25.50309723,175.54251703)
\curveto(25.43794678,175.48001704)(25.33019797,175.44876704)(25.17985079,175.44876704)
\lineto(22.16538983,175.44876704)
\curveto(22.01504265,175.44876704)(21.90729384,175.48001704)(21.84214339,175.54251703)
\curveto(21.77699295,175.5998087)(21.74441773,175.67793369)(21.74441773,175.77689202)
\curveto(21.74441773,175.87585034)(21.77699295,175.9565795)(21.84214339,176.0190795)
\curveto(21.90729384,176.07637116)(22.01504265,176.10501699)(22.16538983,176.10501699)
\lineto(23.0148514,176.10501699)
\lineto(23.0148514,184.56595385)
\lineto(22.16538983,184.56595385)
\curveto(22.01504265,184.56595385)(21.90729384,184.59720385)(21.84214339,184.65970384)
\curveto(21.77699295,184.72220384)(21.74441773,184.802933)(21.74441773,184.90189132)
\curveto(21.74441773,184.99564132)(21.77699295,185.07376631)(21.84214339,185.13626631)
\curveto(21.90729384,185.1987663)(22.01504265,185.2300163)(22.16538983,185.2300163)
\closepath
\moveto(29.1189469,182.05032904)
\curveto(29.1189469,182.80553732)(28.85333355,183.4513706)(28.32210685,183.98782889)
\curveto(27.79589172,184.52949552)(27.15190463,184.80032883)(26.39014559,184.80032883)
\curveto(25.62337497,184.80032883)(24.97437631,184.52949552)(24.44314961,183.98782889)
\curveto(23.9119229,183.44616227)(23.64630955,182.80032898)(23.64630955,182.05032904)
\curveto(23.64630955,181.29512077)(23.9119229,180.64668331)(24.44314961,180.10501669)
\curveto(24.97437631,179.56335006)(25.62337497,179.29251675)(26.39014559,179.29251675)
\curveto(27.14689306,179.29251675)(27.79088015,179.5607459)(28.32210685,180.09720419)
\curveto(28.85333355,180.63887082)(29.1189469,181.28991243)(29.1189469,182.05032904)
\closepath
}
}
{
\newrgbcolor{curcolor}{0 0 0}
\pscustom[linestyle=none,fillstyle=solid,fillcolor=curcolor]
{
\newpath
\moveto(37.0798314,178.46439181)
\lineto(37.0798314,179.40970424)
\curveto(36.1627136,178.60762097)(35.18295115,178.20657933)(34.14054403,178.20657933)
\curveto(33.38379656,178.20657933)(32.79243099,178.40449598)(32.36644731,178.80032929)
\curveto(31.94046363,179.20137092)(31.72747179,179.69095422)(31.72747179,180.26907918)
\curveto(31.72747179,180.90449579)(32.00811986,181.45918325)(32.569416,181.93314155)
\curveto(33.13071214,182.40709985)(33.95010427,182.644079)(35.02759239,182.644079)
\curveto(35.31826361,182.644079)(35.63399269,182.62324566)(35.97477963,182.581579)
\curveto(36.31556657,182.54512067)(36.68391716,182.48522484)(37.0798314,182.40189151)
\lineto(37.0798314,183.46439143)
\curveto(37.0798314,183.82376641)(36.91946107,184.13626638)(36.59872042,184.40189136)
\curveto(36.27797977,184.66751634)(35.7968688,184.80032883)(35.1553875,184.80032883)
\curveto(34.66425337,184.80032883)(33.97516213,184.65189134)(33.08811377,184.35501637)
\curveto(32.92774345,184.30293304)(32.82500621,184.27689137)(32.77990205,184.27689137)
\curveto(32.69971689,184.27689137)(32.62955487,184.30814137)(32.569416,184.37064137)
\curveto(32.5142887,184.43314136)(32.48672505,184.51126635)(32.48672505,184.60501635)
\curveto(32.48672505,184.69355801)(32.51178292,184.7638705)(32.56189864,184.81595383)
\curveto(32.63206066,184.89407883)(32.91521452,185.00084965)(33.41136021,185.13626631)
\curveto(34.19316554,185.35501629)(34.78453112,185.46439128)(35.18545693,185.46439128)
\curveto(35.98229699,185.46439128)(36.603732,185.25866213)(37.04976196,184.84720383)
\curveto(37.49579193,184.44095386)(37.71880691,183.98001639)(37.71880691,183.46439143)
\lineto(37.71880691,179.12064176)
\lineto(38.56075112,179.12064176)
\curveto(38.71610987,179.12064176)(38.82636447,179.08939177)(38.89151492,179.02689177)
\curveto(38.95666536,178.96960011)(38.98924058,178.89147511)(38.98924058,178.79251679)
\curveto(38.98924058,178.6987668)(38.95666536,178.6206418)(38.89151492,178.55814181)
\curveto(38.82636447,178.49564181)(38.71610987,178.46439181)(38.56075112,178.46439181)
\closepath
\moveto(37.0798314,181.73001657)
\curveto(36.78414861,181.81855823)(36.47092532,181.88366239)(36.14016152,181.92532905)
\curveto(35.80939773,181.96699571)(35.46109343,181.98782905)(35.09524862,181.98782905)
\curveto(34.17813083,181.98782905)(33.46147594,181.78209989)(32.94528395,181.37064159)
\curveto(32.55438128,181.06334995)(32.35892995,180.69616248)(32.35892995,180.26907918)
\curveto(32.35892995,179.87324587)(32.50677134,179.53991257)(32.80245413,179.26907925)
\curveto(33.10314849,178.99824594)(33.53915531,178.86282928)(34.1104746,178.86282928)
\curveto(34.65673602,178.86282928)(35.16290485,178.97480844)(35.62898111,179.19876676)
\curveto(36.10006894,179.42793341)(36.5836857,179.78991255)(37.0798314,180.28470418)
\closepath
}
}
{
\newrgbcolor{curcolor}{0 0 0}
\pscustom[linestyle=none,fillstyle=solid,fillcolor=curcolor]
{
\newpath
\moveto(42.12397884,188.26907857)
\lineto(42.12397884,183.9175164)
\curveto(42.88573788,184.94876632)(43.80536147,185.46439128)(44.88284959,185.46439128)
\curveto(45.80497896,185.46439128)(46.59430165,185.11543298)(47.25081767,184.41751636)
\curveto(47.90733369,183.72480808)(48.2355917,182.87324565)(48.2355917,181.86282906)
\curveto(48.2355917,180.8419958)(47.90232212,179.97741253)(47.23578295,179.26907925)
\curveto(46.57425536,178.56074597)(45.78994424,178.20657933)(44.88284959,178.20657933)
\curveto(43.7803036,178.20657933)(42.86068002,178.72220429)(42.12397884,179.75345422)
\lineto(42.12397884,178.46439181)
\lineto(40.64305912,178.46439181)
\curveto(40.49271194,178.46439181)(40.38496312,178.49564181)(40.31981268,178.55814181)
\curveto(40.25466223,178.6206418)(40.22208701,178.6987668)(40.22208701,178.79251679)
\curveto(40.22208701,178.89147511)(40.25466223,178.96960011)(40.31981268,179.02689177)
\curveto(40.38496312,179.08939177)(40.49271194,179.12064176)(40.64305912,179.12064176)
\lineto(41.49252068,179.12064176)
\lineto(41.49252068,187.60501612)
\lineto(40.64305912,187.60501612)
\curveto(40.49271194,187.60501612)(40.38496312,187.63626612)(40.31981268,187.69876611)
\curveto(40.25466223,187.76126611)(40.22208701,187.84199527)(40.22208701,187.94095359)
\curveto(40.22208701,188.03470359)(40.25466223,188.11282858)(40.31981268,188.17532858)
\curveto(40.38496312,188.23782857)(40.49271194,188.26907857)(40.64305912,188.26907857)
\closepath
\moveto(47.60413354,181.83157906)
\curveto(47.60413354,182.659704)(47.33100283,183.36022478)(46.78474141,183.9331414)
\curveto(46.23847999,184.51126635)(45.59950448,184.80032883)(44.86781487,184.80032883)
\curveto(44.13612526,184.80032883)(43.49714975,184.51126635)(42.95088833,183.9331414)
\curveto(42.40462691,183.36022478)(42.1314962,182.659704)(42.1314962,181.83157906)
\curveto(42.1314962,181.00345412)(42.40462691,180.30032917)(42.95088833,179.72220422)
\curveto(43.49714975,179.14928759)(44.13612526,178.86282928)(44.86781487,178.86282928)
\curveto(45.59950448,178.86282928)(46.23847999,179.14928759)(46.78474141,179.72220422)
\curveto(47.33100283,180.30032917)(47.60413354,181.00345412)(47.60413354,181.83157906)
\closepath
}
}
{
\newrgbcolor{curcolor}{0 0 0}
\pscustom[linestyle=none,fillstyle=solid,fillcolor=curcolor]
{
\newpath
\moveto(56.83544994,188.26907857)
\lineto(56.83544994,179.12064176)
\lineto(57.67739415,179.12064176)
\curveto(57.8327529,179.12064176)(57.9430075,179.08939177)(58.00815795,179.02689177)
\curveto(58.07330839,178.96960011)(58.10588361,178.89147511)(58.10588361,178.79251679)
\curveto(58.10588361,178.6987668)(58.07330839,178.6206418)(58.00815795,178.55814181)
\curveto(57.9430075,178.49564181)(57.8327529,178.46439181)(57.67739415,178.46439181)
\lineto(56.19647443,178.46439181)
\lineto(56.19647443,179.76907921)
\curveto(55.46478482,178.72741263)(54.53513809,178.20657933)(53.40753424,178.20657933)
\curveto(52.83621496,178.20657933)(52.28744775,178.36282932)(51.76123262,178.6753293)
\curveto(51.24002907,178.99303761)(50.82657432,179.44355841)(50.52086839,180.02689169)
\curveto(50.22017403,180.61022498)(50.06982685,181.21178744)(50.06982685,181.83157906)
\curveto(50.06982685,182.45657901)(50.22017403,183.05814146)(50.52086839,183.63626642)
\curveto(50.82657432,184.21959971)(51.24002907,184.67012051)(51.76123262,184.98782882)
\curveto(52.28744775,185.30553713)(52.83872075,185.46439128)(53.4150516,185.46439128)
\curveto(54.51759759,185.46439128)(55.44473853,184.94355799)(56.19647443,183.9018914)
\lineto(56.19647443,187.60501612)
\lineto(55.35453022,187.60501612)
\curveto(55.19917147,187.60501612)(55.08891687,187.63626612)(55.02376643,187.69876611)
\curveto(54.95861598,187.76126611)(54.92604076,187.84199527)(54.92604076,187.94095359)
\curveto(54.92604076,188.03470359)(54.95861598,188.11282858)(55.02376643,188.17532858)
\curveto(55.08891687,188.23782857)(55.19917147,188.26907857)(55.35453022,188.26907857)
\closepath
\moveto(56.19647443,181.83157906)
\curveto(56.19647443,182.66491233)(55.92584951,183.36803727)(55.38459966,183.9409539)
\curveto(54.84334981,184.51387052)(54.19936272,184.80032883)(53.4526384,184.80032883)
\curveto(52.7009025,184.80032883)(52.05440962,184.51387052)(51.51315978,183.9409539)
\curveto(50.97190993,183.36803727)(50.701285,182.66491233)(50.701285,181.83157906)
\curveto(50.701285,181.00345412)(50.97190993,180.30032917)(51.51315978,179.72220422)
\curveto(52.05440962,179.14928759)(52.7009025,178.86282928)(53.4526384,178.86282928)
\curveto(54.19936272,178.86282928)(54.84334981,179.14928759)(55.38459966,179.72220422)
\curveto(55.92584951,180.30032917)(56.19647443,181.00345412)(56.19647443,181.83157906)
\closepath
}
}
{
\newrgbcolor{curcolor}{0 0 0}
\pscustom[linewidth=0.95016194,linecolor=curcolor]
{
\newpath
\moveto(143.08810961,200.22108392)
\lineto(322.68297449,200.22108392)
}
}
{
\newrgbcolor{curcolor}{0 0 0}
\pscustom[linestyle=none,fillstyle=solid,fillcolor=curcolor]
{
\newpath
\moveto(313.18135508,200.22108392)
\lineto(309.38070732,196.42043615)
\lineto(322.68297449,200.22108392)
\lineto(309.38070732,204.02173168)
\closepath
}
}
{
\newrgbcolor{curcolor}{0 0 0}
\pscustom[linewidth=1.0135061,linecolor=curcolor]
{
\newpath
\moveto(313.18135508,200.22108392)
\lineto(309.38070732,196.42043615)
\lineto(322.68297449,200.22108392)
\lineto(309.38070732,204.02173168)
\closepath
}
}
{
\newrgbcolor{curcolor}{0 0 0}
\pscustom[linewidth=0.99999871,linecolor=curcolor]
{
\newpath
\moveto(144.51669166,184.14965589)
\lineto(193.44526111,184.14965589)
\lineto(193.44526111,111.64965463)
\lineto(234.51668788,111.64965463)
}
}
{
\newrgbcolor{curcolor}{0 0 0}
\pscustom[linestyle=none,fillstyle=solid,fillcolor=curcolor]
{
\newpath
\moveto(224.5167008,111.64965463)
\lineto(220.51670596,107.64965979)
\lineto(234.51668788,111.64965463)
\lineto(220.51670596,115.64964946)
\closepath
}
}
{
\newrgbcolor{curcolor}{0 0 0}
\pscustom[linewidth=1.06666532,linecolor=curcolor]
{
\newpath
\moveto(224.5167008,111.64965463)
\lineto(220.51670596,107.64965979)
\lineto(234.51668788,111.64965463)
\lineto(220.51670596,115.64964946)
\closepath
}
}
{
\newrgbcolor{curcolor}{0 0 0}
\pscustom[linewidth=1.88976378,linecolor=curcolor]
{
\newpath
\moveto(234.6816548,111.48468061)
\lineto(333.37063647,111.48468061)
\lineto(333.37063647,57.10892008)
\lineto(234.6816548,57.10892008)
\closepath
}
}
{
\newrgbcolor{curcolor}{0 0 0}
\pscustom[linestyle=none,fillstyle=solid,fillcolor=curcolor]
{
\newpath
\moveto(244.28788886,96.40244326)
\lineto(244.28788886,96.62119324)
\curveto(244.28788886,96.78265156)(244.3179583,96.89723488)(244.37809717,96.96494321)
\curveto(244.43823604,97.03265154)(244.51340963,97.06650571)(244.60361794,97.06650571)
\curveto(244.69883782,97.06650571)(244.7765172,97.03265154)(244.83665607,96.96494321)
\curveto(244.89679494,96.89723488)(244.92686438,96.78265156)(244.92686438,96.62119324)
\lineto(244.92686438,95.13681835)
\curveto(244.9218528,94.97536003)(244.88927758,94.86077671)(244.82913871,94.79306838)
\curveto(244.77401141,94.72536005)(244.69883782,94.69150589)(244.60361794,94.69150589)
\curveto(244.5184212,94.69150589)(244.4457534,94.72015172)(244.38561453,94.77744338)
\curveto(244.33048723,94.83994337)(244.29791201,94.94150587)(244.28788886,95.08213086)
\curveto(244.25781942,95.45192249)(244.02227551,95.80348497)(243.58125712,96.13681828)
\curveto(243.14525029,96.47015158)(242.55639051,96.63681824)(241.81467775,96.63681824)
\curveto(240.87751366,96.63681824)(240.16587035,96.33213076)(239.6797478,95.72275581)
\curveto(239.19362525,95.11338085)(238.95056397,94.41546424)(238.95056397,93.62900597)
\curveto(238.95056397,92.7800477)(239.21868311,92.07952692)(239.75492139,91.52744363)
\curveto(240.29115966,90.97536033)(240.98526248,90.69931869)(241.83722983,90.69931869)
\curveto(242.32836395,90.69931869)(242.82701543,90.79306868)(243.33318427,90.98056867)
\curveto(243.84436468,91.16806865)(244.30542937,91.47015196)(244.71637832,91.8868186)
\curveto(244.82162135,91.99098526)(244.91433544,92.04306859)(244.99452061,92.04306859)
\curveto(245.07971734,92.04306859)(245.14987936,92.01181859)(245.20500666,91.94931859)
\curveto(245.26514553,91.89202693)(245.29521497,91.81911027)(245.29521497,91.73056861)
\curveto(245.29521497,91.50661029)(245.04213055,91.22275615)(244.53596171,90.87900617)
\curveto(243.71907536,90.32171455)(242.80947493,90.04306874)(241.80716039,90.04306874)
\curveto(240.78981114,90.04306874)(239.95287851,90.37900621)(239.29636249,91.05088116)
\curveto(238.64485804,91.72796444)(238.31910582,92.58473521)(238.31910582,93.62119347)
\curveto(238.31910582,94.67848505)(238.6523754,95.55608915)(239.31891457,96.25400577)
\curveto(239.9904653,96.95192238)(240.8349153,97.30088069)(241.85226455,97.30088069)
\curveto(242.81949807,97.30088069)(243.63137284,97.00140154)(244.28788886,96.40244326)
\closepath
}
}
{
\newrgbcolor{curcolor}{0 0 0}
\pscustom[linestyle=none,fillstyle=solid,fillcolor=curcolor]
{
\newpath
\moveto(254.42128922,93.66806846)
\curveto(254.42128922,92.66806854)(254.07549071,91.81390194)(253.38389368,91.10556866)
\curveto(252.69730823,90.39723538)(251.86789295,90.04306874)(250.89564785,90.04306874)
\curveto(249.91337961,90.04306874)(249.07895276,90.39723538)(248.39236731,91.10556866)
\curveto(247.70578185,91.81911027)(247.36248913,92.67327687)(247.36248913,93.66806846)
\curveto(247.36248913,94.66806839)(247.70578185,95.52223499)(248.39236731,96.23056827)
\curveto(249.07895276,96.94410988)(249.91337961,97.30088069)(250.89564785,97.30088069)
\curveto(251.86789295,97.30088069)(252.69730823,96.94671405)(253.38389368,96.23838077)
\curveto(254.07549071,95.53004749)(254.42128922,94.67327672)(254.42128922,93.66806846)
\closepath
\moveto(253.78231371,93.66806846)
\curveto(253.78231371,94.49098507)(253.49915985,95.19150585)(252.93285214,95.7696308)
\curveto(252.371556,96.34775576)(251.68998212,96.63681824)(250.8881305,96.63681824)
\curveto(250.08627887,96.63681824)(249.4021992,96.34515159)(248.83589149,95.7618183)
\curveto(248.27459535,95.18369335)(247.99394728,94.48577673)(247.99394728,93.66806846)
\curveto(247.99394728,92.85556852)(248.27459535,92.15765191)(248.83589149,91.57431862)
\curveto(249.4021992,90.99098533)(250.08627887,90.69931869)(250.8881305,90.69931869)
\curveto(251.68998212,90.69931869)(252.371556,90.98838117)(252.93285214,91.56650612)
\curveto(253.49915985,92.14983941)(253.78231371,92.85036019)(253.78231371,93.66806846)
\closepath
}
}
{
\newrgbcolor{curcolor}{0 0 0}
\pscustom[linestyle=none,fillstyle=solid,fillcolor=curcolor]
{
\newpath
\moveto(257.255334,97.06650571)
\lineto(257.255334,96.40244326)
\curveto(257.79157228,97.00140154)(258.33031634,97.30088069)(258.87156619,97.30088069)
\curveto(259.19731841,97.30088069)(259.48297805,97.20973486)(259.72854511,97.02744321)
\curveto(259.97411217,96.85035989)(260.17958665,96.57952658)(260.34496855,96.21494327)
\curveto(260.62561662,96.57952658)(260.90877047,96.85035989)(261.19443011,97.02744321)
\curveto(261.48510133,97.20973486)(261.77577254,97.30088069)(262.06644376,97.30088069)
\curveto(262.52249687,97.30088069)(262.88583589,97.14723487)(263.15646081,96.83994322)
\curveto(263.51228247,96.44410992)(263.6901933,96.01181829)(263.6901933,95.54306832)
\lineto(263.6901933,90.95713117)
\lineto(264.22392579,90.95713117)
\curveto(264.37427297,90.95713117)(264.48202178,90.92588117)(264.54717223,90.86338118)
\curveto(264.61232267,90.80608951)(264.64489789,90.72796452)(264.64489789,90.62900619)
\curveto(264.64489789,90.5352562)(264.61232267,90.45713121)(264.54717223,90.39463121)
\curveto(264.48202178,90.33213122)(264.37427297,90.30088122)(264.22392579,90.30088122)
\lineto(263.05873515,90.30088122)
\lineto(263.05873515,95.48056833)
\curveto(263.05873515,95.81390163)(262.96100948,96.08994328)(262.76555814,96.30869326)
\curveto(262.57010681,96.52744325)(262.34458604,96.63681824)(262.08899584,96.63681824)
\curveto(261.85846349,96.63681824)(261.61540222,96.54567241)(261.35981201,96.36338076)
\curveto(261.10422181,96.18629744)(260.81355059,95.83473497)(260.48779837,95.30869334)
\lineto(260.48779837,90.95713117)
\lineto(261.0140135,90.95713117)
\curveto(261.16436068,90.95713117)(261.27210949,90.92588117)(261.33725994,90.86338118)
\curveto(261.40241038,90.80608951)(261.4349856,90.72796452)(261.4349856,90.62900619)
\curveto(261.4349856,90.5352562)(261.40241038,90.45713121)(261.33725994,90.39463121)
\curveto(261.27210949,90.33213122)(261.16436068,90.30088122)(261.0140135,90.30088122)
\lineto(259.84882285,90.30088122)
\lineto(259.84882285,95.43369333)
\curveto(259.84882285,95.78265164)(259.7485914,96.06910995)(259.5481285,96.29306826)
\curveto(259.35267716,96.52223491)(259.13216796,96.63681824)(258.8866009,96.63681824)
\curveto(258.66108013,96.63681824)(258.43806515,96.56129741)(258.21755595,96.41025576)
\curveto(257.91185002,96.1967141)(257.59110937,95.82952663)(257.255334,95.30869334)
\lineto(257.255334,90.95713117)
\lineto(257.78906649,90.95713117)
\curveto(257.93941367,90.95713117)(258.04716248,90.92588117)(258.11231293,90.86338118)
\curveto(258.17746337,90.80608951)(258.21003859,90.72796452)(258.21003859,90.62900619)
\curveto(258.21003859,90.5352562)(258.17746337,90.45713121)(258.11231293,90.39463121)
\curveto(258.04716248,90.33213122)(257.93941367,90.30088122)(257.78906649,90.30088122)
\lineto(256.09014336,90.30088122)
\curveto(255.93979618,90.30088122)(255.83204737,90.33213122)(255.76689692,90.39463121)
\curveto(255.70174648,90.45713121)(255.66917125,90.5352562)(255.66917125,90.62900619)
\curveto(255.66917125,90.72796452)(255.70174648,90.80608951)(255.76689692,90.86338118)
\curveto(255.83204737,90.92588117)(255.93979618,90.95713117)(256.09014336,90.95713117)
\lineto(256.62387585,90.95713117)
\lineto(256.62387585,96.40244326)
\lineto(256.09014336,96.40244326)
\curveto(255.93979618,96.40244326)(255.83204737,96.43369325)(255.76689692,96.49619325)
\curveto(255.70174648,96.55869324)(255.66917125,96.6394224)(255.66917125,96.73838073)
\curveto(255.66917125,96.83213072)(255.70174648,96.91025572)(255.76689692,96.97275571)
\curveto(255.83204737,97.03525571)(255.93979618,97.06650571)(256.09014336,97.06650571)
\closepath
}
}
{
\newrgbcolor{curcolor}{0 0 0}
\pscustom[linestyle=none,fillstyle=solid,fillcolor=curcolor]
{
\newpath
\moveto(266.99031522,97.06650571)
\lineto(266.99031522,95.8711933)
\curveto(267.36618317,96.34515159)(267.76961477,96.7019224)(268.20061002,96.94150571)
\curveto(268.63160527,97.18108903)(269.14027989,97.30088069)(269.7266339,97.30088069)
\curveto(270.34806891,97.30088069)(270.92189398,97.14983903)(271.44810911,96.84775572)
\curveto(271.97432423,96.54567241)(272.38026162,96.12379744)(272.66592126,95.58213082)
\curveto(272.95659248,95.04567253)(273.10192808,94.4805684)(273.10192808,93.88681845)
\curveto(273.10192808,92.94411018)(272.77617586,92.13681858)(272.12467141,91.46494363)
\curveto(271.47817854,90.79827701)(270.68133849,90.46494371)(269.73415125,90.46494371)
\curveto(268.60654741,90.46494371)(267.69193539,90.94150617)(266.99031522,91.8946311)
\lineto(266.99031522,87.9415064)
\lineto(268.52385646,87.9415064)
\curveto(268.67420364,87.9415064)(268.78195245,87.91286057)(268.84710289,87.8555689)
\curveto(268.91225334,87.79306891)(268.94482856,87.71233975)(268.94482856,87.61338142)
\curveto(268.94482856,87.51963143)(268.91225334,87.44150644)(268.84710289,87.37900644)
\curveto(268.78195245,87.31650644)(268.67420364,87.28525645)(268.52385646,87.28525645)
\lineto(265.5093955,87.28525645)
\curveto(265.35904832,87.28525645)(265.25129951,87.31650644)(265.18614906,87.37900644)
\curveto(265.12099862,87.4362981)(265.0884234,87.5144231)(265.0884234,87.61338142)
\curveto(265.0884234,87.71233975)(265.12099862,87.79306891)(265.18614906,87.8555689)
\curveto(265.25129951,87.91286057)(265.35904832,87.9415064)(265.5093955,87.9415064)
\lineto(266.35885707,87.9415064)
\lineto(266.35885707,96.40244326)
\lineto(265.5093955,96.40244326)
\curveto(265.35904832,96.40244326)(265.25129951,96.43369325)(265.18614906,96.49619325)
\curveto(265.12099862,96.55869324)(265.0884234,96.6394224)(265.0884234,96.73838073)
\curveto(265.0884234,96.83213072)(265.12099862,96.91025572)(265.18614906,96.97275571)
\curveto(265.25129951,97.03525571)(265.35904832,97.06650571)(265.5093955,97.06650571)
\closepath
\moveto(272.46295257,93.88681845)
\curveto(272.46295257,94.64202672)(272.19733922,95.28786001)(271.66611252,95.8243183)
\curveto(271.13989739,96.36598493)(270.4959103,96.63681824)(269.73415125,96.63681824)
\curveto(268.96738064,96.63681824)(268.31838198,96.36598493)(267.78715527,95.8243183)
\curveto(267.25592857,95.28265167)(266.99031522,94.63681839)(266.99031522,93.88681845)
\curveto(266.99031522,93.13161017)(267.25592857,92.48317272)(267.78715527,91.94150609)
\curveto(268.31838198,91.39983947)(268.96738064,91.12900616)(269.73415125,91.12900616)
\curveto(270.49089873,91.12900616)(271.13488581,91.3972353)(271.66611252,91.93369359)
\curveto(272.19733922,92.47536022)(272.46295257,93.12640184)(272.46295257,93.88681845)
\closepath
}
}
{
\newrgbcolor{curcolor}{0 0 0}
\pscustom[linestyle=none,fillstyle=solid,fillcolor=curcolor]
{
\newpath
\moveto(277.85289852,97.06650571)
\lineto(277.85289852,95.41025583)
\curveto(278.67479644,96.18108911)(279.28871409,96.67588073)(279.69465148,96.89463072)
\curveto(280.10560043,97.11858903)(280.48397417,97.23056819)(280.82977268,97.23056819)
\curveto(281.20564063,97.23056819)(281.55394493,97.0977557)(281.87468558,96.83213072)
\curveto(282.20043781,96.57171408)(282.36331392,96.37379742)(282.36331392,96.23838077)
\curveto(282.36331392,96.13942244)(282.3307387,96.05608912)(282.26558825,95.98838079)
\curveto(282.20544938,95.92588079)(282.12777,95.89463079)(282.03255012,95.89463079)
\curveto(281.9824344,95.89463079)(281.93983603,95.90244329)(281.90475502,95.91806829)
\curveto(281.86967401,95.93890162)(281.80452357,95.99879745)(281.70930369,96.09775578)
\curveto(281.53389864,96.28004743)(281.38104568,96.40504742)(281.25074479,96.47275575)
\curveto(281.1204439,96.54046408)(280.9926488,96.57431824)(280.86735948,96.57431824)
\curveto(280.59172298,96.57431824)(280.2584534,96.45973492)(279.86755073,96.23056827)
\curveto(279.48165964,96.00140162)(278.8101089,95.44150583)(277.85289852,94.5508809)
\lineto(277.85289852,90.95713117)
\lineto(280.64935607,90.95713117)
\curveto(280.80471482,90.95713117)(280.91496942,90.92588117)(280.98011986,90.86338118)
\curveto(281.04527031,90.80608951)(281.07784553,90.72796452)(281.07784553,90.62900619)
\curveto(281.07784553,90.5352562)(281.04527031,90.45713121)(280.98011986,90.39463121)
\curveto(280.91496942,90.33213122)(280.80471482,90.30088122)(280.64935607,90.30088122)
\lineto(275.69541649,90.30088122)
\curveto(275.54506931,90.30088122)(275.4373205,90.32952705)(275.37217005,90.38681871)
\curveto(275.30701961,90.44931871)(275.27444439,90.5274437)(275.27444439,90.62119369)
\curveto(275.27444439,90.70973535)(275.30451382,90.78265202)(275.3646527,90.83994368)
\curveto(275.42980314,90.90244367)(275.54005774,90.93369367)(275.69541649,90.93369367)
\lineto(277.22144037,90.93369367)
\lineto(277.22144037,96.40244326)
\lineto(276.05624972,96.40244326)
\curveto(275.90590254,96.40244326)(275.79815373,96.43369325)(275.73300329,96.49619325)
\curveto(275.66785284,96.55869324)(275.63527762,96.6394224)(275.63527762,96.73838073)
\curveto(275.63527762,96.83213072)(275.66534705,96.91025572)(275.72548593,96.97275571)
\curveto(275.79063637,97.03525571)(275.90089097,97.06650571)(276.05624972,97.06650571)
\closepath
}
}
{
\newrgbcolor{curcolor}{0 0 0}
\pscustom[linestyle=none,fillstyle=solid,fillcolor=curcolor]
{
\newpath
\moveto(291.26386652,93.52744347)
\lineto(284.82900722,93.52744347)
\curveto(284.93926182,92.6784852)(285.28004876,91.99358942)(285.85136805,91.47275613)
\curveto(286.4276989,90.95713117)(287.13934222,90.69931869)(287.986298,90.69931869)
\curveto(288.45738583,90.69931869)(288.95102574,90.78004785)(289.46721772,90.94150617)
\curveto(289.98340971,91.10296449)(290.40438181,91.31650614)(290.73013403,91.58213112)
\curveto(290.82535391,91.66025612)(290.90804486,91.69931861)(290.97820688,91.69931861)
\curveto(291.05839204,91.69931861)(291.12855406,91.66546445)(291.18869293,91.59775612)
\curveto(291.2488318,91.53525612)(291.27890124,91.4597353)(291.27890124,91.37119364)
\curveto(291.27890124,91.28265198)(291.23880866,91.19671448)(291.1586235,91.11338116)
\curveto(290.91806801,90.85296451)(290.48957855,90.60817286)(289.87315511,90.37900621)
\curveto(289.26174324,90.1550479)(288.63279087,90.04306874)(287.986298,90.04306874)
\curveto(286.90379831,90.04306874)(285.99920944,90.41025621)(285.2725314,91.14463115)
\curveto(284.55086494,91.88421443)(284.19003171,92.77744353)(284.19003171,93.82431845)
\curveto(284.19003171,94.77744338)(284.52831286,95.59515165)(285.20487517,96.27744327)
\curveto(285.88644905,96.95973488)(286.72839326,97.30088069)(287.73070779,97.30088069)
\curveto(288.76309176,97.30088069)(289.61255333,96.94931821)(290.27909249,96.24619327)
\curveto(290.94563166,95.54827665)(291.27388967,94.64202672)(291.26386652,93.52744347)
\closepath
\moveto(290.62489101,94.19150592)
\curveto(290.49960169,94.9154642)(290.1688379,95.50400582)(289.63259962,95.95713079)
\curveto(289.10137292,96.41025576)(288.46740898,96.63681824)(287.73070779,96.63681824)
\curveto(286.99400661,96.63681824)(286.36004267,96.41285992)(285.82881597,95.96494329)
\curveto(285.29758927,95.51702666)(284.96682547,94.92588087)(284.83652458,94.19150592)
\closepath
}
}
{
\newrgbcolor{curcolor}{0 0 0}
\pscustom[linestyle=none,fillstyle=solid,fillcolor=curcolor]
{
\newpath
\moveto(299.17213127,96.63681824)
\curveto(299.17213127,96.78785989)(299.20220071,96.89723488)(299.26233958,96.96494321)
\curveto(299.32247845,97.03265154)(299.39765204,97.06650571)(299.48786035,97.06650571)
\curveto(299.58308023,97.06650571)(299.66075961,97.03265154)(299.72089848,96.96494321)
\curveto(299.78103735,96.89723488)(299.81110679,96.78265156)(299.81110679,96.62119324)
\lineto(299.81110679,95.49619332)
\curveto(299.81110679,95.33994334)(299.78103735,95.22796418)(299.72089848,95.16025585)
\curveto(299.66075961,95.09254752)(299.58308023,95.05869336)(299.48786035,95.05869336)
\curveto(299.40266362,95.05869336)(299.32999581,95.08733919)(299.26985694,95.14463085)
\curveto(299.21472964,95.20192251)(299.18215442,95.29567251)(299.17213127,95.42588083)
\curveto(299.14206184,95.73838081)(298.98670309,95.99619329)(298.70605502,96.19931827)
\curveto(298.29510606,96.49098492)(297.75135042,96.63681824)(297.07478811,96.63681824)
\curveto(296.36815637,96.63681824)(295.81938916,96.48838075)(295.42848649,96.19150577)
\curveto(295.13280371,95.96754746)(294.98496231,95.71754747)(294.98496231,95.44150583)
\curveto(294.98496231,95.12900585)(295.16036736,94.86858921)(295.51117744,94.66025589)
\curveto(295.75173293,94.51442257)(296.20778604,94.40244341)(296.87933678,94.32431841)
\curveto(297.756362,94.22536009)(298.36526808,94.11338093)(298.70605502,93.98838094)
\curveto(299.19217756,93.80608929)(299.5530108,93.55348514)(299.78855471,93.2305685)
\curveto(300.0291102,92.90765185)(300.14938794,92.55869355)(300.14938794,92.18369358)
\curveto(300.14938794,91.62640195)(299.89129195,91.12900616)(299.37509997,90.69150619)
\curveto(298.85890798,90.25921455)(298.10216051,90.04306874)(297.10485755,90.04306874)
\curveto(296.10755459,90.04306874)(295.29066825,90.30608955)(294.65419852,90.83213118)
\curveto(294.65419852,90.65504786)(294.64417537,90.54046453)(294.62412908,90.4883812)
\curveto(294.60408279,90.43629787)(294.566496,90.39202704)(294.5113687,90.35556871)
\curveto(294.46125297,90.31911038)(294.40361989,90.30088122)(294.33846944,90.30088122)
\curveto(294.24826113,90.30088122)(294.17308754,90.33473538)(294.11294867,90.40244371)
\curveto(294.0528098,90.47015204)(294.02274036,90.5821312)(294.02274036,90.73838119)
\lineto(294.02274036,92.08994358)
\curveto(294.02274036,92.24619357)(294.05030401,92.35817273)(294.10543131,92.42588106)
\curveto(294.16557018,92.49358939)(294.24324956,92.52744355)(294.33846944,92.52744355)
\curveto(294.42867775,92.52744355)(294.50385134,92.49358939)(294.56399021,92.42588106)
\curveto(294.62914065,92.36338106)(294.66171588,92.27744357)(294.66171588,92.16806858)
\curveto(294.66171588,91.92848526)(294.71934896,91.72796444)(294.83461513,91.56650612)
\curveto(295.01002018,91.31650614)(295.28816246,91.10817282)(295.66904198,90.94150617)
\curveto(296.05493308,90.78004785)(296.52602091,90.69931869)(297.08230547,90.69931869)
\curveto(297.90420339,90.69931869)(298.51561526,90.85817284)(298.91654107,91.17588115)
\curveto(299.31746688,91.49358946)(299.51792979,91.82952694)(299.51792979,92.18369358)
\curveto(299.51792979,92.58994354)(299.31496109,92.91546435)(298.90902371,93.160256)
\curveto(298.49807475,93.40504765)(297.89919182,93.56911014)(297.11237491,93.65244346)
\curveto(296.33056957,93.73577679)(295.76927344,93.84515178)(295.42848649,93.98056844)
\curveto(295.08769955,94.1159851)(294.8220862,94.31911008)(294.63164644,94.58994339)
\curveto(294.44120668,94.86077671)(294.3459868,95.15244335)(294.3459868,95.46494333)
\curveto(294.3459868,96.02744328)(294.61160015,96.47275575)(295.14282685,96.80088073)
\curveto(295.67405356,97.13421403)(296.3080175,97.30088069)(297.04471868,97.30088069)
\curveto(297.91673232,97.30088069)(298.62586985,97.07952654)(299.17213127,96.63681824)
\closepath
}
}
{
\newrgbcolor{curcolor}{0 0 0}
\pscustom[linestyle=none,fillstyle=solid,fillcolor=curcolor]
{
\newpath
\moveto(308.4109615,96.63681824)
\curveto(308.4109615,96.78785989)(308.44103093,96.89723488)(308.5011698,96.96494321)
\curveto(308.56130868,97.03265154)(308.63648227,97.06650571)(308.72669057,97.06650571)
\curveto(308.82191045,97.06650571)(308.89958983,97.03265154)(308.9597287,96.96494321)
\curveto(309.01986757,96.89723488)(309.04993701,96.78265156)(309.04993701,96.62119324)
\lineto(309.04993701,95.49619332)
\curveto(309.04993701,95.33994334)(309.01986757,95.22796418)(308.9597287,95.16025585)
\curveto(308.89958983,95.09254752)(308.82191045,95.05869336)(308.72669057,95.05869336)
\curveto(308.64149384,95.05869336)(308.56882604,95.08733919)(308.50868716,95.14463085)
\curveto(308.45355986,95.20192251)(308.42098464,95.29567251)(308.4109615,95.42588083)
\curveto(308.38089206,95.73838081)(308.22553331,95.99619329)(307.94488524,96.19931827)
\curveto(307.53393628,96.49098492)(306.99018065,96.63681824)(306.31361834,96.63681824)
\curveto(305.60698659,96.63681824)(305.05821938,96.48838075)(304.66731672,96.19150577)
\curveto(304.37163393,95.96754746)(304.22379254,95.71754747)(304.22379254,95.44150583)
\curveto(304.22379254,95.12900585)(304.39919758,94.86858921)(304.75000767,94.66025589)
\curveto(304.99056315,94.51442257)(305.44661627,94.40244341)(306.118167,94.32431841)
\curveto(306.99519222,94.22536009)(307.6040983,94.11338093)(307.94488524,93.98838094)
\curveto(308.43100779,93.80608929)(308.79184102,93.55348514)(309.02738493,93.2305685)
\curveto(309.26794042,92.90765185)(309.38821817,92.55869355)(309.38821817,92.18369358)
\curveto(309.38821817,91.62640195)(309.13012217,91.12900616)(308.61393019,90.69150619)
\curveto(308.0977382,90.25921455)(307.34099073,90.04306874)(306.34368777,90.04306874)
\curveto(305.34638481,90.04306874)(304.52949847,90.30608955)(303.89302874,90.83213118)
\curveto(303.89302874,90.65504786)(303.8830056,90.54046453)(303.8629593,90.4883812)
\curveto(303.84291301,90.43629787)(303.80532622,90.39202704)(303.75019892,90.35556871)
\curveto(303.70008319,90.31911038)(303.64245011,90.30088122)(303.57729966,90.30088122)
\curveto(303.48709135,90.30088122)(303.41191776,90.33473538)(303.35177889,90.40244371)
\curveto(303.29164002,90.47015204)(303.26157059,90.5821312)(303.26157059,90.73838119)
\lineto(303.26157059,92.08994358)
\curveto(303.26157059,92.24619357)(303.28913423,92.35817273)(303.34426153,92.42588106)
\curveto(303.40440041,92.49358939)(303.48207978,92.52744355)(303.57729966,92.52744355)
\curveto(303.66750797,92.52744355)(303.74268156,92.49358939)(303.80282043,92.42588106)
\curveto(303.86797088,92.36338106)(303.9005461,92.27744357)(303.9005461,92.16806858)
\curveto(303.9005461,91.92848526)(303.95817919,91.72796444)(304.07344536,91.56650612)
\curveto(304.2488504,91.31650614)(304.52699268,91.10817282)(304.9078722,90.94150617)
\curveto(305.2937633,90.78004785)(305.76485113,90.69931869)(306.3211357,90.69931869)
\curveto(307.14303361,90.69931869)(307.75444548,90.85817284)(308.15537129,91.17588115)
\curveto(308.5562971,91.49358946)(308.75676001,91.82952694)(308.75676001,92.18369358)
\curveto(308.75676001,92.58994354)(308.55379132,92.91546435)(308.14785393,93.160256)
\curveto(307.73690497,93.40504765)(307.13802204,93.56911014)(306.35120513,93.65244346)
\curveto(305.5693998,93.73577679)(305.00810366,93.84515178)(304.66731672,93.98056844)
\curveto(304.32652978,94.1159851)(304.06091642,94.31911008)(303.87047666,94.58994339)
\curveto(303.6800369,94.86077671)(303.58481702,95.15244335)(303.58481702,95.46494333)
\curveto(303.58481702,96.02744328)(303.85043037,96.47275575)(304.38165708,96.80088073)
\curveto(304.91288378,97.13421403)(305.54684772,97.30088069)(306.2835489,97.30088069)
\curveto(307.15556254,97.30088069)(307.86470008,97.07952654)(308.4109615,96.63681824)
\closepath
}
}
{
\newrgbcolor{curcolor}{0 0 0}
\pscustom[linestyle=none,fillstyle=solid,fillcolor=curcolor]
{
\newpath
\moveto(318.98037133,93.52744347)
\lineto(312.54551203,93.52744347)
\curveto(312.65576663,92.6784852)(312.99655357,91.99358942)(313.56787286,91.47275613)
\curveto(314.14420371,90.95713117)(314.85584703,90.69931869)(315.70280281,90.69931869)
\curveto(316.17389064,90.69931869)(316.66753055,90.78004785)(317.18372253,90.94150617)
\curveto(317.69991452,91.10296449)(318.12088662,91.31650614)(318.44663884,91.58213112)
\curveto(318.54185872,91.66025612)(318.62454967,91.69931861)(318.69471169,91.69931861)
\curveto(318.77489685,91.69931861)(318.84505887,91.66546445)(318.90519774,91.59775612)
\curveto(318.96533661,91.53525612)(318.99540605,91.4597353)(318.99540605,91.37119364)
\curveto(318.99540605,91.28265198)(318.95531347,91.19671448)(318.87512831,91.11338116)
\curveto(318.63457282,90.85296451)(318.20608336,90.60817286)(317.58965992,90.37900621)
\curveto(316.97824805,90.1550479)(316.34929568,90.04306874)(315.70280281,90.04306874)
\curveto(314.62030312,90.04306874)(313.71571425,90.41025621)(312.98903621,91.14463115)
\curveto(312.26736975,91.88421443)(311.90653652,92.77744353)(311.90653652,93.82431845)
\curveto(311.90653652,94.77744338)(312.24481767,95.59515165)(312.92137998,96.27744327)
\curveto(313.60295386,96.95973488)(314.44489807,97.30088069)(315.4472126,97.30088069)
\curveto(316.47959657,97.30088069)(317.32905814,96.94931821)(317.9955973,96.24619327)
\curveto(318.66213647,95.54827665)(318.99039448,94.64202672)(318.98037133,93.52744347)
\closepath
\moveto(318.34139582,94.19150592)
\curveto(318.2161065,94.9154642)(317.88534271,95.50400582)(317.34910443,95.95713079)
\curveto(316.81787773,96.41025576)(316.18391379,96.63681824)(315.4472126,96.63681824)
\curveto(314.71051142,96.63681824)(314.07654748,96.41285992)(313.54532078,95.96494329)
\curveto(313.01409408,95.51702666)(312.68333028,94.92588087)(312.55302939,94.19150592)
\closepath
}
}
{
\newrgbcolor{curcolor}{0 0 0}
\pscustom[linestyle=none,fillstyle=solid,fillcolor=curcolor]
{
\newpath
\moveto(327.89596219,100.10556797)
\lineto(327.89596219,90.95713117)
\lineto(328.7379064,90.95713117)
\curveto(328.89326515,90.95713117)(329.00351975,90.92588117)(329.06867019,90.86338118)
\curveto(329.13382064,90.80608951)(329.16639586,90.72796452)(329.16639586,90.62900619)
\curveto(329.16639586,90.5352562)(329.13382064,90.45713121)(329.06867019,90.39463121)
\curveto(329.00351975,90.33213122)(328.89326515,90.30088122)(328.7379064,90.30088122)
\lineto(327.25698667,90.30088122)
\lineto(327.25698667,91.60556862)
\curveto(326.52529707,90.56390203)(325.59565034,90.04306874)(324.46804649,90.04306874)
\curveto(323.8967272,90.04306874)(323.34796,90.19931873)(322.82174487,90.5118187)
\curveto(322.30054131,90.82952701)(321.88708657,91.28004781)(321.58138063,91.8633811)
\curveto(321.28068627,92.44671439)(321.13033909,93.04827684)(321.13033909,93.66806846)
\curveto(321.13033909,94.29306842)(321.28068627,94.89463087)(321.58138063,95.47275583)
\curveto(321.88708657,96.05608912)(322.30054131,96.50660991)(322.82174487,96.82431822)
\curveto(323.34796,97.14202653)(323.89923299,97.30088069)(324.47556385,97.30088069)
\curveto(325.57810983,97.30088069)(326.50525078,96.78004739)(327.25698667,95.73838081)
\lineto(327.25698667,99.44150553)
\lineto(326.41504247,99.44150553)
\curveto(326.25968371,99.44150553)(326.14942912,99.47275552)(326.08427867,99.53525552)
\curveto(326.01912823,99.59775551)(325.986553,99.67848467)(325.986553,99.777443)
\curveto(325.986553,99.87119299)(326.01912823,99.94931799)(326.08427867,100.01181798)
\curveto(326.14942912,100.07431798)(326.25968371,100.10556797)(326.41504247,100.10556797)
\closepath
\moveto(327.25698667,93.66806846)
\curveto(327.25698667,94.50140173)(326.98636175,95.20452668)(326.4451119,95.7774433)
\curveto(325.90386206,96.35035993)(325.25987497,96.63681824)(324.51315064,96.63681824)
\curveto(323.76141474,96.63681824)(323.11492187,96.35035993)(322.57367202,95.7774433)
\curveto(322.03242217,95.20452668)(321.76179725,94.50140173)(321.76179725,93.66806846)
\curveto(321.76179725,92.83994353)(322.03242217,92.13681858)(322.57367202,91.55869362)
\curveto(323.11492187,90.985777)(323.76141474,90.69931869)(324.51315064,90.69931869)
\curveto(325.25987497,90.69931869)(325.90386206,90.985777)(326.4451119,91.55869362)
\curveto(326.98636175,92.13681858)(327.25698667,92.83994353)(327.25698667,93.66806846)
\closepath
}
}
{
\newrgbcolor{curcolor}{0 0 0}
\pscustom[linestyle=none,fillstyle=solid,fillcolor=curcolor]
{
\newpath
\moveto(244.74644776,80.10556949)
\lineto(244.74644776,70.95713269)
\lineto(245.58839197,70.95713269)
\curveto(245.74375072,70.95713269)(245.85400532,70.92588269)(245.91915576,70.86338269)
\curveto(245.98430621,70.80609103)(246.01688143,70.72796604)(246.01688143,70.62900771)
\curveto(246.01688143,70.53525772)(245.98430621,70.45713272)(245.91915576,70.39463273)
\curveto(245.85400532,70.33213273)(245.74375072,70.30088274)(245.58839197,70.30088274)
\lineto(244.10747224,70.30088274)
\lineto(244.10747224,71.60557014)
\curveto(243.37578264,70.56390355)(242.44613591,70.04307026)(241.31853206,70.04307026)
\curveto(240.74721277,70.04307026)(240.19844557,70.19932024)(239.67223044,70.51182022)
\curveto(239.15102688,70.82952853)(238.73757214,71.28004933)(238.4318662,71.86338262)
\curveto(238.13117184,72.44671591)(237.98082466,73.04827836)(237.98082466,73.66806998)
\curveto(237.98082466,74.29306993)(238.13117184,74.89463239)(238.4318662,75.47275734)
\curveto(238.73757214,76.05609063)(239.15102688,76.50661143)(239.67223044,76.82431974)
\curveto(240.19844557,77.14202805)(240.74971856,77.3008822)(241.32604942,77.3008822)
\curveto(242.4285954,77.3008822)(243.35573635,76.78004891)(244.10747224,75.73838232)
\lineto(244.10747224,79.44150704)
\lineto(243.26552804,79.44150704)
\curveto(243.11016929,79.44150704)(242.99991469,79.47275704)(242.93476424,79.53525704)
\curveto(242.8696138,79.59775703)(242.83703857,79.67848619)(242.83703857,79.77744452)
\curveto(242.83703857,79.87119451)(242.8696138,79.9493195)(242.93476424,80.0118195)
\curveto(242.99991469,80.07431949)(243.11016929,80.10556949)(243.26552804,80.10556949)
\closepath
\moveto(244.10747224,73.66806998)
\curveto(244.10747224,74.50140325)(243.83684732,75.2045282)(243.29559747,75.77744482)
\curveto(242.75434763,76.35036144)(242.11036054,76.63681975)(241.36363621,76.63681975)
\curveto(240.61190031,76.63681975)(239.96540744,76.35036144)(239.42415759,75.77744482)
\curveto(238.88290774,75.2045282)(238.61228282,74.50140325)(238.61228282,73.66806998)
\curveto(238.61228282,72.83994504)(238.88290774,72.1368201)(239.42415759,71.55869514)
\curveto(239.96540744,70.98577852)(240.61190031,70.69932021)(241.36363621,70.69932021)
\curveto(242.11036054,70.69932021)(242.75434763,70.98577852)(243.29559747,71.55869514)
\curveto(243.83684732,72.1368201)(244.10747224,72.83994504)(244.10747224,73.66806998)
\closepath
}
}
{
\newrgbcolor{curcolor}{0 0 0}
\pscustom[linestyle=none,fillstyle=solid,fillcolor=curcolor]
{
\newpath
\moveto(252.70733137,70.30088274)
\lineto(252.70733137,71.24619516)
\curveto(251.79021358,70.44411189)(250.81045112,70.04307026)(249.76804401,70.04307026)
\curveto(249.01129653,70.04307026)(248.41993096,70.24098691)(247.99394728,70.63682021)
\curveto(247.56796361,71.03786185)(247.35497177,71.52744514)(247.35497177,72.1055701)
\curveto(247.35497177,72.74098672)(247.63561984,73.29567417)(248.19691598,73.76963247)
\curveto(248.75821211,74.24359077)(249.57760424,74.48056992)(250.65509237,74.48056992)
\curveto(250.94576358,74.48056992)(251.26149266,74.45973659)(251.6022796,74.41806992)
\curveto(251.94306654,74.38161159)(252.31141713,74.32171576)(252.70733137,74.23838244)
\lineto(252.70733137,75.30088236)
\curveto(252.70733137,75.66025733)(252.54696105,75.97275731)(252.2262204,76.23838229)
\curveto(251.90547975,76.50400726)(251.42436877,76.63681975)(250.78288747,76.63681975)
\curveto(250.29175335,76.63681975)(249.60266211,76.48838227)(248.71561375,76.19150729)
\curveto(248.55524342,76.13942396)(248.45250618,76.11338229)(248.40740203,76.11338229)
\curveto(248.32721686,76.11338229)(248.25705485,76.14463229)(248.19691598,76.20713229)
\curveto(248.14178868,76.26963228)(248.11422503,76.34775728)(248.11422503,76.44150727)
\curveto(248.11422503,76.53004893)(248.13928289,76.60036142)(248.18939862,76.65244475)
\curveto(248.25956063,76.73056975)(248.54271449,76.83734057)(249.03886018,76.97275723)
\curveto(249.82066552,77.19150721)(250.41203109,77.3008822)(250.81295691,77.3008822)
\curveto(251.60979696,77.3008822)(252.23123197,77.09515305)(252.67726194,76.68369475)
\curveto(253.1232919,76.27744478)(253.34630689,75.81650732)(253.34630689,75.30088236)
\lineto(253.34630689,70.95713269)
\lineto(254.18825109,70.95713269)
\curveto(254.34360985,70.95713269)(254.45386445,70.92588269)(254.51901489,70.86338269)
\curveto(254.58416533,70.80609103)(254.61674056,70.72796604)(254.61674056,70.62900771)
\curveto(254.61674056,70.53525772)(254.58416533,70.45713272)(254.51901489,70.39463273)
\curveto(254.45386445,70.33213273)(254.34360985,70.30088274)(254.18825109,70.30088274)
\closepath
\moveto(252.70733137,73.56650749)
\curveto(252.41164859,73.65504915)(252.09842529,73.72015331)(251.7676615,73.76181997)
\curveto(251.4368977,73.80348664)(251.0885934,73.82431997)(250.7227486,73.82431997)
\curveto(249.8056308,73.82431997)(249.08897591,73.61859082)(248.57278393,73.20713252)
\curveto(248.18188126,72.89984087)(247.98642992,72.5326534)(247.98642992,72.1055701)
\curveto(247.98642992,71.7097368)(248.13427132,71.37640349)(248.4299541,71.10557017)
\curveto(248.73064846,70.83473686)(249.16665529,70.69932021)(249.73797457,70.69932021)
\curveto(250.28423599,70.69932021)(250.79040483,70.81129936)(251.25648109,71.03525768)
\curveto(251.72756892,71.26442433)(252.21118568,71.62640347)(252.70733137,72.1211951)
\closepath
}
}
{
\newrgbcolor{curcolor}{0 0 0}
\pscustom[linestyle=none,fillstyle=solid,fillcolor=curcolor]
{
\newpath
\moveto(258.83397939,77.06650722)
\lineto(262.26189509,77.06650722)
\curveto(262.41224227,77.06650722)(262.51999108,77.03525722)(262.58514153,76.97275723)
\curveto(262.65029197,76.91025723)(262.6828672,76.82952807)(262.6828672,76.73056975)
\curveto(262.6828672,76.63681975)(262.65029197,76.55869476)(262.58514153,76.49619477)
\curveto(262.51999108,76.43369477)(262.41224227,76.40244477)(262.26189509,76.40244477)
\lineto(258.83397939,76.40244477)
\lineto(258.83397939,72.0352576)
\curveto(258.83397939,71.6550493)(258.979315,71.33734099)(259.26998621,71.08213268)
\curveto(259.565669,70.82692436)(259.99666425,70.69932021)(260.56297196,70.69932021)
\curveto(260.98895564,70.69932021)(261.45002032,70.76442437)(261.94616601,70.89463269)
\curveto(262.44231171,71.03004935)(262.8282028,71.181091)(263.1038393,71.34775766)
\curveto(263.20407075,71.41546598)(263.2867617,71.44932015)(263.35191215,71.44932015)
\curveto(263.43209731,71.44932015)(263.50225933,71.41546598)(263.5623982,71.34775766)
\curveto(263.62253707,71.28525766)(263.65260651,71.20973683)(263.65260651,71.12119517)
\curveto(263.65260651,71.04307018)(263.62003128,70.97015352)(263.55488084,70.90244519)
\curveto(263.39451051,70.7305702)(263.00360785,70.54307022)(262.38217284,70.33994523)
\curveto(261.7657494,70.14202858)(261.17438382,70.04307026)(260.60807611,70.04307026)
\curveto(259.87137493,70.04307026)(259.28502093,70.22275774)(258.84901411,70.58213271)
\curveto(258.41300729,70.94150769)(258.19500388,71.42588265)(258.19500388,72.0352576)
\lineto(258.19500388,76.40244477)
\lineto(257.02981323,76.40244477)
\curveto(256.87946605,76.40244477)(256.77171724,76.43369477)(256.7065668,76.49619477)
\curveto(256.64141635,76.55869476)(256.60884113,76.63942392)(256.60884113,76.73838225)
\curveto(256.60884113,76.83213224)(256.64141635,76.91025723)(256.7065668,76.97275723)
\curveto(256.77171724,77.03525722)(256.87946605,77.06650722)(257.02981323,77.06650722)
\lineto(258.19500388,77.06650722)
\lineto(258.19500388,79.00400708)
\curveto(258.19500388,79.16025706)(258.22507331,79.27223622)(258.28521218,79.33994455)
\curveto(258.34535106,79.40765288)(258.42052465,79.44150704)(258.51073295,79.44150704)
\curveto(258.60595283,79.44150704)(258.68363221,79.40765288)(258.74377108,79.33994455)
\curveto(258.80390995,79.27223622)(258.83397939,79.16025706)(258.83397939,79.00400708)
\closepath
}
}
{
\newrgbcolor{curcolor}{0 0 0}
\pscustom[linestyle=none,fillstyle=solid,fillcolor=curcolor]
{
\newpath
\moveto(271.18500154,70.30088274)
\lineto(271.18500154,71.24619516)
\curveto(270.26788374,70.44411189)(269.28812129,70.04307026)(268.24571417,70.04307026)
\curveto(267.4889667,70.04307026)(266.89760113,70.24098691)(266.47161745,70.63682021)
\curveto(266.04563377,71.03786185)(265.83264194,71.52744514)(265.83264194,72.1055701)
\curveto(265.83264194,72.74098672)(266.11329001,73.29567417)(266.67458614,73.76963247)
\curveto(267.23588228,74.24359077)(268.05527441,74.48056992)(269.13276253,74.48056992)
\curveto(269.42343375,74.48056992)(269.73916283,74.45973659)(270.07994977,74.41806992)
\curveto(270.42073671,74.38161159)(270.7890873,74.32171576)(271.18500154,74.23838244)
\lineto(271.18500154,75.30088236)
\curveto(271.18500154,75.66025733)(271.02463122,75.97275731)(270.70389056,76.23838229)
\curveto(270.38314991,76.50400726)(269.90203894,76.63681975)(269.26055764,76.63681975)
\curveto(268.76942352,76.63681975)(268.08033228,76.48838227)(267.19328391,76.19150729)
\curveto(267.03291359,76.13942396)(266.93017635,76.11338229)(266.8850722,76.11338229)
\curveto(266.80488703,76.11338229)(266.73472502,76.14463229)(266.67458614,76.20713229)
\curveto(266.61945884,76.26963228)(266.59189519,76.34775728)(266.59189519,76.44150727)
\curveto(266.59189519,76.53004893)(266.61695306,76.60036142)(266.66706878,76.65244475)
\curveto(266.7372308,76.73056975)(267.02038466,76.83734057)(267.51653035,76.97275723)
\curveto(268.29833569,77.19150721)(268.88970126,77.3008822)(269.29062707,77.3008822)
\curveto(270.08746713,77.3008822)(270.70890214,77.09515305)(271.1549321,76.68369475)
\curveto(271.60096207,76.27744478)(271.82397705,75.81650732)(271.82397705,75.30088236)
\lineto(271.82397705,70.95713269)
\lineto(272.66592126,70.95713269)
\curveto(272.82128001,70.95713269)(272.93153461,70.92588269)(272.99668506,70.86338269)
\curveto(273.0618355,70.80609103)(273.09441072,70.72796604)(273.09441072,70.62900771)
\curveto(273.09441072,70.53525772)(273.0618355,70.45713272)(272.99668506,70.39463273)
\curveto(272.93153461,70.33213273)(272.82128001,70.30088274)(272.66592126,70.30088274)
\closepath
\moveto(271.18500154,73.56650749)
\curveto(270.88931875,73.65504915)(270.57609546,73.72015331)(270.24533167,73.76181997)
\curveto(269.91456787,73.80348664)(269.56626357,73.82431997)(269.20041877,73.82431997)
\curveto(268.28330097,73.82431997)(267.56664608,73.61859082)(267.05045409,73.20713252)
\curveto(266.65955143,72.89984087)(266.46410009,72.5326534)(266.46410009,72.1055701)
\curveto(266.46410009,71.7097368)(266.61194149,71.37640349)(266.90762427,71.10557017)
\curveto(267.20831863,70.83473686)(267.64432545,70.69932021)(268.21564474,70.69932021)
\curveto(268.76190616,70.69932021)(269.268075,70.81129936)(269.73415125,71.03525768)
\curveto(270.20523908,71.26442433)(270.68885585,71.62640347)(271.18500154,72.1211951)
\closepath
}
}
{
\newrgbcolor{curcolor}{0 0 0}
\pscustom[linewidth=1.88976378,linecolor=curcolor]
{
\newpath
\moveto(322.68297931,200.22107986)
\lineto(466.09932847,200.22107986)
\lineto(466.09932847,178.26081567)
\lineto(322.68297931,178.26081567)
\closepath
}
}
{
\newrgbcolor{curcolor}{0 0 0}
\pscustom[linestyle=none,fillstyle=solid,fillcolor=curcolor]
{
\newpath
\moveto(327.00286788,196.7439892)
\lineto(327.00286788,192.39242703)
\curveto(327.76462692,193.42367695)(328.68425051,193.93930191)(329.76173863,193.93930191)
\curveto(330.683868,193.93930191)(331.47319069,193.59034361)(332.12970671,192.89242699)
\curveto(332.78622273,192.19971871)(333.11448074,191.34815628)(333.11448074,190.33773969)
\curveto(333.11448074,189.31690643)(332.78121116,188.45232316)(332.11467199,187.74398988)
\curveto(331.4531444,187.0356566)(330.66883328,186.68148996)(329.76173863,186.68148996)
\curveto(328.65919264,186.68148996)(327.73956906,187.19711493)(327.00286788,188.22836485)
\lineto(327.00286788,186.93930245)
\lineto(325.52194816,186.93930245)
\curveto(325.37160098,186.93930245)(325.26385216,186.97055244)(325.19870172,187.03305244)
\curveto(325.13355127,187.09555243)(325.10097605,187.17367743)(325.10097605,187.26742742)
\curveto(325.10097605,187.36638575)(325.13355127,187.44451074)(325.19870172,187.5018024)
\curveto(325.26385216,187.5643024)(325.37160098,187.5955524)(325.52194816,187.5955524)
\lineto(326.37140972,187.5955524)
\lineto(326.37140972,196.07992675)
\lineto(325.52194816,196.07992675)
\curveto(325.37160098,196.07992675)(325.26385216,196.11117675)(325.19870172,196.17367674)
\curveto(325.13355127,196.23617674)(325.10097605,196.3169059)(325.10097605,196.41586423)
\curveto(325.10097605,196.50961422)(325.13355127,196.58773921)(325.19870172,196.65023921)
\curveto(325.26385216,196.7127392)(325.37160098,196.7439892)(325.52194816,196.7439892)
\closepath
\moveto(332.48302259,190.30648969)
\curveto(332.48302259,191.13461463)(332.20989187,191.83513541)(331.66363045,192.40805203)
\curveto(331.11736903,192.98617699)(330.47839352,193.27523946)(329.74670391,193.27523946)
\curveto(329.0150143,193.27523946)(328.37603879,192.98617699)(327.82977737,192.40805203)
\curveto(327.28351595,191.83513541)(327.01038524,191.13461463)(327.01038524,190.30648969)
\curveto(327.01038524,189.47836475)(327.28351595,188.77523981)(327.82977737,188.19711485)
\curveto(328.37603879,187.62419823)(329.0150143,187.33773992)(329.74670391,187.33773992)
\curveto(330.47839352,187.33773992)(331.11736903,187.62419823)(331.66363045,188.19711485)
\curveto(332.20989187,188.77523981)(332.48302259,189.47836475)(332.48302259,190.30648969)
\closepath
}
}
{
\newrgbcolor{curcolor}{0 0 0}
\pscustom[linestyle=none,fillstyle=solid,fillcolor=curcolor]
{
\newpath
\moveto(342.81939164,182.55649028)
\lineto(334.42250164,182.55649028)
\curveto(334.27215446,182.55649028)(334.16440565,182.58774028)(334.09925521,182.65024027)
\curveto(334.03410476,182.70753193)(334.00152954,182.78565693)(334.00152954,182.88461525)
\curveto(334.00152954,182.98357358)(334.03410476,183.06430274)(334.09925521,183.12680273)
\curveto(334.16440565,183.1840944)(334.27215446,183.21274023)(334.42250164,183.21274023)
\lineto(342.81939164,183.21274023)
\curveto(342.97475039,183.21274023)(343.08249921,183.1840944)(343.14263808,183.12680273)
\curveto(343.20778852,183.06430274)(343.24036374,182.98357358)(343.24036374,182.88461525)
\curveto(343.24036374,182.78565693)(343.20778852,182.70753193)(343.14263808,182.65024027)
\curveto(343.08249921,182.58774028)(342.97475039,182.55649028)(342.81939164,182.55649028)
\closepath
}
}
{
\newrgbcolor{curcolor}{0 0 0}
\pscustom[linestyle=none,fillstyle=solid,fillcolor=curcolor]
{
\newpath
\moveto(345.81881832,193.70492693)
\lineto(345.81881832,192.71273951)
\curveto(346.25983671,193.17628114)(346.65825674,193.49659361)(347.0140784,193.67367693)
\curveto(347.36990006,193.85076025)(347.77082587,193.93930191)(348.21685584,193.93930191)
\curveto(348.69796681,193.93930191)(349.13647942,193.83253109)(349.53239366,193.61898944)
\curveto(349.81304173,193.46273945)(350.06612615,193.2023228)(350.29164692,192.8377395)
\curveto(350.52217926,192.47836453)(350.63744543,192.10857289)(350.63744543,191.72836458)
\lineto(350.63744543,187.5955524)
\lineto(351.17117792,187.5955524)
\curveto(351.3215251,187.5955524)(351.42927391,187.5643024)(351.49442436,187.5018024)
\curveto(351.5595748,187.44451074)(351.59215002,187.36638575)(351.59215002,187.26742742)
\curveto(351.59215002,187.17367743)(351.5595748,187.09555243)(351.49442436,187.03305244)
\curveto(351.42927391,186.97055244)(351.3215251,186.93930245)(351.17117792,186.93930245)
\lineto(349.47977215,186.93930245)
\curveto(349.32441339,186.93930245)(349.2141588,186.97055244)(349.14900835,187.03305244)
\curveto(349.08385791,187.09555243)(349.05128268,187.17367743)(349.05128268,187.26742742)
\curveto(349.05128268,187.36638575)(349.08385791,187.44451074)(349.14900835,187.5018024)
\curveto(349.2141588,187.5643024)(349.32441339,187.5955524)(349.47977215,187.5955524)
\lineto(350.00598728,187.5955524)
\lineto(350.00598728,191.61898959)
\curveto(350.00598728,192.08253122)(349.84311117,192.47315619)(349.51735894,192.7908645)
\curveto(349.19160672,193.11378114)(348.7555999,193.27523946)(348.20933848,193.27523946)
\curveto(347.79337795,193.27523946)(347.43254471,193.1866978)(347.12683878,193.00961448)
\curveto(346.82113285,192.8377395)(346.38512603,192.40544786)(345.81881832,191.71273958)
\lineto(345.81881832,187.5955524)
\lineto(346.53296742,187.5955524)
\curveto(346.6833146,187.5955524)(346.79106341,187.5643024)(346.85621386,187.5018024)
\curveto(346.9213643,187.44451074)(346.95393952,187.36638575)(346.95393952,187.26742742)
\curveto(346.95393952,187.17367743)(346.9213643,187.09555243)(346.85621386,187.03305244)
\curveto(346.79106341,186.97055244)(346.6833146,186.93930245)(346.53296742,186.93930245)
\lineto(344.47321106,186.93930245)
\curveto(344.32286388,186.93930245)(344.21511506,186.97055244)(344.14996462,187.03305244)
\curveto(344.08481418,187.09555243)(344.05223895,187.17367743)(344.05223895,187.26742742)
\curveto(344.05223895,187.36638575)(344.08481418,187.44451074)(344.14996462,187.5018024)
\curveto(344.21511506,187.5643024)(344.32286388,187.5955524)(344.47321106,187.5955524)
\lineto(345.18736016,187.5955524)
\lineto(345.18736016,193.04086448)
\lineto(344.65362767,193.04086448)
\curveto(344.50328049,193.04086448)(344.39553168,193.07211448)(344.33038124,193.13461448)
\curveto(344.26523079,193.19711447)(344.23265557,193.27784363)(344.23265557,193.37680196)
\curveto(344.23265557,193.47055195)(344.26523079,193.54867694)(344.33038124,193.61117694)
\curveto(344.39553168,193.67367693)(344.50328049,193.70492693)(344.65362767,193.70492693)
\closepath
}
}
{
\newrgbcolor{curcolor}{0 0 0}
\pscustom[linestyle=none,fillstyle=solid,fillcolor=curcolor]
{
\newpath
\moveto(360.51525647,190.1658647)
\lineto(354.08039717,190.1658647)
\curveto(354.19065177,189.31690643)(354.53143871,188.63201065)(355.102758,188.11117736)
\curveto(355.67908885,187.5955524)(356.39073217,187.33773992)(357.23768795,187.33773992)
\curveto(357.70877578,187.33773992)(358.20241569,187.41846908)(358.71860767,187.5799274)
\curveto(359.23479966,187.74138572)(359.65577176,187.95492737)(359.98152398,188.22055235)
\curveto(360.07674386,188.29867734)(360.15943481,188.33773984)(360.22959683,188.33773984)
\curveto(360.30978199,188.33773984)(360.37994401,188.30388568)(360.44008288,188.23617735)
\curveto(360.50022175,188.17367735)(360.53029119,188.09815652)(360.53029119,188.00961486)
\curveto(360.53029119,187.9210732)(360.49019861,187.83513571)(360.41001345,187.75180238)
\curveto(360.16945796,187.49138574)(359.7409685,187.24659409)(359.12454506,187.01742744)
\curveto(358.51313319,186.79346912)(357.88418082,186.68148996)(357.23768795,186.68148996)
\curveto(356.15518826,186.68148996)(355.25059939,187.04867744)(354.52392135,187.78305238)
\curveto(353.80225489,188.52263566)(353.44142166,189.41586476)(353.44142166,190.46273968)
\curveto(353.44142166,191.41586461)(353.77970281,192.23357288)(354.45626512,192.91586449)
\curveto(355.13783901,193.59815611)(355.97978321,193.93930191)(356.98209775,193.93930191)
\curveto(358.01448171,193.93930191)(358.86394328,193.58773944)(359.53048244,192.88461449)
\curveto(360.19702161,192.18669788)(360.52527962,191.28044795)(360.51525647,190.1658647)
\closepath
\moveto(359.87628096,190.82992715)
\curveto(359.75099164,191.55388543)(359.42022785,192.14242705)(358.88398957,192.59555202)
\curveto(358.35276287,193.04867698)(357.71879893,193.27523946)(356.98209775,193.27523946)
\curveto(356.24539656,193.27523946)(355.61143262,193.05128115)(355.08020592,192.60336452)
\curveto(354.54897922,192.15544788)(354.21821542,191.56430209)(354.08791453,190.82992715)
\closepath
}
}
{
\newrgbcolor{curcolor}{0 0 0}
\pscustom[linestyle=none,fillstyle=solid,fillcolor=curcolor]
{
\newpath
\moveto(366.76969871,190.47836468)
\lineto(369.66388192,187.5955524)
\curveto(369.85432168,187.5955524)(369.97459943,187.58513573)(370.02471515,187.5643024)
\curveto(370.07483088,187.5486774)(370.11742925,187.51221907)(370.15251026,187.45492741)
\curveto(370.19260284,187.39763574)(370.21264913,187.33513575)(370.21264913,187.26742742)
\curveto(370.21264913,187.17367743)(370.18007391,187.09555243)(370.11492346,187.03305244)
\curveto(370.04977302,186.97055244)(369.93951842,186.93930245)(369.78415967,186.93930245)
\lineto(367.75447274,186.93930245)
\curveto(367.59911399,186.93930245)(367.48885939,186.97055244)(367.42370894,187.03305244)
\curveto(367.3585585,187.09555243)(367.32598328,187.17367743)(367.32598328,187.26742742)
\curveto(367.32598328,187.36638575)(367.3585585,187.44451074)(367.42370894,187.5018024)
\curveto(367.48885939,187.5643024)(367.59911399,187.5955524)(367.75447274,187.5955524)
\lineto(368.79938564,187.5955524)
\lineto(366.34120925,190.04086471)
\lineto(363.89806757,187.5955524)
\lineto(364.95049783,187.5955524)
\curveto(365.10084501,187.5955524)(365.20859383,187.5643024)(365.27374427,187.5018024)
\curveto(365.33889471,187.44451074)(365.37146994,187.36638575)(365.37146994,187.26742742)
\curveto(365.37146994,187.17367743)(365.33889471,187.09555243)(365.27374427,187.03305244)
\curveto(365.20859383,186.97055244)(365.10084501,186.93930245)(364.95049783,186.93930245)
\lineto(362.89825883,186.93930245)
\curveto(362.74791165,186.93930245)(362.64016284,186.97055244)(362.57501239,187.03305244)
\curveto(362.50986195,187.09555243)(362.47728672,187.17367743)(362.47728672,187.26742742)
\curveto(362.47728672,187.33513575)(362.49482723,187.39763574)(362.52990824,187.45492741)
\curveto(362.57000082,187.51221907)(362.61259919,187.5486774)(362.65770334,187.5643024)
\curveto(362.70781907,187.58513573)(362.82809681,187.5955524)(363.01853657,187.5955524)
\lineto(365.91271978,190.47836468)
\lineto(363.34178301,193.04086448)
\curveto(363.16136639,193.04086448)(363.04610022,193.05128115)(362.99598449,193.07211448)
\curveto(362.95088034,193.09294781)(362.90828197,193.13201031)(362.86818939,193.18930197)
\curveto(362.83310838,193.24659363)(362.81556788,193.30909363)(362.81556788,193.37680196)
\curveto(362.81556788,193.47055195)(362.8481431,193.54867694)(362.91329355,193.61117694)
\curveto(362.97844399,193.67367693)(363.08869859,193.70492693)(363.24405734,193.70492693)
\lineto(364.96553255,193.70492693)
\curveto(365.1208913,193.70492693)(365.22864012,193.67367693)(365.28877899,193.61117694)
\curveto(365.35392943,193.54867694)(365.38650465,193.46794778)(365.38650465,193.36898946)
\curveto(365.38650465,193.27523946)(365.35392943,193.19711447)(365.28877899,193.13461448)
\curveto(365.22864012,193.07211448)(365.1208913,193.04086448)(364.96553255,193.04086448)
\lineto(364.21379665,193.04086448)
\lineto(366.34120925,190.90805214)
\lineto(368.48365656,193.04086448)
\lineto(367.73192066,193.04086448)
\curveto(367.58157348,193.04086448)(367.47131888,193.07211448)(367.40115687,193.13461448)
\curveto(367.33600642,193.19711447)(367.3034312,193.27784363)(367.3034312,193.37680196)
\curveto(367.3034312,193.47055195)(367.33600642,193.54867694)(367.40115687,193.61117694)
\curveto(367.46630731,193.67367693)(367.57656191,193.70492693)(367.73192066,193.70492693)
\lineto(369.45339587,193.70492693)
\curveto(369.60374305,193.70492693)(369.71149186,193.67367693)(369.77664231,193.61117694)
\curveto(369.84179275,193.54867694)(369.87436797,193.46794778)(369.87436797,193.36898946)
\curveto(369.87436797,193.30648946)(369.85432168,193.24659363)(369.8142291,193.18930197)
\curveto(369.77914809,193.13201031)(369.73654973,193.09294781)(369.686434,193.07211448)
\curveto(369.64132985,193.05128115)(369.52856946,193.04086448)(369.34815285,193.04086448)
\closepath
}
}
{
\newrgbcolor{curcolor}{0 0 0}
\pscustom[linestyle=none,fillstyle=solid,fillcolor=curcolor]
{
\newpath
\moveto(374.2795399,193.70492693)
\lineto(377.7074556,193.70492693)
\curveto(377.85780278,193.70492693)(377.96555159,193.67367693)(378.03070204,193.61117694)
\curveto(378.09585248,193.54867694)(378.1284277,193.46794778)(378.1284277,193.36898946)
\curveto(378.1284277,193.27523946)(378.09585248,193.19711447)(378.03070204,193.13461448)
\curveto(377.96555159,193.07211448)(377.85780278,193.04086448)(377.7074556,193.04086448)
\lineto(374.2795399,193.04086448)
\lineto(374.2795399,188.67367731)
\curveto(374.2795399,188.29346901)(374.42487551,187.9757607)(374.71554672,187.72055239)
\curveto(375.01122951,187.46534407)(375.44222476,187.33773992)(376.00853247,187.33773992)
\curveto(376.43451614,187.33773992)(376.89558083,187.40284408)(377.39172652,187.5330524)
\curveto(377.88787222,187.66846906)(378.27376331,187.81951071)(378.54939981,187.98617737)
\curveto(378.64963126,188.05388569)(378.73232221,188.08773986)(378.79747265,188.08773986)
\curveto(378.87765782,188.08773986)(378.94781983,188.05388569)(379.00795871,187.98617737)
\curveto(379.06809758,187.92367737)(379.09816701,187.84815654)(379.09816701,187.75961488)
\curveto(379.09816701,187.68148989)(379.06559179,187.60857323)(379.00044135,187.5408649)
\curveto(378.84007102,187.36898991)(378.44916835,187.18148993)(377.82773334,186.97836494)
\curveto(377.21130991,186.78044829)(376.61994433,186.68148996)(376.05363662,186.68148996)
\curveto(375.31693544,186.68148996)(374.73058144,186.86117745)(374.29457462,187.22055242)
\curveto(373.8585678,187.5799274)(373.64056438,188.06430236)(373.64056438,188.67367731)
\lineto(373.64056438,193.04086448)
\lineto(372.47537374,193.04086448)
\curveto(372.32502656,193.04086448)(372.21727775,193.07211448)(372.1521273,193.13461448)
\curveto(372.08697686,193.19711447)(372.05440164,193.27784363)(372.05440164,193.37680196)
\curveto(372.05440164,193.47055195)(372.08697686,193.54867694)(372.1521273,193.61117694)
\curveto(372.21727775,193.67367693)(372.32502656,193.70492693)(372.47537374,193.70492693)
\lineto(373.64056438,193.70492693)
\lineto(373.64056438,195.64242679)
\curveto(373.64056438,195.79867677)(373.67063382,195.91065593)(373.73077269,195.97836426)
\curveto(373.79091156,196.04607259)(373.86608515,196.07992675)(373.95629346,196.07992675)
\curveto(374.05151334,196.07992675)(374.12919272,196.04607259)(374.18933159,195.97836426)
\curveto(374.24947046,195.91065593)(374.2795399,195.79867677)(374.2795399,195.64242679)
\closepath
}
}
{
\newrgbcolor{curcolor}{0 0 0}
\pscustom[linewidth=1.89141912,linecolor=curcolor]
{
\newpath
\moveto(322.78041449,177.9125188)
\lineto(466.08651109,177.9125188)
\lineto(466.08651109,136.64886879)
\lineto(322.78041449,136.64886879)
\closepath
}
}
{
\newrgbcolor{curcolor}{0 0 0}
\pscustom[linestyle=none,fillstyle=solid,fillcolor=curcolor]
{
\newpath
\moveto(327.32566303,174.36115492)
\lineto(327.32566303,170.00959276)
\curveto(328.08742207,171.04084268)(329.00704566,171.55646764)(330.08453378,171.55646764)
\curveto(331.00666315,171.55646764)(331.79598584,171.20750933)(332.45250186,170.50959272)
\curveto(333.10901788,169.81688444)(333.43727589,168.965322)(333.43727589,167.95490541)
\curveto(333.43727589,166.93407215)(333.10400631,166.06948889)(332.43746714,165.36115561)
\curveto(331.77593955,164.65282233)(330.99162843,164.29865569)(330.08453378,164.29865569)
\curveto(328.98198779,164.29865569)(328.06236421,164.81428065)(327.32566303,165.84553057)
\lineto(327.32566303,164.55646817)
\lineto(325.84474331,164.55646817)
\curveto(325.69439613,164.55646817)(325.58664731,164.58771817)(325.52149687,164.65021816)
\curveto(325.45634642,164.71271816)(325.4237712,164.79084315)(325.4237712,164.88459314)
\curveto(325.4237712,164.98355147)(325.45634642,165.06167646)(325.52149687,165.11896813)
\curveto(325.58664731,165.18146812)(325.69439613,165.21271812)(325.84474331,165.21271812)
\lineto(326.69420487,165.21271812)
\lineto(326.69420487,173.69709248)
\lineto(325.84474331,173.69709248)
\curveto(325.69439613,173.69709248)(325.58664731,173.72834247)(325.52149687,173.79084247)
\curveto(325.45634642,173.85334246)(325.4237712,173.93407162)(325.4237712,174.03302995)
\curveto(325.4237712,174.12677994)(325.45634642,174.20490494)(325.52149687,174.26740493)
\curveto(325.58664731,174.32990493)(325.69439613,174.36115492)(325.84474331,174.36115492)
\closepath
\moveto(332.80581773,167.92365541)
\curveto(332.80581773,168.75178035)(332.53268702,169.45230113)(331.9864256,170.02521775)
\curveto(331.44016418,170.60334271)(330.80118867,170.89240519)(330.06949906,170.89240519)
\curveto(329.33780945,170.89240519)(328.69883394,170.60334271)(328.15257252,170.02521775)
\curveto(327.6063111,169.45230113)(327.33318039,168.75178035)(327.33318039,167.92365541)
\curveto(327.33318039,167.09553048)(327.6063111,166.39240553)(328.15257252,165.81428057)
\curveto(328.69883394,165.24136395)(329.33780945,164.95490564)(330.06949906,164.95490564)
\curveto(330.80118867,164.95490564)(331.44016418,165.24136395)(331.9864256,165.81428057)
\curveto(332.53268702,166.39240553)(332.80581773,167.09553048)(332.80581773,167.92365541)
\closepath
}
}
{
\newrgbcolor{curcolor}{0 0 0}
\pscustom[linestyle=none,fillstyle=solid,fillcolor=curcolor]
{
\newpath
\moveto(343.14218679,160.173656)
\lineto(334.74529679,160.173656)
\curveto(334.59494961,160.173656)(334.4872008,160.204906)(334.42205036,160.26740599)
\curveto(334.35689991,160.32469766)(334.32432469,160.40282265)(334.32432469,160.50178098)
\curveto(334.32432469,160.6007393)(334.35689991,160.68146846)(334.42205036,160.74396846)
\curveto(334.4872008,160.80126012)(334.59494961,160.82990595)(334.74529679,160.82990595)
\lineto(343.14218679,160.82990595)
\curveto(343.29754554,160.82990595)(343.40529435,160.80126012)(343.46543323,160.74396846)
\curveto(343.53058367,160.68146846)(343.56315889,160.6007393)(343.56315889,160.50178098)
\curveto(343.56315889,160.40282265)(343.53058367,160.32469766)(343.46543323,160.26740599)
\curveto(343.40529435,160.204906)(343.29754554,160.173656)(343.14218679,160.173656)
\closepath
}
}
{
\newrgbcolor{curcolor}{0 0 0}
\pscustom[linestyle=none,fillstyle=solid,fillcolor=curcolor]
{
\newpath
\moveto(350.81741076,170.65803021)
\lineto(350.81741076,170.87678019)
\curveto(350.81741076,171.03823851)(350.8474802,171.15282183)(350.90761907,171.22053016)
\curveto(350.96775794,171.28823849)(351.04293153,171.32209266)(351.13313984,171.32209266)
\curveto(351.22835972,171.32209266)(351.3060391,171.28823849)(351.36617797,171.22053016)
\curveto(351.42631684,171.15282183)(351.45638628,171.03823851)(351.45638628,170.87678019)
\lineto(351.45638628,169.3924053)
\curveto(351.4513747,169.23094698)(351.41879948,169.11636366)(351.35866061,169.04865533)
\curveto(351.30353331,168.980947)(351.22835972,168.94709284)(351.13313984,168.94709284)
\curveto(351.0479431,168.94709284)(350.9752753,168.97573867)(350.91513643,169.03303033)
\curveto(350.86000913,169.09553032)(350.82743391,169.19709282)(350.81741076,169.33771781)
\curveto(350.78734132,169.70750944)(350.55179741,170.05907192)(350.11077902,170.39240523)
\curveto(349.67477219,170.72573853)(349.08591241,170.89240519)(348.34419965,170.89240519)
\curveto(347.40703556,170.89240519)(346.69539225,170.58771771)(346.2092697,169.97834276)
\curveto(345.72314715,169.3689678)(345.48008587,168.67105119)(345.48008587,167.88459292)
\curveto(345.48008587,167.03563465)(345.74820501,166.33511387)(346.28444329,165.78303058)
\curveto(346.82068156,165.23094728)(347.51478438,164.95490564)(348.36675173,164.95490564)
\curveto(348.85788585,164.95490564)(349.35653733,165.04865563)(349.86270617,165.23615562)
\curveto(350.37388658,165.4236556)(350.83495126,165.72573891)(351.24590022,166.14240555)
\curveto(351.35114325,166.24657221)(351.44385734,166.29865554)(351.52404251,166.29865554)
\curveto(351.60923924,166.29865554)(351.67940126,166.26740554)(351.73452856,166.20490554)
\curveto(351.79466743,166.14761388)(351.82473687,166.07469722)(351.82473687,165.98615556)
\curveto(351.82473687,165.76219724)(351.57165245,165.4783431)(351.06548361,165.13459312)
\curveto(350.24859726,164.5773015)(349.33899682,164.29865569)(348.33668229,164.29865569)
\curveto(347.31933304,164.29865569)(346.48240041,164.63459316)(345.82588439,165.30646811)
\curveto(345.17437994,165.98355139)(344.84862772,166.84032216)(344.84862772,167.87678042)
\curveto(344.84862772,168.934072)(345.1818973,169.8116761)(345.84843647,170.50959272)
\curveto(346.5199872,171.20750933)(347.3644372,171.55646764)(348.38178645,171.55646764)
\curveto(349.34901997,171.55646764)(350.16089474,171.25698849)(350.81741076,170.65803021)
\closepath
}
}
{
\newrgbcolor{curcolor}{0 0 0}
\pscustom[linestyle=none,fillstyle=solid,fillcolor=curcolor]
{
\newpath
\moveto(360.95081201,167.92365541)
\curveto(360.95081201,166.92365549)(360.60501349,166.06948889)(359.91341647,165.36115561)
\curveto(359.22683101,164.65282233)(358.39741573,164.29865569)(357.42517064,164.29865569)
\curveto(356.4429024,164.29865569)(355.60847555,164.65282233)(354.92189009,165.36115561)
\curveto(354.23530464,166.07469722)(353.89201191,166.92886382)(353.89201191,167.92365541)
\curveto(353.89201191,168.92365534)(354.23530464,169.77782194)(354.92189009,170.48615522)
\curveto(355.60847555,171.19969683)(356.4429024,171.55646764)(357.42517064,171.55646764)
\curveto(358.39741573,171.55646764)(359.22683101,171.202301)(359.91341647,170.49396772)
\curveto(360.60501349,169.78563444)(360.95081201,168.92886367)(360.95081201,167.92365541)
\closepath
\moveto(360.31183649,167.92365541)
\curveto(360.31183649,168.74657202)(360.02868264,169.4470928)(359.46237493,170.02521775)
\curveto(358.90107879,170.60334271)(358.21950491,170.89240519)(357.41765328,170.89240519)
\curveto(356.61580165,170.89240519)(355.93172198,170.60073854)(355.36541427,170.01740525)
\curveto(354.80411814,169.4392803)(354.52347007,168.74136368)(354.52347007,167.92365541)
\curveto(354.52347007,167.11115547)(354.80411814,166.41323886)(355.36541427,165.82990557)
\curveto(355.93172198,165.24657228)(356.61580165,164.95490564)(357.41765328,164.95490564)
\curveto(358.21950491,164.95490564)(358.90107879,165.24396812)(359.46237493,165.82209307)
\curveto(360.02868264,166.40542636)(360.31183649,167.10594714)(360.31183649,167.92365541)
\closepath
}
}
{
\newrgbcolor{curcolor}{0 0 0}
\pscustom[linestyle=none,fillstyle=solid,fillcolor=curcolor]
{
\newpath
\moveto(363.7848559,171.32209266)
\lineto(363.7848559,170.65803021)
\curveto(364.32109418,171.25698849)(364.85983824,171.55646764)(365.40108809,171.55646764)
\curveto(365.72684031,171.55646764)(366.01249995,171.46532181)(366.25806701,171.28303016)
\curveto(366.50363407,171.10594684)(366.70910855,170.83511353)(366.87449045,170.47053022)
\curveto(367.15513852,170.83511353)(367.43829237,171.10594684)(367.72395201,171.28303016)
\curveto(368.01462323,171.46532181)(368.30529444,171.55646764)(368.59596566,171.55646764)
\curveto(369.05201877,171.55646764)(369.41535779,171.40282182)(369.68598271,171.09553017)
\curveto(370.04180437,170.69969687)(370.2197152,170.26740524)(370.2197152,169.79865527)
\lineto(370.2197152,165.21271812)
\lineto(370.75344769,165.21271812)
\curveto(370.90379487,165.21271812)(371.01154368,165.18146812)(371.07669413,165.11896813)
\curveto(371.14184457,165.06167646)(371.17441979,164.98355147)(371.17441979,164.88459314)
\curveto(371.17441979,164.79084315)(371.14184457,164.71271816)(371.07669413,164.65021816)
\curveto(371.01154368,164.58771817)(370.90379487,164.55646817)(370.75344769,164.55646817)
\lineto(369.58825705,164.55646817)
\lineto(369.58825705,169.73615528)
\curveto(369.58825705,170.06948858)(369.49053138,170.34553023)(369.29508004,170.56428021)
\curveto(369.09962871,170.7830302)(368.87410794,170.89240519)(368.61851773,170.89240519)
\curveto(368.38798539,170.89240519)(368.14492412,170.80125936)(367.88933391,170.61896771)
\curveto(367.63374371,170.44188439)(367.34307249,170.09032192)(367.01732027,169.56428029)
\lineto(367.01732027,165.21271812)
\lineto(367.5435354,165.21271812)
\curveto(367.69388258,165.21271812)(367.80163139,165.18146812)(367.86678184,165.11896813)
\curveto(367.93193228,165.06167646)(367.9645075,164.98355147)(367.9645075,164.88459314)
\curveto(367.9645075,164.79084315)(367.93193228,164.71271816)(367.86678184,164.65021816)
\curveto(367.80163139,164.58771817)(367.69388258,164.55646817)(367.5435354,164.55646817)
\lineto(366.37834475,164.55646817)
\lineto(366.37834475,169.68928028)
\curveto(366.37834475,170.03823859)(366.2781133,170.3246969)(366.07765039,170.54865521)
\curveto(365.88219906,170.77782186)(365.66168986,170.89240519)(365.4161228,170.89240519)
\curveto(365.19060203,170.89240519)(364.96758705,170.81688436)(364.74707785,170.66584271)
\curveto(364.44137192,170.45230105)(364.12063127,170.08511358)(363.7848559,169.56428029)
\lineto(363.7848559,165.21271812)
\lineto(364.31858839,165.21271812)
\curveto(364.46893557,165.21271812)(364.57668438,165.18146812)(364.64183483,165.11896813)
\curveto(364.70698527,165.06167646)(364.73956049,164.98355147)(364.73956049,164.88459314)
\curveto(364.73956049,164.79084315)(364.70698527,164.71271816)(364.64183483,164.65021816)
\curveto(364.57668438,164.58771817)(364.46893557,164.55646817)(364.31858839,164.55646817)
\lineto(362.61966526,164.55646817)
\curveto(362.46931808,164.55646817)(362.36156927,164.58771817)(362.29641882,164.65021816)
\curveto(362.23126838,164.71271816)(362.19869315,164.79084315)(362.19869315,164.88459314)
\curveto(362.19869315,164.98355147)(362.23126838,165.06167646)(362.29641882,165.11896813)
\curveto(362.36156927,165.18146812)(362.46931808,165.21271812)(362.61966526,165.21271812)
\lineto(363.15339775,165.21271812)
\lineto(363.15339775,170.65803021)
\lineto(362.61966526,170.65803021)
\curveto(362.46931808,170.65803021)(362.36156927,170.6892802)(362.29641882,170.7517802)
\curveto(362.23126838,170.81428019)(362.19869315,170.89500935)(362.19869315,170.99396768)
\curveto(362.19869315,171.08771767)(362.23126838,171.16584267)(362.29641882,171.22834266)
\curveto(362.36156927,171.29084266)(362.46931808,171.32209266)(362.61966526,171.32209266)
\closepath
}
}
{
\newrgbcolor{curcolor}{0 0 0}
\pscustom[linestyle=none,fillstyle=solid,fillcolor=curcolor]
{
\newpath
\moveto(373.51983535,171.32209266)
\lineto(373.51983535,170.12678025)
\curveto(373.8957033,170.60073854)(374.2991349,170.95750935)(374.73013015,171.19709266)
\curveto(375.1611254,171.43667598)(375.66980003,171.55646764)(376.25615403,171.55646764)
\curveto(376.87758904,171.55646764)(377.45141411,171.40542598)(377.97762924,171.10334267)
\curveto(378.50384437,170.80125936)(378.90978175,170.37938439)(379.19544139,169.83771777)
\curveto(379.48611261,169.30125948)(379.63144822,168.73615535)(379.63144822,168.1424054)
\curveto(379.63144822,167.19969713)(379.30569599,166.39240553)(378.65419155,165.72053058)
\curveto(378.00769867,165.05386396)(377.21085862,164.72053066)(376.26367139,164.72053066)
\curveto(375.13606754,164.72053066)(374.22145553,165.19709312)(373.51983535,166.15021805)
\lineto(373.51983535,162.19709335)
\lineto(375.05337659,162.19709335)
\curveto(375.20372377,162.19709335)(375.31147258,162.16844752)(375.37662302,162.11115585)
\curveto(375.44177347,162.04865586)(375.47434869,161.9679267)(375.47434869,161.86896837)
\curveto(375.47434869,161.77521838)(375.44177347,161.69709339)(375.37662302,161.63459339)
\curveto(375.31147258,161.5720934)(375.20372377,161.5408434)(375.05337659,161.5408434)
\lineto(372.03891563,161.5408434)
\curveto(371.88856845,161.5408434)(371.78081964,161.5720934)(371.71566919,161.63459339)
\curveto(371.65051875,161.69188505)(371.61794353,161.77001005)(371.61794353,161.86896837)
\curveto(371.61794353,161.9679267)(371.65051875,162.04865586)(371.71566919,162.11115585)
\curveto(371.78081964,162.16844752)(371.88856845,162.19709335)(372.03891563,162.19709335)
\lineto(372.8883772,162.19709335)
\lineto(372.8883772,170.65803021)
\lineto(372.03891563,170.65803021)
\curveto(371.88856845,170.65803021)(371.78081964,170.6892802)(371.71566919,170.7517802)
\curveto(371.65051875,170.81428019)(371.61794353,170.89500935)(371.61794353,170.99396768)
\curveto(371.61794353,171.08771767)(371.65051875,171.16584267)(371.71566919,171.22834266)
\curveto(371.78081964,171.29084266)(371.88856845,171.32209266)(372.03891563,171.32209266)
\closepath
\moveto(378.9924727,168.1424054)
\curveto(378.9924727,168.89761367)(378.72685935,169.54344696)(378.19563265,170.07990525)
\curveto(377.66941752,170.62157188)(377.02543043,170.89240519)(376.26367139,170.89240519)
\curveto(375.49690077,170.89240519)(374.84790211,170.62157188)(374.31667541,170.07990525)
\curveto(373.7854487,169.53823862)(373.51983535,168.89240534)(373.51983535,168.1424054)
\curveto(373.51983535,167.38719712)(373.7854487,166.73875967)(374.31667541,166.19709304)
\curveto(374.84790211,165.65542642)(375.49690077,165.38459311)(376.26367139,165.38459311)
\curveto(377.02041886,165.38459311)(377.66440595,165.65282225)(378.19563265,166.18928054)
\curveto(378.72685935,166.73094717)(378.9924727,167.38198879)(378.9924727,168.1424054)
\closepath
}
}
{
\newrgbcolor{curcolor}{0 0 0}
\pscustom[linestyle=none,fillstyle=solid,fillcolor=curcolor]
{
\newpath
\moveto(397.82345609,169.98615526)
\lineto(390.93755525,169.98615526)
\curveto(390.78720807,169.98615526)(390.67945926,170.01740525)(390.61430881,170.07990525)
\curveto(390.54915837,170.14240524)(390.51658315,170.22313441)(390.51658315,170.32209273)
\curveto(390.51658315,170.41584272)(390.54915837,170.49396772)(390.61430881,170.55646771)
\curveto(390.67945926,170.61896771)(390.78720807,170.65021771)(390.93755525,170.65021771)
\lineto(397.82345609,170.65021771)
\curveto(397.97380327,170.65021771)(398.08155208,170.61896771)(398.14670253,170.55646771)
\curveto(398.21185297,170.49396772)(398.24442819,170.41323856)(398.24442819,170.31428023)
\curveto(398.24442819,170.22053024)(398.21185297,170.14240524)(398.14670253,170.07990525)
\curveto(398.08155208,170.01740525)(397.97380327,169.98615526)(397.82345609,169.98615526)
\closepath
\moveto(397.82345609,167.63459294)
\lineto(390.93755525,167.63459294)
\curveto(390.78720807,167.63459294)(390.67945926,167.66323877)(390.61430881,167.72053043)
\curveto(390.54915837,167.78303042)(390.51658315,167.86375958)(390.51658315,167.96271791)
\curveto(390.51658315,168.06167624)(390.54915837,168.13980123)(390.61430881,168.19709289)
\curveto(390.67945926,168.25959289)(390.78720807,168.29084289)(390.93755525,168.29084289)
\lineto(397.82345609,168.29084289)
\curveto(397.97380327,168.29084289)(398.08155208,168.25959289)(398.14670253,168.19709289)
\curveto(398.21185297,168.13980123)(398.24442819,168.06167624)(398.24442819,167.96271791)
\curveto(398.24442819,167.86375958)(398.21185297,167.78303042)(398.14670253,167.72053043)
\curveto(398.08155208,167.66323877)(397.97380327,167.63459294)(397.82345609,167.63459294)
\closepath
}
}
{
\newrgbcolor{curcolor}{0 0 0}
\pscustom[linestyle=none,fillstyle=solid,fillcolor=curcolor]
{
\newpath
\moveto(413.16639198,173.04084253)
\lineto(413.16639198,165.21271812)
\lineto(414.80517624,165.21271812)
\curveto(414.95552342,165.21271812)(415.06327223,165.18146812)(415.12842267,165.11896813)
\curveto(415.19357312,165.06167646)(415.22614834,164.98355147)(415.22614834,164.88459314)
\curveto(415.22614834,164.79084315)(415.19357312,164.71271816)(415.12842267,164.65021816)
\curveto(415.06327223,164.58771817)(414.95552342,164.55646817)(414.80517624,164.55646817)
\lineto(410.8886322,164.55646817)
\curveto(410.73828502,164.55646817)(410.63053621,164.58771817)(410.56538576,164.65021816)
\curveto(410.50023532,164.71271816)(410.4676601,164.79084315)(410.4676601,164.88459314)
\curveto(410.4676601,164.98355147)(410.50023532,165.06167646)(410.56538576,165.11896813)
\curveto(410.63053621,165.18146812)(410.73828502,165.21271812)(410.8886322,165.21271812)
\lineto(412.52741646,165.21271812)
\lineto(412.52741646,173.04084253)
\lineto(409.94144497,173.04084253)
\lineto(409.94144497,170.7674052)
\curveto(409.94144497,170.60594688)(409.91137553,170.49136355)(409.85123666,170.42365522)
\curveto(409.79610936,170.3559469)(409.72093577,170.32209273)(409.62571589,170.32209273)
\curveto(409.53550758,170.32209273)(409.46033399,170.3559469)(409.40019512,170.42365522)
\curveto(409.34005625,170.49136355)(409.30998681,170.60594688)(409.30998681,170.7674052)
\lineto(409.30998681,173.69709248)
\lineto(416.38382163,173.69709248)
\lineto(416.38382163,170.7674052)
\curveto(416.38382163,170.60594688)(416.35375219,170.49136355)(416.29361332,170.42365522)
\curveto(416.23848602,170.3559469)(416.16331243,170.32209273)(416.06809255,170.32209273)
\curveto(415.97287267,170.32209273)(415.89519329,170.3559469)(415.83505442,170.42365522)
\curveto(415.77992712,170.49136355)(415.75236347,170.60594688)(415.75236347,170.7674052)
\lineto(415.75236347,173.04084253)
\closepath
}
}
{
\newrgbcolor{curcolor}{0 0 0}
\pscustom[linestyle=none,fillstyle=solid,fillcolor=curcolor]
{
\newpath
\moveto(327.32566303,154.36115644)
\lineto(327.32566303,150.00959427)
\curveto(328.08742207,151.04084419)(329.00704566,151.55646915)(330.08453378,151.55646915)
\curveto(331.00666315,151.55646915)(331.79598584,151.20751085)(332.45250186,150.50959423)
\curveto(333.10901788,149.81688595)(333.43727589,148.96532352)(333.43727589,147.95490693)
\curveto(333.43727589,146.93407367)(333.10400631,146.0694904)(332.43746714,145.36115712)
\curveto(331.77593955,144.65282385)(330.99162843,144.29865721)(330.08453378,144.29865721)
\curveto(328.98198779,144.29865721)(328.06236421,144.81428217)(327.32566303,145.84553209)
\lineto(327.32566303,144.55646969)
\lineto(325.84474331,144.55646969)
\curveto(325.69439613,144.55646969)(325.58664731,144.58771968)(325.52149687,144.65021968)
\curveto(325.45634642,144.71271967)(325.4237712,144.79084467)(325.4237712,144.88459466)
\curveto(325.4237712,144.98355299)(325.45634642,145.06167798)(325.52149687,145.11896964)
\curveto(325.58664731,145.18146964)(325.69439613,145.21271964)(325.84474331,145.21271964)
\lineto(326.69420487,145.21271964)
\lineto(326.69420487,153.69709399)
\lineto(325.84474331,153.69709399)
\curveto(325.69439613,153.69709399)(325.58664731,153.72834399)(325.52149687,153.79084399)
\curveto(325.45634642,153.85334398)(325.4237712,153.93407314)(325.4237712,154.03303147)
\curveto(325.4237712,154.12678146)(325.45634642,154.20490645)(325.52149687,154.26740645)
\curveto(325.58664731,154.32990644)(325.69439613,154.36115644)(325.84474331,154.36115644)
\closepath
\moveto(332.80581773,147.92365693)
\curveto(332.80581773,148.75178187)(332.53268702,149.45230265)(331.9864256,150.02521927)
\curveto(331.44016418,150.60334423)(330.80118867,150.89240671)(330.06949906,150.89240671)
\curveto(329.33780945,150.89240671)(328.69883394,150.60334423)(328.15257252,150.02521927)
\curveto(327.6063111,149.45230265)(327.33318039,148.75178187)(327.33318039,147.92365693)
\curveto(327.33318039,147.09553199)(327.6063111,146.39240705)(328.15257252,145.81428209)
\curveto(328.69883394,145.24136547)(329.33780945,144.95490716)(330.06949906,144.95490716)
\curveto(330.80118867,144.95490716)(331.44016418,145.24136547)(331.9864256,145.81428209)
\curveto(332.53268702,146.39240705)(332.80581773,147.09553199)(332.80581773,147.92365693)
\closepath
}
}
{
\newrgbcolor{curcolor}{0 0 0}
\pscustom[linestyle=none,fillstyle=solid,fillcolor=curcolor]
{
\newpath
\moveto(343.14218679,140.17365752)
\lineto(334.74529679,140.17365752)
\curveto(334.59494961,140.17365752)(334.4872008,140.20490752)(334.42205036,140.26740751)
\curveto(334.35689991,140.32469917)(334.32432469,140.40282417)(334.32432469,140.50178249)
\curveto(334.32432469,140.60074082)(334.35689991,140.68146998)(334.42205036,140.74396997)
\curveto(334.4872008,140.80126164)(334.59494961,140.82990747)(334.74529679,140.82990747)
\lineto(343.14218679,140.82990747)
\curveto(343.29754554,140.82990747)(343.40529435,140.80126164)(343.46543323,140.74396997)
\curveto(343.53058367,140.68146998)(343.56315889,140.60074082)(343.56315889,140.50178249)
\curveto(343.56315889,140.40282417)(343.53058367,140.32469917)(343.46543323,140.26740751)
\curveto(343.40529435,140.20490752)(343.29754554,140.17365752)(343.14218679,140.17365752)
\closepath
}
}
{
\newrgbcolor{curcolor}{0 0 0}
\pscustom[linestyle=none,fillstyle=solid,fillcolor=curcolor]
{
\newpath
\moveto(351.27596966,154.36115644)
\lineto(351.27596966,145.21271964)
\lineto(352.11791387,145.21271964)
\curveto(352.27327262,145.21271964)(352.38352722,145.18146964)(352.44867766,145.11896964)
\curveto(352.51382811,145.06167798)(352.54640333,144.98355299)(352.54640333,144.88459466)
\curveto(352.54640333,144.79084467)(352.51382811,144.71271967)(352.44867766,144.65021968)
\curveto(352.38352722,144.58771968)(352.27327262,144.55646969)(352.11791387,144.55646969)
\lineto(350.63699414,144.55646969)
\lineto(350.63699414,145.86115709)
\curveto(349.90530454,144.8194905)(348.97565781,144.29865721)(347.84805396,144.29865721)
\curveto(347.27673467,144.29865721)(346.72796747,144.45490719)(346.20175234,144.76740717)
\curveto(345.68054878,145.08511548)(345.26709404,145.53563628)(344.9613881,146.11896957)
\curveto(344.66069374,146.70230286)(344.51034656,147.30386531)(344.51034656,147.92365693)
\curveto(344.51034656,148.54865688)(344.66069374,149.15021934)(344.9613881,149.72834429)
\curveto(345.26709404,150.31167758)(345.68054878,150.76219838)(346.20175234,151.07990669)
\curveto(346.72796747,151.397615)(347.27924046,151.55646915)(347.85557132,151.55646915)
\curveto(348.9581173,151.55646915)(349.88525825,151.03563586)(350.63699414,149.99396927)
\lineto(350.63699414,153.69709399)
\lineto(349.79504994,153.69709399)
\curveto(349.63969118,153.69709399)(349.52943659,153.72834399)(349.46428614,153.79084399)
\curveto(349.3991357,153.85334398)(349.36656047,153.93407314)(349.36656047,154.03303147)
\curveto(349.36656047,154.12678146)(349.3991357,154.20490645)(349.46428614,154.26740645)
\curveto(349.52943659,154.32990644)(349.63969118,154.36115644)(349.79504994,154.36115644)
\closepath
\moveto(350.63699414,147.92365693)
\curveto(350.63699414,148.7569902)(350.36636922,149.46011515)(349.82511937,150.03303177)
\curveto(349.28386953,150.60594839)(348.63988244,150.89240671)(347.89315811,150.89240671)
\curveto(347.14142221,150.89240671)(346.49492934,150.60594839)(345.95367949,150.03303177)
\curveto(345.41242964,149.46011515)(345.14180472,148.7569902)(345.14180472,147.92365693)
\curveto(345.14180472,147.09553199)(345.41242964,146.39240705)(345.95367949,145.81428209)
\curveto(346.49492934,145.24136547)(347.14142221,144.95490716)(347.89315811,144.95490716)
\curveto(348.63988244,144.95490716)(349.28386953,145.24136547)(349.82511937,145.81428209)
\curveto(350.36636922,146.39240705)(350.63699414,147.09553199)(350.63699414,147.92365693)
\closepath
}
}
{
\newrgbcolor{curcolor}{0 0 0}
\pscustom[linestyle=none,fillstyle=solid,fillcolor=curcolor]
{
\newpath
\moveto(359.23685416,144.55646969)
\lineto(359.23685416,145.50178211)
\curveto(358.31973636,144.69969884)(357.3399739,144.29865721)(356.29756679,144.29865721)
\curveto(355.54081932,144.29865721)(354.94945374,144.49657386)(354.52347007,144.89240716)
\curveto(354.09748639,145.2934488)(353.88449455,145.78303209)(353.88449455,146.36115705)
\curveto(353.88449455,146.99657367)(354.16514262,147.55126113)(354.72643876,148.02521942)
\curveto(355.2877349,148.49917772)(356.10712703,148.73615687)(357.18461515,148.73615687)
\curveto(357.47528636,148.73615687)(357.79101544,148.71532354)(358.13180238,148.67365687)
\curveto(358.47258932,148.63719854)(358.84093992,148.57730271)(359.23685416,148.49396939)
\lineto(359.23685416,149.55646931)
\curveto(359.23685416,149.91584428)(359.07648383,150.22834426)(358.75574318,150.49396924)
\curveto(358.43500253,150.75959422)(357.95389155,150.89240671)(357.31241025,150.89240671)
\curveto(356.82127613,150.89240671)(356.13218489,150.74396922)(355.24513653,150.44709424)
\curveto(355.0847662,150.39501091)(354.98202897,150.36896924)(354.93692481,150.36896924)
\curveto(354.85673965,150.36896924)(354.78657763,150.40021924)(354.72643876,150.46271924)
\curveto(354.67131146,150.52521923)(354.64374781,150.60334423)(354.64374781,150.69709422)
\curveto(354.64374781,150.78563588)(354.66880567,150.85594837)(354.7189214,150.9080317)
\curveto(354.78908342,150.9861567)(355.07223727,151.09292752)(355.56838297,151.22834418)
\curveto(356.3501883,151.44709416)(356.94155388,151.55646915)(357.34247969,151.55646915)
\curveto(358.13931974,151.55646915)(358.76075475,151.35074)(359.20678472,150.9392817)
\curveto(359.65281469,150.53303173)(359.87582967,150.07209427)(359.87582967,149.55646931)
\lineto(359.87582967,145.21271964)
\lineto(360.71777388,145.21271964)
\curveto(360.87313263,145.21271964)(360.98338723,145.18146964)(361.04853767,145.11896964)
\curveto(361.11368812,145.06167798)(361.14626334,144.98355299)(361.14626334,144.88459466)
\curveto(361.14626334,144.79084467)(361.11368812,144.71271967)(361.04853767,144.65021968)
\curveto(360.98338723,144.58771968)(360.87313263,144.55646969)(360.71777388,144.55646969)
\closepath
\moveto(359.23685416,147.82209444)
\curveto(358.94117137,147.9106361)(358.62794808,147.97574026)(358.29718428,148.01740692)
\curveto(357.96642049,148.05907359)(357.61811619,148.07990692)(357.25227138,148.07990692)
\curveto(356.33515358,148.07990692)(355.61849869,147.87417777)(355.10230671,147.46271947)
\curveto(354.71140404,147.15542782)(354.51595271,146.78824035)(354.51595271,146.36115705)
\curveto(354.51595271,145.96532375)(354.6637941,145.63199044)(354.95947689,145.36115712)
\curveto(355.26017125,145.09032381)(355.69617807,144.95490716)(356.26749735,144.95490716)
\curveto(356.81375877,144.95490716)(357.31992761,145.06688631)(357.78600387,145.29084463)
\curveto(358.2570917,145.52001128)(358.74070846,145.88199042)(359.23685416,146.37678205)
\closepath
}
}
{
\newrgbcolor{curcolor}{0 0 0}
\pscustom[linestyle=none,fillstyle=solid,fillcolor=curcolor]
{
\newpath
\moveto(365.36350129,151.32209417)
\lineto(368.79141699,151.32209417)
\curveto(368.94176417,151.32209417)(369.04951298,151.29084417)(369.11466343,151.22834418)
\curveto(369.17981387,151.16584418)(369.2123891,151.08511502)(369.2123891,150.9861567)
\curveto(369.2123891,150.89240671)(369.17981387,150.81428171)(369.11466343,150.75178172)
\curveto(369.04951298,150.68928172)(368.94176417,150.65803172)(368.79141699,150.65803172)
\lineto(365.36350129,150.65803172)
\lineto(365.36350129,146.29084455)
\curveto(365.36350129,145.91063625)(365.5088369,145.59292794)(365.79950811,145.33771963)
\curveto(366.0951909,145.08251131)(366.52618615,144.95490716)(367.09249386,144.95490716)
\curveto(367.51847754,144.95490716)(367.97954222,145.02001132)(368.47568791,145.15021964)
\curveto(368.97183361,145.2856363)(369.3577247,145.43667795)(369.6333612,145.60334461)
\curveto(369.73359265,145.67105293)(369.8162836,145.7049071)(369.88143405,145.7049071)
\curveto(369.96161921,145.7049071)(370.03178123,145.67105293)(370.0919201,145.60334461)
\curveto(370.15205897,145.54084461)(370.18212841,145.46532378)(370.18212841,145.37678212)
\curveto(370.18212841,145.29865713)(370.14955318,145.22574047)(370.08440274,145.15803214)
\curveto(369.92403241,144.98615715)(369.53312975,144.79865717)(368.91169474,144.59553218)
\curveto(368.2952713,144.39761553)(367.70390572,144.29865721)(367.13759801,144.29865721)
\curveto(366.40089683,144.29865721)(365.81454283,144.47834469)(365.37853601,144.83771966)
\curveto(364.94252919,145.19709464)(364.72452578,145.6814696)(364.72452578,146.29084455)
\lineto(364.72452578,150.65803172)
\lineto(363.55933513,150.65803172)
\curveto(363.40898795,150.65803172)(363.30123914,150.68928172)(363.23608869,150.75178172)
\curveto(363.17093825,150.81428171)(363.13836303,150.89501087)(363.13836303,150.9939692)
\curveto(363.13836303,151.08771919)(363.17093825,151.16584418)(363.23608869,151.22834418)
\curveto(363.30123914,151.29084417)(363.40898795,151.32209417)(363.55933513,151.32209417)
\lineto(364.72452578,151.32209417)
\lineto(364.72452578,153.25959403)
\curveto(364.72452578,153.41584401)(364.75459521,153.52782317)(364.81473408,153.5955315)
\curveto(364.87487296,153.66323983)(364.95004655,153.69709399)(365.04025485,153.69709399)
\curveto(365.13547473,153.69709399)(365.21315411,153.66323983)(365.27329298,153.5955315)
\curveto(365.33343185,153.52782317)(365.36350129,153.41584401)(365.36350129,153.25959403)
\closepath
}
}
{
\newrgbcolor{curcolor}{0 0 0}
\pscustom[linestyle=none,fillstyle=solid,fillcolor=curcolor]
{
\newpath
\moveto(377.71452167,144.55646969)
\lineto(377.71452167,145.50178211)
\curveto(376.79740387,144.69969884)(375.81764142,144.29865721)(374.77523431,144.29865721)
\curveto(374.01848683,144.29865721)(373.42712126,144.49657386)(373.00113758,144.89240716)
\curveto(372.57515391,145.2934488)(372.36216207,145.78303209)(372.36216207,146.36115705)
\curveto(372.36216207,146.99657367)(372.64281014,147.55126113)(373.20410628,148.02521942)
\curveto(373.76540241,148.49917772)(374.58479454,148.73615687)(375.66228267,148.73615687)
\curveto(375.95295388,148.73615687)(376.26868296,148.71532354)(376.6094699,148.67365687)
\curveto(376.95025684,148.63719854)(377.31860743,148.57730271)(377.71452167,148.49396939)
\lineto(377.71452167,149.55646931)
\curveto(377.71452167,149.91584428)(377.55415135,150.22834426)(377.2334107,150.49396924)
\curveto(376.91267005,150.75959422)(376.43155907,150.89240671)(375.79007777,150.89240671)
\curveto(375.29894365,150.89240671)(374.60985241,150.74396922)(373.72280405,150.44709424)
\curveto(373.56243372,150.39501091)(373.45969648,150.36896924)(373.41459233,150.36896924)
\curveto(373.33440716,150.36896924)(373.26424515,150.40021924)(373.20410628,150.46271924)
\curveto(373.14897898,150.52521923)(373.12141533,150.60334423)(373.12141533,150.69709422)
\curveto(373.12141533,150.78563588)(373.14647319,150.85594837)(373.19658892,150.9080317)
\curveto(373.26675093,150.9861567)(373.54990479,151.09292752)(374.04605048,151.22834418)
\curveto(374.82785582,151.44709416)(375.41922139,151.55646915)(375.82014721,151.55646915)
\curveto(376.61698726,151.55646915)(377.23842227,151.35074)(377.68445224,150.9392817)
\curveto(378.1304822,150.53303173)(378.35349719,150.07209427)(378.35349719,149.55646931)
\lineto(378.35349719,145.21271964)
\lineto(379.19544139,145.21271964)
\curveto(379.35080015,145.21271964)(379.46105475,145.18146964)(379.52620519,145.11896964)
\curveto(379.59135563,145.06167798)(379.62393086,144.98355299)(379.62393086,144.88459466)
\curveto(379.62393086,144.79084467)(379.59135563,144.71271967)(379.52620519,144.65021968)
\curveto(379.46105475,144.58771968)(379.35080015,144.55646969)(379.19544139,144.55646969)
\closepath
\moveto(377.71452167,147.82209444)
\curveto(377.41883889,147.9106361)(377.10561559,147.97574026)(376.7748518,148.01740692)
\curveto(376.444088,148.05907359)(376.0957837,148.07990692)(375.7299389,148.07990692)
\curveto(374.8128211,148.07990692)(374.09616621,147.87417777)(373.57997423,147.46271947)
\curveto(373.18907156,147.15542782)(372.99362022,146.78824035)(372.99362022,146.36115705)
\curveto(372.99362022,145.96532375)(373.14146162,145.63199044)(373.4371444,145.36115712)
\curveto(373.73783876,145.09032381)(374.17384559,144.95490716)(374.74516487,144.95490716)
\curveto(375.29142629,144.95490716)(375.79759513,145.06688631)(376.26367139,145.29084463)
\curveto(376.73475922,145.52001128)(377.21837598,145.88199042)(377.71452167,146.37678205)
\closepath
}
}
{
\newrgbcolor{curcolor}{0 0 0}
\pscustom[linestyle=none,fillstyle=solid,fillcolor=curcolor]
{
\newpath
\moveto(338.49327835,208.4850622)
\lineto(338.49327835,209.43037391)
\curveto(337.54015415,208.62829125)(336.52192584,208.22724992)(335.43859341,208.22724992)
\curveto(334.65213574,208.22724992)(334.03755293,208.42516642)(333.59484497,208.82099942)
\curveto(333.152137,209.22204075)(332.93078302,209.71162367)(332.93078302,210.28974818)
\curveto(332.93078302,210.92516432)(333.22244945,211.47985135)(333.80578229,211.95380928)
\curveto(334.38911513,212.42776722)(335.24067691,212.66474619)(336.36046764,212.66474619)
\curveto(336.66255072,212.66474619)(336.99067544,212.64391287)(337.34484181,212.60224624)
\curveto(337.69900818,212.56578794)(338.08182036,212.50589215)(338.49327835,212.42255889)
\lineto(338.49327835,213.485058)
\curveto(338.49327835,213.84443269)(338.32661182,214.15693243)(337.99327877,214.42255721)
\curveto(337.65994571,214.68818199)(337.15994613,214.82099437)(336.49328003,214.82099437)
\curveto(335.98286379,214.82099437)(335.26671856,214.672557)(334.34484433,214.37568225)
\curveto(334.17817781,214.32359896)(334.07140706,214.29755731)(334.0245321,214.29755731)
\curveto(333.94119884,214.29755731)(333.86828224,214.32880729)(333.80578229,214.39130723)
\curveto(333.74849067,214.45380718)(333.71984486,214.53193212)(333.71984486,214.62568204)
\curveto(333.71984486,214.71422363)(333.74588651,214.78453607)(333.79796979,214.83661936)
\curveto(333.8708864,214.91474429)(334.16515699,215.02151504)(334.68078155,215.15693159)
\curveto(335.49328087,215.37568141)(336.10786369,215.48505631)(336.52453,215.48505631)
\curveto(337.35265431,215.48505631)(337.9984871,215.27932732)(338.46202837,214.86786933)
\curveto(338.92556965,214.46161968)(339.15734029,214.00068256)(339.15734029,213.485058)
\lineto(339.15734029,209.14131165)
\lineto(340.03233955,209.14131165)
\curveto(340.19379775,209.14131165)(340.30838099,209.11006168)(340.37608926,209.04756173)
\curveto(340.44379754,208.99027011)(340.47765168,208.91214518)(340.47765168,208.81318693)
\curveto(340.47765168,208.71943701)(340.44379754,208.64131207)(340.37608926,208.57881212)
\curveto(340.30838099,208.51631218)(340.19379775,208.4850622)(340.03233955,208.4850622)
\closepath
\moveto(338.49327835,211.75068446)
\curveto(338.18598694,211.83922605)(337.86046638,211.90433016)(337.51671667,211.94599679)
\curveto(337.17296696,211.98766342)(336.81098809,212.00849674)(336.43078008,212.00849674)
\curveto(335.47765588,212.00849674)(334.73286484,211.80276775)(334.19640696,211.39130976)
\curveto(333.7901573,211.08401835)(333.58703247,210.71683116)(333.58703247,210.28974818)
\curveto(333.58703247,209.89391518)(333.74067818,209.56058213)(334.04796958,209.28974903)
\curveto(334.36046932,209.01891592)(334.81359394,208.88349937)(335.40734344,208.88349937)
\curveto(335.9750513,208.88349937)(336.50109252,208.99547844)(336.98546711,209.21943658)
\curveto(337.47505004,209.44860306)(337.97765378,209.81058192)(338.49327835,210.30537317)
\closepath
}
}
{
\newrgbcolor{curcolor}{0 0 0}
\pscustom[linestyle=none,fillstyle=solid,fillcolor=curcolor]
{
\newpath
\moveto(345.42295986,215.25068151)
\lineto(345.42295986,213.5944329)
\curveto(346.27712581,214.36526559)(346.91514611,214.86005684)(347.33702075,215.07880666)
\curveto(347.76410373,215.3027648)(348.15733256,215.41474387)(348.51670726,215.41474387)
\curveto(348.90733193,215.41474387)(349.26931079,215.28193149)(349.60264385,215.01630671)
\curveto(349.94118523,214.75589026)(350.11045592,214.55797376)(350.11045592,214.42255721)
\curveto(350.11045592,214.32359896)(350.07660178,214.2402657)(350.00889351,214.17255742)
\curveto(349.94639356,214.11005747)(349.86566446,214.0788075)(349.76670621,214.0788075)
\curveto(349.71462292,214.0788075)(349.67035212,214.08661999)(349.63389382,214.10224498)
\curveto(349.59743552,214.12307829)(349.52972724,214.18297408)(349.43076899,214.28193233)
\curveto(349.24847748,214.46422384)(349.08962345,214.58922373)(348.95420689,214.65693201)
\curveto(348.81879034,214.72464029)(348.68597795,214.75849443)(348.55576973,214.75849443)
\curveto(348.26931164,214.75849443)(347.92295776,214.64391119)(347.5167081,214.41474472)
\curveto(347.11566677,214.18557824)(346.41775069,213.62568288)(345.42295986,212.73505863)
\lineto(345.42295986,209.14131165)
\lineto(348.32920742,209.14131165)
\curveto(348.49066562,209.14131165)(348.60524885,209.11006168)(348.67295713,209.04756173)
\curveto(348.74066541,208.99027011)(348.77451954,208.91214518)(348.77451954,208.81318693)
\curveto(348.77451954,208.71943701)(348.74066541,208.64131207)(348.67295713,208.57881212)
\curveto(348.60524885,208.51631218)(348.49066562,208.4850622)(348.32920742,208.4850622)
\lineto(343.18077425,208.4850622)
\curveto(343.02452438,208.4850622)(342.91254531,208.51370801)(342.84483703,208.57099963)
\curveto(342.77712875,208.63349958)(342.74327462,208.71162451)(342.74327462,208.80537443)
\curveto(342.74327462,208.89391602)(342.77452459,208.96683263)(342.83702454,209.02412425)
\curveto(342.90473281,209.0866242)(343.01931605,209.11787417)(343.18077425,209.11787417)
\lineto(344.76671041,209.11787417)
\lineto(344.76671041,214.58661957)
\lineto(343.55577393,214.58661957)
\curveto(343.39952406,214.58661957)(343.28754499,214.61786954)(343.21983672,214.68036949)
\curveto(343.15212844,214.74286944)(343.1182743,214.82359854)(343.1182743,214.92255679)
\curveto(343.1182743,215.01630671)(343.14952427,215.09443164)(343.21202422,215.15693159)
\curveto(343.2797325,215.21943154)(343.39431574,215.25068151)(343.55577393,215.25068151)
\closepath
}
}
{
\newrgbcolor{curcolor}{0 0 0}
\pscustom[linestyle=none,fillstyle=solid,fillcolor=curcolor]
{
\newpath
\moveto(358.54794867,214.58661957)
\lineto(358.54794867,214.80536939)
\curveto(358.54794867,214.96682758)(358.57919864,215.08141082)(358.64169859,215.1491191)
\curveto(358.70419854,215.21682737)(358.78232347,215.25068151)(358.87607339,215.25068151)
\curveto(358.97503164,215.25068151)(359.05576074,215.21682737)(359.11826069,215.1491191)
\curveto(359.18076064,215.08141082)(359.21201061,214.96682758)(359.21201061,214.80536939)
\lineto(359.21201061,213.32099563)
\curveto(359.20680228,213.15953744)(359.17294814,213.0449542)(359.1104482,212.97724592)
\curveto(359.05315658,212.90953765)(358.97503164,212.87568351)(358.87607339,212.87568351)
\curveto(358.7875318,212.87568351)(358.71201103,212.90432932)(358.64951108,212.96162094)
\curveto(358.59221947,213.02412088)(358.55836533,213.1256833)(358.54794867,213.26630818)
\curveto(358.5166987,213.63609954)(358.27190723,213.98766174)(357.81357429,214.32099479)
\curveto(357.36044967,214.65432785)(356.74847102,214.82099437)(355.97763833,214.82099437)
\curveto(355.00368082,214.82099437)(354.26409811,214.51630713)(353.7588902,213.90693264)
\curveto(353.25368229,213.29755815)(353.00107833,212.59964207)(353.00107833,211.8131844)
\curveto(353.00107833,210.96422678)(353.27972393,210.26370654)(353.83701513,209.71162367)
\curveto(354.39430633,209.1595408)(355.11565989,208.88349937)(356.00107581,208.88349937)
\curveto(356.51149205,208.88349937)(357.02972078,208.97724929)(357.555762,209.16474913)
\curveto(358.08701156,209.35224897)(358.56617782,209.65433205)(358.99326079,210.07099837)
\curveto(359.1026357,210.17516495)(359.19898979,210.22724824)(359.28232305,210.22724824)
\curveto(359.37086464,210.22724824)(359.44378125,210.19599826)(359.50107287,210.13349832)
\curveto(359.56357282,210.0762067)(359.59482279,210.00329009)(359.59482279,209.9147485)
\curveto(359.59482279,209.69079035)(359.33180218,209.40693643)(358.80576095,209.06318672)
\curveto(357.95680333,208.50589552)(357.01149163,208.22724992)(355.96982584,208.22724992)
\curveto(354.91253506,208.22724992)(354.04274412,208.56318714)(353.36045303,209.23506157)
\curveto(352.68337027,209.91214434)(352.34482889,210.76891445)(352.34482889,211.80537191)
\curveto(352.34482889,212.86266269)(352.69118276,213.74026612)(353.38389051,214.4381822)
\curveto(354.08180659,215.13609828)(354.95941002,215.48505631)(356.0167008,215.48505631)
\curveto(357.02190829,215.48505631)(357.86565758,215.1855774)(358.54794867,214.58661957)
\closepath
}
}
{
\newrgbcolor{curcolor}{0 0 0}
\pscustom[linestyle=none,fillstyle=solid,fillcolor=curcolor]
{
\newpath
\moveto(369.77450247,204.10225339)
\lineto(361.04794731,204.10225339)
\curveto(360.89169744,204.10225339)(360.77971837,204.13350336)(360.7120101,204.19600331)
\curveto(360.64430182,204.25329493)(360.61044768,204.33141986)(360.61044768,204.43037811)
\curveto(360.61044768,204.52933636)(360.64430182,204.61006546)(360.7120101,204.67256541)
\curveto(360.77971837,204.72985703)(360.89169744,204.75850284)(361.04794731,204.75850284)
\lineto(369.77450247,204.75850284)
\curveto(369.93596067,204.75850284)(370.04793974,204.72985703)(370.11043969,204.67256541)
\curveto(370.17814797,204.61006546)(370.21200211,204.52933636)(370.21200211,204.43037811)
\curveto(370.21200211,204.33141986)(370.17814797,204.25329493)(370.11043969,204.19600331)
\curveto(370.04793974,204.13350336)(369.93596067,204.10225339)(369.77450247,204.10225339)
\closepath
}
}
{
\newrgbcolor{curcolor}{0 0 0}
\pscustom[linestyle=none,fillstyle=solid,fillcolor=curcolor]
{
\newpath
\moveto(372.54012409,218.28974146)
\lineto(372.54012409,213.93818262)
\curveto(373.33179009,214.96943175)(374.28751846,215.48505631)(375.40730918,215.48505631)
\curveto(376.36564171,215.48505631)(377.18595352,215.13609828)(377.86824461,214.4381822)
\curveto(378.5505357,213.74547444)(378.89168125,212.89391266)(378.89168125,211.88349684)
\curveto(378.89168125,210.86266437)(378.54532737,209.99808176)(377.85261962,209.28974903)
\curveto(377.1651202,208.58141629)(376.35001672,208.22724992)(375.40730918,208.22724992)
\curveto(374.26147681,208.22724992)(373.30574845,208.74287449)(372.54012409,209.77412362)
\lineto(372.54012409,208.4850622)
\lineto(371.00106289,208.4850622)
\curveto(370.84481302,208.4850622)(370.73283394,208.51631218)(370.66512567,208.57881212)
\curveto(370.59741739,208.64131207)(370.56356325,208.71943701)(370.56356325,208.81318693)
\curveto(370.56356325,208.91214518)(370.59741739,208.99027011)(370.66512567,209.04756173)
\curveto(370.73283394,209.11006168)(370.84481302,209.14131165)(371.00106289,209.14131165)
\lineto(371.88387464,209.14131165)
\lineto(371.88387464,217.62567951)
\lineto(371.00106289,217.62567951)
\curveto(370.84481302,217.62567951)(370.73283394,217.65692949)(370.66512567,217.71942944)
\curveto(370.59741739,217.78192938)(370.56356325,217.86265848)(370.56356325,217.96161673)
\curveto(370.56356325,218.05536665)(370.59741739,218.13349159)(370.66512567,218.19599153)
\curveto(370.73283394,218.25849148)(370.84481302,218.28974146)(371.00106289,218.28974146)
\closepath
\moveto(378.2354318,211.85224687)
\curveto(378.2354318,212.68037117)(377.95157787,213.38089142)(377.38387002,213.9538076)
\curveto(376.81616216,214.53193212)(376.15210022,214.82099437)(375.39168419,214.82099437)
\curveto(374.63126817,214.82099437)(373.96720622,214.53193212)(373.39949837,213.9538076)
\curveto(372.83179051,213.38089142)(372.54793659,212.68037117)(372.54793659,211.85224687)
\curveto(372.54793659,211.02412257)(372.83179051,210.32099816)(373.39949837,209.74287364)
\curveto(373.96720622,209.16995746)(374.63126817,208.88349937)(375.39168419,208.88349937)
\curveto(376.15210022,208.88349937)(376.81616216,209.16995746)(377.38387002,209.74287364)
\curveto(377.95157787,210.32099816)(378.2354318,211.02412257)(378.2354318,211.85224687)
\closepath
}
}
{
\newrgbcolor{curcolor}{0 0 0}
\pscustom[linestyle=none,fillstyle=solid,fillcolor=curcolor]
{
\newpath
\moveto(386.86042349,208.4850622)
\lineto(386.86042349,209.44599889)
\curveto(385.96459091,208.63349958)(384.99584173,208.22724992)(383.95417594,208.22724992)
\curveto(383.31355147,208.22724992)(382.82657272,208.40172894)(382.49323966,208.75068698)
\curveto(382.06094836,209.20901993)(381.84480271,209.74287364)(381.84480271,210.35224813)
\lineto(381.84480271,214.58661957)
\lineto(380.96199095,214.58661957)
\curveto(380.80574108,214.58661957)(380.69376201,214.61786954)(380.62605374,214.68036949)
\curveto(380.55834546,214.74286944)(380.52449132,214.82359854)(380.52449132,214.92255679)
\curveto(380.52449132,215.01630671)(380.55834546,215.09443164)(380.62605374,215.15693159)
\curveto(380.69376201,215.21943154)(380.80574108,215.25068151)(380.96199095,215.25068151)
\lineto(382.50105216,215.25068151)
\lineto(382.50105216,210.35224813)
\curveto(382.50105216,209.92516516)(382.63646871,209.57360295)(382.90730182,209.29756152)
\curveto(383.17813492,209.02152008)(383.5166763,208.88349937)(383.92292596,208.88349937)
\curveto(384.9906334,208.88349937)(385.96979924,209.37308229)(386.86042349,210.35224813)
\lineto(386.86042349,214.58661957)
\lineto(385.64948701,214.58661957)
\curveto(385.49323714,214.58661957)(385.38125807,214.61786954)(385.31354979,214.68036949)
\curveto(385.24584152,214.74286944)(385.21198738,214.82359854)(385.21198738,214.92255679)
\curveto(385.21198738,215.01630671)(385.24584152,215.09443164)(385.31354979,215.15693159)
\curveto(385.38125807,215.21943154)(385.49323714,215.25068151)(385.64948701,215.25068151)
\lineto(387.51667294,215.25068151)
\lineto(387.51667294,209.14131165)
\lineto(388.07135997,209.14131165)
\curveto(388.22760984,209.14131165)(388.33958891,209.11006168)(388.40729719,209.04756173)
\curveto(388.47500547,208.99027011)(388.50885961,208.91214518)(388.50885961,208.81318693)
\curveto(388.50885961,208.71943701)(388.47500547,208.64131207)(388.40729719,208.57881212)
\curveto(388.33958891,208.51631218)(388.22760984,208.4850622)(388.07135997,208.4850622)
\closepath
}
}
{
\newrgbcolor{curcolor}{0 0 0}
\pscustom[linestyle=none,fillstyle=solid,fillcolor=curcolor]
{
\newpath
\moveto(393.78229522,214.58661957)
\lineto(393.78229522,209.14131165)
\lineto(396.66510529,209.14131165)
\curveto(396.82135516,209.14131165)(396.93333424,209.11006168)(397.00104251,209.04756173)
\curveto(397.06875079,208.99027011)(397.10260493,208.91214518)(397.10260493,208.81318693)
\curveto(397.10260493,208.71943701)(397.06875079,208.64131207)(397.00104251,208.57881212)
\curveto(396.93333424,208.51631218)(396.82135516,208.4850622)(396.66510529,208.4850622)
\lineto(391.53229711,208.4850622)
\curveto(391.37604724,208.4850622)(391.26406817,208.51631218)(391.19635989,208.57881212)
\curveto(391.12865162,208.64131207)(391.09479748,208.71943701)(391.09479748,208.81318693)
\curveto(391.09479748,208.91214518)(391.12865162,208.99027011)(391.19635989,209.04756173)
\curveto(391.26406817,209.11006168)(391.37604724,209.14131165)(391.53229711,209.14131165)
\lineto(393.11823328,209.14131165)
\lineto(393.11823328,214.58661957)
\lineto(391.69635947,214.58661957)
\curveto(391.5401096,214.58661957)(391.42813053,214.61786954)(391.36042226,214.68036949)
\curveto(391.29271398,214.74286944)(391.25885984,214.82359854)(391.25885984,214.92255679)
\curveto(391.25885984,215.01630671)(391.29271398,215.09443164)(391.36042226,215.15693159)
\curveto(391.42813053,215.21943154)(391.5401096,215.25068151)(391.69635947,215.25068151)
\lineto(393.11823328,215.25068151)
\lineto(393.11823328,216.24286818)
\curveto(393.11823328,216.79495105)(393.34219142,217.27411731)(393.79010771,217.68036697)
\curveto(394.238024,218.08661663)(394.8317735,218.28974146)(395.57135621,218.28974146)
\curveto(396.19114736,218.28974146)(396.85260514,218.23244984)(397.55572955,218.1178666)
\curveto(397.82135432,218.07619997)(397.98020835,218.02672084)(398.03229164,217.96942923)
\curveto(398.08958326,217.91213761)(398.11822907,217.83661684)(398.11822907,217.74286692)
\curveto(398.11822907,217.64911699)(398.0869791,217.57099206)(398.02447915,217.50849211)
\curveto(397.9619792,217.45120049)(397.87864594,217.42255469)(397.77447936,217.42255469)
\curveto(397.73281273,217.42255469)(397.66250029,217.43036718)(397.56354204,217.44599217)
\curveto(396.77708437,217.56578373)(396.11302243,217.62567951)(395.57135621,217.62567951)
\curveto(394.99844003,217.62567951)(394.55573207,217.48505463)(394.24323233,217.20380487)
\curveto(393.93594092,216.92255511)(393.78229522,216.60224288)(393.78229522,216.24286818)
\lineto(393.78229522,215.25068151)
\lineto(396.85260514,215.25068151)
\curveto(397.008855,215.25068151)(397.12083408,215.21943154)(397.18854235,215.15693159)
\curveto(397.25625063,215.09443164)(397.29010477,215.01370254)(397.29010477,214.91474429)
\curveto(397.29010477,214.82099437)(397.25625063,214.74286944)(397.18854235,214.68036949)
\curveto(397.12083408,214.61786954)(397.008855,214.58661957)(396.85260514,214.58661957)
\closepath
}
}
{
\newrgbcolor{curcolor}{0 0 0}
\pscustom[linestyle=none,fillstyle=solid,fillcolor=curcolor]
{
\newpath
\moveto(408.18071955,204.10225339)
\lineto(399.45416439,204.10225339)
\curveto(399.29791452,204.10225339)(399.18593545,204.13350336)(399.11822717,204.19600331)
\curveto(399.0505189,204.25329493)(399.01666476,204.33141986)(399.01666476,204.43037811)
\curveto(399.01666476,204.52933636)(399.0505189,204.61006546)(399.11822717,204.67256541)
\curveto(399.18593545,204.72985703)(399.29791452,204.75850284)(399.45416439,204.75850284)
\lineto(408.18071955,204.75850284)
\curveto(408.34217775,204.75850284)(408.45415682,204.72985703)(408.51665677,204.67256541)
\curveto(408.58436505,204.61006546)(408.61821919,204.52933636)(408.61821919,204.43037811)
\curveto(408.61821919,204.33141986)(408.58436505,204.25329493)(408.51665677,204.19600331)
\curveto(408.45415682,204.13350336)(408.34217775,204.10225339)(408.18071955,204.10225339)
\closepath
}
}
{
\newrgbcolor{curcolor}{0 0 0}
\pscustom[linestyle=none,fillstyle=solid,fillcolor=curcolor]
{
\newpath
\moveto(412.07134022,215.25068151)
\lineto(415.63383723,215.25068151)
\curveto(415.7900871,215.25068151)(415.90206617,215.21943154)(415.96977445,215.15693159)
\curveto(416.03748272,215.09443164)(416.07133686,215.01370254)(416.07133686,214.91474429)
\curveto(416.07133686,214.82099437)(416.03748272,214.74286944)(415.96977445,214.68036949)
\curveto(415.90206617,214.61786954)(415.7900871,214.58661957)(415.63383723,214.58661957)
\lineto(412.07134022,214.58661957)
\lineto(412.07134022,210.21943574)
\curveto(412.07134022,209.83922773)(412.22238176,209.52151966)(412.52446484,209.26631155)
\curveto(412.83175625,209.01110343)(413.27967254,208.88349937)(413.86821371,208.88349937)
\curveto(414.31092167,208.88349937)(414.79008794,208.94860348)(415.3057125,209.0788117)
\curveto(415.82133707,209.21422826)(416.2223784,209.3652698)(416.50883649,209.53193632)
\curveto(416.61300307,209.5996446)(416.6989405,209.63349874)(416.76664878,209.63349874)
\curveto(416.84998204,209.63349874)(416.92289864,209.5996446)(416.98539859,209.53193632)
\curveto(417.04789854,209.46943637)(417.07914851,209.3939156)(417.07914851,209.30537401)
\curveto(417.07914851,209.22724908)(417.04529437,209.15433247)(416.9775861,209.0866242)
\curveto(416.81091957,208.91474934)(416.40466991,208.7272495)(415.75883712,208.52412467)
\curveto(415.11821266,208.32620817)(414.50362985,208.22724992)(413.91508867,208.22724992)
\curveto(413.14946432,208.22724992)(412.54008983,208.40693727)(412.08696521,208.76631197)
\curveto(411.63384059,209.12568666)(411.40727828,209.61006126)(411.40727828,210.21943574)
\lineto(411.40727828,214.58661957)
\lineto(410.1963418,214.58661957)
\curveto(410.04009193,214.58661957)(409.92811286,214.61786954)(409.86040458,214.68036949)
\curveto(409.79269631,214.74286944)(409.75884217,214.82359854)(409.75884217,214.92255679)
\curveto(409.75884217,215.01630671)(409.79269631,215.09443164)(409.86040458,215.15693159)
\curveto(409.92811286,215.21943154)(410.04009193,215.25068151)(410.1963418,215.25068151)
\lineto(411.40727828,215.25068151)
\lineto(411.40727828,217.18817988)
\curveto(411.40727828,217.34442975)(411.43852826,217.45640882)(411.5010282,217.5241171)
\curveto(411.56352815,217.59182538)(411.64165309,217.62567951)(411.73540301,217.62567951)
\curveto(411.83436126,217.62567951)(411.91509036,217.59182538)(411.9775903,217.5241171)
\curveto(412.04009025,217.45640882)(412.07134022,217.34442975)(412.07134022,217.18817988)
\closepath
}
}
{
\newrgbcolor{curcolor}{0 0 0}
\pscustom[linewidth=1.88976378,linecolor=curcolor]
{
\newpath
\moveto(524.70407815,183.60627697)
\lineto(668.12042731,183.60627697)
\lineto(668.12042731,161.64601278)
\lineto(524.70407815,161.64601278)
\closepath
}
}
{
\newrgbcolor{curcolor}{0 0 0}
\pscustom[linestyle=none,fillstyle=solid,fillcolor=curcolor]
{
\newpath
\moveto(528.92652795,180.47745557)
\lineto(528.92652795,176.12589342)
\curveto(529.68828699,177.15714334)(530.60791057,177.6727683)(531.68539869,177.6727683)
\curveto(532.60752805,177.6727683)(533.39685075,177.32380999)(534.05336676,176.62589338)
\curveto(534.70988278,175.9331851)(535.03814079,175.08162267)(535.03814079,174.07120608)
\curveto(535.03814079,173.05037283)(534.70487121,172.18578957)(534.03833204,171.47745629)
\curveto(533.37680446,170.76912301)(532.59249334,170.41495638)(531.68539869,170.41495638)
\curveto(530.58285271,170.41495638)(529.66322913,170.93058133)(528.92652795,171.96183125)
\lineto(528.92652795,170.67276886)
\lineto(527.44560823,170.67276886)
\curveto(527.29526105,170.67276886)(527.18751224,170.70401885)(527.1223618,170.76651885)
\curveto(527.05721135,170.82901884)(527.02463613,170.90714384)(527.02463613,171.00089383)
\curveto(527.02463613,171.09985215)(527.05721135,171.17797715)(527.1223618,171.23526881)
\curveto(527.18751224,171.29776881)(527.29526105,171.3290188)(527.44560823,171.3290188)
\lineto(528.29506979,171.3290188)
\lineto(528.29506979,179.81339313)
\lineto(527.44560823,179.81339313)
\curveto(527.29526105,179.81339313)(527.18751224,179.84464312)(527.1223618,179.90714312)
\curveto(527.05721135,179.96964311)(527.02463613,180.05037227)(527.02463613,180.1493306)
\curveto(527.02463613,180.24308059)(527.05721135,180.32120559)(527.1223618,180.38370558)
\curveto(527.18751224,180.44620558)(527.29526105,180.47745557)(527.44560823,180.47745557)
\closepath
\moveto(534.40668263,174.03995609)
\curveto(534.40668263,174.86808102)(534.13355192,175.5686018)(533.58729051,176.14151842)
\curveto(533.04102909,176.71964337)(532.40205358,177.00870585)(531.67036397,177.00870585)
\curveto(530.93867436,177.00870585)(530.29969885,176.71964337)(529.75343743,176.14151842)
\curveto(529.20717602,175.5686018)(528.93404531,174.86808102)(528.93404531,174.03995609)
\curveto(528.93404531,173.21183115)(529.20717602,172.50870621)(529.75343743,171.93058125)
\curveto(530.29969885,171.35766463)(530.93867436,171.07120632)(531.67036397,171.07120632)
\curveto(532.40205358,171.07120632)(533.04102909,171.35766463)(533.58729051,171.93058125)
\curveto(534.13355192,172.50870621)(534.40668263,173.21183115)(534.40668263,174.03995609)
\closepath
}
}
{
\newrgbcolor{curcolor}{0 0 0}
\pscustom[linestyle=none,fillstyle=solid,fillcolor=curcolor]
{
\newpath
\moveto(544.74305165,166.2899567)
\lineto(536.34616168,166.2899567)
\curveto(536.19581451,166.2899567)(536.08806569,166.3212067)(536.02291525,166.3837067)
\curveto(535.9577648,166.44099836)(535.92518958,166.51912335)(535.92518958,166.61808168)
\curveto(535.92518958,166.71704)(535.9577648,166.79776916)(536.02291525,166.86026916)
\curveto(536.08806569,166.91756082)(536.19581451,166.94620665)(536.34616168,166.94620665)
\lineto(544.74305165,166.94620665)
\curveto(544.8984104,166.94620665)(545.00615921,166.91756082)(545.06629809,166.86026916)
\curveto(545.13144853,166.79776916)(545.16402375,166.71704)(545.16402375,166.61808168)
\curveto(545.16402375,166.51912335)(545.13144853,166.44099836)(545.06629809,166.3837067)
\curveto(545.00615921,166.3212067)(544.8984104,166.2899567)(544.74305165,166.2899567)
\closepath
}
}
{
\newrgbcolor{curcolor}{0 0 0}
\pscustom[linestyle=none,fillstyle=solid,fillcolor=curcolor]
{
\newpath
\moveto(547.74247831,177.43839332)
\lineto(547.74247831,176.4462059)
\curveto(548.18349671,176.90974752)(548.58191673,177.23006)(548.93773839,177.40714332)
\curveto(549.29356005,177.58422664)(549.69448586,177.6727683)(550.14051582,177.6727683)
\curveto(550.6216268,177.6727683)(551.0601394,177.56599747)(551.45605364,177.35245582)
\curveto(551.73670171,177.19620584)(551.98978613,176.93578919)(552.2153069,176.57120589)
\curveto(552.44583924,176.21183091)(552.56110541,175.84203928)(552.56110541,175.46183097)
\lineto(552.56110541,171.3290188)
\lineto(553.0948379,171.3290188)
\curveto(553.24518508,171.3290188)(553.35293389,171.29776881)(553.41808433,171.23526881)
\curveto(553.48323478,171.17797715)(553.51581,171.09985215)(553.51581,171.00089383)
\curveto(553.51581,170.90714384)(553.48323478,170.82901884)(553.41808433,170.76651885)
\curveto(553.35293389,170.70401885)(553.24518508,170.67276886)(553.0948379,170.67276886)
\lineto(551.40343213,170.67276886)
\curveto(551.24807338,170.67276886)(551.13781878,170.70401885)(551.07266834,170.76651885)
\curveto(551.00751789,170.82901884)(550.97494267,170.90714384)(550.97494267,171.00089383)
\curveto(550.97494267,171.09985215)(551.00751789,171.17797715)(551.07266834,171.23526881)
\curveto(551.13781878,171.29776881)(551.24807338,171.3290188)(551.40343213,171.3290188)
\lineto(551.92964726,171.3290188)
\lineto(551.92964726,175.35245598)
\curveto(551.92964726,175.81599761)(551.76677115,176.20662258)(551.44101893,176.52433089)
\curveto(551.1152667,176.84724753)(550.67925988,177.00870585)(550.13299847,177.00870585)
\curveto(549.71703794,177.00870585)(549.35620471,176.92016419)(549.05049877,176.74308087)
\curveto(548.74479284,176.57120589)(548.30878602,176.13891425)(547.74247831,175.44620597)
\lineto(547.74247831,171.3290188)
\lineto(548.45662742,171.3290188)
\curveto(548.6069746,171.3290188)(548.71472341,171.29776881)(548.77987385,171.23526881)
\curveto(548.8450243,171.17797715)(548.87759952,171.09985215)(548.87759952,171.00089383)
\curveto(548.87759952,170.90714384)(548.8450243,170.82901884)(548.77987385,170.76651885)
\curveto(548.71472341,170.70401885)(548.6069746,170.67276886)(548.45662742,170.67276886)
\lineto(546.39687106,170.67276886)
\curveto(546.24652388,170.67276886)(546.13877507,170.70401885)(546.07362462,170.76651885)
\curveto(546.00847418,170.82901884)(545.97589896,170.90714384)(545.97589896,171.00089383)
\curveto(545.97589896,171.09985215)(546.00847418,171.17797715)(546.07362462,171.23526881)
\curveto(546.13877507,171.29776881)(546.24652388,171.3290188)(546.39687106,171.3290188)
\lineto(547.11102016,171.3290188)
\lineto(547.11102016,176.77433087)
\lineto(546.57728767,176.77433087)
\curveto(546.4269405,176.77433087)(546.31919168,176.80558087)(546.25404124,176.86808086)
\curveto(546.1888908,176.93058086)(546.15631557,177.01131002)(546.15631557,177.11026834)
\curveto(546.15631557,177.20401833)(546.1888908,177.28214333)(546.25404124,177.34464332)
\curveto(546.31919168,177.40714332)(546.4269405,177.43839332)(546.57728767,177.43839332)
\closepath
}
}
{
\newrgbcolor{curcolor}{0 0 0}
\pscustom[linestyle=none,fillstyle=solid,fillcolor=curcolor]
{
\newpath
\moveto(562.43891641,173.8993311)
\lineto(556.00405714,173.8993311)
\curveto(556.11431174,173.05037283)(556.45509868,172.36547705)(557.02641796,171.84464376)
\curveto(557.60274881,171.3290188)(558.31439213,171.07120632)(559.16134791,171.07120632)
\curveto(559.63243573,171.07120632)(560.12607564,171.15193548)(560.64226762,171.3133938)
\curveto(561.1584596,171.47485212)(561.57943171,171.68839377)(561.90518393,171.95401875)
\curveto(562.00040381,172.03214375)(562.08309476,172.07120624)(562.15325677,172.07120624)
\curveto(562.23344194,172.07120624)(562.30360395,172.03735208)(562.36374282,171.96964375)
\curveto(562.4238817,171.90714376)(562.45395113,171.83162293)(562.45395113,171.74308127)
\curveto(562.45395113,171.65453961)(562.41385855,171.56860212)(562.33367339,171.48526879)
\curveto(562.0931179,171.22485214)(561.66462844,170.9800605)(561.04820501,170.75089385)
\curveto(560.43679314,170.52693553)(559.80784078,170.41495638)(559.16134791,170.41495638)
\curveto(558.07884821,170.41495638)(557.17425935,170.78214385)(556.44758132,171.51651879)
\curveto(555.72591486,172.25610206)(555.36508163,173.14933116)(555.36508163,174.19620607)
\curveto(555.36508163,175.149331)(555.70336278,175.96703927)(556.37992509,176.64933088)
\curveto(557.06149897,177.33162249)(557.90344317,177.6727683)(558.9057577,177.6727683)
\curveto(559.93814167,177.6727683)(560.78760323,177.32120583)(561.45414239,176.61808088)
\curveto(562.12068155,175.92016427)(562.44893956,175.01391434)(562.43891641,173.8993311)
\closepath
\moveto(561.7999409,174.56339355)
\curveto(561.67465159,175.28735182)(561.34388779,175.87589344)(560.80764952,176.3290184)
\curveto(560.27642282,176.78214337)(559.64245888,177.00870585)(558.9057577,177.00870585)
\curveto(558.16905652,177.00870585)(557.53509258,176.78474753)(557.00386588,176.3368309)
\curveto(556.47263918,175.88891427)(556.14187539,175.29776849)(556.0115745,174.56339355)
\closepath
}
}
{
\newrgbcolor{curcolor}{0 0 0}
\pscustom[linestyle=none,fillstyle=solid,fillcolor=curcolor]
{
\newpath
\moveto(568.69335863,174.21183107)
\lineto(571.58754183,171.3290188)
\curveto(571.77798159,171.3290188)(571.89825933,171.31860214)(571.94837506,171.29776881)
\curveto(571.99849079,171.28214381)(572.04108915,171.24568548)(572.07617016,171.18839381)
\curveto(572.11626274,171.13110215)(572.13630903,171.06860216)(572.13630903,171.00089383)
\curveto(572.13630903,170.90714384)(572.10373381,170.82901884)(572.03858337,170.76651885)
\curveto(571.97343292,170.70401885)(571.86317833,170.67276886)(571.70781957,170.67276886)
\lineto(569.67813265,170.67276886)
\curveto(569.5227739,170.67276886)(569.4125193,170.70401885)(569.34736886,170.76651885)
\curveto(569.28221841,170.82901884)(569.24964319,170.90714384)(569.24964319,171.00089383)
\curveto(569.24964319,171.09985215)(569.28221841,171.17797715)(569.34736886,171.23526881)
\curveto(569.4125193,171.29776881)(569.5227739,171.3290188)(569.67813265,171.3290188)
\lineto(570.72304555,171.3290188)
\lineto(568.26486917,173.77433111)
\lineto(565.8217275,171.3290188)
\lineto(566.87415776,171.3290188)
\curveto(567.02450494,171.3290188)(567.13225375,171.29776881)(567.19740419,171.23526881)
\curveto(567.26255464,171.17797715)(567.29512986,171.09985215)(567.29512986,171.00089383)
\curveto(567.29512986,170.90714384)(567.26255464,170.82901884)(567.19740419,170.76651885)
\curveto(567.13225375,170.70401885)(567.02450494,170.67276886)(566.87415776,170.67276886)
\lineto(564.82191876,170.67276886)
\curveto(564.67157158,170.67276886)(564.56382277,170.70401885)(564.49867232,170.76651885)
\curveto(564.43352188,170.82901884)(564.40094666,170.90714384)(564.40094666,171.00089383)
\curveto(564.40094666,171.06860216)(564.41848716,171.13110215)(564.45356817,171.18839381)
\curveto(564.49366075,171.24568548)(564.53625912,171.28214381)(564.58136327,171.29776881)
\curveto(564.631479,171.31860214)(564.75175674,171.3290188)(564.9421965,171.3290188)
\lineto(567.83637971,174.21183107)
\lineto(565.26544294,176.77433087)
\curveto(565.08502632,176.77433087)(564.96976015,176.78474753)(564.91964443,176.80558087)
\curveto(564.87454027,176.8264142)(564.83194191,176.8654767)(564.79184932,176.92276836)
\curveto(564.75676832,176.98006002)(564.73922781,177.04256001)(564.73922781,177.11026834)
\curveto(564.73922781,177.20401833)(564.77180303,177.28214333)(564.83695348,177.34464332)
\curveto(564.90210392,177.40714332)(565.01235852,177.43839332)(565.16771727,177.43839332)
\lineto(566.88919248,177.43839332)
\curveto(567.04455123,177.43839332)(567.15230004,177.40714332)(567.21243891,177.34464332)
\curveto(567.27758936,177.28214333)(567.31016458,177.20141417)(567.31016458,177.10245584)
\curveto(567.31016458,177.00870585)(567.27758936,176.93058086)(567.21243891,176.86808086)
\curveto(567.15230004,176.80558087)(567.04455123,176.77433087)(566.88919248,176.77433087)
\lineto(566.13745658,176.77433087)
\lineto(568.26486917,174.64151854)
\lineto(570.40731647,176.77433087)
\lineto(569.65558058,176.77433087)
\curveto(569.5052334,176.77433087)(569.3949788,176.80558087)(569.32481678,176.86808086)
\curveto(569.25966634,176.93058086)(569.22709111,177.01131002)(569.22709111,177.11026834)
\curveto(569.22709111,177.20401833)(569.25966634,177.28214333)(569.32481678,177.34464332)
\curveto(569.38996723,177.40714332)(569.50022182,177.43839332)(569.65558058,177.43839332)
\lineto(571.37705578,177.43839332)
\curveto(571.52740296,177.43839332)(571.63515177,177.40714332)(571.70030221,177.34464332)
\curveto(571.76545266,177.28214333)(571.79802788,177.20141417)(571.79802788,177.10245584)
\curveto(571.79802788,177.03995585)(571.77798159,176.98006002)(571.73788901,176.92276836)
\curveto(571.702808,176.8654767)(571.66020963,176.8264142)(571.61009391,176.80558087)
\curveto(571.56498975,176.78474753)(571.45222937,176.77433087)(571.27181275,176.77433087)
\closepath
}
}
{
\newrgbcolor{curcolor}{0 0 0}
\pscustom[linestyle=none,fillstyle=solid,fillcolor=curcolor]
{
\newpath
\moveto(576.20319979,177.43839332)
\lineto(579.63111548,177.43839332)
\curveto(579.78146266,177.43839332)(579.88921147,177.40714332)(579.95436191,177.34464332)
\curveto(580.01951236,177.28214333)(580.05208758,177.20141417)(580.05208758,177.10245584)
\curveto(580.05208758,177.00870585)(580.01951236,176.93058086)(579.95436191,176.86808086)
\curveto(579.88921147,176.80558087)(579.78146266,176.77433087)(579.63111548,176.77433087)
\lineto(576.20319979,176.77433087)
\lineto(576.20319979,172.40714372)
\curveto(576.20319979,172.02693541)(576.3485354,171.70922711)(576.63920661,171.45401879)
\curveto(576.93488939,171.19881048)(577.36588464,171.07120632)(577.93219235,171.07120632)
\curveto(578.35817603,171.07120632)(578.81924071,171.13631048)(579.3153864,171.26651881)
\curveto(579.81153209,171.40193546)(580.19742319,171.55297712)(580.47305968,171.71964377)
\curveto(580.57329113,171.7873521)(580.65598208,171.82120626)(580.72113253,171.82120626)
\curveto(580.80131769,171.82120626)(580.87147971,171.7873521)(580.93161858,171.71964377)
\curveto(580.99175745,171.65714378)(581.02182689,171.58162295)(581.02182689,171.49308129)
\curveto(581.02182689,171.4149563)(580.98925166,171.34203964)(580.92410122,171.27433131)
\curveto(580.76373089,171.10245632)(580.37282823,170.91495634)(579.75139322,170.71183135)
\curveto(579.13496979,170.5139147)(578.54360421,170.41495638)(577.9772965,170.41495638)
\curveto(577.24059533,170.41495638)(576.65424133,170.59464386)(576.21823451,170.95401883)
\curveto(575.78222769,171.3133938)(575.56422428,171.79776877)(575.56422428,172.40714372)
\lineto(575.56422428,176.77433087)
\lineto(574.39903364,176.77433087)
\curveto(574.24868646,176.77433087)(574.14093765,176.80558087)(574.0757872,176.86808086)
\curveto(574.01063676,176.93058086)(573.97806153,177.01131002)(573.97806153,177.11026834)
\curveto(573.97806153,177.20401833)(574.01063676,177.28214333)(574.0757872,177.34464332)
\curveto(574.14093765,177.40714332)(574.24868646,177.43839332)(574.39903364,177.43839332)
\lineto(575.56422428,177.43839332)
\lineto(575.56422428,179.37589316)
\curveto(575.56422428,179.53214315)(575.59429371,179.64412231)(575.65443258,179.71183063)
\curveto(575.71457146,179.77953896)(575.78974505,179.81339313)(575.87995335,179.81339313)
\curveto(575.97517323,179.81339313)(576.05285261,179.77953896)(576.11299148,179.71183063)
\curveto(576.17313035,179.64412231)(576.20319979,179.53214315)(576.20319979,179.37589316)
\closepath
}
}
{
\newrgbcolor{curcolor}{0 0 0}
\pscustom[linewidth=1.89141912,linecolor=curcolor]
{
\newpath
\moveto(524.70407815,161.64601999)
\lineto(668.01017475,161.64601999)
\lineto(668.01017475,120.38236998)
\lineto(524.70407815,120.38236998)
\closepath
}
}
{
\newrgbcolor{curcolor}{0 0 0}
\pscustom[linestyle=none,fillstyle=solid,fillcolor=curcolor]
{
\newpath
\moveto(529.2493231,158.09468018)
\lineto(529.2493231,153.74311802)
\curveto(530.01108214,154.77436794)(530.93070572,155.2899929)(532.00819384,155.2899929)
\curveto(532.9303232,155.2899929)(533.71964589,154.94103459)(534.37616191,154.24311798)
\curveto(535.03267793,153.5504097)(535.36093593,152.69884727)(535.36093593,151.68843069)
\curveto(535.36093593,150.66759743)(535.02766635,149.80301417)(534.36112719,149.09468089)
\curveto(533.6995996,148.38634762)(532.91528848,148.03218098)(532.00819384,148.03218098)
\curveto(530.90564785,148.03218098)(529.98602427,148.54780594)(529.2493231,149.57905585)
\lineto(529.2493231,148.28999346)
\lineto(527.76840338,148.28999346)
\curveto(527.6180562,148.28999346)(527.51030739,148.32124345)(527.44515694,148.38374345)
\curveto(527.3800065,148.44624344)(527.34743128,148.52436844)(527.34743128,148.61811843)
\curveto(527.34743128,148.71707676)(527.3800065,148.79520175)(527.44515694,148.85249341)
\curveto(527.51030739,148.91499341)(527.6180562,148.9462434)(527.76840338,148.9462434)
\lineto(528.61786494,148.9462434)
\lineto(528.61786494,157.43061773)
\lineto(527.76840338,157.43061773)
\curveto(527.6180562,157.43061773)(527.51030739,157.46186773)(527.44515694,157.52436772)
\curveto(527.3800065,157.58686772)(527.34743128,157.66759688)(527.34743128,157.7665552)
\curveto(527.34743128,157.86030519)(527.3800065,157.93843019)(527.44515694,158.00093018)
\curveto(527.51030739,158.06343018)(527.6180562,158.09468018)(527.76840338,158.09468018)
\closepath
\moveto(534.72947778,151.65718069)
\curveto(534.72947778,152.48530562)(534.45634707,153.1858264)(533.91008565,153.75874302)
\curveto(533.36382424,154.33686797)(532.72484872,154.62593045)(531.99315912,154.62593045)
\curveto(531.26146951,154.62593045)(530.622494,154.33686797)(530.07623258,153.75874302)
\curveto(529.52997116,153.1858264)(529.25684045,152.48530562)(529.25684045,151.65718069)
\curveto(529.25684045,150.82905575)(529.52997116,150.12593081)(530.07623258,149.54780586)
\curveto(530.622494,148.97488924)(531.26146951,148.68843092)(531.99315912,148.68843092)
\curveto(532.72484872,148.68843092)(533.36382424,148.97488924)(533.91008565,149.54780586)
\curveto(534.45634707,150.12593081)(534.72947778,150.82905575)(534.72947778,151.65718069)
\closepath
}
}
{
\newrgbcolor{curcolor}{0 0 0}
\pscustom[linestyle=none,fillstyle=solid,fillcolor=curcolor]
{
\newpath
\moveto(545.0658468,143.90718131)
\lineto(536.66895683,143.90718131)
\curveto(536.51860965,143.90718131)(536.41086084,143.9384313)(536.3457104,144.0009313)
\curveto(536.28055995,144.05822296)(536.24798473,144.13634795)(536.24798473,144.23530628)
\curveto(536.24798473,144.3342646)(536.28055995,144.41499377)(536.3457104,144.47749376)
\curveto(536.41086084,144.53478542)(536.51860965,144.56343125)(536.66895683,144.56343125)
\lineto(545.0658468,144.56343125)
\curveto(545.22120555,144.56343125)(545.32895436,144.53478542)(545.38909323,144.47749376)
\curveto(545.45424368,144.41499377)(545.4868189,144.3342646)(545.4868189,144.23530628)
\curveto(545.4868189,144.13634795)(545.45424368,144.05822296)(545.38909323,144.0009313)
\curveto(545.32895436,143.9384313)(545.22120555,143.90718131)(545.0658468,143.90718131)
\closepath
}
}
{
\newrgbcolor{curcolor}{0 0 0}
\pscustom[linestyle=none,fillstyle=solid,fillcolor=curcolor]
{
\newpath
\moveto(552.74107074,154.39155547)
\lineto(552.74107074,154.61030545)
\curveto(552.74107074,154.77176377)(552.77114018,154.8863471)(552.83127905,154.95405543)
\curveto(552.89141792,155.02176375)(552.96659151,155.05561792)(553.05679982,155.05561792)
\curveto(553.1520197,155.05561792)(553.22969907,155.02176375)(553.28983794,154.95405543)
\curveto(553.34997682,154.8863471)(553.38004625,154.77176377)(553.38004625,154.61030545)
\lineto(553.38004625,153.12593057)
\curveto(553.37503468,152.96447225)(553.34245946,152.84988893)(553.28232058,152.7821806)
\curveto(553.22719329,152.71447227)(553.1520197,152.68061811)(553.05679982,152.68061811)
\curveto(552.97160308,152.68061811)(552.89893528,152.70926394)(552.83879641,152.7665556)
\curveto(552.78366911,152.82905559)(552.75109388,152.93061809)(552.74107074,153.07124308)
\curveto(552.7110013,153.44103471)(552.47545739,153.79259718)(552.034439,154.12593049)
\curveto(551.59843218,154.4592638)(551.00957239,154.62593045)(550.26785964,154.62593045)
\curveto(549.33069556,154.62593045)(548.61905224,154.32124298)(548.13292969,153.71186802)
\curveto(547.64680715,153.10249307)(547.40374587,152.40457646)(547.40374587,151.61811819)
\curveto(547.40374587,150.76915993)(547.67186501,150.06863915)(548.20810328,149.51655586)
\curveto(548.74434156,148.96447257)(549.43844437,148.68843092)(550.29041172,148.68843092)
\curveto(550.78154584,148.68843092)(551.28019731,148.78218092)(551.78636615,148.9696809)
\curveto(552.29754656,149.15718089)(552.75861124,149.4592642)(553.1695602,149.87593083)
\curveto(553.27480323,149.98009749)(553.36751732,150.03218082)(553.44770248,150.03218082)
\curveto(553.53289922,150.03218082)(553.60306123,150.00093082)(553.65818853,149.93843082)
\curveto(553.7183274,149.88113916)(553.74839684,149.8082225)(553.74839684,149.71968084)
\curveto(553.74839684,149.49572253)(553.49531242,149.21186838)(552.98914359,148.86811841)
\curveto(552.17225724,148.31082679)(551.26265681,148.03218098)(550.26034228,148.03218098)
\curveto(549.24299303,148.03218098)(548.4060604,148.36811845)(547.74954439,149.0399934)
\curveto(547.09803994,149.71707668)(546.77228772,150.57384744)(546.77228772,151.61030569)
\curveto(546.77228772,152.66759727)(547.1055573,153.54520137)(547.77209646,154.24311798)
\curveto(548.4436472,154.94103459)(549.28809719,155.2899929)(550.30544643,155.2899929)
\curveto(551.27267995,155.2899929)(552.08455472,154.99051376)(552.74107074,154.39155547)
\closepath
}
}
{
\newrgbcolor{curcolor}{0 0 0}
\pscustom[linestyle=none,fillstyle=solid,fillcolor=curcolor]
{
\newpath
\moveto(562.87447195,151.65718069)
\curveto(562.87447195,150.65718077)(562.52867343,149.80301417)(561.83707641,149.09468089)
\curveto(561.15049096,148.38634762)(560.32107568,148.03218098)(559.34883059,148.03218098)
\curveto(558.36656235,148.03218098)(557.53213551,148.38634762)(556.84555006,149.09468089)
\curveto(556.1589646,149.8082225)(555.81567188,150.6623891)(555.81567188,151.65718069)
\curveto(555.81567188,152.65718061)(556.1589646,153.51134721)(556.84555006,154.21968048)
\curveto(557.53213551,154.93322209)(558.36656235,155.2899929)(559.34883059,155.2899929)
\curveto(560.32107568,155.2899929)(561.15049096,154.93582626)(561.83707641,154.22749298)
\curveto(562.52867343,153.51915971)(562.87447195,152.66238894)(562.87447195,151.65718069)
\closepath
\moveto(562.23549643,151.65718069)
\curveto(562.23549643,152.48009729)(561.95234258,153.18061807)(561.38603487,153.75874302)
\curveto(560.82473874,154.33686797)(560.14316486,154.62593045)(559.34131323,154.62593045)
\curveto(558.53946161,154.62593045)(557.85538194,154.33426381)(557.28907424,153.75093052)
\curveto(556.7277781,153.17280557)(556.44713003,152.47488896)(556.44713003,151.65718069)
\curveto(556.44713003,150.84468075)(556.7277781,150.14676414)(557.28907424,149.56343085)
\curveto(557.85538194,148.98009757)(558.53946161,148.68843092)(559.34131323,148.68843092)
\curveto(560.14316486,148.68843092)(560.82473874,148.9774934)(561.38603487,149.55561836)
\curveto(561.95234258,150.13895164)(562.23549643,150.83947242)(562.23549643,151.65718069)
\closepath
}
}
{
\newrgbcolor{curcolor}{0 0 0}
\pscustom[linestyle=none,fillstyle=solid,fillcolor=curcolor]
{
\newpath
\moveto(565.70851583,155.05561792)
\lineto(565.70851583,154.39155547)
\curveto(566.2447541,154.99051376)(566.78349816,155.2899929)(567.32474801,155.2899929)
\curveto(567.65050023,155.2899929)(567.93615987,155.19884707)(568.18172693,155.01655542)
\curveto(568.42729399,154.8394721)(568.63276847,154.56863879)(568.79815037,154.20405549)
\curveto(569.07879843,154.56863879)(569.36195229,154.8394721)(569.64761193,155.01655542)
\curveto(569.93828314,155.19884707)(570.22895436,155.2899929)(570.51962557,155.2899929)
\curveto(570.97567868,155.2899929)(571.3390177,155.13634708)(571.60964262,154.82905544)
\curveto(571.96546428,154.43322213)(572.14337511,154.0009305)(572.14337511,153.53218054)
\lineto(572.14337511,148.9462434)
\lineto(572.67710759,148.9462434)
\curveto(572.82745477,148.9462434)(572.93520358,148.91499341)(573.00035403,148.85249341)
\curveto(573.06550447,148.79520175)(573.09807969,148.71707676)(573.09807969,148.61811843)
\curveto(573.09807969,148.52436844)(573.06550447,148.44624344)(573.00035403,148.38374345)
\curveto(572.93520358,148.32124345)(572.82745477,148.28999346)(572.67710759,148.28999346)
\lineto(571.51191695,148.28999346)
\lineto(571.51191695,153.46968054)
\curveto(571.51191695,153.80301385)(571.41419129,154.0790555)(571.21873995,154.29780548)
\curveto(571.02328862,154.51655546)(570.79776785,154.62593045)(570.54217765,154.62593045)
\curveto(570.3116453,154.62593045)(570.06858403,154.53478463)(569.81299383,154.35249297)
\curveto(569.55740362,154.17540965)(569.26673241,153.82384718)(568.94098019,153.29780556)
\lineto(568.94098019,148.9462434)
\lineto(569.46719531,148.9462434)
\curveto(569.61754249,148.9462434)(569.7252913,148.91499341)(569.79044175,148.85249341)
\curveto(569.85559219,148.79520175)(569.88816742,148.71707676)(569.88816742,148.61811843)
\curveto(569.88816742,148.52436844)(569.85559219,148.44624344)(569.79044175,148.38374345)
\curveto(569.7252913,148.32124345)(569.61754249,148.28999346)(569.46719531,148.28999346)
\lineto(568.30200467,148.28999346)
\lineto(568.30200467,153.42280555)
\curveto(568.30200467,153.77176385)(568.20177322,154.05822216)(568.00131032,154.28218048)
\curveto(567.80585898,154.51134713)(567.58534979,154.62593045)(567.33978273,154.62593045)
\curveto(567.11426196,154.62593045)(566.89124697,154.55040962)(566.67073778,154.39936797)
\curveto(566.36503185,154.18582632)(566.0442912,153.81863885)(565.70851583,153.29780556)
\lineto(565.70851583,148.9462434)
\lineto(566.24224832,148.9462434)
\curveto(566.3925955,148.9462434)(566.50034431,148.91499341)(566.56549475,148.85249341)
\curveto(566.6306452,148.79520175)(566.66322042,148.71707676)(566.66322042,148.61811843)
\curveto(566.66322042,148.52436844)(566.6306452,148.44624344)(566.56549475,148.38374345)
\curveto(566.50034431,148.32124345)(566.3925955,148.28999346)(566.24224832,148.28999346)
\lineto(564.54332519,148.28999346)
\curveto(564.39297801,148.28999346)(564.2852292,148.32124345)(564.22007876,148.38374345)
\curveto(564.15492831,148.44624344)(564.12235309,148.52436844)(564.12235309,148.61811843)
\curveto(564.12235309,148.71707676)(564.15492831,148.79520175)(564.22007876,148.85249341)
\curveto(564.2852292,148.91499341)(564.39297801,148.9462434)(564.54332519,148.9462434)
\lineto(565.07705768,148.9462434)
\lineto(565.07705768,154.39155547)
\lineto(564.54332519,154.39155547)
\curveto(564.39297801,154.39155547)(564.2852292,154.42280547)(564.22007876,154.48530546)
\curveto(564.15492831,154.54780546)(564.12235309,154.62853462)(564.12235309,154.72749294)
\curveto(564.12235309,154.82124294)(564.15492831,154.89936793)(564.22007876,154.96186792)
\curveto(564.2852292,155.02436792)(564.39297801,155.05561792)(564.54332519,155.05561792)
\closepath
}
}
{
\newrgbcolor{curcolor}{0 0 0}
\pscustom[linestyle=none,fillstyle=solid,fillcolor=curcolor]
{
\newpath
\moveto(575.44349525,155.05561792)
\lineto(575.44349525,153.86030551)
\curveto(575.81936319,154.33426381)(576.22279479,154.69103461)(576.65379004,154.93061793)
\curveto(577.08478529,155.17020124)(577.59345991,155.2899929)(578.17981391,155.2899929)
\curveto(578.80124892,155.2899929)(579.37507398,155.13895124)(579.90128911,154.83686793)
\curveto(580.42750424,154.53478463)(580.83344162,154.11290966)(581.11910126,153.57124304)
\curveto(581.40977248,153.03478474)(581.55510808,152.46968062)(581.55510808,151.87593067)
\curveto(581.55510808,150.93322241)(581.22935586,150.12593081)(580.57785142,149.45405586)
\curveto(579.93135855,148.78738925)(579.1345185,148.45405594)(578.18733127,148.45405594)
\curveto(577.05972742,148.45405594)(576.14511542,148.93061841)(575.44349525,149.88374333)
\lineto(575.44349525,145.93061864)
\lineto(576.97703647,145.93061864)
\curveto(577.12738365,145.93061864)(577.23513247,145.90197281)(577.30028291,145.84468115)
\curveto(577.36543335,145.78218116)(577.39800858,145.701452)(577.39800858,145.60249367)
\curveto(577.39800858,145.50874368)(577.36543335,145.43061868)(577.30028291,145.36811869)
\curveto(577.23513247,145.30561869)(577.12738365,145.2743687)(576.97703647,145.2743687)
\lineto(573.96257553,145.2743687)
\curveto(573.81222835,145.2743687)(573.70447954,145.30561869)(573.63932909,145.36811869)
\curveto(573.57417865,145.42541035)(573.54160343,145.50353534)(573.54160343,145.60249367)
\curveto(573.54160343,145.701452)(573.57417865,145.78218116)(573.63932909,145.84468115)
\curveto(573.70447954,145.90197281)(573.81222835,145.93061864)(573.96257553,145.93061864)
\lineto(574.81203709,145.93061864)
\lineto(574.81203709,154.39155547)
\lineto(573.96257553,154.39155547)
\curveto(573.81222835,154.39155547)(573.70447954,154.42280547)(573.63932909,154.48530546)
\curveto(573.57417865,154.54780546)(573.54160343,154.62853462)(573.54160343,154.72749294)
\curveto(573.54160343,154.82124294)(573.57417865,154.89936793)(573.63932909,154.96186792)
\curveto(573.70447954,155.02436792)(573.81222835,155.05561792)(573.96257553,155.05561792)
\closepath
\moveto(580.91613257,151.87593067)
\curveto(580.91613257,152.63113894)(580.65051922,153.27697223)(580.11929252,153.81343052)
\curveto(579.59307739,154.35509714)(578.94909031,154.62593045)(578.18733127,154.62593045)
\curveto(577.42056065,154.62593045)(576.771562,154.35509714)(576.2403353,153.81343052)
\curveto(575.7091086,153.27176389)(575.44349525,152.62593061)(575.44349525,151.87593067)
\curveto(575.44349525,151.1207224)(575.7091086,150.47228495)(576.2403353,149.93061833)
\curveto(576.771562,149.3889517)(577.42056065,149.11811839)(578.18733127,149.11811839)
\curveto(578.94407874,149.11811839)(579.58806582,149.38634754)(580.11929252,149.92280583)
\curveto(580.65051922,150.46447245)(580.91613257,151.11551406)(580.91613257,151.87593067)
\closepath
}
}
{
\newrgbcolor{curcolor}{0 0 0}
\pscustom[linestyle=none,fillstyle=solid,fillcolor=curcolor]
{
\newpath
\moveto(599.74711589,153.71968052)
\lineto(592.86121508,153.71968052)
\curveto(592.7108679,153.71968052)(592.60311909,153.75093052)(592.53796864,153.81343052)
\curveto(592.4728182,153.87593051)(592.44024297,153.95665967)(592.44024297,154.055618)
\curveto(592.44024297,154.14936799)(592.4728182,154.22749298)(592.53796864,154.28999298)
\curveto(592.60311909,154.35249297)(592.7108679,154.38374297)(592.86121508,154.38374297)
\lineto(599.74711589,154.38374297)
\curveto(599.89746307,154.38374297)(600.00521188,154.35249297)(600.07036232,154.28999298)
\curveto(600.13551277,154.22749298)(600.16808799,154.14676382)(600.16808799,154.0478055)
\curveto(600.16808799,153.95405551)(600.13551277,153.87593051)(600.07036232,153.81343052)
\curveto(600.00521188,153.75093052)(599.89746307,153.71968052)(599.74711589,153.71968052)
\closepath
\moveto(599.74711589,151.36811821)
\lineto(592.86121508,151.36811821)
\curveto(592.7108679,151.36811821)(592.60311909,151.39676404)(592.53796864,151.4540557)
\curveto(592.4728182,151.5165557)(592.44024297,151.59728486)(592.44024297,151.69624318)
\curveto(592.44024297,151.79520151)(592.4728182,151.8733265)(592.53796864,151.93061817)
\curveto(592.60311909,151.99311816)(592.7108679,152.02436816)(592.86121508,152.02436816)
\lineto(599.74711589,152.02436816)
\curveto(599.89746307,152.02436816)(600.00521188,151.99311816)(600.07036232,151.93061817)
\curveto(600.13551277,151.8733265)(600.16808799,151.79520151)(600.16808799,151.69624318)
\curveto(600.16808799,151.59728486)(600.13551277,151.5165557)(600.07036232,151.4540557)
\curveto(600.00521188,151.39676404)(599.89746307,151.36811821)(599.74711589,151.36811821)
\closepath
}
}
{
\newrgbcolor{curcolor}{0 0 0}
\pscustom[linestyle=none,fillstyle=solid,fillcolor=curcolor]
{
\newpath
\moveto(613.25581613,152.7040556)
\lineto(613.25581613,148.9462434)
\lineto(615.42081551,148.9462434)
\curveto(615.57116269,148.9462434)(615.6789115,148.91499341)(615.74406195,148.85249341)
\curveto(615.80921239,148.79520175)(615.84178761,148.71707676)(615.84178761,148.61811843)
\curveto(615.84178761,148.52436844)(615.80921239,148.44624344)(615.74406195,148.38374345)
\curveto(615.6789115,148.32124345)(615.57116269,148.28999346)(615.42081551,148.28999346)
\lineto(611.77489641,148.28999346)
\curveto(611.62454923,148.28999346)(611.51680042,148.32124345)(611.45164998,148.38374345)
\curveto(611.38649953,148.44624344)(611.35392431,148.52436844)(611.35392431,148.61811843)
\curveto(611.35392431,148.71707676)(611.38649953,148.79520175)(611.45164998,148.85249341)
\curveto(611.51680042,148.91499341)(611.62454923,148.9462434)(611.77489641,148.9462434)
\lineto(612.62435798,148.9462434)
\lineto(612.62435798,156.77436778)
\lineto(611.77489641,156.77436778)
\curveto(611.62454923,156.77436778)(611.51680042,156.80301361)(611.45164998,156.86030527)
\curveto(611.38649953,156.92280527)(611.35392431,157.00353443)(611.35392431,157.10249275)
\curveto(611.35392431,157.20145108)(611.39151111,157.28478441)(611.4666847,157.35249273)
\curveto(611.52181199,157.40457606)(611.62454923,157.43061773)(611.77489641,157.43061773)
\lineto(618.72845346,157.43061773)
\lineto(618.72845346,155.17280541)
\curveto(618.72845346,155.01655542)(618.69838402,154.90457626)(618.63824515,154.83686793)
\curveto(618.57810628,154.76915961)(618.50293269,154.73530544)(618.41272438,154.73530544)
\curveto(618.3175045,154.73530544)(618.23982512,154.76915961)(618.17968625,154.83686793)
\curveto(618.11954738,154.90457626)(618.08947794,155.01655542)(618.08947794,155.17280541)
\lineto(618.08947794,156.77436778)
\lineto(613.25581613,156.77436778)
\lineto(613.25581613,153.36030555)
\lineto(615.51854118,153.36030555)
\lineto(615.51854118,154.10249299)
\curveto(615.51854118,154.25874298)(615.54861061,154.37072214)(615.60874949,154.43843047)
\curveto(615.66888836,154.50613879)(615.74656773,154.53999296)(615.84178761,154.53999296)
\curveto(615.93199592,154.53999296)(616.00716951,154.50613879)(616.06730838,154.43843047)
\curveto(616.12744725,154.37072214)(616.15751669,154.25874298)(616.15751669,154.10249299)
\lineto(616.15751669,151.96186816)
\curveto(616.15751669,151.80561818)(616.12744725,151.69363902)(616.06730838,151.62593069)
\curveto(616.00716951,151.55822236)(615.93199592,151.5243682)(615.84178761,151.5243682)
\curveto(615.74656773,151.5243682)(615.66888836,151.55822236)(615.60874949,151.62593069)
\curveto(615.54861061,151.69363902)(615.51854118,151.80561818)(615.51854118,151.96186816)
\lineto(615.51854118,152.7040556)
\closepath
}
}
{
\newrgbcolor{curcolor}{0 0 0}
\pscustom[linestyle=none,fillstyle=solid,fillcolor=curcolor]
{
\newpath
\moveto(529.2493231,138.09468177)
\lineto(529.2493231,133.74311962)
\curveto(530.01108214,134.77436953)(530.93070572,135.28999449)(532.00819384,135.28999449)
\curveto(532.9303232,135.28999449)(533.71964589,134.94103619)(534.37616191,134.24311958)
\curveto(535.03267793,133.5504113)(535.36093593,132.69884887)(535.36093593,131.68843228)
\curveto(535.36093593,130.66759903)(535.02766635,129.80301576)(534.36112719,129.09468249)
\curveto(533.6995996,128.38634921)(532.91528848,128.03218257)(532.00819384,128.03218257)
\curveto(530.90564785,128.03218257)(529.98602427,128.54780753)(529.2493231,129.57905745)
\lineto(529.2493231,128.28999505)
\lineto(527.76840338,128.28999505)
\curveto(527.6180562,128.28999505)(527.51030739,128.32124505)(527.44515694,128.38374504)
\curveto(527.3800065,128.44624504)(527.34743128,128.52437003)(527.34743128,128.61812002)
\curveto(527.34743128,128.71707835)(527.3800065,128.79520334)(527.44515694,128.852495)
\curveto(527.51030739,128.914995)(527.6180562,128.946245)(527.76840338,128.946245)
\lineto(528.61786494,128.946245)
\lineto(528.61786494,137.43061932)
\lineto(527.76840338,137.43061932)
\curveto(527.6180562,137.43061932)(527.51030739,137.46186932)(527.44515694,137.52436931)
\curveto(527.3800065,137.58686931)(527.34743128,137.66759847)(527.34743128,137.76655679)
\curveto(527.34743128,137.86030679)(527.3800065,137.93843178)(527.44515694,138.00093178)
\curveto(527.51030739,138.06343177)(527.6180562,138.09468177)(527.76840338,138.09468177)
\closepath
\moveto(534.72947778,131.65718228)
\curveto(534.72947778,132.48530722)(534.45634707,133.18582799)(533.91008565,133.75874461)
\curveto(533.36382424,134.33686957)(532.72484872,134.62593204)(531.99315912,134.62593204)
\curveto(531.26146951,134.62593204)(530.622494,134.33686957)(530.07623258,133.75874461)
\curveto(529.52997116,133.18582799)(529.25684045,132.48530722)(529.25684045,131.65718228)
\curveto(529.25684045,130.82905735)(529.52997116,130.1259324)(530.07623258,129.54780745)
\curveto(530.622494,128.97489083)(531.26146951,128.68843252)(531.99315912,128.68843252)
\curveto(532.72484872,128.68843252)(533.36382424,128.97489083)(533.91008565,129.54780745)
\curveto(534.45634707,130.1259324)(534.72947778,130.82905735)(534.72947778,131.65718228)
\closepath
}
}
{
\newrgbcolor{curcolor}{0 0 0}
\pscustom[linestyle=none,fillstyle=solid,fillcolor=curcolor]
{
\newpath
\moveto(545.0658468,123.9071829)
\lineto(536.66895683,123.9071829)
\curveto(536.51860965,123.9071829)(536.41086084,123.9384329)(536.3457104,124.00093289)
\curveto(536.28055995,124.05822455)(536.24798473,124.13634955)(536.24798473,124.23530787)
\curveto(536.24798473,124.3342662)(536.28055995,124.41499536)(536.3457104,124.47749535)
\curveto(536.41086084,124.53478702)(536.51860965,124.56343285)(536.66895683,124.56343285)
\lineto(545.0658468,124.56343285)
\curveto(545.22120555,124.56343285)(545.32895436,124.53478702)(545.38909323,124.47749535)
\curveto(545.45424368,124.41499536)(545.4868189,124.3342662)(545.4868189,124.23530787)
\curveto(545.4868189,124.13634955)(545.45424368,124.05822455)(545.38909323,124.00093289)
\curveto(545.32895436,123.9384329)(545.22120555,123.9071829)(545.0658468,123.9071829)
\closepath
}
}
{
\newrgbcolor{curcolor}{0 0 0}
\pscustom[linestyle=none,fillstyle=solid,fillcolor=curcolor]
{
\newpath
\moveto(553.19962964,138.09468177)
\lineto(553.19962964,128.946245)
\lineto(554.04157384,128.946245)
\curveto(554.19693259,128.946245)(554.30718719,128.914995)(554.37233763,128.852495)
\curveto(554.43748808,128.79520334)(554.4700633,128.71707835)(554.4700633,128.61812002)
\curveto(554.4700633,128.52437003)(554.43748808,128.44624504)(554.37233763,128.38374504)
\curveto(554.30718719,128.32124505)(554.19693259,128.28999505)(554.04157384,128.28999505)
\lineto(552.56065412,128.28999505)
\lineto(552.56065412,129.59468245)
\curveto(551.82896452,128.55301586)(550.89931779,128.03218257)(549.77171395,128.03218257)
\curveto(549.20039467,128.03218257)(548.65162746,128.18843256)(548.12541233,128.50093253)
\curveto(547.60420878,128.81864084)(547.19075404,129.26916164)(546.8850481,129.85249493)
\curveto(546.58435375,130.43582821)(546.43400657,131.03739066)(546.43400657,131.65718228)
\curveto(546.43400657,132.28218223)(546.58435375,132.88374468)(546.8850481,133.46186964)
\curveto(547.19075404,134.04520292)(547.60420878,134.49572372)(548.12541233,134.81343203)
\curveto(548.65162746,135.13114034)(549.20290045,135.28999449)(549.77923131,135.28999449)
\curveto(550.88177729,135.28999449)(551.80891823,134.7691612)(552.56065412,133.72749462)
\lineto(552.56065412,137.43061932)
\lineto(551.71870992,137.43061932)
\curveto(551.56335117,137.43061932)(551.45309657,137.46186932)(551.38794613,137.52436931)
\curveto(551.32279568,137.58686931)(551.29022046,137.66759847)(551.29022046,137.76655679)
\curveto(551.29022046,137.86030679)(551.32279568,137.93843178)(551.38794613,138.00093178)
\curveto(551.45309657,138.06343177)(551.56335117,138.09468177)(551.71870992,138.09468177)
\closepath
\moveto(552.56065412,131.65718228)
\curveto(552.56065412,132.49051555)(552.2900292,133.19364049)(551.74877936,133.76655711)
\curveto(551.20752951,134.33947373)(550.56354243,134.62593204)(549.8168181,134.62593204)
\curveto(549.06508221,134.62593204)(548.41858933,134.33947373)(547.87733949,133.76655711)
\curveto(547.33608964,133.19364049)(547.06546472,132.49051555)(547.06546472,131.65718228)
\curveto(547.06546472,130.82905735)(547.33608964,130.1259324)(547.87733949,129.54780745)
\curveto(548.41858933,128.97489083)(549.06508221,128.68843252)(549.8168181,128.68843252)
\curveto(550.56354243,128.68843252)(551.20752951,128.97489083)(551.74877936,129.54780745)
\curveto(552.2900292,130.1259324)(552.56065412,130.82905735)(552.56065412,131.65718228)
\closepath
}
}
{
\newrgbcolor{curcolor}{0 0 0}
\pscustom[linestyle=none,fillstyle=solid,fillcolor=curcolor]
{
\newpath
\moveto(561.1605141,128.28999505)
\lineto(561.1605141,129.23530747)
\curveto(560.24339631,128.43322421)(559.26363386,128.03218257)(558.22122675,128.03218257)
\curveto(557.46447928,128.03218257)(556.87311371,128.23009922)(556.44713003,128.62593252)
\curveto(556.02114636,129.02697416)(555.80815452,129.51655745)(555.80815452,130.09468241)
\curveto(555.80815452,130.73009902)(556.08880259,131.28478648)(556.65009872,131.75874477)
\curveto(557.21139486,132.23270307)(558.03078699,132.46968222)(559.1082751,132.46968222)
\curveto(559.39894632,132.46968222)(559.71467539,132.44884889)(560.05546233,132.40718222)
\curveto(560.39624927,132.37072389)(560.76459986,132.31082806)(561.1605141,132.22749474)
\lineto(561.1605141,133.28999465)
\curveto(561.1605141,133.64936962)(561.00014378,133.9618696)(560.67940313,134.22749458)
\curveto(560.35866248,134.49311956)(559.87755151,134.62593204)(559.23607021,134.62593204)
\curveto(558.74493609,134.62593204)(558.05584485,134.47749456)(557.16879649,134.18061958)
\curveto(557.00842617,134.12853625)(556.90568893,134.10249459)(556.86058477,134.10249459)
\curveto(556.78039961,134.10249459)(556.71023759,134.13374458)(556.65009872,134.19624458)
\curveto(556.59497142,134.25874457)(556.56740777,134.33686957)(556.56740777,134.43061956)
\curveto(556.56740777,134.51916122)(556.59246564,134.58947371)(556.64258136,134.64155704)
\curveto(556.71274338,134.71968204)(556.99589724,134.82645286)(557.49204293,134.96186952)
\curveto(558.27384826,135.1806195)(558.86521383,135.28999449)(559.26613964,135.28999449)
\curveto(560.06297969,135.28999449)(560.6844147,135.08426534)(561.13044467,134.67280704)
\curveto(561.57647463,134.26655707)(561.79948961,133.80561961)(561.79948961,133.28999465)
\lineto(561.79948961,128.946245)
\lineto(562.64143382,128.946245)
\curveto(562.79679257,128.946245)(562.90704717,128.914995)(562.97219761,128.852495)
\curveto(563.03734806,128.79520334)(563.06992328,128.71707835)(563.06992328,128.61812002)
\curveto(563.06992328,128.52437003)(563.03734806,128.44624504)(562.97219761,128.38374504)
\curveto(562.90704717,128.32124505)(562.79679257,128.28999505)(562.64143382,128.28999505)
\closepath
\moveto(561.1605141,131.55561979)
\curveto(560.86483132,131.64416145)(560.55160803,131.70926561)(560.22084423,131.75093227)
\curveto(559.89008044,131.79259894)(559.54177614,131.81343227)(559.17593134,131.81343227)
\curveto(558.25881354,131.81343227)(557.54215865,131.60770312)(557.02596667,131.19624482)
\curveto(556.63506401,130.88895318)(556.43961267,130.52176571)(556.43961267,130.09468241)
\curveto(556.43961267,129.6988491)(556.58745407,129.3655158)(556.88313685,129.09468249)
\curveto(557.18383121,128.82384917)(557.61983803,128.68843252)(558.19115731,128.68843252)
\curveto(558.73741873,128.68843252)(559.24358757,128.80041168)(559.70966382,129.02436999)
\curveto(560.18075165,129.25353664)(560.66436841,129.61551578)(561.1605141,130.1103074)
\closepath
}
}
{
\newrgbcolor{curcolor}{0 0 0}
\pscustom[linestyle=none,fillstyle=solid,fillcolor=curcolor]
{
\newpath
\moveto(567.28716121,135.05561951)
\lineto(570.7150769,135.05561951)
\curveto(570.86542408,135.05561951)(570.97317289,135.02436951)(571.03832334,134.96186952)
\curveto(571.10347378,134.89936952)(571.136049,134.81864036)(571.136049,134.71968204)
\curveto(571.136049,134.62593204)(571.10347378,134.54780705)(571.03832334,134.48530706)
\curveto(570.97317289,134.42280706)(570.86542408,134.39155706)(570.7150769,134.39155706)
\lineto(567.28716121,134.39155706)
\lineto(567.28716121,130.02436991)
\curveto(567.28716121,129.64416161)(567.43249682,129.3264533)(567.72316803,129.07124499)
\curveto(568.01885082,128.81603667)(568.44984607,128.68843252)(569.01615378,128.68843252)
\curveto(569.44213745,128.68843252)(569.90320213,128.75353668)(570.39934783,128.883745)
\curveto(570.89549352,129.01916166)(571.28138461,129.17020331)(571.55702111,129.33686997)
\curveto(571.65725256,129.40457829)(571.73994351,129.43843246)(571.80509395,129.43843246)
\curveto(571.88527911,129.43843246)(571.95544113,129.40457829)(572.01558,129.33686997)
\curveto(572.07571887,129.27436997)(572.10578831,129.19884914)(572.10578831,129.11030748)
\curveto(572.10578831,129.03218249)(572.07321309,128.95926583)(572.00806264,128.8915575)
\curveto(571.84769232,128.71968252)(571.45678965,128.53218253)(570.83535465,128.32905755)
\curveto(570.21893121,128.1311409)(569.62756564,128.03218257)(569.06125793,128.03218257)
\curveto(568.32455675,128.03218257)(567.73820275,128.21187006)(567.30219593,128.57124503)
\curveto(566.86618911,128.93062)(566.6481857,129.41499496)(566.6481857,130.02436991)
\lineto(566.6481857,134.39155706)
\lineto(565.48299506,134.39155706)
\curveto(565.33264788,134.39155706)(565.22489907,134.42280706)(565.15974863,134.48530706)
\curveto(565.09459818,134.54780705)(565.06202296,134.62853621)(565.06202296,134.72749454)
\curveto(565.06202296,134.82124453)(565.09459818,134.89936952)(565.15974863,134.96186952)
\curveto(565.22489907,135.02436951)(565.33264788,135.05561951)(565.48299506,135.05561951)
\lineto(566.6481857,135.05561951)
\lineto(566.6481857,136.99311936)
\curveto(566.6481857,137.14936934)(566.67825514,137.2613485)(566.73839401,137.32905683)
\curveto(566.79853288,137.39676516)(566.87370647,137.43061932)(566.96391478,137.43061932)
\curveto(567.05913466,137.43061932)(567.13681403,137.39676516)(567.19695291,137.32905683)
\curveto(567.25709178,137.2613485)(567.28716121,137.14936934)(567.28716121,136.99311936)
\closepath
}
}
{
\newrgbcolor{curcolor}{0 0 0}
\pscustom[linestyle=none,fillstyle=solid,fillcolor=curcolor]
{
\newpath
\moveto(579.63818155,128.28999505)
\lineto(579.63818155,129.23530747)
\curveto(578.72106375,128.43322421)(577.7413013,128.03218257)(576.69889419,128.03218257)
\curveto(575.94214672,128.03218257)(575.35078115,128.23009922)(574.92479748,128.62593252)
\curveto(574.4988138,129.02697416)(574.28582196,129.51655745)(574.28582196,130.09468241)
\curveto(574.28582196,130.73009902)(574.56647003,131.28478648)(575.12776617,131.75874477)
\curveto(575.6890623,132.23270307)(576.50845443,132.46968222)(577.58594255,132.46968222)
\curveto(577.87661376,132.46968222)(578.19234284,132.44884889)(578.53312978,132.40718222)
\curveto(578.87391672,132.37072389)(579.24226731,132.31082806)(579.63818155,132.22749474)
\lineto(579.63818155,133.28999465)
\curveto(579.63818155,133.64936962)(579.47781122,133.9618696)(579.15707057,134.22749458)
\curveto(578.83632993,134.49311956)(578.35521895,134.62593204)(577.71373765,134.62593204)
\curveto(577.22260353,134.62593204)(576.5335123,134.47749456)(575.64646394,134.18061958)
\curveto(575.48609361,134.12853625)(575.38335637,134.10249459)(575.33825222,134.10249459)
\curveto(575.25806706,134.10249459)(575.18790504,134.13374458)(575.12776617,134.19624458)
\curveto(575.07263887,134.25874457)(575.04507522,134.33686957)(575.04507522,134.43061956)
\curveto(575.04507522,134.51916122)(575.07013308,134.58947371)(575.12024881,134.64155704)
\curveto(575.19041083,134.71968204)(575.47356468,134.82645286)(575.96971037,134.96186952)
\curveto(576.75151571,135.1806195)(577.34288128,135.28999449)(577.74380709,135.28999449)
\curveto(578.54064714,135.28999449)(579.16208215,135.08426534)(579.60811211,134.67280704)
\curveto(580.05414208,134.26655707)(580.27715706,133.80561961)(580.27715706,133.28999465)
\lineto(580.27715706,128.946245)
\lineto(581.11910126,128.946245)
\curveto(581.27446002,128.946245)(581.38471461,128.914995)(581.44986506,128.852495)
\curveto(581.5150155,128.79520334)(581.54759073,128.71707835)(581.54759073,128.61812002)
\curveto(581.54759073,128.52437003)(581.5150155,128.44624504)(581.44986506,128.38374504)
\curveto(581.38471461,128.32124505)(581.27446002,128.28999505)(581.11910126,128.28999505)
\closepath
\moveto(579.63818155,131.55561979)
\curveto(579.34249876,131.64416145)(579.02927547,131.70926561)(578.69851168,131.75093227)
\curveto(578.36774788,131.79259894)(578.01944358,131.81343227)(577.65359878,131.81343227)
\curveto(576.73648099,131.81343227)(576.0198261,131.60770312)(575.50363412,131.19624482)
\curveto(575.11273145,130.88895318)(574.91728012,130.52176571)(574.91728012,130.09468241)
\curveto(574.91728012,129.6988491)(575.06512151,129.3655158)(575.3608043,129.09468249)
\curveto(575.66149866,128.82384917)(576.09750548,128.68843252)(576.66882476,128.68843252)
\curveto(577.21508617,128.68843252)(577.72125501,128.80041168)(578.18733127,129.02436999)
\curveto(578.6584191,129.25353664)(579.14203586,129.61551578)(579.63818155,130.1103074)
\closepath
}
}
{
\newrgbcolor{curcolor}{0 0 0}
\pscustom[linestyle=none,fillstyle=solid,fillcolor=curcolor]
{
\newpath
\moveto(540.41697084,192.21853456)
\lineto(540.41697084,193.16384627)
\curveto(539.46384664,192.36176361)(538.44561833,191.96072228)(537.36228591,191.96072228)
\curveto(536.57582824,191.96072228)(535.96124542,192.15863878)(535.51853746,192.55447178)
\curveto(535.0758295,192.95551311)(534.85447552,193.44509603)(534.85447552,194.02322054)
\curveto(534.85447552,194.65863668)(535.14614194,195.21332371)(535.72947478,195.68728164)
\curveto(536.31280763,196.16123958)(537.16436941,196.39821855)(538.28416013,196.39821855)
\curveto(538.58624321,196.39821855)(538.91436794,196.37738523)(539.26853431,196.3357186)
\curveto(539.62270067,196.2992603)(540.00551285,196.23936451)(540.41697084,196.15603125)
\lineto(540.41697084,197.21853036)
\curveto(540.41697084,197.57790505)(540.25030431,197.89040479)(539.91697126,198.15602957)
\curveto(539.58363821,198.42165434)(539.08363863,198.55446673)(538.41697252,198.55446673)
\curveto(537.90655628,198.55446673)(537.19041105,198.40602936)(536.26853683,198.10915461)
\curveto(536.1018703,198.05707132)(535.99509956,198.03102967)(535.9482246,198.03102967)
\curveto(535.86489134,198.03102967)(535.79197473,198.06227965)(535.72947478,198.12477959)
\curveto(535.67218316,198.18727954)(535.64353735,198.26540448)(535.64353735,198.3591544)
\curveto(535.64353735,198.44769599)(535.669579,198.51800843)(535.72166229,198.57009172)
\curveto(535.79457889,198.64821665)(536.08884948,198.7549874)(536.60447405,198.89040395)
\curveto(537.41697336,199.10915377)(538.03155618,199.21852867)(538.4482225,199.21852867)
\curveto(539.2763468,199.21852867)(539.92217959,199.01279968)(540.38572087,198.60134169)
\curveto(540.84926214,198.19509204)(541.08103278,197.73415492)(541.08103278,197.21853036)
\lineto(541.08103278,192.87478401)
\lineto(541.95603205,192.87478401)
\curveto(542.11749024,192.87478401)(542.23207348,192.84353404)(542.29978176,192.78103409)
\curveto(542.36749003,192.72374247)(542.40134417,192.64561754)(542.40134417,192.54665929)
\curveto(542.40134417,192.45290936)(542.36749003,192.37478443)(542.29978176,192.31228448)
\curveto(542.23207348,192.24978454)(542.11749024,192.21853456)(541.95603205,192.21853456)
\closepath
\moveto(540.41697084,195.48415682)
\curveto(540.10967943,195.57269841)(539.78415887,195.63780252)(539.44040916,195.67946915)
\curveto(539.09665945,195.72113578)(538.73468059,195.7419691)(538.35447257,195.7419691)
\curveto(537.40134838,195.7419691)(536.65655734,195.5362401)(536.12009945,195.12478212)
\curveto(535.7138498,194.81749071)(535.51072497,194.45030352)(535.51072497,194.02322054)
\curveto(535.51072497,193.62738754)(535.66437067,193.29405449)(535.97166208,193.02322138)
\curveto(536.28416182,192.75238828)(536.73728643,192.61697173)(537.33103594,192.61697173)
\curveto(537.89874379,192.61697173)(538.42478502,192.7289508)(538.90915961,192.95290894)
\curveto(539.39874253,193.18207542)(539.90134627,193.54405428)(540.41697084,194.03884553)
\closepath
}
}
{
\newrgbcolor{curcolor}{0 0 0}
\pscustom[linestyle=none,fillstyle=solid,fillcolor=curcolor]
{
\newpath
\moveto(547.34665236,198.98415387)
\lineto(547.34665236,197.32790526)
\curveto(548.20081831,198.09873795)(548.8388386,198.5935292)(549.26071325,198.81227902)
\curveto(549.68779622,199.03623716)(550.08102506,199.14821623)(550.44039976,199.14821623)
\curveto(550.83102443,199.14821623)(551.19300329,199.01540385)(551.52633634,198.74977907)
\curveto(551.86487772,198.48936262)(552.03414841,198.29144612)(552.03414841,198.15602957)
\curveto(552.03414841,198.05707132)(552.00029428,197.97373805)(551.932586,197.90602978)
\curveto(551.87008605,197.84352983)(551.78935695,197.81227986)(551.6903987,197.81227986)
\curveto(551.63831541,197.81227986)(551.59404462,197.82009235)(551.55758632,197.83571734)
\curveto(551.52112801,197.85655065)(551.45341974,197.91644644)(551.35446149,198.01540469)
\curveto(551.17216997,198.1976962)(551.01331594,198.32269609)(550.87789939,198.39040437)
\curveto(550.74248283,198.45811265)(550.60967045,198.49196679)(550.47946222,198.49196679)
\curveto(550.19300413,198.49196679)(549.84665025,198.37738355)(549.4404006,198.14821707)
\curveto(549.03935927,197.9190506)(548.34144319,197.35915524)(547.34665236,196.46853099)
\lineto(547.34665236,192.87478401)
\lineto(550.25289991,192.87478401)
\curveto(550.41435811,192.87478401)(550.52894135,192.84353404)(550.59664962,192.78103409)
\curveto(550.6643579,192.72374247)(550.69821204,192.64561754)(550.69821204,192.54665929)
\curveto(550.69821204,192.45290936)(550.6643579,192.37478443)(550.59664962,192.31228448)
\curveto(550.52894135,192.24978454)(550.41435811,192.21853456)(550.25289991,192.21853456)
\lineto(545.10446674,192.21853456)
\curveto(544.94821687,192.21853456)(544.8362378,192.24718037)(544.76852953,192.30447199)
\curveto(544.70082125,192.36697194)(544.66696711,192.44509687)(544.66696711,192.53884679)
\curveto(544.66696711,192.62738838)(544.69821708,192.70030499)(544.76071703,192.75759661)
\curveto(544.82842531,192.82009656)(544.94300855,192.85134653)(545.10446674,192.85134653)
\lineto(546.69040291,192.85134653)
\lineto(546.69040291,198.32009193)
\lineto(545.47946643,198.32009193)
\curveto(545.32321656,198.32009193)(545.21123749,198.3513419)(545.14352921,198.41384185)
\curveto(545.07582093,198.4763418)(545.0419668,198.5570709)(545.0419668,198.65602915)
\curveto(545.0419668,198.74977907)(545.07321677,198.827904)(545.13571672,198.89040395)
\curveto(545.20342499,198.9529039)(545.31800823,198.98415387)(545.47946643,198.98415387)
\closepath
}
}
{
\newrgbcolor{curcolor}{0 0 0}
\pscustom[linestyle=none,fillstyle=solid,fillcolor=curcolor]
{
\newpath
\moveto(560.47164116,198.32009193)
\lineto(560.47164116,198.53884175)
\curveto(560.47164116,198.70029994)(560.50289114,198.81488318)(560.56539108,198.88259146)
\curveto(560.62789103,198.95029973)(560.70601597,198.98415387)(560.79976589,198.98415387)
\curveto(560.89872414,198.98415387)(560.97945324,198.95029973)(561.04195318,198.88259146)
\curveto(561.10445313,198.81488318)(561.13570311,198.70029994)(561.13570311,198.53884175)
\lineto(561.13570311,197.05446799)
\curveto(561.13049478,196.8930098)(561.09664064,196.77842656)(561.03414069,196.71071828)
\curveto(560.97684907,196.64301001)(560.89872414,196.60915587)(560.79976589,196.60915587)
\curveto(560.7112243,196.60915587)(560.63570353,196.63780168)(560.57320358,196.6950933)
\curveto(560.51591196,196.75759324)(560.48205782,196.85915566)(560.47164116,196.99978054)
\curveto(560.44039119,197.3695719)(560.19559973,197.7211341)(559.73726678,198.05446715)
\curveto(559.28414216,198.38780021)(558.67216351,198.55446673)(557.90133083,198.55446673)
\curveto(556.92737331,198.55446673)(556.1877906,198.24977949)(555.68258269,197.640405)
\curveto(555.17737478,197.03103051)(554.92477083,196.33311443)(554.92477083,195.54665676)
\curveto(554.92477083,194.69769914)(555.20341643,193.9971789)(555.76070763,193.44509603)
\curveto(556.31799882,192.89301316)(557.03935238,192.61697173)(557.92476831,192.61697173)
\curveto(558.43518454,192.61697173)(558.95341327,192.71072165)(559.4794545,192.89822149)
\curveto(560.01070405,193.08572133)(560.48987032,193.38780441)(560.91695329,193.80447073)
\curveto(561.0263282,193.90863731)(561.12268228,193.9607206)(561.20601555,193.9607206)
\curveto(561.29455714,193.9607206)(561.36747374,193.92947062)(561.42476536,193.86697068)
\curveto(561.48726531,193.80967906)(561.51851528,193.73676245)(561.51851528,193.64822086)
\curveto(561.51851528,193.42426271)(561.25549467,193.14040879)(560.72945345,192.79665908)
\curveto(559.88049583,192.23936788)(558.93518412,191.96072228)(557.89351833,191.96072228)
\curveto(556.83622755,191.96072228)(555.96643662,192.2966595)(555.28414553,192.96853393)
\curveto(554.60706276,193.64561669)(554.26852138,194.50238681)(554.26852138,195.53884427)
\curveto(554.26852138,196.59613505)(554.61487526,197.47373848)(555.30758301,198.17165455)
\curveto(556.00549909,198.86957063)(556.88310252,199.21852867)(557.94039329,199.21852867)
\curveto(558.94560078,199.21852867)(559.78935007,198.91904976)(560.47164116,198.32009193)
\closepath
}
}
{
\newrgbcolor{curcolor}{0 0 0}
\pscustom[linestyle=none,fillstyle=solid,fillcolor=curcolor]
{
\newpath
\moveto(571.69819497,187.83572575)
\lineto(562.97163981,187.83572575)
\curveto(562.81538994,187.83572575)(562.70341087,187.86697572)(562.63570259,187.92947567)
\curveto(562.56799431,187.98676729)(562.53414018,188.06489222)(562.53414018,188.16385047)
\curveto(562.53414018,188.26280872)(562.56799431,188.34353782)(562.63570259,188.40603777)
\curveto(562.70341087,188.46332939)(562.81538994,188.4919752)(562.97163981,188.4919752)
\lineto(571.69819497,188.4919752)
\curveto(571.85965317,188.4919752)(571.97163224,188.46332939)(572.03413219,188.40603777)
\curveto(572.10184046,188.34353782)(572.1356946,188.26280872)(572.1356946,188.16385047)
\curveto(572.1356946,188.06489222)(572.10184046,187.98676729)(572.03413219,187.92947567)
\curveto(571.97163224,187.86697572)(571.85965317,187.83572575)(571.69819497,187.83572575)
\closepath
}
}
{
\newrgbcolor{curcolor}{0 0 0}
\pscustom[linestyle=none,fillstyle=solid,fillcolor=curcolor]
{
\newpath
\moveto(574.46381659,202.02321382)
\lineto(574.46381659,197.67165498)
\curveto(575.25548259,198.70290411)(576.21121095,199.21852867)(577.33100167,199.21852867)
\curveto(578.2893342,199.21852867)(579.10964601,198.86957063)(579.7919371,198.17165455)
\curveto(580.4742282,197.4789468)(580.81537374,196.62738502)(580.81537374,195.6169692)
\curveto(580.81537374,194.59613673)(580.46901987,193.73155412)(579.77631212,193.02322138)
\curveto(579.0888127,192.31488865)(578.27370922,191.96072228)(577.33100167,191.96072228)
\curveto(576.1851693,191.96072228)(575.22944094,192.47634684)(574.46381659,193.50759598)
\lineto(574.46381659,192.21853456)
\lineto(572.92475538,192.21853456)
\curveto(572.76850551,192.21853456)(572.65652644,192.24978454)(572.58881816,192.31228448)
\curveto(572.52110989,192.37478443)(572.48725575,192.45290936)(572.48725575,192.54665929)
\curveto(572.48725575,192.64561754)(572.52110989,192.72374247)(572.58881816,192.78103409)
\curveto(572.65652644,192.84353404)(572.76850551,192.87478401)(572.92475538,192.87478401)
\lineto(573.80756714,192.87478401)
\lineto(573.80756714,201.35915187)
\lineto(572.92475538,201.35915187)
\curveto(572.76850551,201.35915187)(572.65652644,201.39040185)(572.58881816,201.4529018)
\curveto(572.52110989,201.51540174)(572.48725575,201.59613084)(572.48725575,201.69508909)
\curveto(572.48725575,201.78883901)(572.52110989,201.86696395)(572.58881816,201.92946389)
\curveto(572.65652644,201.99196384)(572.76850551,202.02321382)(572.92475538,202.02321382)
\closepath
\moveto(580.1591243,195.58571923)
\curveto(580.1591243,196.41384353)(579.87527037,197.11436378)(579.30756251,197.68727996)
\curveto(578.73985466,198.26540448)(578.07579271,198.55446673)(577.31537669,198.55446673)
\curveto(576.55496066,198.55446673)(575.89089872,198.26540448)(575.32319086,197.68727996)
\curveto(574.75548301,197.11436378)(574.47162908,196.41384353)(574.47162908,195.58571923)
\curveto(574.47162908,194.75759493)(574.75548301,194.05447052)(575.32319086,193.476346)
\curveto(575.89089872,192.90342982)(576.55496066,192.61697173)(577.31537669,192.61697173)
\curveto(578.07579271,192.61697173)(578.73985466,192.90342982)(579.30756251,193.476346)
\curveto(579.87527037,194.05447052)(580.1591243,194.75759493)(580.1591243,195.58571923)
\closepath
}
}
{
\newrgbcolor{curcolor}{0 0 0}
\pscustom[linestyle=none,fillstyle=solid,fillcolor=curcolor]
{
\newpath
\moveto(588.78411599,192.21853456)
\lineto(588.78411599,193.17947125)
\curveto(587.88828341,192.36697194)(586.91953422,191.96072228)(585.87786843,191.96072228)
\curveto(585.23724397,191.96072228)(584.75026521,192.1352013)(584.41693216,192.48415934)
\curveto(583.98464086,192.94249229)(583.7684952,193.476346)(583.7684952,194.08572049)
\lineto(583.7684952,198.32009193)
\lineto(582.88568345,198.32009193)
\curveto(582.72943358,198.32009193)(582.61745451,198.3513419)(582.54974623,198.41384185)
\curveto(582.48203795,198.4763418)(582.44818381,198.5570709)(582.44818381,198.65602915)
\curveto(582.44818381,198.74977907)(582.48203795,198.827904)(582.54974623,198.89040395)
\curveto(582.61745451,198.9529039)(582.72943358,198.98415387)(582.88568345,198.98415387)
\lineto(584.42474465,198.98415387)
\lineto(584.42474465,194.08572049)
\curveto(584.42474465,193.65863752)(584.56016121,193.30707531)(584.83099431,193.03103388)
\curveto(585.10182742,192.75499244)(585.4403688,192.61697173)(585.84661846,192.61697173)
\curveto(586.91432589,192.61697173)(587.89349174,193.10655465)(588.78411599,194.08572049)
\lineto(588.78411599,198.32009193)
\lineto(587.5731795,198.32009193)
\curveto(587.41692964,198.32009193)(587.30495056,198.3513419)(587.23724229,198.41384185)
\curveto(587.16953401,198.4763418)(587.13567987,198.5570709)(587.13567987,198.65602915)
\curveto(587.13567987,198.74977907)(587.16953401,198.827904)(587.23724229,198.89040395)
\curveto(587.30495056,198.9529039)(587.41692964,198.98415387)(587.5731795,198.98415387)
\lineto(589.44036543,198.98415387)
\lineto(589.44036543,192.87478401)
\lineto(589.99505247,192.87478401)
\curveto(590.15130234,192.87478401)(590.26328141,192.84353404)(590.33098969,192.78103409)
\curveto(590.39869796,192.72374247)(590.4325521,192.64561754)(590.4325521,192.54665929)
\curveto(590.4325521,192.45290936)(590.39869796,192.37478443)(590.33098969,192.31228448)
\curveto(590.26328141,192.24978454)(590.15130234,192.21853456)(589.99505247,192.21853456)
\closepath
}
}
{
\newrgbcolor{curcolor}{0 0 0}
\pscustom[linestyle=none,fillstyle=solid,fillcolor=curcolor]
{
\newpath
\moveto(595.70598771,198.32009193)
\lineto(595.70598771,192.87478401)
\lineto(598.58879779,192.87478401)
\curveto(598.74504766,192.87478401)(598.85702673,192.84353404)(598.92473501,192.78103409)
\curveto(598.99244328,192.72374247)(599.02629742,192.64561754)(599.02629742,192.54665929)
\curveto(599.02629742,192.45290936)(598.99244328,192.37478443)(598.92473501,192.31228448)
\curveto(598.85702673,192.24978454)(598.74504766,192.21853456)(598.58879779,192.21853456)
\lineto(593.45598961,192.21853456)
\curveto(593.29973974,192.21853456)(593.18776066,192.24978454)(593.12005239,192.31228448)
\curveto(593.05234411,192.37478443)(593.01848997,192.45290936)(593.01848997,192.54665929)
\curveto(593.01848997,192.64561754)(593.05234411,192.72374247)(593.12005239,192.78103409)
\curveto(593.18776066,192.84353404)(593.29973974,192.87478401)(593.45598961,192.87478401)
\lineto(595.04192577,192.87478401)
\lineto(595.04192577,198.32009193)
\lineto(593.62005197,198.32009193)
\curveto(593.4638021,198.32009193)(593.35182303,198.3513419)(593.28411475,198.41384185)
\curveto(593.21640647,198.4763418)(593.18255234,198.5570709)(593.18255234,198.65602915)
\curveto(593.18255234,198.74977907)(593.21640647,198.827904)(593.28411475,198.89040395)
\curveto(593.35182303,198.9529039)(593.4638021,198.98415387)(593.62005197,198.98415387)
\lineto(595.04192577,198.98415387)
\lineto(595.04192577,199.97634054)
\curveto(595.04192577,200.52842341)(595.26588392,201.00758967)(595.71380021,201.41383933)
\curveto(596.1617165,201.82008899)(596.755466,202.02321382)(597.49504871,202.02321382)
\curveto(598.11483985,202.02321382)(598.77629763,201.9659222)(599.47942204,201.85133896)
\curveto(599.74504682,201.80967233)(599.90390085,201.7601932)(599.95598414,201.70290158)
\curveto(600.01327576,201.64560997)(600.04192157,201.5700892)(600.04192157,201.47633928)
\curveto(600.04192157,201.38258935)(600.01067159,201.30446442)(599.94817165,201.24196447)
\curveto(599.8856717,201.18467285)(599.80233843,201.15602704)(599.69817186,201.15602704)
\curveto(599.65650522,201.15602704)(599.58619278,201.16383954)(599.48723453,201.17946453)
\curveto(598.70077686,201.29925609)(598.03671492,201.35915187)(597.49504871,201.35915187)
\curveto(596.92213252,201.35915187)(596.47942456,201.21852699)(596.16692483,200.93727723)
\curveto(595.85963342,200.65602747)(595.70598771,200.33571523)(595.70598771,199.97634054)
\lineto(595.70598771,198.98415387)
\lineto(598.77629763,198.98415387)
\curveto(598.9325475,198.98415387)(599.04452657,198.9529039)(599.11223485,198.89040395)
\curveto(599.17994312,198.827904)(599.21379726,198.7471749)(599.21379726,198.64821665)
\curveto(599.21379726,198.55446673)(599.17994312,198.4763418)(599.11223485,198.41384185)
\curveto(599.04452657,198.3513419)(598.9325475,198.32009193)(598.77629763,198.32009193)
\closepath
}
}
{
\newrgbcolor{curcolor}{0 0 0}
\pscustom[linestyle=none,fillstyle=solid,fillcolor=curcolor]
{
\newpath
\moveto(610.10441205,187.83572575)
\lineto(601.37785689,187.83572575)
\curveto(601.22160702,187.83572575)(601.10962795,187.86697572)(601.04191967,187.92947567)
\curveto(600.97421139,187.98676729)(600.94035725,188.06489222)(600.94035725,188.16385047)
\curveto(600.94035725,188.26280872)(600.97421139,188.34353782)(601.04191967,188.40603777)
\curveto(601.10962795,188.46332939)(601.22160702,188.4919752)(601.37785689,188.4919752)
\lineto(610.10441205,188.4919752)
\curveto(610.26587025,188.4919752)(610.37784932,188.46332939)(610.44034926,188.40603777)
\curveto(610.50805754,188.34353782)(610.54191168,188.26280872)(610.54191168,188.16385047)
\curveto(610.54191168,188.06489222)(610.50805754,187.98676729)(610.44034926,187.92947567)
\curveto(610.37784932,187.86697572)(610.26587025,187.83572575)(610.10441205,187.83572575)
\closepath
}
}
{
\newrgbcolor{curcolor}{0 0 0}
\pscustom[linestyle=none,fillstyle=solid,fillcolor=curcolor]
{
\newpath
\moveto(613.99503272,198.98415387)
\lineto(617.55752972,198.98415387)
\curveto(617.71377959,198.98415387)(617.82575866,198.9529039)(617.89346694,198.89040395)
\curveto(617.96117522,198.827904)(617.99502936,198.7471749)(617.99502936,198.64821665)
\curveto(617.99502936,198.55446673)(617.96117522,198.4763418)(617.89346694,198.41384185)
\curveto(617.82575866,198.3513419)(617.71377959,198.32009193)(617.55752972,198.32009193)
\lineto(613.99503272,198.32009193)
\lineto(613.99503272,193.9529081)
\curveto(613.99503272,193.57270009)(614.14607426,193.25499202)(614.44815734,192.9997839)
\curveto(614.75544875,192.74457579)(615.20336504,192.61697173)(615.79190621,192.61697173)
\curveto(616.23461417,192.61697173)(616.71378043,192.68207584)(617.229405,192.81228406)
\curveto(617.74502957,192.94770062)(618.14607089,193.09874215)(618.43252899,193.26540868)
\curveto(618.53669557,193.33311696)(618.62263299,193.3669711)(618.69034127,193.3669711)
\curveto(618.77367453,193.3669711)(618.84659114,193.33311696)(618.90909109,193.26540868)
\curveto(618.97159103,193.20290873)(619.00284101,193.12738796)(619.00284101,193.03884637)
\curveto(619.00284101,192.96072144)(618.96898687,192.88780483)(618.90127859,192.82009656)
\curveto(618.73461207,192.6482217)(618.32836241,192.46072186)(617.68252962,192.25759703)
\curveto(617.04190516,192.05968053)(616.42732234,191.96072228)(615.83878117,191.96072228)
\curveto(615.07315681,191.96072228)(614.46378233,192.14040963)(614.01065771,192.49978433)
\curveto(613.55753309,192.85915902)(613.33097078,193.34353362)(613.33097078,193.9529081)
\lineto(613.33097078,198.32009193)
\lineto(612.1200343,198.32009193)
\curveto(611.96378443,198.32009193)(611.85180536,198.3513419)(611.78409708,198.41384185)
\curveto(611.7163888,198.4763418)(611.68253466,198.5570709)(611.68253466,198.65602915)
\curveto(611.68253466,198.74977907)(611.7163888,198.827904)(611.78409708,198.89040395)
\curveto(611.85180536,198.9529039)(611.96378443,198.98415387)(612.1200343,198.98415387)
\lineto(613.33097078,198.98415387)
\lineto(613.33097078,200.92165224)
\curveto(613.33097078,201.07790211)(613.36222075,201.18988118)(613.4247207,201.25758946)
\curveto(613.48722065,201.32529774)(613.56534558,201.35915187)(613.6590955,201.35915187)
\curveto(613.75805375,201.35915187)(613.83878285,201.32529774)(613.9012828,201.25758946)
\curveto(613.96378275,201.18988118)(613.99503272,201.07790211)(613.99503272,200.92165224)
\closepath
}
}
{
\newrgbcolor{curcolor}{0 0 0}
\pscustom[linestyle=none,fillstyle=solid,fillcolor=curcolor]
{
\newpath
\moveto(467.59928519,200.22107986)
\curveto(467.59928519,201.55753006)(465.98357739,202.22658459)(465.03867894,201.28168614)
\curveto(464.09378049,200.33678769)(464.76283502,198.72107989)(466.09928522,198.72107989)
\curveto(467.43573541,198.72107989)(468.10478994,200.33678769)(467.15989149,201.28168614)
\curveto(466.21499305,202.22658459)(464.59928524,201.55753006)(464.59928524,200.22107986)
\curveto(464.59928524,198.88462967)(466.21499305,198.21557514)(467.15989149,199.16047359)
\curveto(468.10478994,200.10537204)(467.43573541,201.72107984)(466.09928522,201.72107984)
\curveto(464.76283502,201.72107984)(464.09378049,200.10537204)(465.03867894,199.16047359)
\curveto(465.98357739,198.21557514)(467.59928519,198.88462967)(467.59928519,200.22107986)
\closepath
}
}
{
\newrgbcolor{curcolor}{0 0 0}
\pscustom[linewidth=1.03964224,linecolor=curcolor]
{
\newpath
\moveto(524.70407812,183.60627699)
\lineto(466.27711749,183.60627699)
}
}
{
\newrgbcolor{curcolor}{0 0 0}
\pscustom[linestyle=none,fillstyle=solid,fillcolor=curcolor]
{
\newpath
\moveto(514.30765575,183.60627699)
\lineto(510.14908681,179.44770804)
\lineto(524.70407812,183.60627699)
\lineto(510.14908681,187.76484593)
\closepath
}
}
{
\newrgbcolor{curcolor}{0 0 0}
\pscustom[linewidth=1.10895175,linecolor=curcolor]
{
\newpath
\moveto(514.30765575,183.60627699)
\lineto(510.14908681,179.44770804)
\lineto(524.70407812,183.60627699)
\lineto(510.14908681,187.76484593)
\closepath
}
}
{
\newrgbcolor{curcolor}{0 0 0}
\pscustom[linewidth=0.8600164,linecolor=curcolor]
{
\newpath
\moveto(466.32235843,151.29027699)
\lineto(489.85728757,151.29027699)
\lineto(489.85728757,73.50852581)
\lineto(334.45223812,73.50852581)
}
}
{
\newrgbcolor{curcolor}{0 0 0}
\pscustom[linestyle=none,fillstyle=solid,fillcolor=curcolor]
{
\newpath
\moveto(343.05240214,73.50852581)
\lineto(346.49246775,76.94859142)
\lineto(334.45223812,73.50852581)
\lineto(346.49246775,70.0684602)
\closepath
}
}
{
\newrgbcolor{curcolor}{0 0 0}
\pscustom[linewidth=0.91735086,linecolor=curcolor]
{
\newpath
\moveto(343.05240214,73.50852581)
\lineto(346.49246775,76.94859142)
\lineto(334.45223812,73.50852581)
\lineto(346.49246775,70.0684602)
\closepath
}
}
{
\newrgbcolor{curcolor}{0 0 0}
\pscustom[linewidth=1.88976378,linecolor=curcolor]
{
\newpath
\moveto(529.64582188,55.1584107)
\lineto(646.35531568,55.1584107)
\lineto(646.35531568,0.94488593)
\lineto(529.64582188,0.94488593)
\closepath
}
}
{
\newrgbcolor{curcolor}{0 0 0}
\pscustom[linestyle=none,fillstyle=solid,fillcolor=curcolor]
{
\newpath
\moveto(539.14985057,32.26806968)
\lineto(539.14985057,33.2290071)
\curveto(538.28786007,32.41650717)(537.35570756,32.0102572)(536.35339302,32.0102572)
\curveto(535.73696959,32.0102572)(535.26838754,32.18473635)(534.94764689,32.53369466)
\curveto(534.53168636,32.99202796)(534.3237061,33.52588208)(534.3237061,34.13525704)
\lineto(534.3237061,38.36963171)
\lineto(533.47424453,38.36963171)
\curveto(533.32389735,38.36963171)(533.21614854,38.40088171)(533.15099809,38.46338171)
\curveto(533.08584765,38.5258817)(533.05327243,38.60661086)(533.05327243,38.70556919)
\curveto(533.05327243,38.79931918)(533.08584765,38.87744418)(533.15099809,38.93994417)
\curveto(533.21614854,39.00244417)(533.32389735,39.03369416)(533.47424453,39.03369416)
\lineto(534.95516425,39.03369416)
\lineto(534.95516425,34.13525704)
\curveto(534.95516425,33.70817374)(535.08546514,33.35661126)(535.34606692,33.08056962)
\curveto(535.6066687,32.80452797)(535.93242092,32.66650715)(536.32332359,32.66650715)
\curveto(537.35069598,32.66650715)(538.29287164,33.15609044)(539.14985057,34.13525704)
\lineto(539.14985057,38.36963171)
\lineto(537.98465993,38.36963171)
\curveto(537.83431275,38.36963171)(537.72656393,38.40088171)(537.66141349,38.46338171)
\curveto(537.59626304,38.5258817)(537.56368782,38.60661086)(537.56368782,38.70556919)
\curveto(537.56368782,38.79931918)(537.59626304,38.87744418)(537.66141349,38.93994417)
\curveto(537.72656393,39.00244417)(537.83431275,39.03369416)(537.98465993,39.03369416)
\lineto(539.78130873,39.03369416)
\lineto(539.78130873,32.92431963)
\lineto(540.31504121,32.92431963)
\curveto(540.46538839,32.92431963)(540.57313721,32.89306963)(540.63828765,32.83056964)
\curveto(540.7034381,32.77327797)(540.73601332,32.69515298)(540.73601332,32.59619465)
\curveto(540.73601332,32.50244466)(540.7034381,32.42431967)(540.63828765,32.36181967)
\curveto(540.57313721,32.29931968)(540.46538839,32.26806968)(540.31504121,32.26806968)
\closepath
}
}
{
\newrgbcolor{curcolor}{0 0 0}
\pscustom[linestyle=none,fillstyle=solid,fillcolor=curcolor]
{
\newpath
\moveto(544.18648153,39.03369416)
\lineto(544.18648153,38.04150674)
\curveto(544.62749993,38.50504837)(545.02591995,38.82536085)(545.38174161,39.00244417)
\curveto(545.73756327,39.17952749)(546.13848909,39.26806915)(546.58451905,39.26806915)
\curveto(547.06563003,39.26806915)(547.50414264,39.16129832)(547.90005688,38.94775667)
\curveto(548.18070495,38.79150668)(548.43378937,38.53109004)(548.65931014,38.16650673)
\curveto(548.88984248,37.80713176)(549.00510865,37.43734012)(549.00510865,37.05713181)
\lineto(549.00510865,32.92431963)
\lineto(549.53884114,32.92431963)
\curveto(549.68918832,32.92431963)(549.79693713,32.89306963)(549.86208757,32.83056964)
\curveto(549.92723802,32.77327797)(549.95981324,32.69515298)(549.95981324,32.59619465)
\curveto(549.95981324,32.50244466)(549.92723802,32.42431967)(549.86208757,32.36181967)
\curveto(549.79693713,32.29931968)(549.68918832,32.26806968)(549.53884114,32.26806968)
\lineto(547.84743536,32.26806968)
\curveto(547.69207661,32.26806968)(547.58182201,32.29931968)(547.51667157,32.36181967)
\curveto(547.45152112,32.42431967)(547.4189459,32.50244466)(547.4189459,32.59619465)
\curveto(547.4189459,32.69515298)(547.45152112,32.77327797)(547.51667157,32.83056964)
\curveto(547.58182201,32.89306963)(547.69207661,32.92431963)(547.84743536,32.92431963)
\lineto(548.37365049,32.92431963)
\lineto(548.37365049,36.94775682)
\curveto(548.37365049,37.41129845)(548.21077438,37.80192342)(547.88502216,38.11963173)
\curveto(547.55926994,38.44254838)(547.12326311,38.6040067)(546.57700169,38.6040067)
\curveto(546.16104116,38.6040067)(545.80020793,38.51546504)(545.494502,38.33838172)
\curveto(545.18879607,38.16650673)(544.75278924,37.7342151)(544.18648153,37.04150682)
\lineto(544.18648153,32.92431963)
\lineto(544.90063064,32.92431963)
\curveto(545.05097782,32.92431963)(545.15872663,32.89306963)(545.22387707,32.83056964)
\curveto(545.28902752,32.77327797)(545.32160274,32.69515298)(545.32160274,32.59619465)
\curveto(545.32160274,32.50244466)(545.28902752,32.42431967)(545.22387707,32.36181967)
\curveto(545.15872663,32.29931968)(545.05097782,32.26806968)(544.90063064,32.26806968)
\lineto(542.84087427,32.26806968)
\curveto(542.69052709,32.26806968)(542.58277828,32.29931968)(542.51762784,32.36181967)
\curveto(542.45247739,32.42431967)(542.41990217,32.50244466)(542.41990217,32.59619465)
\curveto(542.41990217,32.69515298)(542.45247739,32.77327797)(542.51762784,32.83056964)
\curveto(542.58277828,32.89306963)(542.69052709,32.92431963)(542.84087427,32.92431963)
\lineto(543.55502338,32.92431963)
\lineto(543.55502338,38.36963171)
\lineto(543.02129089,38.36963171)
\curveto(542.87094371,38.36963171)(542.7631949,38.40088171)(542.69804445,38.46338171)
\curveto(542.63289401,38.5258817)(542.60031879,38.60661086)(542.60031879,38.70556919)
\curveto(542.60031879,38.79931918)(542.63289401,38.87744418)(542.69804445,38.93994417)
\curveto(542.7631949,39.00244417)(542.87094371,39.03369416)(543.02129089,39.03369416)
\closepath
}
}
{
\newrgbcolor{curcolor}{0 0 0}
\pscustom[linestyle=none,fillstyle=solid,fillcolor=curcolor]
{
\newpath
\moveto(558.10111347,38.36963171)
\lineto(558.10111347,38.5883817)
\curveto(558.10111347,38.74984002)(558.13118291,38.86442334)(558.19132178,38.93213167)
\curveto(558.25146065,38.99984)(558.32663424,39.03369416)(558.41684255,39.03369416)
\curveto(558.51206243,39.03369416)(558.5897418,38.99984)(558.64988068,38.93213167)
\curveto(558.71001955,38.86442334)(558.74008898,38.74984002)(558.74008898,38.5883817)
\lineto(558.74008898,37.10400681)
\curveto(558.73507741,36.94254849)(558.70250219,36.82796517)(558.64236332,36.76025684)
\curveto(558.58723602,36.69254851)(558.51206243,36.65869434)(558.41684255,36.65869434)
\curveto(558.33164581,36.65869434)(558.25897801,36.68734018)(558.19883914,36.74463184)
\curveto(558.14371184,36.80713183)(558.11113662,36.90869433)(558.10111347,37.04931932)
\curveto(558.07104403,37.41911095)(557.83550012,37.77067343)(557.39448172,38.10400674)
\curveto(556.9584749,38.43734004)(556.36961512,38.6040067)(555.62790236,38.6040067)
\curveto(554.69073827,38.6040067)(553.97909495,38.29931922)(553.49297241,37.68994427)
\curveto(553.00684986,37.08056931)(552.76378858,36.3826527)(552.76378858,35.59619443)
\curveto(552.76378858,34.74723616)(553.03190772,34.04671538)(553.568146,33.49463208)
\curveto(554.10438427,32.94254879)(554.79848709,32.66650715)(555.65045444,32.66650715)
\curveto(556.14158856,32.66650715)(556.64024004,32.76025714)(557.14640888,32.94775713)
\curveto(557.65758929,33.13525711)(558.11865397,33.43734042)(558.52960293,33.85400706)
\curveto(558.63484596,33.95817372)(558.72756005,34.01025705)(558.80774522,34.01025705)
\curveto(558.89294195,34.01025705)(558.96310397,33.97900705)(559.01823127,33.91650705)
\curveto(559.07837014,33.85921539)(559.10843958,33.78629873)(559.10843958,33.69775707)
\curveto(559.10843958,33.47379875)(558.85535516,33.18994461)(558.34918632,32.84619463)
\curveto(557.53229997,32.28890301)(556.62269953,32.0102572)(555.620385,32.0102572)
\curveto(554.60303575,32.0102572)(553.76610312,32.34619467)(553.1095871,33.01806962)
\curveto(552.45808265,33.6951529)(552.13233043,34.55192367)(552.13233043,35.58838193)
\curveto(552.13233043,36.64567351)(552.46560001,37.52327761)(553.13213917,38.22119423)
\curveto(553.80368991,38.91911084)(554.64813991,39.26806915)(555.66548916,39.26806915)
\curveto(556.63272268,39.26806915)(557.44459745,38.96859)(558.10111347,38.36963171)
\closepath
}
}
{
\newrgbcolor{curcolor}{0 0 0}
\pscustom[linestyle=none,fillstyle=solid,fillcolor=curcolor]
{
\newpath
\moveto(568.23451472,35.63525692)
\curveto(568.23451472,34.635257)(567.8887162,33.7810904)(567.19711918,33.07275712)
\curveto(566.51053372,32.36442384)(565.68111844,32.0102572)(564.70887335,32.0102572)
\curveto(563.72660511,32.0102572)(562.89217826,32.36442384)(562.2055928,33.07275712)
\curveto(561.51900735,33.78629873)(561.17571462,34.64046533)(561.17571462,35.63525692)
\curveto(561.17571462,36.63525685)(561.51900735,37.48942345)(562.2055928,38.19775673)
\curveto(562.89217826,38.91129834)(563.72660511,39.26806915)(564.70887335,39.26806915)
\curveto(565.68111844,39.26806915)(566.51053372,38.91390251)(567.19711918,38.20556923)
\curveto(567.8887162,37.49723595)(568.23451472,36.64046518)(568.23451472,35.63525692)
\closepath
\moveto(567.5955392,35.63525692)
\curveto(567.5955392,36.45817353)(567.31238535,37.15869431)(566.74607764,37.73681926)
\curveto(566.1847815,38.31494422)(565.50320762,38.6040067)(564.70135599,38.6040067)
\curveto(563.89950436,38.6040067)(563.21542469,38.31234005)(562.64911698,37.72900676)
\curveto(562.08782085,37.15088181)(561.80717278,36.45296519)(561.80717278,35.63525692)
\curveto(561.80717278,34.82275698)(562.08782085,34.12484037)(562.64911698,33.54150708)
\curveto(563.21542469,32.95817379)(563.89950436,32.66650715)(564.70135599,32.66650715)
\curveto(565.50320762,32.66650715)(566.1847815,32.95556963)(566.74607764,33.53369458)
\curveto(567.31238535,34.11702787)(567.5955392,34.81754865)(567.5955392,35.63525692)
\closepath
}
}
{
\newrgbcolor{curcolor}{0 0 0}
\pscustom[linestyle=none,fillstyle=solid,fillcolor=curcolor]
{
\newpath
\moveto(571.06855861,39.03369416)
\lineto(571.06855861,38.36963171)
\curveto(571.60479689,38.96859)(572.14354095,39.26806915)(572.6847908,39.26806915)
\curveto(573.01054302,39.26806915)(573.29620266,39.17692332)(573.54176972,38.99463167)
\curveto(573.78733678,38.81754835)(573.99281126,38.54671503)(574.15819316,38.18213173)
\curveto(574.43884123,38.54671503)(574.72199508,38.81754835)(575.00765472,38.99463167)
\curveto(575.29832594,39.17692332)(575.58899715,39.26806915)(575.87966837,39.26806915)
\curveto(576.33572148,39.26806915)(576.6990605,39.11442333)(576.96968542,38.80713168)
\curveto(577.32550708,38.41129838)(577.50341791,37.97900674)(577.50341791,37.51025678)
\lineto(577.50341791,32.92431963)
\lineto(578.0371504,32.92431963)
\curveto(578.18749758,32.92431963)(578.29524639,32.89306963)(578.36039684,32.83056964)
\curveto(578.42554728,32.77327797)(578.4581225,32.69515298)(578.4581225,32.59619465)
\curveto(578.4581225,32.50244466)(578.42554728,32.42431967)(578.36039684,32.36181967)
\curveto(578.29524639,32.29931968)(578.18749758,32.26806968)(578.0371504,32.26806968)
\lineto(576.87195976,32.26806968)
\lineto(576.87195976,37.44775678)
\curveto(576.87195976,37.78109009)(576.77423409,38.05713174)(576.57878275,38.27588172)
\curveto(576.38333142,38.49463171)(576.15781065,38.6040067)(575.90222044,38.6040067)
\curveto(575.6716881,38.6040067)(575.42862683,38.51286087)(575.17303662,38.33056922)
\curveto(574.91744642,38.1534859)(574.6267752,37.80192342)(574.30102298,37.2758818)
\lineto(574.30102298,32.92431963)
\lineto(574.82723811,32.92431963)
\curveto(574.97758529,32.92431963)(575.0853341,32.89306963)(575.15048455,32.83056964)
\curveto(575.21563499,32.77327797)(575.24821021,32.69515298)(575.24821021,32.59619465)
\curveto(575.24821021,32.50244466)(575.21563499,32.42431967)(575.15048455,32.36181967)
\curveto(575.0853341,32.29931968)(574.97758529,32.26806968)(574.82723811,32.26806968)
\lineto(573.66204746,32.26806968)
\lineto(573.66204746,37.40088179)
\curveto(573.66204746,37.7498401)(573.56181601,38.03629841)(573.3613531,38.26025672)
\curveto(573.16590177,38.48942337)(572.94539257,38.6040067)(572.69982551,38.6040067)
\curveto(572.47430474,38.6040067)(572.25128976,38.52848587)(572.03078056,38.37744421)
\curveto(571.72507463,38.16390256)(571.40433398,37.79671509)(571.06855861,37.2758818)
\lineto(571.06855861,32.92431963)
\lineto(571.6022911,32.92431963)
\curveto(571.75263828,32.92431963)(571.86038709,32.89306963)(571.92553754,32.83056964)
\curveto(571.99068798,32.77327797)(572.0232632,32.69515298)(572.0232632,32.59619465)
\curveto(572.0232632,32.50244466)(571.99068798,32.42431967)(571.92553754,32.36181967)
\curveto(571.86038709,32.29931968)(571.75263828,32.26806968)(571.6022911,32.26806968)
\lineto(569.90336797,32.26806968)
\curveto(569.75302079,32.26806968)(569.64527197,32.29931968)(569.58012153,32.36181967)
\curveto(569.51497109,32.42431967)(569.48239586,32.50244466)(569.48239586,32.59619465)
\curveto(569.48239586,32.69515298)(569.51497109,32.77327797)(569.58012153,32.83056964)
\curveto(569.64527197,32.89306963)(569.75302079,32.92431963)(569.90336797,32.92431963)
\lineto(570.43710046,32.92431963)
\lineto(570.43710046,38.36963171)
\lineto(569.90336797,38.36963171)
\curveto(569.75302079,38.36963171)(569.64527197,38.40088171)(569.58012153,38.46338171)
\curveto(569.51497109,38.5258817)(569.48239586,38.60661086)(569.48239586,38.70556919)
\curveto(569.48239586,38.79931918)(569.51497109,38.87744418)(569.58012153,38.93994417)
\curveto(569.64527197,39.00244417)(569.75302079,39.03369416)(569.90336797,39.03369416)
\closepath
}
}
{
\newrgbcolor{curcolor}{0 0 0}
\pscustom[linestyle=none,fillstyle=solid,fillcolor=curcolor]
{
\newpath
\moveto(580.80353806,39.03369416)
\lineto(580.80353806,37.83838176)
\curveto(581.17940601,38.31234005)(581.58283761,38.66911086)(582.01383286,38.90869417)
\curveto(582.44482811,39.14827749)(582.95350274,39.26806915)(583.53985674,39.26806915)
\curveto(584.16129175,39.26806915)(584.73511682,39.11702749)(585.26133195,38.81494418)
\curveto(585.78754708,38.51286087)(586.19348446,38.0909859)(586.4791441,37.54931928)
\curveto(586.76981532,37.01286098)(586.91515093,36.44775686)(586.91515093,35.85400691)
\curveto(586.91515093,34.91129864)(586.5893987,34.10400704)(585.93789426,33.43213209)
\curveto(585.29140138,32.76546547)(584.49456133,32.43213217)(583.5473741,32.43213217)
\curveto(582.41977025,32.43213217)(581.50515824,32.90869463)(580.80353806,33.86181956)
\lineto(580.80353806,29.90869486)
\lineto(582.3370793,29.90869486)
\curveto(582.48742648,29.90869486)(582.59517529,29.88004903)(582.66032573,29.82275736)
\curveto(582.72547618,29.76025737)(582.7580514,29.67952821)(582.7580514,29.58056988)
\curveto(582.7580514,29.48681989)(582.72547618,29.40869489)(582.66032573,29.3461949)
\curveto(582.59517529,29.2836949)(582.48742648,29.25244491)(582.3370793,29.25244491)
\lineto(579.32261834,29.25244491)
\curveto(579.17227116,29.25244491)(579.06452235,29.2836949)(578.9993719,29.3461949)
\curveto(578.93422146,29.40348656)(578.90164624,29.48161156)(578.90164624,29.58056988)
\curveto(578.90164624,29.67952821)(578.93422146,29.76025737)(578.9993719,29.82275736)
\curveto(579.06452235,29.88004903)(579.17227116,29.90869486)(579.32261834,29.90869486)
\lineto(580.17207991,29.90869486)
\lineto(580.17207991,38.36963171)
\lineto(579.32261834,38.36963171)
\curveto(579.17227116,38.36963171)(579.06452235,38.40088171)(578.9993719,38.46338171)
\curveto(578.93422146,38.5258817)(578.90164624,38.60661086)(578.90164624,38.70556919)
\curveto(578.90164624,38.79931918)(578.93422146,38.87744418)(578.9993719,38.93994417)
\curveto(579.06452235,39.00244417)(579.17227116,39.03369416)(579.32261834,39.03369416)
\closepath
\moveto(586.27617541,35.85400691)
\curveto(586.27617541,36.60921518)(586.01056206,37.25504847)(585.47933536,37.79150676)
\curveto(584.95312023,38.33317338)(584.30913314,38.6040067)(583.5473741,38.6040067)
\curveto(582.78060348,38.6040067)(582.13160482,38.33317338)(581.60037812,37.79150676)
\curveto(581.06915141,37.24984013)(580.80353806,36.60400685)(580.80353806,35.85400691)
\curveto(580.80353806,35.09879863)(581.06915141,34.45036118)(581.60037812,33.90869455)
\curveto(582.13160482,33.36702793)(582.78060348,33.09619461)(583.5473741,33.09619461)
\curveto(584.30412157,33.09619461)(584.94810866,33.36442376)(585.47933536,33.90088205)
\curveto(586.01056206,34.44254868)(586.27617541,35.0935903)(586.27617541,35.85400691)
\closepath
}
}
{
\newrgbcolor{curcolor}{0 0 0}
\pscustom[linestyle=none,fillstyle=solid,fillcolor=curcolor]
{
\newpath
\moveto(591.6661249,39.03369416)
\lineto(591.6661249,37.37744429)
\curveto(592.48802282,38.14827757)(593.10194047,38.64306919)(593.50787785,38.86181918)
\curveto(593.91882681,39.08577749)(594.29720055,39.19775665)(594.64299906,39.19775665)
\curveto(595.01886701,39.19775665)(595.36717131,39.06494416)(595.68791196,38.79931918)
\curveto(596.01366418,38.53890254)(596.1765403,38.34098588)(596.1765403,38.20556923)
\curveto(596.1765403,38.1066109)(596.14396507,38.02327757)(596.07881463,37.95556925)
\curveto(596.01867576,37.89306925)(595.94099638,37.86181925)(595.8457765,37.86181925)
\curveto(595.79566077,37.86181925)(595.75306241,37.86963175)(595.7179814,37.88525675)
\curveto(595.68290039,37.90609008)(595.61774994,37.96598591)(595.52253006,38.06494424)
\curveto(595.34712502,38.24723589)(595.19427205,38.37223588)(595.06397117,38.43994421)
\curveto(594.93367028,38.50765254)(594.80587517,38.5415067)(594.68058586,38.5415067)
\curveto(594.40494936,38.5415067)(594.07167978,38.42692338)(593.68077711,38.19775673)
\curveto(593.29488601,37.96859008)(592.62333528,37.40869429)(591.6661249,36.51806936)
\lineto(591.6661249,32.92431963)
\lineto(594.46258245,32.92431963)
\curveto(594.6179412,32.92431963)(594.7281958,32.89306963)(594.79334624,32.83056964)
\curveto(594.85849669,32.77327797)(594.89107191,32.69515298)(594.89107191,32.59619465)
\curveto(594.89107191,32.50244466)(594.85849669,32.42431967)(594.79334624,32.36181967)
\curveto(594.7281958,32.29931968)(594.6179412,32.26806968)(594.46258245,32.26806968)
\lineto(589.50864287,32.26806968)
\curveto(589.35829569,32.26806968)(589.25054688,32.29671551)(589.18539643,32.35400717)
\curveto(589.12024599,32.41650717)(589.08767076,32.49463216)(589.08767076,32.58838215)
\curveto(589.08767076,32.67692381)(589.1177402,32.74984047)(589.17787907,32.80713214)
\curveto(589.24302952,32.86963213)(589.35328412,32.90088213)(589.50864287,32.90088213)
\lineto(591.03466674,32.90088213)
\lineto(591.03466674,38.36963171)
\lineto(589.8694761,38.36963171)
\curveto(589.71912892,38.36963171)(589.61138011,38.40088171)(589.54622966,38.46338171)
\curveto(589.48107922,38.5258817)(589.448504,38.60661086)(589.448504,38.70556919)
\curveto(589.448504,38.79931918)(589.47857343,38.87744418)(589.5387123,38.93994417)
\curveto(589.60386275,39.00244417)(589.71411735,39.03369416)(589.8694761,39.03369416)
\closepath
}
}
{
\newrgbcolor{curcolor}{0 0 0}
\pscustom[linestyle=none,fillstyle=solid,fillcolor=curcolor]
{
\newpath
\moveto(605.07708936,35.49463193)
\lineto(598.64223006,35.49463193)
\curveto(598.75248466,34.64567366)(599.0932716,33.96077788)(599.66459089,33.43994459)
\curveto(600.24092174,32.92431963)(600.95256506,32.66650715)(601.79952084,32.66650715)
\curveto(602.27060867,32.66650715)(602.76424858,32.74723631)(603.28044056,32.90869463)
\curveto(603.79663255,33.07015295)(604.21760465,33.2836946)(604.54335688,33.54931958)
\curveto(604.63857676,33.62744457)(604.7212677,33.66650707)(604.79142972,33.66650707)
\curveto(604.87161488,33.66650707)(604.9417769,33.63265291)(605.00191577,33.56494458)
\curveto(605.06205465,33.50244458)(605.09212408,33.42692376)(605.09212408,33.3383821)
\curveto(605.09212408,33.24984044)(605.0520315,33.16390294)(604.97184634,33.08056962)
\curveto(604.73129085,32.82015297)(604.30280139,32.57536132)(603.68637795,32.34619467)
\curveto(603.07496608,32.12223636)(602.44601372,32.0102572)(601.79952084,32.0102572)
\curveto(600.71702115,32.0102572)(599.81243228,32.37744467)(599.08575425,33.11181961)
\curveto(598.36408778,33.85140289)(598.00325455,34.74463199)(598.00325455,35.79150691)
\curveto(598.00325455,36.74463184)(598.3415357,37.56234011)(599.01809801,38.24463172)
\curveto(599.6996719,38.92692334)(600.5416161,39.26806915)(601.54393064,39.26806915)
\curveto(602.5763146,39.26806915)(603.42577617,38.91650667)(604.09231534,38.21338173)
\curveto(604.7588545,37.51546511)(605.08711251,36.60921518)(605.07708936,35.49463193)
\closepath
\moveto(604.43811385,36.15869438)
\curveto(604.31282453,36.88265266)(603.98206074,37.47119428)(603.44582246,37.92431925)
\curveto(602.91459576,38.37744421)(602.28063182,38.6040067)(601.54393064,38.6040067)
\curveto(600.80722945,38.6040067)(600.17326551,38.38004838)(599.64203881,37.93213175)
\curveto(599.11081211,37.48421512)(598.78004831,36.89306933)(598.64974742,36.15869438)
\closepath
}
}
{
\newrgbcolor{curcolor}{0 0 0}
\pscustom[linestyle=none,fillstyle=solid,fillcolor=curcolor]
{
\newpath
\moveto(612.98535412,38.6040067)
\curveto(612.98535412,38.75504835)(613.01542355,38.86442334)(613.07556242,38.93213167)
\curveto(613.1357013,38.99984)(613.21087489,39.03369416)(613.30108319,39.03369416)
\curveto(613.39630307,39.03369416)(613.47398245,38.99984)(613.53412132,38.93213167)
\curveto(613.59426019,38.86442334)(613.62432963,38.74984002)(613.62432963,38.5883817)
\lineto(613.62432963,37.46338178)
\curveto(613.62432963,37.3071318)(613.59426019,37.19515264)(613.53412132,37.12744431)
\curveto(613.47398245,37.05973598)(613.39630307,37.02588182)(613.30108319,37.02588182)
\curveto(613.21588646,37.02588182)(613.14321865,37.05452765)(613.08307978,37.11181931)
\curveto(613.02795248,37.16911097)(612.99537726,37.26286097)(612.98535412,37.39306929)
\curveto(612.95528468,37.70556927)(612.79992593,37.96338175)(612.51927786,38.16650673)
\curveto(612.1083289,38.45817337)(611.56457327,38.6040067)(610.88801096,38.6040067)
\curveto(610.18137921,38.6040067)(609.632612,38.45556921)(609.24170934,38.15869423)
\curveto(608.94602655,37.93473591)(608.79818516,37.68473593)(608.79818516,37.40869429)
\curveto(608.79818516,37.09619431)(608.9735902,36.83577766)(609.32440029,36.62744435)
\curveto(609.56495577,36.48161102)(610.02100889,36.36963187)(610.69255962,36.29150687)
\curveto(611.56958484,36.19254855)(612.17849092,36.08056939)(612.51927786,35.9555694)
\curveto(613.00540041,35.77327775)(613.36623364,35.5206736)(613.60177755,35.19775696)
\curveto(613.84233304,34.87484031)(613.96261078,34.52588201)(613.96261078,34.15088203)
\curveto(613.96261078,33.59359041)(613.70451479,33.09619461)(613.18832281,32.65869465)
\curveto(612.67213082,32.22640301)(611.91538335,32.0102572)(610.91808039,32.0102572)
\curveto(609.92077743,32.0102572)(609.10389109,32.27327801)(608.46742136,32.79931964)
\curveto(608.46742136,32.62223632)(608.45739821,32.50765299)(608.43735192,32.45556966)
\curveto(608.41730563,32.40348633)(608.37971884,32.3592155)(608.32459154,32.32275717)
\curveto(608.27447581,32.28629884)(608.21684273,32.26806968)(608.15169228,32.26806968)
\curveto(608.06148397,32.26806968)(607.98631038,32.30192384)(607.92617151,32.36963217)
\curveto(607.86603264,32.4373405)(607.8359632,32.54931966)(607.8359632,32.70556964)
\lineto(607.8359632,34.05713204)
\curveto(607.8359632,34.21338203)(607.86352685,34.32536119)(607.91865415,34.39306952)
\curveto(607.97879303,34.46077784)(608.0564724,34.49463201)(608.15169228,34.49463201)
\curveto(608.24190059,34.49463201)(608.31707418,34.46077784)(608.37721305,34.39306952)
\curveto(608.4423635,34.33056952)(608.47493872,34.24463203)(608.47493872,34.13525704)
\curveto(608.47493872,33.89567372)(608.5325718,33.6951529)(608.64783798,33.53369458)
\curveto(608.82324302,33.2836946)(609.1013853,33.07536128)(609.48226482,32.90869463)
\curveto(609.86815592,32.74723631)(610.33924375,32.66650715)(610.89552832,32.66650715)
\curveto(611.71742623,32.66650715)(612.3288381,32.8253613)(612.72976391,33.14306961)
\curveto(613.13068972,33.46077792)(613.33115263,33.7967154)(613.33115263,34.15088203)
\curveto(613.33115263,34.557132)(613.12818394,34.88265281)(612.72224655,35.12744446)
\curveto(612.31129759,35.37223611)(611.71241466,35.5362986)(610.92559775,35.61963192)
\curveto(610.14379242,35.70296525)(609.58249628,35.81234024)(609.24170934,35.9477569)
\curveto(608.9009224,36.08317356)(608.63530904,36.28629854)(608.44486928,36.55713185)
\curveto(608.25442952,36.82796517)(608.15920964,37.11963181)(608.15920964,37.43213179)
\curveto(608.15920964,37.99463174)(608.42482299,38.43994421)(608.95604969,38.76806918)
\curveto(609.4872764,39.10140249)(610.12124034,39.26806915)(610.85794152,39.26806915)
\curveto(611.72995516,39.26806915)(612.4390927,39.046715)(612.98535412,38.6040067)
\closepath
}
}
{
\newrgbcolor{curcolor}{0 0 0}
\pscustom[linestyle=none,fillstyle=solid,fillcolor=curcolor]
{
\newpath
\moveto(622.22419141,38.6040067)
\curveto(622.22419141,38.75504835)(622.25426085,38.86442334)(622.31439972,38.93213167)
\curveto(622.37453859,38.99984)(622.44971218,39.03369416)(622.53992049,39.03369416)
\curveto(622.63514037,39.03369416)(622.71281974,38.99984)(622.77295862,38.93213167)
\curveto(622.83309749,38.86442334)(622.86316692,38.74984002)(622.86316692,38.5883817)
\lineto(622.86316692,37.46338178)
\curveto(622.86316692,37.3071318)(622.83309749,37.19515264)(622.77295862,37.12744431)
\curveto(622.71281974,37.05973598)(622.63514037,37.02588182)(622.53992049,37.02588182)
\curveto(622.45472375,37.02588182)(622.38205595,37.05452765)(622.32191708,37.11181931)
\curveto(622.26678978,37.16911097)(622.23421455,37.26286097)(622.22419141,37.39306929)
\curveto(622.19412197,37.70556927)(622.03876322,37.96338175)(621.75811515,38.16650673)
\curveto(621.34716619,38.45817337)(620.80341056,38.6040067)(620.12684825,38.6040067)
\curveto(619.4202165,38.6040067)(618.8714493,38.45556921)(618.48054663,38.15869423)
\curveto(618.18486384,37.93473591)(618.03702245,37.68473593)(618.03702245,37.40869429)
\curveto(618.03702245,37.09619431)(618.21242749,36.83577766)(618.56323758,36.62744435)
\curveto(618.80379307,36.48161102)(619.25984618,36.36963187)(619.93139692,36.29150687)
\curveto(620.80842213,36.19254855)(621.41732821,36.08056939)(621.75811515,35.9555694)
\curveto(622.2442377,35.77327775)(622.60507093,35.5206736)(622.84061485,35.19775696)
\curveto(623.08117033,34.87484031)(623.20144808,34.52588201)(623.20144808,34.15088203)
\curveto(623.20144808,33.59359041)(622.94335209,33.09619461)(622.4271601,32.65869465)
\curveto(621.91096812,32.22640301)(621.15422065,32.0102572)(620.15691769,32.0102572)
\curveto(619.15961473,32.0102572)(618.34272838,32.27327801)(617.70625865,32.79931964)
\curveto(617.70625865,32.62223632)(617.69623551,32.50765299)(617.67618922,32.45556966)
\curveto(617.65614293,32.40348633)(617.61855613,32.3592155)(617.56342883,32.32275717)
\curveto(617.51331311,32.28629884)(617.45568002,32.26806968)(617.39052958,32.26806968)
\curveto(617.30032127,32.26806968)(617.22514768,32.30192384)(617.16500881,32.36963217)
\curveto(617.10486993,32.4373405)(617.0748005,32.54931966)(617.0748005,32.70556964)
\lineto(617.0748005,34.05713204)
\curveto(617.0748005,34.21338203)(617.10236415,34.32536119)(617.15749145,34.39306952)
\curveto(617.21763032,34.46077784)(617.2953097,34.49463201)(617.39052958,34.49463201)
\curveto(617.48073788,34.49463201)(617.55591147,34.46077784)(617.61605035,34.39306952)
\curveto(617.68120079,34.33056952)(617.71377601,34.24463203)(617.71377601,34.13525704)
\curveto(617.71377601,33.89567372)(617.7714091,33.6951529)(617.88667527,33.53369458)
\curveto(618.06208031,33.2836946)(618.3402226,33.07536128)(618.72110212,32.90869463)
\curveto(619.10699321,32.74723631)(619.57808104,32.66650715)(620.13436561,32.66650715)
\curveto(620.95626353,32.66650715)(621.56767539,32.8253613)(621.9686012,33.14306961)
\curveto(622.36952702,33.46077792)(622.56998992,33.7967154)(622.56998992,34.15088203)
\curveto(622.56998992,34.557132)(622.36702123,34.88265281)(621.96108384,35.12744446)
\curveto(621.55013489,35.37223611)(620.95125195,35.5362986)(620.16443504,35.61963192)
\curveto(619.38262971,35.70296525)(618.82133357,35.81234024)(618.48054663,35.9477569)
\curveto(618.13975969,36.08317356)(617.87414634,36.28629854)(617.68370658,36.55713185)
\curveto(617.49326682,36.82796517)(617.39804693,37.11963181)(617.39804693,37.43213179)
\curveto(617.39804693,37.99463174)(617.66366029,38.43994421)(618.19488699,38.76806918)
\curveto(618.72611369,39.10140249)(619.36007763,39.26806915)(620.09677881,39.26806915)
\curveto(620.96879246,39.26806915)(621.67792999,39.046715)(622.22419141,38.6040067)
\closepath
}
}
{
\newrgbcolor{curcolor}{0 0 0}
\pscustom[linestyle=none,fillstyle=solid,fillcolor=curcolor]
{
\newpath
\moveto(632.79359417,35.49463193)
\lineto(626.35873487,35.49463193)
\curveto(626.46898947,34.64567366)(626.80977641,33.96077788)(627.3810957,33.43994459)
\curveto(627.95742655,32.92431963)(628.66906987,32.66650715)(629.51602565,32.66650715)
\curveto(629.98711348,32.66650715)(630.48075339,32.74723631)(630.99694537,32.90869463)
\curveto(631.51313736,33.07015295)(631.93410946,33.2836946)(632.25986169,33.54931958)
\curveto(632.35508157,33.62744457)(632.43777251,33.66650707)(632.50793453,33.66650707)
\curveto(632.58811969,33.66650707)(632.65828171,33.63265291)(632.71842058,33.56494458)
\curveto(632.77855946,33.50244458)(632.80862889,33.42692376)(632.80862889,33.3383821)
\curveto(632.80862889,33.24984044)(632.76853631,33.16390294)(632.68835115,33.08056962)
\curveto(632.44779566,32.82015297)(632.0193062,32.57536132)(631.40288276,32.34619467)
\curveto(630.79147089,32.12223636)(630.16251853,32.0102572)(629.51602565,32.0102572)
\curveto(628.43352596,32.0102572)(627.52893709,32.37744467)(626.80225906,33.11181961)
\curveto(626.08059259,33.85140289)(625.71975936,34.74463199)(625.71975936,35.79150691)
\curveto(625.71975936,36.74463184)(626.05804051,37.56234011)(626.73460282,38.24463172)
\curveto(627.41617671,38.92692334)(628.25812091,39.26806915)(629.26043545,39.26806915)
\curveto(630.29281941,39.26806915)(631.14228098,38.91650667)(631.80882015,38.21338173)
\curveto(632.47535931,37.51546511)(632.80361732,36.60921518)(632.79359417,35.49463193)
\closepath
\moveto(632.15461866,36.15869438)
\curveto(632.02932934,36.88265266)(631.69856555,37.47119428)(631.16232727,37.92431925)
\curveto(630.63110057,38.37744421)(629.99713663,38.6040067)(629.26043545,38.6040067)
\curveto(628.52373426,38.6040067)(627.88977032,38.38004838)(627.35854362,37.93213175)
\curveto(626.82731692,37.48421512)(626.49655312,36.89306933)(626.36625223,36.15869438)
\closepath
}
}
{
\newrgbcolor{curcolor}{0 0 0}
\pscustom[linestyle=none,fillstyle=solid,fillcolor=curcolor]
{
\newpath
\moveto(641.70918503,42.07275643)
\lineto(641.70918503,32.92431963)
\lineto(642.55112924,32.92431963)
\curveto(642.70648799,32.92431963)(642.81674259,32.89306963)(642.88189303,32.83056964)
\curveto(642.94704348,32.77327797)(642.9796187,32.69515298)(642.9796187,32.59619465)
\curveto(642.9796187,32.50244466)(642.94704348,32.42431967)(642.88189303,32.36181967)
\curveto(642.81674259,32.29931968)(642.70648799,32.26806968)(642.55112924,32.26806968)
\lineto(641.07020952,32.26806968)
\lineto(641.07020952,33.57275708)
\curveto(640.33851991,32.53109049)(639.40887318,32.0102572)(638.28126933,32.0102572)
\curveto(637.70995005,32.0102572)(637.16118284,32.16650719)(636.63496771,32.47900716)
\curveto(636.11376415,32.79671547)(635.70030941,33.24723627)(635.39460348,33.83056956)
\curveto(635.09390912,34.41390285)(634.94356194,35.0154653)(634.94356194,35.63525692)
\curveto(634.94356194,36.26025687)(635.09390912,36.86181933)(635.39460348,37.43994429)
\curveto(635.70030941,38.02327757)(636.11376415,38.47379837)(636.63496771,38.79150668)
\curveto(637.16118284,39.10921499)(637.71245583,39.26806915)(638.28878669,39.26806915)
\curveto(639.39133267,39.26806915)(640.31847362,38.74723585)(641.07020952,37.70556927)
\lineto(641.07020952,41.40869398)
\lineto(640.22826531,41.40869398)
\curveto(640.07290656,41.40869398)(639.96265196,41.43994398)(639.89750151,41.50244398)
\curveto(639.83235107,41.56494397)(639.79977585,41.64567313)(639.79977585,41.74463146)
\curveto(639.79977585,41.83838145)(639.83235107,41.91650645)(639.89750151,41.97900644)
\curveto(639.96265196,42.04150644)(640.07290656,42.07275643)(640.22826531,42.07275643)
\closepath
\moveto(641.07020952,35.63525692)
\curveto(641.07020952,36.46859019)(640.79958459,37.17171514)(640.25833474,37.74463176)
\curveto(639.7170849,38.31754839)(639.07309781,38.6040067)(638.32637348,38.6040067)
\curveto(637.57463758,38.6040067)(636.92814471,38.31754839)(636.38689486,37.74463176)
\curveto(635.84564501,37.17171514)(635.57502009,36.46859019)(635.57502009,35.63525692)
\curveto(635.57502009,34.80713199)(635.84564501,34.10400704)(636.38689486,33.52588208)
\curveto(636.92814471,32.95296546)(637.57463758,32.66650715)(638.32637348,32.66650715)
\curveto(639.07309781,32.66650715)(639.7170849,32.95296546)(640.25833474,33.52588208)
\curveto(640.79958459,34.10400704)(641.07020952,34.80713199)(641.07020952,35.63525692)
\closepath
}
}
{
\newrgbcolor{curcolor}{0 0 0}
\pscustom[linestyle=none,fillstyle=solid,fillcolor=curcolor]
{
\newpath
\moveto(540.08200308,22.07275795)
\lineto(540.08200308,12.92432115)
\lineto(540.92394729,12.92432115)
\curveto(541.07930604,12.92432115)(541.18956064,12.89307115)(541.25471109,12.83057115)
\curveto(541.31986153,12.77327949)(541.35243676,12.6951545)(541.35243676,12.59619617)
\curveto(541.35243676,12.50244618)(541.31986153,12.42432118)(541.25471109,12.36182119)
\curveto(541.18956064,12.29932119)(541.07930604,12.26807119)(540.92394729,12.26807119)
\lineto(539.44302757,12.26807119)
\lineto(539.44302757,13.5727586)
\curveto(538.71133796,12.53109201)(537.78169123,12.01025871)(536.65408738,12.01025871)
\curveto(536.0827681,12.01025871)(535.53400089,12.1665087)(535.00778576,12.47900868)
\curveto(534.48658221,12.79671699)(534.07312746,13.24723779)(533.76742153,13.83057108)
\curveto(533.46672717,14.41390437)(533.31637999,15.01546682)(533.31637999,15.63525844)
\curveto(533.31637999,16.26025839)(533.46672717,16.86182085)(533.76742153,17.4399458)
\curveto(534.07312746,18.02327909)(534.48658221,18.47379989)(535.00778576,18.7915082)
\curveto(535.53400089,19.10921651)(536.08527389,19.26807066)(536.66160474,19.26807066)
\curveto(537.76415073,19.26807066)(538.69129167,18.74723737)(539.44302757,17.70557078)
\lineto(539.44302757,21.4086955)
\lineto(538.60108336,21.4086955)
\curveto(538.44572461,21.4086955)(538.33547001,21.4399455)(538.27031957,21.50244549)
\curveto(538.20516912,21.56494549)(538.1725939,21.64567465)(538.1725939,21.74463298)
\curveto(538.1725939,21.83838297)(538.20516912,21.91650796)(538.27031957,21.97900796)
\curveto(538.33547001,22.04150795)(538.44572461,22.07275795)(538.60108336,22.07275795)
\closepath
\moveto(539.44302757,15.63525844)
\curveto(539.44302757,16.46859171)(539.17240265,17.17171666)(538.6311528,17.74463328)
\curveto(538.08990295,18.3175499)(537.44591586,18.60400821)(536.69919154,18.60400821)
\curveto(535.94745564,18.60400821)(535.30096276,18.3175499)(534.75971292,17.74463328)
\curveto(534.21846307,17.17171666)(533.94783815,16.46859171)(533.94783815,15.63525844)
\curveto(533.94783815,14.8071335)(534.21846307,14.10400856)(534.75971292,13.5258836)
\curveto(535.30096276,12.95296698)(535.94745564,12.66650866)(536.69919154,12.66650866)
\curveto(537.44591586,12.66650866)(538.08990295,12.95296698)(538.6311528,13.5258836)
\curveto(539.17240265,14.10400856)(539.44302757,14.8071335)(539.44302757,15.63525844)
\closepath
}
}
{
\newrgbcolor{curcolor}{0 0 0}
\pscustom[linestyle=none,fillstyle=solid,fillcolor=curcolor]
{
\newpath
\moveto(548.0428867,12.26807119)
\lineto(548.0428867,13.21338362)
\curveto(547.1257689,12.41130035)(546.14600644,12.01025871)(545.10359933,12.01025871)
\curveto(544.34685186,12.01025871)(543.75548628,12.20817537)(543.32950261,12.60400867)
\curveto(542.90351893,13.00505031)(542.69052709,13.4946336)(542.69052709,14.07275856)
\curveto(542.69052709,14.70817518)(542.97117516,15.26286263)(543.5324713,15.73682093)
\curveto(544.09376744,16.21077923)(544.91315957,16.44775838)(545.99064769,16.44775838)
\curveto(546.28131891,16.44775838)(546.59704798,16.42692505)(546.93783493,16.38525838)
\curveto(547.27862187,16.34880005)(547.64697246,16.28890422)(548.0428867,16.2055709)
\lineto(548.0428867,17.26807082)
\curveto(548.0428867,17.62744579)(547.88251637,17.93994576)(547.56177572,18.20557074)
\curveto(547.24103507,18.47119572)(546.7599241,18.60400821)(546.1184428,18.60400821)
\curveto(545.62730867,18.60400821)(544.93821743,18.45557073)(544.05116907,18.15869575)
\curveto(543.89079875,18.10661242)(543.78806151,18.08057075)(543.74295735,18.08057075)
\curveto(543.66277219,18.08057075)(543.59261017,18.11182075)(543.5324713,18.17432075)
\curveto(543.477344,18.23682074)(543.44978035,18.31494574)(543.44978035,18.40869573)
\curveto(543.44978035,18.49723739)(543.47483822,18.56754988)(543.52495394,18.61963321)
\curveto(543.59511596,18.69775821)(543.87826981,18.80452903)(544.37441551,18.93994569)
\curveto(545.15622084,19.15869567)(545.74758642,19.26807066)(546.14851223,19.26807066)
\curveto(546.94535228,19.26807066)(547.56678729,19.06234151)(548.01281726,18.65088321)
\curveto(548.45884723,18.24463324)(548.68186221,17.78369578)(548.68186221,17.26807082)
\lineto(548.68186221,12.92432115)
\lineto(549.52380642,12.92432115)
\curveto(549.67916517,12.92432115)(549.78941977,12.89307115)(549.85457022,12.83057115)
\curveto(549.91972066,12.77327949)(549.95229588,12.6951545)(549.95229588,12.59619617)
\curveto(549.95229588,12.50244618)(549.91972066,12.42432118)(549.85457022,12.36182119)
\curveto(549.78941977,12.29932119)(549.67916517,12.26807119)(549.52380642,12.26807119)
\closepath
\moveto(548.0428867,15.53369595)
\curveto(547.74720391,15.62223761)(547.43398062,15.68734177)(547.10321682,15.72900843)
\curveto(546.77245303,15.7706751)(546.42414873,15.79150843)(546.05830392,15.79150843)
\curveto(545.14118613,15.79150843)(544.42453124,15.58577928)(543.90833925,15.17432097)
\curveto(543.51743658,14.86702933)(543.32198525,14.49984186)(543.32198525,14.07275856)
\curveto(543.32198525,13.67692525)(543.46982664,13.34359195)(543.76550943,13.07275863)
\curveto(544.06620379,12.80192532)(544.50221061,12.66650866)(545.07352989,12.66650866)
\curveto(545.61979132,12.66650866)(546.12596015,12.77848782)(546.59203641,13.00244614)
\curveto(547.06312424,13.23161279)(547.546741,13.59359193)(548.0428867,14.08838356)
\closepath
}
}
{
\newrgbcolor{curcolor}{0 0 0}
\pscustom[linestyle=none,fillstyle=solid,fillcolor=curcolor]
{
\newpath
\moveto(554.16953472,19.03369568)
\lineto(557.59745042,19.03369568)
\curveto(557.7477976,19.03369568)(557.85554641,19.00244568)(557.92069685,18.93994569)
\curveto(557.9858473,18.87744569)(558.01842252,18.79671653)(558.01842252,18.69775821)
\curveto(558.01842252,18.60400821)(557.9858473,18.52588322)(557.92069685,18.46338322)
\curveto(557.85554641,18.40088323)(557.7477976,18.36963323)(557.59745042,18.36963323)
\lineto(554.16953472,18.36963323)
\lineto(554.16953472,14.00244606)
\curveto(554.16953472,13.62223776)(554.31487032,13.30452945)(554.60554154,13.04932114)
\curveto(554.90122432,12.79411282)(555.33221957,12.66650866)(555.89852728,12.66650866)
\curveto(556.32451096,12.66650866)(556.78557565,12.73161283)(557.28172134,12.86182115)
\curveto(557.77786703,12.99723781)(558.16375813,13.14827946)(558.43939463,13.31494612)
\curveto(558.53962608,13.38265444)(558.62231703,13.41650861)(558.68746747,13.41650861)
\curveto(558.76765263,13.41650861)(558.83781465,13.38265444)(558.89795352,13.31494612)
\curveto(558.9580924,13.25244612)(558.98816183,13.17692529)(558.98816183,13.08838363)
\curveto(558.98816183,13.01025864)(558.95558661,12.93734198)(558.89043616,12.86963365)
\curveto(558.73006584,12.69775866)(558.33916317,12.51025868)(557.71772816,12.30713369)
\curveto(557.10130472,12.10921704)(556.50993915,12.01025871)(555.94363144,12.01025871)
\curveto(555.20693026,12.01025871)(554.62057626,12.1899462)(554.18456943,12.54932117)
\curveto(553.74856261,12.90869615)(553.5305592,13.39307111)(553.5305592,14.00244606)
\lineto(553.5305592,18.36963323)
\lineto(552.36536856,18.36963323)
\curveto(552.21502138,18.36963323)(552.10727257,18.40088323)(552.04212212,18.46338322)
\curveto(551.97697168,18.52588322)(551.94439645,18.60661238)(551.94439645,18.70557071)
\curveto(551.94439645,18.7993207)(551.97697168,18.87744569)(552.04212212,18.93994569)
\curveto(552.10727257,19.00244568)(552.21502138,19.03369568)(552.36536856,19.03369568)
\lineto(553.5305592,19.03369568)
\lineto(553.5305592,20.97119553)
\curveto(553.5305592,21.12744552)(553.56062864,21.23942468)(553.62076751,21.30713301)
\curveto(553.68090638,21.37484134)(553.75607997,21.4086955)(553.84628828,21.4086955)
\curveto(553.94150816,21.4086955)(554.01918754,21.37484134)(554.07932641,21.30713301)
\curveto(554.13946528,21.23942468)(554.16953472,21.12744552)(554.16953472,20.97119553)
\closepath
}
}
{
\newrgbcolor{curcolor}{0 0 0}
\pscustom[linestyle=none,fillstyle=solid,fillcolor=curcolor]
{
\newpath
\moveto(566.52055687,12.26807119)
\lineto(566.52055687,13.21338362)
\curveto(565.60343907,12.41130035)(564.62367661,12.01025871)(563.5812695,12.01025871)
\curveto(562.82452203,12.01025871)(562.23315645,12.20817537)(561.80717278,12.60400867)
\curveto(561.3811891,13.00505031)(561.16819726,13.4946336)(561.16819726,14.07275856)
\curveto(561.16819726,14.70817518)(561.44884533,15.26286263)(562.01014147,15.73682093)
\curveto(562.57143761,16.21077923)(563.39082974,16.44775838)(564.46831786,16.44775838)
\curveto(564.75898907,16.44775838)(565.07471815,16.42692505)(565.41550509,16.38525838)
\curveto(565.75629203,16.34880005)(566.12464263,16.28890422)(566.52055687,16.2055709)
\lineto(566.52055687,17.26807082)
\curveto(566.52055687,17.62744579)(566.36018654,17.93994576)(566.03944589,18.20557074)
\curveto(565.71870524,18.47119572)(565.23759426,18.60400821)(564.59611296,18.60400821)
\curveto(564.10497884,18.60400821)(563.4158876,18.45557073)(562.52883924,18.15869575)
\curveto(562.36846891,18.10661242)(562.26573167,18.08057075)(562.22062752,18.08057075)
\curveto(562.14044236,18.08057075)(562.07028034,18.11182075)(562.01014147,18.17432075)
\curveto(561.95501417,18.23682074)(561.92745052,18.31494574)(561.92745052,18.40869573)
\curveto(561.92745052,18.49723739)(561.95250838,18.56754988)(562.00262411,18.61963321)
\curveto(562.07278613,18.69775821)(562.35593998,18.80452903)(562.85208568,18.93994569)
\curveto(563.63389101,19.15869567)(564.22525659,19.26807066)(564.6261824,19.26807066)
\curveto(565.42302245,19.26807066)(566.04445746,19.06234151)(566.49048743,18.65088321)
\curveto(566.9365174,18.24463324)(567.15953238,17.78369578)(567.15953238,17.26807082)
\lineto(567.15953238,12.92432115)
\lineto(568.00147659,12.92432115)
\curveto(568.15683534,12.92432115)(568.26708994,12.89307115)(568.33224038,12.83057115)
\curveto(568.39739083,12.77327949)(568.42996605,12.6951545)(568.42996605,12.59619617)
\curveto(568.42996605,12.50244618)(568.39739083,12.42432118)(568.33224038,12.36182119)
\curveto(568.26708994,12.29932119)(568.15683534,12.26807119)(568.00147659,12.26807119)
\closepath
\moveto(566.52055687,15.53369595)
\curveto(566.22487408,15.62223761)(565.91165079,15.68734177)(565.58088699,15.72900843)
\curveto(565.2501232,15.7706751)(564.9018189,15.79150843)(564.53597409,15.79150843)
\curveto(563.61885629,15.79150843)(562.9022014,15.58577928)(562.38600942,15.17432097)
\curveto(561.99510675,14.86702933)(561.79965542,14.49984186)(561.79965542,14.07275856)
\curveto(561.79965542,13.67692525)(561.94749681,13.34359195)(562.2431796,13.07275863)
\curveto(562.54387396,12.80192532)(562.97988078,12.66650866)(563.55120006,12.66650866)
\curveto(564.09746148,12.66650866)(564.60363032,12.77848782)(565.06970658,13.00244614)
\curveto(565.54079441,13.23161279)(566.02441117,13.59359193)(566.52055687,14.08838356)
\closepath
}
}
{
\newrgbcolor{curcolor}{0 0 0}
\pscustom[linewidth=0.88168443,linecolor=curcolor]
{
\newpath
\moveto(667.34348599,134.62275321)
\lineto(692.47178835,134.62275321)
\lineto(692.47178835,135.12783415)
\lineto(692.47178835,18.45520549)
\lineto(648.10459465,18.45520549)
}
}
{
\newrgbcolor{curcolor}{0 0 0}
\pscustom[linestyle=none,fillstyle=solid,fillcolor=curcolor]
{
\newpath
\moveto(656.92143892,18.45520549)
\lineto(660.44817663,21.9819432)
\lineto(648.10459465,18.45520549)
\lineto(660.44817663,14.92846778)
\closepath
}
}
{
\newrgbcolor{curcolor}{0 0 0}
\pscustom[linewidth=0.94046342,linecolor=curcolor]
{
\newpath
\moveto(656.92143892,18.45520549)
\lineto(660.44817663,21.9819432)
\lineto(648.10459465,18.45520549)
\lineto(660.44817663,14.92846778)
\closepath
}
}
{
\newrgbcolor{curcolor}{0 0 0}
\pscustom[linestyle=none,fillstyle=solid,fillcolor=curcolor]
{
\newpath
\moveto(728.07844001,167.26715612)
\lineto(727.23469072,167.26715612)
\lineto(722.39875729,175.44683674)
\lineto(722.39875729,167.92340557)
\lineto(723.60969377,167.92340557)
\curveto(723.77115197,167.92340557)(723.88573521,167.8921556)(723.95344348,167.82965565)
\curveto(724.02115176,167.77236403)(724.0550059,167.6942391)(724.0550059,167.59528085)
\curveto(724.0550059,167.50153092)(724.02115176,167.42340599)(723.95344348,167.36090604)
\curveto(723.88573521,167.2984061)(723.77115197,167.26715612)(723.60969377,167.26715612)
\lineto(721.1956333,167.26715612)
\curveto(721.03417511,167.26715612)(720.91959187,167.2984061)(720.85188359,167.36090604)
\curveto(720.78417532,167.42340599)(720.75032118,167.50153092)(720.75032118,167.59528085)
\curveto(720.75032118,167.6942391)(720.78417532,167.77236403)(720.85188359,167.82965565)
\curveto(720.91959187,167.8921556)(721.03417511,167.92340557)(721.1956333,167.92340557)
\lineto(721.74250784,167.92340557)
\lineto(721.74250784,175.75152399)
\lineto(720.85969609,175.75152399)
\curveto(720.70344622,175.75152399)(720.59146714,175.78016979)(720.52375887,175.83746141)
\curveto(720.45605059,175.89996136)(720.42219645,175.98069046)(720.42219645,176.07964871)
\curveto(720.42219645,176.17860696)(720.45605059,176.25673189)(720.52375887,176.31402351)
\curveto(720.59146714,176.37652346)(720.70344622,176.40777343)(720.85969609,176.40777343)
\lineto(722.57844464,176.40777343)
\lineto(727.42219057,168.21246783)
\lineto(727.42219057,175.75152399)
\lineto(726.21125409,175.75152399)
\curveto(726.05500422,175.75152399)(725.94302514,175.78016979)(725.87531687,175.83746141)
\curveto(725.80760859,175.89996136)(725.77375445,175.98069046)(725.77375445,176.07964871)
\curveto(725.77375445,176.17860696)(725.80500443,176.25673189)(725.86750437,176.31402351)
\curveto(725.93521265,176.37652346)(726.04979589,176.40777343)(726.21125409,176.40777343)
\lineto(728.63312705,176.40777343)
\curveto(728.78937692,176.40777343)(728.90135599,176.37652346)(728.96906427,176.31402351)
\curveto(729.03677254,176.25673189)(729.07062668,176.17860696)(729.07062668,176.07964871)
\curveto(729.07062668,175.98069046)(729.03677254,175.89996136)(728.96906427,175.83746141)
\curveto(728.90135599,175.78016979)(728.78937692,175.75152399)(728.63312705,175.75152399)
\lineto(728.07844001,175.75152399)
\closepath
}
}
{
\newrgbcolor{curcolor}{0 0 0}
\pscustom[linestyle=none,fillstyle=solid,fillcolor=curcolor]
{
\newpath
\moveto(737.64093182,175.75152399)
\lineto(737.64093182,170.2671536)
\curveto(737.64093182,169.34527937)(737.33364041,168.57184252)(736.71905759,167.94684305)
\curveto(736.10447478,167.32184358)(735.3622879,167.00934384)(734.49249697,167.00934384)
\curveto(733.90916412,167.00934384)(733.39093539,167.1369479)(732.93781077,167.39215602)
\curveto(732.48468615,167.64736413)(732.09926981,168.03278048)(731.78156175,168.54840504)
\curveto(731.46385368,169.06923794)(731.30499965,169.64215412)(731.30499965,170.2671536)
\lineto(731.30499965,175.75152399)
\lineto(730.75031261,175.75152399)
\curveto(730.59406274,175.75152399)(730.48208367,175.78016979)(730.4143754,175.83746141)
\curveto(730.34666712,175.89996136)(730.31281298,175.98069046)(730.31281298,176.07964871)
\curveto(730.31281298,176.17860696)(730.34927128,176.26194022)(730.42218789,176.3296485)
\curveto(730.48468784,176.38173179)(730.59406274,176.40777343)(730.75031261,176.40777343)
\lineto(733.17218558,176.40777343)
\curveto(733.32843544,176.40777343)(733.44041452,176.37652346)(733.50812279,176.31402351)
\curveto(733.57583107,176.25673189)(733.60968521,176.17860696)(733.60968521,176.07964871)
\curveto(733.60968521,175.98069046)(733.57583107,175.89996136)(733.50812279,175.83746141)
\curveto(733.44041452,175.78016979)(733.32843544,175.75152399)(733.17218558,175.75152399)
\lineto(731.96124909,175.75152399)
\lineto(731.96124909,170.2671536)
\curveto(731.96124909,169.54319587)(732.20864472,168.92861306)(732.70343597,168.42340515)
\curveto(733.19822722,167.91819724)(733.78416423,167.66559329)(734.46124699,167.66559329)
\curveto(734.89353829,167.66559329)(735.28676713,167.76455154)(735.6409335,167.96246804)
\curveto(736.0003082,168.16038454)(736.31280793,168.46507178)(736.57843271,168.87652977)
\curveto(736.84926582,169.28798776)(736.98468237,169.75152903)(736.98468237,170.2671536)
\lineto(736.98468237,175.75152399)
\lineto(735.77374589,175.75152399)
\curveto(735.61749602,175.75152399)(735.50551695,175.78016979)(735.43780867,175.83746141)
\curveto(735.37010039,175.89996136)(735.33624626,175.98069046)(735.33624626,176.07964871)
\curveto(735.33624626,176.17860696)(735.37010039,176.25673189)(735.43780867,176.31402351)
\curveto(735.50551695,176.37652346)(735.61749602,176.40777343)(735.77374589,176.40777343)
\lineto(738.19561885,176.40777343)
\curveto(738.35186872,176.40777343)(738.46384779,176.37652346)(738.53155607,176.31402351)
\curveto(738.59926434,176.25673189)(738.63311848,176.17860696)(738.63311848,176.07964871)
\curveto(738.63311848,175.98069046)(738.59926434,175.89996136)(738.53155607,175.83746141)
\curveto(738.46384779,175.78016979)(738.35186872,175.75152399)(738.19561885,175.75152399)
\closepath
}
}
{
\newrgbcolor{curcolor}{0 0 0}
\pscustom[linestyle=none,fillstyle=solid,fillcolor=curcolor]
{
\newpath
\moveto(742.92998971,175.75152399)
\lineto(742.92998971,167.92340557)
\lineto(747.28936105,167.92340557)
\lineto(747.28936105,170.53277837)
\curveto(747.28936105,170.68902824)(747.32061102,170.80100732)(747.38311097,170.86871559)
\curveto(747.44561092,170.93642387)(747.52634002,170.97027801)(747.62529827,170.97027801)
\curveto(747.71904819,170.97027801)(747.79717312,170.93642387)(747.85967307,170.86871559)
\curveto(747.92217302,170.80621564)(747.95342299,170.69423657)(747.95342299,170.53277837)
\lineto(747.95342299,167.26715612)
\lineto(740.70342909,167.26715612)
\curveto(740.54717922,167.26715612)(740.43520015,167.2984061)(740.36749187,167.36090604)
\curveto(740.29978359,167.42340599)(740.26592945,167.50153092)(740.26592945,167.59528085)
\curveto(740.26592945,167.6942391)(740.29978359,167.77236403)(740.36749187,167.82965565)
\curveto(740.43520015,167.8921556)(740.54717922,167.92340557)(740.70342909,167.92340557)
\lineto(742.26592777,167.92340557)
\lineto(742.26592777,175.75152399)
\lineto(740.70342909,175.75152399)
\curveto(740.54717922,175.75152399)(740.43520015,175.78016979)(740.36749187,175.83746141)
\curveto(740.29978359,175.89996136)(740.26592945,175.98069046)(740.26592945,176.07964871)
\curveto(740.26592945,176.17860696)(740.29978359,176.25673189)(740.36749187,176.31402351)
\curveto(740.43520015,176.37652346)(740.54717922,176.40777343)(740.70342909,176.40777343)
\lineto(744.4924884,176.40777343)
\curveto(744.64873827,176.40777343)(744.76071734,176.37652346)(744.82842562,176.31402351)
\curveto(744.89613389,176.25673189)(744.92998803,176.17860696)(744.92998803,176.07964871)
\curveto(744.92998803,175.98069046)(744.89613389,175.89996136)(744.82842562,175.83746141)
\curveto(744.76071734,175.78016979)(744.64873827,175.75152399)(744.4924884,175.75152399)
\closepath
}
}
{
\newrgbcolor{curcolor}{0 0 0}
\pscustom[linestyle=none,fillstyle=solid,fillcolor=curcolor]
{
\newpath
\moveto(752.53154489,175.75152399)
\lineto(752.53154489,167.92340557)
\lineto(756.89091622,167.92340557)
\lineto(756.89091622,170.53277837)
\curveto(756.89091622,170.68902824)(756.92216619,170.80100732)(756.98466614,170.86871559)
\curveto(757.04716609,170.93642387)(757.12789519,170.97027801)(757.22685344,170.97027801)
\curveto(757.32060336,170.97027801)(757.39872829,170.93642387)(757.46122824,170.86871559)
\curveto(757.52372819,170.80621564)(757.55497816,170.69423657)(757.55497816,170.53277837)
\lineto(757.55497816,167.26715612)
\lineto(750.30498426,167.26715612)
\curveto(750.14873439,167.26715612)(750.03675532,167.2984061)(749.96904704,167.36090604)
\curveto(749.90133876,167.42340599)(749.86748463,167.50153092)(749.86748463,167.59528085)
\curveto(749.86748463,167.6942391)(749.90133876,167.77236403)(749.96904704,167.82965565)
\curveto(750.03675532,167.8921556)(750.14873439,167.92340557)(750.30498426,167.92340557)
\lineto(751.86748294,167.92340557)
\lineto(751.86748294,175.75152399)
\lineto(750.30498426,175.75152399)
\curveto(750.14873439,175.75152399)(750.03675532,175.78016979)(749.96904704,175.83746141)
\curveto(749.90133876,175.89996136)(749.86748463,175.98069046)(749.86748463,176.07964871)
\curveto(749.86748463,176.17860696)(749.90133876,176.25673189)(749.96904704,176.31402351)
\curveto(750.03675532,176.37652346)(750.14873439,176.40777343)(750.30498426,176.40777343)
\lineto(754.09404357,176.40777343)
\curveto(754.25029344,176.40777343)(754.36227251,176.37652346)(754.42998079,176.31402351)
\curveto(754.49768907,176.25673189)(754.5315432,176.17860696)(754.5315432,176.07964871)
\curveto(754.5315432,175.98069046)(754.49768907,175.89996136)(754.42998079,175.83746141)
\curveto(754.36227251,175.78016979)(754.25029344,175.75152399)(754.09404357,175.75152399)
\closepath
}
}
{
\newrgbcolor{curcolor}{0 0 0}
\pscustom[linewidth=0.97718926,linecolor=curcolor]
{
\newpath
\moveto(718.42194898,172.50347258)
\lineto(666.80392064,172.50347258)
}
}
{
\newrgbcolor{curcolor}{0 0 0}
\pscustom[linestyle=none,fillstyle=solid,fillcolor=curcolor]
{
\newpath
\moveto(708.65005639,172.50347258)
\lineto(704.74129935,168.59471554)
\lineto(718.42194898,172.50347258)
\lineto(704.74129935,176.41222962)
\closepath
}
}
{
\newrgbcolor{curcolor}{0 0 0}
\pscustom[linewidth=1.04233524,linecolor=curcolor]
{
\newpath
\moveto(708.65005639,172.50347258)
\lineto(704.74129935,168.59471554)
\lineto(718.42194898,172.50347258)
\lineto(704.74129935,176.41222962)
\closepath
}
}
\end{pspicture}
}
    \captionsetup{width=0.75\linewidth}
    \caption{ARC Headers, as depicted in comments in the ZFS ARC code \cite[{module/zfs/arc.c}]{zfs}.}
    \label{fig:ARCHeaders}
\end{figure}

Internally the ARC is implemented as a hash table of headers, metadata structures that point to the actual contents of the blocks
stored in the ARC.
These header structures point to a set of buffer structures, which themselves point to ABD, ARC Buffer Data, structures
containing the actual data.
The first and most important of these buffers is the physical buffer, the ABD that contains the contents of the physical block on disk.
Every header that is in the L1 ARC will have at least this buffer, though it is freed when the header moves to the ghost cache.

Every time a block is requested from the ARC by a new consumer, a new entry is added to the consumer list, 
called the buf list within the ARC, with its own data pointer to an ABD containing the requested data.
A consumer is a higher level component of ZFS, such as a particular filesystem's DMU, 
that needs to retrieve the data pointed to by a block pointer.
This consumer buf structure is what the consumer receives back from the ARC when its data is available.

What the ABD that the buf structure points to looks like depends on what the block's data actually looks like on disk.
If the block requested is compressed or encrypted, than the ARC will need to allocate a new ABD to store the uncompressed data,
which will then to pointed to by the new buf structure before it is returned to the consumer.
If the data on disk is not compressed or encrypted, then the buf structure will just point to the physical ABD of the header,
as no changes are needed in order for it to be used by the consumer.

\subsection{ARC Buffer Data}
\label{chapter:ABD}
ARC data is stored in an \texttt{abd\_t}, an ARC Buffer Data structure, which allows the creation of both
large contiguous, or linear, buffers and buffers that might consist internally of a series of equally-sized buffers allocated in different locations, 
known as a scattered buffer \cite[{module/zfs/abd.c}]{zfs}.
This allows ZFS to avoid having to find large contiguous chunks of memory for ARC buffers when the system is running low on memory 
and  avoids having to map the entire buffer into the kernel's memory space when only part of it is accessed.
Both types can be accessed using the same set of functions, allowing the use of scattered buffers instead of linear buffers
wherever it appears useful, without having to change any consumers of those buffers.

ABDs are allocated in terms of pages, attempting to allocate first from the local memory node, 
but the allocation process prioritizes keeping as much of the 
buffer as possible on the same node, which might not be the local node if its memory is very full.
Unlike every other part of ZFS, it uses Linux's \texttt{alloc\_pages\_node} to directly allocate a set of pages
from a specific area of memory (a specific NUMA node, explained in Chapter \ref{whatisnuma}).
It does this to attempt to ensure the series of allocations that it makes are as close together as possible,
regardless of whether they are close to the current process.
This behavior is clever, but undesirable for the purposes of this project, as it means that larger allocations
can be spread out over multiple areas, depending on the available memory.

\subsection{Memory and the ARC}F
The ARC requires large amounts of memory, but gives ZFS comparable or even faster performance than other filesystems in some 
circumstances.
However, this also means that ZFS is significantly affected by memory latency, more so than other filesystems, because they don't typically 
read your files directly from memory.

One development that has recently become quite important in improving high-performance systems, 
is in the inclusion of a memory controller within each physical CPU\cite{lameter_numa_2013}.
Now systems with multiple CPU sockets have multiple memory controllers, 
meaning that the access latency to different parts of memory is no longer uniform,
thus these new architectures came to be referred to as Non-Uniform Memory Access, or NUMA, architectures.

\chapter{What is NUMA?}
\label{whatisnuma}
The traditional models of computing, the Turing Machine and the von Neumann architecture, 
assume that all the memory a computer has is accessible in the same amount of time. 
Programmers have followed this assumption, and have generally not cared about where exactly in memory their program's data are stored,
or on what processors their programs run.

However, as newer systems have gotten faster and faster, with more and more cores and memory, this fundamental assumption could no longer hold
\cite{lameter_numa_2013}.
As processors have become faster and more numerous in systems, the absolute distance between a processor and memory began to matter for latency.
In order for memory to keep up, new system architectures had to be devised to allow memory to be closer and thus keep pace with faster processors.
This was done, as outlined above, by integrating the memory controller directly into the CPU package, 
allowing memory to be directly attached to the processor (Figure \ref{fig:UMAvsNUMA}).
While this allows for much lower latency when accessing that memory from a processor in that socket, 
it makes access to memory attached to any other socket very slow, as it first has to pass through a memory interconnect between sockets (Figure \ref{fig:LocalvsRemote}).
This property of these architectures is known as NUMA, or \textit{Non-Uniform Memory Access}.
A processor and its attached memory are together known as a node.

\begin{figure}[H]
    \centering
    \resizebox{!}{0.25\textheight}{%LaTeX with PSTricks extensions
%%Creator: Inkscape 1.0.2-2 (e86c870879, 2021-01-15)
%%Please note this file requires PSTricks extensions
\psset{xunit=.5pt,yunit=.5pt,runit=.5pt}
\begin{pspicture}(535.30073511,661.85233529)
{
\newrgbcolor{curcolor}{0.80000001 0.80000001 0.80000001}
\pscustom[linestyle=none,fillstyle=solid,fillcolor=curcolor]
{
\newpath
\moveto(84.17337071,496.18731608)
\lineto(84.17337071,170.91819986)
}
}
{
\newrgbcolor{curcolor}{0 0 0}
\pscustom[linewidth=2.64566925,linecolor=curcolor]
{
\newpath
\moveto(84.17337071,496.18731608)
\lineto(84.17337071,170.91819986)
}
}
{
\newrgbcolor{curcolor}{0 0 0}
\pscustom[linewidth=2.64566925,linecolor=curcolor]
{
\newpath
\moveto(451.12736882,497.19746663)
\lineto(451.12736882,162.83696521)
}
}
{
\newrgbcolor{curcolor}{0.80000001 0.80000001 0.80000001}
\pscustom[linestyle=none,fillstyle=solid,fillcolor=curcolor]
{
\newpath
\moveto(0.63137876,183.97502329)
\lineto(534.66934935,183.97502329)
\lineto(534.66934935,-1.39409042)
\lineto(0.63137876,-1.39409042)
\closepath
}
}
{
\newrgbcolor{curcolor}{0 0 0}
\pscustom[linewidth=1.25239939,linecolor=curcolor]
{
\newpath
\moveto(0.63137876,183.97502329)
\lineto(534.66934935,183.97502329)
\lineto(534.66934935,-1.39409042)
\lineto(0.63137876,-1.39409042)
\closepath
}
}
{
\newrgbcolor{curcolor}{0.80000001 0.80000001 0.80000001}
\pscustom[linestyle=none,fillstyle=solid,fillcolor=curcolor]
{
\newpath
\moveto(0.33071092,661.52162627)
\lineto(168.01603192,661.52162627)
\lineto(168.01603192,496.86677287)
\lineto(0.33071092,496.86677287)
\closepath
}
}
{
\newrgbcolor{curcolor}{0 0 0}
\pscustom[linewidth=0.66141731,linecolor=curcolor]
{
\newpath
\moveto(0.33071092,661.52162627)
\lineto(168.01603192,661.52162627)
\lineto(168.01603192,496.86677287)
\lineto(0.33071092,496.86677287)
\closepath
}
}
{
\newrgbcolor{curcolor}{0.80000001 0.80000001 0.80000001}
\pscustom[linestyle=none,fillstyle=solid,fillcolor=curcolor]
{
\newpath
\moveto(367.28472502,661.52162627)
\lineto(534.97004602,661.52162627)
\lineto(534.97004602,496.86677287)
\lineto(367.28472502,496.86677287)
\closepath
}
}
{
\newrgbcolor{curcolor}{0 0 0}
\pscustom[linewidth=0.66141731,linecolor=curcolor]
{
\newpath
\moveto(367.28472502,661.52162627)
\lineto(534.97004602,661.52162627)
\lineto(534.97004602,496.86677287)
\lineto(367.28472502,496.86677287)
\closepath
}
}
{
\newrgbcolor{curcolor}{0.80000001 0.80000001 0.80000001}
\pscustom[linestyle=none,fillstyle=solid,fillcolor=curcolor]
{
\newpath
\moveto(44.80780784,380.72000399)
\lineto(490.49292026,380.72000399)
\lineto(490.49292026,308.95759703)
\lineto(44.80780784,308.95759703)
\closepath
}
}
{
\newrgbcolor{curcolor}{0 0 0}
\pscustom[linewidth=0.71187021,linecolor=curcolor]
{
\newpath
\moveto(44.80780784,380.72000399)
\lineto(490.49292026,380.72000399)
\lineto(490.49292026,308.95759703)
\lineto(44.80780784,308.95759703)
\closepath
}
}
{
\newrgbcolor{curcolor}{0 0 0}
\pscustom[linestyle=none,fillstyle=solid,fillcolor=curcolor]
{
\newpath
\moveto(70.4917694,566.8016652)
\curveto(69.77562585,566.48916619)(69.12458625,566.19619837)(68.53865062,565.92276174)
\curveto(67.96573577,565.64932511)(67.21052985,565.36286769)(66.27303283,565.06338948)
\curveto(65.47876453,564.81599443)(64.61288187,564.60766176)(63.67538485,564.43839147)
\curveto(62.75090863,564.25610038)(61.72877647,564.16495484)(60.60898836,564.16495484)
\curveto(58.49962008,564.16495484)(56.57905328,564.45792266)(54.84728796,565.04385829)
\curveto(53.12854343,565.64281472)(51.63115237,566.57380134)(50.35511476,567.83681815)
\curveto(49.10511874,569.07379338)(48.12855935,570.6427988)(47.42543659,572.54383441)
\curveto(46.72231383,574.45789082)(46.37075245,576.67793583)(46.37075245,579.20396946)
\curveto(46.37075245,581.59979516)(46.70929304,583.74171543)(47.38637421,585.62973025)
\curveto(48.06345539,587.51774507)(49.04001478,589.11279208)(50.31605239,590.41487126)
\curveto(51.55302761,591.67788807)(53.04390828,592.64142667)(54.7886944,593.30548706)
\curveto(56.5465013,593.96954745)(58.49310968,594.30157764)(60.62851955,594.30157764)
\curveto(62.19101458,594.30157764)(63.74699921,594.11277616)(65.29647344,593.73517319)
\curveto(66.85896846,593.35757023)(68.59073378,592.69350984)(70.4917694,591.74299203)
\lineto(70.4917694,587.1531629)
\lineto(70.19880158,587.1531629)
\curveto(68.59724418,588.49430446)(67.00870757,589.47086385)(65.43319175,590.08284107)
\curveto(63.85767594,590.69481829)(62.17148339,591.0008069)(60.37461411,591.0008069)
\curveto(58.90326463,591.0008069)(57.57514386,590.75992225)(56.3902518,590.27815295)
\curveto(55.21838053,589.80940444)(54.17020678,589.0737297)(53.24573056,588.07112873)
\curveto(52.34729592,587.09456933)(51.64417316,585.85759411)(51.13636227,584.36020304)
\curveto(50.64157218,582.87583277)(50.39417714,581.15708824)(50.39417714,579.20396946)
\curveto(50.39417714,577.15970513)(50.66761377,575.40189823)(51.21448703,573.93054875)
\curveto(51.77438108,572.45919927)(52.49052463,571.26128641)(53.36291769,570.33681019)
\curveto(54.27437312,569.37327159)(55.33556765,568.65712804)(56.5465013,568.18837953)
\curveto(57.77045574,567.73265181)(59.05951413,567.50478796)(60.41367649,567.50478796)
\curveto(62.27564972,567.50478796)(64.02043584,567.82379736)(65.64803482,568.46181616)
\curveto(67.2756338,569.09983496)(68.79906645,570.05686316)(70.21833277,571.33290077)
\lineto(70.4917694,571.33290077)
\closepath
}
}
{
\newrgbcolor{curcolor}{0 0 0}
\pscustom[linestyle=none,fillstyle=solid,fillcolor=curcolor]
{
\newpath
\moveto(94.98387833,584.98520105)
\curveto(94.98387833,583.69614265)(94.75601447,582.4982298)(94.30028676,581.39146249)
\curveto(93.85757984,580.29771598)(93.23258183,579.34719817)(92.42529273,578.53990907)
\curveto(91.42269175,577.5373081)(90.23779969,576.78210217)(88.87061655,576.27429129)
\curveto(87.5034334,575.77950119)(85.77817848,575.53210615)(83.69485178,575.53210615)
\lineto(79.82767659,575.53210615)
\lineto(79.82767659,564.69229691)
\lineto(75.9605014,564.69229691)
\lineto(75.9605014,593.77423557)
\lineto(83.85110128,593.77423557)
\curveto(85.59588739,593.77423557)(87.07374727,593.62449646)(88.28468091,593.32501825)
\curveto(89.49561456,593.03856083)(90.56982989,592.58283311)(91.5073269,591.9578351)
\curveto(92.61409421,591.21564996)(93.46695608,590.29117374)(94.06591251,589.18440643)
\curveto(94.67788972,588.07763912)(94.98387833,586.67790399)(94.98387833,584.98520105)
\closepath
\moveto(90.96045364,584.88754511)
\curveto(90.96045364,585.89014609)(90.78467295,586.76253914)(90.43311157,587.50472428)
\curveto(90.08155019,588.24690942)(89.54769772,588.85237624)(88.83155417,589.32112475)
\curveto(88.20655616,589.72476929)(87.49041261,590.01122671)(86.68312351,590.18049701)
\curveto(85.88885521,590.3627881)(84.87974384,590.45393364)(83.6557894,590.45393364)
\lineto(79.82767659,590.45393364)
\lineto(79.82767659,578.83287689)
\lineto(83.08938495,578.83287689)
\curveto(84.65187998,578.83287689)(85.92140719,578.9695952)(86.89796658,579.24303183)
\curveto(87.87452597,579.52948925)(88.66879427,579.97870657)(89.28077149,580.59068379)
\curveto(89.89274871,581.2156818)(90.32243484,581.87323179)(90.56982989,582.56333376)
\curveto(90.83024572,583.25343573)(90.96045364,584.02817285)(90.96045364,584.88754511)
\closepath
}
}
{
\newrgbcolor{curcolor}{0 0 0}
\pscustom[linestyle=none,fillstyle=solid,fillcolor=curcolor]
{
\newpath
\moveto(121.97597799,576.37194722)
\curveto(121.97597799,574.26257894)(121.74160374,572.42013689)(121.27285523,570.84462107)
\curveto(120.81712752,569.28212605)(120.06192159,567.98004686)(119.00723745,566.93838351)
\curveto(118.00463647,565.94880333)(116.8327652,565.22614938)(115.49162364,564.77042166)
\curveto(114.15048208,564.31469395)(112.58798705,564.08683009)(110.80413856,564.08683009)
\curveto(108.9812277,564.08683009)(107.39269109,564.32771474)(106.03852874,564.80948404)
\curveto(104.68436638,565.29125334)(103.54504709,566.0008865)(102.62057087,566.93838351)
\curveto(101.56588673,568.00608844)(100.8041704,569.29514684)(100.3354219,570.8055587)
\curveto(99.87969418,572.31597056)(99.65183032,574.1714334)(99.65183032,576.37194722)
\lineto(99.65183032,593.77423557)
\lineto(103.51900551,593.77423557)
\lineto(103.51900551,576.17663535)
\curveto(103.51900551,574.60111953)(103.62317185,573.35763391)(103.83150452,572.44617847)
\curveto(104.05285798,571.53472304)(104.41744015,570.70790276)(104.92525103,569.96571762)
\curveto(105.49816588,569.11936615)(106.27290299,568.48134735)(107.24946238,568.05166122)
\curveto(108.23904257,567.62197508)(109.42393463,567.40713202)(110.80413856,567.40713202)
\curveto(112.1973633,567.40713202)(113.38225536,567.61546469)(114.35881475,568.03213003)
\curveto(115.33537414,568.46181616)(116.11662165,569.10634536)(116.70255728,569.96571762)
\curveto(117.21036817,570.70790276)(117.56843994,571.55425423)(117.77677261,572.50477204)
\curveto(117.99812608,573.46831064)(118.10880281,574.65971309)(118.10880281,576.07897941)
\lineto(118.10880281,593.77423557)
\lineto(121.97597799,593.77423557)
\closepath
}
}
{
\newrgbcolor{curcolor}{0 0 0}
\pscustom[linestyle=none,fillstyle=solid,fillcolor=curcolor]
{
\newpath
\moveto(437.44581233,561.75089548)
\curveto(436.72966878,561.43839648)(436.07862918,561.14542866)(435.49269355,560.87199203)
\curveto(434.9197787,560.5985554)(434.16457278,560.31209798)(433.22707576,560.01261976)
\curveto(432.43280746,559.76522472)(431.5669248,559.55689205)(430.62942778,559.38762175)
\curveto(429.70495156,559.20533067)(428.6828194,559.11418513)(427.56303129,559.11418513)
\curveto(425.45366301,559.11418513)(423.53309621,559.40715294)(421.80133089,559.99308858)
\curveto(420.08258636,560.592045)(418.5851953,561.52303162)(417.30915769,562.78604843)
\curveto(416.05916167,564.02302366)(415.08260228,565.59202908)(414.37947952,567.4930647)
\curveto(413.67635676,569.4071211)(413.32479538,571.62716612)(413.32479538,574.15319974)
\curveto(413.32479538,576.54902545)(413.66333597,578.69094571)(414.34041714,580.57896053)
\curveto(415.01749832,582.46697535)(415.99405771,584.06202236)(417.27009532,585.36410155)
\curveto(418.50707054,586.62711836)(419.99795121,587.59065696)(421.74273733,588.25471734)
\curveto(423.50054423,588.91877773)(425.44715261,589.25080792)(427.58256248,589.25080792)
\curveto(429.14505751,589.25080792)(430.70104214,589.06200644)(432.25051637,588.68440348)
\curveto(433.8130114,588.30680051)(435.54477671,587.64274013)(437.44581233,586.69222232)
\lineto(437.44581233,582.10239318)
\lineto(437.15284451,582.10239318)
\curveto(435.55128711,583.44353475)(433.9627505,584.42009414)(432.38723468,585.03207135)
\curveto(430.81171887,585.64404857)(429.12552632,585.95003718)(427.32865704,585.95003718)
\curveto(425.85730756,585.95003718)(424.52918679,585.70915253)(423.34429473,585.22738323)
\curveto(422.17242346,584.75863472)(421.12424971,584.02295998)(420.19977349,583.02035901)
\curveto(419.30133885,582.04379962)(418.59821609,580.80682439)(418.0904052,579.30943332)
\curveto(417.59561511,577.82506305)(417.34822007,576.10631852)(417.34822007,574.15319974)
\curveto(417.34822007,572.10893542)(417.6216567,570.35112851)(418.16852996,568.87977903)
\curveto(418.72842401,567.40842955)(419.44456756,566.2105167)(420.31696062,565.28604047)
\curveto(421.22841605,564.32250188)(422.28961058,563.60635832)(423.50054423,563.13760981)
\curveto(424.72449867,562.6818821)(426.01355706,562.45401824)(427.36771942,562.45401824)
\curveto(429.22969265,562.45401824)(430.97447877,562.77302764)(432.60207775,563.41104644)
\curveto(434.22967674,564.04906525)(435.75310938,565.00609345)(437.1723757,566.28213105)
\lineto(437.44581233,566.28213105)
\closepath
}
}
{
\newrgbcolor{curcolor}{0 0 0}
\pscustom[linestyle=none,fillstyle=solid,fillcolor=curcolor]
{
\newpath
\moveto(461.93792126,579.93443133)
\curveto(461.93792126,578.64537294)(461.71005741,577.44746009)(461.25432969,576.34069278)
\curveto(460.81162277,575.24694626)(460.18662476,574.29642845)(459.37933566,573.48913936)
\curveto(458.37673468,572.48653838)(457.19184262,571.73133245)(455.82465948,571.22352157)
\curveto(454.45747633,570.72873148)(452.73222141,570.48133643)(450.64889471,570.48133643)
\lineto(446.78171952,570.48133643)
\lineto(446.78171952,559.6415272)
\lineto(442.91454433,559.6415272)
\lineto(442.91454433,588.72346585)
\lineto(450.80514421,588.72346585)
\curveto(452.54993032,588.72346585)(454.0277902,588.57372674)(455.23872384,588.27424853)
\curveto(456.44965749,587.98779111)(457.52387282,587.53206339)(458.46136983,586.90706538)
\curveto(459.56813714,586.16488025)(460.42099901,585.24040402)(461.01995544,584.13363671)
\curveto(461.63193265,583.02686941)(461.93792126,581.62713428)(461.93792126,579.93443133)
\closepath
\moveto(457.91449657,579.8367754)
\curveto(457.91449657,580.83937637)(457.73871588,581.71176943)(457.3871545,582.45395456)
\curveto(457.03559312,583.1961397)(456.50174065,583.80160652)(455.7855971,584.27035503)
\curveto(455.16059909,584.67399958)(454.44445554,584.960457)(453.63716644,585.12972729)
\curveto(452.84289814,585.31201838)(451.83378677,585.40316392)(450.60983233,585.40316392)
\lineto(446.78171952,585.40316392)
\lineto(446.78171952,573.78210717)
\lineto(450.04342788,573.78210717)
\curveto(451.60592291,573.78210717)(452.87545012,573.91882549)(453.85200951,574.19226212)
\curveto(454.8285689,574.47871954)(455.6228372,574.92793686)(456.23481442,575.53991408)
\curveto(456.84679164,576.16491209)(457.27647777,576.82246208)(457.52387282,577.51256405)
\curveto(457.78428865,578.20266602)(457.91449657,578.97740313)(457.91449657,579.8367754)
\closepath
}
}
{
\newrgbcolor{curcolor}{0 0 0}
\pscustom[linestyle=none,fillstyle=solid,fillcolor=curcolor]
{
\newpath
\moveto(488.93002092,571.32117751)
\curveto(488.93002092,569.21180922)(488.69564667,567.36936717)(488.22689816,565.79385136)
\curveto(487.77117045,564.23135633)(487.01596452,562.92927714)(485.96128038,561.88761379)
\curveto(484.9586794,560.89803361)(483.78680813,560.17537966)(482.44566657,559.71965195)
\curveto(481.10452501,559.26392423)(479.54202998,559.03606037)(477.7581815,559.03606037)
\curveto(475.93527063,559.03606037)(474.34673402,559.27694502)(472.99257167,559.75871432)
\curveto(471.63840931,560.24048362)(470.49909002,560.95011678)(469.5746138,561.88761379)
\curveto(468.51992966,562.95531873)(467.75821333,564.24437712)(467.28946483,565.75478898)
\curveto(466.83373711,567.26520084)(466.60587325,569.12066368)(466.60587325,571.32117751)
\lineto(466.60587325,588.72346585)
\lineto(470.47304844,588.72346585)
\lineto(470.47304844,571.12586563)
\curveto(470.47304844,569.55034981)(470.57721478,568.30686419)(470.78554745,567.39540876)
\curveto(471.00690091,566.48395333)(471.37148308,565.65713304)(471.87929396,564.91494791)
\curveto(472.45220881,564.06859643)(473.22694592,563.43057763)(474.20350531,563.0008915)
\curveto(475.1930855,562.57120537)(476.37797756,562.3563623)(477.7581815,562.3563623)
\curveto(479.15140623,562.3563623)(480.33629829,562.56469497)(481.31285768,562.98136031)
\curveto(482.28941707,563.41104644)(483.07066458,564.05557564)(483.65660021,564.91494791)
\curveto(484.1644111,565.65713304)(484.52248287,566.50348451)(484.73081554,567.45400232)
\curveto(484.95216901,568.41754092)(485.06284574,569.60894338)(485.06284574,571.02820969)
\lineto(485.06284574,588.72346585)
\lineto(488.93002092,588.72346585)
\closepath
}
}
{
\newrgbcolor{curcolor}{0 0 0}
\pscustom[linestyle=none,fillstyle=solid,fillcolor=curcolor]
{
\newpath
\moveto(108.92041076,333.66697357)
\lineto(105.05323557,333.66697357)
\lineto(105.05323557,358.72548754)
\lineto(96.96732382,341.67476057)
\lineto(94.66264365,341.67476057)
\lineto(86.63532546,358.72548754)
\lineto(86.63532546,333.66697357)
\lineto(83.02205572,333.66697357)
\lineto(83.02205572,362.74891222)
\lineto(88.29547643,362.74891222)
\lineto(96.04935799,346.55755753)
\lineto(103.54933411,362.74891222)
\lineto(108.92041076,362.74891222)
\closepath
}
}
{
\newrgbcolor{curcolor}{0 0 0}
\pscustom[linestyle=none,fillstyle=solid,fillcolor=curcolor]
{
\newpath
\moveto(134.70157907,344.1942838)
\lineto(118.6274115,344.1942838)
\curveto(118.6274115,342.85314224)(118.82923378,341.68127097)(119.23287833,340.67867)
\curveto(119.63652287,339.68908981)(120.18990653,338.87529032)(120.89302929,338.23727152)
\curveto(121.57011047,337.61227351)(122.37088917,337.143525)(123.29536539,336.831026)
\curveto(124.23286241,336.51852699)(125.26150496,336.36227749)(126.38129307,336.36227749)
\curveto(127.86566334,336.36227749)(129.35654401,336.65524531)(130.85393508,337.24118094)
\curveto(132.36434693,337.84013737)(133.43856226,338.426073)(134.07658106,338.99898784)
\lineto(134.27189294,338.99898784)
\lineto(134.27189294,334.99509434)
\curveto(133.03491771,334.47426267)(131.7719009,334.03806614)(130.48284251,333.68650476)
\curveto(129.19378411,333.33494338)(127.83962176,333.15916269)(126.42035544,333.15916269)
\curveto(122.8005753,333.15916269)(119.97506346,334.13572208)(117.94381993,336.08884086)
\curveto(115.9125764,338.05498043)(114.89695463,340.84142989)(114.89695463,344.44818924)
\curveto(114.89695463,348.01588622)(115.86700363,350.84790845)(117.80710162,352.94425594)
\curveto(119.7602204,355.04060343)(122.3253164,356.08877718)(125.50238961,356.08877718)
\curveto(128.44508858,356.08877718)(130.71070636,355.22940492)(132.29924297,353.51066039)
\curveto(133.90080037,351.79191586)(134.70157907,349.35051738)(134.70157907,346.18646496)
\closepath
\moveto(131.1273717,347.00677485)
\curveto(131.11435091,348.93385204)(130.62607122,350.42473271)(129.66253262,351.47941686)
\curveto(128.71201481,352.534101)(127.26019652,353.06144307)(125.30707774,353.06144307)
\curveto(123.34093816,353.06144307)(121.77193274,352.48201783)(120.60006147,351.32316735)
\curveto(119.441211,350.16431688)(118.78366101,348.72551937)(118.6274115,347.00677485)
\closepath
}
}
{
\newrgbcolor{curcolor}{0 0 0}
\pscustom[linestyle=none,fillstyle=solid,fillcolor=curcolor]
{
\newpath
\moveto(172.10380632,333.66697357)
\lineto(168.43194301,333.66697357)
\lineto(168.43194301,346.08880902)
\curveto(168.43194301,347.02630603)(168.38637023,347.93125107)(168.29522469,348.80364413)
\curveto(168.21709994,349.67603718)(168.04131925,350.37264955)(167.76788262,350.89348122)
\curveto(167.46840441,351.45337527)(167.03871828,351.87655101)(166.47882422,352.16300843)
\curveto(165.91893017,352.44946585)(165.11164108,352.59269456)(164.05695694,352.59269456)
\curveto(163.02831438,352.59269456)(161.99967182,352.33227872)(160.97102926,351.81144705)
\curveto(159.9423867,351.30363617)(158.91374414,350.65259657)(157.88510159,349.85832827)
\curveto(157.92416396,349.55885005)(157.95671594,349.20728867)(157.98275753,348.80364413)
\curveto(158.00879911,348.41302037)(158.0218199,348.02239661)(158.0218199,347.63177286)
\lineto(158.0218199,333.66697357)
\lineto(154.34995659,333.66697357)
\lineto(154.34995659,346.08880902)
\curveto(154.34995659,347.05234762)(154.30438382,347.96380305)(154.21323828,348.82317531)
\curveto(154.13511353,349.69556837)(153.95933284,350.39218073)(153.68589621,350.91301241)
\curveto(153.38641799,351.47290646)(152.95673186,351.8895718)(152.39683781,352.16300843)
\curveto(151.83694376,352.44946585)(151.02965466,352.59269456)(149.97497052,352.59269456)
\curveto(148.97236955,352.59269456)(147.96325818,352.34529952)(146.94763641,351.85050942)
\curveto(145.94503544,351.35571933)(144.94243446,350.72421093)(143.93983349,349.95598421)
\lineto(143.93983349,333.66697357)
\lineto(140.26797018,333.66697357)
\lineto(140.26797018,355.48331036)
\lineto(143.93983349,355.48331036)
\lineto(143.93983349,353.06144307)
\curveto(145.08566317,354.01196088)(146.22498246,354.75414601)(147.35779135,355.28799848)
\curveto(148.50362104,355.82185095)(149.72106508,356.08877718)(151.01012348,356.08877718)
\curveto(152.49449375,356.08877718)(153.75100017,355.77627818)(154.77964272,355.15128017)
\curveto(155.82130607,354.52628216)(156.59604319,353.6603995)(157.10385407,352.55363219)
\curveto(158.58822435,353.80362821)(159.9423867,354.70206285)(161.16634114,355.2489361)
\curveto(162.39029558,355.80883015)(163.69888516,356.08877718)(165.09210989,356.08877718)
\curveto(167.4879356,356.08877718)(169.25225289,355.35961284)(170.38506179,353.90128415)
\curveto(171.53089147,352.45597625)(172.10380632,350.43124311)(172.10380632,347.82708473)
\closepath
}
}
{
\newrgbcolor{curcolor}{0 0 0}
\pscustom[linestyle=none,fillstyle=solid,fillcolor=curcolor]
{
\newpath
\moveto(197.76778843,344.56537637)
\curveto(197.76778843,341.01070019)(196.85633299,338.20471954)(195.03342213,336.14743442)
\curveto(193.21051127,334.09014931)(190.76911279,333.06150675)(187.7092267,333.06150675)
\curveto(184.62329903,333.06150675)(182.16887976,334.09014931)(180.3459689,336.14743442)
\curveto(178.53607883,338.20471954)(177.63113379,341.01070019)(177.63113379,344.56537637)
\curveto(177.63113379,348.12005255)(178.53607883,350.9260332)(180.3459689,352.98331832)
\curveto(182.16887976,355.05362423)(184.62329903,356.08877718)(187.7092267,356.08877718)
\curveto(190.76911279,356.08877718)(193.21051127,355.05362423)(195.03342213,352.98331832)
\curveto(196.85633299,350.9260332)(197.76778843,348.12005255)(197.76778843,344.56537637)
\closepath
\moveto(193.97873799,344.56537637)
\curveto(193.97873799,347.39088821)(193.42535434,349.4872357)(192.31858703,350.85441885)
\curveto(191.21181972,352.23462278)(189.67536628,352.92472475)(187.7092267,352.92472475)
\curveto(185.71704555,352.92472475)(184.16757131,352.23462278)(183.060804,350.85441885)
\curveto(181.96705748,349.4872357)(181.42018423,347.39088821)(181.42018423,344.56537637)
\curveto(181.42018423,341.83101008)(181.97356788,339.75419377)(183.08033519,338.33492746)
\curveto(184.1871025,336.92868193)(185.73006634,336.22555917)(187.7092267,336.22555917)
\curveto(189.66234548,336.22555917)(191.19228853,336.92217154)(192.29905584,338.31539627)
\curveto(193.41884394,339.72164179)(193.97873799,341.80496849)(193.97873799,344.56537637)
\closepath
}
}
{
\newrgbcolor{curcolor}{0 0 0}
\pscustom[linestyle=none,fillstyle=solid,fillcolor=curcolor]
{
\newpath
\moveto(217.06459758,351.47941686)
\lineto(216.8692857,351.47941686)
\curveto(216.32241245,351.60962477)(215.78855998,351.70077032)(215.2677283,351.75285349)
\curveto(214.75991742,351.81795744)(214.1544506,351.85050942)(213.45132784,351.85050942)
\curveto(212.31851894,351.85050942)(211.22477243,351.59660398)(210.17008828,351.0887931)
\curveto(209.11540414,350.59400301)(208.09978238,349.94947381)(207.12322299,349.15520551)
\lineto(207.12322299,333.66697357)
\lineto(203.45135968,333.66697357)
\lineto(203.45135968,355.48331036)
\lineto(207.12322299,355.48331036)
\lineto(207.12322299,352.26066437)
\curveto(208.58155168,353.43253564)(209.86409968,354.25935592)(210.97086699,354.74112522)
\curveto(212.09065509,355.23591531)(213.22997438,355.48331036)(214.38882485,355.48331036)
\curveto(215.02684365,355.48331036)(215.48908177,355.46377917)(215.77553919,355.42471679)
\curveto(216.06199661,355.39867521)(216.49168274,355.34008165)(217.06459758,355.2489361)
\closepath
}
}
{
\newrgbcolor{curcolor}{0 0 0}
\pscustom[linestyle=none,fillstyle=solid,fillcolor=curcolor]
{
\newpath
\moveto(239.38874674,355.48331036)
\lineto(226.65441229,325.62012419)
\lineto(222.72864353,325.62012419)
\lineto(226.7911306,334.72165771)
\lineto(218.09975202,355.48331036)
\lineto(222.08411434,355.48331036)
\lineto(228.78331176,339.31148685)
\lineto(235.54110274,355.48331036)
\closepath
}
}
{
\newrgbcolor{curcolor}{0 0 0}
\pscustom[linestyle=none,fillstyle=solid,fillcolor=curcolor]
{
\newpath
\moveto(281.00971277,335.77634185)
\curveto(280.29356922,335.46384285)(279.64252963,335.17087503)(279.05659399,334.8974384)
\curveto(278.48367915,334.62400177)(277.72847322,334.33754435)(276.79097621,334.03806614)
\curveto(275.9967079,333.79067109)(275.13082524,333.58233842)(274.19332823,333.41306813)
\curveto(273.268852,333.23077704)(272.24671984,333.1396315)(271.12693174,333.1396315)
\curveto(269.01756346,333.1396315)(267.09699665,333.43259932)(265.36523133,334.01853495)
\curveto(263.64648681,334.61749138)(262.14909574,335.548478)(260.87305814,336.81149481)
\curveto(259.62306212,338.04847004)(258.64650273,339.61747546)(257.94337996,341.51851107)
\curveto(257.2402572,343.43256748)(256.88869582,345.65261249)(256.88869582,348.17864612)
\curveto(256.88869582,350.57447182)(257.22723641,352.71639208)(257.90431759,354.60440691)
\curveto(258.58139877,356.49242173)(259.55795816,358.08746873)(260.83399576,359.38954792)
\curveto(262.07097099,360.65256473)(263.56185166,361.61610333)(265.30663777,362.28016372)
\curveto(267.06444467,362.9442241)(269.01105306,363.2762543)(271.14646293,363.2762543)
\curveto(272.70895795,363.2762543)(274.26494258,363.08745281)(275.81441681,362.70984985)
\curveto(277.37691184,362.33224688)(279.10867716,361.6681865)(281.00971277,360.71766869)
\lineto(281.00971277,356.12783956)
\lineto(280.71674496,356.12783956)
\curveto(279.11518755,357.46898112)(277.52665095,358.44554051)(275.95113513,359.05751773)
\curveto(274.37561931,359.66949495)(272.68942676,359.97548356)(270.89255749,359.97548356)
\curveto(269.421208,359.97548356)(268.09308723,359.73459891)(266.90819517,359.25282961)
\curveto(265.7363239,358.7840811)(264.68815016,358.04840636)(263.76367393,357.04580538)
\curveto(262.86523929,356.06924599)(262.16211653,354.83227076)(261.65430565,353.3348797)
\curveto(261.15951556,351.85050942)(260.91212051,350.1317649)(260.91212051,348.17864612)
\curveto(260.91212051,346.13438179)(261.18555714,344.37657489)(261.7324304,342.90522541)
\curveto(262.29232445,341.43387592)(263.008468,340.23596307)(263.88086106,339.31148685)
\curveto(264.79231649,338.34794825)(265.85351103,337.6318047)(267.06444467,337.16305619)
\curveto(268.28839911,336.70732847)(269.57745751,336.47946461)(270.93161986,336.47946461)
\curveto(272.7935931,336.47946461)(274.53837921,336.79847402)(276.1659782,337.43649282)
\curveto(277.79357718,338.07451162)(279.31700983,339.03153982)(280.73627614,340.30757743)
\lineto(281.00971277,340.30757743)
\closepath
}
}
{
\newrgbcolor{curcolor}{0 0 0}
\pscustom[linestyle=none,fillstyle=solid,fillcolor=curcolor]
{
\newpath
\moveto(304.77916596,344.56537637)
\curveto(304.77916596,341.01070019)(303.86771052,338.20471954)(302.04479966,336.14743442)
\curveto(300.2218888,334.09014931)(297.78049032,333.06150675)(294.72060423,333.06150675)
\curveto(291.63467656,333.06150675)(289.18025729,334.09014931)(287.35734643,336.14743442)
\curveto(285.54745636,338.20471954)(284.64251132,341.01070019)(284.64251132,344.56537637)
\curveto(284.64251132,348.12005255)(285.54745636,350.9260332)(287.35734643,352.98331832)
\curveto(289.18025729,355.05362423)(291.63467656,356.08877718)(294.72060423,356.08877718)
\curveto(297.78049032,356.08877718)(300.2218888,355.05362423)(302.04479966,352.98331832)
\curveto(303.86771052,350.9260332)(304.77916596,348.12005255)(304.77916596,344.56537637)
\closepath
\moveto(300.99011552,344.56537637)
\curveto(300.99011552,347.39088821)(300.43673187,349.4872357)(299.32996456,350.85441885)
\curveto(298.22319725,352.23462278)(296.68674381,352.92472475)(294.72060423,352.92472475)
\curveto(292.72842307,352.92472475)(291.17894884,352.23462278)(290.07218153,350.85441885)
\curveto(288.97843501,349.4872357)(288.43156176,347.39088821)(288.43156176,344.56537637)
\curveto(288.43156176,341.83101008)(288.98494541,339.75419377)(290.09171272,338.33492746)
\curveto(291.19848003,336.92868193)(292.74144387,336.22555917)(294.72060423,336.22555917)
\curveto(296.67372301,336.22555917)(298.20366606,336.92217154)(299.31043337,338.31539627)
\curveto(300.43022147,339.72164179)(300.99011552,341.80496849)(300.99011552,344.56537637)
\closepath
}
}
{
\newrgbcolor{curcolor}{0 0 0}
\pscustom[linestyle=none,fillstyle=solid,fillcolor=curcolor]
{
\newpath
\moveto(328.70486662,333.66697357)
\lineto(325.03300332,333.66697357)
\lineto(325.03300332,346.08880902)
\curveto(325.03300332,347.09140999)(324.97440975,348.02890701)(324.85722263,348.90130006)
\curveto(324.7400355,349.78671391)(324.52519243,350.47681588)(324.21269343,350.97160597)
\curveto(323.88717363,351.51847923)(323.41842512,351.92212378)(322.80644791,352.18253962)
\curveto(322.19447069,352.45597625)(321.40020238,352.59269456)(320.42364299,352.59269456)
\curveto(319.42104202,352.59269456)(318.37286827,352.34529952)(317.27912175,351.85050942)
\curveto(316.18537524,351.35571933)(315.13720149,350.72421093)(314.13460052,349.95598421)
\lineto(314.13460052,333.66697357)
\lineto(310.46273721,333.66697357)
\lineto(310.46273721,355.48331036)
\lineto(314.13460052,355.48331036)
\lineto(314.13460052,353.06144307)
\curveto(315.2804302,354.01196088)(316.46532226,354.75414601)(317.6892767,355.28799848)
\curveto(318.91323113,355.82185095)(320.16973755,356.08877718)(321.45879595,356.08877718)
\curveto(323.81555928,356.08877718)(325.61242855,355.37914402)(326.84940378,353.95987771)
\curveto(328.08637901,352.54061139)(328.70486662,350.49634707)(328.70486662,347.82708473)
\closepath
}
}
{
\newrgbcolor{curcolor}{0 0 0}
\pscustom[linestyle=none,fillstyle=solid,fillcolor=curcolor]
{
\newpath
\moveto(347.10325405,333.86228545)
\curveto(346.41315208,333.67999436)(345.65794615,333.53025526)(344.83763626,333.41306813)
\curveto(344.03034717,333.295881)(343.30769322,333.23728744)(342.66967442,333.23728744)
\curveto(340.44311901,333.23728744)(338.75041606,333.83624386)(337.59156558,335.03415672)
\curveto(336.43271511,336.23206957)(335.85328987,338.15263637)(335.85328987,340.79585712)
\lineto(335.85328987,352.39738268)
\lineto(333.37282902,352.39738268)
\lineto(333.37282902,355.48331036)
\lineto(335.85328987,355.48331036)
\lineto(335.85328987,361.75282165)
\lineto(339.52515318,361.75282165)
\lineto(339.52515318,355.48331036)
\lineto(347.10325405,355.48331036)
\lineto(347.10325405,352.39738268)
\lineto(339.52515318,352.39738268)
\lineto(339.52515318,342.45600809)
\curveto(339.52515318,341.3101784)(339.55119476,340.41174376)(339.60327793,339.76070417)
\curveto(339.6553611,339.12268537)(339.83765218,338.52372894)(340.15015119,337.96383489)
\curveto(340.43660861,337.44300321)(340.82723237,337.05888985)(341.32202246,336.81149481)
\curveto(341.82983334,336.57712055)(342.59806006,336.45993343)(343.62670262,336.45993343)
\curveto(344.22565905,336.45993343)(344.85065706,336.54456857)(345.50169665,336.71383887)
\curveto(346.15273624,336.89612996)(346.62148475,337.04586906)(346.90794217,337.16305619)
\lineto(347.10325405,337.16305619)
\closepath
}
}
{
\newrgbcolor{curcolor}{0 0 0}
\pscustom[linestyle=none,fillstyle=solid,fillcolor=curcolor]
{
\newpath
\moveto(365.1500641,351.47941686)
\lineto(364.95475223,351.47941686)
\curveto(364.40787897,351.60962477)(363.8740265,351.70077032)(363.35319482,351.75285349)
\curveto(362.84538394,351.81795744)(362.23991712,351.85050942)(361.53679436,351.85050942)
\curveto(360.40398546,351.85050942)(359.31023895,351.59660398)(358.2555548,351.0887931)
\curveto(357.20087066,350.59400301)(356.1852489,349.94947381)(355.20868951,349.15520551)
\lineto(355.20868951,333.66697357)
\lineto(351.5368262,333.66697357)
\lineto(351.5368262,355.48331036)
\lineto(355.20868951,355.48331036)
\lineto(355.20868951,352.26066437)
\curveto(356.6670182,353.43253564)(357.9495662,354.25935592)(359.05633351,354.74112522)
\curveto(360.17612161,355.23591531)(361.3154409,355.48331036)(362.47429137,355.48331036)
\curveto(363.11231017,355.48331036)(363.57454829,355.46377917)(363.86100571,355.42471679)
\curveto(364.14746313,355.39867521)(364.57714926,355.34008165)(365.1500641,355.2489361)
\closepath
}
}
{
\newrgbcolor{curcolor}{0 0 0}
\pscustom[linestyle=none,fillstyle=solid,fillcolor=curcolor]
{
\newpath
\moveto(387.20076221,344.56537637)
\curveto(387.20076221,341.01070019)(386.28930678,338.20471954)(384.46639592,336.14743442)
\curveto(382.64348506,334.09014931)(380.20208658,333.06150675)(377.14220049,333.06150675)
\curveto(374.05627281,333.06150675)(371.60185355,334.09014931)(369.77894268,336.14743442)
\curveto(367.96905261,338.20471954)(367.06410758,341.01070019)(367.06410758,344.56537637)
\curveto(367.06410758,348.12005255)(367.96905261,350.9260332)(369.77894268,352.98331832)
\curveto(371.60185355,355.05362423)(374.05627281,356.08877718)(377.14220049,356.08877718)
\curveto(380.20208658,356.08877718)(382.64348506,355.05362423)(384.46639592,352.98331832)
\curveto(386.28930678,350.9260332)(387.20076221,348.12005255)(387.20076221,344.56537637)
\closepath
\moveto(383.41171178,344.56537637)
\curveto(383.41171178,347.39088821)(382.85832812,349.4872357)(381.75156081,350.85441885)
\curveto(380.6447935,352.23462278)(379.10834006,352.92472475)(377.14220049,352.92472475)
\curveto(375.15001933,352.92472475)(373.6005451,352.23462278)(372.49377779,350.85441885)
\curveto(371.40003127,349.4872357)(370.85315801,347.39088821)(370.85315801,344.56537637)
\curveto(370.85315801,341.83101008)(371.40654167,339.75419377)(372.51330898,338.33492746)
\curveto(373.62007629,336.92868193)(375.16304012,336.22555917)(377.14220049,336.22555917)
\curveto(379.09531927,336.22555917)(380.62526232,336.92217154)(381.73202962,338.31539627)
\curveto(382.85181773,339.72164179)(383.41171178,341.80496849)(383.41171178,344.56537637)
\closepath
}
}
{
\newrgbcolor{curcolor}{0 0 0}
\pscustom[linestyle=none,fillstyle=solid,fillcolor=curcolor]
{
\newpath
\moveto(396.59527357,333.66697357)
\lineto(392.92341026,333.66697357)
\lineto(392.92341026,364.05750181)
\lineto(396.59527357,364.05750181)
\closepath
}
}
{
\newrgbcolor{curcolor}{0 0 0}
\pscustom[linestyle=none,fillstyle=solid,fillcolor=curcolor]
{
\newpath
\moveto(407.57179762,333.66697357)
\lineto(403.89993431,333.66697357)
\lineto(403.89993431,364.05750181)
\lineto(407.57179762,364.05750181)
\closepath
}
}
{
\newrgbcolor{curcolor}{0 0 0}
\pscustom[linestyle=none,fillstyle=solid,fillcolor=curcolor]
{
\newpath
\moveto(433.09905659,344.1942838)
\lineto(417.02488902,344.1942838)
\curveto(417.02488902,342.85314224)(417.2267113,341.68127097)(417.63035584,340.67867)
\curveto(418.03400039,339.68908981)(418.58738405,338.87529032)(419.29050681,338.23727152)
\curveto(419.96758799,337.61227351)(420.76836669,337.143525)(421.69284291,336.831026)
\curveto(422.63033992,336.51852699)(423.65898248,336.36227749)(424.77877058,336.36227749)
\curveto(426.26314086,336.36227749)(427.75402153,336.65524531)(429.25141259,337.24118094)
\curveto(430.76182445,337.84013737)(431.83603978,338.426073)(432.47405858,338.99898784)
\lineto(432.66937046,338.99898784)
\lineto(432.66937046,334.99509434)
\curveto(431.43239523,334.47426267)(430.16937842,334.03806614)(428.88032002,333.68650476)
\curveto(427.59126163,333.33494338)(426.23709927,333.15916269)(424.81783296,333.15916269)
\curveto(421.19805282,333.15916269)(418.37254098,334.13572208)(416.34129745,336.08884086)
\curveto(414.31005392,338.05498043)(413.29443215,340.84142989)(413.29443215,344.44818924)
\curveto(413.29443215,348.01588622)(414.26448114,350.84790845)(416.20457913,352.94425594)
\curveto(418.15769791,355.04060343)(420.72279391,356.08877718)(423.89986713,356.08877718)
\curveto(426.8425661,356.08877718)(429.10818388,355.22940492)(430.69672049,353.51066039)
\curveto(432.29827789,351.79191586)(433.09905659,349.35051738)(433.09905659,346.18646496)
\closepath
\moveto(429.52484922,347.00677485)
\curveto(429.51182843,348.93385204)(429.02354874,350.42473271)(428.06001014,351.47941686)
\curveto(427.10949233,352.534101)(425.65767404,353.06144307)(423.70455525,353.06144307)
\curveto(421.73841568,353.06144307)(420.16941026,352.48201783)(418.99753899,351.32316735)
\curveto(417.83868851,350.16431688)(417.18113852,348.72551937)(417.02488902,347.00677485)
\closepath
}
}
{
\newrgbcolor{curcolor}{0 0 0}
\pscustom[linestyle=none,fillstyle=solid,fillcolor=curcolor]
{
\newpath
\moveto(452.278682,351.47941686)
\lineto(452.08337012,351.47941686)
\curveto(451.53649686,351.60962477)(451.00264439,351.70077032)(450.48181272,351.75285349)
\curveto(449.97400184,351.81795744)(449.36853501,351.85050942)(448.66541225,351.85050942)
\curveto(447.53260336,351.85050942)(446.43885684,351.59660398)(445.3841727,351.0887931)
\curveto(444.32948856,350.59400301)(443.31386679,349.94947381)(442.3373074,349.15520551)
\lineto(442.3373074,333.66697357)
\lineto(438.66544409,333.66697357)
\lineto(438.66544409,355.48331036)
\lineto(442.3373074,355.48331036)
\lineto(442.3373074,352.26066437)
\curveto(443.79563609,353.43253564)(445.07818409,354.25935592)(446.1849514,354.74112522)
\curveto(447.3047395,355.23591531)(448.44405879,355.48331036)(449.60290927,355.48331036)
\curveto(450.24092807,355.48331036)(450.70316618,355.46377917)(450.9896236,355.42471679)
\curveto(451.27608102,355.39867521)(451.70576716,355.34008165)(452.278682,355.2489361)
\closepath
}
}
{
\newrgbcolor{curcolor}{0 0 0}
\pscustom[linestyle=none,fillstyle=solid,fillcolor=curcolor]
{
\newpath
\moveto(215.36538684,81.78308262)
\lineto(211.49821166,81.78308262)
\lineto(211.49821166,106.84159659)
\lineto(203.4122999,89.79086963)
\lineto(201.10761974,89.79086963)
\lineto(193.08030155,106.84159659)
\lineto(193.08030155,81.78308262)
\lineto(189.4670318,81.78308262)
\lineto(189.4670318,110.86502128)
\lineto(194.74045251,110.86502128)
\lineto(202.49433408,94.67366658)
\lineto(209.9943102,110.86502128)
\lineto(215.36538684,110.86502128)
\closepath
}
}
{
\newrgbcolor{curcolor}{0 0 0}
\pscustom[linestyle=none,fillstyle=solid,fillcolor=curcolor]
{
\newpath
\moveto(241.14655516,92.31039285)
\lineto(225.07238759,92.31039285)
\curveto(225.07238759,90.96925129)(225.27420986,89.79738002)(225.67785441,88.79477905)
\curveto(226.08149896,87.80519886)(226.63488261,86.99139937)(227.33800538,86.35338057)
\curveto(228.01508655,85.72838256)(228.81586525,85.25963405)(229.74034148,84.94713505)
\curveto(230.67783849,84.63463604)(231.70648105,84.47838654)(232.82626915,84.47838654)
\curveto(234.31063943,84.47838654)(235.8015201,84.77135436)(237.29891116,85.35728999)
\curveto(238.80932302,85.95624642)(239.88353835,86.54218205)(240.52155715,87.1150969)
\lineto(240.71686903,87.1150969)
\lineto(240.71686903,83.11120339)
\curveto(239.4798938,82.59037172)(238.21687699,82.15417519)(236.92781859,81.80261381)
\curveto(235.6387602,81.45105243)(234.28459784,81.27527174)(232.86533153,81.27527174)
\curveto(229.24555139,81.27527174)(226.42003955,82.25183113)(224.38879602,84.20494991)
\curveto(222.35755248,86.17108948)(221.34193072,88.95753895)(221.34193072,92.5642983)
\curveto(221.34193072,96.13199527)(222.31197971,98.9640175)(224.2520777,101.06036499)
\curveto(226.20519648,103.15671249)(228.77029248,104.20488623)(231.9473657,104.20488623)
\curveto(234.89006466,104.20488623)(237.15568245,103.34551397)(238.74421906,101.62676944)
\curveto(240.34577646,99.90802491)(241.14655516,97.46662644)(241.14655516,94.30257401)
\closepath
\moveto(237.57234779,95.1228839)
\curveto(237.559327,97.0499611)(237.0710473,98.54084177)(236.1075087,99.59552591)
\curveto(235.1569909,100.65021005)(233.7051726,101.17755212)(231.75205382,101.17755212)
\curveto(229.78591425,101.17755212)(228.21690883,100.59812688)(227.04503756,99.43927641)
\curveto(225.88618708,98.28042593)(225.22863709,96.84162843)(225.07238759,95.1228839)
\closepath
}
}
{
\newrgbcolor{curcolor}{0 0 0}
\pscustom[linestyle=none,fillstyle=solid,fillcolor=curcolor]
{
\newpath
\moveto(278.5487824,81.78308262)
\lineto(274.87691909,81.78308262)
\lineto(274.87691909,94.20491807)
\curveto(274.87691909,95.14241509)(274.83134632,96.04736012)(274.74020078,96.91975318)
\curveto(274.66207603,97.79214623)(274.48629534,98.4887586)(274.21285871,99.00959027)
\curveto(273.91338049,99.56948432)(273.48369436,99.99266006)(272.92380031,100.27911748)
\curveto(272.36390626,100.5655749)(271.55661716,100.70880361)(270.50193302,100.70880361)
\curveto(269.47329046,100.70880361)(268.44464791,100.44838778)(267.41600535,99.9275561)
\curveto(266.38736279,99.41974522)(265.35872023,98.76870562)(264.33007767,97.97443732)
\curveto(264.36914005,97.67495911)(264.40169203,97.32339773)(264.42773361,96.91975318)
\curveto(264.4537752,96.52912942)(264.46679599,96.13850567)(264.46679599,95.74788191)
\lineto(264.46679599,81.78308262)
\lineto(260.79493268,81.78308262)
\lineto(260.79493268,94.20491807)
\curveto(260.79493268,95.16845667)(260.74935991,96.0799121)(260.65821436,96.93928437)
\curveto(260.58008961,97.81167742)(260.40430892,98.50828979)(260.13087229,99.02912146)
\curveto(259.83139408,99.58901551)(259.40170795,100.00568085)(258.8418139,100.27911748)
\curveto(258.28191985,100.5655749)(257.47463075,100.70880361)(256.41994661,100.70880361)
\curveto(255.41734563,100.70880361)(254.40823426,100.46140857)(253.3926125,99.96661848)
\curveto(252.39001152,99.47182839)(251.38741055,98.84031998)(250.38480957,98.07209326)
\lineto(250.38480957,81.78308262)
\lineto(246.71294626,81.78308262)
\lineto(246.71294626,103.59941941)
\lineto(250.38480957,103.59941941)
\lineto(250.38480957,101.17755212)
\curveto(251.53063926,102.12806993)(252.66995855,102.87025507)(253.80276744,103.40410753)
\curveto(254.94859713,103.93796)(256.16604117,104.20488623)(257.45509956,104.20488623)
\curveto(258.93946984,104.20488623)(260.19597625,103.89238723)(261.22461881,103.26738922)
\curveto(262.26628216,102.64239121)(263.04101928,101.77650855)(263.54883016,100.66974124)
\curveto(265.03320043,101.91973726)(266.38736279,102.8181719)(267.61131722,103.36504516)
\curveto(268.83527166,103.92493921)(270.14386124,104.20488623)(271.53708598,104.20488623)
\curveto(273.93291168,104.20488623)(275.69722898,103.47572189)(276.83003787,102.0173932)
\curveto(277.97586756,100.5720853)(278.5487824,98.54735216)(278.5487824,95.94319379)
\closepath
}
}
{
\newrgbcolor{curcolor}{0 0 0}
\pscustom[linestyle=none,fillstyle=solid,fillcolor=curcolor]
{
\newpath
\moveto(304.21276451,92.68148542)
\curveto(304.21276451,89.12680924)(303.30130908,86.32082859)(301.47839822,84.26354347)
\curveto(299.65548736,82.20625836)(297.21408888,81.1776158)(294.15420279,81.1776158)
\curveto(291.06827511,81.1776158)(288.61385584,82.20625836)(286.79094498,84.26354347)
\curveto(284.98105491,86.32082859)(284.07610988,89.12680924)(284.07610988,92.68148542)
\curveto(284.07610988,96.2361616)(284.98105491,99.04214225)(286.79094498,101.09942737)
\curveto(288.61385584,103.16973328)(291.06827511,104.20488623)(294.15420279,104.20488623)
\curveto(297.21408888,104.20488623)(299.65548736,103.16973328)(301.47839822,101.09942737)
\curveto(303.30130908,99.04214225)(304.21276451,96.2361616)(304.21276451,92.68148542)
\closepath
\moveto(300.42371408,92.68148542)
\curveto(300.42371408,95.50699726)(299.87033042,97.60334475)(298.76356311,98.9705279)
\curveto(297.6567958,100.35073184)(296.12034236,101.04083381)(294.15420279,101.04083381)
\curveto(292.16202163,101.04083381)(290.6125474,100.35073184)(289.50578009,98.9705279)
\curveto(288.41203357,97.60334475)(287.86516031,95.50699726)(287.86516031,92.68148542)
\curveto(287.86516031,89.94711913)(288.41854397,87.87030282)(289.52531128,86.45103651)
\curveto(290.63207859,85.04479099)(292.17504242,84.34166823)(294.15420279,84.34166823)
\curveto(296.10732157,84.34166823)(297.63726461,85.03828059)(298.74403192,86.43150532)
\curveto(299.86382003,87.83775084)(300.42371408,89.92107754)(300.42371408,92.68148542)
\closepath
}
}
{
\newrgbcolor{curcolor}{0 0 0}
\pscustom[linestyle=none,fillstyle=solid,fillcolor=curcolor]
{
\newpath
\moveto(323.50957367,99.59552591)
\lineto(323.31426179,99.59552591)
\curveto(322.76738853,99.72573383)(322.23353606,99.81687937)(321.71270439,99.86896254)
\curveto(321.20489351,99.9340665)(320.59942668,99.96661848)(319.89630392,99.96661848)
\curveto(318.76349503,99.96661848)(317.66974851,99.71271304)(316.61506437,99.20490215)
\curveto(315.56038023,98.71011206)(314.54475846,98.06558286)(313.56819907,97.27131456)
\lineto(313.56819907,81.78308262)
\lineto(309.89633576,81.78308262)
\lineto(309.89633576,103.59941941)
\lineto(313.56819907,103.59941941)
\lineto(313.56819907,100.37677342)
\curveto(315.02652776,101.54864469)(316.30907576,102.37546497)(317.41584307,102.85723427)
\curveto(318.53563117,103.35202436)(319.67495046,103.59941941)(320.83380094,103.59941941)
\curveto(321.47181974,103.59941941)(321.93405785,103.57988822)(322.22051527,103.54082585)
\curveto(322.50697269,103.51478426)(322.93665883,103.4561907)(323.50957367,103.36504516)
\closepath
}
}
{
\newrgbcolor{curcolor}{0 0 0}
\pscustom[linestyle=none,fillstyle=solid,fillcolor=curcolor]
{
\newpath
\moveto(345.83372283,103.59941941)
\lineto(333.09938837,73.73623324)
\lineto(329.17361962,73.73623324)
\lineto(333.23610669,82.83776676)
\lineto(324.54472811,103.59941941)
\lineto(328.52909042,103.59941941)
\lineto(335.22828784,87.4275959)
\lineto(341.98607883,103.59941941)
\closepath
}
}
\end{pspicture}
}
    \resizebox{!}{0.25\textheight}{%LaTeX with PSTricks extensions
%%Creator: Inkscape 1.0.2-2 (e86c870879, 2021-01-15)
%%Please note this file requires PSTricks extensions
\psset{xunit=.5pt,yunit=.5pt,runit=.5pt}
\begin{pspicture}(535.30073511,423.12458669)
{
\newrgbcolor{curcolor}{0 0 0}
\pscustom[linestyle=none,fillstyle=solid,fillcolor=curcolor]
{
\newpath
\moveto(125.21474646,262.51035425)
\lineto(125.21474646,169.57630701)
}
}
{
\newrgbcolor{curcolor}{0 0 0}
\pscustom[linewidth=2.64566925,linecolor=curcolor]
{
\newpath
\moveto(125.21474646,262.51035425)
\lineto(125.21474646,169.57630701)
}
}
{
\newrgbcolor{curcolor}{0.80000001 0.80000001 0.80000001}
\pscustom[linestyle=none,fillstyle=solid,fillcolor=curcolor]
{
\newpath
\moveto(160.38875955,376.99323056)
\lineto(374.91198367,376.99323056)
\lineto(374.91198367,319.25593837)
\lineto(160.38875955,319.25593837)
\closepath
}
}
{
\newrgbcolor{curcolor}{0 0 0}
\pscustom[linewidth=0.4430022,linecolor=curcolor]
{
\newpath
\moveto(160.38875955,376.99323056)
\lineto(374.91198367,376.99323056)
\lineto(374.91198367,319.25593837)
\lineto(160.38875955,319.25593837)
\closepath
}
}
{
\newrgbcolor{curcolor}{0.80000001 0.80000001 0.80000001}
\pscustom[linestyle=none,fillstyle=solid,fillcolor=curcolor]
{
\newpath
\moveto(41.37207917,422.79387724)
\lineto(209.05740017,422.79387724)
\lineto(209.05740017,258.13902384)
\lineto(41.37207917,258.13902384)
\closepath
}
}
{
\newrgbcolor{curcolor}{0 0 0}
\pscustom[linewidth=0.66141731,linecolor=curcolor]
{
\newpath
\moveto(41.37207917,422.79387724)
\lineto(209.05740017,422.79387724)
\lineto(209.05740017,258.13902384)
\lineto(41.37207917,258.13902384)
\closepath
}
}
{
\newrgbcolor{curcolor}{0.80000001 0.80000001 0.80000001}
\pscustom[linestyle=none,fillstyle=solid,fillcolor=curcolor]
{
\newpath
\moveto(326.24335747,422.79387724)
\lineto(493.92867847,422.79387724)
\lineto(493.92867847,258.13902384)
\lineto(326.24335747,258.13902384)
\closepath
}
}
{
\newrgbcolor{curcolor}{0 0 0}
\pscustom[linewidth=0.66141731,linecolor=curcolor]
{
\newpath
\moveto(326.24335747,422.79387724)
\lineto(493.92867847,422.79387724)
\lineto(493.92867847,258.13902384)
\lineto(326.24335747,258.13902384)
\closepath
}
}
{
\newrgbcolor{curcolor}{0 0 0}
\pscustom[linestyle=none,fillstyle=solid,fillcolor=curcolor]
{
\newpath
\moveto(111.53315206,328.07395942)
\curveto(110.81700851,327.76146041)(110.16596892,327.46849259)(109.58003328,327.19505596)
\curveto(109.00711844,326.92161933)(108.25191251,326.63516191)(107.3144155,326.3356837)
\curveto(106.52014719,326.08828865)(105.65426453,325.87995598)(104.71676752,325.71068569)
\curveto(103.79229129,325.5283946)(102.77015913,325.43724906)(101.65037103,325.43724906)
\curveto(99.54100275,325.43724906)(97.62043595,325.73021688)(95.88867063,326.31615251)
\curveto(94.1699261,326.91510894)(92.67253503,327.84609556)(91.39649743,329.10911237)
\curveto(90.14650141,330.3460876)(89.16994202,331.91509302)(88.46681926,333.81612863)
\curveto(87.7636965,335.73018504)(87.41213511,337.95023005)(87.41213511,340.47626368)
\curveto(87.41213511,342.87208938)(87.7506757,345.01400965)(88.42775688,346.90202447)
\curveto(89.10483806,348.79003929)(90.08139745,350.3850863)(91.35743505,351.68716548)
\curveto(92.59441028,352.95018229)(94.08529095,353.91372089)(95.83007706,354.57778128)
\curveto(97.58788397,355.24184167)(99.53449235,355.57387186)(101.66990222,355.57387186)
\curveto(103.23239724,355.57387186)(104.78838187,355.38507038)(106.33785611,355.00746741)
\curveto(107.90035113,354.62986445)(109.63211645,353.96580406)(111.53315206,353.01528625)
\lineto(111.53315206,348.42545712)
\lineto(111.24018425,348.42545712)
\curveto(109.63862685,349.76659868)(108.05009024,350.74315807)(106.47457442,351.35513529)
\curveto(104.8990586,351.96711251)(103.21286606,352.27310112)(101.41599678,352.27310112)
\curveto(99.9446473,352.27310112)(98.61652652,352.03221647)(97.43163446,351.55044717)
\curveto(96.25976319,351.08169866)(95.21158945,350.34602392)(94.28711323,349.34342295)
\curveto(93.38867859,348.36686355)(92.68555582,347.12988833)(92.17774494,345.63249726)
\curveto(91.68295485,344.14812699)(91.4355598,342.42938246)(91.4355598,340.47626368)
\curveto(91.4355598,338.43199935)(91.70899643,336.67419245)(92.25586969,335.20284297)
\curveto(92.81576374,333.73149349)(93.5319073,332.53358063)(94.40430035,331.60910441)
\curveto(95.31575578,330.64556581)(96.37695032,329.92942226)(97.58788397,329.46067375)
\curveto(98.8118384,329.00494603)(100.1008968,328.77708218)(101.45505915,328.77708218)
\curveto(103.31703239,328.77708218)(105.0618185,329.09609158)(106.68941749,329.73411038)
\curveto(108.31701647,330.37212918)(109.84044912,331.32915738)(111.25971544,332.60519499)
\lineto(111.53315206,332.60519499)
\closepath
}
}
{
\newrgbcolor{curcolor}{0 0 0}
\pscustom[linestyle=none,fillstyle=solid,fillcolor=curcolor]
{
\newpath
\moveto(136.025261,346.25749527)
\curveto(136.025261,344.96843687)(135.79739714,343.77052402)(135.34166943,342.66375671)
\curveto(134.8989625,341.5700102)(134.27396449,340.61949239)(133.4666754,339.81220329)
\curveto(132.46407442,338.80960232)(131.27918236,338.05439639)(129.91199921,337.54658551)
\curveto(128.54481607,337.05179541)(126.81956114,336.80440037)(124.73623444,336.80440037)
\lineto(120.86905926,336.80440037)
\lineto(120.86905926,325.96459113)
\lineto(117.00188407,325.96459113)
\lineto(117.00188407,355.04652979)
\lineto(124.89248395,355.04652979)
\curveto(126.63727006,355.04652979)(128.11512993,354.89679068)(129.32606358,354.59731247)
\curveto(130.53699722,354.31085505)(131.61121255,353.85512733)(132.54870957,353.23012932)
\curveto(133.65547688,352.48794418)(134.50833875,351.56346796)(135.10729517,350.45670065)
\curveto(135.71927239,349.34993334)(136.025261,347.95019821)(136.025261,346.25749527)
\closepath
\moveto(132.00183631,346.15983933)
\curveto(132.00183631,347.16244031)(131.82605562,348.03483336)(131.47449424,348.7770185)
\curveto(131.12293286,349.51920364)(130.58908039,350.12467046)(129.87293684,350.59341897)
\curveto(129.24793883,350.99706351)(128.53179527,351.28352093)(127.72450618,351.45279123)
\curveto(126.93023787,351.63508232)(125.9211265,351.72622786)(124.69717207,351.72622786)
\lineto(120.86905926,351.72622786)
\lineto(120.86905926,340.10517111)
\lineto(124.13076762,340.10517111)
\curveto(125.69326265,340.10517111)(126.96278985,340.24188942)(127.93934924,340.51532605)
\curveto(128.91590864,340.80178347)(129.71017694,341.25100079)(130.32215416,341.86297801)
\curveto(130.93413138,342.48797602)(131.36381751,343.14552601)(131.61121255,343.83562798)
\curveto(131.87162839,344.52572995)(132.00183631,345.30046707)(132.00183631,346.15983933)
\closepath
}
}
{
\newrgbcolor{curcolor}{0 0 0}
\pscustom[linestyle=none,fillstyle=solid,fillcolor=curcolor]
{
\newpath
\moveto(163.01736066,337.64424144)
\curveto(163.01736066,335.53487316)(162.78298641,333.69243111)(162.3142379,332.11691529)
\curveto(161.85851018,330.55442027)(161.10330426,329.25234108)(160.04862011,328.21067773)
\curveto(159.04601914,327.22109755)(157.87414787,326.4984436)(156.53300631,326.04271588)
\curveto(155.19186474,325.58698817)(153.62936972,325.35912431)(151.84552123,325.35912431)
\curveto(150.02261037,325.35912431)(148.43407376,325.60000896)(147.0799114,326.08177826)
\curveto(145.72574905,326.56354756)(144.58642976,327.27318072)(143.66195354,328.21067773)
\curveto(142.6072694,329.27838266)(141.84555307,330.56744106)(141.37680456,332.07785292)
\curveto(140.92107685,333.58826478)(140.69321299,335.44372762)(140.69321299,337.64424144)
\lineto(140.69321299,355.04652979)
\lineto(144.56038818,355.04652979)
\lineto(144.56038818,337.44892957)
\curveto(144.56038818,335.87341375)(144.66455451,334.62992813)(144.87288718,333.71847269)
\curveto(145.09424064,332.80701726)(145.45882282,331.98019698)(145.9666337,331.23801184)
\curveto(146.53954854,330.39166037)(147.31428566,329.75364157)(148.29084505,329.32395544)
\curveto(149.28042523,328.8942693)(150.46531729,328.67942624)(151.84552123,328.67942624)
\curveto(153.23874596,328.67942624)(154.42363802,328.88775891)(155.40019741,329.30442425)
\curveto(156.3767568,329.73411038)(157.15800432,330.37863958)(157.74393995,331.23801184)
\curveto(158.25175083,331.98019698)(158.60982261,332.82654845)(158.81815528,333.77706626)
\curveto(159.03950874,334.74060486)(159.15018547,335.93200731)(159.15018547,337.35127363)
\lineto(159.15018547,355.04652979)
\lineto(163.01736066,355.04652979)
\closepath
}
}
{
\newrgbcolor{curcolor}{0 0 0}
\pscustom[linestyle=none,fillstyle=solid,fillcolor=curcolor]
{
\newpath
\moveto(396.40441594,323.02316086)
\curveto(395.68827239,322.71066186)(395.0372328,322.41769404)(394.45129716,322.14425741)
\curveto(393.87838232,321.87082078)(393.12317639,321.58436336)(392.18567938,321.28488515)
\curveto(391.39141107,321.0374901)(390.52552841,320.82915743)(389.5880314,320.65988714)
\curveto(388.66355517,320.47759605)(387.64142301,320.38645051)(386.52163491,320.38645051)
\curveto(384.41226663,320.38645051)(382.49169982,320.67941833)(380.75993451,321.26535396)
\curveto(379.04118998,321.86431039)(377.54379891,322.79529701)(376.26776131,324.05831382)
\curveto(375.01776529,325.29528905)(374.0412059,326.86429447)(373.33808314,328.76533008)
\curveto(372.63496037,330.67938649)(372.28339899,332.8994315)(372.28339899,335.42546513)
\curveto(372.28339899,337.82129083)(372.62193958,339.9632111)(373.29902076,341.85122592)
\curveto(373.97610194,343.73924074)(374.95266133,345.33428774)(376.22869893,346.63636693)
\curveto(377.46567416,347.89938374)(378.95655483,348.86292234)(380.70134094,349.52698273)
\curveto(382.45914785,350.19104311)(384.40575623,350.52307331)(386.5411661,350.52307331)
\curveto(388.10366112,350.52307331)(389.65964575,350.33427182)(391.20911999,349.95666886)
\curveto(392.77161501,349.5790659)(394.50338033,348.91500551)(396.40441594,347.9644877)
\lineto(396.40441594,343.37465857)
\lineto(396.11144813,343.37465857)
\curveto(394.50989073,344.71580013)(392.92135412,345.69235952)(391.3458383,346.30433674)
\curveto(389.77032248,346.91631396)(388.08412994,347.22230257)(386.28726066,347.22230257)
\curveto(384.81591117,347.22230257)(383.4877904,346.98141792)(382.30289834,346.49964862)
\curveto(381.13102707,346.03090011)(380.08285333,345.29522537)(379.1583771,344.29262439)
\curveto(378.25994247,343.316065)(377.5568197,342.07908978)(377.04900882,340.58169871)
\curveto(376.55421873,339.09732844)(376.30682368,337.37858391)(376.30682368,335.42546513)
\curveto(376.30682368,333.3812008)(376.58026031,331.6233939)(377.12713357,330.15204442)
\curveto(377.68702762,328.68069493)(378.40317118,327.48278208)(379.27556423,326.55830586)
\curveto(380.18701966,325.59476726)(381.2482142,324.87862371)(382.45914785,324.4098752)
\curveto(383.68310228,323.95414748)(384.97216068,323.72628363)(386.32632303,323.72628363)
\curveto(388.18829627,323.72628363)(389.93308238,324.04529303)(391.56068137,324.68331183)
\curveto(393.18828035,325.32133063)(394.711713,326.27835883)(396.13097931,327.55439644)
\lineto(396.40441594,327.55439644)
\closepath
}
}
{
\newrgbcolor{curcolor}{0 0 0}
\pscustom[linestyle=none,fillstyle=solid,fillcolor=curcolor]
{
\newpath
\moveto(420.89652488,341.20669672)
\curveto(420.89652488,339.91763832)(420.66866102,338.71972547)(420.21293331,337.61295816)
\curveto(419.77022638,336.51921164)(419.14522837,335.56869384)(418.33793928,334.76140474)
\curveto(417.3353383,333.75880377)(416.15044624,333.00359784)(414.78326309,332.49578695)
\curveto(413.41607995,332.00099686)(411.69082502,331.75360182)(409.60749832,331.75360182)
\lineto(405.74032314,331.75360182)
\lineto(405.74032314,320.91379258)
\lineto(401.87314795,320.91379258)
\lineto(401.87314795,349.99573124)
\lineto(409.76374782,349.99573124)
\curveto(411.50853394,349.99573124)(412.98639381,349.84599213)(414.19732746,349.54651392)
\curveto(415.4082611,349.26005649)(416.48247643,348.80432878)(417.41997345,348.17933077)
\curveto(418.52674076,347.43714563)(419.37960263,346.51266941)(419.97855905,345.4059021)
\curveto(420.59053627,344.29913479)(420.89652488,342.89939966)(420.89652488,341.20669672)
\closepath
\moveto(416.87310019,341.10904078)
\curveto(416.87310019,342.11164175)(416.6973195,342.98403481)(416.34575812,343.72621995)
\curveto(415.99419674,344.46840508)(415.46034427,345.07387191)(414.74420072,345.54262041)
\curveto(414.11920271,345.94626496)(413.40305915,346.23272238)(412.59577006,346.40199268)
\curveto(411.80150175,346.58428376)(410.79239038,346.67542931)(409.56843595,346.67542931)
\lineto(405.74032314,346.67542931)
\lineto(405.74032314,335.05437256)
\lineto(409.0020315,335.05437256)
\curveto(410.56452653,335.05437256)(411.83405373,335.19109087)(412.81061312,335.4645275)
\curveto(413.78717251,335.75098492)(414.58144082,336.20020224)(415.19341804,336.81217946)
\curveto(415.80539526,337.43717747)(416.23508139,338.09472746)(416.48247643,338.78482943)
\curveto(416.74289227,339.4749314)(416.87310019,340.24966852)(416.87310019,341.10904078)
\closepath
}
}
{
\newrgbcolor{curcolor}{0 0 0}
\pscustom[linestyle=none,fillstyle=solid,fillcolor=curcolor]
{
\newpath
\moveto(447.88862454,332.59344289)
\curveto(447.88862454,330.48407461)(447.65425029,328.64163256)(447.18550178,327.06611674)
\curveto(446.72977406,325.50362172)(445.97456813,324.20154253)(444.91988399,323.15987918)
\curveto(443.91728302,322.170299)(442.74541175,321.44764505)(441.40427019,320.99191733)
\curveto(440.06312862,320.53618962)(438.5006336,320.30832576)(436.71678511,320.30832576)
\curveto(434.89387425,320.30832576)(433.30533764,320.54921041)(431.95117528,321.03097971)
\curveto(430.59701293,321.51274901)(429.45769364,322.22238216)(428.53321742,323.15987918)
\curveto(427.47853327,324.22758411)(426.71681695,325.51664251)(426.24806844,327.02705437)
\curveto(425.79234073,328.53746622)(425.56447687,330.39292907)(425.56447687,332.59344289)
\lineto(425.56447687,349.99573124)
\lineto(429.43165206,349.99573124)
\lineto(429.43165206,332.39813102)
\curveto(429.43165206,330.8226152)(429.53581839,329.57912957)(429.74415106,328.66767414)
\curveto(429.96550452,327.75621871)(430.3300867,326.92939843)(430.83789758,326.18721329)
\curveto(431.41081242,325.34086182)(432.18554954,324.70284302)(433.16210893,324.27315688)
\curveto(434.15168911,323.84347075)(435.33658117,323.62862769)(436.71678511,323.62862769)
\curveto(438.11000984,323.62862769)(439.2949019,323.83696036)(440.27146129,324.2536257)
\curveto(441.24802068,324.68331183)(442.0292682,325.32784103)(442.61520383,326.18721329)
\curveto(443.12301471,326.92939843)(443.48108649,327.7757499)(443.68941916,328.72626771)
\curveto(443.91077262,329.68980631)(444.02144935,330.88120876)(444.02144935,332.30047508)
\lineto(444.02144935,349.99573124)
\lineto(447.88862454,349.99573124)
\closepath
}
}
{
\newrgbcolor{curcolor}{0.80000001 0.80000001 0.80000001}
\pscustom[linestyle=none,fillstyle=solid,fillcolor=curcolor]
{
\newpath
\moveto(60.74550283,293.97727683)
\lineto(189.68400535,293.97727683)
\lineto(189.68400535,258.1390743)
\lineto(60.74550283,258.1390743)
\closepath
}
}
{
\newrgbcolor{curcolor}{0 0 0}
\pscustom[linewidth=0.27058431,linecolor=curcolor]
{
\newpath
\moveto(60.74550283,293.97727683)
\lineto(189.68400535,293.97727683)
\lineto(189.68400535,258.1390743)
\lineto(60.74550283,258.1390743)
\closepath
}
}
{
\newrgbcolor{curcolor}{0 0 0}
\pscustom[linestyle=none,fillstyle=solid,fillcolor=curcolor]
{
\newpath
\moveto(79.29351423,270.47888719)
\lineto(78.17472463,270.47888719)
\lineto(78.17472463,282.99313139)
\lineto(75.83543729,274.47798316)
\lineto(75.1686839,274.47798316)
\lineto(72.84634792,282.99313139)
\lineto(72.84634792,270.47888719)
\lineto(71.8010142,270.47888719)
\lineto(71.8010142,285.00243326)
\lineto(73.32663638,285.00243326)
\lineto(75.56986602,276.9164563)
\lineto(77.73963979,285.00243326)
\lineto(79.29351423,285.00243326)
\closepath
}
}
{
\newrgbcolor{curcolor}{0 0 0}
\pscustom[linestyle=none,fillstyle=solid,fillcolor=curcolor]
{
\newpath
\moveto(86.75211123,275.7362353)
\lineto(82.10178882,275.7362353)
\curveto(82.10178882,275.06646801)(82.16017683,274.48123445)(82.27695285,273.98053463)
\curveto(82.39372886,273.48633741)(82.55382502,273.07992522)(82.75724131,272.76129806)
\curveto(82.95312366,272.4491735)(83.18479222,272.21508007)(83.45224697,272.05901779)
\curveto(83.72346869,271.90295551)(84.02105919,271.82492437)(84.34501846,271.82492437)
\curveto(84.77445285,271.82492437)(85.20577073,271.97123276)(85.63897209,272.26384954)
\curveto(86.07594041,272.56296891)(86.3867153,272.85558569)(86.57129675,273.14169987)
\lineto(86.62780128,273.14169987)
\lineto(86.62780128,271.14215189)
\curveto(86.26993928,270.88204809)(85.90454336,270.66421115)(85.53161349,270.48864109)
\curveto(85.15868362,270.31307102)(84.76691892,270.22528599)(84.35631937,270.22528599)
\curveto(83.30910217,270.22528599)(82.49167004,270.71298061)(81.90402298,271.68836987)
\curveto(81.31637592,272.67026173)(81.02255239,274.06181707)(81.02255239,275.8630359)
\curveto(81.02255239,277.64474695)(81.30319153,279.05906138)(81.86446981,280.10597918)
\curveto(82.42951506,281.15289699)(83.17160783,281.67635589)(84.0907481,281.67635589)
\curveto(84.94208295,281.67635589)(85.59753544,281.24718462)(86.05710557,280.38884207)
\curveto(86.52044268,279.53049952)(86.75211123,278.31126294)(86.75211123,276.73113234)
\closepath
\moveto(85.71807842,277.14079583)
\curveto(85.71431145,278.1031799)(85.57305014,278.84772704)(85.29429448,279.37443724)
\curveto(85.0193058,279.90114744)(84.59928883,280.16450254)(84.03424358,280.16450254)
\curveto(83.46543136,280.16450254)(83.01151167,279.87513706)(82.67248452,279.2964061)
\curveto(82.33722434,278.71767514)(82.14699244,277.99913838)(82.10178882,277.14079583)
\closepath
}
}
{
\newrgbcolor{curcolor}{0 0 0}
\pscustom[linestyle=none,fillstyle=solid,fillcolor=curcolor]
{
\newpath
\moveto(97.57272847,270.47888719)
\lineto(96.5104434,270.47888719)
\lineto(96.5104434,276.68236288)
\curveto(96.5104434,277.15054973)(96.49725901,277.60248008)(96.47089023,278.03815395)
\curveto(96.44828842,278.47382782)(96.39743435,278.82171666)(96.31832801,279.08182046)
\curveto(96.23168774,279.36143205)(96.10737779,279.57276639)(95.94539815,279.71582348)
\curveto(95.78341851,279.85888057)(95.54986647,279.93040912)(95.24474204,279.93040912)
\curveto(94.94715154,279.93040912)(94.64956104,279.80035721)(94.35197054,279.54025341)
\curveto(94.05438004,279.2866522)(93.75678955,278.96152245)(93.45919905,278.56486415)
\curveto(93.47049995,278.41530447)(93.47991737,278.2397344)(93.48745131,278.03815395)
\curveto(93.49498525,277.8430761)(93.49875221,277.64799825)(93.49875221,277.4529204)
\lineto(93.49875221,270.47888719)
\lineto(92.43646714,270.47888719)
\lineto(92.43646714,276.68236288)
\curveto(92.43646714,277.16355492)(92.42328275,277.61873657)(92.39691398,278.04790784)
\curveto(92.37431217,278.48358171)(92.32345809,278.83147055)(92.24435176,279.09157435)
\curveto(92.15771149,279.37118594)(92.03340153,279.57926898)(91.87142189,279.71582348)
\curveto(91.70944225,279.85888057)(91.47589022,279.93040912)(91.17076578,279.93040912)
\curveto(90.88070922,279.93040912)(90.58876917,279.80685981)(90.29494564,279.5597612)
\curveto(90.00488908,279.31266258)(89.71483252,278.99728672)(89.42477596,278.61363362)
\lineto(89.42477596,270.47888719)
\lineto(88.36249089,270.47888719)
\lineto(88.36249089,281.37398522)
\lineto(89.42477596,281.37398522)
\lineto(89.42477596,280.16450254)
\curveto(89.75626917,280.63919198)(90.0858789,281.0098399)(90.41360515,281.27644629)
\curveto(90.74509836,281.54305269)(91.0973099,281.67635589)(91.47023976,281.67635589)
\curveto(91.89967416,281.67635589)(92.2631866,281.52029361)(92.5607771,281.20816905)
\curveto(92.86213457,280.89604448)(93.08626918,280.46362191)(93.23318095,279.91090133)
\curveto(93.66261534,280.53515046)(94.05438004,280.98382952)(94.40847507,281.25693851)
\curveto(94.76257009,281.5365501)(95.14115041,281.67635589)(95.54421602,281.67635589)
\curveto(96.2373382,281.67635589)(96.74776241,281.31221057)(97.07548865,280.58391992)
\curveto(97.40698186,279.86213187)(97.57272847,278.85097834)(97.57272847,277.55045932)
\closepath
}
}
{
\newrgbcolor{curcolor}{0 0 0}
\pscustom[linestyle=none,fillstyle=solid,fillcolor=curcolor]
{
\newpath
\moveto(104.99742385,275.92155926)
\curveto(104.99742385,274.14635081)(104.73373606,272.74504157)(104.2063605,271.71763155)
\curveto(103.67898493,270.69022153)(102.97267836,270.17651652)(102.08744081,270.17651652)
\curveto(101.19466931,270.17651652)(100.48459578,270.69022153)(99.95722021,271.71763155)
\curveto(99.43361161,272.74504157)(99.17180731,274.14635081)(99.17180731,275.92155926)
\curveto(99.17180731,277.69676771)(99.43361161,279.09807695)(99.95722021,280.12548697)
\curveto(100.48459578,281.15939958)(101.19466931,281.67635589)(102.08744081,281.67635589)
\curveto(102.97267836,281.67635589)(103.67898493,281.15939958)(104.2063605,280.12548697)
\curveto(104.73373606,279.09807695)(104.99742385,277.69676771)(104.99742385,275.92155926)
\closepath
\moveto(103.90123606,275.92155926)
\curveto(103.90123606,277.33262239)(103.74113991,278.37954019)(103.4209476,279.06231267)
\curveto(103.10075529,279.75158775)(102.65625302,280.09622529)(102.08744081,280.09622529)
\curveto(101.51109465,280.09622529)(101.06282542,279.75158775)(100.74263311,279.06231267)
\curveto(100.42620777,278.37954019)(100.2679951,277.33262239)(100.2679951,275.92155926)
\curveto(100.2679951,274.5560143)(100.42809125,273.51885038)(100.74828356,272.81006752)
\curveto(101.06847587,272.10778726)(101.51486162,271.75664712)(102.08744081,271.75664712)
\curveto(102.65248606,271.75664712)(103.09510484,272.10453596)(103.41529714,272.80031363)
\curveto(103.73925642,273.5025939)(103.90123606,274.54300911)(103.90123606,275.92155926)
\closepath
}
}
{
\newrgbcolor{curcolor}{0 0 0}
\pscustom[linestyle=none,fillstyle=solid,fillcolor=curcolor]
{
\newpath
\moveto(110.58006982,279.37443724)
\lineto(110.5235653,279.37443724)
\curveto(110.36535263,279.43946319)(110.21090692,279.48498135)(110.06022819,279.51099173)
\curveto(109.91331643,279.54350471)(109.7381524,279.5597612)(109.53473611,279.5597612)
\curveto(109.20700986,279.5597612)(108.89058452,279.43296059)(108.58546009,279.17935939)
\curveto(108.28033565,278.93226077)(107.98651212,278.61038232)(107.7039895,278.21372402)
\lineto(107.7039895,270.47888719)
\lineto(106.64170443,270.47888719)
\lineto(106.64170443,281.37398522)
\lineto(107.7039895,281.37398522)
\lineto(107.7039895,279.76459294)
\curveto(108.12588995,280.3498265)(108.49693633,280.76274128)(108.81712864,281.0033373)
\curveto(109.14108792,281.25043591)(109.47069765,281.37398522)(109.80595783,281.37398522)
\curveto(109.99053928,281.37398522)(110.12426665,281.36423133)(110.20713996,281.34472354)
\curveto(110.29001326,281.33171835)(110.41432322,281.30245667)(110.58006982,281.25693851)
\closepath
}
}
{
\newrgbcolor{curcolor}{0 0 0}
\pscustom[linestyle=none,fillstyle=solid,fillcolor=curcolor]
{
\newpath
\moveto(117.03853619,281.37398522)
\lineto(113.35444116,266.46028344)
\lineto(112.2187002,266.46028344)
\lineto(113.39399433,271.00559739)
\lineto(110.87954296,281.37398522)
\lineto(112.03223527,281.37398522)
\lineto(113.97034048,273.29776215)
\lineto(115.92539705,281.37398522)
\closepath
}
}
{
\newrgbcolor{curcolor}{0 0 0}
\pscustom[linestyle=none,fillstyle=solid,fillcolor=curcolor]
{
\newpath
\moveto(129.07965093,271.53230759)
\curveto(128.87246767,271.37624531)(128.68411925,271.22993692)(128.51460568,271.09338243)
\curveto(128.34885907,270.95682793)(128.13037491,270.81377084)(127.85915319,270.66421115)
\curveto(127.62936812,270.54066185)(127.37886472,270.43662032)(127.107643,270.35208659)
\curveto(126.84018825,270.26105026)(126.54448124,270.21553209)(126.22052196,270.21553209)
\curveto(125.61027309,270.21553209)(125.05464526,270.36184048)(124.55363847,270.65445726)
\curveto(124.05639865,270.95357663)(123.62319729,271.41851218)(123.25403439,272.0492639)
\curveto(122.89240543,272.66701043)(122.60988281,273.45057314)(122.40646652,274.39995202)
\curveto(122.20305023,275.35583349)(122.10134208,276.46452595)(122.10134208,277.72602939)
\curveto(122.10134208,278.92250688)(122.19928326,279.99218377)(122.39516561,280.93506005)
\curveto(122.59104797,281.87793634)(122.87357059,282.67450423)(123.24273349,283.32476374)
\curveto(123.60059548,283.95551546)(124.03191335,284.43670749)(124.53668711,284.76833984)
\curveto(125.04522784,285.09997219)(125.6083896,285.26578836)(126.22617241,285.26578836)
\curveto(126.67820861,285.26578836)(127.12836133,285.17150074)(127.57663056,284.98292548)
\curveto(128.02866676,284.79435022)(128.52967355,284.46271787)(129.07965093,283.98802843)
\lineto(129.07965093,281.69586367)
\lineto(128.99489414,281.69586367)
\curveto(128.53155703,282.36563097)(128.0719869,282.8533256)(127.61618373,283.15894756)
\curveto(127.16038056,283.46456953)(126.67255816,283.61738052)(126.15271653,283.61738052)
\curveto(125.72704911,283.61738052)(125.34281834,283.49708251)(125.00002422,283.25648649)
\curveto(124.66099707,283.02239307)(124.35775612,282.65499645)(124.09030136,282.15429663)
\curveto(123.83038055,281.666602)(123.62696426,281.04885547)(123.48005249,280.30105703)
\curveto(123.3369077,279.5597612)(123.2653353,278.70141865)(123.2653353,277.72602939)
\curveto(123.2653353,276.70512196)(123.34444163,275.82727163)(123.5026543,275.09247839)
\curveto(123.66463394,274.35768515)(123.8718172,273.7594464)(124.12420408,273.29776215)
\curveto(124.38789186,272.81657012)(124.69489978,272.45892739)(125.04522784,272.22483397)
\curveto(125.39932286,271.99724314)(125.77225273,271.88344773)(126.16401743,271.88344773)
\curveto(126.70269391,271.88344773)(127.20746766,272.04276131)(127.67833871,272.36138846)
\curveto(128.14920975,272.68001562)(128.58994504,273.15795636)(129.00054459,273.79521067)
\lineto(129.07965093,273.79521067)
\closepath
}
}
{
\newrgbcolor{curcolor}{0 0 0}
\pscustom[linestyle=none,fillstyle=solid,fillcolor=curcolor]
{
\newpath
\moveto(135.95625007,275.92155926)
\curveto(135.95625007,274.14635081)(135.69256229,272.74504157)(135.16518672,271.71763155)
\curveto(134.63781115,270.69022153)(133.93150459,270.17651652)(133.04626703,270.17651652)
\curveto(132.15349553,270.17651652)(131.443422,270.69022153)(130.91604643,271.71763155)
\curveto(130.39243784,272.74504157)(130.13063354,274.14635081)(130.13063354,275.92155926)
\curveto(130.13063354,277.69676771)(130.39243784,279.09807695)(130.91604643,280.12548697)
\curveto(131.443422,281.15939958)(132.15349553,281.67635589)(133.04626703,281.67635589)
\curveto(133.93150459,281.67635589)(134.63781115,281.15939958)(135.16518672,280.12548697)
\curveto(135.69256229,279.09807695)(135.95625007,277.69676771)(135.95625007,275.92155926)
\closepath
\moveto(134.86006228,275.92155926)
\curveto(134.86006228,277.33262239)(134.69996613,278.37954019)(134.37977382,279.06231267)
\curveto(134.05958151,279.75158775)(133.61507925,280.09622529)(133.04626703,280.09622529)
\curveto(132.46992087,280.09622529)(132.02165164,279.75158775)(131.70145933,279.06231267)
\curveto(131.38503399,278.37954019)(131.22682132,277.33262239)(131.22682132,275.92155926)
\curveto(131.22682132,274.5560143)(131.38691748,273.51885038)(131.70710979,272.81006752)
\curveto(132.02730209,272.10778726)(132.47368784,271.75664712)(133.04626703,271.75664712)
\curveto(133.61131228,271.75664712)(134.05393106,272.10453596)(134.37412337,272.80031363)
\curveto(134.69808265,273.5025939)(134.86006228,274.54300911)(134.86006228,275.92155926)
\closepath
}
}
{
\newrgbcolor{curcolor}{0 0 0}
\pscustom[linestyle=none,fillstyle=solid,fillcolor=curcolor]
{
\newpath
\moveto(142.87805878,270.47888719)
\lineto(141.81577371,270.47888719)
\lineto(141.81577371,276.68236288)
\curveto(141.81577371,277.1830627)(141.79882235,277.65124955)(141.76491963,278.08692342)
\curveto(141.73101692,278.52909988)(141.66886194,278.87373742)(141.5784547,279.12083603)
\curveto(141.48428049,279.39394502)(141.34866963,279.59552547)(141.17162212,279.72557737)
\curveto(140.99457461,279.86213187)(140.76478954,279.93040912)(140.48226691,279.93040912)
\curveto(140.19221035,279.93040912)(139.8889694,279.80685981)(139.57254406,279.5597612)
\curveto(139.25611872,279.31266258)(138.95287777,278.99728672)(138.66282121,278.61363362)
\lineto(138.66282121,270.47888719)
\lineto(137.60053614,270.47888719)
\lineto(137.60053614,281.37398522)
\lineto(138.66282121,281.37398522)
\lineto(138.66282121,280.16450254)
\curveto(138.99431442,280.63919198)(139.33710854,281.0098399)(139.69120356,281.27644629)
\curveto(140.04529859,281.54305269)(140.40881103,281.67635589)(140.7817409,281.67635589)
\curveto(141.46356217,281.67635589)(141.9834038,281.32196446)(142.34126579,280.6131816)
\curveto(142.69912778,279.90439874)(142.87805878,278.88349131)(142.87805878,277.55045932)
\closepath
}
}
{
\newrgbcolor{curcolor}{0 0 0}
\pscustom[linestyle=none,fillstyle=solid,fillcolor=curcolor]
{
\newpath
\moveto(148.20078026,270.57642612)
\curveto(148.00113094,270.48538979)(147.78264677,270.41060994)(147.54532777,270.35208659)
\curveto(147.31177573,270.29356323)(147.10270899,270.26430156)(146.91812754,270.26430156)
\curveto(146.27397595,270.26430156)(145.78427007,270.56342093)(145.44900989,271.16165967)
\curveto(145.11374971,271.75989842)(144.94611961,272.71903119)(144.94611961,274.03905799)
\lineto(144.94611961,279.83287019)
\lineto(144.22851215,279.83287019)
\lineto(144.22851215,281.37398522)
\lineto(144.94611961,281.37398522)
\lineto(144.94611961,284.50498474)
\lineto(146.00840469,284.50498474)
\lineto(146.00840469,281.37398522)
\lineto(148.20078026,281.37398522)
\lineto(148.20078026,279.83287019)
\lineto(146.00840469,279.83287019)
\lineto(146.00840469,274.86813886)
\curveto(146.00840469,274.29591049)(146.01593862,273.84723143)(146.0310065,273.52210168)
\curveto(146.04607437,273.20347452)(146.09881193,272.90435515)(146.18921917,272.62474356)
\curveto(146.27209247,272.36463976)(146.38510152,272.17281321)(146.52824632,272.0492639)
\curveto(146.67515808,271.93221719)(146.89740921,271.87369383)(147.19499971,271.87369383)
\curveto(147.36828025,271.87369383)(147.54909474,271.9159607)(147.73744315,272.00049444)
\curveto(147.92579157,272.09153077)(148.06140243,272.16631061)(148.14427573,272.22483397)
\lineto(148.20078026,272.22483397)
\closepath
}
}
{
\newrgbcolor{curcolor}{0 0 0}
\pscustom[linestyle=none,fillstyle=solid,fillcolor=curcolor]
{
\newpath
\moveto(153.42179866,279.37443724)
\lineto(153.36529414,279.37443724)
\curveto(153.20708147,279.43946319)(153.05263576,279.48498135)(152.90195703,279.51099173)
\curveto(152.75504527,279.54350471)(152.57988124,279.5597612)(152.37646495,279.5597612)
\curveto(152.0487387,279.5597612)(151.73231336,279.43296059)(151.42718893,279.17935939)
\curveto(151.12206449,278.93226077)(150.82824096,278.61038232)(150.54571834,278.21372402)
\lineto(150.54571834,270.47888719)
\lineto(149.48343326,270.47888719)
\lineto(149.48343326,281.37398522)
\lineto(150.54571834,281.37398522)
\lineto(150.54571834,279.76459294)
\curveto(150.96761879,280.3498265)(151.33866517,280.76274128)(151.65885748,281.0033373)
\curveto(151.98281676,281.25043591)(152.31242649,281.37398522)(152.64768667,281.37398522)
\curveto(152.83226812,281.37398522)(152.96599549,281.36423133)(153.0488688,281.34472354)
\curveto(153.1317421,281.33171835)(153.25605205,281.30245667)(153.42179866,281.25693851)
\closepath
}
}
{
\newrgbcolor{curcolor}{0 0 0}
\pscustom[linestyle=none,fillstyle=solid,fillcolor=curcolor]
{
\newpath
\moveto(159.8011587,275.92155926)
\curveto(159.8011587,274.14635081)(159.53747091,272.74504157)(159.01009535,271.71763155)
\curveto(158.48271978,270.69022153)(157.77641321,270.17651652)(156.89117566,270.17651652)
\curveto(155.99840416,270.17651652)(155.28833063,270.69022153)(154.76095506,271.71763155)
\curveto(154.23734646,272.74504157)(153.97554216,274.14635081)(153.97554216,275.92155926)
\curveto(153.97554216,277.69676771)(154.23734646,279.09807695)(154.76095506,280.12548697)
\curveto(155.28833063,281.15939958)(155.99840416,281.67635589)(156.89117566,281.67635589)
\curveto(157.77641321,281.67635589)(158.48271978,281.15939958)(159.01009535,280.12548697)
\curveto(159.53747091,279.09807695)(159.8011587,277.69676771)(159.8011587,275.92155926)
\closepath
\moveto(158.70497091,275.92155926)
\curveto(158.70497091,277.33262239)(158.54487476,278.37954019)(158.22468245,279.06231267)
\curveto(157.90449014,279.75158775)(157.45998787,280.09622529)(156.89117566,280.09622529)
\curveto(156.3148295,280.09622529)(155.86656027,279.75158775)(155.54636796,279.06231267)
\curveto(155.22994262,278.37954019)(155.07172995,277.33262239)(155.07172995,275.92155926)
\curveto(155.07172995,274.5560143)(155.2318261,273.51885038)(155.55201841,272.81006752)
\curveto(155.87221072,272.10778726)(156.31859647,271.75664712)(156.89117566,271.75664712)
\curveto(157.45622091,271.75664712)(157.89883969,272.10453596)(158.21903199,272.80031363)
\curveto(158.54299127,273.5025939)(158.70497091,274.54300911)(158.70497091,275.92155926)
\closepath
}
}
{
\newrgbcolor{curcolor}{0 0 0}
\pscustom[linestyle=none,fillstyle=solid,fillcolor=curcolor]
{
\newpath
\moveto(162.51903074,270.47888719)
\lineto(161.45674567,270.47888719)
\lineto(161.45674567,285.65594407)
\lineto(162.51903074,285.65594407)
\closepath
}
}
{
\newrgbcolor{curcolor}{0 0 0}
\pscustom[linestyle=none,fillstyle=solid,fillcolor=curcolor]
{
\newpath
\moveto(165.69458464,270.47888719)
\lineto(164.63229957,270.47888719)
\lineto(164.63229957,285.65594407)
\lineto(165.69458464,285.65594407)
\closepath
}
}
{
\newrgbcolor{curcolor}{0 0 0}
\pscustom[linestyle=none,fillstyle=solid,fillcolor=curcolor]
{
\newpath
\moveto(173.07972018,275.7362353)
\lineto(168.42939776,275.7362353)
\curveto(168.42939776,275.06646801)(168.48778577,274.48123445)(168.60456179,273.98053463)
\curveto(168.72133781,273.48633741)(168.88143396,273.07992522)(169.08485025,272.76129806)
\curveto(169.28073261,272.4491735)(169.51240116,272.21508007)(169.77985591,272.05901779)
\curveto(170.05107763,271.90295551)(170.34866813,271.82492437)(170.67262741,271.82492437)
\curveto(171.1020618,271.82492437)(171.53337967,271.97123276)(171.96658103,272.26384954)
\curveto(172.40354936,272.56296891)(172.71432425,272.85558569)(172.8989057,273.14169987)
\lineto(172.95541022,273.14169987)
\lineto(172.95541022,271.14215189)
\curveto(172.59754823,270.88204809)(172.2321523,270.66421115)(171.85922243,270.48864109)
\curveto(171.48629257,270.31307102)(171.09452786,270.22528599)(170.68392831,270.22528599)
\curveto(169.63671112,270.22528599)(168.81927899,270.71298061)(168.23163193,271.68836987)
\curveto(167.64398487,272.67026173)(167.35016133,274.06181707)(167.35016133,275.8630359)
\curveto(167.35016133,277.64474695)(167.63080048,279.05906138)(168.19207876,280.10597918)
\curveto(168.75712401,281.15289699)(169.49921677,281.67635589)(170.41835705,281.67635589)
\curveto(171.26969189,281.67635589)(171.92514438,281.24718462)(172.38471452,280.38884207)
\curveto(172.84805162,279.53049952)(173.07972018,278.31126294)(173.07972018,276.73113234)
\closepath
\moveto(172.04568737,277.14079583)
\curveto(172.0419204,278.1031799)(171.90065909,278.84772704)(171.62190343,279.37443724)
\curveto(171.34691474,279.90114744)(170.92689777,280.16450254)(170.36185252,280.16450254)
\curveto(169.7930403,280.16450254)(169.33912062,279.87513706)(169.00009347,279.2964061)
\curveto(168.66483328,278.71767514)(168.47460138,277.99913838)(168.42939776,277.14079583)
\closepath
}
}
{
\newrgbcolor{curcolor}{0 0 0}
\pscustom[linestyle=none,fillstyle=solid,fillcolor=curcolor]
{
\newpath
\moveto(178.62846866,279.37443724)
\lineto(178.57196413,279.37443724)
\curveto(178.41375146,279.43946319)(178.25930576,279.48498135)(178.10862703,279.51099173)
\curveto(177.96171526,279.54350471)(177.78655123,279.5597612)(177.58313494,279.5597612)
\curveto(177.2554087,279.5597612)(176.93898336,279.43296059)(176.63385892,279.17935939)
\curveto(176.32873449,278.93226077)(176.03491096,278.61038232)(175.75238833,278.21372402)
\lineto(175.75238833,270.47888719)
\lineto(174.69010326,270.47888719)
\lineto(174.69010326,281.37398522)
\lineto(175.75238833,281.37398522)
\lineto(175.75238833,279.76459294)
\curveto(176.17428879,280.3498265)(176.54533517,280.76274128)(176.86552748,281.0033373)
\curveto(177.18948675,281.25043591)(177.51909648,281.37398522)(177.85435666,281.37398522)
\curveto(178.03893811,281.37398522)(178.17266549,281.36423133)(178.25553879,281.34472354)
\curveto(178.3384121,281.33171835)(178.46272205,281.30245667)(178.62846866,281.25693851)
\closepath
}
}
{
\newrgbcolor{curcolor}{0 0 0}
\pscustom[linestyle=none,fillstyle=solid,fillcolor=curcolor]
{
\newpath
\moveto(218.73842821,339.95904512)
\lineto(213.82215145,339.95904512)
\lineto(213.82215145,341.43821383)
\lineto(215.45254935,341.43821383)
\lineto(215.45254935,352.969891)
\lineto(213.82215145,352.969891)
\lineto(213.82215145,354.44905972)
\lineto(218.73842821,354.44905972)
\lineto(218.73842821,352.969891)
\lineto(217.1080303,352.969891)
\lineto(217.1080303,341.43821383)
\lineto(218.73842821,341.43821383)
\closepath
}
}
{
\newrgbcolor{curcolor}{0 0 0}
\pscustom[linestyle=none,fillstyle=solid,fillcolor=curcolor]
{
\newpath
\moveto(229.23986309,339.95904512)
\lineto(227.66799229,339.95904512)
\lineto(227.66799229,346.14819843)
\curveto(227.66799229,346.64774225)(227.64290924,347.11484816)(227.59274315,347.54951616)
\curveto(227.54257706,347.99067175)(227.4506059,348.3345136)(227.31682966,348.58104172)
\curveto(227.17747941,348.85352017)(226.97681505,349.05463521)(226.71483659,349.18438685)
\curveto(226.45285812,349.32062608)(226.11284351,349.38874569)(225.69479277,349.38874569)
\curveto(225.265594,349.38874569)(224.8168862,349.26548163)(224.34866937,349.01895351)
\curveto(223.88045254,348.77242539)(223.43174474,348.45777766)(223.00254597,348.07501031)
\lineto(223.00254597,339.95904512)
\lineto(221.43067517,339.95904512)
\lineto(221.43067517,350.82898891)
\lineto(223.00254597,350.82898891)
\lineto(223.00254597,349.62229864)
\curveto(223.49305885,350.09589214)(224.00029375,350.46568432)(224.52425068,350.73167518)
\curveto(225.04820762,350.99766605)(225.58609958,351.13066148)(226.13792656,351.13066148)
\curveto(227.14682236,351.13066148)(227.91603573,350.77708826)(228.44556667,350.06994181)
\curveto(228.97509761,349.36279536)(229.23986309,348.34424497)(229.23986309,347.01429064)
\closepath
}
}
{
\newrgbcolor{curcolor}{0 0 0}
\pscustom[linestyle=none,fillstyle=solid,fillcolor=curcolor]
{
\newpath
\moveto(237.11594034,340.05635885)
\curveto(236.82051781,339.9655327)(236.49722523,339.89092551)(236.14606261,339.83253727)
\curveto(235.80047399,339.77414903)(235.49111644,339.74495491)(235.21798995,339.74495491)
\curveto(234.26483426,339.74495491)(233.54021296,340.04338368)(233.04412608,340.64024124)
\curveto(232.5480392,341.23709879)(232.29999575,342.19401715)(232.29999575,343.51099631)
\lineto(232.29999575,349.29143196)
\lineto(231.23814686,349.29143196)
\lineto(231.23814686,350.82898891)
\lineto(232.29999575,350.82898891)
\lineto(232.29999575,353.95275969)
\lineto(233.87186656,353.95275969)
\lineto(233.87186656,350.82898891)
\lineto(237.11594034,350.82898891)
\lineto(237.11594034,349.29143196)
\lineto(233.87186656,349.29143196)
\lineto(233.87186656,344.33816303)
\curveto(233.87186656,343.76725581)(233.88301458,343.31961264)(233.90531062,342.99523354)
\curveto(233.92760665,342.67734201)(234.00564279,342.37891324)(234.13941903,342.09994721)
\curveto(234.26204725,341.84044392)(234.42926755,341.64906025)(234.64107993,341.52579619)
\curveto(234.85846631,341.40901971)(235.1873329,341.35063148)(235.62767968,341.35063148)
\curveto(235.88408414,341.35063148)(236.15163662,341.39280076)(236.43033712,341.47713933)
\curveto(236.70903761,341.56796548)(236.90970197,341.64257267)(237.03233019,341.70096091)
\lineto(237.11594034,341.70096091)
\closepath
}
}
{
\newrgbcolor{curcolor}{0 0 0}
\pscustom[linestyle=none,fillstyle=solid,fillcolor=curcolor]
{
\newpath
\moveto(246.83143913,345.20425524)
\lineto(239.95032386,345.20425524)
\curveto(239.95032386,344.53603428)(240.03672102,343.95215189)(240.20951532,343.45260807)
\curveto(240.38230963,342.95955183)(240.61920505,342.55407795)(240.92020159,342.23618643)
\curveto(241.21005011,341.92478249)(241.55285172,341.69122954)(241.94860642,341.53552757)
\curveto(242.34993514,341.3798256)(242.79028192,341.30197461)(243.26964678,341.30197461)
\curveto(243.90508391,341.30197461)(244.54330805,341.44794521)(245.18431919,341.7398864)
\curveto(245.83090434,342.03831518)(246.29076016,342.33025637)(246.56388665,342.61570998)
\lineto(246.6474968,342.61570998)
\lineto(246.6474968,340.62077849)
\curveto(246.11796585,340.36127521)(245.57728689,340.14394121)(245.02545991,339.96877649)
\curveto(244.47363292,339.79361177)(243.89393589,339.70602942)(243.28636881,339.70602942)
\curveto(241.73679405,339.70602942)(240.52723389,340.19259807)(239.65768834,341.16573539)
\curveto(238.78814279,342.14536028)(238.35337002,343.53370285)(238.35337002,345.33076309)
\curveto(238.35337002,347.10836058)(238.76863376,348.51940969)(239.59916124,349.5639104)
\curveto(240.43526273,350.60841112)(241.53334268,351.13066148)(242.89340111,351.13066148)
\curveto(244.15312735,351.13066148)(245.12300508,350.70248106)(245.80303429,349.84612023)
\curveto(246.48863751,348.98975939)(246.83143913,347.77333775)(246.83143913,346.1968553)
\closepath
\moveto(245.3013734,346.60557297)
\curveto(245.29579939,347.56573512)(245.08677402,348.30856327)(244.67429728,348.83405742)
\curveto(244.26739456,349.35955157)(243.64589245,349.62229864)(242.80979096,349.62229864)
\curveto(241.96811546,349.62229864)(241.29644726,349.33360124)(240.79478637,348.75620643)
\curveto(240.29869948,348.17881163)(240.01721198,347.46193381)(239.95032386,346.60557297)
\closepath
}
}
{
\newrgbcolor{curcolor}{0 0 0}
\pscustom[linestyle=none,fillstyle=solid,fillcolor=curcolor]
{
\newpath
\moveto(255.04195491,348.83405742)
\lineto(254.95834476,348.83405742)
\curveto(254.72423635,348.89893324)(254.49570194,348.94434631)(254.27274154,348.97029664)
\curveto(254.05535515,349.00273455)(253.79616369,349.01895351)(253.49516716,349.01895351)
\curveto(253.01022829,349.01895351)(252.54201146,348.89244566)(252.09051665,348.63942996)
\curveto(251.63902185,348.39290184)(251.20424907,348.07176652)(250.78619833,347.67602402)
\lineto(250.78619833,339.95904512)
\lineto(249.21432753,339.95904512)
\lineto(249.21432753,350.82898891)
\lineto(250.78619833,350.82898891)
\lineto(250.78619833,349.22331234)
\curveto(251.41048744,349.80719473)(251.95952742,350.2191562)(252.43331826,350.45919673)
\curveto(252.91268312,350.70572485)(253.40040899,350.82898891)(253.89649587,350.82898891)
\curveto(254.16962236,350.82898891)(254.36749971,350.81925754)(254.49012793,350.79979479)
\curveto(254.61275615,350.78681963)(254.79669848,350.75762551)(255.04195491,350.71221243)
\closepath
}
}
{
\newrgbcolor{curcolor}{0 0 0}
\pscustom[linestyle=none,fillstyle=solid,fillcolor=curcolor]
{
\newpath
\moveto(263.4280544,340.64024124)
\curveto(262.90409747,340.34830004)(262.40522358,340.12123467)(261.93143273,339.95904512)
\curveto(261.4632159,339.79685556)(260.96434201,339.71576079)(260.43481107,339.71576079)
\curveto(259.76035586,339.71576079)(259.14164076,339.82929348)(258.57866576,340.05635885)
\curveto(258.01569075,340.2899118)(257.53353889,340.64024124)(257.13221018,341.10734715)
\curveto(256.72530745,341.57445306)(256.41037589,342.16482303)(256.1874155,342.87845706)
\curveto(255.9644551,343.59209109)(255.8529749,344.42574539)(255.8529749,345.37941996)
\curveto(255.8529749,347.15701745)(256.27102564,348.5518476)(257.10712713,349.5639104)
\curveto(257.94880263,350.57597321)(259.05803061,351.08200461)(260.43481107,351.08200461)
\curveto(260.96991602,351.08200461)(261.49387295,350.99442226)(262.00668187,350.81925754)
\curveto(262.52506479,350.64409282)(262.99885563,350.43000261)(263.4280544,350.17698691)
\lineto(263.4280544,348.14312993)
\lineto(263.34444425,348.14312993)
\curveto(262.8650794,348.57779793)(262.36899251,348.9119084)(261.8561836,349.14546136)
\curveto(261.34894869,349.37901431)(260.85286181,349.49579079)(260.36792295,349.49579079)
\curveto(259.47608136,349.49579079)(258.7709691,349.14546136)(258.25258618,348.44480249)
\curveto(257.73977726,347.75063121)(257.48337281,346.72883703)(257.48337281,345.37941996)
\curveto(257.48337281,344.06892837)(257.73420325,343.06010936)(258.23586415,342.35296291)
\curveto(258.74309905,341.65230404)(259.45378532,341.30197461)(260.36792295,341.30197461)
\curveto(260.68564151,341.30197461)(261.00893409,341.35063148)(261.33780067,341.44794521)
\curveto(261.66666726,341.54525894)(261.96208979,341.67176679)(262.22406825,341.82746876)
\curveto(262.45260266,341.96370798)(262.66720204,342.10643479)(262.8678664,342.25564918)
\curveto(263.06853076,342.41135115)(263.22739004,342.54434658)(263.34444425,342.65463548)
\lineto(263.4280544,342.65463548)
\closepath
}
}
{
\newrgbcolor{curcolor}{0 0 0}
\pscustom[linestyle=none,fillstyle=solid,fillcolor=curcolor]
{
\newpath
\moveto(273.40274365,345.38915133)
\curveto(273.40274365,343.61804142)(273.01256295,342.21996748)(272.23220156,341.19492951)
\curveto(271.45184017,340.16989153)(270.40671331,339.65737255)(269.09682097,339.65737255)
\curveto(267.77578062,339.65737255)(266.72507975,340.16989153)(265.94471836,341.19492951)
\curveto(265.16993098,342.21996748)(264.78253729,343.61804142)(264.78253729,345.38915133)
\curveto(264.78253729,347.16026124)(265.16993098,348.55833518)(265.94471836,349.58337315)
\curveto(266.72507975,350.6148987)(267.77578062,351.13066148)(269.09682097,351.13066148)
\curveto(270.40671331,351.13066148)(271.45184017,350.6148987)(272.23220156,349.58337315)
\curveto(273.01256295,348.55833518)(273.40274365,347.16026124)(273.40274365,345.38915133)
\closepath
\moveto(271.78070676,345.38915133)
\curveto(271.78070676,346.79695664)(271.54381134,347.84145736)(271.07002049,348.52265348)
\curveto(270.59622965,349.21033718)(269.93849647,349.55417903)(269.09682097,349.55417903)
\curveto(268.24399745,349.55417903)(267.58069027,349.21033718)(267.10689943,348.52265348)
\curveto(266.63868259,347.84145736)(266.40457418,346.79695664)(266.40457418,345.38915133)
\curveto(266.40457418,344.02675909)(266.6414696,342.99198975)(267.11526044,342.2848433)
\curveto(267.58905129,341.58418443)(268.24957146,341.233855)(269.09682097,341.233855)
\curveto(269.93292246,341.233855)(270.58786863,341.58094064)(271.06165948,342.27511192)
\curveto(271.54102433,342.97577079)(271.78070676,344.01378392)(271.78070676,345.38915133)
\closepath
}
}
{
\newrgbcolor{curcolor}{0 0 0}
\pscustom[linestyle=none,fillstyle=solid,fillcolor=curcolor]
{
\newpath
\moveto(283.64498891,339.95904512)
\lineto(282.07311811,339.95904512)
\lineto(282.07311811,346.14819843)
\curveto(282.07311811,346.64774225)(282.04803506,347.11484816)(281.99786897,347.54951616)
\curveto(281.94770289,347.99067175)(281.85573172,348.3345136)(281.72195548,348.58104172)
\curveto(281.58260523,348.85352017)(281.38194088,349.05463521)(281.11996241,349.18438685)
\curveto(280.85798394,349.32062608)(280.51796934,349.38874569)(280.09991859,349.38874569)
\curveto(279.67071983,349.38874569)(279.22201203,349.26548163)(278.75379519,349.01895351)
\curveto(278.28557836,348.77242539)(277.83687056,348.45777766)(277.40767179,348.07501031)
\lineto(277.40767179,339.95904512)
\lineto(275.83580099,339.95904512)
\lineto(275.83580099,350.82898891)
\lineto(277.40767179,350.82898891)
\lineto(277.40767179,349.62229864)
\curveto(277.89818467,350.09589214)(278.40541957,350.46568432)(278.92937651,350.73167518)
\curveto(279.45333344,350.99766605)(279.9912254,351.13066148)(280.54305238,351.13066148)
\curveto(281.55194818,351.13066148)(282.32116155,350.77708826)(282.85069249,350.06994181)
\curveto(283.38022344,349.36279536)(283.64498891,348.34424497)(283.64498891,347.01429064)
\closepath
}
}
{
\newrgbcolor{curcolor}{0 0 0}
\pscustom[linestyle=none,fillstyle=solid,fillcolor=curcolor]
{
\newpath
\moveto(294.48086293,339.95904512)
\lineto(292.90899213,339.95904512)
\lineto(292.90899213,346.14819843)
\curveto(292.90899213,346.64774225)(292.88390908,347.11484816)(292.83374299,347.54951616)
\curveto(292.7835769,347.99067175)(292.69160574,348.3345136)(292.5578295,348.58104172)
\curveto(292.41847925,348.85352017)(292.2178149,349.05463521)(291.95583643,349.18438685)
\curveto(291.69385796,349.32062608)(291.35384336,349.38874569)(290.93579261,349.38874569)
\curveto(290.50659385,349.38874569)(290.05788605,349.26548163)(289.58966921,349.01895351)
\curveto(289.12145238,348.77242539)(288.67274458,348.45777766)(288.24354581,348.07501031)
\lineto(288.24354581,339.95904512)
\lineto(286.67167501,339.95904512)
\lineto(286.67167501,350.82898891)
\lineto(288.24354581,350.82898891)
\lineto(288.24354581,349.62229864)
\curveto(288.73405869,350.09589214)(289.24129359,350.46568432)(289.76525053,350.73167518)
\curveto(290.28920746,350.99766605)(290.82709942,351.13066148)(291.3789264,351.13066148)
\curveto(292.3878222,351.13066148)(293.15703557,350.77708826)(293.68656651,350.06994181)
\curveto(294.21609746,349.36279536)(294.48086293,348.34424497)(294.48086293,347.01429064)
\closepath
}
}
{
\newrgbcolor{curcolor}{0 0 0}
\pscustom[linestyle=none,fillstyle=solid,fillcolor=curcolor]
{
\newpath
\moveto(305.32509796,345.20425524)
\lineto(298.4439827,345.20425524)
\curveto(298.4439827,344.53603428)(298.53037985,343.95215189)(298.70317416,343.45260807)
\curveto(298.87596847,342.95955183)(299.11286389,342.55407795)(299.41386043,342.23618643)
\curveto(299.70370895,341.92478249)(300.04651056,341.69122954)(300.44226526,341.53552757)
\curveto(300.84359398,341.3798256)(301.28394076,341.30197461)(301.76330562,341.30197461)
\curveto(302.39874275,341.30197461)(303.03696689,341.44794521)(303.67797803,341.7398864)
\curveto(304.32456318,342.03831518)(304.784419,342.33025637)(305.05754549,342.61570998)
\lineto(305.14115564,342.61570998)
\lineto(305.14115564,340.62077849)
\curveto(304.61162469,340.36127521)(304.07094573,340.14394121)(303.51911874,339.96877649)
\curveto(302.96729176,339.79361177)(302.38759473,339.70602942)(301.78002765,339.70602942)
\curveto(300.23045288,339.70602942)(299.02089273,340.19259807)(298.15134718,341.16573539)
\curveto(297.28180163,342.14536028)(296.84702885,343.53370285)(296.84702885,345.33076309)
\curveto(296.84702885,347.10836058)(297.26229259,348.51940969)(298.09282007,349.5639104)
\curveto(298.92892156,350.60841112)(300.02700152,351.13066148)(301.38705995,351.13066148)
\curveto(302.64678619,351.13066148)(303.61666392,350.70248106)(304.29669313,349.84612023)
\curveto(304.98229635,348.98975939)(305.32509796,347.77333775)(305.32509796,346.1968553)
\closepath
\moveto(303.79503224,346.60557297)
\curveto(303.78945823,347.56573512)(303.58043285,348.30856327)(303.16795612,348.83405742)
\curveto(302.76105339,349.35955157)(302.13955129,349.62229864)(301.3034498,349.62229864)
\curveto(300.4617743,349.62229864)(299.7901061,349.33360124)(299.28844521,348.75620643)
\curveto(298.79235832,348.17881163)(298.51087082,347.46193381)(298.4439827,346.60557297)
\closepath
}
}
{
\newrgbcolor{curcolor}{0 0 0}
\pscustom[linestyle=none,fillstyle=solid,fillcolor=curcolor]
{
\newpath
\moveto(314.61418467,340.64024124)
\curveto(314.09022774,340.34830004)(313.59135385,340.12123467)(313.11756301,339.95904512)
\curveto(312.64934617,339.79685556)(312.15047228,339.71576079)(311.62094134,339.71576079)
\curveto(310.94648614,339.71576079)(310.32777103,339.82929348)(309.76479603,340.05635885)
\curveto(309.20182103,340.2899118)(308.71966917,340.64024124)(308.31834045,341.10734715)
\curveto(307.91143773,341.57445306)(307.59650617,342.16482303)(307.37354577,342.87845706)
\curveto(307.15058537,343.59209109)(307.03910517,344.42574539)(307.03910517,345.37941996)
\curveto(307.03910517,347.15701745)(307.45715592,348.5518476)(308.29325741,349.5639104)
\curveto(309.13493291,350.57597321)(310.24416088,351.08200461)(311.62094134,351.08200461)
\curveto(312.15604629,351.08200461)(312.68000323,350.99442226)(313.19281214,350.81925754)
\curveto(313.71119506,350.64409282)(314.18498591,350.43000261)(314.61418467,350.17698691)
\lineto(314.61418467,348.14312993)
\lineto(314.53057452,348.14312993)
\curveto(314.05120967,348.57779793)(313.55512279,348.9119084)(313.04231387,349.14546136)
\curveto(312.53507897,349.37901431)(312.03899208,349.49579079)(311.55405322,349.49579079)
\curveto(310.66221163,349.49579079)(309.95709937,349.14546136)(309.43871645,348.44480249)
\curveto(308.92590754,347.75063121)(308.66950308,346.72883703)(308.66950308,345.37941996)
\curveto(308.66950308,344.06892837)(308.92033353,343.06010936)(309.42199442,342.35296291)
\curveto(309.92922932,341.65230404)(310.63991559,341.30197461)(311.55405322,341.30197461)
\curveto(311.87177179,341.30197461)(312.19506436,341.35063148)(312.52393095,341.44794521)
\curveto(312.85279753,341.54525894)(313.14822006,341.67176679)(313.41019853,341.82746876)
\curveto(313.63873293,341.96370798)(313.85333232,342.10643479)(314.05399667,342.25564918)
\curveto(314.25466103,342.41135115)(314.41352032,342.54434658)(314.53057452,342.65463548)
\lineto(314.61418467,342.65463548)
\closepath
}
}
{
\newrgbcolor{curcolor}{0 0 0}
\pscustom[linestyle=none,fillstyle=solid,fillcolor=curcolor]
{
\newpath
\moveto(321.47857972,340.05635885)
\curveto(321.18315719,339.9655327)(320.85986462,339.89092551)(320.50870199,339.83253727)
\curveto(320.16311338,339.77414903)(319.85375582,339.74495491)(319.58062934,339.74495491)
\curveto(318.62747364,339.74495491)(317.90285235,340.04338368)(317.40676546,340.64024124)
\curveto(316.91067858,341.23709879)(316.66263514,342.19401715)(316.66263514,343.51099631)
\lineto(316.66263514,349.29143196)
\lineto(315.60078624,349.29143196)
\lineto(315.60078624,350.82898891)
\lineto(316.66263514,350.82898891)
\lineto(316.66263514,353.95275969)
\lineto(318.23450594,353.95275969)
\lineto(318.23450594,350.82898891)
\lineto(321.47857972,350.82898891)
\lineto(321.47857972,349.29143196)
\lineto(318.23450594,349.29143196)
\lineto(318.23450594,344.33816303)
\curveto(318.23450594,343.76725581)(318.24565396,343.31961264)(318.26795,342.99523354)
\curveto(318.29024604,342.67734201)(318.36828218,342.37891324)(318.50205841,342.09994721)
\curveto(318.62468663,341.84044392)(318.79190693,341.64906025)(319.00371931,341.52579619)
\curveto(319.2211057,341.40901971)(319.54997228,341.35063148)(319.99031907,341.35063148)
\curveto(320.24672352,341.35063148)(320.514276,341.39280076)(320.7929765,341.47713933)
\curveto(321.07167699,341.56796548)(321.27234135,341.64257267)(321.39496957,341.70096091)
\lineto(321.47857972,341.70096091)
\closepath
}
}
{
\newrgbcolor{curcolor}{0.7019608 0.7019608 0.7019608}
\pscustom[linestyle=none,fillstyle=solid,fillcolor=curcolor]
{
\newpath
\moveto(410.08601575,262.18161094)
\lineto(410.08601575,169.2475259)
}
}
{
\newrgbcolor{curcolor}{0 0 0}
\pscustom[linewidth=2.64566925,linecolor=curcolor]
{
\newpath
\moveto(410.08601575,262.18161094)
\lineto(410.08601575,169.2475259)
}
}
{
\newrgbcolor{curcolor}{0.80000001 0.80000001 0.80000001}
\pscustom[linestyle=none,fillstyle=solid,fillcolor=curcolor]
{
\newpath
\moveto(345.61675229,293.97727683)
\lineto(474.55525481,293.97727683)
\lineto(474.55525481,258.1390743)
\lineto(345.61675229,258.1390743)
\closepath
}
}
{
\newrgbcolor{curcolor}{0 0 0}
\pscustom[linewidth=0.27058431,linecolor=curcolor]
{
\newpath
\moveto(345.61675229,293.97727683)
\lineto(474.55525481,293.97727683)
\lineto(474.55525481,258.1390743)
\lineto(345.61675229,258.1390743)
\closepath
}
}
{
\newrgbcolor{curcolor}{0 0 0}
\pscustom[linestyle=none,fillstyle=solid,fillcolor=curcolor]
{
\newpath
\moveto(364.16478815,270.47888719)
\lineto(363.04599855,270.47888719)
\lineto(363.04599855,282.99313139)
\lineto(360.70671122,274.47798316)
\lineto(360.03995782,274.47798316)
\lineto(357.71762184,282.99313139)
\lineto(357.71762184,270.47888719)
\lineto(356.67228813,270.47888719)
\lineto(356.67228813,285.00243326)
\lineto(358.1979103,285.00243326)
\lineto(360.44113995,276.9164563)
\lineto(362.61091371,285.00243326)
\lineto(364.16478815,285.00243326)
\closepath
}
}
{
\newrgbcolor{curcolor}{0 0 0}
\pscustom[linestyle=none,fillstyle=solid,fillcolor=curcolor]
{
\newpath
\moveto(371.62338515,275.7362353)
\lineto(366.97306274,275.7362353)
\curveto(366.97306274,275.06646801)(367.03145075,274.48123445)(367.14822677,273.98053463)
\curveto(367.26500279,273.48633741)(367.42509894,273.07992522)(367.62851523,272.76129806)
\curveto(367.82439759,272.4491735)(368.05606614,272.21508007)(368.32352089,272.05901779)
\curveto(368.59474261,271.90295551)(368.89233311,271.82492437)(369.21629239,271.82492437)
\curveto(369.64572678,271.82492437)(370.07704465,271.97123276)(370.51024601,272.26384954)
\curveto(370.94721434,272.56296891)(371.25798923,272.85558569)(371.44257067,273.14169987)
\lineto(371.4990752,273.14169987)
\lineto(371.4990752,271.14215189)
\curveto(371.14121321,270.88204809)(370.77581728,270.66421115)(370.40288741,270.48864109)
\curveto(370.02995755,270.31307102)(369.63819284,270.22528599)(369.22759329,270.22528599)
\curveto(368.18037609,270.22528599)(367.36294397,270.71298061)(366.7752969,271.68836987)
\curveto(366.18764984,272.67026173)(365.89382631,274.06181707)(365.89382631,275.8630359)
\curveto(365.89382631,277.64474695)(366.17446545,279.05906138)(366.73574374,280.10597918)
\curveto(367.30078899,281.15289699)(368.04288175,281.67635589)(368.96202202,281.67635589)
\curveto(369.81335687,281.67635589)(370.46880936,281.24718462)(370.9283795,280.38884207)
\curveto(371.3917166,279.53049952)(371.62338515,278.31126294)(371.62338515,276.73113234)
\closepath
\moveto(370.58935235,277.14079583)
\curveto(370.58558538,278.1031799)(370.44432407,278.84772704)(370.16556841,279.37443724)
\curveto(369.89057972,279.90114744)(369.47056275,280.16450254)(368.9055175,280.16450254)
\curveto(368.33670528,280.16450254)(367.8827856,279.87513706)(367.54375845,279.2964061)
\curveto(367.20849826,278.71767514)(367.01826636,277.99913838)(366.97306274,277.14079583)
\closepath
}
}
{
\newrgbcolor{curcolor}{0 0 0}
\pscustom[linestyle=none,fillstyle=solid,fillcolor=curcolor]
{
\newpath
\moveto(382.44400239,270.47888719)
\lineto(381.38171732,270.47888719)
\lineto(381.38171732,276.68236288)
\curveto(381.38171732,277.15054973)(381.36853293,277.60248008)(381.34216416,278.03815395)
\curveto(381.31956235,278.47382782)(381.26870827,278.82171666)(381.18960194,279.08182046)
\curveto(381.10296167,279.36143205)(380.97865171,279.57276639)(380.81667207,279.71582348)
\curveto(380.65469243,279.85888057)(380.4211404,279.93040912)(380.11601596,279.93040912)
\curveto(379.81842546,279.93040912)(379.52083496,279.80035721)(379.22324447,279.54025341)
\curveto(378.92565397,279.2866522)(378.62806347,278.96152245)(378.33047297,278.56486415)
\curveto(378.34177388,278.41530447)(378.3511913,278.2397344)(378.35872523,278.03815395)
\curveto(378.36625917,277.8430761)(378.37002614,277.64799825)(378.37002614,277.4529204)
\lineto(378.37002614,270.47888719)
\lineto(377.30774107,270.47888719)
\lineto(377.30774107,276.68236288)
\curveto(377.30774107,277.16355492)(377.29455668,277.61873657)(377.2681879,278.04790784)
\curveto(377.24558609,278.48358171)(377.19473202,278.83147055)(377.11562568,279.09157435)
\curveto(377.02898541,279.37118594)(376.90467545,279.57926898)(376.74269582,279.71582348)
\curveto(376.58071618,279.85888057)(376.34716414,279.93040912)(376.04203971,279.93040912)
\curveto(375.75198314,279.93040912)(375.4600431,279.80685981)(375.16621957,279.5597612)
\curveto(374.87616301,279.31266258)(374.58610644,278.99728672)(374.29604988,278.61363362)
\lineto(374.29604988,270.47888719)
\lineto(373.23376481,270.47888719)
\lineto(373.23376481,281.37398522)
\lineto(374.29604988,281.37398522)
\lineto(374.29604988,280.16450254)
\curveto(374.62754309,280.63919198)(374.95715282,281.0098399)(375.28487907,281.27644629)
\curveto(375.61637228,281.54305269)(375.96858382,281.67635589)(376.34151369,281.67635589)
\curveto(376.77094808,281.67635589)(377.13446052,281.52029361)(377.43205102,281.20816905)
\curveto(377.73340849,280.89604448)(377.9575431,280.46362191)(378.10445487,279.91090133)
\curveto(378.53388926,280.53515046)(378.92565397,280.98382952)(379.27974899,281.25693851)
\curveto(379.63384402,281.5365501)(380.01242433,281.67635589)(380.41548995,281.67635589)
\curveto(381.10861212,281.67635589)(381.61903633,281.31221057)(381.94676257,280.58391992)
\curveto(382.27825579,279.86213187)(382.44400239,278.85097834)(382.44400239,277.55045932)
\closepath
}
}
{
\newrgbcolor{curcolor}{0 0 0}
\pscustom[linestyle=none,fillstyle=solid,fillcolor=curcolor]
{
\newpath
\moveto(389.86869777,275.92155926)
\curveto(389.86869777,274.14635081)(389.60500999,272.74504157)(389.07763442,271.71763155)
\curveto(388.55025885,270.69022153)(387.84395229,270.17651652)(386.95871473,270.17651652)
\curveto(386.06594323,270.17651652)(385.3558697,270.69022153)(384.82849413,271.71763155)
\curveto(384.30488554,272.74504157)(384.04308124,274.14635081)(384.04308124,275.92155926)
\curveto(384.04308124,277.69676771)(384.30488554,279.09807695)(384.82849413,280.12548697)
\curveto(385.3558697,281.15939958)(386.06594323,281.67635589)(386.95871473,281.67635589)
\curveto(387.84395229,281.67635589)(388.55025885,281.15939958)(389.07763442,280.12548697)
\curveto(389.60500999,279.09807695)(389.86869777,277.69676771)(389.86869777,275.92155926)
\closepath
\moveto(388.77250998,275.92155926)
\curveto(388.77250998,277.33262239)(388.61241383,278.37954019)(388.29222152,279.06231267)
\curveto(387.97202921,279.75158775)(387.52752695,280.09622529)(386.95871473,280.09622529)
\curveto(386.38236857,280.09622529)(385.93409934,279.75158775)(385.61390703,279.06231267)
\curveto(385.29748169,278.37954019)(385.13926902,277.33262239)(385.13926902,275.92155926)
\curveto(385.13926902,274.5560143)(385.29936518,273.51885038)(385.61955749,272.81006752)
\curveto(385.93974979,272.10778726)(386.38613554,271.75664712)(386.95871473,271.75664712)
\curveto(387.52375998,271.75664712)(387.96637876,272.10453596)(388.28657107,272.80031363)
\curveto(388.61053034,273.5025939)(388.77250998,274.54300911)(388.77250998,275.92155926)
\closepath
}
}
{
\newrgbcolor{curcolor}{0 0 0}
\pscustom[linestyle=none,fillstyle=solid,fillcolor=curcolor]
{
\newpath
\moveto(395.45134375,279.37443724)
\lineto(395.39483922,279.37443724)
\curveto(395.23662655,279.43946319)(395.08218085,279.48498135)(394.93150212,279.51099173)
\curveto(394.78459035,279.54350471)(394.60942632,279.5597612)(394.40601003,279.5597612)
\curveto(394.07828379,279.5597612)(393.76185845,279.43296059)(393.45673401,279.17935939)
\curveto(393.15160958,278.93226077)(392.85778605,278.61038232)(392.57526342,278.21372402)
\lineto(392.57526342,270.47888719)
\lineto(391.51297835,270.47888719)
\lineto(391.51297835,281.37398522)
\lineto(392.57526342,281.37398522)
\lineto(392.57526342,279.76459294)
\curveto(392.99716387,280.3498265)(393.36821026,280.76274128)(393.68840256,281.0033373)
\curveto(394.01236184,281.25043591)(394.34197157,281.37398522)(394.67723175,281.37398522)
\curveto(394.8618132,281.37398522)(394.99554058,281.36423133)(395.07841388,281.34472354)
\curveto(395.16128718,281.33171835)(395.28559714,281.30245667)(395.45134375,281.25693851)
\closepath
}
}
{
\newrgbcolor{curcolor}{0 0 0}
\pscustom[linestyle=none,fillstyle=solid,fillcolor=curcolor]
{
\newpath
\moveto(401.90981012,281.37398522)
\lineto(398.22571508,266.46028344)
\lineto(397.08997413,266.46028344)
\lineto(398.26526825,271.00559739)
\lineto(395.75081688,281.37398522)
\lineto(396.9035092,281.37398522)
\lineto(398.84161441,273.29776215)
\lineto(400.79667097,281.37398522)
\closepath
}
}
{
\newrgbcolor{curcolor}{0 0 0}
\pscustom[linestyle=none,fillstyle=solid,fillcolor=curcolor]
{
\newpath
\moveto(413.95092485,271.53230759)
\curveto(413.74374159,271.37624531)(413.55539318,271.22993692)(413.3858796,271.09338243)
\curveto(413.22013299,270.95682793)(413.00164883,270.81377084)(412.73042711,270.66421115)
\curveto(412.50064204,270.54066185)(412.25013865,270.43662032)(411.97891693,270.35208659)
\curveto(411.71146217,270.26105026)(411.41575516,270.21553209)(411.09179588,270.21553209)
\curveto(410.48154701,270.21553209)(409.92591918,270.36184048)(409.42491239,270.65445726)
\curveto(408.92767257,270.95357663)(408.49447121,271.41851218)(408.12530832,272.0492639)
\curveto(407.76367936,272.66701043)(407.48115673,273.45057314)(407.27774044,274.39995202)
\curveto(407.07432415,275.35583349)(406.97261601,276.46452595)(406.97261601,277.72602939)
\curveto(406.97261601,278.92250688)(407.07055718,279.99218377)(407.26643954,280.93506005)
\curveto(407.46232189,281.87793634)(407.74484451,282.67450423)(408.11400741,283.32476374)
\curveto(408.4718694,283.95551546)(408.90318728,284.43670749)(409.40796104,284.76833984)
\curveto(409.91650176,285.09997219)(410.47966353,285.26578836)(411.09744634,285.26578836)
\curveto(411.54948254,285.26578836)(411.99963525,285.17150074)(412.44790448,284.98292548)
\curveto(412.89994068,284.79435022)(413.40094747,284.46271787)(413.95092485,283.98802843)
\lineto(413.95092485,281.69586367)
\lineto(413.86616806,281.69586367)
\curveto(413.40283096,282.36563097)(412.94326082,282.8533256)(412.48745765,283.15894756)
\curveto(412.03165448,283.46456953)(411.54383208,283.61738052)(411.02399045,283.61738052)
\curveto(410.59832303,283.61738052)(410.21409226,283.49708251)(409.87129814,283.25648649)
\curveto(409.53227099,283.02239307)(409.22903004,282.65499645)(408.96157529,282.15429663)
\curveto(408.70165447,281.666602)(408.49823818,281.04885547)(408.35132642,280.30105703)
\curveto(408.20818162,279.5597612)(408.13660922,278.70141865)(408.13660922,277.72602939)
\curveto(408.13660922,276.70512196)(408.21571556,275.82727163)(408.37392823,275.09247839)
\curveto(408.53590787,274.35768515)(408.74309112,273.7594464)(408.995478,273.29776215)
\curveto(409.25916579,272.81657012)(409.56617371,272.45892739)(409.91650176,272.22483397)
\curveto(410.27059678,271.99724314)(410.64352665,271.88344773)(411.03529136,271.88344773)
\curveto(411.57396783,271.88344773)(412.07874159,272.04276131)(412.54961263,272.36138846)
\curveto(413.02048367,272.68001562)(413.46121897,273.15795636)(413.87181852,273.79521067)
\lineto(413.95092485,273.79521067)
\closepath
}
}
{
\newrgbcolor{curcolor}{0 0 0}
\pscustom[linestyle=none,fillstyle=solid,fillcolor=curcolor]
{
\newpath
\moveto(420.82752399,275.92155926)
\curveto(420.82752399,274.14635081)(420.56383621,272.74504157)(420.03646064,271.71763155)
\curveto(419.50908508,270.69022153)(418.80277851,270.17651652)(417.91754095,270.17651652)
\curveto(417.02476946,270.17651652)(416.31469593,270.69022153)(415.78732036,271.71763155)
\curveto(415.26371176,272.74504157)(415.00190746,274.14635081)(415.00190746,275.92155926)
\curveto(415.00190746,277.69676771)(415.26371176,279.09807695)(415.78732036,280.12548697)
\curveto(416.31469593,281.15939958)(417.02476946,281.67635589)(417.91754095,281.67635589)
\curveto(418.80277851,281.67635589)(419.50908508,281.15939958)(420.03646064,280.12548697)
\curveto(420.56383621,279.09807695)(420.82752399,277.69676771)(420.82752399,275.92155926)
\closepath
\moveto(419.73133621,275.92155926)
\curveto(419.73133621,277.33262239)(419.57124005,278.37954019)(419.25104774,279.06231267)
\curveto(418.93085544,279.75158775)(418.48635317,280.09622529)(417.91754095,280.09622529)
\curveto(417.3411948,280.09622529)(416.89292557,279.75158775)(416.57273326,279.06231267)
\curveto(416.25630792,278.37954019)(416.09809525,277.33262239)(416.09809525,275.92155926)
\curveto(416.09809525,274.5560143)(416.2581914,273.51885038)(416.57838371,272.81006752)
\curveto(416.89857602,272.10778726)(417.34496177,271.75664712)(417.91754095,271.75664712)
\curveto(418.4825862,271.75664712)(418.92520498,272.10453596)(419.24539729,272.80031363)
\curveto(419.56935657,273.5025939)(419.73133621,274.54300911)(419.73133621,275.92155926)
\closepath
}
}
{
\newrgbcolor{curcolor}{0 0 0}
\pscustom[linestyle=none,fillstyle=solid,fillcolor=curcolor]
{
\newpath
\moveto(427.7493327,270.47888719)
\lineto(426.68704763,270.47888719)
\lineto(426.68704763,276.68236288)
\curveto(426.68704763,277.1830627)(426.67009627,277.65124955)(426.63619356,278.08692342)
\curveto(426.60229084,278.52909988)(426.54013586,278.87373742)(426.44972862,279.12083603)
\curveto(426.35555442,279.39394502)(426.21994356,279.59552547)(426.04289604,279.72557737)
\curveto(425.86584853,279.86213187)(425.63606346,279.93040912)(425.35354084,279.93040912)
\curveto(425.06348428,279.93040912)(424.76024332,279.80685981)(424.44381798,279.5597612)
\curveto(424.12739264,279.31266258)(423.82415169,278.99728672)(423.53409513,278.61363362)
\lineto(423.53409513,270.47888719)
\lineto(422.47181006,270.47888719)
\lineto(422.47181006,281.37398522)
\lineto(423.53409513,281.37398522)
\lineto(423.53409513,280.16450254)
\curveto(423.86558834,280.63919198)(424.20838246,281.0098399)(424.56247749,281.27644629)
\curveto(424.91657251,281.54305269)(425.28008496,281.67635589)(425.65301482,281.67635589)
\curveto(426.33483609,281.67635589)(426.85467772,281.32196446)(427.21253971,280.6131816)
\curveto(427.5704017,279.90439874)(427.7493327,278.88349131)(427.7493327,277.55045932)
\closepath
}
}
{
\newrgbcolor{curcolor}{0 0 0}
\pscustom[linestyle=none,fillstyle=solid,fillcolor=curcolor]
{
\newpath
\moveto(433.07205418,270.57642612)
\curveto(432.87240486,270.48538979)(432.6539207,270.41060994)(432.41660169,270.35208659)
\curveto(432.18304965,270.29356323)(431.97398291,270.26430156)(431.78940146,270.26430156)
\curveto(431.14524988,270.26430156)(430.65554399,270.56342093)(430.32028381,271.16165967)
\curveto(429.98502363,271.75989842)(429.81739354,272.71903119)(429.81739354,274.03905799)
\lineto(429.81739354,279.83287019)
\lineto(429.09978607,279.83287019)
\lineto(429.09978607,281.37398522)
\lineto(429.81739354,281.37398522)
\lineto(429.81739354,284.50498474)
\lineto(430.87967861,284.50498474)
\lineto(430.87967861,281.37398522)
\lineto(433.07205418,281.37398522)
\lineto(433.07205418,279.83287019)
\lineto(430.87967861,279.83287019)
\lineto(430.87967861,274.86813886)
\curveto(430.87967861,274.29591049)(430.88721255,273.84723143)(430.90228042,273.52210168)
\curveto(430.91734829,273.20347452)(430.97008585,272.90435515)(431.06049309,272.62474356)
\curveto(431.14336639,272.36463976)(431.25637544,272.17281321)(431.39952024,272.0492639)
\curveto(431.546432,271.93221719)(431.76868314,271.87369383)(432.06627364,271.87369383)
\curveto(432.23955418,271.87369383)(432.42036866,271.9159607)(432.60871708,272.00049444)
\curveto(432.79706549,272.09153077)(432.93267635,272.16631061)(433.01554966,272.22483397)
\lineto(433.07205418,272.22483397)
\closepath
}
}
{
\newrgbcolor{curcolor}{0 0 0}
\pscustom[linestyle=none,fillstyle=solid,fillcolor=curcolor]
{
\newpath
\moveto(438.29307258,279.37443724)
\lineto(438.23656806,279.37443724)
\curveto(438.07835539,279.43946319)(437.92390969,279.48498135)(437.77323095,279.51099173)
\curveto(437.62631919,279.54350471)(437.45115516,279.5597612)(437.24773887,279.5597612)
\curveto(436.92001263,279.5597612)(436.60358729,279.43296059)(436.29846285,279.17935939)
\curveto(435.99333841,278.93226077)(435.69951488,278.61038232)(435.41699226,278.21372402)
\lineto(435.41699226,270.47888719)
\lineto(434.35470719,270.47888719)
\lineto(434.35470719,281.37398522)
\lineto(435.41699226,281.37398522)
\lineto(435.41699226,279.76459294)
\curveto(435.83889271,280.3498265)(436.20993909,280.76274128)(436.5301314,281.0033373)
\curveto(436.85409068,281.25043591)(437.18370041,281.37398522)(437.51896059,281.37398522)
\curveto(437.70354204,281.37398522)(437.83726942,281.36423133)(437.92014272,281.34472354)
\curveto(438.00301602,281.33171835)(438.12732598,281.30245667)(438.29307258,281.25693851)
\closepath
}
}
{
\newrgbcolor{curcolor}{0 0 0}
\pscustom[linestyle=none,fillstyle=solid,fillcolor=curcolor]
{
\newpath
\moveto(444.67243262,275.92155926)
\curveto(444.67243262,274.14635081)(444.40874484,272.74504157)(443.88136927,271.71763155)
\curveto(443.3539937,270.69022153)(442.64768714,270.17651652)(441.76244958,270.17651652)
\curveto(440.86967808,270.17651652)(440.15960455,270.69022153)(439.63222898,271.71763155)
\curveto(439.10862039,272.74504157)(438.84681609,274.14635081)(438.84681609,275.92155926)
\curveto(438.84681609,277.69676771)(439.10862039,279.09807695)(439.63222898,280.12548697)
\curveto(440.15960455,281.15939958)(440.86967808,281.67635589)(441.76244958,281.67635589)
\curveto(442.64768714,281.67635589)(443.3539937,281.15939958)(443.88136927,280.12548697)
\curveto(444.40874484,279.09807695)(444.67243262,277.69676771)(444.67243262,275.92155926)
\closepath
\moveto(443.57624483,275.92155926)
\curveto(443.57624483,277.33262239)(443.41614868,278.37954019)(443.09595637,279.06231267)
\curveto(442.77576406,279.75158775)(442.3312618,280.09622529)(441.76244958,280.09622529)
\curveto(441.18610342,280.09622529)(440.73783419,279.75158775)(440.41764188,279.06231267)
\curveto(440.10121654,278.37954019)(439.94300387,277.33262239)(439.94300387,275.92155926)
\curveto(439.94300387,274.5560143)(440.10310003,273.51885038)(440.42329234,272.81006752)
\curveto(440.74348464,272.10778726)(441.18987039,271.75664712)(441.76244958,271.75664712)
\curveto(442.32749483,271.75664712)(442.77011361,272.10453596)(443.09030592,272.80031363)
\curveto(443.4142652,273.5025939)(443.57624483,274.54300911)(443.57624483,275.92155926)
\closepath
}
}
{
\newrgbcolor{curcolor}{0 0 0}
\pscustom[linestyle=none,fillstyle=solid,fillcolor=curcolor]
{
\newpath
\moveto(447.39030466,270.47888719)
\lineto(446.32801959,270.47888719)
\lineto(446.32801959,285.65594407)
\lineto(447.39030466,285.65594407)
\closepath
}
}
{
\newrgbcolor{curcolor}{0 0 0}
\pscustom[linestyle=none,fillstyle=solid,fillcolor=curcolor]
{
\newpath
\moveto(450.56585857,270.47888719)
\lineto(449.50357349,270.47888719)
\lineto(449.50357349,285.65594407)
\lineto(450.56585857,285.65594407)
\closepath
}
}
{
\newrgbcolor{curcolor}{0 0 0}
\pscustom[linestyle=none,fillstyle=solid,fillcolor=curcolor]
{
\newpath
\moveto(457.9509941,275.7362353)
\lineto(453.30067169,275.7362353)
\curveto(453.30067169,275.06646801)(453.3590597,274.48123445)(453.47583571,273.98053463)
\curveto(453.59261173,273.48633741)(453.75270789,273.07992522)(453.95612418,272.76129806)
\curveto(454.15200653,272.4491735)(454.38367508,272.21508007)(454.65112984,272.05901779)
\curveto(454.92235156,271.90295551)(455.21994206,271.82492437)(455.54390133,271.82492437)
\curveto(455.97333572,271.82492437)(456.4046536,271.97123276)(456.83785496,272.26384954)
\curveto(457.27482328,272.56296891)(457.58559817,272.85558569)(457.77017962,273.14169987)
\lineto(457.82668414,273.14169987)
\lineto(457.82668414,271.14215189)
\curveto(457.46882215,270.88204809)(457.10342622,270.66421115)(456.73049636,270.48864109)
\curveto(456.35756649,270.31307102)(455.96580179,270.22528599)(455.55520224,270.22528599)
\curveto(454.50798504,270.22528599)(453.69055291,270.71298061)(453.10290585,271.68836987)
\curveto(452.51525879,272.67026173)(452.22143526,274.06181707)(452.22143526,275.8630359)
\curveto(452.22143526,277.64474695)(452.5020744,279.05906138)(453.06335268,280.10597918)
\curveto(453.62839793,281.15289699)(454.37049069,281.67635589)(455.28963097,281.67635589)
\curveto(456.14096581,281.67635589)(456.7964183,281.24718462)(457.25598844,280.38884207)
\curveto(457.71932555,279.53049952)(457.9509941,278.31126294)(457.9509941,276.73113234)
\closepath
\moveto(456.91696129,277.14079583)
\curveto(456.91319432,278.1031799)(456.77193301,278.84772704)(456.49317735,279.37443724)
\curveto(456.21818866,279.90114744)(455.79817169,280.16450254)(455.23312644,280.16450254)
\curveto(454.66431423,280.16450254)(454.21039454,279.87513706)(453.87136739,279.2964061)
\curveto(453.53610721,278.71767514)(453.34587531,277.99913838)(453.30067169,277.14079583)
\closepath
}
}
{
\newrgbcolor{curcolor}{0 0 0}
\pscustom[linestyle=none,fillstyle=solid,fillcolor=curcolor]
{
\newpath
\moveto(463.49974258,279.37443724)
\lineto(463.44323806,279.37443724)
\curveto(463.28502539,279.43946319)(463.13057968,279.48498135)(462.97990095,279.51099173)
\curveto(462.83298918,279.54350471)(462.65782516,279.5597612)(462.45440887,279.5597612)
\curveto(462.12668262,279.5597612)(461.81025728,279.43296059)(461.50513285,279.17935939)
\curveto(461.20000841,278.93226077)(460.90618488,278.61038232)(460.62366226,278.21372402)
\lineto(460.62366226,270.47888719)
\lineto(459.56137718,270.47888719)
\lineto(459.56137718,281.37398522)
\lineto(460.62366226,281.37398522)
\lineto(460.62366226,279.76459294)
\curveto(461.04556271,280.3498265)(461.41660909,280.76274128)(461.7368014,281.0033373)
\curveto(462.06076068,281.25043591)(462.39037041,281.37398522)(462.72563059,281.37398522)
\curveto(462.91021204,281.37398522)(463.04393941,281.36423133)(463.12681272,281.34472354)
\curveto(463.20968602,281.33171835)(463.33399597,281.30245667)(463.49974258,281.25693851)
\closepath
}
}
{
\newrgbcolor{curcolor}{0.80000001 0.80000001 0.80000001}
\pscustom[linestyle=none,fillstyle=solid,fillcolor=curcolor]
{
\newpath
\moveto(0.42503943,183.1639233)
\lineto(250.00445434,183.1639233)
\lineto(250.00445434,0.42505477)
\lineto(0.42503943,0.42505477)
\closepath
}
}
{
\newrgbcolor{curcolor}{0 0 0}
\pscustom[linewidth=0.85007621,linecolor=curcolor]
{
\newpath
\moveto(0.42503943,183.1639233)
\lineto(250.00445434,183.1639233)
\lineto(250.00445434,0.42505477)
\lineto(0.42503943,0.42505477)
\closepath
}
}
{
\newrgbcolor{curcolor}{0 0 0}
\pscustom[linestyle=none,fillstyle=solid,fillcolor=curcolor]
{
\newpath
\moveto(76.46007953,81.53327777)
\lineto(72.85401896,81.53327777)
\lineto(72.85401896,107.59987972)
\lineto(65.31407413,89.86321292)
\lineto(63.16500773,89.86321292)
\lineto(55.67970019,107.59987972)
\lineto(55.67970019,81.53327777)
\lineto(52.31040117,81.53327777)
\lineto(52.31040117,111.78516421)
\lineto(57.22775649,111.78516421)
\lineto(64.45809006,94.94244167)
\lineto(71.45166207,111.78516421)
\lineto(76.46007953,111.78516421)
\closepath
}
}
{
\newrgbcolor{curcolor}{0 0 0}
\pscustom[linestyle=none,fillstyle=solid,fillcolor=curcolor]
{
\newpath
\moveto(100.50048297,92.48409496)
\lineto(85.51165545,92.48409496)
\curveto(85.51165545,91.08900012)(85.69985053,89.86998522)(86.07624069,88.82705025)
\curveto(86.45263085,87.79765989)(86.96864962,86.95112177)(87.624297,86.28743588)
\curveto(88.25566114,85.6372946)(89.00237065,85.14968864)(89.86442553,84.824618)
\curveto(90.73862203,84.49954736)(91.69780986,84.33701204)(92.74198902,84.33701204)
\curveto(94.12613348,84.33701204)(95.51634875,84.64176576)(96.91263483,85.25127321)
\curveto(98.32106252,85.87432527)(99.32274601,86.48383272)(99.9176853,87.07979556)
\lineto(100.09980957,87.07979556)
\lineto(100.09980957,82.91482799)
\curveto(98.94635585,82.37304359)(97.7686189,81.91929915)(96.56659871,81.55359468)
\curveto(95.36457852,81.18789021)(94.10185024,81.00503798)(92.77841387,81.00503798)
\curveto(89.40304404,81.00503798)(86.76831292,82.02088373)(84.8742205,84.05257523)
\curveto(82.98012808,86.09781134)(82.03308187,88.99635788)(82.03308187,92.74821485)
\curveto(82.03308187,96.45943799)(82.93763242,99.40539067)(84.74673351,101.58607288)
\curveto(86.56797622,103.76675509)(88.95987498,104.85709619)(91.9224298,104.85709619)
\curveto(94.66643548,104.85709619)(96.77907703,103.96315193)(98.26035443,102.17526341)
\curveto(99.75377346,100.38737489)(100.50048297,97.84776052)(100.50048297,94.55642029)
\closepath
\moveto(97.1676088,95.40973072)
\curveto(97.15546719,97.414333)(96.70015651,98.96519084)(95.80167677,100.06230425)
\curveto(94.91533865,101.15941766)(93.56154824,101.70797437)(91.74030552,101.70797437)
\curveto(89.90692119,101.70797437)(88.44385622,101.10523922)(87.35111059,99.89976893)
\curveto(86.27050658,98.69429864)(85.65735487,97.19761924)(85.51165545,95.40973072)
\closepath
}
}
{
\newrgbcolor{curcolor}{0 0 0}
\pscustom[linestyle=none,fillstyle=solid,fillcolor=curcolor]
{
\newpath
\moveto(135.37728096,81.53327777)
\lineto(131.95334466,81.53327777)
\lineto(131.95334466,94.45483571)
\curveto(131.95334466,95.43004763)(131.910849,96.37139803)(131.82585767,97.2788869)
\curveto(131.75300797,98.18637577)(131.58909612,98.9110124)(131.33412214,99.4527968)
\curveto(131.05486493,100.03521503)(130.65419153,100.47541486)(130.13210195,100.77339628)
\curveto(129.61001237,101.0713777)(128.85723205,101.22036841)(127.87376099,101.22036841)
\curveto(126.91457316,101.22036841)(125.95538533,100.94947621)(124.9961975,100.40769181)
\curveto(124.03700968,99.87945202)(123.07782185,99.20222152)(122.11863402,98.37600031)
\curveto(122.15505887,98.06447428)(122.18541292,97.69876981)(122.20969616,97.2788869)
\curveto(122.23397939,96.8725486)(122.24612101,96.4662103)(122.24612101,96.059872)
\lineto(122.24612101,81.53327777)
\lineto(118.82218471,81.53327777)
\lineto(118.82218471,94.45483571)
\curveto(118.82218471,95.45713685)(118.77968905,96.40525955)(118.69469772,97.29920381)
\curveto(118.62184801,98.20669268)(118.45793617,98.93132932)(118.20296219,99.47311372)
\curveto(117.92370497,100.05553195)(117.52303158,100.48895947)(117.000942,100.77339628)
\curveto(116.47885242,101.0713777)(115.7260721,101.22036841)(114.74260104,101.22036841)
\curveto(113.80769645,101.22036841)(112.86672105,100.96302082)(111.91967484,100.44832564)
\curveto(110.98477024,99.93363046)(110.04986565,99.27671687)(109.11496106,98.47758488)
\lineto(109.11496106,81.53327777)
\lineto(105.69102476,81.53327777)
\lineto(105.69102476,104.22727183)
\lineto(109.11496106,104.22727183)
\lineto(109.11496106,101.70797437)
\curveto(110.18342345,102.6967309)(111.24581503,103.46877367)(112.30213581,104.02410268)
\curveto(113.3705982,104.57943169)(114.50583949,104.85709619)(115.70785968,104.85709619)
\curveto(117.09200414,104.85709619)(118.26367028,104.53202555)(119.22285811,103.88188427)
\curveto(120.19418756,103.23174299)(120.91661383,102.33102643)(121.39013694,101.17973458)
\curveto(122.7742814,102.48001714)(124.03700968,103.41459523)(125.17832178,103.98346885)
\curveto(126.31963388,104.56588708)(127.53986649,104.85709619)(128.83901963,104.85709619)
\curveto(131.07307735,104.85709619)(132.7182666,104.09859803)(133.77458737,102.58160171)
\curveto(134.84304977,101.07815)(135.37728096,98.97196315)(135.37728096,96.26304115)
\closepath
}
}
{
\newrgbcolor{curcolor}{0 0 0}
\pscustom[linestyle=none,fillstyle=solid,fillcolor=curcolor]
{
\newpath
\moveto(159.30840969,92.87011634)
\curveto(159.30840969,89.17243781)(158.45849642,86.25357435)(156.75866989,84.11352597)
\curveto(155.05884336,81.97347759)(152.78228997,80.9034534)(149.92900972,80.9034534)
\curveto(147.05144624,80.9034534)(144.76275123,81.97347759)(143.0629247,84.11352597)
\curveto(141.37523978,86.25357435)(140.53139733,89.17243781)(140.53139733,92.87011634)
\curveto(140.53139733,96.56779487)(141.37523978,99.48665833)(143.0629247,101.62670671)
\curveto(144.76275123,103.7802997)(147.05144624,104.85709619)(149.92900972,104.85709619)
\curveto(152.78228997,104.85709619)(155.05884336,103.7802997)(156.75866989,101.62670671)
\curveto(158.45849642,99.48665833)(159.30840969,96.56779487)(159.30840969,92.87011634)
\closepath
\moveto(155.77519882,92.87011634)
\curveto(155.77519882,95.80929671)(155.25918006,97.98997892)(154.22714252,99.41216297)
\curveto(153.19510498,100.84789163)(151.76239405,101.56575596)(149.92900972,101.56575596)
\curveto(148.07134215,101.56575596)(146.6264896,100.84789163)(145.59445207,99.41216297)
\curveto(144.57455615,97.98997892)(144.06460819,95.80929671)(144.06460819,92.87011634)
\curveto(144.06460819,90.02574824)(144.58062696,87.86538294)(145.61266449,86.38902045)
\curveto(146.64470203,84.92620257)(148.08348377,84.19479363)(149.92900972,84.19479363)
\curveto(151.75025243,84.19479363)(153.17689256,84.91943027)(154.20893009,86.36870354)
\curveto(155.25310925,87.83152142)(155.77519882,89.99865902)(155.77519882,92.87011634)
\closepath
}
}
{
\newrgbcolor{curcolor}{0 0 0}
\pscustom[linestyle=none,fillstyle=solid,fillcolor=curcolor]
{
\newpath
\moveto(177.30228509,100.06230425)
\lineto(177.12016082,100.06230425)
\curveto(176.61021286,100.19775035)(176.11240651,100.29256262)(175.62674179,100.34674106)
\curveto(175.15321869,100.41446411)(174.58863345,100.44832564)(173.93298607,100.44832564)
\curveto(172.8766653,100.44832564)(171.85676938,100.18420574)(170.87329831,99.65596595)
\curveto(169.88982725,99.14127077)(168.94278104,98.47081258)(168.03215968,97.64459137)
\lineto(168.03215968,81.53327777)
\lineto(164.60822339,81.53327777)
\lineto(164.60822339,104.22727183)
\lineto(168.03215968,104.22727183)
\lineto(168.03215968,100.87498085)
\curveto(169.39202091,102.09399575)(170.58797029,102.95407849)(171.62000783,103.45522906)
\curveto(172.66418698,103.96992424)(173.72657856,104.22727183)(174.80718257,104.22727183)
\curveto(175.40212186,104.22727183)(175.8331493,104.20695491)(176.1002649,104.16632108)
\curveto(176.36738049,104.13923186)(176.76805389,104.07828112)(177.30228509,103.98346885)
\closepath
}
}
{
\newrgbcolor{curcolor}{0 0 0}
\pscustom[linestyle=none,fillstyle=solid,fillcolor=curcolor]
{
\newpath
\moveto(198.11909755,104.22727183)
\lineto(186.24459507,73.16270879)
\lineto(182.58389722,73.16270879)
\lineto(186.37208206,82.63039118)
\lineto(178.267552,104.22727183)
\lineto(181.98288713,104.22727183)
\lineto(188.22974963,87.4048662)
\lineto(194.53124941,104.22727183)
\closepath
}
}
{
\newrgbcolor{curcolor}{0.80000001 0.80000001 0.80000001}
\pscustom[linestyle=none,fillstyle=solid,fillcolor=curcolor]
{
\newpath
\moveto(285.29630331,183.1639233)
\lineto(534.87571821,183.1639233)
\lineto(534.87571821,0.42505477)
\lineto(285.29630331,0.42505477)
\closepath
}
}
{
\newrgbcolor{curcolor}{0 0 0}
\pscustom[linewidth=0.85007621,linecolor=curcolor]
{
\newpath
\moveto(285.29630331,183.1639233)
\lineto(534.87571821,183.1639233)
\lineto(534.87571821,0.42505477)
\lineto(285.29630331,0.42505477)
\closepath
}
}
{
\newrgbcolor{curcolor}{0 0 0}
\pscustom[linestyle=none,fillstyle=solid,fillcolor=curcolor]
{
\newpath
\moveto(361.33132006,80.85380488)
\lineto(357.72525949,80.85380488)
\lineto(357.72525949,106.92040683)
\lineto(350.18531466,89.18374003)
\lineto(348.03624826,89.18374003)
\lineto(340.55094072,106.92040683)
\lineto(340.55094072,80.85380488)
\lineto(337.1816417,80.85380488)
\lineto(337.1816417,111.10569132)
\lineto(342.09899702,111.10569132)
\lineto(349.32933059,94.26296878)
\lineto(356.3229026,111.10569132)
\lineto(361.33132006,111.10569132)
\closepath
}
}
{
\newrgbcolor{curcolor}{0 0 0}
\pscustom[linestyle=none,fillstyle=solid,fillcolor=curcolor]
{
\newpath
\moveto(385.3717235,91.80462207)
\lineto(370.38289598,91.80462207)
\curveto(370.38289598,90.40952724)(370.57109106,89.19051234)(370.94748122,88.14757737)
\curveto(371.32387138,87.11818701)(371.83989015,86.27164888)(372.49553753,85.60796299)
\curveto(373.12690167,84.95782171)(373.87361118,84.47021575)(374.73566606,84.14514511)
\curveto(375.60986256,83.82007447)(376.56905039,83.65753915)(377.61322955,83.65753915)
\curveto(378.99737401,83.65753915)(380.38758928,83.96229288)(381.78387536,84.57180033)
\curveto(383.19230305,85.19485239)(384.19398655,85.80435984)(384.78892583,86.40032268)
\lineto(384.9710501,86.40032268)
\lineto(384.9710501,82.2353551)
\curveto(383.81759638,81.6935707)(382.63985943,81.23982627)(381.43783924,80.8741218)
\curveto(380.23581905,80.50841733)(378.97309077,80.32556509)(377.6496544,80.32556509)
\curveto(374.27428458,80.32556509)(371.63955345,81.34141084)(369.74546103,83.37310234)
\curveto(367.85136861,85.41833845)(366.9043224,88.31688499)(366.9043224,92.06874196)
\curveto(366.9043224,95.7799651)(367.80887295,98.72591778)(369.61797404,100.90659999)
\curveto(371.43921676,103.0872822)(373.83111552,104.17762331)(376.79367033,104.17762331)
\curveto(379.53767601,104.17762331)(381.65031756,103.28367905)(383.13159496,101.49579053)
\curveto(384.62501399,99.70790201)(385.3717235,97.16828763)(385.3717235,93.8769474)
\closepath
\moveto(382.03884934,94.73025783)
\curveto(382.02670772,96.73486011)(381.57139704,98.28571796)(380.6729173,99.38283137)
\curveto(379.78657918,100.47994478)(378.43278877,101.02850148)(376.61154606,101.02850148)
\curveto(374.77816173,101.02850148)(373.31509675,100.42576634)(372.22235112,99.22029605)
\curveto(371.14174711,98.01482576)(370.5285954,96.51814635)(370.38289598,94.73025783)
\closepath
}
}
{
\newrgbcolor{curcolor}{0 0 0}
\pscustom[linestyle=none,fillstyle=solid,fillcolor=curcolor]
{
\newpath
\moveto(420.24852149,80.85380488)
\lineto(416.82458519,80.85380488)
\lineto(416.82458519,93.77536282)
\curveto(416.82458519,94.75057474)(416.78208953,95.69192514)(416.69709821,96.59941401)
\curveto(416.6242485,97.50690288)(416.46033665,98.23153952)(416.20536267,98.77332392)
\curveto(415.92610546,99.35574215)(415.52543206,99.79594197)(415.00334248,100.09392339)
\curveto(414.48125291,100.39190481)(413.72847259,100.54089552)(412.74500152,100.54089552)
\curveto(411.78581369,100.54089552)(410.82662586,100.27000332)(409.86743804,99.72821892)
\curveto(408.90825021,99.19997913)(407.94906238,98.52274863)(406.98987455,97.69652742)
\curveto(407.02629941,97.38500139)(407.05665345,97.01929692)(407.08093669,96.59941401)
\curveto(407.10521992,96.19307571)(407.11736154,95.78673741)(407.11736154,95.38039911)
\lineto(407.11736154,80.85380488)
\lineto(403.69342524,80.85380488)
\lineto(403.69342524,93.77536282)
\curveto(403.69342524,94.77766396)(403.65092958,95.72578666)(403.56593825,96.61973092)
\curveto(403.49308855,97.5272198)(403.3291767,98.25185643)(403.07420272,98.79364083)
\curveto(402.79494551,99.37605906)(402.39427211,99.80948658)(401.87218253,100.09392339)
\curveto(401.35009296,100.39190481)(400.59731263,100.54089552)(399.61384157,100.54089552)
\curveto(398.67893698,100.54089552)(397.73796158,100.28354793)(396.79091537,99.76885275)
\curveto(395.85601078,99.25415757)(394.92110618,98.59724399)(393.98620159,97.798112)
\lineto(393.98620159,80.85380488)
\lineto(390.56226529,80.85380488)
\lineto(390.56226529,103.54779894)
\lineto(393.98620159,103.54779894)
\lineto(393.98620159,101.02850148)
\curveto(395.05466398,102.01725801)(396.11705556,102.78930078)(397.17337634,103.34462979)
\curveto(398.24183873,103.8999588)(399.37708002,104.17762331)(400.57910021,104.17762331)
\curveto(401.96324467,104.17762331)(403.13491081,103.85255267)(404.09409864,103.20241139)
\curveto(405.06542809,102.55227011)(405.78785436,101.65155354)(406.26137747,100.50026169)
\curveto(407.64552193,101.80054425)(408.90825021,102.73512234)(410.04956231,103.30399596)
\curveto(411.19087441,103.88641419)(412.41110702,104.17762331)(413.71026016,104.17762331)
\curveto(415.94431788,104.17762331)(417.58950713,103.41912515)(418.64582791,101.90212883)
\curveto(419.7142903,100.39867712)(420.24852149,98.29249026)(420.24852149,95.58356826)
\closepath
}
}
{
\newrgbcolor{curcolor}{0 0 0}
\pscustom[linestyle=none,fillstyle=solid,fillcolor=curcolor]
{
\newpath
\moveto(444.17965022,92.19064345)
\curveto(444.17965022,88.49296492)(443.32973695,85.57410147)(441.62991042,83.43405309)
\curveto(439.93008389,81.29400471)(437.6535305,80.22398052)(434.80025025,80.22398052)
\curveto(431.92268677,80.22398052)(429.63399176,81.29400471)(427.93416523,83.43405309)
\curveto(426.24648032,85.57410147)(425.40263786,88.49296492)(425.40263786,92.19064345)
\curveto(425.40263786,95.88832198)(426.24648032,98.80718544)(427.93416523,100.94723382)
\curveto(429.63399176,103.10082681)(431.92268677,104.17762331)(434.80025025,104.17762331)
\curveto(437.6535305,104.17762331)(439.93008389,103.10082681)(441.62991042,100.94723382)
\curveto(443.32973695,98.80718544)(444.17965022,95.88832198)(444.17965022,92.19064345)
\closepath
\moveto(440.64643936,92.19064345)
\curveto(440.64643936,95.12982382)(440.13042059,97.31050604)(439.09838305,98.73269009)
\curveto(438.06634551,100.16841875)(436.63363458,100.88628308)(434.80025025,100.88628308)
\curveto(432.94258269,100.88628308)(431.49773013,100.16841875)(430.4656926,98.73269009)
\curveto(429.44579668,97.31050604)(428.93584872,95.12982382)(428.93584872,92.19064345)
\curveto(428.93584872,89.34627535)(429.45186749,87.18591006)(430.48390503,85.70954757)
\curveto(431.51594256,84.24672969)(432.9547243,83.51532075)(434.80025025,83.51532075)
\curveto(436.62149296,83.51532075)(438.04813309,84.23995738)(439.08017062,85.68923065)
\curveto(440.12434978,87.15204853)(440.64643936,89.31918613)(440.64643936,92.19064345)
\closepath
}
}
{
\newrgbcolor{curcolor}{0 0 0}
\pscustom[linestyle=none,fillstyle=solid,fillcolor=curcolor]
{
\newpath
\moveto(462.17352562,99.38283137)
\lineto(461.99140135,99.38283137)
\curveto(461.48145339,99.51827747)(460.98364705,99.61308974)(460.49798232,99.66726818)
\curveto(460.02445922,99.73499123)(459.45987398,99.76885275)(458.8042266,99.76885275)
\curveto(457.74790583,99.76885275)(456.72800991,99.50473286)(455.74453885,98.97649307)
\curveto(454.76106778,98.46179789)(453.81402157,97.79133969)(452.90340022,96.96511848)
\lineto(452.90340022,80.85380488)
\lineto(449.47946392,80.85380488)
\lineto(449.47946392,103.54779894)
\lineto(452.90340022,103.54779894)
\lineto(452.90340022,100.19550797)
\curveto(454.26326144,101.41452287)(455.45921082,102.2746056)(456.49124836,102.77575617)
\curveto(457.53542751,103.29045135)(458.59781909,103.54779894)(459.6784231,103.54779894)
\curveto(460.27336239,103.54779894)(460.70438983,103.52748203)(460.97150543,103.4868482)
\curveto(461.23862103,103.45975898)(461.63929442,103.39880823)(462.17352562,103.30399596)
\closepath
}
}
{
\newrgbcolor{curcolor}{0 0 0}
\pscustom[linestyle=none,fillstyle=solid,fillcolor=curcolor]
{
\newpath
\moveto(482.99033808,103.54779894)
\lineto(471.1158356,72.4832359)
\lineto(467.45513775,72.4832359)
\lineto(471.24332259,81.95091829)
\lineto(463.13879253,103.54779894)
\lineto(466.85412766,103.54779894)
\lineto(473.10099016,86.72539332)
\lineto(479.40248994,103.54779894)
\closepath
}
}
\end{pspicture}
}
    \captionsetup{width=0.8\linewidth}
    \caption{Uniform and Non-Uniform Memory Access}
    \label{fig:UMAvsNUMA}
\end{figure}

\begin{figure}[H]
    \centering
    \resizebox{!}{0.25\textheight}{%LaTeX with PSTricks extensions
%%Creator: Inkscape 1.0.2-2 (e86c870879, 2021-01-15)
%%Please note this file requires PSTricks extensions
\psset{xunit=.5pt,yunit=.5pt,runit=.5pt}
\begin{pspicture}(535.30073511,423.12458669)
{
\newrgbcolor{curcolor}{0.7019608 0.7019608 0.7019608}
\pscustom[linestyle=none,fillstyle=solid,fillcolor=curcolor]
{
\newpath
\moveto(410.08601575,262.18161094)
\lineto(410.08601575,169.2475259)
}
}
{
\newrgbcolor{curcolor}{0 0 0}
\pscustom[linewidth=2.64566925,linecolor=curcolor]
{
\newpath
\moveto(410.08601575,262.18161094)
\lineto(410.08601575,169.2475259)
}
}
{
\newrgbcolor{curcolor}{0 0 0}
\pscustom[linewidth=2.64566925,linecolor=curcolor]
{
\newpath
\moveto(125.21474646,262.51035425)
\lineto(125.21474646,169.57630701)
}
}
{
\newrgbcolor{curcolor}{0.80000001 0.80000001 0.80000001}
\pscustom[linestyle=none,fillstyle=solid,fillcolor=curcolor]
{
\newpath
\moveto(160.38875955,376.99323056)
\lineto(374.91198367,376.99323056)
\lineto(374.91198367,319.25593837)
\lineto(160.38875955,319.25593837)
\closepath
}
}
{
\newrgbcolor{curcolor}{0 0 0}
\pscustom[linewidth=0.4430022,linecolor=curcolor]
{
\newpath
\moveto(160.38875955,376.99323056)
\lineto(374.91198367,376.99323056)
\lineto(374.91198367,319.25593837)
\lineto(160.38875955,319.25593837)
\closepath
}
}
{
\newrgbcolor{curcolor}{0.80000001 0.80000001 0.80000001}
\pscustom[linestyle=none,fillstyle=solid,fillcolor=curcolor]
{
\newpath
\moveto(41.37207917,422.79387724)
\lineto(209.05740017,422.79387724)
\lineto(209.05740017,258.13902384)
\lineto(41.37207917,258.13902384)
\closepath
}
}
{
\newrgbcolor{curcolor}{0 0 0}
\pscustom[linewidth=0.66141731,linecolor=curcolor]
{
\newpath
\moveto(41.37207917,422.79387724)
\lineto(209.05740017,422.79387724)
\lineto(209.05740017,258.13902384)
\lineto(41.37207917,258.13902384)
\closepath
}
}
{
\newrgbcolor{curcolor}{0.80000001 0.80000001 0.80000001}
\pscustom[linestyle=none,fillstyle=solid,fillcolor=curcolor]
{
\newpath
\moveto(326.24335747,422.79387724)
\lineto(493.92867847,422.79387724)
\lineto(493.92867847,258.13902384)
\lineto(326.24335747,258.13902384)
\closepath
}
}
{
\newrgbcolor{curcolor}{0 0 0}
\pscustom[linewidth=0.66141731,linecolor=curcolor]
{
\newpath
\moveto(326.24335747,422.79387724)
\lineto(493.92867847,422.79387724)
\lineto(493.92867847,258.13902384)
\lineto(326.24335747,258.13902384)
\closepath
}
}
{
\newrgbcolor{curcolor}{0 0 0}
\pscustom[linestyle=none,fillstyle=solid,fillcolor=curcolor]
{
\newpath
\moveto(111.53315206,328.07395942)
\curveto(110.81700851,327.76146041)(110.16596892,327.46849259)(109.58003328,327.19505596)
\curveto(109.00711844,326.92161933)(108.25191251,326.63516191)(107.3144155,326.3356837)
\curveto(106.52014719,326.08828865)(105.65426453,325.87995598)(104.71676752,325.71068569)
\curveto(103.79229129,325.5283946)(102.77015913,325.43724906)(101.65037103,325.43724906)
\curveto(99.54100275,325.43724906)(97.62043595,325.73021688)(95.88867063,326.31615251)
\curveto(94.1699261,326.91510894)(92.67253503,327.84609556)(91.39649743,329.10911237)
\curveto(90.14650141,330.3460876)(89.16994202,331.91509302)(88.46681926,333.81612863)
\curveto(87.7636965,335.73018504)(87.41213511,337.95023005)(87.41213511,340.47626368)
\curveto(87.41213511,342.87208938)(87.7506757,345.01400965)(88.42775688,346.90202447)
\curveto(89.10483806,348.79003929)(90.08139745,350.3850863)(91.35743505,351.68716548)
\curveto(92.59441028,352.95018229)(94.08529095,353.91372089)(95.83007706,354.57778128)
\curveto(97.58788397,355.24184167)(99.53449235,355.57387186)(101.66990222,355.57387186)
\curveto(103.23239724,355.57387186)(104.78838187,355.38507038)(106.33785611,355.00746741)
\curveto(107.90035113,354.62986445)(109.63211645,353.96580406)(111.53315206,353.01528625)
\lineto(111.53315206,348.42545712)
\lineto(111.24018425,348.42545712)
\curveto(109.63862685,349.76659868)(108.05009024,350.74315807)(106.47457442,351.35513529)
\curveto(104.8990586,351.96711251)(103.21286606,352.27310112)(101.41599678,352.27310112)
\curveto(99.9446473,352.27310112)(98.61652652,352.03221647)(97.43163446,351.55044717)
\curveto(96.25976319,351.08169866)(95.21158945,350.34602392)(94.28711323,349.34342295)
\curveto(93.38867859,348.36686355)(92.68555582,347.12988833)(92.17774494,345.63249726)
\curveto(91.68295485,344.14812699)(91.4355598,342.42938246)(91.4355598,340.47626368)
\curveto(91.4355598,338.43199935)(91.70899643,336.67419245)(92.25586969,335.20284297)
\curveto(92.81576374,333.73149349)(93.5319073,332.53358063)(94.40430035,331.60910441)
\curveto(95.31575578,330.64556581)(96.37695032,329.92942226)(97.58788397,329.46067375)
\curveto(98.8118384,329.00494603)(100.1008968,328.77708218)(101.45505915,328.77708218)
\curveto(103.31703239,328.77708218)(105.0618185,329.09609158)(106.68941749,329.73411038)
\curveto(108.31701647,330.37212918)(109.84044912,331.32915738)(111.25971544,332.60519499)
\lineto(111.53315206,332.60519499)
\closepath
}
}
{
\newrgbcolor{curcolor}{0 0 0}
\pscustom[linestyle=none,fillstyle=solid,fillcolor=curcolor]
{
\newpath
\moveto(136.025261,346.25749527)
\curveto(136.025261,344.96843687)(135.79739714,343.77052402)(135.34166943,342.66375671)
\curveto(134.8989625,341.5700102)(134.27396449,340.61949239)(133.4666754,339.81220329)
\curveto(132.46407442,338.80960232)(131.27918236,338.05439639)(129.91199921,337.54658551)
\curveto(128.54481607,337.05179541)(126.81956114,336.80440037)(124.73623444,336.80440037)
\lineto(120.86905926,336.80440037)
\lineto(120.86905926,325.96459113)
\lineto(117.00188407,325.96459113)
\lineto(117.00188407,355.04652979)
\lineto(124.89248395,355.04652979)
\curveto(126.63727006,355.04652979)(128.11512993,354.89679068)(129.32606358,354.59731247)
\curveto(130.53699722,354.31085505)(131.61121255,353.85512733)(132.54870957,353.23012932)
\curveto(133.65547688,352.48794418)(134.50833875,351.56346796)(135.10729517,350.45670065)
\curveto(135.71927239,349.34993334)(136.025261,347.95019821)(136.025261,346.25749527)
\closepath
\moveto(132.00183631,346.15983933)
\curveto(132.00183631,347.16244031)(131.82605562,348.03483336)(131.47449424,348.7770185)
\curveto(131.12293286,349.51920364)(130.58908039,350.12467046)(129.87293684,350.59341897)
\curveto(129.24793883,350.99706351)(128.53179527,351.28352093)(127.72450618,351.45279123)
\curveto(126.93023787,351.63508232)(125.9211265,351.72622786)(124.69717207,351.72622786)
\lineto(120.86905926,351.72622786)
\lineto(120.86905926,340.10517111)
\lineto(124.13076762,340.10517111)
\curveto(125.69326265,340.10517111)(126.96278985,340.24188942)(127.93934924,340.51532605)
\curveto(128.91590864,340.80178347)(129.71017694,341.25100079)(130.32215416,341.86297801)
\curveto(130.93413138,342.48797602)(131.36381751,343.14552601)(131.61121255,343.83562798)
\curveto(131.87162839,344.52572995)(132.00183631,345.30046707)(132.00183631,346.15983933)
\closepath
}
}
{
\newrgbcolor{curcolor}{0 0 0}
\pscustom[linestyle=none,fillstyle=solid,fillcolor=curcolor]
{
\newpath
\moveto(163.01736066,337.64424144)
\curveto(163.01736066,335.53487316)(162.78298641,333.69243111)(162.3142379,332.11691529)
\curveto(161.85851018,330.55442027)(161.10330426,329.25234108)(160.04862011,328.21067773)
\curveto(159.04601914,327.22109755)(157.87414787,326.4984436)(156.53300631,326.04271588)
\curveto(155.19186474,325.58698817)(153.62936972,325.35912431)(151.84552123,325.35912431)
\curveto(150.02261037,325.35912431)(148.43407376,325.60000896)(147.0799114,326.08177826)
\curveto(145.72574905,326.56354756)(144.58642976,327.27318072)(143.66195354,328.21067773)
\curveto(142.6072694,329.27838266)(141.84555307,330.56744106)(141.37680456,332.07785292)
\curveto(140.92107685,333.58826478)(140.69321299,335.44372762)(140.69321299,337.64424144)
\lineto(140.69321299,355.04652979)
\lineto(144.56038818,355.04652979)
\lineto(144.56038818,337.44892957)
\curveto(144.56038818,335.87341375)(144.66455451,334.62992813)(144.87288718,333.71847269)
\curveto(145.09424064,332.80701726)(145.45882282,331.98019698)(145.9666337,331.23801184)
\curveto(146.53954854,330.39166037)(147.31428566,329.75364157)(148.29084505,329.32395544)
\curveto(149.28042523,328.8942693)(150.46531729,328.67942624)(151.84552123,328.67942624)
\curveto(153.23874596,328.67942624)(154.42363802,328.88775891)(155.40019741,329.30442425)
\curveto(156.3767568,329.73411038)(157.15800432,330.37863958)(157.74393995,331.23801184)
\curveto(158.25175083,331.98019698)(158.60982261,332.82654845)(158.81815528,333.77706626)
\curveto(159.03950874,334.74060486)(159.15018547,335.93200731)(159.15018547,337.35127363)
\lineto(159.15018547,355.04652979)
\lineto(163.01736066,355.04652979)
\closepath
}
}
{
\newrgbcolor{curcolor}{0 0 0}
\pscustom[linestyle=none,fillstyle=solid,fillcolor=curcolor]
{
\newpath
\moveto(396.40441594,323.02316086)
\curveto(395.68827239,322.71066186)(395.0372328,322.41769404)(394.45129716,322.14425741)
\curveto(393.87838232,321.87082078)(393.12317639,321.58436336)(392.18567938,321.28488515)
\curveto(391.39141107,321.0374901)(390.52552841,320.82915743)(389.5880314,320.65988714)
\curveto(388.66355517,320.47759605)(387.64142301,320.38645051)(386.52163491,320.38645051)
\curveto(384.41226663,320.38645051)(382.49169982,320.67941833)(380.75993451,321.26535396)
\curveto(379.04118998,321.86431039)(377.54379891,322.79529701)(376.26776131,324.05831382)
\curveto(375.01776529,325.29528905)(374.0412059,326.86429447)(373.33808314,328.76533008)
\curveto(372.63496037,330.67938649)(372.28339899,332.8994315)(372.28339899,335.42546513)
\curveto(372.28339899,337.82129083)(372.62193958,339.9632111)(373.29902076,341.85122592)
\curveto(373.97610194,343.73924074)(374.95266133,345.33428774)(376.22869893,346.63636693)
\curveto(377.46567416,347.89938374)(378.95655483,348.86292234)(380.70134094,349.52698273)
\curveto(382.45914785,350.19104311)(384.40575623,350.52307331)(386.5411661,350.52307331)
\curveto(388.10366112,350.52307331)(389.65964575,350.33427182)(391.20911999,349.95666886)
\curveto(392.77161501,349.5790659)(394.50338033,348.91500551)(396.40441594,347.9644877)
\lineto(396.40441594,343.37465857)
\lineto(396.11144813,343.37465857)
\curveto(394.50989073,344.71580013)(392.92135412,345.69235952)(391.3458383,346.30433674)
\curveto(389.77032248,346.91631396)(388.08412994,347.22230257)(386.28726066,347.22230257)
\curveto(384.81591117,347.22230257)(383.4877904,346.98141792)(382.30289834,346.49964862)
\curveto(381.13102707,346.03090011)(380.08285333,345.29522537)(379.1583771,344.29262439)
\curveto(378.25994247,343.316065)(377.5568197,342.07908978)(377.04900882,340.58169871)
\curveto(376.55421873,339.09732844)(376.30682368,337.37858391)(376.30682368,335.42546513)
\curveto(376.30682368,333.3812008)(376.58026031,331.6233939)(377.12713357,330.15204442)
\curveto(377.68702762,328.68069493)(378.40317118,327.48278208)(379.27556423,326.55830586)
\curveto(380.18701966,325.59476726)(381.2482142,324.87862371)(382.45914785,324.4098752)
\curveto(383.68310228,323.95414748)(384.97216068,323.72628363)(386.32632303,323.72628363)
\curveto(388.18829627,323.72628363)(389.93308238,324.04529303)(391.56068137,324.68331183)
\curveto(393.18828035,325.32133063)(394.711713,326.27835883)(396.13097931,327.55439644)
\lineto(396.40441594,327.55439644)
\closepath
}
}
{
\newrgbcolor{curcolor}{0 0 0}
\pscustom[linestyle=none,fillstyle=solid,fillcolor=curcolor]
{
\newpath
\moveto(420.89652488,341.20669672)
\curveto(420.89652488,339.91763832)(420.66866102,338.71972547)(420.21293331,337.61295816)
\curveto(419.77022638,336.51921164)(419.14522837,335.56869384)(418.33793928,334.76140474)
\curveto(417.3353383,333.75880377)(416.15044624,333.00359784)(414.78326309,332.49578695)
\curveto(413.41607995,332.00099686)(411.69082502,331.75360182)(409.60749832,331.75360182)
\lineto(405.74032314,331.75360182)
\lineto(405.74032314,320.91379258)
\lineto(401.87314795,320.91379258)
\lineto(401.87314795,349.99573124)
\lineto(409.76374782,349.99573124)
\curveto(411.50853394,349.99573124)(412.98639381,349.84599213)(414.19732746,349.54651392)
\curveto(415.4082611,349.26005649)(416.48247643,348.80432878)(417.41997345,348.17933077)
\curveto(418.52674076,347.43714563)(419.37960263,346.51266941)(419.97855905,345.4059021)
\curveto(420.59053627,344.29913479)(420.89652488,342.89939966)(420.89652488,341.20669672)
\closepath
\moveto(416.87310019,341.10904078)
\curveto(416.87310019,342.11164175)(416.6973195,342.98403481)(416.34575812,343.72621995)
\curveto(415.99419674,344.46840508)(415.46034427,345.07387191)(414.74420072,345.54262041)
\curveto(414.11920271,345.94626496)(413.40305915,346.23272238)(412.59577006,346.40199268)
\curveto(411.80150175,346.58428376)(410.79239038,346.67542931)(409.56843595,346.67542931)
\lineto(405.74032314,346.67542931)
\lineto(405.74032314,335.05437256)
\lineto(409.0020315,335.05437256)
\curveto(410.56452653,335.05437256)(411.83405373,335.19109087)(412.81061312,335.4645275)
\curveto(413.78717251,335.75098492)(414.58144082,336.20020224)(415.19341804,336.81217946)
\curveto(415.80539526,337.43717747)(416.23508139,338.09472746)(416.48247643,338.78482943)
\curveto(416.74289227,339.4749314)(416.87310019,340.24966852)(416.87310019,341.10904078)
\closepath
}
}
{
\newrgbcolor{curcolor}{0 0 0}
\pscustom[linestyle=none,fillstyle=solid,fillcolor=curcolor]
{
\newpath
\moveto(447.88862454,332.59344289)
\curveto(447.88862454,330.48407461)(447.65425029,328.64163256)(447.18550178,327.06611674)
\curveto(446.72977406,325.50362172)(445.97456813,324.20154253)(444.91988399,323.15987918)
\curveto(443.91728302,322.170299)(442.74541175,321.44764505)(441.40427019,320.99191733)
\curveto(440.06312862,320.53618962)(438.5006336,320.30832576)(436.71678511,320.30832576)
\curveto(434.89387425,320.30832576)(433.30533764,320.54921041)(431.95117528,321.03097971)
\curveto(430.59701293,321.51274901)(429.45769364,322.22238216)(428.53321742,323.15987918)
\curveto(427.47853327,324.22758411)(426.71681695,325.51664251)(426.24806844,327.02705437)
\curveto(425.79234073,328.53746622)(425.56447687,330.39292907)(425.56447687,332.59344289)
\lineto(425.56447687,349.99573124)
\lineto(429.43165206,349.99573124)
\lineto(429.43165206,332.39813102)
\curveto(429.43165206,330.8226152)(429.53581839,329.57912957)(429.74415106,328.66767414)
\curveto(429.96550452,327.75621871)(430.3300867,326.92939843)(430.83789758,326.18721329)
\curveto(431.41081242,325.34086182)(432.18554954,324.70284302)(433.16210893,324.27315688)
\curveto(434.15168911,323.84347075)(435.33658117,323.62862769)(436.71678511,323.62862769)
\curveto(438.11000984,323.62862769)(439.2949019,323.83696036)(440.27146129,324.2536257)
\curveto(441.24802068,324.68331183)(442.0292682,325.32784103)(442.61520383,326.18721329)
\curveto(443.12301471,326.92939843)(443.48108649,327.7757499)(443.68941916,328.72626771)
\curveto(443.91077262,329.68980631)(444.02144935,330.88120876)(444.02144935,332.30047508)
\lineto(444.02144935,349.99573124)
\lineto(447.88862454,349.99573124)
\closepath
}
}
{
\newrgbcolor{curcolor}{0 0 0}
\pscustom[linestyle=none,fillstyle=solid,fillcolor=curcolor]
{
\newpath
\moveto(218.0894034,343.53045381)
\lineto(213.10789174,343.53045381)
\lineto(213.10789174,345.00962173)
\lineto(214.75992367,345.00962173)
\lineto(214.75992367,356.54129273)
\lineto(213.10789174,356.54129273)
\lineto(213.10789174,358.02046066)
\lineto(218.0894034,358.02046066)
\lineto(218.0894034,356.54129273)
\lineto(216.43737147,356.54129273)
\lineto(216.43737147,345.00962173)
\lineto(218.0894034,345.00962173)
\closepath
}
}
{
\newrgbcolor{curcolor}{0 0 0}
\pscustom[linestyle=none,fillstyle=solid,fillcolor=curcolor]
{
\newpath
\moveto(228.73018357,343.53045381)
\lineto(227.13745535,343.53045381)
\lineto(227.13745535,349.71960381)
\curveto(227.13745535,350.21914737)(227.11203947,350.68625303)(227.06120772,351.12092079)
\curveto(227.01037597,351.56207614)(226.91718443,351.90591781)(226.78163309,352.1524458)
\curveto(226.64043378,352.4249241)(226.43710677,352.62603903)(226.17165207,352.75579061)
\curveto(225.90619737,352.89202976)(225.56167105,352.96014933)(225.13807312,352.96014933)
\curveto(224.70317925,352.96014933)(224.24851747,352.83688534)(223.77408779,352.59035735)
\curveto(223.29965811,352.34382936)(222.84499633,352.0291818)(222.41010245,351.64641466)
\lineto(222.41010245,343.53045381)
\lineto(220.81737424,343.53045381)
\lineto(220.81737424,354.40039179)
\lineto(222.41010245,354.40039179)
\lineto(222.41010245,353.19370216)
\curveto(222.90712402,353.6672954)(223.42108951,354.03708738)(223.95199892,354.30307811)
\curveto(224.48290832,354.56906883)(225.02793766,354.70206419)(225.58708693,354.70206419)
\curveto(226.60936993,354.70206419)(227.38879012,354.34849116)(227.9253475,353.64134509)
\curveto(228.46190488,352.93419902)(228.73018357,351.91564918)(228.73018357,350.58569556)
\closepath
}
}
{
\newrgbcolor{curcolor}{0 0 0}
\pscustom[linestyle=none,fillstyle=solid,fillcolor=curcolor]
{
\newpath
\moveto(236.71076889,343.62776749)
\curveto(236.41142636,343.53694139)(236.08384396,343.46233423)(235.7280217,343.40394603)
\curveto(235.37784741,343.34555782)(235.06438494,343.31636371)(234.78763429,343.31636371)
\curveto(233.82183101,343.31636371)(233.0875946,343.61479233)(232.58492506,344.21164956)
\curveto(232.08225551,344.8085068)(231.83092074,345.76542464)(231.83092074,347.0824031)
\lineto(231.83092074,352.86283565)
\lineto(230.754982,352.86283565)
\lineto(230.754982,354.40039179)
\lineto(231.83092074,354.40039179)
\lineto(231.83092074,357.52416089)
\lineto(233.42364896,357.52416089)
\lineto(233.42364896,354.40039179)
\lineto(236.71076889,354.40039179)
\lineto(236.71076889,352.86283565)
\lineto(233.42364896,352.86283565)
\lineto(233.42364896,347.90956938)
\curveto(233.42364896,347.33866246)(233.4349449,346.89101953)(233.45753679,346.5666406)
\curveto(233.48012868,346.24874925)(233.5592003,345.95032063)(233.69475163,345.67135475)
\curveto(233.81900703,345.41185161)(233.9884462,345.22046804)(234.20306915,345.09720405)
\curveto(234.42334007,344.98042763)(234.75657044,344.92203942)(235.20276026,344.92203942)
\curveto(235.46256699,344.92203942)(235.73366967,344.96420868)(236.01606829,345.04854721)
\curveto(236.29846691,345.13937331)(236.50179391,345.21398046)(236.62604931,345.27236867)
\lineto(236.71076889,345.27236867)
\closepath
}
}
{
\newrgbcolor{curcolor}{0 0 0}
\pscustom[linestyle=none,fillstyle=solid,fillcolor=curcolor]
{
\newpath
\moveto(246.55518386,348.77566112)
\lineto(239.58276194,348.77566112)
\curveto(239.58276194,348.10744053)(239.67030551,347.52355845)(239.84539265,347.0240149)
\curveto(240.0204798,346.53095892)(240.26051863,346.12548526)(240.56550914,345.80759391)
\curveto(240.8592037,345.49619013)(241.206554,345.2626373)(241.60756004,345.10693541)
\curveto(242.01421406,344.95123353)(242.46040387,344.87338258)(242.9461295,344.87338258)
\curveto(243.58999835,344.87338258)(244.23669119,345.0193531)(244.88620802,345.31129414)
\curveto(245.54137282,345.60972276)(246.00733054,345.9016638)(246.28408119,346.18711725)
\lineto(246.36880077,346.18711725)
\lineto(246.36880077,344.19218683)
\curveto(245.8322434,343.93268368)(245.28439007,343.7153498)(244.72524081,343.54018518)
\curveto(244.16609154,343.36502055)(243.57870241,343.27743824)(242.96307342,343.27743824)
\curveto(241.39293709,343.27743824)(240.16732708,343.76400664)(239.28624339,344.73714343)
\curveto(238.40515969,345.7167678)(237.96461785,347.10510963)(237.96461785,348.90216891)
\curveto(237.96461785,350.67976545)(238.38539179,352.0908138)(239.22693968,353.13531396)
\curveto(240.07413554,354.17981411)(241.1867861,354.70206419)(242.56489136,354.70206419)
\curveto(243.84133313,354.70206419)(244.82408032,354.273884)(245.51313296,353.41752363)
\curveto(246.20783356,352.56116325)(246.55518386,351.34474226)(246.55518386,349.76826065)
\closepath
\moveto(245.00481544,350.17697811)
\curveto(244.99916747,351.13713974)(244.7873685,351.87996749)(244.36941855,352.40546136)
\curveto(243.95711656,352.93095523)(243.32736764,353.19370216)(242.48017178,353.19370216)
\curveto(241.62732795,353.19370216)(240.94674727,352.90500492)(240.43842976,352.32761042)
\curveto(239.93576021,351.75021592)(239.65053761,351.03333848)(239.58276194,350.17697811)
\closepath
}
}
{
\newrgbcolor{curcolor}{0 0 0}
\pscustom[linestyle=none,fillstyle=solid,fillcolor=curcolor]
{
\newpath
\moveto(254.87464844,352.40546136)
\lineto(254.78992886,352.40546136)
\curveto(254.55271402,352.47033715)(254.32114715,352.5157502)(254.09522825,352.54170051)
\curveto(253.87495733,352.57413841)(253.61232661,352.59035735)(253.3073361,352.59035735)
\curveto(252.81596251,352.59035735)(252.34153282,352.46384957)(251.88404706,352.210834)
\curveto(251.4265613,351.96430602)(250.98601945,351.64317087)(250.56242152,351.24742858)
\lineto(250.56242152,343.53045381)
\lineto(248.9696933,343.53045381)
\lineto(248.9696933,354.40039179)
\lineto(250.56242152,354.40039179)
\lineto(250.56242152,352.79471608)
\curveto(251.19499443,353.37859815)(251.75131971,353.7905594)(252.23139736,354.03059981)
\curveto(252.71712299,354.27712779)(253.21132057,354.40039179)(253.71399012,354.40039179)
\curveto(253.99074076,354.40039179)(254.19124378,354.39066042)(254.31549918,354.37119768)
\curveto(254.43975457,354.35822253)(254.62613766,354.32902842)(254.87464844,354.28361537)
\closepath
}
}
{
\newrgbcolor{curcolor}{0 0 0}
\pscustom[linestyle=none,fillstyle=solid,fillcolor=curcolor]
{
\newpath
\moveto(263.3720217,344.21164956)
\curveto(262.8411123,343.91970853)(262.33561877,343.69264327)(261.85554112,343.53045381)
\curveto(261.38111143,343.36826434)(260.8756179,343.28716961)(260.33906053,343.28716961)
\curveto(259.65565587,343.28716961)(259.02873093,343.40070224)(258.45828572,343.62776749)
\curveto(257.88784051,343.86132032)(257.39929089,344.21164956)(256.99263688,344.67875523)
\curveto(256.5803349,345.14586089)(256.26122445,345.73623054)(256.03530556,346.44986419)
\curveto(255.80938666,347.16349784)(255.69642722,347.99715169)(255.69642722,348.95082575)
\curveto(255.69642722,350.72842229)(256.12002514,352.12325169)(256.967221,353.13531396)
\curveto(257.82006484,354.14737622)(258.94401134,354.65340735)(260.33906053,354.65340735)
\curveto(260.88126588,354.65340735)(261.41217528,354.56582504)(261.93178874,354.39066042)
\curveto(262.45705018,354.2154958)(262.93712783,354.0014057)(263.3720217,353.74839014)
\lineto(263.3720217,351.71453424)
\lineto(263.28730212,351.71453424)
\curveto(262.80157649,352.14920201)(262.29890695,352.48331231)(261.77929349,352.71686514)
\curveto(261.265328,352.95041797)(260.76265846,353.06719438)(260.27128486,353.06719438)
\curveto(259.36760927,353.06719438)(258.65314077,352.71686514)(258.12787933,352.01620664)
\curveto(257.60826587,351.32203573)(257.34845914,350.3002421)(257.34845914,348.95082575)
\curveto(257.34845914,347.64033487)(257.6026179,346.63151639)(258.11093542,345.92437032)
\curveto(258.6249009,345.22371183)(259.34501738,344.87338258)(260.27128486,344.87338258)
\curveto(260.59321928,344.87338258)(260.92080168,344.92203942)(261.25403206,345.0193531)
\curveto(261.58726243,345.11666678)(261.88660496,345.24317457)(262.15205967,345.39887645)
\curveto(262.38362653,345.5351156)(262.60107347,345.67784233)(262.80440048,345.82705664)
\curveto(263.00772748,345.98275853)(263.1686947,346.11575389)(263.28730212,346.22604273)
\lineto(263.3720217,346.22604273)
\closepath
}
}
{
\newrgbcolor{curcolor}{0 0 0}
\pscustom[linestyle=none,fillstyle=solid,fillcolor=curcolor]
{
\newpath
\moveto(273.47906826,348.96055711)
\curveto(273.47906826,347.18944815)(273.08371019,345.79137496)(272.29299406,344.76633754)
\curveto(271.50227792,343.74130011)(270.4432831,343.2287814)(269.11600958,343.2287814)
\curveto(267.77744013,343.2287814)(266.71279733,343.74130011)(265.92208119,344.76633754)
\curveto(265.13701303,345.79137496)(264.74447895,347.18944815)(264.74447895,348.96055711)
\curveto(264.74447895,350.73166608)(265.13701303,352.12973927)(265.92208119,353.15477669)
\curveto(266.71279733,354.18630169)(267.77744013,354.70206419)(269.11600958,354.70206419)
\curveto(270.4432831,354.70206419)(271.50227792,354.18630169)(272.29299406,353.15477669)
\curveto(273.08371019,352.12973927)(273.47906826,350.73166608)(273.47906826,348.96055711)
\closepath
\moveto(271.83550829,348.96055711)
\curveto(271.83550829,350.36836167)(271.59546947,351.41286183)(271.11539181,352.09405759)
\curveto(270.63531416,352.78174092)(269.96885342,353.12558259)(269.11600958,353.12558259)
\curveto(268.25186981,353.12558259)(267.57976109,352.78174092)(267.09968344,352.09405759)
\curveto(266.62525376,351.41286183)(266.38803892,350.36836167)(266.38803892,348.96055711)
\curveto(266.38803892,347.5981656)(266.62807774,346.56339681)(267.1081554,345.85625074)
\curveto(267.58823305,345.15559225)(268.25751778,344.80526301)(269.11600958,344.80526301)
\curveto(269.96320544,344.80526301)(270.6268422,345.15234846)(271.10691985,345.84651938)
\curveto(271.59264548,346.54717787)(271.83550829,347.58519045)(271.83550829,348.96055711)
\closepath
}
}
{
\newrgbcolor{curcolor}{0 0 0}
\pscustom[linestyle=none,fillstyle=solid,fillcolor=curcolor]
{
\newpath
\moveto(283.85721627,343.53045381)
\lineto(282.26448806,343.53045381)
\lineto(282.26448806,349.71960381)
\curveto(282.26448806,350.21914737)(282.23907218,350.68625303)(282.18824043,351.12092079)
\curveto(282.13740868,351.56207614)(282.04421713,351.90591781)(281.90866579,352.1524458)
\curveto(281.76746648,352.4249241)(281.56413948,352.62603903)(281.29868477,352.75579061)
\curveto(281.03323007,352.89202976)(280.68870376,352.96014933)(280.26510583,352.96014933)
\curveto(279.83021195,352.96014933)(279.37555017,352.83688534)(278.90112049,352.59035735)
\curveto(278.42669081,352.34382936)(277.97202903,352.0291818)(277.53713516,351.64641466)
\lineto(277.53713516,343.53045381)
\lineto(275.94440694,343.53045381)
\lineto(275.94440694,354.40039179)
\lineto(277.53713516,354.40039179)
\lineto(277.53713516,353.19370216)
\curveto(278.03415673,353.6672954)(278.54812222,354.03708738)(279.07903162,354.30307811)
\curveto(279.60994103,354.56906883)(280.15497036,354.70206419)(280.71411963,354.70206419)
\curveto(281.73640264,354.70206419)(282.51582283,354.34849116)(283.0523802,353.64134509)
\curveto(283.58893758,352.93419902)(283.85721627,351.91564918)(283.85721627,350.58569556)
\closepath
}
}
{
\newrgbcolor{curcolor}{0 0 0}
\pscustom[linestyle=none,fillstyle=solid,fillcolor=curcolor]
{
\newpath
\moveto(294.83687999,343.53045381)
\lineto(293.24415177,343.53045381)
\lineto(293.24415177,349.71960381)
\curveto(293.24415177,350.21914737)(293.21873589,350.68625303)(293.16790414,351.12092079)
\curveto(293.11707239,351.56207614)(293.02388085,351.90591781)(292.88832951,352.1524458)
\curveto(292.7471302,352.4249241)(292.54380319,352.62603903)(292.27834849,352.75579061)
\curveto(292.01289379,352.89202976)(291.66836747,352.96014933)(291.24476954,352.96014933)
\curveto(290.80987567,352.96014933)(290.35521389,352.83688534)(289.88078421,352.59035735)
\curveto(289.40635453,352.34382936)(288.95169275,352.0291818)(288.51679887,351.64641466)
\lineto(288.51679887,343.53045381)
\lineto(286.92407066,343.53045381)
\lineto(286.92407066,354.40039179)
\lineto(288.51679887,354.40039179)
\lineto(288.51679887,353.19370216)
\curveto(289.01382044,353.6672954)(289.52778593,354.03708738)(290.05869534,354.30307811)
\curveto(290.58960474,354.56906883)(291.13463408,354.70206419)(291.69378335,354.70206419)
\curveto(292.71606635,354.70206419)(293.49548654,354.34849116)(294.03204392,353.64134509)
\curveto(294.5686013,352.93419902)(294.83687999,351.91564918)(294.83687999,350.58569556)
\closepath
}
}
{
\newrgbcolor{curcolor}{0 0 0}
\pscustom[linestyle=none,fillstyle=solid,fillcolor=curcolor]
{
\newpath
\moveto(305.82500893,348.77566112)
\lineto(298.85258701,348.77566112)
\curveto(298.85258701,348.10744053)(298.94013058,347.52355845)(299.11521773,347.0240149)
\curveto(299.29030487,346.53095892)(299.5303437,346.12548526)(299.83533421,345.80759391)
\curveto(300.12902877,345.49619013)(300.47637907,345.2626373)(300.87738511,345.10693541)
\curveto(301.28403913,344.95123353)(301.73022895,344.87338258)(302.21595457,344.87338258)
\curveto(302.85982343,344.87338258)(303.50651627,345.0193531)(304.15603309,345.31129414)
\curveto(304.81119789,345.60972276)(305.27715561,345.9016638)(305.55390626,346.18711725)
\lineto(305.63862585,346.18711725)
\lineto(305.63862585,344.19218683)
\curveto(305.10206847,343.93268368)(304.55421515,343.7153498)(303.99506588,343.54018518)
\curveto(303.43591661,343.36502055)(302.84852748,343.27743824)(302.23289849,343.27743824)
\curveto(300.66276216,343.27743824)(299.43715215,343.76400664)(298.55606846,344.73714343)
\curveto(297.67498476,345.7167678)(297.23444292,347.10510963)(297.23444292,348.90216891)
\curveto(297.23444292,350.67976545)(297.65521686,352.0908138)(298.49676475,353.13531396)
\curveto(299.34396061,354.17981411)(300.45661117,354.70206419)(301.83471644,354.70206419)
\curveto(303.1111582,354.70206419)(304.09390539,354.273884)(304.78295803,353.41752363)
\curveto(305.47765863,352.56116325)(305.82500893,351.34474226)(305.82500893,349.76826065)
\closepath
\moveto(304.27464051,350.17697811)
\curveto(304.26899254,351.13713974)(304.05719357,351.87996749)(303.63924362,352.40546136)
\curveto(303.22694163,352.93095523)(302.59719271,353.19370216)(301.74999685,353.19370216)
\curveto(300.89715302,353.19370216)(300.21657234,352.90500492)(299.70825483,352.32761042)
\curveto(299.20558528,351.75021592)(298.92036268,351.03333848)(298.85258701,350.17697811)
\closepath
}
}
{
\newrgbcolor{curcolor}{0 0 0}
\pscustom[linestyle=none,fillstyle=solid,fillcolor=curcolor]
{
\newpath
\moveto(315.23735281,344.21164956)
\curveto(314.70644341,343.91970853)(314.20094988,343.69264327)(313.72087222,343.53045381)
\curveto(313.24644254,343.36826434)(312.74094901,343.28716961)(312.20439163,343.28716961)
\curveto(311.52098697,343.28716961)(310.89406204,343.40070224)(310.32361683,343.62776749)
\curveto(309.75317161,343.86132032)(309.264622,344.21164956)(308.85796799,344.67875523)
\curveto(308.445666,345.14586089)(308.12655556,345.73623054)(307.90063667,346.44986419)
\curveto(307.67471777,347.16349784)(307.56175832,347.99715169)(307.56175832,348.95082575)
\curveto(307.56175832,350.72842229)(307.98535625,352.12325169)(308.83255211,353.13531396)
\curveto(309.68539594,354.14737622)(310.80934245,354.65340735)(312.20439163,354.65340735)
\curveto(312.74659698,354.65340735)(313.27750639,354.56582504)(313.79711985,354.39066042)
\curveto(314.32238128,354.2154958)(314.80245894,354.0014057)(315.23735281,353.74839014)
\lineto(315.23735281,351.71453424)
\lineto(315.15263323,351.71453424)
\curveto(314.6669076,352.14920201)(314.16423806,352.48331231)(313.6446246,352.71686514)
\curveto(313.13065911,352.95041797)(312.62798956,353.06719438)(312.13661596,353.06719438)
\curveto(311.23294038,353.06719438)(310.51847187,352.71686514)(309.99321044,352.01620664)
\curveto(309.47359698,351.32203573)(309.21379025,350.3002421)(309.21379025,348.95082575)
\curveto(309.21379025,347.64033487)(309.46794901,346.63151639)(309.97626652,345.92437032)
\curveto(310.49023201,345.22371183)(311.21034849,344.87338258)(312.13661596,344.87338258)
\curveto(312.45855039,344.87338258)(312.78613279,344.92203942)(313.11936316,345.0193531)
\curveto(313.45259353,345.11666678)(313.75193607,345.24317457)(314.01739077,345.39887645)
\curveto(314.24895764,345.5351156)(314.46640458,345.67784233)(314.66973159,345.82705664)
\curveto(314.87305859,345.98275853)(315.03402581,346.11575389)(315.15263323,346.22604273)
\lineto(315.23735281,346.22604273)
\closepath
}
}
{
\newrgbcolor{curcolor}{0 0 0}
\pscustom[linestyle=none,fillstyle=solid,fillcolor=curcolor]
{
\newpath
\moveto(322.19283077,343.62776749)
\curveto(321.89348823,343.53694139)(321.56590583,343.46233423)(321.21008357,343.40394603)
\curveto(320.85990928,343.34555782)(320.54644682,343.31636371)(320.26969617,343.31636371)
\curveto(319.30389289,343.31636371)(318.56965648,343.61479233)(318.06698693,344.21164956)
\curveto(317.56431739,344.8085068)(317.31298262,345.76542464)(317.31298262,347.0824031)
\lineto(317.31298262,352.86283565)
\lineto(316.23704388,352.86283565)
\lineto(316.23704388,354.40039179)
\lineto(317.31298262,354.40039179)
\lineto(317.31298262,357.52416089)
\lineto(318.90571083,357.52416089)
\lineto(318.90571083,354.40039179)
\lineto(322.19283077,354.40039179)
\lineto(322.19283077,352.86283565)
\lineto(318.90571083,352.86283565)
\lineto(318.90571083,347.90956938)
\curveto(318.90571083,347.33866246)(318.91700678,346.89101953)(318.93959867,346.5666406)
\curveto(318.96219056,346.24874925)(319.04126217,345.95032063)(319.17681351,345.67135475)
\curveto(319.3010689,345.41185161)(319.47050807,345.22046804)(319.68513103,345.09720405)
\curveto(319.90540195,344.98042763)(320.23863232,344.92203942)(320.68482214,344.92203942)
\curveto(320.94462887,344.92203942)(321.21573155,344.96420868)(321.49813017,345.04854721)
\curveto(321.78052879,345.13937331)(321.98385579,345.21398046)(322.10811118,345.27236867)
\lineto(322.19283077,345.27236867)
\closepath
}
}
{
\newrgbcolor{curcolor}{0.80000001 0.80000001 0.80000001}
\pscustom[linestyle=none,fillstyle=solid,fillcolor=curcolor]
{
\newpath
\moveto(0.42503943,183.1639233)
\lineto(250.00445434,183.1639233)
\lineto(250.00445434,0.42505477)
\lineto(0.42503943,0.42505477)
\closepath
}
}
{
\newrgbcolor{curcolor}{0 0 0}
\pscustom[linewidth=0.85007621,linecolor=curcolor]
{
\newpath
\moveto(0.42503943,183.1639233)
\lineto(250.00445434,183.1639233)
\lineto(250.00445434,0.42505477)
\lineto(0.42503943,0.42505477)
\closepath
}
}
{
\newrgbcolor{curcolor}{0 0 0}
\pscustom[linestyle=none,fillstyle=solid,fillcolor=curcolor]
{
\newpath
\moveto(76.46007953,81.53327777)
\lineto(72.85401896,81.53327777)
\lineto(72.85401896,107.59987972)
\lineto(65.31407413,89.86321292)
\lineto(63.16500773,89.86321292)
\lineto(55.67970019,107.59987972)
\lineto(55.67970019,81.53327777)
\lineto(52.31040117,81.53327777)
\lineto(52.31040117,111.78516421)
\lineto(57.22775649,111.78516421)
\lineto(64.45809006,94.94244167)
\lineto(71.45166207,111.78516421)
\lineto(76.46007953,111.78516421)
\closepath
}
}
{
\newrgbcolor{curcolor}{0 0 0}
\pscustom[linestyle=none,fillstyle=solid,fillcolor=curcolor]
{
\newpath
\moveto(100.50048297,92.48409496)
\lineto(85.51165545,92.48409496)
\curveto(85.51165545,91.08900012)(85.69985053,89.86998522)(86.07624069,88.82705025)
\curveto(86.45263085,87.79765989)(86.96864962,86.95112177)(87.624297,86.28743588)
\curveto(88.25566114,85.6372946)(89.00237065,85.14968864)(89.86442553,84.824618)
\curveto(90.73862203,84.49954736)(91.69780986,84.33701204)(92.74198902,84.33701204)
\curveto(94.12613348,84.33701204)(95.51634875,84.64176576)(96.91263483,85.25127321)
\curveto(98.32106252,85.87432527)(99.32274601,86.48383272)(99.9176853,87.07979556)
\lineto(100.09980957,87.07979556)
\lineto(100.09980957,82.91482799)
\curveto(98.94635585,82.37304359)(97.7686189,81.91929915)(96.56659871,81.55359468)
\curveto(95.36457852,81.18789021)(94.10185024,81.00503798)(92.77841387,81.00503798)
\curveto(89.40304404,81.00503798)(86.76831292,82.02088373)(84.8742205,84.05257523)
\curveto(82.98012808,86.09781134)(82.03308187,88.99635788)(82.03308187,92.74821485)
\curveto(82.03308187,96.45943799)(82.93763242,99.40539067)(84.74673351,101.58607288)
\curveto(86.56797622,103.76675509)(88.95987498,104.85709619)(91.9224298,104.85709619)
\curveto(94.66643548,104.85709619)(96.77907703,103.96315193)(98.26035443,102.17526341)
\curveto(99.75377346,100.38737489)(100.50048297,97.84776052)(100.50048297,94.55642029)
\closepath
\moveto(97.1676088,95.40973072)
\curveto(97.15546719,97.414333)(96.70015651,98.96519084)(95.80167677,100.06230425)
\curveto(94.91533865,101.15941766)(93.56154824,101.70797437)(91.74030552,101.70797437)
\curveto(89.90692119,101.70797437)(88.44385622,101.10523922)(87.35111059,99.89976893)
\curveto(86.27050658,98.69429864)(85.65735487,97.19761924)(85.51165545,95.40973072)
\closepath
}
}
{
\newrgbcolor{curcolor}{0 0 0}
\pscustom[linestyle=none,fillstyle=solid,fillcolor=curcolor]
{
\newpath
\moveto(135.37728096,81.53327777)
\lineto(131.95334466,81.53327777)
\lineto(131.95334466,94.45483571)
\curveto(131.95334466,95.43004763)(131.910849,96.37139803)(131.82585767,97.2788869)
\curveto(131.75300797,98.18637577)(131.58909612,98.9110124)(131.33412214,99.4527968)
\curveto(131.05486493,100.03521503)(130.65419153,100.47541486)(130.13210195,100.77339628)
\curveto(129.61001237,101.0713777)(128.85723205,101.22036841)(127.87376099,101.22036841)
\curveto(126.91457316,101.22036841)(125.95538533,100.94947621)(124.9961975,100.40769181)
\curveto(124.03700968,99.87945202)(123.07782185,99.20222152)(122.11863402,98.37600031)
\curveto(122.15505887,98.06447428)(122.18541292,97.69876981)(122.20969616,97.2788869)
\curveto(122.23397939,96.8725486)(122.24612101,96.4662103)(122.24612101,96.059872)
\lineto(122.24612101,81.53327777)
\lineto(118.82218471,81.53327777)
\lineto(118.82218471,94.45483571)
\curveto(118.82218471,95.45713685)(118.77968905,96.40525955)(118.69469772,97.29920381)
\curveto(118.62184801,98.20669268)(118.45793617,98.93132932)(118.20296219,99.47311372)
\curveto(117.92370497,100.05553195)(117.52303158,100.48895947)(117.000942,100.77339628)
\curveto(116.47885242,101.0713777)(115.7260721,101.22036841)(114.74260104,101.22036841)
\curveto(113.80769645,101.22036841)(112.86672105,100.96302082)(111.91967484,100.44832564)
\curveto(110.98477024,99.93363046)(110.04986565,99.27671687)(109.11496106,98.47758488)
\lineto(109.11496106,81.53327777)
\lineto(105.69102476,81.53327777)
\lineto(105.69102476,104.22727183)
\lineto(109.11496106,104.22727183)
\lineto(109.11496106,101.70797437)
\curveto(110.18342345,102.6967309)(111.24581503,103.46877367)(112.30213581,104.02410268)
\curveto(113.3705982,104.57943169)(114.50583949,104.85709619)(115.70785968,104.85709619)
\curveto(117.09200414,104.85709619)(118.26367028,104.53202555)(119.22285811,103.88188427)
\curveto(120.19418756,103.23174299)(120.91661383,102.33102643)(121.39013694,101.17973458)
\curveto(122.7742814,102.48001714)(124.03700968,103.41459523)(125.17832178,103.98346885)
\curveto(126.31963388,104.56588708)(127.53986649,104.85709619)(128.83901963,104.85709619)
\curveto(131.07307735,104.85709619)(132.7182666,104.09859803)(133.77458737,102.58160171)
\curveto(134.84304977,101.07815)(135.37728096,98.97196315)(135.37728096,96.26304115)
\closepath
}
}
{
\newrgbcolor{curcolor}{0 0 0}
\pscustom[linestyle=none,fillstyle=solid,fillcolor=curcolor]
{
\newpath
\moveto(159.30840969,92.87011634)
\curveto(159.30840969,89.17243781)(158.45849642,86.25357435)(156.75866989,84.11352597)
\curveto(155.05884336,81.97347759)(152.78228997,80.9034534)(149.92900972,80.9034534)
\curveto(147.05144624,80.9034534)(144.76275123,81.97347759)(143.0629247,84.11352597)
\curveto(141.37523978,86.25357435)(140.53139733,89.17243781)(140.53139733,92.87011634)
\curveto(140.53139733,96.56779487)(141.37523978,99.48665833)(143.0629247,101.62670671)
\curveto(144.76275123,103.7802997)(147.05144624,104.85709619)(149.92900972,104.85709619)
\curveto(152.78228997,104.85709619)(155.05884336,103.7802997)(156.75866989,101.62670671)
\curveto(158.45849642,99.48665833)(159.30840969,96.56779487)(159.30840969,92.87011634)
\closepath
\moveto(155.77519882,92.87011634)
\curveto(155.77519882,95.80929671)(155.25918006,97.98997892)(154.22714252,99.41216297)
\curveto(153.19510498,100.84789163)(151.76239405,101.56575596)(149.92900972,101.56575596)
\curveto(148.07134215,101.56575596)(146.6264896,100.84789163)(145.59445207,99.41216297)
\curveto(144.57455615,97.98997892)(144.06460819,95.80929671)(144.06460819,92.87011634)
\curveto(144.06460819,90.02574824)(144.58062696,87.86538294)(145.61266449,86.38902045)
\curveto(146.64470203,84.92620257)(148.08348377,84.19479363)(149.92900972,84.19479363)
\curveto(151.75025243,84.19479363)(153.17689256,84.91943027)(154.20893009,86.36870354)
\curveto(155.25310925,87.83152142)(155.77519882,89.99865902)(155.77519882,92.87011634)
\closepath
}
}
{
\newrgbcolor{curcolor}{0 0 0}
\pscustom[linestyle=none,fillstyle=solid,fillcolor=curcolor]
{
\newpath
\moveto(177.30228509,100.06230425)
\lineto(177.12016082,100.06230425)
\curveto(176.61021286,100.19775035)(176.11240651,100.29256262)(175.62674179,100.34674106)
\curveto(175.15321869,100.41446411)(174.58863345,100.44832564)(173.93298607,100.44832564)
\curveto(172.8766653,100.44832564)(171.85676938,100.18420574)(170.87329831,99.65596595)
\curveto(169.88982725,99.14127077)(168.94278104,98.47081258)(168.03215968,97.64459137)
\lineto(168.03215968,81.53327777)
\lineto(164.60822339,81.53327777)
\lineto(164.60822339,104.22727183)
\lineto(168.03215968,104.22727183)
\lineto(168.03215968,100.87498085)
\curveto(169.39202091,102.09399575)(170.58797029,102.95407849)(171.62000783,103.45522906)
\curveto(172.66418698,103.96992424)(173.72657856,104.22727183)(174.80718257,104.22727183)
\curveto(175.40212186,104.22727183)(175.8331493,104.20695491)(176.1002649,104.16632108)
\curveto(176.36738049,104.13923186)(176.76805389,104.07828112)(177.30228509,103.98346885)
\closepath
}
}
{
\newrgbcolor{curcolor}{0 0 0}
\pscustom[linestyle=none,fillstyle=solid,fillcolor=curcolor]
{
\newpath
\moveto(198.11909755,104.22727183)
\lineto(186.24459507,73.16270879)
\lineto(182.58389722,73.16270879)
\lineto(186.37208206,82.63039118)
\lineto(178.267552,104.22727183)
\lineto(181.98288713,104.22727183)
\lineto(188.22974963,87.4048662)
\lineto(194.53124941,104.22727183)
\closepath
}
}
{
\newrgbcolor{curcolor}{0.80000001 0.80000001 0.80000001}
\pscustom[linestyle=none,fillstyle=solid,fillcolor=curcolor]
{
\newpath
\moveto(285.29630331,183.1639233)
\lineto(534.87571821,183.1639233)
\lineto(534.87571821,0.42505477)
\lineto(285.29630331,0.42505477)
\closepath
}
}
{
\newrgbcolor{curcolor}{0 0 0}
\pscustom[linewidth=0.85007621,linecolor=curcolor]
{
\newpath
\moveto(285.29630331,183.1639233)
\lineto(534.87571821,183.1639233)
\lineto(534.87571821,0.42505477)
\lineto(285.29630331,0.42505477)
\closepath
}
}
{
\newrgbcolor{curcolor}{0 0 0}
\pscustom[linestyle=none,fillstyle=solid,fillcolor=curcolor]
{
\newpath
\moveto(361.33132006,80.85380488)
\lineto(357.72525949,80.85380488)
\lineto(357.72525949,106.92040683)
\lineto(350.18531466,89.18374003)
\lineto(348.03624826,89.18374003)
\lineto(340.55094072,106.92040683)
\lineto(340.55094072,80.85380488)
\lineto(337.1816417,80.85380488)
\lineto(337.1816417,111.10569132)
\lineto(342.09899702,111.10569132)
\lineto(349.32933059,94.26296878)
\lineto(356.3229026,111.10569132)
\lineto(361.33132006,111.10569132)
\closepath
}
}
{
\newrgbcolor{curcolor}{0 0 0}
\pscustom[linestyle=none,fillstyle=solid,fillcolor=curcolor]
{
\newpath
\moveto(385.3717235,91.80462207)
\lineto(370.38289598,91.80462207)
\curveto(370.38289598,90.40952724)(370.57109106,89.19051234)(370.94748122,88.14757737)
\curveto(371.32387138,87.11818701)(371.83989015,86.27164888)(372.49553753,85.60796299)
\curveto(373.12690167,84.95782171)(373.87361118,84.47021575)(374.73566606,84.14514511)
\curveto(375.60986256,83.82007447)(376.56905039,83.65753915)(377.61322955,83.65753915)
\curveto(378.99737401,83.65753915)(380.38758928,83.96229288)(381.78387536,84.57180033)
\curveto(383.19230305,85.19485239)(384.19398655,85.80435984)(384.78892583,86.40032268)
\lineto(384.9710501,86.40032268)
\lineto(384.9710501,82.2353551)
\curveto(383.81759638,81.6935707)(382.63985943,81.23982627)(381.43783924,80.8741218)
\curveto(380.23581905,80.50841733)(378.97309077,80.32556509)(377.6496544,80.32556509)
\curveto(374.27428458,80.32556509)(371.63955345,81.34141084)(369.74546103,83.37310234)
\curveto(367.85136861,85.41833845)(366.9043224,88.31688499)(366.9043224,92.06874196)
\curveto(366.9043224,95.7799651)(367.80887295,98.72591778)(369.61797404,100.90659999)
\curveto(371.43921676,103.0872822)(373.83111552,104.17762331)(376.79367033,104.17762331)
\curveto(379.53767601,104.17762331)(381.65031756,103.28367905)(383.13159496,101.49579053)
\curveto(384.62501399,99.70790201)(385.3717235,97.16828763)(385.3717235,93.8769474)
\closepath
\moveto(382.03884934,94.73025783)
\curveto(382.02670772,96.73486011)(381.57139704,98.28571796)(380.6729173,99.38283137)
\curveto(379.78657918,100.47994478)(378.43278877,101.02850148)(376.61154606,101.02850148)
\curveto(374.77816173,101.02850148)(373.31509675,100.42576634)(372.22235112,99.22029605)
\curveto(371.14174711,98.01482576)(370.5285954,96.51814635)(370.38289598,94.73025783)
\closepath
}
}
{
\newrgbcolor{curcolor}{0 0 0}
\pscustom[linestyle=none,fillstyle=solid,fillcolor=curcolor]
{
\newpath
\moveto(420.24852149,80.85380488)
\lineto(416.82458519,80.85380488)
\lineto(416.82458519,93.77536282)
\curveto(416.82458519,94.75057474)(416.78208953,95.69192514)(416.69709821,96.59941401)
\curveto(416.6242485,97.50690288)(416.46033665,98.23153952)(416.20536267,98.77332392)
\curveto(415.92610546,99.35574215)(415.52543206,99.79594197)(415.00334248,100.09392339)
\curveto(414.48125291,100.39190481)(413.72847259,100.54089552)(412.74500152,100.54089552)
\curveto(411.78581369,100.54089552)(410.82662586,100.27000332)(409.86743804,99.72821892)
\curveto(408.90825021,99.19997913)(407.94906238,98.52274863)(406.98987455,97.69652742)
\curveto(407.02629941,97.38500139)(407.05665345,97.01929692)(407.08093669,96.59941401)
\curveto(407.10521992,96.19307571)(407.11736154,95.78673741)(407.11736154,95.38039911)
\lineto(407.11736154,80.85380488)
\lineto(403.69342524,80.85380488)
\lineto(403.69342524,93.77536282)
\curveto(403.69342524,94.77766396)(403.65092958,95.72578666)(403.56593825,96.61973092)
\curveto(403.49308855,97.5272198)(403.3291767,98.25185643)(403.07420272,98.79364083)
\curveto(402.79494551,99.37605906)(402.39427211,99.80948658)(401.87218253,100.09392339)
\curveto(401.35009296,100.39190481)(400.59731263,100.54089552)(399.61384157,100.54089552)
\curveto(398.67893698,100.54089552)(397.73796158,100.28354793)(396.79091537,99.76885275)
\curveto(395.85601078,99.25415757)(394.92110618,98.59724399)(393.98620159,97.798112)
\lineto(393.98620159,80.85380488)
\lineto(390.56226529,80.85380488)
\lineto(390.56226529,103.54779894)
\lineto(393.98620159,103.54779894)
\lineto(393.98620159,101.02850148)
\curveto(395.05466398,102.01725801)(396.11705556,102.78930078)(397.17337634,103.34462979)
\curveto(398.24183873,103.8999588)(399.37708002,104.17762331)(400.57910021,104.17762331)
\curveto(401.96324467,104.17762331)(403.13491081,103.85255267)(404.09409864,103.20241139)
\curveto(405.06542809,102.55227011)(405.78785436,101.65155354)(406.26137747,100.50026169)
\curveto(407.64552193,101.80054425)(408.90825021,102.73512234)(410.04956231,103.30399596)
\curveto(411.19087441,103.88641419)(412.41110702,104.17762331)(413.71026016,104.17762331)
\curveto(415.94431788,104.17762331)(417.58950713,103.41912515)(418.64582791,101.90212883)
\curveto(419.7142903,100.39867712)(420.24852149,98.29249026)(420.24852149,95.58356826)
\closepath
}
}
{
\newrgbcolor{curcolor}{0 0 0}
\pscustom[linestyle=none,fillstyle=solid,fillcolor=curcolor]
{
\newpath
\moveto(444.17965022,92.19064345)
\curveto(444.17965022,88.49296492)(443.32973695,85.57410147)(441.62991042,83.43405309)
\curveto(439.93008389,81.29400471)(437.6535305,80.22398052)(434.80025025,80.22398052)
\curveto(431.92268677,80.22398052)(429.63399176,81.29400471)(427.93416523,83.43405309)
\curveto(426.24648032,85.57410147)(425.40263786,88.49296492)(425.40263786,92.19064345)
\curveto(425.40263786,95.88832198)(426.24648032,98.80718544)(427.93416523,100.94723382)
\curveto(429.63399176,103.10082681)(431.92268677,104.17762331)(434.80025025,104.17762331)
\curveto(437.6535305,104.17762331)(439.93008389,103.10082681)(441.62991042,100.94723382)
\curveto(443.32973695,98.80718544)(444.17965022,95.88832198)(444.17965022,92.19064345)
\closepath
\moveto(440.64643936,92.19064345)
\curveto(440.64643936,95.12982382)(440.13042059,97.31050604)(439.09838305,98.73269009)
\curveto(438.06634551,100.16841875)(436.63363458,100.88628308)(434.80025025,100.88628308)
\curveto(432.94258269,100.88628308)(431.49773013,100.16841875)(430.4656926,98.73269009)
\curveto(429.44579668,97.31050604)(428.93584872,95.12982382)(428.93584872,92.19064345)
\curveto(428.93584872,89.34627535)(429.45186749,87.18591006)(430.48390503,85.70954757)
\curveto(431.51594256,84.24672969)(432.9547243,83.51532075)(434.80025025,83.51532075)
\curveto(436.62149296,83.51532075)(438.04813309,84.23995738)(439.08017062,85.68923065)
\curveto(440.12434978,87.15204853)(440.64643936,89.31918613)(440.64643936,92.19064345)
\closepath
}
}
{
\newrgbcolor{curcolor}{0 0 0}
\pscustom[linestyle=none,fillstyle=solid,fillcolor=curcolor]
{
\newpath
\moveto(462.17352562,99.38283137)
\lineto(461.99140135,99.38283137)
\curveto(461.48145339,99.51827747)(460.98364705,99.61308974)(460.49798232,99.66726818)
\curveto(460.02445922,99.73499123)(459.45987398,99.76885275)(458.8042266,99.76885275)
\curveto(457.74790583,99.76885275)(456.72800991,99.50473286)(455.74453885,98.97649307)
\curveto(454.76106778,98.46179789)(453.81402157,97.79133969)(452.90340022,96.96511848)
\lineto(452.90340022,80.85380488)
\lineto(449.47946392,80.85380488)
\lineto(449.47946392,103.54779894)
\lineto(452.90340022,103.54779894)
\lineto(452.90340022,100.19550797)
\curveto(454.26326144,101.41452287)(455.45921082,102.2746056)(456.49124836,102.77575617)
\curveto(457.53542751,103.29045135)(458.59781909,103.54779894)(459.6784231,103.54779894)
\curveto(460.27336239,103.54779894)(460.70438983,103.52748203)(460.97150543,103.4868482)
\curveto(461.23862103,103.45975898)(461.63929442,103.39880823)(462.17352562,103.30399596)
\closepath
}
}
{
\newrgbcolor{curcolor}{0 0 0}
\pscustom[linestyle=none,fillstyle=solid,fillcolor=curcolor]
{
\newpath
\moveto(482.99033808,103.54779894)
\lineto(471.1158356,72.4832359)
\lineto(467.45513775,72.4832359)
\lineto(471.24332259,81.95091829)
\lineto(463.13879253,103.54779894)
\lineto(466.85412766,103.54779894)
\lineto(473.10099016,86.72539332)
\lineto(479.40248994,103.54779894)
\closepath
}
}
{
\newrgbcolor{curcolor}{0 1 0}
\pscustom[linestyle=none,fillstyle=solid,fillcolor=curcolor]
{
\newpath
\moveto(50.15367328,319.25592395)
\lineto(200.27580607,319.25592395)
\lineto(200.27580607,274.13378034)
\lineto(50.15367328,274.13378034)
\closepath
}
}
{
\newrgbcolor{curcolor}{0.7019608 0.7019608 0.7019608}
\pscustom[linewidth=0.72824692,linecolor=curcolor]
{
\newpath
\moveto(50.15367328,319.25592395)
\lineto(200.27580607,319.25592395)
\lineto(200.27580607,274.13378034)
\lineto(50.15367328,274.13378034)
\closepath
}
}
{
\newrgbcolor{curcolor}{0 0 0}
\pscustom[linestyle=none,fillstyle=solid,fillcolor=curcolor]
{
\newpath
\moveto(72.09956289,305.63247968)
\curveto(72.09956289,304.46670304)(71.90478775,303.38335504)(71.51523747,302.3824357)
\curveto(71.1368172,301.39329188)(70.60257681,300.5336788)(69.91251631,299.80359646)
\curveto(69.05550569,298.89688129)(68.04267496,298.21390104)(66.87402411,297.75465569)
\curveto(65.70537327,297.30718587)(64.2306472,297.08345096)(62.44984592,297.08345096)
\lineto(59.14423353,297.08345096)
\lineto(59.14423353,287.28032918)
\lineto(55.83862114,287.28032918)
\lineto(55.83862114,313.5809568)
\lineto(62.58340601,313.5809568)
\curveto(64.07482709,313.5809568)(65.338083,313.4455383)(66.37317375,313.1747013)
\curveto(67.4082645,312.91563983)(68.32649016,312.50349657)(69.12785074,311.93827153)
\curveto(70.07390143,311.26706679)(70.80291695,310.43100476)(71.31489732,309.43008542)
\curveto(71.8380077,308.42916607)(72.09956289,307.1632975)(72.09956289,305.63247968)
\closepath
\moveto(68.66039041,305.54416327)
\curveto(68.66039041,306.45087844)(68.5101353,307.23983839)(68.20962508,307.91104312)
\curveto(67.90911486,308.58224786)(67.45278453,309.12980961)(66.84063409,309.55372839)
\curveto(66.3063937,309.91876956)(65.69424326,310.17783104)(65.00418276,310.33091282)
\curveto(64.32525227,310.49577013)(63.46267665,310.57819878)(62.41645589,310.57819878)
\lineto(59.14423353,310.57819878)
\lineto(59.14423353,300.0685457)
\lineto(61.93230054,300.0685457)
\curveto(63.26790151,300.0685457)(64.35307729,300.19218868)(65.1878279,300.43947463)
\curveto(66.0225785,300.69853611)(66.70150899,301.1047916)(67.22461937,301.65824112)
\curveto(67.74772975,302.22346616)(68.11502001,302.81813)(68.32649016,303.44223265)
\curveto(68.54909033,304.0663353)(68.66039041,304.76697884)(68.66039041,305.54416327)
\closepath
}
}
{
\newrgbcolor{curcolor}{0 0 0}
\pscustom[linestyle=none,fillstyle=solid,fillcolor=curcolor]
{
\newpath
\moveto(87.84295877,303.38924281)
\lineto(87.67600865,303.38924281)
\curveto(87.20854832,303.50699802)(86.75221799,303.58942667)(86.30701766,303.63652876)
\curveto(85.87294735,303.69540637)(85.35540198,303.72484517)(84.75438154,303.72484517)
\curveto(83.78607084,303.72484517)(82.85115017,303.4952225)(81.94961952,303.03597716)
\curveto(81.04808886,302.58850733)(80.17994824,302.00561901)(79.34519763,301.28731219)
\lineto(79.34519763,287.28032918)
\lineto(76.20653536,287.28032918)
\lineto(76.20653536,307.01021571)
\lineto(79.34519763,307.01021571)
\lineto(79.34519763,304.0957741)
\curveto(80.59175853,305.15557105)(81.68806433,305.90331668)(82.63411501,306.33901098)
\curveto(83.5912957,306.7864808)(84.56517141,307.01021571)(85.55574212,307.01021571)
\curveto(86.10111252,307.01021571)(86.4962278,306.99255243)(86.74108798,306.95722587)
\curveto(86.98594815,306.93367482)(87.35323842,306.88068498)(87.84295877,306.79825632)
\closepath
}
}
{
\newrgbcolor{curcolor}{0 0 0}
\pscustom[linestyle=none,fillstyle=solid,fillcolor=curcolor]
{
\newpath
\moveto(106.69162721,297.13644081)
\curveto(106.69162721,293.92172339)(105.91252664,291.38409848)(104.35432552,289.52356605)
\curveto(102.79612439,287.66303363)(100.70924788,286.73276742)(98.09369599,286.73276742)
\curveto(95.45588409,286.73276742)(93.35787757,287.66303363)(91.79967645,289.52356605)
\curveto(90.25260533,291.38409848)(89.47906977,293.92172339)(89.47906977,297.13644081)
\curveto(89.47906977,300.35115822)(90.25260533,302.88878313)(91.79967645,304.74931556)
\curveto(93.35787757,306.6216235)(95.45588409,307.55777747)(98.09369599,307.55777747)
\curveto(100.70924788,307.55777747)(102.79612439,306.6216235)(104.35432552,304.74931556)
\curveto(105.91252664,302.88878313)(106.69162721,300.35115822)(106.69162721,297.13644081)
\closepath
\moveto(103.45279487,297.13644081)
\curveto(103.45279487,299.69172901)(102.97976953,301.58758799)(102.03371884,302.82401777)
\curveto(101.08766816,304.07222306)(99.77432721,304.69632571)(98.09369599,304.69632571)
\curveto(96.39080476,304.69632571)(95.06633381,304.07222306)(94.12028312,302.82401777)
\curveto(93.18536245,301.58758799)(92.71790211,299.69172901)(92.71790211,297.13644081)
\curveto(92.71790211,294.66358126)(93.19092745,292.78538555)(94.13697813,291.50185369)
\curveto(95.08302882,290.23009735)(96.40193477,289.59421918)(98.09369599,289.59421918)
\curveto(99.7631972,289.59421918)(101.07097315,290.22420959)(102.01702383,291.48419041)
\curveto(102.97420452,292.75594675)(103.45279487,294.64003021)(103.45279487,297.13644081)
\closepath
}
}
{
\newrgbcolor{curcolor}{0 0 0}
\pscustom[linestyle=none,fillstyle=solid,fillcolor=curcolor]
{
\newpath
\moveto(126.6755571,289.52356605)
\curveto(126.6755571,286.1793179)(125.95767158,283.72412164)(124.52190054,282.15797726)
\curveto(123.08612951,280.59183288)(120.87682291,279.80876069)(117.89398075,279.80876069)
\curveto(116.90341004,279.80876069)(115.93509934,279.88530158)(114.98904865,280.03838336)
\curveto(114.05412798,280.17968962)(113.13033731,280.38576125)(112.21767665,280.65659825)
\lineto(112.21767665,284.04794848)
\lineto(112.38462677,284.04794848)
\curveto(112.89660714,283.83598909)(113.70909773,283.57692762)(114.82209853,283.27076405)
\curveto(115.93509934,282.95282497)(117.04810014,282.79385543)(118.16110095,282.79385543)
\curveto(119.22958172,282.79385543)(120.11441736,282.92927393)(120.81560786,283.20011093)
\curveto(121.51679837,283.47094792)(122.06216877,283.84776462)(122.45171905,284.330561)
\curveto(122.84126933,284.78980635)(123.11951953,285.34325587)(123.28646965,285.99090956)
\curveto(123.45341977,286.63856325)(123.53689483,287.36275783)(123.53689483,288.1634933)
\lineto(123.53689483,289.96514812)
\curveto(122.59084415,289.16441264)(121.68374849,288.56386104)(120.81560786,288.1634933)
\curveto(119.95859725,287.77490109)(118.86229145,287.58060498)(117.52669049,287.58060498)
\curveto(115.30068888,287.58060498)(113.5310176,288.42844254)(112.21767665,290.12411766)
\curveto(110.91546571,291.8315683)(110.26436024,294.23377472)(110.26436024,297.33073691)
\curveto(110.26436024,299.02641203)(110.4869604,300.48657672)(110.93216072,301.71123097)
\curveto(111.38849105,302.94766074)(112.0062065,304.01334545)(112.78530706,304.9082851)
\curveto(113.50875758,305.74434714)(114.38802822,306.39200083)(115.42311897,306.85124617)
\curveto(116.45820972,307.32226704)(117.48773546,307.55777747)(118.5116962,307.55777747)
\curveto(119.59130698,307.55777747)(120.49283763,307.44002226)(121.21628815,307.20451182)
\curveto(121.95086869,306.98077691)(122.72440424,306.63339902)(123.53689483,306.16237816)
\lineto(123.73723498,307.01021571)
\lineto(126.6755571,307.01021571)
\closepath
\moveto(123.53689483,292.7029569)
\lineto(123.53689483,303.45989594)
\curveto(122.70214423,303.86026367)(121.92304367,304.14287619)(121.19959314,304.30773349)
\curveto(120.48727263,304.48436632)(119.77495211,304.57268273)(119.0626316,304.57268273)
\curveto(117.33748035,304.57268273)(115.97961937,303.96035561)(114.98904865,302.73570135)
\curveto(113.99847794,301.5110471)(113.50319258,299.73294333)(113.50319258,297.40139004)
\curveto(113.50319258,295.18759197)(113.87048285,293.50958014)(114.60506338,292.36735453)
\curveto(115.33964391,291.22512893)(116.55837979,290.65401613)(118.26127102,290.65401613)
\curveto(119.17393168,290.65401613)(120.08659234,290.83653672)(120.999253,291.20157789)
\curveto(121.92304367,291.57839458)(122.76892428,292.07885425)(123.53689483,292.7029569)
\closepath
}
}
{
\newrgbcolor{curcolor}{0 0 0}
\pscustom[linestyle=none,fillstyle=solid,fillcolor=curcolor]
{
\newpath
\moveto(144.48913143,303.38924281)
\lineto(144.32218131,303.38924281)
\curveto(143.85472098,303.50699802)(143.39839065,303.58942667)(142.95319032,303.63652876)
\curveto(142.51912001,303.69540637)(142.00157464,303.72484517)(141.4005542,303.72484517)
\curveto(140.4322435,303.72484517)(139.49732283,303.4952225)(138.59579218,303.03597716)
\curveto(137.69426152,302.58850733)(136.8261209,302.00561901)(135.99137029,301.28731219)
\lineto(135.99137029,287.28032918)
\lineto(132.85270803,287.28032918)
\lineto(132.85270803,307.01021571)
\lineto(135.99137029,307.01021571)
\lineto(135.99137029,304.0957741)
\curveto(137.23793119,305.15557105)(138.33423699,305.90331668)(139.28028767,306.33901098)
\curveto(140.23746836,306.7864808)(141.21134407,307.01021571)(142.20191478,307.01021571)
\curveto(142.74728518,307.01021571)(143.14240046,306.99255243)(143.38726064,306.95722587)
\curveto(143.63212082,306.93367482)(143.99941108,306.88068498)(144.48913143,306.79825632)
\closepath
}
}
{
\newrgbcolor{curcolor}{0 0 0}
\pscustom[linestyle=none,fillstyle=solid,fillcolor=curcolor]
{
\newpath
\moveto(161.33440118,287.28032918)
\lineto(158.21243392,287.28032918)
\lineto(158.21243392,289.38225979)
\curveto(157.93418372,289.18207593)(157.55576345,288.89946341)(157.0771731,288.53442224)
\curveto(156.60971276,288.18115659)(156.15338243,287.89854407)(155.70818211,287.68658468)
\curveto(155.18507173,287.41574768)(154.5840513,287.19201277)(153.90512081,287.01537994)
\curveto(153.22619032,286.8269716)(152.43039474,286.73276742)(151.51773408,286.73276742)
\curveto(149.83710287,286.73276742)(148.41246184,287.3215435)(147.243811,288.49909567)
\curveto(146.07516015,289.67664784)(145.49083473,291.17802685)(145.49083473,293.0032327)
\curveto(145.49083473,294.49872395)(145.79134495,295.70571492)(146.39236538,296.62420561)
\curveto(147.00451582,297.55447182)(147.87265645,298.28455417)(148.99678726,298.81445264)
\curveto(150.13204808,299.34435112)(151.49547407,299.70350453)(153.08706522,299.89191287)
\curveto(154.67865637,300.08032122)(156.3871126,300.22162748)(158.21243392,300.31583165)
\lineto(158.21243392,300.82806684)
\curveto(158.21243392,301.58170023)(158.08443883,302.20580288)(157.82844864,302.70037479)
\curveto(157.58358847,303.1949467)(157.22742821,303.58353891)(156.75996787,303.86615143)
\curveto(156.31476755,304.13698843)(155.78052716,304.31950902)(155.15724671,304.41371319)
\curveto(154.53396626,304.50791736)(153.88286079,304.55501945)(153.2039303,304.55501945)
\curveto(152.38030971,304.55501945)(151.46208404,304.43726423)(150.44925331,304.2017538)
\curveto(149.43642258,303.97801889)(148.39020182,303.64830428)(147.31059104,303.21260998)
\lineto(147.14364092,303.21260998)
\lineto(147.14364092,306.58629694)
\curveto(147.75579137,306.76292976)(148.640627,306.95722587)(149.79814784,307.16918526)
\curveto(150.95566868,307.38114465)(152.0964945,307.48712434)(153.22062531,307.48712434)
\curveto(154.53396626,307.48712434)(155.67479209,307.36936913)(156.64310279,307.13385869)
\curveto(157.6225435,306.91012378)(158.46842411,306.52153157)(159.18074462,305.96808205)
\curveto(159.88193513,305.42640805)(160.41617552,304.72576451)(160.78346578,303.86615143)
\curveto(161.15075605,303.00653835)(161.33440118,301.94085364)(161.33440118,300.6690973)
\closepath
\moveto(158.21243392,292.13773186)
\lineto(158.21243392,297.63101272)
\curveto(157.25525323,297.57213511)(156.12555741,297.48381869)(154.82334647,297.36606348)
\curveto(153.53226554,297.24830826)(152.5083048,297.0775632)(151.75146425,296.85382829)
\curveto(150.8499336,296.58299129)(150.12091807,296.15907251)(149.56441767,295.58207195)
\curveto(149.00791727,295.01684691)(148.72966707,294.23377472)(148.72966707,293.23285538)
\curveto(148.72966707,292.1024053)(149.0524373,291.24867998)(149.69797777,290.67167942)
\curveto(150.34351824,290.10645438)(151.32852395,289.82384186)(152.6529949,289.82384186)
\curveto(153.7548657,289.82384186)(154.76213143,290.04757677)(155.67479209,290.49504659)
\curveto(156.58745275,290.95429194)(157.43333336,291.50185369)(158.21243392,292.13773186)
\closepath
}
}
{
\newrgbcolor{curcolor}{0 0 0}
\pscustom[linestyle=none,fillstyle=solid,fillcolor=curcolor]
{
\newpath
\moveto(194.590865,287.28032918)
\lineto(191.45220274,287.28032918)
\lineto(191.45220274,298.51417684)
\curveto(191.45220274,299.3620144)(191.41324771,300.18041315)(191.33533765,300.9693731)
\curveto(191.2685576,301.75833306)(191.11830249,302.38832346)(190.88457233,302.85934433)
\curveto(190.62858214,303.36569176)(190.26129187,303.74839622)(189.78270153,304.00745769)
\curveto(189.30411118,304.26651917)(188.61405068,304.39604991)(187.71252003,304.39604991)
\curveto(186.8332494,304.39604991)(185.95397876,304.16053947)(185.07470813,303.68951861)
\curveto(184.19543749,303.23027326)(183.31616686,302.64149718)(182.43689622,301.92319036)
\curveto(182.47028624,301.65235336)(182.49811126,301.33441428)(182.52037128,300.9693731)
\curveto(182.5426313,300.61610746)(182.5537613,300.26284181)(182.5537613,299.90957616)
\lineto(182.5537613,287.28032918)
\lineto(179.41509904,287.28032918)
\lineto(179.41509904,298.51417684)
\curveto(179.41509904,299.38556544)(179.37614401,300.20985196)(179.29823395,300.98703639)
\curveto(179.2314539,301.77599634)(179.0811988,302.40598675)(178.84746863,302.87700761)
\curveto(178.59147844,303.38335504)(178.22418818,303.76017174)(177.74559783,304.00745769)
\curveto(177.26700748,304.26651917)(176.57694699,304.39604991)(175.67541633,304.39604991)
\curveto(174.81840571,304.39604991)(173.95583009,304.172315)(173.08768946,303.72484517)
\curveto(172.23067884,303.27737535)(171.37366823,302.70626255)(170.51665761,302.01150677)
\lineto(170.51665761,287.28032918)
\lineto(167.37799534,287.28032918)
\lineto(167.37799534,307.01021571)
\lineto(170.51665761,307.01021571)
\lineto(170.51665761,304.81996869)
\curveto(171.49609831,305.67958177)(172.46997402,306.3507865)(173.43828472,306.83358289)
\curveto(174.41772542,307.31637928)(175.45838118,307.55777747)(176.56025197,307.55777747)
\curveto(177.82907289,307.55777747)(178.90311867,307.27516495)(179.7823893,306.70993991)
\curveto(180.67278995,306.14471487)(181.33502542,305.36164268)(181.76909574,304.36072334)
\curveto(183.03791665,305.49117342)(184.19543749,306.30368442)(185.24165825,306.79825632)
\curveto(186.287879,307.30460376)(187.40644481,307.55777747)(188.59735567,307.55777747)
\curveto(190.64527715,307.55777747)(192.15339324,306.89834826)(193.12170394,305.57948983)
\curveto(194.10114465,304.27240693)(194.590865,302.44131331)(194.590865,300.08620898)
\closepath
}
}
{
\newrgbcolor{curcolor}{1 0.40000001 0}
\pscustom[linestyle=none,fillstyle=solid,fillcolor=curcolor]
{
\newpath
\moveto(52.3575999,178.83888562)
\lineto(198.07187944,178.83888562)
\lineto(198.07187944,124.55317477)
\lineto(52.3575999,124.55317477)
\closepath
}
}
{
\newrgbcolor{curcolor}{0.7019608 0.7019608 0.7019608}
\pscustom[linewidth=0.85039368,linecolor=curcolor]
{
\newpath
\moveto(52.3575999,178.83888562)
\lineto(198.07187944,178.83888562)
\lineto(198.07187944,124.55317477)
\lineto(52.3575999,124.55317477)
\closepath
}
}
{
\newrgbcolor{curcolor}{0 0 0}
\pscustom[linestyle=none,fillstyle=solid,fillcolor=curcolor]
{
\newpath
\moveto(106.15231,151.96947195)
\curveto(106.15231,149.3262512)(105.57288476,146.9304255)(104.41403429,144.78199484)
\curveto(103.2682046,142.63356418)(101.73826156,140.96690282)(99.82420515,139.78201076)
\curveto(98.49608438,138.96170087)(97.01171411,138.36925484)(95.37109433,138.00467267)
\curveto(93.74349535,137.6400905)(91.59506469,137.45779941)(88.92580235,137.45779941)
\lineto(81.58207573,137.45779941)
\lineto(81.58207573,166.53973806)
\lineto(88.8476776,166.53973806)
\curveto(91.68621023,166.53973806)(93.93880722,166.33140539)(95.60546858,165.91474005)
\curveto(97.28515074,165.51109551)(98.70441705,164.95120146)(99.86326753,164.2350579)
\curveto(101.84242789,162.99808267)(103.38539173,161.3509525)(104.49215904,159.29366738)
\curveto(105.59892635,157.23638227)(106.15231,154.79498379)(106.15231,151.96947195)
\closepath
\moveto(102.10935413,152.02806552)
\curveto(102.10935413,154.3067041)(101.71221997,156.2272709)(100.91795167,157.78976592)
\curveto(100.12368336,159.35226095)(98.9387913,160.58272578)(97.36327549,161.48116042)
\curveto(96.2174458,162.13220001)(95.00000176,162.58141733)(93.71094337,162.82881238)
\curveto(92.42188497,163.08922822)(90.87892113,163.21943614)(89.08205185,163.21943614)
\lineto(85.44925092,163.21943614)
\lineto(85.44925092,140.77810134)
\lineto(89.08205185,140.77810134)
\curveto(90.94402509,140.77810134)(92.56511368,140.91481965)(93.94531762,141.18825628)
\curveto(95.33854235,141.46169291)(96.61457995,141.96950379)(97.77343043,142.71168893)
\curveto(99.21873833,143.63616515)(100.29946406,144.85360919)(101.01560761,146.36402105)
\curveto(101.74477195,147.87443291)(102.10935413,149.76244773)(102.10935413,152.02806552)
\closepath
}
}
{
\newrgbcolor{curcolor}{0 0 0}
\pscustom[linestyle=none,fillstyle=solid,fillcolor=curcolor]
{
\newpath
\moveto(129.06239412,137.45779941)
\lineto(125.410062,137.45779941)
\lineto(125.410062,139.78201076)
\curveto(125.0845422,139.5606573)(124.64183528,139.24815829)(124.08194123,138.84451374)
\curveto(123.53506797,138.45388999)(123.0012155,138.14139098)(122.48038383,137.90701673)
\curveto(121.86840661,137.60753852)(121.16528385,137.36014347)(120.37101554,137.16483159)
\curveto(119.57674724,136.95649892)(118.64576062,136.85233259)(117.57805568,136.85233259)
\curveto(115.61191611,136.85233259)(113.94525475,137.50337218)(112.5780716,138.80545137)
\curveto(111.21088846,140.10753056)(110.52729688,141.76768152)(110.52729688,143.78590426)
\curveto(110.52729688,145.43954483)(110.87885826,146.774176)(111.58198103,147.78979776)
\curveto(112.29812458,148.81844032)(113.31374634,149.62572942)(114.62884632,150.21166505)
\curveto(115.9569671,150.79760069)(117.5520141,151.19473484)(119.41398734,151.40306751)
\curveto(121.27596058,151.61140018)(123.27465213,151.76764968)(125.410062,151.87181602)
\lineto(125.410062,152.43822046)
\curveto(125.410062,153.27155114)(125.26032289,153.96165311)(124.96084468,154.50852637)
\curveto(124.67438726,155.05539963)(124.25772192,155.48508576)(123.71084866,155.79758477)
\curveto(123.19001698,156.09706298)(122.56501897,156.29888525)(121.83585463,156.40305159)
\curveto(121.10669028,156.50721792)(120.34497396,156.55930109)(119.55070565,156.55930109)
\curveto(118.58716705,156.55930109)(117.51295172,156.42909317)(116.32805966,156.16867733)
\curveto(115.1431676,155.92128229)(113.91921317,155.55670012)(112.65619636,155.07493082)
\lineto(112.46088448,155.07493082)
\lineto(112.46088448,158.80538769)
\curveto(113.17702803,159.00069957)(114.21218098,159.21554263)(115.56634334,159.44991689)
\curveto(116.92050569,159.68429114)(118.25513686,159.80147827)(119.57023684,159.80147827)
\curveto(121.10669028,159.80147827)(122.44132145,159.67127035)(123.57413034,159.41085451)
\curveto(124.71996003,159.16345947)(125.70954021,158.73377333)(126.54287089,158.12179612)
\curveto(127.36318078,157.52283969)(127.98817879,156.74810257)(128.41786492,155.79758477)
\curveto(128.84755105,154.84706696)(129.06239412,153.66868529)(129.06239412,152.26243977)
\closepath
\moveto(125.410062,142.82887606)
\lineto(125.410062,148.90307547)
\curveto(124.2902739,148.83797151)(122.96866352,148.74031557)(121.44523087,148.61010765)
\curveto(119.93481901,148.47989973)(118.73690616,148.29109825)(117.85149231,148.0437032)
\curveto(116.79680817,147.74422499)(115.9439463,147.27547648)(115.29290671,146.63745768)
\curveto(114.64186712,146.01245967)(114.31634732,145.14657701)(114.31634732,144.0398097)
\curveto(114.31634732,142.78981368)(114.69395028,141.84580627)(115.44915621,141.20778747)
\curveto(116.20436214,140.58278946)(117.35670222,140.27029045)(118.90617646,140.27029045)
\curveto(120.19523485,140.27029045)(121.37361652,140.5176855)(122.44132145,141.01247559)
\curveto(123.50902638,141.52028647)(124.49860657,142.1257533)(125.410062,142.82887606)
\closepath
}
}
{
\newrgbcolor{curcolor}{0 0 0}
\pscustom[linestyle=none,fillstyle=solid,fillcolor=curcolor]
{
\newpath
\moveto(147.46077384,137.65311129)
\curveto(146.77067187,137.4708202)(146.01546594,137.32108109)(145.19515605,137.20389397)
\curveto(144.38786695,137.08670684)(143.665213,137.02811328)(143.0271942,137.02811328)
\curveto(140.80063879,137.02811328)(139.10793585,137.6270697)(137.94908537,138.82498256)
\curveto(136.79023489,140.02289541)(136.21080965,141.94346221)(136.21080965,144.58668296)
\lineto(136.21080965,156.18820852)
\lineto(133.7303488,156.18820852)
\lineto(133.7303488,159.2741362)
\lineto(136.21080965,159.2741362)
\lineto(136.21080965,165.54364749)
\lineto(139.88267296,165.54364749)
\lineto(139.88267296,159.2741362)
\lineto(147.46077384,159.2741362)
\lineto(147.46077384,156.18820852)
\lineto(139.88267296,156.18820852)
\lineto(139.88267296,146.24683393)
\curveto(139.88267296,145.10100424)(139.90871455,144.2025696)(139.96079771,143.55153001)
\curveto(140.01288088,142.9135112)(140.19517197,142.31455478)(140.50767097,141.75466073)
\curveto(140.79412839,141.23382905)(141.18475215,140.84971569)(141.67954224,140.60232065)
\curveto(142.18735313,140.36794639)(142.95557985,140.25075927)(143.9842224,140.25075927)
\curveto(144.58317883,140.25075927)(145.20817684,140.33539441)(145.85921643,140.50466471)
\curveto(146.51025603,140.68695579)(146.97900454,140.8366949)(147.26546196,140.95388203)
\lineto(147.46077384,140.95388203)
\closepath
}
}
{
\newrgbcolor{curcolor}{0 0 0}
\pscustom[linestyle=none,fillstyle=solid,fillcolor=curcolor]
{
\newpath
\moveto(168.84742421,137.45779941)
\lineto(165.19509209,137.45779941)
\lineto(165.19509209,139.78201076)
\curveto(164.8695723,139.5606573)(164.42686537,139.24815829)(163.86697132,138.84451374)
\curveto(163.32009806,138.45388999)(162.7862456,138.14139098)(162.26541392,137.90701673)
\curveto(161.6534367,137.60753852)(160.95031394,137.36014347)(160.15604564,137.16483159)
\curveto(159.36177733,136.95649892)(158.43079071,136.85233259)(157.36308578,136.85233259)
\curveto(155.39694621,136.85233259)(153.73028485,137.50337218)(152.3631017,138.80545137)
\curveto(150.99591855,140.10753056)(150.31232698,141.76768152)(150.31232698,143.78590426)
\curveto(150.31232698,145.43954483)(150.66388836,146.774176)(151.36701112,147.78979776)
\curveto(152.08315467,148.81844032)(153.09877644,149.62572942)(154.41387642,150.21166505)
\curveto(155.74199719,150.79760069)(157.3370442,151.19473484)(159.19901743,151.40306751)
\curveto(161.06099067,151.61140018)(163.05968222,151.76764968)(165.19509209,151.87181602)
\lineto(165.19509209,152.43822046)
\curveto(165.19509209,153.27155114)(165.04535299,153.96165311)(164.74587477,154.50852637)
\curveto(164.45941735,155.05539963)(164.04275201,155.48508576)(163.49587875,155.79758477)
\curveto(162.97504708,156.09706298)(162.35004907,156.29888525)(161.62088472,156.40305159)
\curveto(160.89172038,156.50721792)(160.13000405,156.55930109)(159.33573575,156.55930109)
\curveto(158.37219715,156.55930109)(157.29798182,156.42909317)(156.11308976,156.16867733)
\curveto(154.9281977,155.92128229)(153.70424326,155.55670012)(152.44122645,155.07493082)
\lineto(152.24591457,155.07493082)
\lineto(152.24591457,158.80538769)
\curveto(152.96205812,159.00069957)(153.99721108,159.21554263)(155.35137343,159.44991689)
\curveto(156.70553579,159.68429114)(158.04016696,159.80147827)(159.35526694,159.80147827)
\curveto(160.89172038,159.80147827)(162.22635154,159.67127035)(163.35916044,159.41085451)
\curveto(164.50499012,159.16345947)(165.49457031,158.73377333)(166.32790099,158.12179612)
\curveto(167.14821087,157.52283969)(167.77320888,156.74810257)(168.20289502,155.79758477)
\curveto(168.63258115,154.84706696)(168.84742421,153.66868529)(168.84742421,152.26243977)
\closepath
\moveto(165.19509209,142.82887606)
\lineto(165.19509209,148.90307547)
\curveto(164.07530399,148.83797151)(162.75369362,148.74031557)(161.23026097,148.61010765)
\curveto(159.71984911,148.47989973)(158.52193626,148.29109825)(157.63652241,148.0437032)
\curveto(156.58183827,147.74422499)(155.7289764,147.27547648)(155.0779368,146.63745768)
\curveto(154.42689721,146.01245967)(154.10137741,145.14657701)(154.10137741,144.0398097)
\curveto(154.10137741,142.78981368)(154.47898038,141.84580627)(155.23418631,141.20778747)
\curveto(155.98939224,140.58278946)(157.14173232,140.27029045)(158.69120655,140.27029045)
\curveto(159.98026495,140.27029045)(161.15864661,140.5176855)(162.22635154,141.01247559)
\curveto(163.29405648,141.52028647)(164.28363666,142.1257533)(165.19509209,142.82887606)
\closepath
}
}
{
\newrgbcolor{curcolor}{0 0 0}
\pscustom[linestyle=none,fillstyle=solid,fillcolor=curcolor]
{
\newpath
\moveto(125.21474646,176.00486764)
\lineto(125.21474646,268.59652748)
}
}
{
\newrgbcolor{curcolor}{0 0 0}
\pscustom[linewidth=2.64078996,linecolor=curcolor]
{
\newpath
\moveto(125.21474646,176.00486764)
\lineto(125.21474646,268.59652748)
}
}
{
\newrgbcolor{curcolor}{0 0 0}
\pscustom[linestyle=none,fillstyle=solid,fillcolor=curcolor]
{
\newpath
\moveto(125.21474646,242.18862785)
\lineto(135.77790631,231.62546799)
\lineto(125.21474646,268.59652748)
\lineto(114.6515866,231.62546799)
\closepath
}
}
{
\newrgbcolor{curcolor}{0 0 0}
\pscustom[linewidth=2.81684271,linecolor=curcolor]
{
\newpath
\moveto(125.21474646,242.18862785)
\lineto(135.77790631,231.62546799)
\lineto(125.21474646,268.59652748)
\lineto(114.6515866,231.62546799)
\closepath
}
}
\end{pspicture}
}
    \resizebox{!}{0.25\textheight}{%LaTeX with PSTricks extensions
%%Creator: Inkscape 1.0.2-2 (e86c870879, 2021-01-15)
%%Please note this file requires PSTricks extensions
\psset{xunit=.5pt,yunit=.5pt,runit=.5pt}
\begin{pspicture}(535.30073511,423.12458669)
{
\newrgbcolor{curcolor}{0.80000001 0.80000001 0.80000001}
\pscustom[linestyle=none,fillstyle=solid,fillcolor=curcolor]
{
\newpath
\moveto(410.08601575,262.18161094)
\lineto(410.08601575,169.2475259)
}
}
{
\newrgbcolor{curcolor}{0 0 0}
\pscustom[linewidth=2.64566925,linecolor=curcolor]
{
\newpath
\moveto(410.08601575,262.18161094)
\lineto(410.08601575,169.2475259)
}
}
{
\newrgbcolor{curcolor}{0 0 0}
\pscustom[linewidth=2.64566925,linecolor=curcolor]
{
\newpath
\moveto(125.21474646,262.51035425)
\lineto(125.21474646,169.57630701)
}
}
{
\newrgbcolor{curcolor}{0.80000001 0.80000001 0.80000001}
\pscustom[linestyle=none,fillstyle=solid,fillcolor=curcolor]
{
\newpath
\moveto(160.38875955,376.99323056)
\lineto(374.91198367,376.99323056)
\lineto(374.91198367,319.25593837)
\lineto(160.38875955,319.25593837)
\closepath
}
}
{
\newrgbcolor{curcolor}{0 0 0}
\pscustom[linewidth=0.4430022,linecolor=curcolor]
{
\newpath
\moveto(160.38875955,376.99323056)
\lineto(374.91198367,376.99323056)
\lineto(374.91198367,319.25593837)
\lineto(160.38875955,319.25593837)
\closepath
}
}
{
\newrgbcolor{curcolor}{0.80000001 0.80000001 0.80000001}
\pscustom[linestyle=none,fillstyle=solid,fillcolor=curcolor]
{
\newpath
\moveto(41.37207917,422.79387724)
\lineto(209.05740017,422.79387724)
\lineto(209.05740017,258.13902384)
\lineto(41.37207917,258.13902384)
\closepath
}
}
{
\newrgbcolor{curcolor}{0 0 0}
\pscustom[linewidth=0.66141731,linecolor=curcolor]
{
\newpath
\moveto(41.37207917,422.79387724)
\lineto(209.05740017,422.79387724)
\lineto(209.05740017,258.13902384)
\lineto(41.37207917,258.13902384)
\closepath
}
}
{
\newrgbcolor{curcolor}{0.80000001 0.80000001 0.80000001}
\pscustom[linestyle=none,fillstyle=solid,fillcolor=curcolor]
{
\newpath
\moveto(326.24335747,422.79387724)
\lineto(493.92867847,422.79387724)
\lineto(493.92867847,258.13902384)
\lineto(326.24335747,258.13902384)
\closepath
}
}
{
\newrgbcolor{curcolor}{0 0 0}
\pscustom[linewidth=0.66141731,linecolor=curcolor]
{
\newpath
\moveto(326.24335747,422.79387724)
\lineto(493.92867847,422.79387724)
\lineto(493.92867847,258.13902384)
\lineto(326.24335747,258.13902384)
\closepath
}
}
{
\newrgbcolor{curcolor}{0 0 0}
\pscustom[linestyle=none,fillstyle=solid,fillcolor=curcolor]
{
\newpath
\moveto(111.53315206,328.07395942)
\curveto(110.81700851,327.76146041)(110.16596892,327.46849259)(109.58003328,327.19505596)
\curveto(109.00711844,326.92161933)(108.25191251,326.63516191)(107.3144155,326.3356837)
\curveto(106.52014719,326.08828865)(105.65426453,325.87995598)(104.71676752,325.71068569)
\curveto(103.79229129,325.5283946)(102.77015913,325.43724906)(101.65037103,325.43724906)
\curveto(99.54100275,325.43724906)(97.62043595,325.73021688)(95.88867063,326.31615251)
\curveto(94.1699261,326.91510894)(92.67253503,327.84609556)(91.39649743,329.10911237)
\curveto(90.14650141,330.3460876)(89.16994202,331.91509302)(88.46681926,333.81612863)
\curveto(87.7636965,335.73018504)(87.41213511,337.95023005)(87.41213511,340.47626368)
\curveto(87.41213511,342.87208938)(87.7506757,345.01400965)(88.42775688,346.90202447)
\curveto(89.10483806,348.79003929)(90.08139745,350.3850863)(91.35743505,351.68716548)
\curveto(92.59441028,352.95018229)(94.08529095,353.91372089)(95.83007706,354.57778128)
\curveto(97.58788397,355.24184167)(99.53449235,355.57387186)(101.66990222,355.57387186)
\curveto(103.23239724,355.57387186)(104.78838187,355.38507038)(106.33785611,355.00746741)
\curveto(107.90035113,354.62986445)(109.63211645,353.96580406)(111.53315206,353.01528625)
\lineto(111.53315206,348.42545712)
\lineto(111.24018425,348.42545712)
\curveto(109.63862685,349.76659868)(108.05009024,350.74315807)(106.47457442,351.35513529)
\curveto(104.8990586,351.96711251)(103.21286606,352.27310112)(101.41599678,352.27310112)
\curveto(99.9446473,352.27310112)(98.61652652,352.03221647)(97.43163446,351.55044717)
\curveto(96.25976319,351.08169866)(95.21158945,350.34602392)(94.28711323,349.34342295)
\curveto(93.38867859,348.36686355)(92.68555582,347.12988833)(92.17774494,345.63249726)
\curveto(91.68295485,344.14812699)(91.4355598,342.42938246)(91.4355598,340.47626368)
\curveto(91.4355598,338.43199935)(91.70899643,336.67419245)(92.25586969,335.20284297)
\curveto(92.81576374,333.73149349)(93.5319073,332.53358063)(94.40430035,331.60910441)
\curveto(95.31575578,330.64556581)(96.37695032,329.92942226)(97.58788397,329.46067375)
\curveto(98.8118384,329.00494603)(100.1008968,328.77708218)(101.45505915,328.77708218)
\curveto(103.31703239,328.77708218)(105.0618185,329.09609158)(106.68941749,329.73411038)
\curveto(108.31701647,330.37212918)(109.84044912,331.32915738)(111.25971544,332.60519499)
\lineto(111.53315206,332.60519499)
\closepath
}
}
{
\newrgbcolor{curcolor}{0 0 0}
\pscustom[linestyle=none,fillstyle=solid,fillcolor=curcolor]
{
\newpath
\moveto(136.025261,346.25749527)
\curveto(136.025261,344.96843687)(135.79739714,343.77052402)(135.34166943,342.66375671)
\curveto(134.8989625,341.5700102)(134.27396449,340.61949239)(133.4666754,339.81220329)
\curveto(132.46407442,338.80960232)(131.27918236,338.05439639)(129.91199921,337.54658551)
\curveto(128.54481607,337.05179541)(126.81956114,336.80440037)(124.73623444,336.80440037)
\lineto(120.86905926,336.80440037)
\lineto(120.86905926,325.96459113)
\lineto(117.00188407,325.96459113)
\lineto(117.00188407,355.04652979)
\lineto(124.89248395,355.04652979)
\curveto(126.63727006,355.04652979)(128.11512993,354.89679068)(129.32606358,354.59731247)
\curveto(130.53699722,354.31085505)(131.61121255,353.85512733)(132.54870957,353.23012932)
\curveto(133.65547688,352.48794418)(134.50833875,351.56346796)(135.10729517,350.45670065)
\curveto(135.71927239,349.34993334)(136.025261,347.95019821)(136.025261,346.25749527)
\closepath
\moveto(132.00183631,346.15983933)
\curveto(132.00183631,347.16244031)(131.82605562,348.03483336)(131.47449424,348.7770185)
\curveto(131.12293286,349.51920364)(130.58908039,350.12467046)(129.87293684,350.59341897)
\curveto(129.24793883,350.99706351)(128.53179527,351.28352093)(127.72450618,351.45279123)
\curveto(126.93023787,351.63508232)(125.9211265,351.72622786)(124.69717207,351.72622786)
\lineto(120.86905926,351.72622786)
\lineto(120.86905926,340.10517111)
\lineto(124.13076762,340.10517111)
\curveto(125.69326265,340.10517111)(126.96278985,340.24188942)(127.93934924,340.51532605)
\curveto(128.91590864,340.80178347)(129.71017694,341.25100079)(130.32215416,341.86297801)
\curveto(130.93413138,342.48797602)(131.36381751,343.14552601)(131.61121255,343.83562798)
\curveto(131.87162839,344.52572995)(132.00183631,345.30046707)(132.00183631,346.15983933)
\closepath
}
}
{
\newrgbcolor{curcolor}{0 0 0}
\pscustom[linestyle=none,fillstyle=solid,fillcolor=curcolor]
{
\newpath
\moveto(163.01736066,337.64424144)
\curveto(163.01736066,335.53487316)(162.78298641,333.69243111)(162.3142379,332.11691529)
\curveto(161.85851018,330.55442027)(161.10330426,329.25234108)(160.04862011,328.21067773)
\curveto(159.04601914,327.22109755)(157.87414787,326.4984436)(156.53300631,326.04271588)
\curveto(155.19186474,325.58698817)(153.62936972,325.35912431)(151.84552123,325.35912431)
\curveto(150.02261037,325.35912431)(148.43407376,325.60000896)(147.0799114,326.08177826)
\curveto(145.72574905,326.56354756)(144.58642976,327.27318072)(143.66195354,328.21067773)
\curveto(142.6072694,329.27838266)(141.84555307,330.56744106)(141.37680456,332.07785292)
\curveto(140.92107685,333.58826478)(140.69321299,335.44372762)(140.69321299,337.64424144)
\lineto(140.69321299,355.04652979)
\lineto(144.56038818,355.04652979)
\lineto(144.56038818,337.44892957)
\curveto(144.56038818,335.87341375)(144.66455451,334.62992813)(144.87288718,333.71847269)
\curveto(145.09424064,332.80701726)(145.45882282,331.98019698)(145.9666337,331.23801184)
\curveto(146.53954854,330.39166037)(147.31428566,329.75364157)(148.29084505,329.32395544)
\curveto(149.28042523,328.8942693)(150.46531729,328.67942624)(151.84552123,328.67942624)
\curveto(153.23874596,328.67942624)(154.42363802,328.88775891)(155.40019741,329.30442425)
\curveto(156.3767568,329.73411038)(157.15800432,330.37863958)(157.74393995,331.23801184)
\curveto(158.25175083,331.98019698)(158.60982261,332.82654845)(158.81815528,333.77706626)
\curveto(159.03950874,334.74060486)(159.15018547,335.93200731)(159.15018547,337.35127363)
\lineto(159.15018547,355.04652979)
\lineto(163.01736066,355.04652979)
\closepath
}
}
{
\newrgbcolor{curcolor}{0 0 0}
\pscustom[linestyle=none,fillstyle=solid,fillcolor=curcolor]
{
\newpath
\moveto(396.40441594,323.02316086)
\curveto(395.68827239,322.71066186)(395.0372328,322.41769404)(394.45129716,322.14425741)
\curveto(393.87838232,321.87082078)(393.12317639,321.58436336)(392.18567938,321.28488515)
\curveto(391.39141107,321.0374901)(390.52552841,320.82915743)(389.5880314,320.65988714)
\curveto(388.66355517,320.47759605)(387.64142301,320.38645051)(386.52163491,320.38645051)
\curveto(384.41226663,320.38645051)(382.49169982,320.67941833)(380.75993451,321.26535396)
\curveto(379.04118998,321.86431039)(377.54379891,322.79529701)(376.26776131,324.05831382)
\curveto(375.01776529,325.29528905)(374.0412059,326.86429447)(373.33808314,328.76533008)
\curveto(372.63496037,330.67938649)(372.28339899,332.8994315)(372.28339899,335.42546513)
\curveto(372.28339899,337.82129083)(372.62193958,339.9632111)(373.29902076,341.85122592)
\curveto(373.97610194,343.73924074)(374.95266133,345.33428774)(376.22869893,346.63636693)
\curveto(377.46567416,347.89938374)(378.95655483,348.86292234)(380.70134094,349.52698273)
\curveto(382.45914785,350.19104311)(384.40575623,350.52307331)(386.5411661,350.52307331)
\curveto(388.10366112,350.52307331)(389.65964575,350.33427182)(391.20911999,349.95666886)
\curveto(392.77161501,349.5790659)(394.50338033,348.91500551)(396.40441594,347.9644877)
\lineto(396.40441594,343.37465857)
\lineto(396.11144813,343.37465857)
\curveto(394.50989073,344.71580013)(392.92135412,345.69235952)(391.3458383,346.30433674)
\curveto(389.77032248,346.91631396)(388.08412994,347.22230257)(386.28726066,347.22230257)
\curveto(384.81591117,347.22230257)(383.4877904,346.98141792)(382.30289834,346.49964862)
\curveto(381.13102707,346.03090011)(380.08285333,345.29522537)(379.1583771,344.29262439)
\curveto(378.25994247,343.316065)(377.5568197,342.07908978)(377.04900882,340.58169871)
\curveto(376.55421873,339.09732844)(376.30682368,337.37858391)(376.30682368,335.42546513)
\curveto(376.30682368,333.3812008)(376.58026031,331.6233939)(377.12713357,330.15204442)
\curveto(377.68702762,328.68069493)(378.40317118,327.48278208)(379.27556423,326.55830586)
\curveto(380.18701966,325.59476726)(381.2482142,324.87862371)(382.45914785,324.4098752)
\curveto(383.68310228,323.95414748)(384.97216068,323.72628363)(386.32632303,323.72628363)
\curveto(388.18829627,323.72628363)(389.93308238,324.04529303)(391.56068137,324.68331183)
\curveto(393.18828035,325.32133063)(394.711713,326.27835883)(396.13097931,327.55439644)
\lineto(396.40441594,327.55439644)
\closepath
}
}
{
\newrgbcolor{curcolor}{0 0 0}
\pscustom[linestyle=none,fillstyle=solid,fillcolor=curcolor]
{
\newpath
\moveto(420.89652488,341.20669672)
\curveto(420.89652488,339.91763832)(420.66866102,338.71972547)(420.21293331,337.61295816)
\curveto(419.77022638,336.51921164)(419.14522837,335.56869384)(418.33793928,334.76140474)
\curveto(417.3353383,333.75880377)(416.15044624,333.00359784)(414.78326309,332.49578695)
\curveto(413.41607995,332.00099686)(411.69082502,331.75360182)(409.60749832,331.75360182)
\lineto(405.74032314,331.75360182)
\lineto(405.74032314,320.91379258)
\lineto(401.87314795,320.91379258)
\lineto(401.87314795,349.99573124)
\lineto(409.76374782,349.99573124)
\curveto(411.50853394,349.99573124)(412.98639381,349.84599213)(414.19732746,349.54651392)
\curveto(415.4082611,349.26005649)(416.48247643,348.80432878)(417.41997345,348.17933077)
\curveto(418.52674076,347.43714563)(419.37960263,346.51266941)(419.97855905,345.4059021)
\curveto(420.59053627,344.29913479)(420.89652488,342.89939966)(420.89652488,341.20669672)
\closepath
\moveto(416.87310019,341.10904078)
\curveto(416.87310019,342.11164175)(416.6973195,342.98403481)(416.34575812,343.72621995)
\curveto(415.99419674,344.46840508)(415.46034427,345.07387191)(414.74420072,345.54262041)
\curveto(414.11920271,345.94626496)(413.40305915,346.23272238)(412.59577006,346.40199268)
\curveto(411.80150175,346.58428376)(410.79239038,346.67542931)(409.56843595,346.67542931)
\lineto(405.74032314,346.67542931)
\lineto(405.74032314,335.05437256)
\lineto(409.0020315,335.05437256)
\curveto(410.56452653,335.05437256)(411.83405373,335.19109087)(412.81061312,335.4645275)
\curveto(413.78717251,335.75098492)(414.58144082,336.20020224)(415.19341804,336.81217946)
\curveto(415.80539526,337.43717747)(416.23508139,338.09472746)(416.48247643,338.78482943)
\curveto(416.74289227,339.4749314)(416.87310019,340.24966852)(416.87310019,341.10904078)
\closepath
}
}
{
\newrgbcolor{curcolor}{0 0 0}
\pscustom[linestyle=none,fillstyle=solid,fillcolor=curcolor]
{
\newpath
\moveto(447.88862454,332.59344289)
\curveto(447.88862454,330.48407461)(447.65425029,328.64163256)(447.18550178,327.06611674)
\curveto(446.72977406,325.50362172)(445.97456813,324.20154253)(444.91988399,323.15987918)
\curveto(443.91728302,322.170299)(442.74541175,321.44764505)(441.40427019,320.99191733)
\curveto(440.06312862,320.53618962)(438.5006336,320.30832576)(436.71678511,320.30832576)
\curveto(434.89387425,320.30832576)(433.30533764,320.54921041)(431.95117528,321.03097971)
\curveto(430.59701293,321.51274901)(429.45769364,322.22238216)(428.53321742,323.15987918)
\curveto(427.47853327,324.22758411)(426.71681695,325.51664251)(426.24806844,327.02705437)
\curveto(425.79234073,328.53746622)(425.56447687,330.39292907)(425.56447687,332.59344289)
\lineto(425.56447687,349.99573124)
\lineto(429.43165206,349.99573124)
\lineto(429.43165206,332.39813102)
\curveto(429.43165206,330.8226152)(429.53581839,329.57912957)(429.74415106,328.66767414)
\curveto(429.96550452,327.75621871)(430.3300867,326.92939843)(430.83789758,326.18721329)
\curveto(431.41081242,325.34086182)(432.18554954,324.70284302)(433.16210893,324.27315688)
\curveto(434.15168911,323.84347075)(435.33658117,323.62862769)(436.71678511,323.62862769)
\curveto(438.11000984,323.62862769)(439.2949019,323.83696036)(440.27146129,324.2536257)
\curveto(441.24802068,324.68331183)(442.0292682,325.32784103)(442.61520383,326.18721329)
\curveto(443.12301471,326.92939843)(443.48108649,327.7757499)(443.68941916,328.72626771)
\curveto(443.91077262,329.68980631)(444.02144935,330.88120876)(444.02144935,332.30047508)
\lineto(444.02144935,349.99573124)
\lineto(447.88862454,349.99573124)
\closepath
}
}
{
\newrgbcolor{curcolor}{0 0 0}
\pscustom[linestyle=none,fillstyle=solid,fillcolor=curcolor]
{
\newpath
\moveto(224.90445649,338.63244012)
\lineto(220.60794414,338.63244012)
\lineto(220.60794414,340.61160632)
\lineto(222.03280793,340.61160632)
\lineto(222.03280793,356.0412902)
\lineto(220.60794414,356.0412902)
\lineto(220.60794414,358.0204564)
\lineto(224.90445649,358.0204564)
\lineto(224.90445649,356.0412902)
\lineto(223.4795927,356.0412902)
\lineto(223.4795927,340.61160632)
\lineto(224.90445649,340.61160632)
\closepath
}
}
{
\newrgbcolor{curcolor}{0 0 0}
\pscustom[linestyle=none,fillstyle=solid,fillcolor=curcolor]
{
\newpath
\moveto(234.08204078,338.63244012)
\lineto(232.70832595,338.63244012)
\lineto(232.70832595,346.91368818)
\curveto(232.70832595,347.5820908)(232.68640497,348.20709065)(232.642563,348.78868774)
\curveto(232.59872104,349.37896538)(232.51834411,349.83903471)(232.40143221,350.16889575)
\curveto(232.27964898,350.53347899)(232.10428113,350.80257615)(231.87532865,350.97618722)
\curveto(231.64637618,351.15847885)(231.3492251,351.24962466)(230.98387541,351.24962466)
\curveto(230.60878306,351.24962466)(230.21664106,351.08469414)(229.80744941,350.75483311)
\curveto(229.39825776,350.42497207)(229.00611576,350.00396523)(228.63102341,349.49181257)
\lineto(228.63102341,338.63244012)
\lineto(227.25730858,338.63244012)
\lineto(227.25730858,353.17670754)
\lineto(228.63102341,353.17670754)
\lineto(228.63102341,351.56212458)
\curveto(229.05970038,352.19580499)(229.50299134,352.69059654)(229.96089628,353.04649924)
\curveto(230.41880123,353.40240193)(230.88888449,353.58035328)(231.37114608,353.58035328)
\curveto(232.25285667,353.58035328)(232.9251001,353.10726311)(233.38787637,352.16108278)
\curveto(233.85065264,351.21490244)(234.08204078,349.85205554)(234.08204078,348.07254207)
\closepath
}
}
{
\newrgbcolor{curcolor}{0 0 0}
\pscustom[linestyle=none,fillstyle=solid,fillcolor=curcolor]
{
\newpath
\moveto(240.9652289,338.76264842)
\curveto(240.70704845,338.64112068)(240.42451136,338.54129431)(240.11761762,338.46316933)
\curveto(239.81559521,338.38504435)(239.54523644,338.34598186)(239.30654131,338.34598186)
\curveto(238.47354401,338.34598186)(237.84027122,338.74528732)(237.40672292,339.54389824)
\curveto(236.97317462,340.34250916)(236.75640047,341.62289081)(236.75640047,343.38504317)
\lineto(236.75640047,351.11941636)
\lineto(235.82841226,351.11941636)
\lineto(235.82841226,353.17670754)
\lineto(236.75640047,353.17670754)
\lineto(236.75640047,357.35639406)
\lineto(238.13011531,357.35639406)
\lineto(238.13011531,353.17670754)
\lineto(240.9652289,353.17670754)
\lineto(240.9652289,351.11941636)
\lineto(238.13011531,351.11941636)
\lineto(238.13011531,344.49181375)
\curveto(238.13011531,343.72792504)(238.13985797,343.12896684)(238.15934328,342.69493917)
\curveto(238.1788286,342.26959205)(238.24702721,341.87028658)(238.36393911,341.49702278)
\curveto(238.47110835,341.14980064)(238.61724823,340.89372431)(238.80235874,340.7287938)
\curveto(238.99234057,340.57254383)(239.279749,340.49441885)(239.664584,340.49441885)
\curveto(239.88866515,340.49441885)(240.12248895,340.55084245)(240.36605541,340.66368964)
\curveto(240.60962187,340.78521739)(240.78498972,340.88504376)(240.89215896,340.96316874)
\lineto(240.9652289,340.96316874)
\closepath
}
}
{
\newrgbcolor{curcolor}{0 0 0}
\pscustom[linestyle=none,fillstyle=solid,fillcolor=curcolor]
{
\newpath
\moveto(249.45595642,345.65066764)
\lineto(243.44230053,345.65066764)
\curveto(243.44230053,344.75657063)(243.51780613,343.97532081)(243.66881733,343.30691819)
\curveto(243.81982854,342.64719612)(244.02686003,342.10466153)(244.28991181,341.67931441)
\curveto(244.54322092,341.26264784)(244.84280767,340.95014791)(245.18867204,340.74181463)
\curveto(245.53940774,340.53348134)(245.92424275,340.4293147)(246.34317706,340.4293147)
\curveto(246.89850859,340.4293147)(247.45627578,340.62462715)(248.01647864,341.01525206)
\curveto(248.58155283,341.41455752)(248.98343748,341.80518243)(249.22213261,342.18712679)
\lineto(249.29520255,342.18712679)
\lineto(249.29520255,339.51785658)
\curveto(248.83242628,339.17063444)(248.35990735,338.8798359)(247.87764576,338.64546095)
\curveto(247.39538417,338.41108601)(246.88876593,338.29389853)(246.35779105,338.29389853)
\curveto(245.00356153,338.29389853)(243.9464831,338.94494005)(243.18655574,340.24702308)
\curveto(242.42662839,341.55778666)(242.04666471,343.41542511)(242.04666471,345.81993843)
\curveto(242.04666471,348.1984101)(242.40957874,350.08643049)(243.13540679,351.4839996)
\curveto(243.86610617,352.88156872)(244.82575802,353.58035328)(246.01436234,353.58035328)
\curveto(247.11528274,353.58035328)(247.96289402,353.00743675)(248.55719618,351.86160368)
\curveto(249.15636967,350.71577062)(249.45595642,349.08816683)(249.45595642,346.97879233)
\closepath
\moveto(248.11877655,347.5256672)
\curveto(248.11390522,348.81038912)(247.93123038,349.8043125)(247.57075202,350.50743733)
\curveto(247.21514499,351.21056217)(246.67199178,351.56212458)(245.9412924,351.56212458)
\curveto(245.20572169,351.56212458)(244.61872653,351.17583995)(244.1803069,350.40327069)
\curveto(243.7467586,349.63070143)(243.50075648,348.67150026)(243.44230053,347.5256672)
\closepath
}
}
{
\newrgbcolor{curcolor}{0 0 0}
\pscustom[linestyle=none,fillstyle=solid,fillcolor=curcolor]
{
\newpath
\moveto(256.63142372,350.50743733)
\lineto(256.55835378,350.50743733)
\curveto(256.35375795,350.59424287)(256.15403346,350.65500674)(255.95918029,350.68972896)
\curveto(255.76919845,350.73313172)(255.54268164,350.75483311)(255.27962987,350.75483311)
\curveto(254.85582423,350.75483311)(254.44663258,350.58556231)(254.05205491,350.24702073)
\curveto(253.65747725,349.91715969)(253.27751357,349.48747229)(252.91216388,348.95795853)
\lineto(252.91216388,338.63244012)
\lineto(251.53844905,338.63244012)
\lineto(251.53844905,353.17670754)
\lineto(252.91216388,353.17670754)
\lineto(252.91216388,351.02827054)
\curveto(253.45775275,351.80952036)(253.93757867,352.36073551)(254.35164166,352.68191599)
\curveto(254.77057597,353.01177702)(255.19681727,353.17670754)(255.63036557,353.17670754)
\curveto(255.8690607,353.17670754)(256.04199289,353.16368671)(256.14916213,353.13764505)
\curveto(256.25633137,353.12028394)(256.41708523,353.08122145)(256.63142372,353.02045758)
\closepath
}
}
{
\newrgbcolor{curcolor}{0 0 0}
\pscustom[linestyle=none,fillstyle=solid,fillcolor=curcolor]
{
\newpath
\moveto(263.9603386,339.54389824)
\curveto(263.50243365,339.15327333)(263.06644969,338.84945396)(262.65238671,338.63244012)
\curveto(262.24319506,338.41542628)(261.80721109,338.30691937)(261.34443482,338.30691937)
\curveto(260.75500399,338.30691937)(260.21428645,338.45882905)(259.7222822,338.76264842)
\curveto(259.23027795,339.07514835)(258.80890797,339.54389824)(258.45817227,340.16889809)
\curveto(258.10256524,340.79389795)(257.82733514,341.58382832)(257.63248197,342.5386892)
\curveto(257.43762881,343.49355009)(257.34020222,344.60900122)(257.34020222,345.88504259)
\curveto(257.34020222,348.26351425)(257.70555191,350.12983325)(258.43625129,351.4839996)
\curveto(259.171822,352.83816595)(260.14121651,353.51524913)(261.34443482,353.51524913)
\curveto(261.81208242,353.51524913)(262.26998737,353.39806165)(262.71814965,353.16368671)
\curveto(263.17118327,352.92931176)(263.58524625,352.6428535)(263.9603386,352.30431191)
\lineto(263.9603386,349.58295838)
\lineto(263.88726866,349.58295838)
\curveto(263.46833435,350.16455547)(263.03478605,350.61160397)(262.58662376,350.9241039)
\curveto(262.14333281,351.23660383)(261.70978451,351.39285379)(261.28597887,351.39285379)
\curveto(260.5065662,351.39285379)(259.89034306,350.9241039)(259.43730944,349.98660412)
\curveto(258.98914715,349.0577849)(258.76506601,347.69059772)(258.76506601,345.88504259)
\curveto(258.76506601,344.13157077)(258.98427583,342.7817447)(259.42269545,341.83556437)
\curveto(259.86598641,340.89806459)(260.48708088,340.4293147)(261.28597887,340.4293147)
\curveto(261.56364463,340.4293147)(261.84618173,340.49441885)(262.13359015,340.62462715)
\curveto(262.42099857,340.75483546)(262.67917902,340.92410625)(262.90813149,341.13243953)
\curveto(263.10785599,341.31473116)(263.29540216,341.50570334)(263.47077001,341.70535607)
\curveto(263.64613786,341.91368935)(263.78497075,342.0916407)(263.88726866,342.23921011)
\lineto(263.9603386,342.23921011)
\closepath
}
}
{
\newrgbcolor{curcolor}{0 0 0}
\pscustom[linestyle=none,fillstyle=solid,fillcolor=curcolor]
{
\newpath
\moveto(272.6775822,345.89806342)
\curveto(272.6775822,343.52827231)(272.33658915,341.65761302)(271.65460307,340.28608557)
\curveto(270.97261698,338.91455811)(270.05924276,338.22879438)(268.9144804,338.22879438)
\curveto(267.75997538,338.22879438)(266.84172982,338.91455811)(266.15974374,340.28608557)
\curveto(265.48262898,341.65761302)(265.1440716,343.52827231)(265.1440716,345.89806342)
\curveto(265.1440716,348.26785453)(265.48262898,350.13851381)(266.15974374,351.51004126)
\curveto(266.84172982,352.89024927)(267.75997538,353.58035328)(268.9144804,353.58035328)
\curveto(270.05924276,353.58035328)(270.97261698,352.89024927)(271.65460307,351.51004126)
\curveto(272.33658915,350.13851381)(272.6775822,348.26785453)(272.6775822,345.89806342)
\closepath
\moveto(271.2600254,345.89806342)
\curveto(271.2600254,347.78174353)(271.05299391,349.17931264)(270.63893093,350.09077076)
\curveto(270.22486795,351.01090944)(269.6500511,351.47097877)(268.9144804,351.47097877)
\curveto(268.16916703,351.47097877)(267.58947885,351.01090944)(267.17541587,350.09077076)
\curveto(266.76622422,349.17931264)(266.5616284,347.78174353)(266.5616284,345.89806342)
\curveto(266.5616284,344.07514718)(266.76865989,342.69059889)(267.18272287,341.74441856)
\curveto(267.59678585,340.80691878)(268.17403836,340.33816889)(268.9144804,340.33816889)
\curveto(269.64517977,340.33816889)(270.21756095,340.8025785)(270.63162394,341.73139773)
\curveto(271.05055825,342.66889751)(271.2600254,344.05778607)(271.2600254,345.89806342)
\closepath
}
}
{
\newrgbcolor{curcolor}{0 0 0}
\pscustom[linestyle=none,fillstyle=solid,fillcolor=curcolor]
{
\newpath
\moveto(281.62864807,338.63244012)
\lineto(280.25493324,338.63244012)
\lineto(280.25493324,346.91368818)
\curveto(280.25493324,347.5820908)(280.23301226,348.20709065)(280.18917029,348.78868774)
\curveto(280.14532833,349.37896538)(280.0649514,349.83903471)(279.9480395,350.16889575)
\curveto(279.82625627,350.53347899)(279.65088842,350.80257615)(279.42193595,350.97618722)
\curveto(279.19298347,351.15847885)(278.89583239,351.24962466)(278.5304827,351.24962466)
\curveto(278.15539036,351.24962466)(277.76324836,351.08469414)(277.3540567,350.75483311)
\curveto(276.94486505,350.42497207)(276.55272305,350.00396523)(276.1776307,349.49181257)
\lineto(276.1776307,338.63244012)
\lineto(274.80391587,338.63244012)
\lineto(274.80391587,353.17670754)
\lineto(276.1776307,353.17670754)
\lineto(276.1776307,351.56212458)
\curveto(276.60630767,352.19580499)(277.04959863,352.69059654)(277.50750357,353.04649924)
\curveto(277.96540852,353.40240193)(278.43549178,353.58035328)(278.91775337,353.58035328)
\curveto(279.79946396,353.58035328)(280.47170739,353.10726311)(280.93448366,352.16108278)
\curveto(281.39725993,351.21490244)(281.62864807,349.85205554)(281.62864807,348.07254207)
\closepath
}
}
{
\newrgbcolor{curcolor}{0 0 0}
\pscustom[linestyle=none,fillstyle=solid,fillcolor=curcolor]
{
\newpath
\moveto(291.09851131,338.63244012)
\lineto(289.72479648,338.63244012)
\lineto(289.72479648,346.91368818)
\curveto(289.72479648,347.5820908)(289.7028755,348.20709065)(289.65903354,348.78868774)
\curveto(289.61519158,349.37896538)(289.53481464,349.83903471)(289.41790274,350.16889575)
\curveto(289.29611951,350.53347899)(289.12075166,350.80257615)(288.89179919,350.97618722)
\curveto(288.66284672,351.15847885)(288.36569564,351.24962466)(288.00034595,351.24962466)
\curveto(287.6252536,351.24962466)(287.2331116,351.08469414)(286.82391995,350.75483311)
\curveto(286.4147283,350.42497207)(286.0225863,350.00396523)(285.64749395,349.49181257)
\lineto(285.64749395,338.63244012)
\lineto(284.27377911,338.63244012)
\lineto(284.27377911,353.17670754)
\lineto(285.64749395,353.17670754)
\lineto(285.64749395,351.56212458)
\curveto(286.07617092,352.19580499)(286.51946187,352.69059654)(286.97736682,353.04649924)
\curveto(287.43527176,353.40240193)(287.90535503,353.58035328)(288.38761662,353.58035328)
\curveto(289.2693272,353.58035328)(289.94157063,353.10726311)(290.4043469,352.16108278)
\curveto(290.86712318,351.21490244)(291.09851131,349.85205554)(291.09851131,348.07254207)
\closepath
}
}
{
\newrgbcolor{curcolor}{0 0 0}
\pscustom[linestyle=none,fillstyle=solid,fillcolor=curcolor]
{
\newpath
\moveto(300.57568155,345.65066764)
\lineto(294.56202566,345.65066764)
\curveto(294.56202566,344.75657063)(294.63753127,343.97532081)(294.78854247,343.30691819)
\curveto(294.93955368,342.64719612)(295.14658517,342.10466153)(295.40963694,341.67931441)
\curveto(295.66294606,341.26264784)(295.96253281,340.95014791)(296.30839718,340.74181463)
\curveto(296.65913288,340.53348134)(297.04396789,340.4293147)(297.4629022,340.4293147)
\curveto(298.01823373,340.4293147)(298.57600092,340.62462715)(299.13620378,341.01525206)
\curveto(299.70127796,341.41455752)(300.10316262,341.80518243)(300.34185775,342.18712679)
\lineto(300.41492769,342.18712679)
\lineto(300.41492769,339.51785658)
\curveto(299.95215142,339.17063444)(299.47963248,338.8798359)(298.99737089,338.64546095)
\curveto(298.5151093,338.41108601)(298.00849107,338.29389853)(297.47751619,338.29389853)
\curveto(296.12328667,338.29389853)(295.06620824,338.94494005)(294.30628088,340.24702308)
\curveto(293.54635353,341.55778666)(293.16638985,343.41542511)(293.16638985,345.81993843)
\curveto(293.16638985,348.1984101)(293.52930388,350.08643049)(294.25513192,351.4839996)
\curveto(294.9858313,352.88156872)(295.94548315,353.58035328)(297.13408748,353.58035328)
\curveto(298.23500788,353.58035328)(299.08261916,353.00743675)(299.67692132,351.86160368)
\curveto(300.27609481,350.71577062)(300.57568155,349.08816683)(300.57568155,346.97879233)
\closepath
\moveto(299.23850169,347.5256672)
\curveto(299.23363036,348.81038912)(299.05095552,349.8043125)(298.69047716,350.50743733)
\curveto(298.33487012,351.21056217)(297.79171692,351.56212458)(297.06101754,351.56212458)
\curveto(296.32544683,351.56212458)(295.73845166,351.17583995)(295.30003204,350.40327069)
\curveto(294.86648374,349.63070143)(294.62048161,348.67150026)(294.56202566,347.5256672)
\closepath
}
}
{
\newrgbcolor{curcolor}{0 0 0}
\pscustom[linestyle=none,fillstyle=solid,fillcolor=curcolor]
{
\newpath
\moveto(308.69375375,339.54389824)
\curveto(308.23584881,339.15327333)(307.79986485,338.84945396)(307.38580187,338.63244012)
\curveto(306.97661021,338.41542628)(306.54062625,338.30691937)(306.07784998,338.30691937)
\curveto(305.48841915,338.30691937)(304.94770161,338.45882905)(304.45569736,338.76264842)
\curveto(303.96369311,339.07514835)(303.54232313,339.54389824)(303.19158743,340.16889809)
\curveto(302.8359804,340.79389795)(302.5607503,341.58382832)(302.36589713,342.5386892)
\curveto(302.17104397,343.49355009)(302.07361738,344.60900122)(302.07361738,345.88504259)
\curveto(302.07361738,348.26351425)(302.43896707,350.12983325)(303.16966645,351.4839996)
\curveto(303.90523716,352.83816595)(304.87463167,353.51524913)(306.07784998,353.51524913)
\curveto(306.54549758,353.51524913)(307.00340252,353.39806165)(307.45156481,353.16368671)
\curveto(307.90459843,352.92931176)(308.31866141,352.6428535)(308.69375375,352.30431191)
\lineto(308.69375375,349.58295838)
\lineto(308.62068382,349.58295838)
\curveto(308.20174951,350.16455547)(307.76820121,350.61160397)(307.32003892,350.9241039)
\curveto(306.87674797,351.23660383)(306.44319967,351.39285379)(306.01939403,351.39285379)
\curveto(305.23998136,351.39285379)(304.62375821,350.9241039)(304.1707246,349.98660412)
\curveto(303.72256231,349.0577849)(303.49848117,347.69059772)(303.49848117,345.88504259)
\curveto(303.49848117,344.13157077)(303.71769098,342.7817447)(304.15611061,341.83556437)
\curveto(304.59940157,340.89806459)(305.22049604,340.4293147)(306.01939403,340.4293147)
\curveto(306.29705979,340.4293147)(306.57959688,340.49441885)(306.86700531,340.62462715)
\curveto(307.15441373,340.75483546)(307.41259418,340.92410625)(307.64154665,341.13243953)
\curveto(307.84127115,341.31473116)(308.02881732,341.50570334)(308.20418517,341.70535607)
\curveto(308.37955302,341.91368935)(308.5183859,342.0916407)(308.62068382,342.23921011)
\lineto(308.69375375,342.23921011)
\closepath
}
}
{
\newrgbcolor{curcolor}{0 0 0}
\pscustom[linestyle=none,fillstyle=solid,fillcolor=curcolor]
{
\newpath
\moveto(314.69279837,338.76264842)
\curveto(314.43461792,338.64112068)(314.15208083,338.54129431)(313.84518709,338.46316933)
\curveto(313.54316468,338.38504435)(313.27280591,338.34598186)(313.03411078,338.34598186)
\curveto(312.20111348,338.34598186)(311.56784069,338.74528732)(311.13429239,339.54389824)
\curveto(310.70074409,340.34250916)(310.48396994,341.62289081)(310.48396994,343.38504317)
\lineto(310.48396994,351.11941636)
\lineto(309.55598173,351.11941636)
\lineto(309.55598173,353.17670754)
\lineto(310.48396994,353.17670754)
\lineto(310.48396994,357.35639406)
\lineto(311.85768478,357.35639406)
\lineto(311.85768478,353.17670754)
\lineto(314.69279837,353.17670754)
\lineto(314.69279837,351.11941636)
\lineto(311.85768478,351.11941636)
\lineto(311.85768478,344.49181375)
\curveto(311.85768478,343.72792504)(311.86742743,343.12896684)(311.88691275,342.69493917)
\curveto(311.90639807,342.26959205)(311.97459668,341.87028658)(312.09150858,341.49702278)
\curveto(312.19867782,341.14980064)(312.34481769,340.89372431)(312.5299282,340.7287938)
\curveto(312.71991004,340.57254383)(313.00731847,340.49441885)(313.39215347,340.49441885)
\curveto(313.61623461,340.49441885)(313.85005842,340.55084245)(314.09362488,340.66368964)
\curveto(314.33719133,340.78521739)(314.51255919,340.88504376)(314.61972843,340.96316874)
\lineto(314.69279837,340.96316874)
\closepath
}
}
{
\newrgbcolor{curcolor}{0.80000001 0.80000001 0.80000001}
\pscustom[linestyle=none,fillstyle=solid,fillcolor=curcolor]
{
\newpath
\moveto(0.42503943,183.1639233)
\lineto(250.00445434,183.1639233)
\lineto(250.00445434,0.42505477)
\lineto(0.42503943,0.42505477)
\closepath
}
}
{
\newrgbcolor{curcolor}{0 0 0}
\pscustom[linewidth=0.85007621,linecolor=curcolor]
{
\newpath
\moveto(0.42503943,183.1639233)
\lineto(250.00445434,183.1639233)
\lineto(250.00445434,0.42505477)
\lineto(0.42503943,0.42505477)
\closepath
}
}
{
\newrgbcolor{curcolor}{0 0 0}
\pscustom[linestyle=none,fillstyle=solid,fillcolor=curcolor]
{
\newpath
\moveto(76.46007953,81.53327777)
\lineto(72.85401896,81.53327777)
\lineto(72.85401896,107.59987972)
\lineto(65.31407413,89.86321292)
\lineto(63.16500773,89.86321292)
\lineto(55.67970019,107.59987972)
\lineto(55.67970019,81.53327777)
\lineto(52.31040117,81.53327777)
\lineto(52.31040117,111.78516421)
\lineto(57.22775649,111.78516421)
\lineto(64.45809006,94.94244167)
\lineto(71.45166207,111.78516421)
\lineto(76.46007953,111.78516421)
\closepath
}
}
{
\newrgbcolor{curcolor}{0 0 0}
\pscustom[linestyle=none,fillstyle=solid,fillcolor=curcolor]
{
\newpath
\moveto(100.50048297,92.48409496)
\lineto(85.51165545,92.48409496)
\curveto(85.51165545,91.08900012)(85.69985053,89.86998522)(86.07624069,88.82705025)
\curveto(86.45263085,87.79765989)(86.96864962,86.95112177)(87.624297,86.28743588)
\curveto(88.25566114,85.6372946)(89.00237065,85.14968864)(89.86442553,84.824618)
\curveto(90.73862203,84.49954736)(91.69780986,84.33701204)(92.74198902,84.33701204)
\curveto(94.12613348,84.33701204)(95.51634875,84.64176576)(96.91263483,85.25127321)
\curveto(98.32106252,85.87432527)(99.32274601,86.48383272)(99.9176853,87.07979556)
\lineto(100.09980957,87.07979556)
\lineto(100.09980957,82.91482799)
\curveto(98.94635585,82.37304359)(97.7686189,81.91929915)(96.56659871,81.55359468)
\curveto(95.36457852,81.18789021)(94.10185024,81.00503798)(92.77841387,81.00503798)
\curveto(89.40304404,81.00503798)(86.76831292,82.02088373)(84.8742205,84.05257523)
\curveto(82.98012808,86.09781134)(82.03308187,88.99635788)(82.03308187,92.74821485)
\curveto(82.03308187,96.45943799)(82.93763242,99.40539067)(84.74673351,101.58607288)
\curveto(86.56797622,103.76675509)(88.95987498,104.85709619)(91.9224298,104.85709619)
\curveto(94.66643548,104.85709619)(96.77907703,103.96315193)(98.26035443,102.17526341)
\curveto(99.75377346,100.38737489)(100.50048297,97.84776052)(100.50048297,94.55642029)
\closepath
\moveto(97.1676088,95.40973072)
\curveto(97.15546719,97.414333)(96.70015651,98.96519084)(95.80167677,100.06230425)
\curveto(94.91533865,101.15941766)(93.56154824,101.70797437)(91.74030552,101.70797437)
\curveto(89.90692119,101.70797437)(88.44385622,101.10523922)(87.35111059,99.89976893)
\curveto(86.27050658,98.69429864)(85.65735487,97.19761924)(85.51165545,95.40973072)
\closepath
}
}
{
\newrgbcolor{curcolor}{0 0 0}
\pscustom[linestyle=none,fillstyle=solid,fillcolor=curcolor]
{
\newpath
\moveto(135.37728096,81.53327777)
\lineto(131.95334466,81.53327777)
\lineto(131.95334466,94.45483571)
\curveto(131.95334466,95.43004763)(131.910849,96.37139803)(131.82585767,97.2788869)
\curveto(131.75300797,98.18637577)(131.58909612,98.9110124)(131.33412214,99.4527968)
\curveto(131.05486493,100.03521503)(130.65419153,100.47541486)(130.13210195,100.77339628)
\curveto(129.61001237,101.0713777)(128.85723205,101.22036841)(127.87376099,101.22036841)
\curveto(126.91457316,101.22036841)(125.95538533,100.94947621)(124.9961975,100.40769181)
\curveto(124.03700968,99.87945202)(123.07782185,99.20222152)(122.11863402,98.37600031)
\curveto(122.15505887,98.06447428)(122.18541292,97.69876981)(122.20969616,97.2788869)
\curveto(122.23397939,96.8725486)(122.24612101,96.4662103)(122.24612101,96.059872)
\lineto(122.24612101,81.53327777)
\lineto(118.82218471,81.53327777)
\lineto(118.82218471,94.45483571)
\curveto(118.82218471,95.45713685)(118.77968905,96.40525955)(118.69469772,97.29920381)
\curveto(118.62184801,98.20669268)(118.45793617,98.93132932)(118.20296219,99.47311372)
\curveto(117.92370497,100.05553195)(117.52303158,100.48895947)(117.000942,100.77339628)
\curveto(116.47885242,101.0713777)(115.7260721,101.22036841)(114.74260104,101.22036841)
\curveto(113.80769645,101.22036841)(112.86672105,100.96302082)(111.91967484,100.44832564)
\curveto(110.98477024,99.93363046)(110.04986565,99.27671687)(109.11496106,98.47758488)
\lineto(109.11496106,81.53327777)
\lineto(105.69102476,81.53327777)
\lineto(105.69102476,104.22727183)
\lineto(109.11496106,104.22727183)
\lineto(109.11496106,101.70797437)
\curveto(110.18342345,102.6967309)(111.24581503,103.46877367)(112.30213581,104.02410268)
\curveto(113.3705982,104.57943169)(114.50583949,104.85709619)(115.70785968,104.85709619)
\curveto(117.09200414,104.85709619)(118.26367028,104.53202555)(119.22285811,103.88188427)
\curveto(120.19418756,103.23174299)(120.91661383,102.33102643)(121.39013694,101.17973458)
\curveto(122.7742814,102.48001714)(124.03700968,103.41459523)(125.17832178,103.98346885)
\curveto(126.31963388,104.56588708)(127.53986649,104.85709619)(128.83901963,104.85709619)
\curveto(131.07307735,104.85709619)(132.7182666,104.09859803)(133.77458737,102.58160171)
\curveto(134.84304977,101.07815)(135.37728096,98.97196315)(135.37728096,96.26304115)
\closepath
}
}
{
\newrgbcolor{curcolor}{0 0 0}
\pscustom[linestyle=none,fillstyle=solid,fillcolor=curcolor]
{
\newpath
\moveto(159.30840969,92.87011634)
\curveto(159.30840969,89.17243781)(158.45849642,86.25357435)(156.75866989,84.11352597)
\curveto(155.05884336,81.97347759)(152.78228997,80.9034534)(149.92900972,80.9034534)
\curveto(147.05144624,80.9034534)(144.76275123,81.97347759)(143.0629247,84.11352597)
\curveto(141.37523978,86.25357435)(140.53139733,89.17243781)(140.53139733,92.87011634)
\curveto(140.53139733,96.56779487)(141.37523978,99.48665833)(143.0629247,101.62670671)
\curveto(144.76275123,103.7802997)(147.05144624,104.85709619)(149.92900972,104.85709619)
\curveto(152.78228997,104.85709619)(155.05884336,103.7802997)(156.75866989,101.62670671)
\curveto(158.45849642,99.48665833)(159.30840969,96.56779487)(159.30840969,92.87011634)
\closepath
\moveto(155.77519882,92.87011634)
\curveto(155.77519882,95.80929671)(155.25918006,97.98997892)(154.22714252,99.41216297)
\curveto(153.19510498,100.84789163)(151.76239405,101.56575596)(149.92900972,101.56575596)
\curveto(148.07134215,101.56575596)(146.6264896,100.84789163)(145.59445207,99.41216297)
\curveto(144.57455615,97.98997892)(144.06460819,95.80929671)(144.06460819,92.87011634)
\curveto(144.06460819,90.02574824)(144.58062696,87.86538294)(145.61266449,86.38902045)
\curveto(146.64470203,84.92620257)(148.08348377,84.19479363)(149.92900972,84.19479363)
\curveto(151.75025243,84.19479363)(153.17689256,84.91943027)(154.20893009,86.36870354)
\curveto(155.25310925,87.83152142)(155.77519882,89.99865902)(155.77519882,92.87011634)
\closepath
}
}
{
\newrgbcolor{curcolor}{0 0 0}
\pscustom[linestyle=none,fillstyle=solid,fillcolor=curcolor]
{
\newpath
\moveto(177.30228509,100.06230425)
\lineto(177.12016082,100.06230425)
\curveto(176.61021286,100.19775035)(176.11240651,100.29256262)(175.62674179,100.34674106)
\curveto(175.15321869,100.41446411)(174.58863345,100.44832564)(173.93298607,100.44832564)
\curveto(172.8766653,100.44832564)(171.85676938,100.18420574)(170.87329831,99.65596595)
\curveto(169.88982725,99.14127077)(168.94278104,98.47081258)(168.03215968,97.64459137)
\lineto(168.03215968,81.53327777)
\lineto(164.60822339,81.53327777)
\lineto(164.60822339,104.22727183)
\lineto(168.03215968,104.22727183)
\lineto(168.03215968,100.87498085)
\curveto(169.39202091,102.09399575)(170.58797029,102.95407849)(171.62000783,103.45522906)
\curveto(172.66418698,103.96992424)(173.72657856,104.22727183)(174.80718257,104.22727183)
\curveto(175.40212186,104.22727183)(175.8331493,104.20695491)(176.1002649,104.16632108)
\curveto(176.36738049,104.13923186)(176.76805389,104.07828112)(177.30228509,103.98346885)
\closepath
}
}
{
\newrgbcolor{curcolor}{0 0 0}
\pscustom[linestyle=none,fillstyle=solid,fillcolor=curcolor]
{
\newpath
\moveto(198.11909755,104.22727183)
\lineto(186.24459507,73.16270879)
\lineto(182.58389722,73.16270879)
\lineto(186.37208206,82.63039118)
\lineto(178.267552,104.22727183)
\lineto(181.98288713,104.22727183)
\lineto(188.22974963,87.4048662)
\lineto(194.53124941,104.22727183)
\closepath
}
}
{
\newrgbcolor{curcolor}{0.80000001 0.80000001 0.80000001}
\pscustom[linestyle=none,fillstyle=solid,fillcolor=curcolor]
{
\newpath
\moveto(285.29630331,183.1639233)
\lineto(534.87571821,183.1639233)
\lineto(534.87571821,0.42505477)
\lineto(285.29630331,0.42505477)
\closepath
}
}
{
\newrgbcolor{curcolor}{0 0 0}
\pscustom[linewidth=0.85007621,linecolor=curcolor]
{
\newpath
\moveto(285.29630331,183.1639233)
\lineto(534.87571821,183.1639233)
\lineto(534.87571821,0.42505477)
\lineto(285.29630331,0.42505477)
\closepath
}
}
{
\newrgbcolor{curcolor}{0 0 0}
\pscustom[linestyle=none,fillstyle=solid,fillcolor=curcolor]
{
\newpath
\moveto(361.33132006,80.85380488)
\lineto(357.72525949,80.85380488)
\lineto(357.72525949,106.92040683)
\lineto(350.18531466,89.18374003)
\lineto(348.03624826,89.18374003)
\lineto(340.55094072,106.92040683)
\lineto(340.55094072,80.85380488)
\lineto(337.1816417,80.85380488)
\lineto(337.1816417,111.10569132)
\lineto(342.09899702,111.10569132)
\lineto(349.32933059,94.26296878)
\lineto(356.3229026,111.10569132)
\lineto(361.33132006,111.10569132)
\closepath
}
}
{
\newrgbcolor{curcolor}{0 0 0}
\pscustom[linestyle=none,fillstyle=solid,fillcolor=curcolor]
{
\newpath
\moveto(385.3717235,91.80462207)
\lineto(370.38289598,91.80462207)
\curveto(370.38289598,90.40952724)(370.57109106,89.19051234)(370.94748122,88.14757737)
\curveto(371.32387138,87.11818701)(371.83989015,86.27164888)(372.49553753,85.60796299)
\curveto(373.12690167,84.95782171)(373.87361118,84.47021575)(374.73566606,84.14514511)
\curveto(375.60986256,83.82007447)(376.56905039,83.65753915)(377.61322955,83.65753915)
\curveto(378.99737401,83.65753915)(380.38758928,83.96229288)(381.78387536,84.57180033)
\curveto(383.19230305,85.19485239)(384.19398655,85.80435984)(384.78892583,86.40032268)
\lineto(384.9710501,86.40032268)
\lineto(384.9710501,82.2353551)
\curveto(383.81759638,81.6935707)(382.63985943,81.23982627)(381.43783924,80.8741218)
\curveto(380.23581905,80.50841733)(378.97309077,80.32556509)(377.6496544,80.32556509)
\curveto(374.27428458,80.32556509)(371.63955345,81.34141084)(369.74546103,83.37310234)
\curveto(367.85136861,85.41833845)(366.9043224,88.31688499)(366.9043224,92.06874196)
\curveto(366.9043224,95.7799651)(367.80887295,98.72591778)(369.61797404,100.90659999)
\curveto(371.43921676,103.0872822)(373.83111552,104.17762331)(376.79367033,104.17762331)
\curveto(379.53767601,104.17762331)(381.65031756,103.28367905)(383.13159496,101.49579053)
\curveto(384.62501399,99.70790201)(385.3717235,97.16828763)(385.3717235,93.8769474)
\closepath
\moveto(382.03884934,94.73025783)
\curveto(382.02670772,96.73486011)(381.57139704,98.28571796)(380.6729173,99.38283137)
\curveto(379.78657918,100.47994478)(378.43278877,101.02850148)(376.61154606,101.02850148)
\curveto(374.77816173,101.02850148)(373.31509675,100.42576634)(372.22235112,99.22029605)
\curveto(371.14174711,98.01482576)(370.5285954,96.51814635)(370.38289598,94.73025783)
\closepath
}
}
{
\newrgbcolor{curcolor}{0 0 0}
\pscustom[linestyle=none,fillstyle=solid,fillcolor=curcolor]
{
\newpath
\moveto(420.24852149,80.85380488)
\lineto(416.82458519,80.85380488)
\lineto(416.82458519,93.77536282)
\curveto(416.82458519,94.75057474)(416.78208953,95.69192514)(416.69709821,96.59941401)
\curveto(416.6242485,97.50690288)(416.46033665,98.23153952)(416.20536267,98.77332392)
\curveto(415.92610546,99.35574215)(415.52543206,99.79594197)(415.00334248,100.09392339)
\curveto(414.48125291,100.39190481)(413.72847259,100.54089552)(412.74500152,100.54089552)
\curveto(411.78581369,100.54089552)(410.82662586,100.27000332)(409.86743804,99.72821892)
\curveto(408.90825021,99.19997913)(407.94906238,98.52274863)(406.98987455,97.69652742)
\curveto(407.02629941,97.38500139)(407.05665345,97.01929692)(407.08093669,96.59941401)
\curveto(407.10521992,96.19307571)(407.11736154,95.78673741)(407.11736154,95.38039911)
\lineto(407.11736154,80.85380488)
\lineto(403.69342524,80.85380488)
\lineto(403.69342524,93.77536282)
\curveto(403.69342524,94.77766396)(403.65092958,95.72578666)(403.56593825,96.61973092)
\curveto(403.49308855,97.5272198)(403.3291767,98.25185643)(403.07420272,98.79364083)
\curveto(402.79494551,99.37605906)(402.39427211,99.80948658)(401.87218253,100.09392339)
\curveto(401.35009296,100.39190481)(400.59731263,100.54089552)(399.61384157,100.54089552)
\curveto(398.67893698,100.54089552)(397.73796158,100.28354793)(396.79091537,99.76885275)
\curveto(395.85601078,99.25415757)(394.92110618,98.59724399)(393.98620159,97.798112)
\lineto(393.98620159,80.85380488)
\lineto(390.56226529,80.85380488)
\lineto(390.56226529,103.54779894)
\lineto(393.98620159,103.54779894)
\lineto(393.98620159,101.02850148)
\curveto(395.05466398,102.01725801)(396.11705556,102.78930078)(397.17337634,103.34462979)
\curveto(398.24183873,103.8999588)(399.37708002,104.17762331)(400.57910021,104.17762331)
\curveto(401.96324467,104.17762331)(403.13491081,103.85255267)(404.09409864,103.20241139)
\curveto(405.06542809,102.55227011)(405.78785436,101.65155354)(406.26137747,100.50026169)
\curveto(407.64552193,101.80054425)(408.90825021,102.73512234)(410.04956231,103.30399596)
\curveto(411.19087441,103.88641419)(412.41110702,104.17762331)(413.71026016,104.17762331)
\curveto(415.94431788,104.17762331)(417.58950713,103.41912515)(418.64582791,101.90212883)
\curveto(419.7142903,100.39867712)(420.24852149,98.29249026)(420.24852149,95.58356826)
\closepath
}
}
{
\newrgbcolor{curcolor}{0 0 0}
\pscustom[linestyle=none,fillstyle=solid,fillcolor=curcolor]
{
\newpath
\moveto(444.17965022,92.19064345)
\curveto(444.17965022,88.49296492)(443.32973695,85.57410147)(441.62991042,83.43405309)
\curveto(439.93008389,81.29400471)(437.6535305,80.22398052)(434.80025025,80.22398052)
\curveto(431.92268677,80.22398052)(429.63399176,81.29400471)(427.93416523,83.43405309)
\curveto(426.24648032,85.57410147)(425.40263786,88.49296492)(425.40263786,92.19064345)
\curveto(425.40263786,95.88832198)(426.24648032,98.80718544)(427.93416523,100.94723382)
\curveto(429.63399176,103.10082681)(431.92268677,104.17762331)(434.80025025,104.17762331)
\curveto(437.6535305,104.17762331)(439.93008389,103.10082681)(441.62991042,100.94723382)
\curveto(443.32973695,98.80718544)(444.17965022,95.88832198)(444.17965022,92.19064345)
\closepath
\moveto(440.64643936,92.19064345)
\curveto(440.64643936,95.12982382)(440.13042059,97.31050604)(439.09838305,98.73269009)
\curveto(438.06634551,100.16841875)(436.63363458,100.88628308)(434.80025025,100.88628308)
\curveto(432.94258269,100.88628308)(431.49773013,100.16841875)(430.4656926,98.73269009)
\curveto(429.44579668,97.31050604)(428.93584872,95.12982382)(428.93584872,92.19064345)
\curveto(428.93584872,89.34627535)(429.45186749,87.18591006)(430.48390503,85.70954757)
\curveto(431.51594256,84.24672969)(432.9547243,83.51532075)(434.80025025,83.51532075)
\curveto(436.62149296,83.51532075)(438.04813309,84.23995738)(439.08017062,85.68923065)
\curveto(440.12434978,87.15204853)(440.64643936,89.31918613)(440.64643936,92.19064345)
\closepath
}
}
{
\newrgbcolor{curcolor}{0 0 0}
\pscustom[linestyle=none,fillstyle=solid,fillcolor=curcolor]
{
\newpath
\moveto(462.17352562,99.38283137)
\lineto(461.99140135,99.38283137)
\curveto(461.48145339,99.51827747)(460.98364705,99.61308974)(460.49798232,99.66726818)
\curveto(460.02445922,99.73499123)(459.45987398,99.76885275)(458.8042266,99.76885275)
\curveto(457.74790583,99.76885275)(456.72800991,99.50473286)(455.74453885,98.97649307)
\curveto(454.76106778,98.46179789)(453.81402157,97.79133969)(452.90340022,96.96511848)
\lineto(452.90340022,80.85380488)
\lineto(449.47946392,80.85380488)
\lineto(449.47946392,103.54779894)
\lineto(452.90340022,103.54779894)
\lineto(452.90340022,100.19550797)
\curveto(454.26326144,101.41452287)(455.45921082,102.2746056)(456.49124836,102.77575617)
\curveto(457.53542751,103.29045135)(458.59781909,103.54779894)(459.6784231,103.54779894)
\curveto(460.27336239,103.54779894)(460.70438983,103.52748203)(460.97150543,103.4868482)
\curveto(461.23862103,103.45975898)(461.63929442,103.39880823)(462.17352562,103.30399596)
\closepath
}
}
{
\newrgbcolor{curcolor}{0 0 0}
\pscustom[linestyle=none,fillstyle=solid,fillcolor=curcolor]
{
\newpath
\moveto(482.99033808,103.54779894)
\lineto(471.1158356,72.4832359)
\lineto(467.45513775,72.4832359)
\lineto(471.24332259,81.95091829)
\lineto(463.13879253,103.54779894)
\lineto(466.85412766,103.54779894)
\lineto(473.10099016,86.72539332)
\lineto(479.40248994,103.54779894)
\closepath
}
}
{
\newrgbcolor{curcolor}{0 1 0}
\pscustom[linestyle=none,fillstyle=solid,fillcolor=curcolor]
{
\newpath
\moveto(335.02495007,315.09235633)
\lineto(485.14708286,315.09235633)
\lineto(485.14708286,269.97021272)
\lineto(335.02495007,269.97021272)
\closepath
}
}
{
\newrgbcolor{curcolor}{0.7019608 0.7019608 0.7019608}
\pscustom[linewidth=0.72824692,linecolor=curcolor]
{
\newpath
\moveto(335.02495007,315.09235633)
\lineto(485.14708286,315.09235633)
\lineto(485.14708286,269.97021272)
\lineto(335.02495007,269.97021272)
\closepath
}
}
{
\newrgbcolor{curcolor}{0 0 0}
\pscustom[linestyle=none,fillstyle=solid,fillcolor=curcolor]
{
\newpath
\moveto(356.97084153,301.46891206)
\curveto(356.97084153,300.30313541)(356.77606639,299.21978742)(356.38651611,298.21886808)
\curveto(356.00809583,297.22972426)(355.47385545,296.37011118)(354.78379495,295.64002884)
\curveto(353.92678433,294.73331367)(352.9139536,294.05033342)(351.74530275,293.59108807)
\curveto(350.57665191,293.14361825)(349.10192584,292.91988334)(347.32112455,292.91988334)
\lineto(344.01551217,292.91988334)
\lineto(344.01551217,283.11676156)
\lineto(340.70989978,283.11676156)
\lineto(340.70989978,309.41738918)
\lineto(347.45468465,309.41738918)
\curveto(348.94610573,309.41738918)(350.20936164,309.28197068)(351.24445239,309.01113368)
\curveto(352.27954314,308.7520722)(353.1977688,308.33992895)(353.99912938,307.77470391)
\curveto(354.94518006,307.10349917)(355.67419559,306.26743713)(356.18617596,305.26651779)
\curveto(356.70928634,304.26559845)(356.97084153,302.99972987)(356.97084153,301.46891206)
\closepath
\moveto(353.53166904,301.38059565)
\curveto(353.53166904,302.28731081)(353.38141393,303.07627076)(353.08090372,303.7474755)
\curveto(352.7803935,304.41868023)(352.32406317,304.96624199)(351.71191273,305.39016077)
\curveto(351.17767234,305.75520194)(350.5655219,306.01426342)(349.8754614,306.1673452)
\curveto(349.19653091,306.3322025)(348.33395529,306.41463115)(347.28773453,306.41463115)
\lineto(344.01551217,306.41463115)
\lineto(344.01551217,295.90497808)
\lineto(346.80357918,295.90497808)
\curveto(348.13918015,295.90497808)(349.22435593,296.02862105)(350.05910653,296.27590701)
\curveto(350.89385714,296.53496848)(351.57278763,296.94122398)(352.09589801,297.4946735)
\curveto(352.61900838,298.05989854)(352.98629865,298.65456238)(353.1977688,299.27866503)
\curveto(353.42036896,299.90276768)(353.53166904,300.60341122)(353.53166904,301.38059565)
\closepath
}
}
{
\newrgbcolor{curcolor}{0 0 0}
\pscustom[linestyle=none,fillstyle=solid,fillcolor=curcolor]
{
\newpath
\moveto(372.71423741,299.22567518)
\lineto(372.54728729,299.22567518)
\curveto(372.07982695,299.3434304)(371.62349662,299.42585905)(371.1782963,299.47296114)
\curveto(370.74422599,299.53183875)(370.22668061,299.56127755)(369.62566018,299.56127755)
\curveto(368.65734948,299.56127755)(367.7224288,299.33165488)(366.82089815,298.87240953)
\curveto(365.9193675,298.42493971)(365.05122687,297.84205139)(364.21647627,297.12374457)
\lineto(364.21647627,283.11676156)
\lineto(361.077814,283.11676156)
\lineto(361.077814,302.84664809)
\lineto(364.21647627,302.84664809)
\lineto(364.21647627,299.93220648)
\curveto(365.46303717,300.99200343)(366.55934296,301.73974906)(367.50539365,302.17544336)
\curveto(368.46257434,302.62291318)(369.43645004,302.84664809)(370.42702076,302.84664809)
\curveto(370.97239115,302.84664809)(371.36750644,302.82898481)(371.61236662,302.79365825)
\curveto(371.85722679,302.7701072)(372.22451706,302.71711735)(372.71423741,302.6346887)
\closepath
}
}
{
\newrgbcolor{curcolor}{0 0 0}
\pscustom[linestyle=none,fillstyle=solid,fillcolor=curcolor]
{
\newpath
\moveto(391.56290584,292.97287318)
\curveto(391.56290584,289.75815577)(390.78380528,287.22053085)(389.22560416,285.35999843)
\curveto(387.66740303,283.49946601)(385.58052652,282.5691998)(382.96497463,282.5691998)
\curveto(380.32716273,282.5691998)(378.22915621,283.49946601)(376.67095508,285.35999843)
\curveto(375.12388396,287.22053085)(374.35034841,289.75815577)(374.35034841,292.97287318)
\curveto(374.35034841,296.1875906)(375.12388396,298.72521551)(376.67095508,300.58574793)
\curveto(378.22915621,302.45805588)(380.32716273,303.39420985)(382.96497463,303.39420985)
\curveto(385.58052652,303.39420985)(387.66740303,302.45805588)(389.22560416,300.58574793)
\curveto(390.78380528,298.72521551)(391.56290584,296.1875906)(391.56290584,292.97287318)
\closepath
\moveto(388.3240735,292.97287318)
\curveto(388.3240735,295.52816138)(387.85104816,297.42402037)(386.90499748,298.66045014)
\curveto(385.9589468,299.90865544)(384.64560585,300.53275809)(382.96497463,300.53275809)
\curveto(381.2620834,300.53275809)(379.93761244,299.90865544)(378.99156176,298.66045014)
\curveto(378.05664108,297.42402037)(377.58918075,295.52816138)(377.58918075,292.97287318)
\curveto(377.58918075,290.50001364)(378.06220609,288.62181793)(379.00825677,287.33828607)
\curveto(379.95430746,286.06652973)(381.27321341,285.43065156)(382.96497463,285.43065156)
\curveto(384.63447584,285.43065156)(385.94225178,286.06064197)(386.88830247,287.32062279)
\curveto(387.84548316,288.59237913)(388.3240735,290.47646259)(388.3240735,292.97287318)
\closepath
}
}
{
\newrgbcolor{curcolor}{0 0 0}
\pscustom[linestyle=none,fillstyle=solid,fillcolor=curcolor]
{
\newpath
\moveto(411.54683574,285.35999843)
\curveto(411.54683574,282.01575028)(410.82895022,279.56055402)(409.39317918,277.99440964)
\curveto(407.95740814,276.42826526)(405.74810155,275.64519307)(402.76525939,275.64519307)
\curveto(401.77468867,275.64519307)(400.80637798,275.72173396)(399.86032729,275.87481574)
\curveto(398.92540662,276.016122)(398.00161595,276.22219363)(397.08895529,276.49303063)
\lineto(397.08895529,279.88438086)
\lineto(397.25590541,279.88438086)
\curveto(397.76788578,279.67242147)(398.58037637,279.41336)(399.69337717,279.10719643)
\curveto(400.80637798,278.78925735)(401.91937878,278.63028781)(403.03237958,278.63028781)
\curveto(404.10086036,278.63028781)(404.985696,278.7657063)(405.6868865,279.0365433)
\curveto(406.38807701,279.3073803)(406.9334474,279.68419699)(407.32299768,280.16699338)
\curveto(407.71254797,280.62623873)(407.99079817,281.17968824)(408.15774829,281.82734194)
\curveto(408.32469841,282.47499563)(408.40817347,283.19919021)(408.40817347,283.99992568)
\lineto(408.40817347,285.80158049)
\curveto(407.46212279,285.00084502)(406.55502713,284.40029342)(405.6868865,283.99992568)
\curveto(404.82987588,283.61133347)(403.73357009,283.41703736)(402.39796913,283.41703736)
\curveto(400.17196752,283.41703736)(398.40229624,284.26487492)(397.08895529,285.96055004)
\curveto(395.78674435,287.66800068)(395.13563888,290.0702071)(395.13563888,293.16716929)
\curveto(395.13563888,294.86284441)(395.35823904,296.32300909)(395.80343936,297.54766335)
\curveto(396.25976969,298.78409312)(396.87748514,299.84977783)(397.6565857,300.74471748)
\curveto(398.38003622,301.58077951)(399.25930686,302.22843321)(400.29439761,302.68767855)
\curveto(401.32948835,303.15869942)(402.3590141,303.39420985)(403.38297484,303.39420985)
\curveto(404.46258562,303.39420985)(405.36411627,303.27645463)(406.08756679,303.0409442)
\curveto(406.82214732,302.81720929)(407.59568288,302.4698314)(408.40817347,301.99881053)
\lineto(408.60851361,302.84664809)
\lineto(411.54683574,302.84664809)
\closepath
\moveto(408.40817347,288.53938928)
\lineto(408.40817347,299.29632831)
\curveto(407.57342287,299.69669605)(406.7943223,299.97930857)(406.07087178,300.14416587)
\curveto(405.35855126,300.3207987)(404.64623075,300.40911511)(403.93391024,300.40911511)
\curveto(402.20875899,300.40911511)(400.85089801,299.79678798)(399.86032729,298.57213373)
\curveto(398.86975658,297.34747948)(398.37447122,295.56937571)(398.37447122,293.23782242)
\curveto(398.37447122,291.02402435)(398.74176148,289.34601251)(399.47634201,288.20378691)
\curveto(400.21092254,287.06156131)(401.42965843,286.49044851)(403.13254966,286.49044851)
\curveto(404.04521032,286.49044851)(404.95787098,286.6729691)(405.87053163,287.03801027)
\curveto(406.7943223,287.41482696)(407.64020291,287.91528663)(408.40817347,288.53938928)
\closepath
}
}
{
\newrgbcolor{curcolor}{0 0 0}
\pscustom[linestyle=none,fillstyle=solid,fillcolor=curcolor]
{
\newpath
\moveto(429.36041007,299.22567518)
\lineto(429.19345995,299.22567518)
\curveto(428.72599961,299.3434304)(428.26966928,299.42585905)(427.82446896,299.47296114)
\curveto(427.39039865,299.53183875)(426.87285327,299.56127755)(426.27183284,299.56127755)
\curveto(425.30352214,299.56127755)(424.36860146,299.33165488)(423.46707081,298.87240953)
\curveto(422.56554016,298.42493971)(421.69739953,297.84205139)(420.86264893,297.12374457)
\lineto(420.86264893,283.11676156)
\lineto(417.72398666,283.11676156)
\lineto(417.72398666,302.84664809)
\lineto(420.86264893,302.84664809)
\lineto(420.86264893,299.93220648)
\curveto(422.10920983,300.99200343)(423.20551562,301.73974906)(424.15156631,302.17544336)
\curveto(425.108747,302.62291318)(426.0826227,302.84664809)(427.07319342,302.84664809)
\curveto(427.61856381,302.84664809)(428.0136791,302.82898481)(428.25853928,302.79365825)
\curveto(428.50339945,302.7701072)(428.87068972,302.71711735)(429.36041007,302.6346887)
\closepath
}
}
{
\newrgbcolor{curcolor}{0 0 0}
\pscustom[linestyle=none,fillstyle=solid,fillcolor=curcolor]
{
\newpath
\moveto(446.20567982,283.11676156)
\lineto(443.08371256,283.11676156)
\lineto(443.08371256,285.21869217)
\curveto(442.80546236,285.0185083)(442.42704209,284.73589578)(441.94845174,284.37085461)
\curveto(441.4809914,284.01758896)(441.02466107,283.73497644)(440.57946075,283.52301705)
\curveto(440.05635037,283.25218006)(439.45532994,283.02844514)(438.77639945,282.85181232)
\curveto(438.09746896,282.66340397)(437.30167338,282.5691998)(436.38901272,282.5691998)
\curveto(434.70838151,282.5691998)(433.28374048,283.15797588)(432.11508963,284.33552805)
\curveto(430.94643879,285.51308021)(430.36211337,287.01445923)(430.36211337,288.83966508)
\curveto(430.36211337,290.33515633)(430.66262358,291.5421473)(431.26364402,292.46063799)
\curveto(431.87579446,293.3909042)(432.74393509,294.12098655)(433.8680659,294.65088502)
\curveto(435.00332672,295.18078349)(436.36675271,295.5399369)(437.95834386,295.72834525)
\curveto(439.54993501,295.9167536)(441.25839124,296.05805986)(443.08371256,296.15226403)
\lineto(443.08371256,296.66449922)
\curveto(443.08371256,297.41813261)(442.95571747,298.04223526)(442.69972728,298.53680717)
\curveto(442.45486711,299.03137908)(442.09870685,299.41997129)(441.63124651,299.70258381)
\curveto(441.18604619,299.97342081)(440.6518058,300.15594139)(440.02852535,300.25014557)
\curveto(439.4052449,300.34434974)(438.75413943,300.39145183)(438.07520894,300.39145183)
\curveto(437.25158834,300.39145183)(436.33336268,300.27369661)(435.32053195,300.03818618)
\curveto(434.30770122,299.81445127)(433.26148046,299.48473666)(432.18186968,299.04904236)
\lineto(432.01491956,299.04904236)
\lineto(432.01491956,302.42272931)
\curveto(432.62707,302.59936214)(433.51190564,302.79365825)(434.66942648,303.00561764)
\curveto(435.82694732,303.21757702)(436.96777314,303.32355672)(438.09190395,303.32355672)
\curveto(439.4052449,303.32355672)(440.54607073,303.2058015)(441.51438143,302.97029107)
\curveto(442.49382213,302.74655616)(443.33970274,302.35796394)(444.05202326,301.80451443)
\curveto(444.75321377,301.26284043)(445.28745415,300.56219689)(445.65474442,299.70258381)
\curveto(446.02203468,298.84297073)(446.20567982,297.77728602)(446.20567982,296.50552968)
\closepath
\moveto(443.08371256,287.97416424)
\lineto(443.08371256,293.46744509)
\curveto(442.12653187,293.40856748)(440.99683605,293.32025107)(439.69462511,293.20249586)
\curveto(438.40354418,293.08474064)(437.37958344,292.91399558)(436.62274289,292.69026066)
\curveto(435.72121224,292.41942367)(434.99219671,291.99550489)(434.43569631,291.41850432)
\curveto(433.87919591,290.85327929)(433.60094571,290.0702071)(433.60094571,289.06928775)
\curveto(433.60094571,287.93883768)(433.92371594,287.08511236)(434.56925641,286.50811179)
\curveto(435.21479687,285.94288675)(436.19980258,285.66027423)(437.52427354,285.66027423)
\curveto(438.62614434,285.66027423)(439.63341007,285.88400915)(440.54607073,286.33147897)
\curveto(441.45873139,286.79072431)(442.304612,287.33828607)(443.08371256,287.97416424)
\closepath
}
}
{
\newrgbcolor{curcolor}{0 0 0}
\pscustom[linestyle=none,fillstyle=solid,fillcolor=curcolor]
{
\newpath
\moveto(479.46214364,283.11676156)
\lineto(476.32348137,283.11676156)
\lineto(476.32348137,294.35060922)
\curveto(476.32348137,295.19844678)(476.28452634,296.01684553)(476.20661629,296.80580548)
\curveto(476.13983624,297.59476543)(475.98958113,298.22475584)(475.75585096,298.69577671)
\curveto(475.49986078,299.20212414)(475.13257051,299.58482859)(474.65398017,299.84389007)
\curveto(474.17538982,300.10295155)(473.48532932,300.23248228)(472.58379867,300.23248228)
\curveto(471.70452803,300.23248228)(470.8252574,299.99697185)(469.94598676,299.52595099)
\curveto(469.06671613,299.06670564)(468.18744549,298.47792956)(467.30817486,297.75962274)
\curveto(467.34156488,297.48878574)(467.3693899,297.17084665)(467.39164992,296.80580548)
\curveto(467.41390993,296.45253983)(467.42503994,296.09927418)(467.42503994,295.74600853)
\lineto(467.42503994,283.11676156)
\lineto(464.28637767,283.11676156)
\lineto(464.28637767,294.35060922)
\curveto(464.28637767,295.22199782)(464.24742265,296.04628434)(464.16951259,296.82346877)
\curveto(464.10273254,297.61242872)(463.95247743,298.24241912)(463.71874726,298.71343999)
\curveto(463.46275708,299.21978742)(463.09546681,299.59660412)(462.61687647,299.84389007)
\curveto(462.13828612,300.10295155)(461.44822562,300.23248228)(460.54669497,300.23248228)
\curveto(459.68968435,300.23248228)(458.82710873,300.00874737)(457.9589681,299.56127755)
\curveto(457.10195748,299.11380773)(456.24494686,298.54269493)(455.38793624,297.84793915)
\lineto(455.38793624,283.11676156)
\lineto(452.24927397,283.11676156)
\lineto(452.24927397,302.84664809)
\lineto(455.38793624,302.84664809)
\lineto(455.38793624,300.65640106)
\curveto(456.36737695,301.51601415)(457.34125265,302.18721888)(458.30956335,302.67001527)
\curveto(459.28900406,303.15281166)(460.32965981,303.39420985)(461.43153061,303.39420985)
\curveto(462.70035153,303.39420985)(463.7743973,303.11159733)(464.65366794,302.54637229)
\curveto(465.54406858,301.98114725)(466.20630406,301.19807506)(466.64037438,300.19715572)
\curveto(467.90919529,301.3276058)(469.06671613,302.14011679)(470.11293688,302.6346887)
\curveto(471.15915764,303.14103613)(472.27772345,303.39420985)(473.46863431,303.39420985)
\curveto(475.51655579,303.39420985)(477.02467188,302.73478064)(477.99298258,301.41592221)
\curveto(478.97242329,300.10883931)(479.46214364,298.27774569)(479.46214364,295.92264136)
\closepath
}
}
{
\newrgbcolor{curcolor}{1 0.40000001 0}
\pscustom[linestyle=none,fillstyle=solid,fillcolor=curcolor]
{
\newpath
\moveto(52.3575999,178.83888562)
\lineto(198.07187944,178.83888562)
\lineto(198.07187944,124.55317477)
\lineto(52.3575999,124.55317477)
\closepath
}
}
{
\newrgbcolor{curcolor}{0.7019608 0.7019608 0.7019608}
\pscustom[linewidth=0.85039368,linecolor=curcolor]
{
\newpath
\moveto(52.3575999,178.83888562)
\lineto(198.07187944,178.83888562)
\lineto(198.07187944,124.55317477)
\lineto(52.3575999,124.55317477)
\closepath
}
}
{
\newrgbcolor{curcolor}{0 0 0}
\pscustom[linestyle=none,fillstyle=solid,fillcolor=curcolor]
{
\newpath
\moveto(106.15231,151.96947195)
\curveto(106.15231,149.3262512)(105.57288476,146.9304255)(104.41403429,144.78199484)
\curveto(103.2682046,142.63356418)(101.73826156,140.96690282)(99.82420515,139.78201076)
\curveto(98.49608438,138.96170087)(97.01171411,138.36925484)(95.37109433,138.00467267)
\curveto(93.74349535,137.6400905)(91.59506469,137.45779941)(88.92580235,137.45779941)
\lineto(81.58207573,137.45779941)
\lineto(81.58207573,166.53973806)
\lineto(88.8476776,166.53973806)
\curveto(91.68621023,166.53973806)(93.93880722,166.33140539)(95.60546858,165.91474005)
\curveto(97.28515074,165.51109551)(98.70441705,164.95120146)(99.86326753,164.2350579)
\curveto(101.84242789,162.99808267)(103.38539173,161.3509525)(104.49215904,159.29366738)
\curveto(105.59892635,157.23638227)(106.15231,154.79498379)(106.15231,151.96947195)
\closepath
\moveto(102.10935413,152.02806552)
\curveto(102.10935413,154.3067041)(101.71221997,156.2272709)(100.91795167,157.78976592)
\curveto(100.12368336,159.35226095)(98.9387913,160.58272578)(97.36327549,161.48116042)
\curveto(96.2174458,162.13220001)(95.00000176,162.58141733)(93.71094337,162.82881238)
\curveto(92.42188497,163.08922822)(90.87892113,163.21943614)(89.08205185,163.21943614)
\lineto(85.44925092,163.21943614)
\lineto(85.44925092,140.77810134)
\lineto(89.08205185,140.77810134)
\curveto(90.94402509,140.77810134)(92.56511368,140.91481965)(93.94531762,141.18825628)
\curveto(95.33854235,141.46169291)(96.61457995,141.96950379)(97.77343043,142.71168893)
\curveto(99.21873833,143.63616515)(100.29946406,144.85360919)(101.01560761,146.36402105)
\curveto(101.74477195,147.87443291)(102.10935413,149.76244773)(102.10935413,152.02806552)
\closepath
}
}
{
\newrgbcolor{curcolor}{0 0 0}
\pscustom[linestyle=none,fillstyle=solid,fillcolor=curcolor]
{
\newpath
\moveto(129.06239412,137.45779941)
\lineto(125.410062,137.45779941)
\lineto(125.410062,139.78201076)
\curveto(125.0845422,139.5606573)(124.64183528,139.24815829)(124.08194123,138.84451374)
\curveto(123.53506797,138.45388999)(123.0012155,138.14139098)(122.48038383,137.90701673)
\curveto(121.86840661,137.60753852)(121.16528385,137.36014347)(120.37101554,137.16483159)
\curveto(119.57674724,136.95649892)(118.64576062,136.85233259)(117.57805568,136.85233259)
\curveto(115.61191611,136.85233259)(113.94525475,137.50337218)(112.5780716,138.80545137)
\curveto(111.21088846,140.10753056)(110.52729688,141.76768152)(110.52729688,143.78590426)
\curveto(110.52729688,145.43954483)(110.87885826,146.774176)(111.58198103,147.78979776)
\curveto(112.29812458,148.81844032)(113.31374634,149.62572942)(114.62884632,150.21166505)
\curveto(115.9569671,150.79760069)(117.5520141,151.19473484)(119.41398734,151.40306751)
\curveto(121.27596058,151.61140018)(123.27465213,151.76764968)(125.410062,151.87181602)
\lineto(125.410062,152.43822046)
\curveto(125.410062,153.27155114)(125.26032289,153.96165311)(124.96084468,154.50852637)
\curveto(124.67438726,155.05539963)(124.25772192,155.48508576)(123.71084866,155.79758477)
\curveto(123.19001698,156.09706298)(122.56501897,156.29888525)(121.83585463,156.40305159)
\curveto(121.10669028,156.50721792)(120.34497396,156.55930109)(119.55070565,156.55930109)
\curveto(118.58716705,156.55930109)(117.51295172,156.42909317)(116.32805966,156.16867733)
\curveto(115.1431676,155.92128229)(113.91921317,155.55670012)(112.65619636,155.07493082)
\lineto(112.46088448,155.07493082)
\lineto(112.46088448,158.80538769)
\curveto(113.17702803,159.00069957)(114.21218098,159.21554263)(115.56634334,159.44991689)
\curveto(116.92050569,159.68429114)(118.25513686,159.80147827)(119.57023684,159.80147827)
\curveto(121.10669028,159.80147827)(122.44132145,159.67127035)(123.57413034,159.41085451)
\curveto(124.71996003,159.16345947)(125.70954021,158.73377333)(126.54287089,158.12179612)
\curveto(127.36318078,157.52283969)(127.98817879,156.74810257)(128.41786492,155.79758477)
\curveto(128.84755105,154.84706696)(129.06239412,153.66868529)(129.06239412,152.26243977)
\closepath
\moveto(125.410062,142.82887606)
\lineto(125.410062,148.90307547)
\curveto(124.2902739,148.83797151)(122.96866352,148.74031557)(121.44523087,148.61010765)
\curveto(119.93481901,148.47989973)(118.73690616,148.29109825)(117.85149231,148.0437032)
\curveto(116.79680817,147.74422499)(115.9439463,147.27547648)(115.29290671,146.63745768)
\curveto(114.64186712,146.01245967)(114.31634732,145.14657701)(114.31634732,144.0398097)
\curveto(114.31634732,142.78981368)(114.69395028,141.84580627)(115.44915621,141.20778747)
\curveto(116.20436214,140.58278946)(117.35670222,140.27029045)(118.90617646,140.27029045)
\curveto(120.19523485,140.27029045)(121.37361652,140.5176855)(122.44132145,141.01247559)
\curveto(123.50902638,141.52028647)(124.49860657,142.1257533)(125.410062,142.82887606)
\closepath
}
}
{
\newrgbcolor{curcolor}{0 0 0}
\pscustom[linestyle=none,fillstyle=solid,fillcolor=curcolor]
{
\newpath
\moveto(147.46077384,137.65311129)
\curveto(146.77067187,137.4708202)(146.01546594,137.32108109)(145.19515605,137.20389397)
\curveto(144.38786695,137.08670684)(143.665213,137.02811328)(143.0271942,137.02811328)
\curveto(140.80063879,137.02811328)(139.10793585,137.6270697)(137.94908537,138.82498256)
\curveto(136.79023489,140.02289541)(136.21080965,141.94346221)(136.21080965,144.58668296)
\lineto(136.21080965,156.18820852)
\lineto(133.7303488,156.18820852)
\lineto(133.7303488,159.2741362)
\lineto(136.21080965,159.2741362)
\lineto(136.21080965,165.54364749)
\lineto(139.88267296,165.54364749)
\lineto(139.88267296,159.2741362)
\lineto(147.46077384,159.2741362)
\lineto(147.46077384,156.18820852)
\lineto(139.88267296,156.18820852)
\lineto(139.88267296,146.24683393)
\curveto(139.88267296,145.10100424)(139.90871455,144.2025696)(139.96079771,143.55153001)
\curveto(140.01288088,142.9135112)(140.19517197,142.31455478)(140.50767097,141.75466073)
\curveto(140.79412839,141.23382905)(141.18475215,140.84971569)(141.67954224,140.60232065)
\curveto(142.18735313,140.36794639)(142.95557985,140.25075927)(143.9842224,140.25075927)
\curveto(144.58317883,140.25075927)(145.20817684,140.33539441)(145.85921643,140.50466471)
\curveto(146.51025603,140.68695579)(146.97900454,140.8366949)(147.26546196,140.95388203)
\lineto(147.46077384,140.95388203)
\closepath
}
}
{
\newrgbcolor{curcolor}{0 0 0}
\pscustom[linestyle=none,fillstyle=solid,fillcolor=curcolor]
{
\newpath
\moveto(168.84742421,137.45779941)
\lineto(165.19509209,137.45779941)
\lineto(165.19509209,139.78201076)
\curveto(164.8695723,139.5606573)(164.42686537,139.24815829)(163.86697132,138.84451374)
\curveto(163.32009806,138.45388999)(162.7862456,138.14139098)(162.26541392,137.90701673)
\curveto(161.6534367,137.60753852)(160.95031394,137.36014347)(160.15604564,137.16483159)
\curveto(159.36177733,136.95649892)(158.43079071,136.85233259)(157.36308578,136.85233259)
\curveto(155.39694621,136.85233259)(153.73028485,137.50337218)(152.3631017,138.80545137)
\curveto(150.99591855,140.10753056)(150.31232698,141.76768152)(150.31232698,143.78590426)
\curveto(150.31232698,145.43954483)(150.66388836,146.774176)(151.36701112,147.78979776)
\curveto(152.08315467,148.81844032)(153.09877644,149.62572942)(154.41387642,150.21166505)
\curveto(155.74199719,150.79760069)(157.3370442,151.19473484)(159.19901743,151.40306751)
\curveto(161.06099067,151.61140018)(163.05968222,151.76764968)(165.19509209,151.87181602)
\lineto(165.19509209,152.43822046)
\curveto(165.19509209,153.27155114)(165.04535299,153.96165311)(164.74587477,154.50852637)
\curveto(164.45941735,155.05539963)(164.04275201,155.48508576)(163.49587875,155.79758477)
\curveto(162.97504708,156.09706298)(162.35004907,156.29888525)(161.62088472,156.40305159)
\curveto(160.89172038,156.50721792)(160.13000405,156.55930109)(159.33573575,156.55930109)
\curveto(158.37219715,156.55930109)(157.29798182,156.42909317)(156.11308976,156.16867733)
\curveto(154.9281977,155.92128229)(153.70424326,155.55670012)(152.44122645,155.07493082)
\lineto(152.24591457,155.07493082)
\lineto(152.24591457,158.80538769)
\curveto(152.96205812,159.00069957)(153.99721108,159.21554263)(155.35137343,159.44991689)
\curveto(156.70553579,159.68429114)(158.04016696,159.80147827)(159.35526694,159.80147827)
\curveto(160.89172038,159.80147827)(162.22635154,159.67127035)(163.35916044,159.41085451)
\curveto(164.50499012,159.16345947)(165.49457031,158.73377333)(166.32790099,158.12179612)
\curveto(167.14821087,157.52283969)(167.77320888,156.74810257)(168.20289502,155.79758477)
\curveto(168.63258115,154.84706696)(168.84742421,153.66868529)(168.84742421,152.26243977)
\closepath
\moveto(165.19509209,142.82887606)
\lineto(165.19509209,148.90307547)
\curveto(164.07530399,148.83797151)(162.75369362,148.74031557)(161.23026097,148.61010765)
\curveto(159.71984911,148.47989973)(158.52193626,148.29109825)(157.63652241,148.0437032)
\curveto(156.58183827,147.74422499)(155.7289764,147.27547648)(155.0779368,146.63745768)
\curveto(154.42689721,146.01245967)(154.10137741,145.14657701)(154.10137741,144.0398097)
\curveto(154.10137741,142.78981368)(154.47898038,141.84580627)(155.23418631,141.20778747)
\curveto(155.98939224,140.58278946)(157.14173232,140.27029045)(158.69120655,140.27029045)
\curveto(159.98026495,140.27029045)(161.15864661,140.5176855)(162.22635154,141.01247559)
\curveto(163.29405648,141.52028647)(164.28363666,142.1257533)(165.19509209,142.82887606)
\closepath
}
}
{
\newrgbcolor{curcolor}{0 0 0}
\pscustom[linewidth=3.20881881,linecolor=curcolor]
{
\newpath
\moveto(125.21474646,176.00486764)
\lineto(125.21474646,312.64627866)
}
}
{
\newrgbcolor{curcolor}{0 0 0}
\pscustom[linestyle=none,fillstyle=solid,fillcolor=curcolor]
{
\newpath
\moveto(125.21474646,280.55809055)
\lineto(138.0500217,267.7228153)
\lineto(125.21474646,312.64627866)
\lineto(112.37947121,267.7228153)
\closepath
}
}
{
\newrgbcolor{curcolor}{0 0 0}
\pscustom[linewidth=3.42274017,linecolor=curcolor]
{
\newpath
\moveto(125.21474646,280.55809055)
\lineto(138.0500217,267.7228153)
\lineto(125.21474646,312.64627866)
\lineto(112.37947121,267.7228153)
\closepath
}
}
{
\newrgbcolor{curcolor}{0 0 0}
\pscustom[linewidth=3.43438118,linecolor=curcolor]
{
\newpath
\moveto(174.32469921,344.84808023)
\lineto(330.92751496,344.84808023)
}
}
{
\newrgbcolor{curcolor}{0 0 0}
\pscustom[linestyle=none,fillstyle=solid,fillcolor=curcolor]
{
\newpath
\moveto(296.58370311,344.84808023)
\lineto(282.84617838,331.1105555)
\lineto(330.92751496,344.84808023)
\lineto(282.84617838,358.58560497)
\closepath
}
}
{
\newrgbcolor{curcolor}{0 0 0}
\pscustom[linewidth=3.66334004,linecolor=curcolor]
{
\newpath
\moveto(296.58370311,344.84808023)
\lineto(282.84617838,331.1105555)
\lineto(330.92751496,344.84808023)
\lineto(282.84617838,358.58560497)
\closepath
}
}
\end{pspicture}
}
    \captionsetup{width=0.8\linewidth}
    \caption{Local and Remote Memory Accesses}
    \label{fig:LocalvsRemote}
\end{figure}

\section{Measuring the Impact of NUMA}
\label{chapter:measuringnuma}
Tools such as the STREAM benchmark, used to measure the bandwidth of memory based on the throughput of various operations
performed on memory \cite{bergstrom_stream}.
While originally used to measure CPU performance, ``On modern hardware, each of these tests achieve similar bandwidth,
as memory is the primary constraint, not floating-point execution'' \cite{bergstrom_stream}.
STREAM showed a 33\% reduction in bandwidth when accessing memory remotely, from a program running on one node to memory located
on another node. This went up to 39\% when running with multiple threads, which of course caused contention over the interconnect,
as it is shared between the multiple processors on the same CPU socket.

Another tool used to measure the bandwidth and latency of machines with a NUMA architecture is the Intel Memory Latency Checker.
The Intel MLC is likely to have a more reliable result as it is directly intended to measure NUMA latency and bandwidth,
while STREAM only incidentally does so.
Using this tool in my test environment, I found that there was about a 63\% increase in latency
(66.8 nanoseconds locally versus 108.8 remotely) and about a 50\% reduction in bandwidth 
(18405.1 MB/s locally versus 9284.0 remotely).
Thus, there is a significant performance impact from NUMA, to both memory bandwidth and memory latency.

\section{Linux and NUMA}
The most important place for NUMA support is of course the operating system, and it has been supported in Linux since 2.5, with the current system
of APIs and scheduler support added in version 2.6 \cite{dobson_linux_numa}
\cite[{Documentation/admin-guide/mm/numa\_memory\_policy.rst}]{linux}.
There are two main ways in which the operating system has to support NUMA. First, in memory allocation, so that a process can ask for memory from
a specific node, and in the scheduler, so that a process can be kept on the socket closest to the memory that it uses.

\subsection{Memory Policy}
Linux uses memory policy to determine where an allocation should take place \cite[{Documentation/admin-guide/mm/numa\_memory\_policy.rst}]{linux}. 
Typically, a process has a local memory policy, such that it prefers memory from the NUMA node closest to where it is currently running when possible. 
Other configurations are possible for better performance in some scenarios, including preferring a specific node,
or binding all memory allocated to a specific node, such that the process would rather an allocation fail if it was unable to allocate memory 
from a specific node.
Memory in Linux is allocated on a first touch policy, that is the process or thread that first accesses the memory determines the policy used to decide
where that particular allocation will be located.
As an example, if two threads had a local memory policy and one thread running on node 0 were to allocate some memory, but another thread running on a node 1 were to actually write to that
memory, that allocation would be placed closest to the second thread on node 1.

\section{Scheduling}
Linux's scheduler is also aware of NUMA \cite[{Documentation/scheduler/sched-domains.txt}]{linux}.
This allows it to attempt to keep processes within the same NUMA node.
It uses a hierarchy of scheduling domains, which the scheduler tries to keep a process within while still balancing all running processes across
all available processors. 
These include not only the same NUMA node, but also the same physical CPU core or even the same virtual processor when possible.

\section{Memory Reclamation}
Linux generally will allocate all available memory for caching, as files are read from disks \cite{lameter_numa_2013}.
Only when all system memory is running low does it release these memory caches, either those not in use or those least likely to be used,
freeing them for use by the rest of the system.
When memory runs low on a particular node, Linux usually just allocates on another node, unless the system has 4 or more nodes,
at which point it frees memory from the local node when that node is low on memory.
Unlike every other filesystem on Linux, ZFS does not utilize Linux's page cache, with the exception of when files are directly mapped into memory
by a process.
Thus it does not take advantage of this caching mechanism, instead using its own internal cache, the ARC (Section \ref{chapter:ARC}).
This allows ZFS to have much more control of how it caches files and use a very clever caching policy, an Adaptive Replacement Cache.

\section{How Does NUMA Affect Programs?}
NUMA architectures mean that operating systems and programs need to be cognizant of where they allocate memory from in order to achieve maximum performance.
Poor memory and CPU location can lead to substantially increased latency in most cases \cite{lameter_numa_2013}.

\chapter{Mitigating the Effects of NUMA on ZFS}

\section{Test Environment}

All experiments were performed on a Dell PowerEdge 610 with two Xeon X5570 CPUs and 48 GB of RAM split evenly between them.
This system has two NUMA Nodes, each with 4 processors, with each processor capable of hyperthreading
and thus running two processes at once.
The NUMA nodes are connected directly via Intel's Quick Path Interconnect,
their first NUMA interconnect system \cite{kochhar_optimal_2009}.
Each node has 24 GB of DDR3-1066 RAM.
The system has three disks, two root disks, each 67.77 GiB in size,
mirrored by the BIOS (one of which failed during testing),
that were used to store the operating system, the testing scripts, and the test results,
and one ZFS disk, 68.38 GiB in size, used only to store the files used in testing.

Every data point shown is the average of three distinct tests, in order to ensure the effects that are visible in the data are not the result of
any particular test run, but instead are consistent and repeatable.

\subsection{Automatic Testing}
To guarantee a clean test environment, the machine was rebooted between each test.
However, the machine used for testing took around 10 minutes to reboot, so manually running multiple tests in a row
quickly became very tedious.

In order to run many tests in a row, I wrote a collection of scripts in Python 3 and Bash 
to measure the differences between file accesses on different nodes.
Each test run was started by an \texttt{rc.local} script that would run on each boot.
This script would trigger \texttt{autotest.py}, which would then look for new tests to be run based on 
the file names within the test-run folder.
The name of each test run would inform the script which version of ZFS needed to be loaded,
either a modified version, or a standard release version of ZFS to compare against,
which test to run, and what file to test against.

The script automatically rebooted the machine at the end of each test.
However, care was taken to ensure that it would not reboot if there were no more test runs needed, so that the machine
would not enter a boot loop and become inaccessible.

A ZFS filesystem was created with ZFS version 0.8.5 and populated with a set of files full of randomly generated data
of various sizes, which were then used to test the performance of reading files from ZFS.

The \texttt{numactl} program was used to control which processors the test program ran on, as it allows you to set what nodes a program is permitted
to run on before executing it.

\begin{figure}
    \centering
    \resizebox{!}{0.6\textheight}{%LaTeX with PSTricks extensions
%%Creator: Inkscape 1.0.2-2 (e86c870879, 2021-01-15)
%%Please note this file requires PSTricks extensions
\psset{xunit=.5pt,yunit=.5pt,runit=.5pt}
\begin{pspicture}(1122.51968504,1889.76377953)
{
\newrgbcolor{curcolor}{0.7019608 0.7019608 0.7019608}
\pscustom[linestyle=none,fillstyle=solid,fillcolor=curcolor,opacity=0.92623001]
{
\newpath
\moveto(341.61699,1861.19235181)
\lineto(780.90269605,1861.19235181)
\lineto(780.90269605,1716.90662654)
\lineto(341.61699,1716.90662654)
\closepath
}
}
{
\newrgbcolor{curcolor}{0 0 0}
\pscustom[linewidth=1.00157103,linecolor=curcolor]
{
\newpath
\moveto(341.61699,1861.19235181)
\lineto(780.90269605,1861.19235181)
\lineto(780.90269605,1716.90662654)
\lineto(341.61699,1716.90662654)
\closepath
}
}
{
\newrgbcolor{curcolor}{0 0 0}
\pscustom[linestyle=none,fillstyle=solid,fillcolor=curcolor]
{
\newpath
\moveto(460.01994445,1849.55028079)
\lineto(446.01608279,1813.22227146)
\lineto(442.6371873,1813.22227146)
\lineto(456.5824554,1849.55028079)
\closepath
}
}
{
\newrgbcolor{curcolor}{0 0 0}
\pscustom[linestyle=none,fillstyle=solid,fillcolor=curcolor]
{
\newpath
\moveto(483.26205811,1829.68706278)
\lineto(467.18789054,1829.68706278)
\curveto(467.18789054,1828.34592122)(467.38971281,1827.17404995)(467.79335736,1826.17144898)
\curveto(468.19700191,1825.1818688)(468.75038557,1824.3680693)(469.45350833,1823.7300505)
\curveto(470.1305895,1823.10505249)(470.9313682,1822.63630398)(471.85584443,1822.32380498)
\curveto(472.79334144,1822.01130597)(473.821984,1821.85505647)(474.9417721,1821.85505647)
\curveto(476.42614238,1821.85505647)(477.91702305,1822.14802429)(479.41441411,1822.73395992)
\curveto(480.92482597,1823.33291635)(481.9990413,1823.91885198)(482.6370601,1824.49176683)
\lineto(482.83237198,1824.49176683)
\lineto(482.83237198,1820.48787332)
\curveto(481.59539675,1819.96704165)(480.33237994,1819.53084512)(479.04332154,1819.17928374)
\curveto(477.75426315,1818.82772236)(476.40010079,1818.65194167)(474.98083448,1818.65194167)
\curveto(471.36105434,1818.65194167)(468.5355425,1819.62850106)(466.50429897,1821.58161984)
\curveto(464.47305543,1823.54775942)(463.45743367,1826.33420888)(463.45743367,1829.94096823)
\curveto(463.45743367,1833.5086652)(464.42748266,1836.34068743)(466.36758065,1838.43703493)
\curveto(468.32069943,1840.53338242)(470.88579543,1841.58155616)(474.06286865,1841.58155616)
\curveto(477.00556761,1841.58155616)(479.2711854,1840.7221839)(480.85972201,1839.00343937)
\curveto(482.46127941,1837.28469484)(483.26205811,1834.84329637)(483.26205811,1831.67924394)
\closepath
\moveto(479.68785074,1832.49955383)
\curveto(479.67482995,1834.42663103)(479.18655025,1835.9175117)(478.22301166,1836.97219584)
\curveto(477.27249385,1838.02687998)(475.82067555,1838.55422205)(473.86755677,1838.55422205)
\curveto(471.9014172,1838.55422205)(470.33241178,1837.97479681)(469.16054051,1836.81594634)
\curveto(468.00169003,1835.65709586)(467.34414004,1834.21829836)(467.18789054,1832.49955383)
\closepath
}
}
{
\newrgbcolor{curcolor}{0 0 0}
\pscustom[linestyle=none,fillstyle=solid,fillcolor=curcolor]
{
\newpath
\moveto(500.15653454,1819.35506443)
\curveto(499.46643257,1819.17277334)(498.71122664,1819.02303424)(497.89091676,1818.90584711)
\curveto(497.08362766,1818.78865998)(496.36097371,1818.73006642)(495.72295491,1818.73006642)
\curveto(493.4963995,1818.73006642)(491.80369655,1819.32902285)(490.64484608,1820.5269357)
\curveto(489.4859956,1821.72484855)(488.90657036,1823.64541535)(488.90657036,1826.28863611)
\lineto(488.90657036,1837.89016167)
\lineto(486.42610951,1837.89016167)
\lineto(486.42610951,1840.97608934)
\lineto(488.90657036,1840.97608934)
\lineto(488.90657036,1847.24560063)
\lineto(492.57843367,1847.24560063)
\lineto(492.57843367,1840.97608934)
\lineto(500.15653454,1840.97608934)
\lineto(500.15653454,1837.89016167)
\lineto(492.57843367,1837.89016167)
\lineto(492.57843367,1827.94878707)
\curveto(492.57843367,1826.80295738)(492.60447525,1825.90452274)(492.65655842,1825.25348315)
\curveto(492.70864159,1824.61546435)(492.89093268,1824.01650792)(493.20343168,1823.45661387)
\curveto(493.4898891,1822.9357822)(493.88051286,1822.55166884)(494.37530295,1822.30427379)
\curveto(494.88311383,1822.06989954)(495.65134055,1821.95271241)(496.67998311,1821.95271241)
\curveto(497.27893954,1821.95271241)(497.90393755,1822.03734756)(498.55497714,1822.20661785)
\curveto(499.20601674,1822.38890894)(499.67476524,1822.53864804)(499.96122266,1822.65583517)
\lineto(500.15653454,1822.65583517)
\closepath
}
}
{
\newrgbcolor{curcolor}{0 0 0}
\pscustom[linestyle=none,fillstyle=solid,fillcolor=curcolor]
{
\newpath
\moveto(520.72287503,1820.5269357)
\curveto(519.4989206,1819.94100007)(518.33355972,1819.48527235)(517.22679241,1819.15975255)
\curveto(516.1330459,1818.83423276)(514.96768502,1818.67147286)(513.7307098,1818.67147286)
\curveto(512.15519398,1818.67147286)(510.70988608,1818.89933672)(509.3947861,1819.35506443)
\curveto(508.07968612,1819.82381294)(506.95338762,1820.5269357)(506.01589061,1821.46443271)
\curveto(505.0653728,1822.40192973)(504.32969806,1823.58682179)(503.80886639,1825.0191089)
\curveto(503.28803471,1826.451396)(503.02761887,1828.12456776)(503.02761887,1830.03862417)
\curveto(503.02761887,1833.60632114)(504.00417826,1836.40579139)(505.95729705,1838.43703493)
\curveto(507.92343662,1840.46827846)(510.5145742,1841.48390022)(513.7307098,1841.48390022)
\curveto(514.98070582,1841.48390022)(516.20466025,1841.30811953)(517.4025731,1840.95655815)
\curveto(518.61350675,1840.60499677)(519.72027406,1840.17531064)(520.72287503,1839.66749976)
\lineto(520.72287503,1835.5854815)
\lineto(520.52756315,1835.5854815)
\curveto(519.40777505,1836.45787456)(518.24892458,1837.12844534)(517.05101172,1837.59719385)
\curveto(515.86611966,1838.06594236)(514.70726919,1838.30031661)(513.57446029,1838.30031661)
\curveto(511.49113359,1838.30031661)(509.84400342,1837.59719385)(508.63306978,1836.19094833)
\curveto(507.43515692,1834.7977236)(506.8362005,1832.74694888)(506.8362005,1830.03862417)
\curveto(506.8362005,1827.40842421)(507.42213613,1825.38369107)(508.5940074,1823.96442476)
\curveto(509.77889946,1822.55817923)(511.43905043,1821.85505647)(513.57446029,1821.85505647)
\curveto(514.31664543,1821.85505647)(515.07185136,1821.95271241)(515.84007808,1822.14802429)
\curveto(516.6083048,1822.34333617)(517.29840677,1822.59724161)(517.91038399,1822.90974061)
\curveto(518.44423645,1823.18317724)(518.94553694,1823.46963466)(519.41428545,1823.76911288)
\curveto(519.88303396,1824.08161188)(520.25412653,1824.34853812)(520.52756315,1824.56989158)
\lineto(520.72287503,1824.56989158)
\closepath
}
}
{
\newrgbcolor{curcolor}{0 0 0}
\pscustom[linestyle=none,fillstyle=solid,fillcolor=curcolor]
{
\newpath
\moveto(538.61344695,1849.55028079)
\lineto(524.60958528,1813.22227146)
\lineto(521.23068979,1813.22227146)
\lineto(535.17595789,1849.55028079)
\closepath
}
}
{
\newrgbcolor{curcolor}{0 0 0}
\pscustom[linestyle=none,fillstyle=solid,fillcolor=curcolor]
{
\newpath
\moveto(557.20713791,1836.97219584)
\lineto(557.01182603,1836.97219584)
\curveto(556.46495277,1837.10240376)(555.9311003,1837.1935493)(555.41026863,1837.24563247)
\curveto(554.90245774,1837.31073643)(554.29699092,1837.34328841)(553.59386816,1837.34328841)
\curveto(552.46105927,1837.34328841)(551.36731275,1837.08938297)(550.31262861,1836.58157208)
\curveto(549.25794447,1836.08678199)(548.2423227,1835.44225279)(547.26576331,1834.64798449)
\lineto(547.26576331,1819.15975255)
\lineto(543.5939,1819.15975255)
\lineto(543.5939,1840.97608934)
\lineto(547.26576331,1840.97608934)
\lineto(547.26576331,1837.75344335)
\curveto(548.724092,1838.92531462)(550.00664,1839.7521349)(551.11340731,1840.2339042)
\curveto(552.23319541,1840.7286943)(553.3725147,1840.97608934)(554.53136518,1840.97608934)
\curveto(555.16938398,1840.97608934)(555.63162209,1840.95655815)(555.91807951,1840.91749578)
\curveto(556.20453693,1840.89145419)(556.63422306,1840.83286063)(557.20713791,1840.74171509)
\closepath
}
}
{
\newrgbcolor{curcolor}{0 0 0}
\pscustom[linestyle=none,fillstyle=solid,fillcolor=curcolor]
{
\newpath
\moveto(576.79691356,1820.5269357)
\curveto(575.57295912,1819.94100007)(574.40759825,1819.48527235)(573.30083094,1819.15975255)
\curveto(572.20708442,1818.83423276)(571.04172355,1818.67147286)(569.80474832,1818.67147286)
\curveto(568.22923251,1818.67147286)(566.78392461,1818.89933672)(565.46882463,1819.35506443)
\curveto(564.15372465,1819.82381294)(563.02742615,1820.5269357)(562.08992914,1821.46443271)
\curveto(561.13941133,1822.40192973)(560.40373659,1823.58682179)(559.88290491,1825.0191089)
\curveto(559.36207324,1826.451396)(559.1016574,1828.12456776)(559.1016574,1830.03862417)
\curveto(559.1016574,1833.60632114)(560.07821679,1836.40579139)(562.03133557,1838.43703493)
\curveto(563.99747515,1840.46827846)(566.58861273,1841.48390022)(569.80474832,1841.48390022)
\curveto(571.05474434,1841.48390022)(572.27869878,1841.30811953)(573.47661163,1840.95655815)
\curveto(574.68754528,1840.60499677)(575.79431259,1840.17531064)(576.79691356,1839.66749976)
\lineto(576.79691356,1835.5854815)
\lineto(576.60160168,1835.5854815)
\curveto(575.48181358,1836.45787456)(574.3229631,1837.12844534)(573.12505025,1837.59719385)
\curveto(571.94015819,1838.06594236)(570.78130771,1838.30031661)(569.64849882,1838.30031661)
\curveto(567.56517212,1838.30031661)(565.91804195,1837.59719385)(564.7071083,1836.19094833)
\curveto(563.50919545,1834.7977236)(562.91023903,1832.74694888)(562.91023903,1830.03862417)
\curveto(562.91023903,1827.40842421)(563.49617466,1825.38369107)(564.66804593,1823.96442476)
\curveto(565.85293799,1822.55817923)(567.51308895,1821.85505647)(569.64849882,1821.85505647)
\curveto(570.39068396,1821.85505647)(571.14588989,1821.95271241)(571.91411661,1822.14802429)
\curveto(572.68234333,1822.34333617)(573.3724453,1822.59724161)(573.98442252,1822.90974061)
\curveto(574.51827498,1823.18317724)(575.01957547,1823.46963466)(575.48832398,1823.76911288)
\curveto(575.95707248,1824.08161188)(576.32816505,1824.34853812)(576.60160168,1824.56989158)
\lineto(576.79691356,1824.56989158)
\closepath
}
}
{
\newrgbcolor{curcolor}{0 0 0}
\pscustom[linestyle=none,fillstyle=solid,fillcolor=curcolor]
{
\newpath
\moveto(587.50000836,1819.15975255)
\lineto(582.83205447,1819.15975255)
\lineto(582.83205447,1824.72614108)
\lineto(587.50000836,1824.72614108)
\closepath
}
}
{
\newrgbcolor{curcolor}{0 0 0}
\pscustom[linestyle=none,fillstyle=solid,fillcolor=curcolor]
{
\newpath
\moveto(599.76558777,1819.15975255)
\lineto(596.09372446,1819.15975255)
\lineto(596.09372446,1849.55028079)
\lineto(599.76558777,1849.55028079)
\closepath
}
}
{
\newrgbcolor{curcolor}{0 0 0}
\pscustom[linestyle=none,fillstyle=solid,fillcolor=curcolor]
{
\newpath
\moveto(625.62489135,1830.05815535)
\curveto(625.62489135,1826.50347917)(624.71343592,1823.69749852)(622.89052506,1821.64021341)
\curveto(621.0676142,1819.58292829)(618.62621572,1818.55428573)(615.56632963,1818.55428573)
\curveto(612.48040195,1818.55428573)(610.02598269,1819.58292829)(608.20307182,1821.64021341)
\curveto(606.39318175,1823.69749852)(605.48823672,1826.50347917)(605.48823672,1830.05815535)
\curveto(605.48823672,1833.61283154)(606.39318175,1836.41881218)(608.20307182,1838.4760973)
\curveto(610.02598269,1840.54640321)(612.48040195,1841.58155616)(615.56632963,1841.58155616)
\curveto(618.62621572,1841.58155616)(621.0676142,1840.54640321)(622.89052506,1838.4760973)
\curveto(624.71343592,1836.41881218)(625.62489135,1833.61283154)(625.62489135,1830.05815535)
\closepath
\moveto(621.83584092,1830.05815535)
\curveto(621.83584092,1832.88366719)(621.28245726,1834.98001468)(620.17568995,1836.34719783)
\curveto(619.06892264,1837.72740177)(617.5324692,1838.41750374)(615.56632963,1838.41750374)
\curveto(613.57414847,1838.41750374)(612.02467424,1837.72740177)(610.91790693,1836.34719783)
\curveto(609.82416041,1834.98001468)(609.27728715,1832.88366719)(609.27728715,1830.05815535)
\curveto(609.27728715,1827.32378906)(609.83067081,1825.24697275)(610.93743812,1823.82770644)
\curveto(612.04420543,1822.42146092)(613.58716926,1821.71833816)(615.56632963,1821.71833816)
\curveto(617.51944841,1821.71833816)(619.04939146,1822.41495052)(620.15615877,1823.80817525)
\curveto(621.27594687,1825.21442078)(621.83584092,1827.29774748)(621.83584092,1830.05815535)
\closepath
}
}
{
\newrgbcolor{curcolor}{0 0 0}
\pscustom[linestyle=none,fillstyle=solid,fillcolor=curcolor]
{
\newpath
\moveto(647.44122374,1820.5269357)
\curveto(646.2172693,1819.94100007)(645.05190843,1819.48527235)(643.94514112,1819.15975255)
\curveto(642.8513946,1818.83423276)(641.68603373,1818.67147286)(640.4490585,1818.67147286)
\curveto(638.87354268,1818.67147286)(637.42823479,1818.89933672)(636.11313481,1819.35506443)
\curveto(634.79803483,1819.82381294)(633.67173633,1820.5269357)(632.73423931,1821.46443271)
\curveto(631.78372151,1822.40192973)(631.04804677,1823.58682179)(630.52721509,1825.0191089)
\curveto(630.00638342,1826.451396)(629.74596758,1828.12456776)(629.74596758,1830.03862417)
\curveto(629.74596758,1833.60632114)(630.72252697,1836.40579139)(632.67564575,1838.43703493)
\curveto(634.64178532,1840.46827846)(637.23292291,1841.48390022)(640.4490585,1841.48390022)
\curveto(641.69905452,1841.48390022)(642.92300896,1841.30811953)(644.12092181,1840.95655815)
\curveto(645.33185545,1840.60499677)(646.43862276,1840.17531064)(647.44122374,1839.66749976)
\lineto(647.44122374,1835.5854815)
\lineto(647.24591186,1835.5854815)
\curveto(646.12612376,1836.45787456)(644.96727328,1837.12844534)(643.76936043,1837.59719385)
\curveto(642.58446837,1838.06594236)(641.42561789,1838.30031661)(640.292809,1838.30031661)
\curveto(638.2094823,1838.30031661)(636.56235213,1837.59719385)(635.35141848,1836.19094833)
\curveto(634.15350563,1834.7977236)(633.5545492,1832.74694888)(633.5545492,1830.03862417)
\curveto(633.5545492,1827.40842421)(634.14048484,1825.38369107)(635.31235611,1823.96442476)
\curveto(636.49724817,1822.55817923)(638.15739913,1821.85505647)(640.292809,1821.85505647)
\curveto(641.03499414,1821.85505647)(641.79020006,1821.95271241)(642.55842679,1822.14802429)
\curveto(643.32665351,1822.34333617)(644.01675548,1822.59724161)(644.62873269,1822.90974061)
\curveto(645.16258516,1823.18317724)(645.66388565,1823.46963466)(646.13263416,1823.76911288)
\curveto(646.60138266,1824.08161188)(646.97247523,1824.34853812)(647.24591186,1824.56989158)
\lineto(647.44122374,1824.56989158)
\closepath
}
}
{
\newrgbcolor{curcolor}{0 0 0}
\pscustom[linestyle=none,fillstyle=solid,fillcolor=curcolor]
{
\newpath
\moveto(669.1013149,1819.15975255)
\lineto(665.44898278,1819.15975255)
\lineto(665.44898278,1821.4839639)
\curveto(665.12346298,1821.26261044)(664.68075606,1820.95011144)(664.12086201,1820.54646689)
\curveto(663.57398875,1820.15584313)(663.04013628,1819.84334413)(662.51930461,1819.60896987)
\curveto(661.90732739,1819.30949166)(661.20420463,1819.06209661)(660.40993632,1818.86678474)
\curveto(659.61566802,1818.65845207)(658.6846814,1818.55428573)(657.61697647,1818.55428573)
\curveto(655.65083689,1818.55428573)(653.98417553,1819.20532532)(652.61699238,1820.50740451)
\curveto(651.24980924,1821.8094837)(650.56621766,1823.46963466)(650.56621766,1825.4878574)
\curveto(650.56621766,1827.14149797)(650.91777905,1828.47612914)(651.62090181,1829.49175091)
\curveto(652.33704536,1830.52039346)(653.35266713,1831.32768256)(654.66776711,1831.9136182)
\curveto(655.99588788,1832.49955383)(657.59093488,1832.89668798)(659.45290812,1833.10502065)
\curveto(661.31488136,1833.31335332)(663.31357291,1833.46960282)(665.44898278,1833.57376916)
\lineto(665.44898278,1834.14017361)
\curveto(665.44898278,1834.97350429)(665.29924367,1835.66360626)(664.99976546,1836.21047951)
\curveto(664.71330804,1836.75735277)(664.2966427,1837.18703891)(663.74976944,1837.49953791)
\curveto(663.22893776,1837.79901612)(662.60393975,1838.0008384)(661.87477541,1838.10500473)
\curveto(661.14561106,1838.20917107)(660.38389474,1838.26125424)(659.58962643,1838.26125424)
\curveto(658.62608784,1838.26125424)(657.55187251,1838.13104632)(656.36698045,1837.87063048)
\curveto(655.18208838,1837.62323543)(653.95813395,1837.25865326)(652.69511714,1836.77688396)
\lineto(652.49980526,1836.77688396)
\lineto(652.49980526,1840.50734083)
\curveto(653.21594881,1840.70265271)(654.25110177,1840.91749578)(655.60526412,1841.15187003)
\curveto(656.95942648,1841.38624429)(658.29405764,1841.50343141)(659.60915762,1841.50343141)
\curveto(661.14561106,1841.50343141)(662.48024223,1841.37322349)(663.61305112,1841.11280766)
\curveto(664.75888081,1840.86541261)(665.74846099,1840.43572648)(666.58179167,1839.82374926)
\curveto(667.40210156,1839.22479283)(668.02709957,1838.45005572)(668.4567857,1837.49953791)
\curveto(668.88647183,1836.5490201)(669.1013149,1835.37063844)(669.1013149,1833.96439292)
\closepath
\moveto(665.44898278,1824.5308292)
\lineto(665.44898278,1830.60502861)
\curveto(664.32919468,1830.53992465)(663.0075843,1830.44226871)(661.48415165,1830.31206079)
\curveto(659.97373979,1830.18185288)(658.77582694,1829.99305139)(657.89041309,1829.74565635)
\curveto(656.83572895,1829.44617813)(655.98286708,1828.97742963)(655.33182749,1828.33941083)
\curveto(654.6807879,1827.71441282)(654.3552681,1826.84853016)(654.3552681,1825.74176285)
\curveto(654.3552681,1824.49176683)(654.73287106,1823.54775942)(655.48807699,1822.90974061)
\curveto(656.24328292,1822.2847426)(657.395623,1821.9722436)(658.94509724,1821.9722436)
\curveto(660.23415563,1821.9722436)(661.4125373,1822.21963864)(662.48024223,1822.71442874)
\curveto(663.54794716,1823.22223962)(664.53752735,1823.82770644)(665.44898278,1824.5308292)
\closepath
}
}
{
\newrgbcolor{curcolor}{0 0 0}
\pscustom[linestyle=none,fillstyle=solid,fillcolor=curcolor]
{
\newpath
\moveto(679.88252056,1819.15975255)
\lineto(676.21065725,1819.15975255)
\lineto(676.21065725,1849.55028079)
\lineto(679.88252056,1849.55028079)
\closepath
}
}
{
\newrgbcolor{curcolor}{0 0 0}
\pscustom[linestyle=none,fillstyle=solid,fillcolor=curcolor]
{
\newpath
\moveto(437.25473581,1794.89324557)
\lineto(430.75474073,1794.89324557)
\lineto(430.75474073,1796.77866081)
\lineto(437.25473581,1796.77866081)
\closepath
}
}
{
\newrgbcolor{curcolor}{0 0 0}
\pscustom[linestyle=none,fillstyle=solid,fillcolor=curcolor]
{
\newpath
\moveto(458.2443035,1789.04949999)
\lineto(448.43181093,1789.04949999)
\lineto(448.43181093,1804.55990492)
\lineto(450.49430937,1804.55990492)
\lineto(450.49430937,1790.88283194)
\lineto(458.2443035,1790.88283194)
\closepath
}
}
{
\newrgbcolor{curcolor}{0 0 0}
\pscustom[linestyle=none,fillstyle=solid,fillcolor=curcolor]
{
\newpath
\moveto(470.06721,1794.86199559)
\curveto(470.06721,1792.9661637)(469.58109925,1791.46963705)(468.60887777,1790.37241566)
\curveto(467.63665628,1789.27519427)(466.33457393,1788.72658357)(464.70263072,1788.72658357)
\curveto(463.05679863,1788.72658357)(461.74777185,1789.27519427)(460.77555036,1790.37241566)
\curveto(459.81027331,1791.46963705)(459.32763479,1792.9661637)(459.32763479,1794.86199559)
\curveto(459.32763479,1796.75782749)(459.81027331,1798.25435414)(460.77555036,1799.35157553)
\curveto(461.74777185,1800.45574136)(463.05679863,1801.00782428)(464.70263072,1801.00782428)
\curveto(466.33457393,1801.00782428)(467.63665628,1800.45574136)(468.60887777,1799.35157553)
\curveto(469.58109925,1798.25435414)(470.06721,1796.75782749)(470.06721,1794.86199559)
\closepath
\moveto(468.04637819,1794.86199559)
\curveto(468.04637819,1796.3689389)(467.75123953,1797.48699361)(467.16096219,1798.21615972)
\curveto(466.57068486,1798.95227028)(465.75124104,1799.32032556)(464.70263072,1799.32032556)
\curveto(463.64013153,1799.32032556)(462.81374326,1798.95227028)(462.22346593,1798.21615972)
\curveto(461.64013304,1797.48699361)(461.34846659,1796.3689389)(461.34846659,1794.86199559)
\curveto(461.34846659,1793.40366336)(461.64360526,1792.29602531)(462.23388259,1791.53908144)
\curveto(462.82415992,1790.78908201)(463.64707596,1790.41408229)(464.70263072,1790.41408229)
\curveto(465.7442966,1790.41408229)(466.5602682,1790.78560979)(467.15054554,1791.52866478)
\curveto(467.74776731,1792.27866422)(468.04637819,1793.38977449)(468.04637819,1794.86199559)
\closepath
}
}
{
\newrgbcolor{curcolor}{0 0 0}
\pscustom[linestyle=none,fillstyle=solid,fillcolor=curcolor]
{
\newpath
\moveto(482.53595043,1790.37241566)
\curveto(482.53595043,1788.40019493)(482.0880341,1786.95227936)(481.19220144,1786.02866894)
\curveto(480.29636879,1785.10505853)(478.91789761,1784.64325333)(477.05678791,1784.64325333)
\curveto(476.43873282,1784.64325333)(475.83456661,1784.68839218)(475.24428928,1784.77866989)
\curveto(474.66095638,1784.86200316)(474.08456793,1784.98353085)(473.51512392,1785.14325295)
\lineto(473.51512392,1787.14325143)
\lineto(473.61929051,1787.14325143)
\curveto(473.93873471,1787.01825153)(474.44567877,1786.86547387)(475.14012269,1786.68491845)
\curveto(475.83456661,1786.49741859)(476.52901053,1786.40366866)(477.22345445,1786.40366866)
\curveto(477.89012061,1786.40366866)(478.44220352,1786.48352971)(478.87970319,1786.64325181)
\curveto(479.31720286,1786.80297391)(479.65748038,1787.02519597)(479.90053575,1787.30991798)
\curveto(480.14359113,1787.5807511)(480.31720211,1787.90713975)(480.42136869,1788.2890839)
\curveto(480.52553528,1788.67102806)(480.57761857,1789.09811107)(480.57761857,1789.57033293)
\lineto(480.57761857,1790.63283213)
\curveto(479.98734124,1790.16061026)(479.42136945,1789.80644386)(478.87970319,1789.57033293)
\curveto(478.34498137,1789.34116644)(477.66095411,1789.22658319)(476.82762141,1789.22658319)
\curveto(475.43873357,1789.22658319)(474.33456774,1789.72658281)(473.51512392,1790.72658206)
\curveto(472.70262453,1791.73352574)(472.29637484,1793.15019133)(472.29637484,1794.97657884)
\curveto(472.29637484,1795.97657808)(472.43526362,1796.83768854)(472.71304119,1797.55991022)
\curveto(472.9977632,1798.28907634)(473.38317957,1798.91754808)(473.86929032,1799.44532546)
\curveto(474.32067886,1799.93838064)(474.86928956,1800.3203248)(475.5151224,1800.59115793)
\curveto(476.16095525,1800.86893549)(476.80331587,1801.00782428)(477.44220428,1801.00782428)
\curveto(478.11581488,1801.00782428)(478.67831446,1800.93837989)(479.129703,1800.7994911)
\curveto(479.58803599,1800.66754676)(480.07067451,1800.4626858)(480.57761857,1800.18490823)
\lineto(480.70261848,1800.68490786)
\lineto(482.53595043,1800.68490786)
\closepath
\moveto(480.57761857,1792.24741424)
\lineto(480.57761857,1798.59115944)
\curveto(480.05678564,1798.82727037)(479.57067489,1798.99393691)(479.11928634,1799.09115906)
\curveto(478.67484224,1799.19532565)(478.23039813,1799.24740894)(477.78595402,1799.24740894)
\curveto(476.70956595,1799.24740894)(475.86234436,1798.88629811)(475.24428928,1798.16407643)
\curveto(474.62623419,1797.44185475)(474.31720664,1796.39324444)(474.31720664,1795.01824548)
\curveto(474.31720664,1793.71269091)(474.54637314,1792.72310832)(475.00470612,1792.04949772)
\curveto(475.46303911,1791.37588712)(476.2234552,1791.03908182)(477.2859544,1791.03908182)
\curveto(477.85539841,1791.03908182)(478.42484243,1791.14672063)(478.99428644,1791.36199824)
\curveto(479.57067489,1791.5842203)(480.09845227,1791.87935896)(480.57761857,1792.24741424)
\closepath
}
}
{
\newrgbcolor{curcolor}{0 0 0}
\pscustom[linestyle=none,fillstyle=solid,fillcolor=curcolor]
{
\newpath
\moveto(502.9317722,1789.04949999)
\lineto(500.98385701,1789.04949999)
\lineto(500.98385701,1790.28908239)
\curveto(500.81024603,1790.17102692)(500.5741351,1790.00436038)(500.27552421,1789.78908277)
\curveto(499.98385777,1789.58074959)(499.69913576,1789.41408305)(499.42135819,1789.28908314)
\curveto(499.09496955,1789.12936104)(498.71996984,1788.9974167)(498.29635905,1788.89325011)
\curveto(497.87274825,1788.78213908)(497.37622085,1788.72658357)(496.80677684,1788.72658357)
\curveto(495.75816652,1788.72658357)(494.8692783,1789.07380553)(494.14011219,1789.76824945)
\curveto(493.41094607,1790.46269337)(493.04636302,1791.34810936)(493.04636302,1792.42449744)
\curveto(493.04636302,1793.30644122)(493.23386288,1794.01824623)(493.60886259,1794.55991249)
\curveto(493.99080675,1795.10852319)(494.532473,1795.53907842)(495.23386136,1795.85157818)
\curveto(495.94219416,1796.16407794)(496.79288796,1796.37588334)(497.78594276,1796.48699437)
\curveto(498.77899757,1796.59810539)(499.84496898,1796.68143866)(500.98385701,1796.73699418)
\lineto(500.98385701,1797.03907728)
\curveto(500.98385701,1797.48352139)(500.90399596,1797.85157667)(500.74427386,1798.14324311)
\curveto(500.5914962,1798.43490956)(500.36927414,1798.66407605)(500.0776077,1798.83074259)
\curveto(499.79983013,1798.99046469)(499.46649705,1799.0981035)(499.07760845,1799.15365901)
\curveto(498.68871986,1799.20921453)(498.28247017,1799.23699228)(497.85885938,1799.23699228)
\curveto(497.34497088,1799.23699228)(496.77205464,1799.16754789)(496.14011068,1799.02865911)
\curveto(495.50816671,1798.89671476)(494.85538943,1798.70227047)(494.18177882,1798.44532622)
\lineto(494.07761224,1798.44532622)
\lineto(494.07761224,1800.43490805)
\curveto(494.45955639,1800.53907463)(495.01163931,1800.65365788)(495.73386098,1800.77865779)
\curveto(496.45608266,1800.90365769)(497.16788768,1800.96615764)(497.86927603,1800.96615764)
\curveto(498.68871986,1800.96615764)(499.40052488,1800.89671325)(500.00469109,1800.75782447)
\curveto(500.61580173,1800.62588012)(501.14357911,1800.39671363)(501.58802322,1800.07032499)
\curveto(502.02552289,1799.75088078)(502.35885597,1799.33768665)(502.58802246,1798.83074259)
\curveto(502.81718896,1798.32379853)(502.9317722,1797.69532678)(502.9317722,1796.94532735)
\closepath
\moveto(500.98385701,1791.91408116)
\lineto(500.98385701,1795.15366204)
\curveto(500.38663524,1795.11893984)(499.68177466,1795.06685655)(498.86927528,1794.99741216)
\curveto(498.06372033,1794.92796777)(497.42483193,1794.8272734)(496.95261006,1794.69532905)
\curveto(496.39011049,1794.53560695)(495.93524972,1794.28560714)(495.58802776,1793.94532962)
\curveto(495.2408058,1793.61199654)(495.06719482,1793.15019133)(495.06719482,1792.559914)
\curveto(495.06719482,1791.89324784)(495.26858356,1791.389776)(495.67136103,1791.04949848)
\curveto(496.0741385,1790.7161654)(496.68872137,1790.54949886)(497.51510964,1790.54949886)
\curveto(498.20260912,1790.54949886)(498.83108086,1790.6814432)(499.40052488,1790.94533189)
\curveto(499.96996889,1791.21616502)(500.49774627,1791.53908144)(500.98385701,1791.91408116)
\closepath
}
}
{
\newrgbcolor{curcolor}{0 0 0}
\pscustom[linestyle=none,fillstyle=solid,fillcolor=curcolor]
{
\newpath
\moveto(508.68176553,1789.04949999)
\lineto(506.72343368,1789.04949999)
\lineto(506.72343368,1805.25782106)
\lineto(508.68176553,1805.25782106)
\closepath
}
}
{
\newrgbcolor{curcolor}{0 0 0}
\pscustom[linestyle=none,fillstyle=solid,fillcolor=curcolor]
{
\newpath
\moveto(514.53592852,1789.04949999)
\lineto(512.57759666,1789.04949999)
\lineto(512.57759666,1805.25782106)
\lineto(514.53592852,1805.25782106)
\closepath
}
}
{
\newrgbcolor{curcolor}{0 0 0}
\pscustom[linestyle=none,fillstyle=solid,fillcolor=curcolor]
{
\newpath
\moveto(535.82757828,1794.86199559)
\curveto(535.82757828,1792.9661637)(535.34146754,1791.46963705)(534.36924605,1790.37241566)
\curveto(533.39702457,1789.27519427)(532.09494222,1788.72658357)(530.46299901,1788.72658357)
\curveto(528.81716692,1788.72658357)(527.50814013,1789.27519427)(526.53591865,1790.37241566)
\curveto(525.5706416,1791.46963705)(525.08800308,1792.9661637)(525.08800308,1794.86199559)
\curveto(525.08800308,1796.75782749)(525.5706416,1798.25435414)(526.53591865,1799.35157553)
\curveto(527.50814013,1800.45574136)(528.81716692,1801.00782428)(530.46299901,1801.00782428)
\curveto(532.09494222,1801.00782428)(533.39702457,1800.45574136)(534.36924605,1799.35157553)
\curveto(535.34146754,1798.25435414)(535.82757828,1796.75782749)(535.82757828,1794.86199559)
\closepath
\moveto(533.80674648,1794.86199559)
\curveto(533.80674648,1796.3689389)(533.51160781,1797.48699361)(532.92133048,1798.21615972)
\curveto(532.33105315,1798.95227028)(531.51160933,1799.32032556)(530.46299901,1799.32032556)
\curveto(529.40049981,1799.32032556)(528.57411155,1798.95227028)(527.98383422,1798.21615972)
\curveto(527.40050133,1797.48699361)(527.10883488,1796.3689389)(527.10883488,1794.86199559)
\curveto(527.10883488,1793.40366336)(527.40397355,1792.29602531)(527.99425088,1791.53908144)
\curveto(528.58452821,1790.78908201)(529.40744425,1790.41408229)(530.46299901,1790.41408229)
\curveto(531.50466489,1790.41408229)(532.32063649,1790.78560979)(532.91091382,1791.52866478)
\curveto(533.50813559,1792.27866422)(533.80674648,1793.38977449)(533.80674648,1794.86199559)
\closepath
}
}
{
\newrgbcolor{curcolor}{0 0 0}
\pscustom[linestyle=none,fillstyle=solid,fillcolor=curcolor]
{
\newpath
\moveto(548.50465189,1789.04949999)
\lineto(546.54632004,1789.04949999)
\lineto(546.54632004,1790.34116568)
\curveto(545.88659831,1789.82033274)(545.25465435,1789.42102749)(544.65048814,1789.14324992)
\curveto(544.04632193,1788.86547235)(543.37965577,1788.72658357)(542.65048965,1788.72658357)
\curveto(541.42826835,1788.72658357)(540.47688019,1789.09811107)(539.79632514,1789.84116606)
\curveto(539.1157701,1790.59116549)(538.77549258,1791.68838688)(538.77549258,1793.13283024)
\lineto(538.77549258,1800.68490786)
\lineto(540.73382444,1800.68490786)
\lineto(540.73382444,1794.05991287)
\curveto(540.73382444,1793.46963554)(540.76160219,1792.96269148)(540.81715771,1792.53908069)
\curveto(540.87271322,1792.12241433)(540.99076869,1791.76477572)(541.1713241,1791.46616483)
\curveto(541.35882396,1791.16060951)(541.60187933,1790.93838745)(541.90049022,1790.79949867)
\curveto(542.1991011,1790.66060988)(542.63312855,1790.59116549)(543.20257257,1790.59116549)
\curveto(543.70951663,1790.59116549)(544.26159954,1790.72310984)(544.85882131,1790.98699853)
\curveto(545.46298752,1791.25088722)(546.0254871,1791.58769252)(546.54632004,1791.99741443)
\lineto(546.54632004,1800.68490786)
\lineto(548.50465189,1800.68490786)
\closepath
}
}
{
\newrgbcolor{curcolor}{0 0 0}
\pscustom[linestyle=none,fillstyle=solid,fillcolor=curcolor]
{
\newpath
\moveto(558.4004793,1789.15366658)
\curveto(558.03242402,1789.05644443)(557.62964655,1788.97658338)(557.19214688,1788.91408343)
\curveto(556.76159165,1788.85158348)(556.37617527,1788.8203335)(556.03589775,1788.8203335)
\curveto(554.84839865,1788.8203335)(553.94562156,1789.1397777)(553.32756647,1789.77866611)
\curveto(552.70951138,1790.41755451)(552.40048384,1791.44185929)(552.40048384,1792.85158045)
\lineto(552.40048384,1799.03907577)
\lineto(551.07756817,1799.03907577)
\lineto(551.07756817,1800.68490786)
\lineto(552.40048384,1800.68490786)
\lineto(552.40048384,1804.02865533)
\lineto(554.35881569,1804.02865533)
\lineto(554.35881569,1800.68490786)
\lineto(558.4004793,1800.68490786)
\lineto(558.4004793,1799.03907577)
\lineto(554.35881569,1799.03907577)
\lineto(554.35881569,1793.73699645)
\curveto(554.35881569,1793.1258858)(554.37270457,1792.64671949)(554.40048232,1792.29949753)
\curveto(554.42826008,1791.95922001)(554.52548223,1791.63977581)(554.69214877,1791.34116493)
\curveto(554.84492643,1791.06338736)(555.05325961,1790.8585264)(555.3171483,1790.72658206)
\curveto(555.58798143,1790.60158215)(555.99770334,1790.5390822)(556.54631403,1790.5390822)
\curveto(556.86575824,1790.5390822)(557.19909132,1790.58422105)(557.54631328,1790.67449876)
\curveto(557.89353524,1790.77172091)(558.14353505,1790.85158196)(558.29631271,1790.91408191)
\lineto(558.4004793,1790.91408191)
\closepath
}
}
{
\newrgbcolor{curcolor}{0 0 0}
\pscustom[linestyle=none,fillstyle=solid,fillcolor=curcolor]
{
\newpath
\moveto(571.0046351,1795.00782882)
\curveto(571.0046351,1794.06338509)(570.86921854,1793.19880241)(570.59838541,1792.41408078)
\curveto(570.32755228,1791.63630359)(569.94560812,1790.97658187)(569.45255294,1790.43491561)
\curveto(568.99421995,1789.92102711)(568.4525537,1789.52172186)(567.82755417,1789.23699985)
\curveto(567.20949908,1788.95922228)(566.55324958,1788.8203335)(565.85880566,1788.8203335)
\curveto(565.25463945,1788.8203335)(564.70602875,1788.88630567)(564.21297357,1789.01825002)
\curveto(563.72686283,1789.15019436)(563.23033543,1789.35505532)(562.72339137,1789.63283288)
\lineto(562.72339137,1784.75783657)
\lineto(560.76505951,1784.75783657)
\lineto(560.76505951,1800.68490786)
\lineto(562.72339137,1800.68490786)
\lineto(562.72339137,1799.46615878)
\curveto(563.2442243,1799.90365845)(563.8275572,1800.2682415)(564.47339004,1800.55990795)
\curveto(565.12616733,1800.85851884)(565.82061124,1801.00782428)(566.5567218,1801.00782428)
\curveto(567.95949851,1801.00782428)(569.04977547,1800.47657468)(569.82755266,1799.41407548)
\curveto(570.61227429,1798.35852073)(571.0046351,1796.88977184)(571.0046351,1795.00782882)
\closepath
\moveto(568.9838033,1794.95574552)
\curveto(568.9838033,1796.35852224)(568.74422014,1797.40713256)(568.26505384,1798.10157648)
\curveto(567.78588754,1798.7960204)(567.04977698,1799.14324236)(566.05672218,1799.14324236)
\curveto(565.4942226,1799.14324236)(564.92825081,1799.02171467)(564.35880679,1798.7786593)
\curveto(563.78936278,1798.53560393)(563.2442243,1798.21615972)(562.72339137,1797.82032669)
\lineto(562.72339137,1791.22658168)
\curveto(563.2789465,1790.97658187)(563.75464059,1790.80644311)(564.15047362,1790.7161654)
\curveto(564.55325109,1790.62588769)(565.00811186,1790.58074883)(565.51505592,1790.58074883)
\curveto(566.60533287,1790.58074883)(567.45602667,1790.94880411)(568.06713732,1791.68491467)
\curveto(568.67824797,1792.42102522)(568.9838033,1793.51130217)(568.9838033,1794.95574552)
\closepath
}
}
{
\newrgbcolor{curcolor}{0 0 0}
\pscustom[linestyle=none,fillstyle=solid,fillcolor=curcolor]
{
\newpath
\moveto(583.70254036,1789.04949999)
\lineto(581.74420851,1789.04949999)
\lineto(581.74420851,1790.34116568)
\curveto(581.08448679,1789.82033274)(580.45254282,1789.42102749)(579.84837661,1789.14324992)
\curveto(579.2442104,1788.86547235)(578.57754424,1788.72658357)(577.84837812,1788.72658357)
\curveto(576.62615683,1788.72658357)(575.67476866,1789.09811107)(574.99421362,1789.84116606)
\curveto(574.31365858,1790.59116549)(573.97338106,1791.68838688)(573.97338106,1793.13283024)
\lineto(573.97338106,1800.68490786)
\lineto(575.93171291,1800.68490786)
\lineto(575.93171291,1794.05991287)
\curveto(575.93171291,1793.46963554)(575.95949066,1792.96269148)(576.01504618,1792.53908069)
\curveto(576.07060169,1792.12241433)(576.18865716,1791.76477572)(576.36921258,1791.46616483)
\curveto(576.55671243,1791.16060951)(576.79976781,1790.93838745)(577.09837869,1790.79949867)
\curveto(577.39698958,1790.66060988)(577.83101703,1790.59116549)(578.40046104,1790.59116549)
\curveto(578.9074051,1790.59116549)(579.45948802,1790.72310984)(580.05670979,1790.98699853)
\curveto(580.660876,1791.25088722)(581.22337557,1791.58769252)(581.74420851,1791.99741443)
\lineto(581.74420851,1800.68490786)
\lineto(583.70254036,1800.68490786)
\closepath
}
}
{
\newrgbcolor{curcolor}{0 0 0}
\pscustom[linestyle=none,fillstyle=solid,fillcolor=curcolor]
{
\newpath
\moveto(593.59836777,1789.15366658)
\curveto(593.23031249,1789.05644443)(592.82753502,1788.97658338)(592.39003535,1788.91408343)
\curveto(591.95948012,1788.85158348)(591.57406375,1788.8203335)(591.23378623,1788.8203335)
\curveto(590.04628712,1788.8203335)(589.14351003,1789.1397777)(588.52545494,1789.77866611)
\curveto(587.90739985,1790.41755451)(587.59837231,1791.44185929)(587.59837231,1792.85158045)
\lineto(587.59837231,1799.03907577)
\lineto(586.27545664,1799.03907577)
\lineto(586.27545664,1800.68490786)
\lineto(587.59837231,1800.68490786)
\lineto(587.59837231,1804.02865533)
\lineto(589.55670416,1804.02865533)
\lineto(589.55670416,1800.68490786)
\lineto(593.59836777,1800.68490786)
\lineto(593.59836777,1799.03907577)
\lineto(589.55670416,1799.03907577)
\lineto(589.55670416,1793.73699645)
\curveto(589.55670416,1793.1258858)(589.57059304,1792.64671949)(589.5983708,1792.29949753)
\curveto(589.62614855,1791.95922001)(589.7233707,1791.63977581)(589.89003724,1791.34116493)
\curveto(590.04281491,1791.06338736)(590.25114808,1790.8585264)(590.51503677,1790.72658206)
\curveto(590.7858699,1790.60158215)(591.19559181,1790.5390822)(591.74420251,1790.5390822)
\curveto(592.06364671,1790.5390822)(592.39697979,1790.58422105)(592.74420175,1790.67449876)
\curveto(593.09142371,1790.77172091)(593.34142352,1790.85158196)(593.49420118,1790.91408191)
\lineto(593.59836777,1790.91408191)
\closepath
}
}
{
\newrgbcolor{curcolor}{0 0 0}
\pscustom[linestyle=none,fillstyle=solid,fillcolor=curcolor]
{
\newpath
\moveto(609.50461008,1789.15366658)
\curveto(609.1365548,1789.05644443)(608.73377733,1788.97658338)(608.29627766,1788.91408343)
\curveto(607.86572243,1788.85158348)(607.48030606,1788.8203335)(607.14002854,1788.8203335)
\curveto(605.95252944,1788.8203335)(605.04975234,1789.1397777)(604.43169725,1789.77866611)
\curveto(603.81364216,1790.41755451)(603.50461462,1791.44185929)(603.50461462,1792.85158045)
\lineto(603.50461462,1799.03907577)
\lineto(602.18169895,1799.03907577)
\lineto(602.18169895,1800.68490786)
\lineto(603.50461462,1800.68490786)
\lineto(603.50461462,1804.02865533)
\lineto(605.46294647,1804.02865533)
\lineto(605.46294647,1800.68490786)
\lineto(609.50461008,1800.68490786)
\lineto(609.50461008,1799.03907577)
\lineto(605.46294647,1799.03907577)
\lineto(605.46294647,1793.73699645)
\curveto(605.46294647,1793.1258858)(605.47683535,1792.64671949)(605.50461311,1792.29949753)
\curveto(605.53239086,1791.95922001)(605.62961301,1791.63977581)(605.79627955,1791.34116493)
\curveto(605.94905722,1791.06338736)(606.15739039,1790.8585264)(606.42127908,1790.72658206)
\curveto(606.69211221,1790.60158215)(607.10183412,1790.5390822)(607.65044482,1790.5390822)
\curveto(607.96988902,1790.5390822)(608.3032221,1790.58422105)(608.65044406,1790.67449876)
\curveto(608.99766602,1790.77172091)(609.24766583,1790.85158196)(609.40044349,1790.91408191)
\lineto(609.50461008,1790.91408191)
\closepath
}
}
{
\newrgbcolor{curcolor}{0 0 0}
\pscustom[linestyle=none,fillstyle=solid,fillcolor=curcolor]
{
\newpath
\moveto(621.78584946,1794.86199559)
\curveto(621.78584946,1792.9661637)(621.29973872,1791.46963705)(620.32751723,1790.37241566)
\curveto(619.35529574,1789.27519427)(618.0532134,1788.72658357)(616.42127019,1788.72658357)
\curveto(614.7754381,1788.72658357)(613.46641131,1789.27519427)(612.49418982,1790.37241566)
\curveto(611.52891278,1791.46963705)(611.04627425,1792.9661637)(611.04627425,1794.86199559)
\curveto(611.04627425,1796.75782749)(611.52891278,1798.25435414)(612.49418982,1799.35157553)
\curveto(613.46641131,1800.45574136)(614.7754381,1801.00782428)(616.42127019,1801.00782428)
\curveto(618.0532134,1801.00782428)(619.35529574,1800.45574136)(620.32751723,1799.35157553)
\curveto(621.29973872,1798.25435414)(621.78584946,1796.75782749)(621.78584946,1794.86199559)
\closepath
\moveto(619.76501766,1794.86199559)
\curveto(619.76501766,1796.3689389)(619.46987899,1797.48699361)(618.87960166,1798.21615972)
\curveto(618.28932433,1798.95227028)(617.4698805,1799.32032556)(616.42127019,1799.32032556)
\curveto(615.35877099,1799.32032556)(614.53238273,1798.95227028)(613.94210539,1798.21615972)
\curveto(613.3587725,1797.48699361)(613.06710606,1796.3689389)(613.06710606,1794.86199559)
\curveto(613.06710606,1793.40366336)(613.36224472,1792.29602531)(613.95252205,1791.53908144)
\curveto(614.54279938,1790.78908201)(615.36571543,1790.41408229)(616.42127019,1790.41408229)
\curveto(617.46293606,1790.41408229)(618.27890767,1790.78560979)(618.869185,1791.52866478)
\curveto(619.46640677,1792.27866422)(619.76501766,1793.38977449)(619.76501766,1794.86199559)
\closepath
}
}
{
\newrgbcolor{curcolor}{0 0 0}
\pscustom[linestyle=none,fillstyle=solid,fillcolor=curcolor]
{
\newpath
\moveto(634.29624778,1789.04949999)
\lineto(632.33791593,1789.04949999)
\lineto(632.33791593,1805.25782106)
\lineto(634.29624778,1805.25782106)
\closepath
}
}
{
\newrgbcolor{curcolor}{0 0 0}
\pscustom[linestyle=none,fillstyle=solid,fillcolor=curcolor]
{
\newpath
\moveto(648.08791197,1794.86199559)
\curveto(648.08791197,1792.9661637)(647.60180122,1791.46963705)(646.62957974,1790.37241566)
\curveto(645.65735825,1789.27519427)(644.3552759,1788.72658357)(642.72333269,1788.72658357)
\curveto(641.0775006,1788.72658357)(639.76847382,1789.27519427)(638.79625233,1790.37241566)
\curveto(637.83097528,1791.46963705)(637.34833676,1792.9661637)(637.34833676,1794.86199559)
\curveto(637.34833676,1796.75782749)(637.83097528,1798.25435414)(638.79625233,1799.35157553)
\curveto(639.76847382,1800.45574136)(641.0775006,1801.00782428)(642.72333269,1801.00782428)
\curveto(644.3552759,1801.00782428)(645.65735825,1800.45574136)(646.62957974,1799.35157553)
\curveto(647.60180122,1798.25435414)(648.08791197,1796.75782749)(648.08791197,1794.86199559)
\closepath
\moveto(646.06708016,1794.86199559)
\curveto(646.06708016,1796.3689389)(645.7719415,1797.48699361)(645.18166417,1798.21615972)
\curveto(644.59138684,1798.95227028)(643.77194301,1799.32032556)(642.72333269,1799.32032556)
\curveto(641.6608335,1799.32032556)(640.83444523,1798.95227028)(640.2441679,1798.21615972)
\curveto(639.66083501,1797.48699361)(639.36916856,1796.3689389)(639.36916856,1794.86199559)
\curveto(639.36916856,1793.40366336)(639.66430723,1792.29602531)(640.25458456,1791.53908144)
\curveto(640.84486189,1790.78908201)(641.66777794,1790.41408229)(642.72333269,1790.41408229)
\curveto(643.76499857,1790.41408229)(644.58097018,1790.78560979)(645.17124751,1791.52866478)
\curveto(645.76846928,1792.27866422)(646.06708016,1793.38977449)(646.06708016,1794.86199559)
\closepath
}
}
{
\newrgbcolor{curcolor}{0 0 0}
\pscustom[linestyle=none,fillstyle=solid,fillcolor=curcolor]
{
\newpath
\moveto(660.5566524,1790.37241566)
\curveto(660.5566524,1788.40019493)(660.10873607,1786.95227936)(659.21290341,1786.02866894)
\curveto(658.31707076,1785.10505853)(656.93859958,1784.64325333)(655.07748988,1784.64325333)
\curveto(654.45943479,1784.64325333)(653.85526858,1784.68839218)(653.26499125,1784.77866989)
\curveto(652.68165836,1784.86200316)(652.1052699,1784.98353085)(651.53582589,1785.14325295)
\lineto(651.53582589,1787.14325143)
\lineto(651.63999248,1787.14325143)
\curveto(651.95943668,1787.01825153)(652.46638074,1786.86547387)(653.16082466,1786.68491845)
\curveto(653.85526858,1786.49741859)(654.5497125,1786.40366866)(655.24415642,1786.40366866)
\curveto(655.91082258,1786.40366866)(656.4629055,1786.48352971)(656.90040516,1786.64325181)
\curveto(657.33790483,1786.80297391)(657.67818235,1787.02519597)(657.92123773,1787.30991798)
\curveto(658.1642931,1787.5807511)(658.33790408,1787.90713975)(658.44207066,1788.2890839)
\curveto(658.54623725,1788.67102806)(658.59832055,1789.09811107)(658.59832055,1789.57033293)
\lineto(658.59832055,1790.63283213)
\curveto(658.00804322,1790.16061026)(657.44207142,1789.80644386)(656.90040516,1789.57033293)
\curveto(656.36568335,1789.34116644)(655.68165609,1789.22658319)(654.84832338,1789.22658319)
\curveto(653.45943555,1789.22658319)(652.35526971,1789.72658281)(651.53582589,1790.72658206)
\curveto(650.7233265,1791.73352574)(650.31707681,1793.15019133)(650.31707681,1794.97657884)
\curveto(650.31707681,1795.97657808)(650.4559656,1796.83768854)(650.73374316,1797.55991022)
\curveto(651.01846517,1798.28907634)(651.40388154,1798.91754808)(651.88999229,1799.44532546)
\curveto(652.34138084,1799.93838064)(652.88999153,1800.3203248)(653.53582438,1800.59115793)
\curveto(654.18165722,1800.86893549)(654.82401785,1801.00782428)(655.46290625,1801.00782428)
\curveto(656.13651685,1801.00782428)(656.69901643,1800.93837989)(657.15040498,1800.7994911)
\curveto(657.60873796,1800.66754676)(658.09137649,1800.4626858)(658.59832055,1800.18490823)
\lineto(658.72332045,1800.68490786)
\lineto(660.5566524,1800.68490786)
\closepath
\moveto(658.59832055,1792.24741424)
\lineto(658.59832055,1798.59115944)
\curveto(658.07748761,1798.82727037)(657.59137686,1798.99393691)(657.13998832,1799.09115906)
\curveto(656.69554421,1799.19532565)(656.2511001,1799.24740894)(655.80665599,1799.24740894)
\curveto(654.73026792,1799.24740894)(653.88304634,1798.88629811)(653.26499125,1798.16407643)
\curveto(652.64693616,1797.44185475)(652.33790862,1796.39324444)(652.33790862,1795.01824548)
\curveto(652.33790862,1793.71269091)(652.56707511,1792.72310832)(653.0254081,1792.04949772)
\curveto(653.48374108,1791.37588712)(654.24415717,1791.03908182)(655.30665637,1791.03908182)
\curveto(655.87610038,1791.03908182)(656.4455444,1791.14672063)(657.01498841,1791.36199824)
\curveto(657.59137686,1791.5842203)(658.11915424,1791.87935896)(658.59832055,1792.24741424)
\closepath
}
}
{
\newrgbcolor{curcolor}{0 0 0}
\pscustom[linestyle=none,fillstyle=solid,fillcolor=curcolor]
{
\newpath
\moveto(678.17122628,1803.36198916)
\lineto(678.06705969,1803.36198916)
\curveto(677.85178208,1803.42448912)(677.57053229,1803.48698907)(677.22331033,1803.54948902)
\curveto(676.87608837,1803.61893341)(676.57053305,1803.65365561)(676.30664436,1803.65365561)
\curveto(675.46636722,1803.65365561)(674.85525657,1803.46615575)(674.47331241,1803.09115604)
\curveto(674.0983127,1802.72310076)(673.91081284,1802.05296238)(673.91081284,1801.08074089)
\lineto(673.91081284,1800.68490786)
\lineto(677.44206017,1800.68490786)
\lineto(677.44206017,1799.03907577)
\lineto(673.97331279,1799.03907577)
\lineto(673.97331279,1789.04949999)
\lineto(672.01498094,1789.04949999)
\lineto(672.01498094,1799.03907577)
\lineto(670.69206527,1799.03907577)
\lineto(670.69206527,1800.68490786)
\lineto(672.01498094,1800.68490786)
\lineto(672.01498094,1801.07032423)
\curveto(672.01498094,1802.45226763)(672.35873068,1803.51129461)(673.04623016,1804.24740516)
\curveto(673.73372964,1804.99046015)(674.72678444,1805.36198765)(676.02539457,1805.36198765)
\curveto(676.46289424,1805.36198765)(676.85525505,1805.34115433)(677.20247701,1805.2994877)
\curveto(677.55664341,1805.25782106)(677.87955983,1805.20920999)(678.17122628,1805.15365448)
\closepath
}
}
{
\newrgbcolor{curcolor}{0 0 0}
\pscustom[linestyle=none,fillstyle=solid,fillcolor=curcolor]
{
\newpath
\moveto(681.51495059,1802.63282305)
\lineto(679.30661893,1802.63282305)
\lineto(679.30661893,1804.66407151)
\lineto(681.51495059,1804.66407151)
\closepath
\moveto(681.38995068,1789.04949999)
\lineto(679.43161883,1789.04949999)
\lineto(679.43161883,1800.68490786)
\lineto(681.38995068,1800.68490786)
\closepath
}
}
{
\newrgbcolor{curcolor}{0 0 0}
\pscustom[linestyle=none,fillstyle=solid,fillcolor=curcolor]
{
\newpath
\moveto(687.24413529,1789.04949999)
\lineto(685.28580344,1789.04949999)
\lineto(685.28580344,1805.25782106)
\lineto(687.24413529,1805.25782106)
\closepath
}
}
{
\newrgbcolor{curcolor}{0 0 0}
\pscustom[linestyle=none,fillstyle=solid,fillcolor=curcolor]
{
\newpath
\moveto(700.85870186,1794.66407908)
\lineto(692.28579168,1794.66407908)
\curveto(692.28579168,1793.94880184)(692.39343049,1793.32380231)(692.6087081,1792.7890805)
\curveto(692.82398572,1792.26130312)(693.11912438,1791.82727567)(693.4941241,1791.48699815)
\curveto(693.85523494,1791.15366507)(694.28231795,1790.90366526)(694.77537313,1790.73699872)
\curveto(695.27537275,1790.57033218)(695.82398345,1790.4869989)(696.42120522,1790.4869989)
\curveto(697.21287129,1790.4869989)(698.00800957,1790.64324879)(698.80662008,1790.95574855)
\curveto(699.61217503,1791.27519275)(700.18509126,1791.58769252)(700.52536878,1791.89324784)
\lineto(700.62953537,1791.89324784)
\lineto(700.62953537,1789.75783279)
\curveto(699.96981365,1789.48005522)(699.29620304,1789.24741651)(698.60870356,1789.05991665)
\curveto(697.92120408,1788.87241679)(697.19898241,1788.77866686)(696.44203854,1788.77866686)
\curveto(694.51148444,1788.77866686)(693.00454114,1789.2994998)(691.92120862,1790.34116568)
\curveto(690.83787611,1791.389776)(690.29620985,1792.87588599)(690.29620985,1794.79949564)
\curveto(690.29620985,1796.70227198)(690.81357057,1798.2126875)(691.84829201,1799.33074221)
\curveto(692.88995789,1800.44879692)(694.25801241,1801.00782428)(695.95245557,1801.00782428)
\curveto(697.52189883,1801.00782428)(698.73023125,1800.54949129)(699.57745283,1799.63282532)
\curveto(700.43161885,1798.71615935)(700.85870186,1797.414077)(700.85870186,1795.72657827)
\closepath
\moveto(698.9524533,1796.16407794)
\curveto(698.94550886,1797.19185494)(698.68509239,1797.98699323)(698.17120389,1798.54949281)
\curveto(697.66425983,1799.11199238)(696.88995486,1799.39324217)(695.84828899,1799.39324217)
\curveto(694.79967867,1799.39324217)(693.96287375,1799.08421462)(693.33787422,1798.46615953)
\curveto(692.71981913,1797.84810445)(692.36912495,1797.08074392)(692.28579168,1796.16407794)
\closepath
}
}
{
\newrgbcolor{curcolor}{0 0 0}
\pscustom[linestyle=none,fillstyle=solid,fillcolor=curcolor]
{
\newpath
\moveto(437.25473581,1768.22659908)
\lineto(430.75474073,1768.22659908)
\lineto(430.75474073,1770.11201432)
\lineto(437.25473581,1770.11201432)
\closepath
}
}
{
\newrgbcolor{curcolor}{0 0 0}
\pscustom[linestyle=none,fillstyle=solid,fillcolor=curcolor]
{
\newpath
\moveto(462.24430048,1762.3828535)
\lineto(460.18180204,1762.3828535)
\lineto(460.18180204,1775.74742672)
\lineto(455.8693053,1766.6536836)
\lineto(454.64013956,1766.6536836)
\lineto(450.3588928,1775.74742672)
\lineto(450.3588928,1762.3828535)
\lineto(448.43181093,1762.3828535)
\lineto(448.43181093,1777.89325843)
\lineto(451.2443088,1777.89325843)
\lineto(455.37972234,1769.2578483)
\lineto(459.37971931,1777.89325843)
\lineto(462.24430048,1777.89325843)
\closepath
}
}
{
\newrgbcolor{curcolor}{0 0 0}
\pscustom[linestyle=none,fillstyle=solid,fillcolor=curcolor]
{
\newpath
\moveto(475.99429002,1767.99743259)
\lineto(467.42137984,1767.99743259)
\curveto(467.42137984,1767.28215535)(467.52901865,1766.65715582)(467.74429626,1766.122434)
\curveto(467.95957388,1765.59465663)(468.25471254,1765.16062918)(468.62971226,1764.82035166)
\curveto(468.9908231,1764.48701858)(469.41790611,1764.23701876)(469.91096129,1764.07035222)
\curveto(470.41096091,1763.90368568)(470.95957161,1763.82035241)(471.55679338,1763.82035241)
\curveto(472.34845944,1763.82035241)(473.14359773,1763.97660229)(473.94220824,1764.28910206)
\curveto(474.74776318,1764.60854626)(475.32067942,1764.92104602)(475.66095694,1765.22660135)
\lineto(475.76512353,1765.22660135)
\lineto(475.76512353,1763.0911863)
\curveto(475.1054018,1762.81340873)(474.4317912,1762.58077002)(473.74429172,1762.39327016)
\curveto(473.05679224,1762.2057703)(472.33457057,1762.11202037)(471.57762669,1762.11202037)
\curveto(469.6470726,1762.11202037)(468.14012929,1762.63285331)(467.05679678,1763.67451919)
\curveto(465.97346427,1764.72312951)(465.43179801,1766.20923949)(465.43179801,1768.13284915)
\curveto(465.43179801,1770.03562549)(465.94915873,1771.54604101)(466.98388017,1772.66409572)
\curveto(468.02554605,1773.78215043)(469.39360057,1774.34117779)(471.08804373,1774.34117779)
\curveto(472.65748699,1774.34117779)(473.86581941,1773.8828448)(474.71304099,1772.96617883)
\curveto(475.56720701,1772.04951285)(475.99429002,1770.74743051)(475.99429002,1769.05993178)
\closepath
\moveto(474.08804146,1769.49743145)
\curveto(474.08109702,1770.52520845)(473.82068055,1771.32034674)(473.30679205,1771.88284631)
\curveto(472.79984799,1772.44534589)(472.02554302,1772.72659567)(470.98387714,1772.72659567)
\curveto(469.93526683,1772.72659567)(469.0984619,1772.41756813)(468.47346238,1771.79951304)
\curveto(467.85540729,1771.18145795)(467.50471311,1770.41409742)(467.42137984,1769.49743145)
\closepath
}
}
{
\newrgbcolor{curcolor}{0 0 0}
\pscustom[linestyle=none,fillstyle=solid,fillcolor=curcolor]
{
\newpath
\moveto(487.27553109,1765.73701763)
\curveto(487.27553109,1764.67451843)(486.8345592,1763.80299131)(485.95261542,1763.12243627)
\curveto(485.07761608,1762.44188123)(483.87970032,1762.10160371)(482.35886814,1762.10160371)
\curveto(481.49775768,1762.10160371)(480.70609161,1762.20229808)(479.98386994,1762.40368682)
\curveto(479.2685927,1762.61201999)(478.66789871,1762.83771427)(478.18178797,1763.08076964)
\lineto(478.18178797,1765.27868464)
\lineto(478.28595455,1765.27868464)
\curveto(478.90400964,1764.81340722)(479.59150912,1764.44187972)(480.34845299,1764.16410215)
\curveto(481.10539687,1763.89326902)(481.83109076,1763.75785246)(482.52553468,1763.75785246)
\curveto(483.38664514,1763.75785246)(484.06025574,1763.89674124)(484.54636648,1764.17451881)
\curveto(485.03247723,1764.45229638)(485.2755326,1764.88979605)(485.2755326,1765.48701782)
\curveto(485.2755326,1765.94535081)(485.14358826,1766.29257276)(484.87969957,1766.5286837)
\curveto(484.61581088,1766.76479463)(484.10886682,1766.96618337)(483.35886738,1767.13284991)
\curveto(483.08108982,1767.19534986)(482.71650676,1767.26826647)(482.26511821,1767.35159974)
\curveto(481.8206741,1767.43493301)(481.41442441,1767.52521072)(481.04636913,1767.62243287)
\curveto(480.02553657,1767.893266)(479.29984268,1768.28909903)(478.86928745,1768.80993197)
\curveto(478.44567666,1769.33770935)(478.23387126,1769.98354219)(478.23387126,1770.74743051)
\curveto(478.23387126,1771.22659681)(478.33109341,1771.67798536)(478.52553771,1772.10159615)
\curveto(478.72692644,1772.52520694)(479.02900955,1772.90367887)(479.43178702,1773.23701196)
\curveto(479.82067562,1773.5634006)(480.3137308,1773.82034485)(480.91095257,1774.00784471)
\curveto(481.51511878,1774.202289)(482.18872938,1774.29951115)(482.93178437,1774.29951115)
\curveto(483.62622829,1774.29951115)(484.32761665,1774.21270566)(485.03594945,1774.03909468)
\curveto(485.75122668,1773.87242814)(486.34497624,1773.66756719)(486.8171981,1773.42451181)
\lineto(486.8171981,1771.3307634)
\lineto(486.71303151,1771.3307634)
\curveto(486.21303189,1771.69881867)(485.60539346,1772.00784622)(484.89011622,1772.25784603)
\curveto(484.17483899,1772.51479028)(483.47345063,1772.6432624)(482.78595115,1772.6432624)
\curveto(482.07067391,1772.6432624)(481.4665077,1772.50437362)(480.97345252,1772.22659605)
\curveto(480.48039734,1771.95576292)(480.23386975,1771.54951323)(480.23386975,1771.00784698)
\curveto(480.23386975,1770.52868067)(480.38317519,1770.16756983)(480.68178608,1769.92451446)
\curveto(480.97345252,1769.68145909)(481.44567439,1769.48354257)(482.09845167,1769.33076491)
\curveto(482.45956251,1769.24743164)(482.86233998,1769.16409837)(483.30678409,1769.0807651)
\curveto(483.75817264,1768.99743183)(484.13317235,1768.921043)(484.43178324,1768.85159861)
\curveto(485.34150477,1768.64326543)(486.04289313,1768.28562681)(486.53594831,1767.77868275)
\curveto(487.0290035,1767.26479425)(487.27553109,1766.58423921)(487.27553109,1765.73701763)
\closepath
}
}
{
\newrgbcolor{curcolor}{0 0 0}
\pscustom[linestyle=none,fillstyle=solid,fillcolor=curcolor]
{
\newpath
\moveto(498.39010513,1765.73701763)
\curveto(498.39010513,1764.67451843)(497.94913324,1763.80299131)(497.06718946,1763.12243627)
\curveto(496.19219012,1762.44188123)(494.99427436,1762.10160371)(493.47344218,1762.10160371)
\curveto(492.61233172,1762.10160371)(491.82066565,1762.20229808)(491.09844398,1762.40368682)
\curveto(490.38316674,1762.61201999)(489.78247275,1762.83771427)(489.29636201,1763.08076964)
\lineto(489.29636201,1765.27868464)
\lineto(489.4005286,1765.27868464)
\curveto(490.01858368,1764.81340722)(490.70608316,1764.44187972)(491.46302704,1764.16410215)
\curveto(492.21997091,1763.89326902)(492.9456648,1763.75785246)(493.64010872,1763.75785246)
\curveto(494.50121918,1763.75785246)(495.17482978,1763.89674124)(495.66094053,1764.17451881)
\curveto(496.14705127,1764.45229638)(496.39010664,1764.88979605)(496.39010664,1765.48701782)
\curveto(496.39010664,1765.94535081)(496.2581623,1766.29257276)(495.99427361,1766.5286837)
\curveto(495.73038492,1766.76479463)(495.22344086,1766.96618337)(494.47344143,1767.13284991)
\curveto(494.19566386,1767.19534986)(493.8310808,1767.26826647)(493.37969225,1767.35159974)
\curveto(492.93524814,1767.43493301)(492.52899845,1767.52521072)(492.16094317,1767.62243287)
\curveto(491.14011061,1767.893266)(490.41441672,1768.28909903)(489.98386149,1768.80993197)
\curveto(489.5602507,1769.33770935)(489.3484453,1769.98354219)(489.3484453,1770.74743051)
\curveto(489.3484453,1771.22659681)(489.44566745,1771.67798536)(489.64011175,1772.10159615)
\curveto(489.84150048,1772.52520694)(490.14358359,1772.90367887)(490.54636106,1773.23701196)
\curveto(490.93524966,1773.5634006)(491.42830484,1773.82034485)(492.02552661,1774.00784471)
\curveto(492.62969282,1774.202289)(493.30330342,1774.29951115)(494.04635841,1774.29951115)
\curveto(494.74080233,1774.29951115)(495.44219069,1774.21270566)(496.15052349,1774.03909468)
\curveto(496.86580073,1773.87242814)(497.45955028,1773.66756719)(497.93177214,1773.42451181)
\lineto(497.93177214,1771.3307634)
\lineto(497.82760555,1771.3307634)
\curveto(497.32760593,1771.69881867)(496.7199675,1772.00784622)(496.00469027,1772.25784603)
\curveto(495.28941303,1772.51479028)(494.58802467,1772.6432624)(493.90052519,1772.6432624)
\curveto(493.18524796,1772.6432624)(492.58108175,1772.50437362)(492.08802656,1772.22659605)
\curveto(491.59497138,1771.95576292)(491.34844379,1771.54951323)(491.34844379,1771.00784698)
\curveto(491.34844379,1770.52868067)(491.49774923,1770.16756983)(491.79636012,1769.92451446)
\curveto(492.08802656,1769.68145909)(492.56024843,1769.48354257)(493.21302571,1769.33076491)
\curveto(493.57413655,1769.24743164)(493.97691402,1769.16409837)(494.42135813,1769.0807651)
\curveto(494.87274668,1768.99743183)(495.24774639,1768.921043)(495.54635728,1768.85159861)
\curveto(496.45607881,1768.64326543)(497.15746717,1768.28562681)(497.65052235,1767.77868275)
\curveto(498.14357754,1767.26479425)(498.39010513,1766.58423921)(498.39010513,1765.73701763)
\closepath
}
}
{
\newrgbcolor{curcolor}{0 0 0}
\pscustom[linestyle=none,fillstyle=solid,fillcolor=curcolor]
{
\newpath
\moveto(510.23384529,1762.3828535)
\lineto(508.28593009,1762.3828535)
\lineto(508.28593009,1763.6224359)
\curveto(508.11231911,1763.50438043)(507.87620818,1763.33771389)(507.5775973,1763.12243627)
\curveto(507.28593085,1762.9141031)(507.00120884,1762.74743656)(506.72343127,1762.62243665)
\curveto(506.39704263,1762.46271455)(506.02204292,1762.33077021)(505.59843213,1762.22660362)
\curveto(505.17482134,1762.11549259)(504.67829393,1762.05993708)(504.10884992,1762.05993708)
\curveto(503.0602396,1762.05993708)(502.17135139,1762.40715904)(501.44218527,1763.10160296)
\curveto(500.71301916,1763.79604688)(500.3484361,1764.68146287)(500.3484361,1765.75785095)
\curveto(500.3484361,1766.63979472)(500.53593596,1767.35159974)(500.91093567,1767.893266)
\curveto(501.29287983,1768.44187669)(501.83454608,1768.87243192)(502.53593444,1769.18493169)
\curveto(503.24426724,1769.49743145)(504.09496104,1769.70923685)(505.08801585,1769.82034787)
\curveto(506.08107065,1769.9314589)(507.14704207,1770.01479217)(508.28593009,1770.07034768)
\lineto(508.28593009,1770.37243079)
\curveto(508.28593009,1770.8168749)(508.20606904,1771.18493017)(508.04634694,1771.47659662)
\curveto(507.89356928,1771.76826307)(507.67134722,1771.99742956)(507.37968078,1772.1640961)
\curveto(507.10190321,1772.3238182)(506.76857013,1772.43145701)(506.37968153,1772.48701252)
\curveto(505.99079294,1772.54256804)(505.58454325,1772.57034579)(505.16093246,1772.57034579)
\curveto(504.64704396,1772.57034579)(504.07412772,1772.5009014)(503.44218376,1772.36201262)
\curveto(502.81023979,1772.23006827)(502.15746251,1772.03562398)(501.48385191,1771.77867973)
\lineto(501.37968532,1771.77867973)
\lineto(501.37968532,1773.76826155)
\curveto(501.76162947,1773.87242814)(502.31371239,1773.98701139)(503.03593406,1774.11201129)
\curveto(503.75815574,1774.2370112)(504.46996076,1774.29951115)(505.17134912,1774.29951115)
\curveto(505.99079294,1774.29951115)(506.70259796,1774.23006676)(507.30676417,1774.09117798)
\curveto(507.91787482,1773.95923363)(508.44565219,1773.73006714)(508.8900963,1773.4036785)
\curveto(509.32759597,1773.08423429)(509.66092905,1772.67104016)(509.89009555,1772.1640961)
\curveto(510.11926204,1771.65715204)(510.23384529,1771.02868029)(510.23384529,1770.27868086)
\closepath
\moveto(508.28593009,1765.24743467)
\lineto(508.28593009,1768.48701555)
\curveto(507.68870832,1768.45229335)(506.98384774,1768.40021006)(506.17134836,1768.33076567)
\curveto(505.36579341,1768.26132128)(504.72690501,1768.16062691)(504.25468314,1768.02868256)
\curveto(503.69218357,1767.86896046)(503.2373228,1767.61896065)(502.89010084,1767.27868313)
\curveto(502.54287888,1766.94535005)(502.3692679,1766.48354484)(502.3692679,1765.89326751)
\curveto(502.3692679,1765.22660135)(502.57065664,1764.72312951)(502.97343411,1764.38285199)
\curveto(503.37621158,1764.04951891)(503.99079445,1763.88285237)(504.81718272,1763.88285237)
\curveto(505.5046822,1763.88285237)(506.13315394,1764.01479671)(506.70259796,1764.2786854)
\curveto(507.27204197,1764.54951853)(507.79981935,1764.87243495)(508.28593009,1765.24743467)
\closepath
}
}
{
\newrgbcolor{curcolor}{0 0 0}
\pscustom[linestyle=none,fillstyle=solid,fillcolor=curcolor]
{
\newpath
\moveto(523.44217351,1763.70576917)
\curveto(523.44217351,1761.73354844)(522.99425719,1760.28563286)(522.09842453,1759.36202245)
\curveto(521.20259187,1758.43841204)(519.8241207,1757.97660683)(517.96301099,1757.97660683)
\curveto(517.3449559,1757.97660683)(516.74078969,1758.02174569)(516.15051236,1758.1120234)
\curveto(515.56717947,1758.19535667)(514.99079102,1758.31688435)(514.421347,1758.47660646)
\lineto(514.421347,1760.47660494)
\lineto(514.52551359,1760.47660494)
\curveto(514.8449578,1760.35160504)(515.35190186,1760.19882738)(516.04634578,1760.01827196)
\curveto(516.74078969,1759.8307721)(517.43523361,1759.73702217)(518.12967753,1759.73702217)
\curveto(518.79634369,1759.73702217)(519.34842661,1759.81688322)(519.78592628,1759.97660532)
\curveto(520.22342595,1760.13632742)(520.56370347,1760.35854948)(520.80675884,1760.64327148)
\curveto(521.04981421,1760.91410461)(521.22342519,1761.24049325)(521.32759178,1761.62243741)
\curveto(521.43175837,1762.00438156)(521.48384166,1762.43146457)(521.48384166,1762.90368644)
\lineto(521.48384166,1763.96618564)
\curveto(520.89356433,1763.49396377)(520.32759254,1763.13979737)(519.78592628,1762.90368644)
\curveto(519.25120446,1762.67451995)(518.5671772,1762.5599367)(517.7338445,1762.5599367)
\curveto(516.34495666,1762.5599367)(515.24079083,1763.05993632)(514.421347,1764.05993557)
\curveto(513.60884762,1765.06687925)(513.20259793,1766.48354484)(513.20259793,1768.30993235)
\curveto(513.20259793,1769.30993159)(513.34148671,1770.17104205)(513.61926428,1770.89326373)
\curveto(513.90398629,1771.62242984)(514.28940266,1772.25090159)(514.7755134,1772.77867897)
\curveto(515.22690195,1773.27173415)(515.77551265,1773.65367831)(516.42134549,1773.92451144)
\curveto(517.06717834,1774.202289)(517.70953896,1774.34117779)(518.34842737,1774.34117779)
\curveto(519.02203797,1774.34117779)(519.58453754,1774.27173339)(520.03592609,1774.13284461)
\curveto(520.49425908,1774.00090027)(520.9768976,1773.79603931)(521.48384166,1773.51826174)
\lineto(521.60884157,1774.01826136)
\lineto(523.44217351,1774.01826136)
\closepath
\moveto(521.48384166,1765.58076775)
\lineto(521.48384166,1771.92451295)
\curveto(520.96300872,1772.16062388)(520.47689798,1772.32729042)(520.02550943,1772.42451257)
\curveto(519.58106532,1772.52867916)(519.13662122,1772.58076245)(518.69217711,1772.58076245)
\curveto(517.61578903,1772.58076245)(516.76856745,1772.21965161)(516.15051236,1771.49742994)
\curveto(515.53245728,1770.77520826)(515.22342973,1769.72659794)(515.22342973,1768.35159898)
\curveto(515.22342973,1767.04604442)(515.45259622,1766.05646183)(515.91092921,1765.38285123)
\curveto(516.3692622,1764.70924063)(517.12967829,1764.37243533)(518.19217749,1764.37243533)
\curveto(518.7616215,1764.37243533)(519.33106551,1764.48007414)(519.90050953,1764.69535175)
\curveto(520.47689798,1764.91757381)(521.00467536,1765.21271247)(521.48384166,1765.58076775)
\closepath
}
}
{
\newrgbcolor{curcolor}{0 0 0}
\pscustom[linestyle=none,fillstyle=solid,fillcolor=curcolor]
{
\newpath
\moveto(537.03591143,1767.99743259)
\lineto(528.46300125,1767.99743259)
\curveto(528.46300125,1767.28215535)(528.57064006,1766.65715582)(528.78591767,1766.122434)
\curveto(529.00119529,1765.59465663)(529.29633395,1765.16062918)(529.67133367,1764.82035166)
\curveto(530.03244451,1764.48701858)(530.45952752,1764.23701876)(530.9525827,1764.07035222)
\curveto(531.45258232,1763.90368568)(532.00119302,1763.82035241)(532.59841479,1763.82035241)
\curveto(533.39008086,1763.82035241)(534.18521914,1763.97660229)(534.98382965,1764.28910206)
\curveto(535.7893846,1764.60854626)(536.36230083,1764.92104602)(536.70257835,1765.22660135)
\lineto(536.80674494,1765.22660135)
\lineto(536.80674494,1763.0911863)
\curveto(536.14702321,1762.81340873)(535.47341261,1762.58077002)(534.78591313,1762.39327016)
\curveto(534.09841365,1762.2057703)(533.37619198,1762.11202037)(532.61924811,1762.11202037)
\curveto(530.68869401,1762.11202037)(529.18175071,1762.63285331)(528.09841819,1763.67451919)
\curveto(527.01508568,1764.72312951)(526.47341942,1766.20923949)(526.47341942,1768.13284915)
\curveto(526.47341942,1770.03562549)(526.99078014,1771.54604101)(528.02550158,1772.66409572)
\curveto(529.06716746,1773.78215043)(530.43522198,1774.34117779)(532.12966514,1774.34117779)
\curveto(533.6991084,1774.34117779)(534.90744082,1773.8828448)(535.7546624,1772.96617883)
\curveto(536.60882842,1772.04951285)(537.03591143,1770.74743051)(537.03591143,1769.05993178)
\closepath
\moveto(535.12966287,1769.49743145)
\curveto(535.12271843,1770.52520845)(534.86230196,1771.32034674)(534.34841346,1771.88284631)
\curveto(533.8414694,1772.44534589)(533.06716443,1772.72659567)(532.02549855,1772.72659567)
\curveto(530.97688824,1772.72659567)(530.14008331,1772.41756813)(529.51508379,1771.79951304)
\curveto(528.8970287,1771.18145795)(528.54633452,1770.41409742)(528.46300125,1769.49743145)
\closepath
}
}
{
\newrgbcolor{curcolor}{0 0 0}
\pscustom[linestyle=none,fillstyle=solid,fillcolor=curcolor]
{
\newpath
\moveto(549.60881305,1775.96617656)
\lineto(547.40048139,1775.96617656)
\lineto(547.40048139,1777.99742502)
\lineto(549.60881305,1777.99742502)
\closepath
\moveto(549.48381314,1762.3828535)
\lineto(547.52548129,1762.3828535)
\lineto(547.52548129,1774.01826136)
\lineto(549.48381314,1774.01826136)
\closepath
}
}
{
\newrgbcolor{curcolor}{0 0 0}
\pscustom[linestyle=none,fillstyle=solid,fillcolor=curcolor]
{
\newpath
\moveto(559.61923009,1776.69534267)
\lineto(559.51506351,1776.69534267)
\curveto(559.29978589,1776.75784262)(559.0185361,1776.82034258)(558.67131414,1776.88284253)
\curveto(558.32409219,1776.95228692)(558.01853686,1776.98700912)(557.75464817,1776.98700912)
\curveto(556.91437103,1776.98700912)(556.30326038,1776.79950926)(555.92131623,1776.42450954)
\curveto(555.54631651,1776.05645427)(555.35881665,1775.38631588)(555.35881665,1774.4140944)
\lineto(555.35881665,1774.01826136)
\lineto(558.89006398,1774.01826136)
\lineto(558.89006398,1772.37242928)
\lineto(555.4213166,1772.37242928)
\lineto(555.4213166,1762.3828535)
\lineto(553.46298475,1762.3828535)
\lineto(553.46298475,1772.37242928)
\lineto(552.14006909,1772.37242928)
\lineto(552.14006909,1774.01826136)
\lineto(553.46298475,1774.01826136)
\lineto(553.46298475,1774.40367774)
\curveto(553.46298475,1775.78562114)(553.80673449,1776.84464811)(554.49423397,1777.58075867)
\curveto(555.18173345,1778.32381366)(556.17478826,1778.69534116)(557.47339838,1778.69534116)
\curveto(557.91089805,1778.69534116)(558.30325887,1778.67450784)(558.65048083,1778.63284121)
\curveto(559.00464723,1778.59117457)(559.32756365,1778.5425635)(559.61923009,1778.48700798)
\closepath
}
}
{
\newrgbcolor{curcolor}{0 0 0}
\pscustom[linestyle=none,fillstyle=solid,fillcolor=curcolor]
{
\newpath
\moveto(577.40046357,1762.3828535)
\lineto(575.45254838,1762.3828535)
\lineto(575.45254838,1763.6224359)
\curveto(575.2789374,1763.50438043)(575.04282647,1763.33771389)(574.74421558,1763.12243627)
\curveto(574.45254914,1762.9141031)(574.16782713,1762.74743656)(573.89004956,1762.62243665)
\curveto(573.56366092,1762.46271455)(573.1886612,1762.33077021)(572.76505041,1762.22660362)
\curveto(572.34143962,1762.11549259)(571.84491222,1762.05993708)(571.27546821,1762.05993708)
\curveto(570.22685789,1762.05993708)(569.33796967,1762.40715904)(568.60880356,1763.10160296)
\curveto(567.87963744,1763.79604688)(567.51505438,1764.68146287)(567.51505438,1765.75785095)
\curveto(567.51505438,1766.63979472)(567.70255424,1767.35159974)(568.07755396,1767.893266)
\curveto(568.45949811,1768.44187669)(569.00116437,1768.87243192)(569.70255273,1769.18493169)
\curveto(570.41088553,1769.49743145)(571.26157933,1769.70923685)(572.25463413,1769.82034787)
\curveto(573.24768894,1769.9314589)(574.31366035,1770.01479217)(575.45254838,1770.07034768)
\lineto(575.45254838,1770.37243079)
\curveto(575.45254838,1770.8168749)(575.37268733,1771.18493017)(575.21296523,1771.47659662)
\curveto(575.06018757,1771.76826307)(574.83796551,1771.99742956)(574.54629907,1772.1640961)
\curveto(574.2685215,1772.3238182)(573.93518842,1772.43145701)(573.54629982,1772.48701252)
\curveto(573.15741123,1772.54256804)(572.75116153,1772.57034579)(572.32755074,1772.57034579)
\curveto(571.81366224,1772.57034579)(571.24074601,1772.5009014)(570.60880204,1772.36201262)
\curveto(569.97685808,1772.23006827)(569.32408079,1772.03562398)(568.65047019,1771.77867973)
\lineto(568.5463036,1771.77867973)
\lineto(568.5463036,1773.76826155)
\curveto(568.92824776,1773.87242814)(569.48033068,1773.98701139)(570.20255235,1774.11201129)
\curveto(570.92477403,1774.2370112)(571.63657904,1774.29951115)(572.3379674,1774.29951115)
\curveto(573.15741123,1774.29951115)(573.86921624,1774.23006676)(574.47338245,1774.09117798)
\curveto(575.0844931,1773.95923363)(575.61227048,1773.73006714)(576.05671459,1773.4036785)
\curveto(576.49421426,1773.08423429)(576.82754734,1772.67104016)(577.05671383,1772.1640961)
\curveto(577.28588033,1771.65715204)(577.40046357,1771.02868029)(577.40046357,1770.27868086)
\closepath
\moveto(575.45254838,1765.24743467)
\lineto(575.45254838,1768.48701555)
\curveto(574.85532661,1768.45229335)(574.15046603,1768.40021006)(573.33796665,1768.33076567)
\curveto(572.5324117,1768.26132128)(571.89352329,1768.16062691)(571.42130143,1768.02868256)
\curveto(570.85880185,1767.86896046)(570.40394109,1767.61896065)(570.05671913,1767.27868313)
\curveto(569.70949717,1766.94535005)(569.53588619,1766.48354484)(569.53588619,1765.89326751)
\curveto(569.53588619,1765.22660135)(569.73727493,1764.72312951)(570.1400524,1764.38285199)
\curveto(570.54282987,1764.04951891)(571.15741274,1763.88285237)(571.983801,1763.88285237)
\curveto(572.67130048,1763.88285237)(573.29977223,1764.01479671)(573.86921624,1764.2786854)
\curveto(574.43866026,1764.54951853)(574.96643764,1764.87243495)(575.45254838,1765.24743467)
\closepath
}
}
{
\newrgbcolor{curcolor}{0 0 0}
\pscustom[linestyle=none,fillstyle=solid,fillcolor=curcolor]
{
\newpath
\moveto(590.90045825,1762.3828535)
\lineto(588.94212639,1762.3828535)
\lineto(588.94212639,1769.00784849)
\curveto(588.94212639,1769.54257031)(588.91087642,1770.04256993)(588.84837647,1770.50784735)
\curveto(588.78587651,1770.98006922)(588.67129327,1771.3481245)(588.50462673,1771.61201318)
\curveto(588.33101575,1771.90367963)(588.08101594,1772.11895725)(587.75462729,1772.25784603)
\curveto(587.42823865,1772.40367925)(587.00462786,1772.47659586)(586.48379492,1772.47659586)
\curveto(585.9490731,1772.47659586)(585.39004575,1772.34465152)(584.80671286,1772.08076283)
\curveto(584.22337996,1771.81687414)(583.66435261,1771.48006884)(583.12963079,1771.07034693)
\lineto(583.12963079,1762.3828535)
\lineto(581.17129894,1762.3828535)
\lineto(581.17129894,1774.01826136)
\lineto(583.12963079,1774.01826136)
\lineto(583.12963079,1772.72659567)
\curveto(583.74074144,1773.23353974)(584.37268541,1773.62937277)(585.02546269,1773.91409478)
\curveto(585.67823998,1774.19881678)(586.34837836,1774.34117779)(587.03587784,1774.34117779)
\curveto(588.29282133,1774.34117779)(589.25115394,1773.96270585)(589.91087566,1773.20576198)
\curveto(590.57059738,1772.44881811)(590.90045825,1771.35854115)(590.90045825,1769.93493112)
\closepath
}
}
{
\newrgbcolor{curcolor}{0 0 0}
\pscustom[linestyle=none,fillstyle=solid,fillcolor=curcolor]
{
\newpath
\moveto(604.52544614,1774.01826136)
\lineto(597.73378461,1758.09119008)
\lineto(595.6400362,1758.09119008)
\lineto(597.80670122,1762.94535307)
\lineto(593.17128806,1774.01826136)
\lineto(595.29628646,1774.01826136)
\lineto(598.86920042,1765.39326789)
\lineto(602.47336436,1774.01826136)
\closepath
}
}
{
\newrgbcolor{curcolor}{0 0 0}
\pscustom[linestyle=none,fillstyle=solid,fillcolor=curcolor]
{
\newpath
\moveto(622.90042536,1765.73701763)
\curveto(622.90042536,1764.67451843)(622.45945347,1763.80299131)(621.57750969,1763.12243627)
\curveto(620.70251035,1762.44188123)(619.50459459,1762.10160371)(617.98376241,1762.10160371)
\curveto(617.12265195,1762.10160371)(616.33098588,1762.20229808)(615.60876421,1762.40368682)
\curveto(614.89348697,1762.61201999)(614.29279298,1762.83771427)(613.80668224,1763.08076964)
\lineto(613.80668224,1765.27868464)
\lineto(613.91084882,1765.27868464)
\curveto(614.52890391,1764.81340722)(615.21640339,1764.44187972)(615.97334726,1764.16410215)
\curveto(616.73029114,1763.89326902)(617.45598503,1763.75785246)(618.15042895,1763.75785246)
\curveto(619.01153941,1763.75785246)(619.68515001,1763.89674124)(620.17126075,1764.17451881)
\curveto(620.6573715,1764.45229638)(620.90042687,1764.88979605)(620.90042687,1765.48701782)
\curveto(620.90042687,1765.94535081)(620.76848252,1766.29257276)(620.50459384,1766.5286837)
\curveto(620.24070515,1766.76479463)(619.73376109,1766.96618337)(618.98376165,1767.13284991)
\curveto(618.70598409,1767.19534986)(618.34140103,1767.26826647)(617.89001248,1767.35159974)
\curveto(617.44556837,1767.43493301)(617.03931868,1767.52521072)(616.6712634,1767.62243287)
\curveto(615.65043084,1767.893266)(614.92473695,1768.28909903)(614.49418172,1768.80993197)
\curveto(614.07057093,1769.33770935)(613.85876553,1769.98354219)(613.85876553,1770.74743051)
\curveto(613.85876553,1771.22659681)(613.95598768,1771.67798536)(614.15043198,1772.10159615)
\curveto(614.35182071,1772.52520694)(614.65390382,1772.90367887)(615.05668129,1773.23701196)
\curveto(615.44556989,1773.5634006)(615.93862507,1773.82034485)(616.53584684,1774.00784471)
\curveto(617.14001305,1774.202289)(617.81362365,1774.29951115)(618.55667864,1774.29951115)
\curveto(619.25112256,1774.29951115)(619.95251092,1774.21270566)(620.66084372,1774.03909468)
\curveto(621.37612095,1773.87242814)(621.9698705,1773.66756719)(622.44209237,1773.42451181)
\lineto(622.44209237,1771.3307634)
\lineto(622.33792578,1771.3307634)
\curveto(621.83792616,1771.69881867)(621.23028773,1772.00784622)(620.51501049,1772.25784603)
\curveto(619.79973326,1772.51479028)(619.0983449,1772.6432624)(618.41084542,1772.6432624)
\curveto(617.69556818,1772.6432624)(617.09140197,1772.50437362)(616.59834679,1772.22659605)
\curveto(616.10529161,1771.95576292)(615.85876402,1771.54951323)(615.85876402,1771.00784698)
\curveto(615.85876402,1770.52868067)(616.00806946,1770.16756983)(616.30668034,1769.92451446)
\curveto(616.59834679,1769.68145909)(617.07056866,1769.48354257)(617.72334594,1769.33076491)
\curveto(618.08445678,1769.24743164)(618.48723425,1769.16409837)(618.93167836,1769.0807651)
\curveto(619.38306691,1768.99743183)(619.75806662,1768.921043)(620.05667751,1768.85159861)
\curveto(620.96639904,1768.64326543)(621.6677874,1768.28562681)(622.16084258,1767.77868275)
\curveto(622.65389777,1767.26479425)(622.90042536,1766.58423921)(622.90042536,1765.73701763)
\closepath
}
}
{
\newrgbcolor{curcolor}{0 0 0}
\pscustom[linestyle=none,fillstyle=solid,fillcolor=curcolor]
{
\newpath
\moveto(634.30666584,1763.11201962)
\curveto(633.65388856,1762.79951985)(633.03236125,1762.55646448)(632.44208392,1762.3828535)
\curveto(631.85875103,1762.20924252)(631.23722372,1762.12243703)(630.577502,1762.12243703)
\curveto(629.73722486,1762.12243703)(628.96639211,1762.24396472)(628.26500375,1762.48702009)
\curveto(627.56361539,1762.7370199)(626.9629214,1763.11201962)(626.46292178,1763.61201924)
\curveto(625.95597772,1764.11201886)(625.5636169,1764.74396283)(625.28583934,1765.50785114)
\curveto(625.00806177,1766.27173945)(624.86917298,1767.16409988)(624.86917298,1768.18493244)
\curveto(624.86917298,1770.08770878)(625.39000592,1771.58076321)(626.4316718,1772.66409572)
\curveto(627.48028212,1773.74742824)(628.86222552,1774.28909449)(630.577502,1774.28909449)
\curveto(631.24416816,1774.28909449)(631.89694545,1774.19534456)(632.53583385,1774.00784471)
\curveto(633.1816667,1773.82034485)(633.77194403,1773.59117835)(634.30666584,1773.32034523)
\lineto(634.30666584,1771.14326354)
\lineto(634.20249926,1771.14326354)
\curveto(633.60527749,1771.60854097)(632.9872224,1771.96617958)(632.34833399,1772.21617939)
\curveto(631.71639003,1772.46617921)(631.09833494,1772.59117911)(630.49416873,1772.59117911)
\curveto(629.38305846,1772.59117911)(628.5045869,1772.21617939)(627.85875406,1771.46617996)
\curveto(627.21986565,1770.72312497)(626.90042145,1769.6293758)(626.90042145,1768.18493244)
\curveto(626.90042145,1766.78215573)(627.21292121,1765.70229543)(627.83792074,1764.94535156)
\curveto(628.4698647,1764.19535213)(629.3552807,1763.82035241)(630.49416873,1763.82035241)
\curveto(630.89000176,1763.82035241)(631.29277924,1763.87243571)(631.70250115,1763.97660229)
\curveto(632.11222306,1764.08076888)(632.48027834,1764.21618545)(632.80666698,1764.38285199)
\curveto(633.09138899,1764.52868521)(633.35874989,1764.68146287)(633.60874971,1764.84118497)
\curveto(633.85874952,1765.00785151)(634.05666603,1765.15021252)(634.20249926,1765.26826798)
\lineto(634.30666584,1765.26826798)
\closepath
}
}
{
\newrgbcolor{curcolor}{0 0 0}
\pscustom[linestyle=none,fillstyle=solid,fillcolor=curcolor]
{
\newpath
\moveto(644.0774909,1771.88284631)
\lineto(643.97332431,1771.88284631)
\curveto(643.68165787,1771.95229071)(643.39693586,1772.00090178)(643.11915829,1772.02867954)
\curveto(642.84832517,1772.06340173)(642.52540874,1772.08076283)(642.15040903,1772.08076283)
\curveto(641.54624282,1772.08076283)(640.96290993,1771.94534627)(640.40041035,1771.67451314)
\curveto(639.83791078,1771.41062445)(639.29624452,1771.06687471)(638.77541158,1770.64326392)
\lineto(638.77541158,1762.3828535)
\lineto(636.81707973,1762.3828535)
\lineto(636.81707973,1774.01826136)
\lineto(638.77541158,1774.01826136)
\lineto(638.77541158,1772.29951266)
\curveto(639.55318877,1772.92451219)(640.23721603,1773.36548408)(640.82749336,1773.62242833)
\curveto(641.42471513,1773.88631702)(642.03235356,1774.01826136)(642.65040865,1774.01826136)
\curveto(642.99068617,1774.01826136)(643.23721376,1774.00784471)(643.38999142,1773.98701139)
\curveto(643.54276909,1773.97312251)(643.77193558,1773.94187253)(644.0774909,1773.89326146)
\closepath
}
}
{
\newrgbcolor{curcolor}{0 0 0}
\pscustom[linestyle=none,fillstyle=solid,fillcolor=curcolor]
{
\newpath
\moveto(648.02540731,1775.96617656)
\lineto(645.81707564,1775.96617656)
\lineto(645.81707564,1777.99742502)
\lineto(648.02540731,1777.99742502)
\closepath
\moveto(647.9004074,1762.3828535)
\lineto(645.94207555,1762.3828535)
\lineto(645.94207555,1774.01826136)
\lineto(647.9004074,1774.01826136)
\closepath
}
}
{
\newrgbcolor{curcolor}{0 0 0}
\pscustom[linestyle=none,fillstyle=solid,fillcolor=curcolor]
{
\newpath
\moveto(662.01498801,1768.34118233)
\curveto(662.01498801,1767.3967386)(661.87957144,1766.53215592)(661.60873832,1765.74743429)
\curveto(661.33790519,1764.9696571)(660.95596103,1764.30993538)(660.46290585,1763.76826912)
\curveto(660.00457286,1763.25438062)(659.46290661,1762.85507537)(658.83790708,1762.57035336)
\curveto(658.21985199,1762.29257579)(657.56360249,1762.15368701)(656.86915857,1762.15368701)
\curveto(656.26499236,1762.15368701)(655.71638166,1762.21965918)(655.22332648,1762.35160352)
\curveto(654.73721574,1762.48354787)(654.24068833,1762.68840882)(653.73374427,1762.96618639)
\lineto(653.73374427,1758.09119008)
\lineto(651.77541242,1758.09119008)
\lineto(651.77541242,1774.01826136)
\lineto(653.73374427,1774.01826136)
\lineto(653.73374427,1772.79951229)
\curveto(654.25457721,1773.23701196)(654.8379101,1773.60159501)(655.48374295,1773.89326146)
\curveto(656.13652023,1774.19187234)(656.83096415,1774.34117779)(657.56707471,1774.34117779)
\curveto(658.96985142,1774.34117779)(660.06012838,1773.80992819)(660.83790557,1772.74742899)
\curveto(661.62262719,1771.69187424)(662.01498801,1770.22312535)(662.01498801,1768.34118233)
\closepath
\moveto(659.9941562,1768.28909903)
\curveto(659.9941562,1769.69187575)(659.75457305,1770.74048607)(659.27540675,1771.43492999)
\curveto(658.79624044,1772.1293739)(658.06012989,1772.47659586)(657.06707508,1772.47659586)
\curveto(656.50457551,1772.47659586)(655.93860372,1772.35506818)(655.3691597,1772.11201281)
\curveto(654.79971569,1771.86895743)(654.25457721,1771.54951323)(653.73374427,1771.1536802)
\lineto(653.73374427,1764.55993519)
\curveto(654.28929941,1764.30993538)(654.76499349,1764.13979662)(655.16082653,1764.04951891)
\curveto(655.563604,1763.9592412)(656.01846477,1763.91410234)(656.52540883,1763.91410234)
\curveto(657.61568578,1763.91410234)(658.46637958,1764.28215762)(659.07749023,1765.01826817)
\curveto(659.68860088,1765.75437873)(659.9941562,1766.84465568)(659.9941562,1768.28909903)
\closepath
}
}
{
\newrgbcolor{curcolor}{0 0 0}
\pscustom[linestyle=none,fillstyle=solid,fillcolor=curcolor]
{
\newpath
\moveto(671.10872933,1762.48702009)
\curveto(670.74067405,1762.38979794)(670.33789658,1762.30993689)(669.90039691,1762.24743694)
\curveto(669.46984168,1762.18493698)(669.08442531,1762.15368701)(668.74414779,1762.15368701)
\curveto(667.55664868,1762.15368701)(666.65387159,1762.47313121)(666.0358165,1763.11201962)
\curveto(665.41776141,1763.75090802)(665.10873387,1764.7752128)(665.10873387,1766.18493396)
\lineto(665.10873387,1772.37242928)
\lineto(663.7858182,1772.37242928)
\lineto(663.7858182,1774.01826136)
\lineto(665.10873387,1774.01826136)
\lineto(665.10873387,1777.36200883)
\lineto(667.06706572,1777.36200883)
\lineto(667.06706572,1774.01826136)
\lineto(671.10872933,1774.01826136)
\lineto(671.10872933,1772.37242928)
\lineto(667.06706572,1772.37242928)
\lineto(667.06706572,1767.07034995)
\curveto(667.06706572,1766.45923931)(667.0809546,1765.980073)(667.10873236,1765.63285104)
\curveto(667.13651011,1765.29257352)(667.23373226,1764.97312932)(667.4003988,1764.67451843)
\curveto(667.55317646,1764.39674087)(667.76150964,1764.19187991)(668.02539833,1764.05993557)
\curveto(668.29623146,1763.93493566)(668.70595337,1763.87243571)(669.25456407,1763.87243571)
\curveto(669.57400827,1763.87243571)(669.90734135,1763.91757456)(670.25456331,1764.00785227)
\curveto(670.60178527,1764.10507442)(670.85178508,1764.18493547)(671.00456274,1764.24743542)
\lineto(671.10872933,1764.24743542)
\closepath
}
}
{
\newrgbcolor{curcolor}{0 0 0}
\pscustom[linestyle=none,fillstyle=solid,fillcolor=curcolor]
{
\newpath
\moveto(687.23370706,1776.69534267)
\lineto(687.12954047,1776.69534267)
\curveto(686.91426285,1776.75784262)(686.63301307,1776.82034258)(686.28579111,1776.88284253)
\curveto(685.93856915,1776.95228692)(685.63301382,1776.98700912)(685.36912513,1776.98700912)
\curveto(684.52884799,1776.98700912)(683.91773734,1776.79950926)(683.53579319,1776.42450954)
\curveto(683.16079347,1776.05645427)(682.97329361,1775.38631588)(682.97329361,1774.4140944)
\lineto(682.97329361,1774.01826136)
\lineto(686.50454094,1774.01826136)
\lineto(686.50454094,1772.37242928)
\lineto(683.03579357,1772.37242928)
\lineto(683.03579357,1762.3828535)
\lineto(681.07746171,1762.3828535)
\lineto(681.07746171,1772.37242928)
\lineto(679.75454605,1772.37242928)
\lineto(679.75454605,1774.01826136)
\lineto(681.07746171,1774.01826136)
\lineto(681.07746171,1774.40367774)
\curveto(681.07746171,1775.78562114)(681.42121145,1776.84464811)(682.10871093,1777.58075867)
\curveto(682.79621041,1778.32381366)(683.78926522,1778.69534116)(685.08787535,1778.69534116)
\curveto(685.52537502,1778.69534116)(685.91773583,1778.67450784)(686.26495779,1778.63284121)
\curveto(686.61912419,1778.59117457)(686.94204061,1778.5425635)(687.23370706,1778.48700798)
\closepath
}
}
{
\newrgbcolor{curcolor}{0 0 0}
\pscustom[linestyle=none,fillstyle=solid,fillcolor=curcolor]
{
\newpath
\moveto(697.51495495,1762.3828535)
\lineto(695.56703976,1762.3828535)
\lineto(695.56703976,1763.6224359)
\curveto(695.39342878,1763.50438043)(695.15731785,1763.33771389)(694.85870696,1763.12243627)
\curveto(694.56704052,1762.9141031)(694.28231851,1762.74743656)(694.00454094,1762.62243665)
\curveto(693.6781523,1762.46271455)(693.30315258,1762.33077021)(692.87954179,1762.22660362)
\curveto(692.455931,1762.11549259)(691.9594036,1762.05993708)(691.38995959,1762.05993708)
\curveto(690.34134927,1762.05993708)(689.45246105,1762.40715904)(688.72329494,1763.10160296)
\curveto(687.99412882,1763.79604688)(687.62954576,1764.68146287)(687.62954576,1765.75785095)
\curveto(687.62954576,1766.63979472)(687.81704562,1767.35159974)(688.19204534,1767.893266)
\curveto(688.57398949,1768.44187669)(689.11565575,1768.87243192)(689.81704411,1769.18493169)
\curveto(690.52537691,1769.49743145)(691.37607071,1769.70923685)(692.36912551,1769.82034787)
\curveto(693.36218032,1769.9314589)(694.42815173,1770.01479217)(695.56703976,1770.07034768)
\lineto(695.56703976,1770.37243079)
\curveto(695.56703976,1770.8168749)(695.48717871,1771.18493017)(695.32745661,1771.47659662)
\curveto(695.17467895,1771.76826307)(694.95245689,1771.99742956)(694.66079045,1772.1640961)
\curveto(694.38301288,1772.3238182)(694.0496798,1772.43145701)(693.6607912,1772.48701252)
\curveto(693.27190261,1772.54256804)(692.86565291,1772.57034579)(692.44204212,1772.57034579)
\curveto(691.92815362,1772.57034579)(691.35523739,1772.5009014)(690.72329342,1772.36201262)
\curveto(690.09134946,1772.23006827)(689.43857217,1772.03562398)(688.76496157,1771.77867973)
\lineto(688.66079498,1771.77867973)
\lineto(688.66079498,1773.76826155)
\curveto(689.04273914,1773.87242814)(689.59482206,1773.98701139)(690.31704373,1774.11201129)
\curveto(691.03926541,1774.2370112)(691.75107042,1774.29951115)(692.45245878,1774.29951115)
\curveto(693.27190261,1774.29951115)(693.98370762,1774.23006676)(694.58787383,1774.09117798)
\curveto(695.19898448,1773.95923363)(695.72676186,1773.73006714)(696.17120597,1773.4036785)
\curveto(696.60870564,1773.08423429)(696.94203872,1772.67104016)(697.17120521,1772.1640961)
\curveto(697.40037171,1771.65715204)(697.51495495,1771.02868029)(697.51495495,1770.27868086)
\closepath
\moveto(695.56703976,1765.24743467)
\lineto(695.56703976,1768.48701555)
\curveto(694.96981799,1768.45229335)(694.26495741,1768.40021006)(693.45245803,1768.33076567)
\curveto(692.64690308,1768.26132128)(692.00801467,1768.16062691)(691.53579281,1768.02868256)
\curveto(690.97329324,1767.86896046)(690.51843247,1767.61896065)(690.17121051,1767.27868313)
\curveto(689.82398855,1766.94535005)(689.65037757,1766.48354484)(689.65037757,1765.89326751)
\curveto(689.65037757,1765.22660135)(689.85176631,1764.72312951)(690.25454378,1764.38285199)
\curveto(690.65732125,1764.04951891)(691.27190412,1763.88285237)(692.09829238,1763.88285237)
\curveto(692.78579186,1763.88285237)(693.41426361,1764.01479671)(693.98370762,1764.2786854)
\curveto(694.55315164,1764.54951853)(695.08092902,1764.87243495)(695.56703976,1765.24743467)
\closepath
}
}
{
\newrgbcolor{curcolor}{0 0 0}
\pscustom[linestyle=none,fillstyle=solid,fillcolor=curcolor]
{
\newpath
\moveto(703.38994098,1775.96617656)
\lineto(701.18160931,1775.96617656)
\lineto(701.18160931,1777.99742502)
\lineto(703.38994098,1777.99742502)
\closepath
\moveto(703.26494107,1762.3828535)
\lineto(701.30660922,1762.3828535)
\lineto(701.30660922,1774.01826136)
\lineto(703.26494107,1774.01826136)
\closepath
}
}
{
\newrgbcolor{curcolor}{0 0 0}
\pscustom[linestyle=none,fillstyle=solid,fillcolor=curcolor]
{
\newpath
\moveto(709.11912568,1762.3828535)
\lineto(707.16079383,1762.3828535)
\lineto(707.16079383,1778.59117457)
\lineto(709.11912568,1778.59117457)
\closepath
}
}
{
\newrgbcolor{curcolor}{0 0 0}
\pscustom[linestyle=none,fillstyle=solid,fillcolor=curcolor]
{
\newpath
\moveto(721.30661,1765.73701763)
\curveto(721.30661,1764.67451843)(720.86563811,1763.80299131)(719.98369433,1763.12243627)
\curveto(719.10869499,1762.44188123)(717.91077923,1762.10160371)(716.38994705,1762.10160371)
\curveto(715.52883659,1762.10160371)(714.73717052,1762.20229808)(714.01494885,1762.40368682)
\curveto(713.29967161,1762.61201999)(712.69897762,1762.83771427)(712.21286688,1763.08076964)
\lineto(712.21286688,1765.27868464)
\lineto(712.31703346,1765.27868464)
\curveto(712.93508855,1764.81340722)(713.62258803,1764.44187972)(714.3795319,1764.16410215)
\curveto(715.13647578,1763.89326902)(715.86216967,1763.75785246)(716.55661359,1763.75785246)
\curveto(717.41772405,1763.75785246)(718.09133465,1763.89674124)(718.57744539,1764.17451881)
\curveto(719.06355614,1764.45229638)(719.30661151,1764.88979605)(719.30661151,1765.48701782)
\curveto(719.30661151,1765.94535081)(719.17466716,1766.29257276)(718.91077848,1766.5286837)
\curveto(718.64688979,1766.76479463)(718.13994573,1766.96618337)(717.38994629,1767.13284991)
\curveto(717.11216872,1767.19534986)(716.74758567,1767.26826647)(716.29619712,1767.35159974)
\curveto(715.85175301,1767.43493301)(715.44550332,1767.52521072)(715.07744804,1767.62243287)
\curveto(714.05661548,1767.893266)(713.33092159,1768.28909903)(712.90036636,1768.80993197)
\curveto(712.47675557,1769.33770935)(712.26495017,1769.98354219)(712.26495017,1770.74743051)
\curveto(712.26495017,1771.22659681)(712.36217232,1771.67798536)(712.55661662,1772.10159615)
\curveto(712.75800535,1772.52520694)(713.06008846,1772.90367887)(713.46286593,1773.23701196)
\curveto(713.85175452,1773.5634006)(714.34480971,1773.82034485)(714.94203148,1774.00784471)
\curveto(715.54619769,1774.202289)(716.21980829,1774.29951115)(716.96286328,1774.29951115)
\curveto(717.6573072,1774.29951115)(718.35869556,1774.21270566)(719.06702836,1774.03909468)
\curveto(719.78230559,1773.87242814)(720.37605514,1773.66756719)(720.84827701,1773.42451181)
\lineto(720.84827701,1771.3307634)
\lineto(720.74411042,1771.3307634)
\curveto(720.2441108,1771.69881867)(719.63647237,1772.00784622)(718.92119513,1772.25784603)
\curveto(718.2059179,1772.51479028)(717.50452954,1772.6432624)(716.81703006,1772.6432624)
\curveto(716.10175282,1772.6432624)(715.49758661,1772.50437362)(715.00453143,1772.22659605)
\curveto(714.51147625,1771.95576292)(714.26494866,1771.54951323)(714.26494866,1771.00784698)
\curveto(714.26494866,1770.52868067)(714.4142541,1770.16756983)(714.71286498,1769.92451446)
\curveto(715.00453143,1769.68145909)(715.4767533,1769.48354257)(716.12953058,1769.33076491)
\curveto(716.49064142,1769.24743164)(716.89341889,1769.16409837)(717.337863,1769.0807651)
\curveto(717.78925155,1768.99743183)(718.16425126,1768.921043)(718.46286215,1768.85159861)
\curveto(719.37258368,1768.64326543)(720.07397204,1768.28562681)(720.56702722,1767.77868275)
\curveto(721.0600824,1767.26479425)(721.30661,1766.58423921)(721.30661,1765.73701763)
\closepath
}
}
{
\newrgbcolor{curcolor}{0.7019608 0.7019608 0.7019608}
\pscustom[linestyle=none,fillstyle=solid,fillcolor=curcolor,opacity=0.92623001]
{
\newpath
\moveto(356.97411693,1580.67296641)
\lineto(765.54556747,1580.67296641)
\lineto(765.54556747,1426.38724594)
\lineto(356.97411693,1426.38724594)
\closepath
}
}
{
\newrgbcolor{curcolor}{0 0 0}
\pscustom[linewidth=1.00157103,linecolor=curcolor]
{
\newpath
\moveto(356.97411693,1580.67296641)
\lineto(765.54556747,1580.67296641)
\lineto(765.54556747,1426.38724594)
\lineto(356.97411693,1426.38724594)
\closepath
}
}
{
\newrgbcolor{curcolor}{0 0 0}
\pscustom[linestyle=none,fillstyle=solid,fillcolor=curcolor]
{
\newpath
\moveto(467.0999665,1548.08792088)
\lineto(463.44763438,1548.08792088)
\lineto(463.44763438,1550.41213223)
\curveto(463.12211458,1550.19077877)(462.67940765,1549.87827977)(462.1195136,1549.47463522)
\curveto(461.57264035,1549.08401146)(461.03878788,1548.77151246)(460.5179562,1548.5371382)
\curveto(459.90597899,1548.23765999)(459.20285622,1547.99026494)(458.40858792,1547.79495307)
\curveto(457.61431961,1547.5866204)(456.683333,1547.48245406)(455.61562806,1547.48245406)
\curveto(453.64948849,1547.48245406)(451.98282713,1548.13349365)(450.61564398,1549.43557284)
\curveto(449.24846083,1550.73765203)(448.56486926,1552.39780299)(448.56486926,1554.41602574)
\curveto(448.56486926,1556.0696663)(448.91643064,1557.40429747)(449.6195534,1558.41991924)
\curveto(450.33569696,1559.4485618)(451.35131872,1560.25585089)(452.6664187,1560.84178653)
\curveto(453.99453947,1561.42772216)(455.58958648,1561.82485631)(457.45155972,1562.03318898)
\curveto(459.31353295,1562.24152165)(461.31222451,1562.39777116)(463.44763438,1562.50193749)
\lineto(463.44763438,1563.06834194)
\curveto(463.44763438,1563.90167262)(463.29789527,1564.59177459)(462.99841706,1565.13864785)
\curveto(462.71195963,1565.6855211)(462.29529429,1566.11520724)(461.74842104,1566.42770624)
\curveto(461.22758936,1566.72718445)(460.60259135,1566.92900673)(459.87342701,1567.03317306)
\curveto(459.14426266,1567.1373394)(458.38254634,1567.18942257)(457.58827803,1567.18942257)
\curveto(456.62473943,1567.18942257)(455.5505241,1567.05921465)(454.36563204,1566.79879881)
\curveto(453.18073998,1566.55140376)(451.95678554,1566.18682159)(450.69376873,1565.70505229)
\lineto(450.49845685,1565.70505229)
\lineto(450.49845685,1569.43550916)
\curveto(451.21460041,1569.63082104)(452.24975336,1569.84566411)(453.60391572,1570.08003836)
\curveto(454.95807807,1570.31441262)(456.29270924,1570.43159974)(457.60780922,1570.43159974)
\curveto(459.14426266,1570.43159974)(460.47889383,1570.30139182)(461.61170272,1570.04097599)
\curveto(462.75753241,1569.79358094)(463.74711259,1569.36389481)(464.58044327,1568.75191759)
\curveto(465.40075316,1568.15296116)(466.02575117,1567.37822405)(466.4554373,1566.42770624)
\curveto(466.88512343,1565.47718843)(467.0999665,1564.29880677)(467.0999665,1562.89256125)
\closepath
\moveto(463.44763438,1553.45899753)
\lineto(463.44763438,1559.53319694)
\curveto(462.32784627,1559.46809298)(461.0062359,1559.37043704)(459.48280325,1559.24022913)
\curveto(457.97239139,1559.11002121)(456.77447854,1558.92121972)(455.88906469,1558.67382468)
\curveto(454.83438055,1558.37434647)(453.98151868,1557.90559796)(453.33047909,1557.26757916)
\curveto(452.67943949,1556.64258115)(452.3539197,1555.77669849)(452.3539197,1554.66993118)
\curveto(452.3539197,1553.41993516)(452.73152266,1552.47592775)(453.48672859,1551.83790894)
\curveto(454.24193452,1551.21291093)(455.3942746,1550.90041193)(456.94374883,1550.90041193)
\curveto(458.23280723,1550.90041193)(459.41118889,1551.14780697)(460.47889383,1551.64259707)
\curveto(461.54659876,1552.15040795)(462.53617894,1552.75587477)(463.44763438,1553.45899753)
\closepath
}
}
{
\newrgbcolor{curcolor}{0 0 0}
\pscustom[linestyle=none,fillstyle=solid,fillcolor=curcolor]
{
\newpath
\moveto(492.2561372,1548.08792088)
\lineto(488.58427389,1548.08792088)
\lineto(488.58427389,1550.50978817)
\curveto(487.34729866,1549.53322878)(486.1624066,1548.78453325)(485.02959771,1548.26370157)
\curveto(483.89678881,1547.7428699)(482.64679279,1547.48245406)(481.27960965,1547.48245406)
\curveto(478.98795027,1547.48245406)(477.20410179,1548.17906643)(475.92806418,1549.57229116)
\curveto(474.65202658,1550.97853668)(474.01400778,1553.0358218)(474.01400778,1555.74414651)
\lineto(474.01400778,1569.90425767)
\lineto(477.68587109,1569.90425767)
\lineto(477.68587109,1557.48242222)
\curveto(477.68587109,1556.37565491)(477.73795425,1555.42513711)(477.84212059,1554.6308688)
\curveto(477.94628692,1553.84962129)(478.16764039,1553.17905051)(478.50618098,1552.61915646)
\curveto(478.85774236,1552.04624161)(479.31347007,1551.62957627)(479.87336412,1551.36916044)
\curveto(480.43325817,1551.1087446)(481.24705767,1550.97853668)(482.3147626,1550.97853668)
\curveto(483.26528041,1550.97853668)(484.30043336,1551.22593173)(485.42022146,1551.72072182)
\curveto(486.55303035,1552.21551191)(487.6077145,1552.84702031)(488.58427389,1553.61524703)
\lineto(488.58427389,1569.90425767)
\lineto(492.2561372,1569.90425767)
\closepath
}
}
{
\newrgbcolor{curcolor}{0 0 0}
\pscustom[linestyle=none,fillstyle=solid,fillcolor=curcolor]
{
\newpath
\moveto(510.81076331,1548.28323276)
\curveto(510.12066134,1548.10094168)(509.36545541,1547.95120257)(508.54514552,1547.83401544)
\curveto(507.73785643,1547.71682831)(507.01520248,1547.65823475)(506.37718368,1547.65823475)
\curveto(504.15062827,1547.65823475)(502.45792532,1548.25719118)(501.29907485,1549.45510403)
\curveto(500.14022437,1550.65301688)(499.56079913,1552.57358368)(499.56079913,1555.21680444)
\lineto(499.56079913,1566.81833)
\lineto(497.08033828,1566.81833)
\lineto(497.08033828,1569.90425767)
\lineto(499.56079913,1569.90425767)
\lineto(499.56079913,1576.17376896)
\lineto(503.23266244,1576.17376896)
\lineto(503.23266244,1569.90425767)
\lineto(510.81076331,1569.90425767)
\lineto(510.81076331,1566.81833)
\lineto(503.23266244,1566.81833)
\lineto(503.23266244,1556.8769554)
\curveto(503.23266244,1555.73112571)(503.25870402,1554.83269108)(503.31078719,1554.18165148)
\curveto(503.36287036,1553.54363268)(503.54516144,1552.94467625)(503.85766045,1552.3847822)
\curveto(504.14411787,1551.86395053)(504.53474163,1551.47983717)(505.02953172,1551.23244212)
\curveto(505.5373426,1550.99806787)(506.30556932,1550.88088074)(507.33421188,1550.88088074)
\curveto(507.93316831,1550.88088074)(508.55816632,1550.96551589)(509.20920591,1551.13478618)
\curveto(509.8602455,1551.31707727)(510.32899401,1551.46681638)(510.61545143,1551.5840035)
\lineto(510.81076331,1551.5840035)
\closepath
}
}
{
\newrgbcolor{curcolor}{0 0 0}
\pscustom[linestyle=none,fillstyle=solid,fillcolor=curcolor]
{
\newpath
\moveto(533.83803347,1558.98632368)
\curveto(533.83803347,1555.4316475)(532.92657803,1552.62566685)(531.10366717,1550.56838174)
\curveto(529.28075631,1548.51109662)(526.83935783,1547.48245406)(523.77947174,1547.48245406)
\curveto(520.69354407,1547.48245406)(518.2391248,1548.51109662)(516.41621394,1550.56838174)
\curveto(514.60632386,1552.62566685)(513.70137883,1555.4316475)(513.70137883,1558.98632368)
\curveto(513.70137883,1562.54099987)(514.60632386,1565.34698052)(516.41621394,1567.40426563)
\curveto(518.2391248,1569.47457154)(520.69354407,1570.50972449)(523.77947174,1570.50972449)
\curveto(526.83935783,1570.50972449)(529.28075631,1569.47457154)(531.10366717,1567.40426563)
\curveto(532.92657803,1565.34698052)(533.83803347,1562.54099987)(533.83803347,1558.98632368)
\closepath
\moveto(530.04898303,1558.98632368)
\curveto(530.04898303,1561.81183552)(529.49559937,1563.90818301)(528.38883207,1565.27536616)
\curveto(527.28206476,1566.6555701)(525.74561131,1567.34567207)(523.77947174,1567.34567207)
\curveto(521.78729058,1567.34567207)(520.23781635,1566.6555701)(519.13104904,1565.27536616)
\curveto(518.03730252,1563.90818301)(517.49042927,1561.81183552)(517.49042927,1558.98632368)
\curveto(517.49042927,1556.25195739)(518.04381292,1554.17514109)(519.15058023,1552.75587477)
\curveto(520.25734754,1551.34962925)(521.80031138,1550.64650649)(523.77947174,1550.64650649)
\curveto(525.73259052,1550.64650649)(527.26253357,1551.34311885)(528.36930088,1552.73634358)
\curveto(529.48908898,1554.14258911)(530.04898303,1556.22591581)(530.04898303,1558.98632368)
\closepath
}
}
{
\newrgbcolor{curcolor}{0 0 0}
\pscustom[linestyle=none,fillstyle=solid,fillcolor=curcolor]
{
\newpath
\moveto(550.84970086,1548.28323276)
\curveto(550.15959889,1548.10094168)(549.40439296,1547.95120257)(548.58408307,1547.83401544)
\curveto(547.77679397,1547.71682831)(547.05414003,1547.65823475)(546.41612122,1547.65823475)
\curveto(544.18956581,1547.65823475)(542.49686287,1548.25719118)(541.33801239,1549.45510403)
\curveto(540.17916191,1550.65301688)(539.59973668,1552.57358368)(539.59973668,1555.21680444)
\lineto(539.59973668,1566.81833)
\lineto(537.11927582,1566.81833)
\lineto(537.11927582,1569.90425767)
\lineto(539.59973668,1569.90425767)
\lineto(539.59973668,1576.17376896)
\lineto(543.27159999,1576.17376896)
\lineto(543.27159999,1569.90425767)
\lineto(550.84970086,1569.90425767)
\lineto(550.84970086,1566.81833)
\lineto(543.27159999,1566.81833)
\lineto(543.27159999,1556.8769554)
\curveto(543.27159999,1555.73112571)(543.29764157,1554.83269108)(543.34972474,1554.18165148)
\curveto(543.4018079,1553.54363268)(543.58409899,1552.94467625)(543.896598,1552.3847822)
\curveto(544.18305542,1551.86395053)(544.57367917,1551.47983717)(545.06846926,1551.23244212)
\curveto(545.57628015,1550.99806787)(546.34450687,1550.88088074)(547.37314943,1550.88088074)
\curveto(547.97210585,1550.88088074)(548.59710386,1550.96551589)(549.24814346,1551.13478618)
\curveto(549.89918305,1551.31707727)(550.36793156,1551.46681638)(550.65438898,1551.5840035)
\lineto(550.84970086,1551.5840035)
\closepath
}
}
{
\newrgbcolor{curcolor}{0 0 0}
\pscustom[linestyle=none,fillstyle=solid,fillcolor=curcolor]
{
\newpath
\moveto(573.54494082,1558.61523112)
\lineto(557.47077325,1558.61523112)
\curveto(557.47077325,1557.27408955)(557.67259552,1556.10221828)(558.07624007,1555.09961731)
\curveto(558.47988462,1554.11003713)(559.03326827,1553.29623763)(559.73639103,1552.65821883)
\curveto(560.41347221,1552.03322082)(561.21425091,1551.56447231)(562.13872714,1551.25197331)
\curveto(563.07622415,1550.9394743)(564.10486671,1550.7832248)(565.22465481,1550.7832248)
\curveto(566.70902508,1550.7832248)(568.19990575,1551.07619262)(569.69729682,1551.66212825)
\curveto(571.20770868,1552.26108468)(572.28192401,1552.84702031)(572.91994281,1553.41993516)
\lineto(573.11525469,1553.41993516)
\lineto(573.11525469,1549.41604165)
\curveto(571.87827946,1548.89520998)(570.61526265,1548.45901345)(569.32620425,1548.10745207)
\curveto(568.03714586,1547.75589069)(566.6829835,1547.58011)(565.26371719,1547.58011)
\curveto(561.64393704,1547.58011)(558.81842521,1548.55666939)(556.78718167,1550.50978817)
\curveto(554.75593814,1552.47592775)(553.74031638,1555.26237721)(553.74031638,1558.86913656)
\curveto(553.74031638,1562.43683353)(554.71036537,1565.26885576)(556.65046336,1567.36520326)
\curveto(558.60358214,1569.46155075)(561.16867814,1570.50972449)(564.34575136,1570.50972449)
\curveto(567.28845032,1570.50972449)(569.55406811,1569.65035223)(571.14260472,1567.9316077)
\curveto(572.74416212,1566.21286317)(573.54494082,1563.7714647)(573.54494082,1560.60741227)
\closepath
\moveto(569.97073345,1561.42772216)
\curveto(569.95771266,1563.35479936)(569.46943296,1564.84568003)(568.50589436,1565.90036417)
\curveto(567.55537656,1566.95504831)(566.10355826,1567.48239038)(564.15043948,1567.48239038)
\curveto(562.18429991,1567.48239038)(560.61529449,1566.90296514)(559.44342322,1565.74411467)
\curveto(558.28457274,1564.58526419)(557.62702275,1563.14646669)(557.47077325,1561.42772216)
\closepath
}
}
{
\newrgbcolor{curcolor}{0 0 0}
\pscustom[linestyle=none,fillstyle=solid,fillcolor=curcolor]
{
\newpath
\moveto(594.6972234,1554.37696336)
\curveto(594.6972234,1552.3847822)(593.87040312,1550.75067282)(592.21676255,1549.47463522)
\curveto(590.57614277,1548.19859761)(588.33005618,1547.56057881)(585.47850275,1547.56057881)
\curveto(583.86392456,1547.56057881)(582.37955429,1547.74938029)(581.02539193,1548.12698326)
\curveto(579.68425037,1548.51760702)(578.55795187,1548.94078275)(577.64649644,1549.39651047)
\lineto(577.64649644,1553.5175911)
\lineto(577.84180832,1553.5175911)
\curveto(579.0006588,1552.64519804)(580.28971719,1551.94858567)(581.70898351,1551.427754)
\curveto(583.12824982,1550.91994312)(584.48892257,1550.66603767)(585.79100176,1550.66603767)
\curveto(587.40557995,1550.66603767)(588.66859676,1550.92645351)(589.5800522,1551.44728519)
\curveto(590.49150763,1551.96811686)(590.94723534,1552.78842675)(590.94723534,1553.90821485)
\curveto(590.94723534,1554.76758712)(590.6998403,1555.41862671)(590.20505021,1555.86133363)
\curveto(589.71026011,1556.30404056)(588.75974231,1556.68164352)(587.35349678,1556.99414253)
\curveto(586.83266511,1557.11132965)(586.14907354,1557.24804797)(585.30272206,1557.40429747)
\curveto(584.46939138,1557.56054697)(583.70767506,1557.72981727)(583.01757309,1557.91210835)
\curveto(581.10351668,1558.41991924)(579.74284393,1559.16210437)(578.93555484,1560.13866376)
\curveto(578.14128653,1561.12824395)(577.74415238,1562.33917759)(577.74415238,1563.7714647)
\curveto(577.74415238,1564.66989934)(577.92644347,1565.51625081)(578.29102564,1566.31051911)
\curveto(578.6686286,1567.10478742)(579.23503305,1567.81442058)(579.99023898,1568.43941859)
\curveto(580.71940332,1569.0513958)(581.64387955,1569.5331651)(582.76366765,1569.88472648)
\curveto(583.89647654,1570.24930866)(585.15949335,1570.43159974)(586.55271808,1570.43159974)
\curveto(587.85479727,1570.43159974)(589.16989725,1570.26883984)(590.49801802,1569.94332005)
\curveto(591.83915959,1569.63082104)(592.95243729,1569.24670768)(593.83785114,1568.79097997)
\lineto(593.83785114,1564.86521122)
\lineto(593.64253926,1564.86521122)
\curveto(592.70504225,1565.55531319)(591.56572296,1566.13473842)(590.22458139,1566.60348693)
\curveto(588.88343983,1567.08525623)(587.56833985,1567.32614088)(586.27928145,1567.32614088)
\curveto(584.93813989,1567.32614088)(583.805331,1567.06572504)(582.88085477,1566.54489337)
\curveto(581.95637855,1566.03708248)(581.49414044,1565.27536616)(581.49414044,1564.25974439)
\curveto(581.49414044,1563.36130975)(581.77408747,1562.68422858)(582.33398152,1562.22850086)
\curveto(582.88085477,1561.77277315)(583.76626862,1561.40168058)(584.99022306,1561.11522316)
\curveto(585.66730424,1560.95897365)(586.42251017,1560.80272415)(587.25584085,1560.64647465)
\curveto(588.10219232,1560.49022515)(588.80531508,1560.34699643)(589.36520913,1560.21678852)
\curveto(591.07093286,1559.82616476)(592.38603284,1559.15559398)(593.31050907,1558.20507617)
\curveto(594.23498529,1557.24153757)(594.6972234,1555.96549997)(594.6972234,1554.37696336)
\closepath
}
}
{
\newrgbcolor{curcolor}{0 0 0}
\pscustom[linestyle=none,fillstyle=solid,fillcolor=curcolor]
{
\newpath
\moveto(611.27919131,1548.28323276)
\curveto(610.58908934,1548.10094168)(609.83388342,1547.95120257)(609.01357353,1547.83401544)
\curveto(608.20628443,1547.71682831)(607.48363048,1547.65823475)(606.84561168,1547.65823475)
\curveto(604.61905627,1547.65823475)(602.92635333,1548.25719118)(601.76750285,1549.45510403)
\curveto(600.60865237,1550.65301688)(600.02922713,1552.57358368)(600.02922713,1555.21680444)
\lineto(600.02922713,1566.81833)
\lineto(597.54876628,1566.81833)
\lineto(597.54876628,1569.90425767)
\lineto(600.02922713,1569.90425767)
\lineto(600.02922713,1576.17376896)
\lineto(603.70109044,1576.17376896)
\lineto(603.70109044,1569.90425767)
\lineto(611.27919131,1569.90425767)
\lineto(611.27919131,1566.81833)
\lineto(603.70109044,1566.81833)
\lineto(603.70109044,1556.8769554)
\curveto(603.70109044,1555.73112571)(603.72713203,1554.83269108)(603.77921519,1554.18165148)
\curveto(603.83129836,1553.54363268)(604.01358945,1552.94467625)(604.32608845,1552.3847822)
\curveto(604.61254587,1551.86395053)(605.00316963,1551.47983717)(605.49795972,1551.23244212)
\curveto(606.0057706,1550.99806787)(606.77399733,1550.88088074)(607.80263988,1550.88088074)
\curveto(608.40159631,1550.88088074)(609.02659432,1550.96551589)(609.67763391,1551.13478618)
\curveto(610.32867351,1551.31707727)(610.79742201,1551.46681638)(611.08387944,1551.5840035)
\lineto(611.27919131,1551.5840035)
\closepath
}
}
{
\newrgbcolor{curcolor}{0 0 0}
\pscustom[linestyle=none,fillstyle=solid,fillcolor=curcolor]
{
\newpath
\moveto(621.70885254,1548.08792088)
\lineto(617.04089865,1548.08792088)
\lineto(617.04089865,1553.65430941)
\lineto(621.70885254,1553.65430941)
\closepath
}
}
{
\newrgbcolor{curcolor}{0 0 0}
\pscustom[linestyle=none,fillstyle=solid,fillcolor=curcolor]
{
\newpath
\moveto(649.46266389,1559.25976031)
\curveto(649.46266389,1557.48893262)(649.20875844,1555.86784403)(648.70094756,1554.39649455)
\curveto(648.19313668,1552.93816586)(647.47699312,1551.70119063)(646.5525169,1550.68556886)
\curveto(645.69314464,1549.72203026)(644.67752287,1548.97333473)(643.5056516,1548.43948226)
\curveto(642.34680112,1547.91865059)(641.11633629,1547.65823475)(639.8142571,1547.65823475)
\curveto(638.68144821,1547.65823475)(637.65280565,1547.78193227)(636.72832943,1548.02932732)
\curveto(635.816874,1548.27672237)(634.88588738,1548.66083573)(633.93536957,1549.1816674)
\lineto(633.93536957,1540.0410715)
\lineto(630.26350626,1540.0410715)
\lineto(630.26350626,1569.90425767)
\lineto(633.93536957,1569.90425767)
\lineto(633.93536957,1567.6191087)
\curveto(634.91192896,1568.43941859)(636.00567548,1569.12301016)(637.21660913,1569.66988342)
\curveto(638.44056356,1570.22977747)(639.74264275,1570.50972449)(641.12284669,1570.50972449)
\curveto(643.75304665,1570.50972449)(645.79731097,1569.51363392)(647.25563966,1567.52145276)
\curveto(648.72698914,1565.54229239)(649.46266389,1562.78839491)(649.46266389,1559.25976031)
\closepath
\moveto(645.67361345,1559.16210437)
\curveto(645.67361345,1561.79230433)(645.22439613,1563.75844391)(644.32596149,1565.06052309)
\curveto(643.42752685,1566.36260228)(642.04732291,1567.01364188)(640.18534967,1567.01364188)
\curveto(639.13066553,1567.01364188)(638.06947099,1566.78577802)(637.00176606,1566.3300503)
\curveto(635.93406113,1565.87432259)(634.91192896,1565.27536616)(633.93536957,1564.53318102)
\lineto(633.93536957,1552.16993914)
\curveto(634.97703292,1551.70119063)(635.86895717,1551.38218123)(636.6111423,1551.21291093)
\curveto(637.36634823,1551.04364064)(638.2192101,1550.95900549)(639.16972791,1550.95900549)
\curveto(641.21399223,1550.95900549)(642.80903924,1551.64910746)(643.95486892,1553.0293114)
\curveto(645.10069861,1554.40951534)(645.67361345,1556.45377966)(645.67361345,1559.16210437)
\closepath
}
}
{
\newrgbcolor{curcolor}{0 0 0}
\pscustom[linestyle=none,fillstyle=solid,fillcolor=curcolor]
{
\newpath
\moveto(673.95477555,1569.90425767)
\lineto(661.22044109,1540.0410715)
\lineto(657.29467234,1540.0410715)
\lineto(661.35715941,1549.14260503)
\lineto(652.66578083,1569.90425767)
\lineto(656.65014314,1569.90425767)
\lineto(663.34934056,1553.73243416)
\lineto(670.10713155,1569.90425767)
\closepath
}
}
{
\newrgbcolor{curcolor}{0 0 0}
\pscustom[linestyle=none,fillstyle=solid,fillcolor=curcolor]
{
\newpath
\moveto(390.25998113,1526.0397727)
\lineto(383.75998605,1526.0397727)
\lineto(383.75998605,1527.92518794)
\lineto(390.25998113,1527.92518794)
\closepath
}
}
{
\newrgbcolor{curcolor}{0 0 0}
\pscustom[linestyle=none,fillstyle=solid,fillcolor=curcolor]
{
\newpath
\moveto(414.37454645,1521.34185959)
\curveto(413.52732487,1520.95297099)(412.60024224,1520.61269347)(411.59329856,1520.32102703)
\curveto(410.59329932,1520.03630502)(409.62455005,1519.89394402)(408.68705076,1519.89394402)
\curveto(407.47871834,1519.89394402)(406.37108029,1520.06061056)(405.3641366,1520.39394364)
\curveto(404.35719292,1520.72727672)(403.49955468,1521.22727634)(402.79122188,1521.8939425)
\curveto(402.07594465,1522.5675531)(401.52386173,1523.40783025)(401.13497314,1524.41477393)
\curveto(400.74608454,1525.42866205)(400.55164025,1526.61268893)(400.55164025,1527.96685458)
\curveto(400.55164025,1530.44601937)(401.27386192,1532.400879)(402.71830527,1533.83143347)
\curveto(404.16969306,1535.26893238)(406.15927489,1535.98768184)(408.68705076,1535.98768184)
\curveto(409.56899453,1535.98768184)(410.46829941,1535.88004303)(411.38496538,1535.66476542)
\curveto(412.3085758,1535.45643224)(413.3016306,1535.09879362)(414.3641298,1534.59184956)
\lineto(414.3641298,1532.14393475)
\lineto(414.17662994,1532.14393475)
\curveto(413.96135232,1532.31060129)(413.64885256,1532.52935112)(413.23913065,1532.80018425)
\curveto(412.82940873,1533.07101738)(412.42663126,1533.29671165)(412.03079823,1533.47726707)
\curveto(411.55163192,1533.69254469)(411.00649345,1533.86962789)(410.3953828,1534.00851667)
\curveto(409.79121659,1534.15434989)(409.10371711,1534.22726651)(408.33288436,1534.22726651)
\curveto(406.59677456,1534.22726651)(405.2217756,1533.66823915)(404.20788748,1532.55018444)
\curveto(403.2009438,1531.43907417)(402.69747196,1529.93213087)(402.69747196,1528.02935453)
\curveto(402.69747196,1526.0224116)(403.22524933,1524.45991278)(404.28080409,1523.34185807)
\curveto(405.33635885,1522.2307478)(406.77385776,1521.67519267)(408.59330083,1521.67519267)
\curveto(409.25996699,1521.67519267)(409.92316093,1521.74116484)(410.58288266,1521.87310919)
\curveto(411.24954882,1522.00505353)(411.83288171,1522.17519229)(412.33288133,1522.38352547)
\lineto(412.33288133,1526.18560592)
\lineto(408.17663448,1526.18560592)
\lineto(408.17663448,1527.99810455)
\lineto(414.37454645,1527.99810455)
\closepath
}
}
{
\newrgbcolor{curcolor}{0 0 0}
\pscustom[linestyle=none,fillstyle=solid,fillcolor=curcolor]
{
\newpath
\moveto(425.08287367,1529.69601993)
\lineto(424.97870708,1529.69601993)
\curveto(424.68704063,1529.76546433)(424.40231863,1529.8140754)(424.12454106,1529.84185316)
\curveto(423.85370793,1529.87657535)(423.53079151,1529.89393645)(423.15579179,1529.89393645)
\curveto(422.55162558,1529.89393645)(421.96829269,1529.75851989)(421.40579312,1529.48768676)
\curveto(420.84329354,1529.22379807)(420.30162729,1528.88004833)(419.78079435,1528.45643754)
\lineto(419.78079435,1520.19602712)
\lineto(417.82246249,1520.19602712)
\lineto(417.82246249,1531.83143498)
\lineto(419.78079435,1531.83143498)
\lineto(419.78079435,1530.11268629)
\curveto(420.55857154,1530.73768581)(421.2425988,1531.1786577)(421.83287613,1531.43560195)
\curveto(422.4300979,1531.69949064)(423.03773633,1531.83143498)(423.65579141,1531.83143498)
\curveto(423.99606893,1531.83143498)(424.24259653,1531.82101833)(424.39537419,1531.80018501)
\curveto(424.54815185,1531.78629613)(424.77731834,1531.75504615)(425.08287367,1531.70643508)
\closepath
}
}
{
\newrgbcolor{curcolor}{0 0 0}
\pscustom[linestyle=none,fillstyle=solid,fillcolor=curcolor]
{
\newpath
\moveto(435.59327905,1520.19602712)
\lineto(433.64536386,1520.19602712)
\lineto(433.64536386,1521.43560952)
\curveto(433.47175288,1521.31755405)(433.23564195,1521.15088751)(432.93703106,1520.93560989)
\curveto(432.64536462,1520.72727672)(432.36064261,1520.56061018)(432.08286504,1520.43561027)
\curveto(431.7564764,1520.27588817)(431.38147669,1520.14394383)(430.9578659,1520.03977724)
\curveto(430.5342551,1519.92866621)(430.0377277,1519.8731107)(429.46828369,1519.8731107)
\curveto(428.41967337,1519.8731107)(427.53078515,1520.22033266)(426.80161904,1520.91477658)
\curveto(426.07245292,1521.6092205)(425.70786987,1522.49463649)(425.70786987,1523.57102457)
\curveto(425.70786987,1524.45296834)(425.89536973,1525.16477336)(426.27036944,1525.70643962)
\curveto(426.6523136,1526.25505031)(427.19397985,1526.68560554)(427.89536821,1526.99810531)
\curveto(428.60370101,1527.31060507)(429.45439481,1527.52241047)(430.44744961,1527.63352149)
\curveto(431.44050442,1527.74463252)(432.50647583,1527.82796579)(433.64536386,1527.88352131)
\lineto(433.64536386,1528.18560441)
\curveto(433.64536386,1528.63004852)(433.56550281,1528.9981038)(433.40578071,1529.28977024)
\curveto(433.25300305,1529.58143669)(433.03078099,1529.81060318)(432.73911455,1529.97726972)
\curveto(432.46133698,1530.13699182)(432.1280039,1530.24463063)(431.7391153,1530.30018614)
\curveto(431.35022671,1530.35574166)(430.94397702,1530.38351941)(430.52036623,1530.38351941)
\curveto(430.00647773,1530.38351941)(429.43356149,1530.31407502)(428.80161753,1530.17518624)
\curveto(428.16967356,1530.04324189)(427.51689628,1529.8487976)(426.84328567,1529.59185335)
\lineto(426.73911909,1529.59185335)
\lineto(426.73911909,1531.58143517)
\curveto(427.12106324,1531.68560176)(427.67314616,1531.80018501)(428.39536783,1531.92518491)
\curveto(429.11758951,1532.05018482)(429.82939453,1532.11268477)(430.53078288,1532.11268477)
\curveto(431.35022671,1532.11268477)(432.06203173,1532.04324038)(432.66619794,1531.9043516)
\curveto(433.27730858,1531.77240725)(433.80508596,1531.54324076)(434.24953007,1531.21685212)
\curveto(434.68702974,1530.89740791)(435.02036282,1530.48421378)(435.24952931,1529.97726972)
\curveto(435.47869581,1529.47032566)(435.59327905,1528.84185391)(435.59327905,1528.09185448)
\closepath
\moveto(433.64536386,1523.06060829)
\lineto(433.64536386,1526.30018917)
\curveto(433.04814209,1526.26546697)(432.34328151,1526.21338368)(431.53078213,1526.14393929)
\curveto(430.72522718,1526.0744949)(430.08633878,1525.97380053)(429.61411691,1525.84185618)
\curveto(429.05161734,1525.68213408)(428.59675657,1525.43213427)(428.24953461,1525.09185675)
\curveto(427.90231265,1524.75852367)(427.72870167,1524.29671846)(427.72870167,1523.70644113)
\curveto(427.72870167,1523.03977497)(427.93009041,1522.53630313)(428.33286788,1522.19602561)
\curveto(428.73564535,1521.86269253)(429.35022822,1521.69602599)(430.17661649,1521.69602599)
\curveto(430.86411597,1521.69602599)(431.49258771,1521.82797033)(432.06203173,1522.09185902)
\curveto(432.63147574,1522.36269215)(433.15925312,1522.68560857)(433.64536386,1523.06060829)
\closepath
}
}
{
\newrgbcolor{curcolor}{0 0 0}
\pscustom[linestyle=none,fillstyle=solid,fillcolor=curcolor]
{
\newpath
\moveto(449.60369001,1526.10227265)
\curveto(449.60369001,1525.13005117)(449.46480123,1524.25505183)(449.18702366,1523.47727464)
\curveto(448.91619053,1522.69949745)(448.54813525,1522.04672017)(448.08285783,1521.51894279)
\curveto(447.58980264,1520.97033209)(447.04813639,1520.55713796)(446.45785906,1520.27936039)
\curveto(445.86758172,1520.00852726)(445.21827666,1519.8731107)(444.50994386,1519.8731107)
\curveto(443.85022214,1519.8731107)(443.27383369,1519.95297175)(442.7807785,1520.11269385)
\curveto(442.28772332,1520.26547151)(441.80161258,1520.47380469)(441.32244627,1520.73769338)
\lineto(441.19744637,1520.19602712)
\lineto(439.36411442,1520.19602712)
\lineto(439.36411442,1536.40434819)
\lineto(441.32244627,1536.40434819)
\lineto(441.32244627,1530.61268591)
\curveto(441.87105697,1531.06407445)(442.45438986,1531.43212973)(443.07244495,1531.71685174)
\curveto(443.69050004,1532.00851818)(444.38494396,1532.15435141)(445.15577671,1532.15435141)
\curveto(446.53077567,1532.15435141)(447.61410818,1531.62657403)(448.40577425,1530.57101927)
\curveto(449.20438476,1529.51546451)(449.60369001,1528.02588231)(449.60369001,1526.10227265)
\closepath
\moveto(447.5828582,1526.05018936)
\curveto(447.5828582,1527.4390772)(447.35369171,1528.49115973)(446.89535872,1529.20643697)
\curveto(446.43702574,1529.92865865)(445.69744296,1530.28976948)(444.6766104,1530.28976948)
\curveto(444.10716639,1530.28976948)(443.53077794,1530.16476958)(442.94744505,1529.91476977)
\curveto(442.36411215,1529.6717144)(441.8224459,1529.35574241)(441.32244627,1528.96685382)
\lineto(441.32244627,1522.3001922)
\curveto(441.87800141,1522.05019238)(442.35369549,1521.87658141)(442.74952853,1521.77935926)
\curveto(443.152306,1521.68213711)(443.60716677,1521.63352603)(444.11411083,1521.63352603)
\curveto(445.19744334,1521.63352603)(446.04466492,1521.98769243)(446.65577557,1522.69602523)
\curveto(447.27383066,1523.41130247)(447.5828582,1524.52935718)(447.5828582,1526.05018936)
\closepath
}
}
{
\newrgbcolor{curcolor}{0 0 0}
\pscustom[linestyle=none,fillstyle=solid,fillcolor=curcolor]
{
\newpath
\moveto(460.96826295,1523.55019125)
\curveto(460.96826295,1522.48769205)(460.52729106,1521.61616494)(459.64534728,1520.93560989)
\curveto(458.77034794,1520.25505485)(457.57243218,1519.91477733)(456.0516,1519.91477733)
\curveto(455.19048954,1519.91477733)(454.39882347,1520.0154717)(453.6766018,1520.21686044)
\curveto(452.96132456,1520.42519361)(452.36063057,1520.65088789)(451.87451983,1520.89394326)
\lineto(451.87451983,1523.09185826)
\lineto(451.97868641,1523.09185826)
\curveto(452.5967415,1522.62658084)(453.28424098,1522.25505334)(454.04118485,1521.97727577)
\curveto(454.79812873,1521.70644265)(455.52382262,1521.57102608)(456.21826654,1521.57102608)
\curveto(457.079377,1521.57102608)(457.7529876,1521.70991486)(458.23909834,1521.98769243)
\curveto(458.72520909,1522.26547)(458.96826446,1522.70296967)(458.96826446,1523.30019144)
\curveto(458.96826446,1523.75852443)(458.83632011,1524.10574639)(458.57243143,1524.34185732)
\curveto(458.30854274,1524.57796825)(457.80159868,1524.77935699)(457.05159924,1524.94602353)
\curveto(456.77382167,1525.00852348)(456.40923862,1525.08144009)(455.95785007,1525.16477336)
\curveto(455.51340596,1525.24810663)(455.10715627,1525.33838434)(454.73910099,1525.43560649)
\curveto(453.71826843,1525.70643962)(452.99257454,1526.10227265)(452.56201931,1526.62310559)
\curveto(452.13840852,1527.15088297)(451.92660312,1527.79671582)(451.92660312,1528.56060413)
\curveto(451.92660312,1529.03977043)(452.02382527,1529.49115898)(452.21826957,1529.91476977)
\curveto(452.4196583,1530.33838056)(452.72174141,1530.71685249)(453.12451888,1531.05018558)
\curveto(453.51340747,1531.37657422)(454.00646266,1531.63351847)(454.60368443,1531.82101833)
\curveto(455.20785064,1532.01546262)(455.88146124,1532.11268477)(456.62451623,1532.11268477)
\curveto(457.31896015,1532.11268477)(458.02034851,1532.02587928)(458.72868131,1531.8522683)
\curveto(459.44395854,1531.68560176)(460.03770809,1531.48074081)(460.50992996,1531.23768543)
\lineto(460.50992996,1529.14393702)
\lineto(460.40576337,1529.14393702)
\curveto(459.90576375,1529.5119923)(459.29812532,1529.82101984)(458.58284808,1530.07101965)
\curveto(457.86757085,1530.3279639)(457.16618249,1530.45643603)(456.47868301,1530.45643603)
\curveto(455.76340577,1530.45643603)(455.15923956,1530.31754724)(454.66618438,1530.03976967)
\curveto(454.1731292,1529.76893655)(453.92660161,1529.36268685)(453.92660161,1528.8210206)
\curveto(453.92660161,1528.34185429)(454.07590705,1527.98074345)(454.37451793,1527.73768808)
\curveto(454.66618438,1527.49463271)(455.13840625,1527.29671619)(455.79118353,1527.14393853)
\curveto(456.15229437,1527.06060526)(456.55507184,1526.97727199)(456.99951595,1526.89393872)
\curveto(457.4509045,1526.81060545)(457.82590421,1526.73421662)(458.1245151,1526.66477223)
\curveto(459.03423663,1526.45643905)(459.73562499,1526.09880043)(460.22868017,1525.59185637)
\curveto(460.72173535,1525.07796787)(460.96826295,1524.39741283)(460.96826295,1523.55019125)
\closepath
}
}
{
\newrgbcolor{curcolor}{0 0 0}
\pscustom[linestyle=none,fillstyle=solid,fillcolor=curcolor]
{
\newpath
\moveto(480.99949537,1520.19602712)
\lineto(479.04116352,1520.19602712)
\lineto(479.04116352,1526.82102211)
\curveto(479.04116352,1527.35574393)(479.00991355,1527.85574355)(478.94741359,1528.32102097)
\curveto(478.88491364,1528.79324284)(478.77033039,1529.16129812)(478.60366385,1529.42518681)
\curveto(478.43005287,1529.71685325)(478.18005306,1529.93213087)(477.85366442,1530.07101965)
\curveto(477.52727578,1530.21685287)(477.10366499,1530.28976948)(476.58283205,1530.28976948)
\curveto(476.04811023,1530.28976948)(475.48908288,1530.15782514)(474.90574998,1529.89393645)
\curveto(474.32241709,1529.63004776)(473.76338974,1529.29324246)(473.22866792,1528.88352055)
\lineto(473.22866792,1520.19602712)
\lineto(471.27033607,1520.19602712)
\lineto(471.27033607,1531.83143498)
\lineto(473.22866792,1531.83143498)
\lineto(473.22866792,1530.5397693)
\curveto(473.83977857,1531.04671336)(474.47172253,1531.44254639)(475.12449982,1531.7272684)
\curveto(475.7772771,1532.0119904)(476.44741548,1532.15435141)(477.13491496,1532.15435141)
\curveto(478.39185846,1532.15435141)(479.35019107,1531.77587947)(480.00991279,1531.0189356)
\curveto(480.66963451,1530.26199173)(480.99949537,1529.17171477)(480.99949537,1527.74810474)
\closepath
}
}
{
\newrgbcolor{curcolor}{0 0 0}
\pscustom[linestyle=none,fillstyle=solid,fillcolor=curcolor]
{
\newpath
\moveto(494.50990338,1525.81060621)
\lineto(485.9369932,1525.81060621)
\curveto(485.9369932,1525.09532897)(486.04463201,1524.47032944)(486.25990962,1523.93560763)
\curveto(486.47518724,1523.40783025)(486.7703259,1522.9738028)(487.14532562,1522.63352528)
\curveto(487.50643646,1522.3001922)(487.93351947,1522.05019238)(488.42657465,1521.88352584)
\curveto(488.92657427,1521.7168593)(489.47518497,1521.63352603)(490.07240674,1521.63352603)
\curveto(490.86407281,1521.63352603)(491.65921109,1521.78977592)(492.4578216,1522.10227568)
\curveto(493.26337655,1522.42171988)(493.83629278,1522.73421965)(494.1765703,1523.03977497)
\lineto(494.28073689,1523.03977497)
\lineto(494.28073689,1520.90435992)
\curveto(493.62101517,1520.62658235)(492.94740456,1520.39394364)(492.25990508,1520.20644378)
\curveto(491.5724056,1520.01894392)(490.85018393,1519.92519399)(490.09324006,1519.92519399)
\curveto(488.16268596,1519.92519399)(486.65574266,1520.44602693)(485.57241014,1521.48769281)
\curveto(484.48907763,1522.53630313)(483.94741137,1524.02241312)(483.94741137,1525.94602277)
\curveto(483.94741137,1527.84879911)(484.46477209,1529.35921463)(485.49949353,1530.47726934)
\curveto(486.54115941,1531.59532405)(487.90921393,1532.15435141)(489.60365709,1532.15435141)
\curveto(491.17310035,1532.15435141)(492.38143277,1531.69601842)(493.22865435,1530.77935245)
\curveto(494.08282037,1529.86268647)(494.50990338,1528.56060413)(494.50990338,1526.8731054)
\closepath
\moveto(492.60365482,1527.31060507)
\curveto(492.59671038,1528.33838207)(492.33629392,1529.13352036)(491.82240542,1529.69601993)
\curveto(491.31546135,1530.25851951)(490.54115638,1530.5397693)(489.49949051,1530.5397693)
\curveto(488.45088019,1530.5397693)(487.61407527,1530.23074175)(486.98907574,1529.61268666)
\curveto(486.37102065,1528.99463158)(486.02032647,1528.22727104)(485.9369932,1527.31060507)
\closepath
}
}
{
\newrgbcolor{curcolor}{0 0 0}
\pscustom[linestyle=none,fillstyle=solid,fillcolor=curcolor]
{
\newpath
\moveto(507.55155618,1520.19602712)
\lineto(505.08280805,1520.19602712)
\lineto(501.78072721,1524.66477374)
\lineto(498.45781306,1520.19602712)
\lineto(496.17656479,1520.19602712)
\lineto(500.71822802,1525.99810606)
\lineto(496.21823142,1531.83143498)
\lineto(498.68697955,1531.83143498)
\lineto(501.96822707,1527.43560498)
\lineto(505.25989125,1531.83143498)
\lineto(507.55155618,1531.83143498)
\lineto(502.97864297,1526.10227265)
\closepath
}
}
{
\newrgbcolor{curcolor}{0 0 0}
\pscustom[linestyle=none,fillstyle=solid,fillcolor=curcolor]
{
\newpath
\moveto(516.14530154,1520.30019371)
\curveto(515.77724626,1520.20297156)(515.37446879,1520.12311051)(514.93696912,1520.06061056)
\curveto(514.50641389,1519.9981106)(514.12099752,1519.96686063)(513.78072,1519.96686063)
\curveto(512.59322089,1519.96686063)(511.6904438,1520.28630483)(511.07238871,1520.92519324)
\curveto(510.45433362,1521.56408164)(510.14530608,1522.58838642)(510.14530608,1523.99810758)
\lineto(510.14530608,1530.1856029)
\lineto(508.82239041,1530.1856029)
\lineto(508.82239041,1531.83143498)
\lineto(510.14530608,1531.83143498)
\lineto(510.14530608,1535.17518246)
\lineto(512.10363793,1535.17518246)
\lineto(512.10363793,1531.83143498)
\lineto(516.14530154,1531.83143498)
\lineto(516.14530154,1530.1856029)
\lineto(512.10363793,1530.1856029)
\lineto(512.10363793,1524.88352357)
\curveto(512.10363793,1524.27241293)(512.11752681,1523.79324662)(512.14530457,1523.44602466)
\curveto(512.17308232,1523.10574714)(512.27030447,1522.78630294)(512.43697101,1522.48769205)
\curveto(512.58974867,1522.20991449)(512.79808185,1522.00505353)(513.06197054,1521.87310919)
\curveto(513.33280367,1521.74810928)(513.74252558,1521.68560933)(514.29113628,1521.68560933)
\curveto(514.61058048,1521.68560933)(514.94391356,1521.73074818)(515.29113552,1521.82102589)
\curveto(515.63835748,1521.91824804)(515.88835729,1521.99810909)(516.04113495,1522.06060904)
\lineto(516.14530154,1522.06060904)
\closepath
}
}
{
\newrgbcolor{curcolor}{0 0 0}
\pscustom[linestyle=none,fillstyle=solid,fillcolor=curcolor]
{
\newpath
\moveto(532.05154385,1520.30019371)
\curveto(531.68348857,1520.20297156)(531.2807111,1520.12311051)(530.84321143,1520.06061056)
\curveto(530.4126562,1519.9981106)(530.02723983,1519.96686063)(529.68696231,1519.96686063)
\curveto(528.4994632,1519.96686063)(527.59668611,1520.28630483)(526.97863102,1520.92519324)
\curveto(526.36057593,1521.56408164)(526.05154839,1522.58838642)(526.05154839,1523.99810758)
\lineto(526.05154839,1530.1856029)
\lineto(524.72863272,1530.1856029)
\lineto(524.72863272,1531.83143498)
\lineto(526.05154839,1531.83143498)
\lineto(526.05154839,1535.17518246)
\lineto(528.00988024,1535.17518246)
\lineto(528.00988024,1531.83143498)
\lineto(532.05154385,1531.83143498)
\lineto(532.05154385,1530.1856029)
\lineto(528.00988024,1530.1856029)
\lineto(528.00988024,1524.88352357)
\curveto(528.00988024,1524.27241293)(528.02376912,1523.79324662)(528.05154688,1523.44602466)
\curveto(528.07932463,1523.10574714)(528.17654678,1522.78630294)(528.34321332,1522.48769205)
\curveto(528.49599099,1522.20991449)(528.70432416,1522.00505353)(528.96821285,1521.87310919)
\curveto(529.23904598,1521.74810928)(529.64876789,1521.68560933)(530.19737859,1521.68560933)
\curveto(530.51682279,1521.68560933)(530.85015587,1521.73074818)(531.19737783,1521.82102589)
\curveto(531.54459979,1521.91824804)(531.7945996,1521.99810909)(531.94737726,1522.06060904)
\lineto(532.05154385,1522.06060904)
\closepath
}
}
{
\newrgbcolor{curcolor}{0 0 0}
\pscustom[linestyle=none,fillstyle=solid,fillcolor=curcolor]
{
\newpath
\moveto(544.15570003,1525.81060621)
\lineto(535.58278985,1525.81060621)
\curveto(535.58278985,1525.09532897)(535.69042866,1524.47032944)(535.90570627,1523.93560763)
\curveto(536.12098389,1523.40783025)(536.41612255,1522.9738028)(536.79112227,1522.63352528)
\curveto(537.15223311,1522.3001922)(537.57931612,1522.05019238)(538.0723713,1521.88352584)
\curveto(538.57237092,1521.7168593)(539.12098162,1521.63352603)(539.71820339,1521.63352603)
\curveto(540.50986946,1521.63352603)(541.30500774,1521.78977592)(542.10361825,1522.10227568)
\curveto(542.9091732,1522.42171988)(543.48208943,1522.73421965)(543.82236695,1523.03977497)
\lineto(543.92653354,1523.03977497)
\lineto(543.92653354,1520.90435992)
\curveto(543.26681181,1520.62658235)(542.59320121,1520.39394364)(541.90570173,1520.20644378)
\curveto(541.21820225,1520.01894392)(540.49598058,1519.92519399)(539.7390367,1519.92519399)
\curveto(537.80848261,1519.92519399)(536.30153931,1520.44602693)(535.21820679,1521.48769281)
\curveto(534.13487428,1522.53630313)(533.59320802,1524.02241312)(533.59320802,1525.94602277)
\curveto(533.59320802,1527.84879911)(534.11056874,1529.35921463)(535.14529018,1530.47726934)
\curveto(536.18695606,1531.59532405)(537.55501058,1532.15435141)(539.24945374,1532.15435141)
\curveto(540.818897,1532.15435141)(542.02722942,1531.69601842)(542.874451,1530.77935245)
\curveto(543.72861702,1529.86268647)(544.15570003,1528.56060413)(544.15570003,1526.8731054)
\closepath
\moveto(542.24945147,1527.31060507)
\curveto(542.24250703,1528.33838207)(541.98209056,1529.13352036)(541.46820206,1529.69601993)
\curveto(540.961258,1530.25851951)(540.18695303,1530.5397693)(539.14528715,1530.5397693)
\curveto(538.09667684,1530.5397693)(537.25987191,1530.23074175)(536.63487239,1529.61268666)
\curveto(536.0168173,1528.99463158)(535.66612312,1528.22727104)(535.58278985,1527.31060507)
\closepath
}
}
{
\newrgbcolor{curcolor}{0 0 0}
\pscustom[linestyle=none,fillstyle=solid,fillcolor=curcolor]
{
\newpath
\moveto(555.4369447,1523.55019125)
\curveto(555.4369447,1522.48769205)(554.99597281,1521.61616494)(554.11402904,1520.93560989)
\curveto(553.2390297,1520.25505485)(552.04111394,1519.91477733)(550.52028175,1519.91477733)
\curveto(549.6591713,1519.91477733)(548.86750523,1520.0154717)(548.14528355,1520.21686044)
\curveto(547.43000632,1520.42519361)(546.82931233,1520.65088789)(546.34320158,1520.89394326)
\lineto(546.34320158,1523.09185826)
\lineto(546.44736817,1523.09185826)
\curveto(547.06542326,1522.62658084)(547.75292274,1522.25505334)(548.50986661,1521.97727577)
\curveto(549.26681048,1521.70644265)(549.99250438,1521.57102608)(550.6869483,1521.57102608)
\curveto(551.54805876,1521.57102608)(552.22166936,1521.70991486)(552.7077801,1521.98769243)
\curveto(553.19389084,1522.26547)(553.43694622,1522.70296967)(553.43694622,1523.30019144)
\curveto(553.43694622,1523.75852443)(553.30500187,1524.10574639)(553.04111318,1524.34185732)
\curveto(552.77722449,1524.57796825)(552.27028043,1524.77935699)(551.520281,1524.94602353)
\curveto(551.24250343,1525.00852348)(550.87792037,1525.08144009)(550.42653183,1525.16477336)
\curveto(549.98208772,1525.24810663)(549.57583803,1525.33838434)(549.20778275,1525.43560649)
\curveto(548.18695019,1525.70643962)(547.46125629,1526.10227265)(547.03070106,1526.62310559)
\curveto(546.60709027,1527.15088297)(546.39528488,1527.79671582)(546.39528488,1528.56060413)
\curveto(546.39528488,1529.03977043)(546.49250702,1529.49115898)(546.68695132,1529.91476977)
\curveto(546.88834006,1530.33838056)(547.19042316,1530.71685249)(547.59320064,1531.05018558)
\curveto(547.98208923,1531.37657422)(548.47514441,1531.63351847)(549.07236618,1531.82101833)
\curveto(549.67653239,1532.01546262)(550.35014299,1532.11268477)(551.09319799,1532.11268477)
\curveto(551.78764191,1532.11268477)(552.48903027,1532.02587928)(553.19736306,1531.8522683)
\curveto(553.9126403,1531.68560176)(554.50638985,1531.48074081)(554.97861172,1531.23768543)
\lineto(554.97861172,1529.14393702)
\lineto(554.87444513,1529.14393702)
\curveto(554.37444551,1529.5119923)(553.76680708,1529.82101984)(553.05152984,1530.07101965)
\curveto(552.3362526,1530.3279639)(551.63486425,1530.45643603)(550.94736477,1530.45643603)
\curveto(550.23208753,1530.45643603)(549.62792132,1530.31754724)(549.13486614,1530.03976967)
\curveto(548.64181095,1529.76893655)(548.39528336,1529.36268685)(548.39528336,1528.8210206)
\curveto(548.39528336,1528.34185429)(548.54458881,1527.98074345)(548.84319969,1527.73768808)
\curveto(549.13486614,1527.49463271)(549.607088,1527.29671619)(550.25986529,1527.14393853)
\curveto(550.62097612,1527.06060526)(551.0237536,1526.97727199)(551.4681977,1526.89393872)
\curveto(551.91958625,1526.81060545)(552.29458597,1526.73421662)(552.59319685,1526.66477223)
\curveto(553.50291839,1526.45643905)(554.20430675,1526.09880043)(554.69736193,1525.59185637)
\curveto(555.19041711,1525.07796787)(555.4369447,1524.39741283)(555.4369447,1523.55019125)
\closepath
}
}
{
\newrgbcolor{curcolor}{0 0 0}
\pscustom[linestyle=none,fillstyle=solid,fillcolor=curcolor]
{
\newpath
\moveto(564.28068713,1520.30019371)
\curveto(563.91263185,1520.20297156)(563.50985438,1520.12311051)(563.07235471,1520.06061056)
\curveto(562.64179948,1519.9981106)(562.2563831,1519.96686063)(561.91610558,1519.96686063)
\curveto(560.72860648,1519.96686063)(559.82582939,1520.28630483)(559.2077743,1520.92519324)
\curveto(558.58971921,1521.56408164)(558.28069167,1522.58838642)(558.28069167,1523.99810758)
\lineto(558.28069167,1530.1856029)
\lineto(556.957776,1530.1856029)
\lineto(556.957776,1531.83143498)
\lineto(558.28069167,1531.83143498)
\lineto(558.28069167,1535.17518246)
\lineto(560.23902352,1535.17518246)
\lineto(560.23902352,1531.83143498)
\lineto(564.28068713,1531.83143498)
\lineto(564.28068713,1530.1856029)
\lineto(560.23902352,1530.1856029)
\lineto(560.23902352,1524.88352357)
\curveto(560.23902352,1524.27241293)(560.2529124,1523.79324662)(560.28069015,1523.44602466)
\curveto(560.30846791,1523.10574714)(560.40569006,1522.78630294)(560.5723566,1522.48769205)
\curveto(560.72513426,1522.20991449)(560.93346744,1522.00505353)(561.19735613,1521.87310919)
\curveto(561.46818926,1521.74810928)(561.87791117,1521.68560933)(562.42652186,1521.68560933)
\curveto(562.74596607,1521.68560933)(563.07929915,1521.73074818)(563.42652111,1521.82102589)
\curveto(563.77374307,1521.91824804)(564.02374288,1521.99810909)(564.17652054,1522.06060904)
\lineto(564.28068713,1522.06060904)
\closepath
}
}
{
\newrgbcolor{curcolor}{0 0 0}
\pscustom[linestyle=none,fillstyle=solid,fillcolor=curcolor]
{
\newpath
\moveto(580.40566486,1534.50851629)
\lineto(580.30149827,1534.50851629)
\curveto(580.08622065,1534.57101625)(579.80497087,1534.6335162)(579.45774891,1534.69601615)
\curveto(579.11052695,1534.76546054)(578.80497162,1534.80018274)(578.54108293,1534.80018274)
\curveto(577.70080579,1534.80018274)(577.08969514,1534.61268288)(576.70775099,1534.23768316)
\curveto(576.33275127,1533.86962789)(576.14525141,1533.19948951)(576.14525141,1532.22726802)
\lineto(576.14525141,1531.83143498)
\lineto(579.67649874,1531.83143498)
\lineto(579.67649874,1530.1856029)
\lineto(576.20775137,1530.1856029)
\lineto(576.20775137,1520.19602712)
\lineto(574.24941951,1520.19602712)
\lineto(574.24941951,1530.1856029)
\lineto(572.92650385,1530.1856029)
\lineto(572.92650385,1531.83143498)
\lineto(574.24941951,1531.83143498)
\lineto(574.24941951,1532.21685136)
\curveto(574.24941951,1533.59879476)(574.59316925,1534.65782174)(575.28066873,1535.39393229)
\curveto(575.96816821,1536.13698728)(576.96122302,1536.50851478)(578.25983315,1536.50851478)
\curveto(578.69733281,1536.50851478)(579.08969363,1536.48768146)(579.43691559,1536.44601483)
\curveto(579.79108199,1536.40434819)(580.11399841,1536.35573712)(580.40566486,1536.3001816)
\closepath
}
}
{
\newrgbcolor{curcolor}{0 0 0}
\pscustom[linestyle=none,fillstyle=solid,fillcolor=curcolor]
{
\newpath
\moveto(583.749418,1533.77935018)
\lineto(581.54108634,1533.77935018)
\lineto(581.54108634,1535.81059864)
\lineto(583.749418,1535.81059864)
\closepath
\moveto(583.62441809,1520.19602712)
\lineto(581.66608624,1520.19602712)
\lineto(581.66608624,1531.83143498)
\lineto(583.62441809,1531.83143498)
\closepath
}
}
{
\newrgbcolor{curcolor}{0 0 0}
\pscustom[linestyle=none,fillstyle=solid,fillcolor=curcolor]
{
\newpath
\moveto(589.47857387,1520.19602712)
\lineto(587.52024202,1520.19602712)
\lineto(587.52024202,1536.40434819)
\lineto(589.47857387,1536.40434819)
\closepath
}
}
{
\newrgbcolor{curcolor}{0 0 0}
\pscustom[linestyle=none,fillstyle=solid,fillcolor=curcolor]
{
\newpath
\moveto(603.09315485,1525.81060621)
\lineto(594.52024467,1525.81060621)
\curveto(594.52024467,1525.09532897)(594.62788348,1524.47032944)(594.8431611,1523.93560763)
\curveto(595.05843871,1523.40783025)(595.35357738,1522.9738028)(595.72857709,1522.63352528)
\curveto(596.08968793,1522.3001922)(596.51677094,1522.05019238)(597.00982612,1521.88352584)
\curveto(597.50982575,1521.7168593)(598.05843644,1521.63352603)(598.65565821,1521.63352603)
\curveto(599.44732428,1521.63352603)(600.24246257,1521.78977592)(601.04107307,1522.10227568)
\curveto(601.84662802,1522.42171988)(602.41954425,1522.73421965)(602.75982177,1523.03977497)
\lineto(602.86398836,1523.03977497)
\lineto(602.86398836,1520.90435992)
\curveto(602.20426664,1520.62658235)(601.53065604,1520.39394364)(600.84315656,1520.20644378)
\curveto(600.15565708,1520.01894392)(599.4334354,1519.92519399)(598.67649153,1519.92519399)
\curveto(596.74593743,1519.92519399)(595.23899413,1520.44602693)(594.15566162,1521.48769281)
\curveto(593.0723291,1522.53630313)(592.53066285,1524.02241312)(592.53066285,1525.94602277)
\curveto(592.53066285,1527.84879911)(593.04802357,1529.35921463)(594.082745,1530.47726934)
\curveto(595.12441088,1531.59532405)(596.4924654,1532.15435141)(598.18690857,1532.15435141)
\curveto(599.75635182,1532.15435141)(600.96468424,1531.69601842)(601.81190582,1530.77935245)
\curveto(602.66607184,1529.86268647)(603.09315485,1528.56060413)(603.09315485,1526.8731054)
\closepath
\moveto(601.1869063,1527.31060507)
\curveto(601.17996186,1528.33838207)(600.91954539,1529.13352036)(600.40565689,1529.69601993)
\curveto(599.89871283,1530.25851951)(599.12440786,1530.5397693)(598.08274198,1530.5397693)
\curveto(597.03413166,1530.5397693)(596.19732674,1530.23074175)(595.57232721,1529.61268666)
\curveto(594.95427212,1528.99463158)(594.60357794,1528.22727104)(594.52024467,1527.31060507)
\closepath
}
}
{
\newrgbcolor{curcolor}{0 0 0}
\pscustom[linestyle=none,fillstyle=solid,fillcolor=curcolor]
{
\newpath
\moveto(390.25998113,1499.37312621)
\lineto(383.75998605,1499.37312621)
\lineto(383.75998605,1501.25854145)
\lineto(390.25998113,1501.25854145)
\closepath
}
}
{
\newrgbcolor{curcolor}{0 0 0}
\pscustom[linestyle=none,fillstyle=solid,fillcolor=curcolor]
{
\newpath
\moveto(414.54121299,1501.26895811)
\curveto(414.54121299,1499.85923695)(414.23218545,1498.58146014)(413.61413036,1497.43562767)
\curveto(413.00301971,1496.28979521)(412.18704811,1495.40090699)(411.16621555,1494.76896302)
\curveto(410.45788275,1494.33146336)(409.66621668,1494.01549137)(408.79121735,1493.82104708)
\curveto(407.92316245,1493.62660278)(406.77732998,1493.52938063)(405.35371995,1493.52938063)
\lineto(401.43705624,1493.52938063)
\lineto(401.43705624,1509.03978556)
\lineto(405.31205331,1509.03978556)
\curveto(406.82594105,1509.03978556)(408.02732903,1508.92867453)(408.91621725,1508.70645248)
\curveto(409.81204991,1508.49117487)(410.56899378,1508.19256398)(411.18704887,1507.81061982)
\curveto(412.24260362,1507.1508981)(413.06551967,1506.27242654)(413.655797,1505.17520515)
\curveto(414.24607433,1504.07798376)(414.54121299,1502.77590141)(414.54121299,1501.26895811)
\closepath
\moveto(412.38496463,1501.30020808)
\curveto(412.38496463,1502.51548494)(412.17315923,1503.53978972)(411.74954844,1504.37312243)
\curveto(411.32593765,1505.20645513)(410.69399368,1505.86270463)(409.85371654,1506.34187094)
\curveto(409.24260589,1506.6890929)(408.59330083,1506.92867605)(407.90580135,1507.06062039)
\curveto(407.21830187,1507.19950918)(406.39538582,1507.26895357)(405.43705322,1507.26895357)
\lineto(403.49955468,1507.26895357)
\lineto(403.49955468,1495.30021262)
\lineto(405.43705322,1495.30021262)
\curveto(406.43010802,1495.30021262)(407.2946907,1495.37312923)(408.03080125,1495.51896246)
\curveto(408.77385625,1495.66479568)(409.45441129,1495.93562881)(410.07246638,1496.33146184)
\curveto(410.84329913,1496.82451703)(411.41968758,1497.47382209)(411.80163173,1498.27937704)
\curveto(412.19052033,1499.08493198)(412.38496463,1500.09187566)(412.38496463,1501.30020808)
\closepath
}
}
{
\newrgbcolor{curcolor}{0 0 0}
\pscustom[linestyle=none,fillstyle=solid,fillcolor=curcolor]
{
\newpath
\moveto(427.45787019,1499.14395971)
\lineto(418.88496001,1499.14395971)
\curveto(418.88496001,1498.42868248)(418.99259882,1497.80368295)(419.20787643,1497.26896113)
\curveto(419.42315405,1496.74118375)(419.71829271,1496.30715631)(420.09329243,1495.96687879)
\curveto(420.45440327,1495.6335457)(420.88148628,1495.38354589)(421.37454146,1495.21687935)
\curveto(421.87454108,1495.05021281)(422.42315178,1494.96687954)(423.02037355,1494.96687954)
\curveto(423.81203962,1494.96687954)(424.6071779,1495.12312942)(425.40578841,1495.43562919)
\curveto(426.21134336,1495.75507339)(426.78425959,1496.06757315)(427.12453711,1496.37312848)
\lineto(427.2287037,1496.37312848)
\lineto(427.2287037,1494.23771343)
\curveto(426.56898197,1493.95993586)(425.89537137,1493.72729715)(425.20787189,1493.53979729)
\curveto(424.52037241,1493.35229743)(423.79815074,1493.2585475)(423.04120687,1493.2585475)
\curveto(421.11065277,1493.2585475)(419.60370947,1493.77938044)(418.52037695,1494.82104632)
\curveto(417.43704444,1495.86965664)(416.89537818,1497.35576662)(416.89537818,1499.27937628)
\curveto(416.89537818,1501.18215262)(417.4127389,1502.69256814)(418.44746034,1503.81062285)
\curveto(419.48912622,1504.92867756)(420.85718074,1505.48770492)(422.5516239,1505.48770492)
\curveto(424.12106716,1505.48770492)(425.32939958,1505.02937193)(426.17662116,1504.11270596)
\curveto(427.03078718,1503.19603998)(427.45787019,1501.89395763)(427.45787019,1500.20645891)
\closepath
\moveto(425.55162163,1500.64395858)
\curveto(425.54467719,1501.67173558)(425.28426072,1502.46687387)(424.77037222,1503.02937344)
\curveto(424.26342816,1503.59187302)(423.48912319,1503.8731228)(422.44745731,1503.8731228)
\curveto(421.398847,1503.8731228)(420.56204207,1503.56409526)(419.93704255,1502.94604017)
\curveto(419.31898746,1502.32798508)(418.96829328,1501.56062455)(418.88496001,1500.64395858)
\closepath
}
}
{
\newrgbcolor{curcolor}{0 0 0}
\pscustom[linestyle=none,fillstyle=solid,fillcolor=curcolor]
{
\newpath
\moveto(436.46827964,1493.63354722)
\curveto(436.10022437,1493.53632507)(435.69744689,1493.45646402)(435.25994722,1493.39396407)
\curveto(434.82939199,1493.33146411)(434.44397562,1493.30021414)(434.1036981,1493.30021414)
\curveto(432.916199,1493.30021414)(432.0134219,1493.61965834)(431.39536681,1494.25854674)
\curveto(430.77731173,1494.89743515)(430.46828418,1495.92173993)(430.46828418,1497.33146109)
\lineto(430.46828418,1503.5189564)
\lineto(429.14536852,1503.5189564)
\lineto(429.14536852,1505.16478849)
\lineto(430.46828418,1505.16478849)
\lineto(430.46828418,1508.50853596)
\lineto(432.42661603,1508.50853596)
\lineto(432.42661603,1505.16478849)
\lineto(436.46827964,1505.16478849)
\lineto(436.46827964,1503.5189564)
\lineto(432.42661603,1503.5189564)
\lineto(432.42661603,1498.21687708)
\curveto(432.42661603,1497.60576643)(432.44050491,1497.12660013)(432.46828267,1496.77937817)
\curveto(432.49606043,1496.43910065)(432.59328257,1496.11965645)(432.75994911,1495.82104556)
\curveto(432.91272678,1495.54326799)(433.12105995,1495.33840704)(433.38494864,1495.20646269)
\curveto(433.65578177,1495.08146279)(434.06550368,1495.01896284)(434.61411438,1495.01896284)
\curveto(434.93355858,1495.01896284)(435.26689166,1495.06410169)(435.61411362,1495.1543794)
\curveto(435.96133558,1495.25160155)(436.21133539,1495.3314626)(436.36411305,1495.39396255)
\lineto(436.46827964,1495.39396255)
\closepath
}
}
{
\newrgbcolor{curcolor}{0 0 0}
\pscustom[linestyle=none,fillstyle=solid,fillcolor=curcolor]
{
\newpath
\moveto(448.57243943,1499.14395971)
\lineto(439.99952925,1499.14395971)
\curveto(439.99952925,1498.42868248)(440.10716805,1497.80368295)(440.32244567,1497.26896113)
\curveto(440.53772328,1496.74118375)(440.83286195,1496.30715631)(441.20786166,1495.96687879)
\curveto(441.5689725,1495.6335457)(441.99605551,1495.38354589)(442.4891107,1495.21687935)
\curveto(442.98911032,1495.05021281)(443.53772101,1494.96687954)(444.13494278,1494.96687954)
\curveto(444.92660885,1494.96687954)(445.72174714,1495.12312942)(446.52035765,1495.43562919)
\curveto(447.32591259,1495.75507339)(447.89882882,1496.06757315)(448.23910635,1496.37312848)
\lineto(448.34327293,1496.37312848)
\lineto(448.34327293,1494.23771343)
\curveto(447.68355121,1493.95993586)(447.00994061,1493.72729715)(446.32244113,1493.53979729)
\curveto(445.63494165,1493.35229743)(444.91271997,1493.2585475)(444.1557761,1493.2585475)
\curveto(442.22522201,1493.2585475)(440.7182787,1493.77938044)(439.63494619,1494.82104632)
\curveto(438.55161367,1495.86965664)(438.00994742,1497.35576662)(438.00994742,1499.27937628)
\curveto(438.00994742,1501.18215262)(438.52730814,1502.69256814)(439.56202958,1503.81062285)
\curveto(440.60369546,1504.92867756)(441.97174998,1505.48770492)(443.66619314,1505.48770492)
\curveto(445.2356364,1505.48770492)(446.44396881,1505.02937193)(447.2911904,1504.11270596)
\curveto(448.14535642,1503.19603998)(448.57243943,1501.89395763)(448.57243943,1500.20645891)
\closepath
\moveto(446.66619087,1500.64395858)
\curveto(446.65924643,1501.67173558)(446.39882996,1502.46687387)(445.88494146,1503.02937344)
\curveto(445.3779974,1503.59187302)(444.60369243,1503.8731228)(443.56202655,1503.8731228)
\curveto(442.51341623,1503.8731228)(441.67661131,1503.56409526)(441.05161178,1502.94604017)
\curveto(440.43355669,1502.32798508)(440.08286252,1501.56062455)(439.99952925,1500.64395858)
\closepath
}
}
{
\newrgbcolor{curcolor}{0 0 0}
\pscustom[linestyle=none,fillstyle=solid,fillcolor=curcolor]
{
\newpath
\moveto(458.80159435,1503.02937344)
\lineto(458.69742776,1503.02937344)
\curveto(458.40576132,1503.09881783)(458.12103931,1503.14742891)(457.84326174,1503.17520666)
\curveto(457.57242862,1503.20992886)(457.24951219,1503.22728996)(456.87451248,1503.22728996)
\curveto(456.27034627,1503.22728996)(455.68701337,1503.09187339)(455.1245138,1502.82104027)
\curveto(454.56201423,1502.55715158)(454.02034797,1502.21340184)(453.49951503,1501.78979105)
\lineto(453.49951503,1493.52938063)
\lineto(451.54118318,1493.52938063)
\lineto(451.54118318,1505.16478849)
\lineto(453.49951503,1505.16478849)
\lineto(453.49951503,1503.44603979)
\curveto(454.27729222,1504.07103932)(454.96131948,1504.51201121)(455.55159681,1504.76895546)
\curveto(456.14881858,1505.03284415)(456.75645701,1505.16478849)(457.3745121,1505.16478849)
\curveto(457.71478962,1505.16478849)(457.96131721,1505.15437183)(458.11409487,1505.13353852)
\curveto(458.26687253,1505.11964964)(458.49603903,1505.08839966)(458.80159435,1505.03978859)
\closepath
}
}
{
\newrgbcolor{curcolor}{0 0 0}
\pscustom[linestyle=none,fillstyle=solid,fillcolor=curcolor]
{
\newpath
\moveto(477.6244995,1493.52938063)
\lineto(475.66616765,1493.52938063)
\lineto(475.66616765,1500.15437562)
\curveto(475.66616765,1500.65437524)(475.64186211,1501.13701376)(475.59325104,1501.60229119)
\curveto(475.5515844,1502.06756861)(475.45783447,1502.43909611)(475.31200125,1502.71687368)
\curveto(475.15227915,1503.01548456)(474.92311266,1503.24117884)(474.62450177,1503.3939565)
\curveto(474.32589089,1503.54673416)(473.89533566,1503.62312299)(473.33283608,1503.62312299)
\curveto(472.78422538,1503.62312299)(472.23561469,1503.48423421)(471.68700399,1503.20645664)
\curveto(471.1383933,1502.93562351)(470.5897826,1502.58840155)(470.0411719,1502.16479076)
\curveto(470.06200522,1502.00506866)(470.07936632,1501.8175688)(470.0932552,1501.60229119)
\curveto(470.10714408,1501.39395801)(470.11408852,1501.18562484)(470.11408852,1500.97729166)
\lineto(470.11408852,1493.52938063)
\lineto(468.15575666,1493.52938063)
\lineto(468.15575666,1500.15437562)
\curveto(468.15575666,1500.66826412)(468.13145113,1501.15437486)(468.08284005,1501.61270785)
\curveto(468.04117342,1502.07798527)(467.94742349,1502.44951277)(467.80159027,1502.72729034)
\curveto(467.64186816,1503.02590122)(467.41270167,1503.24812328)(467.11409079,1503.3939565)
\curveto(466.8154799,1503.54673416)(466.38492467,1503.62312299)(465.8224251,1503.62312299)
\curveto(465.28770328,1503.62312299)(464.74950924,1503.49117865)(464.20784298,1503.22728996)
\curveto(463.67312117,1502.96340127)(463.13839935,1502.62659597)(462.60367753,1502.21687406)
\lineto(462.60367753,1493.52938063)
\lineto(460.64534568,1493.52938063)
\lineto(460.64534568,1505.16478849)
\lineto(462.60367753,1505.16478849)
\lineto(462.60367753,1503.8731228)
\curveto(463.21478818,1504.38006686)(463.82242661,1504.7758999)(464.42659282,1505.06062191)
\curveto(465.03770347,1505.34534391)(465.68700853,1505.48770492)(466.37450801,1505.48770492)
\curveto(467.16617408,1505.48770492)(467.83631246,1505.32103837)(468.38492316,1504.98770529)
\curveto(468.94047829,1504.65437221)(469.35367242,1504.19256701)(469.62450555,1503.60228968)
\curveto(470.41617162,1504.26895584)(471.1383933,1504.74812214)(471.79117058,1505.03978859)
\curveto(472.44394786,1505.33839947)(473.141864,1505.48770492)(473.884919,1505.48770492)
\curveto(475.16269581,1505.48770492)(476.10366732,1505.09881632)(476.70783353,1504.32103913)
\curveto(477.31894418,1503.55020638)(477.6244995,1502.47034609)(477.6244995,1501.08145825)
\closepath
}
}
{
\newrgbcolor{curcolor}{0 0 0}
\pscustom[linestyle=none,fillstyle=solid,fillcolor=curcolor]
{
\newpath
\moveto(483.49949105,1507.11270369)
\lineto(481.29115939,1507.11270369)
\lineto(481.29115939,1509.14395215)
\lineto(483.49949105,1509.14395215)
\closepath
\moveto(483.37449114,1493.52938063)
\lineto(481.41615929,1493.52938063)
\lineto(481.41615929,1505.16478849)
\lineto(483.37449114,1505.16478849)
\closepath
}
}
{
\newrgbcolor{curcolor}{0 0 0}
\pscustom[linestyle=none,fillstyle=solid,fillcolor=curcolor]
{
\newpath
\moveto(496.97864826,1493.52938063)
\lineto(495.02031641,1493.52938063)
\lineto(495.02031641,1500.15437562)
\curveto(495.02031641,1500.68909743)(494.98906644,1501.18909706)(494.92656648,1501.65437448)
\curveto(494.86406653,1502.12659635)(494.74948328,1502.49465162)(494.58281674,1502.75854031)
\curveto(494.40920576,1503.05020676)(494.15920595,1503.26548437)(493.83281731,1503.40437316)
\curveto(493.50642867,1503.55020638)(493.08281788,1503.62312299)(492.56198494,1503.62312299)
\curveto(492.02726312,1503.62312299)(491.46823577,1503.49117865)(490.88490287,1503.22728996)
\curveto(490.30156998,1502.96340127)(489.74254263,1502.62659597)(489.20782081,1502.21687406)
\lineto(489.20782081,1493.52938063)
\lineto(487.24948896,1493.52938063)
\lineto(487.24948896,1505.16478849)
\lineto(489.20782081,1505.16478849)
\lineto(489.20782081,1503.8731228)
\curveto(489.81893146,1504.38006686)(490.45087542,1504.7758999)(491.10365271,1505.06062191)
\curveto(491.75642999,1505.34534391)(492.42656837,1505.48770492)(493.11406785,1505.48770492)
\curveto(494.37101135,1505.48770492)(495.32934396,1505.10923298)(495.98906568,1504.35228911)
\curveto(496.6487874,1503.59534524)(496.97864826,1502.50506828)(496.97864826,1501.08145825)
\closepath
}
}
{
\newrgbcolor{curcolor}{0 0 0}
\pscustom[linestyle=none,fillstyle=solid,fillcolor=curcolor]
{
\newpath
\moveto(510.48904906,1499.14395971)
\lineto(501.91613888,1499.14395971)
\curveto(501.91613888,1498.42868248)(502.02377769,1497.80368295)(502.2390553,1497.26896113)
\curveto(502.45433292,1496.74118375)(502.74947158,1496.30715631)(503.1244713,1495.96687879)
\curveto(503.48558214,1495.6335457)(503.91266515,1495.38354589)(504.40572033,1495.21687935)
\curveto(504.90571995,1495.05021281)(505.45433065,1494.96687954)(506.05155242,1494.96687954)
\curveto(506.84321849,1494.96687954)(507.63835678,1495.12312942)(508.43696728,1495.43562919)
\curveto(509.24252223,1495.75507339)(509.81543846,1496.06757315)(510.15571598,1496.37312848)
\lineto(510.25988257,1496.37312848)
\lineto(510.25988257,1494.23771343)
\curveto(509.60016085,1493.95993586)(508.92655024,1493.72729715)(508.23905077,1493.53979729)
\curveto(507.55155129,1493.35229743)(506.82932961,1493.2585475)(506.07238574,1493.2585475)
\curveto(504.14183164,1493.2585475)(502.63488834,1493.77938044)(501.55155582,1494.82104632)
\curveto(500.46822331,1495.86965664)(499.92655705,1497.35576662)(499.92655705,1499.27937628)
\curveto(499.92655705,1501.18215262)(500.44391777,1502.69256814)(501.47863921,1503.81062285)
\curveto(502.52030509,1504.92867756)(503.88835961,1505.48770492)(505.58280277,1505.48770492)
\curveto(507.15224603,1505.48770492)(508.36057845,1505.02937193)(509.20780003,1504.11270596)
\curveto(510.06196605,1503.19603998)(510.48904906,1501.89395763)(510.48904906,1500.20645891)
\closepath
\moveto(508.5828005,1500.64395858)
\curveto(508.57585607,1501.67173558)(508.3154396,1502.46687387)(507.8015511,1503.02937344)
\curveto(507.29460704,1503.59187302)(506.52030207,1503.8731228)(505.47863619,1503.8731228)
\curveto(504.43002587,1503.8731228)(503.59322095,1503.56409526)(502.96822142,1502.94604017)
\curveto(502.35016633,1502.32798508)(501.99947215,1501.56062455)(501.91613888,1500.64395858)
\closepath
}
}
{
\newrgbcolor{curcolor}{0 0 0}
\pscustom[linestyle=none,fillstyle=solid,fillcolor=curcolor]
{
\newpath
\moveto(521.77029373,1496.88354476)
\curveto(521.77029373,1495.82104556)(521.32932185,1494.94951844)(520.44737807,1494.2689634)
\curveto(519.57237873,1493.58840836)(518.37446297,1493.24813084)(516.85363079,1493.24813084)
\curveto(515.99252033,1493.24813084)(515.20085426,1493.34882521)(514.47863258,1493.55021395)
\curveto(513.76335535,1493.75854712)(513.16266136,1493.9842414)(512.67655061,1494.22729677)
\lineto(512.67655061,1496.42521177)
\lineto(512.7807172,1496.42521177)
\curveto(513.39877229,1495.95993435)(514.08627177,1495.58840685)(514.84321564,1495.31062928)
\curveto(515.60015951,1495.03979615)(516.32585341,1494.90437959)(517.02029733,1494.90437959)
\curveto(517.88140779,1494.90437959)(518.55501839,1495.04326837)(519.04112913,1495.32104594)
\curveto(519.52723988,1495.59882351)(519.77029525,1496.03632318)(519.77029525,1496.63354495)
\curveto(519.77029525,1497.09187793)(519.6383509,1497.43909989)(519.37446221,1497.67521083)
\curveto(519.11057352,1497.91132176)(518.60362946,1498.1127105)(517.85363003,1498.27937704)
\curveto(517.57585246,1498.34187699)(517.21126941,1498.4147936)(516.75988086,1498.49812687)
\curveto(516.31543675,1498.58146014)(515.90918706,1498.67173785)(515.54113178,1498.76896)
\curveto(514.52029922,1499.03979313)(513.79460532,1499.43562616)(513.36405009,1499.9564591)
\curveto(512.9404393,1500.48423648)(512.72863391,1501.13006932)(512.72863391,1501.89395763)
\curveto(512.72863391,1502.37312394)(512.82585606,1502.82451249)(513.02030035,1503.24812328)
\curveto(513.22168909,1503.67173407)(513.5237722,1504.050206)(513.92654967,1504.38353908)
\curveto(514.31543826,1504.70992773)(514.80849345,1504.96687198)(515.40571522,1505.15437183)
\curveto(516.00988143,1505.34881613)(516.68349203,1505.44603828)(517.42654702,1505.44603828)
\curveto(518.12099094,1505.44603828)(518.8223793,1505.35923279)(519.5307121,1505.18562181)
\curveto(520.24598933,1505.01895527)(520.83973888,1504.81409431)(521.31196075,1504.57103894)
\lineto(521.31196075,1502.47729053)
\lineto(521.20779416,1502.47729053)
\curveto(520.70779454,1502.8453458)(520.10015611,1503.15437335)(519.38487887,1503.40437316)
\curveto(518.66960164,1503.66131741)(517.96821328,1503.78978953)(517.2807138,1503.78978953)
\curveto(516.56543656,1503.78978953)(515.96127035,1503.65090075)(515.46821517,1503.37312318)
\curveto(514.97515999,1503.10229005)(514.7286324,1502.69604036)(514.7286324,1502.1543741)
\curveto(514.7286324,1501.6752078)(514.87793784,1501.31409696)(515.17654872,1501.07104159)
\curveto(515.46821517,1500.82798622)(515.94043703,1500.6300697)(516.59321432,1500.47729204)
\curveto(516.95432516,1500.39395877)(517.35710263,1500.3106255)(517.80154674,1500.22729223)
\curveto(518.25293528,1500.14395896)(518.627935,1500.06757013)(518.92654589,1499.99812574)
\curveto(519.83626742,1499.78979256)(520.53765578,1499.43215394)(521.03071096,1498.92520988)
\curveto(521.52376614,1498.41132138)(521.77029373,1497.73076634)(521.77029373,1496.88354476)
\closepath
}
}
{
\newrgbcolor{curcolor}{0 0 0}
\pscustom[linestyle=none,fillstyle=solid,fillcolor=curcolor]
{
\newpath
\moveto(542.31194965,1499.48770945)
\curveto(542.31194965,1498.54326572)(542.17653309,1497.67868305)(541.90569996,1496.89396142)
\curveto(541.63486683,1496.11618423)(541.25292268,1495.4564625)(540.75986749,1494.91479625)
\curveto(540.30153451,1494.40090775)(539.75986825,1494.00160249)(539.13486872,1493.71688049)
\curveto(538.51681363,1493.43910292)(537.86056413,1493.30021414)(537.16612021,1493.30021414)
\curveto(536.561954,1493.30021414)(536.01334331,1493.36618631)(535.52028812,1493.49813065)
\curveto(535.03417738,1493.630075)(534.53764998,1493.83493595)(534.03070592,1494.11271352)
\lineto(534.03070592,1489.23771721)
\lineto(532.07237407,1489.23771721)
\lineto(532.07237407,1505.16478849)
\lineto(534.03070592,1505.16478849)
\lineto(534.03070592,1503.94603942)
\curveto(534.55153886,1504.38353908)(535.13487175,1504.74812214)(535.78070459,1505.03978859)
\curveto(536.43348188,1505.33839947)(537.1279258,1505.48770492)(537.86403635,1505.48770492)
\curveto(539.26681307,1505.48770492)(540.35709002,1504.95645532)(541.13486721,1503.89395612)
\curveto(541.91958884,1502.83840136)(542.31194965,1501.36965248)(542.31194965,1499.48770945)
\closepath
\moveto(540.29111785,1499.43562616)
\curveto(540.29111785,1500.83840288)(540.0515347,1501.8870132)(539.57236839,1502.58145711)
\curveto(539.09320209,1503.27590103)(538.35709153,1503.62312299)(537.36403673,1503.62312299)
\curveto(536.80153715,1503.62312299)(536.23556536,1503.50159531)(535.66612135,1503.25853994)
\curveto(535.09667733,1503.01548456)(534.55153886,1502.69604036)(534.03070592,1502.30020733)
\lineto(534.03070592,1495.70646232)
\curveto(534.58626105,1495.4564625)(535.06195514,1495.28632374)(535.45778817,1495.19604603)
\curveto(535.86056564,1495.10576833)(536.31542641,1495.06062947)(536.82237047,1495.06062947)
\curveto(537.91264742,1495.06062947)(538.76334123,1495.42868475)(539.37445187,1496.1647953)
\curveto(539.98556252,1496.90090586)(540.29111785,1497.99118281)(540.29111785,1499.43562616)
\closepath
}
}
{
\newrgbcolor{curcolor}{0 0 0}
\pscustom[linestyle=none,fillstyle=solid,fillcolor=curcolor]
{
\newpath
\moveto(554.4056887,1493.52938063)
\lineto(552.45777351,1493.52938063)
\lineto(552.45777351,1494.76896302)
\curveto(552.28416253,1494.65090756)(552.0480516,1494.48424102)(551.74944071,1494.2689634)
\curveto(551.45777427,1494.06063023)(551.17305226,1493.89396369)(550.89527469,1493.76896378)
\curveto(550.56888605,1493.60924168)(550.19388633,1493.47729734)(549.77027554,1493.37313075)
\curveto(549.34666475,1493.26201972)(548.85013735,1493.20646421)(548.28069334,1493.20646421)
\curveto(547.23208302,1493.20646421)(546.3431948,1493.55368617)(545.61402869,1494.24813009)
\curveto(544.88486257,1494.942574)(544.52027952,1495.82799)(544.52027952,1496.90437808)
\curveto(544.52027952,1497.78632185)(544.70777937,1498.49812687)(545.08277909,1499.03979313)
\curveto(545.46472325,1499.58840382)(546.0063895,1500.01895905)(546.70777786,1500.33145882)
\curveto(547.41611066,1500.64395858)(548.26680446,1500.85576398)(549.25985926,1500.966875)
\curveto(550.25291407,1501.07798603)(551.31888548,1501.1613193)(552.45777351,1501.21687481)
\lineto(552.45777351,1501.51895792)
\curveto(552.45777351,1501.96340203)(552.37791246,1502.3314573)(552.21819036,1502.62312375)
\curveto(552.0654127,1502.9147902)(551.84319064,1503.14395669)(551.5515242,1503.31062323)
\curveto(551.27374663,1503.47034533)(550.94041355,1503.57798414)(550.55152495,1503.63353965)
\curveto(550.16263636,1503.68909517)(549.75638667,1503.71687292)(549.33277587,1503.71687292)
\curveto(548.81888737,1503.71687292)(548.24597114,1503.64742853)(547.61402718,1503.50853975)
\curveto(546.98208321,1503.3765954)(546.32930592,1503.1821511)(545.65569532,1502.92520685)
\lineto(545.55152874,1502.92520685)
\lineto(545.55152874,1504.91478868)
\curveto(545.93347289,1505.01895527)(546.48555581,1505.13353852)(547.20777748,1505.25853842)
\curveto(547.92999916,1505.38353833)(548.64180418,1505.44603828)(549.34319253,1505.44603828)
\curveto(550.16263636,1505.44603828)(550.87444138,1505.37659389)(551.47860758,1505.2377051)
\curveto(552.08971823,1505.10576076)(552.61749561,1504.87659427)(553.06193972,1504.55020562)
\curveto(553.49943939,1504.23076142)(553.83277247,1503.81756729)(554.06193896,1503.31062323)
\curveto(554.29110546,1502.80367917)(554.4056887,1502.17520742)(554.4056887,1501.42520799)
\closepath
\moveto(552.45777351,1496.3939618)
\lineto(552.45777351,1499.63354268)
\curveto(551.86055174,1499.59882048)(551.15569116,1499.54673719)(550.34319178,1499.4772928)
\curveto(549.53763683,1499.4078484)(548.89874843,1499.30715404)(548.42652656,1499.17520969)
\curveto(547.86402699,1499.01548759)(547.40916622,1498.76548778)(547.06194426,1498.42521026)
\curveto(546.7147223,1498.09187718)(546.54111132,1497.63007197)(546.54111132,1497.03979464)
\curveto(546.54111132,1496.37312848)(546.74250006,1495.86965664)(547.14527753,1495.52937912)
\curveto(547.548055,1495.19604603)(548.16263787,1495.02937949)(548.98902613,1495.02937949)
\curveto(549.67652561,1495.02937949)(550.30499736,1495.16132384)(550.87444138,1495.42521253)
\curveto(551.44388539,1495.69604566)(551.97166277,1496.01896208)(552.45777351,1496.3939618)
\closepath
}
}
{
\newrgbcolor{curcolor}{0 0 0}
\pscustom[linestyle=none,fillstyle=solid,fillcolor=curcolor]
{
\newpath
\moveto(565.43693525,1503.02937344)
\lineto(565.33276866,1503.02937344)
\curveto(565.04110221,1503.09881783)(564.7563802,1503.14742891)(564.47860264,1503.17520666)
\curveto(564.20776951,1503.20992886)(563.88485309,1503.22728996)(563.50985337,1503.22728996)
\curveto(562.90568716,1503.22728996)(562.32235427,1503.09187339)(561.75985469,1502.82104027)
\curveto(561.19735512,1502.55715158)(560.65568886,1502.21340184)(560.13485592,1501.78979105)
\lineto(560.13485592,1493.52938063)
\lineto(558.17652407,1493.52938063)
\lineto(558.17652407,1505.16478849)
\lineto(560.13485592,1505.16478849)
\lineto(560.13485592,1503.44603979)
\curveto(560.91263311,1504.07103932)(561.59666037,1504.51201121)(562.1869377,1504.76895546)
\curveto(562.78415947,1505.03284415)(563.3917979,1505.16478849)(564.00985299,1505.16478849)
\curveto(564.35013051,1505.16478849)(564.5966581,1505.15437183)(564.74943577,1505.13353852)
\curveto(564.90221343,1505.11964964)(565.13137992,1505.08839966)(565.43693525,1505.03978859)
\closepath
}
}
{
\newrgbcolor{curcolor}{0 0 0}
\pscustom[linestyle=none,fillstyle=solid,fillcolor=curcolor]
{
\newpath
\moveto(575.94734063,1493.52938063)
\lineto(573.99942544,1493.52938063)
\lineto(573.99942544,1494.76896302)
\curveto(573.82581446,1494.65090756)(573.58970353,1494.48424102)(573.29109264,1494.2689634)
\curveto(572.9994262,1494.06063023)(572.71470419,1493.89396369)(572.43692662,1493.76896378)
\curveto(572.11053798,1493.60924168)(571.73553826,1493.47729734)(571.31192747,1493.37313075)
\curveto(570.88831668,1493.26201972)(570.39178928,1493.20646421)(569.82234527,1493.20646421)
\curveto(568.77373495,1493.20646421)(567.88484673,1493.55368617)(567.15568062,1494.24813009)
\curveto(566.4265145,1494.942574)(566.06193144,1495.82799)(566.06193144,1496.90437808)
\curveto(566.06193144,1497.78632185)(566.2494313,1498.49812687)(566.62443102,1499.03979313)
\curveto(567.00637517,1499.58840382)(567.54804143,1500.01895905)(568.24942979,1500.33145882)
\curveto(568.95776259,1500.64395858)(569.80845639,1500.85576398)(570.80151119,1500.966875)
\curveto(571.794566,1501.07798603)(572.86053741,1501.1613193)(573.99942544,1501.21687481)
\lineto(573.99942544,1501.51895792)
\curveto(573.99942544,1501.96340203)(573.91956439,1502.3314573)(573.75984229,1502.62312375)
\curveto(573.60706462,1502.9147902)(573.38484257,1503.14395669)(573.09317612,1503.31062323)
\curveto(572.81539856,1503.47034533)(572.48206548,1503.57798414)(572.09317688,1503.63353965)
\curveto(571.70428829,1503.68909517)(571.29803859,1503.71687292)(570.8744278,1503.71687292)
\curveto(570.3605393,1503.71687292)(569.78762307,1503.64742853)(569.1556791,1503.50853975)
\curveto(568.52373514,1503.3765954)(567.87095785,1503.1821511)(567.19734725,1502.92520685)
\lineto(567.09318066,1502.92520685)
\lineto(567.09318066,1504.91478868)
\curveto(567.47512482,1505.01895527)(568.02720774,1505.13353852)(568.74942941,1505.25853842)
\curveto(569.47165109,1505.38353833)(570.1834561,1505.44603828)(570.88484446,1505.44603828)
\curveto(571.70428829,1505.44603828)(572.4160933,1505.37659389)(573.02025951,1505.2377051)
\curveto(573.63137016,1505.10576076)(574.15914754,1504.87659427)(574.60359165,1504.55020562)
\curveto(575.04109132,1504.23076142)(575.3744244,1503.81756729)(575.60359089,1503.31062323)
\curveto(575.83275739,1502.80367917)(575.94734063,1502.17520742)(575.94734063,1501.42520799)
\closepath
\moveto(573.99942544,1496.3939618)
\lineto(573.99942544,1499.63354268)
\curveto(573.40220367,1499.59882048)(572.69734309,1499.54673719)(571.88484371,1499.4772928)
\curveto(571.07928876,1499.4078484)(570.44040035,1499.30715404)(569.96817849,1499.17520969)
\curveto(569.40567891,1499.01548759)(568.95081815,1498.76548778)(568.60359619,1498.42521026)
\curveto(568.25637423,1498.09187718)(568.08276325,1497.63007197)(568.08276325,1497.03979464)
\curveto(568.08276325,1496.37312848)(568.28415199,1495.86965664)(568.68692946,1495.52937912)
\curveto(569.08970693,1495.19604603)(569.7042898,1495.02937949)(570.53067806,1495.02937949)
\curveto(571.21817754,1495.02937949)(571.84664929,1495.16132384)(572.4160933,1495.42521253)
\curveto(572.98553732,1495.69604566)(573.5133147,1496.01896208)(573.99942544,1496.3939618)
\closepath
}
}
{
\newrgbcolor{curcolor}{0 0 0}
\pscustom[linestyle=none,fillstyle=solid,fillcolor=curcolor]
{
\newpath
\moveto(596.69732982,1493.52938063)
\lineto(594.73899797,1493.52938063)
\lineto(594.73899797,1500.15437562)
\curveto(594.73899797,1500.65437524)(594.71469243,1501.13701376)(594.66608136,1501.60229119)
\curveto(594.62441472,1502.06756861)(594.53066479,1502.43909611)(594.38483157,1502.71687368)
\curveto(594.22510947,1503.01548456)(593.99594298,1503.24117884)(593.69733209,1503.3939565)
\curveto(593.39872121,1503.54673416)(592.96816598,1503.62312299)(592.4056664,1503.62312299)
\curveto(591.85705571,1503.62312299)(591.30844501,1503.48423421)(590.75983431,1503.20645664)
\curveto(590.21122362,1502.93562351)(589.66261292,1502.58840155)(589.11400222,1502.16479076)
\curveto(589.13483554,1502.00506866)(589.15219664,1501.8175688)(589.16608552,1501.60229119)
\curveto(589.1799744,1501.39395801)(589.18691884,1501.18562484)(589.18691884,1500.97729166)
\lineto(589.18691884,1493.52938063)
\lineto(587.22858698,1493.52938063)
\lineto(587.22858698,1500.15437562)
\curveto(587.22858698,1500.66826412)(587.20428145,1501.15437486)(587.15567037,1501.61270785)
\curveto(587.11400374,1502.07798527)(587.02025381,1502.44951277)(586.87442059,1502.72729034)
\curveto(586.71469848,1503.02590122)(586.48553199,1503.24812328)(586.18692111,1503.3939565)
\curveto(585.88831022,1503.54673416)(585.45775499,1503.62312299)(584.89525542,1503.62312299)
\curveto(584.3605336,1503.62312299)(583.82233956,1503.49117865)(583.2806733,1503.22728996)
\curveto(582.74595149,1502.96340127)(582.21122967,1502.62659597)(581.67650785,1502.21687406)
\lineto(581.67650785,1493.52938063)
\lineto(579.718176,1493.52938063)
\lineto(579.718176,1505.16478849)
\lineto(581.67650785,1505.16478849)
\lineto(581.67650785,1503.8731228)
\curveto(582.2876185,1504.38006686)(582.89525693,1504.7758999)(583.49942314,1505.06062191)
\curveto(584.11053379,1505.34534391)(584.75983885,1505.48770492)(585.44733833,1505.48770492)
\curveto(586.2390044,1505.48770492)(586.90914278,1505.32103837)(587.45775348,1504.98770529)
\curveto(588.01330861,1504.65437221)(588.42650275,1504.19256701)(588.69733587,1503.60228968)
\curveto(589.48900194,1504.26895584)(590.21122362,1504.74812214)(590.8640009,1505.03978859)
\curveto(591.51677818,1505.33839947)(592.21469432,1505.48770492)(592.95774932,1505.48770492)
\curveto(594.23552613,1505.48770492)(595.17649764,1505.09881632)(595.78066385,1504.32103913)
\curveto(596.3917745,1503.55020638)(596.69732982,1502.47034609)(596.69732982,1501.08145825)
\closepath
}
}
{
\newrgbcolor{curcolor}{0 0 0}
\pscustom[linestyle=none,fillstyle=solid,fillcolor=curcolor]
{
\newpath
\moveto(610.20772505,1499.14395971)
\lineto(601.63481487,1499.14395971)
\curveto(601.63481487,1498.42868248)(601.74245368,1497.80368295)(601.95773129,1497.26896113)
\curveto(602.17300891,1496.74118375)(602.46814757,1496.30715631)(602.84314729,1495.96687879)
\curveto(603.20425813,1495.6335457)(603.63134114,1495.38354589)(604.12439632,1495.21687935)
\curveto(604.62439594,1495.05021281)(605.17300664,1494.96687954)(605.77022841,1494.96687954)
\curveto(606.56189448,1494.96687954)(607.35703276,1495.12312942)(608.15564327,1495.43562919)
\curveto(608.96119822,1495.75507339)(609.53411445,1496.06757315)(609.87439197,1496.37312848)
\lineto(609.97855856,1496.37312848)
\lineto(609.97855856,1494.23771343)
\curveto(609.31883684,1493.95993586)(608.64522623,1493.72729715)(607.95772675,1493.53979729)
\curveto(607.27022727,1493.35229743)(606.5480056,1493.2585475)(605.79106173,1493.2585475)
\curveto(603.86050763,1493.2585475)(602.35356433,1493.77938044)(601.27023181,1494.82104632)
\curveto(600.1868993,1495.86965664)(599.64523304,1497.35576662)(599.64523304,1499.27937628)
\curveto(599.64523304,1501.18215262)(600.16259376,1502.69256814)(601.1973152,1503.81062285)
\curveto(602.23898108,1504.92867756)(603.6070356,1505.48770492)(605.30147876,1505.48770492)
\curveto(606.87092202,1505.48770492)(608.07925444,1505.02937193)(608.92647602,1504.11270596)
\curveto(609.78064204,1503.19603998)(610.20772505,1501.89395763)(610.20772505,1500.20645891)
\closepath
\moveto(608.30147649,1500.64395858)
\curveto(608.29453205,1501.67173558)(608.03411559,1502.46687387)(607.52022708,1503.02937344)
\curveto(607.01328302,1503.59187302)(606.23897805,1503.8731228)(605.19731218,1503.8731228)
\curveto(604.14870186,1503.8731228)(603.31189694,1503.56409526)(602.68689741,1502.94604017)
\curveto(602.06884232,1502.32798508)(601.71814814,1501.56062455)(601.63481487,1500.64395858)
\closepath
}
}
{
\newrgbcolor{curcolor}{0 0 0}
\pscustom[linestyle=none,fillstyle=solid,fillcolor=curcolor]
{
\newpath
\moveto(619.21813811,1493.63354722)
\curveto(618.85008283,1493.53632507)(618.44730536,1493.45646402)(618.00980569,1493.39396407)
\curveto(617.57925046,1493.33146411)(617.19383408,1493.30021414)(616.85355656,1493.30021414)
\curveto(615.66605746,1493.30021414)(614.76328037,1493.61965834)(614.14522528,1494.25854674)
\curveto(613.52717019,1494.89743515)(613.21814265,1495.92173993)(613.21814265,1497.33146109)
\lineto(613.21814265,1503.5189564)
\lineto(611.89522698,1503.5189564)
\lineto(611.89522698,1505.16478849)
\lineto(613.21814265,1505.16478849)
\lineto(613.21814265,1508.50853596)
\lineto(615.1764745,1508.50853596)
\lineto(615.1764745,1505.16478849)
\lineto(619.21813811,1505.16478849)
\lineto(619.21813811,1503.5189564)
\lineto(615.1764745,1503.5189564)
\lineto(615.1764745,1498.21687708)
\curveto(615.1764745,1497.60576643)(615.19036338,1497.12660013)(615.21814113,1496.77937817)
\curveto(615.24591889,1496.43910065)(615.34314104,1496.11965645)(615.50980758,1495.82104556)
\curveto(615.66258524,1495.54326799)(615.87091842,1495.33840704)(616.13480711,1495.20646269)
\curveto(616.40564024,1495.08146279)(616.81536215,1495.01896284)(617.36397284,1495.01896284)
\curveto(617.68341705,1495.01896284)(618.01675013,1495.06410169)(618.36397209,1495.1543794)
\curveto(618.71119405,1495.25160155)(618.96119386,1495.3314626)(619.11397152,1495.39396255)
\lineto(619.21813811,1495.39396255)
\closepath
}
}
{
\newrgbcolor{curcolor}{0 0 0}
\pscustom[linestyle=none,fillstyle=solid,fillcolor=curcolor]
{
\newpath
\moveto(631.32230871,1499.14395971)
\lineto(622.74939852,1499.14395971)
\curveto(622.74939852,1498.42868248)(622.85703733,1497.80368295)(623.07231495,1497.26896113)
\curveto(623.28759256,1496.74118375)(623.58273123,1496.30715631)(623.95773094,1495.96687879)
\curveto(624.31884178,1495.6335457)(624.74592479,1495.38354589)(625.23897997,1495.21687935)
\curveto(625.7389796,1495.05021281)(626.28759029,1494.96687954)(626.88481206,1494.96687954)
\curveto(627.67647813,1494.96687954)(628.47161642,1495.12312942)(629.27022692,1495.43562919)
\curveto(630.07578187,1495.75507339)(630.6486981,1496.06757315)(630.98897562,1496.37312848)
\lineto(631.09314221,1496.37312848)
\lineto(631.09314221,1494.23771343)
\curveto(630.43342049,1493.95993586)(629.75980989,1493.72729715)(629.07231041,1493.53979729)
\curveto(628.38481093,1493.35229743)(627.66258925,1493.2585475)(626.90564538,1493.2585475)
\curveto(624.97509129,1493.2585475)(623.46814798,1493.77938044)(622.38481547,1494.82104632)
\curveto(621.30148295,1495.86965664)(620.7598167,1497.35576662)(620.7598167,1499.27937628)
\curveto(620.7598167,1501.18215262)(621.27717742,1502.69256814)(622.31189886,1503.81062285)
\curveto(623.35356473,1504.92867756)(624.72161925,1505.48770492)(626.41606242,1505.48770492)
\curveto(627.98550567,1505.48770492)(629.19383809,1505.02937193)(630.04105967,1504.11270596)
\curveto(630.8952257,1503.19603998)(631.32230871,1501.89395763)(631.32230871,1500.20645891)
\closepath
\moveto(629.41606015,1500.64395858)
\curveto(629.40911571,1501.67173558)(629.14869924,1502.46687387)(628.63481074,1503.02937344)
\curveto(628.12786668,1503.59187302)(627.35356171,1503.8731228)(626.31189583,1503.8731228)
\curveto(625.26328551,1503.8731228)(624.42648059,1503.56409526)(623.80148106,1502.94604017)
\curveto(623.18342597,1502.32798508)(622.83273179,1501.56062455)(622.74939852,1500.64395858)
\closepath
}
}
{
\newrgbcolor{curcolor}{0 0 0}
\pscustom[linestyle=none,fillstyle=solid,fillcolor=curcolor]
{
\newpath
\moveto(641.551442,1503.02937344)
\lineto(641.44727542,1503.02937344)
\curveto(641.15560897,1503.09881783)(640.87088696,1503.14742891)(640.5931094,1503.17520666)
\curveto(640.32227627,1503.20992886)(639.99935985,1503.22728996)(639.62436013,1503.22728996)
\curveto(639.02019392,1503.22728996)(638.43686103,1503.09187339)(637.87436145,1502.82104027)
\curveto(637.31186188,1502.55715158)(636.77019562,1502.21340184)(636.24936268,1501.78979105)
\lineto(636.24936268,1493.52938063)
\lineto(634.29103083,1493.52938063)
\lineto(634.29103083,1505.16478849)
\lineto(636.24936268,1505.16478849)
\lineto(636.24936268,1503.44603979)
\curveto(637.02713987,1504.07103932)(637.71116713,1504.51201121)(638.30144446,1504.76895546)
\curveto(638.89866623,1505.03284415)(639.50630466,1505.16478849)(640.12435975,1505.16478849)
\curveto(640.46463727,1505.16478849)(640.71116486,1505.15437183)(640.86394252,1505.13353852)
\curveto(641.01672019,1505.11964964)(641.24588668,1505.08839966)(641.551442,1505.03978859)
\closepath
}
}
{
\newrgbcolor{curcolor}{0 0 0}
\pscustom[linestyle=none,fillstyle=solid,fillcolor=curcolor]
{
\newpath
\moveto(651.70770146,1496.88354476)
\curveto(651.70770146,1495.82104556)(651.26672957,1494.94951844)(650.3847858,1494.2689634)
\curveto(649.50978646,1493.58840836)(648.3118707,1493.24813084)(646.79103851,1493.24813084)
\curveto(645.92992805,1493.24813084)(645.13826199,1493.34882521)(644.41604031,1493.55021395)
\curveto(643.70076307,1493.75854712)(643.10006908,1493.9842414)(642.61395834,1494.22729677)
\lineto(642.61395834,1496.42521177)
\lineto(642.71812493,1496.42521177)
\curveto(643.33618002,1495.95993435)(644.0236795,1495.58840685)(644.78062337,1495.31062928)
\curveto(645.53756724,1495.03979615)(646.26326114,1494.90437959)(646.95770505,1494.90437959)
\curveto(647.81881551,1494.90437959)(648.49242612,1495.04326837)(648.97853686,1495.32104594)
\curveto(649.4646476,1495.59882351)(649.70770297,1496.03632318)(649.70770297,1496.63354495)
\curveto(649.70770297,1497.09187793)(649.57575863,1497.43909989)(649.31186994,1497.67521083)
\curveto(649.04798125,1497.91132176)(648.54103719,1498.1127105)(647.79103776,1498.27937704)
\curveto(647.51326019,1498.34187699)(647.14867713,1498.4147936)(646.69728858,1498.49812687)
\curveto(646.25284448,1498.58146014)(645.84659478,1498.67173785)(645.47853951,1498.76896)
\curveto(644.45770695,1499.03979313)(643.73201305,1499.43562616)(643.30145782,1499.9564591)
\curveto(642.87784703,1500.48423648)(642.66604163,1501.13006932)(642.66604163,1501.89395763)
\curveto(642.66604163,1502.37312394)(642.76326378,1502.82451249)(642.95770808,1503.24812328)
\curveto(643.15909682,1503.67173407)(643.46117992,1504.050206)(643.8639574,1504.38353908)
\curveto(644.25284599,1504.70992773)(644.74590117,1504.96687198)(645.34312294,1505.15437183)
\curveto(645.94728915,1505.34881613)(646.62089975,1505.44603828)(647.36395475,1505.44603828)
\curveto(648.05839867,1505.44603828)(648.75978702,1505.35923279)(649.46811982,1505.18562181)
\curveto(650.18339706,1505.01895527)(650.77714661,1504.81409431)(651.24936847,1504.57103894)
\lineto(651.24936847,1502.47729053)
\lineto(651.14520189,1502.47729053)
\curveto(650.64520226,1502.8453458)(650.03756384,1503.15437335)(649.3222866,1503.40437316)
\curveto(648.60700936,1503.66131741)(647.905621,1503.78978953)(647.21812152,1503.78978953)
\curveto(646.50284429,1503.78978953)(645.89867808,1503.65090075)(645.4056229,1503.37312318)
\curveto(644.91256771,1503.10229005)(644.66604012,1502.69604036)(644.66604012,1502.1543741)
\curveto(644.66604012,1501.6752078)(644.81534556,1501.31409696)(645.11395645,1501.07104159)
\curveto(645.4056229,1500.82798622)(645.87784476,1500.6300697)(646.53062204,1500.47729204)
\curveto(646.89173288,1500.39395877)(647.29451036,1500.3106255)(647.73895446,1500.22729223)
\curveto(648.19034301,1500.14395896)(648.56534273,1500.06757013)(648.86395361,1499.99812574)
\curveto(649.77367515,1499.78979256)(650.4750635,1499.43215394)(650.96811869,1498.92520988)
\curveto(651.46117387,1498.41132138)(651.70770146,1497.73076634)(651.70770146,1496.88354476)
\closepath
}
}
{
\newrgbcolor{curcolor}{0 0 0}
\pscustom[linestyle=none,fillstyle=solid,fillcolor=curcolor]
{
\newpath
\moveto(447.30160489,1482.37313907)
\lineto(443.27035794,1466.86273414)
\lineto(440.94744303,1466.86273414)
\lineto(437.68702883,1479.7377244)
\lineto(434.49953124,1466.86273414)
\lineto(432.22869963,1466.86273414)
\lineto(428.12453607,1482.37313907)
\lineto(430.2391178,1482.37313907)
\lineto(433.499532,1469.47731549)
\lineto(436.70786291,1482.37313907)
\lineto(438.80161132,1482.37313907)
\lineto(442.04119221,1469.35231559)
\lineto(445.28077309,1482.37313907)
\closepath
}
}
{
\newrgbcolor{curcolor}{0 0 0}
\pscustom[linestyle=none,fillstyle=solid,fillcolor=curcolor]
{
\newpath
\moveto(459.91617662,1466.86273414)
\lineto(457.95784477,1466.86273414)
\lineto(457.95784477,1473.48772913)
\curveto(457.95784477,1474.02245094)(457.92659479,1474.52245056)(457.86409484,1474.98772799)
\curveto(457.80159488,1475.45994986)(457.68701164,1475.82800513)(457.5203451,1476.09189382)
\curveto(457.34673412,1476.38356027)(457.09673431,1476.59883788)(456.77034567,1476.73772667)
\curveto(456.44395702,1476.88355989)(456.02034623,1476.9564765)(455.49951329,1476.9564765)
\curveto(454.96479148,1476.9564765)(454.40576412,1476.82453216)(453.82243123,1476.56064347)
\curveto(453.23909834,1476.29675478)(452.68007098,1475.95994948)(452.14534916,1475.55022756)
\lineto(452.14534916,1466.86273414)
\lineto(450.18701731,1466.86273414)
\lineto(450.18701731,1483.07105521)
\lineto(452.14534916,1483.07105521)
\lineto(452.14534916,1477.20647631)
\curveto(452.75645981,1477.71342037)(453.38840378,1478.10925341)(454.04118106,1478.39397541)
\curveto(454.69395835,1478.67869742)(455.36409673,1478.82105842)(456.05159621,1478.82105842)
\curveto(457.3085397,1478.82105842)(458.26687231,1478.44258649)(458.92659403,1477.68564262)
\curveto(459.58631576,1476.92869874)(459.91617662,1475.83842179)(459.91617662,1474.41481176)
\closepath
}
}
{
\newrgbcolor{curcolor}{0 0 0}
\pscustom[linestyle=none,fillstyle=solid,fillcolor=curcolor]
{
\newpath
\moveto(472.72866849,1466.86273414)
\lineto(470.78075329,1466.86273414)
\lineto(470.78075329,1468.10231653)
\curveto(470.60714232,1467.98426107)(470.37103138,1467.81759453)(470.0724205,1467.60231691)
\curveto(469.78075405,1467.39398374)(469.49603204,1467.22731719)(469.21825448,1467.10231729)
\curveto(468.89186584,1466.94259519)(468.51686612,1466.81065084)(468.09325533,1466.70648426)
\curveto(467.66964454,1466.59537323)(467.17311714,1466.53981772)(466.60367312,1466.53981772)
\curveto(465.5550628,1466.53981772)(464.66617459,1466.88703967)(463.93700847,1467.58148359)
\curveto(463.20784236,1468.27592751)(462.8432593,1469.16134351)(462.8432593,1470.23773158)
\curveto(462.8432593,1471.11967536)(463.03075916,1471.83148038)(463.40575887,1472.37314664)
\curveto(463.78770303,1472.92175733)(464.32936929,1473.35231256)(465.03075765,1473.66481232)
\curveto(465.73909044,1473.97731209)(466.58978424,1474.18911748)(467.58283905,1474.30022851)
\curveto(468.57589385,1474.41133954)(469.64186527,1474.49467281)(470.78075329,1474.55022832)
\lineto(470.78075329,1474.85231143)
\curveto(470.78075329,1475.29675553)(470.70089224,1475.66481081)(470.54117014,1475.95647726)
\curveto(470.38839248,1476.2481437)(470.16617043,1476.4773102)(469.87450398,1476.64397674)
\curveto(469.59672641,1476.80369884)(469.26339333,1476.91133765)(468.87450474,1476.96689316)
\curveto(468.48561614,1477.02244867)(468.07936645,1477.05022643)(467.65575566,1477.05022643)
\curveto(467.14186716,1477.05022643)(466.56895093,1476.98078204)(465.93700696,1476.84189325)
\curveto(465.30506299,1476.70994891)(464.65228571,1476.51550461)(463.97867511,1476.25856036)
\lineto(463.87450852,1476.25856036)
\lineto(463.87450852,1478.24814219)
\curveto(464.25645268,1478.35230878)(464.80853559,1478.46689202)(465.53075727,1478.59189193)
\curveto(466.25297894,1478.71689184)(466.96478396,1478.77939179)(467.66617232,1478.77939179)
\curveto(468.48561614,1478.77939179)(469.19742116,1478.7099474)(469.80158737,1478.57105861)
\curveto(470.41269802,1478.43911427)(470.9404754,1478.20994777)(471.3849195,1477.88355913)
\curveto(471.82241917,1477.56411493)(472.15575225,1477.1509208)(472.38491875,1476.64397674)
\curveto(472.61408524,1476.13703268)(472.72866849,1475.50856093)(472.72866849,1474.7585615)
\closepath
\moveto(470.78075329,1469.7273153)
\lineto(470.78075329,1472.96689619)
\curveto(470.18353152,1472.93217399)(469.47867095,1472.8800907)(468.66617156,1472.8106463)
\curveto(467.86061662,1472.74120191)(467.22172821,1472.64050754)(466.74950634,1472.5085632)
\curveto(466.18700677,1472.3488411)(465.732146,1472.09884129)(465.38492404,1471.75856377)
\curveto(465.03770208,1471.42523069)(464.8640911,1470.96342548)(464.8640911,1470.37314815)
\curveto(464.8640911,1469.70648199)(465.06547984,1469.20301014)(465.46825731,1468.86273262)
\curveto(465.87103479,1468.52939954)(466.48561766,1468.362733)(467.31200592,1468.362733)
\curveto(467.9995054,1468.362733)(468.62797715,1468.49467735)(469.19742116,1468.75856604)
\curveto(469.76686517,1469.02939916)(470.29464255,1469.35231559)(470.78075329,1469.7273153)
\closepath
}
}
{
\newrgbcolor{curcolor}{0 0 0}
\pscustom[linestyle=none,fillstyle=solid,fillcolor=curcolor]
{
\newpath
\moveto(482.54115874,1466.96690073)
\curveto(482.17310347,1466.86967858)(481.77032599,1466.78981753)(481.33282632,1466.72731757)
\curveto(480.90227109,1466.66481762)(480.51685472,1466.63356764)(480.1765772,1466.63356764)
\curveto(478.9890781,1466.63356764)(478.086301,1466.95301185)(477.46824591,1467.59190025)
\curveto(476.85019083,1468.23078866)(476.54116328,1469.25509344)(476.54116328,1470.66481459)
\lineto(476.54116328,1476.85230991)
\lineto(475.21824762,1476.85230991)
\lineto(475.21824762,1478.498142)
\lineto(476.54116328,1478.498142)
\lineto(476.54116328,1481.84188947)
\lineto(478.49949513,1481.84188947)
\lineto(478.49949513,1478.498142)
\lineto(482.54115874,1478.498142)
\lineto(482.54115874,1476.85230991)
\lineto(478.49949513,1476.85230991)
\lineto(478.49949513,1471.55023059)
\curveto(478.49949513,1470.93911994)(478.51338401,1470.45995364)(478.54116177,1470.11273168)
\curveto(478.56893953,1469.77245416)(478.66616167,1469.45300996)(478.83282822,1469.15439907)
\curveto(478.98560588,1468.8766215)(479.19393905,1468.67176055)(479.45782774,1468.5398162)
\curveto(479.72866087,1468.4148163)(480.13838278,1468.35231634)(480.68699348,1468.35231634)
\curveto(481.00643768,1468.35231634)(481.33977076,1468.3974552)(481.68699272,1468.48773291)
\curveto(482.03421468,1468.58495506)(482.28421449,1468.66481611)(482.43699215,1468.72731606)
\lineto(482.54115874,1468.72731606)
\closepath
}
}
{
\newrgbcolor{curcolor}{0 0 0}
\pscustom[linestyle=none,fillstyle=solid,fillcolor=curcolor]
{
\newpath
\moveto(502.46822414,1478.498142)
\lineto(497.75989436,1466.86273414)
\lineto(495.79114585,1466.86273414)
\lineto(491.11406606,1478.498142)
\lineto(493.23906445,1478.498142)
\lineto(496.84322839,1469.23773234)
\lineto(500.41614235,1478.498142)
\closepath
}
}
{
\newrgbcolor{curcolor}{0 0 0}
\pscustom[linestyle=none,fillstyle=solid,fillcolor=curcolor]
{
\newpath
\moveto(514.58279714,1472.47731322)
\lineto(506.00988696,1472.47731322)
\curveto(506.00988696,1471.76203599)(506.11752577,1471.13703646)(506.33280338,1470.60231464)
\curveto(506.548081,1470.07453726)(506.84321966,1469.64050981)(507.21821938,1469.30023229)
\curveto(507.57933022,1468.96689921)(508.00641323,1468.7168994)(508.49946841,1468.55023286)
\curveto(508.99946803,1468.38356632)(509.54807873,1468.30023305)(510.1453005,1468.30023305)
\curveto(510.93696657,1468.30023305)(511.73210485,1468.45648293)(512.53071536,1468.7689827)
\curveto(513.33627031,1469.0884269)(513.90918654,1469.40092666)(514.24946406,1469.70648199)
\lineto(514.35363065,1469.70648199)
\lineto(514.35363065,1467.57106693)
\curveto(513.69390892,1467.29328937)(513.02029832,1467.06065065)(512.33279884,1466.8731508)
\curveto(511.64529936,1466.68565094)(510.92307769,1466.59190101)(510.16613382,1466.59190101)
\curveto(508.23557972,1466.59190101)(506.72863642,1467.11273395)(505.6453039,1468.15439983)
\curveto(504.56197139,1469.20301014)(504.02030513,1470.68912013)(504.02030513,1472.61272979)
\curveto(504.02030513,1474.51550613)(504.53766585,1476.02592165)(505.57238729,1477.14397636)
\curveto(506.61405317,1478.26203107)(507.98210769,1478.82105842)(509.67655085,1478.82105842)
\curveto(511.24599411,1478.82105842)(512.45432653,1478.36272544)(513.30154811,1477.44605946)
\curveto(514.15571413,1476.52939349)(514.58279714,1475.22731114)(514.58279714,1473.53981242)
\closepath
\moveto(512.67654858,1473.97731209)
\curveto(512.66960414,1475.00508909)(512.40918767,1475.80022738)(511.89529917,1476.36272695)
\curveto(511.38835511,1476.92522652)(510.61405014,1477.20647631)(509.57238426,1477.20647631)
\curveto(508.52377395,1477.20647631)(507.68696902,1476.89744877)(507.0619695,1476.27939368)
\curveto(506.44391441,1475.66133859)(506.09322023,1474.89397806)(506.00988696,1473.97731209)
\closepath
}
}
{
\newrgbcolor{curcolor}{0 0 0}
\pscustom[linestyle=none,fillstyle=solid,fillcolor=curcolor]
{
\newpath
\moveto(524.81195927,1476.36272695)
\lineto(524.70779269,1476.36272695)
\curveto(524.41612624,1476.43217134)(524.13140423,1476.48078242)(523.85362667,1476.50856017)
\curveto(523.58279354,1476.54328237)(523.25987712,1476.56064347)(522.8848774,1476.56064347)
\curveto(522.28071119,1476.56064347)(521.6973783,1476.4252269)(521.13487872,1476.15439377)
\curveto(520.57237915,1475.89050509)(520.03071289,1475.54675535)(519.50987995,1475.12314455)
\lineto(519.50987995,1466.86273414)
\lineto(517.5515481,1466.86273414)
\lineto(517.5515481,1478.498142)
\lineto(519.50987995,1478.498142)
\lineto(519.50987995,1476.7793933)
\curveto(520.28765714,1477.40439283)(520.9716844,1477.84536472)(521.56196173,1478.10230897)
\curveto(522.1591835,1478.36619766)(522.76682193,1478.498142)(523.38487702,1478.498142)
\curveto(523.72515454,1478.498142)(523.97168213,1478.48772534)(524.12445979,1478.46689202)
\curveto(524.27723746,1478.45300315)(524.50640395,1478.42175317)(524.81195927,1478.3731421)
\closepath
}
}
{
\newrgbcolor{curcolor}{0 0 0}
\pscustom[linestyle=none,fillstyle=solid,fillcolor=curcolor]
{
\newpath
\moveto(534.96820431,1470.21689827)
\curveto(534.96820431,1469.15439907)(534.52723243,1468.28287195)(533.64528865,1467.60231691)
\curveto(532.77028931,1466.92176187)(531.57237355,1466.58148435)(530.05154137,1466.58148435)
\curveto(529.19043091,1466.58148435)(528.39876484,1466.68217872)(527.67654316,1466.88356745)
\curveto(526.96126593,1467.09190063)(526.36057194,1467.3175949)(525.87446119,1467.56065028)
\lineto(525.87446119,1469.75856528)
\lineto(525.97862778,1469.75856528)
\curveto(526.59668287,1469.29328785)(527.28418235,1468.92176036)(528.04112622,1468.64398279)
\curveto(528.79807009,1468.37314966)(529.52376399,1468.2377331)(530.21820791,1468.2377331)
\curveto(531.07931837,1468.2377331)(531.75292897,1468.37662188)(532.23903971,1468.65439945)
\curveto(532.72515046,1468.93217702)(532.96820583,1469.36967669)(532.96820583,1469.96689846)
\curveto(532.96820583,1470.42523144)(532.83626148,1470.7724534)(532.57237279,1471.00856433)
\curveto(532.3084841,1471.24467527)(531.80154004,1471.446064)(531.05154061,1471.61273054)
\curveto(530.77376304,1471.6752305)(530.40917999,1471.74814711)(529.95779144,1471.83148038)
\curveto(529.51334733,1471.91481365)(529.10709764,1472.00509136)(528.73904236,1472.10231351)
\curveto(527.7182098,1472.37314664)(526.9925159,1472.76897967)(526.56196067,1473.28981261)
\curveto(526.13834988,1473.81758999)(525.92654449,1474.46342283)(525.92654449,1475.22731114)
\curveto(525.92654449,1475.70647745)(526.02376664,1476.15786599)(526.21821093,1476.58147678)
\curveto(526.41959967,1477.00508758)(526.72168277,1477.38355951)(527.12446025,1477.71689259)
\curveto(527.51334884,1478.04328123)(528.00640403,1478.30022548)(528.6036258,1478.48772534)
\curveto(529.20779201,1478.68216964)(529.88140261,1478.77939179)(530.6244576,1478.77939179)
\curveto(531.31890152,1478.77939179)(532.02028988,1478.6925863)(532.72862267,1478.51897532)
\curveto(533.44389991,1478.35230878)(534.03764946,1478.14744782)(534.50987133,1477.90439245)
\lineto(534.50987133,1475.81064403)
\lineto(534.40570474,1475.81064403)
\curveto(533.90570512,1476.17869931)(533.29806669,1476.48772686)(532.58278945,1476.73772667)
\curveto(531.86751222,1476.99467092)(531.16612386,1477.12314304)(530.47862438,1477.12314304)
\curveto(529.76334714,1477.12314304)(529.15918093,1476.98425426)(528.66612575,1476.70647669)
\curveto(528.17307057,1476.43564356)(527.92654297,1476.02939387)(527.92654297,1475.48772761)
\curveto(527.92654297,1475.00856131)(528.07584842,1474.64745047)(528.3744593,1474.4043951)
\curveto(528.66612575,1474.16133973)(529.13834761,1473.96342321)(529.7911249,1473.81064555)
\curveto(530.15223574,1473.72731228)(530.55501321,1473.64397901)(530.99945732,1473.56064574)
\curveto(531.45084586,1473.47731247)(531.82584558,1473.40092364)(532.12445647,1473.33147924)
\curveto(533.034178,1473.12314607)(533.73556636,1472.76550745)(534.22862154,1472.25856339)
\curveto(534.72167672,1471.74467489)(534.96820431,1471.06411985)(534.96820431,1470.21689827)
\closepath
}
}
{
\newrgbcolor{curcolor}{0 0 0}
\pscustom[linestyle=none,fillstyle=solid,fillcolor=curcolor]
{
\newpath
\moveto(539.87444972,1480.44605719)
\lineto(537.66611806,1480.44605719)
\lineto(537.66611806,1482.47730566)
\lineto(539.87444972,1482.47730566)
\closepath
\moveto(539.74944981,1466.86273414)
\lineto(537.79111796,1466.86273414)
\lineto(537.79111796,1478.498142)
\lineto(539.74944981,1478.498142)
\closepath
}
}
{
\newrgbcolor{curcolor}{0 0 0}
\pscustom[linestyle=none,fillstyle=solid,fillcolor=curcolor]
{
\newpath
\moveto(553.54109958,1472.67522974)
\curveto(553.54109958,1470.77939784)(553.05498884,1469.2828712)(552.08276735,1468.1856498)
\curveto(551.11054587,1467.08842841)(549.80846352,1466.53981772)(548.17652031,1466.53981772)
\curveto(546.53068822,1466.53981772)(545.22166143,1467.08842841)(544.24943994,1468.1856498)
\curveto(543.2841629,1469.2828712)(542.80152437,1470.77939784)(542.80152437,1472.67522974)
\curveto(542.80152437,1474.57106164)(543.2841629,1476.06758828)(544.24943994,1477.16480968)
\curveto(545.22166143,1478.26897551)(546.53068822,1478.82105842)(548.17652031,1478.82105842)
\curveto(549.80846352,1478.82105842)(551.11054587,1478.26897551)(552.08276735,1477.16480968)
\curveto(553.05498884,1476.06758828)(553.54109958,1474.57106164)(553.54109958,1472.67522974)
\closepath
\moveto(551.52026778,1472.67522974)
\curveto(551.52026778,1474.18217304)(551.22512911,1475.30022775)(550.63485178,1476.02939387)
\curveto(550.04457445,1476.76550442)(549.22513062,1477.1335597)(548.17652031,1477.1335597)
\curveto(547.11402111,1477.1335597)(546.28763285,1476.76550442)(545.69735552,1476.02939387)
\curveto(545.11402262,1475.30022775)(544.82235618,1474.18217304)(544.82235618,1472.67522974)
\curveto(544.82235618,1471.21689751)(545.11749484,1470.10925946)(545.70777217,1469.35231559)
\curveto(546.29804951,1468.60231615)(547.12096555,1468.22731644)(548.17652031,1468.22731644)
\curveto(549.21818619,1468.22731644)(550.03415779,1468.59884394)(550.62443512,1469.34189893)
\curveto(551.22165689,1470.09189836)(551.52026778,1471.20300863)(551.52026778,1472.67522974)
\closepath
}
}
{
\newrgbcolor{curcolor}{0 0 0}
\pscustom[linestyle=none,fillstyle=solid,fillcolor=curcolor]
{
\newpath
\moveto(566.30150646,1466.86273414)
\lineto(564.34317461,1466.86273414)
\lineto(564.34317461,1473.48772913)
\curveto(564.34317461,1474.02245094)(564.31192463,1474.52245056)(564.24942468,1474.98772799)
\curveto(564.18692472,1475.45994986)(564.07234148,1475.82800513)(563.90567494,1476.09189382)
\curveto(563.73206396,1476.38356027)(563.48206415,1476.59883788)(563.1556755,1476.73772667)
\curveto(562.82928686,1476.88355989)(562.40567607,1476.9564765)(561.88484313,1476.9564765)
\curveto(561.35012132,1476.9564765)(560.79109396,1476.82453216)(560.20776107,1476.56064347)
\curveto(559.62442818,1476.29675478)(559.06540082,1475.95994948)(558.530679,1475.55022756)
\lineto(558.530679,1466.86273414)
\lineto(556.57234715,1466.86273414)
\lineto(556.57234715,1478.498142)
\lineto(558.530679,1478.498142)
\lineto(558.530679,1477.20647631)
\curveto(559.14178965,1477.71342037)(559.77373362,1478.10925341)(560.4265109,1478.39397541)
\curveto(561.07928819,1478.67869742)(561.74942657,1478.82105842)(562.43692605,1478.82105842)
\curveto(563.69386954,1478.82105842)(564.65220215,1478.44258649)(565.31192387,1477.68564262)
\curveto(565.9716456,1476.92869874)(566.30150646,1475.83842179)(566.30150646,1474.41481176)
\closepath
}
}
{
\newrgbcolor{curcolor}{0 0 0}
\pscustom[linestyle=none,fillstyle=solid,fillcolor=curcolor]
{
\newpath
\moveto(587.48899767,1472.67522974)
\curveto(587.48899767,1470.77939784)(587.00288692,1469.2828712)(586.03066544,1468.1856498)
\curveto(585.05844395,1467.08842841)(583.7563616,1466.53981772)(582.12441839,1466.53981772)
\curveto(580.4785863,1466.53981772)(579.16955952,1467.08842841)(578.19733803,1468.1856498)
\curveto(577.23206098,1469.2828712)(576.74942246,1470.77939784)(576.74942246,1472.67522974)
\curveto(576.74942246,1474.57106164)(577.23206098,1476.06758828)(578.19733803,1477.16480968)
\curveto(579.16955952,1478.26897551)(580.4785863,1478.82105842)(582.12441839,1478.82105842)
\curveto(583.7563616,1478.82105842)(585.05844395,1478.26897551)(586.03066544,1477.16480968)
\curveto(587.00288692,1476.06758828)(587.48899767,1474.57106164)(587.48899767,1472.67522974)
\closepath
\moveto(585.46816586,1472.67522974)
\curveto(585.46816586,1474.18217304)(585.1730272,1475.30022775)(584.58274986,1476.02939387)
\curveto(583.99247253,1476.76550442)(583.17302871,1477.1335597)(582.12441839,1477.1335597)
\curveto(581.06191919,1477.1335597)(580.23553093,1476.76550442)(579.6452536,1476.02939387)
\curveto(579.06192071,1475.30022775)(578.77025426,1474.18217304)(578.77025426,1472.67522974)
\curveto(578.77025426,1471.21689751)(579.06539293,1470.10925946)(579.65567026,1469.35231559)
\curveto(580.24594759,1468.60231615)(581.06886363,1468.22731644)(582.12441839,1468.22731644)
\curveto(583.16608427,1468.22731644)(583.98205587,1468.59884394)(584.57233321,1469.34189893)
\curveto(585.16955498,1470.09189836)(585.46816586,1471.20300863)(585.46816586,1472.67522974)
\closepath
}
}
{
\newrgbcolor{curcolor}{0 0 0}
\pscustom[linestyle=none,fillstyle=solid,fillcolor=curcolor]
{
\newpath
\moveto(596.78065717,1481.17522331)
\lineto(596.67649058,1481.17522331)
\curveto(596.46121296,1481.23772326)(596.17996318,1481.30022321)(595.83274122,1481.36272317)
\curveto(595.48551926,1481.43216756)(595.17996393,1481.46688976)(594.91607524,1481.46688976)
\curveto(594.0757981,1481.46688976)(593.46468745,1481.2793899)(593.0827433,1480.90439018)
\curveto(592.70774358,1480.5363349)(592.52024372,1479.86619652)(592.52024372,1478.89397504)
\lineto(592.52024372,1478.498142)
\lineto(596.05149105,1478.498142)
\lineto(596.05149105,1476.85230991)
\lineto(592.58274368,1476.85230991)
\lineto(592.58274368,1466.86273414)
\lineto(590.62441182,1466.86273414)
\lineto(590.62441182,1476.85230991)
\lineto(589.30149616,1476.85230991)
\lineto(589.30149616,1478.498142)
\lineto(590.62441182,1478.498142)
\lineto(590.62441182,1478.88355838)
\curveto(590.62441182,1480.26550178)(590.96816156,1481.32452875)(591.65566104,1482.06063931)
\curveto(592.34316052,1482.8036943)(593.33621533,1483.1752218)(594.63482546,1483.1752218)
\curveto(595.07232513,1483.1752218)(595.46468594,1483.15438848)(595.8119079,1483.11272184)
\curveto(596.1660743,1483.07105521)(596.48899072,1483.02244413)(596.78065717,1482.96688862)
\closepath
}
}
{
\newrgbcolor{curcolor}{0 0 0}
\pscustom[linestyle=none,fillstyle=solid,fillcolor=curcolor]
{
\newpath
\moveto(617.00980546,1466.86273414)
\lineto(604.90564795,1466.86273414)
\lineto(604.90564795,1468.77939935)
\lineto(614.41605742,1480.53980712)
\lineto(605.25981435,1480.53980712)
\lineto(605.25981435,1482.37313907)
\lineto(616.78063897,1482.37313907)
\lineto(616.78063897,1480.50855715)
\lineto(607.17647957,1468.69606608)
\lineto(617.00980546,1468.69606608)
\closepath
}
}
{
\newrgbcolor{curcolor}{0 0 0}
\pscustom[linestyle=none,fillstyle=solid,fillcolor=curcolor]
{
\newpath
\moveto(630.19730782,1480.53980712)
\lineto(622.35356375,1480.53980712)
\lineto(622.35356375,1476.16481043)
\lineto(629.09314199,1476.16481043)
\lineto(629.09314199,1474.33147849)
\lineto(622.35356375,1474.33147849)
\lineto(622.35356375,1466.86273414)
\lineto(620.29106531,1466.86273414)
\lineto(620.29106531,1482.37313907)
\lineto(630.19730782,1482.37313907)
\closepath
}
}
{
\newrgbcolor{curcolor}{0 0 0}
\pscustom[linestyle=none,fillstyle=solid,fillcolor=curcolor]
{
\newpath
\moveto(643.82228628,1471.28981412)
\curveto(643.82228628,1470.68564791)(643.67992528,1470.08842614)(643.39520327,1469.49814881)
\curveto(643.1174257,1468.90787148)(642.72506489,1468.40787186)(642.21812083,1467.99814995)
\curveto(641.66256569,1467.55370584)(641.01326063,1467.20648388)(640.27020564,1466.95648407)
\curveto(639.53409508,1466.70648426)(638.64520687,1466.58148435)(637.60354099,1466.58148435)
\curveto(636.48548628,1466.58148435)(635.4785426,1466.68565094)(634.58270994,1466.89398411)
\curveto(633.69382172,1467.10231729)(632.78757241,1467.41134483)(631.863962,1467.82106675)
\lineto(631.863962,1470.40439812)
\lineto(632.00979522,1470.40439812)
\curveto(632.79451685,1469.75162084)(633.70076616,1469.248149)(634.72854316,1468.8939826)
\curveto(635.75632016,1468.5398162)(636.72159721,1468.362733)(637.62437431,1468.362733)
\curveto(638.90215112,1468.362733)(639.89520592,1468.60231615)(640.60353872,1469.08148246)
\curveto(641.31881595,1469.56064876)(641.67645457,1470.19953717)(641.67645457,1470.99814768)
\curveto(641.67645457,1471.68564716)(641.50631581,1472.19259122)(641.16603829,1472.51897986)
\curveto(640.83270521,1472.8453685)(640.32228893,1473.09884053)(639.63478945,1473.27939595)
\curveto(639.11395651,1473.41828473)(638.54798472,1473.53286798)(637.93687407,1473.62314569)
\curveto(637.33270786,1473.7134234)(636.69034723,1473.82800665)(636.00979219,1473.96689543)
\curveto(634.63479323,1474.25856188)(633.61396067,1474.75508928)(632.94729451,1475.45647764)
\curveto(632.28757279,1476.16481043)(631.95771193,1477.08494863)(631.95771193,1478.21689221)
\curveto(631.95771193,1479.51550234)(632.50632262,1480.57800154)(633.60354401,1481.4043898)
\curveto(634.70076541,1482.23772251)(636.09312546,1482.65438886)(637.78062419,1482.65438886)
\curveto(638.87090114,1482.65438886)(639.87090038,1482.55022227)(640.78062192,1482.34188909)
\curveto(641.69034345,1482.13355592)(642.4958984,1481.87661167)(643.19728676,1481.57105634)
\lineto(643.19728676,1479.13355819)
\lineto(643.05145353,1479.13355819)
\curveto(642.4611762,1479.63355781)(641.68339901,1480.04675194)(640.71812196,1480.37314058)
\curveto(639.75978936,1480.70647366)(638.77715121,1480.8731402)(637.77020753,1480.8731402)
\curveto(636.6660417,1480.8731402)(635.77715348,1480.64397371)(635.10354288,1480.18564072)
\curveto(634.43687672,1479.72730774)(634.10354364,1479.13703041)(634.10354364,1478.41480873)
\curveto(634.10354364,1477.76897589)(634.27021018,1477.26203183)(634.60354326,1476.89397655)
\curveto(634.93687634,1476.52592127)(635.52368145,1476.24467148)(636.36395859,1476.05022719)
\curveto(636.8084027,1475.95300504)(637.44034667,1475.83494957)(638.25979049,1475.69606079)
\curveto(639.07923432,1475.557172)(639.77367823,1475.414811)(640.34312225,1475.26897778)
\curveto(641.49589915,1474.96342245)(642.36395405,1474.50161725)(642.94728694,1473.88356216)
\curveto(643.53061984,1473.26550707)(643.82228628,1472.40092439)(643.82228628,1471.28981412)
\closepath
}
}
{
\newrgbcolor{curcolor}{0 0 0}
\pscustom[linestyle=none,fillstyle=solid,fillcolor=curcolor]
{
\newpath
\moveto(660.52020062,1466.96690073)
\curveto(660.15214534,1466.86967858)(659.74936787,1466.78981753)(659.3118682,1466.72731757)
\curveto(658.88131297,1466.66481762)(658.49589659,1466.63356764)(658.15561907,1466.63356764)
\curveto(656.96811997,1466.63356764)(656.06534288,1466.95301185)(655.44728779,1467.59190025)
\curveto(654.8292327,1468.23078866)(654.52020516,1469.25509344)(654.52020516,1470.66481459)
\lineto(654.52020516,1476.85230991)
\lineto(653.19728949,1476.85230991)
\lineto(653.19728949,1478.498142)
\lineto(654.52020516,1478.498142)
\lineto(654.52020516,1481.84188947)
\lineto(656.47853701,1481.84188947)
\lineto(656.47853701,1478.498142)
\lineto(660.52020062,1478.498142)
\lineto(660.52020062,1476.85230991)
\lineto(656.47853701,1476.85230991)
\lineto(656.47853701,1471.55023059)
\curveto(656.47853701,1470.93911994)(656.49242589,1470.45995364)(656.52020364,1470.11273168)
\curveto(656.5479814,1469.77245416)(656.64520355,1469.45300996)(656.81187009,1469.15439907)
\curveto(656.96464775,1468.8766215)(657.17298093,1468.67176055)(657.43686962,1468.5398162)
\curveto(657.70770274,1468.4148163)(658.11742466,1468.35231634)(658.66603535,1468.35231634)
\curveto(658.98547955,1468.35231634)(659.31881264,1468.3974552)(659.6660346,1468.48773291)
\curveto(660.01325655,1468.58495506)(660.26325637,1468.66481611)(660.41603403,1468.72731606)
\lineto(660.52020062,1468.72731606)
\closepath
}
}
{
\newrgbcolor{curcolor}{0 0 0}
\pscustom[linestyle=none,fillstyle=solid,fillcolor=curcolor]
{
\newpath
\moveto(672.80143999,1472.67522974)
\curveto(672.80143999,1470.77939784)(672.31532925,1469.2828712)(671.34310776,1468.1856498)
\curveto(670.37088628,1467.08842841)(669.06880393,1466.53981772)(667.43686072,1466.53981772)
\curveto(665.79102863,1466.53981772)(664.48200184,1467.08842841)(663.50978036,1468.1856498)
\curveto(662.54450331,1469.2828712)(662.06186479,1470.77939784)(662.06186479,1472.67522974)
\curveto(662.06186479,1474.57106164)(662.54450331,1476.06758828)(663.50978036,1477.16480968)
\curveto(664.48200184,1478.26897551)(665.79102863,1478.82105842)(667.43686072,1478.82105842)
\curveto(669.06880393,1478.82105842)(670.37088628,1478.26897551)(671.34310776,1477.16480968)
\curveto(672.31532925,1476.06758828)(672.80143999,1474.57106164)(672.80143999,1472.67522974)
\closepath
\moveto(670.78060819,1472.67522974)
\curveto(670.78060819,1474.18217304)(670.48546952,1475.30022775)(669.89519219,1476.02939387)
\curveto(669.30491486,1476.76550442)(668.48547104,1477.1335597)(667.43686072,1477.1335597)
\curveto(666.37436152,1477.1335597)(665.54797326,1476.76550442)(664.95769593,1476.02939387)
\curveto(664.37436304,1475.30022775)(664.08269659,1474.18217304)(664.08269659,1472.67522974)
\curveto(664.08269659,1471.21689751)(664.37783526,1470.10925946)(664.96811259,1469.35231559)
\curveto(665.55838992,1468.60231615)(666.38130596,1468.22731644)(667.43686072,1468.22731644)
\curveto(668.4785266,1468.22731644)(669.2944982,1468.59884394)(669.88477553,1469.34189893)
\curveto(670.48199731,1470.09189836)(670.78060819,1471.20300863)(670.78060819,1472.67522974)
\closepath
}
}
{
\newrgbcolor{curcolor}{0 0 0}
\pscustom[linestyle=none,fillstyle=solid,fillcolor=curcolor]
{
\newpath
\moveto(685.31183832,1466.86273414)
\lineto(683.35350646,1466.86273414)
\lineto(683.35350646,1483.07105521)
\lineto(685.31183832,1483.07105521)
\closepath
}
}
{
\newrgbcolor{curcolor}{0 0 0}
\pscustom[linestyle=none,fillstyle=solid,fillcolor=curcolor]
{
\newpath
\moveto(699.10348808,1472.67522974)
\curveto(699.10348808,1470.77939784)(698.61737734,1469.2828712)(697.64515585,1468.1856498)
\curveto(696.67293437,1467.08842841)(695.37085202,1466.53981772)(693.73890881,1466.53981772)
\curveto(692.09307672,1466.53981772)(690.78404993,1467.08842841)(689.81182845,1468.1856498)
\curveto(688.8465514,1469.2828712)(688.36391288,1470.77939784)(688.36391288,1472.67522974)
\curveto(688.36391288,1474.57106164)(688.8465514,1476.06758828)(689.81182845,1477.16480968)
\curveto(690.78404993,1478.26897551)(692.09307672,1478.82105842)(693.73890881,1478.82105842)
\curveto(695.37085202,1478.82105842)(696.67293437,1478.26897551)(697.64515585,1477.16480968)
\curveto(698.61737734,1476.06758828)(699.10348808,1474.57106164)(699.10348808,1472.67522974)
\closepath
\moveto(697.08265628,1472.67522974)
\curveto(697.08265628,1474.18217304)(696.78751761,1475.30022775)(696.19724028,1476.02939387)
\curveto(695.60696295,1476.76550442)(694.78751913,1477.1335597)(693.73890881,1477.1335597)
\curveto(692.67640961,1477.1335597)(691.85002135,1476.76550442)(691.25974402,1476.02939387)
\curveto(690.67641113,1475.30022775)(690.38474468,1474.18217304)(690.38474468,1472.67522974)
\curveto(690.38474468,1471.21689751)(690.67988335,1470.10925946)(691.27016068,1469.35231559)
\curveto(691.86043801,1468.60231615)(692.68335405,1468.22731644)(693.73890881,1468.22731644)
\curveto(694.78057469,1468.22731644)(695.59654629,1468.59884394)(696.18682362,1469.34189893)
\curveto(696.78404539,1470.09189836)(697.08265628,1471.20300863)(697.08265628,1472.67522974)
\closepath
}
}
{
\newrgbcolor{curcolor}{0 0 0}
\pscustom[linestyle=none,fillstyle=solid,fillcolor=curcolor]
{
\newpath
\moveto(711.1764099,1466.86273414)
\lineto(709.22849471,1466.86273414)
\lineto(709.22849471,1468.10231653)
\curveto(709.05488373,1467.98426107)(708.81877279,1467.81759453)(708.52016191,1467.60231691)
\curveto(708.22849546,1467.39398374)(707.94377346,1467.22731719)(707.66599589,1467.10231729)
\curveto(707.33960725,1466.94259519)(706.96460753,1466.81065084)(706.54099674,1466.70648426)
\curveto(706.11738595,1466.59537323)(705.62085855,1466.53981772)(705.05141453,1466.53981772)
\curveto(704.00280421,1466.53981772)(703.113916,1466.88703967)(702.38474988,1467.58148359)
\curveto(701.65558377,1468.27592751)(701.29100071,1469.16134351)(701.29100071,1470.23773158)
\curveto(701.29100071,1471.11967536)(701.47850057,1471.83148038)(701.85350029,1472.37314664)
\curveto(702.23544444,1472.92175733)(702.7771107,1473.35231256)(703.47849906,1473.66481232)
\curveto(704.18683185,1473.97731209)(705.03752565,1474.18911748)(706.03058046,1474.30022851)
\curveto(707.02363526,1474.41133954)(708.08960668,1474.49467281)(709.22849471,1474.55022832)
\lineto(709.22849471,1474.85231143)
\curveto(709.22849471,1475.29675553)(709.14863366,1475.66481081)(708.98891155,1475.95647726)
\curveto(708.83613389,1476.2481437)(708.61391184,1476.4773102)(708.32224539,1476.64397674)
\curveto(708.04446782,1476.80369884)(707.71113474,1476.91133765)(707.32224615,1476.96689316)
\curveto(706.93335755,1477.02244867)(706.52710786,1477.05022643)(706.10349707,1477.05022643)
\curveto(705.58960857,1477.05022643)(705.01669234,1476.98078204)(704.38474837,1476.84189325)
\curveto(703.7528044,1476.70994891)(703.10002712,1476.51550461)(702.42641652,1476.25856036)
\lineto(702.32224993,1476.25856036)
\lineto(702.32224993,1478.24814219)
\curveto(702.70419409,1478.35230878)(703.256277,1478.46689202)(703.97849868,1478.59189193)
\curveto(704.70072035,1478.71689184)(705.41252537,1478.77939179)(706.11391373,1478.77939179)
\curveto(706.93335755,1478.77939179)(707.64516257,1478.7099474)(708.24932878,1478.57105861)
\curveto(708.86043943,1478.43911427)(709.38821681,1478.20994777)(709.83266092,1477.88355913)
\curveto(710.27016058,1477.56411493)(710.60349367,1477.1509208)(710.83266016,1476.64397674)
\curveto(711.06182665,1476.13703268)(711.1764099,1475.50856093)(711.1764099,1474.7585615)
\closepath
\moveto(709.22849471,1469.7273153)
\lineto(709.22849471,1472.96689619)
\curveto(708.63127294,1472.93217399)(707.92641236,1472.8800907)(707.11391297,1472.8106463)
\curveto(706.30835803,1472.74120191)(705.66946962,1472.64050754)(705.19724776,1472.5085632)
\curveto(704.63474818,1472.3488411)(704.17988741,1472.09884129)(703.83266545,1471.75856377)
\curveto(703.4854435,1471.42523069)(703.31183252,1470.96342548)(703.31183252,1470.37314815)
\curveto(703.31183252,1469.70648199)(703.51322125,1469.20301014)(703.91599873,1468.86273262)
\curveto(704.3187762,1468.52939954)(704.93335907,1468.362733)(705.75974733,1468.362733)
\curveto(706.44724681,1468.362733)(707.07571856,1468.49467735)(707.64516257,1468.75856604)
\curveto(708.21460658,1469.02939916)(708.74238396,1469.35231559)(709.22849471,1469.7273153)
\closepath
}
}
{
\newrgbcolor{curcolor}{0 0 0}
\pscustom[linestyle=none,fillstyle=solid,fillcolor=curcolor]
{
\newpath
\moveto(724.38472371,1466.86273414)
\lineto(722.42639186,1466.86273414)
\lineto(722.42639186,1468.08148322)
\curveto(721.86389228,1467.59537247)(721.27708717,1467.21690054)(720.66597652,1466.94606741)
\curveto(720.05486587,1466.67523428)(719.39167193,1466.53981772)(718.67639469,1466.53981772)
\curveto(717.28750686,1466.53981772)(716.18334102,1467.07453953)(715.3638972,1468.14398317)
\curveto(714.55139782,1469.2134268)(714.14514812,1470.69606457)(714.14514812,1472.59189647)
\curveto(714.14514812,1473.57800683)(714.28403691,1474.45647839)(714.56181447,1475.22731114)
\curveto(714.84653648,1475.99814389)(715.22848064,1476.6543934)(715.70764694,1477.19605965)
\curveto(716.17986881,1477.72383703)(716.7284795,1478.1266145)(717.35347903,1478.40439207)
\curveto(717.98542299,1478.68216964)(718.63820028,1478.82105842)(719.31181088,1478.82105842)
\curveto(719.92292153,1478.82105842)(720.46458779,1478.75508625)(720.93680965,1478.62314191)
\curveto(721.40903152,1478.498142)(721.90555892,1478.30022548)(722.42639186,1478.02939236)
\lineto(722.42639186,1483.07105521)
\lineto(724.38472371,1483.07105521)
\closepath
\moveto(722.42639186,1469.7273153)
\lineto(722.42639186,1476.40439359)
\curveto(721.89861448,1476.64050452)(721.42639261,1476.80369884)(721.00972626,1476.89397655)
\curveto(720.59305991,1476.98425426)(720.13819914,1477.02939311)(719.64514396,1477.02939311)
\curveto(718.54792257,1477.02939311)(717.69375655,1476.64744896)(717.0826459,1475.88356065)
\curveto(716.47153525,1475.11967234)(716.16597993,1474.03633982)(716.16597993,1472.6335631)
\curveto(716.16597993,1471.25161971)(716.40209086,1470.19953717)(716.87431272,1469.47731549)
\curveto(717.34653459,1468.76203826)(718.10347846,1468.40439964)(719.14514434,1468.40439964)
\curveto(719.70069947,1468.40439964)(720.26319905,1468.52592732)(720.83264306,1468.7689827)
\curveto(721.40208708,1469.01898251)(721.93333667,1469.33842671)(722.42639186,1469.7273153)
\closepath
}
}
{
\newrgbcolor{curcolor}{0 0 0}
\pscustom[linestyle=none,fillstyle=solid,fillcolor=curcolor]
{
\newpath
\moveto(442.21827387,1440.19608765)
\lineto(436.0932785,1440.19608765)
\lineto(436.0932785,1441.77941978)
\lineto(438.12452697,1441.77941978)
\lineto(438.12452697,1454.12316044)
\lineto(436.0932785,1454.12316044)
\lineto(436.0932785,1455.70649258)
\lineto(442.21827387,1455.70649258)
\lineto(442.21827387,1454.12316044)
\lineto(440.18702541,1454.12316044)
\lineto(440.18702541,1441.77941978)
\lineto(442.21827387,1441.77941978)
\closepath
}
}
{
\newrgbcolor{curcolor}{0 0 0}
\pscustom[linestyle=none,fillstyle=solid,fillcolor=curcolor]
{
\newpath
\moveto(451.83284703,1454.50857682)
\lineto(451.72868044,1454.50857682)
\curveto(451.51340283,1454.57107677)(451.23215304,1454.63357672)(450.88493108,1454.69607668)
\curveto(450.53770912,1454.76552107)(450.2321538,1454.80024326)(449.96826511,1454.80024326)
\curveto(449.12798797,1454.80024326)(448.51687732,1454.61274341)(448.13493316,1454.23774369)
\curveto(447.75993345,1453.86968841)(447.57243359,1453.19955003)(447.57243359,1452.22732854)
\lineto(447.57243359,1451.83149551)
\lineto(451.10368092,1451.83149551)
\lineto(451.10368092,1450.18566342)
\lineto(447.63493354,1450.18566342)
\lineto(447.63493354,1440.19608765)
\lineto(445.67660169,1440.19608765)
\lineto(445.67660169,1450.18566342)
\lineto(444.35368602,1450.18566342)
\lineto(444.35368602,1451.83149551)
\lineto(445.67660169,1451.83149551)
\lineto(445.67660169,1452.21691188)
\curveto(445.67660169,1453.59885528)(446.02035143,1454.65788226)(446.70785091,1455.39399281)
\curveto(447.39535039,1456.13704781)(448.38840519,1456.5085753)(449.68701532,1456.5085753)
\curveto(450.12451499,1456.5085753)(450.5168758,1456.48774199)(450.86409776,1456.44607535)
\curveto(451.21826416,1456.40440872)(451.54118058,1456.35579764)(451.83284703,1456.30024213)
\closepath
}
}
{
\newrgbcolor{curcolor}{0 0 0}
\pscustom[linestyle=none,fillstyle=solid,fillcolor=curcolor]
{
\newpath
\moveto(472.56200215,1440.19608765)
\lineto(470.00992075,1440.19608765)
\lineto(462.65575965,1454.07107715)
\lineto(462.65575965,1440.19608765)
\lineto(460.72867777,1440.19608765)
\lineto(460.72867777,1455.70649258)
\lineto(463.92659202,1455.70649258)
\lineto(470.63492028,1443.03983549)
\lineto(470.63492028,1455.70649258)
\lineto(472.56200215,1455.70649258)
\closepath
}
}
{
\newrgbcolor{curcolor}{0 0 0}
\pscustom[linestyle=none,fillstyle=solid,fillcolor=curcolor]
{
\newpath
\moveto(488.36407151,1446.4252496)
\curveto(488.36407151,1445.30025045)(488.2390716,1444.31761231)(487.98907179,1443.47733516)
\curveto(487.74601642,1442.64400246)(487.34323895,1441.94955854)(486.78073937,1441.39400341)
\curveto(486.24601756,1440.86622603)(485.62101803,1440.48080965)(484.90574079,1440.23775428)
\curveto(484.19046355,1439.99469891)(483.35713085,1439.87317122)(482.40574268,1439.87317122)
\curveto(481.4335212,1439.87317122)(480.58629961,1440.00164335)(479.86407794,1440.2585876)
\curveto(479.14185626,1440.51553185)(478.53421783,1440.89400378)(478.04116265,1441.39400341)
\curveto(477.47866308,1441.96344742)(477.07241338,1442.6509469)(476.82241357,1443.45650185)
\curveto(476.5793582,1444.26205679)(476.45783052,1445.25163938)(476.45783052,1446.4252496)
\lineto(476.45783052,1455.70649258)
\lineto(478.52032896,1455.70649258)
\lineto(478.52032896,1446.32108301)
\curveto(478.52032896,1445.48080587)(478.57588447,1444.81761193)(478.6869955,1444.33150118)
\curveto(478.80505096,1443.84539044)(478.99949526,1443.40441855)(479.27032839,1443.00858552)
\curveto(479.57588371,1442.55719697)(479.98907784,1442.21691945)(480.50991078,1441.98775296)
\curveto(481.03768816,1441.75858646)(481.66963213,1441.64400322)(482.40574268,1441.64400322)
\curveto(483.14879768,1441.64400322)(483.78074164,1441.75511424)(484.30157458,1441.9773363)
\curveto(484.82240752,1442.20650279)(485.23907387,1442.55025253)(485.55157364,1443.00858552)
\curveto(485.82240676,1443.40441855)(486.01337884,1443.8558071)(486.12448987,1444.36275116)
\curveto(486.24254534,1444.87663966)(486.30157307,1445.51205585)(486.30157307,1446.26899972)
\lineto(486.30157307,1455.70649258)
\lineto(488.36407151,1455.70649258)
\closepath
}
}
{
\newrgbcolor{curcolor}{0 0 0}
\pscustom[linestyle=none,fillstyle=solid,fillcolor=curcolor]
{
\newpath
\moveto(506.11405772,1440.19608765)
\lineto(504.05155928,1440.19608765)
\lineto(504.05155928,1453.56066087)
\lineto(499.73906254,1444.46691775)
\lineto(498.5098968,1444.46691775)
\lineto(494.22865004,1453.56066087)
\lineto(494.22865004,1440.19608765)
\lineto(492.30156817,1440.19608765)
\lineto(492.30156817,1455.70649258)
\lineto(495.11406604,1455.70649258)
\lineto(499.24947958,1447.07108244)
\lineto(503.24947655,1455.70649258)
\lineto(506.11405772,1455.70649258)
\closepath
}
}
{
\newrgbcolor{curcolor}{0 0 0}
\pscustom[linestyle=none,fillstyle=solid,fillcolor=curcolor]
{
\newpath
\moveto(522.50988039,1440.19608765)
\lineto(520.31196539,1440.19608765)
\lineto(518.79113321,1444.51900104)
\lineto(512.08280495,1444.51900104)
\lineto(510.56197277,1440.19608765)
\lineto(508.46822435,1440.19608765)
\lineto(514.11405341,1455.70649258)
\lineto(516.86405133,1455.70649258)
\closepath
\moveto(518.15571702,1446.28983304)
\lineto(515.43696908,1453.90441061)
\lineto(512.70780448,1446.28983304)
\closepath
}
}
{
\newrgbcolor{curcolor}{0 0 0}
\pscustom[linestyle=none,fillstyle=solid,fillcolor=curcolor]
{
\newpath
\moveto(544.0619433,1444.96691737)
\curveto(544.0619433,1444.19608462)(543.91611008,1443.51552958)(543.62444364,1442.92525225)
\curveto(543.33277719,1442.33497492)(542.94041637,1441.84886417)(542.44736119,1441.46692002)
\curveto(541.8640283,1441.00858703)(541.22166768,1440.68219839)(540.52027932,1440.48775409)
\curveto(539.8258354,1440.29330979)(538.9404194,1440.19608765)(537.86403133,1440.19608765)
\lineto(532.36403549,1440.19608765)
\lineto(532.36403549,1455.70649258)
\lineto(536.95778201,1455.70649258)
\curveto(538.0897256,1455.70649258)(538.93694718,1455.66482594)(539.49944676,1455.58149267)
\curveto(540.06194633,1455.4981594)(540.60014037,1455.32454842)(541.11402887,1455.06065973)
\curveto(541.68347288,1454.76204885)(542.09666701,1454.37663247)(542.35361126,1453.90441061)
\curveto(542.61055551,1453.43913318)(542.73902764,1452.88010583)(542.73902764,1452.22732854)
\curveto(542.73902764,1451.49121799)(542.55152778,1450.86274624)(542.17652806,1450.3419133)
\curveto(541.80152835,1449.8280248)(541.30152873,1449.41483067)(540.6765292,1449.10233091)
\lineto(540.6765292,1449.01899764)
\curveto(541.72513952,1448.80372002)(542.55152778,1448.34191482)(543.15569399,1447.63358202)
\curveto(543.7598602,1446.93219366)(544.0619433,1446.04330544)(544.0619433,1444.96691737)
\closepath
\moveto(540.59319593,1451.95649541)
\curveto(540.59319593,1452.33149513)(540.53069598,1452.64746711)(540.40569607,1452.90441136)
\curveto(540.28069616,1453.16135561)(540.07930743,1453.36968879)(539.80152986,1453.52941089)
\curveto(539.47514122,1453.71691075)(539.07930818,1453.831494)(538.61403076,1453.87316063)
\curveto(538.14875333,1453.92177171)(537.57236488,1453.94607724)(536.8848654,1453.94607724)
\lineto(534.42653393,1453.94607724)
\lineto(534.42653393,1449.46691397)
\lineto(537.09319858,1449.46691397)
\curveto(537.73903142,1449.46691397)(538.25291992,1449.49816394)(538.63486408,1449.56066389)
\curveto(539.01680823,1449.63010829)(539.37097463,1449.76899707)(539.69736327,1449.97733025)
\curveto(540.02375191,1450.18566342)(540.25291841,1450.45302433)(540.38486275,1450.77941297)
\curveto(540.52375154,1451.11274605)(540.59319593,1451.50510687)(540.59319593,1451.95649541)
\closepath
\moveto(541.91611159,1444.8835841)
\curveto(541.91611159,1445.50858363)(541.82236167,1446.00511103)(541.63486181,1446.37316631)
\curveto(541.44736195,1446.74122158)(541.10708443,1447.05372135)(540.61402925,1447.3106656)
\curveto(540.28069616,1447.48427658)(539.87444647,1447.5953876)(539.39528017,1447.64399868)
\curveto(538.9230583,1447.69955419)(538.34666985,1447.72733195)(537.66611481,1447.72733195)
\lineto(534.42653393,1447.72733195)
\lineto(534.42653393,1441.95650298)
\lineto(537.15569853,1441.95650298)
\curveto(538.05847562,1441.95650298)(538.7980584,1442.00164184)(539.37444685,1442.09191954)
\curveto(539.9508353,1442.18914169)(540.42305717,1442.36275267)(540.79111245,1442.61275248)
\curveto(541.18000104,1442.88358561)(541.46472305,1443.19261316)(541.64527847,1443.53983512)
\curveto(541.82583388,1443.88705708)(541.91611159,1444.3349734)(541.91611159,1444.8835841)
\closepath
}
}
{
\newrgbcolor{curcolor}{0 0 0}
\pscustom[linestyle=none,fillstyle=solid,fillcolor=curcolor]
{
\newpath
\moveto(555.87443966,1440.19608765)
\lineto(553.92652447,1440.19608765)
\lineto(553.92652447,1441.43567004)
\curveto(553.75291349,1441.31761457)(553.51680256,1441.15094803)(553.21819167,1440.93567042)
\curveto(552.92652523,1440.72733724)(552.64180322,1440.5606707)(552.36402565,1440.4356708)
\curveto(552.03763701,1440.2759487)(551.6626373,1440.14400435)(551.23902651,1440.03983776)
\curveto(550.81541571,1439.92872674)(550.31888831,1439.87317122)(549.7494443,1439.87317122)
\curveto(548.70083398,1439.87317122)(547.81194576,1440.22039318)(547.08277965,1440.9148371)
\curveto(546.35361353,1441.60928102)(545.98903048,1442.49469702)(545.98903048,1443.57108509)
\curveto(545.98903048,1444.45302887)(546.17653034,1445.16483389)(546.55153005,1445.70650014)
\curveto(546.93347421,1446.25511084)(547.47514046,1446.68566607)(548.17652882,1446.99816583)
\curveto(548.88486162,1447.3106656)(549.73555542,1447.52247099)(550.72861022,1447.63358202)
\curveto(551.72166503,1447.74469305)(552.78763644,1447.82802632)(553.92652447,1447.88358183)
\lineto(553.92652447,1448.18566493)
\curveto(553.92652447,1448.63010904)(553.84666342,1448.99816432)(553.68694132,1449.28983077)
\curveto(553.53416366,1449.58149721)(553.3119416,1449.8106637)(553.02027516,1449.97733025)
\curveto(552.74249759,1450.13705235)(552.40916451,1450.24469115)(552.02027591,1450.30024667)
\curveto(551.63138732,1450.35580218)(551.22513763,1450.38357994)(550.80152684,1450.38357994)
\curveto(550.28763834,1450.38357994)(549.7147221,1450.31413555)(549.08277814,1450.17524676)
\curveto(548.45083417,1450.04330242)(547.79805689,1449.84885812)(547.12444628,1449.59191387)
\lineto(547.0202797,1449.59191387)
\lineto(547.0202797,1451.5814957)
\curveto(547.40222385,1451.68566229)(547.95430677,1451.80024553)(548.67652844,1451.92524544)
\curveto(549.39875012,1452.05024534)(550.11055514,1452.1127453)(550.81194349,1452.1127453)
\curveto(551.63138732,1452.1127453)(552.34319234,1452.0433009)(552.94735855,1451.90441212)
\curveto(553.55846919,1451.77246778)(554.08624657,1451.54330128)(554.53069068,1451.21691264)
\curveto(554.96819035,1450.89746844)(555.30152343,1450.48427431)(555.53068992,1449.97733025)
\curveto(555.75985642,1449.47038618)(555.87443966,1448.84191444)(555.87443966,1448.09191501)
\closepath
\moveto(553.92652447,1443.06066881)
\lineto(553.92652447,1446.30024969)
\curveto(553.3293027,1446.2655275)(552.62444212,1446.2134442)(551.81194274,1446.14399981)
\curveto(551.00638779,1446.07455542)(550.36749939,1445.97386105)(549.89527752,1445.84191671)
\curveto(549.33277795,1445.68219461)(548.87791718,1445.4321948)(548.53069522,1445.09191727)
\curveto(548.18347326,1444.75858419)(548.00986228,1444.29677899)(548.00986228,1443.70650166)
\curveto(548.00986228,1443.03983549)(548.21125102,1442.53636365)(548.61402849,1442.19608613)
\curveto(549.01680596,1441.86275305)(549.63138883,1441.69608651)(550.4577771,1441.69608651)
\curveto(551.14527658,1441.69608651)(551.77374832,1441.82803086)(552.34319234,1442.09191954)
\curveto(552.91263635,1442.36275267)(553.44041373,1442.6856691)(553.92652447,1443.06066881)
\closepath
}
}
{
\newrgbcolor{curcolor}{0 0 0}
\pscustom[linestyle=none,fillstyle=solid,fillcolor=curcolor]
{
\newpath
\moveto(561.62442578,1440.19608765)
\lineto(559.66609393,1440.19608765)
\lineto(559.66609393,1456.40440872)
\lineto(561.62442578,1456.40440872)
\closepath
}
}
{
\newrgbcolor{curcolor}{0 0 0}
\pscustom[linestyle=none,fillstyle=solid,fillcolor=curcolor]
{
\newpath
\moveto(574.54109063,1440.19608765)
\lineto(572.59317544,1440.19608765)
\lineto(572.59317544,1441.43567004)
\curveto(572.41956446,1441.31761457)(572.18345353,1441.15094803)(571.88484264,1440.93567042)
\curveto(571.5931762,1440.72733724)(571.30845419,1440.5606707)(571.03067662,1440.4356708)
\curveto(570.70428798,1440.2759487)(570.32928826,1440.14400435)(569.90567747,1440.03983776)
\curveto(569.48206668,1439.92872674)(568.98553928,1439.87317122)(568.41609527,1439.87317122)
\curveto(567.36748495,1439.87317122)(566.47859673,1440.22039318)(565.74943062,1440.9148371)
\curveto(565.0202645,1441.60928102)(564.65568144,1442.49469702)(564.65568144,1443.57108509)
\curveto(564.65568144,1444.45302887)(564.8431813,1445.16483389)(565.21818102,1445.70650014)
\curveto(565.60012517,1446.25511084)(566.14179143,1446.68566607)(566.84317979,1446.99816583)
\curveto(567.55151259,1447.3106656)(568.40220639,1447.52247099)(569.39526119,1447.63358202)
\curveto(570.388316,1447.74469305)(571.45428741,1447.82802632)(572.59317544,1447.88358183)
\lineto(572.59317544,1448.18566493)
\curveto(572.59317544,1448.63010904)(572.51331439,1448.99816432)(572.35359229,1449.28983077)
\curveto(572.20081462,1449.58149721)(571.97859257,1449.8106637)(571.68692612,1449.97733025)
\curveto(571.40914856,1450.13705235)(571.07581548,1450.24469115)(570.68692688,1450.30024667)
\curveto(570.29803829,1450.35580218)(569.89178859,1450.38357994)(569.4681778,1450.38357994)
\curveto(568.9542893,1450.38357994)(568.38137307,1450.31413555)(567.7494291,1450.17524676)
\curveto(567.11748514,1450.04330242)(566.46470785,1449.84885812)(565.79109725,1449.59191387)
\lineto(565.68693066,1449.59191387)
\lineto(565.68693066,1451.5814957)
\curveto(566.06887482,1451.68566229)(566.62095774,1451.80024553)(567.34317941,1451.92524544)
\curveto(568.06540109,1452.05024534)(568.7772061,1452.1127453)(569.47859446,1452.1127453)
\curveto(570.29803829,1452.1127453)(571.0098433,1452.0433009)(571.61400951,1451.90441212)
\curveto(572.22512016,1451.77246778)(572.75289754,1451.54330128)(573.19734165,1451.21691264)
\curveto(573.63484132,1450.89746844)(573.9681744,1450.48427431)(574.19734089,1449.97733025)
\curveto(574.42650739,1449.47038618)(574.54109063,1448.84191444)(574.54109063,1448.09191501)
\closepath
\moveto(572.59317544,1443.06066881)
\lineto(572.59317544,1446.30024969)
\curveto(571.99595367,1446.2655275)(571.29109309,1446.2134442)(570.47859371,1446.14399981)
\curveto(569.67303876,1446.07455542)(569.03415035,1445.97386105)(568.56192849,1445.84191671)
\curveto(567.99942891,1445.68219461)(567.54456815,1445.4321948)(567.19734619,1445.09191727)
\curveto(566.85012423,1444.75858419)(566.67651325,1444.29677899)(566.67651325,1443.70650166)
\curveto(566.67651325,1443.03983549)(566.87790199,1442.53636365)(567.28067946,1442.19608613)
\curveto(567.68345693,1441.86275305)(568.2980398,1441.69608651)(569.12442806,1441.69608651)
\curveto(569.81192754,1441.69608651)(570.44039929,1441.82803086)(571.0098433,1442.09191954)
\curveto(571.57928732,1442.36275267)(572.1070647,1442.6856691)(572.59317544,1443.06066881)
\closepath
}
}
{
\newrgbcolor{curcolor}{0 0 0}
\pscustom[linestyle=none,fillstyle=solid,fillcolor=curcolor]
{
\newpath
\moveto(588.04108531,1440.19608765)
\lineto(586.08275345,1440.19608765)
\lineto(586.08275345,1446.82108263)
\curveto(586.08275345,1447.35580445)(586.05150348,1447.85580407)(585.98900353,1448.3210815)
\curveto(585.92650357,1448.79330336)(585.81192033,1449.16135864)(585.64525379,1449.42524733)
\curveto(585.47164281,1449.71691378)(585.22164299,1449.93219139)(584.89525435,1450.07108017)
\curveto(584.56886571,1450.2169134)(584.14525492,1450.28983001)(583.62442198,1450.28983001)
\curveto(583.08970016,1450.28983001)(582.53067281,1450.15788566)(581.94733992,1449.89399698)
\curveto(581.36400702,1449.63010829)(580.80497967,1449.29330299)(580.27025785,1448.88358107)
\lineto(580.27025785,1440.19608765)
\lineto(578.311926,1440.19608765)
\lineto(578.311926,1451.83149551)
\lineto(580.27025785,1451.83149551)
\lineto(580.27025785,1450.53982982)
\curveto(580.8813685,1451.04677388)(581.51331247,1451.44260691)(582.16608975,1451.72732892)
\curveto(582.81886703,1452.01205093)(583.48900542,1452.15441193)(584.1765049,1452.15441193)
\curveto(585.43344839,1452.15441193)(586.391781,1451.77594)(587.05150272,1451.01899612)
\curveto(587.71122444,1450.26205225)(588.04108531,1449.1717753)(588.04108531,1447.74816527)
\closepath
}
}
{
\newrgbcolor{curcolor}{0 0 0}
\pscustom[linestyle=none,fillstyle=solid,fillcolor=curcolor]
{
\newpath
\moveto(600.41606309,1440.92525376)
\curveto(599.7632858,1440.612754)(599.1417585,1440.36969863)(598.55148117,1440.19608765)
\curveto(597.96814827,1440.02247667)(597.34662097,1439.93567118)(596.68689924,1439.93567118)
\curveto(595.8466221,1439.93567118)(595.07578935,1440.05719886)(594.37440099,1440.30025423)
\curveto(593.67301263,1440.55025404)(593.07231864,1440.92525376)(592.57231902,1441.42525338)
\curveto(592.06537496,1441.925253)(591.67301415,1442.55719697)(591.39523658,1443.32108528)
\curveto(591.11745901,1444.08497359)(590.97857023,1444.97733403)(590.97857023,1445.99816659)
\curveto(590.97857023,1447.90094293)(591.49940317,1449.39399735)(592.54106905,1450.47732987)
\curveto(593.58967936,1451.56066238)(594.97162276,1452.10232864)(596.68689924,1452.10232864)
\curveto(597.35356541,1452.10232864)(598.00634269,1452.00857871)(598.6452311,1451.82107885)
\curveto(599.29106394,1451.63357899)(599.88134127,1451.4044125)(600.41606309,1451.13357937)
\lineto(600.41606309,1448.95649768)
\lineto(600.3118965,1448.95649768)
\curveto(599.71467473,1449.42177511)(599.09661964,1449.77941373)(598.45773124,1450.02941354)
\curveto(597.82578727,1450.27941335)(597.20773218,1450.40441326)(596.60356597,1450.40441326)
\curveto(595.4924557,1450.40441326)(594.61398415,1450.02941354)(593.9681513,1449.27941411)
\curveto(593.32926289,1448.53635911)(593.00981869,1447.44260994)(593.00981869,1445.99816659)
\curveto(593.00981869,1444.59538987)(593.32231846,1443.51552958)(593.94731798,1442.75858571)
\curveto(594.57926195,1442.00858627)(595.46467795,1441.63358656)(596.60356597,1441.63358656)
\curveto(596.99939901,1441.63358656)(597.40217648,1441.68566985)(597.81189839,1441.78983644)
\curveto(598.2216203,1441.89400303)(598.58967558,1442.02941959)(598.91606422,1442.19608613)
\curveto(599.20078623,1442.34191936)(599.46814714,1442.49469702)(599.71814695,1442.65441912)
\curveto(599.96814676,1442.82108566)(600.16606328,1442.96344666)(600.3118965,1443.08150213)
\lineto(600.41606309,1443.08150213)
\closepath
}
}
{
\newrgbcolor{curcolor}{0 0 0}
\pscustom[linestyle=none,fillstyle=solid,fillcolor=curcolor]
{
\newpath
\moveto(605.03065647,1453.7794107)
\lineto(602.8223248,1453.7794107)
\lineto(602.8223248,1455.81065917)
\lineto(605.03065647,1455.81065917)
\closepath
\moveto(604.90565656,1440.19608765)
\lineto(602.94732471,1440.19608765)
\lineto(602.94732471,1451.83149551)
\lineto(604.90565656,1451.83149551)
\closepath
}
}
{
\newrgbcolor{curcolor}{0 0 0}
\pscustom[linestyle=none,fillstyle=solid,fillcolor=curcolor]
{
\newpath
\moveto(618.50980647,1440.19608765)
\lineto(616.55147462,1440.19608765)
\lineto(616.55147462,1446.82108263)
\curveto(616.55147462,1447.35580445)(616.52022464,1447.85580407)(616.45772469,1448.3210815)
\curveto(616.39522474,1448.79330336)(616.28064149,1449.16135864)(616.11397495,1449.42524733)
\curveto(615.94036397,1449.71691378)(615.69036416,1449.93219139)(615.36397552,1450.07108017)
\curveto(615.03758688,1450.2169134)(614.61397608,1450.28983001)(614.09314315,1450.28983001)
\curveto(613.55842133,1450.28983001)(612.99939397,1450.15788566)(612.41606108,1449.89399698)
\curveto(611.83272819,1449.63010829)(611.27370083,1449.29330299)(610.73897902,1448.88358107)
\lineto(610.73897902,1440.19608765)
\lineto(608.78064716,1440.19608765)
\lineto(608.78064716,1451.83149551)
\lineto(610.73897902,1451.83149551)
\lineto(610.73897902,1450.53982982)
\curveto(611.35008967,1451.04677388)(611.98203363,1451.44260691)(612.63481092,1451.72732892)
\curveto(613.2875882,1452.01205093)(613.95772658,1452.15441193)(614.64522606,1452.15441193)
\curveto(615.90216955,1452.15441193)(616.86050216,1451.77594)(617.52022389,1451.01899612)
\curveto(618.17994561,1450.26205225)(618.50980647,1449.1717753)(618.50980647,1447.74816527)
\closepath
}
}
{
\newrgbcolor{curcolor}{0 0 0}
\pscustom[linestyle=none,fillstyle=solid,fillcolor=curcolor]
{
\newpath
\moveto(631.71813137,1441.51900331)
\curveto(631.71813137,1439.54678258)(631.27021505,1438.09886701)(630.37438239,1437.1752566)
\curveto(629.47854973,1436.25164619)(628.10007856,1435.78984098)(626.23896885,1435.78984098)
\curveto(625.62091376,1435.78984098)(625.01674755,1435.83497983)(624.42647022,1435.92525754)
\curveto(623.84313733,1436.00859081)(623.26674888,1436.1301185)(622.69730487,1436.2898406)
\lineto(622.69730487,1438.28983909)
\lineto(622.80147145,1438.28983909)
\curveto(623.12091566,1438.16483918)(623.62785972,1438.01206152)(624.32230364,1437.8315061)
\curveto(625.01674755,1437.64400624)(625.71119147,1437.55025631)(626.40563539,1437.55025631)
\curveto(627.07230156,1437.55025631)(627.62438447,1437.63011736)(628.06188414,1437.78983947)
\curveto(628.49938381,1437.94956157)(628.83966133,1438.17178362)(629.0827167,1438.45650563)
\curveto(629.32577207,1438.72733876)(629.49938305,1439.0537274)(629.60354964,1439.43567155)
\curveto(629.70771623,1439.81761571)(629.75979952,1440.24469872)(629.75979952,1440.71692058)
\lineto(629.75979952,1441.77941978)
\curveto(629.16952219,1441.30719792)(628.6035504,1440.95303152)(628.06188414,1440.71692058)
\curveto(627.52716232,1440.48775409)(626.84313506,1440.37317084)(626.00980236,1440.37317084)
\curveto(624.62091452,1440.37317084)(623.51674869,1440.87317047)(622.69730487,1441.87316971)
\curveto(621.88480548,1442.88011339)(621.47855579,1444.29677899)(621.47855579,1446.12316649)
\curveto(621.47855579,1447.12316574)(621.61744457,1447.9842762)(621.89522214,1448.70649787)
\curveto(622.17994415,1449.43566399)(622.56536052,1450.06413574)(623.05147126,1450.59191311)
\curveto(623.50285981,1451.0849683)(624.05147051,1451.46691245)(624.69730335,1451.73774558)
\curveto(625.3431362,1452.01552315)(625.98549682,1452.15441193)(626.62438523,1452.15441193)
\curveto(627.29799583,1452.15441193)(627.8604954,1452.08496754)(628.31188395,1451.94607876)
\curveto(628.77021694,1451.81413441)(629.25285546,1451.60927346)(629.75979952,1451.33149589)
\lineto(629.88479943,1451.83149551)
\lineto(631.71813137,1451.83149551)
\closepath
\moveto(629.75979952,1443.39400189)
\lineto(629.75979952,1449.73774709)
\curveto(629.23896658,1449.97385803)(628.75285584,1450.14052457)(628.30146729,1450.23774672)
\curveto(627.85702318,1450.3419133)(627.41257908,1450.3939966)(626.96813497,1450.3939966)
\curveto(625.89174689,1450.3939966)(625.04452531,1450.03288576)(624.42647022,1449.31066408)
\curveto(623.80841514,1448.58844241)(623.49938759,1447.53983209)(623.49938759,1446.16483313)
\curveto(623.49938759,1444.85927856)(623.72855409,1443.86969598)(624.18688707,1443.19608538)
\curveto(624.64522006,1442.52247477)(625.40563615,1442.18566947)(626.46813535,1442.18566947)
\curveto(627.03757936,1442.18566947)(627.60702337,1442.29330828)(628.17646739,1442.5085859)
\curveto(628.75285584,1442.73080795)(629.28063322,1443.02594662)(629.75979952,1443.39400189)
\closepath
}
}
{
\newrgbcolor{curcolor}{0 0 0}
\pscustom[linestyle=none,fillstyle=solid,fillcolor=curcolor]
{
\newpath
\moveto(645.17644398,1453.7794107)
\lineto(642.96811232,1453.7794107)
\lineto(642.96811232,1455.81065917)
\lineto(645.17644398,1455.81065917)
\closepath
\moveto(645.05144408,1440.19608765)
\lineto(643.09311223,1440.19608765)
\lineto(643.09311223,1451.83149551)
\lineto(645.05144408,1451.83149551)
\closepath
}
}
{
\newrgbcolor{curcolor}{0 0 0}
\pscustom[linestyle=none,fillstyle=solid,fillcolor=curcolor]
{
\newpath
\moveto(657.23895723,1443.55025177)
\curveto(657.23895723,1442.48775258)(656.79798534,1441.61622546)(655.91604156,1440.93567042)
\curveto(655.04104222,1440.25511538)(653.84312646,1439.91483786)(652.32229428,1439.91483786)
\curveto(651.46118382,1439.91483786)(650.66951775,1440.01553223)(649.94729608,1440.21692096)
\curveto(649.23201884,1440.42525414)(648.63132485,1440.65094841)(648.14521411,1440.89400378)
\lineto(648.14521411,1443.09191879)
\lineto(648.2493807,1443.09191879)
\curveto(648.86743578,1442.62664136)(649.55493526,1442.25511387)(650.31187914,1441.9773363)
\curveto(651.06882301,1441.70650317)(651.7945169,1441.57108661)(652.48896082,1441.57108661)
\curveto(653.35007128,1441.57108661)(654.02368188,1441.70997539)(654.50979263,1441.98775296)
\curveto(654.99590337,1442.26553052)(655.23895874,1442.70303019)(655.23895874,1443.30025196)
\curveto(655.23895874,1443.75858495)(655.1070144,1444.10580691)(654.84312571,1444.34191784)
\curveto(654.57923702,1444.57802877)(654.07229296,1444.77941751)(653.32229352,1444.94608405)
\curveto(653.04451596,1445.008584)(652.6799329,1445.08150062)(652.22854435,1445.16483389)
\curveto(651.78410024,1445.24816716)(651.37785055,1445.33844487)(651.00979527,1445.43566701)
\curveto(649.98896271,1445.70650014)(649.26326882,1446.10233318)(648.83271359,1446.62316612)
\curveto(648.4091028,1447.15094349)(648.1972974,1447.79677634)(648.1972974,1448.56066465)
\curveto(648.1972974,1449.03983095)(648.29451955,1449.4912195)(648.48896385,1449.91483029)
\curveto(648.69035258,1450.33844108)(648.99243569,1450.71691302)(649.39521316,1451.0502461)
\curveto(649.78410176,1451.37663474)(650.27715694,1451.63357899)(650.87437871,1451.82107885)
\curveto(651.47854492,1452.01552315)(652.15215552,1452.1127453)(652.89521051,1452.1127453)
\curveto(653.58965443,1452.1127453)(654.29104279,1452.02593981)(654.99937559,1451.85232883)
\curveto(655.71465283,1451.68566229)(656.30840238,1451.48080133)(656.78062424,1451.23774596)
\lineto(656.78062424,1449.14399754)
\lineto(656.67645765,1449.14399754)
\curveto(656.17645803,1449.51205282)(655.5688196,1449.82108036)(654.85354237,1450.07108017)
\curveto(654.13826513,1450.32802442)(653.43687677,1450.45649655)(652.74937729,1450.45649655)
\curveto(652.03410005,1450.45649655)(651.42993385,1450.31760777)(650.93687866,1450.0398302)
\curveto(650.44382348,1449.76899707)(650.19729589,1449.36274738)(650.19729589,1448.82108112)
\curveto(650.19729589,1448.34191482)(650.34660133,1447.98080398)(650.64521222,1447.73774861)
\curveto(650.93687866,1447.49469323)(651.40910053,1447.29677672)(652.06187781,1447.14399906)
\curveto(652.42298865,1447.06066579)(652.82576612,1446.97733252)(653.27021023,1446.89399924)
\curveto(653.72159878,1446.81066597)(654.09659849,1446.73427714)(654.39520938,1446.66483275)
\curveto(655.30493091,1446.45649958)(656.00631927,1446.09886096)(656.49937445,1445.5919169)
\curveto(656.99242964,1445.0780284)(657.23895723,1444.39747336)(657.23895723,1443.55025177)
\closepath
}
}
{
\newrgbcolor{curcolor}{0 0 0}
\pscustom[linestyle=none,fillstyle=solid,fillcolor=curcolor]
{
\newpath
\moveto(677.28059911,1445.81066673)
\lineto(668.70768893,1445.81066673)
\curveto(668.70768893,1445.09538949)(668.81532773,1444.47038997)(669.03060535,1443.93566815)
\curveto(669.24588296,1443.40789077)(669.54102163,1442.97386332)(669.91602134,1442.6335858)
\curveto(670.27713218,1442.30025272)(670.70421519,1442.05025291)(671.19727037,1441.88358637)
\curveto(671.69727,1441.71691983)(672.24588069,1441.63358656)(672.84310246,1441.63358656)
\curveto(673.63476853,1441.63358656)(674.42990682,1441.78983644)(675.22851733,1442.1023362)
\curveto(676.03407227,1442.42178041)(676.6069885,1442.73428017)(676.94726602,1443.03983549)
\lineto(677.05143261,1443.03983549)
\lineto(677.05143261,1440.90442044)
\curveto(676.39171089,1440.62664288)(675.71810029,1440.39400416)(675.03060081,1440.2065043)
\curveto(674.34310133,1440.01900445)(673.62087965,1439.92525452)(672.86393578,1439.92525452)
\curveto(670.93338169,1439.92525452)(669.42643838,1440.44608746)(668.34310587,1441.48775334)
\curveto(667.25977335,1442.53636365)(666.7181071,1444.02247364)(666.7181071,1445.9460833)
\curveto(666.7181071,1447.84885963)(667.23546782,1449.35927516)(668.27018926,1450.47732987)
\curveto(669.31185513,1451.59538458)(670.67990966,1452.15441193)(672.37435282,1452.15441193)
\curveto(673.94379607,1452.15441193)(675.15212849,1451.69607895)(675.99935008,1450.77941297)
\curveto(676.8535161,1449.862747)(677.28059911,1448.56066465)(677.28059911,1446.87316593)
\closepath
\moveto(675.37435055,1447.3106656)
\curveto(675.36740611,1448.3384426)(675.10698964,1449.13358088)(674.59310114,1449.69608046)
\curveto(674.08615708,1450.25858003)(673.31185211,1450.53982982)(672.27018623,1450.53982982)
\curveto(671.22157591,1450.53982982)(670.38477099,1450.23080228)(669.75977146,1449.61274719)
\curveto(669.14171637,1448.9946921)(668.7910222,1448.22733157)(668.70768893,1447.3106656)
\closepath
}
}
{
\newrgbcolor{curcolor}{0 0 0}
\pscustom[linestyle=none,fillstyle=solid,fillcolor=curcolor]
{
\newpath
\moveto(689.97850937,1440.19608765)
\lineto(688.02017752,1440.19608765)
\lineto(688.02017752,1446.82108263)
\curveto(688.02017752,1447.35580445)(687.98892754,1447.85580407)(687.92642759,1448.3210815)
\curveto(687.86392764,1448.79330336)(687.74934439,1449.16135864)(687.58267785,1449.42524733)
\curveto(687.40906687,1449.71691378)(687.15906706,1449.93219139)(686.83267842,1450.07108017)
\curveto(686.50628978,1450.2169134)(686.08267899,1450.28983001)(685.56184605,1450.28983001)
\curveto(685.02712423,1450.28983001)(684.46809687,1450.15788566)(683.88476398,1449.89399698)
\curveto(683.30143109,1449.63010829)(682.74240374,1449.29330299)(682.20768192,1448.88358107)
\lineto(682.20768192,1440.19608765)
\lineto(680.24935007,1440.19608765)
\lineto(680.24935007,1451.83149551)
\lineto(682.20768192,1451.83149551)
\lineto(682.20768192,1450.53982982)
\curveto(682.81879257,1451.04677388)(683.45073653,1451.44260691)(684.10351382,1451.72732892)
\curveto(684.7562911,1452.01205093)(685.42642948,1452.15441193)(686.11392896,1452.15441193)
\curveto(687.37087246,1452.15441193)(688.32920506,1451.77594)(688.98892679,1451.01899612)
\curveto(689.64864851,1450.26205225)(689.97850937,1449.1717753)(689.97850937,1447.74816527)
\closepath
}
}
{
\newrgbcolor{curcolor}{0 0 0}
\pscustom[linestyle=none,fillstyle=solid,fillcolor=curcolor]
{
\newpath
\moveto(702.79098682,1440.19608765)
\lineto(700.84307163,1440.19608765)
\lineto(700.84307163,1441.43567004)
\curveto(700.66946065,1441.31761457)(700.43334972,1441.15094803)(700.13473883,1440.93567042)
\curveto(699.84307239,1440.72733724)(699.55835038,1440.5606707)(699.28057281,1440.4356708)
\curveto(698.95418417,1440.2759487)(698.57918446,1440.14400435)(698.15557366,1440.03983776)
\curveto(697.73196287,1439.92872674)(697.23543547,1439.87317122)(696.66599146,1439.87317122)
\curveto(695.61738114,1439.87317122)(694.72849292,1440.22039318)(693.99932681,1440.9148371)
\curveto(693.27016069,1441.60928102)(692.90557764,1442.49469702)(692.90557764,1443.57108509)
\curveto(692.90557764,1444.45302887)(693.09307749,1445.16483389)(693.46807721,1445.70650014)
\curveto(693.85002137,1446.25511084)(694.39168762,1446.68566607)(695.09307598,1446.99816583)
\curveto(695.80140878,1447.3106656)(696.65210258,1447.52247099)(697.64515738,1447.63358202)
\curveto(698.63821219,1447.74469305)(699.7041836,1447.82802632)(700.84307163,1447.88358183)
\lineto(700.84307163,1448.18566493)
\curveto(700.84307163,1448.63010904)(700.76321058,1448.99816432)(700.60348848,1449.28983077)
\curveto(700.45071082,1449.58149721)(700.22848876,1449.8106637)(699.93682232,1449.97733025)
\curveto(699.65904475,1450.13705235)(699.32571167,1450.24469115)(698.93682307,1450.30024667)
\curveto(698.54793448,1450.35580218)(698.14168479,1450.38357994)(697.718074,1450.38357994)
\curveto(697.2041855,1450.38357994)(696.63126926,1450.31413555)(695.9993253,1450.17524676)
\curveto(695.36738133,1450.04330242)(694.71460405,1449.84885812)(694.04099344,1449.59191387)
\lineto(693.93682686,1449.59191387)
\lineto(693.93682686,1451.5814957)
\curveto(694.31877101,1451.68566229)(694.87085393,1451.80024553)(695.5930756,1451.92524544)
\curveto(696.31529728,1452.05024534)(697.0271023,1452.1127453)(697.72849065,1452.1127453)
\curveto(698.54793448,1452.1127453)(699.2597395,1452.0433009)(699.86390571,1451.90441212)
\curveto(700.47501635,1451.77246778)(701.00279373,1451.54330128)(701.44723784,1451.21691264)
\curveto(701.88473751,1450.89746844)(702.21807059,1450.48427431)(702.44723708,1449.97733025)
\curveto(702.67640358,1449.47038618)(702.79098682,1448.84191444)(702.79098682,1448.09191501)
\closepath
\moveto(700.84307163,1443.06066881)
\lineto(700.84307163,1446.30024969)
\curveto(700.24584986,1446.2655275)(699.54098928,1446.2134442)(698.7284899,1446.14399981)
\curveto(697.92293495,1446.07455542)(697.28404655,1445.97386105)(696.81182468,1445.84191671)
\curveto(696.24932511,1445.68219461)(695.79446434,1445.4321948)(695.44724238,1445.09191727)
\curveto(695.10002042,1444.75858419)(694.92640944,1444.29677899)(694.92640944,1443.70650166)
\curveto(694.92640944,1443.03983549)(695.12779818,1442.53636365)(695.53057565,1442.19608613)
\curveto(695.93335312,1441.86275305)(696.54793599,1441.69608651)(697.37432426,1441.69608651)
\curveto(698.06182374,1441.69608651)(698.69029548,1441.82803086)(699.2597395,1442.09191954)
\curveto(699.82918351,1442.36275267)(700.35696089,1442.6856691)(700.84307163,1443.06066881)
\closepath
}
}
{
\newrgbcolor{curcolor}{0 0 0}
\pscustom[linestyle=none,fillstyle=solid,fillcolor=curcolor]
{
\newpath
\moveto(716.80138336,1446.10233318)
\curveto(716.80138336,1445.13011169)(716.66249458,1444.25511235)(716.38471701,1443.47733516)
\curveto(716.11388388,1442.69955797)(715.7458286,1442.04678069)(715.28055118,1441.51900331)
\curveto(714.787496,1440.97039262)(714.24582974,1440.55719848)(713.65555241,1440.27942092)
\curveto(713.06527508,1440.00858779)(712.41597001,1439.87317122)(711.70763721,1439.87317122)
\curveto(711.04791549,1439.87317122)(710.47152704,1439.95303227)(709.97847186,1440.11275438)
\curveto(709.48541667,1440.26553204)(708.99930593,1440.47386521)(708.52013963,1440.7377539)
\lineto(708.39513972,1440.19608765)
\lineto(706.56180777,1440.19608765)
\lineto(706.56180777,1456.40440872)
\lineto(708.52013963,1456.40440872)
\lineto(708.52013963,1450.61274643)
\curveto(709.06875032,1451.06413498)(709.65208321,1451.43219026)(710.2701383,1451.71691226)
\curveto(710.88819339,1452.00857871)(711.58263731,1452.15441193)(712.35347006,1452.15441193)
\curveto(713.72846902,1452.15441193)(714.81180153,1451.62663455)(715.6034676,1450.5710798)
\curveto(716.40207811,1449.51552504)(716.80138336,1448.02594283)(716.80138336,1446.10233318)
\closepath
\moveto(714.78055156,1446.05024988)
\curveto(714.78055156,1447.43913772)(714.55138506,1448.49122026)(714.09305208,1449.2064975)
\curveto(713.63471909,1449.92871917)(712.89513632,1450.28983001)(711.87430376,1450.28983001)
\curveto(711.30485974,1450.28983001)(710.72847129,1450.1648301)(710.1451384,1449.91483029)
\curveto(709.5618055,1449.67177492)(709.02013925,1449.35580294)(708.52013963,1448.96691434)
\lineto(708.52013963,1442.30025272)
\curveto(709.07569476,1442.05025291)(709.55138885,1441.87664193)(709.94722188,1441.77941978)
\curveto(710.34999935,1441.68219763)(710.80486012,1441.63358656)(711.31180418,1441.63358656)
\curveto(712.39513669,1441.63358656)(713.24235828,1441.98775296)(713.85346892,1442.69608575)
\curveto(714.47152401,1443.41136299)(714.78055156,1444.5294177)(714.78055156,1446.05024988)
\closepath
}
}
{
\newrgbcolor{curcolor}{0 0 0}
\pscustom[linestyle=none,fillstyle=solid,fillcolor=curcolor]
{
\newpath
\moveto(721.83264217,1440.19608765)
\lineto(719.87431032,1440.19608765)
\lineto(719.87431032,1456.40440872)
\lineto(721.83264217,1456.40440872)
\closepath
}
}
{
\newrgbcolor{curcolor}{0 0 0}
\pscustom[linestyle=none,fillstyle=solid,fillcolor=curcolor]
{
\newpath
\moveto(735.44720874,1445.81066673)
\lineto(726.87429856,1445.81066673)
\curveto(726.87429856,1445.09538949)(726.98193737,1444.47038997)(727.19721498,1443.93566815)
\curveto(727.4124926,1443.40789077)(727.70763126,1442.97386332)(728.08263098,1442.6335858)
\curveto(728.44374182,1442.30025272)(728.87082483,1442.05025291)(729.36388001,1441.88358637)
\curveto(729.86387963,1441.71691983)(730.41249033,1441.63358656)(731.0097121,1441.63358656)
\curveto(731.80137817,1441.63358656)(732.59651645,1441.78983644)(733.39512696,1442.1023362)
\curveto(734.20068191,1442.42178041)(734.77359814,1442.73428017)(735.11387566,1443.03983549)
\lineto(735.21804225,1443.03983549)
\lineto(735.21804225,1440.90442044)
\curveto(734.55832053,1440.62664288)(733.88470992,1440.39400416)(733.19721044,1440.2065043)
\curveto(732.50971096,1440.01900445)(731.78748929,1439.92525452)(731.03054542,1439.92525452)
\curveto(729.09999132,1439.92525452)(727.59304802,1440.44608746)(726.5097155,1441.48775334)
\curveto(725.42638299,1442.53636365)(724.88471673,1444.02247364)(724.88471673,1445.9460833)
\curveto(724.88471673,1447.84885963)(725.40207745,1449.35927516)(726.43679889,1450.47732987)
\curveto(727.47846477,1451.59538458)(728.84651929,1452.15441193)(730.54096245,1452.15441193)
\curveto(732.11040571,1452.15441193)(733.31873813,1451.69607895)(734.16595971,1450.77941297)
\curveto(735.02012573,1449.862747)(735.44720874,1448.56066465)(735.44720874,1446.87316593)
\closepath
\moveto(733.54096018,1447.3106656)
\curveto(733.53401575,1448.3384426)(733.27359928,1449.13358088)(732.75971078,1449.69608046)
\curveto(732.25276671,1450.25858003)(731.47846174,1450.53982982)(730.43679587,1450.53982982)
\curveto(729.38818555,1450.53982982)(728.55138063,1450.23080228)(727.9263811,1449.61274719)
\curveto(727.30832601,1448.9946921)(726.95763183,1448.22733157)(726.87429856,1447.3106656)
\closepath
}
}
{
\newrgbcolor{curcolor}{0 0 0}
\pscustom[linestyle=none,fillstyle=solid,fillcolor=curcolor]
{
\newpath
\moveto(747.85345256,1440.19608765)
\lineto(745.89512071,1440.19608765)
\lineto(745.89512071,1441.41483672)
\curveto(745.33262114,1440.92872598)(744.74581603,1440.55025404)(744.13470538,1440.27942092)
\curveto(743.52359473,1440.00858779)(742.86040078,1439.87317122)(742.14512355,1439.87317122)
\curveto(740.75623571,1439.87317122)(739.65206988,1440.40789304)(738.83262605,1441.47733668)
\curveto(738.02012667,1442.54678031)(737.61387698,1444.02941808)(737.61387698,1445.92524998)
\curveto(737.61387698,1446.91136034)(737.75276576,1447.7898319)(738.03054333,1448.56066465)
\curveto(738.31526533,1449.3314974)(738.69720949,1449.9877469)(739.17637579,1450.52941316)
\curveto(739.64859766,1451.05719054)(740.19720836,1451.45996801)(740.82220788,1451.73774558)
\curveto(741.45415185,1452.01552315)(742.10692913,1452.15441193)(742.78053973,1452.15441193)
\curveto(743.39165038,1452.15441193)(743.93331664,1452.08843976)(744.4055385,1451.95649541)
\curveto(744.87776037,1451.83149551)(745.37428777,1451.63357899)(745.89512071,1451.36274586)
\lineto(745.89512071,1456.40440872)
\lineto(747.85345256,1456.40440872)
\closepath
\moveto(745.89512071,1443.06066881)
\lineto(745.89512071,1449.73774709)
\curveto(745.36734333,1449.97385803)(744.89512147,1450.13705235)(744.47845512,1450.22733006)
\curveto(744.06178876,1450.31760777)(743.606928,1450.36274662)(743.11387282,1450.36274662)
\curveto(742.01665142,1450.36274662)(741.1624854,1449.98080247)(740.55137475,1449.21691415)
\curveto(739.94026411,1448.45302584)(739.63470878,1447.36969333)(739.63470878,1445.96691661)
\curveto(739.63470878,1444.58497321)(739.87081971,1443.53289068)(740.34304158,1442.810669)
\curveto(740.81526344,1442.09539176)(741.57220732,1441.73775315)(742.61387319,1441.73775315)
\curveto(743.16942833,1441.73775315)(743.7319279,1441.85928083)(744.30137192,1442.1023362)
\curveto(744.87081593,1442.35233601)(745.40206553,1442.67178022)(745.89512071,1443.06066881)
\closepath
}
}
{
\newrgbcolor{curcolor}{0.7019608 0.7019608 0.7019608}
\pscustom[linestyle=none,fillstyle=solid,fillcolor=curcolor,opacity=0.92623001]
{
\newpath
\moveto(355.90269133,999.63421174)
\lineto(766.61699489,999.63421174)
\lineto(766.61699489,863.20564305)
\lineto(355.90269133,863.20564305)
\closepath
}
}
{
\newrgbcolor{curcolor}{0 0 0}
\pscustom[linewidth=1.00157103,linecolor=curcolor]
{
\newpath
\moveto(355.90269133,999.63421174)
\lineto(766.61699489,999.63421174)
\lineto(766.61699489,863.20564305)
\lineto(355.90269133,863.20564305)
\closepath
}
}
{
\newrgbcolor{curcolor}{0 0 0}
\pscustom[linestyle=none,fillstyle=solid,fillcolor=curcolor]
{
\newpath
\moveto(508.87718489,967.66720662)
\curveto(508.18708292,967.48491554)(507.43187699,967.33517643)(506.6115671,967.2179893)
\curveto(505.804278,967.10080218)(505.08162405,967.04220861)(504.44360525,967.04220861)
\curveto(502.21704984,967.04220861)(500.5243469,967.64116504)(499.36549642,968.83907789)
\curveto(498.20664594,970.03699075)(497.62722071,971.95755755)(497.62722071,974.6007783)
\lineto(497.62722071,986.20230386)
\lineto(495.14675985,986.20230386)
\lineto(495.14675985,989.28823153)
\lineto(497.62722071,989.28823153)
\lineto(497.62722071,995.55774282)
\lineto(501.29908401,995.55774282)
\lineto(501.29908401,989.28823153)
\lineto(508.87718489,989.28823153)
\lineto(508.87718489,986.20230386)
\lineto(501.29908401,986.20230386)
\lineto(501.29908401,976.26092926)
\curveto(501.29908401,975.11509958)(501.3251256,974.21666494)(501.37720877,973.56562534)
\curveto(501.42929193,972.92760654)(501.61158302,972.32865012)(501.92408202,971.76875607)
\curveto(502.21053945,971.24792439)(502.6011632,970.86381103)(503.09595329,970.61641598)
\curveto(503.60376418,970.38204173)(504.3719909,970.2648546)(505.40063346,970.2648546)
\curveto(505.99958988,970.2648546)(506.62458789,970.34948975)(507.27562749,970.51876005)
\curveto(507.92666708,970.70105113)(508.39541559,970.85079024)(508.68187301,970.96797737)
\lineto(508.87718489,970.96797737)
\closepath
}
}
{
\newrgbcolor{curcolor}{0 0 0}
\pscustom[linestyle=none,fillstyle=solid,fillcolor=curcolor]
{
\newpath
\moveto(531.57242485,977.99920498)
\lineto(515.49825728,977.99920498)
\curveto(515.49825728,976.65806342)(515.70007955,975.48619215)(516.1037241,974.48359117)
\curveto(516.50736865,973.49401099)(517.0607523,972.6802115)(517.76387506,972.0421927)
\curveto(518.44095624,971.41719468)(519.24173494,970.94844618)(520.16621116,970.63594717)
\curveto(521.10370818,970.32344817)(522.13235074,970.16719866)(523.25213884,970.16719866)
\curveto(524.73650911,970.16719866)(526.22738978,970.46016648)(527.72478085,971.04610212)
\curveto(529.23519271,971.64505854)(530.30940804,972.23099418)(530.94742684,972.80390902)
\lineto(531.14273872,972.80390902)
\lineto(531.14273872,968.80001552)
\curveto(529.90576349,968.27918384)(528.64274668,967.84298732)(527.35368828,967.49142593)
\curveto(526.06462988,967.13986455)(524.71046753,966.96408386)(523.29120122,966.96408386)
\curveto(519.67142107,966.96408386)(516.84590924,967.94064325)(514.8146657,969.89376204)
\curveto(512.78342217,971.85990161)(511.7678004,974.64635107)(511.7678004,978.25311042)
\curveto(511.7678004,981.82080739)(512.7378494,984.65282963)(514.67794739,986.74917712)
\curveto(516.63106617,988.84552461)(519.19616217,989.89369836)(522.37323539,989.89369836)
\curveto(525.31593435,989.89369836)(527.58155214,989.03432609)(529.17008875,987.31558157)
\curveto(530.77164615,985.59683704)(531.57242485,983.15543856)(531.57242485,979.99138614)
\closepath
\moveto(527.99821748,980.81169602)
\curveto(527.98519669,982.73877322)(527.49691699,984.22965389)(526.53337839,985.28433803)
\curveto(525.58286059,986.33902217)(524.13104229,986.86636425)(522.17792351,986.86636425)
\curveto(520.21178394,986.86636425)(518.64277852,986.28693901)(517.47090725,985.12808853)
\curveto(516.31205677,983.96923805)(515.65450678,982.53044055)(515.49825728,980.81169602)
\closepath
}
}
{
\newrgbcolor{curcolor}{0 0 0}
\pscustom[linestyle=none,fillstyle=solid,fillcolor=curcolor]
{
\newpath
\moveto(552.72470022,973.76093722)
\curveto(552.72470022,971.76875607)(551.89787994,970.13464669)(550.24423937,968.85860908)
\curveto(548.60361959,967.58257148)(546.357533,966.94455268)(543.50597957,966.94455268)
\curveto(541.89140138,966.94455268)(540.40703111,967.13335416)(539.05286875,967.51095712)
\curveto(537.71172719,967.90158088)(536.58542869,968.32475661)(535.67397326,968.78048433)
\lineto(535.67397326,972.90156496)
\lineto(535.86928514,972.90156496)
\curveto(537.02813562,972.0291719)(538.31719401,971.33255954)(539.73646033,970.81172786)
\curveto(541.15572664,970.30391698)(542.51639939,970.05001154)(543.81847858,970.05001154)
\curveto(545.43305677,970.05001154)(546.69607358,970.31042738)(547.60752902,970.83125905)
\curveto(548.51898445,971.35209073)(548.97471216,972.17240061)(548.97471216,973.29218872)
\curveto(548.97471216,974.15156098)(548.72731712,974.80260057)(548.23252703,975.2453075)
\curveto(547.73773693,975.68801442)(546.78721913,976.06561738)(545.38097361,976.37811639)
\curveto(544.86014193,976.49530352)(544.17655036,976.63202183)(543.33019888,976.78827133)
\curveto(542.4968682,976.94452084)(541.73515188,977.11379113)(541.04504991,977.29608222)
\curveto(539.1309935,977.8038931)(537.77032075,978.54607824)(536.96303166,979.52263763)
\curveto(536.16876335,980.51221781)(535.7716292,981.72315145)(535.7716292,983.15543856)
\curveto(535.7716292,984.0538732)(535.95392029,984.90022467)(536.31850246,985.69449298)
\curveto(536.69610542,986.48876128)(537.26250987,987.19839444)(538.0177158,987.82339245)
\curveto(538.74688014,988.43536967)(539.67135637,988.91713897)(540.79114447,989.26870035)
\curveto(541.92395336,989.63328252)(543.18697017,989.81557361)(544.5801949,989.81557361)
\curveto(545.88227409,989.81557361)(547.19737407,989.65281371)(548.52549484,989.32729391)
\curveto(549.86663641,989.01479491)(550.97991411,988.63068155)(551.86532796,988.17495383)
\lineto(551.86532796,984.24918508)
\lineto(551.67001608,984.24918508)
\curveto(550.73251907,984.93928705)(549.59319978,985.51871229)(548.25205821,985.98746079)
\curveto(546.91091665,986.46923009)(545.59581667,986.71011474)(544.30675828,986.71011474)
\curveto(542.96561671,986.71011474)(541.83280782,986.44969891)(540.9083316,985.92886723)
\curveto(539.98385537,985.42105635)(539.52161726,984.65934002)(539.52161726,983.64371826)
\curveto(539.52161726,982.74528362)(539.80156429,982.06820244)(540.36145834,981.61247472)
\curveto(540.9083316,981.15674701)(541.79374544,980.78565444)(543.01769988,980.49919702)
\curveto(543.69478106,980.34294752)(544.44998699,980.18669801)(545.28331767,980.03044851)
\curveto(546.12966914,979.87419901)(546.8327919,979.7309703)(547.39268595,979.60076238)
\curveto(549.09840969,979.21013862)(550.41350967,978.53956784)(551.33798589,977.58905003)
\curveto(552.26246211,976.62551144)(552.72470022,975.34947383)(552.72470022,973.76093722)
\closepath
}
}
{
\newrgbcolor{curcolor}{0 0 0}
\pscustom[linestyle=none,fillstyle=solid,fillcolor=curcolor]
{
\newpath
\moveto(569.30668255,967.66720662)
\curveto(568.61658058,967.48491554)(567.86137465,967.33517643)(567.04106477,967.2179893)
\curveto(566.23377567,967.10080218)(565.51112172,967.04220861)(564.87310292,967.04220861)
\curveto(562.64654751,967.04220861)(560.95384456,967.64116504)(559.79499409,968.83907789)
\curveto(558.63614361,970.03699075)(558.05671837,971.95755755)(558.05671837,974.6007783)
\lineto(558.05671837,986.20230386)
\lineto(555.57625752,986.20230386)
\lineto(555.57625752,989.28823153)
\lineto(558.05671837,989.28823153)
\lineto(558.05671837,995.55774282)
\lineto(561.72858168,995.55774282)
\lineto(561.72858168,989.28823153)
\lineto(569.30668255,989.28823153)
\lineto(569.30668255,986.20230386)
\lineto(561.72858168,986.20230386)
\lineto(561.72858168,976.26092926)
\curveto(561.72858168,975.11509958)(561.75462326,974.21666494)(561.80670643,973.56562534)
\curveto(561.8587896,972.92760654)(562.04108069,972.32865012)(562.35357969,971.76875607)
\curveto(562.64003711,971.24792439)(563.03066087,970.86381103)(563.52545096,970.61641598)
\curveto(564.03326184,970.38204173)(564.80148856,970.2648546)(565.83013112,970.2648546)
\curveto(566.42908755,970.2648546)(567.05408556,970.34948975)(567.70512515,970.51876005)
\curveto(568.35616475,970.70105113)(568.82491325,970.85079024)(569.11137067,970.96797737)
\lineto(569.30668255,970.96797737)
\closepath
}
}
{
\newrgbcolor{curcolor}{0 0 0}
\pscustom[linestyle=none,fillstyle=solid,fillcolor=curcolor]
{
\newpath
\moveto(579.73633657,967.47189475)
\lineto(575.06838268,967.47189475)
\lineto(575.06838268,973.03828327)
\lineto(579.73633657,973.03828327)
\closepath
}
}
{
\newrgbcolor{curcolor}{0 0 0}
\pscustom[linestyle=none,fillstyle=solid,fillcolor=curcolor]
{
\newpath
\moveto(603.87688538,973.76093722)
\curveto(603.87688538,971.76875607)(603.05006509,970.13464669)(601.39642453,968.85860908)
\curveto(599.75580475,967.58257148)(597.50971815,966.94455268)(594.65816473,966.94455268)
\curveto(593.04358654,966.94455268)(591.55921626,967.13335416)(590.20505391,967.51095712)
\curveto(588.86391234,967.90158088)(587.73761385,968.32475661)(586.82615842,968.78048433)
\lineto(586.82615842,972.90156496)
\lineto(587.02147029,972.90156496)
\curveto(588.18032077,972.0291719)(589.46937917,971.33255954)(590.88864548,970.81172786)
\curveto(592.3079118,970.30391698)(593.66858455,970.05001154)(594.97066373,970.05001154)
\curveto(596.58524193,970.05001154)(597.84825874,970.31042738)(598.75971417,970.83125905)
\curveto(599.6711696,971.35209073)(600.12689732,972.17240061)(600.12689732,973.29218872)
\curveto(600.12689732,974.15156098)(599.87950227,974.80260057)(599.38471218,975.2453075)
\curveto(598.88992209,975.68801442)(597.93940428,976.06561738)(596.53315876,976.37811639)
\curveto(596.01232708,976.49530352)(595.32873551,976.63202183)(594.48238404,976.78827133)
\curveto(593.64905336,976.94452084)(592.88733703,977.11379113)(592.19723507,977.29608222)
\curveto(590.28317866,977.8038931)(588.92250591,978.54607824)(588.11521681,979.52263763)
\curveto(587.32094851,980.51221781)(586.92381436,981.72315145)(586.92381436,983.15543856)
\curveto(586.92381436,984.0538732)(587.10610544,984.90022467)(587.47068761,985.69449298)
\curveto(587.84829058,986.48876128)(588.41469503,987.19839444)(589.16990095,987.82339245)
\curveto(589.8990653,988.43536967)(590.82354152,988.91713897)(591.94332962,989.26870035)
\curveto(593.07613852,989.63328252)(594.33915533,989.81557361)(595.73238006,989.81557361)
\curveto(597.03445925,989.81557361)(598.34955923,989.65281371)(599.67768,989.32729391)
\curveto(601.01882156,989.01479491)(602.13209927,988.63068155)(603.01751311,988.17495383)
\lineto(603.01751311,984.24918508)
\lineto(602.82220124,984.24918508)
\curveto(601.88470422,984.93928705)(600.74538493,985.51871229)(599.40424337,985.98746079)
\curveto(598.06310181,986.46923009)(596.74800183,986.71011474)(595.45894343,986.71011474)
\curveto(594.11780187,986.71011474)(592.98499297,986.44969891)(592.06051675,985.92886723)
\curveto(591.13604053,985.42105635)(590.67380242,984.65934002)(590.67380242,983.64371826)
\curveto(590.67380242,982.74528362)(590.95374944,982.06820244)(591.51364349,981.61247472)
\curveto(592.06051675,981.15674701)(592.9459306,980.78565444)(594.16988503,980.49919702)
\curveto(594.84696621,980.34294752)(595.60217214,980.18669801)(596.43550282,980.03044851)
\curveto(597.28185429,979.87419901)(597.98497705,979.7309703)(598.5448711,979.60076238)
\curveto(600.25059484,979.21013862)(601.56569482,978.53956784)(602.49017104,977.58905003)
\curveto(603.41464727,976.62551144)(603.87688538,975.34947383)(603.87688538,973.76093722)
\closepath
}
}
{
\newrgbcolor{curcolor}{0 0 0}
\pscustom[linestyle=none,fillstyle=solid,fillcolor=curcolor]
{
\newpath
\moveto(627.37290098,967.47189475)
\lineto(623.70103768,967.47189475)
\lineto(623.70103768,979.8937302)
\curveto(623.70103768,980.89633117)(623.64244411,981.83382819)(623.52525699,982.70622124)
\curveto(623.40806986,983.59163509)(623.19322679,984.28173706)(622.88072779,984.77652715)
\curveto(622.55520799,985.32340041)(622.08645948,985.72704496)(621.47448226,985.98746079)
\curveto(620.86250505,986.26089742)(620.06823674,986.39761574)(619.09167735,986.39761574)
\curveto(618.08907638,986.39761574)(617.04090263,986.15022069)(615.94715611,985.6554306)
\curveto(614.8534096,985.16064051)(613.80523585,984.5291321)(612.80263488,983.76090538)
\lineto(612.80263488,967.47189475)
\lineto(609.13077157,967.47189475)
\lineto(609.13077157,997.86242299)
\lineto(612.80263488,997.86242299)
\lineto(612.80263488,986.86636425)
\curveto(613.94846456,987.81688205)(615.13335662,988.55906719)(616.35731106,989.09291966)
\curveto(617.58126549,989.62677212)(618.83777191,989.89369836)(620.12683031,989.89369836)
\curveto(622.48359364,989.89369836)(624.28046291,989.1840652)(625.51743814,987.76479889)
\curveto(626.75441337,986.34553257)(627.37290098,984.30126825)(627.37290098,981.63200591)
\closepath
}
}
{
\newrgbcolor{curcolor}{0 0 0}
\pscustom[linestyle=none,fillstyle=solid,fillcolor=curcolor]
{
\newpath
\moveto(386.97351281,953.81055374)
\lineto(380.47351773,953.81055374)
\lineto(380.47351773,955.69596898)
\lineto(386.97351281,955.69596898)
\closepath
}
}
{
\newrgbcolor{curcolor}{0 0 0}
\pscustom[linestyle=none,fillstyle=solid,fillcolor=curcolor]
{
\newpath
\moveto(408.36933019,947.96680816)
\lineto(398.15058792,947.96680816)
\lineto(398.15058792,963.4772131)
\lineto(408.36933019,963.4772131)
\lineto(408.36933019,961.64388115)
\lineto(400.21308636,961.64388115)
\lineto(400.21308636,957.39388436)
\lineto(408.36933019,957.39388436)
\lineto(408.36933019,955.56055242)
\lineto(400.21308636,955.56055242)
\lineto(400.21308636,949.80014011)
\lineto(408.36933019,949.80014011)
\closepath
}
}
{
\newrgbcolor{curcolor}{0 0 0}
\pscustom[linestyle=none,fillstyle=solid,fillcolor=curcolor]
{
\newpath
\moveto(413.58807584,961.55013122)
\lineto(411.37974417,961.55013122)
\lineto(411.37974417,963.58137968)
\lineto(413.58807584,963.58137968)
\closepath
\moveto(413.46307593,947.96680816)
\lineto(411.50474408,947.96680816)
\lineto(411.50474408,959.60221603)
\lineto(413.46307593,959.60221603)
\closepath
}
}
{
\newrgbcolor{curcolor}{0 0 0}
\pscustom[linestyle=none,fillstyle=solid,fillcolor=curcolor]
{
\newpath
\moveto(423.37973584,948.07097475)
\curveto(423.01168056,947.9737526)(422.60890309,947.89389155)(422.17140342,947.8313916)
\curveto(421.74084819,947.76889165)(421.35543182,947.73764167)(421.0151543,947.73764167)
\curveto(419.82765519,947.73764167)(418.9248781,948.05708587)(418.30682301,948.69597428)
\curveto(417.68876792,949.33486268)(417.37974038,950.35916746)(417.37974038,951.76888862)
\lineto(417.37974038,957.95638394)
\lineto(416.05682471,957.95638394)
\lineto(416.05682471,959.60221603)
\lineto(417.37974038,959.60221603)
\lineto(417.37974038,962.9459635)
\lineto(419.33807223,962.9459635)
\lineto(419.33807223,959.60221603)
\lineto(423.37973584,959.60221603)
\lineto(423.37973584,957.95638394)
\lineto(419.33807223,957.95638394)
\lineto(419.33807223,952.65430462)
\curveto(419.33807223,952.04319397)(419.35196111,951.56402766)(419.37973887,951.2168057)
\curveto(419.40751662,950.87652818)(419.50473877,950.55708398)(419.67140531,950.2584731)
\curveto(419.82418297,949.98069553)(420.03251615,949.77583457)(420.29640484,949.64389023)
\curveto(420.56723797,949.51889032)(420.97695988,949.45639037)(421.52557058,949.45639037)
\curveto(421.84501478,949.45639037)(422.17834786,949.50152922)(422.52556982,949.59180693)
\curveto(422.87279178,949.68902908)(423.12279159,949.76889013)(423.27556925,949.83139009)
\lineto(423.37973584,949.83139009)
\closepath
}
}
{
\newrgbcolor{curcolor}{0 0 0}
\pscustom[linestyle=none,fillstyle=solid,fillcolor=curcolor]
{
\newpath
\moveto(435.47347536,947.96680816)
\lineto(433.51514351,947.96680816)
\lineto(433.51514351,954.59180315)
\curveto(433.51514351,955.12652497)(433.48389353,955.62652459)(433.42139358,956.09180202)
\curveto(433.35889363,956.56402388)(433.24431038,956.93207916)(433.07764384,957.19596785)
\curveto(432.90403286,957.48763429)(432.65403305,957.70291191)(432.32764441,957.84180069)
\curveto(432.00125577,957.98763392)(431.57764497,958.06055053)(431.05681204,958.06055053)
\curveto(430.52209022,958.06055053)(429.96306286,957.92860618)(429.37972997,957.66471749)
\curveto(428.79639708,957.4008288)(428.23736972,957.0640235)(427.70264791,956.65430159)
\lineto(427.70264791,947.96680816)
\lineto(425.74431605,947.96680816)
\lineto(425.74431605,964.17512923)
\lineto(427.70264791,964.17512923)
\lineto(427.70264791,958.31055034)
\curveto(428.31375856,958.8174944)(428.94570252,959.21332743)(429.59847981,959.49804944)
\curveto(430.25125709,959.78277145)(430.92139547,959.92513245)(431.60889495,959.92513245)
\curveto(432.86583844,959.92513245)(433.82417105,959.54666051)(434.48389278,958.78971664)
\curveto(435.1436145,958.03277277)(435.47347536,956.94249582)(435.47347536,955.51888578)
\closepath
}
}
{
\newrgbcolor{curcolor}{0 0 0}
\pscustom[linestyle=none,fillstyle=solid,fillcolor=curcolor]
{
\newpath
\moveto(448.98387976,953.58138725)
\lineto(440.41096958,953.58138725)
\curveto(440.41096958,952.86611001)(440.51860839,952.24111048)(440.73388601,951.70638867)
\curveto(440.94916362,951.17861129)(441.24430229,950.74458384)(441.619302,950.40430632)
\curveto(441.98041284,950.07097324)(442.40749585,949.82097343)(442.90055103,949.65430689)
\curveto(443.40055066,949.48764035)(443.94916135,949.40430708)(444.54638312,949.40430708)
\curveto(445.33804919,949.40430708)(446.13318748,949.56055696)(446.93179798,949.87305672)
\curveto(447.73735293,950.19250092)(448.31026916,950.50500069)(448.65054668,950.81055601)
\lineto(448.75471327,950.81055601)
\lineto(448.75471327,948.67514096)
\curveto(448.09499155,948.39736339)(447.42138095,948.16472468)(446.73388147,947.97722482)
\curveto(446.04638199,947.78972496)(445.32416031,947.69597503)(444.56721644,947.69597503)
\curveto(442.63666234,947.69597503)(441.12971904,948.21680797)(440.04638653,949.25847385)
\curveto(438.96305401,950.30708417)(438.42138776,951.79319416)(438.42138776,953.71680381)
\curveto(438.42138776,955.61958015)(438.93874848,957.12999568)(439.97346991,958.24805038)
\curveto(441.01513579,959.36610509)(442.38319031,959.92513245)(444.07763348,959.92513245)
\curveto(445.64707673,959.92513245)(446.85540915,959.46679946)(447.70263073,958.55013349)
\curveto(448.55679675,957.63346752)(448.98387976,956.33138517)(448.98387976,954.64388644)
\closepath
\moveto(447.07763121,955.08138611)
\curveto(447.07068677,956.10916311)(446.8102703,956.9043014)(446.2963818,957.46680098)
\curveto(445.78943774,958.02930055)(445.01513277,958.31055034)(443.97346689,958.31055034)
\curveto(442.92485657,958.31055034)(442.08805165,958.00152279)(441.46305212,957.38346771)
\curveto(440.84499703,956.76541262)(440.49430285,955.99805209)(440.41096958,955.08138611)
\closepath
}
}
{
\newrgbcolor{curcolor}{0 0 0}
\pscustom[linestyle=none,fillstyle=solid,fillcolor=curcolor]
{
\newpath
\moveto(459.2130419,957.46680098)
\lineto(459.10887531,957.46680098)
\curveto(458.81720886,957.53624537)(458.53248686,957.58485644)(458.25470929,957.6126342)
\curveto(457.98387616,957.64735639)(457.66095974,957.66471749)(457.28596002,957.66471749)
\curveto(456.68179381,957.66471749)(456.09846092,957.52930093)(455.53596135,957.2584678)
\curveto(454.97346177,956.99457911)(454.43179552,956.65082937)(453.91096258,956.22721858)
\lineto(453.91096258,947.96680816)
\lineto(451.95263073,947.96680816)
\lineto(451.95263073,959.60221603)
\lineto(453.91096258,959.60221603)
\lineto(453.91096258,957.88346733)
\curveto(454.68873977,958.50846685)(455.37276703,958.94943874)(455.96304436,959.20638299)
\curveto(456.56026613,959.47027168)(457.16790456,959.60221603)(457.78595965,959.60221603)
\curveto(458.12623717,959.60221603)(458.37276476,959.59179937)(458.52554242,959.57096605)
\curveto(458.67832008,959.55707717)(458.90748657,959.5258272)(459.2130419,959.47721612)
\closepath
}
}
{
\newrgbcolor{curcolor}{0 0 0}
\pscustom[linestyle=none,fillstyle=solid,fillcolor=curcolor]
{
\newpath
\moveto(474.81719795,962.27929733)
\lineto(474.71303136,962.27929733)
\curveto(474.49775375,962.34179729)(474.21650396,962.40429724)(473.869282,962.46679719)
\curveto(473.52206004,962.53624158)(473.21650471,962.57096378)(472.95261603,962.57096378)
\curveto(472.11233888,962.57096378)(471.50122823,962.38346392)(471.11928408,962.00846421)
\curveto(470.74428436,961.64040893)(470.5567845,960.97027055)(470.5567845,959.99804906)
\lineto(470.5567845,959.60221603)
\lineto(474.08803183,959.60221603)
\lineto(474.08803183,957.95638394)
\lineto(470.61928446,957.95638394)
\lineto(470.61928446,947.96680816)
\lineto(468.66095261,947.96680816)
\lineto(468.66095261,957.95638394)
\lineto(467.33803694,957.95638394)
\lineto(467.33803694,959.60221603)
\lineto(468.66095261,959.60221603)
\lineto(468.66095261,959.9876324)
\curveto(468.66095261,961.3695758)(469.00470235,962.42860278)(469.69220183,963.16471333)
\curveto(470.37970131,963.90776833)(471.37275611,964.27929582)(472.67136624,964.27929582)
\curveto(473.10886591,964.27929582)(473.50122672,964.2584625)(473.84844868,964.21679587)
\curveto(474.20261508,964.17512923)(474.5255315,964.12651816)(474.81719795,964.07096265)
\closepath
}
}
{
\newrgbcolor{curcolor}{0 0 0}
\pscustom[linestyle=none,fillstyle=solid,fillcolor=curcolor]
{
\newpath
\moveto(478.16094388,961.55013122)
\lineto(475.95261222,961.55013122)
\lineto(475.95261222,963.58137968)
\lineto(478.16094388,963.58137968)
\closepath
\moveto(478.03594398,947.96680816)
\lineto(476.07761213,947.96680816)
\lineto(476.07761213,959.60221603)
\lineto(478.03594398,959.60221603)
\closepath
}
}
{
\newrgbcolor{curcolor}{0 0 0}
\pscustom[linestyle=none,fillstyle=solid,fillcolor=curcolor]
{
\newpath
\moveto(491.82760096,953.77930377)
\curveto(491.82760096,951.88347187)(491.34149021,950.38694522)(490.36926873,949.28972383)
\curveto(489.39704724,948.19250244)(488.09496489,947.64389174)(486.46302168,947.64389174)
\curveto(484.81718959,947.64389174)(483.5081628,948.19250244)(482.53594132,949.28972383)
\curveto(481.57066427,950.38694522)(481.08802575,951.88347187)(481.08802575,953.77930377)
\curveto(481.08802575,955.67513566)(481.57066427,957.17166231)(482.53594132,958.2688837)
\curveto(483.5081628,959.37304953)(484.81718959,959.92513245)(486.46302168,959.92513245)
\curveto(488.09496489,959.92513245)(489.39704724,959.37304953)(490.36926873,958.2688837)
\curveto(491.34149021,957.17166231)(491.82760096,955.67513566)(491.82760096,953.77930377)
\closepath
\moveto(489.80676915,953.77930377)
\curveto(489.80676915,955.28624707)(489.51163049,956.40430178)(488.92135315,957.13346789)
\curveto(488.33107582,957.86957845)(487.511632,958.23763373)(486.46302168,958.23763373)
\curveto(485.40052248,958.23763373)(484.57413422,957.86957845)(483.98385689,957.13346789)
\curveto(483.400524,956.40430178)(483.10885755,955.28624707)(483.10885755,953.77930377)
\curveto(483.10885755,952.32097154)(483.40399622,951.21333348)(483.99427355,950.45638961)
\curveto(484.58455088,949.70639018)(485.40746692,949.33139046)(486.46302168,949.33139046)
\curveto(487.50468756,949.33139046)(488.32065916,949.70291796)(488.9109365,950.44597295)
\curveto(489.50815827,951.19597239)(489.80676915,952.30708266)(489.80676915,953.77930377)
\closepath
}
}
{
\newrgbcolor{curcolor}{0 0 0}
\pscustom[linestyle=none,fillstyle=solid,fillcolor=curcolor]
{
\newpath
\moveto(497.84843344,947.96680816)
\lineto(495.35885199,947.96680816)
\lineto(495.35885199,950.93555592)
\lineto(497.84843344,950.93555592)
\closepath
}
}
{
\newrgbcolor{curcolor}{0 0 0}
\pscustom[linestyle=none,fillstyle=solid,fillcolor=curcolor]
{
\newpath
\moveto(510.72341195,951.32097229)
\curveto(510.72341195,950.2584731)(510.28244006,949.38694598)(509.40049629,948.70639094)
\curveto(508.52549695,948.0258359)(507.32758119,947.68555838)(505.80674901,947.68555838)
\curveto(504.94563855,947.68555838)(504.15397248,947.78625274)(503.4317508,947.98764148)
\curveto(502.71647357,948.19597466)(502.11577958,948.42166893)(501.62966883,948.6647243)
\lineto(501.62966883,950.86263931)
\lineto(501.73383542,950.86263931)
\curveto(502.35189051,950.39736188)(503.03938999,950.02583438)(503.79633386,949.74805682)
\curveto(504.55327773,949.47722369)(505.27897163,949.34180712)(505.97341555,949.34180712)
\curveto(506.83452601,949.34180712)(507.50813661,949.48069591)(507.99424735,949.75847347)
\curveto(508.48035809,950.03625104)(508.72341347,950.47375071)(508.72341347,951.07097248)
\curveto(508.72341347,951.52930547)(508.59146912,951.87652743)(508.32758043,952.11263836)
\curveto(508.06369174,952.34874929)(507.55674768,952.55013803)(506.80674825,952.71680457)
\curveto(506.52897068,952.77930452)(506.16438762,952.85222113)(505.71299908,952.9355544)
\curveto(505.26855497,953.01888767)(504.86230528,953.10916538)(504.49425,953.20638753)
\curveto(503.47341744,953.47722066)(502.74772354,953.87305369)(502.31716831,954.39388663)
\curveto(501.89355752,954.92166401)(501.68175213,955.56749686)(501.68175213,956.33138517)
\curveto(501.68175213,956.81055147)(501.77897428,957.26194002)(501.97341857,957.68555081)
\curveto(502.17480731,958.1091616)(502.47689041,958.48763354)(502.87966789,958.82096662)
\curveto(503.26855648,959.14735526)(503.76161166,959.40429951)(504.35883343,959.59179937)
\curveto(504.96299964,959.78624367)(505.63661025,959.88346581)(506.37966524,959.88346581)
\curveto(507.07410916,959.88346581)(507.77549752,959.79666032)(508.48383031,959.62304934)
\curveto(509.19910755,959.4563828)(509.7928571,959.25152185)(510.26507897,959.00846648)
\lineto(510.26507897,956.91471806)
\lineto(510.16091238,956.91471806)
\curveto(509.66091276,957.28277334)(509.05327433,957.59180088)(508.33799709,957.84180069)
\curveto(507.62271985,958.09874494)(506.9213315,958.22721707)(506.23383202,958.22721707)
\curveto(505.51855478,958.22721707)(504.91438857,958.08832828)(504.42133339,957.81055072)
\curveto(503.92827821,957.53971759)(503.68175061,957.13346789)(503.68175061,956.59180164)
\curveto(503.68175061,956.11263533)(503.83105606,955.7515245)(504.12966694,955.50846912)
\curveto(504.42133339,955.26541375)(504.89355525,955.06749724)(505.54633254,954.91471957)
\curveto(505.90744337,954.8313863)(506.31022085,954.74805303)(506.75466496,954.66471976)
\curveto(507.2060535,954.58138649)(507.58105322,954.50499766)(507.8796641,954.43555327)
\curveto(508.78938564,954.22722009)(509.490774,953.86958148)(509.98382918,953.36263741)
\curveto(510.47688436,952.84874891)(510.72341195,952.16819387)(510.72341195,951.32097229)
\closepath
}
}
{
\newrgbcolor{curcolor}{0 0 0}
\pscustom[linestyle=none,fillstyle=solid,fillcolor=curcolor]
{
\newpath
\moveto(523.25465159,947.96680816)
\lineto(521.29631974,947.96680816)
\lineto(521.29631974,954.59180315)
\curveto(521.29631974,955.12652497)(521.26506976,955.62652459)(521.20256981,956.09180202)
\curveto(521.14006986,956.56402388)(521.02548661,956.93207916)(520.85882007,957.19596785)
\curveto(520.68520909,957.48763429)(520.43520928,957.70291191)(520.10882064,957.84180069)
\curveto(519.78243199,957.98763392)(519.3588212,958.06055053)(518.83798826,958.06055053)
\curveto(518.30326645,958.06055053)(517.74423909,957.92860618)(517.1609062,957.66471749)
\curveto(516.57757331,957.4008288)(516.01854595,957.0640235)(515.48382414,956.65430159)
\lineto(515.48382414,947.96680816)
\lineto(513.52549228,947.96680816)
\lineto(513.52549228,964.17512923)
\lineto(515.48382414,964.17512923)
\lineto(515.48382414,958.31055034)
\curveto(516.09493478,958.8174944)(516.72687875,959.21332743)(517.37965603,959.49804944)
\curveto(518.03243332,959.78277145)(518.7025717,959.92513245)(519.39007118,959.92513245)
\curveto(520.64701467,959.92513245)(521.60534728,959.54666051)(522.26506901,958.78971664)
\curveto(522.92479073,958.03277277)(523.25465159,956.94249582)(523.25465159,955.51888578)
\closepath
}
}
{
\newrgbcolor{curcolor}{0 0 0}
\pscustom[linestyle=none,fillstyle=solid,fillcolor=curcolor]
{
\newpath
\moveto(544.4421428,953.77930377)
\curveto(544.4421428,951.88347187)(543.95603205,950.38694522)(542.98381057,949.28972383)
\curveto(542.01158908,948.19250244)(540.70950673,947.64389174)(539.07756352,947.64389174)
\curveto(537.43173143,947.64389174)(536.12270465,948.19250244)(535.15048316,949.28972383)
\curveto(534.18520611,950.38694522)(533.70256759,951.88347187)(533.70256759,953.77930377)
\curveto(533.70256759,955.67513566)(534.18520611,957.17166231)(535.15048316,958.2688837)
\curveto(536.12270465,959.37304953)(537.43173143,959.92513245)(539.07756352,959.92513245)
\curveto(540.70950673,959.92513245)(542.01158908,959.37304953)(542.98381057,958.2688837)
\curveto(543.95603205,957.17166231)(544.4421428,955.67513566)(544.4421428,953.77930377)
\closepath
\moveto(542.42131099,953.77930377)
\curveto(542.42131099,955.28624707)(542.12617233,956.40430178)(541.535895,957.13346789)
\curveto(540.94561766,957.86957845)(540.12617384,958.23763373)(539.07756352,958.23763373)
\curveto(538.01506433,958.23763373)(537.18867606,957.86957845)(536.59839873,957.13346789)
\curveto(536.01506584,956.40430178)(535.72339939,955.28624707)(535.72339939,953.77930377)
\curveto(535.72339939,952.32097154)(536.01853806,951.21333348)(536.60881539,950.45638961)
\curveto(537.19909272,949.70639018)(538.02200877,949.33139046)(539.07756352,949.33139046)
\curveto(540.1192294,949.33139046)(540.93520101,949.70291796)(541.52547834,950.44597295)
\curveto(542.12270011,951.19597239)(542.42131099,952.30708266)(542.42131099,953.77930377)
\closepath
}
}
{
\newrgbcolor{curcolor}{0 0 0}
\pscustom[linestyle=none,fillstyle=solid,fillcolor=curcolor]
{
\newpath
\moveto(554.73380154,957.46680098)
\lineto(554.62963495,957.46680098)
\curveto(554.33796851,957.53624537)(554.0532465,957.58485644)(553.77546893,957.6126342)
\curveto(553.5046358,957.64735639)(553.18171938,957.66471749)(552.80671967,957.66471749)
\curveto(552.20255346,957.66471749)(551.61922056,957.52930093)(551.05672099,957.2584678)
\curveto(550.49422142,956.99457911)(549.95255516,956.65082937)(549.43172222,956.22721858)
\lineto(549.43172222,947.96680816)
\lineto(547.47339037,947.96680816)
\lineto(547.47339037,959.60221603)
\lineto(549.43172222,959.60221603)
\lineto(549.43172222,957.88346733)
\curveto(550.20949941,958.50846685)(550.89352667,958.94943874)(551.483804,959.20638299)
\curveto(552.08102577,959.47027168)(552.6886642,959.60221603)(553.30671929,959.60221603)
\curveto(553.64699681,959.60221603)(553.8935244,959.59179937)(554.04630206,959.57096605)
\curveto(554.19907972,959.55707717)(554.42824622,959.5258272)(554.73380154,959.47721612)
\closepath
}
}
{
\newrgbcolor{curcolor}{0 0 0}
\pscustom[linestyle=none,fillstyle=solid,fillcolor=curcolor]
{
\newpath
\moveto(572.39003216,951.32097229)
\curveto(572.39003216,950.2584731)(571.94906027,949.38694598)(571.0671165,948.70639094)
\curveto(570.19211716,948.0258359)(568.9942014,947.68555838)(567.47336922,947.68555838)
\curveto(566.61225876,947.68555838)(565.82059269,947.78625274)(565.09837101,947.98764148)
\curveto(564.38309378,948.19597466)(563.78239979,948.42166893)(563.29628904,948.6647243)
\lineto(563.29628904,950.86263931)
\lineto(563.40045563,950.86263931)
\curveto(564.01851072,950.39736188)(564.7060102,950.02583438)(565.46295407,949.74805682)
\curveto(566.21989794,949.47722369)(566.94559184,949.34180712)(567.64003576,949.34180712)
\curveto(568.50114622,949.34180712)(569.17475682,949.48069591)(569.66086756,949.75847347)
\curveto(570.1469783,950.03625104)(570.39003368,950.47375071)(570.39003368,951.07097248)
\curveto(570.39003368,951.52930547)(570.25808933,951.87652743)(569.99420064,952.11263836)
\curveto(569.73031195,952.34874929)(569.22336789,952.55013803)(568.47336846,952.71680457)
\curveto(568.19559089,952.77930452)(567.83100783,952.85222113)(567.37961929,952.9355544)
\curveto(566.93517518,953.01888767)(566.52892549,953.10916538)(566.16087021,953.20638753)
\curveto(565.14003765,953.47722066)(564.41434375,953.87305369)(563.98378852,954.39388663)
\curveto(563.56017773,954.92166401)(563.34837234,955.56749686)(563.34837234,956.33138517)
\curveto(563.34837234,956.81055147)(563.44559449,957.26194002)(563.64003878,957.68555081)
\curveto(563.84142752,958.1091616)(564.14351062,958.48763354)(564.5462881,958.82096662)
\curveto(564.93517669,959.14735526)(565.42823187,959.40429951)(566.02545364,959.59179937)
\curveto(566.62961985,959.78624367)(567.30323046,959.88346581)(568.04628545,959.88346581)
\curveto(568.74072937,959.88346581)(569.44211773,959.79666032)(570.15045052,959.62304934)
\curveto(570.86572776,959.4563828)(571.45947731,959.25152185)(571.93169918,959.00846648)
\lineto(571.93169918,956.91471806)
\lineto(571.82753259,956.91471806)
\curveto(571.32753297,957.28277334)(570.71989454,957.59180088)(570.0046173,957.84180069)
\curveto(569.28934006,958.09874494)(568.58795171,958.22721707)(567.90045223,958.22721707)
\curveto(567.18517499,958.22721707)(566.58100878,958.08832828)(566.0879536,957.81055072)
\curveto(565.59489841,957.53971759)(565.34837082,957.13346789)(565.34837082,956.59180164)
\curveto(565.34837082,956.11263533)(565.49767627,955.7515245)(565.79628715,955.50846912)
\curveto(566.0879536,955.26541375)(566.56017546,955.06749724)(567.21295275,954.91471957)
\curveto(567.57406358,954.8313863)(567.97684106,954.74805303)(568.42128517,954.66471976)
\curveto(568.87267371,954.58138649)(569.24767343,954.50499766)(569.54628431,954.43555327)
\curveto(570.45600585,954.22722009)(571.15739421,953.86958148)(571.65044939,953.36263741)
\curveto(572.14350457,952.84874891)(572.39003216,952.16819387)(572.39003216,951.32097229)
\closepath
}
}
{
\newrgbcolor{curcolor}{0 0 0}
\pscustom[linestyle=none,fillstyle=solid,fillcolor=curcolor]
{
\newpath
\moveto(577.29627757,961.55013122)
\lineto(575.08794591,961.55013122)
\lineto(575.08794591,963.58137968)
\lineto(577.29627757,963.58137968)
\closepath
\moveto(577.17127766,947.96680816)
\lineto(575.21294581,947.96680816)
\lineto(575.21294581,959.60221603)
\lineto(577.17127766,959.60221603)
\closepath
}
}
{
\newrgbcolor{curcolor}{0 0 0}
\pscustom[linestyle=none,fillstyle=solid,fillcolor=curcolor]
{
\newpath
\moveto(598.02543651,947.96680816)
\lineto(596.06710465,947.96680816)
\lineto(596.06710465,954.59180315)
\curveto(596.06710465,955.09180277)(596.04279912,955.5744413)(595.99418804,956.03971872)
\curveto(595.95252141,956.50499615)(595.85877148,956.87652364)(595.71293825,957.15430121)
\curveto(595.55321615,957.4529121)(595.32404966,957.67860637)(595.02543877,957.83138403)
\curveto(594.72682789,957.9841617)(594.29627266,958.06055053)(593.73377309,958.06055053)
\curveto(593.18516239,958.06055053)(592.63655169,957.92166174)(592.087941,957.64388418)
\curveto(591.5393303,957.37305105)(590.99071961,957.02582909)(590.44210891,956.6022183)
\curveto(590.46294223,956.4424962)(590.48030332,956.25499634)(590.4941922,956.03971872)
\curveto(590.50808108,955.83138555)(590.51502552,955.62305237)(590.51502552,955.4147192)
\lineto(590.51502552,947.96680816)
\lineto(588.55669367,947.96680816)
\lineto(588.55669367,954.59180315)
\curveto(588.55669367,955.10569165)(588.53238813,955.59180239)(588.48377706,956.05013538)
\curveto(588.44211042,956.51541281)(588.34836049,956.8869403)(588.20252727,957.16471787)
\curveto(588.04280517,957.46332876)(587.81363868,957.68555081)(587.51502779,957.83138403)
\curveto(587.21641691,957.9841617)(586.78586168,958.06055053)(586.2233621,958.06055053)
\curveto(585.68864028,958.06055053)(585.15044625,957.92860618)(584.60877999,957.66471749)
\curveto(584.07405817,957.4008288)(583.53933635,957.0640235)(583.00461454,956.65430159)
\lineto(583.00461454,947.96680816)
\lineto(581.04628268,947.96680816)
\lineto(581.04628268,959.60221603)
\lineto(583.00461454,959.60221603)
\lineto(583.00461454,958.31055034)
\curveto(583.61572518,958.8174944)(584.22336361,959.21332743)(584.82752982,959.49804944)
\curveto(585.43864047,959.78277145)(586.08794554,959.92513245)(586.77544502,959.92513245)
\curveto(587.56711108,959.92513245)(588.23724947,959.75846591)(588.78586016,959.42513283)
\curveto(589.3414153,959.09179975)(589.75460943,958.62999454)(590.02544256,958.03971721)
\curveto(590.81710863,958.70638337)(591.5393303,959.18554968)(592.19210759,959.47721612)
\curveto(592.84488487,959.77582701)(593.54280101,959.92513245)(594.285856,959.92513245)
\curveto(595.56363281,959.92513245)(596.50460432,959.53624385)(597.10877053,958.75846667)
\curveto(597.71988118,957.98763392)(598.02543651,956.90777362)(598.02543651,955.51888578)
\closepath
}
}
{
\newrgbcolor{curcolor}{0 0 0}
\pscustom[linestyle=none,fillstyle=solid,fillcolor=curcolor]
{
\newpath
\moveto(612.03584578,953.92513699)
\curveto(612.03584578,952.98069326)(611.90042921,952.11611058)(611.62959608,951.33138895)
\curveto(611.35876295,950.55361176)(610.9768188,949.89389004)(610.48376362,949.35222378)
\curveto(610.02543063,948.83833528)(609.48376437,948.43903003)(608.85876485,948.15430802)
\curveto(608.24070976,947.87653045)(607.58446025,947.73764167)(606.89001634,947.73764167)
\curveto(606.28585013,947.73764167)(605.73723943,947.80361384)(605.24418425,947.93555819)
\curveto(604.7580735,948.06750253)(604.2615461,948.27236349)(603.75460204,948.55014106)
\lineto(603.75460204,943.67514474)
\lineto(601.79627019,943.67514474)
\lineto(601.79627019,959.60221603)
\lineto(603.75460204,959.60221603)
\lineto(603.75460204,958.38346695)
\curveto(604.27543498,958.82096662)(604.85876787,959.18554968)(605.50460072,959.47721612)
\curveto(606.157378,959.77582701)(606.85182192,959.92513245)(607.58793247,959.92513245)
\curveto(608.99070919,959.92513245)(610.08098614,959.39388285)(610.85876333,958.33138366)
\curveto(611.64348496,957.2758289)(612.03584578,955.80708001)(612.03584578,953.92513699)
\closepath
\moveto(610.01501397,953.87305369)
\curveto(610.01501397,955.27583041)(609.77543082,956.32444073)(609.29626451,957.01888465)
\curveto(608.81709821,957.71332857)(608.08098766,958.06055053)(607.08793285,958.06055053)
\curveto(606.52543328,958.06055053)(605.95946148,957.93902284)(605.39001747,957.69596747)
\curveto(604.82057346,957.4529121)(604.27543498,957.13346789)(603.75460204,956.73763486)
\lineto(603.75460204,950.14388985)
\curveto(604.31015718,949.89389004)(604.78585126,949.72375128)(605.18168429,949.63347357)
\curveto(605.58446177,949.54319586)(606.03932253,949.498057)(606.5462666,949.498057)
\curveto(607.63654355,949.498057)(608.48723735,949.86611228)(609.098348,950.60222284)
\curveto(609.70945865,951.33833339)(610.01501397,952.42861034)(610.01501397,953.87305369)
\closepath
}
}
{
\newrgbcolor{curcolor}{0 0 0}
\pscustom[linestyle=none,fillstyle=solid,fillcolor=curcolor]
{
\newpath
\moveto(617.06709017,947.96680816)
\lineto(615.10875832,947.96680816)
\lineto(615.10875832,964.17512923)
\lineto(617.06709017,964.17512923)
\closepath
}
}
{
\newrgbcolor{curcolor}{0 0 0}
\pscustom[linestyle=none,fillstyle=solid,fillcolor=curcolor]
{
\newpath
\moveto(630.68167116,953.58138725)
\lineto(622.10876098,953.58138725)
\curveto(622.10876098,952.86611001)(622.21639978,952.24111048)(622.4316774,951.70638867)
\curveto(622.64695501,951.17861129)(622.94209368,950.74458384)(623.3170934,950.40430632)
\curveto(623.67820423,950.07097324)(624.10528724,949.82097343)(624.59834243,949.65430689)
\curveto(625.09834205,949.48764035)(625.64695274,949.40430708)(626.24417451,949.40430708)
\curveto(627.03584058,949.40430708)(627.83097887,949.56055696)(628.62958938,949.87305672)
\curveto(629.43514432,950.19250092)(630.00806056,950.50500069)(630.34833808,950.81055601)
\lineto(630.45250466,950.81055601)
\lineto(630.45250466,948.67514096)
\curveto(629.79278294,948.39736339)(629.11917234,948.16472468)(628.43167286,947.97722482)
\curveto(627.74417338,947.78972496)(627.0219517,947.69597503)(626.26500783,947.69597503)
\curveto(624.33445374,947.69597503)(622.82751043,948.21680797)(621.74417792,949.25847385)
\curveto(620.6608454,950.30708417)(620.11917915,951.79319416)(620.11917915,953.71680381)
\curveto(620.11917915,955.61958015)(620.63653987,957.12999568)(621.67126131,958.24805038)
\curveto(622.71292719,959.36610509)(624.08098171,959.92513245)(625.77542487,959.92513245)
\curveto(627.34486813,959.92513245)(628.55320054,959.46679946)(629.40042213,958.55013349)
\curveto(630.25458815,957.63346752)(630.68167116,956.33138517)(630.68167116,954.64388644)
\closepath
\moveto(628.7754226,955.08138611)
\curveto(628.76847816,956.10916311)(628.50806169,956.9043014)(627.99417319,957.46680098)
\curveto(627.48722913,958.02930055)(626.71292416,958.31055034)(625.67125828,958.31055034)
\curveto(624.62264796,958.31055034)(623.78584304,958.00152279)(623.16084351,957.38346771)
\curveto(622.54278843,956.76541262)(622.19209425,955.99805209)(622.10876098,955.08138611)
\closepath
}
}
{
\newrgbcolor{curcolor}{0 0 0}
\pscustom[linestyle=none,fillstyle=solid,fillcolor=curcolor]
{
\newpath
\moveto(640.91080446,957.46680098)
\lineto(640.80663787,957.46680098)
\curveto(640.51497142,957.53624537)(640.23024942,957.58485644)(639.95247185,957.6126342)
\curveto(639.68163872,957.64735639)(639.3587223,957.66471749)(638.98372258,957.66471749)
\curveto(638.37955637,957.66471749)(637.79622348,957.52930093)(637.2337239,957.2584678)
\curveto(636.67122433,956.99457911)(636.12955807,956.65082937)(635.60872513,956.22721858)
\lineto(635.60872513,947.96680816)
\lineto(633.65039328,947.96680816)
\lineto(633.65039328,959.60221603)
\lineto(635.60872513,959.60221603)
\lineto(635.60872513,957.88346733)
\curveto(636.38650232,958.50846685)(637.07052958,958.94943874)(637.66080691,959.20638299)
\curveto(638.25802868,959.47027168)(638.86566711,959.60221603)(639.4837222,959.60221603)
\curveto(639.82399972,959.60221603)(640.07052731,959.59179937)(640.22330498,959.57096605)
\curveto(640.37608264,959.55707717)(640.60524913,959.5258272)(640.91080446,959.47721612)
\closepath
}
}
{
\newrgbcolor{curcolor}{0 0 0}
\pscustom[linestyle=none,fillstyle=solid,fillcolor=curcolor]
{
\newpath
\moveto(652.49414617,953.58138725)
\lineto(643.92123599,953.58138725)
\curveto(643.92123599,952.86611001)(644.02887479,952.24111048)(644.24415241,951.70638867)
\curveto(644.45943002,951.17861129)(644.75456869,950.74458384)(645.1295684,950.40430632)
\curveto(645.49067924,950.07097324)(645.91776225,949.82097343)(646.41081744,949.65430689)
\curveto(646.91081706,949.48764035)(647.45942775,949.40430708)(648.05664952,949.40430708)
\curveto(648.84831559,949.40430708)(649.64345388,949.56055696)(650.44206439,949.87305672)
\curveto(651.24761933,950.19250092)(651.82053556,950.50500069)(652.16081308,950.81055601)
\lineto(652.26497967,950.81055601)
\lineto(652.26497967,948.67514096)
\curveto(651.60525795,948.39736339)(650.93164735,948.16472468)(650.24414787,947.97722482)
\curveto(649.55664839,947.78972496)(648.83442671,947.69597503)(648.07748284,947.69597503)
\curveto(646.14692875,947.69597503)(644.63998544,948.21680797)(643.55665293,949.25847385)
\curveto(642.47332041,950.30708417)(641.93165416,951.79319416)(641.93165416,953.71680381)
\curveto(641.93165416,955.61958015)(642.44901488,957.12999568)(643.48373632,958.24805038)
\curveto(644.52540219,959.36610509)(645.89345672,959.92513245)(647.58789988,959.92513245)
\curveto(649.15734313,959.92513245)(650.36567555,959.46679946)(651.21289714,958.55013349)
\curveto(652.06706316,957.63346752)(652.49414617,956.33138517)(652.49414617,954.64388644)
\closepath
\moveto(650.58789761,955.08138611)
\curveto(650.58095317,956.10916311)(650.3205367,956.9043014)(649.8066482,957.46680098)
\curveto(649.29970414,958.02930055)(648.52539917,958.31055034)(647.48373329,958.31055034)
\curveto(646.43512297,958.31055034)(645.59831805,958.00152279)(644.97331852,957.38346771)
\curveto(644.35526343,956.76541262)(644.00456926,955.99805209)(643.92123599,955.08138611)
\closepath
}
}
{
\newrgbcolor{curcolor}{0 0 0}
\pscustom[linestyle=none,fillstyle=solid,fillcolor=curcolor]
{
\newpath
\moveto(664.50455695,947.96680816)
\lineto(662.55664176,947.96680816)
\lineto(662.55664176,949.20639056)
\curveto(662.38303078,949.08833509)(662.14691985,948.92166855)(661.84830896,948.70639094)
\curveto(661.55664252,948.49805776)(661.27192051,948.33139122)(660.99414294,948.20639132)
\curveto(660.6677543,948.04666921)(660.29275458,947.91472487)(659.86914379,947.81055828)
\curveto(659.445533,947.69944725)(658.9490056,947.64389174)(658.37956159,947.64389174)
\curveto(657.33095127,947.64389174)(656.44206305,947.9911137)(655.71289694,948.68555762)
\curveto(654.98373082,949.38000154)(654.61914777,950.26541754)(654.61914777,951.34180561)
\curveto(654.61914777,952.22374939)(654.80664762,952.9355544)(655.18164734,953.47722066)
\curveto(655.56359149,954.02583136)(656.10525775,954.45638659)(656.80664611,954.76888635)
\curveto(657.51497891,955.08138611)(658.36567271,955.29319151)(659.35872751,955.40430254)
\curveto(660.35178232,955.51541356)(661.41775373,955.59874683)(662.55664176,955.65430235)
\lineto(662.55664176,955.95638545)
\curveto(662.55664176,956.40082956)(662.47678071,956.76888484)(662.31705861,957.06055128)
\curveto(662.16428095,957.35221773)(661.94205889,957.58138422)(661.65039245,957.74805076)
\curveto(661.37261488,957.90777286)(661.0392818,958.01541167)(660.6503932,958.07096719)
\curveto(660.26150461,958.1265227)(659.85525491,958.15430046)(659.43164412,958.15430046)
\curveto(658.91775562,958.15430046)(658.34483939,958.08485606)(657.71289542,957.94596728)
\curveto(657.08095146,957.81402294)(656.42817417,957.61957864)(655.75456357,957.36263439)
\lineto(655.65039698,957.36263439)
\lineto(655.65039698,959.35221622)
\curveto(656.03234114,959.4563828)(656.58442406,959.57096605)(657.30664573,959.69596596)
\curveto(658.02886741,959.82096586)(658.74067242,959.88346581)(659.44206078,959.88346581)
\curveto(660.26150461,959.88346581)(660.97330962,959.81402142)(661.57747583,959.67513264)
\curveto(662.18858648,959.54318829)(662.71636386,959.3140218)(663.16080797,958.98763316)
\curveto(663.59830764,958.66818896)(663.93164072,958.25499482)(664.16080721,957.74805076)
\curveto(664.38997371,957.2411067)(664.50455695,956.61263496)(664.50455695,955.86263552)
\closepath
\moveto(662.55664176,950.83138933)
\lineto(662.55664176,954.07097021)
\curveto(661.95941999,954.03624802)(661.25455941,953.98416472)(660.44206003,953.91472033)
\curveto(659.63650508,953.84527594)(658.99761667,953.74458157)(658.52539481,953.61263723)
\curveto(657.96289524,953.45291512)(657.50803447,953.20291531)(657.16081251,952.86263779)
\curveto(656.81359055,952.52930471)(656.63997957,952.06749951)(656.63997957,951.47722217)
\curveto(656.63997957,950.81055601)(656.84136831,950.30708417)(657.24414578,949.96680665)
\curveto(657.64692325,949.63347357)(658.26150612,949.46680703)(659.08789438,949.46680703)
\curveto(659.77539386,949.46680703)(660.40386561,949.59875137)(660.97330962,949.86264006)
\curveto(661.54275364,950.13347319)(662.07053102,950.45638961)(662.55664176,950.83138933)
\closepath
}
}
{
\newrgbcolor{curcolor}{0 0 0}
\pscustom[linestyle=none,fillstyle=solid,fillcolor=curcolor]
{
\newpath
\moveto(677.71287076,947.96680816)
\lineto(675.75453891,947.96680816)
\lineto(675.75453891,949.18555724)
\curveto(675.19203934,948.6994465)(674.60523423,948.32097456)(673.99412358,948.05014143)
\curveto(673.38301293,947.7793083)(672.71981898,947.64389174)(672.00454175,947.64389174)
\curveto(670.61565391,947.64389174)(669.51148808,948.17861356)(668.69204425,949.24805719)
\curveto(667.87954487,950.31750083)(667.47329518,951.8001386)(667.47329518,953.6959705)
\curveto(667.47329518,954.68208086)(667.61218396,955.56055242)(667.88996153,956.33138517)
\curveto(668.17468353,957.10221792)(668.55662769,957.75846742)(669.03579399,958.30013368)
\curveto(669.50801586,958.82791106)(670.05662656,959.23068853)(670.68162608,959.5084661)
\curveto(671.31357005,959.78624367)(671.96634733,959.92513245)(672.63995793,959.92513245)
\curveto(673.25106858,959.92513245)(673.79273484,959.85916028)(674.2649567,959.72721593)
\curveto(674.73717857,959.60221603)(675.23370597,959.40429951)(675.75453891,959.13346638)
\lineto(675.75453891,964.17512923)
\lineto(677.71287076,964.17512923)
\closepath
\moveto(675.75453891,950.83138933)
\lineto(675.75453891,957.50846761)
\curveto(675.22676153,957.74457854)(674.75453967,957.90777286)(674.33787332,957.99805057)
\curveto(673.92120696,958.08832828)(673.4663462,958.13346714)(672.97329102,958.13346714)
\curveto(671.87606962,958.13346714)(671.0219036,957.75152298)(670.41079295,956.98763467)
\curveto(669.79968231,956.22374636)(669.49412698,955.14041385)(669.49412698,953.73763713)
\curveto(669.49412698,952.35569373)(669.73023791,951.30361119)(670.20245978,950.58138952)
\curveto(670.67468164,949.86611228)(671.43162552,949.50847366)(672.47329139,949.50847366)
\curveto(673.02884653,949.50847366)(673.5913461,949.63000135)(674.16079012,949.87305672)
\curveto(674.73023413,950.12305653)(675.26148373,950.44250073)(675.75453891,950.83138933)
\closepath
}
}
{
\newrgbcolor{curcolor}{0 0 0}
\pscustom[linestyle=none,fillstyle=solid,fillcolor=curcolor]
{
\newpath
\moveto(684.76496138,947.96680816)
\lineto(682.27537993,947.96680816)
\lineto(682.27537993,950.93555592)
\lineto(684.76496138,950.93555592)
\closepath
}
}
{
\newrgbcolor{curcolor}{0 0 0}
\pscustom[linestyle=none,fillstyle=solid,fillcolor=curcolor]
{
\newpath
\moveto(697.63995431,951.32097229)
\curveto(697.63995431,950.2584731)(697.19898242,949.38694598)(696.31703864,948.70639094)
\curveto(695.4420393,948.0258359)(694.24412354,947.68555838)(692.72329136,947.68555838)
\curveto(691.8621809,947.68555838)(691.07051483,947.78625274)(690.34829316,947.98764148)
\curveto(689.63301592,948.19597466)(689.03232193,948.42166893)(688.54621119,948.6647243)
\lineto(688.54621119,950.86263931)
\lineto(688.65037777,950.86263931)
\curveto(689.26843286,950.39736188)(689.95593234,950.02583438)(690.71287621,949.74805682)
\curveto(691.46982009,949.47722369)(692.19551398,949.34180712)(692.8899579,949.34180712)
\curveto(693.75106836,949.34180712)(694.42467896,949.48069591)(694.9107897,949.75847347)
\curveto(695.39690045,950.03625104)(695.63995582,950.47375071)(695.63995582,951.07097248)
\curveto(695.63995582,951.52930547)(695.50801148,951.87652743)(695.24412279,952.11263836)
\curveto(694.9802341,952.34874929)(694.47329004,952.55013803)(693.7232906,952.71680457)
\curveto(693.44551304,952.77930452)(693.08092998,952.85222113)(692.62954143,952.9355544)
\curveto(692.18509732,953.01888767)(691.77884763,953.10916538)(691.41079235,953.20638753)
\curveto(690.38995979,953.47722066)(689.6642659,953.87305369)(689.23371067,954.39388663)
\curveto(688.81009988,954.92166401)(688.59829448,955.56749686)(688.59829448,956.33138517)
\curveto(688.59829448,956.81055147)(688.69551663,957.26194002)(688.88996093,957.68555081)
\curveto(689.09134966,958.1091616)(689.39343277,958.48763354)(689.79621024,958.82096662)
\curveto(690.18509884,959.14735526)(690.67815402,959.40429951)(691.27537579,959.59179937)
\curveto(691.879542,959.78624367)(692.5531526,959.88346581)(693.29620759,959.88346581)
\curveto(693.99065151,959.88346581)(694.69203987,959.79666032)(695.40037267,959.62304934)
\curveto(696.1156499,959.4563828)(696.70939946,959.25152185)(697.18162132,959.00846648)
\lineto(697.18162132,956.91471806)
\lineto(697.07745473,956.91471806)
\curveto(696.57745511,957.28277334)(695.96981668,957.59180088)(695.25453944,957.84180069)
\curveto(694.53926221,958.09874494)(693.83787385,958.22721707)(693.15037437,958.22721707)
\curveto(692.43509713,958.22721707)(691.83093092,958.08832828)(691.33787574,957.81055072)
\curveto(690.84482056,957.53971759)(690.59829297,957.13346789)(690.59829297,956.59180164)
\curveto(690.59829297,956.11263533)(690.74759841,955.7515245)(691.0462093,955.50846912)
\curveto(691.33787574,955.26541375)(691.81009761,955.06749724)(692.46287489,954.91471957)
\curveto(692.82398573,954.8313863)(693.2267632,954.74805303)(693.67120731,954.66471976)
\curveto(694.12259586,954.58138649)(694.49759557,954.50499766)(694.79620646,954.43555327)
\curveto(695.70592799,954.22722009)(696.40731635,953.86958148)(696.90037153,953.36263741)
\curveto(697.39342672,952.84874891)(697.63995431,952.16819387)(697.63995431,951.32097229)
\closepath
}
}
{
\newrgbcolor{curcolor}{0 0 0}
\pscustom[linestyle=none,fillstyle=solid,fillcolor=curcolor]
{
\newpath
\moveto(710.17117953,947.96680816)
\lineto(708.21284767,947.96680816)
\lineto(708.21284767,954.59180315)
\curveto(708.21284767,955.12652497)(708.1815977,955.62652459)(708.11909774,956.09180202)
\curveto(708.05659779,956.56402388)(707.94201455,956.93207916)(707.775348,957.19596785)
\curveto(707.60173703,957.48763429)(707.35173721,957.70291191)(707.02534857,957.84180069)
\curveto(706.69895993,957.98763392)(706.27534914,958.06055053)(705.7545162,958.06055053)
\curveto(705.21979438,958.06055053)(704.66076703,957.92860618)(704.07743414,957.66471749)
\curveto(703.49410124,957.4008288)(702.93507389,957.0640235)(702.40035207,956.65430159)
\lineto(702.40035207,947.96680816)
\lineto(700.44202022,947.96680816)
\lineto(700.44202022,964.17512923)
\lineto(702.40035207,964.17512923)
\lineto(702.40035207,958.31055034)
\curveto(703.01146272,958.8174944)(703.64340669,959.21332743)(704.29618397,959.49804944)
\curveto(704.94896125,959.78277145)(705.61909964,959.92513245)(706.30659912,959.92513245)
\curveto(707.56354261,959.92513245)(708.52187522,959.54666051)(709.18159694,958.78971664)
\curveto(709.84131866,958.03277277)(710.17117953,956.94249582)(710.17117953,955.51888578)
\closepath
}
}
{
\newrgbcolor{curcolor}{0 0 0}
\pscustom[linestyle=none,fillstyle=solid,fillcolor=curcolor]
{
\newpath
\moveto(386.97351281,927.14390725)
\lineto(380.47351773,927.14390725)
\lineto(380.47351773,929.02932249)
\lineto(386.97351281,929.02932249)
\closepath
}
}
{
\newrgbcolor{curcolor}{0 0 0}
\pscustom[linestyle=none,fillstyle=solid,fillcolor=curcolor]
{
\newpath
\moveto(410.98391155,921.30016167)
\lineto(408.30683024,921.30016167)
\lineto(403.11933416,927.46682367)
\lineto(400.21308636,927.46682367)
\lineto(400.21308636,921.30016167)
\lineto(398.15058792,921.30016167)
\lineto(398.15058792,936.8105666)
\lineto(402.49433464,936.8105666)
\curveto(403.43183393,936.8105666)(404.21308334,936.74806665)(404.83808286,936.62306675)
\curveto(405.46308239,936.50501128)(406.02558197,936.28973366)(406.52558159,935.9772339)
\curveto(407.08808116,935.6230675)(407.52558083,935.17515117)(407.83808059,934.63348492)
\curveto(408.1575248,934.0987631)(408.3172469,933.41820806)(408.3172469,932.5918198)
\curveto(408.3172469,931.47376509)(408.03599711,930.53626579)(407.47349754,929.77932192)
\curveto(406.91099796,929.02932249)(406.13669299,928.4633507)(405.15058263,928.08140654)
\closepath
\moveto(406.16099853,932.44598657)
\curveto(406.16099853,932.89043068)(406.08113748,933.28279149)(405.92141538,933.62306901)
\curveto(405.76863772,933.97029097)(405.51169347,934.26195742)(405.15058263,934.49806835)
\curveto(404.85197174,934.69945709)(404.49780534,934.83834587)(404.08808343,934.9147347)
\curveto(403.67836152,934.99806797)(403.195723,935.03973461)(402.64016786,935.03973461)
\lineto(400.21308636,935.03973461)
\lineto(400.21308636,929.18557237)
\lineto(402.29641812,929.18557237)
\curveto(402.9491954,929.18557237)(403.51863942,929.24112789)(404.00475016,929.35223891)
\curveto(404.4908609,929.47029438)(404.90405504,929.68557199)(405.24433256,929.99807176)
\curveto(405.55683232,930.2897382)(405.78599881,930.62307128)(405.93183204,930.998071)
\curveto(406.0846097,931.38001516)(406.16099853,931.86265368)(406.16099853,932.44598657)
\closepath
}
}
{
\newrgbcolor{curcolor}{0 0 0}
\pscustom[linestyle=none,fillstyle=solid,fillcolor=curcolor]
{
\newpath
\moveto(422.2651529,921.30016167)
\lineto(420.30682105,921.30016167)
\lineto(420.30682105,922.59182736)
\curveto(419.64709933,922.07099442)(419.01515536,921.67168917)(418.41098915,921.3939116)
\curveto(417.80682294,921.11613403)(417.14015678,920.97724525)(416.41099067,920.97724525)
\curveto(415.18876937,920.97724525)(414.2373812,921.34877275)(413.55682616,922.09182774)
\curveto(412.87627112,922.84182717)(412.5359936,923.93904856)(412.5359936,925.38349192)
\lineto(412.5359936,932.93556954)
\lineto(414.49432545,932.93556954)
\lineto(414.49432545,926.31057455)
\curveto(414.49432545,925.72029722)(414.52210321,925.21335316)(414.57765872,924.78974236)
\curveto(414.63321423,924.37307601)(414.7512697,924.01543739)(414.93182512,923.71682651)
\curveto(415.11932498,923.41127119)(415.36238035,923.18904913)(415.66099123,923.05016035)
\curveto(415.95960212,922.91127156)(416.39362957,922.84182717)(416.96307358,922.84182717)
\curveto(417.47001764,922.84182717)(418.02210056,922.97377152)(418.61932233,923.23766021)
\curveto(419.22348854,923.50154889)(419.78598811,923.8383542)(420.30682105,924.24807611)
\lineto(420.30682105,932.93556954)
\lineto(422.2651529,932.93556954)
\closepath
}
}
{
\newrgbcolor{curcolor}{0 0 0}
\pscustom[linestyle=none,fillstyle=solid,fillcolor=curcolor]
{
\newpath
\moveto(435.84847752,921.30016167)
\lineto(433.89014567,921.30016167)
\lineto(433.89014567,927.92515666)
\curveto(433.89014567,928.45987848)(433.8588957,928.9598781)(433.79639574,929.42515552)
\curveto(433.73389579,929.89737739)(433.61931254,930.26543267)(433.452646,930.52932136)
\curveto(433.27903502,930.8209878)(433.02903521,931.03626542)(432.70264657,931.1751542)
\curveto(432.37625793,931.32098742)(431.95264714,931.39390403)(431.4318142,931.39390403)
\curveto(430.89709238,931.39390403)(430.33806503,931.26195969)(429.75473213,930.998071)
\curveto(429.17139924,930.73418231)(428.61237189,930.39737701)(428.07765007,929.9876551)
\lineto(428.07765007,921.30016167)
\lineto(426.11931822,921.30016167)
\lineto(426.11931822,932.93556954)
\lineto(428.07765007,932.93556954)
\lineto(428.07765007,931.64390385)
\curveto(428.68876072,932.15084791)(429.32070468,932.54668094)(429.97348197,932.83140295)
\curveto(430.62625925,933.11612495)(431.29639763,933.25848596)(431.98389711,933.25848596)
\curveto(433.24084061,933.25848596)(434.19917322,932.88001402)(434.85889494,932.12307015)
\curveto(435.51861666,931.36612628)(435.84847752,930.27584933)(435.84847752,928.85223929)
\closepath
}
}
{
\newrgbcolor{curcolor}{0 0 0}
\pscustom[linestyle=none,fillstyle=solid,fillcolor=curcolor]
{
\newpath
\moveto(447.93180328,924.6543258)
\curveto(447.93180328,923.5918266)(447.49083139,922.72029949)(446.60888761,922.03974445)
\curveto(445.73388827,921.3591894)(444.53597251,921.01891188)(443.01514033,921.01891188)
\curveto(442.15402987,921.01891188)(441.3623638,921.11960625)(440.64014213,921.32099499)
\curveto(439.92486489,921.52932816)(439.3241709,921.75502244)(438.83806016,921.99807781)
\lineto(438.83806016,924.19599281)
\lineto(438.94222675,924.19599281)
\curveto(439.56028183,923.73071539)(440.24778131,923.35918789)(441.00472519,923.08141032)
\curveto(441.76166906,922.8105772)(442.48736295,922.67516063)(443.18180687,922.67516063)
\curveto(444.04291733,922.67516063)(444.71652793,922.81404941)(445.20263868,923.09182698)
\curveto(445.68874942,923.36960455)(445.93180479,923.80710422)(445.93180479,924.40432599)
\curveto(445.93180479,924.86265898)(445.79986045,925.20988094)(445.53597176,925.44599187)
\curveto(445.27208307,925.6821028)(444.76513901,925.88349154)(444.01513957,926.05015808)
\curveto(443.73736201,926.11265803)(443.37277895,926.18557464)(442.9213904,926.26890791)
\curveto(442.47694629,926.35224118)(442.0706966,926.44251889)(441.70264132,926.53974104)
\curveto(440.68180876,926.81057417)(439.95611487,927.2064072)(439.52555964,927.72724014)
\curveto(439.10194885,928.25501752)(438.89014345,928.90085037)(438.89014345,929.66473868)
\curveto(438.89014345,930.14390498)(438.9873656,930.59529353)(439.1818099,931.01890432)
\curveto(439.38319863,931.44251511)(439.68528174,931.82098705)(440.08805921,932.15432013)
\curveto(440.47694781,932.48070877)(440.97000299,932.73765302)(441.56722476,932.92515288)
\curveto(442.17139097,933.11959717)(442.84500157,933.21681932)(443.58805656,933.21681932)
\curveto(444.28250048,933.21681932)(444.98388884,933.13001383)(445.69222164,932.95640285)
\curveto(446.40749888,932.78973631)(447.00124843,932.58487536)(447.47347029,932.34181998)
\lineto(447.47347029,930.24807157)
\lineto(447.3693037,930.24807157)
\curveto(446.86930408,930.61612685)(446.26166565,930.92515439)(445.54638842,931.1751542)
\curveto(444.83111118,931.43209845)(444.12972282,931.56057058)(443.44222334,931.56057058)
\curveto(442.7269461,931.56057058)(442.12277989,931.42168179)(441.62972471,931.14390422)
\curveto(441.13666953,930.8730711)(440.89014194,930.4668214)(440.89014194,929.92515515)
\curveto(440.89014194,929.44598884)(441.03944738,929.084878)(441.33805827,928.84182263)
\curveto(441.62972471,928.59876726)(442.10194658,928.40085074)(442.75472386,928.24807308)
\curveto(443.1158347,928.16473981)(443.51861217,928.08140654)(443.96305628,927.99807327)
\curveto(444.41444483,927.91474)(444.78944454,927.83835117)(445.08805543,927.76890678)
\curveto(445.99777696,927.5605736)(446.69916532,927.20293498)(447.1922205,926.69599092)
\curveto(447.68527569,926.18210242)(447.93180328,925.50154738)(447.93180328,924.6543258)
\closepath
}
}
{
\newrgbcolor{curcolor}{0 0 0}
\pscustom[linestyle=none,fillstyle=solid,fillcolor=curcolor]
{
\newpath
\moveto(464.2755385,921.40432826)
\curveto(463.90748322,921.30710611)(463.50470575,921.22724506)(463.06720608,921.16474511)
\curveto(462.63665085,921.10224515)(462.25123447,921.07099518)(461.91095695,921.07099518)
\curveto(460.72345785,921.07099518)(459.82068075,921.39043938)(459.20262567,922.02932779)
\curveto(458.58457058,922.66821619)(458.27554303,923.69252097)(458.27554303,925.10224213)
\lineto(458.27554303,931.28973745)
\lineto(456.95262737,931.28973745)
\lineto(456.95262737,932.93556954)
\lineto(458.27554303,932.93556954)
\lineto(458.27554303,936.27931701)
\lineto(460.23387489,936.27931701)
\lineto(460.23387489,932.93556954)
\lineto(464.2755385,932.93556954)
\lineto(464.2755385,931.28973745)
\lineto(460.23387489,931.28973745)
\lineto(460.23387489,925.98765812)
\curveto(460.23387489,925.37654748)(460.24776376,924.89738117)(460.27554152,924.55015921)
\curveto(460.30331928,924.20988169)(460.40054143,923.89043749)(460.56720797,923.5918266)
\curveto(460.71998563,923.31404904)(460.92831881,923.10918808)(461.19220749,922.97724374)
\curveto(461.46304062,922.85224383)(461.87276254,922.78974388)(462.42137323,922.78974388)
\curveto(462.74081743,922.78974388)(463.07415052,922.83488273)(463.42137248,922.92516044)
\curveto(463.76859443,923.02238259)(464.01859425,923.10224364)(464.17137191,923.16474359)
\lineto(464.2755385,923.16474359)
\closepath
}
}
{
\newrgbcolor{curcolor}{0 0 0}
\pscustom[linestyle=none,fillstyle=solid,fillcolor=curcolor]
{
\newpath
\moveto(476.37969468,926.91474076)
\lineto(467.80678449,926.91474076)
\curveto(467.80678449,926.19946352)(467.9144233,925.57446399)(468.12970092,925.03974218)
\curveto(468.34497853,924.5119648)(468.6401172,924.07793735)(469.01511691,923.73765983)
\curveto(469.37622775,923.40432675)(469.80331076,923.15432694)(470.29636594,922.98766039)
\curveto(470.79636557,922.82099385)(471.34497626,922.73766058)(471.94219803,922.73766058)
\curveto(472.7338641,922.73766058)(473.52900239,922.89391047)(474.32761289,923.20641023)
\curveto(475.13316784,923.52585443)(475.70608407,923.8383542)(476.04636159,924.14390952)
\lineto(476.15052818,924.14390952)
\lineto(476.15052818,922.00849447)
\curveto(475.49080646,921.7307169)(474.81719586,921.49807819)(474.12969638,921.31057833)
\curveto(473.4421969,921.12307847)(472.71997522,921.02932854)(471.96303135,921.02932854)
\curveto(470.03247725,921.02932854)(468.52553395,921.55016148)(467.44220144,922.59182736)
\curveto(466.35886892,923.64043768)(465.81720267,925.12654767)(465.81720267,927.05015732)
\curveto(465.81720267,928.95293366)(466.33456339,930.46334918)(467.36928483,931.58140389)
\curveto(468.4109507,932.6994586)(469.77900522,933.25848596)(471.47344839,933.25848596)
\curveto(473.04289164,933.25848596)(474.25122406,932.80015297)(475.09844564,931.883487)
\curveto(475.95261166,930.96682102)(476.37969468,929.66473868)(476.37969468,927.97723995)
\closepath
\moveto(474.47344612,928.41473962)
\curveto(474.46650168,929.44251662)(474.20608521,930.23765491)(473.69219671,930.80015448)
\curveto(473.18525265,931.36265406)(472.41094768,931.64390385)(471.3692818,931.64390385)
\curveto(470.32067148,931.64390385)(469.48386656,931.3348763)(468.85886703,930.71682121)
\curveto(468.24081194,930.09876613)(467.89011776,929.3314056)(467.80678449,928.41473962)
\closepath
}
}
{
\newrgbcolor{curcolor}{0 0 0}
\pscustom[linestyle=none,fillstyle=solid,fillcolor=curcolor]
{
\newpath
\moveto(487.66093214,924.6543258)
\curveto(487.66093214,923.5918266)(487.21996025,922.72029949)(486.33801647,922.03974445)
\curveto(485.46301713,921.3591894)(484.26510137,921.01891188)(482.74426919,921.01891188)
\curveto(481.88315873,921.01891188)(481.09149266,921.11960625)(480.36927099,921.32099499)
\curveto(479.65399375,921.52932816)(479.05329976,921.75502244)(478.56718902,921.99807781)
\lineto(478.56718902,924.19599281)
\lineto(478.67135561,924.19599281)
\curveto(479.28941069,923.73071539)(479.97691017,923.35918789)(480.73385405,923.08141032)
\curveto(481.49079792,922.8105772)(482.21649181,922.67516063)(482.91093573,922.67516063)
\curveto(483.77204619,922.67516063)(484.44565679,922.81404941)(484.93176754,923.09182698)
\curveto(485.41787828,923.36960455)(485.66093365,923.80710422)(485.66093365,924.40432599)
\curveto(485.66093365,924.86265898)(485.52898931,925.20988094)(485.26510062,925.44599187)
\curveto(485.00121193,925.6821028)(484.49426787,925.88349154)(483.74426843,926.05015808)
\curveto(483.46649087,926.11265803)(483.10190781,926.18557464)(482.65051926,926.26890791)
\curveto(482.20607515,926.35224118)(481.79982546,926.44251889)(481.43177018,926.53974104)
\curveto(480.41093762,926.81057417)(479.68524373,927.2064072)(479.2546885,927.72724014)
\curveto(478.83107771,928.25501752)(478.61927231,928.90085037)(478.61927231,929.66473868)
\curveto(478.61927231,930.14390498)(478.71649446,930.59529353)(478.91093876,931.01890432)
\curveto(479.11232749,931.44251511)(479.4144106,931.82098705)(479.81718807,932.15432013)
\curveto(480.20607667,932.48070877)(480.69913185,932.73765302)(481.29635362,932.92515288)
\curveto(481.90051983,933.11959717)(482.57413043,933.21681932)(483.31718542,933.21681932)
\curveto(484.01162934,933.21681932)(484.7130177,933.13001383)(485.4213505,932.95640285)
\curveto(486.13662774,932.78973631)(486.73037729,932.58487536)(487.20259915,932.34181998)
\lineto(487.20259915,930.24807157)
\lineto(487.09843256,930.24807157)
\curveto(486.59843294,930.61612685)(485.99079451,930.92515439)(485.27551728,931.1751542)
\curveto(484.56024004,931.43209845)(483.85885168,931.56057058)(483.1713522,931.56057058)
\curveto(482.45607496,931.56057058)(481.85190876,931.42168179)(481.35885357,931.14390422)
\curveto(480.86579839,930.8730711)(480.6192708,930.4668214)(480.6192708,929.92515515)
\curveto(480.6192708,929.44598884)(480.76857624,929.084878)(481.06718713,928.84182263)
\curveto(481.35885357,928.59876726)(481.83107544,928.40085074)(482.48385272,928.24807308)
\curveto(482.84496356,928.16473981)(483.24774103,928.08140654)(483.69218514,927.99807327)
\curveto(484.14357369,927.91474)(484.5185734,927.83835117)(484.81718429,927.76890678)
\curveto(485.72690582,927.5605736)(486.42829418,927.20293498)(486.92134936,926.69599092)
\curveto(487.41440455,926.18210242)(487.66093214,925.50154738)(487.66093214,924.6543258)
\closepath
}
}
{
\newrgbcolor{curcolor}{0 0 0}
\pscustom[linestyle=none,fillstyle=solid,fillcolor=curcolor]
{
\newpath
\moveto(496.50468177,921.40432826)
\curveto(496.1366265,921.30710611)(495.73384902,921.22724506)(495.29634935,921.16474511)
\curveto(494.86579412,921.10224515)(494.48037775,921.07099518)(494.14010023,921.07099518)
\curveto(492.95260113,921.07099518)(492.04982403,921.39043938)(491.43176894,922.02932779)
\curveto(490.81371386,922.66821619)(490.50468631,923.69252097)(490.50468631,925.10224213)
\lineto(490.50468631,931.28973745)
\lineto(489.18177065,931.28973745)
\lineto(489.18177065,932.93556954)
\lineto(490.50468631,932.93556954)
\lineto(490.50468631,936.27931701)
\lineto(492.46301816,936.27931701)
\lineto(492.46301816,932.93556954)
\lineto(496.50468177,932.93556954)
\lineto(496.50468177,931.28973745)
\lineto(492.46301816,931.28973745)
\lineto(492.46301816,925.98765812)
\curveto(492.46301816,925.37654748)(492.47690704,924.89738117)(492.5046848,924.55015921)
\curveto(492.53246256,924.20988169)(492.62968471,923.89043749)(492.79635125,923.5918266)
\curveto(492.94912891,923.31404904)(493.15746208,923.10918808)(493.42135077,922.97724374)
\curveto(493.6921839,922.85224383)(494.10190581,922.78974388)(494.65051651,922.78974388)
\curveto(494.96996071,922.78974388)(495.30329379,922.83488273)(495.65051575,922.92516044)
\curveto(495.99773771,923.02238259)(496.24773752,923.10224364)(496.40051519,923.16474359)
\lineto(496.50468177,923.16474359)
\closepath
}
}
{
\newrgbcolor{curcolor}{0 0 0}
\pscustom[linestyle=none,fillstyle=solid,fillcolor=curcolor]
{
\newpath
\moveto(512.41091688,921.40432826)
\curveto(512.0428616,921.30710611)(511.64008412,921.22724506)(511.20258446,921.16474511)
\curveto(510.77202923,921.10224515)(510.38661285,921.07099518)(510.04633533,921.07099518)
\curveto(508.85883623,921.07099518)(507.95605913,921.39043938)(507.33800405,922.02932779)
\curveto(506.71994896,922.66821619)(506.41092141,923.69252097)(506.41092141,925.10224213)
\lineto(506.41092141,931.28973745)
\lineto(505.08800575,931.28973745)
\lineto(505.08800575,932.93556954)
\lineto(506.41092141,932.93556954)
\lineto(506.41092141,936.27931701)
\lineto(508.36925327,936.27931701)
\lineto(508.36925327,932.93556954)
\lineto(512.41091688,932.93556954)
\lineto(512.41091688,931.28973745)
\lineto(508.36925327,931.28973745)
\lineto(508.36925327,925.98765812)
\curveto(508.36925327,925.37654748)(508.38314214,924.89738117)(508.4109199,924.55015921)
\curveto(508.43869766,924.20988169)(508.53591981,923.89043749)(508.70258635,923.5918266)
\curveto(508.85536401,923.31404904)(509.06369719,923.10918808)(509.32758587,922.97724374)
\curveto(509.598419,922.85224383)(510.00814092,922.78974388)(510.55675161,922.78974388)
\curveto(510.87619581,922.78974388)(511.2095289,922.83488273)(511.55675085,922.92516044)
\curveto(511.90397281,923.02238259)(512.15397262,923.10224364)(512.30675029,923.16474359)
\lineto(512.41091688,923.16474359)
\closepath
}
}
{
\newrgbcolor{curcolor}{0 0 0}
\pscustom[linestyle=none,fillstyle=solid,fillcolor=curcolor]
{
\newpath
\moveto(529.41090268,932.93556954)
\lineto(526.37965498,921.30016167)
\lineto(524.56715635,921.30016167)
\lineto(521.57757528,930.26890489)
\lineto(518.60882752,921.30016167)
\lineto(516.80674555,921.30016167)
\lineto(513.74424787,932.93556954)
\lineto(515.78591299,932.93556954)
\lineto(517.92132804,923.92515969)
\lineto(520.82757584,932.93556954)
\lineto(522.44215796,932.93556954)
\lineto(525.42132237,923.92515969)
\lineto(527.44215417,932.93556954)
\closepath
}
}
{
\newrgbcolor{curcolor}{0 0 0}
\pscustom[linestyle=none,fillstyle=solid,fillcolor=curcolor]
{
\newpath
\moveto(534.33797563,934.88348473)
\lineto(532.12964397,934.88348473)
\lineto(532.12964397,936.91473319)
\lineto(534.33797563,936.91473319)
\closepath
\moveto(534.21297573,921.30016167)
\lineto(532.25464388,921.30016167)
\lineto(532.25464388,932.93556954)
\lineto(534.21297573,932.93556954)
\closepath
}
}
{
\newrgbcolor{curcolor}{0 0 0}
\pscustom[linestyle=none,fillstyle=solid,fillcolor=curcolor]
{
\newpath
\moveto(546.69214091,922.02932779)
\curveto(546.03936362,921.71682802)(545.41783632,921.47377265)(544.82755898,921.30016167)
\curveto(544.24422609,921.12655069)(543.62269878,921.0397452)(542.96297706,921.0397452)
\curveto(542.12269992,921.0397452)(541.35186717,921.16127289)(540.65047881,921.40432826)
\curveto(539.94909045,921.65432807)(539.34839646,922.02932779)(538.84839684,922.52932741)
\curveto(538.34145278,923.02932703)(537.94909197,923.661271)(537.6713144,924.42515931)
\curveto(537.39353683,925.18904762)(537.25464805,926.08140805)(537.25464805,927.10224062)
\curveto(537.25464805,929.00501695)(537.77548099,930.49807138)(538.81714686,931.58140389)
\curveto(539.86575718,932.66473641)(541.24770058,933.20640266)(542.96297706,933.20640266)
\curveto(543.62964322,933.20640266)(544.28242051,933.11265273)(544.92130891,932.92515288)
\curveto(545.56714176,932.73765302)(546.15741909,932.50848652)(546.69214091,932.2376534)
\lineto(546.69214091,930.06057171)
\lineto(546.58797432,930.06057171)
\curveto(545.99075255,930.52584914)(545.37269746,930.88348775)(544.73380906,931.13348757)
\curveto(544.10186509,931.38348738)(543.48381,931.50848728)(542.87964379,931.50848728)
\curveto(541.76853352,931.50848728)(540.89006196,931.13348757)(540.24422912,930.38348813)
\curveto(539.60534071,929.64043314)(539.28589651,928.54668397)(539.28589651,927.10224062)
\curveto(539.28589651,925.6994639)(539.59839627,924.6196036)(540.2233958,923.86265973)
\curveto(540.85533977,923.1126603)(541.74075576,922.73766058)(542.87964379,922.73766058)
\curveto(543.27547683,922.73766058)(543.6782543,922.78974388)(544.08797621,922.89391047)
\curveto(544.49769812,922.99807705)(544.8657534,923.13349362)(545.19214204,923.30016016)
\curveto(545.47686405,923.44599338)(545.74422496,923.59877104)(545.99422477,923.75849314)
\curveto(546.24422458,923.92515969)(546.4421411,924.06752069)(546.58797432,924.18557616)
\lineto(546.69214091,924.18557616)
\closepath
}
}
{
\newrgbcolor{curcolor}{0 0 0}
\pscustom[linestyle=none,fillstyle=solid,fillcolor=curcolor]
{
\newpath
\moveto(558.94213076,926.91474076)
\lineto(550.36922058,926.91474076)
\curveto(550.36922058,926.19946352)(550.47685938,925.57446399)(550.692137,925.03974218)
\curveto(550.90741461,924.5119648)(551.20255328,924.07793735)(551.57755299,923.73765983)
\curveto(551.93866383,923.40432675)(552.36574684,923.15432694)(552.85880203,922.98766039)
\curveto(553.35880165,922.82099385)(553.90741234,922.73766058)(554.50463411,922.73766058)
\curveto(555.29630018,922.73766058)(556.09143847,922.89391047)(556.89004898,923.20641023)
\curveto(557.69560392,923.52585443)(558.26852015,923.8383542)(558.60879768,924.14390952)
\lineto(558.71296426,924.14390952)
\lineto(558.71296426,922.00849447)
\curveto(558.05324254,921.7307169)(557.37963194,921.49807819)(556.69213246,921.31057833)
\curveto(556.00463298,921.12307847)(555.2824113,921.02932854)(554.52546743,921.02932854)
\curveto(552.59491334,921.02932854)(551.08797003,921.55016148)(550.00463752,922.59182736)
\curveto(548.921305,923.64043768)(548.37963875,925.12654767)(548.37963875,927.05015732)
\curveto(548.37963875,928.95293366)(548.89699947,930.46334918)(549.93172091,931.58140389)
\curveto(550.97338679,932.6994586)(552.34144131,933.25848596)(554.03588447,933.25848596)
\curveto(555.60532773,933.25848596)(556.81366014,932.80015297)(557.66088173,931.883487)
\curveto(558.51504775,930.96682102)(558.94213076,929.66473868)(558.94213076,927.97723995)
\closepath
\moveto(557.0358822,928.41473962)
\curveto(557.02893776,929.44251662)(556.76852129,930.23765491)(556.25463279,930.80015448)
\curveto(555.74768873,931.36265406)(554.97338376,931.64390385)(553.93171788,931.64390385)
\curveto(552.88310756,931.64390385)(552.04630264,931.3348763)(551.42130311,930.71682121)
\curveto(550.80324802,930.09876613)(550.45255385,929.3314056)(550.36922058,928.41473962)
\closepath
}
}
{
\newrgbcolor{curcolor}{0 0 0}
\pscustom[linestyle=none,fillstyle=solid,fillcolor=curcolor]
{
\newpath
\moveto(575.4525294,921.40432826)
\curveto(575.08447412,921.30710611)(574.68169664,921.22724506)(574.24419698,921.16474511)
\curveto(573.81364175,921.10224515)(573.42822537,921.07099518)(573.08794785,921.07099518)
\curveto(571.90044875,921.07099518)(570.99767165,921.39043938)(570.37961657,922.02932779)
\curveto(569.76156148,922.66821619)(569.45253393,923.69252097)(569.45253393,925.10224213)
\lineto(569.45253393,931.28973745)
\lineto(568.12961827,931.28973745)
\lineto(568.12961827,932.93556954)
\lineto(569.45253393,932.93556954)
\lineto(569.45253393,936.27931701)
\lineto(571.41086579,936.27931701)
\lineto(571.41086579,932.93556954)
\lineto(575.4525294,932.93556954)
\lineto(575.4525294,931.28973745)
\lineto(571.41086579,931.28973745)
\lineto(571.41086579,925.98765812)
\curveto(571.41086579,925.37654748)(571.42475466,924.89738117)(571.45253242,924.55015921)
\curveto(571.48031018,924.20988169)(571.57753233,923.89043749)(571.74419887,923.5918266)
\curveto(571.89697653,923.31404904)(572.10530971,923.10918808)(572.36919839,922.97724374)
\curveto(572.64003152,922.85224383)(573.04975344,922.78974388)(573.59836413,922.78974388)
\curveto(573.91780833,922.78974388)(574.25114142,922.83488273)(574.59836337,922.92516044)
\curveto(574.94558533,923.02238259)(575.19558514,923.10224364)(575.34836281,923.16474359)
\lineto(575.4525294,923.16474359)
\closepath
}
}
{
\newrgbcolor{curcolor}{0 0 0}
\pscustom[linestyle=none,fillstyle=solid,fillcolor=curcolor]
{
\newpath
\moveto(587.73376877,927.11265727)
\curveto(587.73376877,925.21682537)(587.24765803,923.72029873)(586.27543654,922.62307734)
\curveto(585.30321506,921.52585595)(584.00113271,920.97724525)(582.3691895,920.97724525)
\curveto(580.72335741,920.97724525)(579.41433062,921.52585595)(578.44210914,922.62307734)
\curveto(577.47683209,923.72029873)(576.99419357,925.21682537)(576.99419357,927.11265727)
\curveto(576.99419357,929.00848917)(577.47683209,930.50501582)(578.44210914,931.60223721)
\curveto(579.41433062,932.70640304)(580.72335741,933.25848596)(582.3691895,933.25848596)
\curveto(584.00113271,933.25848596)(585.30321506,932.70640304)(586.27543654,931.60223721)
\curveto(587.24765803,930.50501582)(587.73376877,929.00848917)(587.73376877,927.11265727)
\closepath
\moveto(585.71293697,927.11265727)
\curveto(585.71293697,928.61960058)(585.4177983,929.73765529)(584.82752097,930.4668214)
\curveto(584.23724364,931.20293196)(583.41779982,931.57098723)(582.3691895,931.57098723)
\curveto(581.3066903,931.57098723)(580.48030204,931.20293196)(579.89002471,930.4668214)
\curveto(579.30669182,929.73765529)(579.01502537,928.61960058)(579.01502537,927.11265727)
\curveto(579.01502537,925.65432504)(579.31016404,924.54668699)(579.90044137,923.78974312)
\curveto(580.4907187,923.03974369)(581.31363474,922.66474397)(582.3691895,922.66474397)
\curveto(583.41085538,922.66474397)(584.22682698,923.03627147)(584.81710431,923.77932646)
\curveto(585.41432608,924.52932589)(585.71293697,925.64043617)(585.71293697,927.11265727)
\closepath
}
}
{
\newrgbcolor{curcolor}{0 0 0}
\pscustom[linestyle=none,fillstyle=solid,fillcolor=curcolor]
{
\newpath
\moveto(608.50459193,927.2584905)
\curveto(608.50459193,926.31404677)(608.36917537,925.44946409)(608.09834224,924.66474246)
\curveto(607.82750911,923.88696527)(607.44556495,923.22724355)(606.95250977,922.68557729)
\curveto(606.49417679,922.17168879)(605.95251053,921.77238354)(605.327511,921.48766153)
\curveto(604.70945591,921.20988396)(604.05320641,921.07099518)(603.35876249,921.07099518)
\curveto(602.75459628,921.07099518)(602.20598558,921.13696735)(601.7129304,921.26891169)
\curveto(601.22681966,921.40085604)(600.73029226,921.605717)(600.2233482,921.88349456)
\lineto(600.2233482,917.00849825)
\lineto(598.26501634,917.00849825)
\lineto(598.26501634,932.93556954)
\lineto(600.2233482,932.93556954)
\lineto(600.2233482,931.71682046)
\curveto(600.74418114,932.15432013)(601.32751403,932.51890318)(601.97334687,932.81056963)
\curveto(602.62612416,933.10918051)(603.32056807,933.25848596)(604.05667863,933.25848596)
\curveto(605.45945535,933.25848596)(606.5497323,932.72723636)(607.32750949,931.66473716)
\curveto(608.11223112,930.60918241)(608.50459193,929.14043352)(608.50459193,927.2584905)
\closepath
\moveto(606.48376013,927.2064072)
\curveto(606.48376013,928.60918392)(606.24417697,929.65779424)(605.76501067,930.35223816)
\curveto(605.28584437,931.04668208)(604.54973381,931.39390403)(603.55667901,931.39390403)
\curveto(602.99417943,931.39390403)(602.42820764,931.27237635)(601.85876363,931.02932098)
\curveto(601.28931961,930.78626561)(600.74418114,930.4668214)(600.2233482,930.07098837)
\lineto(600.2233482,923.47724336)
\curveto(600.77890333,923.22724355)(601.25459742,923.05710479)(601.65043045,922.96682708)
\curveto(602.05320792,922.87654937)(602.50806869,922.83141051)(603.01501275,922.83141051)
\curveto(604.1052897,922.83141051)(604.9559835,923.19946579)(605.56709415,923.93557634)
\curveto(606.1782048,924.6716869)(606.48376013,925.76196385)(606.48376013,927.2064072)
\closepath
}
}
{
\newrgbcolor{curcolor}{0 0 0}
\pscustom[linestyle=none,fillstyle=solid,fillcolor=curcolor]
{
\newpath
\moveto(618.81708233,930.80015448)
\lineto(618.71291574,930.80015448)
\curveto(618.4212493,930.86959888)(618.13652729,930.91820995)(617.85874972,930.94598771)
\curveto(617.58791659,930.9807099)(617.26500017,930.998071)(616.89000045,930.998071)
\curveto(616.28583425,930.998071)(615.70250135,930.86265444)(615.14000178,930.59182131)
\curveto(614.5775022,930.32793262)(614.03583595,929.98418288)(613.51500301,929.56057209)
\lineto(613.51500301,921.30016167)
\lineto(611.55667116,921.30016167)
\lineto(611.55667116,932.93556954)
\lineto(613.51500301,932.93556954)
\lineto(613.51500301,931.21682084)
\curveto(614.2927802,931.84182036)(614.97680746,932.28279225)(615.56708479,932.5397365)
\curveto(616.16430656,932.80362519)(616.77194499,932.93556954)(617.39000008,932.93556954)
\curveto(617.7302776,932.93556954)(617.97680519,932.92515288)(618.12958285,932.90431956)
\curveto(618.28236051,932.89043068)(618.51152701,932.8591807)(618.81708233,932.81056963)
\closepath
}
}
{
\newrgbcolor{curcolor}{0 0 0}
\pscustom[linestyle=none,fillstyle=solid,fillcolor=curcolor]
{
\newpath
\moveto(622.76499873,934.88348473)
\lineto(620.55666707,934.88348473)
\lineto(620.55666707,936.91473319)
\lineto(622.76499873,936.91473319)
\closepath
\moveto(622.63999883,921.30016167)
\lineto(620.68166698,921.30016167)
\lineto(620.68166698,932.93556954)
\lineto(622.63999883,932.93556954)
\closepath
}
}
{
\newrgbcolor{curcolor}{0 0 0}
\pscustom[linestyle=none,fillstyle=solid,fillcolor=curcolor]
{
\newpath
\moveto(643.49414325,921.30016167)
\lineto(641.5358114,921.30016167)
\lineto(641.5358114,927.92515666)
\curveto(641.5358114,928.42515628)(641.51150586,928.9077948)(641.46289479,929.37307223)
\curveto(641.42122815,929.83834966)(641.32747822,930.20987715)(641.181645,930.48765472)
\curveto(641.0219229,930.78626561)(640.79275641,931.01195988)(640.49414552,931.16473754)
\curveto(640.19553464,931.3175152)(639.76497941,931.39390403)(639.20247983,931.39390403)
\curveto(638.65386914,931.39390403)(638.10525844,931.25501525)(637.55664774,930.97723768)
\curveto(637.00803705,930.70640455)(636.45942635,930.3591826)(635.91081566,929.9355718)
\curveto(635.93164897,929.7758497)(635.94901007,929.58834985)(635.96289895,929.37307223)
\curveto(635.97678783,929.16473905)(635.98373227,928.95640588)(635.98373227,928.7480727)
\lineto(635.98373227,921.30016167)
\lineto(634.02540042,921.30016167)
\lineto(634.02540042,927.92515666)
\curveto(634.02540042,928.43904516)(634.00109488,928.9251559)(633.9524838,929.38348889)
\curveto(633.91081717,929.84876631)(633.81706724,930.22029381)(633.67123402,930.49807138)
\curveto(633.51151192,930.79668226)(633.28234542,931.01890432)(632.98373454,931.16473754)
\curveto(632.68512365,931.3175152)(632.25456842,931.39390403)(631.69206885,931.39390403)
\curveto(631.15734703,931.39390403)(630.61915299,931.26195969)(630.07748674,930.998071)
\curveto(629.54276492,930.73418231)(629.0080431,930.39737701)(628.47332128,929.9876551)
\lineto(628.47332128,921.30016167)
\lineto(626.51498943,921.30016167)
\lineto(626.51498943,932.93556954)
\lineto(628.47332128,932.93556954)
\lineto(628.47332128,931.64390385)
\curveto(629.08443193,932.15084791)(629.69207036,932.54668094)(630.29623657,932.83140295)
\curveto(630.90734722,933.11612495)(631.55665228,933.25848596)(632.24415176,933.25848596)
\curveto(633.03581783,933.25848596)(633.70595621,933.09181942)(634.25456691,932.75848634)
\curveto(634.81012204,932.42515325)(635.22331618,931.96334805)(635.4941493,931.37307072)
\curveto(636.28581537,932.03973688)(637.00803705,932.51890318)(637.66081433,932.81056963)
\curveto(638.31359162,933.10918051)(639.01150775,933.25848596)(639.75456275,933.25848596)
\curveto(641.03233956,933.25848596)(641.97331107,932.86959736)(642.57747728,932.09182017)
\curveto(643.18858793,931.32098742)(643.49414325,930.24112713)(643.49414325,928.85223929)
\closepath
}
}
{
\newrgbcolor{curcolor}{0 0 0}
\pscustom[linestyle=none,fillstyle=solid,fillcolor=curcolor]
{
\newpath
\moveto(657.00456732,926.91474076)
\lineto(648.43165714,926.91474076)
\curveto(648.43165714,926.19946352)(648.53929594,925.57446399)(648.75457356,925.03974218)
\curveto(648.96985117,924.5119648)(649.26498984,924.07793735)(649.63998956,923.73765983)
\curveto(650.00110039,923.40432675)(650.4281834,923.15432694)(650.92123859,922.98766039)
\curveto(651.42123821,922.82099385)(651.96984891,922.73766058)(652.56707068,922.73766058)
\curveto(653.35873674,922.73766058)(654.15387503,922.89391047)(654.95248554,923.20641023)
\curveto(655.75804048,923.52585443)(656.33095672,923.8383542)(656.67123424,924.14390952)
\lineto(656.77540082,924.14390952)
\lineto(656.77540082,922.00849447)
\curveto(656.1156791,921.7307169)(655.4420685,921.49807819)(654.75456902,921.31057833)
\curveto(654.06706954,921.12307847)(653.34484786,921.02932854)(652.58790399,921.02932854)
\curveto(650.6573499,921.02932854)(649.15040659,921.55016148)(648.06707408,922.59182736)
\curveto(646.98374157,923.64043768)(646.44207531,925.12654767)(646.44207531,927.05015732)
\curveto(646.44207531,928.95293366)(646.95943603,930.46334918)(647.99415747,931.58140389)
\curveto(649.03582335,932.6994586)(650.40387787,933.25848596)(652.09832103,933.25848596)
\curveto(653.66776429,933.25848596)(654.87609671,932.80015297)(655.72331829,931.883487)
\curveto(656.57748431,930.96682102)(657.00456732,929.66473868)(657.00456732,927.97723995)
\closepath
\moveto(655.09831876,928.41473962)
\curveto(655.09137432,929.44251662)(654.83095785,930.23765491)(654.31706935,930.80015448)
\curveto(653.81012529,931.36265406)(653.03582032,931.64390385)(651.99415444,931.64390385)
\curveto(650.94554412,931.64390385)(650.1087392,931.3348763)(649.48373967,930.71682121)
\curveto(648.86568459,930.09876613)(648.51499041,929.3314056)(648.43165714,928.41473962)
\closepath
}
}
{
\newrgbcolor{curcolor}{0 0 0}
\pscustom[linestyle=none,fillstyle=solid,fillcolor=curcolor]
{
\newpath
\moveto(673.51495154,921.40432826)
\curveto(673.14689626,921.30710611)(672.74411879,921.22724506)(672.30661912,921.16474511)
\curveto(671.87606389,921.10224515)(671.49064752,921.07099518)(671.15036999,921.07099518)
\curveto(669.96287089,921.07099518)(669.0600938,921.39043938)(668.44203871,922.02932779)
\curveto(667.82398362,922.66821619)(667.51495608,923.69252097)(667.51495608,925.10224213)
\lineto(667.51495608,931.28973745)
\lineto(666.19204041,931.28973745)
\lineto(666.19204041,932.93556954)
\lineto(667.51495608,932.93556954)
\lineto(667.51495608,936.27931701)
\lineto(669.47328793,936.27931701)
\lineto(669.47328793,932.93556954)
\lineto(673.51495154,932.93556954)
\lineto(673.51495154,931.28973745)
\lineto(669.47328793,931.28973745)
\lineto(669.47328793,925.98765812)
\curveto(669.47328793,925.37654748)(669.48717681,924.89738117)(669.51495457,924.55015921)
\curveto(669.54273232,924.20988169)(669.63995447,923.89043749)(669.80662101,923.5918266)
\curveto(669.95939867,923.31404904)(670.16773185,923.10918808)(670.43162054,922.97724374)
\curveto(670.70245367,922.85224383)(671.11217558,922.78974388)(671.66078628,922.78974388)
\curveto(671.98023048,922.78974388)(672.31356356,922.83488273)(672.66078552,922.92516044)
\curveto(673.00800748,923.02238259)(673.25800729,923.10224364)(673.41078495,923.16474359)
\lineto(673.51495154,923.16474359)
\closepath
}
}
{
\newrgbcolor{curcolor}{0 0 0}
\pscustom[linestyle=none,fillstyle=solid,fillcolor=curcolor]
{
\newpath
\moveto(685.60869106,921.30016167)
\lineto(683.65035921,921.30016167)
\lineto(683.65035921,927.92515666)
\curveto(683.65035921,928.45987848)(683.61910923,928.9598781)(683.55660928,929.42515552)
\curveto(683.49410933,929.89737739)(683.37952608,930.26543267)(683.21285954,930.52932136)
\curveto(683.03924856,930.8209878)(682.78924875,931.03626542)(682.46286011,931.1751542)
\curveto(682.13647146,931.32098742)(681.71286067,931.39390403)(681.19202773,931.39390403)
\curveto(680.65730592,931.39390403)(680.09827856,931.26195969)(679.51494567,930.998071)
\curveto(678.93161278,930.73418231)(678.37258542,930.39737701)(677.83786361,929.9876551)
\lineto(677.83786361,921.30016167)
\lineto(675.87953175,921.30016167)
\lineto(675.87953175,937.50848274)
\lineto(677.83786361,937.50848274)
\lineto(677.83786361,931.64390385)
\curveto(678.44897425,932.15084791)(679.08091822,932.54668094)(679.7336955,932.83140295)
\curveto(680.38647279,933.11612495)(681.05661117,933.25848596)(681.74411065,933.25848596)
\curveto(683.00105414,933.25848596)(683.95938675,932.88001402)(684.61910848,932.12307015)
\curveto(685.2788302,931.36612628)(685.60869106,930.27584933)(685.60869106,928.85223929)
\closepath
}
}
{
\newrgbcolor{curcolor}{0 0 0}
\pscustom[linestyle=none,fillstyle=solid,fillcolor=curcolor]
{
\newpath
\moveto(699.11911349,926.91474076)
\lineto(690.54620331,926.91474076)
\curveto(690.54620331,926.19946352)(690.65384211,925.57446399)(690.86911973,925.03974218)
\curveto(691.08439734,924.5119648)(691.37953601,924.07793735)(691.75453572,923.73765983)
\curveto(692.11564656,923.40432675)(692.54272957,923.15432694)(693.03578475,922.98766039)
\curveto(693.53578438,922.82099385)(694.08439507,922.73766058)(694.68161684,922.73766058)
\curveto(695.47328291,922.73766058)(696.2684212,922.89391047)(697.0670317,923.20641023)
\curveto(697.87258665,923.52585443)(698.44550288,923.8383542)(698.7857804,924.14390952)
\lineto(698.88994699,924.14390952)
\lineto(698.88994699,922.00849447)
\curveto(698.23022527,921.7307169)(697.55661467,921.49807819)(696.86911519,921.31057833)
\curveto(696.18161571,921.12307847)(695.45939403,921.02932854)(694.70245016,921.02932854)
\curveto(692.77189607,921.02932854)(691.26495276,921.55016148)(690.18162025,922.59182736)
\curveto(689.09828773,923.64043768)(688.55662148,925.12654767)(688.55662148,927.05015732)
\curveto(688.55662148,928.95293366)(689.0739822,930.46334918)(690.10870364,931.58140389)
\curveto(691.15036951,932.6994586)(692.51842404,933.25848596)(694.2128672,933.25848596)
\curveto(695.78231045,933.25848596)(696.99064287,932.80015297)(697.83786446,931.883487)
\curveto(698.69203048,930.96682102)(699.11911349,929.66473868)(699.11911349,927.97723995)
\closepath
\moveto(697.21286493,928.41473962)
\curveto(697.20592049,929.44251662)(696.94550402,930.23765491)(696.43161552,930.80015448)
\curveto(695.92467146,931.36265406)(695.15036649,931.64390385)(694.10870061,931.64390385)
\curveto(693.06009029,931.64390385)(692.22328537,931.3348763)(691.59828584,930.71682121)
\curveto(690.98023075,930.09876613)(690.62953658,929.3314056)(690.54620331,928.41473962)
\closepath
}
}
{
\newrgbcolor{curcolor}{0 0 0}
\pscustom[linestyle=none,fillstyle=solid,fillcolor=curcolor]
{
\newpath
\moveto(721.97324291,921.30016167)
\lineto(719.7753279,921.30016167)
\lineto(718.25449572,925.62307507)
\lineto(711.54616746,925.62307507)
\lineto(710.02533528,921.30016167)
\lineto(707.93158686,921.30016167)
\lineto(713.57741593,936.8105666)
\lineto(716.32741385,936.8105666)
\closepath
\moveto(717.61907953,927.39390706)
\lineto(714.90033159,935.00848463)
\lineto(712.17116699,927.39390706)
\closepath
}
}
{
\newrgbcolor{curcolor}{0 0 0}
\pscustom[linestyle=none,fillstyle=solid,fillcolor=curcolor]
{
\newpath
\moveto(737.16073604,921.30016167)
\lineto(734.48365473,921.30016167)
\lineto(729.29615866,927.46682367)
\lineto(726.38991086,927.46682367)
\lineto(726.38991086,921.30016167)
\lineto(724.32741242,921.30016167)
\lineto(724.32741242,936.8105666)
\lineto(728.67115913,936.8105666)
\curveto(729.60865842,936.8105666)(730.38990783,936.74806665)(731.01490736,936.62306675)
\curveto(731.63990689,936.50501128)(732.20240646,936.28973366)(732.70240608,935.9772339)
\curveto(733.26490566,935.6230675)(733.70240533,935.17515117)(734.01490509,934.63348492)
\curveto(734.33434929,934.0987631)(734.49407139,933.41820806)(734.49407139,932.5918198)
\curveto(734.49407139,931.47376509)(734.21282161,930.53626579)(733.65032203,929.77932192)
\curveto(733.08782246,929.02932249)(732.31351749,928.4633507)(731.32740712,928.08140654)
\closepath
\moveto(732.33782302,932.44598657)
\curveto(732.33782302,932.89043068)(732.25796197,933.28279149)(732.09823987,933.62306901)
\curveto(731.94546221,933.97029097)(731.68851796,934.26195742)(731.32740712,934.49806835)
\curveto(731.02879624,934.69945709)(730.67462984,934.83834587)(730.26490793,934.9147347)
\curveto(729.85518601,934.99806797)(729.37254749,935.03973461)(728.81699235,935.03973461)
\lineto(726.38991086,935.03973461)
\lineto(726.38991086,929.18557237)
\lineto(728.47324261,929.18557237)
\curveto(729.1260199,929.18557237)(729.69546391,929.24112789)(730.18157466,929.35223891)
\curveto(730.6676854,929.47029438)(731.08087953,929.68557199)(731.42115705,929.99807176)
\curveto(731.73365681,930.2897382)(731.96282331,930.62307128)(732.10865653,930.998071)
\curveto(732.26143419,931.38001516)(732.33782302,931.86265368)(732.33782302,932.44598657)
\closepath
}
}
{
\newrgbcolor{curcolor}{0 0 0}
\pscustom[linestyle=none,fillstyle=solid,fillcolor=curcolor]
{
\newpath
\moveto(751.13990079,922.42516082)
\curveto(750.75795664,922.25849428)(750.41073468,922.1022444)(750.09823491,921.95641117)
\curveto(749.79267959,921.81057795)(749.38990212,921.65780029)(748.88990249,921.49807819)
\curveto(748.4662917,921.36613384)(748.0044865,921.25502282)(747.50448688,921.16474511)
\curveto(747.01143169,921.06752296)(746.46629322,921.01891188)(745.86907145,921.01891188)
\curveto(744.7440723,921.01891188)(743.71976752,921.17516177)(742.7961571,921.48766153)
\curveto(741.87949113,921.80710573)(741.08088062,922.30363313)(740.40032558,922.97724374)
\curveto(739.73365942,923.63696546)(739.21282648,924.47377038)(738.83782677,925.4876585)
\curveto(738.46282705,926.50849106)(738.27532719,927.69251795)(738.27532719,929.03973915)
\curveto(738.27532719,930.31751596)(738.45588261,931.45987621)(738.81699345,932.46681989)
\curveto(739.17810429,933.47376357)(739.69893723,934.32445737)(740.37949227,935.01890129)
\curveto(741.03921399,935.69251189)(741.83435228,936.20640039)(742.76490713,936.56056679)
\curveto(743.70240642,936.91473319)(744.74060008,937.09181639)(745.87948811,937.09181639)
\curveto(746.71282081,937.09181639)(747.54268129,936.99112202)(748.36906956,936.78973329)
\curveto(749.20240226,936.58834455)(750.12601267,936.23417815)(751.13990079,935.72723409)
\lineto(751.13990079,933.27931928)
\lineto(750.98365091,933.27931928)
\curveto(750.12948489,933.99459651)(749.28226331,934.51542945)(748.44198617,934.84181809)
\curveto(747.60170902,935.16820673)(746.70240415,935.33140106)(745.74407154,935.33140106)
\curveto(744.95934991,935.33140106)(744.25101712,935.20292893)(743.61907315,934.94598468)
\curveto(742.99407362,934.69598487)(742.43504627,934.30362406)(741.94199108,933.76890224)
\curveto(741.46282478,933.2480693)(741.08782506,932.58834758)(740.81699194,931.78973707)
\curveto(740.55310325,930.998071)(740.4211589,930.08140503)(740.4211589,929.03973915)
\curveto(740.4211589,927.9494622)(740.56699212,927.01196291)(740.85865857,926.22724128)
\curveto(741.15726946,925.44251965)(741.53921361,924.80363124)(742.00449104,924.31057606)
\curveto(742.49060178,923.79668756)(743.05657357,923.4147434)(743.70240642,923.16474359)
\curveto(744.3551837,922.92168822)(745.04268318,922.80016054)(745.76490486,922.80016054)
\curveto(746.75795966,922.80016054)(747.68851451,922.9702993)(748.55656941,923.31057682)
\curveto(749.42462431,923.65085434)(750.2371237,924.16127062)(750.99406757,924.84182566)
\lineto(751.13990079,924.84182566)
\closepath
}
}
{
\newrgbcolor{curcolor}{0 0 0}
\pscustom[linestyle=none,fillstyle=solid,fillcolor=curcolor]
{
\newpath
\moveto(386.97351281,900.47726076)
\lineto(380.47351773,900.47726076)
\lineto(380.47351773,902.362676)
\lineto(386.97351281,902.362676)
\closepath
}
}
{
\newrgbcolor{curcolor}{0 0 0}
\pscustom[linestyle=none,fillstyle=solid,fillcolor=curcolor]
{
\newpath
\moveto(409.71307918,908.36267146)
\curveto(410.34502314,907.66822754)(410.82766167,906.81753374)(411.16099475,905.81059006)
\curveto(411.50127227,904.80364637)(411.67141103,903.66128613)(411.67141103,902.38350932)
\curveto(411.67141103,901.10573251)(411.49780005,899.95990004)(411.15057809,898.94601192)
\curveto(410.81030057,897.93906823)(410.33113426,897.09879109)(409.71307918,896.42518049)
\curveto(409.07419077,895.72379213)(408.3172469,895.19601475)(407.44224756,894.84184836)
\curveto(406.57419266,894.48768196)(405.58113786,894.31059876)(404.46308315,894.31059876)
\curveto(403.37280619,894.31059876)(402.37975139,894.49115418)(401.48391874,894.85226501)
\curveto(400.59503052,895.21337585)(399.83808665,895.73768101)(399.21308712,896.42518049)
\curveto(398.58808759,897.11267997)(398.10544907,897.95642933)(397.76517155,898.95642858)
\curveto(397.43183847,899.95642782)(397.26517193,901.09878807)(397.26517193,902.38350932)
\curveto(397.26517193,903.64739725)(397.43183847,904.77934084)(397.76517155,905.77934008)
\curveto(398.09850463,906.78628376)(398.58461537,907.64739422)(399.22350378,908.36267146)
\curveto(399.83461443,909.0432265)(400.5915583,909.56405944)(401.49433539,909.92517028)
\curveto(402.40405693,910.28628112)(403.39363951,910.46683653)(404.46308315,910.46683653)
\curveto(405.57419342,910.46683653)(406.57072044,910.2828089)(407.45266422,909.91475362)
\curveto(408.34155244,909.55364278)(409.09502409,909.03628206)(409.71307918,908.36267146)
\closepath
\moveto(409.52557932,902.38350932)
\curveto(409.52557932,904.39739668)(409.07419077,905.94947884)(408.17141368,907.03975579)
\curveto(407.26863658,908.13697719)(406.03599862,908.68558788)(404.47349981,908.68558788)
\curveto(402.89711211,908.68558788)(401.65752971,908.13697719)(400.75475262,907.03975579)
\curveto(399.85891996,905.94947884)(399.41100364,904.39739668)(399.41100364,902.38350932)
\curveto(399.41100364,900.34878863)(399.86933662,898.78976204)(400.7860026,897.70642952)
\curveto(401.70266857,896.63004145)(402.93183431,896.09184741)(404.47349981,896.09184741)
\curveto(406.01516531,896.09184741)(407.24085882,896.63004145)(408.15058036,897.70642952)
\curveto(409.06724633,898.78976204)(409.52557932,900.34878863)(409.52557932,902.38350932)
\closepath
}
}
{
\newrgbcolor{curcolor}{0 0 0}
\pscustom[linestyle=none,fillstyle=solid,fillcolor=curcolor]
{
\newpath
\moveto(424.43181685,894.63351518)
\lineto(422.473485,894.63351518)
\lineto(422.473485,895.92518087)
\curveto(421.81376327,895.40434793)(421.18181931,895.00504268)(420.5776531,894.72726511)
\curveto(419.97348689,894.44948754)(419.30682072,894.31059876)(418.57765461,894.31059876)
\curveto(417.35543331,894.31059876)(416.40404514,894.68212625)(415.7234901,895.42518125)
\curveto(415.04293506,896.17518068)(414.70265754,897.27240207)(414.70265754,898.71684542)
\lineto(414.70265754,906.26892304)
\lineto(416.66098939,906.26892304)
\lineto(416.66098939,899.64392806)
\curveto(416.66098939,899.05365072)(416.68876715,898.54670666)(416.74432266,898.12309587)
\curveto(416.79987818,897.70642952)(416.91793364,897.3487909)(417.09848906,897.05018002)
\curveto(417.28598892,896.74462469)(417.52904429,896.52240264)(417.82765518,896.38351386)
\curveto(418.12626606,896.24462507)(418.56029351,896.17518068)(419.12973753,896.17518068)
\curveto(419.63668159,896.17518068)(420.1887645,896.30712502)(420.78598627,896.57101371)
\curveto(421.39015248,896.8349024)(421.95265206,897.1717077)(422.473485,897.58142962)
\lineto(422.473485,906.26892304)
\lineto(424.43181685,906.26892304)
\closepath
}
}
{
\newrgbcolor{curcolor}{0 0 0}
\pscustom[linestyle=none,fillstyle=solid,fillcolor=curcolor]
{
\newpath
\moveto(434.32764426,894.73768177)
\curveto(433.95958898,894.64045962)(433.55681151,894.56059857)(433.11931184,894.49809862)
\curveto(432.68875661,894.43559866)(432.30334023,894.40434869)(431.96306271,894.40434869)
\curveto(430.77556361,894.40434869)(429.87278652,894.72379289)(429.25473143,895.36268129)
\curveto(428.63667634,896.0015697)(428.3276488,897.02587448)(428.3276488,898.43559564)
\lineto(428.3276488,904.62309096)
\lineto(427.00473313,904.62309096)
\lineto(427.00473313,906.26892304)
\lineto(428.3276488,906.26892304)
\lineto(428.3276488,909.61267051)
\lineto(430.28598065,909.61267051)
\lineto(430.28598065,906.26892304)
\lineto(434.32764426,906.26892304)
\lineto(434.32764426,904.62309096)
\lineto(430.28598065,904.62309096)
\lineto(430.28598065,899.32101163)
\curveto(430.28598065,898.70990098)(430.29986953,898.23073468)(430.32764728,897.88351272)
\curveto(430.35542504,897.5432352)(430.45264719,897.223791)(430.61931373,896.92518011)
\curveto(430.77209139,896.64740254)(430.98042457,896.44254159)(431.24431326,896.31059724)
\curveto(431.51514638,896.18559734)(431.9248683,896.12309739)(432.47347899,896.12309739)
\curveto(432.7929232,896.12309739)(433.12625628,896.16823624)(433.47347824,896.25851395)
\curveto(433.8207002,896.3557361)(434.07070001,896.43559715)(434.22347767,896.4980971)
\lineto(434.32764426,896.4980971)
\closepath
}
}
{
\newrgbcolor{curcolor}{0 0 0}
\pscustom[linestyle=none,fillstyle=solid,fillcolor=curcolor]
{
\newpath
\moveto(446.93180006,900.59184401)
\curveto(446.93180006,899.64740028)(446.79638349,898.7828176)(446.52555037,897.99809597)
\curveto(446.25471724,897.22031878)(445.87277308,896.56059705)(445.3797179,896.0189308)
\curveto(444.92138491,895.5050423)(444.37971866,895.10573704)(443.75471913,894.82101504)
\curveto(443.13666404,894.54323747)(442.48041454,894.40434869)(441.78597062,894.40434869)
\curveto(441.18180441,894.40434869)(440.63319371,894.47032086)(440.14013853,894.6022652)
\curveto(439.65402779,894.73420955)(439.15750038,894.9390705)(438.65055632,895.21684807)
\lineto(438.65055632,890.34185176)
\lineto(436.69222447,890.34185176)
\lineto(436.69222447,906.26892304)
\lineto(438.65055632,906.26892304)
\lineto(438.65055632,905.05017397)
\curveto(439.17138926,905.48767363)(439.75472215,905.85225669)(440.400555,906.14392314)
\curveto(441.05333228,906.44253402)(441.7477762,906.59183947)(442.48388676,906.59183947)
\curveto(443.88666347,906.59183947)(444.97694043,906.06058987)(445.75471761,904.99809067)
\curveto(446.53943924,903.94253591)(446.93180006,902.47378703)(446.93180006,900.59184401)
\closepath
\moveto(444.91096825,900.53976071)
\curveto(444.91096825,901.94253743)(444.6713851,902.99114775)(444.1922188,903.68559166)
\curveto(443.71305249,904.38003558)(442.97694194,904.72725754)(441.98388713,904.72725754)
\curveto(441.42138756,904.72725754)(440.85541577,904.60572986)(440.28597175,904.36267449)
\curveto(439.71652774,904.11961911)(439.17138926,903.80017491)(438.65055632,903.40434188)
\lineto(438.65055632,896.81059687)
\curveto(439.20611146,896.56059705)(439.68180554,896.39045829)(440.07763858,896.30018059)
\curveto(440.48041605,896.20990288)(440.93527682,896.16476402)(441.44222088,896.16476402)
\curveto(442.53249783,896.16476402)(443.38319163,896.5328193)(443.99430228,897.26892985)
\curveto(444.60541293,898.00504041)(444.91096825,899.09531736)(444.91096825,900.53976071)
\closepath
}
}
{
\newrgbcolor{curcolor}{0 0 0}
\pscustom[linestyle=none,fillstyle=solid,fillcolor=curcolor]
{
\newpath
\moveto(459.62970532,894.63351518)
\lineto(457.67137347,894.63351518)
\lineto(457.67137347,895.92518087)
\curveto(457.01165174,895.40434793)(456.37970778,895.00504268)(455.77554157,894.72726511)
\curveto(455.17137536,894.44948754)(454.5047092,894.31059876)(453.77554308,894.31059876)
\curveto(452.55332178,894.31059876)(451.60193361,894.68212625)(450.92137857,895.42518125)
\curveto(450.24082353,896.17518068)(449.90054601,897.27240207)(449.90054601,898.71684542)
\lineto(449.90054601,906.26892304)
\lineto(451.85887786,906.26892304)
\lineto(451.85887786,899.64392806)
\curveto(451.85887786,899.05365072)(451.88665562,898.54670666)(451.94221114,898.12309587)
\curveto(451.99776665,897.70642952)(452.11582211,897.3487909)(452.29637753,897.05018002)
\curveto(452.48387739,896.74462469)(452.72693276,896.52240264)(453.02554365,896.38351386)
\curveto(453.32415453,896.24462507)(453.75818198,896.17518068)(454.327626,896.17518068)
\curveto(454.83457006,896.17518068)(455.38665297,896.30712502)(455.98387474,896.57101371)
\curveto(456.58804095,896.8349024)(457.15054053,897.1717077)(457.67137347,897.58142962)
\lineto(457.67137347,906.26892304)
\lineto(459.62970532,906.26892304)
\closepath
}
}
{
\newrgbcolor{curcolor}{0 0 0}
\pscustom[linestyle=none,fillstyle=solid,fillcolor=curcolor]
{
\newpath
\moveto(469.52553273,894.73768177)
\curveto(469.15747745,894.64045962)(468.75469998,894.56059857)(468.31720031,894.49809862)
\curveto(467.88664508,894.43559866)(467.5012287,894.40434869)(467.16095118,894.40434869)
\curveto(465.97345208,894.40434869)(465.07067499,894.72379289)(464.4526199,895.36268129)
\curveto(463.83456481,896.0015697)(463.52553727,897.02587448)(463.52553727,898.43559564)
\lineto(463.52553727,904.62309096)
\lineto(462.2026216,904.62309096)
\lineto(462.2026216,906.26892304)
\lineto(463.52553727,906.26892304)
\lineto(463.52553727,909.61267051)
\lineto(465.48386912,909.61267051)
\lineto(465.48386912,906.26892304)
\lineto(469.52553273,906.26892304)
\lineto(469.52553273,904.62309096)
\lineto(465.48386912,904.62309096)
\lineto(465.48386912,899.32101163)
\curveto(465.48386912,898.70990098)(465.497758,898.23073468)(465.52553575,897.88351272)
\curveto(465.55331351,897.5432352)(465.65053566,897.223791)(465.8172022,896.92518011)
\curveto(465.96997986,896.64740254)(466.17831304,896.44254159)(466.44220173,896.31059724)
\curveto(466.71303486,896.18559734)(467.12275677,896.12309739)(467.67136746,896.12309739)
\curveto(467.99081167,896.12309739)(468.32414475,896.16823624)(468.67136671,896.25851395)
\curveto(469.01858867,896.3557361)(469.26858848,896.43559715)(469.42136614,896.4980971)
\lineto(469.52553273,896.4980971)
\closepath
}
}
{
\newrgbcolor{curcolor}{0 0 0}
\pscustom[linestyle=none,fillstyle=solid,fillcolor=curcolor]
{
\newpath
\moveto(480.20260665,897.98767931)
\curveto(480.20260665,896.92518011)(479.76163477,896.05365299)(478.87969099,895.37309795)
\curveto(478.00469165,894.69254291)(476.80677589,894.35226539)(475.28594371,894.35226539)
\curveto(474.42483325,894.35226539)(473.63316718,894.45295976)(472.9109455,894.6543485)
\curveto(472.19566827,894.86268167)(471.59497428,895.08837595)(471.10886353,895.33143132)
\lineto(471.10886353,897.52934632)
\lineto(471.21303012,897.52934632)
\curveto(471.83108521,897.0640689)(472.51858469,896.6925414)(473.27552856,896.41476383)
\curveto(474.03247243,896.1439307)(474.75816633,896.00851414)(475.45261025,896.00851414)
\curveto(476.31372071,896.00851414)(476.98733131,896.14740292)(477.47344205,896.42518049)
\curveto(477.9595528,896.70295806)(478.20260817,897.14045773)(478.20260817,897.7376795)
\curveto(478.20260817,898.19601248)(478.07066382,898.54323444)(477.80677513,898.77934538)
\curveto(477.54288644,899.01545631)(477.03594238,899.21684505)(476.28594295,899.38351159)
\curveto(476.00816538,899.44601154)(475.64358233,899.51892815)(475.19219378,899.60226142)
\curveto(474.74774967,899.68559469)(474.34149998,899.7758724)(473.9734447,899.87309455)
\curveto(472.95261214,900.14392768)(472.22691824,900.53976071)(471.79636301,901.06059365)
\curveto(471.37275222,901.58837103)(471.16094683,902.23420387)(471.16094683,902.99809218)
\curveto(471.16094683,903.47725849)(471.25816898,903.92864704)(471.45261327,904.35225783)
\curveto(471.65400201,904.77586862)(471.95608512,905.15434055)(472.35886259,905.48767363)
\curveto(472.74775118,905.81406228)(473.24080637,906.07100653)(473.83802814,906.25850638)
\curveto(474.44219435,906.45295068)(475.11580495,906.55017283)(475.85885994,906.55017283)
\curveto(476.55330386,906.55017283)(477.25469222,906.46336734)(477.96302502,906.28975636)
\curveto(478.67830225,906.12308982)(479.2720518,905.91822886)(479.74427367,905.67517349)
\lineto(479.74427367,903.58142508)
\lineto(479.64010708,903.58142508)
\curveto(479.14010746,903.94948035)(478.53246903,904.2585079)(477.81719179,904.50850771)
\curveto(477.10191456,904.76545196)(476.4005262,904.89392408)(475.71302672,904.89392408)
\curveto(474.99774948,904.89392408)(474.39358327,904.7550353)(473.90052809,904.47725773)
\curveto(473.40747291,904.2064246)(473.16094532,903.80017491)(473.16094532,903.25850865)
\curveto(473.16094532,902.77934235)(473.31025076,902.41823151)(473.60886164,902.17517614)
\curveto(473.90052809,901.93212077)(474.37274995,901.73420425)(475.02552724,901.58142659)
\curveto(475.38663808,901.49809332)(475.78941555,901.41476005)(476.23385966,901.33142678)
\curveto(476.6852482,901.24809351)(477.06024792,901.17170468)(477.35885881,901.10226029)
\curveto(478.26858034,900.89392711)(478.9699687,900.53628849)(479.46302388,900.02934443)
\curveto(479.95607906,899.51545593)(480.20260665,898.83490089)(480.20260665,897.98767931)
\closepath
}
}
{
\newrgbcolor{curcolor}{0 0 0}
\pscustom[linestyle=none,fillstyle=solid,fillcolor=curcolor]
{
\newpath
\moveto(496.54634908,894.73768177)
\curveto(496.1782938,894.64045962)(495.77551633,894.56059857)(495.33801666,894.49809862)
\curveto(494.90746143,894.43559866)(494.52204506,894.40434869)(494.18176754,894.40434869)
\curveto(492.99426843,894.40434869)(492.09149134,894.72379289)(491.47343625,895.36268129)
\curveto(490.85538116,896.0015697)(490.54635362,897.02587448)(490.54635362,898.43559564)
\lineto(490.54635362,904.62309096)
\lineto(489.22343795,904.62309096)
\lineto(489.22343795,906.26892304)
\lineto(490.54635362,906.26892304)
\lineto(490.54635362,909.61267051)
\lineto(492.50468547,909.61267051)
\lineto(492.50468547,906.26892304)
\lineto(496.54634908,906.26892304)
\lineto(496.54634908,904.62309096)
\lineto(492.50468547,904.62309096)
\lineto(492.50468547,899.32101163)
\curveto(492.50468547,898.70990098)(492.51857435,898.23073468)(492.54635211,897.88351272)
\curveto(492.57412986,897.5432352)(492.67135201,897.223791)(492.83801855,896.92518011)
\curveto(492.99079622,896.64740254)(493.19912939,896.44254159)(493.46301808,896.31059724)
\curveto(493.73385121,896.18559734)(494.14357312,896.12309739)(494.69218382,896.12309739)
\curveto(495.01162802,896.12309739)(495.3449611,896.16823624)(495.69218306,896.25851395)
\curveto(496.03940502,896.3557361)(496.28940483,896.43559715)(496.44218249,896.4980971)
\lineto(496.54634908,896.4980971)
\closepath
}
}
{
\newrgbcolor{curcolor}{0 0 0}
\pscustom[linestyle=none,fillstyle=solid,fillcolor=curcolor]
{
\newpath
\moveto(508.82758125,900.44601078)
\curveto(508.82758125,898.55017888)(508.34147051,897.05365224)(507.36924902,895.95643085)
\curveto(506.39702753,894.85920945)(505.09494519,894.31059876)(503.46300198,894.31059876)
\curveto(501.81716989,894.31059876)(500.5081431,894.85920945)(499.53592161,895.95643085)
\curveto(498.57064457,897.05365224)(498.08800604,898.55017888)(498.08800604,900.44601078)
\curveto(498.08800604,902.34184268)(498.57064457,903.83836933)(499.53592161,904.93559072)
\curveto(500.5081431,906.03975655)(501.81716989,906.59183947)(503.46300198,906.59183947)
\curveto(505.09494519,906.59183947)(506.39702753,906.03975655)(507.36924902,904.93559072)
\curveto(508.34147051,903.83836933)(508.82758125,902.34184268)(508.82758125,900.44601078)
\closepath
\moveto(506.80674945,900.44601078)
\curveto(506.80674945,901.95295409)(506.51161078,903.0710088)(505.92133345,903.80017491)
\curveto(505.33105612,904.53628547)(504.51161229,904.90434074)(503.46300198,904.90434074)
\curveto(502.40050278,904.90434074)(501.57411452,904.53628547)(500.98383719,903.80017491)
\curveto(500.40050429,903.0710088)(500.10883785,901.95295409)(500.10883785,900.44601078)
\curveto(500.10883785,898.98767855)(500.40397651,897.8800405)(500.99425384,897.12309663)
\curveto(501.58453118,896.3730972)(502.40744722,895.99809748)(503.46300198,895.99809748)
\curveto(504.50466785,895.99809748)(505.32063946,896.36962498)(505.91091679,897.11267997)
\curveto(506.50813856,897.8626794)(506.80674945,898.97378967)(506.80674945,900.44601078)
\closepath
}
}
{
\newrgbcolor{curcolor}{0 0 0}
\pscustom[linestyle=none,fillstyle=solid,fillcolor=curcolor]
{
\newpath
\moveto(528.40048865,894.63351518)
\lineto(526.45257345,894.63351518)
\lineto(526.45257345,895.87309758)
\curveto(526.27896247,895.75504211)(526.04285154,895.58837557)(525.74424066,895.37309795)
\curveto(525.45257421,895.16476478)(525.1678522,894.99809824)(524.89007464,894.87309833)
\curveto(524.56368599,894.71337623)(524.18868628,894.58143189)(523.76507549,894.4772653)
\curveto(523.3414647,894.36615427)(522.84493729,894.31059876)(522.27549328,894.31059876)
\curveto(521.22688296,894.31059876)(520.33799475,894.65782072)(519.60882863,895.35226464)
\curveto(518.87966252,896.04670855)(518.51507946,896.93212455)(518.51507946,898.00851263)
\curveto(518.51507946,898.8904564)(518.70257932,899.60226142)(519.07757903,900.14392768)
\curveto(519.45952319,900.69253837)(520.00118945,901.1230936)(520.7025778,901.43559337)
\curveto(521.4109106,901.74809313)(522.2616044,901.95989853)(523.25465921,902.07100955)
\curveto(524.24771401,902.18212058)(525.31368543,902.26545385)(526.45257345,902.32100936)
\lineto(526.45257345,902.62309247)
\curveto(526.45257345,903.06753658)(526.3727124,903.43559185)(526.2129903,903.7272583)
\curveto(526.06021264,904.01892475)(525.83799059,904.24809124)(525.54632414,904.41475778)
\curveto(525.26854657,904.57447988)(524.93521349,904.68211869)(524.5463249,904.7376742)
\curveto(524.1574363,904.79322972)(523.75118661,904.82100747)(523.32757582,904.82100747)
\curveto(522.81368732,904.82100747)(522.24077109,904.75156308)(521.60882712,904.6126743)
\curveto(520.97688315,904.48072995)(520.32410587,904.28628565)(519.65049527,904.0293414)
\lineto(519.54632868,904.0293414)
\lineto(519.54632868,906.01892323)
\curveto(519.92827283,906.12308982)(520.48035575,906.23767307)(521.20257743,906.36267297)
\curveto(521.9247991,906.48767288)(522.63660412,906.55017283)(523.33799248,906.55017283)
\curveto(524.1574363,906.55017283)(524.86924132,906.48072844)(525.47340753,906.34183965)
\curveto(526.08451818,906.20989531)(526.61229556,905.98072882)(527.05673966,905.65434017)
\curveto(527.49423933,905.33489597)(527.82757241,904.92170184)(528.05673891,904.41475778)
\curveto(528.2859054,903.90781372)(528.40048865,903.27934197)(528.40048865,902.52934254)
\closepath
\moveto(526.45257345,897.49809635)
\lineto(526.45257345,900.73767723)
\curveto(525.85535168,900.70295503)(525.15049111,900.65087174)(524.33799172,900.58142735)
\curveto(523.53243677,900.51198295)(522.89354837,900.41128859)(522.4213265,900.27934424)
\curveto(521.85882693,900.11962214)(521.40396616,899.86962233)(521.0567442,899.52934481)
\curveto(520.70952224,899.19601173)(520.53591126,898.73420652)(520.53591126,898.14392919)
\curveto(520.53591126,897.47726303)(520.7373,896.97379119)(521.14007747,896.63351367)
\curveto(521.54285495,896.30018059)(522.15743781,896.13351404)(522.98382608,896.13351404)
\curveto(523.67132556,896.13351404)(524.29979731,896.26545839)(524.86924132,896.52934708)
\curveto(525.43868533,896.80018021)(525.96646271,897.12309663)(526.45257345,897.49809635)
\closepath
}
}
{
\newrgbcolor{curcolor}{0 0 0}
\pscustom[linestyle=none,fillstyle=solid,fillcolor=curcolor]
{
\newpath
\moveto(547.98380331,897.98767931)
\curveto(547.98380331,896.92518011)(547.54283142,896.05365299)(546.66088764,895.37309795)
\curveto(545.7858883,894.69254291)(544.58797254,894.35226539)(543.06714036,894.35226539)
\curveto(542.2060299,894.35226539)(541.41436383,894.45295976)(540.69214216,894.6543485)
\curveto(539.97686492,894.86268167)(539.37617093,895.08837595)(538.89006019,895.33143132)
\lineto(538.89006019,897.52934632)
\lineto(538.99422678,897.52934632)
\curveto(539.61228186,897.0640689)(540.29978134,896.6925414)(541.05672522,896.41476383)
\curveto(541.81366909,896.1439307)(542.53936298,896.00851414)(543.2338069,896.00851414)
\curveto(544.09491736,896.00851414)(544.76852796,896.14740292)(545.25463871,896.42518049)
\curveto(545.74074945,896.70295806)(545.98380482,897.14045773)(545.98380482,897.7376795)
\curveto(545.98380482,898.19601248)(545.85186048,898.54323444)(545.58797179,898.77934538)
\curveto(545.3240831,899.01545631)(544.81713904,899.21684505)(544.06713961,899.38351159)
\curveto(543.78936204,899.44601154)(543.42477898,899.51892815)(542.97339043,899.60226142)
\curveto(542.52894632,899.68559469)(542.12269663,899.7758724)(541.75464135,899.87309455)
\curveto(540.73380879,900.14392768)(540.0081149,900.53976071)(539.57755967,901.06059365)
\curveto(539.15394888,901.58837103)(538.94214348,902.23420387)(538.94214348,902.99809218)
\curveto(538.94214348,903.47725849)(539.03936563,903.92864704)(539.23380993,904.35225783)
\curveto(539.43519866,904.77586862)(539.73728177,905.15434055)(540.14005924,905.48767363)
\curveto(540.52894784,905.81406228)(541.02200302,906.07100653)(541.61922479,906.25850638)
\curveto(542.223391,906.45295068)(542.8970016,906.55017283)(543.64005659,906.55017283)
\curveto(544.33450051,906.55017283)(545.03588887,906.46336734)(545.74422167,906.28975636)
\curveto(546.45949891,906.12308982)(547.05324846,905.91822886)(547.52547032,905.67517349)
\lineto(547.52547032,903.58142508)
\lineto(547.42130373,903.58142508)
\curveto(546.92130411,903.94948035)(546.31366568,904.2585079)(545.59838845,904.50850771)
\curveto(544.88311121,904.76545196)(544.18172285,904.89392408)(543.49422337,904.89392408)
\curveto(542.77894614,904.89392408)(542.17477993,904.7550353)(541.68172474,904.47725773)
\curveto(541.18866956,904.2064246)(540.94214197,903.80017491)(540.94214197,903.25850865)
\curveto(540.94214197,902.77934235)(541.09144741,902.41823151)(541.3900583,902.17517614)
\curveto(541.68172474,901.93212077)(542.15394661,901.73420425)(542.80672389,901.58142659)
\curveto(543.16783473,901.49809332)(543.5706122,901.41476005)(544.01505631,901.33142678)
\curveto(544.46644486,901.24809351)(544.84144457,901.17170468)(545.14005546,901.10226029)
\curveto(546.04977699,900.89392711)(546.75116535,900.53628849)(547.24422053,900.02934443)
\curveto(547.73727572,899.51545593)(547.98380331,898.83490089)(547.98380331,897.98767931)
\closepath
}
}
{
\newrgbcolor{curcolor}{0 0 0}
\pscustom[linestyle=none,fillstyle=solid,fillcolor=curcolor]
{
\newpath
\moveto(561.02545923,900.59184401)
\curveto(561.02545923,899.64740028)(560.89004266,898.7828176)(560.61920953,897.99809597)
\curveto(560.3483764,897.22031878)(559.96643225,896.56059705)(559.47337707,896.0189308)
\curveto(559.01504408,895.5050423)(558.47337782,895.10573704)(557.8483783,894.82101504)
\curveto(557.23032321,894.54323747)(556.5740737,894.40434869)(555.87962979,894.40434869)
\curveto(555.27546358,894.40434869)(554.72685288,894.47032086)(554.2337977,894.6022652)
\curveto(553.74768695,894.73420955)(553.25115955,894.9390705)(552.74421549,895.21684807)
\lineto(552.74421549,890.34185176)
\lineto(550.78588364,890.34185176)
\lineto(550.78588364,906.26892304)
\lineto(552.74421549,906.26892304)
\lineto(552.74421549,905.05017397)
\curveto(553.26504843,905.48767363)(553.84838132,905.85225669)(554.49421417,906.14392314)
\curveto(555.14699145,906.44253402)(555.84143537,906.59183947)(556.57754592,906.59183947)
\curveto(557.98032264,906.59183947)(559.07059959,906.06058987)(559.84837678,904.99809067)
\curveto(560.63309841,903.94253591)(561.02545923,902.47378703)(561.02545923,900.59184401)
\closepath
\moveto(559.00462742,900.53976071)
\curveto(559.00462742,901.94253743)(558.76504427,902.99114775)(558.28587797,903.68559166)
\curveto(557.80671166,904.38003558)(557.07060111,904.72725754)(556.0775463,904.72725754)
\curveto(555.51504673,904.72725754)(554.94907493,904.60572986)(554.37963092,904.36267449)
\curveto(553.81018691,904.11961911)(553.26504843,903.80017491)(552.74421549,903.40434188)
\lineto(552.74421549,896.81059687)
\curveto(553.29977063,896.56059705)(553.77546471,896.39045829)(554.17129774,896.30018059)
\curveto(554.57407522,896.20990288)(555.02893598,896.16476402)(555.53588005,896.16476402)
\curveto(556.626157,896.16476402)(557.4768508,896.5328193)(558.08796145,897.26892985)
\curveto(558.6990721,898.00504041)(559.00462742,899.09531736)(559.00462742,900.53976071)
\closepath
}
}
{
\newrgbcolor{curcolor}{0 0 0}
\pscustom[linestyle=none,fillstyle=solid,fillcolor=curcolor]
{
\newpath
\moveto(573.81711442,900.24809427)
\lineto(565.24420424,900.24809427)
\curveto(565.24420424,899.53281703)(565.35184304,898.9078175)(565.56712066,898.37309568)
\curveto(565.78239827,897.84531831)(566.07753694,897.41129086)(566.45253665,897.07101334)
\curveto(566.81364749,896.73768025)(567.2407305,896.48768044)(567.73378569,896.3210139)
\curveto(568.23378531,896.15434736)(568.782396,896.07101409)(569.37961777,896.07101409)
\curveto(570.17128384,896.07101409)(570.96642213,896.22726397)(571.76503264,896.53976374)
\curveto(572.57058758,896.85920794)(573.14350381,897.1717077)(573.48378134,897.47726303)
\lineto(573.58794792,897.47726303)
\lineto(573.58794792,895.34184798)
\curveto(572.9282262,895.06407041)(572.2546156,894.8314317)(571.56711612,894.64393184)
\curveto(570.87961664,894.45643198)(570.15739496,894.36268205)(569.40045109,894.36268205)
\curveto(567.469897,894.36268205)(565.96295369,894.88351499)(564.87962118,895.92518087)
\curveto(563.79628866,896.97379119)(563.25462241,898.45990117)(563.25462241,900.38351083)
\curveto(563.25462241,902.28628717)(563.77198313,903.79670269)(564.80670457,904.9147574)
\curveto(565.84837045,906.03281211)(567.21642497,906.59183947)(568.91086813,906.59183947)
\curveto(570.48031139,906.59183947)(571.6886438,906.13350648)(572.53586539,905.21684051)
\curveto(573.39003141,904.30017453)(573.81711442,902.99809218)(573.81711442,901.31059346)
\closepath
\moveto(571.91086586,901.74809313)
\curveto(571.90392142,902.77587013)(571.64350495,903.57100842)(571.12961645,904.13350799)
\curveto(570.62267239,904.69600757)(569.84836742,904.97725735)(568.80670154,904.97725735)
\curveto(567.75809122,904.97725735)(566.9212863,904.66822981)(566.29628677,904.05017472)
\curveto(565.67823168,903.43211963)(565.32753751,902.6647591)(565.24420424,901.74809313)
\closepath
}
}
{
\newrgbcolor{curcolor}{0 0 0}
\pscustom[linestyle=none,fillstyle=solid,fillcolor=curcolor]
{
\newpath
\moveto(585.39002553,895.36268129)
\curveto(584.73724825,895.05018153)(584.11572094,894.80712616)(583.52544361,894.63351518)
\curveto(582.94211072,894.4599042)(582.32058341,894.37309871)(581.66086169,894.37309871)
\curveto(580.82058455,894.37309871)(580.0497518,894.4946264)(579.34836344,894.73768177)
\curveto(578.64697508,894.98768158)(578.04628109,895.36268129)(577.54628147,895.86268092)
\curveto(577.03933741,896.36268054)(576.64697659,896.9946245)(576.36919903,897.75851282)
\curveto(576.09142146,898.52240113)(575.95253267,899.41476156)(575.95253267,900.43559412)
\curveto(575.95253267,902.33837046)(576.47336561,903.83142489)(577.51503149,904.9147574)
\curveto(578.56364181,905.99808991)(579.94558521,906.53975617)(581.66086169,906.53975617)
\curveto(582.32752785,906.53975617)(582.98030513,906.44600624)(583.61919354,906.25850638)
\curveto(584.26502639,906.07100653)(584.85530372,905.84184003)(585.39002553,905.5710069)
\lineto(585.39002553,903.39392522)
\lineto(585.28585895,903.39392522)
\curveto(584.68863718,903.85920264)(584.07058209,904.21684126)(583.43169368,904.46684107)
\curveto(582.79974972,904.71684088)(582.18169463,904.84184079)(581.57752842,904.84184079)
\curveto(580.46641815,904.84184079)(579.58794659,904.46684107)(578.94211375,903.71684164)
\curveto(578.30322534,902.97378665)(577.98378114,901.88003747)(577.98378114,900.43559412)
\curveto(577.98378114,899.03281741)(578.2962809,897.95295711)(578.92128043,897.19601324)
\curveto(579.55322439,896.44601381)(580.43864039,896.07101409)(581.57752842,896.07101409)
\curveto(581.97336145,896.07101409)(582.37613893,896.12309739)(582.78586084,896.22726397)
\curveto(583.19558275,896.33143056)(583.56363803,896.46684713)(583.89002667,896.63351367)
\curveto(584.17474868,896.77934689)(584.44210958,896.93212455)(584.6921094,897.09184665)
\curveto(584.94210921,897.25851319)(585.14002572,897.4008742)(585.28585895,897.51892966)
\lineto(585.39002553,897.51892966)
\closepath
}
}
{
\newrgbcolor{curcolor}{0 0 0}
\pscustom[linestyle=none,fillstyle=solid,fillcolor=curcolor]
{
\newpath
\moveto(590.00460449,908.21683824)
\lineto(587.79627283,908.21683824)
\lineto(587.79627283,910.2480867)
\lineto(590.00460449,910.2480867)
\closepath
\moveto(589.87960459,894.63351518)
\lineto(587.92127274,894.63351518)
\lineto(587.92127274,906.26892304)
\lineto(589.87960459,906.26892304)
\closepath
}
}
{
\newrgbcolor{curcolor}{0 0 0}
\pscustom[linestyle=none,fillstyle=solid,fillcolor=curcolor]
{
\newpath
\moveto(600.01502154,908.94600435)
\lineto(599.91085495,908.94600435)
\curveto(599.69557734,909.0085043)(599.41432755,909.07100426)(599.06710559,909.13350421)
\curveto(598.71988363,909.2029486)(598.41432831,909.2376708)(598.15043962,909.2376708)
\curveto(597.31016248,909.2376708)(596.69905183,909.05017094)(596.31710767,908.67517122)
\curveto(595.94210795,908.30711595)(595.7546081,907.63697756)(595.7546081,906.66475608)
\lineto(595.7546081,906.26892304)
\lineto(599.28585543,906.26892304)
\lineto(599.28585543,904.62309096)
\lineto(595.81710805,904.62309096)
\lineto(595.81710805,894.63351518)
\lineto(593.8587762,894.63351518)
\lineto(593.8587762,904.62309096)
\lineto(592.53586053,904.62309096)
\lineto(592.53586053,906.26892304)
\lineto(593.8587762,906.26892304)
\lineto(593.8587762,906.65433942)
\curveto(593.8587762,908.03628282)(594.20252594,909.09530979)(594.89002542,909.83142035)
\curveto(595.5775249,910.57447534)(596.5705797,910.94600284)(597.86918983,910.94600284)
\curveto(598.3066895,910.94600284)(598.69905031,910.92516952)(599.04627227,910.88350289)
\curveto(599.40043867,910.84183625)(599.72335509,910.79322518)(600.01502154,910.73766966)
\closepath
}
}
{
\newrgbcolor{curcolor}{0 0 0}
\pscustom[linestyle=none,fillstyle=solid,fillcolor=curcolor]
{
\newpath
\moveto(603.35876027,908.21683824)
\lineto(601.1504286,908.21683824)
\lineto(601.1504286,910.2480867)
\lineto(603.35876027,910.2480867)
\closepath
\moveto(603.23376036,894.63351518)
\lineto(601.27542851,894.63351518)
\lineto(601.27542851,906.26892304)
\lineto(603.23376036,906.26892304)
\closepath
}
}
{
\newrgbcolor{curcolor}{0 0 0}
\pscustom[linestyle=none,fillstyle=solid,fillcolor=curcolor]
{
\newpath
\moveto(616.84834135,900.24809427)
\lineto(608.27543117,900.24809427)
\curveto(608.27543117,899.53281703)(608.38306997,898.9078175)(608.59834759,898.37309568)
\curveto(608.8136252,897.84531831)(609.10876387,897.41129086)(609.48376359,897.07101334)
\curveto(609.84487442,896.73768025)(610.27195743,896.48768044)(610.76501262,896.3210139)
\curveto(611.26501224,896.15434736)(611.81362293,896.07101409)(612.41084471,896.07101409)
\curveto(613.20251077,896.07101409)(613.99764906,896.22726397)(614.79625957,896.53976374)
\curveto(615.60181451,896.85920794)(616.17473075,897.1717077)(616.51500827,897.47726303)
\lineto(616.61917485,897.47726303)
\lineto(616.61917485,895.34184798)
\curveto(615.95945313,895.06407041)(615.28584253,894.8314317)(614.59834305,894.64393184)
\curveto(613.91084357,894.45643198)(613.18862189,894.36268205)(612.43167802,894.36268205)
\curveto(610.50112393,894.36268205)(608.99418062,894.88351499)(607.91084811,895.92518087)
\curveto(606.8275156,896.97379119)(606.28584934,898.45990117)(606.28584934,900.38351083)
\curveto(606.28584934,902.28628717)(606.80321006,903.79670269)(607.8379315,904.9147574)
\curveto(608.87959738,906.03281211)(610.2476519,906.59183947)(611.94209506,906.59183947)
\curveto(613.51153832,906.59183947)(614.71987074,906.13350648)(615.56709232,905.21684051)
\curveto(616.42125834,904.30017453)(616.84834135,902.99809218)(616.84834135,901.31059346)
\closepath
\moveto(614.94209279,901.74809313)
\curveto(614.93514835,902.77587013)(614.67473188,903.57100842)(614.16084338,904.13350799)
\curveto(613.65389932,904.69600757)(612.87959435,904.97725735)(611.83792847,904.97725735)
\curveto(610.78931815,904.97725735)(609.95251323,904.66822981)(609.3275137,904.05017472)
\curveto(608.70945862,903.43211963)(608.35876444,902.6647591)(608.27543117,901.74809313)
\closepath
}
}
{
\newrgbcolor{curcolor}{0 0 0}
\pscustom[linestyle=none,fillstyle=solid,fillcolor=curcolor]
{
\newpath
\moveto(629.25457075,894.63351518)
\lineto(627.2962389,894.63351518)
\lineto(627.2962389,895.85226426)
\curveto(626.73373932,895.36615351)(626.14693421,894.98768158)(625.53582356,894.71684845)
\curveto(624.92471292,894.44601532)(624.26151897,894.31059876)(623.54624174,894.31059876)
\curveto(622.1573539,894.31059876)(621.05318807,894.84532057)(620.23374424,895.91476421)
\curveto(619.42124486,896.98420785)(619.01499516,898.46684561)(619.01499516,900.36267751)
\curveto(619.01499516,901.34878788)(619.15388395,902.22725943)(619.43166152,902.99809218)
\curveto(619.71638352,903.76892493)(620.09832768,904.42517444)(620.57749398,904.9668407)
\curveto(621.04971585,905.49461807)(621.59832654,905.89739555)(622.22332607,906.17517311)
\curveto(622.85527004,906.45295068)(623.50804732,906.59183947)(624.18165792,906.59183947)
\curveto(624.79276857,906.59183947)(625.33443483,906.52586729)(625.80665669,906.39392295)
\curveto(626.27887856,906.26892304)(626.77540596,906.07100653)(627.2962389,905.8001734)
\lineto(627.2962389,910.84183625)
\lineto(629.25457075,910.84183625)
\closepath
\moveto(627.2962389,897.49809635)
\lineto(627.2962389,904.17517463)
\curveto(626.76846152,904.41128556)(626.29623966,904.57447988)(625.8795733,904.66475759)
\curveto(625.46290695,904.7550353)(625.00804619,904.80017415)(624.514991,904.80017415)
\curveto(623.41776961,904.80017415)(622.56360359,904.41823)(621.95249294,903.65434169)
\curveto(621.34138229,902.89045338)(621.03582697,901.80712086)(621.03582697,900.40434415)
\curveto(621.03582697,899.02240075)(621.2719379,897.97031821)(621.74415977,897.24809653)
\curveto(622.21638163,896.5328193)(622.9733255,896.17518068)(624.01499138,896.17518068)
\curveto(624.57054652,896.17518068)(625.13304609,896.29670837)(625.7024901,896.53976374)
\curveto(626.27193412,896.78976355)(626.80318372,897.10920775)(627.2962389,897.49809635)
\closepath
}
}
{
\newrgbcolor{curcolor}{0 0 0}
\pscustom[linestyle=none,fillstyle=solid,fillcolor=curcolor]
{
\newpath
\moveto(642.58791229,894.63351518)
\lineto(640.62958044,894.63351518)
\lineto(640.62958044,910.84183625)
\lineto(642.58791229,910.84183625)
\closepath
}
}
{
\newrgbcolor{curcolor}{0 0 0}
\pscustom[linestyle=none,fillstyle=solid,fillcolor=curcolor]
{
\newpath
\moveto(656.37956206,900.44601078)
\curveto(656.37956206,898.55017888)(655.89345131,897.05365224)(654.92122983,895.95643085)
\curveto(653.94900834,894.85920945)(652.64692599,894.31059876)(651.01498278,894.31059876)
\curveto(649.3691507,894.31059876)(648.06012391,894.85920945)(647.08790242,895.95643085)
\curveto(646.12262537,897.05365224)(645.63998685,898.55017888)(645.63998685,900.44601078)
\curveto(645.63998685,902.34184268)(646.12262537,903.83836933)(647.08790242,904.93559072)
\curveto(648.06012391,906.03975655)(649.3691507,906.59183947)(651.01498278,906.59183947)
\curveto(652.64692599,906.59183947)(653.94900834,906.03975655)(654.92122983,904.93559072)
\curveto(655.89345131,903.83836933)(656.37956206,902.34184268)(656.37956206,900.44601078)
\closepath
\moveto(654.35873025,900.44601078)
\curveto(654.35873025,901.95295409)(654.06359159,903.0710088)(653.47331426,903.80017491)
\curveto(652.88303693,904.53628547)(652.0635931,904.90434074)(651.01498278,904.90434074)
\curveto(649.95248359,904.90434074)(649.12609532,904.53628547)(648.53581799,903.80017491)
\curveto(647.9524851,903.0710088)(647.66081865,901.95295409)(647.66081865,900.44601078)
\curveto(647.66081865,898.98767855)(647.95595732,897.8800405)(648.54623465,897.12309663)
\curveto(649.13651198,896.3730972)(649.95942803,895.99809748)(651.01498278,895.99809748)
\curveto(652.05664866,895.99809748)(652.87262027,896.36962498)(653.4628976,897.11267997)
\curveto(654.06011937,897.8626794)(654.35873025,898.97378967)(654.35873025,900.44601078)
\closepath
}
}
{
\newrgbcolor{curcolor}{0 0 0}
\pscustom[linestyle=none,fillstyle=solid,fillcolor=curcolor]
{
\newpath
\moveto(668.84828807,895.95643085)
\curveto(668.84828807,893.98421012)(668.40037174,892.53629454)(667.50453909,891.61268413)
\curveto(666.60870643,890.68907372)(665.23023525,890.22726851)(663.36912555,890.22726851)
\curveto(662.75107046,890.22726851)(662.14690425,890.27240737)(661.55662692,890.36268508)
\curveto(660.97329403,890.44601835)(660.39690558,890.56754603)(659.82746156,890.72726813)
\lineto(659.82746156,892.72726662)
\lineto(659.93162815,892.72726662)
\curveto(660.25107235,892.60226672)(660.75801641,892.44948905)(661.45246033,892.26893364)
\curveto(662.14690425,892.08143378)(662.84134817,891.98768385)(663.53579209,891.98768385)
\curveto(664.20245825,891.98768385)(664.75454117,892.0675449)(665.19204084,892.227267)
\curveto(665.62954051,892.3869891)(665.96981803,892.60921116)(666.2128734,892.89393316)
\curveto(666.45592877,893.16476629)(666.62953975,893.49115493)(666.73370634,893.87309909)
\curveto(666.83787292,894.25504324)(666.88995622,894.68212625)(666.88995622,895.15434812)
\lineto(666.88995622,896.21684731)
\curveto(666.29967889,895.74462545)(665.73370709,895.39045905)(665.19204084,895.15434812)
\curveto(664.65731902,894.92518163)(663.97329176,894.81059838)(663.13995906,894.81059838)
\curveto(661.75107122,894.81059838)(660.64690539,895.310598)(659.82746156,896.31059724)
\curveto(659.01496218,897.31754093)(658.60871248,898.73420652)(658.60871248,900.56059403)
\curveto(658.60871248,901.56059327)(658.74760127,902.42170373)(659.02537884,903.14392541)
\curveto(659.31010084,903.87309152)(659.69551722,904.50156327)(660.18162796,905.02934065)
\curveto(660.63301651,905.52239583)(661.1816272,905.90433999)(661.82746005,906.17517311)
\curveto(662.47329289,906.45295068)(663.11565352,906.59183947)(663.75454192,906.59183947)
\curveto(664.42815253,906.59183947)(664.9906521,906.52239507)(665.44204065,906.38350629)
\curveto(665.90037363,906.25156195)(666.38301216,906.04670099)(666.88995622,905.76892342)
\lineto(667.01495612,906.26892304)
\lineto(668.84828807,906.26892304)
\closepath
\moveto(666.88995622,897.83142943)
\lineto(666.88995622,904.17517463)
\curveto(666.36912328,904.41128556)(665.88301254,904.5779521)(665.43162399,904.67517425)
\curveto(664.98717988,904.77934084)(664.54273577,904.83142413)(664.09829166,904.83142413)
\curveto(663.02190359,904.83142413)(662.17468201,904.47031329)(661.55662692,903.74809162)
\curveto(660.93857183,903.02586994)(660.62954429,901.97725962)(660.62954429,900.60226066)
\curveto(660.62954429,899.2967061)(660.85871078,898.30712351)(661.31704377,897.63351291)
\curveto(661.77537675,896.95990231)(662.53579285,896.62309701)(663.59829204,896.62309701)
\curveto(664.16773606,896.62309701)(664.73718007,896.73073582)(665.30662408,896.94601343)
\curveto(665.88301254,897.16823548)(666.41078991,897.46337415)(666.88995622,897.83142943)
\closepath
}
}
{
\newrgbcolor{curcolor}{0 0 0}
\pscustom[linestyle=none,fillstyle=solid,fillcolor=curcolor]
{
\newpath
\moveto(686.46287637,908.94600435)
\lineto(686.35870978,908.94600435)
\curveto(686.14343217,909.0085043)(685.86218238,909.07100426)(685.51496042,909.13350421)
\curveto(685.16773846,909.2029486)(684.86218314,909.2376708)(684.59829445,909.2376708)
\curveto(683.75801731,909.2376708)(683.14690666,909.05017094)(682.7649625,908.67517122)
\curveto(682.38996279,908.30711595)(682.20246293,907.63697756)(682.20246293,906.66475608)
\lineto(682.20246293,906.26892304)
\lineto(685.73371026,906.26892304)
\lineto(685.73371026,904.62309096)
\lineto(682.26496288,904.62309096)
\lineto(682.26496288,894.63351518)
\lineto(680.30663103,894.63351518)
\lineto(680.30663103,904.62309096)
\lineto(678.98371536,904.62309096)
\lineto(678.98371536,906.26892304)
\lineto(680.30663103,906.26892304)
\lineto(680.30663103,906.65433942)
\curveto(680.30663103,908.03628282)(680.65038077,909.09530979)(681.33788025,909.83142035)
\curveto(682.02537973,910.57447534)(683.01843453,910.94600284)(684.31704466,910.94600284)
\curveto(684.75454433,910.94600284)(685.14690514,910.92516952)(685.4941271,910.88350289)
\curveto(685.8482935,910.84183625)(686.17120992,910.79322518)(686.46287637,910.73766966)
\closepath
}
}
{
\newrgbcolor{curcolor}{0 0 0}
\pscustom[linestyle=none,fillstyle=solid,fillcolor=curcolor]
{
\newpath
\moveto(689.80660068,908.21683824)
\lineto(687.59826902,908.21683824)
\lineto(687.59826902,910.2480867)
\lineto(689.80660068,910.2480867)
\closepath
\moveto(689.68160077,894.63351518)
\lineto(687.72326892,894.63351518)
\lineto(687.72326892,906.26892304)
\lineto(689.68160077,906.26892304)
\closepath
}
}
{
\newrgbcolor{curcolor}{0 0 0}
\pscustom[linestyle=none,fillstyle=solid,fillcolor=curcolor]
{
\newpath
\moveto(695.53578538,894.63351518)
\lineto(693.57745353,894.63351518)
\lineto(693.57745353,910.84183625)
\lineto(695.53578538,910.84183625)
\closepath
}
}
{
\newrgbcolor{curcolor}{0 0 0}
\pscustom[linestyle=none,fillstyle=solid,fillcolor=curcolor]
{
\newpath
\moveto(709.15035195,900.24809427)
\lineto(700.57744177,900.24809427)
\curveto(700.57744177,899.53281703)(700.68508058,898.9078175)(700.90035819,898.37309568)
\curveto(701.11563581,897.84531831)(701.41077447,897.41129086)(701.78577419,897.07101334)
\curveto(702.14688503,896.73768025)(702.57396804,896.48768044)(703.06702322,896.3210139)
\curveto(703.56702284,896.15434736)(704.11563354,896.07101409)(704.71285531,896.07101409)
\curveto(705.50452138,896.07101409)(706.29965966,896.22726397)(707.09827017,896.53976374)
\curveto(707.90382512,896.85920794)(708.47674135,897.1717077)(708.81701887,897.47726303)
\lineto(708.92118546,897.47726303)
\lineto(708.92118546,895.34184798)
\curveto(708.26146374,895.06407041)(707.58785313,894.8314317)(706.90035365,894.64393184)
\curveto(706.21285417,894.45643198)(705.4906325,894.36268205)(704.73368863,894.36268205)
\curveto(702.80313453,894.36268205)(701.29619123,894.88351499)(700.21285871,895.92518087)
\curveto(699.1295262,896.97379119)(698.58785994,898.45990117)(698.58785994,900.38351083)
\curveto(698.58785994,902.28628717)(699.10522066,903.79670269)(700.1399421,904.9147574)
\curveto(701.18160798,906.03281211)(702.5496625,906.59183947)(704.24410566,906.59183947)
\curveto(705.81354892,906.59183947)(707.02188134,906.13350648)(707.86910292,905.21684051)
\curveto(708.72326894,904.30017453)(709.15035195,902.99809218)(709.15035195,901.31059346)
\closepath
\moveto(707.24410339,901.74809313)
\curveto(707.23715895,902.77587013)(706.97674248,903.57100842)(706.46285398,904.13350799)
\curveto(705.95590992,904.69600757)(705.18160495,904.97725735)(704.13993908,904.97725735)
\curveto(703.09132876,904.97725735)(702.25452384,904.66822981)(701.62952431,904.05017472)
\curveto(701.01146922,903.43211963)(700.66077504,902.6647591)(700.57744177,901.74809313)
\closepath
}
}
{
\newrgbcolor{curcolor}{0.7019608 0.7019608 0.7019608}
\pscustom[linestyle=none,fillstyle=solid,fillcolor=curcolor,opacity=0.92623001]
{
\newpath
\moveto(356.97411725,1290.15361816)
\lineto(765.54556779,1290.15361816)
\lineto(765.54556779,1135.86789769)
\lineto(356.97411725,1135.86789769)
\closepath
}
}
{
\newrgbcolor{curcolor}{0 0 0}
\pscustom[linewidth=1.00157103,linecolor=curcolor]
{
\newpath
\moveto(356.97411725,1290.15361816)
\lineto(765.54556779,1290.15361816)
\lineto(765.54556779,1135.86789769)
\lineto(356.97411725,1135.86789769)
\closepath
}
}
{
\newrgbcolor{curcolor}{0 0 0}
\pscustom[linestyle=none,fillstyle=solid,fillcolor=curcolor]
{
\newpath
\moveto(502.56861659,1255.42571961)
\lineto(498.89675328,1255.42571961)
\lineto(498.89675328,1267.84755506)
\curveto(498.89675328,1268.85015604)(498.83815972,1269.78765305)(498.72097259,1270.66004611)
\curveto(498.60378547,1271.54545995)(498.3889424,1272.23556192)(498.07644339,1272.73035202)
\curveto(497.7509236,1273.27722527)(497.28217509,1273.68086982)(496.67019787,1273.94128566)
\curveto(496.05822065,1274.21472229)(495.26395235,1274.3514406)(494.28739296,1274.3514406)
\curveto(493.28479198,1274.3514406)(492.23661824,1274.10404556)(491.14287172,1273.60925547)
\curveto(490.0491252,1273.11446538)(489.00095146,1272.48295697)(487.99835048,1271.71473025)
\lineto(487.99835048,1255.42571961)
\lineto(484.32648717,1255.42571961)
\lineto(484.32648717,1277.2420564)
\lineto(487.99835048,1277.2420564)
\lineto(487.99835048,1274.82018911)
\curveto(489.14418017,1275.77070692)(490.32907223,1276.51289206)(491.55302666,1277.04674452)
\curveto(492.7769811,1277.58059699)(494.03348752,1277.84752322)(495.32254591,1277.84752322)
\curveto(497.67930924,1277.84752322)(499.47617852,1277.13789007)(500.71315375,1275.71862375)
\curveto(501.95012898,1274.29935744)(502.56861659,1272.25509311)(502.56861659,1269.58583078)
\closepath
}
}
{
\newrgbcolor{curcolor}{0 0 0}
\pscustom[linestyle=none,fillstyle=solid,fillcolor=curcolor]
{
\newpath
\moveto(527.72478599,1255.42571961)
\lineto(524.05292268,1255.42571961)
\lineto(524.05292268,1257.8475869)
\curveto(522.81594745,1256.87102751)(521.63105539,1256.12233198)(520.4982465,1255.6015003)
\curveto(519.3654376,1255.08066863)(518.11544158,1254.82025279)(516.74825844,1254.82025279)
\curveto(514.45659907,1254.82025279)(512.67275058,1255.51686516)(511.39671298,1256.91008989)
\curveto(510.12067537,1258.31633541)(509.48265657,1260.37362053)(509.48265657,1263.08194524)
\lineto(509.48265657,1277.2420564)
\lineto(513.15451988,1277.2420564)
\lineto(513.15451988,1264.82022095)
\curveto(513.15451988,1263.71345364)(513.20660305,1262.76293583)(513.31076938,1261.96866753)
\curveto(513.41493572,1261.18742002)(513.63628918,1260.51684924)(513.97482977,1259.95695519)
\curveto(514.32639115,1259.38404034)(514.78211886,1258.967375)(515.34201291,1258.70695917)
\curveto(515.90190697,1258.44654333)(516.71570646,1258.31633541)(517.78341139,1258.31633541)
\curveto(518.7339292,1258.31633541)(519.76908215,1258.56373045)(520.88887025,1259.05852055)
\curveto(522.02167915,1259.55331064)(523.07636329,1260.18481904)(524.05292268,1260.95304576)
\lineto(524.05292268,1277.2420564)
\lineto(527.72478599,1277.2420564)
\closepath
}
}
{
\newrgbcolor{curcolor}{0 0 0}
\pscustom[linestyle=none,fillstyle=solid,fillcolor=curcolor]
{
\newpath
\moveto(566.78716111,1255.42571961)
\lineto(563.1152978,1255.42571961)
\lineto(563.1152978,1267.84755506)
\curveto(563.1152978,1268.78505208)(563.06972503,1269.68999711)(562.97857949,1270.56239017)
\curveto(562.90045474,1271.43478322)(562.72467404,1272.13139559)(562.45123742,1272.65222726)
\curveto(562.1517592,1273.21212132)(561.72207307,1273.63529705)(561.16217902,1273.92175447)
\curveto(560.60228497,1274.20821189)(559.79499587,1274.3514406)(558.74031173,1274.3514406)
\curveto(557.71166917,1274.3514406)(556.68302661,1274.09102477)(555.65438406,1273.57019309)
\curveto(554.6257415,1273.06238221)(553.59709894,1272.41134261)(552.56845638,1271.61707431)
\curveto(552.60751876,1271.3175961)(552.64007074,1270.96603472)(552.66611232,1270.56239017)
\curveto(552.6921539,1270.17176641)(552.7051747,1269.78114266)(552.7051747,1269.3905189)
\lineto(552.7051747,1255.42571961)
\lineto(549.03331139,1255.42571961)
\lineto(549.03331139,1267.84755506)
\curveto(549.03331139,1268.81109366)(548.98773862,1269.72254909)(548.89659307,1270.58192136)
\curveto(548.81846832,1271.45431441)(548.64268763,1272.15092678)(548.369251,1272.67175845)
\curveto(548.06977279,1273.2316525)(547.64008666,1273.64831784)(547.08019261,1273.92175447)
\curveto(546.52029856,1274.20821189)(545.71300946,1274.3514406)(544.65832532,1274.3514406)
\curveto(543.65572434,1274.3514406)(542.64661297,1274.10404556)(541.63099121,1273.60925547)
\curveto(540.62839023,1273.11446538)(539.62578926,1272.48295697)(538.62318828,1271.71473025)
\lineto(538.62318828,1255.42571961)
\lineto(534.95132497,1255.42571961)
\lineto(534.95132497,1277.2420564)
\lineto(538.62318828,1277.2420564)
\lineto(538.62318828,1274.82018911)
\curveto(539.76901797,1275.77070692)(540.90833726,1276.51289206)(542.04114615,1277.04674452)
\curveto(543.18697583,1277.58059699)(544.40441988,1277.84752322)(545.69347827,1277.84752322)
\curveto(547.17784854,1277.84752322)(548.43435496,1277.53502422)(549.46299752,1276.91002621)
\curveto(550.50466087,1276.2850282)(551.27939799,1275.41914554)(551.78720887,1274.31237823)
\curveto(553.27157914,1275.56237425)(554.6257415,1276.46080889)(555.84969593,1277.00768215)
\curveto(557.07365037,1277.5675762)(558.38223995,1277.84752322)(559.77546468,1277.84752322)
\curveto(562.17129039,1277.84752322)(563.93560769,1277.11835888)(565.06841658,1275.66003019)
\curveto(566.21424627,1274.21472229)(566.78716111,1272.18998915)(566.78716111,1269.58583078)
\closepath
}
}
{
\newrgbcolor{curcolor}{0 0 0}
\pscustom[linestyle=none,fillstyle=solid,fillcolor=curcolor]
{
\newpath
\moveto(590.81051984,1255.42571961)
\lineto(587.15818772,1255.42571961)
\lineto(587.15818772,1257.74993096)
\curveto(586.83266792,1257.5285775)(586.389961,1257.2160785)(585.83006695,1256.81243395)
\curveto(585.28319369,1256.42181019)(584.74934122,1256.10931119)(584.22850955,1255.87493693)
\curveto(583.61653233,1255.57545872)(582.91340957,1255.32806367)(582.11914126,1255.1327518)
\curveto(581.32487296,1254.92441913)(580.39388634,1254.82025279)(579.32618141,1254.82025279)
\curveto(577.36004183,1254.82025279)(575.69338047,1255.47129238)(574.32619733,1256.77337157)
\curveto(572.95901418,1258.07545076)(572.27542261,1259.73560172)(572.27542261,1261.75382446)
\curveto(572.27542261,1263.40746503)(572.62698399,1264.7420962)(573.33010675,1265.75771797)
\curveto(574.0462503,1266.78636052)(575.06187207,1267.59364962)(576.37697205,1268.17958526)
\curveto(577.70509282,1268.76552089)(579.30013982,1269.16265504)(581.16211306,1269.37098771)
\curveto(583.0240863,1269.57932038)(585.02277785,1269.73556988)(587.15818772,1269.83973622)
\lineto(587.15818772,1270.40614067)
\curveto(587.15818772,1271.23947135)(587.00844861,1271.92957332)(586.7089704,1272.47644657)
\curveto(586.42251298,1273.02331983)(586.00584764,1273.45300596)(585.45897438,1273.76550497)
\curveto(584.9381427,1274.06498318)(584.31314469,1274.26680546)(583.58398035,1274.37097179)
\curveto(582.854816,1274.47513813)(582.09309968,1274.52722129)(581.29883138,1274.52722129)
\curveto(580.33529278,1274.52722129)(579.26107745,1274.39701338)(578.07618539,1274.13659754)
\curveto(576.89129333,1273.88920249)(575.66733889,1273.52462032)(574.40432208,1273.04285102)
\lineto(574.2090102,1273.04285102)
\lineto(574.2090102,1276.77330789)
\curveto(574.92515375,1276.96861977)(575.96030671,1277.18346284)(577.31446906,1277.41783709)
\curveto(578.66863142,1277.65221134)(580.00326258,1277.76939847)(581.31836256,1277.76939847)
\curveto(582.854816,1277.76939847)(584.18944717,1277.63919055)(585.32225606,1277.37877472)
\curveto(586.46808575,1277.13137967)(587.45766593,1276.70169354)(588.29099661,1276.08971632)
\curveto(589.1113065,1275.49075989)(589.73630451,1274.71602278)(590.16599064,1273.76550497)
\curveto(590.59567677,1272.81498716)(590.81051984,1271.6366055)(590.81051984,1270.23035998)
\closepath
\moveto(587.15818772,1260.79679626)
\lineto(587.15818772,1266.87099567)
\curveto(586.03839962,1266.80589171)(584.71678924,1266.70823577)(583.19335659,1266.57802785)
\curveto(581.68294474,1266.44781994)(580.48503188,1266.25901845)(579.59961804,1266.01162341)
\curveto(578.54493389,1265.71214519)(577.69207203,1265.24339669)(577.04103243,1264.60537789)
\curveto(576.38999284,1263.98037988)(576.06447304,1263.11449722)(576.06447304,1262.00772991)
\curveto(576.06447304,1260.75773389)(576.44207601,1259.81372647)(577.19728193,1259.17570767)
\curveto(577.95248786,1258.55070966)(579.10482794,1258.23821066)(580.65430218,1258.23821066)
\curveto(581.94336057,1258.23821066)(583.12174224,1258.4856057)(584.18944717,1258.98039579)
\curveto(585.25715211,1259.48820668)(586.24673229,1260.0936735)(587.15818772,1260.79679626)
\closepath
}
}
{
\newrgbcolor{curcolor}{0 0 0}
\pscustom[linestyle=none,fillstyle=solid,fillcolor=curcolor]
{
\newpath
\moveto(614.01357536,1256.79290276)
\curveto(612.78962093,1256.20696713)(611.62426005,1255.75123941)(610.51749274,1255.42571961)
\curveto(609.42374623,1255.10019982)(608.25838535,1254.93743992)(607.02141013,1254.93743992)
\curveto(605.44589431,1254.93743992)(604.00058641,1255.16530377)(602.68548643,1255.62103149)
\curveto(601.37038645,1256.08978)(600.24408795,1256.79290276)(599.30659094,1257.73039977)
\curveto(598.35607313,1258.66789679)(597.62039839,1259.85278885)(597.09956672,1261.28507596)
\curveto(596.57873504,1262.71736306)(596.3183192,1264.39053482)(596.3183192,1266.30459122)
\curveto(596.3183192,1269.8722882)(597.29487859,1272.67175845)(599.24799738,1274.70300198)
\curveto(601.21413695,1276.73424552)(603.80527453,1277.74986728)(607.02141013,1277.74986728)
\curveto(608.27140615,1277.74986728)(609.49536058,1277.57408659)(610.69327344,1277.22252521)
\curveto(611.90420708,1276.87096383)(613.01097439,1276.4412777)(614.01357536,1275.93346682)
\lineto(614.01357536,1271.85144856)
\lineto(613.81826349,1271.85144856)
\curveto(612.69847538,1272.72384162)(611.53962491,1273.3944124)(610.34171205,1273.86316091)
\curveto(609.15681999,1274.33190942)(607.99796952,1274.56628367)(606.86516062,1274.56628367)
\curveto(604.78183392,1274.56628367)(603.13470375,1273.86316091)(601.92377011,1272.45691539)
\curveto(600.72585725,1271.06369066)(600.12690083,1269.01291594)(600.12690083,1266.30459122)
\curveto(600.12690083,1263.67439127)(600.71283646,1261.64965813)(601.88470773,1260.23039181)
\curveto(603.06959979,1258.82414629)(604.72975076,1258.12102353)(606.86516062,1258.12102353)
\curveto(607.60734576,1258.12102353)(608.36255169,1258.21867947)(609.13077841,1258.41399135)
\curveto(609.89900513,1258.60930323)(610.5891071,1258.86320867)(611.20108432,1259.17570767)
\curveto(611.73493679,1259.4491443)(612.23623727,1259.73560172)(612.70498578,1260.03507994)
\curveto(613.17373429,1260.34757894)(613.54482686,1260.61450518)(613.81826349,1260.83585864)
\lineto(614.01357536,1260.83585864)
\closepath
}
}
{
\newrgbcolor{curcolor}{0 0 0}
\pscustom[linestyle=none,fillstyle=solid,fillcolor=curcolor]
{
\newpath
\moveto(630.04868443,1255.62103149)
\curveto(629.35858246,1255.4387404)(628.60337654,1255.2890013)(627.78306665,1255.17181417)
\curveto(626.97577755,1255.05462704)(626.2531236,1254.99603348)(625.6151048,1254.99603348)
\curveto(623.38854939,1254.99603348)(621.69584645,1255.59498991)(620.53699597,1256.79290276)
\curveto(619.37814549,1257.99081561)(618.79872025,1259.91138241)(618.79872025,1262.55460316)
\lineto(618.79872025,1274.15612873)
\lineto(616.3182594,1274.15612873)
\lineto(616.3182594,1277.2420564)
\lineto(618.79872025,1277.2420564)
\lineto(618.79872025,1283.51156769)
\lineto(622.47058356,1283.51156769)
\lineto(622.47058356,1277.2420564)
\lineto(630.04868443,1277.2420564)
\lineto(630.04868443,1274.15612873)
\lineto(622.47058356,1274.15612873)
\lineto(622.47058356,1264.21475413)
\curveto(622.47058356,1263.06892444)(622.49662515,1262.1704898)(622.54870831,1261.51945021)
\curveto(622.60079148,1260.88143141)(622.78308257,1260.28247498)(623.09558157,1259.72258093)
\curveto(623.38203899,1259.20174926)(623.77266275,1258.8176359)(624.26745284,1258.57024085)
\curveto(624.77526372,1258.3358666)(625.54349045,1258.21867947)(626.572133,1258.21867947)
\curveto(627.17108943,1258.21867947)(627.79608744,1258.30331462)(628.44712703,1258.47258491)
\curveto(629.09816663,1258.654876)(629.56691513,1258.8046151)(629.85337256,1258.92180223)
\lineto(630.04868443,1258.92180223)
\closepath
}
}
{
\newrgbcolor{curcolor}{0 0 0}
\pscustom[linestyle=none,fillstyle=solid,fillcolor=curcolor]
{
\newpath
\moveto(638.19318227,1255.42571961)
\lineto(634.52131896,1255.42571961)
\lineto(634.52131896,1285.81624785)
\lineto(638.19318227,1285.81624785)
\closepath
}
}
{
\newrgbcolor{curcolor}{0 0 0}
\pscustom[linestyle=none,fillstyle=solid,fillcolor=curcolor]
{
\newpath
\moveto(386.18706702,1235.52039562)
\lineto(379.68707194,1235.52039562)
\lineto(379.68707194,1237.40581086)
\lineto(386.18706702,1237.40581086)
\closepath
}
}
{
\newrgbcolor{curcolor}{0 0 0}
\pscustom[linestyle=none,fillstyle=solid,fillcolor=curcolor]
{
\newpath
\moveto(409.06204995,1234.44747976)
\curveto(409.06204995,1233.67664701)(408.91621673,1232.99609197)(408.62455028,1232.40581464)
\curveto(408.33288384,1231.81553731)(407.94052302,1231.32942657)(407.44746784,1230.94748241)
\curveto(406.86413495,1230.48914942)(406.22177432,1230.16276078)(405.52038597,1229.96831648)
\curveto(404.82594205,1229.77387219)(403.94052605,1229.67665004)(402.86413797,1229.67665004)
\lineto(397.36414214,1229.67665004)
\lineto(397.36414214,1245.18705497)
\lineto(401.95788866,1245.18705497)
\curveto(403.08983225,1245.18705497)(403.93705383,1245.14538834)(404.4995534,1245.06205506)
\curveto(405.06205298,1244.97872179)(405.60024702,1244.80511081)(406.11413552,1244.54122213)
\curveto(406.68357953,1244.24261124)(407.09677366,1243.85719487)(407.35371791,1243.384973)
\curveto(407.61066216,1242.91969557)(407.73913429,1242.36066822)(407.73913429,1241.70789094)
\curveto(407.73913429,1240.97178038)(407.55163443,1240.34330863)(407.17663471,1239.8224757)
\curveto(406.801635,1239.3085872)(406.30163537,1238.89539306)(405.67663585,1238.5828933)
\lineto(405.67663585,1238.49956003)
\curveto(406.72524616,1238.28428241)(407.55163443,1237.82247721)(408.15580064,1237.11414441)
\curveto(408.75996685,1236.41275605)(409.06204995,1235.52386784)(409.06204995,1234.44747976)
\closepath
\moveto(405.59330258,1241.43705781)
\curveto(405.59330258,1241.81205752)(405.53080262,1242.12802951)(405.40580272,1242.38497376)
\curveto(405.28080281,1242.64191801)(405.07941408,1242.85025118)(404.80163651,1243.00997328)
\curveto(404.47524787,1243.19747314)(404.07941483,1243.31205639)(403.61413741,1243.35372302)
\curveto(403.14885998,1243.4023341)(402.57247153,1243.42663964)(401.88497205,1243.42663964)
\lineto(399.42664058,1243.42663964)
\lineto(399.42664058,1238.94747636)
\lineto(402.09330522,1238.94747636)
\curveto(402.73913807,1238.94747636)(403.25302657,1238.97872633)(403.63497072,1239.04122629)
\curveto(404.01691488,1239.11067068)(404.37108128,1239.24955946)(404.69746992,1239.45789264)
\curveto(405.02385856,1239.66622581)(405.25302506,1239.93358672)(405.3849694,1240.25997536)
\curveto(405.52385818,1240.59330845)(405.59330258,1240.98566926)(405.59330258,1241.43705781)
\closepath
\moveto(406.91621824,1234.36414649)
\curveto(406.91621824,1234.98914602)(406.82246831,1235.48567342)(406.63496846,1235.8537287)
\curveto(406.4474686,1236.22178398)(406.10719108,1236.53428374)(405.61413589,1236.79122799)
\curveto(405.28080281,1236.96483897)(404.87455312,1237.07595)(404.39538682,1237.12456107)
\curveto(403.92316495,1237.18011658)(403.3467765,1237.20789434)(402.66622146,1237.20789434)
\lineto(399.42664058,1237.20789434)
\lineto(399.42664058,1231.43706537)
\lineto(402.15580518,1231.43706537)
\curveto(403.05858227,1231.43706537)(403.79816505,1231.48220423)(404.3745535,1231.57248194)
\curveto(404.95094195,1231.66970409)(405.42316382,1231.84331507)(405.79121909,1232.09331488)
\curveto(406.18010769,1232.364148)(406.4648297,1232.67317555)(406.64538511,1233.02039751)
\curveto(406.82594053,1233.36761947)(406.91621824,1233.8155358)(406.91621824,1234.36414649)
\closepath
}
}
{
\newrgbcolor{curcolor}{0 0 0}
\pscustom[linestyle=none,fillstyle=solid,fillcolor=curcolor]
{
\newpath
\moveto(420.87454091,1229.67665004)
\lineto(418.92662571,1229.67665004)
\lineto(418.92662571,1230.91623243)
\curveto(418.75301473,1230.79817697)(418.5169038,1230.63151043)(418.21829292,1230.41623281)
\curveto(417.92662647,1230.20789964)(417.64190446,1230.0412331)(417.3641269,1229.91623319)
\curveto(417.03773825,1229.75651109)(416.66273854,1229.62456674)(416.23912775,1229.52040016)
\curveto(415.81551696,1229.40928913)(415.31898955,1229.35373362)(414.74954554,1229.35373362)
\curveto(413.70093522,1229.35373362)(412.81204701,1229.70095558)(412.08288089,1230.39539949)
\curveto(411.35371478,1231.08984341)(410.98913172,1231.97525941)(410.98913172,1233.05164748)
\curveto(410.98913172,1233.93359126)(411.17663158,1234.64539628)(411.55163129,1235.18706254)
\curveto(411.93357545,1235.73567323)(412.47524171,1236.16622846)(413.17663006,1236.47872823)
\curveto(413.88496286,1236.79122799)(414.73565666,1237.00303338)(415.72871147,1237.11414441)
\curveto(416.72176627,1237.22525544)(417.78773769,1237.30858871)(418.92662571,1237.36414422)
\lineto(418.92662571,1237.66622733)
\curveto(418.92662571,1238.11067144)(418.84676466,1238.47872671)(418.68704256,1238.77039316)
\curveto(418.5342649,1239.0620596)(418.31204285,1239.2912261)(418.0203764,1239.45789264)
\curveto(417.74259883,1239.61761474)(417.40926575,1239.72525355)(417.02037716,1239.78080906)
\curveto(416.63148856,1239.83636457)(416.22523887,1239.86414233)(415.80162808,1239.86414233)
\curveto(415.28773958,1239.86414233)(414.71482334,1239.79469794)(414.08287938,1239.65580916)
\curveto(413.45093541,1239.52386481)(412.79815813,1239.32942051)(412.12454753,1239.07247626)
\lineto(412.02038094,1239.07247626)
\lineto(412.02038094,1241.06205809)
\curveto(412.40232509,1241.16622468)(412.95440801,1241.28080793)(413.67662969,1241.40580783)
\curveto(414.39885136,1241.53080774)(415.11065638,1241.59330769)(415.81204474,1241.59330769)
\curveto(416.63148856,1241.59330769)(417.34329358,1241.5238633)(417.94745979,1241.38497451)
\curveto(418.55857044,1241.25303017)(419.08634781,1241.02386368)(419.53079192,1240.69747503)
\curveto(419.96829159,1240.37803083)(420.30162467,1239.9648367)(420.53079117,1239.45789264)
\curveto(420.75995766,1238.95094858)(420.87454091,1238.32247683)(420.87454091,1237.5724774)
\closepath
\moveto(418.92662571,1232.5412312)
\lineto(418.92662571,1235.78081209)
\curveto(418.32940394,1235.74608989)(417.62454337,1235.6940066)(416.81204398,1235.6245622)
\curveto(416.00648903,1235.55511781)(415.36760063,1235.45442344)(414.89537876,1235.3224791)
\curveto(414.33287919,1235.162757)(413.87801842,1234.91275719)(413.53079646,1234.57247967)
\curveto(413.1835745,1234.23914659)(413.00996352,1233.77734138)(413.00996352,1233.18706405)
\curveto(413.00996352,1232.52039789)(413.21135226,1232.01692605)(413.61412973,1231.67664853)
\curveto(414.01690721,1231.34331544)(414.63149007,1231.1766489)(415.45787834,1231.1766489)
\curveto(416.14537782,1231.1766489)(416.77384956,1231.30859325)(417.34329358,1231.57248194)
\curveto(417.91273759,1231.84331507)(418.44051497,1232.16623149)(418.92662571,1232.5412312)
\closepath
}
}
{
\newrgbcolor{curcolor}{0 0 0}
\pscustom[linestyle=none,fillstyle=solid,fillcolor=curcolor]
{
\newpath
\moveto(432.95786638,1233.03081417)
\curveto(432.95786638,1231.96831497)(432.51689449,1231.09678785)(431.63495072,1230.41623281)
\curveto(430.75995138,1229.73567777)(429.56203562,1229.39540025)(428.04120343,1229.39540025)
\curveto(427.18009297,1229.39540025)(426.38842691,1229.49609462)(425.66620523,1229.69748336)
\curveto(424.95092799,1229.90581653)(424.350234,1230.13151081)(423.86412326,1230.37456618)
\lineto(423.86412326,1232.57248118)
\lineto(423.96828985,1232.57248118)
\curveto(424.58634494,1232.10720375)(425.27384442,1231.73567626)(426.03078829,1231.45789869)
\curveto(426.78773216,1231.18706556)(427.51342606,1231.051649)(428.20786997,1231.051649)
\curveto(429.06898043,1231.051649)(429.74259104,1231.19053778)(430.22870178,1231.46831535)
\curveto(430.71481252,1231.74609292)(430.95786789,1232.18359259)(430.95786789,1232.78081436)
\curveto(430.95786789,1233.23914734)(430.82592355,1233.5863693)(430.56203486,1233.82248023)
\curveto(430.29814617,1234.05859117)(429.79120211,1234.2599799)(429.04120268,1234.42664644)
\curveto(428.76342511,1234.4891464)(428.39884205,1234.56206301)(427.94745351,1234.64539628)
\curveto(427.5030094,1234.72872955)(427.0967597,1234.81900726)(426.72870443,1234.91622941)
\curveto(425.70787187,1235.18706254)(424.98217797,1235.58289557)(424.55162274,1236.10372851)
\curveto(424.12801195,1236.63150589)(423.91620655,1237.27733873)(423.91620655,1238.04122704)
\curveto(423.91620655,1238.52039335)(424.0134287,1238.97178189)(424.207873,1239.39539269)
\curveto(424.40926174,1239.81900348)(424.71134484,1240.19747541)(425.11412232,1240.53080849)
\curveto(425.50301091,1240.85719713)(425.99606609,1241.11414139)(426.59328786,1241.30164124)
\curveto(427.19745407,1241.49608554)(427.87106467,1241.59330769)(428.61411967,1241.59330769)
\curveto(429.30856359,1241.59330769)(430.00995194,1241.5065022)(430.71828474,1241.33289122)
\curveto(431.43356198,1241.16622468)(432.02731153,1240.96136372)(432.49953339,1240.71830835)
\lineto(432.49953339,1238.62455994)
\lineto(432.39536681,1238.62455994)
\curveto(431.89536718,1238.99261521)(431.28772876,1239.30164276)(430.57245152,1239.55164257)
\curveto(429.85717428,1239.80858682)(429.15578592,1239.93705894)(428.46828644,1239.93705894)
\curveto(427.75300921,1239.93705894)(427.148843,1239.79817016)(426.65578782,1239.52039259)
\curveto(426.16273263,1239.24955946)(425.91620504,1238.84330977)(425.91620504,1238.30164351)
\curveto(425.91620504,1237.82247721)(426.06551048,1237.46136637)(426.36412137,1237.218311)
\curveto(426.65578782,1236.97525563)(427.12800968,1236.77733911)(427.78078696,1236.62456145)
\curveto(428.1418978,1236.54122818)(428.54467528,1236.45789491)(428.98911938,1236.37456164)
\curveto(429.44050793,1236.29122837)(429.81550765,1236.21483954)(430.11411853,1236.14539514)
\curveto(431.02384007,1235.93706197)(431.72522842,1235.57942335)(432.21828361,1235.07247929)
\curveto(432.71133879,1234.55859079)(432.95786638,1233.87803575)(432.95786638,1233.03081417)
\closepath
}
}
{
\newrgbcolor{curcolor}{0 0 0}
\pscustom[linestyle=none,fillstyle=solid,fillcolor=curcolor]
{
\newpath
\moveto(445.49952628,1235.29122912)
\lineto(436.9266161,1235.29122912)
\curveto(436.9266161,1234.57595189)(437.03425491,1233.95095236)(437.24953252,1233.41623054)
\curveto(437.46481014,1232.88845316)(437.7599488,1232.45442571)(438.13494852,1232.11414819)
\curveto(438.49605936,1231.78081511)(438.92314237,1231.5308153)(439.41619755,1231.36414876)
\curveto(439.91619717,1231.19748222)(440.46480787,1231.11414895)(441.06202964,1231.11414895)
\curveto(441.85369571,1231.11414895)(442.64883399,1231.27039883)(443.4474445,1231.5828986)
\curveto(444.25299945,1231.9023428)(444.82591568,1232.21484256)(445.1661932,1232.52039789)
\lineto(445.27035979,1232.52039789)
\lineto(445.27035979,1230.38498284)
\curveto(444.61063806,1230.10720527)(443.93702746,1229.87456656)(443.24952798,1229.6870667)
\curveto(442.5620285,1229.49956684)(441.83980683,1229.40581691)(441.08286296,1229.40581691)
\curveto(439.15230886,1229.40581691)(437.64536556,1229.92664985)(436.56203304,1230.96831573)
\curveto(435.47870053,1232.01692605)(434.93703427,1233.50303603)(434.93703427,1235.42664569)
\curveto(434.93703427,1237.32942203)(435.45439499,1238.83983755)(436.48911643,1239.95789226)
\curveto(437.53078231,1241.07594697)(438.89883683,1241.63497432)(440.59327999,1241.63497432)
\curveto(442.16272325,1241.63497432)(443.37105567,1241.17664134)(444.21827725,1240.25997536)
\curveto(445.07244327,1239.34330939)(445.49952628,1238.04122704)(445.49952628,1236.35372832)
\closepath
\moveto(443.59327772,1236.79122799)
\curveto(443.58633328,1237.81900499)(443.32591681,1238.61414328)(442.81202831,1239.17664285)
\curveto(442.30508425,1239.73914243)(441.53077928,1240.02039221)(440.48911341,1240.02039221)
\curveto(439.44050309,1240.02039221)(438.60369816,1239.71136467)(437.97869864,1239.09330958)
\curveto(437.36064355,1238.47525449)(437.00994937,1237.70789396)(436.9266161,1236.79122799)
\closepath
}
}
{
\newrgbcolor{curcolor}{0 0 0}
\pscustom[linestyle=none,fillstyle=solid,fillcolor=curcolor]
{
\newpath
\moveto(457.90576289,1229.67665004)
\lineto(455.94743104,1229.67665004)
\lineto(455.94743104,1230.89539912)
\curveto(455.38493147,1230.40928837)(454.79812636,1230.03081644)(454.18701571,1229.75998331)
\curveto(453.57590506,1229.48915018)(452.91271111,1229.35373362)(452.19743388,1229.35373362)
\curveto(450.80854604,1229.35373362)(449.70438021,1229.88845543)(448.88493638,1230.95789907)
\curveto(448.072437,1232.0273427)(447.66618731,1233.50998047)(447.66618731,1235.40581237)
\curveto(447.66618731,1236.39192274)(447.80507609,1237.27039429)(448.08285366,1238.04122704)
\curveto(448.36757566,1238.81205979)(448.74951982,1239.4683093)(449.22868612,1240.00997555)
\curveto(449.70090799,1240.53775293)(450.24951869,1240.94053041)(450.87451821,1241.21830797)
\curveto(451.50646218,1241.49608554)(452.15923946,1241.63497432)(452.83285006,1241.63497432)
\curveto(453.44396071,1241.63497432)(453.98562697,1241.56900215)(454.45784883,1241.43705781)
\curveto(454.9300707,1241.3120579)(455.4265981,1241.11414139)(455.94743104,1240.84330826)
\lineto(455.94743104,1245.88497111)
\lineto(457.90576289,1245.88497111)
\closepath
\moveto(455.94743104,1232.5412312)
\lineto(455.94743104,1239.21830949)
\curveto(455.41965366,1239.45442042)(454.9474318,1239.61761474)(454.53076545,1239.70789245)
\curveto(454.11409909,1239.79817016)(453.65923833,1239.84330901)(453.16618315,1239.84330901)
\curveto(452.06896175,1239.84330901)(451.21479573,1239.46136486)(450.60368508,1238.69747655)
\curveto(449.99257444,1237.93358824)(449.68701911,1236.85025572)(449.68701911,1235.44747901)
\curveto(449.68701911,1234.06553561)(449.92313004,1233.01345307)(450.39535191,1232.29123139)
\curveto(450.86757377,1231.57595416)(451.62451764,1231.21831554)(452.66618352,1231.21831554)
\curveto(453.22173866,1231.21831554)(453.78423823,1231.33984322)(454.35368225,1231.5828986)
\curveto(454.92312626,1231.83289841)(455.45437586,1232.15234261)(455.94743104,1232.5412312)
\closepath
}
}
{
\newrgbcolor{curcolor}{0 0 0}
\pscustom[linestyle=none,fillstyle=solid,fillcolor=curcolor]
{
\newpath
\moveto(479.1765768,1235.48914564)
\curveto(479.1765768,1233.59331374)(478.69046606,1232.0967871)(477.71824457,1230.9995657)
\curveto(476.74602308,1229.90234431)(475.44394074,1229.35373362)(473.81199753,1229.35373362)
\curveto(472.16616544,1229.35373362)(470.85713865,1229.90234431)(469.88491716,1230.9995657)
\curveto(468.91964012,1232.0967871)(468.43700159,1233.59331374)(468.43700159,1235.48914564)
\curveto(468.43700159,1237.38497754)(468.91964012,1238.88150419)(469.88491716,1239.97872558)
\curveto(470.85713865,1241.08289141)(472.16616544,1241.63497432)(473.81199753,1241.63497432)
\curveto(475.44394074,1241.63497432)(476.74602308,1241.08289141)(477.71824457,1239.97872558)
\curveto(478.69046606,1238.88150419)(479.1765768,1237.38497754)(479.1765768,1235.48914564)
\closepath
\moveto(477.155745,1235.48914564)
\curveto(477.155745,1236.99608895)(476.86060633,1238.11414365)(476.270329,1238.84330977)
\curveto(475.68005167,1239.57942032)(474.86060784,1239.9474756)(473.81199753,1239.9474756)
\curveto(472.74949833,1239.9474756)(471.92311007,1239.57942032)(471.33283273,1238.84330977)
\curveto(470.74949984,1238.11414365)(470.4578334,1236.99608895)(470.4578334,1235.48914564)
\curveto(470.4578334,1234.03081341)(470.75297206,1232.92317536)(471.34324939,1232.16623149)
\curveto(471.93352672,1231.41623206)(472.75644277,1231.04123234)(473.81199753,1231.04123234)
\curveto(474.8536634,1231.04123234)(475.66963501,1231.41275984)(476.25991234,1232.15581483)
\curveto(476.85713411,1232.90581426)(477.155745,1234.01692453)(477.155745,1235.48914564)
\closepath
}
}
{
\newrgbcolor{curcolor}{0 0 0}
\pscustom[linestyle=none,fillstyle=solid,fillcolor=curcolor]
{
\newpath
\moveto(491.93698368,1229.67665004)
\lineto(489.97865182,1229.67665004)
\lineto(489.97865182,1236.30164503)
\curveto(489.97865182,1236.83636684)(489.94740185,1237.33636647)(489.8849019,1237.80164389)
\curveto(489.82240194,1238.27386576)(489.7078187,1238.64192103)(489.54115216,1238.90580972)
\curveto(489.36754118,1239.19747617)(489.11754137,1239.41275378)(488.79115272,1239.55164257)
\curveto(488.46476408,1239.69747579)(488.04115329,1239.7703924)(487.52032035,1239.7703924)
\curveto(486.98559853,1239.7703924)(486.42657118,1239.63844806)(485.84323829,1239.37455937)
\curveto(485.25990539,1239.11067068)(484.70087804,1238.77386538)(484.16615622,1238.36414347)
\lineto(484.16615622,1229.67665004)
\lineto(482.20782437,1229.67665004)
\lineto(482.20782437,1241.3120579)
\lineto(484.16615622,1241.3120579)
\lineto(484.16615622,1240.02039221)
\curveto(484.77726687,1240.52733627)(485.40921084,1240.92316931)(486.06198812,1241.20789131)
\curveto(486.71476541,1241.49261332)(487.38490379,1241.63497432)(488.07240327,1241.63497432)
\curveto(489.32934676,1241.63497432)(490.28767937,1241.25650239)(490.94740109,1240.49955852)
\curveto(491.60712281,1239.74261464)(491.93698368,1238.65233769)(491.93698368,1237.22872766)
\closepath
}
}
{
\newrgbcolor{curcolor}{0 0 0}
\pscustom[linestyle=none,fillstyle=solid,fillcolor=curcolor]
{
\newpath
\moveto(514.42655723,1243.35372302)
\lineto(508.88489476,1243.35372302)
\lineto(508.88489476,1229.67665004)
\lineto(506.82239632,1229.67665004)
\lineto(506.82239632,1243.35372302)
\lineto(501.28073384,1243.35372302)
\lineto(501.28073384,1245.18705497)
\lineto(514.42655723,1245.18705497)
\closepath
}
}
{
\newrgbcolor{curcolor}{0 0 0}
\pscustom[linestyle=none,fillstyle=solid,fillcolor=curcolor]
{
\newpath
\moveto(523.80155226,1235.29122912)
\lineto(515.22864208,1235.29122912)
\curveto(515.22864208,1234.57595189)(515.33628089,1233.95095236)(515.55155851,1233.41623054)
\curveto(515.76683612,1232.88845316)(516.06197479,1232.45442571)(516.4369745,1232.11414819)
\curveto(516.79808534,1231.78081511)(517.22516835,1231.5308153)(517.71822353,1231.36414876)
\curveto(518.21822315,1231.19748222)(518.76683385,1231.11414895)(519.36405562,1231.11414895)
\curveto(520.15572169,1231.11414895)(520.95085998,1231.27039883)(521.74947048,1231.5828986)
\curveto(522.55502543,1231.9023428)(523.12794166,1232.21484256)(523.46821918,1232.52039789)
\lineto(523.57238577,1232.52039789)
\lineto(523.57238577,1230.38498284)
\curveto(522.91266405,1230.10720527)(522.23905345,1229.87456656)(521.55155397,1229.6870667)
\curveto(520.86405449,1229.49956684)(520.14183281,1229.40581691)(519.38488894,1229.40581691)
\curveto(517.45433484,1229.40581691)(515.94739154,1229.92664985)(514.86405903,1230.96831573)
\curveto(513.78072651,1232.01692605)(513.23906025,1233.50303603)(513.23906025,1235.42664569)
\curveto(513.23906025,1237.32942203)(513.75642097,1238.83983755)(514.79114241,1239.95789226)
\curveto(515.83280829,1241.07594697)(517.20086281,1241.63497432)(518.89530598,1241.63497432)
\curveto(520.46474923,1241.63497432)(521.67308165,1241.17664134)(522.52030323,1240.25997536)
\curveto(523.37446925,1239.34330939)(523.80155226,1238.04122704)(523.80155226,1236.35372832)
\closepath
\moveto(521.89530371,1236.79122799)
\curveto(521.88835927,1237.81900499)(521.6279428,1238.61414328)(521.1140543,1239.17664285)
\curveto(520.60711024,1239.73914243)(519.83280527,1240.02039221)(518.79113939,1240.02039221)
\curveto(517.74252907,1240.02039221)(516.90572415,1239.71136467)(516.28072462,1239.09330958)
\curveto(515.66266953,1238.47525449)(515.31197535,1237.70789396)(515.22864208,1236.79122799)
\closepath
}
}
{
\newrgbcolor{curcolor}{0 0 0}
\pscustom[linestyle=none,fillstyle=solid,fillcolor=curcolor]
{
\newpath
\moveto(535.08278252,1233.03081417)
\curveto(535.08278252,1231.96831497)(534.64181063,1231.09678785)(533.75986685,1230.41623281)
\curveto(532.88486751,1229.73567777)(531.68695175,1229.39540025)(530.16611957,1229.39540025)
\curveto(529.30500911,1229.39540025)(528.51334304,1229.49609462)(527.79112137,1229.69748336)
\curveto(527.07584413,1229.90581653)(526.47515014,1230.13151081)(525.9890394,1230.37456618)
\lineto(525.9890394,1232.57248118)
\lineto(526.09320599,1232.57248118)
\curveto(526.71126107,1232.10720375)(527.39876055,1231.73567626)(528.15570443,1231.45789869)
\curveto(528.9126483,1231.18706556)(529.63834219,1231.051649)(530.33278611,1231.051649)
\curveto(531.19389657,1231.051649)(531.86750717,1231.19053778)(532.35361792,1231.46831535)
\curveto(532.83972866,1231.74609292)(533.08278403,1232.18359259)(533.08278403,1232.78081436)
\curveto(533.08278403,1233.23914734)(532.95083969,1233.5863693)(532.686951,1233.82248023)
\curveto(532.42306231,1234.05859117)(531.91611825,1234.2599799)(531.16611881,1234.42664644)
\curveto(530.88834125,1234.4891464)(530.52375819,1234.56206301)(530.07236964,1234.64539628)
\curveto(529.62792553,1234.72872955)(529.22167584,1234.81900726)(528.85362056,1234.91622941)
\curveto(527.832788,1235.18706254)(527.10709411,1235.58289557)(526.67653888,1236.10372851)
\curveto(526.25292809,1236.63150589)(526.04112269,1237.27733873)(526.04112269,1238.04122704)
\curveto(526.04112269,1238.52039335)(526.13834484,1238.97178189)(526.33278914,1239.39539269)
\curveto(526.53417787,1239.81900348)(526.83626098,1240.19747541)(527.23903845,1240.53080849)
\curveto(527.62792705,1240.85719713)(528.12098223,1241.11414139)(528.718204,1241.30164124)
\curveto(529.32237021,1241.49608554)(529.99598081,1241.59330769)(530.7390358,1241.59330769)
\curveto(531.43347972,1241.59330769)(532.13486808,1241.5065022)(532.84320088,1241.33289122)
\curveto(533.55847812,1241.16622468)(534.15222767,1240.96136372)(534.62444953,1240.71830835)
\lineto(534.62444953,1238.62455994)
\lineto(534.52028294,1238.62455994)
\curveto(534.02028332,1238.99261521)(533.41264489,1239.30164276)(532.69736766,1239.55164257)
\curveto(531.98209042,1239.80858682)(531.28070206,1239.93705894)(530.59320258,1239.93705894)
\curveto(529.87792534,1239.93705894)(529.27375913,1239.79817016)(528.78070395,1239.52039259)
\curveto(528.28764877,1239.24955946)(528.04112118,1238.84330977)(528.04112118,1238.30164351)
\curveto(528.04112118,1237.82247721)(528.19042662,1237.46136637)(528.48903751,1237.218311)
\curveto(528.78070395,1236.97525563)(529.25292582,1236.77733911)(529.9057031,1236.62456145)
\curveto(530.26681394,1236.54122818)(530.66959141,1236.45789491)(531.11403552,1236.37456164)
\curveto(531.56542407,1236.29122837)(531.94042378,1236.21483954)(532.23903467,1236.14539514)
\curveto(533.1487562,1235.93706197)(533.85014456,1235.57942335)(534.34319974,1235.07247929)
\curveto(534.83625493,1234.55859079)(535.08278252,1233.87803575)(535.08278252,1233.03081417)
\closepath
}
}
{
\newrgbcolor{curcolor}{0 0 0}
\pscustom[linestyle=none,fillstyle=solid,fillcolor=curcolor]
{
\newpath
\moveto(543.92652494,1229.78081663)
\curveto(543.55846967,1229.68359448)(543.15569219,1229.60373343)(542.71819253,1229.54123347)
\curveto(542.2876373,1229.47873352)(541.90222092,1229.44748354)(541.5619434,1229.44748354)
\curveto(540.3744443,1229.44748354)(539.4716672,1229.76692775)(538.85361212,1230.40581615)
\curveto(538.23555703,1231.04470456)(537.92652948,1232.06900934)(537.92652948,1233.47873049)
\lineto(537.92652948,1239.66622581)
\lineto(536.60361382,1239.66622581)
\lineto(536.60361382,1241.3120579)
\lineto(537.92652948,1241.3120579)
\lineto(537.92652948,1244.65580537)
\lineto(539.88486134,1244.65580537)
\lineto(539.88486134,1241.3120579)
\lineto(543.92652494,1241.3120579)
\lineto(543.92652494,1239.66622581)
\lineto(539.88486134,1239.66622581)
\lineto(539.88486134,1234.36414649)
\curveto(539.88486134,1233.75303584)(539.89875021,1233.27386954)(539.92652797,1232.92664758)
\curveto(539.95430573,1232.58637006)(540.05152788,1232.26692586)(540.21819442,1231.96831497)
\curveto(540.37097208,1231.6905374)(540.57930525,1231.48567645)(540.84319394,1231.3537321)
\curveto(541.11402707,1231.2287322)(541.52374898,1231.16623224)(542.07235968,1231.16623224)
\curveto(542.39180388,1231.16623224)(542.72513696,1231.2113711)(543.07235892,1231.30164881)
\curveto(543.41958088,1231.39887096)(543.66958069,1231.47873201)(543.82235836,1231.54123196)
\lineto(543.92652494,1231.54123196)
\closepath
}
}
{
\newrgbcolor{curcolor}{0 0 0}
\pscustom[linestyle=none,fillstyle=solid,fillcolor=curcolor]
{
\newpath
\moveto(564.0931807,1240.49955852)
\curveto(564.0931807,1239.81205904)(563.97165301,1239.17317063)(563.72859764,1238.5828933)
\curveto(563.49248671,1237.99956041)(563.15915363,1237.49261635)(562.7285984,1237.06206112)
\curveto(562.19387658,1236.5273393)(561.56193261,1236.12456183)(560.8327665,1235.8537287)
\curveto(560.10360038,1235.58984001)(559.18346219,1235.45789566)(558.07235192,1235.45789566)
\lineto(556.00985348,1235.45789566)
\lineto(556.00985348,1229.67665004)
\lineto(553.94735504,1229.67665004)
\lineto(553.94735504,1245.18705497)
\lineto(558.15568519,1245.18705497)
\curveto(559.08624004,1245.18705497)(559.87443389,1245.10719392)(560.52026673,1244.94747182)
\curveto(561.16609958,1244.79469416)(561.73901581,1244.55163878)(562.23901543,1244.2183057)
\curveto(562.82929277,1243.82247267)(563.28415353,1243.32941749)(563.60359774,1242.73914016)
\curveto(563.92998638,1242.14886282)(564.0931807,1241.40233561)(564.0931807,1240.49955852)
\closepath
\moveto(561.94734899,1240.44747522)
\curveto(561.94734899,1240.98219704)(561.85359906,1241.44747447)(561.6660992,1241.8433075)
\curveto(561.47859934,1242.23914053)(561.19387734,1242.56205696)(560.81193318,1242.81205677)
\curveto(560.4786001,1243.02733438)(560.09665594,1243.18011204)(559.66610071,1243.27038975)
\curveto(559.24248992,1243.3676119)(558.70429589,1243.41622298)(558.0515186,1243.41622298)
\lineto(556.00985348,1243.41622298)
\lineto(556.00985348,1237.218311)
\lineto(557.7494355,1237.218311)
\curveto(558.5827682,1237.218311)(559.25985102,1237.29122761)(559.78068396,1237.43706083)
\curveto(560.3015169,1237.5898385)(560.72512769,1237.82942165)(561.05151633,1238.15581029)
\curveto(561.37790497,1238.48914337)(561.60707147,1238.83983755)(561.73901581,1239.20789283)
\curveto(561.8779046,1239.5759481)(561.94734899,1239.98914224)(561.94734899,1240.44747522)
\closepath
}
}
{
\newrgbcolor{curcolor}{0 0 0}
\pscustom[linestyle=none,fillstyle=solid,fillcolor=curcolor]
{
\newpath
\moveto(575.17651018,1229.67665004)
\lineto(573.22859499,1229.67665004)
\lineto(573.22859499,1230.91623243)
\curveto(573.05498401,1230.79817697)(572.81887307,1230.63151043)(572.52026219,1230.41623281)
\curveto(572.22859574,1230.20789964)(571.94387374,1230.0412331)(571.66609617,1229.91623319)
\curveto(571.33970753,1229.75651109)(570.96470781,1229.62456674)(570.54109702,1229.52040016)
\curveto(570.11748623,1229.40928913)(569.62095883,1229.35373362)(569.05151481,1229.35373362)
\curveto(568.0029045,1229.35373362)(567.11401628,1229.70095558)(566.38485016,1230.39539949)
\curveto(565.65568405,1231.08984341)(565.29110099,1231.97525941)(565.29110099,1233.05164748)
\curveto(565.29110099,1233.93359126)(565.47860085,1234.64539628)(565.85360057,1235.18706254)
\curveto(566.23554472,1235.73567323)(566.77721098,1236.16622846)(567.47859934,1236.47872823)
\curveto(568.18693213,1236.79122799)(569.03762593,1237.00303338)(570.03068074,1237.11414441)
\curveto(571.02373554,1237.22525544)(572.08970696,1237.30858871)(573.22859499,1237.36414422)
\lineto(573.22859499,1237.66622733)
\curveto(573.22859499,1238.11067144)(573.14873394,1238.47872671)(572.98901183,1238.77039316)
\curveto(572.83623417,1239.0620596)(572.61401212,1239.2912261)(572.32234567,1239.45789264)
\curveto(572.0445681,1239.61761474)(571.71123502,1239.72525355)(571.32234643,1239.78080906)
\curveto(570.93345783,1239.83636457)(570.52720814,1239.86414233)(570.10359735,1239.86414233)
\curveto(569.58970885,1239.86414233)(569.01679262,1239.79469794)(568.38484865,1239.65580916)
\curveto(567.75290468,1239.52386481)(567.1001274,1239.32942051)(566.4265168,1239.07247626)
\lineto(566.32235021,1239.07247626)
\lineto(566.32235021,1241.06205809)
\curveto(566.70429437,1241.16622468)(567.25637728,1241.28080793)(567.97859896,1241.40580783)
\curveto(568.70082063,1241.53080774)(569.41262565,1241.59330769)(570.11401401,1241.59330769)
\curveto(570.93345783,1241.59330769)(571.64526285,1241.5238633)(572.24942906,1241.38497451)
\curveto(572.86053971,1241.25303017)(573.38831709,1241.02386368)(573.8327612,1240.69747503)
\curveto(574.27026086,1240.37803083)(574.60359395,1239.9648367)(574.83276044,1239.45789264)
\curveto(575.06192693,1238.95094858)(575.17651018,1238.32247683)(575.17651018,1237.5724774)
\closepath
\moveto(573.22859499,1232.5412312)
\lineto(573.22859499,1235.78081209)
\curveto(572.63137322,1235.74608989)(571.92651264,1235.6940066)(571.11401325,1235.6245622)
\curveto(570.30845831,1235.55511781)(569.6695699,1235.45442344)(569.19734804,1235.3224791)
\curveto(568.63484846,1235.162757)(568.17998769,1234.91275719)(567.83276574,1234.57247967)
\curveto(567.48554378,1234.23914659)(567.3119328,1233.77734138)(567.3119328,1233.18706405)
\curveto(567.3119328,1232.52039789)(567.51332153,1232.01692605)(567.91609901,1231.67664853)
\curveto(568.31887648,1231.34331544)(568.93345935,1231.1766489)(569.75984761,1231.1766489)
\curveto(570.44734709,1231.1766489)(571.07581884,1231.30859325)(571.64526285,1231.57248194)
\curveto(572.21470686,1231.84331507)(572.74248424,1232.16623149)(573.22859499,1232.5412312)
\closepath
}
}
{
\newrgbcolor{curcolor}{0 0 0}
\pscustom[linestyle=none,fillstyle=solid,fillcolor=curcolor]
{
\newpath
\moveto(586.2077423,1239.17664285)
\lineto(586.10357572,1239.17664285)
\curveto(585.81190927,1239.24608724)(585.52718726,1239.29469832)(585.24940969,1239.32247607)
\curveto(584.97857657,1239.35719827)(584.65566014,1239.37455937)(584.28066043,1239.37455937)
\curveto(583.67649422,1239.37455937)(583.09316133,1239.2391428)(582.53066175,1238.96830968)
\curveto(581.96816218,1238.70442099)(581.42649592,1238.36067125)(580.90566298,1237.93706046)
\lineto(580.90566298,1229.67665004)
\lineto(578.94733113,1229.67665004)
\lineto(578.94733113,1241.3120579)
\lineto(580.90566298,1241.3120579)
\lineto(580.90566298,1239.5933092)
\curveto(581.68344017,1240.21830873)(582.36746743,1240.65928062)(582.95774476,1240.91622487)
\curveto(583.55496653,1241.18011356)(584.16260496,1241.3120579)(584.78066005,1241.3120579)
\curveto(585.12093757,1241.3120579)(585.36746516,1241.30164124)(585.52024282,1241.28080793)
\curveto(585.67302049,1241.26691905)(585.90218698,1241.23566907)(586.2077423,1241.187058)
\closepath
}
}
{
\newrgbcolor{curcolor}{0 0 0}
\pscustom[linestyle=none,fillstyle=solid,fillcolor=curcolor]
{
\newpath
\moveto(596.71816211,1229.67665004)
\lineto(594.77024691,1229.67665004)
\lineto(594.77024691,1230.91623243)
\curveto(594.59663594,1230.79817697)(594.360525,1230.63151043)(594.06191412,1230.41623281)
\curveto(593.77024767,1230.20789964)(593.48552566,1230.0412331)(593.2077481,1229.91623319)
\curveto(592.88135946,1229.75651109)(592.50635974,1229.62456674)(592.08274895,1229.52040016)
\curveto(591.65913816,1229.40928913)(591.16261076,1229.35373362)(590.59316674,1229.35373362)
\curveto(589.54455642,1229.35373362)(588.65566821,1229.70095558)(587.92650209,1230.39539949)
\curveto(587.19733598,1231.08984341)(586.83275292,1231.97525941)(586.83275292,1233.05164748)
\curveto(586.83275292,1233.93359126)(587.02025278,1234.64539628)(587.39525249,1235.18706254)
\curveto(587.77719665,1235.73567323)(588.31886291,1236.16622846)(589.02025127,1236.47872823)
\curveto(589.72858406,1236.79122799)(590.57927786,1237.00303338)(591.57233267,1237.11414441)
\curveto(592.56538747,1237.22525544)(593.63135889,1237.30858871)(594.77024691,1237.36414422)
\lineto(594.77024691,1237.66622733)
\curveto(594.77024691,1238.11067144)(594.69038586,1238.47872671)(594.53066376,1238.77039316)
\curveto(594.3778861,1239.0620596)(594.15566405,1239.2912261)(593.8639976,1239.45789264)
\curveto(593.58622003,1239.61761474)(593.25288695,1239.72525355)(592.86399836,1239.78080906)
\curveto(592.47510976,1239.83636457)(592.06886007,1239.86414233)(591.64524928,1239.86414233)
\curveto(591.13136078,1239.86414233)(590.55844455,1239.79469794)(589.92650058,1239.65580916)
\curveto(589.29455661,1239.52386481)(588.64177933,1239.32942051)(587.96816873,1239.07247626)
\lineto(587.86400214,1239.07247626)
\lineto(587.86400214,1241.06205809)
\curveto(588.2459463,1241.16622468)(588.79802921,1241.28080793)(589.52025089,1241.40580783)
\curveto(590.24247256,1241.53080774)(590.95427758,1241.59330769)(591.65566594,1241.59330769)
\curveto(592.47510976,1241.59330769)(593.18691478,1241.5238633)(593.79108099,1241.38497451)
\curveto(594.40219164,1241.25303017)(594.92996902,1241.02386368)(595.37441312,1240.69747503)
\curveto(595.81191279,1240.37803083)(596.14524587,1239.9648367)(596.37441237,1239.45789264)
\curveto(596.60357886,1238.95094858)(596.71816211,1238.32247683)(596.71816211,1237.5724774)
\closepath
\moveto(594.77024691,1232.5412312)
\lineto(594.77024691,1235.78081209)
\curveto(594.17302514,1235.74608989)(593.46816457,1235.6940066)(592.65566518,1235.6245622)
\curveto(591.85011024,1235.55511781)(591.21122183,1235.45442344)(590.73899996,1235.3224791)
\curveto(590.17650039,1235.162757)(589.72163962,1234.91275719)(589.37441766,1234.57247967)
\curveto(589.0271957,1234.23914659)(588.85358472,1233.77734138)(588.85358472,1233.18706405)
\curveto(588.85358472,1232.52039789)(589.05497346,1232.01692605)(589.45775093,1231.67664853)
\curveto(589.86052841,1231.34331544)(590.47511128,1231.1766489)(591.30149954,1231.1766489)
\curveto(591.98899902,1231.1766489)(592.61747077,1231.30859325)(593.18691478,1231.57248194)
\curveto(593.75635879,1231.84331507)(594.28413617,1232.16623149)(594.77024691,1232.5412312)
\closepath
}
}
{
\newrgbcolor{curcolor}{0 0 0}
\pscustom[linestyle=none,fillstyle=solid,fillcolor=curcolor]
{
\newpath
\moveto(617.46813688,1229.67665004)
\lineto(615.50980503,1229.67665004)
\lineto(615.50980503,1236.30164503)
\curveto(615.50980503,1236.80164465)(615.48549949,1237.28428317)(615.43688842,1237.7495606)
\curveto(615.39522178,1238.21483802)(615.30147185,1238.58636552)(615.15563863,1238.86414309)
\curveto(614.99591653,1239.16275397)(614.76675003,1239.38844825)(614.46813915,1239.54122591)
\curveto(614.16952826,1239.69400357)(613.73897303,1239.7703924)(613.17647346,1239.7703924)
\curveto(612.62786276,1239.7703924)(612.07925207,1239.63150362)(611.53064137,1239.35372605)
\curveto(610.98203067,1239.08289292)(610.43341998,1238.73567096)(609.88480928,1238.31206017)
\curveto(609.9056426,1238.15233807)(609.9230037,1237.96483821)(609.93689258,1237.7495606)
\curveto(609.95078146,1237.54122742)(609.95772589,1237.33289425)(609.95772589,1237.12456107)
\lineto(609.95772589,1229.67665004)
\lineto(607.99939404,1229.67665004)
\lineto(607.99939404,1236.30164503)
\curveto(607.99939404,1236.81553353)(607.97508851,1237.30164427)(607.92647743,1237.75997726)
\curveto(607.8848108,1238.22525468)(607.79106087,1238.59678218)(607.64522764,1238.87455975)
\curveto(607.48550554,1239.17317063)(607.25633905,1239.39539269)(606.95772816,1239.54122591)
\curveto(606.65911728,1239.69400357)(606.22856205,1239.7703924)(605.66606247,1239.7703924)
\curveto(605.13134066,1239.7703924)(604.59314662,1239.63844806)(604.05148036,1239.37455937)
\curveto(603.51675855,1239.11067068)(602.98203673,1238.77386538)(602.44731491,1238.36414347)
\lineto(602.44731491,1229.67665004)
\lineto(600.48898306,1229.67665004)
\lineto(600.48898306,1241.3120579)
\lineto(602.44731491,1241.3120579)
\lineto(602.44731491,1240.02039221)
\curveto(603.05842556,1240.52733627)(603.66606399,1240.92316931)(604.2702302,1241.20789131)
\curveto(604.88134085,1241.49261332)(605.53064591,1241.63497432)(606.21814539,1241.63497432)
\curveto(607.00981146,1241.63497432)(607.67994984,1241.46830778)(608.22856054,1241.1349747)
\curveto(608.78411567,1240.80164162)(609.1973098,1240.33983642)(609.46814293,1239.74955908)
\curveto(610.259809,1240.41622525)(610.98203067,1240.89539155)(611.63480796,1241.187058)
\curveto(612.28758524,1241.48566888)(612.98550138,1241.63497432)(613.72855637,1241.63497432)
\curveto(615.00633319,1241.63497432)(615.9473047,1241.24608573)(616.55147091,1240.46830854)
\curveto(617.16258155,1239.69747579)(617.46813688,1238.6176155)(617.46813688,1237.22872766)
\closepath
}
}
{
\newrgbcolor{curcolor}{0 0 0}
\pscustom[linestyle=none,fillstyle=solid,fillcolor=curcolor]
{
\newpath
\moveto(630.97854653,1235.29122912)
\lineto(622.40563635,1235.29122912)
\curveto(622.40563635,1234.57595189)(622.51327515,1233.95095236)(622.72855277,1233.41623054)
\curveto(622.94383038,1232.88845316)(623.23896905,1232.45442571)(623.61396877,1232.11414819)
\curveto(623.9750796,1231.78081511)(624.40216261,1231.5308153)(624.8952178,1231.36414876)
\curveto(625.39521742,1231.19748222)(625.94382811,1231.11414895)(626.54104988,1231.11414895)
\curveto(627.33271595,1231.11414895)(628.12785424,1231.27039883)(628.92646475,1231.5828986)
\curveto(629.73201969,1231.9023428)(630.30493593,1232.21484256)(630.64521345,1232.52039789)
\lineto(630.74938003,1232.52039789)
\lineto(630.74938003,1230.38498284)
\curveto(630.08965831,1230.10720527)(629.41604771,1229.87456656)(628.72854823,1229.6870667)
\curveto(628.04104875,1229.49956684)(627.31882707,1229.40581691)(626.5618832,1229.40581691)
\curveto(624.63132911,1229.40581691)(623.1243858,1229.92664985)(622.04105329,1230.96831573)
\curveto(620.95772078,1232.01692605)(620.41605452,1233.50303603)(620.41605452,1235.42664569)
\curveto(620.41605452,1237.32942203)(620.93341524,1238.83983755)(621.96813668,1239.95789226)
\curveto(623.00980256,1241.07594697)(624.37785708,1241.63497432)(626.07230024,1241.63497432)
\curveto(627.6417435,1241.63497432)(628.85007592,1241.17664134)(629.6972975,1240.25997536)
\curveto(630.55146352,1239.34330939)(630.97854653,1238.04122704)(630.97854653,1236.35372832)
\closepath
\moveto(629.07229797,1236.79122799)
\curveto(629.06535353,1237.81900499)(628.80493706,1238.61414328)(628.29104856,1239.17664285)
\curveto(627.7841045,1239.73914243)(627.00979953,1240.02039221)(625.96813365,1240.02039221)
\curveto(624.91952333,1240.02039221)(624.08271841,1239.71136467)(623.45771888,1239.09330958)
\curveto(622.8396638,1238.47525449)(622.48896962,1237.70789396)(622.40563635,1236.79122799)
\closepath
}
}
{
\newrgbcolor{curcolor}{0 0 0}
\pscustom[linestyle=none,fillstyle=solid,fillcolor=curcolor]
{
\newpath
\moveto(639.98895958,1229.78081663)
\curveto(639.62090431,1229.68359448)(639.21812683,1229.60373343)(638.78062716,1229.54123347)
\curveto(638.35007193,1229.47873352)(637.96465556,1229.44748354)(637.62437804,1229.44748354)
\curveto(636.43687894,1229.44748354)(635.53410184,1229.76692775)(634.91604676,1230.40581615)
\curveto(634.29799167,1231.04470456)(633.98896412,1232.06900934)(633.98896412,1233.47873049)
\lineto(633.98896412,1239.66622581)
\lineto(632.66604846,1239.66622581)
\lineto(632.66604846,1241.3120579)
\lineto(633.98896412,1241.3120579)
\lineto(633.98896412,1244.65580537)
\lineto(635.94729597,1244.65580537)
\lineto(635.94729597,1241.3120579)
\lineto(639.98895958,1241.3120579)
\lineto(639.98895958,1239.66622581)
\lineto(635.94729597,1239.66622581)
\lineto(635.94729597,1234.36414649)
\curveto(635.94729597,1233.75303584)(635.96118485,1233.27386954)(635.98896261,1232.92664758)
\curveto(636.01674037,1232.58637006)(636.11396252,1232.26692586)(636.28062906,1231.96831497)
\curveto(636.43340672,1231.6905374)(636.64173989,1231.48567645)(636.90562858,1231.3537321)
\curveto(637.17646171,1231.2287322)(637.58618362,1231.16623224)(638.13479432,1231.16623224)
\curveto(638.45423852,1231.16623224)(638.7875716,1231.2113711)(639.13479356,1231.30164881)
\curveto(639.48201552,1231.39887096)(639.73201533,1231.47873201)(639.884793,1231.54123196)
\lineto(639.98895958,1231.54123196)
\closepath
}
}
{
\newrgbcolor{curcolor}{0 0 0}
\pscustom[linestyle=none,fillstyle=solid,fillcolor=curcolor]
{
\newpath
\moveto(652.09311576,1235.29122912)
\lineto(643.52020558,1235.29122912)
\curveto(643.52020558,1234.57595189)(643.62784439,1233.95095236)(643.843122,1233.41623054)
\curveto(644.05839962,1232.88845316)(644.35353829,1232.45442571)(644.728538,1232.11414819)
\curveto(645.08964884,1231.78081511)(645.51673185,1231.5308153)(646.00978703,1231.36414876)
\curveto(646.50978665,1231.19748222)(647.05839735,1231.11414895)(647.65561912,1231.11414895)
\curveto(648.44728519,1231.11414895)(649.24242348,1231.27039883)(650.04103398,1231.5828986)
\curveto(650.84658893,1231.9023428)(651.41950516,1232.21484256)(651.75978268,1232.52039789)
\lineto(651.86394927,1232.52039789)
\lineto(651.86394927,1230.38498284)
\curveto(651.20422755,1230.10720527)(650.53061695,1229.87456656)(649.84311747,1229.6870667)
\curveto(649.15561799,1229.49956684)(648.43339631,1229.40581691)(647.67645244,1229.40581691)
\curveto(645.74589834,1229.40581691)(644.23895504,1229.92664985)(643.15562253,1230.96831573)
\curveto(642.07229001,1232.01692605)(641.53062375,1233.50303603)(641.53062375,1235.42664569)
\curveto(641.53062375,1237.32942203)(642.04798447,1238.83983755)(643.08270591,1239.95789226)
\curveto(644.12437179,1241.07594697)(645.49242631,1241.63497432)(647.18686948,1241.63497432)
\curveto(648.75631273,1241.63497432)(649.96464515,1241.17664134)(650.81186673,1240.25997536)
\curveto(651.66603275,1239.34330939)(652.09311576,1238.04122704)(652.09311576,1236.35372832)
\closepath
\moveto(650.18686721,1236.79122799)
\curveto(650.17992277,1237.81900499)(649.9195063,1238.61414328)(649.4056178,1239.17664285)
\curveto(648.89867374,1239.73914243)(648.12436877,1240.02039221)(647.08270289,1240.02039221)
\curveto(646.03409257,1240.02039221)(645.19728765,1239.71136467)(644.57228812,1239.09330958)
\curveto(643.95423303,1238.47525449)(643.60353885,1237.70789396)(643.52020558,1236.79122799)
\closepath
}
}
{
\newrgbcolor{curcolor}{0 0 0}
\pscustom[linestyle=none,fillstyle=solid,fillcolor=curcolor]
{
\newpath
\moveto(662.3222779,1239.17664285)
\lineto(662.21811131,1239.17664285)
\curveto(661.92644486,1239.24608724)(661.64172286,1239.29469832)(661.36394529,1239.32247607)
\curveto(661.09311216,1239.35719827)(660.77019574,1239.37455937)(660.39519602,1239.37455937)
\curveto(659.79102981,1239.37455937)(659.20769692,1239.2391428)(658.64519735,1238.96830968)
\curveto(658.08269777,1238.70442099)(657.54103152,1238.36067125)(657.02019858,1237.93706046)
\lineto(657.02019858,1229.67665004)
\lineto(655.06186672,1229.67665004)
\lineto(655.06186672,1241.3120579)
\lineto(657.02019858,1241.3120579)
\lineto(657.02019858,1239.5933092)
\curveto(657.79797577,1240.21830873)(658.48200303,1240.65928062)(659.07228036,1240.91622487)
\curveto(659.66950213,1241.18011356)(660.27714056,1241.3120579)(660.89519564,1241.3120579)
\curveto(661.23547316,1241.3120579)(661.48200076,1241.30164124)(661.63477842,1241.28080793)
\curveto(661.78755608,1241.26691905)(662.01672257,1241.23566907)(662.3222779,1241.187058)
\closepath
}
}
{
\newrgbcolor{curcolor}{0 0 0}
\pscustom[linestyle=none,fillstyle=solid,fillcolor=curcolor]
{
\newpath
\moveto(672.47850852,1233.03081417)
\curveto(672.47850852,1231.96831497)(672.03753663,1231.09678785)(671.15559285,1230.41623281)
\curveto(670.28059352,1229.73567777)(669.08267775,1229.39540025)(667.56184557,1229.39540025)
\curveto(666.70073511,1229.39540025)(665.90906904,1229.49609462)(665.18684737,1229.69748336)
\curveto(664.47157013,1229.90581653)(663.87087614,1230.13151081)(663.3847654,1230.37456618)
\lineto(663.3847654,1232.57248118)
\lineto(663.48893199,1232.57248118)
\curveto(664.10698707,1232.10720375)(664.79448655,1231.73567626)(665.55143043,1231.45789869)
\curveto(666.3083743,1231.18706556)(667.03406819,1231.051649)(667.72851211,1231.051649)
\curveto(668.58962257,1231.051649)(669.26323317,1231.19053778)(669.74934392,1231.46831535)
\curveto(670.23545466,1231.74609292)(670.47851003,1232.18359259)(670.47851003,1232.78081436)
\curveto(670.47851003,1233.23914734)(670.34656569,1233.5863693)(670.082677,1233.82248023)
\curveto(669.81878831,1234.05859117)(669.31184425,1234.2599799)(668.56184482,1234.42664644)
\curveto(668.28406725,1234.4891464)(667.91948419,1234.56206301)(667.46809564,1234.64539628)
\curveto(667.02365153,1234.72872955)(666.61740184,1234.81900726)(666.24934657,1234.91622941)
\curveto(665.228514,1235.18706254)(664.50282011,1235.58289557)(664.07226488,1236.10372851)
\curveto(663.64865409,1236.63150589)(663.43684869,1237.27733873)(663.43684869,1238.04122704)
\curveto(663.43684869,1238.52039335)(663.53407084,1238.97178189)(663.72851514,1239.39539269)
\curveto(663.92990388,1239.81900348)(664.23198698,1240.19747541)(664.63476445,1240.53080849)
\curveto(665.02365305,1240.85719713)(665.51670823,1241.11414139)(666.11393,1241.30164124)
\curveto(666.71809621,1241.49608554)(667.39170681,1241.59330769)(668.13476181,1241.59330769)
\curveto(668.82920572,1241.59330769)(669.53059408,1241.5065022)(670.23892688,1241.33289122)
\curveto(670.95420412,1241.16622468)(671.54795367,1240.96136372)(672.02017553,1240.71830835)
\lineto(672.02017553,1238.62455994)
\lineto(671.91600894,1238.62455994)
\curveto(671.41600932,1238.99261521)(670.80837089,1239.30164276)(670.09309366,1239.55164257)
\curveto(669.37781642,1239.80858682)(668.67642806,1239.93705894)(667.98892858,1239.93705894)
\curveto(667.27365135,1239.93705894)(666.66948514,1239.79817016)(666.17642995,1239.52039259)
\curveto(665.68337477,1239.24955946)(665.43684718,1238.84330977)(665.43684718,1238.30164351)
\curveto(665.43684718,1237.82247721)(665.58615262,1237.46136637)(665.88476351,1237.218311)
\curveto(666.17642995,1236.97525563)(666.64865182,1236.77733911)(667.3014291,1236.62456145)
\curveto(667.66253994,1236.54122818)(668.06531741,1236.45789491)(668.50976152,1236.37456164)
\curveto(668.96115007,1236.29122837)(669.33614979,1236.21483954)(669.63476067,1236.14539514)
\curveto(670.5444822,1235.93706197)(671.24587056,1235.57942335)(671.73892575,1235.07247929)
\curveto(672.23198093,1234.55859079)(672.47850852,1233.87803575)(672.47850852,1233.03081417)
\closepath
}
}
{
\newrgbcolor{curcolor}{0 0 0}
\pscustom[linestyle=none,fillstyle=solid,fillcolor=curcolor]
{
\newpath
\moveto(679.34310019,1232.64539779)
\lineto(676.40560242,1225.82248629)
\lineto(674.88477023,1225.82248629)
\lineto(676.69726886,1232.64539779)
\closepath
}
}
{
\newrgbcolor{curcolor}{0 0 0}
\pscustom[linestyle=none,fillstyle=solid,fillcolor=curcolor]
{
\newpath
\moveto(386.18706702,1208.85374913)
\lineto(379.68707194,1208.85374913)
\lineto(379.68707194,1210.73916437)
\lineto(386.18706702,1210.73916437)
\closepath
}
}
{
\newrgbcolor{curcolor}{0 0 0}
\pscustom[linestyle=none,fillstyle=solid,fillcolor=curcolor]
{
\newpath
\moveto(409.34329974,1204.1350027)
\curveto(408.96135558,1203.96833615)(408.61413362,1203.81208627)(408.30163386,1203.66625305)
\curveto(407.99607854,1203.52041983)(407.59330106,1203.36764216)(407.09330144,1203.20792006)
\curveto(406.66969065,1203.07597572)(406.20788545,1202.96486469)(405.70788582,1202.87458698)
\curveto(405.21483064,1202.77736483)(404.66969216,1202.72875376)(404.07247039,1202.72875376)
\curveto(402.94747125,1202.72875376)(401.92316646,1202.88500364)(400.99955605,1203.1975034)
\curveto(400.08289008,1203.51694761)(399.28427957,1204.01347501)(398.60372453,1204.68708561)
\curveto(397.93705837,1205.34680733)(397.41622543,1206.18361226)(397.04122571,1207.19750038)
\curveto(396.666226,1208.21833294)(396.47872614,1209.40235982)(396.47872614,1210.74958102)
\curveto(396.47872614,1212.02735784)(396.65928156,1213.16971808)(397.0203924,1214.17666176)
\curveto(397.38150323,1215.18360545)(397.90233617,1216.03429925)(398.58289121,1216.72874317)
\curveto(399.24261294,1217.40235377)(400.03775122,1217.91624227)(400.96830608,1218.27040867)
\curveto(401.90580537,1218.62457507)(402.94399903,1218.80165827)(404.08288705,1218.80165827)
\curveto(404.91621976,1218.80165827)(405.74608024,1218.7009639)(406.5724685,1218.49957516)
\curveto(407.40580121,1218.29818642)(408.32941162,1217.94402003)(409.34329974,1217.43707596)
\lineto(409.34329974,1214.98916115)
\lineto(409.18704986,1214.98916115)
\curveto(408.33288384,1215.70443839)(407.48566226,1216.22527133)(406.64538511,1216.55165997)
\curveto(405.80510797,1216.87804861)(404.9058031,1217.04124293)(403.94747049,1217.04124293)
\curveto(403.16274886,1217.04124293)(402.45441606,1216.91277081)(401.8224721,1216.65582656)
\curveto(401.19747257,1216.40582674)(400.63844521,1216.01346593)(400.14539003,1215.47874411)
\curveto(399.66622373,1214.95791117)(399.29122401,1214.29818945)(399.02039088,1213.49957894)
\curveto(398.75650219,1212.70791288)(398.62455785,1211.7912469)(398.62455785,1210.74958102)
\curveto(398.62455785,1209.65930407)(398.77039107,1208.72180478)(399.06205752,1207.93708315)
\curveto(399.3606684,1207.15236152)(399.74261256,1206.51347312)(400.20788998,1206.02041794)
\curveto(400.69400073,1205.50652944)(401.25997252,1205.12458528)(401.90580537,1204.87458547)
\curveto(402.55858265,1204.6315301)(403.24608213,1204.51000241)(403.96830381,1204.51000241)
\curveto(404.96135861,1204.51000241)(405.89191346,1204.68014117)(406.75996836,1205.02041869)
\curveto(407.62802326,1205.36069621)(408.44052264,1205.87111249)(409.19746652,1206.55166753)
\lineto(409.34329974,1206.55166753)
\closepath
}
}
{
\newrgbcolor{curcolor}{0 0 0}
\pscustom[linestyle=none,fillstyle=solid,fillcolor=curcolor]
{
\newpath
\moveto(422.02037414,1208.82249915)
\curveto(422.02037414,1206.92666725)(421.5342634,1205.4301406)(420.56204191,1204.33291921)
\curveto(419.58982042,1203.23569782)(418.28773807,1202.68708712)(416.65579486,1202.68708712)
\curveto(415.00996278,1202.68708712)(413.70093599,1203.23569782)(412.7287145,1204.33291921)
\curveto(411.76343745,1205.4301406)(411.28079893,1206.92666725)(411.28079893,1208.82249915)
\curveto(411.28079893,1210.71833105)(411.76343745,1212.21485769)(412.7287145,1213.31207909)
\curveto(413.70093599,1214.41624492)(415.00996278,1214.96832783)(416.65579486,1214.96832783)
\curveto(418.28773807,1214.96832783)(419.58982042,1214.41624492)(420.56204191,1213.31207909)
\curveto(421.5342634,1212.21485769)(422.02037414,1210.71833105)(422.02037414,1208.82249915)
\closepath
\moveto(419.99954233,1208.82249915)
\curveto(419.99954233,1210.32944245)(419.70440367,1211.44749716)(419.11412634,1212.17666328)
\curveto(418.52384901,1212.91277383)(417.70440518,1213.28082911)(416.65579486,1213.28082911)
\curveto(415.59329567,1213.28082911)(414.7669074,1212.91277383)(414.17663007,1212.17666328)
\curveto(413.59329718,1211.44749716)(413.30163073,1210.32944245)(413.30163073,1208.82249915)
\curveto(413.30163073,1207.36416692)(413.5967694,1206.25652887)(414.18704673,1205.499585)
\curveto(414.77732406,1204.74958556)(415.60024011,1204.37458585)(416.65579486,1204.37458585)
\curveto(417.69746074,1204.37458585)(418.51343235,1204.74611334)(419.10370968,1205.48916834)
\curveto(419.70093145,1206.23916777)(419.99954233,1207.35027804)(419.99954233,1208.82249915)
\closepath
}
}
{
\newrgbcolor{curcolor}{0 0 0}
\pscustom[linestyle=none,fillstyle=solid,fillcolor=curcolor]
{
\newpath
\moveto(434.78078101,1203.01000355)
\lineto(432.82244916,1203.01000355)
\lineto(432.82244916,1209.63499853)
\curveto(432.82244916,1210.16972035)(432.79119919,1210.66971997)(432.72869923,1211.1349974)
\curveto(432.66619928,1211.60721926)(432.55161603,1211.97527454)(432.38494949,1212.23916323)
\curveto(432.21133851,1212.53082968)(431.9613387,1212.74610729)(431.63495006,1212.88499608)
\curveto(431.30856142,1213.0308293)(430.88495063,1213.10374591)(430.36411769,1213.10374591)
\curveto(429.82939587,1213.10374591)(429.27036852,1212.97180157)(428.68703563,1212.70791288)
\curveto(428.10370273,1212.44402419)(427.54467538,1212.10721889)(427.00995356,1211.69749697)
\lineto(427.00995356,1203.01000355)
\lineto(425.05162171,1203.01000355)
\lineto(425.05162171,1214.64541141)
\lineto(427.00995356,1214.64541141)
\lineto(427.00995356,1213.35374572)
\curveto(427.62106421,1213.86068978)(428.25300818,1214.25652282)(428.90578546,1214.54124482)
\curveto(429.55856274,1214.82596683)(430.22870113,1214.96832783)(430.91620061,1214.96832783)
\curveto(432.1731441,1214.96832783)(433.13147671,1214.5898559)(433.79119843,1213.83291202)
\curveto(434.45092015,1213.07596815)(434.78078101,1211.9856912)(434.78078101,1210.56208117)
\closepath
}
}
{
\newrgbcolor{curcolor}{0 0 0}
\pscustom[linestyle=none,fillstyle=solid,fillcolor=curcolor]
{
\newpath
\moveto(444.81202138,1217.32249272)
\lineto(444.7078548,1217.32249272)
\curveto(444.49257718,1217.38499267)(444.21132739,1217.44749262)(443.86410543,1217.50999258)
\curveto(443.51688347,1217.57943697)(443.21132815,1217.61415916)(442.94743946,1217.61415916)
\curveto(442.10716232,1217.61415916)(441.49605167,1217.42665931)(441.11410751,1217.05165959)
\curveto(440.7391078,1216.68360431)(440.55160794,1216.01346593)(440.55160794,1215.04124444)
\lineto(440.55160794,1214.64541141)
\lineto(444.08285527,1214.64541141)
\lineto(444.08285527,1212.99957932)
\lineto(440.61410789,1212.99957932)
\lineto(440.61410789,1203.01000355)
\lineto(438.65577604,1203.01000355)
\lineto(438.65577604,1212.99957932)
\lineto(437.33286038,1212.99957932)
\lineto(437.33286038,1214.64541141)
\lineto(438.65577604,1214.64541141)
\lineto(438.65577604,1215.03082779)
\curveto(438.65577604,1216.41277118)(438.99952578,1217.47179816)(439.68702526,1218.20790871)
\curveto(440.37452474,1218.95096371)(441.36757955,1219.3224912)(442.66618967,1219.3224912)
\curveto(443.10368934,1219.3224912)(443.49605016,1219.30165789)(443.84327212,1219.25999125)
\curveto(444.19743852,1219.21832462)(444.52035494,1219.16971354)(444.81202138,1219.11415803)
\closepath
}
}
{
\newrgbcolor{curcolor}{0 0 0}
\pscustom[linestyle=none,fillstyle=solid,fillcolor=curcolor]
{
\newpath
\moveto(448.15576732,1216.5933266)
\lineto(445.94743566,1216.5933266)
\lineto(445.94743566,1218.62457507)
\lineto(448.15576732,1218.62457507)
\closepath
\moveto(448.03076741,1203.01000355)
\lineto(446.07243556,1203.01000355)
\lineto(446.07243556,1214.64541141)
\lineto(448.03076741,1214.64541141)
\closepath
}
}
{
\newrgbcolor{curcolor}{0 0 0}
\pscustom[linestyle=none,fillstyle=solid,fillcolor=curcolor]
{
\newpath
\moveto(461.63492453,1203.01000355)
\lineto(459.67659268,1203.01000355)
\lineto(459.67659268,1209.63499853)
\curveto(459.67659268,1210.16972035)(459.6453427,1210.66971997)(459.58284275,1211.1349974)
\curveto(459.5203428,1211.60721926)(459.40575955,1211.97527454)(459.23909301,1212.23916323)
\curveto(459.06548203,1212.53082968)(458.81548222,1212.74610729)(458.48909358,1212.88499608)
\curveto(458.16270494,1213.0308293)(457.73909415,1213.10374591)(457.21826121,1213.10374591)
\curveto(456.68353939,1213.10374591)(456.12451204,1212.97180157)(455.54117914,1212.70791288)
\curveto(454.95784625,1212.44402419)(454.3988189,1212.10721889)(453.86409708,1211.69749697)
\lineto(453.86409708,1203.01000355)
\lineto(451.90576523,1203.01000355)
\lineto(451.90576523,1214.64541141)
\lineto(453.86409708,1214.64541141)
\lineto(453.86409708,1213.35374572)
\curveto(454.47520773,1213.86068978)(455.10715169,1214.25652282)(455.75992898,1214.54124482)
\curveto(456.41270626,1214.82596683)(457.08284464,1214.96832783)(457.77034412,1214.96832783)
\curveto(459.02728762,1214.96832783)(459.98562023,1214.5898559)(460.64534195,1213.83291202)
\curveto(461.30506367,1213.07596815)(461.63492453,1211.9856912)(461.63492453,1210.56208117)
\closepath
}
}
{
\newrgbcolor{curcolor}{0 0 0}
\pscustom[linestyle=none,fillstyle=solid,fillcolor=curcolor]
{
\newpath
\moveto(475.14533254,1208.62458263)
\lineto(466.57242236,1208.62458263)
\curveto(466.57242236,1207.9093054)(466.68006117,1207.28430587)(466.89533878,1206.74958405)
\curveto(467.1106164,1206.22180667)(467.40575506,1205.78777922)(467.78075478,1205.4475017)
\curveto(468.14186562,1205.11416862)(468.56894863,1204.86416881)(469.06200381,1204.69750227)
\curveto(469.56200343,1204.53083573)(470.11061413,1204.44750246)(470.7078359,1204.44750246)
\curveto(471.49950197,1204.44750246)(472.29464025,1204.60375234)(473.09325076,1204.9162521)
\curveto(473.89880571,1205.23569631)(474.47172194,1205.54819607)(474.81199946,1205.85375139)
\lineto(474.91616605,1205.85375139)
\lineto(474.91616605,1203.71833634)
\curveto(474.25644432,1203.44055878)(473.58283372,1203.20792006)(472.89533424,1203.02042021)
\curveto(472.20783476,1202.83292035)(471.48561309,1202.73917042)(470.72866922,1202.73917042)
\curveto(468.79811512,1202.73917042)(467.29117182,1203.26000336)(466.2078393,1204.30166924)
\curveto(465.12450679,1205.35027955)(464.58284053,1206.83638954)(464.58284053,1208.7599992)
\curveto(464.58284053,1210.66277553)(465.10020125,1212.17319106)(466.13492269,1213.29124577)
\curveto(467.17658857,1214.40930048)(468.54464309,1214.96832783)(470.23908625,1214.96832783)
\curveto(471.80852951,1214.96832783)(473.01686193,1214.50999485)(473.86408351,1213.59332887)
\curveto(474.71824953,1212.6766629)(475.14533254,1211.37458055)(475.14533254,1209.68708183)
\closepath
\moveto(473.23908398,1210.1245815)
\curveto(473.23213954,1211.1523585)(472.97172307,1211.94749678)(472.45783457,1212.50999636)
\curveto(471.95089051,1213.07249593)(471.17658554,1213.35374572)(470.13491967,1213.35374572)
\curveto(469.08630935,1213.35374572)(468.24950442,1213.04471818)(467.6245049,1212.42666309)
\curveto(467.00644981,1211.808608)(466.65575563,1211.04124747)(466.57242236,1210.1245815)
\closepath
}
}
{
\newrgbcolor{curcolor}{0 0 0}
\pscustom[linestyle=none,fillstyle=solid,fillcolor=curcolor]
{
\newpath
\moveto(486.42657,1206.36416768)
\curveto(486.42657,1205.30166848)(485.98559812,1204.43014136)(485.10365434,1203.74958632)
\curveto(484.228655,1203.06903128)(483.03073924,1202.72875376)(481.50990706,1202.72875376)
\curveto(480.6487966,1202.72875376)(479.85713053,1202.82944813)(479.13490885,1203.03083686)
\curveto(478.41963162,1203.23917004)(477.81893763,1203.46486431)(477.33282688,1203.70791968)
\lineto(477.33282688,1205.90583469)
\lineto(477.43699347,1205.90583469)
\curveto(478.05504856,1205.44055726)(478.74254804,1205.06902977)(479.49949191,1204.7912522)
\curveto(480.25643578,1204.52041907)(480.98212968,1204.38500251)(481.6765736,1204.38500251)
\curveto(482.53768406,1204.38500251)(483.21129466,1204.52389129)(483.6974054,1204.80166886)
\curveto(484.18351615,1205.07944643)(484.42657152,1205.51694609)(484.42657152,1206.11416786)
\curveto(484.42657152,1206.57250085)(484.29462717,1206.91972281)(484.03073848,1207.15583374)
\curveto(483.76684979,1207.39194468)(483.25990573,1207.59333341)(482.5099063,1207.75999995)
\curveto(482.23212873,1207.82249991)(481.86754568,1207.89541652)(481.41615713,1207.97874979)
\curveto(480.97171302,1208.06208306)(480.56546333,1208.15236077)(480.19740805,1208.24958292)
\curveto(479.17657549,1208.52041604)(478.45088159,1208.91624908)(478.02032636,1209.43708202)
\curveto(477.59671557,1209.9648594)(477.38491018,1210.61069224)(477.38491018,1211.37458055)
\curveto(477.38491018,1211.85374686)(477.48213233,1212.3051354)(477.67657662,1212.72874619)
\curveto(477.87796536,1213.15235698)(478.18004847,1213.53082892)(478.58282594,1213.864162)
\curveto(478.97171453,1214.19055064)(479.46476972,1214.44749489)(480.06199149,1214.63499475)
\curveto(480.6661577,1214.82943905)(481.3397683,1214.9266612)(482.08282329,1214.9266612)
\curveto(482.77726721,1214.9266612)(483.47865557,1214.83985571)(484.18698836,1214.66624473)
\curveto(484.9022656,1214.49957819)(485.49601515,1214.29471723)(485.96823702,1214.05166186)
\lineto(485.96823702,1211.95791344)
\lineto(485.86407043,1211.95791344)
\curveto(485.36407081,1212.32596872)(484.75643238,1212.63499626)(484.04115514,1212.88499608)
\curveto(483.32587791,1213.14194033)(482.62448955,1213.27041245)(481.93699007,1213.27041245)
\curveto(481.22171283,1213.27041245)(480.61754662,1213.13152367)(480.12449144,1212.8537461)
\curveto(479.63143626,1212.58291297)(479.38490866,1212.17666328)(479.38490866,1211.63499702)
\curveto(479.38490866,1211.15583072)(479.53421411,1210.79471988)(479.83282499,1210.55166451)
\curveto(480.12449144,1210.30860914)(480.5967133,1210.11069262)(481.24949059,1209.95791496)
\curveto(481.61060143,1209.87458169)(482.0133789,1209.79124842)(482.45782301,1209.70791515)
\curveto(482.90921155,1209.62458188)(483.28421127,1209.54819304)(483.58282216,1209.47874865)
\curveto(484.49254369,1209.27041548)(485.19393205,1208.91277686)(485.68698723,1208.4058328)
\curveto(486.18004241,1207.8919443)(486.42657,1207.21138926)(486.42657,1206.36416768)
\closepath
}
}
{
\newrgbcolor{curcolor}{0 0 0}
\pscustom[linestyle=none,fillstyle=solid,fillcolor=curcolor]
{
\newpath
\moveto(506.96822592,1208.96833237)
\curveto(506.96822592,1208.02388864)(506.83280936,1207.15930596)(506.56197623,1206.37458433)
\curveto(506.2911431,1205.59680714)(505.90919894,1204.93708542)(505.41614376,1204.39541916)
\curveto(504.95781078,1203.88153066)(504.41614452,1203.48222541)(503.79114499,1203.1975034)
\curveto(503.1730899,1202.91972584)(502.5168404,1202.78083705)(501.82239648,1202.78083705)
\curveto(501.21823027,1202.78083705)(500.66961958,1202.84680923)(500.17656439,1202.97875357)
\curveto(499.69045365,1203.11069791)(499.19392625,1203.31555887)(498.68698219,1203.59333644)
\lineto(498.68698219,1198.71834013)
\lineto(496.72865033,1198.71834013)
\lineto(496.72865033,1214.64541141)
\lineto(498.68698219,1214.64541141)
\lineto(498.68698219,1213.42666233)
\curveto(499.20781513,1213.864162)(499.79114802,1214.22874506)(500.43698086,1214.5204115)
\curveto(501.08975815,1214.81902239)(501.78420207,1214.96832783)(502.52031262,1214.96832783)
\curveto(503.92308934,1214.96832783)(505.01336629,1214.43707823)(505.79114348,1213.37457904)
\curveto(506.57586511,1212.31902428)(506.96822592,1210.85027539)(506.96822592,1208.96833237)
\closepath
\moveto(504.94739412,1208.91624908)
\curveto(504.94739412,1210.31902579)(504.70781096,1211.36763611)(504.22864466,1212.06208003)
\curveto(503.74947836,1212.75652395)(503.0133678,1213.10374591)(502.020313,1213.10374591)
\curveto(501.45781342,1213.10374591)(500.89184163,1212.98221822)(500.32239762,1212.73916285)
\curveto(499.7529536,1212.49610748)(499.20781513,1212.17666328)(498.68698219,1211.78083024)
\lineto(498.68698219,1205.18708523)
\curveto(499.24253732,1204.93708542)(499.71823141,1204.76694666)(500.11406444,1204.67666895)
\curveto(500.51684191,1204.58639124)(500.97170268,1204.54125239)(501.47864674,1204.54125239)
\curveto(502.56892369,1204.54125239)(503.41961749,1204.90930766)(504.03072814,1205.64541822)
\curveto(504.64183879,1206.38152877)(504.94739412,1207.47180573)(504.94739412,1208.91624908)
\closepath
}
}
{
\newrgbcolor{curcolor}{0 0 0}
\pscustom[linestyle=none,fillstyle=solid,fillcolor=curcolor]
{
\newpath
\moveto(517.28071632,1212.50999636)
\lineto(517.17654973,1212.50999636)
\curveto(516.88488329,1212.57944075)(516.60016128,1212.62805183)(516.32238371,1212.65582958)
\curveto(516.05155058,1212.69055178)(515.72863416,1212.70791288)(515.35363445,1212.70791288)
\curveto(514.74946824,1212.70791288)(514.16613534,1212.57249631)(513.60363577,1212.30166318)
\curveto(513.04113619,1212.03777449)(512.49946994,1211.69402475)(511.978637,1211.27041396)
\lineto(511.978637,1203.01000355)
\lineto(510.02030515,1203.01000355)
\lineto(510.02030515,1214.64541141)
\lineto(511.978637,1214.64541141)
\lineto(511.978637,1212.92666271)
\curveto(512.75641419,1213.55166224)(513.44044145,1213.99263413)(514.03071878,1214.24957838)
\curveto(514.62794055,1214.51346707)(515.23557898,1214.64541141)(515.85363407,1214.64541141)
\curveto(516.19391159,1214.64541141)(516.44043918,1214.63499475)(516.59321684,1214.61416143)
\curveto(516.7459945,1214.60027256)(516.975161,1214.56902258)(517.28071632,1214.5204115)
\closepath
}
}
{
\newrgbcolor{curcolor}{0 0 0}
\pscustom[linestyle=none,fillstyle=solid,fillcolor=curcolor]
{
\newpath
\moveto(529.04112681,1208.82249915)
\curveto(529.04112681,1206.92666725)(528.55501607,1205.4301406)(527.58279458,1204.33291921)
\curveto(526.6105731,1203.23569782)(525.30849075,1202.68708712)(523.67654754,1202.68708712)
\curveto(522.03071545,1202.68708712)(520.72168866,1203.23569782)(519.74946718,1204.33291921)
\curveto(518.78419013,1205.4301406)(518.3015516,1206.92666725)(518.3015516,1208.82249915)
\curveto(518.3015516,1210.71833105)(518.78419013,1212.21485769)(519.74946718,1213.31207909)
\curveto(520.72168866,1214.41624492)(522.03071545,1214.96832783)(523.67654754,1214.96832783)
\curveto(525.30849075,1214.96832783)(526.6105731,1214.41624492)(527.58279458,1213.31207909)
\curveto(528.55501607,1212.21485769)(529.04112681,1210.71833105)(529.04112681,1208.82249915)
\closepath
\moveto(527.02029501,1208.82249915)
\curveto(527.02029501,1210.32944245)(526.72515634,1211.44749716)(526.13487901,1212.17666328)
\curveto(525.54460168,1212.91277383)(524.72515786,1213.28082911)(523.67654754,1213.28082911)
\curveto(522.61404834,1213.28082911)(521.78766008,1212.91277383)(521.19738275,1212.17666328)
\curveto(520.61404985,1211.44749716)(520.32238341,1210.32944245)(520.32238341,1208.82249915)
\curveto(520.32238341,1207.36416692)(520.61752207,1206.25652887)(521.20779941,1205.499585)
\curveto(521.79807674,1204.74958556)(522.62099278,1204.37458585)(523.67654754,1204.37458585)
\curveto(524.71821342,1204.37458585)(525.53418502,1204.74611334)(526.12446235,1205.48916834)
\curveto(526.72168412,1206.23916777)(527.02029501,1207.35027804)(527.02029501,1208.82249915)
\closepath
}
}
{
\newrgbcolor{curcolor}{0 0 0}
\pscustom[linestyle=none,fillstyle=solid,fillcolor=curcolor]
{
\newpath
\moveto(540.67653454,1203.73916966)
\curveto(540.02375726,1203.4266699)(539.40222995,1203.18361453)(538.81195262,1203.01000355)
\curveto(538.22861973,1202.83639257)(537.60709242,1202.74958708)(536.94737069,1202.74958708)
\curveto(536.10709355,1202.74958708)(535.3362608,1202.87111476)(534.63487244,1203.11417013)
\curveto(533.93348409,1203.36416995)(533.3327901,1203.73916966)(532.83279047,1204.23916928)
\curveto(532.32584641,1204.7391689)(531.9334856,1205.37111287)(531.65570803,1206.13500118)
\curveto(531.37793046,1206.89888949)(531.23904168,1207.79124993)(531.23904168,1208.81208249)
\curveto(531.23904168,1210.71485883)(531.75987462,1212.20791325)(532.8015405,1213.29124577)
\curveto(533.85015082,1214.37457828)(535.23209421,1214.91624454)(536.94737069,1214.91624454)
\curveto(537.61403686,1214.91624454)(538.26681414,1214.82249461)(538.90570255,1214.63499475)
\curveto(539.55153539,1214.44749489)(540.14181272,1214.2183284)(540.67653454,1213.94749527)
\lineto(540.67653454,1211.77041359)
\lineto(540.57236795,1211.77041359)
\curveto(539.97514618,1212.23569101)(539.35709109,1212.59332963)(538.71820269,1212.84332944)
\curveto(538.08625872,1213.09332925)(537.46820363,1213.21832916)(536.86403742,1213.21832916)
\curveto(535.75292715,1213.21832916)(534.8744556,1212.84332944)(534.22862275,1212.09333001)
\curveto(533.58973435,1211.35027501)(533.27029014,1210.25652584)(533.27029014,1208.81208249)
\curveto(533.27029014,1207.40930577)(533.58278991,1206.32944548)(534.20778943,1205.57250161)
\curveto(534.8397334,1204.82250218)(535.7251494,1204.44750246)(536.86403742,1204.44750246)
\curveto(537.25987046,1204.44750246)(537.66264793,1204.49958575)(538.07236984,1204.60375234)
\curveto(538.48209176,1204.70791893)(538.85014703,1204.84333549)(539.17653567,1205.01000203)
\curveto(539.46125768,1205.15583526)(539.72861859,1205.30861292)(539.9786184,1205.46833502)
\curveto(540.22861821,1205.63500156)(540.42653473,1205.77736256)(540.57236795,1205.89541803)
\lineto(540.67653454,1205.89541803)
\closepath
}
}
{
\newrgbcolor{curcolor}{0 0 0}
\pscustom[linestyle=none,fillstyle=solid,fillcolor=curcolor]
{
\newpath
\moveto(552.92652439,1208.62458263)
\lineto(544.35361421,1208.62458263)
\curveto(544.35361421,1207.9093054)(544.46125302,1207.28430587)(544.67653063,1206.74958405)
\curveto(544.89180825,1206.22180667)(545.18694691,1205.78777922)(545.56194663,1205.4475017)
\curveto(545.92305747,1205.11416862)(546.35014048,1204.86416881)(546.84319566,1204.69750227)
\curveto(547.34319528,1204.53083573)(547.89180598,1204.44750246)(548.48902775,1204.44750246)
\curveto(549.28069381,1204.44750246)(550.0758321,1204.60375234)(550.87444261,1204.9162521)
\curveto(551.67999755,1205.23569631)(552.25291379,1205.54819607)(552.59319131,1205.85375139)
\lineto(552.6973579,1205.85375139)
\lineto(552.6973579,1203.71833634)
\curveto(552.03763617,1203.44055878)(551.36402557,1203.20792006)(550.67652609,1203.02042021)
\curveto(549.98902661,1202.83292035)(549.26680494,1202.73917042)(548.50986106,1202.73917042)
\curveto(546.57930697,1202.73917042)(545.07236366,1203.26000336)(543.98903115,1204.30166924)
\curveto(542.90569864,1205.35027955)(542.36403238,1206.83638954)(542.36403238,1208.7599992)
\curveto(542.36403238,1210.66277553)(542.8813931,1212.17319106)(543.91611454,1213.29124577)
\curveto(544.95778042,1214.40930048)(546.32583494,1214.96832783)(548.0202781,1214.96832783)
\curveto(549.58972136,1214.96832783)(550.79805378,1214.50999485)(551.64527536,1213.59332887)
\curveto(552.49944138,1212.6766629)(552.92652439,1211.37458055)(552.92652439,1209.68708183)
\closepath
\moveto(551.02027583,1210.1245815)
\curveto(551.01333139,1211.1523585)(550.75291492,1211.94749678)(550.23902642,1212.50999636)
\curveto(549.73208236,1213.07249593)(548.95777739,1213.35374572)(547.91611151,1213.35374572)
\curveto(546.8675012,1213.35374572)(546.03069627,1213.04471818)(545.40569675,1212.42666309)
\curveto(544.78764166,1211.808608)(544.43694748,1211.04124747)(544.35361421,1210.1245815)
\closepath
}
}
{
\newrgbcolor{curcolor}{0 0 0}
\pscustom[linestyle=none,fillstyle=solid,fillcolor=curcolor]
{
\newpath
\moveto(564.20776906,1206.36416768)
\curveto(564.20776906,1205.30166848)(563.76679717,1204.43014136)(562.8848534,1203.74958632)
\curveto(562.00985406,1203.06903128)(560.8119383,1202.72875376)(559.29110611,1202.72875376)
\curveto(558.42999565,1202.72875376)(557.63832959,1202.82944813)(556.91610791,1203.03083686)
\curveto(556.20083067,1203.23917004)(555.60013668,1203.46486431)(555.11402594,1203.70791968)
\lineto(555.11402594,1205.90583469)
\lineto(555.21819253,1205.90583469)
\curveto(555.83624762,1205.44055726)(556.5237471,1205.06902977)(557.28069097,1204.7912522)
\curveto(558.03763484,1204.52041907)(558.76332874,1204.38500251)(559.45777265,1204.38500251)
\curveto(560.31888311,1204.38500251)(560.99249372,1204.52389129)(561.47860446,1204.80166886)
\curveto(561.9647152,1205.07944643)(562.20777057,1205.51694609)(562.20777057,1206.11416786)
\curveto(562.20777057,1206.57250085)(562.07582623,1206.91972281)(561.81193754,1207.15583374)
\curveto(561.54804885,1207.39194468)(561.04110479,1207.59333341)(560.29110536,1207.75999995)
\curveto(560.01332779,1207.82249991)(559.64874473,1207.89541652)(559.19735619,1207.97874979)
\curveto(558.75291208,1208.06208306)(558.34666238,1208.15236077)(557.97860711,1208.24958292)
\curveto(556.95777455,1208.52041604)(556.23208065,1208.91624908)(555.80152542,1209.43708202)
\curveto(555.37791463,1209.9648594)(555.16610924,1210.61069224)(555.16610924,1211.37458055)
\curveto(555.16610924,1211.85374686)(555.26333138,1212.3051354)(555.45777568,1212.72874619)
\curveto(555.65916442,1213.15235698)(555.96124752,1213.53082892)(556.364025,1213.864162)
\curveto(556.75291359,1214.19055064)(557.24596877,1214.44749489)(557.84319054,1214.63499475)
\curveto(558.44735675,1214.82943905)(559.12096735,1214.9266612)(559.86402235,1214.9266612)
\curveto(560.55846627,1214.9266612)(561.25985462,1214.83985571)(561.96818742,1214.66624473)
\curveto(562.68346466,1214.49957819)(563.27721421,1214.29471723)(563.74943607,1214.05166186)
\lineto(563.74943607,1211.95791344)
\lineto(563.64526949,1211.95791344)
\curveto(563.14526987,1212.32596872)(562.53763144,1212.63499626)(561.8223542,1212.88499608)
\curveto(561.10707696,1213.14194033)(560.4056886,1213.27041245)(559.71818912,1213.27041245)
\curveto(559.00291189,1213.27041245)(558.39874568,1213.13152367)(557.9056905,1212.8537461)
\curveto(557.41263531,1212.58291297)(557.16610772,1212.17666328)(557.16610772,1211.63499702)
\curveto(557.16610772,1211.15583072)(557.31541316,1210.79471988)(557.61402405,1210.55166451)
\curveto(557.9056905,1210.30860914)(558.37791236,1210.11069262)(559.03068964,1209.95791496)
\curveto(559.39180048,1209.87458169)(559.79457796,1209.79124842)(560.23902206,1209.70791515)
\curveto(560.69041061,1209.62458188)(561.06541033,1209.54819304)(561.36402121,1209.47874865)
\curveto(562.27374275,1209.27041548)(562.9751311,1208.91277686)(563.46818629,1208.4058328)
\curveto(563.96124147,1207.8919443)(564.20776906,1207.21138926)(564.20776906,1206.36416768)
\closepath
}
}
{
\newrgbcolor{curcolor}{0 0 0}
\pscustom[linestyle=none,fillstyle=solid,fillcolor=curcolor]
{
\newpath
\moveto(575.3223431,1206.36416768)
\curveto(575.3223431,1205.30166848)(574.88137121,1204.43014136)(573.99942744,1203.74958632)
\curveto(573.1244281,1203.06903128)(571.92651234,1202.72875376)(570.40568016,1202.72875376)
\curveto(569.5445697,1202.72875376)(568.75290363,1202.82944813)(568.03068195,1203.03083686)
\curveto(567.31540472,1203.23917004)(566.71471073,1203.46486431)(566.22859998,1203.70791968)
\lineto(566.22859998,1205.90583469)
\lineto(566.33276657,1205.90583469)
\curveto(566.95082166,1205.44055726)(567.63832114,1205.06902977)(568.39526501,1204.7912522)
\curveto(569.15220888,1204.52041907)(569.87790278,1204.38500251)(570.5723467,1204.38500251)
\curveto(571.43345716,1204.38500251)(572.10706776,1204.52389129)(572.5931785,1204.80166886)
\curveto(573.07928924,1205.07944643)(573.32234462,1205.51694609)(573.32234462,1206.11416786)
\curveto(573.32234462,1206.57250085)(573.19040027,1206.91972281)(572.92651158,1207.15583374)
\curveto(572.66262289,1207.39194468)(572.15567883,1207.59333341)(571.4056794,1207.75999995)
\curveto(571.12790183,1207.82249991)(570.76331877,1207.89541652)(570.31193023,1207.97874979)
\curveto(569.86748612,1208.06208306)(569.46123643,1208.15236077)(569.09318115,1208.24958292)
\curveto(568.07234859,1208.52041604)(567.34665469,1208.91624908)(566.91609946,1209.43708202)
\curveto(566.49248867,1209.9648594)(566.28068328,1210.61069224)(566.28068328,1211.37458055)
\curveto(566.28068328,1211.85374686)(566.37790543,1212.3051354)(566.57234972,1212.72874619)
\curveto(566.77373846,1213.15235698)(567.07582156,1213.53082892)(567.47859904,1213.864162)
\curveto(567.86748763,1214.19055064)(568.36054281,1214.44749489)(568.95776458,1214.63499475)
\curveto(569.56193079,1214.82943905)(570.2355414,1214.9266612)(570.97859639,1214.9266612)
\curveto(571.67304031,1214.9266612)(572.37442867,1214.83985571)(573.08276146,1214.66624473)
\curveto(573.7980387,1214.49957819)(574.39178825,1214.29471723)(574.86401012,1214.05166186)
\lineto(574.86401012,1211.95791344)
\lineto(574.75984353,1211.95791344)
\curveto(574.25984391,1212.32596872)(573.65220548,1212.63499626)(572.93692824,1212.88499608)
\curveto(572.221651,1213.14194033)(571.52026265,1213.27041245)(570.83276317,1213.27041245)
\curveto(570.11748593,1213.27041245)(569.51331972,1213.13152367)(569.02026454,1212.8537461)
\curveto(568.52720936,1212.58291297)(568.28068176,1212.17666328)(568.28068176,1211.63499702)
\curveto(568.28068176,1211.15583072)(568.42998721,1210.79471988)(568.72859809,1210.55166451)
\curveto(569.02026454,1210.30860914)(569.4924864,1210.11069262)(570.14526369,1209.95791496)
\curveto(570.50637452,1209.87458169)(570.909152,1209.79124842)(571.35359611,1209.70791515)
\curveto(571.80498465,1209.62458188)(572.17998437,1209.54819304)(572.47859525,1209.47874865)
\curveto(573.38831679,1209.27041548)(574.08970515,1208.91277686)(574.58276033,1208.4058328)
\curveto(575.07581551,1207.8919443)(575.3223431,1207.21138926)(575.3223431,1206.36416768)
\closepath
}
}
{
\newrgbcolor{curcolor}{0 0 0}
\pscustom[linestyle=none,fillstyle=solid,fillcolor=curcolor]
{
\newpath
\moveto(594.66606884,1203.01000355)
\lineto(592.71815365,1203.01000355)
\lineto(592.71815365,1204.24958594)
\curveto(592.54454267,1204.13153048)(592.30843174,1203.96486394)(592.00982085,1203.74958632)
\curveto(591.71815441,1203.54125314)(591.4334324,1203.3745866)(591.15565483,1203.2495867)
\curveto(590.82926619,1203.0898646)(590.45426647,1202.95792025)(590.03065568,1202.85375366)
\curveto(589.60704489,1202.74264264)(589.11051749,1202.68708712)(588.54107348,1202.68708712)
\curveto(587.49246316,1202.68708712)(586.60357494,1203.03430908)(585.87440883,1203.728753)
\curveto(585.14524271,1204.42319692)(584.78065965,1205.30861292)(584.78065965,1206.38500099)
\curveto(584.78065965,1207.26694477)(584.96815951,1207.97874979)(585.34315923,1208.52041604)
\curveto(585.72510338,1209.06902674)(586.26676964,1209.49958197)(586.968158,1209.81208173)
\curveto(587.6764908,1210.1245815)(588.5271846,1210.33638689)(589.5202394,1210.44749792)
\curveto(590.51329421,1210.55860895)(591.57926562,1210.64194222)(592.71815365,1210.69749773)
\lineto(592.71815365,1210.99958084)
\curveto(592.71815365,1211.44402494)(592.6382926,1211.81208022)(592.4785705,1212.10374667)
\curveto(592.32579284,1212.39541311)(592.10357078,1212.62457961)(591.81190433,1212.79124615)
\curveto(591.53412677,1212.95096825)(591.20079369,1213.05860706)(590.81190509,1213.11416257)
\curveto(590.4230165,1213.16971808)(590.0167668,1213.19749584)(589.59315601,1213.19749584)
\curveto(589.07926751,1213.19749584)(588.50635128,1213.12805145)(587.87440731,1212.98916266)
\curveto(587.24246335,1212.85721832)(586.58968606,1212.66277402)(585.91607546,1212.40582977)
\lineto(585.81190887,1212.40582977)
\lineto(585.81190887,1214.3954116)
\curveto(586.19385303,1214.49957819)(586.74593595,1214.61416143)(587.46815762,1214.73916134)
\curveto(588.1903793,1214.86416124)(588.90218431,1214.9266612)(589.60357267,1214.9266612)
\curveto(590.4230165,1214.9266612)(591.13482151,1214.85721681)(591.73898772,1214.71832802)
\curveto(592.35009837,1214.58638368)(592.87787575,1214.35721718)(593.32231986,1214.03082854)
\curveto(593.75981953,1213.71138434)(594.09315261,1213.29819021)(594.3223191,1212.79124615)
\curveto(594.5514856,1212.28430209)(594.66606884,1211.65583034)(594.66606884,1210.90583091)
\closepath
\moveto(592.71815365,1205.87458471)
\lineto(592.71815365,1209.11416559)
\curveto(592.12093188,1209.0794434)(591.4160713,1209.0273601)(590.60357192,1208.95791571)
\curveto(589.79801697,1208.88847132)(589.15912856,1208.78777695)(588.6869067,1208.65583261)
\curveto(588.12440712,1208.49611051)(587.66954636,1208.2461107)(587.3223244,1207.90583318)
\curveto(586.97510244,1207.57250009)(586.80149146,1207.11069489)(586.80149146,1206.52041756)
\curveto(586.80149146,1205.85375139)(587.0028802,1205.35027955)(587.40565767,1205.01000203)
\curveto(587.80843514,1204.67666895)(588.42301801,1204.51000241)(589.24940627,1204.51000241)
\curveto(589.93690575,1204.51000241)(590.5653775,1204.64194676)(591.13482151,1204.90583545)
\curveto(591.70426553,1205.17666857)(592.23204291,1205.499585)(592.71815365,1205.87458471)
\closepath
}
}
{
\newrgbcolor{curcolor}{0 0 0}
\pscustom[linestyle=none,fillstyle=solid,fillcolor=curcolor]
{
\newpath
\moveto(608.16606352,1203.01000355)
\lineto(606.20773166,1203.01000355)
\lineto(606.20773166,1209.63499853)
\curveto(606.20773166,1210.16972035)(606.17648169,1210.66971997)(606.11398174,1211.1349974)
\curveto(606.05148178,1211.60721926)(605.93689854,1211.97527454)(605.770232,1212.23916323)
\curveto(605.59662102,1212.53082968)(605.3466212,1212.74610729)(605.02023256,1212.88499608)
\curveto(604.69384392,1213.0308293)(604.27023313,1213.10374591)(603.74940019,1213.10374591)
\curveto(603.21467837,1213.10374591)(602.65565102,1212.97180157)(602.07231813,1212.70791288)
\curveto(601.48898523,1212.44402419)(600.92995788,1212.10721889)(600.39523606,1211.69749697)
\lineto(600.39523606,1203.01000355)
\lineto(598.43690421,1203.01000355)
\lineto(598.43690421,1214.64541141)
\lineto(600.39523606,1214.64541141)
\lineto(600.39523606,1213.35374572)
\curveto(601.00634671,1213.86068978)(601.63829068,1214.25652282)(602.29106796,1214.54124482)
\curveto(602.94384524,1214.82596683)(603.61398363,1214.96832783)(604.30148311,1214.96832783)
\curveto(605.5584266,1214.96832783)(606.51675921,1214.5898559)(607.17648093,1213.83291202)
\curveto(607.83620265,1213.07596815)(608.16606352,1211.9856912)(608.16606352,1210.56208117)
\closepath
}
}
{
\newrgbcolor{curcolor}{0 0 0}
\pscustom[linestyle=none,fillstyle=solid,fillcolor=curcolor]
{
\newpath
\moveto(621.37438842,1203.01000355)
\lineto(619.41605657,1203.01000355)
\lineto(619.41605657,1204.22875262)
\curveto(618.85355699,1203.74264188)(618.26675188,1203.36416995)(617.65564123,1203.09333682)
\curveto(617.04453058,1202.82250369)(616.38133664,1202.68708712)(615.6660594,1202.68708712)
\curveto(614.27717157,1202.68708712)(613.17300574,1203.22180894)(612.35356191,1204.29125258)
\curveto(611.54106253,1205.36069621)(611.13481283,1206.84333398)(611.13481283,1208.73916588)
\curveto(611.13481283,1209.72527624)(611.27370162,1210.6037478)(611.55147918,1211.37458055)
\curveto(611.83620119,1212.1454133)(612.21814535,1212.80166281)(612.69731165,1213.34332906)
\curveto(613.16953352,1213.87110644)(613.71814421,1214.27388391)(614.34314374,1214.55166148)
\curveto(614.97508771,1214.82943905)(615.62786499,1214.96832783)(616.30147559,1214.96832783)
\curveto(616.91258624,1214.96832783)(617.4542525,1214.90235566)(617.92647436,1214.77041132)
\curveto(618.39869623,1214.64541141)(618.89522363,1214.44749489)(619.41605657,1214.17666176)
\lineto(619.41605657,1219.21832462)
\lineto(621.37438842,1219.21832462)
\closepath
\moveto(619.41605657,1205.87458471)
\lineto(619.41605657,1212.55166299)
\curveto(618.88827919,1212.78777393)(618.41605732,1212.95096825)(617.99939097,1213.04124596)
\curveto(617.58272462,1213.13152367)(617.12786385,1213.17666252)(616.63480867,1213.17666252)
\curveto(615.53758728,1213.17666252)(614.68342126,1212.79471837)(614.07231061,1212.03083005)
\curveto(613.46119996,1211.26694174)(613.15564464,1210.18360923)(613.15564464,1208.78083251)
\curveto(613.15564464,1207.39888911)(613.39175557,1206.34680658)(613.86397743,1205.6245849)
\curveto(614.3361993,1204.90930766)(615.09314317,1204.55166905)(616.13480905,1204.55166905)
\curveto(616.69036419,1204.55166905)(617.25286376,1204.67319673)(617.82230777,1204.9162521)
\curveto(618.39175179,1205.16625191)(618.92300139,1205.48569612)(619.41605657,1205.87458471)
\closepath
}
}
{
\newrgbcolor{curcolor}{0 0 0}
\pscustom[linestyle=none,fillstyle=solid,fillcolor=curcolor]
{
\newpath
\moveto(634.83270103,1216.5933266)
\lineto(632.62436937,1216.5933266)
\lineto(632.62436937,1218.62457507)
\lineto(634.83270103,1218.62457507)
\closepath
\moveto(634.70770112,1203.01000355)
\lineto(632.74936927,1203.01000355)
\lineto(632.74936927,1214.64541141)
\lineto(634.70770112,1214.64541141)
\closepath
}
}
{
\newrgbcolor{curcolor}{0 0 0}
\pscustom[linestyle=none,fillstyle=solid,fillcolor=curcolor]
{
\newpath
\moveto(644.62438266,1203.11417013)
\curveto(644.25632738,1203.01694799)(643.85354991,1202.93708693)(643.41605024,1202.87458698)
\curveto(642.98549501,1202.81208703)(642.60007863,1202.78083705)(642.25980111,1202.78083705)
\curveto(641.07230201,1202.78083705)(640.16952492,1203.10028126)(639.55146983,1203.73916966)
\curveto(638.93341474,1204.37805807)(638.6243872,1205.40236285)(638.6243872,1206.812084)
\lineto(638.6243872,1212.99957932)
\lineto(637.30147153,1212.99957932)
\lineto(637.30147153,1214.64541141)
\lineto(638.6243872,1214.64541141)
\lineto(638.6243872,1217.98915888)
\lineto(640.58271905,1217.98915888)
\lineto(640.58271905,1214.64541141)
\lineto(644.62438266,1214.64541141)
\lineto(644.62438266,1212.99957932)
\lineto(640.58271905,1212.99957932)
\lineto(640.58271905,1207.6975)
\curveto(640.58271905,1207.08638935)(640.59660793,1206.60722305)(640.62438568,1206.26000109)
\curveto(640.65216344,1205.91972357)(640.74938559,1205.60027936)(640.91605213,1205.30166848)
\curveto(641.06882979,1205.02389091)(641.27716297,1204.81902996)(641.54105166,1204.68708561)
\curveto(641.81188479,1204.56208571)(642.2216067,1204.49958575)(642.77021739,1204.49958575)
\curveto(643.0896616,1204.49958575)(643.42299468,1204.54472461)(643.77021664,1204.63500232)
\curveto(644.1174386,1204.73222447)(644.36743841,1204.81208552)(644.52021607,1204.87458547)
\lineto(644.62438266,1204.87458547)
\closepath
}
}
{
\newrgbcolor{curcolor}{0 0 0}
\pscustom[linestyle=none,fillstyle=solid,fillcolor=curcolor]
{
\newpath
\moveto(655.30145658,1206.36416768)
\curveto(655.30145658,1205.30166848)(654.8604847,1204.43014136)(653.97854092,1203.74958632)
\curveto(653.10354158,1203.06903128)(651.90562582,1202.72875376)(650.38479364,1202.72875376)
\curveto(649.52368318,1202.72875376)(648.73201711,1202.82944813)(648.00979543,1203.03083686)
\curveto(647.2945182,1203.23917004)(646.69382421,1203.46486431)(646.20771346,1203.70791968)
\lineto(646.20771346,1205.90583469)
\lineto(646.31188005,1205.90583469)
\curveto(646.92993514,1205.44055726)(647.61743462,1205.06902977)(648.37437849,1204.7912522)
\curveto(649.13132236,1204.52041907)(649.85701626,1204.38500251)(650.55146018,1204.38500251)
\curveto(651.41257064,1204.38500251)(652.08618124,1204.52389129)(652.57229198,1204.80166886)
\curveto(653.05840273,1205.07944643)(653.3014581,1205.51694609)(653.3014581,1206.11416786)
\curveto(653.3014581,1206.57250085)(653.16951375,1206.91972281)(652.90562506,1207.15583374)
\curveto(652.64173637,1207.39194468)(652.13479231,1207.59333341)(651.38479288,1207.75999995)
\curveto(651.10701531,1207.82249991)(650.74243226,1207.89541652)(650.29104371,1207.97874979)
\curveto(649.8465996,1208.06208306)(649.44034991,1208.15236077)(649.07229463,1208.24958292)
\curveto(648.05146207,1208.52041604)(647.32576817,1208.91624908)(646.89521294,1209.43708202)
\curveto(646.47160215,1209.9648594)(646.25979676,1210.61069224)(646.25979676,1211.37458055)
\curveto(646.25979676,1211.85374686)(646.35701891,1212.3051354)(646.5514632,1212.72874619)
\curveto(646.75285194,1213.15235698)(647.05493505,1213.53082892)(647.45771252,1213.864162)
\curveto(647.84660111,1214.19055064)(648.3396563,1214.44749489)(648.93687807,1214.63499475)
\curveto(649.54104428,1214.82943905)(650.21465488,1214.9266612)(650.95770987,1214.9266612)
\curveto(651.65215379,1214.9266612)(652.35354215,1214.83985571)(653.06187495,1214.66624473)
\curveto(653.77715218,1214.49957819)(654.37090173,1214.29471723)(654.8431236,1214.05166186)
\lineto(654.8431236,1211.95791344)
\lineto(654.73895701,1211.95791344)
\curveto(654.23895739,1212.32596872)(653.63131896,1212.63499626)(652.91604172,1212.88499608)
\curveto(652.20076449,1213.14194033)(651.49937613,1213.27041245)(650.81187665,1213.27041245)
\curveto(650.09659941,1213.27041245)(649.4924332,1213.13152367)(648.99937802,1212.8537461)
\curveto(648.50632284,1212.58291297)(648.25979525,1212.17666328)(648.25979525,1211.63499702)
\curveto(648.25979525,1211.15583072)(648.40910069,1210.79471988)(648.70771157,1210.55166451)
\curveto(648.99937802,1210.30860914)(649.47159988,1210.11069262)(650.12437717,1209.95791496)
\curveto(650.48548801,1209.87458169)(650.88826548,1209.79124842)(651.33270959,1209.70791515)
\curveto(651.78409813,1209.62458188)(652.15909785,1209.54819304)(652.45770874,1209.47874865)
\curveto(653.36743027,1209.27041548)(654.06881863,1208.91277686)(654.56187381,1208.4058328)
\curveto(655.05492899,1207.8919443)(655.30145658,1207.21138926)(655.30145658,1206.36416768)
\closepath
}
}
{
\newrgbcolor{curcolor}{0 0 0}
\pscustom[linestyle=none,fillstyle=solid,fillcolor=curcolor]
{
\newpath
\moveto(682.58267632,1203.01000355)
\lineto(680.62434447,1203.01000355)
\lineto(680.62434447,1209.63499853)
\curveto(680.62434447,1210.13499816)(680.60003893,1210.61763668)(680.55142786,1211.08291411)
\curveto(680.50976122,1211.54819153)(680.41601129,1211.91971903)(680.27017807,1212.1974966)
\curveto(680.11045597,1212.49610748)(679.88128947,1212.72180175)(679.58267859,1212.87457942)
\curveto(679.2840677,1213.02735708)(678.85351247,1213.10374591)(678.2910129,1213.10374591)
\curveto(677.7424022,1213.10374591)(677.19379151,1212.96485713)(676.64518081,1212.68707956)
\curveto(676.09657011,1212.41624643)(675.54795942,1212.06902447)(674.99934872,1211.64541368)
\curveto(675.02018204,1211.48569158)(675.03754314,1211.29819172)(675.05143202,1211.08291411)
\curveto(675.06532089,1210.87458093)(675.07226533,1210.66624775)(675.07226533,1210.45791458)
\lineto(675.07226533,1203.01000355)
\lineto(673.11393348,1203.01000355)
\lineto(673.11393348,1209.63499853)
\curveto(673.11393348,1210.14888703)(673.08962794,1210.63499778)(673.04101687,1211.09333076)
\curveto(672.99935024,1211.55860819)(672.90560031,1211.93013569)(672.75976708,1212.20791325)
\curveto(672.60004498,1212.50652414)(672.37087849,1212.72874619)(672.0722676,1212.87457942)
\curveto(671.77365672,1213.02735708)(671.34310149,1213.10374591)(670.78060191,1213.10374591)
\curveto(670.2458801,1213.10374591)(669.70768606,1212.97180157)(669.1660198,1212.70791288)
\curveto(668.63129798,1212.44402419)(668.09657617,1212.10721889)(667.56185435,1211.69749697)
\lineto(667.56185435,1203.01000355)
\lineto(665.6035225,1203.01000355)
\lineto(665.6035225,1214.64541141)
\lineto(667.56185435,1214.64541141)
\lineto(667.56185435,1213.35374572)
\curveto(668.172965,1213.86068978)(668.78060343,1214.25652282)(669.38476964,1214.54124482)
\curveto(669.99588029,1214.82596683)(670.64518535,1214.96832783)(671.33268483,1214.96832783)
\curveto(672.1243509,1214.96832783)(672.79448928,1214.80166129)(673.34309998,1214.46832821)
\curveto(673.89865511,1214.13499513)(674.31184924,1213.67318992)(674.58268237,1213.08291259)
\curveto(675.37434844,1213.74957875)(676.09657011,1214.22874506)(676.7493474,1214.5204115)
\curveto(677.40212468,1214.81902239)(678.10004082,1214.96832783)(678.84309581,1214.96832783)
\curveto(680.12087263,1214.96832783)(681.06184414,1214.57943924)(681.66601035,1213.80166205)
\curveto(682.27712099,1213.0308293)(682.58267632,1211.950969)(682.58267632,1210.56208117)
\closepath
}
}
{
\newrgbcolor{curcolor}{0 0 0}
\pscustom[linestyle=none,fillstyle=solid,fillcolor=curcolor]
{
\newpath
\moveto(696.09307155,1208.62458263)
\lineto(687.52016137,1208.62458263)
\curveto(687.52016137,1207.9093054)(687.62780018,1207.28430587)(687.84307779,1206.74958405)
\curveto(688.05835541,1206.22180667)(688.35349407,1205.78777922)(688.72849379,1205.4475017)
\curveto(689.08960463,1205.11416862)(689.51668764,1204.86416881)(690.00974282,1204.69750227)
\curveto(690.50974244,1204.53083573)(691.05835314,1204.44750246)(691.65557491,1204.44750246)
\curveto(692.44724097,1204.44750246)(693.24237926,1204.60375234)(694.04098977,1204.9162521)
\curveto(694.84654471,1205.23569631)(695.41946095,1205.54819607)(695.75973847,1205.85375139)
\lineto(695.86390506,1205.85375139)
\lineto(695.86390506,1203.71833634)
\curveto(695.20418333,1203.44055878)(694.53057273,1203.20792006)(693.84307325,1203.02042021)
\curveto(693.15557377,1202.83292035)(692.4333521,1202.73917042)(691.67640822,1202.73917042)
\curveto(689.74585413,1202.73917042)(688.23891082,1203.26000336)(687.15557831,1204.30166924)
\curveto(686.0722458,1205.35027955)(685.53057954,1206.83638954)(685.53057954,1208.7599992)
\curveto(685.53057954,1210.66277553)(686.04794026,1212.17319106)(687.0826617,1213.29124577)
\curveto(688.12432758,1214.40930048)(689.4923821,1214.96832783)(691.18682526,1214.96832783)
\curveto(692.75626852,1214.96832783)(693.96460094,1214.50999485)(694.81182252,1213.59332887)
\curveto(695.66598854,1212.6766629)(696.09307155,1211.37458055)(696.09307155,1209.68708183)
\closepath
\moveto(694.18682299,1210.1245815)
\curveto(694.17987855,1211.1523585)(693.91946208,1211.94749678)(693.40557358,1212.50999636)
\curveto(692.89862952,1213.07249593)(692.12432455,1213.35374572)(691.08265867,1213.35374572)
\curveto(690.03404836,1213.35374572)(689.19724343,1213.04471818)(688.57224391,1212.42666309)
\curveto(687.95418882,1211.808608)(687.60349464,1211.04124747)(687.52016137,1210.1245815)
\closepath
}
}
{
\newrgbcolor{curcolor}{0 0 0}
\pscustom[linestyle=none,fillstyle=solid,fillcolor=curcolor]
{
\newpath
\moveto(716.04097633,1203.01000355)
\lineto(714.08264448,1203.01000355)
\lineto(714.08264448,1209.63499853)
\curveto(714.08264448,1210.13499816)(714.05833894,1210.61763668)(714.00972787,1211.08291411)
\curveto(713.96806123,1211.54819153)(713.8743113,1211.91971903)(713.72847808,1212.1974966)
\curveto(713.56875598,1212.49610748)(713.33958949,1212.72180175)(713.0409786,1212.87457942)
\curveto(712.74236772,1213.02735708)(712.31181249,1213.10374591)(711.74931291,1213.10374591)
\curveto(711.20070221,1213.10374591)(710.65209152,1212.96485713)(710.10348082,1212.68707956)
\curveto(709.55487013,1212.41624643)(709.00625943,1212.06902447)(708.45764873,1211.64541368)
\curveto(708.47848205,1211.48569158)(708.49584315,1211.29819172)(708.50973203,1211.08291411)
\curveto(708.52362091,1210.87458093)(708.53056535,1210.66624775)(708.53056535,1210.45791458)
\lineto(708.53056535,1203.01000355)
\lineto(706.57223349,1203.01000355)
\lineto(706.57223349,1209.63499853)
\curveto(706.57223349,1210.14888703)(706.54792796,1210.63499778)(706.49931688,1211.09333076)
\curveto(706.45765025,1211.55860819)(706.36390032,1211.93013569)(706.2180671,1212.20791325)
\curveto(706.05834499,1212.50652414)(705.8291785,1212.72874619)(705.53056762,1212.87457942)
\curveto(705.23195673,1213.02735708)(704.8014015,1213.10374591)(704.23890193,1213.10374591)
\curveto(703.70418011,1213.10374591)(703.16598607,1212.97180157)(702.62431981,1212.70791288)
\curveto(702.089598,1212.44402419)(701.55487618,1212.10721889)(701.02015436,1211.69749697)
\lineto(701.02015436,1203.01000355)
\lineto(699.06182251,1203.01000355)
\lineto(699.06182251,1214.64541141)
\lineto(701.02015436,1214.64541141)
\lineto(701.02015436,1213.35374572)
\curveto(701.63126501,1213.86068978)(702.23890344,1214.25652282)(702.84306965,1214.54124482)
\curveto(703.4541803,1214.82596683)(704.10348536,1214.96832783)(704.79098484,1214.96832783)
\curveto(705.58265091,1214.96832783)(706.25278929,1214.80166129)(706.80139999,1214.46832821)
\curveto(707.35695512,1214.13499513)(707.77014925,1213.67318992)(708.04098238,1213.08291259)
\curveto(708.83264845,1213.74957875)(709.55487013,1214.22874506)(710.20764741,1214.5204115)
\curveto(710.86042469,1214.81902239)(711.55834083,1214.96832783)(712.30139583,1214.96832783)
\curveto(713.57917264,1214.96832783)(714.52014415,1214.57943924)(715.12431036,1213.80166205)
\curveto(715.73542101,1213.0308293)(716.04097633,1211.950969)(716.04097633,1210.56208117)
\closepath
}
}
{
\newrgbcolor{curcolor}{0 0 0}
\pscustom[linestyle=none,fillstyle=solid,fillcolor=curcolor]
{
\newpath
\moveto(729.7284836,1208.82249915)
\curveto(729.7284836,1206.92666725)(729.24237285,1205.4301406)(728.27015137,1204.33291921)
\curveto(727.29792988,1203.23569782)(725.99584753,1202.68708712)(724.36390432,1202.68708712)
\curveto(722.71807223,1202.68708712)(721.40904545,1203.23569782)(720.43682396,1204.33291921)
\curveto(719.47154691,1205.4301406)(718.98890839,1206.92666725)(718.98890839,1208.82249915)
\curveto(718.98890839,1210.71833105)(719.47154691,1212.21485769)(720.43682396,1213.31207909)
\curveto(721.40904545,1214.41624492)(722.71807223,1214.96832783)(724.36390432,1214.96832783)
\curveto(725.99584753,1214.96832783)(727.29792988,1214.41624492)(728.27015137,1213.31207909)
\curveto(729.24237285,1212.21485769)(729.7284836,1210.71833105)(729.7284836,1208.82249915)
\closepath
\moveto(727.70765179,1208.82249915)
\curveto(727.70765179,1210.32944245)(727.41251313,1211.44749716)(726.82223579,1212.17666328)
\curveto(726.23195846,1212.91277383)(725.41251464,1213.28082911)(724.36390432,1213.28082911)
\curveto(723.30140513,1213.28082911)(722.47501686,1212.91277383)(721.88473953,1212.17666328)
\curveto(721.30140664,1211.44749716)(721.00974019,1210.32944245)(721.00974019,1208.82249915)
\curveto(721.00974019,1207.36416692)(721.30487886,1206.25652887)(721.89515619,1205.499585)
\curveto(722.48543352,1204.74958556)(723.30834956,1204.37458585)(724.36390432,1204.37458585)
\curveto(725.4055702,1204.37458585)(726.2215418,1204.74611334)(726.81181914,1205.48916834)
\curveto(727.40904091,1206.23916777)(727.70765179,1207.35027804)(727.70765179,1208.82249915)
\closepath
}
}
{
\newrgbcolor{curcolor}{0 0 0}
\pscustom[linestyle=none,fillstyle=solid,fillcolor=curcolor]
{
\newpath
\moveto(740.02012792,1212.50999636)
\lineto(739.91596133,1212.50999636)
\curveto(739.62429489,1212.57944075)(739.33957288,1212.62805183)(739.06179531,1212.65582958)
\curveto(738.79096219,1212.69055178)(738.46804576,1212.70791288)(738.09304605,1212.70791288)
\curveto(737.48887984,1212.70791288)(736.90554695,1212.57249631)(736.34304737,1212.30166318)
\curveto(735.7805478,1212.03777449)(735.23888154,1211.69402475)(734.7180486,1211.27041396)
\lineto(734.7180486,1203.01000355)
\lineto(732.75971675,1203.01000355)
\lineto(732.75971675,1214.64541141)
\lineto(734.7180486,1214.64541141)
\lineto(734.7180486,1212.92666271)
\curveto(735.49582579,1213.55166224)(736.17985305,1213.99263413)(736.77013038,1214.24957838)
\curveto(737.36735215,1214.51346707)(737.97499058,1214.64541141)(738.59304567,1214.64541141)
\curveto(738.93332319,1214.64541141)(739.17985078,1214.63499475)(739.33262844,1214.61416143)
\curveto(739.4854061,1214.60027256)(739.7145726,1214.56902258)(740.02012792,1214.5204115)
\closepath
}
}
{
\newrgbcolor{curcolor}{0 0 0}
\pscustom[linestyle=none,fillstyle=solid,fillcolor=curcolor]
{
\newpath
\moveto(751.92635722,1214.64541141)
\lineto(745.13469569,1198.71834013)
\lineto(743.04094728,1198.71834013)
\lineto(745.2076123,1203.57250312)
\lineto(740.57219914,1214.64541141)
\lineto(742.69719754,1214.64541141)
\lineto(746.2701115,1206.02041794)
\lineto(749.87427544,1214.64541141)
\closepath
}
}
{
\newrgbcolor{curcolor}{0.7019608 0.7019608 0.7019608}
\pscustom[linestyle=none,fillstyle=solid,fillcolor=curcolor,opacity=0.92623001]
{
\newpath
\moveto(355.90269133,454.30976449)
\lineto(766.61699489,454.30976449)
\lineto(766.61699489,317.88119581)
\lineto(355.90269133,317.88119581)
\closepath
}
}
{
\newrgbcolor{curcolor}{0 0 0}
\pscustom[linewidth=1.00157103,linecolor=curcolor]
{
\newpath
\moveto(355.90269133,454.30976449)
\lineto(766.61699489,454.30976449)
\lineto(766.61699489,317.88119581)
\lineto(355.90269133,317.88119581)
\closepath
}
}
{
\newrgbcolor{curcolor}{0 0 0}
\pscustom[linestyle=none,fillstyle=solid,fillcolor=curcolor]
{
\newpath
\moveto(498.19364375,447.88955304)
\curveto(499.37853581,446.58747385)(500.28348085,444.99242685)(500.90847886,443.10441203)
\curveto(501.54649766,441.2163972)(501.86550706,439.07447694)(501.86550706,436.67865124)
\curveto(501.86550706,434.28282553)(501.53998726,432.13439487)(500.88894767,430.23335926)
\curveto(500.25092887,428.34534444)(499.35249423,426.76982862)(498.19364375,425.50681181)
\curveto(496.9957309,424.19171183)(495.57646458,423.20213164)(493.93584481,422.53807126)
\curveto(492.30824582,421.87401087)(490.44627259,421.54198068)(488.34992509,421.54198068)
\curveto(486.30566077,421.54198068)(484.44368753,421.88052127)(482.76400538,422.55760245)
\curveto(481.09734402,423.23468362)(479.6780777,424.21775341)(478.50620644,425.50681181)
\curveto(477.33433517,426.7958702)(476.42939013,428.37789642)(475.79137133,430.25289045)
\curveto(475.16637332,432.12788448)(474.85387431,434.26980474)(474.85387431,436.67865124)
\curveto(474.85387431,439.04843536)(475.16637332,441.17082443)(475.79137133,443.04581846)
\curveto(476.41636934,444.93383329)(477.32782477,446.54841148)(478.52573762,447.88955304)
\curveto(479.67156731,449.16559065)(481.09083362,450.14215004)(482.78353657,450.81923121)
\curveto(484.4892603,451.49631239)(486.34472314,451.83485298)(488.34992509,451.83485298)
\curveto(490.43325179,451.83485298)(492.30173543,451.489802)(493.955376,450.79970003)
\curveto(495.62203736,450.12261885)(497.03479327,449.15256985)(498.19364375,447.88955304)
\closepath
\moveto(497.84208237,436.67865124)
\curveto(497.84208237,440.45468088)(496.9957309,443.36482786)(495.30302796,445.40909219)
\curveto(493.61032501,447.46637731)(491.29913445,448.49501986)(488.36945628,448.49501986)
\curveto(485.41373653,448.49501986)(483.08952518,447.46637731)(481.39682223,445.40909219)
\curveto(479.71714008,443.36482786)(478.877299,440.45468088)(478.877299,436.67865124)
\curveto(478.877299,432.86355922)(479.73667127,429.94039144)(481.4554158,427.90914791)
\curveto(483.17416032,425.89092517)(485.47884048,424.8818138)(488.36945628,424.8818138)
\curveto(491.26007208,424.8818138)(493.55824184,425.89092517)(495.26396558,427.90914791)
\curveto(496.98271011,429.94039144)(497.84208237,432.86355922)(497.84208237,436.67865124)
\closepath
}
}
{
\newrgbcolor{curcolor}{0 0 0}
\pscustom[linestyle=none,fillstyle=solid,fillcolor=curcolor]
{
\newpath
\moveto(526.90449036,433.31928693)
\curveto(526.90449036,431.54845924)(526.65058492,429.92737065)(526.14277403,428.45602117)
\curveto(525.63496315,426.99769248)(524.9188196,425.76071725)(523.99434338,424.74509548)
\curveto(523.13497111,423.78155688)(522.11934934,423.03286135)(520.94747808,422.49900888)
\curveto(519.7886276,421.97817721)(518.55816277,421.71776137)(517.25608358,421.71776137)
\curveto(516.12327469,421.71776137)(515.09463213,421.84145889)(514.1701559,422.08885394)
\curveto(513.25870047,422.33624898)(512.32771385,422.72036235)(511.37719605,423.24119402)
\lineto(511.37719605,414.10059812)
\lineto(507.70533274,414.10059812)
\lineto(507.70533274,443.96378429)
\lineto(511.37719605,443.96378429)
\lineto(511.37719605,441.67863532)
\curveto(512.35375544,442.4989452)(513.44750196,443.18253678)(514.6584356,443.72941004)
\curveto(515.88239004,444.28930409)(517.18446922,444.56925111)(518.56467316,444.56925111)
\curveto(521.19487312,444.56925111)(523.23913745,443.57316053)(524.69746614,441.58097938)
\curveto(526.16881562,439.60181901)(526.90449036,436.84792153)(526.90449036,433.31928693)
\closepath
\moveto(523.11543992,433.22163099)
\curveto(523.11543992,435.85183095)(522.6662226,437.81797053)(521.76778796,439.12004971)
\curveto(520.86935332,440.4221289)(519.48914939,441.07316849)(517.62717615,441.07316849)
\curveto(516.57249201,441.07316849)(515.51129747,440.84530464)(514.44359253,440.38957692)
\curveto(513.3758876,439.93384921)(512.35375544,439.33489278)(511.37719605,438.59270764)
\lineto(511.37719605,426.22946576)
\curveto(512.4188594,425.76071725)(513.31078364,425.44170785)(514.05296878,425.27243755)
\curveto(514.80817471,425.10316726)(515.66103657,425.01853211)(516.61155438,425.01853211)
\curveto(518.65581871,425.01853211)(520.25086571,425.70863408)(521.3966954,427.08883802)
\curveto(522.54252508,428.46904196)(523.11543992,430.51330628)(523.11543992,433.22163099)
\closepath
}
}
{
\newrgbcolor{curcolor}{0 0 0}
\pscustom[linestyle=none,fillstyle=solid,fillcolor=curcolor]
{
\newpath
\moveto(551.82628726,433.31928693)
\curveto(551.82628726,431.54845924)(551.57238182,429.92737065)(551.06457094,428.45602117)
\curveto(550.55676006,426.99769248)(549.8406165,425.76071725)(548.91614028,424.74509548)
\curveto(548.05676802,423.78155688)(547.04114625,423.03286135)(545.86927498,422.49900888)
\curveto(544.7104245,421.97817721)(543.47995967,421.71776137)(542.17788048,421.71776137)
\curveto(541.04507159,421.71776137)(540.01642903,421.84145889)(539.09195281,422.08885394)
\curveto(538.18049738,422.33624898)(537.24951076,422.72036235)(536.29899295,423.24119402)
\lineto(536.29899295,414.10059812)
\lineto(532.62712964,414.10059812)
\lineto(532.62712964,443.96378429)
\lineto(536.29899295,443.96378429)
\lineto(536.29899295,441.67863532)
\curveto(537.27555234,442.4989452)(538.36929886,443.18253678)(539.5802325,443.72941004)
\curveto(540.80418694,444.28930409)(542.10626613,444.56925111)(543.48647007,444.56925111)
\curveto(546.11667003,444.56925111)(548.16093435,443.57316053)(549.61926304,441.58097938)
\curveto(551.09061252,439.60181901)(551.82628726,436.84792153)(551.82628726,433.31928693)
\closepath
\moveto(548.03723683,433.22163099)
\curveto(548.03723683,435.85183095)(547.58801951,437.81797053)(546.68958487,439.12004971)
\curveto(545.79115023,440.4221289)(544.41094629,441.07316849)(542.54897305,441.07316849)
\curveto(541.49428891,441.07316849)(540.43309437,440.84530464)(539.36538944,440.38957692)
\curveto(538.2976845,439.93384921)(537.27555234,439.33489278)(536.29899295,438.59270764)
\lineto(536.29899295,426.22946576)
\curveto(537.3406563,425.76071725)(538.23258054,425.44170785)(538.97476568,425.27243755)
\curveto(539.72997161,425.10316726)(540.58283348,425.01853211)(541.53335129,425.01853211)
\curveto(543.57761561,425.01853211)(545.17266261,425.70863408)(546.3184923,427.08883802)
\curveto(547.46432198,428.46904196)(548.03723683,430.51330628)(548.03723683,433.22163099)
\closepath
}
}
{
\newrgbcolor{curcolor}{0 0 0}
\pscustom[linestyle=none,fillstyle=solid,fillcolor=curcolor]
{
\newpath
\moveto(576.14261374,433.0458503)
\curveto(576.14261374,429.49117412)(575.23115831,426.68519347)(573.40824745,424.62790835)
\curveto(571.58533658,422.57062324)(569.14393811,421.54198068)(566.08405202,421.54198068)
\curveto(562.99812434,421.54198068)(560.54370507,422.57062324)(558.72079421,424.62790835)
\curveto(556.91090414,426.68519347)(556.0059591,429.49117412)(556.0059591,433.0458503)
\curveto(556.0059591,436.60052648)(556.91090414,439.40650713)(558.72079421,441.46379225)
\curveto(560.54370507,443.53409816)(562.99812434,444.56925111)(566.08405202,444.56925111)
\curveto(569.14393811,444.56925111)(571.58533658,443.53409816)(573.40824745,441.46379225)
\curveto(575.23115831,439.40650713)(576.14261374,436.60052648)(576.14261374,433.0458503)
\closepath
\moveto(572.3535633,433.0458503)
\curveto(572.3535633,435.87136214)(571.80017965,437.96770963)(570.69341234,439.33489278)
\curveto(569.58664503,440.71509672)(568.05019159,441.40519869)(566.08405202,441.40519869)
\curveto(564.09187086,441.40519869)(562.54239663,440.71509672)(561.43562932,439.33489278)
\curveto(560.3418828,437.96770963)(559.79500954,435.87136214)(559.79500954,433.0458503)
\curveto(559.79500954,430.31148401)(560.34839319,428.2346677)(561.4551605,426.81540139)
\curveto(562.56192781,425.40915587)(564.10489165,424.70603311)(566.08405202,424.70603311)
\curveto(568.0371708,424.70603311)(569.56711384,425.40264547)(570.67388115,426.7958702)
\curveto(571.79366925,428.20211572)(572.3535633,430.28544242)(572.3535633,433.0458503)
\closepath
}
}
{
\newrgbcolor{curcolor}{0 0 0}
\pscustom[linestyle=none,fillstyle=solid,fillcolor=curcolor]
{
\newpath
\moveto(597.41207287,428.43648998)
\curveto(597.41207287,426.44430882)(596.58525258,424.81019944)(594.93161201,423.53416184)
\curveto(593.29099224,422.25812423)(591.04490564,421.62010543)(588.19335222,421.62010543)
\curveto(586.57877403,421.62010543)(585.09440375,421.80890691)(583.7402414,422.18650988)
\curveto(582.39909983,422.57713363)(581.27280134,423.00030937)(580.3613459,423.45603709)
\lineto(580.3613459,427.57711771)
\lineto(580.55665778,427.57711771)
\curveto(581.71550826,426.70472466)(583.00456666,426.00811229)(584.42383297,425.48728062)
\curveto(585.84309928,424.97946974)(587.20377204,424.72556429)(588.50585122,424.72556429)
\curveto(590.12042942,424.72556429)(591.38344623,424.98598013)(592.29490166,425.50681181)
\curveto(593.20635709,426.02764348)(593.66208481,426.84795337)(593.66208481,427.96774147)
\curveto(593.66208481,428.82711373)(593.41468976,429.47815333)(592.91989967,429.92086025)
\curveto(592.42510958,430.36356718)(591.47459177,430.74117014)(590.06834625,431.05366915)
\curveto(589.54751457,431.17085627)(588.863923,431.30757459)(588.01757153,431.46382409)
\curveto(587.18424085,431.62007359)(586.42252452,431.78934389)(585.73242255,431.97163497)
\curveto(583.81836615,432.47944586)(582.4576934,433.22163099)(581.6504043,434.19819038)
\curveto(580.856136,435.18777057)(580.45900184,436.39870421)(580.45900184,437.83099132)
\curveto(580.45900184,438.72942596)(580.64129293,439.57577743)(581.0058751,440.37004573)
\curveto(581.38347807,441.16431404)(581.94988251,441.87394719)(582.70508844,442.4989452)
\curveto(583.43425279,443.11092242)(584.35872901,443.59269172)(585.47851711,443.9442531)
\curveto(586.61132601,444.30883528)(587.87434282,444.49112636)(589.26756755,444.49112636)
\curveto(590.56964674,444.49112636)(591.88474671,444.32836646)(593.21286749,444.00284667)
\curveto(594.55400905,443.69034766)(595.66728675,443.3062343)(596.5527006,442.85050659)
\lineto(596.5527006,438.92473783)
\lineto(596.35738872,438.92473783)
\curveto(595.41989171,439.6148398)(594.28057242,440.19426504)(592.93943086,440.66301355)
\curveto(591.59828929,441.14478285)(590.28318931,441.3856675)(588.99413092,441.3856675)
\curveto(587.65298936,441.3856675)(586.52018046,441.12525166)(585.59570424,440.60441999)
\curveto(584.67122802,440.0966091)(584.2089899,439.33489278)(584.2089899,438.31927101)
\curveto(584.2089899,437.42083637)(584.48893693,436.7437552)(585.04883098,436.28802748)
\curveto(585.59570424,435.83229976)(586.48111809,435.4612072)(587.70507252,435.17474977)
\curveto(588.3821537,435.01850027)(589.13735963,434.86225077)(589.97069031,434.70600127)
\curveto(590.81704178,434.54975176)(591.52016454,434.40652305)(592.08005859,434.27631513)
\curveto(593.78578233,433.88569138)(595.10088231,433.2151206)(596.02535853,432.26460279)
\curveto(596.94983475,431.30106419)(597.41207287,430.02502659)(597.41207287,428.43648998)
\closepath
}
}
{
\newrgbcolor{curcolor}{0 0 0}
\pscustom[linestyle=none,fillstyle=solid,fillcolor=curcolor]
{
\newpath
\moveto(606.6112662,447.61611641)
\lineto(602.47065439,447.61611641)
\lineto(602.47065439,451.42469804)
\lineto(606.6112662,451.42469804)
\closepath
\moveto(606.37689195,422.1474475)
\lineto(602.70502864,422.1474475)
\lineto(602.70502864,443.96378429)
\lineto(606.37689195,443.96378429)
\closepath
}
}
{
\newrgbcolor{curcolor}{0 0 0}
\pscustom[linestyle=none,fillstyle=solid,fillcolor=curcolor]
{
\newpath
\moveto(624.97057925,422.34275938)
\curveto(624.28047728,422.16046829)(623.52527135,422.01072919)(622.70496146,421.89354206)
\curveto(621.89767237,421.77635493)(621.17501842,421.71776137)(620.53699961,421.71776137)
\curveto(618.3104442,421.71776137)(616.61774126,422.3167178)(615.45889078,423.51463065)
\curveto(614.30004031,424.7125435)(613.72061507,426.6331103)(613.72061507,429.27633105)
\lineto(613.72061507,440.87785662)
\lineto(611.24015422,440.87785662)
\lineto(611.24015422,443.96378429)
\lineto(613.72061507,443.96378429)
\lineto(613.72061507,450.23329558)
\lineto(617.39247838,450.23329558)
\lineto(617.39247838,443.96378429)
\lineto(624.97057925,443.96378429)
\lineto(624.97057925,440.87785662)
\lineto(617.39247838,440.87785662)
\lineto(617.39247838,430.93648202)
\curveto(617.39247838,429.79065233)(617.41851996,428.89221769)(617.47060313,428.2411781)
\curveto(617.5226863,427.6031593)(617.70497738,427.00420287)(618.01747639,426.44430882)
\curveto(618.30393381,425.92347715)(618.69455756,425.53936379)(619.18934766,425.29196874)
\curveto(619.69715854,425.05759449)(620.46538526,424.94040736)(621.49402782,424.94040736)
\curveto(622.09298424,424.94040736)(622.71798225,425.02504251)(623.36902185,425.1943128)
\curveto(624.02006144,425.37660389)(624.48880995,425.52634299)(624.77526737,425.64353012)
\lineto(624.97057925,425.64353012)
\closepath
}
}
{
\newrgbcolor{curcolor}{0 0 0}
\pscustom[linestyle=none,fillstyle=solid,fillcolor=curcolor]
{
\newpath
\moveto(647.66582642,432.67475773)
\lineto(631.59165885,432.67475773)
\curveto(631.59165885,431.33361617)(631.79348112,430.1617449)(632.19712567,429.15914393)
\curveto(632.60077022,428.16956375)(633.15415387,427.35576425)(633.85727663,426.71774545)
\curveto(634.53435781,426.09274744)(635.33513651,425.62399893)(636.25961274,425.31149993)
\curveto(637.19710975,424.99900092)(638.22575231,424.84275142)(639.34554041,424.84275142)
\curveto(640.82991068,424.84275142)(642.32079135,425.13571924)(643.81818242,425.72165487)
\curveto(645.32859428,426.3206113)(646.40280961,426.90654693)(647.04082841,427.47946178)
\lineto(647.23614029,427.47946178)
\lineto(647.23614029,423.47556827)
\curveto(645.99916506,422.9547366)(644.73614825,422.51854007)(643.44708985,422.16697869)
\curveto(642.15803146,421.81541731)(640.8038691,421.63963662)(639.38460279,421.63963662)
\curveto(635.76482264,421.63963662)(632.93931081,422.61619601)(630.90806727,424.56931479)
\curveto(628.87682374,426.53545436)(627.86120198,429.32190383)(627.86120198,432.92866318)
\curveto(627.86120198,436.49636015)(628.83125097,439.32838238)(630.77134896,441.42472987)
\curveto(632.72446774,443.52107737)(635.28956374,444.56925111)(638.46663696,444.56925111)
\curveto(641.40933592,444.56925111)(643.67495371,443.70987885)(645.26349032,441.99113432)
\curveto(646.86504772,440.27238979)(647.66582642,437.83099132)(647.66582642,434.66693889)
\closepath
\moveto(644.09161905,435.48724878)
\curveto(644.07859826,437.41432598)(643.59031856,438.90520665)(642.62677996,439.95989079)
\curveto(641.67626216,441.01457493)(640.22444386,441.541917)(638.27132508,441.541917)
\curveto(636.30518551,441.541917)(634.73618009,440.96249176)(633.56430882,439.80364129)
\curveto(632.40545834,438.64479081)(631.74790835,437.20599331)(631.59165885,435.48724878)
\closepath
}
}
{
\newrgbcolor{curcolor}{0 0 0}
\pscustom[linestyle=none,fillstyle=solid,fillcolor=curcolor]
{
\newpath
\moveto(409.64538685,408.4861065)
\lineto(403.14539177,408.4861065)
\lineto(403.14539177,410.37152174)
\lineto(409.64538685,410.37152174)
\closepath
}
}
{
\newrgbcolor{curcolor}{0 0 0}
\pscustom[linestyle=none,fillstyle=solid,fillcolor=curcolor]
{
\newpath
\moveto(426.29120782,402.64236092)
\lineto(420.16621246,402.64236092)
\lineto(420.16621246,404.22569305)
\lineto(422.19746092,404.22569305)
\lineto(422.19746092,416.56943372)
\lineto(420.16621246,416.56943372)
\lineto(420.16621246,418.15276585)
\lineto(426.29120782,418.15276585)
\lineto(426.29120782,416.56943372)
\lineto(424.25995936,416.56943372)
\lineto(424.25995936,404.22569305)
\lineto(426.29120782,404.22569305)
\closepath
}
}
{
\newrgbcolor{curcolor}{0 0 0}
\pscustom[linestyle=none,fillstyle=solid,fillcolor=curcolor]
{
\newpath
\moveto(435.90578279,416.95485009)
\lineto(435.8016162,416.95485009)
\curveto(435.58633859,417.01735004)(435.3050888,417.07985)(434.95786684,417.14234995)
\curveto(434.61064488,417.21179434)(434.30508955,417.24651654)(434.04120087,417.24651654)
\curveto(433.20092372,417.24651654)(432.58981307,417.05901668)(432.20786892,416.68401696)
\curveto(431.8328692,416.31596169)(431.64536934,415.6458233)(431.64536934,414.67360182)
\lineto(431.64536934,414.27776878)
\lineto(435.17661667,414.27776878)
\lineto(435.17661667,412.63193669)
\lineto(431.7078693,412.63193669)
\lineto(431.7078693,402.64236092)
\lineto(429.74953745,402.64236092)
\lineto(429.74953745,412.63193669)
\lineto(428.42662178,412.63193669)
\lineto(428.42662178,414.27776878)
\lineto(429.74953745,414.27776878)
\lineto(429.74953745,414.66318516)
\curveto(429.74953745,416.04512856)(430.09328719,417.10415553)(430.78078667,417.84026609)
\curveto(431.46828615,418.58332108)(432.46134095,418.95484858)(433.75995108,418.95484858)
\curveto(434.19745075,418.95484858)(434.58981156,418.93401526)(434.93703352,418.89234863)
\curveto(435.29119992,418.85068199)(435.61411634,418.80207092)(435.90578279,418.7465154)
\closepath
}
}
{
\newrgbcolor{curcolor}{0 0 0}
\pscustom[linestyle=none,fillstyle=solid,fillcolor=curcolor]
{
\newpath
\moveto(454.56201921,408.45485652)
\curveto(454.56201921,406.55902462)(454.07590847,405.06249798)(453.10368698,403.96527658)
\curveto(452.13146549,402.86805519)(450.82938314,402.3194445)(449.19743993,402.3194445)
\curveto(447.55160785,402.3194445)(446.24258106,402.86805519)(445.27035957,403.96527658)
\curveto(444.30508252,405.06249798)(443.822444,406.55902462)(443.822444,408.45485652)
\curveto(443.822444,410.35068842)(444.30508252,411.84721507)(445.27035957,412.94443646)
\curveto(446.24258106,414.04860229)(447.55160785,414.60068521)(449.19743993,414.60068521)
\curveto(450.82938314,414.60068521)(452.13146549,414.04860229)(453.10368698,412.94443646)
\curveto(454.07590847,411.84721507)(454.56201921,410.35068842)(454.56201921,408.45485652)
\closepath
\moveto(452.5411874,408.45485652)
\curveto(452.5411874,409.96179983)(452.24604874,411.07985454)(451.65577141,411.80902065)
\curveto(451.06549408,412.5451312)(450.24605025,412.91318648)(449.19743993,412.91318648)
\curveto(448.13494074,412.91318648)(447.30855247,412.5451312)(446.71827514,411.80902065)
\curveto(446.13494225,411.07985454)(445.8432758,409.96179983)(445.8432758,408.45485652)
\curveto(445.8432758,406.99652429)(446.13841447,405.88888624)(446.7286918,405.13194237)
\curveto(447.31896913,404.38194294)(448.14188518,404.00694322)(449.19743993,404.00694322)
\curveto(450.23910581,404.00694322)(451.05507742,404.37847072)(451.64535475,405.12152571)
\curveto(452.24257652,405.87152514)(452.5411874,406.98263541)(452.5411874,408.45485652)
\closepath
}
}
{
\newrgbcolor{curcolor}{0 0 0}
\pscustom[linestyle=none,fillstyle=solid,fillcolor=curcolor]
{
\newpath
\moveto(467.83284237,408.60068974)
\curveto(467.83284237,407.65624601)(467.6974258,406.79166334)(467.42659267,406.00694171)
\curveto(467.15575954,405.22916452)(466.77381539,404.56944279)(466.28076021,404.02777654)
\curveto(465.82242722,403.51388804)(465.28076096,403.11458278)(464.65576144,402.82986078)
\curveto(464.03770635,402.55208321)(463.38145684,402.41319443)(462.68701292,402.41319443)
\curveto(462.08284672,402.41319443)(461.53423602,402.4791666)(461.04118084,402.61111094)
\curveto(460.55507009,402.74305529)(460.05854269,402.94791624)(459.55159863,403.22569381)
\lineto(459.55159863,398.3506975)
\lineto(457.59326678,398.3506975)
\lineto(457.59326678,414.27776878)
\lineto(459.55159863,414.27776878)
\lineto(459.55159863,413.0590197)
\curveto(460.07243157,413.49651937)(460.65576446,413.86110243)(461.30159731,414.15276888)
\curveto(461.95437459,414.45137976)(462.64881851,414.60068521)(463.38492906,414.60068521)
\curveto(464.78770578,414.60068521)(465.87798273,414.06943561)(466.65575992,413.00693641)
\curveto(467.44048155,411.95138165)(467.83284237,410.48263277)(467.83284237,408.60068974)
\closepath
\moveto(465.81201056,408.54860645)
\curveto(465.81201056,409.95138317)(465.57242741,410.99999348)(465.0932611,411.6944374)
\curveto(464.6140948,412.38888132)(463.87798425,412.73610328)(462.88492944,412.73610328)
\curveto(462.32242987,412.73610328)(461.75645807,412.6145756)(461.18701406,412.37152023)
\curveto(460.61757005,412.12846485)(460.07243157,411.80902065)(459.55159863,411.41318762)
\lineto(459.55159863,404.81944261)
\curveto(460.10715377,404.56944279)(460.58284785,404.39930403)(460.97868088,404.30902632)
\curveto(461.38145836,404.21874862)(461.83631912,404.17360976)(462.34326319,404.17360976)
\curveto(463.43354014,404.17360976)(464.28423394,404.54166504)(464.89534459,405.27777559)
\curveto(465.50645524,406.01388615)(465.81201056,407.1041631)(465.81201056,408.54860645)
\closepath
}
}
{
\newrgbcolor{curcolor}{0 0 0}
\pscustom[linestyle=none,fillstyle=solid,fillcolor=curcolor]
{
\newpath
\moveto(481.12449718,408.60068974)
\curveto(481.12449718,407.65624601)(480.98908061,406.79166334)(480.71824748,406.00694171)
\curveto(480.44741436,405.22916452)(480.0654702,404.56944279)(479.57241502,404.02777654)
\curveto(479.11408203,403.51388804)(478.57241577,403.11458278)(477.94741625,402.82986078)
\curveto(477.32936116,402.55208321)(476.67311166,402.41319443)(475.97866774,402.41319443)
\curveto(475.37450153,402.41319443)(474.82589083,402.4791666)(474.33283565,402.61111094)
\curveto(473.84672491,402.74305529)(473.3501975,402.94791624)(472.84325344,403.22569381)
\lineto(472.84325344,398.3506975)
\lineto(470.88492159,398.3506975)
\lineto(470.88492159,414.27776878)
\lineto(472.84325344,414.27776878)
\lineto(472.84325344,413.0590197)
\curveto(473.36408638,413.49651937)(473.94741927,413.86110243)(474.59325212,414.15276888)
\curveto(475.2460294,414.45137976)(475.94047332,414.60068521)(476.67658388,414.60068521)
\curveto(478.07936059,414.60068521)(479.16963755,414.06943561)(479.94741473,413.00693641)
\curveto(480.73213636,411.95138165)(481.12449718,410.48263277)(481.12449718,408.60068974)
\closepath
\moveto(479.10366537,408.54860645)
\curveto(479.10366537,409.95138317)(478.86408222,410.99999348)(478.38491592,411.6944374)
\curveto(477.90574961,412.38888132)(477.16963906,412.73610328)(476.17658425,412.73610328)
\curveto(475.61408468,412.73610328)(475.04811289,412.6145756)(474.47866887,412.37152023)
\curveto(473.90922486,412.12846485)(473.36408638,411.80902065)(472.84325344,411.41318762)
\lineto(472.84325344,404.81944261)
\curveto(473.39880858,404.56944279)(473.87450266,404.39930403)(474.2703357,404.30902632)
\curveto(474.67311317,404.21874862)(475.12797394,404.17360976)(475.634918,404.17360976)
\curveto(476.72519495,404.17360976)(477.57588875,404.54166504)(478.1869994,405.27777559)
\curveto(478.79811005,406.01388615)(479.10366537,407.1041631)(479.10366537,408.54860645)
\closepath
}
}
{
\newrgbcolor{curcolor}{0 0 0}
\pscustom[linestyle=none,fillstyle=solid,fillcolor=curcolor]
{
\newpath
\moveto(494.09324278,408.45485652)
\curveto(494.09324278,406.55902462)(493.60713203,405.06249798)(492.63491055,403.96527658)
\curveto(491.66268906,402.86805519)(490.36060671,402.3194445)(488.7286635,402.3194445)
\curveto(487.08283141,402.3194445)(485.77380463,402.86805519)(484.80158314,403.96527658)
\curveto(483.83630609,405.06249798)(483.35366757,406.55902462)(483.35366757,408.45485652)
\curveto(483.35366757,410.35068842)(483.83630609,411.84721507)(484.80158314,412.94443646)
\curveto(485.77380463,414.04860229)(487.08283141,414.60068521)(488.7286635,414.60068521)
\curveto(490.36060671,414.60068521)(491.66268906,414.04860229)(492.63491055,412.94443646)
\curveto(493.60713203,411.84721507)(494.09324278,410.35068842)(494.09324278,408.45485652)
\closepath
\moveto(492.07241097,408.45485652)
\curveto(492.07241097,409.96179983)(491.77727231,411.07985454)(491.18699497,411.80902065)
\curveto(490.59671764,412.5451312)(489.77727382,412.91318648)(488.7286635,412.91318648)
\curveto(487.6661643,412.91318648)(486.83977604,412.5451312)(486.24949871,411.80902065)
\curveto(485.66616582,411.07985454)(485.37449937,409.96179983)(485.37449937,408.45485652)
\curveto(485.37449937,406.99652429)(485.66963804,405.88888624)(486.25991537,405.13194237)
\curveto(486.8501927,404.38194294)(487.67310874,404.00694322)(488.7286635,404.00694322)
\curveto(489.77032938,404.00694322)(490.58630098,404.37847072)(491.17657832,405.12152571)
\curveto(491.77380009,405.87152514)(492.07241097,406.98263541)(492.07241097,408.45485652)
\closepath
}
}
{
\newrgbcolor{curcolor}{0 0 0}
\pscustom[linestyle=none,fillstyle=solid,fillcolor=curcolor]
{
\newpath
\moveto(505.43698406,405.99652505)
\curveto(505.43698406,404.93402585)(504.99601217,404.06249873)(504.11406839,403.38194369)
\curveto(503.23906905,402.70138865)(502.04115329,402.36111113)(500.52032111,402.36111113)
\curveto(499.65921065,402.36111113)(498.86754458,402.4618055)(498.14532291,402.66319424)
\curveto(497.43004567,402.87152741)(496.82935168,403.09722169)(496.34324094,403.34027706)
\lineto(496.34324094,405.53819206)
\lineto(496.44740753,405.53819206)
\curveto(497.06546261,405.07291464)(497.75296209,404.70138714)(498.50990596,404.42360957)
\curveto(499.26684984,404.15277644)(499.99254373,404.01735988)(500.68698765,404.01735988)
\curveto(501.54809811,404.01735988)(502.22170871,404.15624866)(502.70781946,404.43402623)
\curveto(503.1939302,404.7118038)(503.43698557,405.14930347)(503.43698557,405.74652524)
\curveto(503.43698557,406.20485822)(503.30504123,406.55208018)(503.04115254,406.78819112)
\curveto(502.77726385,407.02430205)(502.27031979,407.22569078)(501.52032035,407.39235733)
\curveto(501.24254279,407.45485728)(500.87795973,407.52777389)(500.42657118,407.61110716)
\curveto(499.98212707,407.69444043)(499.57587738,407.78471814)(499.2078221,407.88194029)
\curveto(498.18698954,408.15277342)(497.46129565,408.54860645)(497.03074042,409.06943939)
\curveto(496.60712963,409.59721677)(496.39532423,410.24304961)(496.39532423,411.00693792)
\curveto(496.39532423,411.48610423)(496.49254638,411.93749278)(496.68699068,412.36110357)
\curveto(496.88837941,412.78471436)(497.19046252,413.16318629)(497.59323999,413.49651937)
\curveto(497.98212859,413.82290802)(498.47518377,414.07985227)(499.07240554,414.26735212)
\curveto(499.67657175,414.46179642)(500.35018235,414.55901857)(501.09323734,414.55901857)
\curveto(501.78768126,414.55901857)(502.48906962,414.47221308)(503.19740242,414.2986021)
\curveto(503.91267965,414.13193556)(504.50642921,413.9270746)(504.97865107,413.68401923)
\lineto(504.97865107,411.59027082)
\lineto(504.87448448,411.59027082)
\curveto(504.37448486,411.95832609)(503.76684643,412.26735364)(503.0515692,412.51735345)
\curveto(502.33629196,412.7742977)(501.6349036,412.90276982)(500.94740412,412.90276982)
\curveto(500.23212688,412.90276982)(499.62796067,412.76388104)(499.13490549,412.48610347)
\curveto(498.64185031,412.21527034)(498.39532272,411.80902065)(498.39532272,411.26735439)
\curveto(498.39532272,410.78818809)(498.54462816,410.42707725)(498.84323905,410.18402188)
\curveto(499.13490549,409.94096651)(499.60712736,409.74304999)(500.25990464,409.59027233)
\curveto(500.62101548,409.50693906)(501.02379295,409.42360579)(501.46823706,409.34027252)
\curveto(501.91962561,409.25693925)(502.29462532,409.18055042)(502.59323621,409.11110603)
\curveto(503.50295774,408.90277285)(504.2043461,408.54513423)(504.69740128,408.03819017)
\curveto(505.19045647,407.52430167)(505.43698406,406.84374663)(505.43698406,405.99652505)
\closepath
}
}
{
\newrgbcolor{curcolor}{0 0 0}
\pscustom[linestyle=none,fillstyle=solid,fillcolor=curcolor]
{
\newpath
\moveto(510.34322946,416.22568398)
\lineto(508.1348978,416.22568398)
\lineto(508.1348978,418.25693244)
\lineto(510.34322946,418.25693244)
\closepath
\moveto(510.21822956,402.64236092)
\lineto(508.25989771,402.64236092)
\lineto(508.25989771,414.27776878)
\lineto(510.21822956,414.27776878)
\closepath
}
}
{
\newrgbcolor{curcolor}{0 0 0}
\pscustom[linestyle=none,fillstyle=solid,fillcolor=curcolor]
{
\newpath
\moveto(520.13488226,402.74652751)
\curveto(519.76682698,402.64930536)(519.36404951,402.56944431)(518.92654984,402.50694435)
\curveto(518.49599461,402.4444444)(518.11057823,402.41319443)(517.77030071,402.41319443)
\curveto(516.58280161,402.41319443)(515.68002452,402.73263863)(515.06196943,403.37152703)
\curveto(514.44391434,404.01041544)(514.1348868,405.03472022)(514.1348868,406.44444138)
\lineto(514.1348868,412.63193669)
\lineto(512.81197113,412.63193669)
\lineto(512.81197113,414.27776878)
\lineto(514.1348868,414.27776878)
\lineto(514.1348868,417.62151625)
\lineto(516.09321865,417.62151625)
\lineto(516.09321865,414.27776878)
\lineto(520.13488226,414.27776878)
\lineto(520.13488226,412.63193669)
\lineto(516.09321865,412.63193669)
\lineto(516.09321865,407.32985737)
\curveto(516.09321865,406.71874672)(516.10710753,406.23958042)(516.13488528,405.89235846)
\curveto(516.16266304,405.55208094)(516.25988519,405.23263674)(516.42655173,404.93402585)
\curveto(516.57932939,404.65624828)(516.78766257,404.45138733)(517.05155126,404.31944298)
\curveto(517.32238438,404.19444308)(517.7321063,404.13194313)(518.28071699,404.13194313)
\curveto(518.6001612,404.13194313)(518.93349428,404.17708198)(519.28071624,404.26735969)
\curveto(519.6279382,404.36458184)(519.87793801,404.44444289)(520.03071567,404.50694284)
\lineto(520.13488226,404.50694284)
\closepath
}
}
{
\newrgbcolor{curcolor}{0 0 0}
\pscustom[linestyle=none,fillstyle=solid,fillcolor=curcolor]
{
\newpath
\moveto(532.23903844,408.25694)
\lineto(523.66612826,408.25694)
\curveto(523.66612826,407.54166277)(523.77376706,406.91666324)(523.98904468,406.38194142)
\curveto(524.20432229,405.85416404)(524.49946096,405.4201366)(524.87446067,405.07985907)
\curveto(525.23557151,404.74652599)(525.66265452,404.49652618)(526.15570971,404.32985964)
\curveto(526.65570933,404.1631931)(527.20432002,404.07985983)(527.80154179,404.07985983)
\curveto(528.59320786,404.07985983)(529.38834615,404.23610971)(530.18695666,404.54860948)
\curveto(530.9925116,404.86805368)(531.56542784,405.18055344)(531.90570536,405.48610877)
\lineto(532.00987194,405.48610877)
\lineto(532.00987194,403.35069372)
\curveto(531.35015022,403.07291615)(530.67653962,402.84027744)(529.98904014,402.65277758)
\curveto(529.30154066,402.46527772)(528.57931898,402.37152779)(527.82237511,402.37152779)
\curveto(525.89182102,402.37152779)(524.38487771,402.89236073)(523.3015452,403.93402661)
\curveto(522.21821268,404.98263693)(521.67654643,406.46874691)(521.67654643,408.39235657)
\curveto(521.67654643,410.29513291)(522.19390715,411.80554843)(523.22862859,412.92360314)
\curveto(524.27029447,414.04165785)(525.63834899,414.60068521)(527.33279215,414.60068521)
\curveto(528.90223541,414.60068521)(530.11056782,414.14235222)(530.95778941,413.22568625)
\curveto(531.81195543,412.30902027)(532.23903844,411.00693792)(532.23903844,409.3194392)
\closepath
\moveto(530.33278988,409.75693887)
\curveto(530.32584544,410.78471587)(530.06542897,411.57985416)(529.55154047,412.14235373)
\curveto(529.04459641,412.70485331)(528.27029144,412.98610309)(527.22862556,412.98610309)
\curveto(526.18001524,412.98610309)(525.34321032,412.67707555)(524.71821079,412.05902046)
\curveto(524.10015571,411.44096537)(523.74946153,410.67360484)(523.66612826,409.75693887)
\closepath
}
}
{
\newrgbcolor{curcolor}{0 0 0}
\pscustom[linestyle=none,fillstyle=solid,fillcolor=curcolor]
{
\newpath
\moveto(552.94736498,408.60068974)
\curveto(552.94736498,407.65624601)(552.81194842,406.79166334)(552.54111529,406.00694171)
\curveto(552.27028216,405.22916452)(551.88833801,404.56944279)(551.39528282,404.02777654)
\curveto(550.93694984,403.51388804)(550.39528358,403.11458278)(549.77028405,402.82986078)
\curveto(549.15222897,402.55208321)(548.49597946,402.41319443)(547.80153554,402.41319443)
\curveto(547.19736933,402.41319443)(546.64875864,402.4791666)(546.15570346,402.61111094)
\curveto(545.66959271,402.74305529)(545.17306531,402.94791624)(544.66612125,403.22569381)
\lineto(544.66612125,398.3506975)
\lineto(542.7077894,398.3506975)
\lineto(542.7077894,414.27776878)
\lineto(544.66612125,414.27776878)
\lineto(544.66612125,413.0590197)
\curveto(545.18695419,413.49651937)(545.77028708,413.86110243)(546.41611992,414.15276888)
\curveto(547.06889721,414.45137976)(547.76334113,414.60068521)(548.49945168,414.60068521)
\curveto(549.9022284,414.60068521)(550.99250535,414.06943561)(551.77028254,413.00693641)
\curveto(552.55500417,411.95138165)(552.94736498,410.48263277)(552.94736498,408.60068974)
\closepath
\moveto(550.92653318,408.54860645)
\curveto(550.92653318,409.95138317)(550.68695003,410.99999348)(550.20778372,411.6944374)
\curveto(549.72861742,412.38888132)(548.99250686,412.73610328)(547.99945206,412.73610328)
\curveto(547.43695249,412.73610328)(546.87098069,412.6145756)(546.30153668,412.37152023)
\curveto(545.73209266,412.12846485)(545.18695419,411.80902065)(544.66612125,411.41318762)
\lineto(544.66612125,404.81944261)
\curveto(545.22167638,404.56944279)(545.69737047,404.39930403)(546.0932035,404.30902632)
\curveto(546.49598098,404.21874862)(546.95084174,404.17360976)(547.4577858,404.17360976)
\curveto(548.54806276,404.17360976)(549.39875656,404.54166504)(550.00986721,405.27777559)
\curveto(550.62097785,406.01388615)(550.92653318,407.1041631)(550.92653318,408.54860645)
\closepath
}
}
{
\newrgbcolor{curcolor}{0 0 0}
\pscustom[linestyle=none,fillstyle=solid,fillcolor=curcolor]
{
\newpath
\moveto(565.04110404,402.64236092)
\lineto(563.09318884,402.64236092)
\lineto(563.09318884,403.88194331)
\curveto(562.91957786,403.76388785)(562.68346693,403.59722131)(562.38485605,403.38194369)
\curveto(562.0931896,403.17361052)(561.80846759,403.00694398)(561.53069002,402.88194407)
\curveto(561.20430138,402.72222197)(560.82930167,402.59027763)(560.40569088,402.48611104)
\curveto(559.98208009,402.37500001)(559.48555268,402.3194445)(558.91610867,402.3194445)
\curveto(557.86749835,402.3194445)(556.97861014,402.66666646)(556.24944402,403.36111038)
\curveto(555.52027791,404.05555429)(555.15569485,404.94097029)(555.15569485,406.01735837)
\curveto(555.15569485,406.89930214)(555.34319471,407.61110716)(555.71819442,408.15277342)
\curveto(556.10013858,408.70138411)(556.64180483,409.13193934)(557.34319319,409.44443911)
\curveto(558.05152599,409.75693887)(558.90221979,409.96874427)(559.8952746,410.07985529)
\curveto(560.8883294,410.19096632)(561.95430082,410.27429959)(563.09318884,410.3298551)
\lineto(563.09318884,410.63193821)
\curveto(563.09318884,411.07638232)(563.01332779,411.44443759)(562.85360569,411.73610404)
\curveto(562.70082803,412.02777049)(562.47860597,412.25693698)(562.18693953,412.42360352)
\curveto(561.90916196,412.58332562)(561.57582888,412.69096443)(561.18694028,412.74651994)
\curveto(560.79805169,412.80207545)(560.391802,412.82985321)(559.96819121,412.82985321)
\curveto(559.45430271,412.82985321)(558.88138647,412.76040882)(558.24944251,412.62152004)
\curveto(557.61749854,412.48957569)(556.96472126,412.29513139)(556.29111066,412.03818714)
\lineto(556.18694407,412.03818714)
\lineto(556.18694407,414.02776897)
\curveto(556.56888822,414.13193556)(557.12097114,414.24651881)(557.84319281,414.37151871)
\curveto(558.56541449,414.49651862)(559.27721951,414.55901857)(559.97860787,414.55901857)
\curveto(560.79805169,414.55901857)(561.50985671,414.48957418)(562.11402292,414.35068539)
\curveto(562.72513357,414.21874105)(563.25291094,413.98957456)(563.69735505,413.66318591)
\curveto(564.13485472,413.34374171)(564.4681878,412.93054758)(564.6973543,412.42360352)
\curveto(564.92652079,411.91665946)(565.04110404,411.28818771)(565.04110404,410.53818828)
\closepath
\moveto(563.09318884,405.50694209)
\lineto(563.09318884,408.74652297)
\curveto(562.49596707,408.71180077)(561.79110649,408.65971748)(560.97860711,408.59027309)
\curveto(560.17305216,408.52082869)(559.53416376,408.42013433)(559.06194189,408.28818998)
\curveto(558.49944232,408.12846788)(558.04458155,407.87846807)(557.69735959,407.53819055)
\curveto(557.35013763,407.20485747)(557.17652665,406.74305226)(557.17652665,406.15277493)
\curveto(557.17652665,405.48610877)(557.37791539,404.98263693)(557.78069286,404.64235941)
\curveto(558.18347033,404.30902632)(558.7980532,404.14235978)(559.62444147,404.14235978)
\curveto(560.31194095,404.14235978)(560.94041269,404.27430413)(561.50985671,404.53819282)
\curveto(562.07930072,404.80902595)(562.6070781,405.13194237)(563.09318884,405.50694209)
\closepath
}
}
{
\newrgbcolor{curcolor}{0 0 0}
\pscustom[linestyle=none,fillstyle=solid,fillcolor=curcolor]
{
\newpath
\moveto(576.07233616,412.14235373)
\lineto(575.96816957,412.14235373)
\curveto(575.67650313,412.21179812)(575.39178112,412.2604092)(575.11400355,412.28818695)
\curveto(574.84317042,412.32290915)(574.520254,412.34027025)(574.14525428,412.34027025)
\curveto(573.54108807,412.34027025)(572.95775518,412.20485368)(572.39525561,411.93402056)
\curveto(571.83275603,411.67013187)(571.29108978,411.32638213)(570.77025684,410.90277134)
\lineto(570.77025684,402.64236092)
\lineto(568.81192499,402.64236092)
\lineto(568.81192499,414.27776878)
\lineto(570.77025684,414.27776878)
\lineto(570.77025684,412.55902008)
\curveto(571.54803403,413.18401961)(572.23206129,413.6249915)(572.82233862,413.88193575)
\curveto(573.41956039,414.14582444)(574.02719882,414.27776878)(574.64525391,414.27776878)
\curveto(574.98553143,414.27776878)(575.23205902,414.26735212)(575.38483668,414.24651881)
\curveto(575.53761434,414.23262993)(575.76678083,414.20137995)(576.07233616,414.15276888)
\closepath
}
}
{
\newrgbcolor{curcolor}{0 0 0}
\pscustom[linestyle=none,fillstyle=solid,fillcolor=curcolor]
{
\newpath
\moveto(586.58275596,402.64236092)
\lineto(584.63484077,402.64236092)
\lineto(584.63484077,403.88194331)
\curveto(584.46122979,403.76388785)(584.22511886,403.59722131)(583.92650797,403.38194369)
\curveto(583.63484153,403.17361052)(583.35011952,403.00694398)(583.07234195,402.88194407)
\curveto(582.74595331,402.72222197)(582.37095359,402.59027763)(581.9473428,402.48611104)
\curveto(581.52373201,402.37500001)(581.02720461,402.3194445)(580.4577606,402.3194445)
\curveto(579.40915028,402.3194445)(578.52026206,402.66666646)(577.79109595,403.36111038)
\curveto(577.06192983,404.05555429)(576.69734678,404.94097029)(576.69734678,406.01735837)
\curveto(576.69734678,406.89930214)(576.88484663,407.61110716)(577.25984635,408.15277342)
\curveto(577.64179051,408.70138411)(578.18345676,409.13193934)(578.88484512,409.44443911)
\curveto(579.59317792,409.75693887)(580.44387172,409.96874427)(581.43692652,410.07985529)
\curveto(582.42998133,410.19096632)(583.49595274,410.27429959)(584.63484077,410.3298551)
\lineto(584.63484077,410.63193821)
\curveto(584.63484077,411.07638232)(584.55497972,411.44443759)(584.39525762,411.73610404)
\curveto(584.24247996,412.02777049)(584.0202579,412.25693698)(583.72859146,412.42360352)
\curveto(583.45081389,412.58332562)(583.11748081,412.69096443)(582.72859221,412.74651994)
\curveto(582.33970362,412.80207545)(581.93345393,412.82985321)(581.50984314,412.82985321)
\curveto(580.99595464,412.82985321)(580.4230384,412.76040882)(579.79109444,412.62152004)
\curveto(579.15915047,412.48957569)(578.50637319,412.29513139)(577.83276258,412.03818714)
\lineto(577.728596,412.03818714)
\lineto(577.728596,414.02776897)
\curveto(578.11054015,414.13193556)(578.66262307,414.24651881)(579.38484474,414.37151871)
\curveto(580.10706642,414.49651862)(580.81887144,414.55901857)(581.52025979,414.55901857)
\curveto(582.33970362,414.55901857)(583.05150864,414.48957418)(583.65567485,414.35068539)
\curveto(584.26678549,414.21874105)(584.79456287,413.98957456)(585.23900698,413.66318591)
\curveto(585.67650665,413.34374171)(586.00983973,412.93054758)(586.23900622,412.42360352)
\curveto(586.46817272,411.91665946)(586.58275596,411.28818771)(586.58275596,410.53818828)
\closepath
\moveto(584.63484077,405.50694209)
\lineto(584.63484077,408.74652297)
\curveto(584.037619,408.71180077)(583.33275842,408.65971748)(582.52025904,408.59027309)
\curveto(581.71470409,408.52082869)(581.07581569,408.42013433)(580.60359382,408.28818998)
\curveto(580.04109425,408.12846788)(579.58623348,407.87846807)(579.23901152,407.53819055)
\curveto(578.89178956,407.20485747)(578.71817858,406.74305226)(578.71817858,406.15277493)
\curveto(578.71817858,405.48610877)(578.91956732,404.98263693)(579.32234479,404.64235941)
\curveto(579.72512226,404.30902632)(580.33970513,404.14235978)(581.1660934,404.14235978)
\curveto(581.85359288,404.14235978)(582.48206462,404.27430413)(583.05150864,404.53819282)
\curveto(583.62095265,404.80902595)(584.14873003,405.13194237)(584.63484077,405.50694209)
\closepath
}
}
{
\newrgbcolor{curcolor}{0 0 0}
\pscustom[linestyle=none,fillstyle=solid,fillcolor=curcolor]
{
\newpath
\moveto(607.33273074,402.64236092)
\lineto(605.37439888,402.64236092)
\lineto(605.37439888,409.26735591)
\curveto(605.37439888,409.76735553)(605.35009335,410.24999405)(605.30148227,410.71527148)
\curveto(605.25981564,411.1805489)(605.16606571,411.5520764)(605.02023248,411.82985397)
\curveto(604.86051038,412.12846485)(604.63134389,412.35415913)(604.332733,412.50693679)
\curveto(604.03412212,412.65971445)(603.60356689,412.73610328)(603.04106732,412.73610328)
\curveto(602.49245662,412.73610328)(601.94384592,412.5972145)(601.39523523,412.31943693)
\curveto(600.84662453,412.0486038)(600.29801384,411.70138184)(599.74940314,411.27777105)
\curveto(599.77023646,411.11804895)(599.78759755,410.93054909)(599.80148643,410.71527148)
\curveto(599.81537531,410.5069383)(599.82231975,410.29860513)(599.82231975,410.09027195)
\lineto(599.82231975,402.64236092)
\lineto(597.8639879,402.64236092)
\lineto(597.8639879,409.26735591)
\curveto(597.8639879,409.78124441)(597.83968236,410.26735515)(597.79107129,410.72568814)
\curveto(597.74940465,411.19096556)(597.65565472,411.56249306)(597.5098215,411.84027063)
\curveto(597.3500994,412.13888151)(597.12093291,412.36110357)(596.82232202,412.50693679)
\curveto(596.52371113,412.65971445)(596.09315591,412.73610328)(595.53065633,412.73610328)
\curveto(594.99593451,412.73610328)(594.45774048,412.60415894)(593.91607422,412.34027025)
\curveto(593.3813524,412.07638156)(592.84663058,411.73957626)(592.31190877,411.32985435)
\lineto(592.31190877,402.64236092)
\lineto(590.35357691,402.64236092)
\lineto(590.35357691,414.27776878)
\lineto(592.31190877,414.27776878)
\lineto(592.31190877,412.98610309)
\curveto(592.92301941,413.49304715)(593.53065784,413.88888019)(594.13482405,414.1736022)
\curveto(594.7459347,414.4583242)(595.39523977,414.60068521)(596.08273925,414.60068521)
\curveto(596.87440531,414.60068521)(597.5445437,414.43401866)(598.09315439,414.10068558)
\curveto(598.64870953,413.7673525)(599.06190366,413.3055473)(599.33273679,412.71526996)
\curveto(600.12440286,413.38193613)(600.84662453,413.86110243)(601.49940182,414.15276888)
\curveto(602.1521791,414.45137976)(602.85009524,414.60068521)(603.59315023,414.60068521)
\curveto(604.87092704,414.60068521)(605.81189855,414.21179661)(606.41606476,413.43401942)
\curveto(607.02717541,412.66318667)(607.33273074,411.58332638)(607.33273074,410.19443854)
\closepath
}
}
{
\newrgbcolor{curcolor}{0 0 0}
\pscustom[linestyle=none,fillstyle=solid,fillcolor=curcolor]
{
\newpath
\moveto(620.84314038,408.25694)
\lineto(612.2702302,408.25694)
\curveto(612.2702302,407.54166277)(612.37786901,406.91666324)(612.59314663,406.38194142)
\curveto(612.80842424,405.85416404)(613.10356291,405.4201366)(613.47856262,405.07985907)
\curveto(613.83967346,404.74652599)(614.26675647,404.49652618)(614.75981165,404.32985964)
\curveto(615.25981127,404.1631931)(615.80842197,404.07985983)(616.40564374,404.07985983)
\curveto(617.19730981,404.07985983)(617.9924481,404.23610971)(618.7910586,404.54860948)
\curveto(619.59661355,404.86805368)(620.16952978,405.18055344)(620.5098073,405.48610877)
\lineto(620.61397389,405.48610877)
\lineto(620.61397389,403.35069372)
\curveto(619.95425217,403.07291615)(619.28064157,402.84027744)(618.59314209,402.65277758)
\curveto(617.90564261,402.46527772)(617.18342093,402.37152779)(616.42647706,402.37152779)
\curveto(614.49592296,402.37152779)(612.98897966,402.89236073)(611.90564715,403.93402661)
\curveto(610.82231463,404.98263693)(610.28064837,406.46874691)(610.28064837,408.39235657)
\curveto(610.28064837,410.29513291)(610.79800909,411.80554843)(611.83273053,412.92360314)
\curveto(612.87439641,414.04165785)(614.24245093,414.60068521)(615.9368941,414.60068521)
\curveto(617.50633735,414.60068521)(618.71466977,414.14235222)(619.56189135,413.22568625)
\curveto(620.41605737,412.30902027)(620.84314038,411.00693792)(620.84314038,409.3194392)
\closepath
\moveto(618.93689183,409.75693887)
\curveto(618.92994739,410.78471587)(618.66953092,411.57985416)(618.15564242,412.14235373)
\curveto(617.64869836,412.70485331)(616.87439339,412.98610309)(615.83272751,412.98610309)
\curveto(614.78411719,412.98610309)(613.94731227,412.67707555)(613.32231274,412.05902046)
\curveto(612.70425765,411.44096537)(612.35356347,410.67360484)(612.2702302,409.75693887)
\closepath
}
}
{
\newrgbcolor{curcolor}{0 0 0}
\pscustom[linestyle=none,fillstyle=solid,fillcolor=curcolor]
{
\newpath
\moveto(629.85355344,402.74652751)
\curveto(629.48549816,402.64930536)(629.08272069,402.56944431)(628.64522102,402.50694435)
\curveto(628.21466579,402.4444444)(627.82924942,402.41319443)(627.4889719,402.41319443)
\curveto(626.30147279,402.41319443)(625.3986957,402.73263863)(624.78064061,403.37152703)
\curveto(624.16258552,404.01041544)(623.85355798,405.03472022)(623.85355798,406.44444138)
\lineto(623.85355798,412.63193669)
\lineto(622.53064231,412.63193669)
\lineto(622.53064231,414.27776878)
\lineto(623.85355798,414.27776878)
\lineto(623.85355798,417.62151625)
\lineto(625.81188983,417.62151625)
\lineto(625.81188983,414.27776878)
\lineto(629.85355344,414.27776878)
\lineto(629.85355344,412.63193669)
\lineto(625.81188983,412.63193669)
\lineto(625.81188983,407.32985737)
\curveto(625.81188983,406.71874672)(625.82577871,406.23958042)(625.85355647,405.89235846)
\curveto(625.88133422,405.55208094)(625.97855637,405.23263674)(626.14522291,404.93402585)
\curveto(626.29800057,404.65624828)(626.50633375,404.45138733)(626.77022244,404.31944298)
\curveto(627.04105557,404.19444308)(627.45077748,404.13194313)(627.99938818,404.13194313)
\curveto(628.31883238,404.13194313)(628.65216546,404.17708198)(628.99938742,404.26735969)
\curveto(629.34660938,404.36458184)(629.59660919,404.44444289)(629.74938685,404.50694284)
\lineto(629.85355344,404.50694284)
\closepath
}
}
{
\newrgbcolor{curcolor}{0 0 0}
\pscustom[linestyle=none,fillstyle=solid,fillcolor=curcolor]
{
\newpath
\moveto(641.95770962,408.25694)
\lineto(633.38479944,408.25694)
\curveto(633.38479944,407.54166277)(633.49243825,406.91666324)(633.70771586,406.38194142)
\curveto(633.92299348,405.85416404)(634.21813214,405.4201366)(634.59313186,405.07985907)
\curveto(634.9542427,404.74652599)(635.38132571,404.49652618)(635.87438089,404.32985964)
\curveto(636.37438051,404.1631931)(636.92299121,404.07985983)(637.52021298,404.07985983)
\curveto(638.31187904,404.07985983)(639.10701733,404.23610971)(639.90562784,404.54860948)
\curveto(640.71118278,404.86805368)(641.28409902,405.18055344)(641.62437654,405.48610877)
\lineto(641.72854313,405.48610877)
\lineto(641.72854313,403.35069372)
\curveto(641.0688214,403.07291615)(640.3952108,402.84027744)(639.70771132,402.65277758)
\curveto(639.02021184,402.46527772)(638.29799017,402.37152779)(637.54104629,402.37152779)
\curveto(635.6104922,402.37152779)(634.10354889,402.89236073)(633.02021638,403.93402661)
\curveto(631.93688387,404.98263693)(631.39521761,406.46874691)(631.39521761,408.39235657)
\curveto(631.39521761,410.29513291)(631.91257833,411.80554843)(632.94729977,412.92360314)
\curveto(633.98896565,414.04165785)(635.35702017,414.60068521)(637.05146333,414.60068521)
\curveto(638.62090659,414.60068521)(639.82923901,414.14235222)(640.67646059,413.22568625)
\curveto(641.53062661,412.30902027)(641.95770962,411.00693792)(641.95770962,409.3194392)
\closepath
\moveto(640.05146106,409.75693887)
\curveto(640.04451662,410.78471587)(639.78410015,411.57985416)(639.27021165,412.14235373)
\curveto(638.76326759,412.70485331)(637.98896262,412.98610309)(636.94729674,412.98610309)
\curveto(635.89868643,412.98610309)(635.0618815,412.67707555)(634.43688198,412.05902046)
\curveto(633.81882689,411.44096537)(633.46813271,410.67360484)(633.38479944,409.75693887)
\closepath
}
}
{
\newrgbcolor{curcolor}{0 0 0}
\pscustom[linestyle=none,fillstyle=solid,fillcolor=curcolor]
{
\newpath
\moveto(652.18685734,412.14235373)
\lineto(652.08269075,412.14235373)
\curveto(651.7910243,412.21179812)(651.5063023,412.2604092)(651.22852473,412.28818695)
\curveto(650.9576916,412.32290915)(650.63477518,412.34027025)(650.25977546,412.34027025)
\curveto(649.65560925,412.34027025)(649.07227636,412.20485368)(648.50977678,411.93402056)
\curveto(647.94727721,411.67013187)(647.40561095,411.32638213)(646.88477801,410.90277134)
\lineto(646.88477801,402.64236092)
\lineto(644.92644616,402.64236092)
\lineto(644.92644616,414.27776878)
\lineto(646.88477801,414.27776878)
\lineto(646.88477801,412.55902008)
\curveto(647.6625552,413.18401961)(648.34658246,413.6249915)(648.93685979,413.88193575)
\curveto(649.53408157,414.14582444)(650.14171999,414.27776878)(650.75977508,414.27776878)
\curveto(651.1000526,414.27776878)(651.34658019,414.26735212)(651.49935786,414.24651881)
\curveto(651.65213552,414.23262993)(651.88130201,414.20137995)(652.18685734,414.15276888)
\closepath
}
}
{
\newrgbcolor{curcolor}{0 0 0}
\pscustom[linestyle=none,fillstyle=solid,fillcolor=curcolor]
{
\newpath
\moveto(663.63478816,416.22568398)
\lineto(661.42645649,416.22568398)
\lineto(661.42645649,418.25693244)
\lineto(663.63478816,418.25693244)
\closepath
\moveto(663.50978825,402.64236092)
\lineto(661.5514564,402.64236092)
\lineto(661.5514564,414.27776878)
\lineto(663.50978825,414.27776878)
\closepath
}
}
{
\newrgbcolor{curcolor}{0 0 0}
\pscustom[linestyle=none,fillstyle=solid,fillcolor=curcolor]
{
\newpath
\moveto(675.69727257,405.99652505)
\curveto(675.69727257,404.93402585)(675.25630068,404.06249873)(674.3743569,403.38194369)
\curveto(673.49935756,402.70138865)(672.3014418,402.36111113)(670.78060962,402.36111113)
\curveto(669.91949916,402.36111113)(669.12783309,402.4618055)(668.40561142,402.66319424)
\curveto(667.69033418,402.87152741)(667.08964019,403.09722169)(666.60352945,403.34027706)
\lineto(666.60352945,405.53819206)
\lineto(666.70769603,405.53819206)
\curveto(667.32575112,405.07291464)(668.0132506,404.70138714)(668.77019447,404.42360957)
\curveto(669.52713835,404.15277644)(670.25283224,404.01735988)(670.94727616,404.01735988)
\curveto(671.80838662,404.01735988)(672.48199722,404.15624866)(672.96810796,404.43402623)
\curveto(673.45421871,404.7118038)(673.69727408,405.14930347)(673.69727408,405.74652524)
\curveto(673.69727408,406.20485822)(673.56532973,406.55208018)(673.30144105,406.78819112)
\curveto(673.03755236,407.02430205)(672.5306083,407.22569078)(671.78060886,407.39235733)
\curveto(671.50283129,407.45485728)(671.13824824,407.52777389)(670.68685969,407.61110716)
\curveto(670.24241558,407.69444043)(669.83616589,407.78471814)(669.46811061,407.88194029)
\curveto(668.44727805,408.15277342)(667.72158416,408.54860645)(667.29102893,409.06943939)
\curveto(666.86741814,409.59721677)(666.65561274,410.24304961)(666.65561274,411.00693792)
\curveto(666.65561274,411.48610423)(666.75283489,411.93749278)(666.94727919,412.36110357)
\curveto(667.14866792,412.78471436)(667.45075103,413.16318629)(667.8535285,413.49651937)
\curveto(668.24241709,413.82290802)(668.73547228,414.07985227)(669.33269405,414.26735212)
\curveto(669.93686026,414.46179642)(670.61047086,414.55901857)(671.35352585,414.55901857)
\curveto(672.04796977,414.55901857)(672.74935813,414.47221308)(673.45769093,414.2986021)
\curveto(674.17296816,414.13193556)(674.76671771,413.9270746)(675.23893958,413.68401923)
\lineto(675.23893958,411.59027082)
\lineto(675.13477299,411.59027082)
\curveto(674.63477337,411.95832609)(674.02713494,412.26735364)(673.3118577,412.51735345)
\curveto(672.59658047,412.7742977)(671.89519211,412.90276982)(671.20769263,412.90276982)
\curveto(670.49241539,412.90276982)(669.88824918,412.76388104)(669.395194,412.48610347)
\curveto(668.90213882,412.21527034)(668.65561123,411.80902065)(668.65561123,411.26735439)
\curveto(668.65561123,410.78818809)(668.80491667,410.42707725)(669.10352755,410.18402188)
\curveto(669.395194,409.94096651)(669.86741587,409.74304999)(670.52019315,409.59027233)
\curveto(670.88130399,409.50693906)(671.28408146,409.42360579)(671.72852557,409.34027252)
\curveto(672.17991412,409.25693925)(672.55491383,409.18055042)(672.85352472,409.11110603)
\curveto(673.76324625,408.90277285)(674.46463461,408.54513423)(674.95768979,408.03819017)
\curveto(675.45074497,407.52430167)(675.69727257,406.84374663)(675.69727257,405.99652505)
\closepath
}
}
{
\newrgbcolor{curcolor}{0 0 0}
\pscustom[linestyle=none,fillstyle=solid,fillcolor=curcolor]
{
\newpath
\moveto(694.31183219,405.99652505)
\curveto(694.31183219,404.93402585)(693.8708603,404.06249873)(692.98891652,403.38194369)
\curveto(692.11391719,402.70138865)(690.91600143,402.36111113)(689.39516924,402.36111113)
\curveto(688.53405878,402.36111113)(687.74239272,402.4618055)(687.02017104,402.66319424)
\curveto(686.3048938,402.87152741)(685.70419981,403.09722169)(685.21808907,403.34027706)
\lineto(685.21808907,405.53819206)
\lineto(685.32225566,405.53819206)
\curveto(685.94031075,405.07291464)(686.62781023,404.70138714)(687.3847541,404.42360957)
\curveto(688.14169797,404.15277644)(688.86739186,404.01735988)(689.56183578,404.01735988)
\curveto(690.42294624,404.01735988)(691.09655684,404.15624866)(691.58266759,404.43402623)
\curveto(692.06877833,404.7118038)(692.3118337,405.14930347)(692.3118337,405.74652524)
\curveto(692.3118337,406.20485822)(692.17988936,406.55208018)(691.91600067,406.78819112)
\curveto(691.65211198,407.02430205)(691.14516792,407.22569078)(690.39516849,407.39235733)
\curveto(690.11739092,407.45485728)(689.75280786,407.52777389)(689.30141931,407.61110716)
\curveto(688.85697521,407.69444043)(688.45072551,407.78471814)(688.08267024,407.88194029)
\curveto(687.06183768,408.15277342)(686.33614378,408.54860645)(685.90558855,409.06943939)
\curveto(685.48197776,409.59721677)(685.27017236,410.24304961)(685.27017236,411.00693792)
\curveto(685.27017236,411.48610423)(685.36739451,411.93749278)(685.56183881,412.36110357)
\curveto(685.76322755,412.78471436)(686.06531065,413.16318629)(686.46808812,413.49651937)
\curveto(686.85697672,413.82290802)(687.3500319,414.07985227)(687.94725367,414.26735212)
\curveto(688.55141988,414.46179642)(689.22503048,414.55901857)(689.96808548,414.55901857)
\curveto(690.6625294,414.55901857)(691.36391775,414.47221308)(692.07225055,414.2986021)
\curveto(692.78752779,414.13193556)(693.38127734,413.9270746)(693.8534992,413.68401923)
\lineto(693.8534992,411.59027082)
\lineto(693.74933262,411.59027082)
\curveto(693.24933299,411.95832609)(692.64169456,412.26735364)(691.92641733,412.51735345)
\curveto(691.21114009,412.7742977)(690.50975173,412.90276982)(689.82225225,412.90276982)
\curveto(689.10697502,412.90276982)(688.50280881,412.76388104)(688.00975362,412.48610347)
\curveto(687.51669844,412.21527034)(687.27017085,411.80902065)(687.27017085,411.26735439)
\curveto(687.27017085,410.78818809)(687.41947629,410.42707725)(687.71808718,410.18402188)
\curveto(688.00975362,409.94096651)(688.48197549,409.74304999)(689.13475277,409.59027233)
\curveto(689.49586361,409.50693906)(689.89864108,409.42360579)(690.34308519,409.34027252)
\curveto(690.79447374,409.25693925)(691.16947346,409.18055042)(691.46808434,409.11110603)
\curveto(692.37780588,408.90277285)(693.07919423,408.54513423)(693.57224942,408.03819017)
\curveto(694.0653046,407.52430167)(694.31183219,406.84374663)(694.31183219,405.99652505)
\closepath
}
}
{
\newrgbcolor{curcolor}{0 0 0}
\pscustom[linestyle=none,fillstyle=solid,fillcolor=curcolor]
{
\newpath
\moveto(706.8535029,408.25694)
\lineto(698.28059272,408.25694)
\curveto(698.28059272,407.54166277)(698.38823153,406.91666324)(698.60350915,406.38194142)
\curveto(698.81878676,405.85416404)(699.11392543,405.4201366)(699.48892514,405.07985907)
\curveto(699.85003598,404.74652599)(700.27711899,404.49652618)(700.77017417,404.32985964)
\curveto(701.27017379,404.1631931)(701.81878449,404.07985983)(702.41600626,404.07985983)
\curveto(703.20767233,404.07985983)(704.00281062,404.23610971)(704.80142112,404.54860948)
\curveto(705.60697607,404.86805368)(706.1798923,405.18055344)(706.52016982,405.48610877)
\lineto(706.62433641,405.48610877)
\lineto(706.62433641,403.35069372)
\curveto(705.96461469,403.07291615)(705.29100409,402.84027744)(704.60350461,402.65277758)
\curveto(703.91600513,402.46527772)(703.19378345,402.37152779)(702.43683958,402.37152779)
\curveto(700.50628548,402.37152779)(698.99934218,402.89236073)(697.91600967,403.93402661)
\curveto(696.83267715,404.98263693)(696.29101089,406.46874691)(696.29101089,408.39235657)
\curveto(696.29101089,410.29513291)(696.80837161,411.80554843)(697.84309305,412.92360314)
\curveto(698.88475893,414.04165785)(700.25281345,414.60068521)(701.94725662,414.60068521)
\curveto(703.51669987,414.60068521)(704.72503229,414.14235222)(705.57225387,413.22568625)
\curveto(706.42641989,412.30902027)(706.8535029,411.00693792)(706.8535029,409.3194392)
\closepath
\moveto(704.94725435,409.75693887)
\curveto(704.94030991,410.78471587)(704.67989344,411.57985416)(704.16600494,412.14235373)
\curveto(703.65906088,412.70485331)(702.88475591,412.98610309)(701.84309003,412.98610309)
\curveto(700.79447971,412.98610309)(699.95767479,412.67707555)(699.33267526,412.05902046)
\curveto(698.71462017,411.44096537)(698.36392599,410.67360484)(698.28059272,409.75693887)
\closepath
}
}
{
\newrgbcolor{curcolor}{0 0 0}
\pscustom[linestyle=none,fillstyle=solid,fillcolor=curcolor]
{
\newpath
\moveto(715.86391596,402.74652751)
\curveto(715.49586068,402.64930536)(715.09308321,402.56944431)(714.65558354,402.50694435)
\curveto(714.22502831,402.4444444)(713.83961194,402.41319443)(713.49933442,402.41319443)
\curveto(712.31183531,402.41319443)(711.40905822,402.73263863)(710.79100313,403.37152703)
\curveto(710.17294804,404.01041544)(709.8639205,405.03472022)(709.8639205,406.44444138)
\lineto(709.8639205,412.63193669)
\lineto(708.54100483,412.63193669)
\lineto(708.54100483,414.27776878)
\lineto(709.8639205,414.27776878)
\lineto(709.8639205,417.62151625)
\lineto(711.82225235,417.62151625)
\lineto(711.82225235,414.27776878)
\lineto(715.86391596,414.27776878)
\lineto(715.86391596,412.63193669)
\lineto(711.82225235,412.63193669)
\lineto(711.82225235,407.32985737)
\curveto(711.82225235,406.71874672)(711.83614123,406.23958042)(711.86391899,405.89235846)
\curveto(711.89169674,405.55208094)(711.98891889,405.23263674)(712.15558543,404.93402585)
\curveto(712.30836309,404.65624828)(712.51669627,404.45138733)(712.78058496,404.31944298)
\curveto(713.05141809,404.19444308)(713.46114,404.13194313)(714.0097507,404.13194313)
\curveto(714.3291949,404.13194313)(714.66252798,404.17708198)(715.00974994,404.26735969)
\curveto(715.3569719,404.36458184)(715.60697171,404.44444289)(715.75974937,404.50694284)
\lineto(715.86391596,404.50694284)
\closepath
}
}
{
\newrgbcolor{curcolor}{0 0 0}
\pscustom[linestyle=none,fillstyle=solid,fillcolor=curcolor]
{
\newpath
\moveto(722.2909931,405.61110867)
\lineto(719.35349532,398.78819717)
\lineto(717.83266314,398.78819717)
\lineto(719.64516177,405.61110867)
\closepath
}
}
{
\newrgbcolor{curcolor}{0 0 0}
\pscustom[linestyle=none,fillstyle=solid,fillcolor=curcolor]
{
\newpath
\moveto(462.02034594,376.07988102)
\curveto(461.65229066,375.98265887)(461.24951319,375.90279782)(460.81201352,375.84029786)
\curveto(460.38145829,375.77779791)(459.99604192,375.74654793)(459.6557644,375.74654793)
\curveto(458.4682653,375.74654793)(457.5654882,376.06599214)(456.94743311,376.70488054)
\curveto(456.32937802,377.34376895)(456.02035048,378.36807373)(456.02035048,379.77779488)
\lineto(456.02035048,385.9652902)
\lineto(454.69743482,385.9652902)
\lineto(454.69743482,387.61112229)
\lineto(456.02035048,387.61112229)
\lineto(456.02035048,390.95486976)
\lineto(457.97868233,390.95486976)
\lineto(457.97868233,387.61112229)
\lineto(462.02034594,387.61112229)
\lineto(462.02034594,385.9652902)
\lineto(457.97868233,385.9652902)
\lineto(457.97868233,380.66321088)
\curveto(457.97868233,380.05210023)(457.99257121,379.57293393)(458.02034897,379.22571197)
\curveto(458.04812672,378.88543445)(458.14534887,378.56599025)(458.31201541,378.26737936)
\curveto(458.46479308,377.98960179)(458.67312625,377.78474084)(458.93701494,377.65279649)
\curveto(459.20784807,377.52779659)(459.61756998,377.46529663)(460.16618068,377.46529663)
\curveto(460.48562488,377.46529663)(460.81895796,377.51043549)(461.16617992,377.6007132)
\curveto(461.51340188,377.69793535)(461.76340169,377.7777964)(461.91617935,377.84029635)
\lineto(462.02034594,377.84029635)
\closepath
}
}
{
\newrgbcolor{curcolor}{0 0 0}
\pscustom[linestyle=none,fillstyle=solid,fillcolor=curcolor]
{
\newpath
\moveto(474.12450573,381.59029351)
\lineto(465.55159554,381.59029351)
\curveto(465.55159554,380.87501628)(465.65923435,380.25001675)(465.87451197,379.71529493)
\curveto(466.08978958,379.18751755)(466.38492825,378.7534901)(466.75992796,378.41321258)
\curveto(467.1210388,378.0798795)(467.54812181,377.82987969)(468.04117699,377.66321315)
\curveto(468.54117662,377.49654661)(469.08978731,377.41321334)(469.68700908,377.41321334)
\curveto(470.47867515,377.41321334)(471.27381344,377.56946322)(472.07242394,377.88196299)
\curveto(472.87797889,378.20140719)(473.45089512,378.51390695)(473.79117264,378.81946228)
\lineto(473.89533923,378.81946228)
\lineto(473.89533923,376.68404722)
\curveto(473.23561751,376.40626966)(472.56200691,376.17363094)(471.87450743,375.98613109)
\curveto(471.18700795,375.79863123)(470.46478627,375.7048813)(469.7078424,375.7048813)
\curveto(467.77728831,375.7048813)(466.270345,376.22571424)(465.18701249,377.26738012)
\curveto(464.10367997,378.31599043)(463.56201372,379.80210042)(463.56201372,381.72571008)
\curveto(463.56201372,383.62848642)(464.07937444,385.13890194)(465.11409588,386.25695665)
\curveto(466.15576175,387.37501136)(467.52381627,387.93403871)(469.21825944,387.93403871)
\curveto(470.78770269,387.93403871)(471.99603511,387.47570573)(472.84325669,386.55903975)
\curveto(473.69742272,385.64237378)(474.12450573,384.34029143)(474.12450573,382.65279271)
\closepath
\moveto(472.21825717,383.09029238)
\curveto(472.21131273,384.11806938)(471.95089626,384.91320767)(471.43700776,385.47570724)
\curveto(470.9300637,386.03820681)(470.15575873,386.3194566)(469.11409285,386.3194566)
\curveto(468.06548253,386.3194566)(467.22867761,386.01042906)(466.60367808,385.39237397)
\curveto(465.98562299,384.77431888)(465.63492882,384.00695835)(465.55159554,383.09029238)
\closepath
}
}
{
\newrgbcolor{curcolor}{0 0 0}
\pscustom[linestyle=none,fillstyle=solid,fillcolor=curcolor]
{
\newpath
\moveto(485.40574319,379.32987856)
\curveto(485.40574319,378.26737936)(484.9647713,377.39585224)(484.08282752,376.7152972)
\curveto(483.20782818,376.03474216)(482.00991242,375.69446464)(480.48908024,375.69446464)
\curveto(479.62796978,375.69446464)(478.83630371,375.79515901)(478.11408204,375.99654774)
\curveto(477.3988048,376.20488092)(476.79811081,376.43057519)(476.31200007,376.67363057)
\lineto(476.31200007,378.87154557)
\lineto(476.41616666,378.87154557)
\curveto(477.03422174,378.40626814)(477.72172122,378.03474065)(478.4786651,377.75696308)
\curveto(479.23560897,377.48612995)(479.96130286,377.35071339)(480.65574678,377.35071339)
\curveto(481.51685724,377.35071339)(482.19046784,377.48960217)(482.67657859,377.76737974)
\curveto(483.16268933,378.04515731)(483.4057447,378.48265698)(483.4057447,379.07987875)
\curveto(483.4057447,379.53821173)(483.27380036,379.88543369)(483.00991167,380.12154462)
\curveto(482.74602298,380.35765556)(482.23907892,380.55904429)(481.48907948,380.72571083)
\curveto(481.21130192,380.78821079)(480.84671886,380.8611274)(480.39533031,380.94446067)
\curveto(479.9508862,381.02779394)(479.54463651,381.11807165)(479.17658123,381.2152938)
\curveto(478.15574867,381.48612692)(477.43005478,381.88195996)(476.99949955,382.4027929)
\curveto(476.57588876,382.93057028)(476.36408336,383.57640312)(476.36408336,384.34029143)
\curveto(476.36408336,384.81945774)(476.46130551,385.27084628)(476.65574981,385.69445707)
\curveto(476.85713854,386.11806787)(477.15922165,386.4965398)(477.56199912,386.82987288)
\curveto(477.95088772,387.15626152)(478.4439429,387.41320577)(479.04116467,387.60070563)
\curveto(479.64533088,387.79514993)(480.31894148,387.89237208)(481.06199647,387.89237208)
\curveto(481.75644039,387.89237208)(482.45782875,387.80556659)(483.16616155,387.63195561)
\curveto(483.88143879,387.46528907)(484.47518834,387.26042811)(484.9474102,387.01737274)
\lineto(484.9474102,384.92362432)
\lineto(484.84324361,384.92362432)
\curveto(484.34324399,385.2916796)(483.73560556,385.60070715)(483.02032833,385.85070696)
\curveto(482.30505109,386.10765121)(481.60366273,386.23612333)(480.91616325,386.23612333)
\curveto(480.20088602,386.23612333)(479.59671981,386.09723455)(479.10366462,385.81945698)
\curveto(478.61060944,385.54862385)(478.36408185,385.14237416)(478.36408185,384.6007079)
\curveto(478.36408185,384.1215416)(478.51338729,383.76043076)(478.81199818,383.51737539)
\curveto(479.10366462,383.27432002)(479.57588649,383.0764035)(480.22866377,382.92362584)
\curveto(480.58977461,382.84029257)(480.99255208,382.7569593)(481.43699619,382.67362603)
\curveto(481.88838474,382.59029276)(482.26338445,382.51390393)(482.56199534,382.44445953)
\curveto(483.47171687,382.23612636)(484.17310523,381.87848774)(484.66616041,381.37154368)
\curveto(485.1592156,380.85765518)(485.40574319,380.17710014)(485.40574319,379.32987856)
\closepath
}
}
{
\newrgbcolor{curcolor}{0 0 0}
\pscustom[linestyle=none,fillstyle=solid,fillcolor=curcolor]
{
\newpath
\moveto(494.24948562,376.07988102)
\curveto(493.88143034,375.98265887)(493.47865286,375.90279782)(493.0411532,375.84029786)
\curveto(492.61059797,375.77779791)(492.22518159,375.74654793)(491.88490407,375.74654793)
\curveto(490.69740497,375.74654793)(489.79462787,376.06599214)(489.17657279,376.70488054)
\curveto(488.5585177,377.34376895)(488.24949015,378.36807373)(488.24949015,379.77779488)
\lineto(488.24949015,385.9652902)
\lineto(486.92657449,385.9652902)
\lineto(486.92657449,387.61112229)
\lineto(488.24949015,387.61112229)
\lineto(488.24949015,390.95486976)
\lineto(490.20782201,390.95486976)
\lineto(490.20782201,387.61112229)
\lineto(494.24948562,387.61112229)
\lineto(494.24948562,385.9652902)
\lineto(490.20782201,385.9652902)
\lineto(490.20782201,380.66321088)
\curveto(490.20782201,380.05210023)(490.22171088,379.57293393)(490.24948864,379.22571197)
\curveto(490.2772664,378.88543445)(490.37448855,378.56599025)(490.54115509,378.26737936)
\curveto(490.69393275,377.98960179)(490.90226593,377.78474084)(491.16615461,377.65279649)
\curveto(491.43698774,377.52779659)(491.84670966,377.46529663)(492.39532035,377.46529663)
\curveto(492.71476455,377.46529663)(493.04809764,377.51043549)(493.39531959,377.6007132)
\curveto(493.74254155,377.69793535)(493.99254136,377.7777964)(494.14531903,377.84029635)
\lineto(494.24948562,377.84029635)
\closepath
}
}
{
\newrgbcolor{curcolor}{0 0 0}
\pscustom[linestyle=none,fillstyle=solid,fillcolor=curcolor]
{
\newpath
\moveto(513.84322514,375.97571443)
\lineto(511.88489328,375.97571443)
\lineto(511.88489328,382.60070942)
\curveto(511.88489328,383.13543123)(511.85364331,383.63543085)(511.79114335,384.10070828)
\curveto(511.7286434,384.57293015)(511.61406016,384.94098542)(511.44739362,385.20487411)
\curveto(511.27378264,385.49654056)(511.02378282,385.71181817)(510.69739418,385.85070696)
\curveto(510.37100554,385.99654018)(509.94739475,386.06945679)(509.42656181,386.06945679)
\curveto(508.89183999,386.06945679)(508.33281264,385.93751245)(507.74947975,385.67362376)
\curveto(507.16614685,385.40973507)(506.6071195,385.07292977)(506.07239768,384.66320785)
\lineto(506.07239768,375.97571443)
\lineto(504.11406583,375.97571443)
\lineto(504.11406583,387.61112229)
\lineto(506.07239768,387.61112229)
\lineto(506.07239768,386.3194566)
\curveto(506.68350833,386.82640066)(507.3154523,387.2222337)(507.96822958,387.5069557)
\curveto(508.62100686,387.79167771)(509.29114525,387.93403871)(509.97864473,387.93403871)
\curveto(511.23558822,387.93403871)(512.19392083,387.55556678)(512.85364255,386.79862291)
\curveto(513.51336427,386.04167903)(513.84322514,384.95140208)(513.84322514,383.52779205)
\closepath
}
}
{
\newrgbcolor{curcolor}{0 0 0}
\pscustom[linestyle=none,fillstyle=solid,fillcolor=curcolor]
{
\newpath
\moveto(527.35362594,381.59029351)
\lineto(518.78071575,381.59029351)
\curveto(518.78071575,380.87501628)(518.88835456,380.25001675)(519.10363218,379.71529493)
\curveto(519.31890979,379.18751755)(519.61404846,378.7534901)(519.98904817,378.41321258)
\curveto(520.35015901,378.0798795)(520.77724202,377.82987969)(521.2702972,377.66321315)
\curveto(521.77029683,377.49654661)(522.31890752,377.41321334)(522.91612929,377.41321334)
\curveto(523.70779536,377.41321334)(524.50293365,377.56946322)(525.30154415,377.88196299)
\curveto(526.1070991,378.20140719)(526.68001533,378.51390695)(527.02029285,378.81946228)
\lineto(527.12445944,378.81946228)
\lineto(527.12445944,376.68404722)
\curveto(526.46473772,376.40626966)(525.79112712,376.17363094)(525.10362764,375.98613109)
\curveto(524.41612816,375.79863123)(523.69390648,375.7048813)(522.93696261,375.7048813)
\curveto(521.00640851,375.7048813)(519.49946521,376.22571424)(518.4161327,377.26738012)
\curveto(517.33280018,378.31599043)(516.79113393,379.80210042)(516.79113393,381.72571008)
\curveto(516.79113393,383.62848642)(517.30849465,385.13890194)(518.34321609,386.25695665)
\curveto(519.38488196,387.37501136)(520.75293648,387.93403871)(522.44737965,387.93403871)
\curveto(524.0168229,387.93403871)(525.22515532,387.47570573)(526.0723769,386.55903975)
\curveto(526.92654292,385.64237378)(527.35362594,384.34029143)(527.35362594,382.65279271)
\closepath
\moveto(525.44737738,383.09029238)
\curveto(525.44043294,384.11806938)(525.18001647,384.91320767)(524.66612797,385.47570724)
\curveto(524.15918391,386.03820681)(523.38487894,386.3194566)(522.34321306,386.3194566)
\curveto(521.29460274,386.3194566)(520.45779782,386.01042906)(519.83279829,385.39237397)
\curveto(519.2147432,384.77431888)(518.86404902,384.00695835)(518.78071575,383.09029238)
\closepath
}
}
{
\newrgbcolor{curcolor}{0 0 0}
\pscustom[linestyle=none,fillstyle=solid,fillcolor=curcolor]
{
\newpath
\moveto(540.06195286,381.59029351)
\lineto(531.48904268,381.59029351)
\curveto(531.48904268,380.87501628)(531.59668149,380.25001675)(531.8119591,379.71529493)
\curveto(532.02723672,379.18751755)(532.32237538,378.7534901)(532.6973751,378.41321258)
\curveto(533.05848594,378.0798795)(533.48556895,377.82987969)(533.97862413,377.66321315)
\curveto(534.47862375,377.49654661)(535.02723445,377.41321334)(535.62445622,377.41321334)
\curveto(536.41612229,377.41321334)(537.21126057,377.56946322)(538.00987108,377.88196299)
\curveto(538.81542603,378.20140719)(539.38834226,378.51390695)(539.72861978,378.81946228)
\lineto(539.83278637,378.81946228)
\lineto(539.83278637,376.68404722)
\curveto(539.17306464,376.40626966)(538.49945404,376.17363094)(537.81195456,375.98613109)
\curveto(537.12445508,375.79863123)(536.40223341,375.7048813)(535.64528954,375.7048813)
\curveto(533.71473544,375.7048813)(532.20779214,376.22571424)(531.12445962,377.26738012)
\curveto(530.04112711,378.31599043)(529.49946085,379.80210042)(529.49946085,381.72571008)
\curveto(529.49946085,383.62848642)(530.01682157,385.13890194)(531.05154301,386.25695665)
\curveto(532.09320889,387.37501136)(533.46126341,387.93403871)(535.15570657,387.93403871)
\curveto(536.72514983,387.93403871)(537.93348225,387.47570573)(538.78070383,386.55903975)
\curveto(539.63486985,385.64237378)(540.06195286,384.34029143)(540.06195286,382.65279271)
\closepath
\moveto(538.1557043,383.09029238)
\curveto(538.14875986,384.11806938)(537.88834339,384.91320767)(537.37445489,385.47570724)
\curveto(536.86751083,386.03820681)(536.09320586,386.3194566)(535.05153998,386.3194566)
\curveto(534.00292967,386.3194566)(533.16612474,386.01042906)(532.54112522,385.39237397)
\curveto(531.92307013,384.77431888)(531.57237595,384.00695835)(531.48904268,383.09029238)
\closepath
}
}
{
\newrgbcolor{curcolor}{0 0 0}
\pscustom[linestyle=none,fillstyle=solid,fillcolor=curcolor]
{
\newpath
\moveto(552.46819668,375.97571443)
\lineto(550.50986483,375.97571443)
\lineto(550.50986483,377.19446351)
\curveto(549.94736525,376.70835276)(549.36056014,376.32988083)(548.74944949,376.0590477)
\curveto(548.13833885,375.78821457)(547.4751449,375.652798)(546.75986767,375.652798)
\curveto(545.37097983,375.652798)(544.266814,376.18751982)(543.44737017,377.25696346)
\curveto(542.63487079,378.32640709)(542.22862109,379.80904486)(542.22862109,381.70487676)
\curveto(542.22862109,382.69098712)(542.36750988,383.56945868)(542.64528745,384.34029143)
\curveto(542.93000945,385.11112418)(543.31195361,385.76737369)(543.79111991,386.30903994)
\curveto(544.26334178,386.83681732)(544.81195247,387.23959479)(545.436952,387.51737236)
\curveto(546.06889597,387.79514993)(546.72167325,387.93403871)(547.39528385,387.93403871)
\curveto(548.0063945,387.93403871)(548.54806076,387.86806654)(549.02028262,387.7361222)
\curveto(549.49250449,387.61112229)(549.98903189,387.41320577)(550.50986483,387.14237265)
\lineto(550.50986483,392.1840355)
\lineto(552.46819668,392.1840355)
\closepath
\moveto(550.50986483,378.84029559)
\lineto(550.50986483,385.51737388)
\curveto(549.98208745,385.75348481)(549.50986559,385.91667913)(549.09319923,386.00695684)
\curveto(548.67653288,386.09723455)(548.22167212,386.1423734)(547.72861693,386.1423734)
\curveto(546.63139554,386.1423734)(545.77722952,385.76042925)(545.16611887,384.99654094)
\curveto(544.55500822,384.23265262)(544.2494529,383.14932011)(544.2494529,381.74654339)
\curveto(544.2494529,380.3646)(544.48556383,379.31251746)(544.9577857,378.59029578)
\curveto(545.43000756,377.87501855)(546.18695143,377.51737993)(547.22861731,377.51737993)
\curveto(547.78417245,377.51737993)(548.34667202,377.63890761)(548.91611604,377.88196299)
\curveto(549.48556005,378.1319628)(550.01680965,378.451407)(550.50986483,378.84029559)
\closepath
}
}
{
\newrgbcolor{curcolor}{0 0 0}
\pscustom[linestyle=none,fillstyle=solid,fillcolor=curcolor]
{
\newpath
\moveto(564.63485234,379.32987856)
\curveto(564.63485234,378.26737936)(564.19388046,377.39585224)(563.31193668,376.7152972)
\curveto(562.43693734,376.03474216)(561.23902158,375.69446464)(559.7181894,375.69446464)
\curveto(558.85707894,375.69446464)(558.06541287,375.79515901)(557.34319119,375.99654774)
\curveto(556.62791396,376.20488092)(556.02721997,376.43057519)(555.54110922,376.67363057)
\lineto(555.54110922,378.87154557)
\lineto(555.64527581,378.87154557)
\curveto(556.2633309,378.40626814)(556.95083038,378.03474065)(557.70777425,377.75696308)
\curveto(558.46471812,377.48612995)(559.19041202,377.35071339)(559.88485594,377.35071339)
\curveto(560.7459664,377.35071339)(561.419577,377.48960217)(561.90568774,377.76737974)
\curveto(562.39179849,378.04515731)(562.63485386,378.48265698)(562.63485386,379.07987875)
\curveto(562.63485386,379.53821173)(562.50290951,379.88543369)(562.23902082,380.12154462)
\curveto(561.97513213,380.35765556)(561.46818807,380.55904429)(560.71818864,380.72571083)
\curveto(560.44041107,380.78821079)(560.07582802,380.8611274)(559.62443947,380.94446067)
\curveto(559.17999536,381.02779394)(558.77374567,381.11807165)(558.40569039,381.2152938)
\curveto(557.38485783,381.48612692)(556.65916393,381.88195996)(556.2286087,382.4027929)
\curveto(555.80499791,382.93057028)(555.59319252,383.57640312)(555.59319252,384.34029143)
\curveto(555.59319252,384.81945774)(555.69041467,385.27084628)(555.88485896,385.69445707)
\curveto(556.0862477,386.11806787)(556.38833081,386.4965398)(556.79110828,386.82987288)
\curveto(557.17999687,387.15626152)(557.67305206,387.41320577)(558.27027383,387.60070563)
\curveto(558.87444004,387.79514993)(559.54805064,387.89237208)(560.29110563,387.89237208)
\curveto(560.98554955,387.89237208)(561.68693791,387.80556659)(562.39527071,387.63195561)
\curveto(563.11054794,387.46528907)(563.70429749,387.26042811)(564.17651936,387.01737274)
\lineto(564.17651936,384.92362432)
\lineto(564.07235277,384.92362432)
\curveto(563.57235315,385.2916796)(562.96471472,385.60070715)(562.24943748,385.85070696)
\curveto(561.53416025,386.10765121)(560.83277189,386.23612333)(560.14527241,386.23612333)
\curveto(559.42999517,386.23612333)(558.82582896,386.09723455)(558.33277378,385.81945698)
\curveto(557.8397186,385.54862385)(557.59319101,385.14237416)(557.59319101,384.6007079)
\curveto(557.59319101,384.1215416)(557.74249645,383.76043076)(558.04110733,383.51737539)
\curveto(558.33277378,383.27432002)(558.80499564,383.0764035)(559.45777293,382.92362584)
\curveto(559.81888377,382.84029257)(560.22166124,382.7569593)(560.66610535,382.67362603)
\curveto(561.11749389,382.59029276)(561.49249361,382.51390393)(561.7911045,382.44445953)
\curveto(562.70082603,382.23612636)(563.40221439,381.87848774)(563.89526957,381.37154368)
\curveto(564.38832475,380.85765518)(564.63485234,380.17710014)(564.63485234,379.32987856)
\closepath
}
}
{
\newrgbcolor{curcolor}{0 0 0}
\pscustom[linestyle=none,fillstyle=solid,fillcolor=curcolor]
{
\newpath
\moveto(580.97858035,376.07988102)
\curveto(580.61052508,375.98265887)(580.2077476,375.90279782)(579.77024793,375.84029786)
\curveto(579.3396927,375.77779791)(578.95427633,375.74654793)(578.61399881,375.74654793)
\curveto(577.42649971,375.74654793)(576.52372261,376.06599214)(575.90566752,376.70488054)
\curveto(575.28761244,377.34376895)(574.97858489,378.36807373)(574.97858489,379.77779488)
\lineto(574.97858489,385.9652902)
\lineto(573.65566923,385.9652902)
\lineto(573.65566923,387.61112229)
\lineto(574.97858489,387.61112229)
\lineto(574.97858489,390.95486976)
\lineto(576.93691674,390.95486976)
\lineto(576.93691674,387.61112229)
\lineto(580.97858035,387.61112229)
\lineto(580.97858035,385.9652902)
\lineto(576.93691674,385.9652902)
\lineto(576.93691674,380.66321088)
\curveto(576.93691674,380.05210023)(576.95080562,379.57293393)(576.97858338,379.22571197)
\curveto(577.00636114,378.88543445)(577.10358328,378.56599025)(577.27024983,378.26737936)
\curveto(577.42302749,377.98960179)(577.63136066,377.78474084)(577.89524935,377.65279649)
\curveto(578.16608248,377.52779659)(578.57580439,377.46529663)(579.12441509,377.46529663)
\curveto(579.44385929,377.46529663)(579.77719237,377.51043549)(580.12441433,377.6007132)
\curveto(580.47163629,377.69793535)(580.7216361,377.7777964)(580.87441377,377.84029635)
\lineto(580.97858035,377.84029635)
\closepath
}
}
{
\newrgbcolor{curcolor}{0 0 0}
\pscustom[linestyle=none,fillstyle=solid,fillcolor=curcolor]
{
\newpath
\moveto(593.25983415,381.78821003)
\curveto(593.25983415,379.89237813)(592.77372341,378.39585149)(591.80150192,377.29863009)
\curveto(590.82928043,376.2014087)(589.52719808,375.652798)(587.89525488,375.652798)
\curveto(586.24942279,375.652798)(584.940396,376.2014087)(583.96817451,377.29863009)
\curveto(583.00289747,378.39585149)(582.52025894,379.89237813)(582.52025894,381.78821003)
\curveto(582.52025894,383.68404193)(583.00289747,385.18056857)(583.96817451,386.27778997)
\curveto(584.940396,387.3819558)(586.24942279,387.93403871)(587.89525488,387.93403871)
\curveto(589.52719808,387.93403871)(590.82928043,387.3819558)(591.80150192,386.27778997)
\curveto(592.77372341,385.18056857)(593.25983415,383.68404193)(593.25983415,381.78821003)
\closepath
\moveto(591.23900235,381.78821003)
\curveto(591.23900235,383.29515333)(590.94386368,384.41320804)(590.35358635,385.14237416)
\curveto(589.76330902,385.87848471)(588.94386519,386.24653999)(587.89525488,386.24653999)
\curveto(586.83275568,386.24653999)(586.00636742,385.87848471)(585.41609008,385.14237416)
\curveto(584.83275719,384.41320804)(584.54109075,383.29515333)(584.54109075,381.78821003)
\curveto(584.54109075,380.3298778)(584.83622941,379.22223975)(585.42650674,378.46529588)
\curveto(586.01678407,377.71529644)(586.83970012,377.34029673)(587.89525488,377.34029673)
\curveto(588.93692075,377.34029673)(589.75289236,377.71182422)(590.34316969,378.45487922)
\curveto(590.94039146,379.20487865)(591.23900235,380.31598892)(591.23900235,381.78821003)
\closepath
}
}
{
\newrgbcolor{curcolor}{0 0 0}
\pscustom[linestyle=none,fillstyle=solid,fillcolor=curcolor]
{
\newpath
\moveto(614.03064289,381.88195996)
\curveto(614.03064289,380.90973847)(613.89175411,380.03473913)(613.61397654,379.25696194)
\curveto(613.34314341,378.47918476)(612.97508813,377.82640747)(612.50981071,377.29863009)
\curveto(612.01675552,376.7500194)(611.47508927,376.33682527)(610.88481194,376.0590477)
\curveto(610.2945346,375.78821457)(609.64522954,375.652798)(608.93689674,375.652798)
\curveto(608.27717502,375.652798)(607.70078657,375.73265906)(607.20773138,375.89238116)
\curveto(606.7146762,376.04515882)(606.22856546,376.25349199)(605.74939915,376.51738068)
\lineto(605.62439925,375.97571443)
\lineto(603.7910673,375.97571443)
\lineto(603.7910673,392.1840355)
\lineto(605.74939915,392.1840355)
\lineto(605.74939915,386.39237321)
\curveto(606.29800985,386.84376176)(606.88134274,387.21181704)(607.49939783,387.49653904)
\curveto(608.11745292,387.78820549)(608.81189684,387.93403871)(609.58272959,387.93403871)
\curveto(610.95772855,387.93403871)(612.04106106,387.40626133)(612.83272713,386.35070658)
\curveto(613.63133764,385.29515182)(614.03064289,383.80556961)(614.03064289,381.88195996)
\closepath
\moveto(612.00981108,381.82987666)
\curveto(612.00981108,383.2187645)(611.78064459,384.27084704)(611.3223116,384.98612428)
\curveto(610.86397862,385.70834595)(610.12439584,386.06945679)(609.10356328,386.06945679)
\curveto(608.53411927,386.06945679)(607.95773082,385.94445689)(607.37439792,385.69445707)
\curveto(606.79106503,385.4514017)(606.24939878,385.13542972)(605.74939915,384.74654112)
\lineto(605.74939915,378.0798795)
\curveto(606.30495429,377.82987969)(606.78064837,377.65626871)(607.17648141,377.55904656)
\curveto(607.57925888,377.46182441)(608.03411965,377.41321334)(608.54106371,377.41321334)
\curveto(609.62439622,377.41321334)(610.4716178,377.76737974)(611.08272845,378.47571254)
\curveto(611.70078354,379.19098977)(612.00981108,380.30904448)(612.00981108,381.82987666)
\closepath
}
}
{
\newrgbcolor{curcolor}{0 0 0}
\pscustom[linestyle=none,fillstyle=solid,fillcolor=curcolor]
{
\newpath
\moveto(626.82229808,381.59029351)
\lineto(618.2493879,381.59029351)
\curveto(618.2493879,380.87501628)(618.35702671,380.25001675)(618.57230432,379.71529493)
\curveto(618.78758194,379.18751755)(619.0827206,378.7534901)(619.45772032,378.41321258)
\curveto(619.81883116,378.0798795)(620.24591417,377.82987969)(620.73896935,377.66321315)
\curveto(621.23896897,377.49654661)(621.78757967,377.41321334)(622.38480144,377.41321334)
\curveto(623.1764675,377.41321334)(623.97160579,377.56946322)(624.7702163,377.88196299)
\curveto(625.57577124,378.20140719)(626.14868748,378.51390695)(626.488965,378.81946228)
\lineto(626.59313159,378.81946228)
\lineto(626.59313159,376.68404722)
\curveto(625.93340986,376.40626966)(625.25979926,376.17363094)(624.57229978,375.98613109)
\curveto(623.8848003,375.79863123)(623.16257863,375.7048813)(622.40563475,375.7048813)
\curveto(620.47508066,375.7048813)(618.96813735,376.22571424)(617.88480484,377.26738012)
\curveto(616.80147233,378.31599043)(616.25980607,379.80210042)(616.25980607,381.72571008)
\curveto(616.25980607,383.62848642)(616.77716679,385.13890194)(617.81188823,386.25695665)
\curveto(618.85355411,387.37501136)(620.22160863,387.93403871)(621.91605179,387.93403871)
\curveto(623.48549505,387.93403871)(624.69382747,387.47570573)(625.54104905,386.55903975)
\curveto(626.39521507,385.64237378)(626.82229808,384.34029143)(626.82229808,382.65279271)
\closepath
\moveto(624.91604952,383.09029238)
\curveto(624.90910508,384.11806938)(624.64868861,384.91320767)(624.13480011,385.47570724)
\curveto(623.62785605,386.03820681)(622.85355108,386.3194566)(621.8118852,386.3194566)
\curveto(620.76327489,386.3194566)(619.92646996,386.01042906)(619.30147044,385.39237397)
\curveto(618.68341535,384.77431888)(618.33272117,384.00695835)(618.2493879,383.09029238)
\closepath
}
}
{
\newrgbcolor{curcolor}{0 0 0}
\pscustom[linestyle=none,fillstyle=solid,fillcolor=curcolor]
{
\newpath
\moveto(644.55146021,385.47570724)
\lineto(644.44729363,385.47570724)
\curveto(644.15562718,385.54515163)(643.87090517,385.59376271)(643.59312761,385.62154046)
\curveto(643.32229448,385.65626266)(642.99937805,385.67362376)(642.62437834,385.67362376)
\curveto(642.02021213,385.67362376)(641.43687924,385.53820719)(640.87437966,385.26737406)
\curveto(640.31188009,385.00348537)(639.77021383,384.65973564)(639.24938089,384.23612484)
\lineto(639.24938089,375.97571443)
\lineto(637.29104904,375.97571443)
\lineto(637.29104904,387.61112229)
\lineto(639.24938089,387.61112229)
\lineto(639.24938089,385.89237359)
\curveto(640.02715808,386.51737312)(640.71118534,386.95834501)(641.30146267,387.21528926)
\curveto(641.89868444,387.47917795)(642.50632287,387.61112229)(643.12437796,387.61112229)
\curveto(643.46465548,387.61112229)(643.71118307,387.60070563)(643.86396073,387.57987231)
\curveto(644.0167384,387.56598344)(644.24590489,387.53473346)(644.55146021,387.48612239)
\closepath
}
}
{
\newrgbcolor{curcolor}{0 0 0}
\pscustom[linestyle=none,fillstyle=solid,fillcolor=curcolor]
{
\newpath
\moveto(656.13478751,381.59029351)
\lineto(647.56187733,381.59029351)
\curveto(647.56187733,380.87501628)(647.66951613,380.25001675)(647.88479375,379.71529493)
\curveto(648.10007136,379.18751755)(648.39521003,378.7534901)(648.77020974,378.41321258)
\curveto(649.13132058,378.0798795)(649.55840359,377.82987969)(650.05145878,377.66321315)
\curveto(650.5514584,377.49654661)(651.10006909,377.41321334)(651.69729086,377.41321334)
\curveto(652.48895693,377.41321334)(653.28409522,377.56946322)(654.08270573,377.88196299)
\curveto(654.88826067,378.20140719)(655.4611769,378.51390695)(655.80145442,378.81946228)
\lineto(655.90562101,378.81946228)
\lineto(655.90562101,376.68404722)
\curveto(655.24589929,376.40626966)(654.57228869,376.17363094)(653.88478921,375.98613109)
\curveto(653.19728973,375.79863123)(652.47506805,375.7048813)(651.71812418,375.7048813)
\curveto(649.78757009,375.7048813)(648.28062678,376.22571424)(647.19729427,377.26738012)
\curveto(646.11396175,378.31599043)(645.5722955,379.80210042)(645.5722955,381.72571008)
\curveto(645.5722955,383.62848642)(646.08965622,385.13890194)(647.12437766,386.25695665)
\curveto(648.16604353,387.37501136)(649.53409806,387.93403871)(651.22854122,387.93403871)
\curveto(652.79798448,387.93403871)(654.00631689,387.47570573)(654.85353848,386.55903975)
\curveto(655.7077045,385.64237378)(656.13478751,384.34029143)(656.13478751,382.65279271)
\closepath
\moveto(654.22853895,383.09029238)
\curveto(654.22159451,384.11806938)(653.96117804,384.91320767)(653.44728954,385.47570724)
\curveto(652.94034548,386.03820681)(652.16604051,386.3194566)(651.12437463,386.3194566)
\curveto(650.07576431,386.3194566)(649.23895939,386.01042906)(648.61395986,385.39237397)
\curveto(647.99590477,384.77431888)(647.6452106,384.00695835)(647.56187733,383.09029238)
\closepath
}
}
{
\newrgbcolor{curcolor}{0 0 0}
\pscustom[linestyle=none,fillstyle=solid,fillcolor=curcolor]
{
\newpath
\moveto(669.34311405,381.93404325)
\curveto(669.34311405,380.98959952)(669.20769749,380.12501684)(668.93686436,379.34029522)
\curveto(668.66603123,378.56251803)(668.28408708,377.9027963)(667.79103189,377.36113005)
\curveto(667.33269891,376.84724155)(666.79103265,376.44793629)(666.16603312,376.16321429)
\curveto(665.54797804,375.88543672)(664.89172853,375.74654793)(664.19728461,375.74654793)
\curveto(663.5931184,375.74654793)(663.04450771,375.81252011)(662.55145252,375.94446445)
\curveto(662.06534178,376.0764088)(661.56881438,376.28126975)(661.06187032,376.55904732)
\lineto(661.06187032,371.68405101)
\lineto(659.10353847,371.68405101)
\lineto(659.10353847,387.61112229)
\lineto(661.06187032,387.61112229)
\lineto(661.06187032,386.39237321)
\curveto(661.58270326,386.82987288)(662.16603615,387.19445594)(662.81186899,387.48612239)
\curveto(663.46464628,387.78473327)(664.1590902,387.93403871)(664.89520075,387.93403871)
\curveto(666.29797747,387.93403871)(667.38825442,387.40278912)(668.16603161,386.34028992)
\curveto(668.95075324,385.28473516)(669.34311405,383.81598627)(669.34311405,381.93404325)
\closepath
\moveto(667.32228225,381.88195996)
\curveto(667.32228225,383.28473668)(667.0826991,384.33334699)(666.60353279,385.02779091)
\curveto(666.12436649,385.72223483)(665.38825593,386.06945679)(664.39520113,386.06945679)
\curveto(663.83270156,386.06945679)(663.26672976,385.9479291)(662.69728575,385.70487373)
\curveto(662.12784173,385.46181836)(661.58270326,385.14237416)(661.06187032,384.74654112)
\lineto(661.06187032,378.15279611)
\curveto(661.61742545,377.9027963)(662.09311954,377.73265754)(662.48895257,377.64237983)
\curveto(662.89173005,377.55210212)(663.34659081,377.50696327)(663.85353487,377.50696327)
\curveto(664.94381183,377.50696327)(665.79450563,377.87501855)(666.40561628,378.6111291)
\curveto(667.01672692,379.34723965)(667.32228225,380.43751661)(667.32228225,381.88195996)
\closepath
}
}
{
\newrgbcolor{curcolor}{0 0 0}
\pscustom[linestyle=none,fillstyle=solid,fillcolor=curcolor]
{
\newpath
\moveto(682.13475483,381.59029351)
\lineto(673.56184465,381.59029351)
\curveto(673.56184465,380.87501628)(673.66948345,380.25001675)(673.88476107,379.71529493)
\curveto(674.10003868,379.18751755)(674.39517735,378.7534901)(674.77017706,378.41321258)
\curveto(675.1312879,378.0798795)(675.55837091,377.82987969)(676.05142609,377.66321315)
\curveto(676.55142572,377.49654661)(677.10003641,377.41321334)(677.69725818,377.41321334)
\curveto(678.48892425,377.41321334)(679.28406254,377.56946322)(680.08267305,377.88196299)
\curveto(680.88822799,378.20140719)(681.46114422,378.51390695)(681.80142174,378.81946228)
\lineto(681.90558833,378.81946228)
\lineto(681.90558833,376.68404722)
\curveto(681.24586661,376.40626966)(680.57225601,376.17363094)(679.88475653,375.98613109)
\curveto(679.19725705,375.79863123)(678.47503537,375.7048813)(677.7180915,375.7048813)
\curveto(675.78753741,375.7048813)(674.2805941,376.22571424)(673.19726159,377.26738012)
\curveto(672.11392907,378.31599043)(671.57226282,379.80210042)(671.57226282,381.72571008)
\curveto(671.57226282,383.62848642)(672.08962354,385.13890194)(673.12434498,386.25695665)
\curveto(674.16601085,387.37501136)(675.53406538,387.93403871)(677.22850854,387.93403871)
\curveto(678.79795179,387.93403871)(680.00628421,387.47570573)(680.8535058,386.55903975)
\curveto(681.70767182,385.64237378)(682.13475483,384.34029143)(682.13475483,382.65279271)
\closepath
\moveto(680.22850627,383.09029238)
\curveto(680.22156183,384.11806938)(679.96114536,384.91320767)(679.44725686,385.47570724)
\curveto(678.9403128,386.03820681)(678.16600783,386.3194566)(677.12434195,386.3194566)
\curveto(676.07573163,386.3194566)(675.23892671,386.01042906)(674.61392718,385.39237397)
\curveto(673.99587209,384.77431888)(673.64517792,384.00695835)(673.56184465,383.09029238)
\closepath
}
}
{
\newrgbcolor{curcolor}{0 0 0}
\pscustom[linestyle=none,fillstyle=solid,fillcolor=curcolor]
{
\newpath
\moveto(694.14516561,375.97571443)
\lineto(692.19725042,375.97571443)
\lineto(692.19725042,377.21529682)
\curveto(692.02363944,377.09724136)(691.78752851,376.93057482)(691.48891762,376.7152972)
\curveto(691.19725118,376.50696403)(690.91252917,376.34029748)(690.6347516,376.21529758)
\curveto(690.30836296,376.05557548)(689.93336324,375.92363113)(689.50975245,375.81946455)
\curveto(689.08614166,375.70835352)(688.58961426,375.652798)(688.02017025,375.652798)
\curveto(686.97155993,375.652798)(686.08267171,376.00001996)(685.3535056,376.69446388)
\curveto(684.62433948,377.3889078)(684.25975642,378.2743238)(684.25975642,379.35071187)
\curveto(684.25975642,380.23265565)(684.44725628,380.94446067)(684.822256,381.48612692)
\curveto(685.20420015,382.03473762)(685.74586641,382.46529285)(686.44725477,382.77779261)
\curveto(687.15558757,383.09029238)(688.00628137,383.30209777)(688.99933617,383.4132088)
\curveto(689.99239098,383.52431983)(691.05836239,383.6076531)(692.19725042,383.66320861)
\lineto(692.19725042,383.96529172)
\curveto(692.19725042,384.40973582)(692.11738937,384.7777911)(691.95766727,385.06945755)
\curveto(691.80488961,385.36112399)(691.58266755,385.59029049)(691.29100111,385.75695703)
\curveto(691.01322354,385.91667913)(690.67989046,386.02431794)(690.29100186,386.07987345)
\curveto(689.90211327,386.13542896)(689.49586357,386.16320672)(689.07225278,386.16320672)
\curveto(688.55836428,386.16320672)(687.98544805,386.09376233)(687.35350408,385.95487354)
\curveto(686.72156012,385.8229292)(686.06878283,385.6284849)(685.39517223,385.37154065)
\lineto(685.29100564,385.37154065)
\lineto(685.29100564,387.36112248)
\curveto(685.6729498,387.46528907)(686.22503272,387.57987231)(686.94725439,387.70487222)
\curveto(687.66947607,387.82987213)(688.38128108,387.89237208)(689.08266944,387.89237208)
\curveto(689.90211327,387.89237208)(690.61391828,387.82292769)(691.21808449,387.6840389)
\curveto(691.82919514,387.55209456)(692.35697252,387.32292806)(692.80141663,386.99653942)
\curveto(693.2389163,386.67709522)(693.57224938,386.26390109)(693.80141587,385.75695703)
\curveto(694.03058237,385.25001297)(694.14516561,384.62154122)(694.14516561,383.87154179)
\closepath
\moveto(692.19725042,378.84029559)
\lineto(692.19725042,382.07987648)
\curveto(691.60002865,382.04515428)(690.89516807,381.99307099)(690.08266869,381.92362659)
\curveto(689.27711374,381.8541822)(688.63822533,381.75348783)(688.16600347,381.62154349)
\curveto(687.6035039,381.46182139)(687.14864313,381.21182158)(686.80142117,380.87154406)
\curveto(686.45419921,380.53821098)(686.28058823,380.07640577)(686.28058823,379.48612844)
\curveto(686.28058823,378.81946228)(686.48197697,378.31599043)(686.88475444,377.97571291)
\curveto(687.28753191,377.64237983)(687.90211478,377.47571329)(688.72850304,377.47571329)
\curveto(689.41600252,377.47571329)(690.04447427,377.60765764)(690.61391828,377.87154633)
\curveto(691.1833623,378.14237945)(691.71113968,378.46529588)(692.19725042,378.84029559)
\closepath
}
}
{
\newrgbcolor{curcolor}{0 0 0}
\pscustom[linestyle=none,fillstyle=solid,fillcolor=curcolor]
{
\newpath
\moveto(703.95764866,376.07988102)
\curveto(703.58959338,375.98265887)(703.18681591,375.90279782)(702.74931624,375.84029786)
\curveto(702.31876101,375.77779791)(701.93334463,375.74654793)(701.59306711,375.74654793)
\curveto(700.40556801,375.74654793)(699.50279092,376.06599214)(698.88473583,376.70488054)
\curveto(698.26668074,377.34376895)(697.9576532,378.36807373)(697.9576532,379.77779488)
\lineto(697.9576532,385.9652902)
\lineto(696.63473753,385.9652902)
\lineto(696.63473753,387.61112229)
\lineto(697.9576532,387.61112229)
\lineto(697.9576532,390.95486976)
\lineto(699.91598505,390.95486976)
\lineto(699.91598505,387.61112229)
\lineto(703.95764866,387.61112229)
\lineto(703.95764866,385.9652902)
\lineto(699.91598505,385.9652902)
\lineto(699.91598505,380.66321088)
\curveto(699.91598505,380.05210023)(699.92987393,379.57293393)(699.95765168,379.22571197)
\curveto(699.98542944,378.88543445)(700.08265159,378.56599025)(700.24931813,378.26737936)
\curveto(700.40209579,377.98960179)(700.61042897,377.78474084)(700.87431766,377.65279649)
\curveto(701.14515079,377.52779659)(701.5548727,377.46529663)(702.10348339,377.46529663)
\curveto(702.4229276,377.46529663)(702.75626068,377.51043549)(703.10348264,377.6007132)
\curveto(703.4507046,377.69793535)(703.70070441,377.7777964)(703.85348207,377.84029635)
\lineto(703.95764866,377.84029635)
\closepath
}
}
{
\newrgbcolor{curcolor}{0 0 0}
\pscustom[linestyle=none,fillstyle=solid,fillcolor=curcolor]
{
\newpath
\moveto(716.06180484,381.59029351)
\lineto(707.48889466,381.59029351)
\curveto(707.48889466,380.87501628)(707.59653346,380.25001675)(707.81181108,379.71529493)
\curveto(708.02708869,379.18751755)(708.32222736,378.7534901)(708.69722708,378.41321258)
\curveto(709.05833791,378.0798795)(709.48542092,377.82987969)(709.97847611,377.66321315)
\curveto(710.47847573,377.49654661)(711.02708643,377.41321334)(711.6243082,377.41321334)
\curveto(712.41597426,377.41321334)(713.21111255,377.56946322)(714.00972306,377.88196299)
\curveto(714.815278,378.20140719)(715.38819424,378.51390695)(715.72847176,378.81946228)
\lineto(715.83263834,378.81946228)
\lineto(715.83263834,376.68404722)
\curveto(715.17291662,376.40626966)(714.49930602,376.17363094)(713.81180654,375.98613109)
\curveto(713.12430706,375.79863123)(712.40208538,375.7048813)(711.64514151,375.7048813)
\curveto(709.71458742,375.7048813)(708.20764411,376.22571424)(707.1243116,377.26738012)
\curveto(706.04097909,378.31599043)(705.49931283,379.80210042)(705.49931283,381.72571008)
\curveto(705.49931283,383.62848642)(706.01667355,385.13890194)(707.05139499,386.25695665)
\curveto(708.09306087,387.37501136)(709.46111539,387.93403871)(711.15555855,387.93403871)
\curveto(712.72500181,387.93403871)(713.93333423,387.47570573)(714.78055581,386.55903975)
\curveto(715.63472183,385.64237378)(716.06180484,384.34029143)(716.06180484,382.65279271)
\closepath
\moveto(714.15555628,383.09029238)
\curveto(714.14861184,384.11806938)(713.88819537,384.91320767)(713.37430687,385.47570724)
\curveto(712.86736281,386.03820681)(712.09305784,386.3194566)(711.05139196,386.3194566)
\curveto(710.00278164,386.3194566)(709.16597672,386.01042906)(708.54097719,385.39237397)
\curveto(707.92292211,384.77431888)(707.57222793,384.00695835)(707.48889466,383.09029238)
\closepath
}
}
{
\newrgbcolor{curcolor}{0 0 0}
\pscustom[linestyle=none,fillstyle=solid,fillcolor=curcolor]
{
\newpath
\moveto(728.46804866,375.97571443)
\lineto(726.50971681,375.97571443)
\lineto(726.50971681,377.19446351)
\curveto(725.94721723,376.70835276)(725.36041212,376.32988083)(724.74930147,376.0590477)
\curveto(724.13819082,375.78821457)(723.47499688,375.652798)(722.75971964,375.652798)
\curveto(721.37083181,375.652798)(720.26666597,376.18751982)(719.44722215,377.25696346)
\curveto(718.63472276,378.32640709)(718.22847307,379.80904486)(718.22847307,381.70487676)
\curveto(718.22847307,382.69098712)(718.36736186,383.56945868)(718.64513942,384.34029143)
\curveto(718.92986143,385.11112418)(719.31180559,385.76737369)(719.79097189,386.30903994)
\curveto(720.26319376,386.83681732)(720.81180445,387.23959479)(721.43680398,387.51737236)
\curveto(722.06874794,387.79514993)(722.72152523,387.93403871)(723.39513583,387.93403871)
\curveto(724.00624648,387.93403871)(724.54791274,387.86806654)(725.0201346,387.7361222)
\curveto(725.49235647,387.61112229)(725.98888387,387.41320577)(726.50971681,387.14237265)
\lineto(726.50971681,392.1840355)
\lineto(728.46804866,392.1840355)
\closepath
\moveto(726.50971681,378.84029559)
\lineto(726.50971681,385.51737388)
\curveto(725.98193943,385.75348481)(725.50971756,385.91667913)(725.09305121,386.00695684)
\curveto(724.67638486,386.09723455)(724.22152409,386.1423734)(723.72846891,386.1423734)
\curveto(722.63124752,386.1423734)(721.7770815,385.76042925)(721.16597085,384.99654094)
\curveto(720.5548602,384.23265262)(720.24930488,383.14932011)(720.24930488,381.74654339)
\curveto(720.24930488,380.3646)(720.48541581,379.31251746)(720.95763767,378.59029578)
\curveto(721.42985954,377.87501855)(722.18680341,377.51737993)(723.22846929,377.51737993)
\curveto(723.78402442,377.51737993)(724.346524,377.63890761)(724.91596801,377.88196299)
\curveto(725.48541203,378.1319628)(726.01666162,378.451407)(726.50971681,378.84029559)
\closepath
}
}
{
\newrgbcolor{curcolor}{0 0 0}
\pscustom[linestyle=none,fillstyle=solid,fillcolor=curcolor]
{
\newpath
\moveto(470.61408944,360.9444758)
\lineto(467.58284173,349.30906794)
\lineto(465.7703431,349.30906794)
\lineto(462.78076203,358.27781115)
\lineto(459.81201428,349.30906794)
\lineto(458.00993231,349.30906794)
\lineto(454.94743463,360.9444758)
\lineto(456.98909975,360.9444758)
\lineto(459.1245148,351.93406595)
\lineto(462.0307626,360.9444758)
\lineto(463.64534471,360.9444758)
\lineto(466.62450912,351.93406595)
\lineto(468.64534093,360.9444758)
\closepath
}
}
{
\newrgbcolor{curcolor}{0 0 0}
\pscustom[linestyle=none,fillstyle=solid,fillcolor=curcolor]
{
\newpath
\moveto(475.54116599,362.89239099)
\lineto(473.33283433,362.89239099)
\lineto(473.33283433,364.92363946)
\lineto(475.54116599,364.92363946)
\closepath
\moveto(475.41616609,349.30906794)
\lineto(473.45783424,349.30906794)
\lineto(473.45783424,360.9444758)
\lineto(475.41616609,360.9444758)
\closepath
}
}
{
\newrgbcolor{curcolor}{0 0 0}
\pscustom[linestyle=none,fillstyle=solid,fillcolor=curcolor]
{
\newpath
\moveto(485.332826,349.41323452)
\curveto(484.96477072,349.31601237)(484.56199325,349.23615132)(484.12449358,349.17365137)
\curveto(483.69393835,349.11115142)(483.30852197,349.07990144)(482.96824445,349.07990144)
\curveto(481.78074535,349.07990144)(480.87796826,349.39934564)(480.25991317,350.03823405)
\curveto(479.64185808,350.67712246)(479.33283054,351.70142724)(479.33283054,353.11114839)
\lineto(479.33283054,359.29864371)
\lineto(478.00991487,359.29864371)
\lineto(478.00991487,360.9444758)
\lineto(479.33283054,360.9444758)
\lineto(479.33283054,364.28822327)
\lineto(481.29116239,364.28822327)
\lineto(481.29116239,360.9444758)
\lineto(485.332826,360.9444758)
\lineto(485.332826,359.29864371)
\lineto(481.29116239,359.29864371)
\lineto(481.29116239,353.99656439)
\curveto(481.29116239,353.38545374)(481.30505127,352.90628744)(481.33282902,352.55906548)
\curveto(481.36060678,352.21878796)(481.45782893,351.89934375)(481.62449547,351.60073287)
\curveto(481.77727313,351.3229553)(481.98560631,351.11809434)(482.249495,350.98615)
\curveto(482.52032812,350.86115009)(482.93005004,350.79865014)(483.47866073,350.79865014)
\curveto(483.79810494,350.79865014)(484.13143802,350.843789)(484.47865998,350.93406671)
\curveto(484.82588194,351.03128885)(485.07588175,351.11114991)(485.22865941,351.17364986)
\lineto(485.332826,351.17364986)
\closepath
}
}
{
\newrgbcolor{curcolor}{0 0 0}
\pscustom[linestyle=none,fillstyle=solid,fillcolor=curcolor]
{
\newpath
\moveto(497.42657273,349.30906794)
\lineto(495.46824087,349.30906794)
\lineto(495.46824087,355.93406292)
\curveto(495.46824087,356.46878474)(495.4369909,356.96878436)(495.37449095,357.43406179)
\curveto(495.31199099,357.90628365)(495.19740775,358.27433893)(495.03074121,358.53822762)
\curveto(494.85713023,358.82989407)(494.60713042,359.04517168)(494.28074177,359.18406046)
\curveto(493.95435313,359.32989369)(493.53074234,359.4028103)(493.0099094,359.4028103)
\curveto(492.47518758,359.4028103)(491.91616023,359.27086595)(491.33282734,359.00697727)
\curveto(490.74949444,358.74308858)(490.19046709,358.40628328)(489.65574527,357.99656136)
\lineto(489.65574527,349.30906794)
\lineto(487.69741342,349.30906794)
\lineto(487.69741342,365.51738901)
\lineto(489.65574527,365.51738901)
\lineto(489.65574527,359.65281011)
\curveto(490.26685592,360.15975417)(490.89879989,360.5555872)(491.55157717,360.84030921)
\curveto(492.20435446,361.12503122)(492.87449284,361.26739222)(493.56199232,361.26739222)
\curveto(494.81893581,361.26739222)(495.77726842,360.88892029)(496.43699014,360.13197641)
\curveto(497.09671187,359.37503254)(497.42657273,358.28475559)(497.42657273,356.86114556)
\closepath
}
}
{
\newrgbcolor{curcolor}{0 0 0}
\pscustom[linestyle=none,fillstyle=solid,fillcolor=curcolor]
{
\newpath
\moveto(518.61404952,355.12156354)
\curveto(518.61404952,353.22573164)(518.12793877,351.72920499)(517.15571729,350.6319836)
\curveto(516.1834958,349.53476221)(514.88141345,348.98615151)(513.24947024,348.98615151)
\curveto(511.60363815,348.98615151)(510.29461137,349.53476221)(509.32238988,350.6319836)
\curveto(508.35711283,351.72920499)(507.87447431,353.22573164)(507.87447431,355.12156354)
\curveto(507.87447431,357.01739544)(508.35711283,358.51392208)(509.32238988,359.61114347)
\curveto(510.29461137,360.71530931)(511.60363815,361.26739222)(513.24947024,361.26739222)
\curveto(514.88141345,361.26739222)(516.1834958,360.71530931)(517.15571729,359.61114347)
\curveto(518.12793877,358.51392208)(518.61404952,357.01739544)(518.61404952,355.12156354)
\closepath
\moveto(516.59321771,355.12156354)
\curveto(516.59321771,356.62850684)(516.29807905,357.74656155)(515.70780172,358.47572767)
\curveto(515.11752438,359.21183822)(514.29808056,359.5798935)(513.24947024,359.5798935)
\curveto(512.18697105,359.5798935)(511.36058278,359.21183822)(510.77030545,358.47572767)
\curveto(510.18697256,357.74656155)(509.89530611,356.62850684)(509.89530611,355.12156354)
\curveto(509.89530611,353.66323131)(510.19044478,352.55559326)(510.78072211,351.79864939)
\curveto(511.37099944,351.04864995)(512.19391548,350.67365024)(513.24947024,350.67365024)
\curveto(514.29113612,350.67365024)(515.10710773,351.04517773)(515.69738506,351.78823273)
\curveto(516.29460683,352.53823216)(516.59321771,353.64934243)(516.59321771,355.12156354)
\closepath
}
}
{
\newrgbcolor{curcolor}{0 0 0}
\pscustom[linestyle=none,fillstyle=solid,fillcolor=curcolor]
{
\newpath
\moveto(531.88487267,355.26739676)
\curveto(531.88487267,354.32295303)(531.74945611,353.45837035)(531.47862298,352.67364872)
\curveto(531.20778985,351.89587153)(530.8258457,351.23614981)(530.33279051,350.69448355)
\curveto(529.87445753,350.18059505)(529.33279127,349.7812898)(528.70779174,349.49656779)
\curveto(528.08973666,349.21879023)(527.43348715,349.07990144)(526.73904323,349.07990144)
\curveto(526.13487702,349.07990144)(525.58626633,349.14587361)(525.09321114,349.27781796)
\curveto(524.6071004,349.4097623)(524.110573,349.61462326)(523.60362894,349.89240083)
\lineto(523.60362894,345.01740452)
\lineto(521.64529709,345.01740452)
\lineto(521.64529709,360.9444758)
\lineto(523.60362894,360.9444758)
\lineto(523.60362894,359.72572672)
\curveto(524.12446188,360.16322639)(524.70779477,360.52780945)(525.35362761,360.81947589)
\curveto(526.0064049,361.11808678)(526.70084882,361.26739222)(527.43695937,361.26739222)
\curveto(528.83973609,361.26739222)(529.93001304,360.73614262)(530.70779023,359.67364343)
\curveto(531.49251186,358.61808867)(531.88487267,357.14933978)(531.88487267,355.26739676)
\closepath
\moveto(529.86404087,355.21531347)
\curveto(529.86404087,356.61809018)(529.62445772,357.6667005)(529.14529141,358.36114442)
\curveto(528.66612511,359.05558834)(527.93001455,359.4028103)(526.93695975,359.4028103)
\curveto(526.37446018,359.4028103)(525.80848838,359.28128261)(525.23904437,359.03822724)
\curveto(524.66960035,358.79517187)(524.12446188,358.47572767)(523.60362894,358.07989463)
\lineto(523.60362894,351.48614962)
\curveto(524.15918407,351.23614981)(524.63487816,351.06601105)(525.03071119,350.97573334)
\curveto(525.43348867,350.88545563)(525.88834943,350.84031678)(526.39529349,350.84031678)
\curveto(527.48557045,350.84031678)(528.33626425,351.20837205)(528.9473749,351.94448261)
\curveto(529.55848554,352.68059316)(529.86404087,353.77087012)(529.86404087,355.21531347)
\closepath
}
}
{
\newrgbcolor{curcolor}{0 0 0}
\pscustom[linestyle=none,fillstyle=solid,fillcolor=curcolor]
{
\newpath
\moveto(545.17652749,355.26739676)
\curveto(545.17652749,354.32295303)(545.04111092,353.45837035)(544.77027779,352.67364872)
\curveto(544.49944466,351.89587153)(544.11750051,351.23614981)(543.62444533,350.69448355)
\curveto(543.16611234,350.18059505)(542.62444608,349.7812898)(541.99944656,349.49656779)
\curveto(541.38139147,349.21879023)(540.72514196,349.07990144)(540.03069805,349.07990144)
\curveto(539.42653184,349.07990144)(538.87792114,349.14587361)(538.38486596,349.27781796)
\curveto(537.89875521,349.4097623)(537.40222781,349.61462326)(536.89528375,349.89240083)
\lineto(536.89528375,345.01740452)
\lineto(534.9369519,345.01740452)
\lineto(534.9369519,360.9444758)
\lineto(536.89528375,360.9444758)
\lineto(536.89528375,359.72572672)
\curveto(537.41611669,360.16322639)(537.99944958,360.52780945)(538.64528243,360.81947589)
\curveto(539.29805971,361.11808678)(539.99250363,361.26739222)(540.72861418,361.26739222)
\curveto(542.1313909,361.26739222)(543.22166785,360.73614262)(543.99944504,359.67364343)
\curveto(544.78416667,358.61808867)(545.17652749,357.14933978)(545.17652749,355.26739676)
\closepath
\moveto(543.15569568,355.21531347)
\curveto(543.15569568,356.61809018)(542.91611253,357.6667005)(542.43694622,358.36114442)
\curveto(541.95777992,359.05558834)(541.22166937,359.4028103)(540.22861456,359.4028103)
\curveto(539.66611499,359.4028103)(539.10014319,359.28128261)(538.53069918,359.03822724)
\curveto(537.96125517,358.79517187)(537.41611669,358.47572767)(536.89528375,358.07989463)
\lineto(536.89528375,351.48614962)
\curveto(537.45083889,351.23614981)(537.92653297,351.06601105)(538.322366,350.97573334)
\curveto(538.72514348,350.88545563)(539.18000424,350.84031678)(539.68694831,350.84031678)
\curveto(540.77722526,350.84031678)(541.62791906,351.20837205)(542.23902971,351.94448261)
\curveto(542.85014036,352.68059316)(543.15569568,353.77087012)(543.15569568,355.21531347)
\closepath
}
}
{
\newrgbcolor{curcolor}{0 0 0}
\pscustom[linestyle=none,fillstyle=solid,fillcolor=curcolor]
{
\newpath
\moveto(558.14526587,355.12156354)
\curveto(558.14526587,353.22573164)(557.65915513,351.72920499)(556.68693364,350.6319836)
\curveto(555.71471216,349.53476221)(554.41262981,348.98615151)(552.7806866,348.98615151)
\curveto(551.13485451,348.98615151)(549.82582772,349.53476221)(548.85360624,350.6319836)
\curveto(547.88832919,351.72920499)(547.40569067,353.22573164)(547.40569067,355.12156354)
\curveto(547.40569067,357.01739544)(547.88832919,358.51392208)(548.85360624,359.61114347)
\curveto(549.82582772,360.71530931)(551.13485451,361.26739222)(552.7806866,361.26739222)
\curveto(554.41262981,361.26739222)(555.71471216,360.71530931)(556.68693364,359.61114347)
\curveto(557.65915513,358.51392208)(558.14526587,357.01739544)(558.14526587,355.12156354)
\closepath
\moveto(556.12443407,355.12156354)
\curveto(556.12443407,356.62850684)(555.8292954,357.74656155)(555.23901807,358.47572767)
\curveto(554.64874074,359.21183822)(553.82929692,359.5798935)(552.7806866,359.5798935)
\curveto(551.7181874,359.5798935)(550.89179914,359.21183822)(550.30152181,358.47572767)
\curveto(549.71818892,357.74656155)(549.42652247,356.62850684)(549.42652247,355.12156354)
\curveto(549.42652247,353.66323131)(549.72166114,352.55559326)(550.31193847,351.79864939)
\curveto(550.9022158,351.04864995)(551.72513184,350.67365024)(552.7806866,350.67365024)
\curveto(553.82235248,350.67365024)(554.63832408,351.04517773)(555.22860141,351.78823273)
\curveto(555.82582319,352.53823216)(556.12443407,353.64934243)(556.12443407,355.12156354)
\closepath
}
}
{
\newrgbcolor{curcolor}{0 0 0}
\pscustom[linestyle=none,fillstyle=solid,fillcolor=curcolor]
{
\newpath
\moveto(569.48900716,352.66323206)
\curveto(569.48900716,351.60073287)(569.04803527,350.72920575)(568.16609149,350.04865071)
\curveto(567.29109215,349.36809567)(566.09317639,349.02781815)(564.57234421,349.02781815)
\curveto(563.71123375,349.02781815)(562.91956768,349.12851252)(562.19734601,349.32990125)
\curveto(561.48206877,349.53823443)(560.88137478,349.7639287)(560.39526404,350.00698407)
\lineto(560.39526404,352.20489908)
\lineto(560.49943062,352.20489908)
\curveto(561.11748571,351.73962165)(561.80498519,351.36809416)(562.56192906,351.09031659)
\curveto(563.31887294,350.81948346)(564.04456683,350.6840669)(564.73901075,350.6840669)
\curveto(565.60012121,350.6840669)(566.27373181,350.82295568)(566.75984255,351.10073325)
\curveto(567.2459533,351.37851081)(567.48900867,351.81601048)(567.48900867,352.41323225)
\curveto(567.48900867,352.87156524)(567.35706432,353.2187872)(567.09317564,353.45489813)
\curveto(566.82928695,353.69100906)(566.32234289,353.8923978)(565.57234345,354.05906434)
\curveto(565.29456589,354.12156429)(564.92998283,354.19448091)(564.47859428,354.27781418)
\curveto(564.03415017,354.36114745)(563.62790048,354.45142516)(563.2598452,354.5486473)
\curveto(562.23901264,354.81948043)(561.51331875,355.21531347)(561.08276352,355.73614641)
\curveto(560.65915273,356.26392378)(560.44734733,356.90975663)(560.44734733,357.67364494)
\curveto(560.44734733,358.15281124)(560.54456948,358.60419979)(560.73901378,359.02781058)
\curveto(560.94040251,359.45142137)(561.24248562,359.82989331)(561.64526309,360.16322639)
\curveto(562.03415169,360.48961503)(562.52720687,360.74655928)(563.12442864,360.93405914)
\curveto(563.72859485,361.12850344)(564.40220545,361.22572559)(565.14526044,361.22572559)
\curveto(565.83970436,361.22572559)(566.54109272,361.1389201)(567.24942552,360.96530912)
\curveto(567.96470275,360.79864258)(568.5584523,360.59378162)(569.03067417,360.35072625)
\lineto(569.03067417,358.25697783)
\lineto(568.92650758,358.25697783)
\curveto(568.42650796,358.62503311)(567.81886953,358.93406065)(567.10359229,359.18406046)
\curveto(566.38831506,359.44100471)(565.6869267,359.56947684)(564.99942722,359.56947684)
\curveto(564.28414998,359.56947684)(563.67998377,359.43058806)(563.18692859,359.15281049)
\curveto(562.69387341,358.88197736)(562.44734582,358.47572767)(562.44734582,357.93406141)
\curveto(562.44734582,357.45489511)(562.59665126,357.09378427)(562.89526214,356.8507289)
\curveto(563.18692859,356.60767352)(563.65915046,356.40975701)(564.31192774,356.25697935)
\curveto(564.67303858,356.17364608)(565.07581605,356.0903128)(565.52026016,356.00697953)
\curveto(565.97164871,355.92364626)(566.34664842,355.84725743)(566.64525931,355.77781304)
\curveto(567.55498084,355.56947987)(568.2563692,355.21184125)(568.74942438,354.70489719)
\curveto(569.24247957,354.19100869)(569.48900716,353.51045365)(569.48900716,352.66323206)
\closepath
}
}
{
\newrgbcolor{curcolor}{0 0 0}
\pscustom[linestyle=none,fillstyle=solid,fillcolor=curcolor]
{
\newpath
\moveto(574.39526698,362.89239099)
\lineto(572.18693532,362.89239099)
\lineto(572.18693532,364.92363946)
\lineto(574.39526698,364.92363946)
\closepath
\moveto(574.27026707,349.30906794)
\lineto(572.31193522,349.30906794)
\lineto(572.31193522,360.9444758)
\lineto(574.27026707,360.9444758)
\closepath
}
}
{
\newrgbcolor{curcolor}{0 0 0}
\pscustom[linestyle=none,fillstyle=solid,fillcolor=curcolor]
{
\newpath
\moveto(584.18691977,349.41323452)
\curveto(583.8188645,349.31601237)(583.41608702,349.23615132)(582.97858735,349.17365137)
\curveto(582.54803212,349.11115142)(582.16261575,349.07990144)(581.82233823,349.07990144)
\curveto(580.63483913,349.07990144)(579.73206203,349.39934564)(579.11400695,350.03823405)
\curveto(578.49595186,350.67712246)(578.18692431,351.70142724)(578.18692431,353.11114839)
\lineto(578.18692431,359.29864371)
\lineto(576.86400865,359.29864371)
\lineto(576.86400865,360.9444758)
\lineto(578.18692431,360.9444758)
\lineto(578.18692431,364.28822327)
\lineto(580.14525617,364.28822327)
\lineto(580.14525617,360.9444758)
\lineto(584.18691977,360.9444758)
\lineto(584.18691977,359.29864371)
\lineto(580.14525617,359.29864371)
\lineto(580.14525617,353.99656439)
\curveto(580.14525617,353.38545374)(580.15914504,352.90628744)(580.1869228,352.55906548)
\curveto(580.21470056,352.21878796)(580.31192271,351.89934375)(580.47858925,351.60073287)
\curveto(580.63136691,351.3229553)(580.83970008,351.11809434)(581.10358877,350.98615)
\curveto(581.3744219,350.86115009)(581.78414381,350.79865014)(582.33275451,350.79865014)
\curveto(582.65219871,350.79865014)(582.98553179,350.843789)(583.33275375,350.93406671)
\curveto(583.67997571,351.03128885)(583.92997552,351.11114991)(584.08275319,351.17364986)
\lineto(584.18691977,351.17364986)
\closepath
}
}
{
\newrgbcolor{curcolor}{0 0 0}
\pscustom[linestyle=none,fillstyle=solid,fillcolor=curcolor]
{
\newpath
\moveto(596.29107595,354.92364702)
\lineto(587.71816577,354.92364702)
\curveto(587.71816577,354.20836978)(587.82580458,353.58337026)(588.0410822,353.04864844)
\curveto(588.25635981,352.52087106)(588.55149848,352.08684361)(588.92649819,351.74656609)
\curveto(589.28760903,351.41323301)(589.71469204,351.1632332)(590.20774722,350.99656666)
\curveto(590.70774684,350.82990012)(591.25635754,350.74656685)(591.85357931,350.74656685)
\curveto(592.64524538,350.74656685)(593.44038367,350.90281673)(594.23899417,351.21531649)
\curveto(595.04454912,351.5347607)(595.61746535,351.84726046)(595.95774287,352.15281578)
\lineto(596.06190946,352.15281578)
\lineto(596.06190946,350.01740073)
\curveto(595.40218774,349.73962317)(594.72857714,349.50698445)(594.04107766,349.31948459)
\curveto(593.35357818,349.13198474)(592.6313565,349.03823481)(591.87441263,349.03823481)
\curveto(589.94385853,349.03823481)(588.43691523,349.55906775)(587.35358272,350.60073362)
\curveto(586.2702502,351.64934394)(585.72858394,353.13545393)(585.72858394,355.05906359)
\curveto(585.72858394,356.96183992)(586.24594466,358.47225545)(587.2806661,359.59031016)
\curveto(588.32233198,360.70836487)(589.6903865,361.26739222)(591.38482967,361.26739222)
\curveto(592.95427292,361.26739222)(594.16260534,360.80905923)(595.00982692,359.89239326)
\curveto(595.86399294,358.97572729)(596.29107595,357.67364494)(596.29107595,355.98614622)
\closepath
\moveto(594.3848274,356.42364589)
\curveto(594.37788296,357.45142289)(594.11746649,358.24656117)(593.60357799,358.80906075)
\curveto(593.09663393,359.37156032)(592.32232896,359.65281011)(591.28066308,359.65281011)
\curveto(590.23205276,359.65281011)(589.39524784,359.34378257)(588.77024831,358.72572748)
\curveto(588.15219322,358.10767239)(587.80149904,357.34031186)(587.71816577,356.42364589)
\closepath
}
}
{
\newrgbcolor{curcolor}{0 0 0}
\pscustom[linestyle=none,fillstyle=solid,fillcolor=curcolor]
{
\newpath
\moveto(618.74938676,349.30906794)
\lineto(616.19730536,349.30906794)
\lineto(608.84314425,363.18405744)
\lineto(608.84314425,349.30906794)
\lineto(606.91606238,349.30906794)
\lineto(606.91606238,364.81947287)
\lineto(610.11397663,364.81947287)
\lineto(616.82230488,352.15281578)
\lineto(616.82230488,364.81947287)
\lineto(618.74938676,364.81947287)
\closepath
}
}
{
\newrgbcolor{curcolor}{0 0 0}
\pscustom[linestyle=none,fillstyle=solid,fillcolor=curcolor]
{
\newpath
\moveto(634.55147053,355.53822989)
\curveto(634.55147053,354.41323074)(634.42647063,353.4305926)(634.17647081,352.59031545)
\curveto(633.93341544,351.75698275)(633.53063797,351.06253883)(632.9681384,350.5069837)
\curveto(632.43341658,349.97920632)(631.80841705,349.59378994)(631.09313981,349.35073457)
\curveto(630.37786258,349.1076792)(629.54452987,348.98615151)(628.59314171,348.98615151)
\curveto(627.62092022,348.98615151)(626.77369864,349.11462364)(626.05147696,349.37156789)
\curveto(625.32925529,349.62851214)(624.72161686,350.00698407)(624.22856167,350.5069837)
\curveto(623.6660621,351.07642771)(623.25981241,351.76392719)(623.0098126,352.56948214)
\curveto(622.76675722,353.37503708)(622.64522954,354.36461967)(622.64522954,355.53822989)
\lineto(622.64522954,364.81947287)
\lineto(624.70772798,364.81947287)
\lineto(624.70772798,355.4340633)
\curveto(624.70772798,354.59378616)(624.76328349,353.93059222)(624.87439452,353.44448147)
\curveto(624.99244999,352.95837073)(625.18689428,352.51739884)(625.45772741,352.12156581)
\curveto(625.76328274,351.67017726)(626.17647687,351.32989974)(626.69730981,351.10073325)
\curveto(627.22508718,350.87156675)(627.85703115,350.75698351)(628.59314171,350.75698351)
\curveto(629.3361967,350.75698351)(629.96814067,350.86809453)(630.4889736,351.09031659)
\curveto(631.00980654,351.31948308)(631.4264729,351.66323282)(631.73897266,352.12156581)
\curveto(632.00980579,352.51739884)(632.20077786,352.96878739)(632.31188889,353.47573145)
\curveto(632.42994436,353.98961995)(632.48897209,354.62503614)(632.48897209,355.38198001)
\lineto(632.48897209,364.81947287)
\lineto(634.55147053,364.81947287)
\closepath
}
}
{
\newrgbcolor{curcolor}{0 0 0}
\pscustom[linestyle=none,fillstyle=solid,fillcolor=curcolor]
{
\newpath
\moveto(652.30145674,349.30906794)
\lineto(650.2389583,349.30906794)
\lineto(650.2389583,362.67364116)
\lineto(645.92646156,353.57989804)
\lineto(644.69729583,353.57989804)
\lineto(640.41604907,362.67364116)
\lineto(640.41604907,349.30906794)
\lineto(638.48896719,349.30906794)
\lineto(638.48896719,364.81947287)
\lineto(641.30146506,364.81947287)
\lineto(645.4368786,356.18406273)
\lineto(649.43687557,364.81947287)
\lineto(652.30145674,364.81947287)
\closepath
}
}
{
\newrgbcolor{curcolor}{0 0 0}
\pscustom[linestyle=none,fillstyle=solid,fillcolor=curcolor]
{
\newpath
\moveto(668.697265,349.30906794)
\lineto(666.49935,349.30906794)
\lineto(664.97851781,353.63198133)
\lineto(658.27018955,353.63198133)
\lineto(656.74935737,349.30906794)
\lineto(654.65560896,349.30906794)
\lineto(660.30143802,364.81947287)
\lineto(663.05143594,364.81947287)
\closepath
\moveto(664.34310163,355.40281333)
\lineto(661.62435368,363.0173909)
\lineto(658.89518908,355.40281333)
\closepath
}
}
{
\newrgbcolor{curcolor}{0 0 0}
\pscustom[linestyle=none,fillstyle=solid,fillcolor=curcolor]
{
\newpath
\moveto(688.12434393,349.30906794)
\lineto(686.16601208,349.30906794)
\lineto(686.16601208,355.93406292)
\curveto(686.16601208,356.46878474)(686.13476211,356.96878436)(686.07226215,357.43406179)
\curveto(686.0097622,357.90628365)(685.89517895,358.27433893)(685.72851241,358.53822762)
\curveto(685.55490143,358.82989407)(685.30490162,359.04517168)(684.97851298,359.18406046)
\curveto(684.65212434,359.32989369)(684.22851355,359.4028103)(683.70768061,359.4028103)
\curveto(683.17295879,359.4028103)(682.61393144,359.27086595)(682.03059854,359.00697727)
\curveto(681.44726565,358.74308858)(680.8882383,358.40628328)(680.35351648,357.99656136)
\lineto(680.35351648,349.30906794)
\lineto(678.39518463,349.30906794)
\lineto(678.39518463,360.9444758)
\lineto(680.35351648,360.9444758)
\lineto(680.35351648,359.65281011)
\curveto(680.96462713,360.15975417)(681.5965711,360.5555872)(682.24934838,360.84030921)
\curveto(682.90212566,361.12503122)(683.57226404,361.26739222)(684.25976352,361.26739222)
\curveto(685.51670702,361.26739222)(686.47503963,360.88892029)(687.13476135,360.13197641)
\curveto(687.79448307,359.37503254)(688.12434393,358.28475559)(688.12434393,356.86114556)
\closepath
}
}
{
\newrgbcolor{curcolor}{0 0 0}
\pscustom[linestyle=none,fillstyle=solid,fillcolor=curcolor]
{
\newpath
\moveto(701.81182072,355.12156354)
\curveto(701.81182072,353.22573164)(701.32570998,351.72920499)(700.35348849,350.6319836)
\curveto(699.38126701,349.53476221)(698.07918466,348.98615151)(696.44724145,348.98615151)
\curveto(694.80140936,348.98615151)(693.49238257,349.53476221)(692.52016109,350.6319836)
\curveto(691.55488404,351.72920499)(691.07224552,353.22573164)(691.07224552,355.12156354)
\curveto(691.07224552,357.01739544)(691.55488404,358.51392208)(692.52016109,359.61114347)
\curveto(693.49238257,360.71530931)(694.80140936,361.26739222)(696.44724145,361.26739222)
\curveto(698.07918466,361.26739222)(699.38126701,360.71530931)(700.35348849,359.61114347)
\curveto(701.32570998,358.51392208)(701.81182072,357.01739544)(701.81182072,355.12156354)
\closepath
\moveto(699.79098892,355.12156354)
\curveto(699.79098892,356.62850684)(699.49585025,357.74656155)(698.90557292,358.47572767)
\curveto(698.31529559,359.21183822)(697.49585177,359.5798935)(696.44724145,359.5798935)
\curveto(695.38474225,359.5798935)(694.55835399,359.21183822)(693.96807666,358.47572767)
\curveto(693.38474377,357.74656155)(693.09307732,356.62850684)(693.09307732,355.12156354)
\curveto(693.09307732,353.66323131)(693.38821599,352.55559326)(693.97849332,351.79864939)
\curveto(694.56877065,351.04864995)(695.39168669,350.67365024)(696.44724145,350.67365024)
\curveto(697.48890733,350.67365024)(698.30487893,351.04517773)(698.89515626,351.78823273)
\curveto(699.49237803,352.53823216)(699.79098892,353.64934243)(699.79098892,355.12156354)
\closepath
}
}
{
\newrgbcolor{curcolor}{0 0 0}
\pscustom[linestyle=none,fillstyle=solid,fillcolor=curcolor]
{
\newpath
\moveto(714.28057557,349.30906794)
\lineto(712.32224372,349.30906794)
\lineto(712.32224372,350.52781701)
\curveto(711.75974415,350.04170627)(711.17293903,349.66323433)(710.56182839,349.39240121)
\curveto(709.95071774,349.12156808)(709.28752379,348.98615151)(708.57224656,348.98615151)
\curveto(707.18335872,348.98615151)(706.07919289,349.52087333)(705.25974906,350.59031697)
\curveto(704.44724968,351.6597606)(704.04099999,353.14239837)(704.04099999,355.03823027)
\curveto(704.04099999,356.02434063)(704.17988877,356.90281219)(704.45766634,357.67364494)
\curveto(704.74238834,358.44447769)(705.1243325,359.10072719)(705.6034988,359.64239345)
\curveto(706.07572067,360.17017083)(706.62433136,360.5729483)(707.24933089,360.85072587)
\curveto(707.88127486,361.12850344)(708.53405214,361.26739222)(709.20766274,361.26739222)
\curveto(709.81877339,361.26739222)(710.36043965,361.20142005)(710.83266151,361.0694757)
\curveto(711.30488338,360.9444758)(711.80141078,360.74655928)(712.32224372,360.47572615)
\lineto(712.32224372,365.51738901)
\lineto(714.28057557,365.51738901)
\closepath
\moveto(712.32224372,352.1736491)
\lineto(712.32224372,358.85072738)
\curveto(711.79446634,359.08683832)(711.32224448,359.25003264)(710.90557813,359.34031035)
\curveto(710.48891177,359.43058806)(710.03405101,359.47572691)(709.54099582,359.47572691)
\curveto(708.44377443,359.47572691)(707.58960841,359.09378275)(706.97849776,358.32989444)
\curveto(706.36738711,357.56600613)(706.06183179,356.48267362)(706.06183179,355.0798969)
\curveto(706.06183179,353.6979535)(706.29794272,352.64587097)(706.77016459,351.92364929)
\curveto(707.24238645,351.20837205)(707.99933032,350.85073344)(709.0409962,350.85073344)
\curveto(709.59655134,350.85073344)(710.15905091,350.97226112)(710.72849493,351.21531649)
\curveto(711.29793894,351.4653163)(711.82918854,351.78476051)(712.32224372,352.1736491)
\closepath
}
}
{
\newrgbcolor{curcolor}{0 0 0}
\pscustom[linestyle=none,fillstyle=solid,fillcolor=curcolor]
{
\newpath
\moveto(727.87429907,354.92364702)
\lineto(719.30138889,354.92364702)
\curveto(719.30138889,354.20836978)(719.4090277,353.58337026)(719.62430531,353.04864844)
\curveto(719.83958293,352.52087106)(720.13472159,352.08684361)(720.50972131,351.74656609)
\curveto(720.87083215,351.41323301)(721.29791516,351.1632332)(721.79097034,350.99656666)
\curveto(722.29096996,350.82990012)(722.83958066,350.74656685)(723.43680243,350.74656685)
\curveto(724.2284685,350.74656685)(725.02360678,350.90281673)(725.82221729,351.21531649)
\curveto(726.62777224,351.5347607)(727.20068847,351.84726046)(727.54096599,352.15281578)
\lineto(727.64513258,352.15281578)
\lineto(727.64513258,350.01740073)
\curveto(726.98541085,349.73962317)(726.31180025,349.50698445)(725.62430077,349.31948459)
\curveto(724.93680129,349.13198474)(724.21457962,349.03823481)(723.45763575,349.03823481)
\curveto(721.52708165,349.03823481)(720.02013835,349.55906775)(718.93680583,350.60073362)
\curveto(717.85347332,351.64934394)(717.31180706,353.13545393)(717.31180706,355.05906359)
\curveto(717.31180706,356.96183992)(717.82916778,358.47225545)(718.86388922,359.59031016)
\curveto(719.9055551,360.70836487)(721.27360962,361.26739222)(722.96805278,361.26739222)
\curveto(724.53749604,361.26739222)(725.74582846,360.80905923)(726.59305004,359.89239326)
\curveto(727.44721606,358.97572729)(727.87429907,357.67364494)(727.87429907,355.98614622)
\closepath
\moveto(725.96805051,356.42364589)
\curveto(725.96110607,357.45142289)(725.7006896,358.24656117)(725.1868011,358.80906075)
\curveto(724.67985704,359.37156032)(723.90555207,359.65281011)(722.8638862,359.65281011)
\curveto(721.81527588,359.65281011)(720.97847095,359.34378257)(720.35347143,358.72572748)
\curveto(719.73541634,358.10767239)(719.38472216,357.34031186)(719.30138889,356.42364589)
\closepath
}
}
{
\newrgbcolor{curcolor}{0 0 0}
\pscustom[linestyle=none,fillstyle=solid,fillcolor=curcolor]
{
\newpath
\moveto(409.64538685,328.48616702)
\lineto(403.14539177,328.48616702)
\lineto(403.14539177,330.37158226)
\lineto(409.64538685,330.37158226)
\closepath
}
}
{
\newrgbcolor{curcolor}{0 0 0}
\pscustom[linestyle=none,fillstyle=solid,fillcolor=curcolor]
{
\newpath
\moveto(432.65578634,322.64242144)
\lineto(430.10370494,322.64242144)
\lineto(422.74954384,336.51741095)
\lineto(422.74954384,322.64242144)
\lineto(420.82246196,322.64242144)
\lineto(420.82246196,338.15282638)
\lineto(424.02037621,338.15282638)
\lineto(430.72870447,325.48616929)
\lineto(430.72870447,338.15282638)
\lineto(432.65578634,338.15282638)
\closepath
}
}
{
\newrgbcolor{curcolor}{0 0 0}
\pscustom[linestyle=none,fillstyle=solid,fillcolor=curcolor]
{
\newpath
\moveto(446.54119228,328.45491705)
\curveto(446.54119228,326.55908515)(446.05508154,325.0625585)(445.08286005,323.96533711)
\curveto(444.11063857,322.86811572)(442.80855622,322.31950502)(441.17661301,322.31950502)
\curveto(439.53078092,322.31950502)(438.22175413,322.86811572)(437.24953265,323.96533711)
\curveto(436.2842556,325.0625585)(435.80161707,326.55908515)(435.80161707,328.45491705)
\curveto(435.80161707,330.35074895)(436.2842556,331.84727559)(437.24953265,332.94449698)
\curveto(438.22175413,334.04866281)(439.53078092,334.60074573)(441.17661301,334.60074573)
\curveto(442.80855622,334.60074573)(444.11063857,334.04866281)(445.08286005,332.94449698)
\curveto(446.05508154,331.84727559)(446.54119228,330.35074895)(446.54119228,328.45491705)
\closepath
\moveto(444.52036048,328.45491705)
\curveto(444.52036048,329.96186035)(444.22522181,331.07991506)(443.63494448,331.80908118)
\curveto(443.04466715,332.54519173)(442.22522333,332.91324701)(441.17661301,332.91324701)
\curveto(440.11411381,332.91324701)(439.28772555,332.54519173)(438.69744822,331.80908118)
\curveto(438.11411533,331.07991506)(437.82244888,329.96186035)(437.82244888,328.45491705)
\curveto(437.82244888,326.99658482)(438.11758754,325.88894677)(438.70786488,325.13200289)
\curveto(439.29814221,324.38200346)(440.12105825,324.00700374)(441.17661301,324.00700374)
\curveto(442.21827889,324.00700374)(443.03425049,324.37853124)(443.62452782,325.12158623)
\curveto(444.22174959,325.87158567)(444.52036048,326.98269594)(444.52036048,328.45491705)
\closepath
}
}
{
\newrgbcolor{curcolor}{0 0 0}
\pscustom[linestyle=none,fillstyle=solid,fillcolor=curcolor]
{
\newpath
\moveto(459.00993271,322.64242144)
\lineto(457.05160086,322.64242144)
\lineto(457.05160086,323.86117052)
\curveto(456.48910129,323.37505978)(455.90229618,322.99658784)(455.29118553,322.72575471)
\curveto(454.68007488,322.45492159)(454.01688094,322.31950502)(453.3016037,322.31950502)
\curveto(451.91271586,322.31950502)(450.80855003,322.85422684)(449.9891062,323.92367047)
\curveto(449.17660682,324.99311411)(448.77035713,326.47575188)(448.77035713,328.37158378)
\curveto(448.77035713,329.35769414)(448.90924591,330.2361657)(449.18702348,331.00699845)
\curveto(449.47174548,331.7778312)(449.85368964,332.4340807)(450.33285594,332.97574696)
\curveto(450.80507781,333.50352434)(451.35368851,333.90630181)(451.97868803,334.18407938)
\curveto(452.610632,334.46185695)(453.26340928,334.60074573)(453.93701988,334.60074573)
\curveto(454.54813053,334.60074573)(455.08979679,334.53477356)(455.56201866,334.40282921)
\curveto(456.03424052,334.27782931)(456.53076792,334.07991279)(457.05160086,333.80907966)
\lineto(457.05160086,338.85074251)
\lineto(459.00993271,338.85074251)
\closepath
\moveto(457.05160086,325.50700261)
\lineto(457.05160086,332.18408089)
\curveto(456.52382348,332.42019182)(456.05160162,332.58338614)(455.63493527,332.67366385)
\curveto(455.21826892,332.76394156)(454.76340815,332.80908042)(454.27035297,332.80908042)
\curveto(453.17313157,332.80908042)(452.31896555,332.42713626)(451.7078549,331.66324795)
\curveto(451.09674426,330.89935964)(450.79118893,329.81602713)(450.79118893,328.41325041)
\curveto(450.79118893,327.03130701)(451.02729986,325.97922447)(451.49952173,325.2570028)
\curveto(451.97174359,324.54172556)(452.72868747,324.18408694)(453.77035334,324.18408694)
\curveto(454.32590848,324.18408694)(454.88840805,324.30561463)(455.45785207,324.54867)
\curveto(456.02729608,324.79866981)(456.55854568,325.11811402)(457.05160086,325.50700261)
\closepath
}
}
{
\newrgbcolor{curcolor}{0 0 0}
\pscustom[linestyle=none,fillstyle=solid,fillcolor=curcolor]
{
\newpath
\moveto(472.60367063,328.25700053)
\lineto(464.03076045,328.25700053)
\curveto(464.03076045,327.54172329)(464.13839926,326.91672377)(464.35367687,326.38200195)
\curveto(464.56895449,325.85422457)(464.86409315,325.42019712)(465.23909287,325.0799196)
\curveto(465.60020371,324.74658652)(466.02728672,324.49658671)(466.5203419,324.32992017)
\curveto(467.02034152,324.16325363)(467.56895222,324.07992036)(468.16617399,324.07992036)
\curveto(468.95784006,324.07992036)(469.75297834,324.23617024)(470.55158885,324.54867)
\curveto(471.3571438,324.8681142)(471.93006003,325.18061397)(472.27033755,325.48616929)
\lineto(472.37450414,325.48616929)
\lineto(472.37450414,323.35075424)
\curveto(471.71478241,323.07297667)(471.04117181,322.84033796)(470.35367233,322.6528381)
\curveto(469.66617285,322.46533824)(468.94395118,322.37158832)(468.1870073,322.37158832)
\curveto(466.25645321,322.37158832)(464.74950991,322.89242125)(463.66617739,323.93408713)
\curveto(462.58284488,324.98269745)(462.04117862,326.46880744)(462.04117862,328.39241709)
\curveto(462.04117862,330.29519343)(462.55853934,331.80560896)(463.59326078,332.92366367)
\curveto(464.63492666,334.04171837)(466.00298118,334.60074573)(467.69742434,334.60074573)
\curveto(469.2668676,334.60074573)(470.47520002,334.14241274)(471.3224216,333.22574677)
\curveto(472.17658762,332.3090808)(472.60367063,331.00699845)(472.60367063,329.31949973)
\closepath
\moveto(470.69742207,329.75699939)
\curveto(470.69047763,330.78477639)(470.43006116,331.57991468)(469.91617266,332.14241426)
\curveto(469.4092286,332.70491383)(468.63492363,332.98616362)(467.59325775,332.98616362)
\curveto(466.54464744,332.98616362)(465.70784251,332.67713607)(465.08284299,332.05908099)
\curveto(464.4647879,331.4410259)(464.11409372,330.67366537)(464.03076045,329.75699939)
\closepath
}
}
{
\newrgbcolor{curcolor}{0 0 0}
\pscustom[linestyle=none,fillstyle=solid,fillcolor=curcolor]
{
\newpath
\moveto(493.30157331,330.40283224)
\curveto(493.30157331,327.61811212)(492.86407364,325.57297478)(491.9890743,324.26742021)
\curveto(491.1210194,322.96881009)(489.77032598,322.31950502)(487.93699403,322.31950502)
\curveto(486.07588433,322.31950502)(484.71477425,322.97922674)(483.85366379,324.29867019)
\curveto(482.99949777,325.61811364)(482.57241476,327.64588988)(482.57241476,330.38199892)
\curveto(482.57241476,333.13894128)(483.00644221,335.17366196)(483.87449711,336.48616097)
\curveto(484.74255201,337.80560442)(486.09671765,338.46532614)(487.93699403,338.46532614)
\curveto(489.79810374,338.46532614)(491.1557416,337.79518776)(492.00990762,336.45491099)
\curveto(492.87101808,335.12157867)(493.30157331,333.10421908)(493.30157331,330.40283224)
\closepath
\moveto(490.56199205,325.67366915)
\curveto(490.80504742,326.23616872)(490.96824174,326.89589045)(491.05157501,327.65283432)
\curveto(491.14185272,328.41672263)(491.18699158,329.3333886)(491.18699158,330.40283224)
\curveto(491.18699158,331.458387)(491.14185272,332.37505297)(491.05157501,333.15283016)
\curveto(490.96824174,333.93060735)(490.8015752,334.59032907)(490.55157539,335.13199533)
\curveto(490.30852002,335.66671715)(489.97518694,336.06949462)(489.55157615,336.34032775)
\curveto(489.1349098,336.61116088)(488.59671576,336.74657744)(487.93699403,336.74657744)
\curveto(487.28421675,336.74657744)(486.74255049,336.61116088)(486.31199526,336.34032775)
\curveto(485.88838447,336.06949462)(485.55157917,335.65977271)(485.30157936,335.11116201)
\curveto(485.06546843,334.59727351)(484.90227411,333.92713513)(484.8119964,333.10074686)
\curveto(484.72866313,332.2743586)(484.68699649,331.36810929)(484.68699649,330.38199892)
\curveto(484.68699649,329.29866641)(484.72519091,328.39241709)(484.80157974,327.66325098)
\curveto(484.87796857,326.93408486)(485.04116289,326.28130758)(485.2911627,325.70491913)
\curveto(485.5203292,325.16325287)(485.84324562,324.75005874)(486.25991197,324.46533673)
\curveto(486.68352276,324.18061472)(487.24255012,324.03825372)(487.93699403,324.03825372)
\curveto(488.58977132,324.03825372)(489.13143758,324.17367029)(489.56199281,324.44450341)
\curveto(489.99254804,324.71533654)(490.32588112,325.12505845)(490.56199205,325.67366915)
\closepath
}
}
{
\newrgbcolor{curcolor}{0 0 0}
\pscustom[linestyle=none,fillstyle=solid,fillcolor=curcolor]
{
\newpath
\moveto(506.23906456,336.2257445)
\lineto(504.0307329,336.2257445)
\lineto(504.0307329,338.25699296)
\lineto(506.23906456,338.25699296)
\closepath
\moveto(506.11406465,322.64242144)
\lineto(504.1557328,322.64242144)
\lineto(504.1557328,334.27782931)
\lineto(506.11406465,334.27782931)
\closepath
}
}
{
\newrgbcolor{curcolor}{0 0 0}
\pscustom[linestyle=none,fillstyle=solid,fillcolor=curcolor]
{
\newpath
\moveto(516.2494744,336.95491062)
\lineto(516.14530781,336.95491062)
\curveto(515.93003019,337.01741057)(515.64878041,337.07991052)(515.30155845,337.14241047)
\curveto(514.95433649,337.21185487)(514.64878116,337.24657706)(514.38489247,337.24657706)
\curveto(513.54461533,337.24657706)(512.93350468,337.0590772)(512.55156053,336.68407749)
\curveto(512.17656081,336.31602221)(511.98906095,335.64588383)(511.98906095,334.67366234)
\lineto(511.98906095,334.27782931)
\lineto(515.52030828,334.27782931)
\lineto(515.52030828,332.63199722)
\lineto(512.05156091,332.63199722)
\lineto(512.05156091,322.64242144)
\lineto(510.09322905,322.64242144)
\lineto(510.09322905,332.63199722)
\lineto(508.77031339,332.63199722)
\lineto(508.77031339,334.27782931)
\lineto(510.09322905,334.27782931)
\lineto(510.09322905,334.66324568)
\curveto(510.09322905,336.04518908)(510.43697879,337.10421606)(511.12447827,337.84032661)
\curveto(511.81197775,338.58338161)(512.80503256,338.9549091)(514.10364269,338.9549091)
\curveto(514.54114236,338.9549091)(514.93350317,338.93407578)(515.28072513,338.89240915)
\curveto(515.63489153,338.85074251)(515.95780795,338.80213144)(516.2494744,338.74657593)
\closepath
}
}
{
\newrgbcolor{curcolor}{0 0 0}
\pscustom[linestyle=none,fillstyle=solid,fillcolor=curcolor]
{
\newpath
\moveto(534.3536243,322.64242144)
\lineto(525.95779731,322.64242144)
\lineto(525.95779731,324.22575358)
\lineto(529.18696154,324.22575358)
\lineto(529.18696154,334.62157905)
\lineto(525.95779731,334.62157905)
\lineto(525.95779731,336.03824464)
\curveto(526.39529698,336.03824464)(526.86404663,336.07296684)(527.36404625,336.14241123)
\curveto(527.86404587,336.21880006)(528.24251781,336.32643887)(528.49946206,336.46532765)
\curveto(528.81890626,336.63893863)(529.06890607,336.85768847)(529.24946149,337.12157716)
\curveto(529.43696135,337.39241028)(529.54460016,337.75352112)(529.57237791,338.20490967)
\lineto(531.18696003,338.20490967)
\lineto(531.18696003,324.22575358)
\lineto(534.3536243,324.22575358)
\closepath
}
}
{
\newrgbcolor{curcolor}{0 0 0}
\pscustom[linestyle=none,fillstyle=solid,fillcolor=curcolor]
{
\newpath
\moveto(542.61403728,325.6111692)
\lineto(539.67653951,318.78825769)
\lineto(538.15570732,318.78825769)
\lineto(539.96820595,325.6111692)
\closepath
}
}
{
\newrgbcolor{curcolor}{0 0 0}
\pscustom[linestyle=none,fillstyle=solid,fillcolor=curcolor]
{
\newpath
\moveto(565.80151367,322.64242144)
\lineto(563.24943226,322.64242144)
\lineto(555.89527116,336.51741095)
\lineto(555.89527116,322.64242144)
\lineto(553.96818929,322.64242144)
\lineto(553.96818929,338.15282638)
\lineto(557.16610353,338.15282638)
\lineto(563.87443179,325.48616929)
\lineto(563.87443179,338.15282638)
\lineto(565.80151367,338.15282638)
\closepath
}
}
{
\newrgbcolor{curcolor}{0 0 0}
\pscustom[linestyle=none,fillstyle=solid,fillcolor=curcolor]
{
\newpath
\moveto(579.6869178,328.45491705)
\curveto(579.6869178,326.55908515)(579.20080706,325.0625585)(578.22858557,323.96533711)
\curveto(577.25636409,322.86811572)(575.95428174,322.31950502)(574.32233853,322.31950502)
\curveto(572.67650644,322.31950502)(571.36747965,322.86811572)(570.39525817,323.96533711)
\curveto(569.42998112,325.0625585)(568.9473426,326.55908515)(568.9473426,328.45491705)
\curveto(568.9473426,330.35074895)(569.42998112,331.84727559)(570.39525817,332.94449698)
\curveto(571.36747965,334.04866281)(572.67650644,334.60074573)(574.32233853,334.60074573)
\curveto(575.95428174,334.60074573)(577.25636409,334.04866281)(578.22858557,332.94449698)
\curveto(579.20080706,331.84727559)(579.6869178,330.35074895)(579.6869178,328.45491705)
\closepath
\moveto(577.666086,328.45491705)
\curveto(577.666086,329.96186035)(577.37094733,331.07991506)(576.78067,331.80908118)
\curveto(576.19039267,332.54519173)(575.37094885,332.91324701)(574.32233853,332.91324701)
\curveto(573.25983933,332.91324701)(572.43345107,332.54519173)(571.84317374,331.80908118)
\curveto(571.25984085,331.07991506)(570.9681744,329.96186035)(570.9681744,328.45491705)
\curveto(570.9681744,326.99658482)(571.26331307,325.88894677)(571.8535904,325.13200289)
\curveto(572.44386773,324.38200346)(573.26678377,324.00700374)(574.32233853,324.00700374)
\curveto(575.36400441,324.00700374)(576.17997601,324.37853124)(576.77025334,325.12158623)
\curveto(577.36747511,325.87158567)(577.666086,326.98269594)(577.666086,328.45491705)
\closepath
}
}
{
\newrgbcolor{curcolor}{0 0 0}
\pscustom[linestyle=none,fillstyle=solid,fillcolor=curcolor]
{
\newpath
\moveto(592.15565823,322.64242144)
\lineto(590.19732638,322.64242144)
\lineto(590.19732638,323.86117052)
\curveto(589.63482681,323.37505978)(589.0480217,322.99658784)(588.43691105,322.72575471)
\curveto(587.8258004,322.45492159)(587.16260646,322.31950502)(586.44732922,322.31950502)
\curveto(585.05844138,322.31950502)(583.95427555,322.85422684)(583.13483173,323.92367047)
\curveto(582.32233234,324.99311411)(581.91608265,326.47575188)(581.91608265,328.37158378)
\curveto(581.91608265,329.35769414)(582.05497143,330.2361657)(582.332749,331.00699845)
\curveto(582.61747101,331.7778312)(582.99941516,332.4340807)(583.47858147,332.97574696)
\curveto(583.95080333,333.50352434)(584.49941403,333.90630181)(585.12441355,334.18407938)
\curveto(585.75635752,334.46185695)(586.4091348,334.60074573)(587.08274541,334.60074573)
\curveto(587.69385605,334.60074573)(588.23552231,334.53477356)(588.70774418,334.40282921)
\curveto(589.17996604,334.27782931)(589.67649344,334.07991279)(590.19732638,333.80907966)
\lineto(590.19732638,338.85074251)
\lineto(592.15565823,338.85074251)
\closepath
\moveto(590.19732638,325.50700261)
\lineto(590.19732638,332.18408089)
\curveto(589.669549,332.42019182)(589.19732714,332.58338614)(588.78066079,332.67366385)
\curveto(588.36399444,332.76394156)(587.90913367,332.80908042)(587.41607849,332.80908042)
\curveto(586.31885709,332.80908042)(585.46469107,332.42713626)(584.85358042,331.66324795)
\curveto(584.24246978,330.89935964)(583.93691445,329.81602713)(583.93691445,328.41325041)
\curveto(583.93691445,327.03130701)(584.17302538,325.97922447)(584.64524725,325.2570028)
\curveto(585.11746911,324.54172556)(585.87441299,324.18408694)(586.91607886,324.18408694)
\curveto(587.471634,324.18408694)(588.03413357,324.30561463)(588.60357759,324.54867)
\curveto(589.1730216,324.79866981)(589.7042712,325.11811402)(590.19732638,325.50700261)
\closepath
}
}
{
\newrgbcolor{curcolor}{0 0 0}
\pscustom[linestyle=none,fillstyle=solid,fillcolor=curcolor]
{
\newpath
\moveto(605.74939615,328.25700053)
\lineto(597.17648597,328.25700053)
\curveto(597.17648597,327.54172329)(597.28412478,326.91672377)(597.49940239,326.38200195)
\curveto(597.71468001,325.85422457)(598.00981867,325.42019712)(598.38481839,325.0799196)
\curveto(598.74592923,324.74658652)(599.17301224,324.49658671)(599.66606742,324.32992017)
\curveto(600.16606704,324.16325363)(600.71467774,324.07992036)(601.31189951,324.07992036)
\curveto(602.10356558,324.07992036)(602.89870386,324.23617024)(603.69731437,324.54867)
\curveto(604.50286932,324.8681142)(605.07578555,325.18061397)(605.41606307,325.48616929)
\lineto(605.52022966,325.48616929)
\lineto(605.52022966,323.35075424)
\curveto(604.86050793,323.07297667)(604.18689733,322.84033796)(603.49939785,322.6528381)
\curveto(602.81189837,322.46533824)(602.0896767,322.37158832)(601.33273283,322.37158832)
\curveto(599.40217873,322.37158832)(597.89523543,322.89242125)(596.81190291,323.93408713)
\curveto(595.7285704,324.98269745)(595.18690414,326.46880744)(595.18690414,328.39241709)
\curveto(595.18690414,330.29519343)(595.70426486,331.80560896)(596.7389863,332.92366367)
\curveto(597.78065218,334.04171837)(599.1487067,334.60074573)(600.84314986,334.60074573)
\curveto(602.41259312,334.60074573)(603.62092554,334.14241274)(604.46814712,333.22574677)
\curveto(605.32231314,332.3090808)(605.74939615,331.00699845)(605.74939615,329.31949973)
\closepath
\moveto(603.84314759,329.75699939)
\curveto(603.83620315,330.78477639)(603.57578668,331.57991468)(603.06189818,332.14241426)
\curveto(602.55495412,332.70491383)(601.78064915,332.98616362)(600.73898327,332.98616362)
\curveto(599.69037296,332.98616362)(598.85356803,332.67713607)(598.22856851,332.05908099)
\curveto(597.61051342,331.4410259)(597.25981924,330.67366537)(597.17648597,329.75699939)
\closepath
}
}
{
\newrgbcolor{curcolor}{0 0 0}
\pscustom[linestyle=none,fillstyle=solid,fillcolor=curcolor]
{
\newpath
\moveto(625.58272336,322.64242144)
\lineto(617.18689638,322.64242144)
\lineto(617.18689638,324.22575358)
\lineto(620.4160606,324.22575358)
\lineto(620.4160606,334.62157905)
\lineto(617.18689638,334.62157905)
\lineto(617.18689638,336.03824464)
\curveto(617.62439605,336.03824464)(618.09314569,336.07296684)(618.59314531,336.14241123)
\curveto(619.09314494,336.21880006)(619.47161687,336.32643887)(619.72856112,336.46532765)
\curveto(620.04800532,336.63893863)(620.29800514,336.85768847)(620.47856055,337.12157716)
\curveto(620.66606041,337.39241028)(620.77369922,337.75352112)(620.80147698,338.20490967)
\lineto(622.41605909,338.20490967)
\lineto(622.41605909,324.22575358)
\lineto(625.58272336,324.22575358)
\closepath
}
}
{
\newrgbcolor{curcolor}{0 0 0}
\pscustom[linestyle=none,fillstyle=solid,fillcolor=curcolor]
{
\newpath
\moveto(639.38479008,336.2257445)
\lineto(637.17645842,336.2257445)
\lineto(637.17645842,338.25699296)
\lineto(639.38479008,338.25699296)
\closepath
\moveto(639.25979017,322.64242144)
\lineto(637.30145832,322.64242144)
\lineto(637.30145832,334.27782931)
\lineto(639.25979017,334.27782931)
\closepath
}
}
{
\newrgbcolor{curcolor}{0 0 0}
\pscustom[linestyle=none,fillstyle=solid,fillcolor=curcolor]
{
\newpath
\moveto(649.39519271,336.95491062)
\lineto(649.29102612,336.95491062)
\curveto(649.0757485,337.01741057)(648.79449872,337.07991052)(648.44727676,337.14241047)
\curveto(648.1000548,337.21185487)(647.79449947,337.24657706)(647.53061078,337.24657706)
\curveto(646.69033364,337.24657706)(646.07922299,337.0590772)(645.69727884,336.68407749)
\curveto(645.32227912,336.31602221)(645.13477926,335.64588383)(645.13477926,334.67366234)
\lineto(645.13477926,334.27782931)
\lineto(648.66602659,334.27782931)
\lineto(648.66602659,332.63199722)
\lineto(645.19727922,332.63199722)
\lineto(645.19727922,322.64242144)
\lineto(643.23894737,322.64242144)
\lineto(643.23894737,332.63199722)
\lineto(641.9160317,332.63199722)
\lineto(641.9160317,334.27782931)
\lineto(643.23894737,334.27782931)
\lineto(643.23894737,334.66324568)
\curveto(643.23894737,336.04518908)(643.58269711,337.10421606)(644.27019658,337.84032661)
\curveto(644.95769606,338.58338161)(645.95075087,338.9549091)(647.249361,338.9549091)
\curveto(647.68686067,338.9549091)(648.07922148,338.93407578)(648.42644344,338.89240915)
\curveto(648.78060984,338.85074251)(649.10352626,338.80213144)(649.39519271,338.74657593)
\closepath
}
}
{
\newrgbcolor{curcolor}{0 0 0}
\pscustom[linestyle=none,fillstyle=solid,fillcolor=curcolor]
{
\newpath
\moveto(668.36393971,330.40283224)
\curveto(668.36393971,327.61811212)(667.92644004,325.57297478)(667.0514407,324.26742021)
\curveto(666.1833858,322.96881009)(664.83269238,322.31950502)(662.99936043,322.31950502)
\curveto(661.13825073,322.31950502)(659.77714065,322.97922674)(658.91603019,324.29867019)
\curveto(658.06186417,325.61811364)(657.63478116,327.64588988)(657.63478116,330.38199892)
\curveto(657.63478116,333.13894128)(658.06880861,335.17366196)(658.9368635,336.48616097)
\curveto(659.8049184,337.80560442)(661.15908404,338.46532614)(662.99936043,338.46532614)
\curveto(664.86047013,338.46532614)(666.218108,337.79518776)(667.07227402,336.45491099)
\curveto(667.93338448,335.12157867)(668.36393971,333.10421908)(668.36393971,330.40283224)
\closepath
\moveto(665.62435844,325.67366915)
\curveto(665.86741382,326.23616872)(666.03060814,326.89589045)(666.11394141,327.65283432)
\curveto(666.20421912,328.41672263)(666.24935797,329.3333886)(666.24935797,330.40283224)
\curveto(666.24935797,331.458387)(666.20421912,332.37505297)(666.11394141,333.15283016)
\curveto(666.03060814,333.93060735)(665.8639416,334.59032907)(665.61394179,335.13199533)
\curveto(665.37088641,335.66671715)(665.03755333,336.06949462)(664.61394254,336.34032775)
\curveto(664.19727619,336.61116088)(663.65908215,336.74657744)(662.99936043,336.74657744)
\curveto(662.34658315,336.74657744)(661.80491689,336.61116088)(661.37436166,336.34032775)
\curveto(660.95075087,336.06949462)(660.61394557,335.65977271)(660.36394576,335.11116201)
\curveto(660.12783483,334.59727351)(659.9646405,333.92713513)(659.87436279,333.10074686)
\curveto(659.79102952,332.2743586)(659.74936289,331.36810929)(659.74936289,330.38199892)
\curveto(659.74936289,329.29866641)(659.7875573,328.39241709)(659.86394614,327.66325098)
\curveto(659.94033497,326.93408486)(660.10352929,326.28130758)(660.3535291,325.70491913)
\curveto(660.58269559,325.16325287)(660.90561201,324.75005874)(661.32227837,324.46533673)
\curveto(661.74588916,324.18061472)(662.30491651,324.03825372)(662.99936043,324.03825372)
\curveto(663.65213771,324.03825372)(664.19380397,324.17367029)(664.6243592,324.44450341)
\curveto(665.05491443,324.71533654)(665.38824751,325.12505845)(665.62435844,325.67366915)
\closepath
}
}
{
\newrgbcolor{curcolor}{0.7019608 0.7019608 0.7019608}
\pscustom[linestyle=none,fillstyle=solid,fillcolor=curcolor,opacity=0.92623001]
{
\newpath
\moveto(355.90269183,181.64750308)
\lineto(766.6169954,181.64750308)
\lineto(766.6169954,45.21893439)
\lineto(355.90269183,45.21893439)
\closepath
}
}
{
\newrgbcolor{curcolor}{0 0 0}
\pscustom[linewidth=1.00157103,linecolor=curcolor]
{
\newpath
\moveto(355.90269183,181.64750308)
\lineto(766.6169954,181.64750308)
\lineto(766.6169954,45.21893439)
\lineto(355.90269183,45.21893439)
\closepath
}
}
{
\newrgbcolor{curcolor}{0 0 0}
\pscustom[linestyle=none,fillstyle=solid,fillcolor=curcolor]
{
\newpath
\moveto(517.73459943,149.48518609)
\lineto(512.71508416,149.48518609)
\lineto(502.98855263,161.04764927)
\lineto(497.53935123,161.04764927)
\lineto(497.53935123,149.48518609)
\lineto(493.67217604,149.48518609)
\lineto(493.67217604,178.56712474)
\lineto(501.81668136,178.56712474)
\curveto(503.57448827,178.56712474)(505.03932735,178.44993761)(506.21119862,178.21556336)
\curveto(507.38306989,177.9942099)(508.43775403,177.59056535)(509.37525105,177.00462972)
\curveto(510.42993519,176.34056933)(511.25024508,175.50072825)(511.83618071,174.48510649)
\curveto(512.43513714,173.48250551)(512.73461535,172.20646791)(512.73461535,170.65699368)
\curveto(512.73461535,168.56064618)(512.20727328,166.80283928)(511.15258914,165.38357297)
\curveto(510.097905,163.97732744)(508.6460867,162.91613291)(506.79713426,162.19998935)
\closepath
\moveto(508.69165947,170.38355705)
\curveto(508.69165947,171.21688773)(508.54192037,171.95256247)(508.24244215,172.59058127)
\curveto(507.95598473,173.24162086)(507.47421543,173.78849412)(506.79713426,174.23120105)
\curveto(506.2372402,174.60880401)(505.57317982,174.86921985)(504.8049531,175.01244856)
\curveto(504.03672638,175.16869806)(503.13178134,175.24682281)(502.09011799,175.24682281)
\lineto(497.53935123,175.24682281)
\lineto(497.53935123,164.27029526)
\lineto(501.44558879,164.27029526)
\curveto(502.66954323,164.27029526)(503.73724816,164.3744616)(504.6487036,164.58279427)
\curveto(505.56015903,164.80414773)(506.33489614,165.20779228)(506.97291495,165.79372791)
\curveto(507.55885058,166.34060117)(507.98853671,166.96559918)(508.26197334,167.66872194)
\curveto(508.54843076,168.38486549)(508.69165947,169.28981053)(508.69165947,170.38355705)
\closepath
}
}
{
\newrgbcolor{curcolor}{0 0 0}
\pscustom[linestyle=none,fillstyle=solid,fillcolor=curcolor]
{
\newpath
\moveto(538.47672163,160.01249632)
\lineto(522.40255406,160.01249632)
\curveto(522.40255406,158.67135475)(522.60437633,157.49948348)(523.00802088,156.49688251)
\curveto(523.41166543,155.50730233)(523.96504908,154.69350284)(524.66817184,154.05548403)
\curveto(525.34525302,153.43048602)(526.14603172,152.96173752)(527.07050795,152.64923851)
\curveto(528.00800496,152.33673951)(529.03664752,152.18049)(530.15643562,152.18049)
\curveto(531.64080589,152.18049)(533.13168656,152.47345782)(534.62907763,153.05939346)
\curveto(536.13948949,153.65834988)(537.21370482,154.24428552)(537.85172362,154.81720036)
\lineto(538.0470355,154.81720036)
\lineto(538.0470355,150.81330686)
\curveto(536.81006027,150.29247518)(535.54704346,149.85627865)(534.25798506,149.50471727)
\curveto(532.96892667,149.15315589)(531.61476431,148.9773752)(530.195498,148.9773752)
\curveto(526.57571785,148.9773752)(523.75020602,149.95393459)(521.71896248,151.90705337)
\curveto(519.68771895,153.87319295)(518.67209719,156.65964241)(518.67209719,160.26640176)
\curveto(518.67209719,163.83409873)(519.64214618,166.66612097)(521.58224417,168.76246846)
\curveto(523.53536295,170.85881595)(526.10045895,171.9069897)(529.27753217,171.9069897)
\curveto(532.22023113,171.9069897)(534.48584892,171.04761743)(536.07438553,169.3288729)
\curveto(537.67594293,167.61012838)(538.47672163,165.1687299)(538.47672163,162.00467747)
\closepath
\moveto(534.90251426,162.82498736)
\curveto(534.88949347,164.75206456)(534.40121377,166.24294523)(533.43767517,167.29762937)
\curveto(532.48715737,168.35231351)(531.03533907,168.87965558)(529.08222029,168.87965558)
\curveto(527.11608072,168.87965558)(525.5470753,168.30023035)(524.37520403,167.14137987)
\curveto(523.21635355,165.98252939)(522.55880356,164.54373189)(522.40255406,162.82498736)
\closepath
}
}
{
\newrgbcolor{curcolor}{0 0 0}
\pscustom[linestyle=none,fillstyle=solid,fillcolor=curcolor]
{
\newpath
\moveto(563.24226855,160.55936958)
\curveto(563.24226855,158.73645871)(562.98185271,157.09583894)(562.46102104,155.63751025)
\curveto(561.95321016,154.17918156)(561.26310819,152.95522712)(560.39071513,151.96564694)
\curveto(559.46623891,150.93700438)(558.45061714,150.16226726)(557.34384983,149.64143559)
\curveto(556.23708252,149.1336247)(555.01963848,148.87971926)(553.69151771,148.87971926)
\curveto(552.45454248,148.87971926)(551.37381676,149.02945837)(550.44934053,149.32893658)
\curveto(549.52486431,149.615394)(548.61340888,150.00601776)(547.71497424,150.50080785)
\lineto(547.48059999,149.48518609)
\lineto(544.04311093,149.48518609)
\lineto(544.04311093,179.87571432)
\lineto(547.71497424,179.87571432)
\lineto(547.71497424,169.0163739)
\curveto(548.7436168,169.86272537)(549.83736332,170.55282734)(550.99621379,171.08667981)
\curveto(552.15506427,171.63355307)(553.45714346,171.9069897)(554.90245136,171.9069897)
\curveto(557.48056815,171.9069897)(559.51181168,170.91740951)(560.99618195,168.93824915)
\curveto(562.49357302,166.95908878)(563.24226855,164.16612893)(563.24226855,160.55936958)
\closepath
\moveto(559.45321812,160.46171364)
\curveto(559.45321812,163.06587201)(559.02353198,165.03852198)(558.16415972,166.37966354)
\curveto(557.30478746,167.7338259)(555.91807312,168.41090708)(554.00401672,168.41090708)
\curveto(552.93631178,168.41090708)(551.85558606,168.17653282)(550.76183954,167.70778432)
\curveto(549.66809302,167.2520566)(548.65247125,166.65961057)(547.71497424,165.93044622)
\lineto(547.71497424,153.43048602)
\curveto(548.75663759,152.96173752)(549.64856183,152.63621772)(550.39074697,152.45392663)
\curveto(551.1459529,152.27163555)(551.99881477,152.18049)(552.94933257,152.18049)
\curveto(554.98057611,152.18049)(556.56911272,152.84455039)(557.7149424,154.17267116)
\curveto(558.87379288,155.51381272)(559.45321812,157.61016022)(559.45321812,160.46171364)
\closepath
}
}
{
\newrgbcolor{curcolor}{0 0 0}
\pscustom[linestyle=none,fillstyle=solid,fillcolor=curcolor]
{
\newpath
\moveto(587.55859863,160.38358889)
\curveto(587.55859863,156.8289127)(586.6471432,154.02293205)(584.82423234,151.96564694)
\curveto(583.00132148,149.90836182)(580.559923,148.87971926)(577.50003691,148.87971926)
\curveto(574.41410923,148.87971926)(571.95968997,149.90836182)(570.1367791,151.96564694)
\curveto(568.32688903,154.02293205)(567.421944,156.8289127)(567.421944,160.38358889)
\curveto(567.421944,163.93826507)(568.32688903,166.74424572)(570.1367791,168.80153083)
\curveto(571.95968997,170.87183674)(574.41410923,171.9069897)(577.50003691,171.9069897)
\curveto(580.559923,171.9069897)(583.00132148,170.87183674)(584.82423234,168.80153083)
\curveto(586.6471432,166.74424572)(587.55859863,163.93826507)(587.55859863,160.38358889)
\closepath
\moveto(583.7695482,160.38358889)
\curveto(583.7695482,163.20910072)(583.21616454,165.30544821)(582.10939723,166.67263136)
\curveto(581.00262992,168.0528353)(579.46617648,168.74293727)(577.50003691,168.74293727)
\curveto(575.50785575,168.74293727)(573.95838152,168.0528353)(572.85161421,166.67263136)
\curveto(571.75786769,165.30544821)(571.21099443,163.20910072)(571.21099443,160.38358889)
\curveto(571.21099443,157.64922259)(571.76437809,155.57240629)(572.8711454,154.15313997)
\curveto(573.97791271,152.74689445)(575.52087654,152.04377169)(577.50003691,152.04377169)
\curveto(579.45315569,152.04377169)(580.98309874,152.74038405)(582.08986605,154.13360878)
\curveto(583.20965415,155.53985431)(583.7695482,157.62318101)(583.7695482,160.38358889)
\closepath
}
}
{
\newrgbcolor{curcolor}{0 0 0}
\pscustom[linestyle=none,fillstyle=solid,fillcolor=curcolor]
{
\newpath
\moveto(611.83586068,160.38358889)
\curveto(611.83586068,156.8289127)(610.92440525,154.02293205)(609.10149439,151.96564694)
\curveto(607.27858353,149.90836182)(604.83718505,148.87971926)(601.77729896,148.87971926)
\curveto(598.69137128,148.87971926)(596.23695202,149.90836182)(594.41404115,151.96564694)
\curveto(592.60415108,154.02293205)(591.69920605,156.8289127)(591.69920605,160.38358889)
\curveto(591.69920605,163.93826507)(592.60415108,166.74424572)(594.41404115,168.80153083)
\curveto(596.23695202,170.87183674)(598.69137128,171.9069897)(601.77729896,171.9069897)
\curveto(604.83718505,171.9069897)(607.27858353,170.87183674)(609.10149439,168.80153083)
\curveto(610.92440525,166.74424572)(611.83586068,163.93826507)(611.83586068,160.38358889)
\closepath
\moveto(608.04681025,160.38358889)
\curveto(608.04681025,163.20910072)(607.49342659,165.30544821)(606.38665928,166.67263136)
\curveto(605.27989197,168.0528353)(603.74343853,168.74293727)(601.77729896,168.74293727)
\curveto(599.7851178,168.74293727)(598.23564357,168.0528353)(597.12887626,166.67263136)
\curveto(596.03512974,165.30544821)(595.48825648,163.20910072)(595.48825648,160.38358889)
\curveto(595.48825648,157.64922259)(596.04164014,155.57240629)(597.14840745,154.15313997)
\curveto(598.25517476,152.74689445)(599.79813859,152.04377169)(601.77729896,152.04377169)
\curveto(603.73041774,152.04377169)(605.26036079,152.74038405)(606.36712809,154.13360878)
\curveto(607.4869162,155.53985431)(608.04681025,157.62318101)(608.04681025,160.38358889)
\closepath
}
}
{
\newrgbcolor{curcolor}{0 0 0}
\pscustom[linestyle=none,fillstyle=solid,fillcolor=curcolor]
{
\newpath
\moveto(628.84752807,149.68049796)
\curveto(628.1574261,149.49820688)(627.40222018,149.34846777)(626.58191029,149.23128064)
\curveto(625.77462119,149.11409352)(625.05196724,149.05549995)(624.41394844,149.05549995)
\curveto(622.18739303,149.05549995)(620.49469009,149.65445638)(619.33583961,150.85236923)
\curveto(618.17698913,152.05028208)(617.59756389,153.97084889)(617.59756389,156.61406964)
\lineto(617.59756389,168.2155952)
\lineto(615.11710304,168.2155952)
\lineto(615.11710304,171.30152287)
\lineto(617.59756389,171.30152287)
\lineto(617.59756389,177.57103416)
\lineto(621.2694272,177.57103416)
\lineto(621.2694272,171.30152287)
\lineto(628.84752807,171.30152287)
\lineto(628.84752807,168.2155952)
\lineto(621.2694272,168.2155952)
\lineto(621.2694272,158.2742206)
\curveto(621.2694272,157.12839092)(621.29546879,156.22995628)(621.34755195,155.57891668)
\curveto(621.39963512,154.94089788)(621.58192621,154.34194145)(621.89442521,153.7820474)
\curveto(622.18088263,153.26121573)(622.57150639,152.87710237)(623.06629648,152.62970732)
\curveto(623.57410736,152.39533307)(624.34233409,152.27814594)(625.37097664,152.27814594)
\curveto(625.96993307,152.27814594)(626.59493108,152.36278109)(627.24597067,152.53205138)
\curveto(627.89701027,152.71434247)(628.36575877,152.86408158)(628.6522162,152.9812687)
\lineto(628.84752807,152.9812687)
\closepath
}
}
{
\newrgbcolor{curcolor}{0 0 0}
\pscustom[linestyle=none,fillstyle=solid,fillcolor=curcolor]
{
\newpath
\moveto(387.53080274,135.82384508)
\lineto(381.03080766,135.82384508)
\lineto(381.03080766,137.70926032)
\lineto(387.53080274,137.70926032)
\closepath
}
}
{
\newrgbcolor{curcolor}{0 0 0}
\pscustom[linestyle=none,fillstyle=solid,fillcolor=curcolor]
{
\newpath
\moveto(411.81203461,137.71967698)
\curveto(411.81203461,136.30995582)(411.50300706,135.03217901)(410.88495197,133.88634655)
\curveto(410.27384133,132.74051408)(409.45786972,131.85162586)(408.43703716,131.2196819)
\curveto(407.72870436,130.78218223)(406.93703829,130.46621025)(406.06203896,130.27176595)
\curveto(405.19398406,130.07732165)(404.04815159,129.9800995)(402.62454156,129.9800995)
\lineto(398.70787785,129.9800995)
\lineto(398.70787785,145.49050443)
\lineto(402.58287492,145.49050443)
\curveto(404.09676267,145.49050443)(405.29815065,145.37939341)(406.18703886,145.15717135)
\curveto(407.08287152,144.94189374)(407.83981539,144.64328285)(408.45787048,144.2613387)
\curveto(409.51342523,143.60161697)(410.33634128,142.72314542)(410.92661861,141.62592402)
\curveto(411.51689594,140.52870263)(411.81203461,139.22662028)(411.81203461,137.71967698)
\closepath
\moveto(409.65578624,137.75092696)
\curveto(409.65578624,138.96620381)(409.44398084,139.99050859)(409.02037005,140.8238413)
\curveto(408.59675926,141.657174)(407.9648153,142.3134235)(407.12453815,142.79258981)
\curveto(406.5134275,143.13981177)(405.86412244,143.37939492)(405.17662296,143.51133926)
\curveto(404.48912348,143.65022805)(403.66620744,143.71967244)(402.70787483,143.71967244)
\lineto(400.77037629,143.71967244)
\lineto(400.77037629,131.7509315)
\lineto(402.70787483,131.7509315)
\curveto(403.70092963,131.7509315)(404.56551231,131.82384811)(405.30162287,131.96968133)
\curveto(406.04467786,132.11551455)(406.7252329,132.38634768)(407.34328799,132.78218072)
\curveto(408.11412074,133.2752359)(408.69050919,133.92454096)(409.07245335,134.73009591)
\curveto(409.46134194,135.53565085)(409.65578624,136.54259454)(409.65578624,137.75092696)
\closepath
}
}
{
\newrgbcolor{curcolor}{0 0 0}
\pscustom[linestyle=none,fillstyle=solid,fillcolor=curcolor]
{
\newpath
\moveto(424.7286918,135.59467859)
\lineto(416.15578162,135.59467859)
\curveto(416.15578162,134.87940135)(416.26342043,134.25440182)(416.47869804,133.71968001)
\curveto(416.69397566,133.19190263)(416.98911432,132.75787518)(417.36411404,132.41759766)
\curveto(417.72522488,132.08426458)(418.15230789,131.83426477)(418.64536307,131.66759823)
\curveto(419.14536269,131.50093168)(419.69397339,131.41759841)(420.29119516,131.41759841)
\curveto(421.08286123,131.41759841)(421.87799951,131.5738483)(422.67661002,131.88634806)
\curveto(423.48216497,132.20579226)(424.0550812,132.51829203)(424.39535872,132.82384735)
\lineto(424.49952531,132.82384735)
\lineto(424.49952531,130.6884323)
\curveto(423.83980359,130.41065473)(423.16619298,130.17801602)(422.4786935,129.99051616)
\curveto(421.79119402,129.8030163)(421.06897235,129.70926637)(420.31202848,129.70926637)
\curveto(418.38147438,129.70926637)(416.87453108,130.23009931)(415.79119856,131.27176519)
\curveto(414.70786605,132.32037551)(414.16619979,133.8064855)(414.16619979,135.73009515)
\curveto(414.16619979,137.63287149)(414.68356051,139.14328701)(415.71828195,140.26134172)
\curveto(416.75994783,141.37939643)(418.12800235,141.93842379)(419.82244551,141.93842379)
\curveto(421.39188877,141.93842379)(422.60022119,141.4800908)(423.44744277,140.56342483)
\curveto(424.30160879,139.64675886)(424.7286918,138.34467651)(424.7286918,136.65717778)
\closepath
\moveto(422.82244324,137.09467745)
\curveto(422.81549881,138.12245445)(422.55508234,138.91759274)(422.04119384,139.48009231)
\curveto(421.53424977,140.04259189)(420.7599448,140.32384168)(419.71827893,140.32384168)
\curveto(418.66966861,140.32384168)(417.83286369,140.01481413)(417.20786416,139.39675904)
\curveto(416.58980907,138.77870396)(416.23911489,138.01134343)(416.15578162,137.09467745)
\closepath
}
}
{
\newrgbcolor{curcolor}{0 0 0}
\pscustom[linestyle=none,fillstyle=solid,fillcolor=curcolor]
{
\newpath
\moveto(429.67660433,129.9800995)
\lineto(427.71827248,129.9800995)
\lineto(427.71827248,146.18842057)
\lineto(429.67660433,146.18842057)
\closepath
}
}
{
\newrgbcolor{curcolor}{0 0 0}
\pscustom[linestyle=none,fillstyle=solid,fillcolor=curcolor]
{
\newpath
\moveto(443.29117811,135.59467859)
\lineto(434.71826792,135.59467859)
\curveto(434.71826792,134.87940135)(434.82590673,134.25440182)(435.04118435,133.71968001)
\curveto(435.25646196,133.19190263)(435.55160063,132.75787518)(435.92660034,132.41759766)
\curveto(436.28771118,132.08426458)(436.71479419,131.83426477)(437.20784937,131.66759823)
\curveto(437.707849,131.50093168)(438.25645969,131.41759841)(438.85368146,131.41759841)
\curveto(439.64534753,131.41759841)(440.44048582,131.5738483)(441.23909632,131.88634806)
\curveto(442.04465127,132.20579226)(442.6175675,132.51829203)(442.95784502,132.82384735)
\lineto(443.06201161,132.82384735)
\lineto(443.06201161,130.6884323)
\curveto(442.40228989,130.41065473)(441.72867929,130.17801602)(441.04117981,129.99051616)
\curveto(440.35368033,129.8030163)(439.63145865,129.70926637)(438.87451478,129.70926637)
\curveto(436.94396069,129.70926637)(435.43701738,130.23009931)(434.35368487,131.27176519)
\curveto(433.27035235,132.32037551)(432.7286861,133.8064855)(432.7286861,135.73009515)
\curveto(432.7286861,137.63287149)(433.24604682,139.14328701)(434.28076826,140.26134172)
\curveto(435.32243413,141.37939643)(436.69048865,141.93842379)(438.38493182,141.93842379)
\curveto(439.95437507,141.93842379)(441.16270749,141.4800908)(442.00992907,140.56342483)
\curveto(442.8640951,139.64675886)(443.29117811,138.34467651)(443.29117811,136.65717778)
\closepath
\moveto(441.38492955,137.09467745)
\curveto(441.37798511,138.12245445)(441.11756864,138.91759274)(440.60368014,139.48009231)
\curveto(440.09673608,140.04259189)(439.32243111,140.32384168)(438.28076523,140.32384168)
\curveto(437.23215491,140.32384168)(436.39534999,140.01481413)(435.77035046,139.39675904)
\curveto(435.15229537,138.77870396)(434.80160119,138.01134343)(434.71826792,137.09467745)
\closepath
}
}
{
\newrgbcolor{curcolor}{0 0 0}
\pscustom[linestyle=none,fillstyle=solid,fillcolor=curcolor]
{
\newpath
\moveto(452.30158756,130.08426609)
\curveto(451.93353228,129.98704394)(451.53075481,129.90718289)(451.09325514,129.84468294)
\curveto(450.66269991,129.78218298)(450.27728353,129.75093301)(449.93700601,129.75093301)
\curveto(448.74950691,129.75093301)(447.84672982,130.07037721)(447.22867473,130.70926562)
\curveto(446.61061964,131.34815402)(446.3015921,132.3724588)(446.3015921,133.78217996)
\lineto(446.3015921,139.96967528)
\lineto(444.97867643,139.96967528)
\lineto(444.97867643,141.61550737)
\lineto(446.3015921,141.61550737)
\lineto(446.3015921,144.95925484)
\lineto(448.25992395,144.95925484)
\lineto(448.25992395,141.61550737)
\lineto(452.30158756,141.61550737)
\lineto(452.30158756,139.96967528)
\lineto(448.25992395,139.96967528)
\lineto(448.25992395,134.66759596)
\curveto(448.25992395,134.05648531)(448.27381283,133.577319)(448.30159058,133.23009704)
\curveto(448.32936834,132.88981952)(448.42659049,132.57037532)(448.59325703,132.27176443)
\curveto(448.74603469,131.99398687)(448.95436787,131.78912591)(449.21825656,131.65718157)
\curveto(449.48908969,131.53218166)(449.8988116,131.46968171)(450.44742229,131.46968171)
\curveto(450.7668665,131.46968171)(451.10019958,131.51482056)(451.44742154,131.60509827)
\curveto(451.7946435,131.70232042)(452.04464331,131.78218147)(452.19742097,131.84468142)
\lineto(452.30158756,131.84468142)
\closepath
}
}
{
\newrgbcolor{curcolor}{0 0 0}
\pscustom[linestyle=none,fillstyle=solid,fillcolor=curcolor]
{
\newpath
\moveto(464.40574374,135.59467859)
\lineto(455.83283356,135.59467859)
\curveto(455.83283356,134.87940135)(455.94047236,134.25440182)(456.15574998,133.71968001)
\curveto(456.37102759,133.19190263)(456.66616626,132.75787518)(457.04116598,132.41759766)
\curveto(457.40227681,132.08426458)(457.82935982,131.83426477)(458.32241501,131.66759823)
\curveto(458.82241463,131.50093168)(459.37102532,131.41759841)(459.96824709,131.41759841)
\curveto(460.75991316,131.41759841)(461.55505145,131.5738483)(462.35366196,131.88634806)
\curveto(463.1592169,132.20579226)(463.73213314,132.51829203)(464.07241066,132.82384735)
\lineto(464.17657724,132.82384735)
\lineto(464.17657724,130.6884323)
\curveto(463.51685552,130.41065473)(462.84324492,130.17801602)(462.15574544,129.99051616)
\curveto(461.46824596,129.8030163)(460.74602428,129.70926637)(459.98908041,129.70926637)
\curveto(458.05852632,129.70926637)(456.55158301,130.23009931)(455.4682505,131.27176519)
\curveto(454.38491798,132.32037551)(453.84325173,133.8064855)(453.84325173,135.73009515)
\curveto(453.84325173,137.63287149)(454.36061245,139.14328701)(455.39533389,140.26134172)
\curveto(456.43699977,141.37939643)(457.80505429,141.93842379)(459.49949745,141.93842379)
\curveto(461.06894071,141.93842379)(462.27727312,141.4800908)(463.12449471,140.56342483)
\curveto(463.97866073,139.64675886)(464.40574374,138.34467651)(464.40574374,136.65717778)
\closepath
\moveto(462.49949518,137.09467745)
\curveto(462.49255074,138.12245445)(462.23213427,138.91759274)(461.71824577,139.48009231)
\curveto(461.21130171,140.04259189)(460.43699674,140.32384168)(459.39533086,140.32384168)
\curveto(458.34672054,140.32384168)(457.50991562,140.01481413)(456.88491609,139.39675904)
\curveto(456.26686101,138.77870396)(455.91616683,138.01134343)(455.83283356,137.09467745)
\closepath
}
}
{
\newrgbcolor{curcolor}{0 0 0}
\pscustom[linestyle=none,fillstyle=solid,fillcolor=curcolor]
{
\newpath
\moveto(486.093229,143.65717249)
\lineto(480.55156653,143.65717249)
\lineto(480.55156653,129.9800995)
\lineto(478.48906809,129.9800995)
\lineto(478.48906809,143.65717249)
\lineto(472.94740561,143.65717249)
\lineto(472.94740561,145.49050443)
\lineto(486.093229,145.49050443)
\closepath
}
}
{
\newrgbcolor{curcolor}{0 0 0}
\pscustom[linestyle=none,fillstyle=solid,fillcolor=curcolor]
{
\newpath
\moveto(495.46821682,135.59467859)
\lineto(486.89530664,135.59467859)
\curveto(486.89530664,134.87940135)(487.00294545,134.25440182)(487.21822307,133.71968001)
\curveto(487.43350068,133.19190263)(487.72863935,132.75787518)(488.10363906,132.41759766)
\curveto(488.4647499,132.08426458)(488.89183291,131.83426477)(489.38488809,131.66759823)
\curveto(489.88488771,131.50093168)(490.43349841,131.41759841)(491.03072018,131.41759841)
\curveto(491.82238625,131.41759841)(492.61752454,131.5738483)(493.41613504,131.88634806)
\curveto(494.22168999,132.20579226)(494.79460622,132.51829203)(495.13488374,132.82384735)
\lineto(495.23905033,132.82384735)
\lineto(495.23905033,130.6884323)
\curveto(494.57932861,130.41065473)(493.90571801,130.17801602)(493.21821853,129.99051616)
\curveto(492.53071905,129.8030163)(491.80849737,129.70926637)(491.0515535,129.70926637)
\curveto(489.1209994,129.70926637)(487.6140561,130.23009931)(486.53072359,131.27176519)
\curveto(485.44739107,132.32037551)(484.90572481,133.8064855)(484.90572481,135.73009515)
\curveto(484.90572481,137.63287149)(485.42308553,139.14328701)(486.45780697,140.26134172)
\curveto(487.49947285,141.37939643)(488.86752737,141.93842379)(490.56197054,141.93842379)
\curveto(492.13141379,141.93842379)(493.33974621,141.4800908)(494.18696779,140.56342483)
\curveto(495.04113381,139.64675886)(495.46821682,138.34467651)(495.46821682,136.65717778)
\closepath
\moveto(493.56196827,137.09467745)
\curveto(493.55502383,138.12245445)(493.29460736,138.91759274)(492.78071886,139.48009231)
\curveto(492.2737748,140.04259189)(491.49946983,140.32384168)(490.45780395,140.32384168)
\curveto(489.40919363,140.32384168)(488.57238871,140.01481413)(487.94738918,139.39675904)
\curveto(487.32933409,138.77870396)(486.97863991,138.01134343)(486.89530664,137.09467745)
\closepath
}
}
{
\newrgbcolor{curcolor}{0 0 0}
\pscustom[linestyle=none,fillstyle=solid,fillcolor=curcolor]
{
\newpath
\moveto(506.7494615,133.33426363)
\curveto(506.7494615,132.27176443)(506.30848961,131.40023732)(505.42654583,130.71968228)
\curveto(504.55154649,130.03912723)(503.35363073,129.69884971)(501.83279855,129.69884971)
\curveto(500.97168809,129.69884971)(500.18002202,129.79954408)(499.45780035,130.00093282)
\curveto(498.74252311,130.209266)(498.14182912,130.43496027)(497.65571838,130.67801564)
\lineto(497.65571838,132.87593064)
\lineto(497.75988496,132.87593064)
\curveto(498.37794005,132.41065322)(499.06543953,132.03912572)(499.8223834,131.76134815)
\curveto(500.57932727,131.49051503)(501.30502117,131.35509846)(501.99946509,131.35509846)
\curveto(502.86057555,131.35509846)(503.53418615,131.49398725)(504.02029689,131.77176481)
\curveto(504.50640764,132.04954238)(504.74946301,132.48704205)(504.74946301,133.08426382)
\curveto(504.74946301,133.54259681)(504.61751866,133.88981877)(504.35362997,134.1259297)
\curveto(504.08974129,134.36204063)(503.58279722,134.56342937)(502.83279779,134.73009591)
\curveto(502.55502022,134.79259586)(502.19043717,134.86551247)(501.73904862,134.94884574)
\curveto(501.29460451,135.03217901)(500.88835482,135.12245672)(500.52029954,135.21967887)
\curveto(499.49946698,135.490512)(498.77377309,135.88634503)(498.34321786,136.40717797)
\curveto(497.91960706,136.93495535)(497.70780167,137.5807882)(497.70780167,138.34467651)
\curveto(497.70780167,138.82384281)(497.80502382,139.27523136)(497.99946812,139.69884215)
\curveto(498.20085685,140.12245294)(498.50293996,140.50092488)(498.90571743,140.83425796)
\curveto(499.29460602,141.1606466)(499.78766121,141.41759085)(500.38488298,141.60509071)
\curveto(500.98904919,141.799535)(501.66265979,141.89675715)(502.40571478,141.89675715)
\curveto(503.1001587,141.89675715)(503.80154706,141.80995166)(504.50987986,141.63634068)
\curveto(505.22515709,141.46967414)(505.81890664,141.26481319)(506.29112851,141.02175781)
\lineto(506.29112851,138.9280094)
\lineto(506.18696192,138.9280094)
\curveto(505.6869623,139.29606468)(505.07932387,139.60509222)(504.36404663,139.85509203)
\curveto(503.6487694,140.11203628)(502.94738104,140.24050841)(502.25988156,140.24050841)
\curveto(501.54460432,140.24050841)(500.94043811,140.10161962)(500.44738293,139.82384205)
\curveto(499.95432775,139.55300893)(499.70780016,139.14675923)(499.70780016,138.60509298)
\curveto(499.70780016,138.12592667)(499.8571056,137.76481583)(500.15571648,137.52176046)
\curveto(500.44738293,137.27870509)(500.91960479,137.08078857)(501.57238208,136.92801091)
\curveto(501.93349292,136.84467764)(502.33627039,136.76134437)(502.7807145,136.6780111)
\curveto(503.23210305,136.59467783)(503.60710276,136.518289)(503.90571365,136.44884461)
\curveto(504.81543518,136.24051143)(505.51682354,135.88287281)(506.00987872,135.37592875)
\curveto(506.5029339,134.86204025)(506.7494615,134.18148521)(506.7494615,133.33426363)
\closepath
}
}
{
\newrgbcolor{curcolor}{0 0 0}
\pscustom[linestyle=none,fillstyle=solid,fillcolor=curcolor]
{
\newpath
\moveto(515.59320392,130.08426609)
\curveto(515.22514864,129.98704394)(514.82237117,129.90718289)(514.3848715,129.84468294)
\curveto(513.95431627,129.78218298)(513.5688999,129.75093301)(513.22862238,129.75093301)
\curveto(512.04112328,129.75093301)(511.13834618,130.07037721)(510.52029109,130.70926562)
\curveto(509.90223601,131.34815402)(509.59320846,132.3724588)(509.59320846,133.78217996)
\lineto(509.59320846,139.96967528)
\lineto(508.2702928,139.96967528)
\lineto(508.2702928,141.61550737)
\lineto(509.59320846,141.61550737)
\lineto(509.59320846,144.95925484)
\lineto(511.55154031,144.95925484)
\lineto(511.55154031,141.61550737)
\lineto(515.59320392,141.61550737)
\lineto(515.59320392,139.96967528)
\lineto(511.55154031,139.96967528)
\lineto(511.55154031,134.66759596)
\curveto(511.55154031,134.05648531)(511.56542919,133.577319)(511.59320695,133.23009704)
\curveto(511.6209847,132.88981952)(511.71820685,132.57037532)(511.88487339,132.27176443)
\curveto(512.03765106,131.99398687)(512.24598423,131.78912591)(512.50987292,131.65718157)
\curveto(512.78070605,131.53218166)(513.19042796,131.46968171)(513.73903866,131.46968171)
\curveto(514.05848286,131.46968171)(514.39181594,131.51482056)(514.7390379,131.60509827)
\curveto(515.08625986,131.70232042)(515.33625967,131.78218147)(515.48903733,131.84468142)
\lineto(515.59320392,131.84468142)
\closepath
}
}
{
\newrgbcolor{curcolor}{0 0 0}
\pscustom[linestyle=none,fillstyle=solid,fillcolor=curcolor]
{
\newpath
\moveto(531.71819607,144.29258867)
\lineto(531.61402948,144.29258867)
\curveto(531.39875186,144.35508863)(531.11750208,144.41758858)(530.77028012,144.48008853)
\curveto(530.42305816,144.54953292)(530.11750283,144.58425512)(529.85361414,144.58425512)
\curveto(529.013337,144.58425512)(528.40222635,144.39675526)(528.0202822,144.02175555)
\curveto(527.64528248,143.65370027)(527.45778262,142.98356189)(527.45778262,142.0113404)
\lineto(527.45778262,141.61550737)
\lineto(530.98902995,141.61550737)
\lineto(530.98902995,139.96967528)
\lineto(527.52028258,139.96967528)
\lineto(527.52028258,129.9800995)
\lineto(525.56195072,129.9800995)
\lineto(525.56195072,139.96967528)
\lineto(524.23903506,139.96967528)
\lineto(524.23903506,141.61550737)
\lineto(525.56195072,141.61550737)
\lineto(525.56195072,142.00092374)
\curveto(525.56195072,143.38286714)(525.90570046,144.44189412)(526.59319994,145.17800467)
\curveto(527.28069942,145.92105966)(528.27375423,146.29258716)(529.57236436,146.29258716)
\curveto(530.00986403,146.29258716)(530.40222484,146.27175384)(530.7494468,146.23008721)
\curveto(531.1036132,146.18842057)(531.42652962,146.1398095)(531.71819607,146.08425398)
\closepath
}
}
{
\newrgbcolor{curcolor}{0 0 0}
\pscustom[linestyle=none,fillstyle=solid,fillcolor=curcolor]
{
\newpath
\moveto(535.06193479,143.56342256)
\lineto(532.85360313,143.56342256)
\lineto(532.85360313,145.59467102)
\lineto(535.06193479,145.59467102)
\closepath
\moveto(534.93693489,129.9800995)
\lineto(532.97860304,129.9800995)
\lineto(532.97860304,141.61550737)
\lineto(534.93693489,141.61550737)
\closepath
}
}
{
\newrgbcolor{curcolor}{0 0 0}
\pscustom[linestyle=none,fillstyle=solid,fillcolor=curcolor]
{
\newpath
\moveto(540.79110508,129.9800995)
\lineto(538.83277323,129.9800995)
\lineto(538.83277323,146.18842057)
\lineto(540.79110508,146.18842057)
\closepath
}
}
{
\newrgbcolor{curcolor}{0 0 0}
\pscustom[linestyle=none,fillstyle=solid,fillcolor=curcolor]
{
\newpath
\moveto(554.40567165,135.59467859)
\lineto(545.83276147,135.59467859)
\curveto(545.83276147,134.87940135)(545.94040027,134.25440182)(546.15567789,133.71968001)
\curveto(546.3709555,133.19190263)(546.66609417,132.75787518)(547.04109389,132.41759766)
\curveto(547.40220472,132.08426458)(547.82928773,131.83426477)(548.32234292,131.66759823)
\curveto(548.82234254,131.50093168)(549.37095323,131.41759841)(549.96817501,131.41759841)
\curveto(550.75984107,131.41759841)(551.55497936,131.5738483)(552.35358987,131.88634806)
\curveto(553.15914481,132.20579226)(553.73206105,132.51829203)(554.07233857,132.82384735)
\lineto(554.17650515,132.82384735)
\lineto(554.17650515,130.6884323)
\curveto(553.51678343,130.41065473)(552.84317283,130.17801602)(552.15567335,129.99051616)
\curveto(551.46817387,129.8030163)(550.74595219,129.70926637)(549.98900832,129.70926637)
\curveto(548.05845423,129.70926637)(546.55151092,130.23009931)(545.46817841,131.27176519)
\curveto(544.3848459,132.32037551)(543.84317964,133.8064855)(543.84317964,135.73009515)
\curveto(543.84317964,137.63287149)(544.36054036,139.14328701)(545.3952618,140.26134172)
\curveto(546.43692768,141.37939643)(547.8049822,141.93842379)(549.49942536,141.93842379)
\curveto(551.06886862,141.93842379)(552.27720104,141.4800908)(553.12442262,140.56342483)
\curveto(553.97858864,139.64675886)(554.40567165,138.34467651)(554.40567165,136.65717778)
\closepath
\moveto(552.49942309,137.09467745)
\curveto(552.49247865,138.12245445)(552.23206218,138.91759274)(551.71817368,139.48009231)
\curveto(551.21122962,140.04259189)(550.43692465,140.32384168)(549.39525877,140.32384168)
\curveto(548.34664845,140.32384168)(547.50984353,140.01481413)(546.884844,139.39675904)
\curveto(546.26678892,138.77870396)(545.91609474,138.01134343)(545.83276147,137.09467745)
\closepath
}
}
{
\newrgbcolor{curcolor}{0 0 0}
\pscustom[linestyle=none,fillstyle=solid,fillcolor=curcolor]
{
\newpath
\moveto(387.53080274,109.15719859)
\lineto(381.03080766,109.15719859)
\lineto(381.03080766,111.04261383)
\lineto(387.53080274,111.04261383)
\closepath
}
}
{
\newrgbcolor{curcolor}{0 0 0}
\pscustom[linestyle=none,fillstyle=solid,fillcolor=curcolor]
{
\newpath
\moveto(411.54120148,103.31345301)
\lineto(408.86412017,103.31345301)
\lineto(403.67662409,109.48011501)
\lineto(400.77037629,109.48011501)
\lineto(400.77037629,103.31345301)
\lineto(398.70787785,103.31345301)
\lineto(398.70787785,118.82385794)
\lineto(403.05162457,118.82385794)
\curveto(403.98912386,118.82385794)(404.77037327,118.76135799)(405.39537279,118.63635808)
\curveto(406.02037232,118.51830262)(406.5828719,118.303025)(407.08287152,117.99052524)
\curveto(407.64537109,117.63635884)(408.08287076,117.18844251)(408.39537052,116.64677626)
\curveto(408.71481473,116.11205444)(408.87453683,115.4314994)(408.87453683,114.60511113)
\curveto(408.87453683,113.48705642)(408.59328704,112.54955713)(408.03078747,111.79261326)
\curveto(407.46828789,111.04261383)(406.69398292,110.47664203)(405.70787256,110.09469788)
\closepath
\moveto(406.71828846,114.45927791)
\curveto(406.71828846,114.90372202)(406.63842741,115.29608283)(406.47870531,115.63636035)
\curveto(406.32592765,115.98358231)(406.0689834,116.27524876)(405.70787256,116.51135969)
\curveto(405.40926167,116.71274843)(405.05509527,116.85163721)(404.64537336,116.92802604)
\curveto(404.23565145,117.01135931)(403.75301293,117.05302595)(403.19745779,117.05302595)
\lineto(400.77037629,117.05302595)
\lineto(400.77037629,111.19886371)
\lineto(402.85370805,111.19886371)
\curveto(403.50648533,111.19886371)(404.07592935,111.25441922)(404.56204009,111.36553025)
\curveto(405.04815084,111.48358572)(405.46134497,111.69886333)(405.80162249,112.0113631)
\curveto(406.11412225,112.30302954)(406.34328874,112.63636262)(406.48912197,113.01136234)
\curveto(406.64189963,113.3933065)(406.71828846,113.87594502)(406.71828846,114.45927791)
\closepath
}
}
{
\newrgbcolor{curcolor}{0 0 0}
\pscustom[linestyle=none,fillstyle=solid,fillcolor=curcolor]
{
\newpath
\moveto(422.60369517,108.9280321)
\lineto(414.03078499,108.9280321)
\curveto(414.03078499,108.21275486)(414.13842379,107.58775533)(414.35370141,107.05303351)
\curveto(414.56897902,106.52525614)(414.86411769,106.09122869)(415.2391174,105.75095117)
\curveto(415.60022824,105.41761808)(416.02731125,105.16761827)(416.52036644,105.00095173)
\curveto(417.02036606,104.83428519)(417.56897675,104.75095192)(418.16619852,104.75095192)
\curveto(418.95786459,104.75095192)(419.75300288,104.9072018)(420.55161339,105.21970157)
\curveto(421.35716833,105.53914577)(421.93008456,105.85164553)(422.27036209,106.15720086)
\lineto(422.37452867,106.15720086)
\lineto(422.37452867,104.02178581)
\curveto(421.71480695,103.74400824)(421.04119635,103.51136953)(420.35369687,103.32386967)
\curveto(419.66619739,103.13636981)(418.94397571,103.04261988)(418.18703184,103.04261988)
\curveto(416.25647775,103.04261988)(414.74953444,103.56345282)(413.66620193,104.6051187)
\curveto(412.58286941,105.65372902)(412.04120316,107.139839)(412.04120316,109.06344866)
\curveto(412.04120316,110.966225)(412.55856388,112.47664052)(413.59328532,113.59469523)
\curveto(414.6349512,114.71274994)(416.00300572,115.2717773)(417.69744888,115.2717773)
\curveto(419.26689214,115.2717773)(420.47522455,114.81344431)(421.32244614,113.89677834)
\curveto(422.17661216,112.98011236)(422.60369517,111.67803001)(422.60369517,109.99053129)
\closepath
\moveto(420.69744661,110.42803096)
\curveto(420.69050217,111.45580796)(420.4300857,112.25094625)(419.9161972,112.81344582)
\curveto(419.40925314,113.3759454)(418.63494817,113.65719518)(417.59328229,113.65719518)
\curveto(416.54467197,113.65719518)(415.70786705,113.34816764)(415.08286752,112.73011255)
\curveto(414.46481243,112.11205746)(414.11411826,111.34469693)(414.03078499,110.42803096)
\closepath
}
}
{
\newrgbcolor{curcolor}{0 0 0}
\pscustom[linestyle=none,fillstyle=solid,fillcolor=curcolor]
{
\newpath
\moveto(435.8120145,109.21969854)
\curveto(435.8120145,108.24747705)(435.67312572,107.37247772)(435.39534815,106.59470053)
\curveto(435.12451502,105.81692334)(434.75645975,105.16414605)(434.29118232,104.63636868)
\curveto(433.79812714,104.08775798)(433.25646088,103.67456385)(432.66618355,103.39678628)
\curveto(432.07590622,103.12595315)(431.42660116,102.99053659)(430.71826836,102.99053659)
\curveto(430.05854664,102.99053659)(429.48215818,103.07039764)(428.989103,103.23011974)
\curveto(428.49604782,103.3828974)(428.00993707,103.59123058)(427.53077077,103.85511927)
\lineto(427.40577086,103.31345301)
\lineto(425.57243892,103.31345301)
\lineto(425.57243892,119.52177408)
\lineto(427.53077077,119.52177408)
\lineto(427.53077077,113.7301118)
\curveto(428.07938147,114.18150034)(428.66271436,114.54955562)(429.28076945,114.83427763)
\curveto(429.89882453,115.12594407)(430.59326845,115.2717773)(431.3641012,115.2717773)
\curveto(432.73910016,115.2717773)(433.82243268,114.74399992)(434.61409874,113.68844516)
\curveto(435.41270925,112.6328904)(435.8120145,111.1433082)(435.8120145,109.21969854)
\closepath
\moveto(433.7911827,109.16761525)
\curveto(433.7911827,110.55650309)(433.56201621,111.60858562)(433.10368322,112.32386286)
\curveto(432.64535023,113.04608454)(431.90576746,113.40719537)(430.8849349,113.40719537)
\curveto(430.31549089,113.40719537)(429.73910243,113.28219547)(429.15576954,113.03219566)
\curveto(428.57243665,112.78914029)(428.03077039,112.4731683)(427.53077077,112.08427971)
\lineto(427.53077077,105.41761808)
\curveto(428.08632591,105.16761827)(428.56201999,104.99400729)(428.95785302,104.89678515)
\curveto(429.3606305,104.799563)(429.81549126,104.75095192)(430.32243532,104.75095192)
\curveto(431.40576784,104.75095192)(432.25298942,105.10511832)(432.86410007,105.81345112)
\curveto(433.48215516,106.52872836)(433.7911827,107.64678306)(433.7911827,109.16761525)
\closepath
}
}
{
\newrgbcolor{curcolor}{0 0 0}
\pscustom[linestyle=none,fillstyle=solid,fillcolor=curcolor]
{
\newpath
\moveto(448.78075289,109.12594861)
\curveto(448.78075289,107.23011671)(448.29464215,105.73359007)(447.32242066,104.63636868)
\curveto(446.35019918,103.53914728)(445.04811683,102.99053659)(443.41617362,102.99053659)
\curveto(441.77034153,102.99053659)(440.46131474,103.53914728)(439.48909326,104.63636868)
\curveto(438.52381621,105.73359007)(438.04117769,107.23011671)(438.04117769,109.12594861)
\curveto(438.04117769,111.02178051)(438.52381621,112.51830716)(439.48909326,113.61552855)
\curveto(440.46131474,114.71969438)(441.77034153,115.2717773)(443.41617362,115.2717773)
\curveto(445.04811683,115.2717773)(446.35019918,114.71969438)(447.32242066,113.61552855)
\curveto(448.29464215,112.51830716)(448.78075289,111.02178051)(448.78075289,109.12594861)
\closepath
\moveto(446.75992109,109.12594861)
\curveto(446.75992109,110.63289192)(446.46478242,111.75094663)(445.87450509,112.48011274)
\curveto(445.28422776,113.2162233)(444.46478394,113.58427857)(443.41617362,113.58427857)
\curveto(442.35367442,113.58427857)(441.52728616,113.2162233)(440.93700883,112.48011274)
\curveto(440.35367594,111.75094663)(440.06200949,110.63289192)(440.06200949,109.12594861)
\curveto(440.06200949,107.66761638)(440.35714816,106.55997833)(440.94742549,105.80303446)
\curveto(441.53770282,105.05303503)(442.36061886,104.67803531)(443.41617362,104.67803531)
\curveto(444.4578395,104.67803531)(445.2738111,105.04956281)(445.86408843,105.7926178)
\curveto(446.4613102,106.54261723)(446.75992109,107.6537275)(446.75992109,109.12594861)
\closepath
}
}
{
\newrgbcolor{curcolor}{0 0 0}
\pscustom[linestyle=none,fillstyle=solid,fillcolor=curcolor]
{
\newpath
\moveto(461.72865963,109.12594861)
\curveto(461.72865963,107.23011671)(461.24254889,105.73359007)(460.2703274,104.63636868)
\curveto(459.29810591,103.53914728)(457.99602356,102.99053659)(456.36408035,102.99053659)
\curveto(454.71824827,102.99053659)(453.40922148,103.53914728)(452.43699999,104.63636868)
\curveto(451.47172294,105.73359007)(450.98908442,107.23011671)(450.98908442,109.12594861)
\curveto(450.98908442,111.02178051)(451.47172294,112.51830716)(452.43699999,113.61552855)
\curveto(453.40922148,114.71969438)(454.71824827,115.2717773)(456.36408035,115.2717773)
\curveto(457.99602356,115.2717773)(459.29810591,114.71969438)(460.2703274,113.61552855)
\curveto(461.24254889,112.51830716)(461.72865963,111.02178051)(461.72865963,109.12594861)
\closepath
\moveto(459.70782782,109.12594861)
\curveto(459.70782782,110.63289192)(459.41268916,111.75094663)(458.82241183,112.48011274)
\curveto(458.2321345,113.2162233)(457.41269067,113.58427857)(456.36408035,113.58427857)
\curveto(455.30158116,113.58427857)(454.47519289,113.2162233)(453.88491556,112.48011274)
\curveto(453.30158267,111.75094663)(453.00991622,110.63289192)(453.00991622,109.12594861)
\curveto(453.00991622,107.66761638)(453.30505489,106.55997833)(453.89533222,105.80303446)
\curveto(454.48560955,105.05303503)(455.3085256,104.67803531)(456.36408035,104.67803531)
\curveto(457.40574623,104.67803531)(458.22171784,105.04956281)(458.81199517,105.7926178)
\curveto(459.40921694,106.54261723)(459.70782782,107.6537275)(459.70782782,109.12594861)
\closepath
}
}
{
\newrgbcolor{curcolor}{0 0 0}
\pscustom[linestyle=none,fillstyle=solid,fillcolor=curcolor]
{
\newpath
\moveto(470.80156929,103.4176196)
\curveto(470.43351402,103.32039745)(470.03073654,103.2405364)(469.59323688,103.17803645)
\curveto(469.16268165,103.11553649)(468.77726527,103.08428652)(468.43698775,103.08428652)
\curveto(467.24948865,103.08428652)(466.34671155,103.40373072)(465.72865647,104.04261912)
\curveto(465.11060138,104.68150753)(464.80157383,105.70581231)(464.80157383,107.11553347)
\lineto(464.80157383,113.30302879)
\lineto(463.47865817,113.30302879)
\lineto(463.47865817,114.94886087)
\lineto(464.80157383,114.94886087)
\lineto(464.80157383,118.29260834)
\lineto(466.75990569,118.29260834)
\lineto(466.75990569,114.94886087)
\lineto(470.80156929,114.94886087)
\lineto(470.80156929,113.30302879)
\lineto(466.75990569,113.30302879)
\lineto(466.75990569,108.00094946)
\curveto(466.75990569,107.38983881)(466.77379456,106.91067251)(466.80157232,106.56345055)
\curveto(466.82935008,106.22317303)(466.92657223,105.90372883)(467.09323877,105.60511794)
\curveto(467.24601643,105.32734038)(467.4543496,105.12247942)(467.71823829,104.99053507)
\curveto(467.98907142,104.86553517)(468.39879333,104.80303522)(468.94740403,104.80303522)
\curveto(469.26684823,104.80303522)(469.60018131,104.84817407)(469.94740327,104.93845178)
\curveto(470.29462523,105.03567393)(470.54462504,105.11553498)(470.69740271,105.17803493)
\lineto(470.80156929,105.17803493)
\closepath
}
}
{
\newrgbcolor{curcolor}{0 0 0}
\pscustom[linestyle=none,fillstyle=solid,fillcolor=curcolor]
{
\newpath
\moveto(497.64530333,103.31345301)
\lineto(495.68697148,103.31345301)
\lineto(495.68697148,109.938448)
\curveto(495.68697148,110.43844762)(495.66266594,110.92108614)(495.61405487,111.38636357)
\curveto(495.57238823,111.85164099)(495.4786383,112.22316849)(495.33280508,112.50094606)
\curveto(495.17308298,112.79955694)(494.94391649,113.02525122)(494.6453056,113.17802888)
\curveto(494.34669471,113.33080654)(493.91613949,113.40719537)(493.35363991,113.40719537)
\curveto(492.80502921,113.40719537)(492.25641852,113.26830659)(491.70780782,112.99052902)
\curveto(491.15919713,112.71969589)(490.61058643,112.37247393)(490.06197573,111.94886314)
\curveto(490.08280905,111.78914104)(490.10017015,111.60164118)(490.11405903,111.38636357)
\curveto(490.12794791,111.17803039)(490.13489235,110.96969722)(490.13489235,110.76136404)
\lineto(490.13489235,103.31345301)
\lineto(488.17656049,103.31345301)
\lineto(488.17656049,109.938448)
\curveto(488.17656049,110.4523365)(488.15225496,110.93844724)(488.10364388,111.39678023)
\curveto(488.06197725,111.86205765)(487.96822732,112.23358515)(487.8223941,112.51136272)
\curveto(487.66267199,112.8099736)(487.4335055,113.03219566)(487.13489462,113.17802888)
\curveto(486.83628373,113.33080654)(486.4057285,113.40719537)(485.84322893,113.40719537)
\curveto(485.30850711,113.40719537)(484.77031307,113.27525103)(484.22864681,113.01136234)
\curveto(483.693925,112.74747365)(483.15920318,112.41066835)(482.62448136,112.00094644)
\lineto(482.62448136,103.31345301)
\lineto(480.66614951,103.31345301)
\lineto(480.66614951,114.94886087)
\lineto(482.62448136,114.94886087)
\lineto(482.62448136,113.65719518)
\curveto(483.23559201,114.16413925)(483.84323044,114.55997228)(484.44739665,114.84469429)
\curveto(485.0585073,115.12941629)(485.70781236,115.2717773)(486.39531184,115.2717773)
\curveto(487.18697791,115.2717773)(487.85711629,115.10511076)(488.40572699,114.77177767)
\curveto(488.96128212,114.43844459)(489.37447625,113.97663939)(489.64530938,113.38636206)
\curveto(490.43697545,114.05302822)(491.15919713,114.53219452)(491.81197441,114.82386097)
\curveto(492.46475169,115.12247185)(493.16266783,115.2717773)(493.90572283,115.2717773)
\curveto(495.18349964,115.2717773)(496.12447115,114.8828887)(496.72863736,114.10511151)
\curveto(497.33974801,113.33427876)(497.64530333,112.25441847)(497.64530333,110.86553063)
\closepath
}
}
{
\newrgbcolor{curcolor}{0 0 0}
\pscustom[linestyle=none,fillstyle=solid,fillcolor=curcolor]
{
\newpath
\moveto(510.45779684,103.31345301)
\lineto(508.50988165,103.31345301)
\lineto(508.50988165,104.55303541)
\curveto(508.33627067,104.43497994)(508.10015974,104.2683134)(507.80154885,104.05303578)
\curveto(507.5098824,103.84470261)(507.2251604,103.67803607)(506.94738283,103.55303616)
\curveto(506.62099419,103.39331406)(506.24599447,103.26136972)(505.82238368,103.15720313)
\curveto(505.39877289,103.0460921)(504.90224549,102.99053659)(504.33280147,102.99053659)
\curveto(503.28419116,102.99053659)(502.39530294,103.33775855)(501.66613682,104.03220247)
\curveto(500.93697071,104.72664639)(500.57238765,105.61206238)(500.57238765,106.68845046)
\curveto(500.57238765,107.57039423)(500.75988751,108.28219925)(501.13488723,108.82386551)
\curveto(501.51683138,109.3724762)(502.05849764,109.80303143)(502.759886,110.1155312)
\curveto(503.46821879,110.42803096)(504.3189126,110.63983636)(505.3119674,110.75094738)
\curveto(506.3050222,110.86205841)(507.37099362,110.94539168)(508.50988165,111.00094719)
\lineto(508.50988165,111.3030303)
\curveto(508.50988165,111.74747441)(508.4300206,112.11552968)(508.2702985,112.40719613)
\curveto(508.11752083,112.69886258)(507.89529878,112.92802907)(507.60363233,113.09469561)
\curveto(507.32585477,113.25441771)(506.99252168,113.36205652)(506.60363309,113.41761203)
\curveto(506.21474449,113.47316755)(505.8084948,113.5009453)(505.38488401,113.5009453)
\curveto(504.87099551,113.5009453)(504.29807928,113.43150091)(503.66613531,113.29261213)
\curveto(503.03419135,113.16066778)(502.38141406,112.96622348)(501.70780346,112.70927923)
\lineto(501.60363687,112.70927923)
\lineto(501.60363687,114.69886106)
\curveto(501.98558103,114.80302765)(502.53766394,114.9176109)(503.25988562,115.0426108)
\curveto(503.9821073,115.16761071)(504.69391231,115.23011066)(505.39530067,115.23011066)
\curveto(506.21474449,115.23011066)(506.92654951,115.16066627)(507.53071572,115.02177749)
\curveto(508.14182637,114.88983314)(508.66960375,114.66066665)(509.11404786,114.33427801)
\curveto(509.55154753,114.0148338)(509.88488061,113.60163967)(510.1140471,113.09469561)
\curveto(510.34321359,112.58775155)(510.45779684,111.9592798)(510.45779684,111.20928037)
\closepath
\moveto(508.50988165,106.17803418)
\lineto(508.50988165,109.41761506)
\curveto(507.91265988,109.38289286)(507.2077993,109.33080957)(506.39529991,109.26136518)
\curveto(505.58974497,109.19192078)(504.95085656,109.09122642)(504.4786347,108.95928207)
\curveto(503.91613512,108.79955997)(503.46127436,108.54956016)(503.1140524,108.20928264)
\curveto(502.76683044,107.87594956)(502.59321946,107.41414435)(502.59321946,106.82386702)
\curveto(502.59321946,106.15720086)(502.79460819,105.65372902)(503.19738567,105.3134515)
\curveto(503.60016314,104.98011842)(504.21474601,104.81345188)(505.04113427,104.81345188)
\curveto(505.72863375,104.81345188)(506.3571055,104.94539622)(506.92654951,105.20928491)
\curveto(507.49599353,105.48011804)(508.0237709,105.80303446)(508.50988165,106.17803418)
\closepath
}
}
{
\newrgbcolor{curcolor}{0 0 0}
\pscustom[linestyle=none,fillstyle=solid,fillcolor=curcolor]
{
\newpath
\moveto(522.83277795,104.04261912)
\curveto(522.18000066,103.73011936)(521.55847336,103.48706399)(520.96819602,103.31345301)
\curveto(520.38486313,103.13984203)(519.76333583,103.05303654)(519.1036141,103.05303654)
\curveto(518.26333696,103.05303654)(517.49250421,103.17456423)(516.79111585,103.4176196)
\curveto(516.08972749,103.66761941)(515.4890335,104.04261912)(514.98903388,104.54261875)
\curveto(514.48208982,105.04261837)(514.08972901,105.67456233)(513.81195144,106.43845065)
\curveto(513.53417387,107.20233896)(513.39528509,108.09469939)(513.39528509,109.11553195)
\curveto(513.39528509,111.01830829)(513.91611803,112.51136272)(514.95778391,113.59469523)
\curveto(516.00639422,114.67802775)(517.38833762,115.219694)(519.1036141,115.219694)
\curveto(519.77028026,115.219694)(520.42305755,115.12594407)(521.06194595,114.93844421)
\curveto(521.7077788,114.75094436)(522.29805613,114.52177786)(522.83277795,114.25094474)
\lineto(522.83277795,112.07386305)
\lineto(522.72861136,112.07386305)
\curveto(522.13138959,112.53914047)(521.5133345,112.89677909)(520.8744461,113.1467789)
\curveto(520.24250213,113.39677871)(519.62444704,113.52177862)(519.02028083,113.52177862)
\curveto(517.90917056,113.52177862)(517.030699,113.1467789)(516.38486616,112.39677947)
\curveto(515.74597775,111.65372448)(515.42653355,110.55997531)(515.42653355,109.11553195)
\curveto(515.42653355,107.71275524)(515.73903331,106.63289494)(516.36403284,105.87595107)
\curveto(516.99597681,105.12595164)(517.8813928,104.75095192)(519.02028083,104.75095192)
\curveto(519.41611387,104.75095192)(519.81889134,104.80303522)(520.22861325,104.9072018)
\curveto(520.63833516,105.01136839)(521.00639044,105.14678496)(521.33277908,105.3134515)
\curveto(521.61750109,105.45928472)(521.884862,105.61206238)(522.13486181,105.77178448)
\curveto(522.38486162,105.93845102)(522.58277814,106.08081203)(522.72861136,106.19886749)
\lineto(522.83277795,106.19886749)
\closepath
}
}
{
\newrgbcolor{curcolor}{0 0 0}
\pscustom[linestyle=none,fillstyle=solid,fillcolor=curcolor]
{
\newpath
\moveto(535.07235114,103.31345301)
\lineto(533.11401929,103.31345301)
\lineto(533.11401929,109.938448)
\curveto(533.11401929,110.47316982)(533.08276931,110.97316944)(533.02026936,111.43844686)
\curveto(532.9577694,111.91066873)(532.84318616,112.278724)(532.67651962,112.54261269)
\curveto(532.50290864,112.83427914)(532.25290883,113.04955676)(531.92652018,113.18844554)
\curveto(531.60013154,113.33427876)(531.17652075,113.40719537)(530.65568781,113.40719537)
\curveto(530.120966,113.40719537)(529.56193864,113.27525103)(528.97860575,113.01136234)
\curveto(528.39527286,112.74747365)(527.8362455,112.41066835)(527.30152368,112.00094644)
\lineto(527.30152368,103.31345301)
\lineto(525.34319183,103.31345301)
\lineto(525.34319183,119.52177408)
\lineto(527.30152368,119.52177408)
\lineto(527.30152368,113.65719518)
\curveto(527.91263433,114.16413925)(528.5445783,114.55997228)(529.19735558,114.84469429)
\curveto(529.85013287,115.12941629)(530.52027125,115.2717773)(531.20777073,115.2717773)
\curveto(532.46471422,115.2717773)(533.42304683,114.89330536)(534.08276855,114.13636149)
\curveto(534.74249028,113.37941762)(535.07235114,112.28914066)(535.07235114,110.86553063)
\closepath
}
}
{
\newrgbcolor{curcolor}{0 0 0}
\pscustom[linestyle=none,fillstyle=solid,fillcolor=curcolor]
{
\newpath
\moveto(540.94734826,116.89677607)
\lineto(538.73901659,116.89677607)
\lineto(538.73901659,118.92802453)
\lineto(540.94734826,118.92802453)
\closepath
\moveto(540.82234835,103.31345301)
\lineto(538.8640165,103.31345301)
\lineto(538.8640165,114.94886087)
\lineto(540.82234835,114.94886087)
\closepath
}
}
{
\newrgbcolor{curcolor}{0 0 0}
\pscustom[linestyle=none,fillstyle=solid,fillcolor=curcolor]
{
\newpath
\moveto(554.42651268,103.31345301)
\lineto(552.46818083,103.31345301)
\lineto(552.46818083,109.938448)
\curveto(552.46818083,110.47316982)(552.43693085,110.97316944)(552.3744309,111.43844686)
\curveto(552.31193095,111.91066873)(552.1973477,112.278724)(552.03068116,112.54261269)
\curveto(551.85707018,112.83427914)(551.60707037,113.04955676)(551.28068173,113.18844554)
\curveto(550.95429308,113.33427876)(550.53068229,113.40719537)(550.00984935,113.40719537)
\curveto(549.47512754,113.40719537)(548.91610018,113.27525103)(548.33276729,113.01136234)
\curveto(547.7494344,112.74747365)(547.19040704,112.41066835)(546.65568522,112.00094644)
\lineto(546.65568522,103.31345301)
\lineto(544.69735337,103.31345301)
\lineto(544.69735337,114.94886087)
\lineto(546.65568522,114.94886087)
\lineto(546.65568522,113.65719518)
\curveto(547.26679587,114.16413925)(547.89873984,114.55997228)(548.55151712,114.84469429)
\curveto(549.20429441,115.12941629)(549.87443279,115.2717773)(550.56193227,115.2717773)
\curveto(551.81887576,115.2717773)(552.77720837,114.89330536)(553.43693009,114.13636149)
\curveto(554.09665182,113.37941762)(554.42651268,112.28914066)(554.42651268,110.86553063)
\closepath
}
}
{
\newrgbcolor{curcolor}{0 0 0}
\pscustom[linestyle=none,fillstyle=solid,fillcolor=curcolor]
{
\newpath
\moveto(567.93692069,108.9280321)
\lineto(559.36401051,108.9280321)
\curveto(559.36401051,108.21275486)(559.47164931,107.58775533)(559.68692693,107.05303351)
\curveto(559.90220454,106.52525614)(560.19734321,106.09122869)(560.57234293,105.75095117)
\curveto(560.93345376,105.41761808)(561.36053677,105.16761827)(561.85359196,105.00095173)
\curveto(562.35359158,104.83428519)(562.90220227,104.75095192)(563.49942404,104.75095192)
\curveto(564.29109011,104.75095192)(565.0862284,104.9072018)(565.88483891,105.21970157)
\curveto(566.69039385,105.53914577)(567.26331009,105.85164553)(567.60358761,106.15720086)
\lineto(567.70775419,106.15720086)
\lineto(567.70775419,104.02178581)
\curveto(567.04803247,103.74400824)(566.37442187,103.51136953)(565.68692239,103.32386967)
\curveto(564.99942291,103.13636981)(564.27720123,103.04261988)(563.52025736,103.04261988)
\curveto(561.58970327,103.04261988)(560.08275996,103.56345282)(558.99942745,104.6051187)
\curveto(557.91609493,105.65372902)(557.37442868,107.139839)(557.37442868,109.06344866)
\curveto(557.37442868,110.966225)(557.8917894,112.47664052)(558.92651084,113.59469523)
\curveto(559.96817672,114.71274994)(561.33623124,115.2717773)(563.0306744,115.2717773)
\curveto(564.60011766,115.2717773)(565.80845007,114.81344431)(566.65567166,113.89677834)
\curveto(567.50983768,112.98011236)(567.93692069,111.67803001)(567.93692069,109.99053129)
\closepath
\moveto(566.03067213,110.42803096)
\curveto(566.02372769,111.45580796)(565.76331122,112.25094625)(565.24942272,112.81344582)
\curveto(564.74247866,113.3759454)(563.96817369,113.65719518)(562.92650781,113.65719518)
\curveto(561.87789749,113.65719518)(561.04109257,113.34816764)(560.41609304,112.73011255)
\curveto(559.79803796,112.11205746)(559.44734378,111.34469693)(559.36401051,110.42803096)
\closepath
}
}
{
\newrgbcolor{curcolor}{0 0 0}
\pscustom[linestyle=none,fillstyle=solid,fillcolor=curcolor]
{
\newpath
\moveto(584.66606916,117.62594218)
\lineto(584.56190257,117.62594218)
\curveto(584.34662496,117.68844213)(584.06537517,117.75094209)(583.71815321,117.81344204)
\curveto(583.37093125,117.88288643)(583.06537593,117.91760863)(582.80148724,117.91760863)
\curveto(581.9612101,117.91760863)(581.35009945,117.73010877)(580.96815529,117.35510905)
\curveto(580.59315557,116.98705378)(580.40565572,116.31691539)(580.40565572,115.34469391)
\lineto(580.40565572,114.94886087)
\lineto(583.93690305,114.94886087)
\lineto(583.93690305,113.30302879)
\lineto(580.46815567,113.30302879)
\lineto(580.46815567,103.31345301)
\lineto(578.50982382,103.31345301)
\lineto(578.50982382,113.30302879)
\lineto(577.18690815,113.30302879)
\lineto(577.18690815,114.94886087)
\lineto(578.50982382,114.94886087)
\lineto(578.50982382,115.33427725)
\curveto(578.50982382,116.71622065)(578.85357356,117.77524762)(579.54107304,118.51135818)
\curveto(580.22857252,119.25441317)(581.22162732,119.62594067)(582.52023745,119.62594067)
\curveto(582.95773712,119.62594067)(583.35009793,119.60510735)(583.69731989,119.56344072)
\curveto(584.05148629,119.52177408)(584.37440271,119.47316301)(584.66606916,119.41760749)
\closepath
}
}
{
\newrgbcolor{curcolor}{0 0 0}
\pscustom[linestyle=none,fillstyle=solid,fillcolor=curcolor]
{
\newpath
\moveto(595.82231639,109.12594861)
\curveto(595.82231639,107.23011671)(595.33620565,105.73359007)(594.36398416,104.63636868)
\curveto(593.39176268,103.53914728)(592.08968033,102.99053659)(590.45773712,102.99053659)
\curveto(588.81190503,102.99053659)(587.50287824,103.53914728)(586.53065676,104.63636868)
\curveto(585.56537971,105.73359007)(585.08274119,107.23011671)(585.08274119,109.12594861)
\curveto(585.08274119,111.02178051)(585.56537971,112.51830716)(586.53065676,113.61552855)
\curveto(587.50287824,114.71969438)(588.81190503,115.2717773)(590.45773712,115.2717773)
\curveto(592.08968033,115.2717773)(593.39176268,114.71969438)(594.36398416,113.61552855)
\curveto(595.33620565,112.51830716)(595.82231639,111.02178051)(595.82231639,109.12594861)
\closepath
\moveto(593.80148459,109.12594861)
\curveto(593.80148459,110.63289192)(593.50634592,111.75094663)(592.91606859,112.48011274)
\curveto(592.32579126,113.2162233)(591.50634744,113.58427857)(590.45773712,113.58427857)
\curveto(589.39523792,113.58427857)(588.56884966,113.2162233)(587.97857233,112.48011274)
\curveto(587.39523944,111.75094663)(587.10357299,110.63289192)(587.10357299,109.12594861)
\curveto(587.10357299,107.66761638)(587.39871166,106.55997833)(587.98898899,105.80303446)
\curveto(588.57926632,105.05303503)(589.40218236,104.67803531)(590.45773712,104.67803531)
\curveto(591.499403,104.67803531)(592.3153746,105.04956281)(592.90565193,105.7926178)
\curveto(593.5028737,106.54261723)(593.80148459,107.6537275)(593.80148459,109.12594861)
\closepath
}
}
{
\newrgbcolor{curcolor}{0 0 0}
\pscustom[linestyle=none,fillstyle=solid,fillcolor=curcolor]
{
\newpath
\moveto(606.11397514,112.81344582)
\lineto(606.00980855,112.81344582)
\curveto(605.7181421,112.88289021)(605.4334201,112.93150129)(605.15564253,112.95927905)
\curveto(604.8848094,112.99400124)(604.56189298,113.01136234)(604.18689326,113.01136234)
\curveto(603.58272705,113.01136234)(602.99939416,112.87594578)(602.43689459,112.60511265)
\curveto(601.87439501,112.34122396)(601.33272876,111.99747422)(600.81189582,111.57386343)
\lineto(600.81189582,103.31345301)
\lineto(598.85356396,103.31345301)
\lineto(598.85356396,114.94886087)
\lineto(600.81189582,114.94886087)
\lineto(600.81189582,113.23011217)
\curveto(601.58967301,113.8551117)(602.27370027,114.29608359)(602.8639776,114.55302784)
\curveto(603.46119937,114.81691653)(604.0688378,114.94886087)(604.68689288,114.94886087)
\curveto(605.0271704,114.94886087)(605.273698,114.93844421)(605.42647566,114.9176109)
\curveto(605.57925332,114.90372202)(605.80841981,114.87247204)(606.11397514,114.82386097)
\closepath
}
}
{
\newrgbcolor{curcolor}{0 0 0}
\pscustom[linestyle=none,fillstyle=solid,fillcolor=curcolor]
{
\newpath
\moveto(621.49937414,103.4176196)
\curveto(621.13131887,103.32039745)(620.72854139,103.2405364)(620.29104172,103.17803645)
\curveto(619.86048649,103.11553649)(619.47507012,103.08428652)(619.1347926,103.08428652)
\curveto(617.9472935,103.08428652)(617.0445164,103.40373072)(616.42646132,104.04261912)
\curveto(615.80840623,104.68150753)(615.49937868,105.70581231)(615.49937868,107.11553347)
\lineto(615.49937868,113.30302879)
\lineto(614.17646302,113.30302879)
\lineto(614.17646302,114.94886087)
\lineto(615.49937868,114.94886087)
\lineto(615.49937868,118.29260834)
\lineto(617.45771053,118.29260834)
\lineto(617.45771053,114.94886087)
\lineto(621.49937414,114.94886087)
\lineto(621.49937414,113.30302879)
\lineto(617.45771053,113.30302879)
\lineto(617.45771053,108.00094946)
\curveto(617.45771053,107.38983881)(617.47159941,106.91067251)(617.49937717,106.56345055)
\curveto(617.52715493,106.22317303)(617.62437708,105.90372883)(617.79104362,105.60511794)
\curveto(617.94382128,105.32734038)(618.15215445,105.12247942)(618.41604314,104.99053507)
\curveto(618.68687627,104.86553517)(619.09659818,104.80303522)(619.64520888,104.80303522)
\curveto(619.96465308,104.80303522)(620.29798616,104.84817407)(620.64520812,104.93845178)
\curveto(620.99243008,105.03567393)(621.24242989,105.11553498)(621.39520756,105.17803493)
\lineto(621.49937414,105.17803493)
\closepath
}
}
{
\newrgbcolor{curcolor}{0 0 0}
\pscustom[linestyle=none,fillstyle=solid,fillcolor=curcolor]
{
\newpath
\moveto(633.59311366,103.31345301)
\lineto(631.63478181,103.31345301)
\lineto(631.63478181,109.938448)
\curveto(631.63478181,110.47316982)(631.60353184,110.97316944)(631.54103188,111.43844686)
\curveto(631.47853193,111.91066873)(631.36394868,112.278724)(631.19728214,112.54261269)
\curveto(631.02367116,112.83427914)(630.77367135,113.04955676)(630.44728271,113.18844554)
\curveto(630.12089407,113.33427876)(629.69728328,113.40719537)(629.17645034,113.40719537)
\curveto(628.64172852,113.40719537)(628.08270117,113.27525103)(627.49936827,113.01136234)
\curveto(626.91603538,112.74747365)(626.35700803,112.41066835)(625.82228621,112.00094644)
\lineto(625.82228621,103.31345301)
\lineto(623.86395436,103.31345301)
\lineto(623.86395436,119.52177408)
\lineto(625.82228621,119.52177408)
\lineto(625.82228621,113.65719518)
\curveto(626.43339686,114.16413925)(627.06534083,114.55997228)(627.71811811,114.84469429)
\curveto(628.37089539,115.12941629)(629.04103378,115.2717773)(629.72853325,115.2717773)
\curveto(630.98547675,115.2717773)(631.94380936,114.89330536)(632.60353108,114.13636149)
\curveto(633.2632528,113.37941762)(633.59311366,112.28914066)(633.59311366,110.86553063)
\closepath
}
}
{
\newrgbcolor{curcolor}{0 0 0}
\pscustom[linestyle=none,fillstyle=solid,fillcolor=curcolor]
{
\newpath
\moveto(647.10350725,108.9280321)
\lineto(638.53059707,108.9280321)
\curveto(638.53059707,108.21275486)(638.63823588,107.58775533)(638.8535135,107.05303351)
\curveto(639.06879111,106.52525614)(639.36392978,106.09122869)(639.73892949,105.75095117)
\curveto(640.10004033,105.41761808)(640.52712334,105.16761827)(641.02017852,105.00095173)
\curveto(641.52017815,104.83428519)(642.06878884,104.75095192)(642.66601061,104.75095192)
\curveto(643.45767668,104.75095192)(644.25281497,104.9072018)(645.05142547,105.21970157)
\curveto(645.85698042,105.53914577)(646.42989665,105.85164553)(646.77017417,106.15720086)
\lineto(646.87434076,106.15720086)
\lineto(646.87434076,104.02178581)
\curveto(646.21461904,103.74400824)(645.54100844,103.51136953)(644.85350896,103.32386967)
\curveto(644.16600948,103.13636981)(643.4437878,103.04261988)(642.68684393,103.04261988)
\curveto(640.75628983,103.04261988)(639.24934653,103.56345282)(638.16601402,104.6051187)
\curveto(637.0826815,105.65372902)(636.54101525,107.139839)(636.54101525,109.06344866)
\curveto(636.54101525,110.966225)(637.05837597,112.47664052)(638.09309741,113.59469523)
\curveto(639.13476328,114.71274994)(640.5028178,115.2717773)(642.19726097,115.2717773)
\curveto(643.76670422,115.2717773)(644.97503664,114.81344431)(645.82225822,113.89677834)
\curveto(646.67642424,112.98011236)(647.10350725,111.67803001)(647.10350725,109.99053129)
\closepath
\moveto(645.1972587,110.42803096)
\curveto(645.19031426,111.45580796)(644.92989779,112.25094625)(644.41600929,112.81344582)
\curveto(643.90906523,113.3759454)(643.13476026,113.65719518)(642.09309438,113.65719518)
\curveto(641.04448406,113.65719518)(640.20767914,113.34816764)(639.58267961,112.73011255)
\curveto(638.96462452,112.11205746)(638.61393034,111.34469693)(638.53059707,110.42803096)
\closepath
}
}
{
\newrgbcolor{curcolor}{0 0 0}
\pscustom[linestyle=none,fillstyle=solid,fillcolor=curcolor]
{
\newpath
\moveto(667.30141752,103.31345301)
\lineto(665.34308567,103.31345301)
\lineto(665.34308567,109.938448)
\curveto(665.34308567,110.47316982)(665.31183569,110.97316944)(665.24933574,111.43844686)
\curveto(665.18683579,111.91066873)(665.07225254,112.278724)(664.905586,112.54261269)
\curveto(664.73197502,112.83427914)(664.48197521,113.04955676)(664.15558657,113.18844554)
\curveto(663.82919793,113.33427876)(663.40558714,113.40719537)(662.8847542,113.40719537)
\curveto(662.35003238,113.40719537)(661.79100502,113.27525103)(661.20767213,113.01136234)
\curveto(660.62433924,112.74747365)(660.06531188,112.41066835)(659.53059007,112.00094644)
\lineto(659.53059007,103.31345301)
\lineto(657.57225822,103.31345301)
\lineto(657.57225822,114.94886087)
\lineto(659.53059007,114.94886087)
\lineto(659.53059007,113.65719518)
\curveto(660.14170072,114.16413925)(660.77364468,114.55997228)(661.42642197,114.84469429)
\curveto(662.07919925,115.12941629)(662.74933763,115.2717773)(663.43683711,115.2717773)
\curveto(664.69378061,115.2717773)(665.65211321,114.89330536)(666.31183494,114.13636149)
\curveto(666.97155666,113.37941762)(667.30141752,112.28914066)(667.30141752,110.86553063)
\closepath
}
}
{
\newrgbcolor{curcolor}{0 0 0}
\pscustom[linestyle=none,fillstyle=solid,fillcolor=curcolor]
{
\newpath
\moveto(680.81183995,108.9280321)
\lineto(672.23892977,108.9280321)
\curveto(672.23892977,108.21275486)(672.34656857,107.58775533)(672.56184619,107.05303351)
\curveto(672.7771238,106.52525614)(673.07226247,106.09122869)(673.44726219,105.75095117)
\curveto(673.80837302,105.41761808)(674.23545603,105.16761827)(674.72851122,105.00095173)
\curveto(675.22851084,104.83428519)(675.77712153,104.75095192)(676.3743433,104.75095192)
\curveto(677.16600937,104.75095192)(677.96114766,104.9072018)(678.75975817,105.21970157)
\curveto(679.56531311,105.53914577)(680.13822935,105.85164553)(680.47850687,106.15720086)
\lineto(680.58267345,106.15720086)
\lineto(680.58267345,104.02178581)
\curveto(679.92295173,103.74400824)(679.24934113,103.51136953)(678.56184165,103.32386967)
\curveto(677.87434217,103.13636981)(677.15212049,103.04261988)(676.39517662,103.04261988)
\curveto(674.46462253,103.04261988)(672.95767922,103.56345282)(671.87434671,104.6051187)
\curveto(670.7910142,105.65372902)(670.24934794,107.139839)(670.24934794,109.06344866)
\curveto(670.24934794,110.966225)(670.76670866,112.47664052)(671.8014301,113.59469523)
\curveto(672.84309598,114.71274994)(674.2111505,115.2717773)(675.90559366,115.2717773)
\curveto(677.47503692,115.2717773)(678.68336934,114.81344431)(679.53059092,113.89677834)
\curveto(680.38475694,112.98011236)(680.81183995,111.67803001)(680.81183995,109.99053129)
\closepath
\moveto(678.90559139,110.42803096)
\curveto(678.89864695,111.45580796)(678.63823048,112.25094625)(678.12434198,112.81344582)
\curveto(677.61739792,113.3759454)(676.84309295,113.65719518)(675.80142707,113.65719518)
\curveto(674.75281675,113.65719518)(673.91601183,113.34816764)(673.2910123,112.73011255)
\curveto(672.67295722,112.11205746)(672.32226304,111.34469693)(672.23892977,110.42803096)
\closepath
}
}
{
\newrgbcolor{curcolor}{0 0 0}
\pscustom[linestyle=none,fillstyle=solid,fillcolor=curcolor]
{
\newpath
\moveto(693.85347112,103.31345301)
\lineto(691.38472299,103.31345301)
\lineto(688.08264215,107.78219963)
\lineto(684.759728,103.31345301)
\lineto(682.47847972,103.31345301)
\lineto(687.02014295,109.11553195)
\lineto(682.52014636,114.94886087)
\lineto(684.98889449,114.94886087)
\lineto(688.27014201,110.55303087)
\lineto(691.56180619,114.94886087)
\lineto(693.85347112,114.94886087)
\lineto(689.28055791,109.21969854)
\closepath
}
}
{
\newrgbcolor{curcolor}{0 0 0}
\pscustom[linestyle=none,fillstyle=solid,fillcolor=curcolor]
{
\newpath
\moveto(702.44721648,103.4176196)
\curveto(702.0791612,103.32039745)(701.67638373,103.2405364)(701.23888406,103.17803645)
\curveto(700.80832883,103.11553649)(700.42291245,103.08428652)(700.08263493,103.08428652)
\curveto(698.89513583,103.08428652)(697.99235874,103.40373072)(697.37430365,104.04261912)
\curveto(696.75624856,104.68150753)(696.44722102,105.70581231)(696.44722102,107.11553347)
\lineto(696.44722102,113.30302879)
\lineto(695.12430535,113.30302879)
\lineto(695.12430535,114.94886087)
\lineto(696.44722102,114.94886087)
\lineto(696.44722102,118.29260834)
\lineto(698.40555287,118.29260834)
\lineto(698.40555287,114.94886087)
\lineto(702.44721648,114.94886087)
\lineto(702.44721648,113.30302879)
\lineto(698.40555287,113.30302879)
\lineto(698.40555287,108.00094946)
\curveto(698.40555287,107.38983881)(698.41944175,106.91067251)(698.44721951,106.56345055)
\curveto(698.47499726,106.22317303)(698.57221941,105.90372883)(698.73888595,105.60511794)
\curveto(698.89166361,105.32734038)(699.09999679,105.12247942)(699.36388548,104.99053507)
\curveto(699.63471861,104.86553517)(700.04444052,104.80303522)(700.59305121,104.80303522)
\curveto(700.91249542,104.80303522)(701.2458285,104.84817407)(701.59305046,104.93845178)
\curveto(701.94027242,105.03567393)(702.19027223,105.11553498)(702.34304989,105.17803493)
\lineto(702.44721648,105.17803493)
\closepath
}
}
{
\newrgbcolor{curcolor}{0 0 0}
\pscustom[linestyle=none,fillstyle=solid,fillcolor=curcolor]
{
\newpath
\moveto(718.35345879,103.4176196)
\curveto(717.98540351,103.32039745)(717.58262604,103.2405364)(717.14512637,103.17803645)
\curveto(716.71457114,103.11553649)(716.32915477,103.08428652)(715.98887725,103.08428652)
\curveto(714.80137814,103.08428652)(713.89860105,103.40373072)(713.28054596,104.04261912)
\curveto(712.66249087,104.68150753)(712.35346333,105.70581231)(712.35346333,107.11553347)
\lineto(712.35346333,113.30302879)
\lineto(711.03054766,113.30302879)
\lineto(711.03054766,114.94886087)
\lineto(712.35346333,114.94886087)
\lineto(712.35346333,118.29260834)
\lineto(714.31179518,118.29260834)
\lineto(714.31179518,114.94886087)
\lineto(718.35345879,114.94886087)
\lineto(718.35345879,113.30302879)
\lineto(714.31179518,113.30302879)
\lineto(714.31179518,108.00094946)
\curveto(714.31179518,107.38983881)(714.32568406,106.91067251)(714.35346182,106.56345055)
\curveto(714.38123957,106.22317303)(714.47846172,105.90372883)(714.64512826,105.60511794)
\curveto(714.79790592,105.32734038)(715.0062391,105.12247942)(715.27012779,104.99053507)
\curveto(715.54096092,104.86553517)(715.95068283,104.80303522)(716.49929353,104.80303522)
\curveto(716.81873773,104.80303522)(717.15207081,104.84817407)(717.49929277,104.93845178)
\curveto(717.84651473,105.03567393)(718.09651454,105.11553498)(718.2492922,105.17803493)
\lineto(718.35345879,105.17803493)
\closepath
}
}
{
\newrgbcolor{curcolor}{0 0 0}
\pscustom[linestyle=none,fillstyle=solid,fillcolor=curcolor]
{
\newpath
\moveto(730.45761497,108.9280321)
\lineto(721.88470479,108.9280321)
\curveto(721.88470479,108.21275486)(721.9923436,107.58775533)(722.20762121,107.05303351)
\curveto(722.42289883,106.52525614)(722.71803749,106.09122869)(723.09303721,105.75095117)
\curveto(723.45414805,105.41761808)(723.88123106,105.16761827)(724.37428624,105.00095173)
\curveto(724.87428586,104.83428519)(725.42289656,104.75095192)(726.02011833,104.75095192)
\curveto(726.81178439,104.75095192)(727.60692268,104.9072018)(728.40553319,105.21970157)
\curveto(729.21108813,105.53914577)(729.78400437,105.85164553)(730.12428189,106.15720086)
\lineto(730.22844848,106.15720086)
\lineto(730.22844848,104.02178581)
\curveto(729.56872675,103.74400824)(728.89511615,103.51136953)(728.20761667,103.32386967)
\curveto(727.52011719,103.13636981)(726.79789552,103.04261988)(726.04095164,103.04261988)
\curveto(724.11039755,103.04261988)(722.60345424,103.56345282)(721.52012173,104.6051187)
\curveto(720.43678922,105.65372902)(719.89512296,107.139839)(719.89512296,109.06344866)
\curveto(719.89512296,110.966225)(720.41248368,112.47664052)(721.44720512,113.59469523)
\curveto(722.488871,114.71274994)(723.85692552,115.2717773)(725.55136868,115.2717773)
\curveto(727.12081194,115.2717773)(728.32914436,114.81344431)(729.17636594,113.89677834)
\curveto(730.03053196,112.98011236)(730.45761497,111.67803001)(730.45761497,109.99053129)
\closepath
\moveto(728.55136641,110.42803096)
\curveto(728.54442197,111.45580796)(728.2840055,112.25094625)(727.770117,112.81344582)
\curveto(727.26317294,113.3759454)(726.48886797,113.65719518)(725.44720209,113.65719518)
\curveto(724.39859178,113.65719518)(723.56178685,113.34816764)(722.93678733,112.73011255)
\curveto(722.31873224,112.11205746)(721.96803806,111.34469693)(721.88470479,110.42803096)
\closepath
}
}
{
\newrgbcolor{curcolor}{0 0 0}
\pscustom[linestyle=none,fillstyle=solid,fillcolor=curcolor]
{
\newpath
\moveto(741.73885964,106.66761714)
\curveto(741.73885964,105.60511794)(741.29788775,104.73359082)(740.41594397,104.05303578)
\curveto(739.54094464,103.37248074)(738.34302888,103.03220322)(736.82219669,103.03220322)
\curveto(735.96108623,103.03220322)(735.16942017,103.13289759)(734.44719849,103.33428633)
\curveto(733.73192125,103.5426195)(733.13122726,103.76831378)(732.64511652,104.01136915)
\lineto(732.64511652,106.20928415)
\lineto(732.74928311,106.20928415)
\curveto(733.3673382,105.74400673)(734.05483768,105.37247923)(734.81178155,105.09470166)
\curveto(735.56872542,104.82386853)(736.29441932,104.68845197)(736.98886323,104.68845197)
\curveto(737.84997369,104.68845197)(738.5235843,104.82734075)(739.00969504,105.10511832)
\curveto(739.49580578,105.38289589)(739.73886115,105.82039556)(739.73886115,106.41761733)
\curveto(739.73886115,106.87595031)(739.60691681,107.22317227)(739.34302812,107.45928321)
\curveto(739.07913943,107.69539414)(738.57219537,107.89678288)(737.82219594,108.06344942)
\curveto(737.54441837,108.12594937)(737.17983531,108.19886598)(736.72844676,108.28219925)
\curveto(736.28400266,108.36553252)(735.87775296,108.45581023)(735.50969769,108.55303238)
\curveto(734.48886513,108.82386551)(733.76317123,109.21969854)(733.332616,109.74053148)
\curveto(732.90900521,110.26830886)(732.69719981,110.9141417)(732.69719981,111.67803001)
\curveto(732.69719981,112.15719632)(732.79442196,112.60858487)(732.98886626,113.03219566)
\curveto(733.190255,113.45580645)(733.4923381,113.83427838)(733.89511557,114.16761146)
\curveto(734.28400417,114.49400011)(734.77705935,114.75094436)(735.37428112,114.93844421)
\curveto(735.97844733,115.13288851)(736.65205793,115.23011066)(737.39511293,115.23011066)
\curveto(738.08955685,115.23011066)(738.7909452,115.14330517)(739.499278,114.96969419)
\curveto(740.21455524,114.80302765)(740.80830479,114.59816669)(741.28052665,114.35511132)
\lineto(741.28052665,112.26136291)
\lineto(741.17636007,112.26136291)
\curveto(740.67636044,112.62941818)(740.06872202,112.93844573)(739.35344478,113.18844554)
\curveto(738.63816754,113.44538979)(737.93677918,113.57386191)(737.2492797,113.57386191)
\curveto(736.53400247,113.57386191)(735.92983626,113.43497313)(735.43678108,113.15719556)
\curveto(734.94372589,112.88636243)(734.6971983,112.48011274)(734.6971983,111.93844648)
\curveto(734.6971983,111.45928018)(734.84650374,111.09816934)(735.14511463,110.85511397)
\curveto(735.43678108,110.6120586)(735.90900294,110.41414208)(736.56178022,110.26136442)
\curveto(736.92289106,110.17803115)(737.32566854,110.09469788)(737.77011264,110.01136461)
\curveto(738.22150119,109.92803134)(738.59650091,109.85164251)(738.89511179,109.78219812)
\curveto(739.80483333,109.57386494)(740.50622168,109.21622632)(740.99927687,108.70928226)
\curveto(741.49233205,108.19539376)(741.73885964,107.51483872)(741.73885964,106.66761714)
\closepath
}
}
{
\newrgbcolor{curcolor}{0 0 0}
\pscustom[linestyle=none,fillstyle=solid,fillcolor=curcolor]
{
\newpath
\moveto(750.58261649,103.4176196)
\curveto(750.21456121,103.32039745)(749.81178373,103.2405364)(749.37428407,103.17803645)
\curveto(748.94372884,103.11553649)(748.55831246,103.08428652)(748.21803494,103.08428652)
\curveto(747.03053584,103.08428652)(746.12775874,103.40373072)(745.50970366,104.04261912)
\curveto(744.89164857,104.68150753)(744.58262102,105.70581231)(744.58262102,107.11553347)
\lineto(744.58262102,113.30302879)
\lineto(743.25970536,113.30302879)
\lineto(743.25970536,114.94886087)
\lineto(744.58262102,114.94886087)
\lineto(744.58262102,118.29260834)
\lineto(746.54095288,118.29260834)
\lineto(746.54095288,114.94886087)
\lineto(750.58261649,114.94886087)
\lineto(750.58261649,113.30302879)
\lineto(746.54095288,113.30302879)
\lineto(746.54095288,108.00094946)
\curveto(746.54095288,107.38983881)(746.55484175,106.91067251)(746.58261951,106.56345055)
\curveto(746.61039727,106.22317303)(746.70761942,105.90372883)(746.87428596,105.60511794)
\curveto(747.02706362,105.32734038)(747.2353968,105.12247942)(747.49928548,104.99053507)
\curveto(747.77011861,104.86553517)(748.17984053,104.80303522)(748.72845122,104.80303522)
\curveto(749.04789542,104.80303522)(749.38122851,104.84817407)(749.72845046,104.93845178)
\curveto(750.07567242,105.03567393)(750.32567223,105.11553498)(750.4784499,105.17803493)
\lineto(750.58261649,105.17803493)
\closepath
}
}
{
\newrgbcolor{curcolor}{0.7019608 0.7019608 0.7019608}
\pscustom[linestyle=none,fillstyle=solid,fillcolor=curcolor,opacity=0.92623001]
{
\newpath
\moveto(355.90269063,726.97196396)
\lineto(766.61699419,726.97196396)
\lineto(766.61699419,590.54339528)
\lineto(355.90269063,590.54339528)
\closepath
}
}
{
\newrgbcolor{curcolor}{0 0 0}
\pscustom[linewidth=1.00157103,linecolor=curcolor]
{
\newpath
\moveto(355.90269063,726.97196396)
\lineto(766.61699419,726.97196396)
\lineto(766.61699419,590.54339528)
\lineto(355.90269063,590.54339528)
\closepath
}
}
{
\newrgbcolor{curcolor}{0 0 0}
\pscustom[linestyle=none,fillstyle=solid,fillcolor=curcolor]
{
\newpath
\moveto(419.80523925,713.07408693)
\lineto(419.60992737,713.07408693)
\curveto(419.20628283,713.19127406)(418.67894075,713.30846118)(418.02790116,713.42564831)
\curveto(417.37686157,713.55585623)(416.80394672,713.62096019)(416.30915663,713.62096019)
\curveto(414.73364082,713.62096019)(413.58781113,713.26939881)(412.87166758,712.56627605)
\curveto(412.16854482,711.87617408)(411.81698344,710.61966766)(411.81698344,708.7967568)
\lineto(411.81698344,708.05457166)
\lineto(418.4380561,708.05457166)
\lineto(418.4380561,704.96864399)
\lineto(411.93417056,704.96864399)
\lineto(411.93417056,686.23823487)
\lineto(408.26230725,686.23823487)
\lineto(408.26230725,704.96864399)
\lineto(405.7818464,704.96864399)
\lineto(405.7818464,708.05457166)
\lineto(408.26230725,708.05457166)
\lineto(408.26230725,708.77722561)
\curveto(408.26230725,711.36836319)(408.90683645,713.35403396)(410.19589485,714.73423789)
\curveto(411.48495324,716.12746262)(413.34692648,716.82407499)(415.78181456,716.82407499)
\curveto(416.60212445,716.82407499)(417.33779919,716.78501261)(417.98883879,716.70688786)
\curveto(418.65289917,716.62876311)(419.25836599,716.53761757)(419.80523925,716.43345123)
\closepath
}
}
{
\newrgbcolor{curcolor}{0 0 0}
\pscustom[linestyle=none,fillstyle=solid,fillcolor=curcolor]
{
\newpath
\moveto(426.07475026,711.70690378)
\lineto(421.93413844,711.70690378)
\lineto(421.93413844,715.51548541)
\lineto(426.07475026,715.51548541)
\closepath
\moveto(425.840376,686.23823487)
\lineto(422.1685127,686.23823487)
\lineto(422.1685127,708.05457166)
\lineto(425.840376,708.05457166)
\closepath
}
}
{
\newrgbcolor{curcolor}{0 0 0}
\pscustom[linestyle=none,fillstyle=solid,fillcolor=curcolor]
{
\newpath
\moveto(451.69966878,697.13663767)
\curveto(451.69966878,693.58196149)(450.78821334,690.77598084)(448.96530248,688.71869573)
\curveto(447.14239162,686.66141061)(444.70099314,685.63276805)(441.64110705,685.63276805)
\curveto(438.55517938,685.63276805)(436.10076011,686.66141061)(434.27784925,688.71869573)
\curveto(432.46795918,690.77598084)(431.56301414,693.58196149)(431.56301414,697.13663767)
\curveto(431.56301414,700.69131386)(432.46795918,703.4972945)(434.27784925,705.55457962)
\curveto(436.10076011,707.62488553)(438.55517938,708.66003848)(441.64110705,708.66003848)
\curveto(444.70099314,708.66003848)(447.14239162,707.62488553)(448.96530248,705.55457962)
\curveto(450.78821334,703.4972945)(451.69966878,700.69131386)(451.69966878,697.13663767)
\closepath
\moveto(447.91061834,697.13663767)
\curveto(447.91061834,699.96214951)(447.35723469,702.058497)(446.25046738,703.42568015)
\curveto(445.14370007,704.80588409)(443.60724663,705.49598606)(441.64110705,705.49598606)
\curveto(439.64892589,705.49598606)(438.09945166,704.80588409)(436.99268435,703.42568015)
\curveto(435.89893783,702.058497)(435.35206458,699.96214951)(435.35206458,697.13663767)
\curveto(435.35206458,694.40227138)(435.90544823,692.32545508)(437.01221554,690.90618876)
\curveto(438.11898285,689.49994324)(439.66194669,688.79682048)(441.64110705,688.79682048)
\curveto(443.59422583,688.79682048)(445.12416888,689.49343284)(446.23093619,690.88665757)
\curveto(447.35072429,692.2929031)(447.91061834,694.3762298)(447.91061834,697.13663767)
\closepath
}
}
{
\newrgbcolor{curcolor}{0 0 0}
\pscustom[linestyle=none,fillstyle=solid,fillcolor=curcolor]
{
\newpath
\moveto(484.62924675,716.62876311)
\lineto(470.62538508,680.30075378)
\lineto(467.24648959,680.30075378)
\lineto(481.19175769,716.62876311)
\closepath
}
}
{
\newrgbcolor{curcolor}{0 0 0}
\pscustom[linestyle=none,fillstyle=solid,fillcolor=curcolor]
{
\newpath
\moveto(519.25804442,692.52727735)
\curveto(519.25804442,690.53509619)(518.43122414,688.90098681)(516.77758357,687.62494921)
\curveto(515.13696379,686.3489116)(512.8908772,685.7108928)(510.03932377,685.7108928)
\curveto(508.42474558,685.7108928)(506.94037531,685.89969428)(505.58621295,686.27729725)
\curveto(504.24507139,686.66792101)(503.11877289,687.09109674)(502.20731746,687.54682446)
\lineto(502.20731746,691.66790509)
\lineto(502.40262934,691.66790509)
\curveto(503.56147982,690.79551203)(504.85053821,690.09889966)(506.26980453,689.57806799)
\curveto(507.68907084,689.07025711)(509.04974359,688.81635166)(510.35182278,688.81635166)
\curveto(511.96640097,688.81635166)(513.22941778,689.0767675)(514.14087322,689.59759918)
\curveto(515.05232865,690.11843085)(515.50805636,690.93874074)(515.50805636,692.05852884)
\curveto(515.50805636,692.91790111)(515.26066132,693.5689407)(514.76587123,694.01164762)
\curveto(514.27108113,694.45435455)(513.32056333,694.83195751)(511.9143178,695.14445652)
\curveto(511.39348613,695.26164364)(510.70989456,695.39836196)(509.86354308,695.55461146)
\curveto(509.0302124,695.71086096)(508.26849608,695.88013126)(507.57839411,696.06242234)
\curveto(505.6643377,696.57023323)(504.30366495,697.31241836)(503.49637586,698.28897775)
\curveto(502.70210755,699.27855794)(502.3049734,700.48949158)(502.3049734,701.92177869)
\curveto(502.3049734,702.82021333)(502.48726449,703.6665648)(502.85184666,704.4608331)
\curveto(503.22944962,705.25510141)(503.79585407,705.96473457)(504.55106,706.58973258)
\curveto(505.28022434,707.20170979)(506.20470057,707.68347909)(507.32448867,708.03504047)
\curveto(508.45729756,708.39962265)(509.72031437,708.58191373)(511.1135391,708.58191373)
\curveto(512.41561829,708.58191373)(513.73071827,708.41915383)(515.05883904,708.09363404)
\curveto(516.39998061,707.78113503)(517.51325831,707.39702167)(518.39867216,706.94129396)
\lineto(518.39867216,703.01552521)
\lineto(518.20336028,703.01552521)
\curveto(517.26586327,703.70562717)(516.12654398,704.28505241)(514.78540241,704.75380092)
\curveto(513.44426085,705.23557022)(512.12916087,705.47645487)(510.84010247,705.47645487)
\curveto(509.49896091,705.47645487)(508.36615202,705.21603903)(507.44167579,704.69520736)
\curveto(506.51719957,704.18739647)(506.05496146,703.42568015)(506.05496146,702.41005838)
\curveto(506.05496146,701.51162374)(506.33490849,700.83454257)(506.89480254,700.37881485)
\curveto(507.44167579,699.92308713)(508.32708964,699.55199457)(509.55104408,699.26553715)
\curveto(510.22812526,699.10928764)(510.98333119,698.95303814)(511.81666187,698.79678864)
\curveto(512.66301334,698.64053914)(513.3661361,698.49731042)(513.92603015,698.36710251)
\curveto(515.63175388,697.97647875)(516.94685386,697.30590797)(517.87133009,696.35539016)
\curveto(518.79580631,695.39185156)(519.25804442,694.11581396)(519.25804442,692.52727735)
\closepath
}
}
{
\newrgbcolor{curcolor}{0 0 0}
\pscustom[linestyle=none,fillstyle=solid,fillcolor=curcolor]
{
\newpath
\moveto(528.45723776,711.70690378)
\lineto(524.31662594,711.70690378)
\lineto(524.31662594,715.51548541)
\lineto(528.45723776,715.51548541)
\closepath
\moveto(528.2228635,686.23823487)
\lineto(524.5510002,686.23823487)
\lineto(524.5510002,708.05457166)
\lineto(528.2228635,708.05457166)
\closepath
}
}
{
\newrgbcolor{curcolor}{0 0 0}
\pscustom[linestyle=none,fillstyle=solid,fillcolor=curcolor]
{
\newpath
\moveto(567.32429801,686.23823487)
\lineto(563.6524347,686.23823487)
\lineto(563.6524347,698.66007032)
\curveto(563.6524347,699.59756734)(563.60686193,700.50251237)(563.51571639,701.37490543)
\curveto(563.43759163,702.24729848)(563.26181094,702.94391085)(562.98837431,703.46474253)
\curveto(562.6888961,704.02463658)(562.25920997,704.44781231)(561.69931592,704.73426973)
\curveto(561.13942187,705.02072715)(560.33213277,705.16395587)(559.27744863,705.16395587)
\curveto(558.24880607,705.16395587)(557.22016351,704.90354003)(556.19152096,704.38270835)
\curveto(555.1628784,703.87489747)(554.13423584,703.22385788)(553.10559328,702.42958957)
\curveto(553.14465566,702.13011136)(553.17720764,701.77854998)(553.20324922,701.37490543)
\curveto(553.2292908,700.98428167)(553.2423116,700.59365792)(553.2423116,700.20303416)
\lineto(553.2423116,686.23823487)
\lineto(549.57044829,686.23823487)
\lineto(549.57044829,698.66007032)
\curveto(549.57044829,699.62360892)(549.52487551,700.53506435)(549.43372997,701.39443662)
\curveto(549.35560522,702.26682967)(549.17982453,702.96344204)(548.9063879,703.48427371)
\curveto(548.60690969,704.04416776)(548.17722356,704.4608331)(547.61732951,704.73426973)
\curveto(547.05743545,705.02072715)(546.25014636,705.16395587)(545.19546222,705.16395587)
\curveto(544.19286124,705.16395587)(543.18374987,704.91656082)(542.1681281,704.42177073)
\curveto(541.16552713,703.92698064)(540.16292616,703.29547223)(539.16032518,702.52724551)
\lineto(539.16032518,686.23823487)
\lineto(535.48846187,686.23823487)
\lineto(535.48846187,708.05457166)
\lineto(539.16032518,708.05457166)
\lineto(539.16032518,705.63270437)
\curveto(540.30615487,706.58322218)(541.44547416,707.32540732)(542.57828305,707.85925978)
\curveto(543.72411273,708.39311225)(544.94155677,708.66003848)(546.23061517,708.66003848)
\curveto(547.71498544,708.66003848)(548.97149186,708.34753948)(550.00013442,707.72254147)
\curveto(551.04179777,707.09754346)(551.81653489,706.2316608)(552.32434577,705.12489349)
\curveto(553.80871604,706.37488951)(555.1628784,707.27332415)(556.38683283,707.82019741)
\curveto(557.61078727,708.38009146)(558.91937685,708.66003848)(560.31260158,708.66003848)
\curveto(562.70842729,708.66003848)(564.47274459,707.93087414)(565.60555348,706.47254545)
\curveto(566.75138317,705.02723755)(567.32429801,703.00250441)(567.32429801,700.39834604)
\closepath
}
}
{
\newrgbcolor{curcolor}{0 0 0}
\pscustom[linestyle=none,fillstyle=solid,fillcolor=curcolor]
{
\newpath
\moveto(593.59375415,697.4100743)
\curveto(593.59375415,695.63924661)(593.33984871,694.01815802)(592.83203783,692.54680854)
\curveto(592.32422694,691.08847985)(591.60808339,689.85150462)(590.68360717,688.83588285)
\curveto(589.8242349,687.87234425)(588.80861314,687.12364872)(587.63674187,686.58979625)
\curveto(586.47789139,686.06896458)(585.24742656,685.80854874)(583.94534737,685.80854874)
\curveto(582.81253848,685.80854874)(581.78389592,685.93224626)(580.8594197,686.17964131)
\curveto(579.94796427,686.42703636)(579.01697765,686.81114972)(578.06645984,687.33198139)
\lineto(578.06645984,678.19138549)
\lineto(574.39459653,678.19138549)
\lineto(574.39459653,708.05457166)
\lineto(578.06645984,708.05457166)
\lineto(578.06645984,705.76942269)
\curveto(579.04301923,706.58973258)(580.13676575,707.27332415)(581.34769939,707.82019741)
\curveto(582.57165383,708.38009146)(583.87373302,708.66003848)(585.25393696,708.66003848)
\curveto(587.88413691,708.66003848)(589.92840124,707.66394791)(591.38672993,705.67176675)
\curveto(592.85807941,703.69260638)(593.59375415,700.9387089)(593.59375415,697.4100743)
\closepath
\moveto(589.80470372,697.31241836)
\curveto(589.80470372,699.94261832)(589.3554864,701.9087579)(588.45705176,703.21083708)
\curveto(587.55861712,704.51291627)(586.17841318,705.16395587)(584.31643994,705.16395587)
\curveto(583.2617558,705.16395587)(582.20056126,704.93609201)(581.13285633,704.48036429)
\curveto(580.06515139,704.02463658)(579.04301923,703.42568015)(578.06645984,702.68349501)
\lineto(578.06645984,690.32025313)
\curveto(579.10812319,689.85150462)(580.00004743,689.53249522)(580.74223257,689.36322492)
\curveto(581.4974385,689.19395463)(582.35030037,689.10931948)(583.30081817,689.10931948)
\curveto(585.3450825,689.10931948)(586.9401295,689.79942145)(588.08595919,691.17962539)
\curveto(589.23178887,692.55982933)(589.80470372,694.60409365)(589.80470372,697.31241836)
\closepath
}
}
{
\newrgbcolor{curcolor}{0 0 0}
\pscustom[linestyle=none,fillstyle=solid,fillcolor=curcolor]
{
\newpath
\moveto(603.02731551,686.23823487)
\lineto(599.35545221,686.23823487)
\lineto(599.35545221,716.62876311)
\lineto(603.02731551,716.62876311)
\closepath
}
}
{
\newrgbcolor{curcolor}{0 0 0}
\pscustom[linestyle=none,fillstyle=solid,fillcolor=curcolor]
{
\newpath
\moveto(628.55457449,696.76554511)
\lineto(612.48040692,696.76554511)
\curveto(612.48040692,695.42440354)(612.68222919,694.25253227)(613.08587374,693.2499313)
\curveto(613.48951829,692.26035112)(614.04290194,691.44655162)(614.7460247,690.80853282)
\curveto(615.42310588,690.18353481)(616.22388458,689.7147863)(617.14836081,689.4022873)
\curveto(618.08585782,689.08978829)(619.11450038,688.93353879)(620.23428848,688.93353879)
\curveto(621.71865875,688.93353879)(623.20953942,689.22650661)(624.70693049,689.81244224)
\curveto(626.21734235,690.41139867)(627.29155768,690.9973343)(627.92957648,691.57024915)
\lineto(628.12488836,691.57024915)
\lineto(628.12488836,687.56635564)
\curveto(626.88791313,687.04552397)(625.62489632,686.60932744)(624.33583792,686.25776606)
\curveto(623.04677953,685.90620468)(621.69261717,685.73042399)(620.27335086,685.73042399)
\curveto(616.65357071,685.73042399)(613.82805888,686.70698338)(611.79681534,688.66010216)
\curveto(609.76557181,690.62624174)(608.74995005,693.4126912)(608.74995005,697.01945055)
\curveto(608.74995005,700.58714752)(609.71999904,703.41916975)(611.66009703,705.51551725)
\curveto(613.61321581,707.61186474)(616.17831181,708.66003848)(619.35538503,708.66003848)
\curveto(622.29808399,708.66003848)(624.56370178,707.80066622)(626.15223839,706.08192169)
\curveto(627.75379579,704.36317716)(628.55457449,701.92177869)(628.55457449,698.75772626)
\closepath
\moveto(624.98036712,699.57803615)
\curveto(624.96734633,701.50511335)(624.47906663,702.99599402)(623.51552803,704.05067816)
\curveto(622.56501023,705.1053623)(621.11319193,705.63270437)(619.16007315,705.63270437)
\curveto(617.19393358,705.63270437)(615.62492816,705.05327913)(614.45305689,703.89442866)
\curveto(613.29420641,702.73557818)(612.63665642,701.29678068)(612.48040692,699.57803615)
\closepath
}
}
{
\newrgbcolor{curcolor}{0 0 0}
\pscustom[linestyle=none,fillstyle=solid,fillcolor=curcolor]
{
\newpath
\moveto(647.73419989,704.05067816)
\lineto(647.53888802,704.05067816)
\curveto(646.99201476,704.18088608)(646.45816229,704.27203162)(645.93733062,704.32411479)
\curveto(645.42951973,704.38921875)(644.82405291,704.42177073)(644.12093015,704.42177073)
\curveto(642.98812126,704.42177073)(641.89437474,704.16786529)(640.8396906,703.6600544)
\curveto(639.78500645,703.16526431)(638.76938469,702.52073511)(637.7928253,701.72646681)
\lineto(637.7928253,686.23823487)
\lineto(634.12096199,686.23823487)
\lineto(634.12096199,708.05457166)
\lineto(637.7928253,708.05457166)
\lineto(637.7928253,704.83192567)
\curveto(639.25115399,706.00379694)(640.53370199,706.83061723)(641.6404693,707.31238652)
\curveto(642.7602574,707.80717662)(643.89957669,708.05457166)(645.05842716,708.05457166)
\curveto(645.69644597,708.05457166)(646.15868408,708.03504047)(646.4451415,707.9959781)
\curveto(646.73159892,707.96993651)(647.16128505,707.91134295)(647.73419989,707.82019741)
\closepath
}
}
{
\newrgbcolor{curcolor}{0 0 0}
\pscustom[linestyle=none,fillstyle=solid,fillcolor=curcolor]
{
\newpath
\moveto(669.45286781,696.76554511)
\lineto(653.37870024,696.76554511)
\curveto(653.37870024,695.42440354)(653.58052251,694.25253227)(653.98416706,693.2499313)
\curveto(654.38781161,692.26035112)(654.94119527,691.44655162)(655.64431803,690.80853282)
\curveto(656.3213992,690.18353481)(657.1221779,689.7147863)(658.04665413,689.4022873)
\curveto(658.98415114,689.08978829)(660.0127937,688.93353879)(661.1325818,688.93353879)
\curveto(662.61695208,688.93353879)(664.10783275,689.22650661)(665.60522381,689.81244224)
\curveto(667.11563567,690.41139867)(668.189851,690.9973343)(668.8278698,691.57024915)
\lineto(669.02318168,691.57024915)
\lineto(669.02318168,687.56635564)
\curveto(667.78620645,687.04552397)(666.52318964,686.60932744)(665.23413124,686.25776606)
\curveto(663.94507285,685.90620468)(662.59091049,685.73042399)(661.17164418,685.73042399)
\curveto(657.55186404,685.73042399)(654.7263522,686.70698338)(652.69510867,688.66010216)
\curveto(650.66386513,690.62624174)(649.64824337,693.4126912)(649.64824337,697.01945055)
\curveto(649.64824337,700.58714752)(650.61829236,703.41916975)(652.55839035,705.51551725)
\curveto(654.51150913,707.61186474)(657.07660513,708.66003848)(660.25367835,708.66003848)
\curveto(663.19637731,708.66003848)(665.4619951,707.80066622)(667.05053171,706.08192169)
\curveto(668.65208911,704.36317716)(669.45286781,701.92177869)(669.45286781,698.75772626)
\closepath
\moveto(665.87866044,699.57803615)
\curveto(665.86563965,701.50511335)(665.37735995,702.99599402)(664.41382136,704.05067816)
\curveto(663.46330355,705.1053623)(662.01148525,705.63270437)(660.05836647,705.63270437)
\curveto(658.0922269,705.63270437)(656.52322148,705.05327913)(655.35135021,703.89442866)
\curveto(654.19249973,702.73557818)(653.53494974,701.29678068)(653.37870024,699.57803615)
\closepath
}
}
{
\newrgbcolor{curcolor}{0 0 0}
\pscustom[linestyle=none,fillstyle=solid,fillcolor=curcolor]
{
\newpath
\moveto(691.97235517,686.23823487)
\lineto(688.32002305,686.23823487)
\lineto(688.32002305,688.56244622)
\curveto(687.99450325,688.34109276)(687.55179633,688.02859376)(686.99190228,687.62494921)
\curveto(686.44502902,687.23432545)(685.91117655,686.92182645)(685.39034488,686.68745219)
\curveto(684.77836766,686.38797398)(684.0752449,686.14057893)(683.28097659,685.94526706)
\curveto(682.48670829,685.73693439)(681.55572167,685.63276805)(680.48801673,685.63276805)
\curveto(678.52187716,685.63276805)(676.8552158,686.28380764)(675.48803265,687.58588683)
\curveto(674.12084951,688.88796602)(673.43725793,690.54811698)(673.43725793,692.56633973)
\curveto(673.43725793,694.21998029)(673.78881931,695.55461146)(674.49194207,696.57023323)
\curveto(675.20808563,697.59887579)(676.22370739,698.40616488)(677.53880737,698.99210052)
\curveto(678.86692815,699.57803615)(680.46197515,699.9751703)(682.32394839,700.18350297)
\curveto(684.18592163,700.39183564)(686.18461318,700.54808515)(688.32002305,700.65225148)
\lineto(688.32002305,701.21865593)
\curveto(688.32002305,702.05198661)(688.17028394,702.74208858)(687.87080573,703.28896183)
\curveto(687.58434831,703.83583509)(687.16768297,704.26552123)(686.62080971,704.57802023)
\curveto(686.09997803,704.87749844)(685.47498002,705.07932072)(684.74581568,705.18348705)
\curveto(684.01665133,705.28765339)(683.25493501,705.33973656)(682.4606667,705.33973656)
\curveto(681.4971281,705.33973656)(680.42291277,705.20952864)(679.23802071,704.9491128)
\curveto(678.05312865,704.70171775)(676.82917422,704.33713558)(675.5661574,703.85536628)
\lineto(675.37084553,703.85536628)
\lineto(675.37084553,707.58582315)
\curveto(676.08698908,707.78113503)(677.12214203,707.9959781)(678.47630439,708.23035235)
\curveto(679.83046674,708.46472661)(681.16509791,708.58191373)(682.48019789,708.58191373)
\curveto(684.01665133,708.58191373)(685.3512825,708.45170581)(686.48409139,708.19128998)
\curveto(687.62992108,707.94389493)(688.61950126,707.5142088)(689.45283194,706.90223158)
\curveto(690.27314183,706.30327515)(690.89813984,705.52853804)(691.32782597,704.57802023)
\curveto(691.7575121,703.62750242)(691.97235517,702.44912076)(691.97235517,701.04287524)
\closepath
\moveto(688.32002305,691.60931152)
\lineto(688.32002305,697.68351093)
\curveto(687.20023495,697.61840697)(685.87862457,697.52075103)(684.35519192,697.39054312)
\curveto(682.84478006,697.2603352)(681.64686721,697.07153371)(680.76145336,696.82413867)
\curveto(679.70676922,696.52466046)(678.85390735,696.05591195)(678.20286776,695.41789315)
\curveto(677.55182817,694.79289514)(677.22630837,693.92701248)(677.22630837,692.82024517)
\curveto(677.22630837,691.57024915)(677.60391133,690.62624174)(678.35911726,689.98822293)
\curveto(679.11432319,689.36322492)(680.26666327,689.05072592)(681.81613751,689.05072592)
\curveto(683.1051959,689.05072592)(684.28357757,689.29812096)(685.3512825,689.79291106)
\curveto(686.41898743,690.30072194)(687.40856762,690.90618876)(688.32002305,691.60931152)
\closepath
}
}
{
\newrgbcolor{curcolor}{0 0 0}
\pscustom[linestyle=none,fillstyle=solid,fillcolor=curcolor]
{
\newpath
\moveto(716.7378913,686.23823487)
\lineto(713.06602799,686.23823487)
\lineto(713.06602799,688.52338385)
\curveto(712.01134385,687.61192842)(710.91108693,686.90229526)(709.76525725,686.39448438)
\curveto(708.61942756,685.88667349)(707.37594194,685.63276805)(706.03480038,685.63276805)
\curveto(703.430642,685.63276805)(701.36033609,686.63536903)(699.82388265,688.64057097)
\curveto(698.30045,690.64577292)(697.53873368,693.42571199)(697.53873368,696.98038817)
\curveto(697.53873368,698.82934062)(697.79914952,700.47647079)(698.31998119,701.92177869)
\curveto(698.85383366,703.36708659)(699.56997721,704.59755142)(700.46841185,705.61317318)
\curveto(701.3538257,706.60275337)(702.38246826,707.3579593)(703.55433952,707.87879097)
\curveto(704.73923159,708.39962265)(705.96318602,708.66003848)(707.22620283,708.66003848)
\curveto(708.37203252,708.66003848)(709.38765428,708.53634096)(710.27306813,708.28894592)
\curveto(711.15848198,708.05457166)(712.0894686,707.68347909)(713.06602799,707.17566821)
\lineto(713.06602799,716.62876311)
\lineto(716.7378913,716.62876311)
\closepath
\moveto(713.06602799,691.60931152)
\lineto(713.06602799,704.12880291)
\curveto(712.07644781,704.57150983)(711.19103396,704.87749844)(710.40978645,705.04676874)
\curveto(709.62853893,705.21603903)(708.77567707,705.30067418)(707.85120084,705.30067418)
\curveto(705.79391573,705.30067418)(704.19235833,704.58453063)(703.04652864,703.15224352)
\curveto(701.90069896,701.71995641)(701.32778411,699.68871288)(701.32778411,697.05851292)
\curveto(701.32778411,694.46737534)(701.77049104,692.49472537)(702.65590488,691.14056301)
\curveto(703.54131873,689.79942145)(704.96058505,689.12885067)(706.91370383,689.12885067)
\curveto(707.95536718,689.12885067)(709.01005132,689.35671453)(710.07775625,689.81244224)
\curveto(711.14546119,690.28119075)(712.14155177,690.88014718)(713.06602799,691.60931152)
\closepath
}
}
{
\newrgbcolor{curcolor}{0 0 0}
\pscustom[linestyle=none,fillstyle=solid,fillcolor=curcolor]
{
\newpath
\moveto(426.29640635,648.2911686)
\lineto(419.79641127,648.2911686)
\lineto(419.79641127,650.17658384)
\lineto(426.29640635,650.17658384)
\closepath
}
}
{
\newrgbcolor{curcolor}{0 0 0}
\pscustom[linestyle=none,fillstyle=solid,fillcolor=curcolor]
{
\newpath
\moveto(447.61930712,653.2703315)
\curveto(447.61930712,652.58283202)(447.49777943,651.94394361)(447.25472406,651.35366628)
\curveto(447.01861313,650.77033339)(446.68528005,650.26338933)(446.25472482,649.8328341)
\curveto(445.720003,649.29811228)(445.08805903,648.89533481)(444.35889292,648.62450168)
\curveto(443.6297268,648.36061299)(442.70958861,648.22866865)(441.59847834,648.22866865)
\lineto(439.5359799,648.22866865)
\lineto(439.5359799,642.44742302)
\lineto(437.47348146,642.44742302)
\lineto(437.47348146,657.95782795)
\lineto(441.68181161,657.95782795)
\curveto(442.61236646,657.95782795)(443.40056031,657.8779669)(444.04639316,657.7182448)
\curveto(444.692226,657.56546714)(445.26514223,657.32241177)(445.76514186,656.98907868)
\curveto(446.35541919,656.59324565)(446.81027995,656.10019047)(447.12972416,655.50991314)
\curveto(447.4561128,654.91963581)(447.61930712,654.17310859)(447.61930712,653.2703315)
\closepath
\moveto(445.47347541,653.2182482)
\curveto(445.47347541,653.75297002)(445.37972548,654.21824745)(445.19222562,654.61408048)
\curveto(445.00472576,655.00991351)(444.72000376,655.33282994)(444.3380596,655.58282975)
\curveto(444.00472652,655.79810736)(443.62278237,655.95088503)(443.19222714,656.04116273)
\curveto(442.76861635,656.13838488)(442.23042231,656.18699596)(441.57764502,656.18699596)
\lineto(439.5359799,656.18699596)
\lineto(439.5359799,649.98908398)
\lineto(441.27556192,649.98908398)
\curveto(442.10889462,649.98908398)(442.78597744,650.06200059)(443.30681038,650.20783381)
\curveto(443.82764332,650.36061148)(444.25125411,650.60019463)(444.57764275,650.92658327)
\curveto(444.9040314,651.25991635)(445.13319789,651.61061053)(445.26514223,651.97866581)
\curveto(445.40403102,652.34672109)(445.47347541,652.75991522)(445.47347541,653.2182482)
\closepath
}
}
{
\newrgbcolor{curcolor}{0 0 0}
\pscustom[linestyle=none,fillstyle=solid,fillcolor=curcolor]
{
\newpath
\moveto(459.40054733,648.0620021)
\lineto(450.82763715,648.0620021)
\curveto(450.82763715,647.34672487)(450.93527596,646.72172534)(451.15055357,646.18700352)
\curveto(451.36583119,645.65922614)(451.66096985,645.2251987)(452.03596957,644.88492118)
\curveto(452.39708041,644.55158809)(452.82416342,644.30158828)(453.3172186,644.13492174)
\curveto(453.81721822,643.9682552)(454.36582892,643.88492193)(454.96305069,643.88492193)
\curveto(455.75471676,643.88492193)(456.54985504,644.04117181)(457.34846555,644.35367158)
\curveto(458.1540205,644.67311578)(458.72693673,644.98561554)(459.06721425,645.29117087)
\lineto(459.17138084,645.29117087)
\lineto(459.17138084,643.15575582)
\curveto(458.51165912,642.87797825)(457.83804851,642.64533954)(457.15054903,642.45783968)
\curveto(456.46304956,642.27033982)(455.74082788,642.17658989)(454.98388401,642.17658989)
\curveto(453.05332991,642.17658989)(451.54638661,642.69742283)(450.46305409,643.73908871)
\curveto(449.37972158,644.78769903)(448.83805532,646.27380901)(448.83805532,648.19741867)
\curveto(448.83805532,650.10019501)(449.35541604,651.61061053)(450.39013748,652.72866524)
\curveto(451.43180336,653.84671995)(452.79985788,654.40574731)(454.49430104,654.40574731)
\curveto(456.0637443,654.40574731)(457.27207672,653.94741432)(458.1192983,653.03074835)
\curveto(458.97346432,652.11408237)(459.40054733,650.81200002)(459.40054733,649.1245013)
\closepath
\moveto(457.49429877,649.56200097)
\curveto(457.48735434,650.58977797)(457.22693787,651.38491626)(456.71304937,651.94741583)
\curveto(456.20610531,652.50991541)(455.43180034,652.79116519)(454.39013446,652.79116519)
\curveto(453.34152414,652.79116519)(452.50471922,652.48213765)(451.87971969,651.86408256)
\curveto(451.2616646,651.24602747)(450.91097042,650.47866694)(450.82763715,649.56200097)
\closepath
}
}
{
\newrgbcolor{curcolor}{0 0 0}
\pscustom[linestyle=none,fillstyle=solid,fillcolor=curcolor]
{
\newpath
\moveto(469.62970586,651.94741583)
\lineto(469.52553927,651.94741583)
\curveto(469.23387283,652.01686022)(468.94915082,652.0654713)(468.67137325,652.09324905)
\curveto(468.40054013,652.12797125)(468.0776237,652.14533235)(467.70262399,652.14533235)
\curveto(467.09845778,652.14533235)(466.51512489,652.00991578)(465.95262531,651.73908266)
\curveto(465.39012574,651.47519397)(464.84845948,651.13144423)(464.32762654,650.70783344)
\lineto(464.32762654,642.44742302)
\lineto(462.36929469,642.44742302)
\lineto(462.36929469,654.08283088)
\lineto(464.32762654,654.08283088)
\lineto(464.32762654,652.36408218)
\curveto(465.10540373,652.98908171)(465.78943099,653.4300536)(466.37970832,653.68699785)
\curveto(466.97693009,653.95088654)(467.58456852,654.08283088)(468.20262361,654.08283088)
\curveto(468.54290113,654.08283088)(468.78942872,654.07241422)(468.94220638,654.05158091)
\curveto(469.09498404,654.03769203)(469.32415054,654.00644205)(469.62970586,653.95783098)
\closepath
}
}
{
\newrgbcolor{curcolor}{0 0 0}
\pscustom[linestyle=none,fillstyle=solid,fillcolor=curcolor]
{
\newpath
\moveto(477.73386552,656.75991219)
\lineto(477.62969893,656.75991219)
\curveto(477.41442131,656.82241214)(477.13317153,656.8849121)(476.78594957,656.94741205)
\curveto(476.43872761,657.01685644)(476.13317228,657.05157864)(475.86928359,657.05157864)
\curveto(475.02900645,657.05157864)(474.4178958,656.86407878)(474.03595165,656.48907906)
\curveto(473.66095193,656.12102379)(473.47345207,655.4508854)(473.47345207,654.47866392)
\lineto(473.47345207,654.08283088)
\lineto(477.0046994,654.08283088)
\lineto(477.0046994,652.43699879)
\lineto(473.53595203,652.43699879)
\lineto(473.53595203,642.44742302)
\lineto(471.57762017,642.44742302)
\lineto(471.57762017,652.43699879)
\lineto(470.25470451,652.43699879)
\lineto(470.25470451,654.08283088)
\lineto(471.57762017,654.08283088)
\lineto(471.57762017,654.46824726)
\curveto(471.57762017,655.85019066)(471.92136991,656.90921763)(472.60886939,657.64532819)
\curveto(473.29636887,658.38838318)(474.28942368,658.75991068)(475.58803381,658.75991068)
\curveto(476.02553348,658.75991068)(476.41789429,658.73907736)(476.76511625,658.69741073)
\curveto(477.11928265,658.65574409)(477.44219907,658.60713302)(477.73386552,658.5515775)
\closepath
}
}
{
\newrgbcolor{curcolor}{0 0 0}
\pscustom[linestyle=none,fillstyle=solid,fillcolor=curcolor]
{
\newpath
\moveto(488.89010554,648.25991862)
\curveto(488.89010554,646.36408672)(488.4039948,644.86756008)(487.43177331,643.77033868)
\curveto(486.45955182,642.67311729)(485.15746948,642.1245066)(483.52552627,642.1245066)
\curveto(481.87969418,642.1245066)(480.57066739,642.67311729)(479.5984459,643.77033868)
\curveto(478.63316886,644.86756008)(478.15053033,646.36408672)(478.15053033,648.25991862)
\curveto(478.15053033,650.15575052)(478.63316886,651.65227717)(479.5984459,652.74949856)
\curveto(480.57066739,653.85366439)(481.87969418,654.40574731)(483.52552627,654.40574731)
\curveto(485.15746948,654.40574731)(486.45955182,653.85366439)(487.43177331,652.74949856)
\curveto(488.4039948,651.65227717)(488.89010554,650.15575052)(488.89010554,648.25991862)
\closepath
\moveto(486.86927374,648.25991862)
\curveto(486.86927374,649.76686193)(486.57413507,650.88491664)(485.98385774,651.61408275)
\curveto(485.39358041,652.3501933)(484.57413658,652.71824858)(483.52552627,652.71824858)
\curveto(482.46302707,652.71824858)(481.63663881,652.3501933)(481.04636148,651.61408275)
\curveto(480.46302858,650.88491664)(480.17136214,649.76686193)(480.17136214,648.25991862)
\curveto(480.17136214,646.80158639)(480.4665008,645.69394834)(481.05677813,644.93700447)
\curveto(481.64705547,644.18700504)(482.46997151,643.81200532)(483.52552627,643.81200532)
\curveto(484.56719215,643.81200532)(485.38316375,644.18353282)(485.97344108,644.92658781)
\curveto(486.57066285,645.67658724)(486.86927374,646.78769751)(486.86927374,648.25991862)
\closepath
}
}
{
\newrgbcolor{curcolor}{0 0 0}
\pscustom[linestyle=none,fillstyle=solid,fillcolor=curcolor]
{
\newpath
\moveto(499.18176429,651.94741583)
\lineto(499.0775977,651.94741583)
\curveto(498.78593125,652.01686022)(498.50120924,652.0654713)(498.22343168,652.09324905)
\curveto(497.95259855,652.12797125)(497.62968213,652.14533235)(497.25468241,652.14533235)
\curveto(496.6505162,652.14533235)(496.06718331,652.00991578)(495.50468373,651.73908266)
\curveto(494.94218416,651.47519397)(494.4005179,651.13144423)(493.87968496,650.70783344)
\lineto(493.87968496,642.44742302)
\lineto(491.92135311,642.44742302)
\lineto(491.92135311,654.08283088)
\lineto(493.87968496,654.08283088)
\lineto(493.87968496,652.36408218)
\curveto(494.65746215,652.98908171)(495.34148941,653.4300536)(495.93176674,653.68699785)
\curveto(496.52898851,653.95088654)(497.13662694,654.08283088)(497.75468203,654.08283088)
\curveto(498.09495955,654.08283088)(498.34148714,654.07241422)(498.49426481,654.05158091)
\curveto(498.64704247,654.03769203)(498.87620896,654.00644205)(499.18176429,653.95783098)
\closepath
}
}
{
\newrgbcolor{curcolor}{0 0 0}
\pscustom[linestyle=none,fillstyle=solid,fillcolor=curcolor]
{
\newpath
\moveto(518.00466943,642.44742302)
\lineto(516.04633758,642.44742302)
\lineto(516.04633758,649.07241801)
\curveto(516.04633758,649.57241763)(516.02203205,650.05505615)(515.97342097,650.52033358)
\curveto(515.93175434,650.985611)(515.83800441,651.3571385)(515.69217118,651.63491607)
\curveto(515.53244908,651.93352695)(515.30328259,652.15922123)(515.0046717,652.31199889)
\curveto(514.70606082,652.46477655)(514.27550559,652.54116538)(513.71300601,652.54116538)
\curveto(513.16439532,652.54116538)(512.61578462,652.4022766)(512.06717393,652.12449903)
\curveto(511.51856323,651.8536659)(510.96995253,651.50644394)(510.42134184,651.08283315)
\curveto(510.44217516,650.92311105)(510.45953625,650.73561119)(510.47342513,650.52033358)
\curveto(510.48731401,650.3120004)(510.49425845,650.10366723)(510.49425845,649.89533405)
\lineto(510.49425845,642.44742302)
\lineto(508.5359266,642.44742302)
\lineto(508.5359266,649.07241801)
\curveto(508.5359266,649.58630651)(508.51162106,650.07241725)(508.46300999,650.53075024)
\curveto(508.42134335,650.99602766)(508.32759342,651.36755516)(508.1817602,651.64533273)
\curveto(508.0220381,651.94394361)(507.7928716,652.16616567)(507.49426072,652.31199889)
\curveto(507.19564983,652.46477655)(506.7650946,652.54116538)(506.20259503,652.54116538)
\curveto(505.66787321,652.54116538)(505.12967917,652.40922104)(504.58801292,652.14533235)
\curveto(504.0532911,651.88144366)(503.51856928,651.54463836)(502.98384746,651.13491645)
\lineto(502.98384746,642.44742302)
\lineto(501.02551561,642.44742302)
\lineto(501.02551561,654.08283088)
\lineto(502.98384746,654.08283088)
\lineto(502.98384746,652.79116519)
\curveto(503.59495811,653.29810925)(504.20259654,653.69394229)(504.80676275,653.9786643)
\curveto(505.4178734,654.2633863)(506.06717847,654.40574731)(506.75467795,654.40574731)
\curveto(507.54634401,654.40574731)(508.21648239,654.23908076)(508.76509309,653.90574768)
\curveto(509.32064823,653.5724146)(509.73384236,653.1106094)(510.00467549,652.52033207)
\curveto(510.79634155,653.18699823)(511.51856323,653.66616453)(512.17134051,653.95783098)
\curveto(512.8241178,654.25644186)(513.52203394,654.40574731)(514.26508893,654.40574731)
\curveto(515.54286574,654.40574731)(516.48383725,654.01685871)(517.08800346,653.23908152)
\curveto(517.69911411,652.46824877)(518.00466943,651.38838848)(518.00466943,649.99950064)
\closepath
}
}
{
\newrgbcolor{curcolor}{0 0 0}
\pscustom[linestyle=none,fillstyle=solid,fillcolor=curcolor]
{
\newpath
\moveto(535.317158,642.55158961)
\curveto(534.94910273,642.45436746)(534.54632525,642.37450641)(534.10882559,642.31200645)
\curveto(533.67827036,642.2495065)(533.29285398,642.21825653)(532.95257646,642.21825653)
\curveto(531.76507736,642.21825653)(530.86230026,642.53770073)(530.24424518,643.17658913)
\curveto(529.62619009,643.81547754)(529.31716254,644.83978232)(529.31716254,646.24950348)
\lineto(529.31716254,652.43699879)
\lineto(527.99424688,652.43699879)
\lineto(527.99424688,654.08283088)
\lineto(529.31716254,654.08283088)
\lineto(529.31716254,657.42657835)
\lineto(531.2754944,657.42657835)
\lineto(531.2754944,654.08283088)
\lineto(535.317158,654.08283088)
\lineto(535.317158,652.43699879)
\lineto(531.2754944,652.43699879)
\lineto(531.2754944,647.13491947)
\curveto(531.2754944,646.52380882)(531.28938327,646.04464252)(531.31716103,645.69742056)
\curveto(531.34493879,645.35714304)(531.44216094,645.03769884)(531.60882748,644.73908795)
\curveto(531.76160514,644.46131038)(531.96993831,644.25644943)(532.233827,644.12450508)
\curveto(532.50466013,643.99950518)(532.91438204,643.93700523)(533.46299274,643.93700523)
\curveto(533.78243694,643.93700523)(534.11577002,643.98214408)(534.46299198,644.07242179)
\curveto(534.81021394,644.16964394)(535.06021375,644.24950499)(535.21299142,644.31200494)
\lineto(535.317158,644.31200494)
\closepath
}
}
{
\newrgbcolor{curcolor}{0 0 0}
\pscustom[linestyle=none,fillstyle=solid,fillcolor=curcolor]
{
\newpath
\moveto(547.42130698,648.0620021)
\lineto(538.84839679,648.0620021)
\curveto(538.84839679,647.34672487)(538.9560356,646.72172534)(539.17131322,646.18700352)
\curveto(539.38659083,645.65922614)(539.6817295,645.2251987)(540.05672921,644.88492118)
\curveto(540.41784005,644.55158809)(540.84492306,644.30158828)(541.33797824,644.13492174)
\curveto(541.83797787,643.9682552)(542.38658856,643.88492193)(542.98381033,643.88492193)
\curveto(543.7754764,643.88492193)(544.57061469,644.04117181)(545.36922519,644.35367158)
\curveto(546.17478014,644.67311578)(546.74769637,644.98561554)(547.08797389,645.29117087)
\lineto(547.19214048,645.29117087)
\lineto(547.19214048,643.15575582)
\curveto(546.53241876,642.87797825)(545.85880816,642.64533954)(545.17130868,642.45783968)
\curveto(544.4838092,642.27033982)(543.76158752,642.17658989)(543.00464365,642.17658989)
\curveto(541.07408956,642.17658989)(539.56714625,642.69742283)(538.48381374,643.73908871)
\curveto(537.40048122,644.78769903)(536.85881497,646.27380901)(536.85881497,648.19741867)
\curveto(536.85881497,650.10019501)(537.37617569,651.61061053)(538.41089713,652.72866524)
\curveto(539.452563,653.84671995)(540.82061752,654.40574731)(542.51506069,654.40574731)
\curveto(544.08450394,654.40574731)(545.29283636,653.94741432)(546.14005794,653.03074835)
\curveto(546.99422397,652.11408237)(547.42130698,650.81200002)(547.42130698,649.1245013)
\closepath
\moveto(545.51505842,649.56200097)
\curveto(545.50811398,650.58977797)(545.24769751,651.38491626)(544.73380901,651.94741583)
\curveto(544.22686495,652.50991541)(543.45255998,652.79116519)(542.4108941,652.79116519)
\curveto(541.36228378,652.79116519)(540.52547886,652.48213765)(539.90047933,651.86408256)
\curveto(539.28242424,651.24602747)(538.93173006,650.47866694)(538.84839679,649.56200097)
\closepath
}
}
{
\newrgbcolor{curcolor}{0 0 0}
\pscustom[linestyle=none,fillstyle=solid,fillcolor=curcolor]
{
\newpath
\moveto(558.70255165,645.80158715)
\curveto(558.70255165,644.73908795)(558.26157976,643.86756083)(557.37963598,643.18700579)
\curveto(556.50463664,642.50645075)(555.30672088,642.16617323)(553.7858887,642.16617323)
\curveto(552.92477824,642.16617323)(552.13311217,642.2668676)(551.4108905,642.46825634)
\curveto(550.69561326,642.67658951)(550.09491927,642.90228379)(549.60880853,643.14533916)
\lineto(549.60880853,645.34325416)
\lineto(549.71297511,645.34325416)
\curveto(550.3310302,644.87797674)(551.01852968,644.50644924)(551.77547355,644.22867167)
\curveto(552.53241743,643.95783854)(553.25811132,643.82242198)(553.95255524,643.82242198)
\curveto(554.8136657,643.82242198)(555.4872763,643.96131076)(555.97338705,644.23908833)
\curveto(556.45949779,644.5168659)(556.70255316,644.95436557)(556.70255316,645.55158734)
\curveto(556.70255316,646.00992032)(556.57060882,646.35714228)(556.30672013,646.59325322)
\curveto(556.04283144,646.82936415)(555.53588738,647.03075288)(554.78588794,647.19741943)
\curveto(554.50811038,647.25991938)(554.14352732,647.33283599)(553.69213877,647.41616926)
\curveto(553.24769466,647.49950253)(552.84144497,647.58978024)(552.47338969,647.68700239)
\curveto(551.45255713,647.95783552)(550.72686324,648.35366855)(550.29630801,648.87450149)
\curveto(549.87269722,649.40227887)(549.66089182,650.04811171)(549.66089182,650.81200002)
\curveto(549.66089182,651.29116633)(549.75811397,651.74255488)(549.95255827,652.16616567)
\curveto(550.153947,652.58977646)(550.45603011,652.96824839)(550.85880758,653.30158147)
\curveto(551.24769618,653.62797012)(551.74075136,653.88491437)(552.33797313,654.07241422)
\curveto(552.94213934,654.26685852)(553.61574994,654.36408067)(554.35880493,654.36408067)
\curveto(555.05324885,654.36408067)(555.75463721,654.27727518)(556.46297001,654.1036642)
\curveto(557.17824724,653.93699766)(557.7719968,653.7321367)(558.24421866,653.48908133)
\lineto(558.24421866,651.39533292)
\lineto(558.14005207,651.39533292)
\curveto(557.64005245,651.76338819)(557.03241402,652.07241574)(556.31713679,652.32241555)
\curveto(555.60185955,652.5793598)(554.90047119,652.70783192)(554.21297171,652.70783192)
\curveto(553.49769447,652.70783192)(552.89352826,652.56894314)(552.40047308,652.29116557)
\curveto(551.9074179,652.02033244)(551.66089031,651.61408275)(551.66089031,651.07241649)
\curveto(551.66089031,650.59325019)(551.81019575,650.23213935)(552.10880664,649.98908398)
\curveto(552.40047308,649.74602861)(552.87269495,649.54811209)(553.52547223,649.39533443)
\curveto(553.88658307,649.31200116)(554.28936054,649.22866789)(554.73380465,649.14533462)
\curveto(555.1851932,649.06200135)(555.56019291,648.98561252)(555.8588038,648.91616813)
\curveto(556.76852533,648.70783495)(557.46991369,648.35019633)(557.96296887,647.84325227)
\curveto(558.45602406,647.32936377)(558.70255165,646.64880873)(558.70255165,645.80158715)
\closepath
}
}
{
\newrgbcolor{curcolor}{0 0 0}
\pscustom[linestyle=none,fillstyle=solid,fillcolor=curcolor]
{
\newpath
\moveto(567.54629407,642.55158961)
\curveto(567.1782388,642.45436746)(566.77546132,642.37450641)(566.33796165,642.31200645)
\curveto(565.90740642,642.2495065)(565.52199005,642.21825653)(565.18171253,642.21825653)
\curveto(563.99421343,642.21825653)(563.09143633,642.53770073)(562.47338124,643.17658913)
\curveto(561.85532616,643.81547754)(561.54629861,644.83978232)(561.54629861,646.24950348)
\lineto(561.54629861,652.43699879)
\lineto(560.22338295,652.43699879)
\lineto(560.22338295,654.08283088)
\lineto(561.54629861,654.08283088)
\lineto(561.54629861,657.42657835)
\lineto(563.50463046,657.42657835)
\lineto(563.50463046,654.08283088)
\lineto(567.54629407,654.08283088)
\lineto(567.54629407,652.43699879)
\lineto(563.50463046,652.43699879)
\lineto(563.50463046,647.13491947)
\curveto(563.50463046,646.52380882)(563.51851934,646.04464252)(563.5462971,645.69742056)
\curveto(563.57407486,645.35714304)(563.67129701,645.03769884)(563.83796355,644.73908795)
\curveto(563.99074121,644.46131038)(564.19907438,644.25644943)(564.46296307,644.12450508)
\curveto(564.7337962,643.99950518)(565.14351811,643.93700523)(565.69212881,643.93700523)
\curveto(566.01157301,643.93700523)(566.34490609,643.98214408)(566.69212805,644.07242179)
\curveto(567.03935001,644.16964394)(567.28934982,644.24950499)(567.44212749,644.31200494)
\lineto(567.54629407,644.31200494)
\closepath
}
}
{
\newrgbcolor{curcolor}{0 0 0}
\pscustom[linestyle=none,fillstyle=solid,fillcolor=curcolor]
{
\newpath
\moveto(586.45253411,642.44742302)
\lineto(584.50461892,642.44742302)
\lineto(584.50461892,643.68700541)
\curveto(584.33100794,643.56894995)(584.09489701,643.40228341)(583.79628612,643.18700579)
\curveto(583.50461968,642.97867262)(583.21989767,642.81200608)(582.9421201,642.68700617)
\curveto(582.61573146,642.52728407)(582.24073175,642.39533973)(581.81712095,642.29117314)
\curveto(581.39351016,642.18006211)(580.89698276,642.1245066)(580.32753875,642.1245066)
\curveto(579.27892843,642.1245066)(578.39004021,642.47172856)(577.6608741,643.16617248)
\curveto(576.93170798,643.86061639)(576.56712493,644.74603239)(576.56712493,645.82242047)
\curveto(576.56712493,646.70436424)(576.75462478,647.41616926)(577.1296245,647.95783552)
\curveto(577.51156866,648.50644621)(578.05323491,648.93700144)(578.75462327,649.24950121)
\curveto(579.46295607,649.56200097)(580.31364987,649.77380637)(581.30670467,649.88491739)
\curveto(582.29975948,649.99602842)(583.36573089,650.07936169)(584.50461892,650.1349172)
\lineto(584.50461892,650.43700031)
\curveto(584.50461892,650.88144442)(584.42475787,651.24949969)(584.26503577,651.54116614)
\curveto(584.11225811,651.83283259)(583.89003605,652.06199908)(583.59836961,652.22866562)
\curveto(583.32059204,652.38838772)(582.98725896,652.49602653)(582.59837036,652.55158204)
\curveto(582.20948177,652.60713756)(581.80323208,652.63491531)(581.37962129,652.63491531)
\curveto(580.86573279,652.63491531)(580.29281655,652.56547092)(579.66087259,652.42658214)
\curveto(579.02892862,652.29463779)(578.37615134,652.10019349)(577.70254073,651.84324924)
\lineto(577.59837415,651.84324924)
\lineto(577.59837415,653.83283107)
\curveto(577.9803183,653.93699766)(578.53240122,654.05158091)(579.25462289,654.17658081)
\curveto(579.97684457,654.30158072)(580.68864959,654.36408067)(581.39003794,654.36408067)
\curveto(582.20948177,654.36408067)(582.92128679,654.29463628)(583.525453,654.15574749)
\curveto(584.13656364,654.02380315)(584.66434102,653.79463666)(585.10878513,653.46824801)
\curveto(585.5462848,653.14880381)(585.87961788,652.73560968)(586.10878437,652.22866562)
\curveto(586.33795087,651.72172156)(586.45253411,651.09324981)(586.45253411,650.34325038)
\closepath
\moveto(584.50461892,645.31200419)
\lineto(584.50461892,648.55158507)
\curveto(583.90739715,648.51686287)(583.20253657,648.46477958)(582.39003719,648.39533519)
\curveto(581.58448224,648.32589079)(580.94559384,648.22519643)(580.47337197,648.09325208)
\curveto(579.9108724,647.93352998)(579.45601163,647.68353017)(579.10878967,647.34325265)
\curveto(578.76156771,647.00991957)(578.58795673,646.54811436)(578.58795673,645.95783703)
\curveto(578.58795673,645.29117087)(578.78934547,644.78769903)(579.19212294,644.44742151)
\curveto(579.59490041,644.11408842)(580.20948328,643.94742188)(581.03587155,643.94742188)
\curveto(581.72337103,643.94742188)(582.35184277,644.07936623)(582.92128679,644.34325492)
\curveto(583.4907308,644.61408805)(584.01850818,644.93700447)(584.50461892,645.31200419)
\closepath
}
}
{
\newrgbcolor{curcolor}{0 0 0}
\pscustom[linestyle=none,fillstyle=solid,fillcolor=curcolor]
{
\newpath
\moveto(598.53586319,645.80158715)
\curveto(598.53586319,644.73908795)(598.0948913,643.86756083)(597.21294753,643.18700579)
\curveto(596.33794819,642.50645075)(595.14003243,642.16617323)(593.61920025,642.16617323)
\curveto(592.75808979,642.16617323)(591.96642372,642.2668676)(591.24420204,642.46825634)
\curveto(590.52892481,642.67658951)(589.92823082,642.90228379)(589.44212007,643.14533916)
\lineto(589.44212007,645.34325416)
\lineto(589.54628666,645.34325416)
\curveto(590.16434175,644.87797674)(590.85184123,644.50644924)(591.6087851,644.22867167)
\curveto(592.36572897,643.95783854)(593.09142287,643.82242198)(593.78586679,643.82242198)
\curveto(594.64697725,643.82242198)(595.32058785,643.96131076)(595.80669859,644.23908833)
\curveto(596.29280933,644.5168659)(596.53586471,644.95436557)(596.53586471,645.55158734)
\curveto(596.53586471,646.00992032)(596.40392036,646.35714228)(596.14003167,646.59325322)
\curveto(595.87614298,646.82936415)(595.36919892,647.03075288)(594.61919949,647.19741943)
\curveto(594.34142192,647.25991938)(593.97683886,647.33283599)(593.52545032,647.41616926)
\curveto(593.08100621,647.49950253)(592.67475652,647.58978024)(592.30670124,647.68700239)
\curveto(591.28586868,647.95783552)(590.56017478,648.35366855)(590.12961955,648.87450149)
\curveto(589.70600876,649.40227887)(589.49420337,650.04811171)(589.49420337,650.81200002)
\curveto(589.49420337,651.29116633)(589.59142552,651.74255488)(589.78586981,652.16616567)
\curveto(589.98725855,652.58977646)(590.28934165,652.96824839)(590.69211913,653.30158147)
\curveto(591.08100772,653.62797012)(591.5740629,653.88491437)(592.17128468,654.07241422)
\curveto(592.77545088,654.26685852)(593.44906149,654.36408067)(594.19211648,654.36408067)
\curveto(594.8865604,654.36408067)(595.58794876,654.27727518)(596.29628155,654.1036642)
\curveto(597.01155879,653.93699766)(597.60530834,653.7321367)(598.07753021,653.48908133)
\lineto(598.07753021,651.39533292)
\lineto(597.97336362,651.39533292)
\curveto(597.473364,651.76338819)(596.86572557,652.07241574)(596.15044833,652.32241555)
\curveto(595.43517109,652.5793598)(594.73378274,652.70783192)(594.04628326,652.70783192)
\curveto(593.33100602,652.70783192)(592.72683981,652.56894314)(592.23378463,652.29116557)
\curveto(591.74072945,652.02033244)(591.49420185,651.61408275)(591.49420185,651.07241649)
\curveto(591.49420185,650.59325019)(591.6435073,650.23213935)(591.94211818,649.98908398)
\curveto(592.23378463,649.74602861)(592.70600649,649.54811209)(593.35878378,649.39533443)
\curveto(593.71989461,649.31200116)(594.12267209,649.22866789)(594.5671162,649.14533462)
\curveto(595.01850474,649.06200135)(595.39350446,648.98561252)(595.69211535,648.91616813)
\curveto(596.60183688,648.70783495)(597.30322524,648.35019633)(597.79628042,647.84325227)
\curveto(598.2893356,647.32936377)(598.53586319,646.64880873)(598.53586319,645.80158715)
\closepath
}
}
{
\newrgbcolor{curcolor}{0 0 0}
\pscustom[linestyle=none,fillstyle=solid,fillcolor=curcolor]
{
\newpath
\moveto(616.09834028,651.94741583)
\lineto(615.99417369,651.94741583)
\curveto(615.70250725,652.01686022)(615.41778524,652.0654713)(615.14000767,652.09324905)
\curveto(614.86917454,652.12797125)(614.54625812,652.14533235)(614.1712584,652.14533235)
\curveto(613.5670922,652.14533235)(612.9837593,652.00991578)(612.42125973,651.73908266)
\curveto(611.85876015,651.47519397)(611.3170939,651.13144423)(610.79626096,650.70783344)
\lineto(610.79626096,642.44742302)
\lineto(608.83792911,642.44742302)
\lineto(608.83792911,654.08283088)
\lineto(610.79626096,654.08283088)
\lineto(610.79626096,652.36408218)
\curveto(611.57403815,652.98908171)(612.25806541,653.4300536)(612.84834274,653.68699785)
\curveto(613.44556451,653.95088654)(614.05320294,654.08283088)(614.67125803,654.08283088)
\curveto(615.01153555,654.08283088)(615.25806314,654.07241422)(615.4108408,654.05158091)
\curveto(615.56361846,654.03769203)(615.79278496,654.00644205)(616.09834028,653.95783098)
\closepath
}
}
{
\newrgbcolor{curcolor}{0 0 0}
\pscustom[linestyle=none,fillstyle=solid,fillcolor=curcolor]
{
\newpath
\moveto(627.68166757,648.0620021)
\lineto(619.10875739,648.0620021)
\curveto(619.10875739,647.34672487)(619.2163962,646.72172534)(619.43167381,646.18700352)
\curveto(619.64695143,645.65922614)(619.94209009,645.2251987)(620.31708981,644.88492118)
\curveto(620.67820065,644.55158809)(621.10528366,644.30158828)(621.59833884,644.13492174)
\curveto(622.09833846,643.9682552)(622.64694916,643.88492193)(623.24417093,643.88492193)
\curveto(624.035837,643.88492193)(624.83097529,644.04117181)(625.62958579,644.35367158)
\curveto(626.43514074,644.67311578)(627.00805697,644.98561554)(627.34833449,645.29117087)
\lineto(627.45250108,645.29117087)
\lineto(627.45250108,643.15575582)
\curveto(626.79277936,642.87797825)(626.11916876,642.64533954)(625.43166928,642.45783968)
\curveto(624.7441698,642.27033982)(624.02194812,642.17658989)(623.26500425,642.17658989)
\curveto(621.33445015,642.17658989)(619.82750685,642.69742283)(618.74417433,643.73908871)
\curveto(617.66084182,644.78769903)(617.11917556,646.27380901)(617.11917556,648.19741867)
\curveto(617.11917556,650.10019501)(617.63653628,651.61061053)(618.67125772,652.72866524)
\curveto(619.7129236,653.84671995)(621.08097812,654.40574731)(622.77542128,654.40574731)
\curveto(624.34486454,654.40574731)(625.55319696,653.94741432)(626.40041854,653.03074835)
\curveto(627.25458456,652.11408237)(627.68166757,650.81200002)(627.68166757,649.1245013)
\closepath
\moveto(625.77541902,649.56200097)
\curveto(625.76847458,650.58977797)(625.50805811,651.38491626)(624.99416961,651.94741583)
\curveto(624.48722555,652.50991541)(623.71292058,652.79116519)(622.6712547,652.79116519)
\curveto(621.62264438,652.79116519)(620.78583946,652.48213765)(620.16083993,651.86408256)
\curveto(619.54278484,651.24602747)(619.19209066,650.47866694)(619.10875739,649.56200097)
\closepath
}
}
{
\newrgbcolor{curcolor}{0 0 0}
\pscustom[linestyle=none,fillstyle=solid,fillcolor=curcolor]
{
\newpath
\moveto(640.08791139,638.1557596)
\lineto(638.12957954,638.1557596)
\lineto(638.12957954,643.75992203)
\curveto(637.52541333,643.23908909)(636.92471934,642.85020049)(636.32749757,642.59325624)
\curveto(635.7302758,642.34325643)(635.08444296,642.21825653)(634.38999904,642.21825653)
\curveto(633.00805564,642.21825653)(631.90388981,642.74950612)(631.07750154,643.81200532)
\curveto(630.25805772,644.88144896)(629.84833581,646.35367006)(629.84833581,648.22866865)
\curveto(629.84833581,649.22866789)(629.99069681,650.11061167)(630.27541882,650.87449998)
\curveto(630.56708526,651.64533273)(630.94902942,652.29116557)(631.42125128,652.81199851)
\curveto(631.87958427,653.31894257)(632.41777831,653.71130339)(633.0358334,653.98908095)
\curveto(633.65388848,654.26685852)(634.30666577,654.40574731)(634.99416525,654.40574731)
\curveto(635.61916477,654.40574731)(636.17124769,654.33630291)(636.65041399,654.19741413)
\curveto(637.13652474,654.05852535)(637.62957992,653.85366439)(638.12957954,653.58283126)
\lineto(638.25457945,654.08283088)
\lineto(640.08791139,654.08283088)
\closepath
\moveto(638.12957954,645.40575411)
\lineto(638.12957954,651.98908247)
\curveto(637.58791328,652.23213784)(637.10874698,652.4022766)(636.69208063,652.49949875)
\curveto(636.27541428,652.5967209)(635.82402573,652.64533197)(635.33791499,652.64533197)
\curveto(634.2059714,652.64533197)(633.34486094,652.2599156)(632.75458361,651.48908285)
\curveto(632.16430628,650.72519453)(631.86916761,649.66963978)(631.86916761,648.32241857)
\curveto(631.86916761,646.96130849)(632.10527854,645.91269818)(632.57750041,645.17658762)
\curveto(633.05666671,644.44742151)(633.80666615,644.08283845)(634.82749871,644.08283845)
\curveto(635.39694272,644.08283845)(635.96638673,644.20436613)(636.53583075,644.44742151)
\curveto(637.10527476,644.69742132)(637.63652436,645.01686552)(638.12957954,645.40575411)
\closepath
}
}
{
\newrgbcolor{curcolor}{0 0 0}
\pscustom[linestyle=none,fillstyle=solid,fillcolor=curcolor]
{
\newpath
\moveto(653.58789938,642.44742302)
\lineto(651.62956753,642.44742302)
\lineto(651.62956753,643.73908871)
\curveto(650.96984581,643.21825577)(650.33790184,642.81895052)(649.73373563,642.54117295)
\curveto(649.12956942,642.26339538)(648.46290326,642.1245066)(647.73373714,642.1245066)
\curveto(646.51151585,642.1245066)(645.56012768,642.49603409)(644.87957264,643.23908909)
\curveto(644.1990176,643.98908852)(643.85874008,645.08630991)(643.85874008,646.53075326)
\lineto(643.85874008,654.08283088)
\lineto(645.81707193,654.08283088)
\lineto(645.81707193,647.4578359)
\curveto(645.81707193,646.86755856)(645.84484968,646.3606145)(645.9004052,645.93700371)
\curveto(645.95596071,645.52033736)(646.07401618,645.16269874)(646.2545716,644.86408786)
\curveto(646.44207145,644.55853253)(646.68512683,644.33631048)(646.98373771,644.1974217)
\curveto(647.2823486,644.05853291)(647.71637605,643.98908852)(648.28582006,643.98908852)
\curveto(648.79276412,643.98908852)(649.34484704,644.12103286)(649.94206881,644.38492155)
\curveto(650.54623502,644.64881024)(651.10873459,644.98561554)(651.62956753,645.39533746)
\lineto(651.62956753,654.08283088)
\lineto(653.58789938,654.08283088)
\closepath
}
}
{
\newrgbcolor{curcolor}{0 0 0}
\pscustom[linestyle=none,fillstyle=solid,fillcolor=curcolor]
{
\newpath
\moveto(667.18162624,648.0620021)
\lineto(658.60871606,648.0620021)
\curveto(658.60871606,647.34672487)(658.71635487,646.72172534)(658.93163248,646.18700352)
\curveto(659.1469101,645.65922614)(659.44204876,645.2251987)(659.81704848,644.88492118)
\curveto(660.17815932,644.55158809)(660.60524233,644.30158828)(661.09829751,644.13492174)
\curveto(661.59829713,643.9682552)(662.14690783,643.88492193)(662.7441296,643.88492193)
\curveto(663.53579567,643.88492193)(664.33093395,644.04117181)(665.12954446,644.35367158)
\curveto(665.93509941,644.67311578)(666.50801564,644.98561554)(666.84829316,645.29117087)
\lineto(666.95245975,645.29117087)
\lineto(666.95245975,643.15575582)
\curveto(666.29273803,642.87797825)(665.61912742,642.64533954)(664.93162794,642.45783968)
\curveto(664.24412846,642.27033982)(663.52190679,642.17658989)(662.76496292,642.17658989)
\curveto(660.83440882,642.17658989)(659.32746552,642.69742283)(658.244133,643.73908871)
\curveto(657.16080049,644.78769903)(656.61913423,646.27380901)(656.61913423,648.19741867)
\curveto(656.61913423,650.10019501)(657.13649495,651.61061053)(658.17121639,652.72866524)
\curveto(659.21288227,653.84671995)(660.58093679,654.40574731)(662.27537995,654.40574731)
\curveto(663.84482321,654.40574731)(665.05315563,653.94741432)(665.90037721,653.03074835)
\curveto(666.75454323,652.11408237)(667.18162624,650.81200002)(667.18162624,649.1245013)
\closepath
\moveto(665.27537768,649.56200097)
\curveto(665.26843324,650.58977797)(665.00801678,651.38491626)(664.49412828,651.94741583)
\curveto(663.98718421,652.50991541)(663.21287924,652.79116519)(662.17121337,652.79116519)
\curveto(661.12260305,652.79116519)(660.28579813,652.48213765)(659.6607986,651.86408256)
\curveto(659.04274351,651.24602747)(658.69204933,650.47866694)(658.60871606,649.56200097)
\closepath
}
}
{
\newrgbcolor{curcolor}{0 0 0}
\pscustom[linestyle=none,fillstyle=solid,fillcolor=curcolor]
{
\newpath
\moveto(678.46287091,645.80158715)
\curveto(678.46287091,644.73908795)(678.02189903,643.86756083)(677.13995525,643.18700579)
\curveto(676.26495591,642.50645075)(675.06704015,642.16617323)(673.54620797,642.16617323)
\curveto(672.68509751,642.16617323)(671.89343144,642.2668676)(671.17120976,642.46825634)
\curveto(670.45593253,642.67658951)(669.85523854,642.90228379)(669.36912779,643.14533916)
\lineto(669.36912779,645.34325416)
\lineto(669.47329438,645.34325416)
\curveto(670.09134947,644.87797674)(670.77884895,644.50644924)(671.53579282,644.22867167)
\curveto(672.29273669,643.95783854)(673.01843059,643.82242198)(673.71287451,643.82242198)
\curveto(674.57398497,643.82242198)(675.24759557,643.96131076)(675.73370631,644.23908833)
\curveto(676.21981706,644.5168659)(676.46287243,644.95436557)(676.46287243,645.55158734)
\curveto(676.46287243,646.00992032)(676.33092808,646.35714228)(676.06703939,646.59325322)
\curveto(675.8031507,646.82936415)(675.29620664,647.03075288)(674.54620721,647.19741943)
\curveto(674.26842964,647.25991938)(673.90384659,647.33283599)(673.45245804,647.41616926)
\curveto(673.00801393,647.49950253)(672.60176424,647.58978024)(672.23370896,647.68700239)
\curveto(671.2128764,647.95783552)(670.4871825,648.35366855)(670.05662727,648.87450149)
\curveto(669.63301648,649.40227887)(669.42121109,650.04811171)(669.42121109,650.81200002)
\curveto(669.42121109,651.29116633)(669.51843324,651.74255488)(669.71287753,652.16616567)
\curveto(669.91426627,652.58977646)(670.21634937,652.96824839)(670.61912685,653.30158147)
\curveto(671.00801544,653.62797012)(671.50107063,653.88491437)(672.0982924,654.07241422)
\curveto(672.70245861,654.26685852)(673.37606921,654.36408067)(674.1191242,654.36408067)
\curveto(674.81356812,654.36408067)(675.51495648,654.27727518)(676.22328927,654.1036642)
\curveto(676.93856651,653.93699766)(677.53231606,653.7321367)(678.00453793,653.48908133)
\lineto(678.00453793,651.39533292)
\lineto(677.90037134,651.39533292)
\curveto(677.40037172,651.76338819)(676.79273329,652.07241574)(676.07745605,652.32241555)
\curveto(675.36217882,652.5793598)(674.66079046,652.70783192)(673.97329098,652.70783192)
\curveto(673.25801374,652.70783192)(672.65384753,652.56894314)(672.16079235,652.29116557)
\curveto(671.66773717,652.02033244)(671.42120957,651.61408275)(671.42120957,651.07241649)
\curveto(671.42120957,650.59325019)(671.57051502,650.23213935)(671.8691259,649.98908398)
\curveto(672.16079235,649.74602861)(672.63301421,649.54811209)(673.2857915,649.39533443)
\curveto(673.64690234,649.31200116)(674.04967981,649.22866789)(674.49412392,649.14533462)
\curveto(674.94551246,649.06200135)(675.32051218,648.98561252)(675.61912307,648.91616813)
\curveto(676.5288446,648.70783495)(677.23023296,648.35019633)(677.72328814,647.84325227)
\curveto(678.21634332,647.32936377)(678.46287091,646.64880873)(678.46287091,645.80158715)
\closepath
}
}
{
\newrgbcolor{curcolor}{0 0 0}
\pscustom[linestyle=none,fillstyle=solid,fillcolor=curcolor]
{
\newpath
\moveto(687.30662776,642.55158961)
\curveto(686.93857248,642.45436746)(686.53579501,642.37450641)(686.09829534,642.31200645)
\curveto(685.66774011,642.2495065)(685.28232373,642.21825653)(684.94204621,642.21825653)
\curveto(683.75454711,642.21825653)(682.85177002,642.53770073)(682.23371493,643.17658913)
\curveto(681.61565984,643.81547754)(681.3066323,644.83978232)(681.3066323,646.24950348)
\lineto(681.3066323,652.43699879)
\lineto(679.98371663,652.43699879)
\lineto(679.98371663,654.08283088)
\lineto(681.3066323,654.08283088)
\lineto(681.3066323,657.42657835)
\lineto(683.26496415,657.42657835)
\lineto(683.26496415,654.08283088)
\lineto(687.30662776,654.08283088)
\lineto(687.30662776,652.43699879)
\lineto(683.26496415,652.43699879)
\lineto(683.26496415,647.13491947)
\curveto(683.26496415,646.52380882)(683.27885303,646.04464252)(683.30663078,645.69742056)
\curveto(683.33440854,645.35714304)(683.43163069,645.03769884)(683.59829723,644.73908795)
\curveto(683.75107489,644.46131038)(683.95940807,644.25644943)(684.22329676,644.12450508)
\curveto(684.49412989,643.99950518)(684.9038518,643.93700523)(685.45246249,643.93700523)
\curveto(685.7719067,643.93700523)(686.10523978,643.98214408)(686.45246174,644.07242179)
\curveto(686.7996837,644.16964394)(687.04968351,644.24950499)(687.20246117,644.31200494)
\lineto(687.30662776,644.31200494)
\closepath
}
}
{
\newrgbcolor{curcolor}{0 0 0}
\pscustom[linestyle=none,fillstyle=solid,fillcolor=curcolor]
{
\newpath
\moveto(699.41078394,648.0620021)
\lineto(690.83787376,648.0620021)
\curveto(690.83787376,647.34672487)(690.94551256,646.72172534)(691.16079018,646.18700352)
\curveto(691.37606779,645.65922614)(691.67120646,645.2251987)(692.04620618,644.88492118)
\curveto(692.40731701,644.55158809)(692.83440002,644.30158828)(693.32745521,644.13492174)
\curveto(693.82745483,643.9682552)(694.37606552,643.88492193)(694.97328729,643.88492193)
\curveto(695.76495336,643.88492193)(696.56009165,644.04117181)(697.35870216,644.35367158)
\curveto(698.1642571,644.67311578)(698.73717334,644.98561554)(699.07745086,645.29117087)
\lineto(699.18161744,645.29117087)
\lineto(699.18161744,643.15575582)
\curveto(698.52189572,642.87797825)(697.84828512,642.64533954)(697.16078564,642.45783968)
\curveto(696.47328616,642.27033982)(695.75106448,642.17658989)(694.99412061,642.17658989)
\curveto(693.06356652,642.17658989)(691.55662321,642.69742283)(690.4732907,643.73908871)
\curveto(689.38995819,644.78769903)(688.84829193,646.27380901)(688.84829193,648.19741867)
\curveto(688.84829193,650.10019501)(689.36565265,651.61061053)(690.40037409,652.72866524)
\curveto(691.44203997,653.84671995)(692.81009449,654.40574731)(694.50453765,654.40574731)
\curveto(696.07398091,654.40574731)(697.28231333,653.94741432)(698.12953491,653.03074835)
\curveto(698.98370093,652.11408237)(699.41078394,650.81200002)(699.41078394,649.1245013)
\closepath
\moveto(697.50453538,649.56200097)
\curveto(697.49759094,650.58977797)(697.23717447,651.38491626)(696.72328597,651.94741583)
\curveto(696.21634191,652.50991541)(695.44203694,652.79116519)(694.40037106,652.79116519)
\curveto(693.35176074,652.79116519)(692.51495582,652.48213765)(691.88995629,651.86408256)
\curveto(691.27190121,651.24602747)(690.92120703,650.47866694)(690.83787376,649.56200097)
\closepath
}
}
{
\newrgbcolor{curcolor}{0 0 0}
\pscustom[linestyle=none,fillstyle=solid,fillcolor=curcolor]
{
\newpath
\moveto(711.81702776,642.44742302)
\lineto(709.85869591,642.44742302)
\lineto(709.85869591,643.6661721)
\curveto(709.29619633,643.18006135)(708.70939122,642.80158942)(708.09828057,642.53075629)
\curveto(707.48716992,642.25992316)(706.82397598,642.1245066)(706.10869874,642.1245066)
\curveto(704.71981091,642.1245066)(703.61564507,642.65922841)(702.79620125,643.72867205)
\curveto(701.98370186,644.79811569)(701.57745217,646.28075345)(701.57745217,648.17658535)
\curveto(701.57745217,649.16269572)(701.71634096,650.04116727)(701.99411852,650.81200002)
\curveto(702.27884053,651.58283277)(702.66078469,652.23908228)(703.13995099,652.78074853)
\curveto(703.61217285,653.30852591)(704.16078355,653.71130339)(704.78578308,653.98908095)
\curveto(705.41772704,654.26685852)(706.07050433,654.40574731)(706.74411493,654.40574731)
\curveto(707.35522558,654.40574731)(707.89689184,654.33977513)(708.3691137,654.20783079)
\curveto(708.84133557,654.08283088)(709.33786297,653.88491437)(709.85869591,653.61408124)
\lineto(709.85869591,658.65574409)
\lineto(711.81702776,658.65574409)
\closepath
\moveto(709.85869591,645.31200419)
\lineto(709.85869591,651.98908247)
\curveto(709.33091853,652.2251934)(708.85869666,652.38838772)(708.44203031,652.47866543)
\curveto(708.02536396,652.56894314)(707.57050319,652.61408199)(707.07744801,652.61408199)
\curveto(705.98022662,652.61408199)(705.1260606,652.23213784)(704.51494995,651.46824953)
\curveto(703.9038393,650.70436122)(703.59828398,649.6210287)(703.59828398,648.21825199)
\curveto(703.59828398,646.83630859)(703.83439491,645.78422605)(704.30661677,645.06200437)
\curveto(704.77883864,644.34672714)(705.53578251,643.98908852)(706.57744839,643.98908852)
\curveto(707.13300352,643.98908852)(707.6955031,644.11061621)(708.26494711,644.35367158)
\curveto(708.83439113,644.60367139)(709.36564072,644.92311559)(709.85869591,645.31200419)
\closepath
}
}
{
\newrgbcolor{curcolor}{0 0 0}
\pscustom[linewidth=2.05937378,linecolor=curcolor]
{
\newpath
\moveto(561.25984252,1715.68450772)
\lineto(561.25984252,1585.29111685)
}
}
{
\newrgbcolor{curcolor}{0 0 0}
\pscustom[linestyle=none,fillstyle=solid,fillcolor=curcolor]
{
\newpath
\moveto(561.25984252,1605.88485466)
\lineto(553.0223474,1614.12234978)
\lineto(561.25984252,1585.29111685)
\lineto(569.49733764,1614.12234978)
\closepath
}
}
{
\newrgbcolor{curcolor}{0 0 0}
\pscustom[linewidth=2.19666543,linecolor=curcolor]
{
\newpath
\moveto(561.25984252,1605.88485466)
\lineto(553.0223474,1614.12234978)
\lineto(561.25984252,1585.29111685)
\lineto(569.49733764,1614.12234978)
\closepath
}
}
{
\newrgbcolor{curcolor}{0 0 0}
\pscustom[linewidth=2.05937378,linecolor=curcolor]
{
\newpath
\moveto(561.25984252,1426.28110866)
\lineto(561.25984252,1295.88771024)
}
}
{
\newrgbcolor{curcolor}{0 0 0}
\pscustom[linestyle=none,fillstyle=solid,fillcolor=curcolor]
{
\newpath
\moveto(561.25984252,1316.48144804)
\lineto(553.0223474,1324.71894316)
\lineto(561.25984252,1295.88771024)
\lineto(569.49733764,1324.71894316)
\closepath
}
}
{
\newrgbcolor{curcolor}{0 0 0}
\pscustom[linewidth=2.19666543,linecolor=curcolor]
{
\newpath
\moveto(561.25984252,1316.48144804)
\lineto(553.0223474,1324.71894316)
\lineto(561.25984252,1295.88771024)
\lineto(569.49733764,1324.71894316)
\closepath
}
}
{
\newrgbcolor{curcolor}{0 0 0}
\pscustom[linewidth=2.05937378,linecolor=curcolor]
{
\newpath
\moveto(561.25984252,1135.20037795)
\lineto(561.25984252,1004.80697953)
}
}
{
\newrgbcolor{curcolor}{0 0 0}
\pscustom[linestyle=none,fillstyle=solid,fillcolor=curcolor]
{
\newpath
\moveto(561.25984252,1025.40071733)
\lineto(553.0223474,1033.63821245)
\lineto(561.25984252,1004.80697953)
\lineto(569.49733764,1033.63821245)
\closepath
}
}
{
\newrgbcolor{curcolor}{0 0 0}
\pscustom[linewidth=2.19666543,linecolor=curcolor]
{
\newpath
\moveto(561.25984252,1025.40071733)
\lineto(553.0223474,1033.63821245)
\lineto(561.25984252,1004.80697953)
\lineto(569.49733764,1033.63821245)
\closepath
}
}
{
\newrgbcolor{curcolor}{0 0 0}
\pscustom[linewidth=2.05937378,linecolor=curcolor]
{
\newpath
\moveto(561.25984252,862.96429606)
\lineto(561.25984252,732.57089764)
}
}
{
\newrgbcolor{curcolor}{0 0 0}
\pscustom[linestyle=none,fillstyle=solid,fillcolor=curcolor]
{
\newpath
\moveto(561.25984252,753.16463544)
\lineto(553.0223474,761.40213056)
\lineto(561.25984252,732.57089764)
\lineto(569.49733764,761.40213056)
\closepath
}
}
{
\newrgbcolor{curcolor}{0 0 0}
\pscustom[linewidth=2.19666543,linecolor=curcolor]
{
\newpath
\moveto(561.25984252,753.16463544)
\lineto(553.0223474,761.40213056)
\lineto(561.25984252,732.57089764)
\lineto(569.49733764,761.40213056)
\closepath
}
}
{
\newrgbcolor{curcolor}{0 0 0}
\pscustom[linewidth=2.05937378,linecolor=curcolor]
{
\newpath
\moveto(561.25984252,589.71802205)
\lineto(561.25984252,459.32462362)
}
}
{
\newrgbcolor{curcolor}{0 0 0}
\pscustom[linestyle=none,fillstyle=solid,fillcolor=curcolor]
{
\newpath
\moveto(561.25984252,479.91836143)
\lineto(553.0223474,488.15585655)
\lineto(561.25984252,459.32462362)
\lineto(569.49733764,488.15585655)
\closepath
}
}
{
\newrgbcolor{curcolor}{0 0 0}
\pscustom[linewidth=2.19666543,linecolor=curcolor]
{
\newpath
\moveto(561.25984252,479.91836143)
\lineto(553.0223474,488.15585655)
\lineto(561.25984252,459.32462362)
\lineto(569.49733764,488.15585655)
\closepath
}
}
{
\newrgbcolor{curcolor}{0 0 0}
\pscustom[linewidth=2.05937378,linecolor=curcolor]
{
\newpath
\moveto(561.25984252,316.97684409)
\lineto(561.25984252,186.58344567)
}
}
{
\newrgbcolor{curcolor}{0 0 0}
\pscustom[linestyle=none,fillstyle=solid,fillcolor=curcolor]
{
\newpath
\moveto(561.25984252,207.17718347)
\lineto(553.0223474,215.4146786)
\lineto(561.25984252,186.58344567)
\lineto(569.49733764,215.4146786)
\closepath
}
}
{
\newrgbcolor{curcolor}{0 0 0}
\pscustom[linewidth=2.19666543,linecolor=curcolor]
{
\newpath
\moveto(561.25984252,207.17718347)
\lineto(553.0223474,215.4146786)
\lineto(561.25984252,186.58344567)
\lineto(569.49733764,215.4146786)
\closepath
}
}
{
\newrgbcolor{curcolor}{0 0 0}
\pscustom[linewidth=2.05984258,linecolor=curcolor]
{
\newpath
\moveto(767.14284094,382.62088819)
\curveto(859.52363614,555.95393725)(951.90456641,729.28723974)(951.90418987,869.28794888)
\curveto(951.90381334,1009.28865803)(859.52470957,1115.95319581)(767.14284094,1222.62092598)
}
}
{
\newrgbcolor{curcolor}{0 0 0}
\pscustom[linestyle=none,fillstyle=solid,fillcolor=curcolor]
{
\newpath
\moveto(780.62808512,1207.05033396)
\lineto(792.2504196,1206.21619482)
\lineto(767.14284094,1222.62092598)
\lineto(779.79394598,1195.42799947)
\closepath
}
}
{
\newrgbcolor{curcolor}{0 0 0}
\pscustom[linewidth=2.19716549,linecolor=curcolor]
{
\newpath
\moveto(780.62808512,1207.05033396)
\lineto(792.2504196,1206.21619482)
\lineto(767.14284094,1222.62092598)
\lineto(779.79394598,1195.42799947)
\closepath
}
}
{
\newrgbcolor{curcolor}{0 0 0}
\pscustom[linewidth=2.0787402,linecolor=curcolor]
{
\newpath
\moveto(354.28571339,104.04948661)
\curveto(252.38135776,391.66739145)(150.47661679,679.2863839)(148.09632337,959.28942703)
\curveto(145.71602995,1239.29247015)(242.85694607,1511.66797963)(340.00000252,1784.04949039)
}
}
{
\newrgbcolor{curcolor}{0 0 0}
\pscustom[linestyle=none,fillstyle=solid,fillcolor=curcolor]
{
\newpath
\moveto(333.01711692,1764.47002688)
\lineto(338.05574808,1753.84508723)
\lineto(340.00000252,1784.04949039)
\lineto(322.39217727,1759.43139571)
\closepath
}
}
{
\newrgbcolor{curcolor}{0 0 0}
\pscustom[linewidth=2.21732295,linecolor=curcolor]
{
\newpath
\moveto(333.01711692,1764.47002688)
\lineto(338.05574808,1753.84508723)
\lineto(340.00000252,1784.04949039)
\lineto(322.39217727,1759.43139571)
\closepath
}
}
\end{pspicture}
}
    \captionsetup{width=0.75\linewidth}
    \caption{A flowchart of how automatic testing worked.}
    \label{fig:testflowchart}
\end{figure}

Either fio\cite{fio} or simpleread (Listing \ref{lst:simplereadinline}) was used as the test program, 
with fio measuring itself and simpleread timed by the time command.

fio is the Flexible I/O Tester, written by Jen Axboe\cite{fio}.
It is a powerful tool for benchmarking the performance of every possible kind of read and write operation on Linux.
I simply used its sequential read capabilities with known files on my test ZFS filesystem to get more accurate measures of
latency and bandwidth, as fio accurately times each individual read operation in order to calculate its statistics.
Through these measurements it can calculate various percentiles of latency and also read bandwidth. 
These metrics were invaluable for detecting differences in read operations between my prototype and normal ZFS.

\section{Demonstrating the Effects}
As a direct demonstration of the effects of NUMA on ZFS, consider this simple file-reading program (Listing \ref{lst:simplereadinline}).
It takes a filename and a read size, how many bytes to read at once, and reads through the entire file sequentially before exiting.

\singlespacing
\lstinputlisting[caption={Simple Read Program},label={lst:simplereadinline},language=C]{code/simpleread.c}
\doublespacing

In order to measure the effects of memory latency on ZFS, I generated files full of random data on a ZFS filesystem.
To get a sense of how file size affected the results of my tests,
I generated files ranging in size from 128 KB to 16 GB, doubling in size each time.
However, the tests presented typically show results from 250 MB and up, as this is where the effects of NUMA become most visible.
Tests involved reading these files with various programs, one file per boot to make sure the ARC and other kernel caches were fully flushed between tests.
I ensured the ARC data being read was located on the
right node by taking advantage of the fact that ZFS will initially load data into the ARC on the same node as the program that first requests it.
Thus I simply ran this program once, bound to the correct node, before taking any measurements in order to ensure a consistent state of the ARC.

Using run time as a coarse measurement of latency, we find that this program takes less time if run on the same node as the ARC data it is consuming, when that ARC data is 1 GB or larger.
It improves from about 10\% less time for a 1 GB file up to a maximum of 15\% less time for a 16 GB file
(Figure \ref{fig:OldZFS}).

\begin{figure}[H]
    \centering
    \captionsetup{width=0.75\linewidth}
    \resizebox{0.75\linewidth}{!}{%LaTeX with PSTricks extensions
%%Creator: Inkscape 1.0.2-2 (e86c870879, 2021-01-15)
%%Please note this file requires PSTricks extensions
\psset{xunit=.5pt,yunit=.5pt,runit=.5pt}
\begin{pspicture}(600,480)
{
\newrgbcolor{curcolor}{0 0 0}
\pscustom[linewidth=1,linecolor=curcolor]
{
\newpath
\moveto(105.1,57.6)
\lineto(114.1,57.6)
\moveto(575,57.6)
\lineto(566,57.6)
}
}
{
\newrgbcolor{curcolor}{0 0 0}
\pscustom[linestyle=none,fillstyle=solid,fillcolor=curcolor]
{
\newpath
\moveto(53.92109375,57.93632812)
\curveto(53.92109375,58.95195312)(54.02460937,59.76835937)(54.23164062,60.38554687)
\curveto(54.44257812,61.00664062)(54.753125,61.48515625)(55.16328125,61.82109375)
\curveto(55.57734375,62.15703125)(56.096875,62.325)(56.721875,62.325)
\curveto(57.1828125,62.325)(57.58710937,62.23125)(57.93476562,62.04375)
\curveto(58.28242187,61.86015625)(58.56953125,61.59257812)(58.79609375,61.24101562)
\curveto(59.02265625,60.89335937)(59.20039062,60.46757812)(59.32929687,59.96367187)
\curveto(59.45820312,59.46367187)(59.52265625,58.78789062)(59.52265625,57.93632812)
\curveto(59.52265625,56.92851562)(59.41914062,56.1140625)(59.21210937,55.49296875)
\curveto(59.00507812,54.87578125)(58.69453125,54.39726562)(58.28046875,54.05742187)
\curveto(57.8703125,53.72148437)(57.35078125,53.55351562)(56.721875,53.55351562)
\curveto(55.89375,53.55351562)(55.24335937,53.85039062)(54.77070312,54.44414062)
\curveto(54.20429687,55.15898437)(53.92109375,56.32304687)(53.92109375,57.93632812)
\closepath
\moveto(55.00507812,57.93632812)
\curveto(55.00507812,56.52617187)(55.16914062,55.58671875)(55.49726562,55.11796875)
\curveto(55.82929687,54.653125)(56.2375,54.42070312)(56.721875,54.42070312)
\curveto(57.20625,54.42070312)(57.6125,54.65507812)(57.940625,55.12382812)
\curveto(58.27265625,55.59257812)(58.43867187,56.53007812)(58.43867187,57.93632812)
\curveto(58.43867187,59.35039062)(58.27265625,60.28984375)(57.940625,60.7546875)
\curveto(57.6125,61.21953125)(57.20234375,61.45195312)(56.71015625,61.45195312)
\curveto(56.22578125,61.45195312)(55.8390625,61.246875)(55.55,60.83671875)
\curveto(55.18671875,60.31328125)(55.00507812,59.34648437)(55.00507812,57.93632812)
\closepath
}
}
{
\newrgbcolor{curcolor}{0 0 0}
\pscustom[linestyle=none,fillstyle=solid,fillcolor=curcolor]
{
\newpath
\moveto(61.18671875,53.7)
\lineto(61.18671875,54.90117187)
\lineto(62.38789062,54.90117187)
\lineto(62.38789062,53.7)
\closepath
}
}
{
\newrgbcolor{curcolor}{0 0 0}
\pscustom[linestyle=none,fillstyle=solid,fillcolor=curcolor]
{
\newpath
\moveto(63.92890625,57.93632812)
\curveto(63.92890625,58.95195312)(64.03242187,59.76835937)(64.23945312,60.38554687)
\curveto(64.45039062,61.00664062)(64.7609375,61.48515625)(65.17109375,61.82109375)
\curveto(65.58515625,62.15703125)(66.1046875,62.325)(66.7296875,62.325)
\curveto(67.190625,62.325)(67.59492187,62.23125)(67.94257812,62.04375)
\curveto(68.29023437,61.86015625)(68.57734375,61.59257812)(68.80390625,61.24101562)
\curveto(69.03046875,60.89335937)(69.20820312,60.46757812)(69.33710937,59.96367187)
\curveto(69.46601562,59.46367187)(69.53046875,58.78789062)(69.53046875,57.93632812)
\curveto(69.53046875,56.92851562)(69.42695312,56.1140625)(69.21992187,55.49296875)
\curveto(69.01289062,54.87578125)(68.70234375,54.39726562)(68.28828125,54.05742187)
\curveto(67.878125,53.72148437)(67.35859375,53.55351562)(66.7296875,53.55351562)
\curveto(65.9015625,53.55351562)(65.25117187,53.85039062)(64.77851562,54.44414062)
\curveto(64.21210937,55.15898437)(63.92890625,56.32304687)(63.92890625,57.93632812)
\closepath
\moveto(65.01289062,57.93632812)
\curveto(65.01289062,56.52617187)(65.17695312,55.58671875)(65.50507812,55.11796875)
\curveto(65.83710937,54.653125)(66.2453125,54.42070312)(66.7296875,54.42070312)
\curveto(67.2140625,54.42070312)(67.6203125,54.65507812)(67.9484375,55.12382812)
\curveto(68.28046875,55.59257812)(68.44648437,56.53007812)(68.44648437,57.93632812)
\curveto(68.44648437,59.35039062)(68.28046875,60.28984375)(67.9484375,60.7546875)
\curveto(67.6203125,61.21953125)(67.21015625,61.45195312)(66.71796875,61.45195312)
\curveto(66.23359375,61.45195312)(65.846875,61.246875)(65.5578125,60.83671875)
\curveto(65.19453125,60.31328125)(65.01289062,59.34648437)(65.01289062,57.93632812)
\closepath
}
}
{
\newrgbcolor{curcolor}{0 0 0}
\pscustom[linestyle=none,fillstyle=solid,fillcolor=curcolor]
{
\newpath
\moveto(70.60859375,55.96757812)
\lineto(71.66328125,56.10820312)
\curveto(71.784375,55.51054687)(71.98945312,55.07890625)(72.27851562,54.81328125)
\curveto(72.57148437,54.5515625)(72.92695312,54.42070312)(73.34492187,54.42070312)
\curveto(73.84101562,54.42070312)(74.25898437,54.59257812)(74.59882812,54.93632812)
\curveto(74.94257812,55.28007812)(75.11445312,55.70585937)(75.11445312,56.21367187)
\curveto(75.11445312,56.69804687)(74.95625,57.09648437)(74.63984375,57.40898437)
\curveto(74.3234375,57.72539062)(73.92109375,57.88359375)(73.4328125,57.88359375)
\curveto(73.23359375,57.88359375)(72.98554687,57.84453125)(72.68867187,57.76640625)
\lineto(72.80585937,58.6921875)
\curveto(72.87617187,58.684375)(72.9328125,58.68046875)(72.97578125,58.68046875)
\curveto(73.425,58.68046875)(73.82929687,58.79765625)(74.18867187,59.03203125)
\curveto(74.54804687,59.26640625)(74.72773437,59.62773437)(74.72773437,60.11601562)
\curveto(74.72773437,60.50273437)(74.596875,60.82304687)(74.33515625,61.07695312)
\curveto(74.0734375,61.33085937)(73.73554687,61.4578125)(73.32148437,61.4578125)
\curveto(72.91132812,61.4578125)(72.56953125,61.32890625)(72.29609375,61.07109375)
\curveto(72.02265625,60.81328125)(71.846875,60.4265625)(71.76875,59.9109375)
\lineto(70.7140625,60.0984375)
\curveto(70.84296875,60.80546875)(71.1359375,61.35234375)(71.59296875,61.7390625)
\curveto(72.05,62.1296875)(72.61835937,62.325)(73.29804687,62.325)
\curveto(73.76679687,62.325)(74.1984375,62.2234375)(74.59296875,62.0203125)
\curveto(74.9875,61.82109375)(75.28828125,61.54765625)(75.4953125,61.2)
\curveto(75.70625,60.85234375)(75.81171875,60.48320312)(75.81171875,60.09257812)
\curveto(75.81171875,59.72148437)(75.71210937,59.38359375)(75.51289062,59.07890625)
\curveto(75.31367187,58.77421875)(75.01875,58.53203125)(74.628125,58.35234375)
\curveto(75.1359375,58.23515625)(75.53046875,57.99101562)(75.81171875,57.61992187)
\curveto(76.09296875,57.25273437)(76.23359375,56.79179687)(76.23359375,56.23710937)
\curveto(76.23359375,55.48710937)(75.96015625,54.85039062)(75.41328125,54.32695312)
\curveto(74.86640625,53.80742187)(74.175,53.54765625)(73.3390625,53.54765625)
\curveto(72.58515625,53.54765625)(71.95820312,53.77226562)(71.45820312,54.22148437)
\curveto(70.96210937,54.67070312)(70.67890625,55.25273437)(70.60859375,55.96757812)
\closepath
}
}
{
\newrgbcolor{curcolor}{0 0 0}
\pscustom[linestyle=none,fillstyle=solid,fillcolor=curcolor]
{
\newpath
\moveto(81.24921875,53.7)
\lineto(80.19453125,53.7)
\lineto(80.19453125,60.42070312)
\curveto(79.940625,60.17851562)(79.60664062,59.93632812)(79.19257812,59.69414062)
\curveto(78.78242187,59.45195312)(78.41328125,59.2703125)(78.08515625,59.14921875)
\lineto(78.08515625,60.16875)
\curveto(78.675,60.44609375)(79.190625,60.78203125)(79.63203125,61.1765625)
\curveto(80.0734375,61.57109375)(80.3859375,61.95390625)(80.56953125,62.325)
\lineto(81.24921875,62.325)
\closepath
}
}
{
\newrgbcolor{curcolor}{0 0 0}
\pscustom[linestyle=none,fillstyle=solid,fillcolor=curcolor]
{
\newpath
\moveto(89.49335937,54.71367187)
\lineto(89.49335937,53.7)
\lineto(83.815625,53.7)
\curveto(83.8078125,53.95390625)(83.84882812,54.19804687)(83.93867187,54.43242187)
\curveto(84.08320312,54.81914062)(84.31367187,55.2)(84.63007812,55.575)
\curveto(84.95039062,55.95)(85.41132812,56.38359375)(86.01289062,56.87578125)
\curveto(86.94648437,57.64140625)(87.57734375,58.246875)(87.90546875,58.6921875)
\curveto(88.23359375,59.14140625)(88.39765625,59.56523437)(88.39765625,59.96367187)
\curveto(88.39765625,60.38164062)(88.24726562,60.73320312)(87.94648437,61.01835937)
\curveto(87.64960937,61.30742187)(87.2609375,61.45195312)(86.78046875,61.45195312)
\curveto(86.27265625,61.45195312)(85.86640625,61.29960937)(85.56171875,60.99492187)
\curveto(85.25703125,60.69023437)(85.10273437,60.26835937)(85.09882812,59.72929687)
\lineto(84.01484375,59.840625)
\curveto(84.0890625,60.64921875)(84.36835937,61.26445312)(84.85273437,61.68632812)
\curveto(85.33710937,62.11210937)(85.9875,62.325)(86.80390625,62.325)
\curveto(87.628125,62.325)(88.28046875,62.09648437)(88.7609375,61.63945312)
\curveto(89.24140625,61.18242187)(89.48164062,60.61601562)(89.48164062,59.94023437)
\curveto(89.48164062,59.59648437)(89.41132812,59.25859375)(89.27070312,58.9265625)
\curveto(89.13007812,58.59453125)(88.89570312,58.24492187)(88.56757812,57.87773437)
\curveto(88.24335937,57.51054687)(87.70234375,57.00664062)(86.94453125,56.36601562)
\curveto(86.31171875,55.83476562)(85.90546875,55.4734375)(85.72578125,55.28203125)
\curveto(85.54609375,55.09453125)(85.39765625,54.90507812)(85.28046875,54.71367187)
\closepath
}
}
{
\newrgbcolor{curcolor}{0 0 0}
\pscustom[linestyle=none,fillstyle=solid,fillcolor=curcolor]
{
\newpath
\moveto(90.62421875,55.95)
\lineto(91.73164062,56.04375)
\curveto(91.81367187,55.5046875)(92.003125,55.0984375)(92.3,54.825)
\curveto(92.60078125,54.55546875)(92.96210937,54.42070312)(93.38398437,54.42070312)
\curveto(93.89179687,54.42070312)(94.32148437,54.61210937)(94.67304687,54.99492187)
\curveto(95.02460937,55.37773437)(95.20039062,55.88554687)(95.20039062,56.51835937)
\curveto(95.20039062,57.11992187)(95.03046875,57.59453125)(94.690625,57.9421875)
\curveto(94.3546875,58.28984375)(93.91328125,58.46367187)(93.36640625,58.46367187)
\curveto(93.0265625,58.46367187)(92.71992187,58.38554687)(92.44648437,58.22929687)
\curveto(92.17304687,58.07695312)(91.95820312,57.87773437)(91.80195312,57.63164062)
\lineto(90.81171875,57.76054687)
\lineto(91.64375,62.17265625)
\lineto(95.91523437,62.17265625)
\lineto(95.91523437,61.16484375)
\lineto(92.4875,61.16484375)
\lineto(92.02460937,58.85625)
\curveto(92.54023437,59.215625)(93.08125,59.3953125)(93.64765625,59.3953125)
\curveto(94.39765625,59.3953125)(95.03046875,59.13554687)(95.54609375,58.61601562)
\curveto(96.06171875,58.09648437)(96.31953125,57.42851562)(96.31953125,56.61210937)
\curveto(96.31953125,55.83476562)(96.09296875,55.16289062)(95.63984375,54.59648437)
\curveto(95.0890625,53.90117187)(94.33710937,53.55351562)(93.38398437,53.55351562)
\curveto(92.60273437,53.55351562)(91.9640625,53.77226562)(91.46796875,54.20976562)
\curveto(90.97578125,54.64726562)(90.69453125,55.22734375)(90.62421875,55.95)
\closepath
}
}
{
\newrgbcolor{curcolor}{0 0 0}
\pscustom[linewidth=1,linecolor=curcolor]
{
\newpath
\moveto(105.1,103.6)
\lineto(114.1,103.6)
\moveto(575,103.6)
\lineto(566,103.6)
}
}
{
\newrgbcolor{curcolor}{0 0 0}
\pscustom[linestyle=none,fillstyle=solid,fillcolor=curcolor]
{
\newpath
\moveto(60.59492187,103.93632812)
\curveto(60.59492187,104.95195312)(60.6984375,105.76835937)(60.90546875,106.38554687)
\curveto(61.11640625,107.00664062)(61.42695312,107.48515625)(61.83710937,107.82109375)
\curveto(62.25117187,108.15703125)(62.77070312,108.325)(63.39570312,108.325)
\curveto(63.85664062,108.325)(64.2609375,108.23125)(64.60859375,108.04375)
\curveto(64.95625,107.86015625)(65.24335937,107.59257812)(65.46992187,107.24101562)
\curveto(65.69648437,106.89335937)(65.87421875,106.46757812)(66.003125,105.96367187)
\curveto(66.13203125,105.46367187)(66.19648437,104.78789062)(66.19648437,103.93632812)
\curveto(66.19648437,102.92851562)(66.09296875,102.1140625)(65.8859375,101.49296875)
\curveto(65.67890625,100.87578125)(65.36835937,100.39726562)(64.95429687,100.05742187)
\curveto(64.54414062,99.72148437)(64.02460937,99.55351562)(63.39570312,99.55351562)
\curveto(62.56757812,99.55351562)(61.9171875,99.85039062)(61.44453125,100.44414062)
\curveto(60.878125,101.15898437)(60.59492187,102.32304687)(60.59492187,103.93632812)
\closepath
\moveto(61.67890625,103.93632812)
\curveto(61.67890625,102.52617187)(61.84296875,101.58671875)(62.17109375,101.11796875)
\curveto(62.503125,100.653125)(62.91132812,100.42070312)(63.39570312,100.42070312)
\curveto(63.88007812,100.42070312)(64.28632812,100.65507812)(64.61445312,101.12382812)
\curveto(64.94648437,101.59257812)(65.1125,102.53007812)(65.1125,103.93632812)
\curveto(65.1125,105.35039062)(64.94648437,106.28984375)(64.61445312,106.7546875)
\curveto(64.28632812,107.21953125)(63.87617187,107.45195312)(63.38398437,107.45195312)
\curveto(62.89960937,107.45195312)(62.51289062,107.246875)(62.22382812,106.83671875)
\curveto(61.86054687,106.31328125)(61.67890625,105.34648437)(61.67890625,103.93632812)
\closepath
}
}
{
\newrgbcolor{curcolor}{0 0 0}
\pscustom[linestyle=none,fillstyle=solid,fillcolor=curcolor]
{
\newpath
\moveto(67.86054687,99.7)
\lineto(67.86054687,100.90117187)
\lineto(69.06171875,100.90117187)
\lineto(69.06171875,99.7)
\closepath
}
}
{
\newrgbcolor{curcolor}{0 0 0}
\pscustom[linestyle=none,fillstyle=solid,fillcolor=curcolor]
{
\newpath
\moveto(70.60273437,103.93632812)
\curveto(70.60273437,104.95195312)(70.70625,105.76835937)(70.91328125,106.38554687)
\curveto(71.12421875,107.00664062)(71.43476562,107.48515625)(71.84492187,107.82109375)
\curveto(72.25898437,108.15703125)(72.77851562,108.325)(73.40351562,108.325)
\curveto(73.86445312,108.325)(74.26875,108.23125)(74.61640625,108.04375)
\curveto(74.9640625,107.86015625)(75.25117187,107.59257812)(75.47773437,107.24101562)
\curveto(75.70429687,106.89335937)(75.88203125,106.46757812)(76.0109375,105.96367187)
\curveto(76.13984375,105.46367187)(76.20429687,104.78789062)(76.20429687,103.93632812)
\curveto(76.20429687,102.92851562)(76.10078125,102.1140625)(75.89375,101.49296875)
\curveto(75.68671875,100.87578125)(75.37617187,100.39726562)(74.96210937,100.05742187)
\curveto(74.55195312,99.72148437)(74.03242187,99.55351562)(73.40351562,99.55351562)
\curveto(72.57539062,99.55351562)(71.925,99.85039062)(71.45234375,100.44414062)
\curveto(70.8859375,101.15898437)(70.60273437,102.32304687)(70.60273437,103.93632812)
\closepath
\moveto(71.68671875,103.93632812)
\curveto(71.68671875,102.52617187)(71.85078125,101.58671875)(72.17890625,101.11796875)
\curveto(72.5109375,100.653125)(72.91914062,100.42070312)(73.40351562,100.42070312)
\curveto(73.88789062,100.42070312)(74.29414062,100.65507812)(74.62226562,101.12382812)
\curveto(74.95429687,101.59257812)(75.1203125,102.53007812)(75.1203125,103.93632812)
\curveto(75.1203125,105.35039062)(74.95429687,106.28984375)(74.62226562,106.7546875)
\curveto(74.29414062,107.21953125)(73.88398437,107.45195312)(73.39179687,107.45195312)
\curveto(72.90742187,107.45195312)(72.52070312,107.246875)(72.23164062,106.83671875)
\curveto(71.86835937,106.31328125)(71.68671875,105.34648437)(71.68671875,103.93632812)
\closepath
}
}
{
\newrgbcolor{curcolor}{0 0 0}
\pscustom[linestyle=none,fillstyle=solid,fillcolor=curcolor]
{
\newpath
\moveto(82.74921875,106.18632812)
\lineto(81.70039062,106.10429687)
\curveto(81.60664062,106.51835937)(81.47382812,106.81914062)(81.30195312,107.00664062)
\curveto(81.01679687,107.30742187)(80.66523437,107.4578125)(80.24726562,107.4578125)
\curveto(79.91132812,107.4578125)(79.61640625,107.3640625)(79.3625,107.1765625)
\curveto(79.03046875,106.934375)(78.76875,106.58085937)(78.57734375,106.11601562)
\curveto(78.3859375,105.65117187)(78.28632812,104.9890625)(78.27851562,104.1296875)
\curveto(78.53242187,104.51640625)(78.84296875,104.80351562)(79.21015625,104.99101562)
\curveto(79.57734375,105.17851562)(79.96210937,105.27226562)(80.36445312,105.27226562)
\curveto(81.06757812,105.27226562)(81.66523437,105.0125)(82.15742187,104.49296875)
\curveto(82.65351562,103.97734375)(82.9015625,103.309375)(82.9015625,102.4890625)
\curveto(82.9015625,101.95)(82.784375,101.44804687)(82.55,100.98320312)
\curveto(82.31953125,100.52226562)(82.00117187,100.16875)(81.59492187,99.92265625)
\curveto(81.18867187,99.6765625)(80.72773437,99.55351562)(80.21210937,99.55351562)
\curveto(79.33320312,99.55351562)(78.61640625,99.87578125)(78.06171875,100.5203125)
\curveto(77.50703125,101.16875)(77.2296875,102.23515625)(77.2296875,103.71953125)
\curveto(77.2296875,105.3796875)(77.53632812,106.58671875)(78.14960937,107.340625)
\curveto(78.68476562,107.996875)(79.40546875,108.325)(80.31171875,108.325)
\curveto(80.9875,108.325)(81.54023437,108.13554687)(81.96992187,107.75664062)
\curveto(82.40351562,107.37773437)(82.66328125,106.85429687)(82.74921875,106.18632812)
\closepath
\moveto(78.44257812,102.48320312)
\curveto(78.44257812,102.11992187)(78.51875,101.77226562)(78.67109375,101.44023437)
\curveto(78.82734375,101.10820312)(79.04414062,100.85429687)(79.32148437,100.67851562)
\curveto(79.59882812,100.50664062)(79.88984375,100.42070312)(80.19453125,100.42070312)
\curveto(80.63984375,100.42070312)(81.02265625,100.60039062)(81.34296875,100.95976562)
\curveto(81.66328125,101.31914062)(81.8234375,101.80742187)(81.8234375,102.42460937)
\curveto(81.8234375,103.01835937)(81.66523437,103.48515625)(81.34882812,103.825)
\curveto(81.03242187,104.16875)(80.63398437,104.340625)(80.15351562,104.340625)
\curveto(79.67695312,104.340625)(79.27265625,104.16875)(78.940625,103.825)
\curveto(78.60859375,103.48515625)(78.44257812,103.03789062)(78.44257812,102.48320312)
\closepath
}
}
{
\newrgbcolor{curcolor}{0 0 0}
\pscustom[linestyle=none,fillstyle=solid,fillcolor=curcolor]
{
\newpath
\moveto(89.49335937,100.71367187)
\lineto(89.49335937,99.7)
\lineto(83.815625,99.7)
\curveto(83.8078125,99.95390625)(83.84882812,100.19804687)(83.93867187,100.43242187)
\curveto(84.08320312,100.81914062)(84.31367187,101.2)(84.63007812,101.575)
\curveto(84.95039062,101.95)(85.41132812,102.38359375)(86.01289062,102.87578125)
\curveto(86.94648437,103.64140625)(87.57734375,104.246875)(87.90546875,104.6921875)
\curveto(88.23359375,105.14140625)(88.39765625,105.56523437)(88.39765625,105.96367187)
\curveto(88.39765625,106.38164062)(88.24726562,106.73320312)(87.94648437,107.01835937)
\curveto(87.64960937,107.30742187)(87.2609375,107.45195312)(86.78046875,107.45195312)
\curveto(86.27265625,107.45195312)(85.86640625,107.29960937)(85.56171875,106.99492187)
\curveto(85.25703125,106.69023437)(85.10273437,106.26835937)(85.09882812,105.72929687)
\lineto(84.01484375,105.840625)
\curveto(84.0890625,106.64921875)(84.36835937,107.26445312)(84.85273437,107.68632812)
\curveto(85.33710937,108.11210937)(85.9875,108.325)(86.80390625,108.325)
\curveto(87.628125,108.325)(88.28046875,108.09648437)(88.7609375,107.63945312)
\curveto(89.24140625,107.18242187)(89.48164062,106.61601562)(89.48164062,105.94023437)
\curveto(89.48164062,105.59648437)(89.41132812,105.25859375)(89.27070312,104.9265625)
\curveto(89.13007812,104.59453125)(88.89570312,104.24492187)(88.56757812,103.87773437)
\curveto(88.24335937,103.51054687)(87.70234375,103.00664062)(86.94453125,102.36601562)
\curveto(86.31171875,101.83476562)(85.90546875,101.4734375)(85.72578125,101.28203125)
\curveto(85.54609375,101.09453125)(85.39765625,100.90507812)(85.28046875,100.71367187)
\closepath
}
}
{
\newrgbcolor{curcolor}{0 0 0}
\pscustom[linestyle=none,fillstyle=solid,fillcolor=curcolor]
{
\newpath
\moveto(90.62421875,101.95)
\lineto(91.73164062,102.04375)
\curveto(91.81367187,101.5046875)(92.003125,101.0984375)(92.3,100.825)
\curveto(92.60078125,100.55546875)(92.96210937,100.42070312)(93.38398437,100.42070312)
\curveto(93.89179687,100.42070312)(94.32148437,100.61210937)(94.67304687,100.99492187)
\curveto(95.02460937,101.37773437)(95.20039062,101.88554687)(95.20039062,102.51835937)
\curveto(95.20039062,103.11992187)(95.03046875,103.59453125)(94.690625,103.9421875)
\curveto(94.3546875,104.28984375)(93.91328125,104.46367187)(93.36640625,104.46367187)
\curveto(93.0265625,104.46367187)(92.71992187,104.38554687)(92.44648437,104.22929687)
\curveto(92.17304687,104.07695312)(91.95820312,103.87773437)(91.80195312,103.63164062)
\lineto(90.81171875,103.76054687)
\lineto(91.64375,108.17265625)
\lineto(95.91523437,108.17265625)
\lineto(95.91523437,107.16484375)
\lineto(92.4875,107.16484375)
\lineto(92.02460937,104.85625)
\curveto(92.54023437,105.215625)(93.08125,105.3953125)(93.64765625,105.3953125)
\curveto(94.39765625,105.3953125)(95.03046875,105.13554687)(95.54609375,104.61601562)
\curveto(96.06171875,104.09648437)(96.31953125,103.42851562)(96.31953125,102.61210937)
\curveto(96.31953125,101.83476562)(96.09296875,101.16289062)(95.63984375,100.59648437)
\curveto(95.0890625,99.90117187)(94.33710937,99.55351562)(93.38398437,99.55351562)
\curveto(92.60273437,99.55351562)(91.9640625,99.77226562)(91.46796875,100.20976562)
\curveto(90.97578125,100.64726562)(90.69453125,101.22734375)(90.62421875,101.95)
\closepath
}
}
{
\newrgbcolor{curcolor}{0 0 0}
\pscustom[linewidth=1,linecolor=curcolor]
{
\newpath
\moveto(105.1,149.7)
\lineto(114.1,149.7)
\moveto(575,149.7)
\lineto(566,149.7)
}
}
{
\newrgbcolor{curcolor}{0 0 0}
\pscustom[linestyle=none,fillstyle=solid,fillcolor=curcolor]
{
\newpath
\moveto(67.26875,150.03632813)
\curveto(67.26875,151.05195313)(67.37226562,151.86835938)(67.57929687,152.48554688)
\curveto(67.79023437,153.10664063)(68.10078125,153.58515625)(68.5109375,153.92109375)
\curveto(68.925,154.25703125)(69.44453125,154.425)(70.06953125,154.425)
\curveto(70.53046875,154.425)(70.93476562,154.33125)(71.28242187,154.14375)
\curveto(71.63007812,153.96015625)(71.9171875,153.69257813)(72.14375,153.34101563)
\curveto(72.3703125,152.99335938)(72.54804687,152.56757813)(72.67695312,152.06367188)
\curveto(72.80585937,151.56367188)(72.8703125,150.88789063)(72.8703125,150.03632813)
\curveto(72.8703125,149.02851563)(72.76679687,148.2140625)(72.55976562,147.59296875)
\curveto(72.35273437,146.97578125)(72.0421875,146.49726563)(71.628125,146.15742188)
\curveto(71.21796875,145.82148438)(70.6984375,145.65351563)(70.06953125,145.65351563)
\curveto(69.24140625,145.65351563)(68.59101562,145.95039063)(68.11835937,146.54414063)
\curveto(67.55195312,147.25898438)(67.26875,148.42304688)(67.26875,150.03632813)
\closepath
\moveto(68.35273437,150.03632813)
\curveto(68.35273437,148.62617188)(68.51679687,147.68671875)(68.84492187,147.21796875)
\curveto(69.17695312,146.753125)(69.58515625,146.52070313)(70.06953125,146.52070313)
\curveto(70.55390625,146.52070313)(70.96015625,146.75507813)(71.28828125,147.22382813)
\curveto(71.6203125,147.69257813)(71.78632812,148.63007813)(71.78632812,150.03632813)
\curveto(71.78632812,151.45039063)(71.6203125,152.38984375)(71.28828125,152.8546875)
\curveto(70.96015625,153.31953125)(70.55,153.55195313)(70.0578125,153.55195313)
\curveto(69.5734375,153.55195313)(69.18671875,153.346875)(68.89765625,152.93671875)
\curveto(68.534375,152.41328125)(68.35273437,151.44648438)(68.35273437,150.03632813)
\closepath
}
}
{
\newrgbcolor{curcolor}{0 0 0}
\pscustom[linestyle=none,fillstyle=solid,fillcolor=curcolor]
{
\newpath
\moveto(74.534375,145.8)
\lineto(74.534375,147.00117188)
\lineto(75.73554687,147.00117188)
\lineto(75.73554687,145.8)
\closepath
}
}
{
\newrgbcolor{curcolor}{0 0 0}
\pscustom[linestyle=none,fillstyle=solid,fillcolor=curcolor]
{
\newpath
\moveto(81.24921875,145.8)
\lineto(80.19453125,145.8)
\lineto(80.19453125,152.52070313)
\curveto(79.940625,152.27851563)(79.60664062,152.03632813)(79.19257812,151.79414063)
\curveto(78.78242187,151.55195313)(78.41328125,151.3703125)(78.08515625,151.24921875)
\lineto(78.08515625,152.26875)
\curveto(78.675,152.54609375)(79.190625,152.88203125)(79.63203125,153.2765625)
\curveto(80.0734375,153.67109375)(80.3859375,154.05390625)(80.56953125,154.425)
\lineto(81.24921875,154.425)
\closepath
}
}
{
\newrgbcolor{curcolor}{0 0 0}
\pscustom[linestyle=none,fillstyle=solid,fillcolor=curcolor]
{
\newpath
\moveto(89.49335937,146.81367188)
\lineto(89.49335937,145.8)
\lineto(83.815625,145.8)
\curveto(83.8078125,146.05390625)(83.84882812,146.29804688)(83.93867187,146.53242188)
\curveto(84.08320312,146.91914063)(84.31367187,147.3)(84.63007812,147.675)
\curveto(84.95039062,148.05)(85.41132812,148.48359375)(86.01289062,148.97578125)
\curveto(86.94648437,149.74140625)(87.57734375,150.346875)(87.90546875,150.7921875)
\curveto(88.23359375,151.24140625)(88.39765625,151.66523438)(88.39765625,152.06367188)
\curveto(88.39765625,152.48164063)(88.24726562,152.83320313)(87.94648437,153.11835938)
\curveto(87.64960937,153.40742188)(87.2609375,153.55195313)(86.78046875,153.55195313)
\curveto(86.27265625,153.55195313)(85.86640625,153.39960938)(85.56171875,153.09492188)
\curveto(85.25703125,152.79023438)(85.10273437,152.36835938)(85.09882812,151.82929688)
\lineto(84.01484375,151.940625)
\curveto(84.0890625,152.74921875)(84.36835937,153.36445313)(84.85273437,153.78632813)
\curveto(85.33710937,154.21210938)(85.9875,154.425)(86.80390625,154.425)
\curveto(87.628125,154.425)(88.28046875,154.19648438)(88.7609375,153.73945313)
\curveto(89.24140625,153.28242188)(89.48164062,152.71601563)(89.48164062,152.04023438)
\curveto(89.48164062,151.69648438)(89.41132812,151.35859375)(89.27070312,151.0265625)
\curveto(89.13007812,150.69453125)(88.89570312,150.34492188)(88.56757812,149.97773438)
\curveto(88.24335937,149.61054688)(87.70234375,149.10664063)(86.94453125,148.46601563)
\curveto(86.31171875,147.93476563)(85.90546875,147.5734375)(85.72578125,147.38203125)
\curveto(85.54609375,147.19453125)(85.39765625,147.00507813)(85.28046875,146.81367188)
\closepath
}
}
{
\newrgbcolor{curcolor}{0 0 0}
\pscustom[linestyle=none,fillstyle=solid,fillcolor=curcolor]
{
\newpath
\moveto(90.62421875,148.05)
\lineto(91.73164062,148.14375)
\curveto(91.81367187,147.6046875)(92.003125,147.1984375)(92.3,146.925)
\curveto(92.60078125,146.65546875)(92.96210937,146.52070313)(93.38398437,146.52070313)
\curveto(93.89179687,146.52070313)(94.32148437,146.71210938)(94.67304687,147.09492188)
\curveto(95.02460937,147.47773438)(95.20039062,147.98554688)(95.20039062,148.61835938)
\curveto(95.20039062,149.21992188)(95.03046875,149.69453125)(94.690625,150.0421875)
\curveto(94.3546875,150.38984375)(93.91328125,150.56367188)(93.36640625,150.56367188)
\curveto(93.0265625,150.56367188)(92.71992187,150.48554688)(92.44648437,150.32929688)
\curveto(92.17304687,150.17695313)(91.95820312,149.97773438)(91.80195312,149.73164063)
\lineto(90.81171875,149.86054688)
\lineto(91.64375,154.27265625)
\lineto(95.91523437,154.27265625)
\lineto(95.91523437,153.26484375)
\lineto(92.4875,153.26484375)
\lineto(92.02460937,150.95625)
\curveto(92.54023437,151.315625)(93.08125,151.4953125)(93.64765625,151.4953125)
\curveto(94.39765625,151.4953125)(95.03046875,151.23554688)(95.54609375,150.71601563)
\curveto(96.06171875,150.19648438)(96.31953125,149.52851563)(96.31953125,148.71210938)
\curveto(96.31953125,147.93476563)(96.09296875,147.26289063)(95.63984375,146.69648438)
\curveto(95.0890625,146.00117188)(94.33710937,145.65351563)(93.38398437,145.65351563)
\curveto(92.60273437,145.65351563)(91.9640625,145.87226563)(91.46796875,146.30976563)
\curveto(90.97578125,146.74726563)(90.69453125,147.32734375)(90.62421875,148.05)
\closepath
}
}
{
\newrgbcolor{curcolor}{0 0 0}
\pscustom[linewidth=1,linecolor=curcolor]
{
\newpath
\moveto(105.1,195.7)
\lineto(114.1,195.7)
\moveto(575,195.7)
\lineto(566,195.7)
}
}
{
\newrgbcolor{curcolor}{0 0 0}
\pscustom[linestyle=none,fillstyle=solid,fillcolor=curcolor]
{
\newpath
\moveto(73.94257812,196.03632813)
\curveto(73.94257812,197.05195313)(74.04609375,197.86835938)(74.253125,198.48554688)
\curveto(74.4640625,199.10664063)(74.77460937,199.58515625)(75.18476562,199.92109375)
\curveto(75.59882812,200.25703125)(76.11835937,200.425)(76.74335937,200.425)
\curveto(77.20429687,200.425)(77.60859375,200.33125)(77.95625,200.14375)
\curveto(78.30390625,199.96015625)(78.59101562,199.69257813)(78.81757812,199.34101563)
\curveto(79.04414062,198.99335938)(79.221875,198.56757813)(79.35078125,198.06367188)
\curveto(79.4796875,197.56367188)(79.54414062,196.88789063)(79.54414062,196.03632813)
\curveto(79.54414062,195.02851563)(79.440625,194.2140625)(79.23359375,193.59296875)
\curveto(79.0265625,192.97578125)(78.71601562,192.49726563)(78.30195312,192.15742188)
\curveto(77.89179687,191.82148438)(77.37226562,191.65351563)(76.74335937,191.65351563)
\curveto(75.91523437,191.65351563)(75.26484375,191.95039063)(74.7921875,192.54414063)
\curveto(74.22578125,193.25898438)(73.94257812,194.42304688)(73.94257812,196.03632813)
\closepath
\moveto(75.0265625,196.03632813)
\curveto(75.0265625,194.62617188)(75.190625,193.68671875)(75.51875,193.21796875)
\curveto(75.85078125,192.753125)(76.25898437,192.52070313)(76.74335937,192.52070313)
\curveto(77.22773437,192.52070313)(77.63398437,192.75507813)(77.96210937,193.22382813)
\curveto(78.29414062,193.69257813)(78.46015625,194.63007813)(78.46015625,196.03632813)
\curveto(78.46015625,197.45039063)(78.29414062,198.38984375)(77.96210937,198.8546875)
\curveto(77.63398437,199.31953125)(77.22382812,199.55195313)(76.73164062,199.55195313)
\curveto(76.24726562,199.55195313)(75.86054687,199.346875)(75.57148437,198.93671875)
\curveto(75.20820312,198.41328125)(75.0265625,197.44648438)(75.0265625,196.03632813)
\closepath
}
}
{
\newrgbcolor{curcolor}{0 0 0}
\pscustom[linestyle=none,fillstyle=solid,fillcolor=curcolor]
{
\newpath
\moveto(81.20820312,191.8)
\lineto(81.20820312,193.00117188)
\lineto(82.409375,193.00117188)
\lineto(82.409375,191.8)
\closepath
}
}
{
\newrgbcolor{curcolor}{0 0 0}
\pscustom[linestyle=none,fillstyle=solid,fillcolor=curcolor]
{
\newpath
\moveto(89.49335937,192.81367188)
\lineto(89.49335937,191.8)
\lineto(83.815625,191.8)
\curveto(83.8078125,192.05390625)(83.84882812,192.29804688)(83.93867187,192.53242188)
\curveto(84.08320312,192.91914063)(84.31367187,193.3)(84.63007812,193.675)
\curveto(84.95039062,194.05)(85.41132812,194.48359375)(86.01289062,194.97578125)
\curveto(86.94648437,195.74140625)(87.57734375,196.346875)(87.90546875,196.7921875)
\curveto(88.23359375,197.24140625)(88.39765625,197.66523438)(88.39765625,198.06367188)
\curveto(88.39765625,198.48164063)(88.24726562,198.83320313)(87.94648437,199.11835938)
\curveto(87.64960937,199.40742188)(87.2609375,199.55195313)(86.78046875,199.55195313)
\curveto(86.27265625,199.55195313)(85.86640625,199.39960938)(85.56171875,199.09492188)
\curveto(85.25703125,198.79023438)(85.10273437,198.36835938)(85.09882812,197.82929688)
\lineto(84.01484375,197.940625)
\curveto(84.0890625,198.74921875)(84.36835937,199.36445313)(84.85273437,199.78632813)
\curveto(85.33710937,200.21210938)(85.9875,200.425)(86.80390625,200.425)
\curveto(87.628125,200.425)(88.28046875,200.19648438)(88.7609375,199.73945313)
\curveto(89.24140625,199.28242188)(89.48164062,198.71601563)(89.48164062,198.04023438)
\curveto(89.48164062,197.69648438)(89.41132812,197.35859375)(89.27070312,197.0265625)
\curveto(89.13007812,196.69453125)(88.89570312,196.34492188)(88.56757812,195.97773438)
\curveto(88.24335937,195.61054688)(87.70234375,195.10664063)(86.94453125,194.46601563)
\curveto(86.31171875,193.93476563)(85.90546875,193.5734375)(85.72578125,193.38203125)
\curveto(85.54609375,193.19453125)(85.39765625,193.00507813)(85.28046875,192.81367188)
\closepath
}
}
{
\newrgbcolor{curcolor}{0 0 0}
\pscustom[linestyle=none,fillstyle=solid,fillcolor=curcolor]
{
\newpath
\moveto(90.62421875,194.05)
\lineto(91.73164062,194.14375)
\curveto(91.81367187,193.6046875)(92.003125,193.1984375)(92.3,192.925)
\curveto(92.60078125,192.65546875)(92.96210937,192.52070313)(93.38398437,192.52070313)
\curveto(93.89179687,192.52070313)(94.32148437,192.71210938)(94.67304687,193.09492188)
\curveto(95.02460937,193.47773438)(95.20039062,193.98554688)(95.20039062,194.61835938)
\curveto(95.20039062,195.21992188)(95.03046875,195.69453125)(94.690625,196.0421875)
\curveto(94.3546875,196.38984375)(93.91328125,196.56367188)(93.36640625,196.56367188)
\curveto(93.0265625,196.56367188)(92.71992187,196.48554688)(92.44648437,196.32929688)
\curveto(92.17304687,196.17695313)(91.95820312,195.97773438)(91.80195312,195.73164063)
\lineto(90.81171875,195.86054688)
\lineto(91.64375,200.27265625)
\lineto(95.91523437,200.27265625)
\lineto(95.91523437,199.26484375)
\lineto(92.4875,199.26484375)
\lineto(92.02460937,196.95625)
\curveto(92.54023437,197.315625)(93.08125,197.4953125)(93.64765625,197.4953125)
\curveto(94.39765625,197.4953125)(95.03046875,197.23554688)(95.54609375,196.71601563)
\curveto(96.06171875,196.19648438)(96.31953125,195.52851563)(96.31953125,194.71210938)
\curveto(96.31953125,193.93476563)(96.09296875,193.26289063)(95.63984375,192.69648438)
\curveto(95.0890625,192.00117188)(94.33710937,191.65351563)(93.38398437,191.65351563)
\curveto(92.60273437,191.65351563)(91.9640625,191.87226563)(91.46796875,192.30976563)
\curveto(90.97578125,192.74726563)(90.69453125,193.32734375)(90.62421875,194.05)
\closepath
}
}
{
\newrgbcolor{curcolor}{0 0 0}
\pscustom[linewidth=1,linecolor=curcolor]
{
\newpath
\moveto(105.1,241.8)
\lineto(114.1,241.8)
\moveto(575,241.8)
\lineto(566,241.8)
}
}
{
\newrgbcolor{curcolor}{0 0 0}
\pscustom[linestyle=none,fillstyle=solid,fillcolor=curcolor]
{
\newpath
\moveto(80.61640625,242.13632813)
\curveto(80.61640625,243.15195313)(80.71992187,243.96835938)(80.92695312,244.58554688)
\curveto(81.13789062,245.20664063)(81.4484375,245.68515625)(81.85859375,246.02109375)
\curveto(82.27265625,246.35703125)(82.7921875,246.525)(83.4171875,246.525)
\curveto(83.878125,246.525)(84.28242187,246.43125)(84.63007812,246.24375)
\curveto(84.97773437,246.06015625)(85.26484375,245.79257813)(85.49140625,245.44101563)
\curveto(85.71796875,245.09335938)(85.89570312,244.66757813)(86.02460937,244.16367188)
\curveto(86.15351562,243.66367188)(86.21796875,242.98789063)(86.21796875,242.13632813)
\curveto(86.21796875,241.12851563)(86.11445312,240.3140625)(85.90742187,239.69296875)
\curveto(85.70039062,239.07578125)(85.38984375,238.59726563)(84.97578125,238.25742188)
\curveto(84.565625,237.92148438)(84.04609375,237.75351563)(83.4171875,237.75351563)
\curveto(82.5890625,237.75351563)(81.93867187,238.05039063)(81.46601562,238.64414063)
\curveto(80.89960937,239.35898438)(80.61640625,240.52304688)(80.61640625,242.13632813)
\closepath
\moveto(81.70039062,242.13632813)
\curveto(81.70039062,240.72617188)(81.86445312,239.78671875)(82.19257812,239.31796875)
\curveto(82.52460937,238.853125)(82.9328125,238.62070313)(83.4171875,238.62070313)
\curveto(83.9015625,238.62070313)(84.3078125,238.85507813)(84.6359375,239.32382813)
\curveto(84.96796875,239.79257813)(85.13398437,240.73007813)(85.13398437,242.13632813)
\curveto(85.13398437,243.55039063)(84.96796875,244.48984375)(84.6359375,244.9546875)
\curveto(84.3078125,245.41953125)(83.89765625,245.65195313)(83.40546875,245.65195313)
\curveto(82.92109375,245.65195313)(82.534375,245.446875)(82.2453125,245.03671875)
\curveto(81.88203125,244.51328125)(81.70039062,243.54648438)(81.70039062,242.13632813)
\closepath
}
}
{
\newrgbcolor{curcolor}{0 0 0}
\pscustom[linestyle=none,fillstyle=solid,fillcolor=curcolor]
{
\newpath
\moveto(87.88203125,237.9)
\lineto(87.88203125,239.10117188)
\lineto(89.08320312,239.10117188)
\lineto(89.08320312,237.9)
\closepath
}
}
{
\newrgbcolor{curcolor}{0 0 0}
\pscustom[linestyle=none,fillstyle=solid,fillcolor=curcolor]
{
\newpath
\moveto(90.62421875,240.15)
\lineto(91.73164062,240.24375)
\curveto(91.81367187,239.7046875)(92.003125,239.2984375)(92.3,239.025)
\curveto(92.60078125,238.75546875)(92.96210937,238.62070313)(93.38398437,238.62070313)
\curveto(93.89179687,238.62070313)(94.32148437,238.81210938)(94.67304687,239.19492188)
\curveto(95.02460937,239.57773438)(95.20039062,240.08554688)(95.20039062,240.71835938)
\curveto(95.20039062,241.31992188)(95.03046875,241.79453125)(94.690625,242.1421875)
\curveto(94.3546875,242.48984375)(93.91328125,242.66367188)(93.36640625,242.66367188)
\curveto(93.0265625,242.66367188)(92.71992187,242.58554688)(92.44648437,242.42929688)
\curveto(92.17304687,242.27695313)(91.95820312,242.07773438)(91.80195312,241.83164063)
\lineto(90.81171875,241.96054688)
\lineto(91.64375,246.37265625)
\lineto(95.91523437,246.37265625)
\lineto(95.91523437,245.36484375)
\lineto(92.4875,245.36484375)
\lineto(92.02460937,243.05625)
\curveto(92.54023437,243.415625)(93.08125,243.5953125)(93.64765625,243.5953125)
\curveto(94.39765625,243.5953125)(95.03046875,243.33554688)(95.54609375,242.81601563)
\curveto(96.06171875,242.29648438)(96.31953125,241.62851563)(96.31953125,240.81210938)
\curveto(96.31953125,240.03476563)(96.09296875,239.36289063)(95.63984375,238.79648438)
\curveto(95.0890625,238.10117188)(94.33710937,237.75351563)(93.38398437,237.75351563)
\curveto(92.60273437,237.75351563)(91.9640625,237.97226563)(91.46796875,238.40976563)
\curveto(90.97578125,238.84726563)(90.69453125,239.42734375)(90.62421875,240.15)
\closepath
}
}
{
\newrgbcolor{curcolor}{0 0 0}
\pscustom[linewidth=1,linecolor=curcolor]
{
\newpath
\moveto(105.1,287.8)
\lineto(114.1,287.8)
\moveto(575,287.8)
\lineto(566,287.8)
}
}
{
\newrgbcolor{curcolor}{0 0 0}
\pscustom[linestyle=none,fillstyle=solid,fillcolor=curcolor]
{
\newpath
\moveto(94.596875,283.9)
\lineto(93.5421875,283.9)
\lineto(93.5421875,290.62070312)
\curveto(93.28828125,290.37851562)(92.95429687,290.13632812)(92.54023437,289.89414062)
\curveto(92.13007812,289.65195312)(91.7609375,289.4703125)(91.4328125,289.34921875)
\lineto(91.4328125,290.36875)
\curveto(92.02265625,290.64609375)(92.53828125,290.98203125)(92.9796875,291.3765625)
\curveto(93.42109375,291.77109375)(93.73359375,292.15390625)(93.9171875,292.525)
\lineto(94.596875,292.525)
\closepath
}
}
{
\newrgbcolor{curcolor}{0 0 0}
\pscustom[linewidth=1,linecolor=curcolor]
{
\newpath
\moveto(105.1,333.8)
\lineto(114.1,333.8)
\moveto(575,333.8)
\lineto(566,333.8)
}
}
{
\newrgbcolor{curcolor}{0 0 0}
\pscustom[linestyle=none,fillstyle=solid,fillcolor=curcolor]
{
\newpath
\moveto(96.1671875,330.91367187)
\lineto(96.1671875,329.9)
\lineto(90.48945312,329.9)
\curveto(90.48164062,330.15390625)(90.52265625,330.39804687)(90.6125,330.63242187)
\curveto(90.75703125,331.01914062)(90.9875,331.4)(91.30390625,331.775)
\curveto(91.62421875,332.15)(92.08515625,332.58359375)(92.68671875,333.07578125)
\curveto(93.6203125,333.84140625)(94.25117187,334.446875)(94.57929687,334.8921875)
\curveto(94.90742187,335.34140625)(95.07148437,335.76523437)(95.07148437,336.16367187)
\curveto(95.07148437,336.58164062)(94.92109375,336.93320312)(94.6203125,337.21835937)
\curveto(94.3234375,337.50742187)(93.93476562,337.65195312)(93.45429687,337.65195312)
\curveto(92.94648437,337.65195312)(92.54023437,337.49960937)(92.23554687,337.19492187)
\curveto(91.93085937,336.89023437)(91.7765625,336.46835937)(91.77265625,335.92929687)
\lineto(90.68867187,336.040625)
\curveto(90.76289062,336.84921875)(91.0421875,337.46445312)(91.5265625,337.88632812)
\curveto(92.0109375,338.31210937)(92.66132812,338.525)(93.47773437,338.525)
\curveto(94.30195312,338.525)(94.95429687,338.29648437)(95.43476562,337.83945312)
\curveto(95.91523437,337.38242187)(96.15546875,336.81601562)(96.15546875,336.14023437)
\curveto(96.15546875,335.79648437)(96.08515625,335.45859375)(95.94453125,335.1265625)
\curveto(95.80390625,334.79453125)(95.56953125,334.44492187)(95.24140625,334.07773437)
\curveto(94.9171875,333.71054687)(94.37617187,333.20664062)(93.61835937,332.56601562)
\curveto(92.98554687,332.03476562)(92.57929687,331.6734375)(92.39960937,331.48203125)
\curveto(92.21992187,331.29453125)(92.07148437,331.10507812)(91.95429687,330.91367187)
\closepath
}
}
{
\newrgbcolor{curcolor}{0 0 0}
\pscustom[linewidth=1,linecolor=curcolor]
{
\newpath
\moveto(105.1,379.9)
\lineto(114.1,379.9)
\moveto(575,379.9)
\lineto(566,379.9)
}
}
{
\newrgbcolor{curcolor}{0 0 0}
\pscustom[linestyle=none,fillstyle=solid,fillcolor=curcolor]
{
\newpath
\moveto(94.00507812,376)
\lineto(94.00507812,378.05664062)
\lineto(90.27851562,378.05664062)
\lineto(90.27851562,379.0234375)
\lineto(94.1984375,384.58984375)
\lineto(95.05976562,384.58984375)
\lineto(95.05976562,379.0234375)
\lineto(96.21992187,379.0234375)
\lineto(96.21992187,378.05664062)
\lineto(95.05976562,378.05664062)
\lineto(95.05976562,376)
\closepath
\moveto(94.00507812,379.0234375)
\lineto(94.00507812,382.89648438)
\lineto(91.315625,379.0234375)
\closepath
}
}
{
\newrgbcolor{curcolor}{0 0 0}
\pscustom[linewidth=1,linecolor=curcolor]
{
\newpath
\moveto(105.1,425.9)
\lineto(114.1,425.9)
\moveto(575,425.9)
\lineto(566,425.9)
}
}
{
\newrgbcolor{curcolor}{0 0 0}
\pscustom[linestyle=none,fillstyle=solid,fillcolor=curcolor]
{
\newpath
\moveto(92.24726562,426.65820312)
\curveto(91.80976562,426.81835938)(91.48554687,427.046875)(91.27460937,427.34375)
\curveto(91.06367187,427.640625)(90.95820312,427.99609375)(90.95820312,428.41015625)
\curveto(90.95820312,429.03515625)(91.1828125,429.56054688)(91.63203125,429.98632812)
\curveto(92.08125,430.41210938)(92.67890625,430.625)(93.425,430.625)
\curveto(94.175,430.625)(94.77851562,430.40625)(95.23554687,429.96875)
\curveto(95.69257812,429.53515625)(95.92109375,429.00585938)(95.92109375,428.38085938)
\curveto(95.92109375,427.98242188)(95.815625,427.63476562)(95.6046875,427.33789062)
\curveto(95.39765625,427.04492188)(95.08125,426.81835938)(94.65546875,426.65820312)
\curveto(95.1828125,426.48632812)(95.58320312,426.20898438)(95.85664062,425.82617188)
\curveto(96.13398437,425.44335938)(96.27265625,424.98632812)(96.27265625,424.45507812)
\curveto(96.27265625,423.72070312)(96.01289062,423.10351562)(95.49335937,422.60351562)
\curveto(94.97382812,422.10351562)(94.29023437,421.85351562)(93.44257812,421.85351562)
\curveto(92.59492187,421.85351562)(91.91132812,422.10351562)(91.39179687,422.60351562)
\curveto(90.87226562,423.10742188)(90.6125,423.734375)(90.6125,424.484375)
\curveto(90.6125,425.04296875)(90.753125,425.50976562)(91.034375,425.88476562)
\curveto(91.31953125,426.26367188)(91.72382812,426.52148438)(92.24726562,426.65820312)
\closepath
\moveto(92.03632812,428.4453125)
\curveto(92.03632812,428.0390625)(92.1671875,427.70703125)(92.42890625,427.44921875)
\curveto(92.690625,427.19140625)(93.03046875,427.0625)(93.4484375,427.0625)
\curveto(93.8546875,427.0625)(94.18671875,427.18945312)(94.44453125,427.44335938)
\curveto(94.70625,427.70117188)(94.83710937,428.015625)(94.83710937,428.38671875)
\curveto(94.83710937,428.7734375)(94.70234375,429.09765625)(94.4328125,429.359375)
\curveto(94.1671875,429.625)(93.83515625,429.7578125)(93.43671875,429.7578125)
\curveto(93.034375,429.7578125)(92.70039062,429.62890625)(92.43476562,429.37109375)
\curveto(92.16914062,429.11328125)(92.03632812,428.8046875)(92.03632812,428.4453125)
\closepath
\moveto(91.69648437,424.47851562)
\curveto(91.69648437,424.17773438)(91.76679687,423.88671875)(91.90742187,423.60546875)
\curveto(92.05195312,423.32421875)(92.26484375,423.10546875)(92.54609375,422.94921875)
\curveto(92.82734375,422.796875)(93.13007812,422.72070312)(93.45429687,422.72070312)
\curveto(93.95820312,422.72070312)(94.37421875,422.8828125)(94.70234375,423.20703125)
\curveto(95.03046875,423.53125)(95.19453125,423.94335938)(95.19453125,424.44335938)
\curveto(95.19453125,424.95117188)(95.02460937,425.37109375)(94.68476562,425.703125)
\curveto(94.34882812,426.03515625)(93.92695312,426.20117188)(93.41914062,426.20117188)
\curveto(92.92304687,426.20117188)(92.5109375,426.03710938)(92.1828125,425.70898438)
\curveto(91.85859375,425.38085938)(91.69648437,424.97070312)(91.69648437,424.47851562)
\closepath
}
}
{
\newrgbcolor{curcolor}{0 0 0}
\pscustom[linewidth=1,linecolor=curcolor]
{
\newpath
\moveto(105.1,57.6)
\lineto(105.1,66.6)
\moveto(105.1,425.9)
\lineto(105.1,416.9)
}
}
{
\newrgbcolor{curcolor}{0 0 0}
\pscustom[linestyle=none,fillstyle=solid,fillcolor=curcolor]
{
\newpath
\moveto(96.13222656,36.71367187)
\lineto(96.13222656,35.7)
\lineto(90.45449219,35.7)
\curveto(90.44667969,35.95390625)(90.48769531,36.19804687)(90.57753906,36.43242187)
\curveto(90.72207031,36.81914062)(90.95253906,37.2)(91.26894531,37.575)
\curveto(91.58925781,37.95)(92.05019531,38.38359375)(92.65175781,38.87578125)
\curveto(93.58535156,39.64140625)(94.21621094,40.246875)(94.54433594,40.6921875)
\curveto(94.87246094,41.14140625)(95.03652344,41.56523437)(95.03652344,41.96367187)
\curveto(95.03652344,42.38164062)(94.88613281,42.73320312)(94.58535156,43.01835937)
\curveto(94.28847656,43.30742187)(93.89980469,43.45195312)(93.41933594,43.45195312)
\curveto(92.91152344,43.45195312)(92.50527344,43.29960937)(92.20058594,42.99492187)
\curveto(91.89589844,42.69023437)(91.74160156,42.26835937)(91.73769531,41.72929687)
\lineto(90.65371094,41.840625)
\curveto(90.72792969,42.64921875)(91.00722656,43.26445312)(91.49160156,43.68632812)
\curveto(91.97597656,44.11210937)(92.62636719,44.325)(93.44277344,44.325)
\curveto(94.26699219,44.325)(94.91933594,44.09648437)(95.39980469,43.63945312)
\curveto(95.88027344,43.18242187)(96.12050781,42.61601562)(96.12050781,41.94023437)
\curveto(96.12050781,41.59648437)(96.05019531,41.25859375)(95.90957031,40.9265625)
\curveto(95.76894531,40.59453125)(95.53457031,40.24492187)(95.20644531,39.87773437)
\curveto(94.88222656,39.51054687)(94.34121094,39.00664062)(93.58339844,38.36601562)
\curveto(92.95058594,37.83476562)(92.54433594,37.4734375)(92.36464844,37.28203125)
\curveto(92.18496094,37.09453125)(92.03652344,36.90507812)(91.91933594,36.71367187)
\closepath
}
}
{
\newrgbcolor{curcolor}{0 0 0}
\pscustom[linestyle=none,fillstyle=solid,fillcolor=curcolor]
{
\newpath
\moveto(97.26308594,37.95)
\lineto(98.37050781,38.04375)
\curveto(98.45253906,37.5046875)(98.64199219,37.0984375)(98.93886719,36.825)
\curveto(99.23964844,36.55546875)(99.60097656,36.42070312)(100.02285156,36.42070312)
\curveto(100.53066406,36.42070312)(100.96035156,36.61210937)(101.31191406,36.99492187)
\curveto(101.66347656,37.37773437)(101.83925781,37.88554687)(101.83925781,38.51835937)
\curveto(101.83925781,39.11992187)(101.66933594,39.59453125)(101.32949219,39.9421875)
\curveto(100.99355469,40.28984375)(100.55214844,40.46367187)(100.00527344,40.46367187)
\curveto(99.66542969,40.46367187)(99.35878906,40.38554687)(99.08535156,40.22929687)
\curveto(98.81191406,40.07695312)(98.59707031,39.87773437)(98.44082031,39.63164062)
\lineto(97.45058594,39.76054687)
\lineto(98.28261719,44.17265625)
\lineto(102.55410156,44.17265625)
\lineto(102.55410156,43.16484375)
\lineto(99.12636719,43.16484375)
\lineto(98.66347656,40.85625)
\curveto(99.17910156,41.215625)(99.72011719,41.3953125)(100.28652344,41.3953125)
\curveto(101.03652344,41.3953125)(101.66933594,41.13554687)(102.18496094,40.61601562)
\curveto(102.70058594,40.09648437)(102.95839844,39.42851562)(102.95839844,38.61210937)
\curveto(102.95839844,37.83476562)(102.73183594,37.16289062)(102.27871094,36.59648437)
\curveto(101.72792969,35.90117187)(100.97597656,35.55351562)(100.02285156,35.55351562)
\curveto(99.24160156,35.55351562)(98.60292969,35.77226562)(98.10683594,36.20976562)
\curveto(97.61464844,36.64726562)(97.33339844,37.22734375)(97.26308594,37.95)
\closepath
}
}
{
\newrgbcolor{curcolor}{0 0 0}
\pscustom[linestyle=none,fillstyle=solid,fillcolor=curcolor]
{
\newpath
\moveto(103.93691406,39.93632812)
\curveto(103.93691406,40.95195312)(104.04042969,41.76835937)(104.24746094,42.38554687)
\curveto(104.45839844,43.00664062)(104.76894531,43.48515625)(105.17910156,43.82109375)
\curveto(105.59316406,44.15703125)(106.11269531,44.325)(106.73769531,44.325)
\curveto(107.19863281,44.325)(107.60292969,44.23125)(107.95058594,44.04375)
\curveto(108.29824219,43.86015625)(108.58535156,43.59257812)(108.81191406,43.24101562)
\curveto(109.03847656,42.89335937)(109.21621094,42.46757812)(109.34511719,41.96367187)
\curveto(109.47402344,41.46367187)(109.53847656,40.78789062)(109.53847656,39.93632812)
\curveto(109.53847656,38.92851562)(109.43496094,38.1140625)(109.22792969,37.49296875)
\curveto(109.02089844,36.87578125)(108.71035156,36.39726562)(108.29628906,36.05742187)
\curveto(107.88613281,35.72148437)(107.36660156,35.55351562)(106.73769531,35.55351562)
\curveto(105.90957031,35.55351562)(105.25917969,35.85039062)(104.78652344,36.44414062)
\curveto(104.22011719,37.15898437)(103.93691406,38.32304687)(103.93691406,39.93632812)
\closepath
\moveto(105.02089844,39.93632812)
\curveto(105.02089844,38.52617187)(105.18496094,37.58671875)(105.51308594,37.11796875)
\curveto(105.84511719,36.653125)(106.25332031,36.42070312)(106.73769531,36.42070312)
\curveto(107.22207031,36.42070312)(107.62832031,36.65507812)(107.95644531,37.12382812)
\curveto(108.28847656,37.59257812)(108.45449219,38.53007812)(108.45449219,39.93632812)
\curveto(108.45449219,41.35039062)(108.28847656,42.28984375)(107.95644531,42.7546875)
\curveto(107.62832031,43.21953125)(107.21816406,43.45195312)(106.72597656,43.45195312)
\curveto(106.24160156,43.45195312)(105.85488281,43.246875)(105.56582031,42.83671875)
\curveto(105.20253906,42.31328125)(105.02089844,41.34648437)(105.02089844,39.93632812)
\closepath
}
}
{
\newrgbcolor{curcolor}{0 0 0}
\pscustom[linestyle=none,fillstyle=solid,fillcolor=curcolor]
{
\newpath
\moveto(111.00332031,35.7)
\lineto(111.00332031,44.28984375)
\lineto(112.71425781,44.28984375)
\lineto(114.74746094,38.2078125)
\curveto(114.93496094,37.64140625)(115.07167969,37.21757812)(115.15761719,36.93632812)
\curveto(115.25527344,37.24882812)(115.40761719,37.7078125)(115.61464844,38.31328125)
\lineto(117.67128906,44.28984375)
\lineto(119.20058594,44.28984375)
\lineto(119.20058594,35.7)
\lineto(118.10488281,35.7)
\lineto(118.10488281,42.88945312)
\lineto(115.60878906,35.7)
\lineto(114.58339844,35.7)
\lineto(112.09902344,43.0125)
\lineto(112.09902344,35.7)
\closepath
}
}
{
\newrgbcolor{curcolor}{0 0 0}
\pscustom[linewidth=1,linecolor=curcolor]
{
\newpath
\moveto(183.4,57.6)
\lineto(183.4,66.6)
\moveto(183.4,425.9)
\lineto(183.4,416.9)
}
}
{
\newrgbcolor{curcolor}{0 0 0}
\pscustom[linestyle=none,fillstyle=solid,fillcolor=curcolor]
{
\newpath
\moveto(168.88925781,37.95)
\lineto(169.99667969,38.04375)
\curveto(170.07871094,37.5046875)(170.26816406,37.0984375)(170.56503906,36.825)
\curveto(170.86582031,36.55546875)(171.22714844,36.42070312)(171.64902344,36.42070312)
\curveto(172.15683594,36.42070312)(172.58652344,36.61210937)(172.93808594,36.99492187)
\curveto(173.28964844,37.37773437)(173.46542969,37.88554687)(173.46542969,38.51835937)
\curveto(173.46542969,39.11992187)(173.29550781,39.59453125)(172.95566406,39.9421875)
\curveto(172.61972656,40.28984375)(172.17832031,40.46367187)(171.63144531,40.46367187)
\curveto(171.29160156,40.46367187)(170.98496094,40.38554687)(170.71152344,40.22929687)
\curveto(170.43808594,40.07695312)(170.22324219,39.87773437)(170.06699219,39.63164062)
\lineto(169.07675781,39.76054687)
\lineto(169.90878906,44.17265625)
\lineto(174.18027344,44.17265625)
\lineto(174.18027344,43.16484375)
\lineto(170.75253906,43.16484375)
\lineto(170.28964844,40.85625)
\curveto(170.80527344,41.215625)(171.34628906,41.3953125)(171.91269531,41.3953125)
\curveto(172.66269531,41.3953125)(173.29550781,41.13554687)(173.81113281,40.61601562)
\curveto(174.32675781,40.09648437)(174.58457031,39.42851562)(174.58457031,38.61210937)
\curveto(174.58457031,37.83476562)(174.35800781,37.16289062)(173.90488281,36.59648437)
\curveto(173.35410156,35.90117187)(172.60214844,35.55351562)(171.64902344,35.55351562)
\curveto(170.86777344,35.55351562)(170.22910156,35.77226562)(169.73300781,36.20976562)
\curveto(169.24082031,36.64726562)(168.95957031,37.22734375)(168.88925781,37.95)
\closepath
}
}
{
\newrgbcolor{curcolor}{0 0 0}
\pscustom[linestyle=none,fillstyle=solid,fillcolor=curcolor]
{
\newpath
\moveto(175.56308594,39.93632812)
\curveto(175.56308594,40.95195312)(175.66660156,41.76835937)(175.87363281,42.38554687)
\curveto(176.08457031,43.00664062)(176.39511719,43.48515625)(176.80527344,43.82109375)
\curveto(177.21933594,44.15703125)(177.73886719,44.325)(178.36386719,44.325)
\curveto(178.82480469,44.325)(179.22910156,44.23125)(179.57675781,44.04375)
\curveto(179.92441406,43.86015625)(180.21152344,43.59257812)(180.43808594,43.24101562)
\curveto(180.66464844,42.89335937)(180.84238281,42.46757812)(180.97128906,41.96367187)
\curveto(181.10019531,41.46367187)(181.16464844,40.78789062)(181.16464844,39.93632812)
\curveto(181.16464844,38.92851562)(181.06113281,38.1140625)(180.85410156,37.49296875)
\curveto(180.64707031,36.87578125)(180.33652344,36.39726562)(179.92246094,36.05742187)
\curveto(179.51230469,35.72148437)(178.99277344,35.55351562)(178.36386719,35.55351562)
\curveto(177.53574219,35.55351562)(176.88535156,35.85039062)(176.41269531,36.44414062)
\curveto(175.84628906,37.15898437)(175.56308594,38.32304687)(175.56308594,39.93632812)
\closepath
\moveto(176.64707031,39.93632812)
\curveto(176.64707031,38.52617187)(176.81113281,37.58671875)(177.13925781,37.11796875)
\curveto(177.47128906,36.653125)(177.87949219,36.42070312)(178.36386719,36.42070312)
\curveto(178.84824219,36.42070312)(179.25449219,36.65507812)(179.58261719,37.12382812)
\curveto(179.91464844,37.59257812)(180.08066406,38.53007812)(180.08066406,39.93632812)
\curveto(180.08066406,41.35039062)(179.91464844,42.28984375)(179.58261719,42.7546875)
\curveto(179.25449219,43.21953125)(178.84433594,43.45195312)(178.35214844,43.45195312)
\curveto(177.86777344,43.45195312)(177.48105469,43.246875)(177.19199219,42.83671875)
\curveto(176.82871094,42.31328125)(176.64707031,41.34648437)(176.64707031,39.93632812)
\closepath
}
}
{
\newrgbcolor{curcolor}{0 0 0}
\pscustom[linestyle=none,fillstyle=solid,fillcolor=curcolor]
{
\newpath
\moveto(182.23691406,39.93632812)
\curveto(182.23691406,40.95195312)(182.34042969,41.76835937)(182.54746094,42.38554687)
\curveto(182.75839844,43.00664062)(183.06894531,43.48515625)(183.47910156,43.82109375)
\curveto(183.89316406,44.15703125)(184.41269531,44.325)(185.03769531,44.325)
\curveto(185.49863281,44.325)(185.90292969,44.23125)(186.25058594,44.04375)
\curveto(186.59824219,43.86015625)(186.88535156,43.59257812)(187.11191406,43.24101562)
\curveto(187.33847656,42.89335937)(187.51621094,42.46757812)(187.64511719,41.96367187)
\curveto(187.77402344,41.46367187)(187.83847656,40.78789062)(187.83847656,39.93632812)
\curveto(187.83847656,38.92851562)(187.73496094,38.1140625)(187.52792969,37.49296875)
\curveto(187.32089844,36.87578125)(187.01035156,36.39726562)(186.59628906,36.05742187)
\curveto(186.18613281,35.72148437)(185.66660156,35.55351562)(185.03769531,35.55351562)
\curveto(184.20957031,35.55351562)(183.55917969,35.85039062)(183.08652344,36.44414062)
\curveto(182.52011719,37.15898437)(182.23691406,38.32304687)(182.23691406,39.93632812)
\closepath
\moveto(183.32089844,39.93632812)
\curveto(183.32089844,38.52617187)(183.48496094,37.58671875)(183.81308594,37.11796875)
\curveto(184.14511719,36.653125)(184.55332031,36.42070312)(185.03769531,36.42070312)
\curveto(185.52207031,36.42070312)(185.92832031,36.65507812)(186.25644531,37.12382812)
\curveto(186.58847656,37.59257812)(186.75449219,38.53007812)(186.75449219,39.93632812)
\curveto(186.75449219,41.35039062)(186.58847656,42.28984375)(186.25644531,42.7546875)
\curveto(185.92832031,43.21953125)(185.51816406,43.45195312)(185.02597656,43.45195312)
\curveto(184.54160156,43.45195312)(184.15488281,43.246875)(183.86582031,42.83671875)
\curveto(183.50253906,42.31328125)(183.32089844,41.34648437)(183.32089844,39.93632812)
\closepath
}
}
{
\newrgbcolor{curcolor}{0 0 0}
\pscustom[linestyle=none,fillstyle=solid,fillcolor=curcolor]
{
\newpath
\moveto(189.30332031,35.7)
\lineto(189.30332031,44.28984375)
\lineto(191.01425781,44.28984375)
\lineto(193.04746094,38.2078125)
\curveto(193.23496094,37.64140625)(193.37167969,37.21757812)(193.45761719,36.93632812)
\curveto(193.55527344,37.24882812)(193.70761719,37.7078125)(193.91464844,38.31328125)
\lineto(195.97128906,44.28984375)
\lineto(197.50058594,44.28984375)
\lineto(197.50058594,35.7)
\lineto(196.40488281,35.7)
\lineto(196.40488281,42.88945312)
\lineto(193.90878906,35.7)
\lineto(192.88339844,35.7)
\lineto(190.39902344,43.0125)
\lineto(190.39902344,35.7)
\closepath
}
}
{
\newrgbcolor{curcolor}{0 0 0}
\pscustom[linewidth=1,linecolor=curcolor]
{
\newpath
\moveto(261.7,57.6)
\lineto(261.7,66.6)
\moveto(261.7,425.9)
\lineto(261.7,416.9)
}
}
{
\newrgbcolor{curcolor}{0 0 0}
\pscustom[linestyle=none,fillstyle=solid,fillcolor=curcolor]
{
\newpath
\moveto(258.16679687,35.7)
\lineto(257.11210937,35.7)
\lineto(257.11210937,42.42070312)
\curveto(256.85820312,42.17851562)(256.52421875,41.93632812)(256.11015625,41.69414062)
\curveto(255.7,41.45195312)(255.33085937,41.2703125)(255.00273437,41.14921875)
\lineto(255.00273437,42.16875)
\curveto(255.59257812,42.44609375)(256.10820312,42.78203125)(256.54960937,43.1765625)
\curveto(256.99101562,43.57109375)(257.30351562,43.95390625)(257.48710937,44.325)
\lineto(258.16679687,44.325)
\closepath
}
}
{
\newrgbcolor{curcolor}{0 0 0}
\pscustom[linestyle=none,fillstyle=solid,fillcolor=curcolor]
{
\newpath
\moveto(265.31523437,39.06914062)
\lineto(265.31523437,40.07695312)
\lineto(268.95390625,40.0828125)
\lineto(268.95390625,36.8953125)
\curveto(268.3953125,36.45)(267.81914062,36.1140625)(267.22539062,35.8875)
\curveto(266.63164062,35.66484375)(266.02226562,35.55351562)(265.39726562,35.55351562)
\curveto(264.55351562,35.55351562)(263.7859375,35.73320312)(263.09453125,36.09257812)
\curveto(262.40703125,36.45585937)(261.8875,36.97929687)(261.5359375,37.66289062)
\curveto(261.184375,38.34648437)(261.00859375,39.11015625)(261.00859375,39.95390625)
\curveto(261.00859375,40.78984375)(261.18242187,41.56914062)(261.53007812,42.29179687)
\curveto(261.88164062,43.01835937)(262.38554687,43.55742187)(263.04179687,43.90898437)
\curveto(263.69804687,44.26054687)(264.45390625,44.43632812)(265.309375,44.43632812)
\curveto(265.93046875,44.43632812)(266.49101562,44.33476562)(266.99101562,44.13164062)
\curveto(267.49492187,43.93242187)(267.88945312,43.653125)(268.17460937,43.29375)
\curveto(268.45976562,42.934375)(268.6765625,42.465625)(268.825,41.8875)
\lineto(267.79960937,41.60625)
\curveto(267.67070312,42.04375)(267.51054687,42.3875)(267.31914062,42.6375)
\curveto(267.12773437,42.8875)(266.85429687,43.08671875)(266.49882812,43.23515625)
\curveto(266.14335937,43.3875)(265.74882812,43.46367187)(265.31523437,43.46367187)
\curveto(264.79570312,43.46367187)(264.34648437,43.38359375)(263.96757812,43.2234375)
\curveto(263.58867187,43.0671875)(263.28203125,42.86015625)(263.04765625,42.60234375)
\curveto(262.8171875,42.34453125)(262.6375,42.06132812)(262.50859375,41.75273437)
\curveto(262.28984375,41.22148437)(262.18046875,40.6453125)(262.18046875,40.02421875)
\curveto(262.18046875,39.25859375)(262.31132812,38.61796875)(262.57304687,38.10234375)
\curveto(262.83867187,37.58671875)(263.2234375,37.20390625)(263.72734375,36.95390625)
\curveto(264.23125,36.70390625)(264.76640625,36.57890625)(265.3328125,36.57890625)
\curveto(265.825,36.57890625)(266.30546875,36.67265625)(266.77421875,36.86015625)
\curveto(267.24296875,37.0515625)(267.5984375,37.2546875)(267.840625,37.46953125)
\lineto(267.840625,39.06914062)
\closepath
}
}
{
\newrgbcolor{curcolor}{0 0 0}
\pscustom[linewidth=1,linecolor=curcolor]
{
\newpath
\moveto(340.1,57.6)
\lineto(340.1,66.6)
\moveto(340.1,425.9)
\lineto(340.1,416.9)
}
}
{
\newrgbcolor{curcolor}{0 0 0}
\pscustom[linestyle=none,fillstyle=solid,fillcolor=curcolor]
{
\newpath
\moveto(338.13710938,36.71367187)
\lineto(338.13710938,35.7)
\lineto(332.459375,35.7)
\curveto(332.4515625,35.95390625)(332.49257813,36.19804687)(332.58242188,36.43242187)
\curveto(332.72695313,36.81914062)(332.95742188,37.2)(333.27382813,37.575)
\curveto(333.59414063,37.95)(334.05507813,38.38359375)(334.65664063,38.87578125)
\curveto(335.59023438,39.64140625)(336.22109375,40.246875)(336.54921875,40.6921875)
\curveto(336.87734375,41.14140625)(337.04140625,41.56523437)(337.04140625,41.96367187)
\curveto(337.04140625,42.38164062)(336.89101563,42.73320312)(336.59023438,43.01835937)
\curveto(336.29335938,43.30742187)(335.9046875,43.45195312)(335.42421875,43.45195312)
\curveto(334.91640625,43.45195312)(334.51015625,43.29960937)(334.20546875,42.99492187)
\curveto(333.90078125,42.69023437)(333.74648438,42.26835937)(333.74257813,41.72929687)
\lineto(332.65859375,41.840625)
\curveto(332.7328125,42.64921875)(333.01210938,43.26445312)(333.49648438,43.68632812)
\curveto(333.98085938,44.11210937)(334.63125,44.325)(335.44765625,44.325)
\curveto(336.271875,44.325)(336.92421875,44.09648437)(337.4046875,43.63945312)
\curveto(337.88515625,43.18242187)(338.12539063,42.61601562)(338.12539063,41.94023437)
\curveto(338.12539063,41.59648437)(338.05507813,41.25859375)(337.91445313,40.9265625)
\curveto(337.77382813,40.59453125)(337.53945313,40.24492187)(337.21132813,39.87773437)
\curveto(336.88710938,39.51054687)(336.34609375,39.00664062)(335.58828125,38.36601562)
\curveto(334.95546875,37.83476562)(334.54921875,37.4734375)(334.36953125,37.28203125)
\curveto(334.18984375,37.09453125)(334.04140625,36.90507812)(333.92421875,36.71367187)
\closepath
}
}
{
\newrgbcolor{curcolor}{0 0 0}
\pscustom[linestyle=none,fillstyle=solid,fillcolor=curcolor]
{
\newpath
\moveto(343.71523438,39.06914062)
\lineto(343.71523438,40.07695312)
\lineto(347.35390625,40.0828125)
\lineto(347.35390625,36.8953125)
\curveto(346.7953125,36.45)(346.21914063,36.1140625)(345.62539063,35.8875)
\curveto(345.03164063,35.66484375)(344.42226563,35.55351562)(343.79726563,35.55351562)
\curveto(342.95351563,35.55351562)(342.1859375,35.73320312)(341.49453125,36.09257812)
\curveto(340.80703125,36.45585937)(340.2875,36.97929687)(339.9359375,37.66289062)
\curveto(339.584375,38.34648437)(339.40859375,39.11015625)(339.40859375,39.95390625)
\curveto(339.40859375,40.78984375)(339.58242188,41.56914062)(339.93007813,42.29179687)
\curveto(340.28164063,43.01835937)(340.78554688,43.55742187)(341.44179688,43.90898437)
\curveto(342.09804688,44.26054687)(342.85390625,44.43632812)(343.709375,44.43632812)
\curveto(344.33046875,44.43632812)(344.89101563,44.33476562)(345.39101563,44.13164062)
\curveto(345.89492188,43.93242187)(346.28945313,43.653125)(346.57460938,43.29375)
\curveto(346.85976563,42.934375)(347.0765625,42.465625)(347.225,41.8875)
\lineto(346.19960938,41.60625)
\curveto(346.07070313,42.04375)(345.91054688,42.3875)(345.71914063,42.6375)
\curveto(345.52773438,42.8875)(345.25429688,43.08671875)(344.89882813,43.23515625)
\curveto(344.54335938,43.3875)(344.14882813,43.46367187)(343.71523438,43.46367187)
\curveto(343.19570313,43.46367187)(342.74648438,43.38359375)(342.36757813,43.2234375)
\curveto(341.98867188,43.0671875)(341.68203125,42.86015625)(341.44765625,42.60234375)
\curveto(341.2171875,42.34453125)(341.0375,42.06132812)(340.90859375,41.75273437)
\curveto(340.68984375,41.22148437)(340.58046875,40.6453125)(340.58046875,40.02421875)
\curveto(340.58046875,39.25859375)(340.71132813,38.61796875)(340.97304688,38.10234375)
\curveto(341.23867188,37.58671875)(341.6234375,37.20390625)(342.12734375,36.95390625)
\curveto(342.63125,36.70390625)(343.16640625,36.57890625)(343.7328125,36.57890625)
\curveto(344.225,36.57890625)(344.70546875,36.67265625)(345.17421875,36.86015625)
\curveto(345.64296875,37.0515625)(345.9984375,37.2546875)(346.240625,37.46953125)
\lineto(346.240625,39.06914062)
\closepath
}
}
{
\newrgbcolor{curcolor}{0 0 0}
\pscustom[linewidth=1,linecolor=curcolor]
{
\newpath
\moveto(418.4,57.6)
\lineto(418.4,66.6)
\moveto(418.4,425.9)
\lineto(418.4,416.9)
}
}
{
\newrgbcolor{curcolor}{0 0 0}
\pscustom[linestyle=none,fillstyle=solid,fillcolor=curcolor]
{
\newpath
\moveto(414.275,35.7)
\lineto(414.275,37.75664062)
\lineto(410.5484375,37.75664062)
\lineto(410.5484375,38.7234375)
\lineto(414.46835937,44.28984375)
\lineto(415.3296875,44.28984375)
\lineto(415.3296875,38.7234375)
\lineto(416.48984375,38.7234375)
\lineto(416.48984375,37.75664062)
\lineto(415.3296875,37.75664062)
\lineto(415.3296875,35.7)
\closepath
\moveto(414.275,38.7234375)
\lineto(414.275,42.59648437)
\lineto(411.58554687,38.7234375)
\closepath
}
}
{
\newrgbcolor{curcolor}{0 0 0}
\pscustom[linestyle=none,fillstyle=solid,fillcolor=curcolor]
{
\newpath
\moveto(422.01523437,39.06914062)
\lineto(422.01523437,40.07695312)
\lineto(425.65390625,40.0828125)
\lineto(425.65390625,36.8953125)
\curveto(425.0953125,36.45)(424.51914062,36.1140625)(423.92539062,35.8875)
\curveto(423.33164062,35.66484375)(422.72226562,35.55351562)(422.09726562,35.55351562)
\curveto(421.25351562,35.55351562)(420.4859375,35.73320312)(419.79453125,36.09257812)
\curveto(419.10703125,36.45585937)(418.5875,36.97929687)(418.2359375,37.66289062)
\curveto(417.884375,38.34648437)(417.70859375,39.11015625)(417.70859375,39.95390625)
\curveto(417.70859375,40.78984375)(417.88242187,41.56914062)(418.23007812,42.29179687)
\curveto(418.58164062,43.01835937)(419.08554687,43.55742187)(419.74179687,43.90898437)
\curveto(420.39804687,44.26054687)(421.15390625,44.43632812)(422.009375,44.43632812)
\curveto(422.63046875,44.43632812)(423.19101562,44.33476562)(423.69101562,44.13164062)
\curveto(424.19492187,43.93242187)(424.58945312,43.653125)(424.87460937,43.29375)
\curveto(425.15976562,42.934375)(425.3765625,42.465625)(425.525,41.8875)
\lineto(424.49960937,41.60625)
\curveto(424.37070312,42.04375)(424.21054687,42.3875)(424.01914062,42.6375)
\curveto(423.82773437,42.8875)(423.55429687,43.08671875)(423.19882812,43.23515625)
\curveto(422.84335937,43.3875)(422.44882812,43.46367187)(422.01523437,43.46367187)
\curveto(421.49570312,43.46367187)(421.04648437,43.38359375)(420.66757812,43.2234375)
\curveto(420.28867187,43.0671875)(419.98203125,42.86015625)(419.74765625,42.60234375)
\curveto(419.5171875,42.34453125)(419.3375,42.06132812)(419.20859375,41.75273437)
\curveto(418.98984375,41.22148437)(418.88046875,40.6453125)(418.88046875,40.02421875)
\curveto(418.88046875,39.25859375)(419.01132812,38.61796875)(419.27304687,38.10234375)
\curveto(419.53867187,37.58671875)(419.9234375,37.20390625)(420.42734375,36.95390625)
\curveto(420.93125,36.70390625)(421.46640625,36.57890625)(422.0328125,36.57890625)
\curveto(422.525,36.57890625)(423.00546875,36.67265625)(423.47421875,36.86015625)
\curveto(423.94296875,37.0515625)(424.2984375,37.2546875)(424.540625,37.46953125)
\lineto(424.540625,39.06914062)
\closepath
}
}
{
\newrgbcolor{curcolor}{0 0 0}
\pscustom[linewidth=1,linecolor=curcolor]
{
\newpath
\moveto(496.7,57.6)
\lineto(496.7,66.6)
\moveto(496.7,425.9)
\lineto(496.7,416.9)
}
}
{
\newrgbcolor{curcolor}{0 0 0}
\pscustom[linestyle=none,fillstyle=solid,fillcolor=curcolor]
{
\newpath
\moveto(490.8171875,40.35820312)
\curveto(490.3796875,40.51835937)(490.05546875,40.746875)(489.84453125,41.04375)
\curveto(489.63359375,41.340625)(489.528125,41.69609375)(489.528125,42.11015625)
\curveto(489.528125,42.73515625)(489.75273437,43.26054687)(490.20195312,43.68632812)
\curveto(490.65117187,44.11210937)(491.24882812,44.325)(491.99492187,44.325)
\curveto(492.74492187,44.325)(493.3484375,44.10625)(493.80546875,43.66875)
\curveto(494.2625,43.23515625)(494.49101562,42.70585937)(494.49101562,42.08085937)
\curveto(494.49101562,41.68242187)(494.38554687,41.33476562)(494.17460937,41.03789062)
\curveto(493.96757812,40.74492187)(493.65117187,40.51835937)(493.22539062,40.35820312)
\curveto(493.75273437,40.18632812)(494.153125,39.90898437)(494.4265625,39.52617187)
\curveto(494.70390625,39.14335937)(494.84257812,38.68632812)(494.84257812,38.15507812)
\curveto(494.84257812,37.42070312)(494.5828125,36.80351562)(494.06328125,36.30351562)
\curveto(493.54375,35.80351562)(492.86015625,35.55351562)(492.0125,35.55351562)
\curveto(491.16484375,35.55351562)(490.48125,35.80351562)(489.96171875,36.30351562)
\curveto(489.4421875,36.80742187)(489.18242187,37.434375)(489.18242187,38.184375)
\curveto(489.18242187,38.74296875)(489.32304687,39.20976562)(489.60429687,39.58476562)
\curveto(489.88945312,39.96367187)(490.29375,40.22148437)(490.8171875,40.35820312)
\closepath
\moveto(490.60625,42.1453125)
\curveto(490.60625,41.7390625)(490.73710937,41.40703125)(490.99882812,41.14921875)
\curveto(491.26054687,40.89140625)(491.60039062,40.7625)(492.01835937,40.7625)
\curveto(492.42460937,40.7625)(492.75664062,40.88945312)(493.01445312,41.14335937)
\curveto(493.27617187,41.40117187)(493.40703125,41.715625)(493.40703125,42.08671875)
\curveto(493.40703125,42.4734375)(493.27226562,42.79765625)(493.00273437,43.059375)
\curveto(492.73710937,43.325)(492.40507812,43.4578125)(492.00664062,43.4578125)
\curveto(491.60429687,43.4578125)(491.2703125,43.32890625)(491.0046875,43.07109375)
\curveto(490.7390625,42.81328125)(490.60625,42.5046875)(490.60625,42.1453125)
\closepath
\moveto(490.26640625,38.17851562)
\curveto(490.26640625,37.87773437)(490.33671875,37.58671875)(490.47734375,37.30546875)
\curveto(490.621875,37.02421875)(490.83476562,36.80546875)(491.11601562,36.64921875)
\curveto(491.39726562,36.496875)(491.7,36.42070312)(492.02421875,36.42070312)
\curveto(492.528125,36.42070312)(492.94414062,36.5828125)(493.27226562,36.90703125)
\curveto(493.60039062,37.23125)(493.76445312,37.64335937)(493.76445312,38.14335937)
\curveto(493.76445312,38.65117187)(493.59453125,39.07109375)(493.2546875,39.403125)
\curveto(492.91875,39.73515625)(492.496875,39.90117187)(491.9890625,39.90117187)
\curveto(491.49296875,39.90117187)(491.08085937,39.73710937)(490.75273437,39.40898437)
\curveto(490.42851562,39.08085937)(490.26640625,38.67070312)(490.26640625,38.17851562)
\closepath
}
}
{
\newrgbcolor{curcolor}{0 0 0}
\pscustom[linestyle=none,fillstyle=solid,fillcolor=curcolor]
{
\newpath
\moveto(500.31523437,39.06914062)
\lineto(500.31523437,40.07695312)
\lineto(503.95390625,40.0828125)
\lineto(503.95390625,36.8953125)
\curveto(503.3953125,36.45)(502.81914062,36.1140625)(502.22539062,35.8875)
\curveto(501.63164062,35.66484375)(501.02226562,35.55351562)(500.39726562,35.55351562)
\curveto(499.55351562,35.55351562)(498.7859375,35.73320312)(498.09453125,36.09257812)
\curveto(497.40703125,36.45585937)(496.8875,36.97929687)(496.5359375,37.66289062)
\curveto(496.184375,38.34648437)(496.00859375,39.11015625)(496.00859375,39.95390625)
\curveto(496.00859375,40.78984375)(496.18242187,41.56914062)(496.53007812,42.29179687)
\curveto(496.88164062,43.01835937)(497.38554687,43.55742187)(498.04179687,43.90898437)
\curveto(498.69804687,44.26054687)(499.45390625,44.43632812)(500.309375,44.43632812)
\curveto(500.93046875,44.43632812)(501.49101562,44.33476562)(501.99101562,44.13164062)
\curveto(502.49492187,43.93242187)(502.88945312,43.653125)(503.17460937,43.29375)
\curveto(503.45976562,42.934375)(503.6765625,42.465625)(503.825,41.8875)
\lineto(502.79960937,41.60625)
\curveto(502.67070312,42.04375)(502.51054687,42.3875)(502.31914062,42.6375)
\curveto(502.12773437,42.8875)(501.85429687,43.08671875)(501.49882812,43.23515625)
\curveto(501.14335937,43.3875)(500.74882812,43.46367187)(500.31523437,43.46367187)
\curveto(499.79570312,43.46367187)(499.34648437,43.38359375)(498.96757812,43.2234375)
\curveto(498.58867187,43.0671875)(498.28203125,42.86015625)(498.04765625,42.60234375)
\curveto(497.8171875,42.34453125)(497.6375,42.06132812)(497.50859375,41.75273437)
\curveto(497.28984375,41.22148437)(497.18046875,40.6453125)(497.18046875,40.02421875)
\curveto(497.18046875,39.25859375)(497.31132812,38.61796875)(497.57304687,38.10234375)
\curveto(497.83867187,37.58671875)(498.2234375,37.20390625)(498.72734375,36.95390625)
\curveto(499.23125,36.70390625)(499.76640625,36.57890625)(500.3328125,36.57890625)
\curveto(500.825,36.57890625)(501.30546875,36.67265625)(501.77421875,36.86015625)
\curveto(502.24296875,37.0515625)(502.5984375,37.2546875)(502.840625,37.46953125)
\lineto(502.840625,39.06914062)
\closepath
}
}
{
\newrgbcolor{curcolor}{0 0 0}
\pscustom[linewidth=1,linecolor=curcolor]
{
\newpath
\moveto(575,57.6)
\lineto(575,66.6)
\moveto(575,425.9)
\lineto(575,416.9)
}
}
{
\newrgbcolor{curcolor}{0 0 0}
\pscustom[linestyle=none,fillstyle=solid,fillcolor=curcolor]
{
\newpath
\moveto(568.12988281,35.7)
\lineto(567.07519531,35.7)
\lineto(567.07519531,42.42070312)
\curveto(566.82128906,42.17851562)(566.48730469,41.93632812)(566.07324219,41.69414062)
\curveto(565.66308594,41.45195312)(565.29394531,41.2703125)(564.96582031,41.14921875)
\lineto(564.96582031,42.16875)
\curveto(565.55566406,42.44609375)(566.07128906,42.78203125)(566.51269531,43.1765625)
\curveto(566.95410156,43.57109375)(567.26660156,43.95390625)(567.45019531,44.325)
\lineto(568.12988281,44.325)
\closepath
}
}
{
\newrgbcolor{curcolor}{0 0 0}
\pscustom[linestyle=none,fillstyle=solid,fillcolor=curcolor]
{
\newpath
\moveto(576.30371094,42.18632812)
\lineto(575.25488281,42.10429687)
\curveto(575.16113281,42.51835937)(575.02832031,42.81914062)(574.85644531,43.00664062)
\curveto(574.57128906,43.30742187)(574.21972656,43.4578125)(573.80175781,43.4578125)
\curveto(573.46582031,43.4578125)(573.17089844,43.3640625)(572.91699219,43.1765625)
\curveto(572.58496094,42.934375)(572.32324219,42.58085937)(572.13183594,42.11601562)
\curveto(571.94042969,41.65117187)(571.84082031,40.9890625)(571.83300781,40.1296875)
\curveto(572.08691406,40.51640625)(572.39746094,40.80351562)(572.76464844,40.99101562)
\curveto(573.13183594,41.17851562)(573.51660156,41.27226562)(573.91894531,41.27226562)
\curveto(574.62207031,41.27226562)(575.21972656,41.0125)(575.71191406,40.49296875)
\curveto(576.20800781,39.97734375)(576.45605469,39.309375)(576.45605469,38.4890625)
\curveto(576.45605469,37.95)(576.33886719,37.44804687)(576.10449219,36.98320312)
\curveto(575.87402344,36.52226562)(575.55566406,36.16875)(575.14941406,35.92265625)
\curveto(574.74316406,35.6765625)(574.28222656,35.55351562)(573.76660156,35.55351562)
\curveto(572.88769531,35.55351562)(572.17089844,35.87578125)(571.61621094,36.5203125)
\curveto(571.06152344,37.16875)(570.78417969,38.23515625)(570.78417969,39.71953125)
\curveto(570.78417969,41.3796875)(571.09082031,42.58671875)(571.70410156,43.340625)
\curveto(572.23925781,43.996875)(572.95996094,44.325)(573.86621094,44.325)
\curveto(574.54199219,44.325)(575.09472656,44.13554687)(575.52441406,43.75664062)
\curveto(575.95800781,43.37773437)(576.21777344,42.85429687)(576.30371094,42.18632812)
\closepath
\moveto(571.99707031,38.48320312)
\curveto(571.99707031,38.11992187)(572.07324219,37.77226562)(572.22558594,37.44023437)
\curveto(572.38183594,37.10820312)(572.59863281,36.85429687)(572.87597656,36.67851562)
\curveto(573.15332031,36.50664062)(573.44433594,36.42070312)(573.74902344,36.42070312)
\curveto(574.19433594,36.42070312)(574.57714844,36.60039062)(574.89746094,36.95976562)
\curveto(575.21777344,37.31914062)(575.37792969,37.80742187)(575.37792969,38.42460937)
\curveto(575.37792969,39.01835937)(575.21972656,39.48515625)(574.90332031,39.825)
\curveto(574.58691406,40.16875)(574.18847656,40.340625)(573.70800781,40.340625)
\curveto(573.23144531,40.340625)(572.82714844,40.16875)(572.49511719,39.825)
\curveto(572.16308594,39.48515625)(571.99707031,39.03789062)(571.99707031,38.48320312)
\closepath
}
}
{
\newrgbcolor{curcolor}{0 0 0}
\pscustom[linestyle=none,fillstyle=solid,fillcolor=curcolor]
{
\newpath
\moveto(581.95214844,39.06914062)
\lineto(581.95214844,40.07695312)
\lineto(585.59082031,40.0828125)
\lineto(585.59082031,36.8953125)
\curveto(585.03222656,36.45)(584.45605469,36.1140625)(583.86230469,35.8875)
\curveto(583.26855469,35.66484375)(582.65917969,35.55351562)(582.03417969,35.55351562)
\curveto(581.19042969,35.55351562)(580.42285156,35.73320312)(579.73144531,36.09257812)
\curveto(579.04394531,36.45585937)(578.52441406,36.97929687)(578.17285156,37.66289062)
\curveto(577.82128906,38.34648437)(577.64550781,39.11015625)(577.64550781,39.95390625)
\curveto(577.64550781,40.78984375)(577.81933594,41.56914062)(578.16699219,42.29179687)
\curveto(578.51855469,43.01835937)(579.02246094,43.55742187)(579.67871094,43.90898437)
\curveto(580.33496094,44.26054687)(581.09082031,44.43632812)(581.94628906,44.43632812)
\curveto(582.56738281,44.43632812)(583.12792969,44.33476562)(583.62792969,44.13164062)
\curveto(584.13183594,43.93242187)(584.52636719,43.653125)(584.81152344,43.29375)
\curveto(585.09667969,42.934375)(585.31347656,42.465625)(585.46191406,41.8875)
\lineto(584.43652344,41.60625)
\curveto(584.30761719,42.04375)(584.14746094,42.3875)(583.95605469,42.6375)
\curveto(583.76464844,42.8875)(583.49121094,43.08671875)(583.13574219,43.23515625)
\curveto(582.78027344,43.3875)(582.38574219,43.46367187)(581.95214844,43.46367187)
\curveto(581.43261719,43.46367187)(580.98339844,43.38359375)(580.60449219,43.2234375)
\curveto(580.22558594,43.0671875)(579.91894531,42.86015625)(579.68457031,42.60234375)
\curveto(579.45410156,42.34453125)(579.27441406,42.06132812)(579.14550781,41.75273437)
\curveto(578.92675781,41.22148437)(578.81738281,40.6453125)(578.81738281,40.02421875)
\curveto(578.81738281,39.25859375)(578.94824219,38.61796875)(579.20996094,38.10234375)
\curveto(579.47558594,37.58671875)(579.86035156,37.20390625)(580.36425781,36.95390625)
\curveto(580.86816406,36.70390625)(581.40332031,36.57890625)(581.96972656,36.57890625)
\curveto(582.46191406,36.57890625)(582.94238281,36.67265625)(583.41113281,36.86015625)
\curveto(583.87988281,37.0515625)(584.23535156,37.2546875)(584.47753906,37.46953125)
\lineto(584.47753906,39.06914062)
\closepath
}
}
{
\newrgbcolor{curcolor}{0 0 0}
\pscustom[linewidth=1,linecolor=curcolor]
{
\newpath
\moveto(105.1,425.9)
\lineto(105.1,57.6)
\lineto(575,57.6)
\lineto(575,425.9)
\closepath
}
}
{
\newrgbcolor{curcolor}{0 0 0}
\pscustom[linestyle=none,fillstyle=solid,fillcolor=curcolor]
{
\newpath
\moveto(16.3,192.29082031)
\lineto(7.71015625,192.29082031)
\lineto(7.71015625,196.09941406)
\curveto(7.71015625,196.86503906)(7.78828125,197.44707031)(7.94453125,197.84550781)
\curveto(8.096875,198.24394531)(8.36835938,198.56230469)(8.75898438,198.80058594)
\curveto(9.14960938,199.03886719)(9.58125,199.15800781)(10.05390625,199.15800781)
\curveto(10.66328125,199.15800781)(11.17695313,198.96074219)(11.59492188,198.56621094)
\curveto(12.01289063,198.17167969)(12.27851563,197.56230469)(12.39179688,196.73808594)
\curveto(12.53632813,197.03886719)(12.67890625,197.26738281)(12.81953125,197.42363281)
\curveto(13.12421875,197.75566406)(13.50507813,198.07011719)(13.96210938,198.36699219)
\lineto(16.3,199.86113281)
\lineto(16.3,198.43144531)
\lineto(14.51289063,197.29472656)
\curveto(13.99726563,196.96269531)(13.60273438,196.68925781)(13.32929688,196.47441406)
\curveto(13.05585938,196.25957031)(12.86445313,196.06621094)(12.75507813,195.89433594)
\curveto(12.64570313,195.72636719)(12.56953125,195.55449219)(12.5265625,195.37871094)
\curveto(12.49921875,195.24980469)(12.48554688,195.03886719)(12.48554688,194.74589844)
\lineto(12.48554688,193.42753906)
\lineto(16.3,193.42753906)
\closepath
\moveto(11.50117188,193.42753906)
\lineto(11.50117188,195.87089844)
\curveto(11.50117188,196.39042969)(11.4484375,196.79667969)(11.34296875,197.08964844)
\curveto(11.23359375,197.38261719)(11.06171875,197.60527344)(10.82734375,197.75761719)
\curveto(10.5890625,197.90996094)(10.33125,197.98613281)(10.05390625,197.98613281)
\curveto(9.64765625,197.98613281)(9.31367188,197.83769531)(9.05195313,197.54082031)
\curveto(8.79023438,197.24785156)(8.659375,196.78300781)(8.659375,196.14628906)
\lineto(8.659375,193.42753906)
\closepath
}
}
{
\newrgbcolor{curcolor}{0 0 0}
\pscustom[linestyle=none,fillstyle=solid,fillcolor=curcolor]
{
\newpath
\moveto(16.3,204.88261719)
\lineto(15.3859375,204.88261719)
\curveto(16.0890625,204.39824219)(16.440625,203.74003906)(16.440625,202.90800781)
\curveto(16.440625,202.54082031)(16.3703125,202.19707031)(16.2296875,201.87675781)
\curveto(16.0890625,201.56035156)(15.91328125,201.32402344)(15.70234375,201.16777344)
\curveto(15.4875,201.01542969)(15.22578125,200.90800781)(14.9171875,200.84550781)
\curveto(14.71015625,200.80253906)(14.38203125,200.78105469)(13.9328125,200.78105469)
\lineto(10.07734375,200.78105469)
\lineto(10.07734375,201.83574219)
\lineto(13.52851563,201.83574219)
\curveto(14.07929688,201.83574219)(14.45039063,201.85722656)(14.64179688,201.90019531)
\curveto(14.91914063,201.96660156)(15.13789063,202.10722656)(15.29804688,202.32207031)
\curveto(15.45429688,202.53691406)(15.53242188,202.80253906)(15.53242188,203.11894531)
\curveto(15.53242188,203.43535156)(15.45234375,203.73222656)(15.2921875,204.00957031)
\curveto(15.128125,204.28691406)(14.90742188,204.48222656)(14.63007813,204.59550781)
\curveto(14.34882813,204.71269531)(13.94257813,204.77128906)(13.41132813,204.77128906)
\lineto(10.07734375,204.77128906)
\lineto(10.07734375,205.82597656)
\lineto(16.3,205.82597656)
\closepath
}
}
{
\newrgbcolor{curcolor}{0 0 0}
\pscustom[linestyle=none,fillstyle=solid,fillcolor=curcolor]
{
\newpath
\moveto(16.3,207.47832031)
\lineto(10.07734375,207.47832031)
\lineto(10.07734375,208.42753906)
\lineto(10.96210938,208.42753906)
\curveto(10.27851563,208.88457031)(9.93671875,209.54472656)(9.93671875,210.40800781)
\curveto(9.93671875,210.78300781)(10.00507813,211.12675781)(10.14179688,211.43925781)
\curveto(10.27460938,211.75566406)(10.45039063,211.99199219)(10.66914063,212.14824219)
\curveto(10.88789063,212.30449219)(11.14765625,212.41386719)(11.4484375,212.47636719)
\curveto(11.64375,212.51542969)(11.98554688,212.53496094)(12.47382813,212.53496094)
\lineto(16.3,212.53496094)
\lineto(16.3,211.48027344)
\lineto(12.51484375,211.48027344)
\curveto(12.08515625,211.48027344)(11.76484375,211.43925781)(11.55390625,211.35722656)
\curveto(11.3390625,211.27519531)(11.16914063,211.12871094)(11.04414063,210.91777344)
\curveto(10.91523438,210.71074219)(10.85078125,210.46660156)(10.85078125,210.18535156)
\curveto(10.85078125,209.73613281)(10.99335938,209.34746094)(11.27851563,209.01933594)
\curveto(11.56367188,208.69511719)(12.1046875,208.53300781)(12.9015625,208.53300781)
\lineto(16.3,208.53300781)
\closepath
}
}
{
\newrgbcolor{curcolor}{0 0 0}
\pscustom[linestyle=none,fillstyle=solid,fillcolor=curcolor]
{
\newpath
\moveto(15.35664063,216.45488281)
\lineto(16.28828125,216.60722656)
\curveto(16.35078125,216.31035156)(16.38203125,216.04472656)(16.38203125,215.81035156)
\curveto(16.38203125,215.42753906)(16.32148438,215.13066406)(16.20039063,214.91972656)
\curveto(16.07929688,214.70878906)(15.92109375,214.56035156)(15.72578125,214.47441406)
\curveto(15.5265625,214.38847656)(15.11054688,214.34550781)(14.47773438,214.34550781)
\lineto(10.89765625,214.34550781)
\lineto(10.89765625,213.57207031)
\lineto(10.07734375,213.57207031)
\lineto(10.07734375,214.34550781)
\lineto(8.53632813,214.34550781)
\lineto(7.90351563,215.39433594)
\lineto(10.07734375,215.39433594)
\lineto(10.07734375,216.45488281)
\lineto(10.89765625,216.45488281)
\lineto(10.89765625,215.39433594)
\lineto(14.53632813,215.39433594)
\curveto(14.83710938,215.39433594)(15.03046875,215.41191406)(15.11640625,215.44707031)
\curveto(15.20234375,215.48613281)(15.27070313,215.54667969)(15.32148438,215.62871094)
\curveto(15.37226563,215.71464844)(15.39765625,215.83574219)(15.39765625,215.99199219)
\curveto(15.39765625,216.10917969)(15.38398438,216.26347656)(15.35664063,216.45488281)
\closepath
}
}
{
\newrgbcolor{curcolor}{0 0 0}
\pscustom[linestyle=none,fillstyle=solid,fillcolor=curcolor]
{
\newpath
\moveto(8.92304688,217.49199219)
\lineto(7.71015625,217.49199219)
\lineto(7.71015625,218.54667969)
\lineto(8.92304688,218.54667969)
\closepath
\moveto(16.3,217.49199219)
\lineto(10.07734375,217.49199219)
\lineto(10.07734375,218.54667969)
\lineto(16.3,218.54667969)
\closepath
}
}
{
\newrgbcolor{curcolor}{0 0 0}
\pscustom[linestyle=none,fillstyle=solid,fillcolor=curcolor]
{
\newpath
\moveto(16.3,220.15214844)
\lineto(10.07734375,220.15214844)
\lineto(10.07734375,221.09550781)
\lineto(10.95039063,221.09550781)
\curveto(10.64570313,221.29082031)(10.4015625,221.55058594)(10.21796875,221.87480469)
\curveto(10.03046875,222.19902344)(9.93671875,222.56816406)(9.93671875,222.98222656)
\curveto(9.93671875,223.44316406)(10.03242188,223.82011719)(10.22382813,224.11308594)
\curveto(10.41523438,224.40996094)(10.6828125,224.61894531)(11.0265625,224.74003906)
\curveto(10.3,225.23222656)(9.93671875,225.87285156)(9.93671875,226.66191406)
\curveto(9.93671875,227.27910156)(10.10859375,227.75371094)(10.45234375,228.08574219)
\curveto(10.7921875,228.41777344)(11.31757813,228.58378906)(12.02851563,228.58378906)
\lineto(16.3,228.58378906)
\lineto(16.3,227.53496094)
\lineto(12.38007813,227.53496094)
\curveto(11.95820313,227.53496094)(11.65546875,227.49980469)(11.471875,227.42949219)
\curveto(11.284375,227.36308594)(11.13398438,227.24003906)(11.02070313,227.06035156)
\curveto(10.90742188,226.88066406)(10.85078125,226.66972656)(10.85078125,226.42753906)
\curveto(10.85078125,225.99003906)(10.99726563,225.62675781)(11.29023438,225.33769531)
\curveto(11.57929688,225.04863281)(12.04414063,224.90410156)(12.68476563,224.90410156)
\lineto(16.3,224.90410156)
\lineto(16.3,223.84941406)
\lineto(12.25703125,223.84941406)
\curveto(11.78828125,223.84941406)(11.43671875,223.76347656)(11.20234375,223.59160156)
\curveto(10.96796875,223.41972656)(10.85078125,223.13847656)(10.85078125,222.74785156)
\curveto(10.85078125,222.45097656)(10.92890625,222.17558594)(11.08515625,221.92167969)
\curveto(11.24140625,221.67167969)(11.46992188,221.49003906)(11.77070313,221.37675781)
\curveto(12.07148438,221.26347656)(12.50507813,221.20683594)(13.07148438,221.20683594)
\lineto(16.3,221.20683594)
\closepath
}
}
{
\newrgbcolor{curcolor}{0 0 0}
\pscustom[linestyle=none,fillstyle=solid,fillcolor=curcolor]
{
\newpath
\moveto(14.29609375,234.40800781)
\lineto(14.43085938,235.49785156)
\curveto(15.06757813,235.32597656)(15.56171875,235.00761719)(15.91328125,234.54277344)
\curveto(16.26484375,234.07792969)(16.440625,233.48417969)(16.440625,232.76152344)
\curveto(16.440625,231.85136719)(16.16132813,231.12871094)(15.60273438,230.59355469)
\curveto(15.04023438,230.06230469)(14.253125,229.79667969)(13.24140625,229.79667969)
\curveto(12.19453125,229.79667969)(11.38203125,230.06621094)(10.80390625,230.60527344)
\curveto(10.22578125,231.14433594)(9.93671875,231.84355469)(9.93671875,232.70292969)
\curveto(9.93671875,233.53496094)(10.21992188,234.21464844)(10.78632813,234.74199219)
\curveto(11.35273438,235.26933594)(12.14960938,235.53300781)(13.17695313,235.53300781)
\curveto(13.23945313,235.53300781)(13.33320313,235.53105469)(13.45820313,235.52714844)
\lineto(13.45820313,230.88652344)
\curveto(14.14179688,230.92558594)(14.66523438,231.11894531)(15.02851563,231.46660156)
\curveto(15.39179688,231.81425781)(15.5734375,232.24785156)(15.5734375,232.76738281)
\curveto(15.5734375,233.15410156)(15.471875,233.48417969)(15.26875,233.75761719)
\curveto(15.065625,234.03105469)(14.74140625,234.24785156)(14.29609375,234.40800781)
\closepath
\moveto(12.59101563,230.94511719)
\lineto(12.59101563,234.41972656)
\curveto(12.06757813,234.37285156)(11.675,234.24003906)(11.41328125,234.02128906)
\curveto(11.00703125,233.68535156)(10.80390625,233.24980469)(10.80390625,232.71464844)
\curveto(10.80390625,232.23027344)(10.96601563,231.82207031)(11.29023438,231.49003906)
\curveto(11.61445313,231.16191406)(12.04804688,230.98027344)(12.59101563,230.94511719)
\closepath
}
}
{
\newrgbcolor{curcolor}{0 0 0}
\pscustom[linestyle=none,fillstyle=solid,fillcolor=curcolor]
{
\newpath
\moveto(18.82539063,242.17167969)
\curveto(18.09101563,241.58964844)(17.23164063,241.09746094)(16.24726563,240.69511719)
\curveto(15.26289063,240.29277344)(14.24335938,240.09160156)(13.18867188,240.09160156)
\curveto(12.25898438,240.09160156)(11.36835938,240.24199219)(10.51679688,240.54277344)
\curveto(9.52851563,240.89433594)(8.54414063,241.43730469)(7.56367188,242.17167969)
\lineto(7.56367188,242.92753906)
\curveto(8.37617188,242.45488281)(8.95625,242.14238281)(9.30390625,241.99003906)
\curveto(9.84296875,241.75175781)(10.40546875,241.56425781)(10.99140625,241.42753906)
\curveto(11.721875,241.25957031)(12.45625,241.17558594)(13.19453125,241.17558594)
\curveto(15.0734375,241.17558594)(16.95039063,241.75957031)(18.82539063,242.92753906)
\closepath
}
}
{
\newrgbcolor{curcolor}{0 0 0}
\pscustom[linestyle=none,fillstyle=solid,fillcolor=curcolor]
{
\newpath
\moveto(14.44257813,243.73027344)
\lineto(14.27851563,244.77324219)
\curveto(14.69648438,244.83183594)(15.01679688,244.99394531)(15.23945313,245.25957031)
\curveto(15.46210938,245.52910156)(15.5734375,245.90410156)(15.5734375,246.38457031)
\curveto(15.5734375,246.86894531)(15.47578125,247.22832031)(15.28046875,247.46269531)
\curveto(15.08125,247.69707031)(14.84882813,247.81425781)(14.58320313,247.81425781)
\curveto(14.34492188,247.81425781)(14.15742188,247.71074219)(14.02070313,247.50371094)
\curveto(13.92695313,247.35917969)(13.8078125,246.99980469)(13.66328125,246.42558594)
\curveto(13.46796875,245.65214844)(13.3,245.11503906)(13.159375,244.81425781)
\curveto(13.01484375,244.51738281)(12.81757813,244.29082031)(12.56757813,244.13457031)
\curveto(12.31367188,243.98222656)(12.034375,243.90605469)(11.7296875,243.90605469)
\curveto(11.45234375,243.90605469)(11.19648438,243.96855469)(10.96210938,244.09355469)
\curveto(10.72382813,244.22246094)(10.5265625,244.39628906)(10.3703125,244.61503906)
\curveto(10.24921875,244.77910156)(10.14765625,245.00175781)(10.065625,245.28300781)
\curveto(9.9796875,245.56816406)(9.93671875,245.87285156)(9.93671875,246.19707031)
\curveto(9.93671875,246.68535156)(10.00703125,247.11308594)(10.14765625,247.48027344)
\curveto(10.28828125,247.85136719)(10.4796875,248.12480469)(10.721875,248.30058594)
\curveto(10.96015625,248.47636719)(11.28046875,248.59746094)(11.6828125,248.66386719)
\lineto(11.8234375,247.63261719)
\curveto(11.503125,247.58574219)(11.253125,247.44902344)(11.0734375,247.22246094)
\curveto(10.89375,246.99980469)(10.80390625,246.68339844)(10.80390625,246.27324219)
\curveto(10.80390625,245.78886719)(10.88398438,245.44316406)(11.04414063,245.23613281)
\curveto(11.20429688,245.02910156)(11.39179688,244.92558594)(11.60664063,244.92558594)
\curveto(11.74335938,244.92558594)(11.86640625,244.96855469)(11.97578125,245.05449219)
\curveto(12.0890625,245.14042969)(12.1828125,245.27519531)(12.25703125,245.45878906)
\curveto(12.29609375,245.56425781)(12.3859375,245.87480469)(12.5265625,246.39042969)
\curveto(12.72578125,247.13652344)(12.88984375,247.65605469)(13.01875,247.94902344)
\curveto(13.14375,248.24589844)(13.32734375,248.47832031)(13.56953125,248.64628906)
\curveto(13.81171875,248.81425781)(14.1125,248.89824219)(14.471875,248.89824219)
\curveto(14.8234375,248.89824219)(15.15546875,248.79472656)(15.46796875,248.58769531)
\curveto(15.7765625,248.38457031)(16.01679688,248.08964844)(16.18867188,247.70292969)
\curveto(16.35664063,247.31621094)(16.440625,246.87871094)(16.440625,246.39042969)
\curveto(16.440625,245.58183594)(16.27265625,244.96464844)(15.93671875,244.53886719)
\curveto(15.60078125,244.11699219)(15.10273438,243.84746094)(14.44257813,243.73027344)
\closepath
}
}
{
\newrgbcolor{curcolor}{0 0 0}
\pscustom[linestyle=none,fillstyle=solid,fillcolor=curcolor]
{
\newpath
\moveto(14.29609375,254.41191406)
\lineto(14.43085938,255.50175781)
\curveto(15.06757813,255.32988281)(15.56171875,255.01152344)(15.91328125,254.54667969)
\curveto(16.26484375,254.08183594)(16.440625,253.48808594)(16.440625,252.76542969)
\curveto(16.440625,251.85527344)(16.16132813,251.13261719)(15.60273438,250.59746094)
\curveto(15.04023438,250.06621094)(14.253125,249.80058594)(13.24140625,249.80058594)
\curveto(12.19453125,249.80058594)(11.38203125,250.07011719)(10.80390625,250.60917969)
\curveto(10.22578125,251.14824219)(9.93671875,251.84746094)(9.93671875,252.70683594)
\curveto(9.93671875,253.53886719)(10.21992188,254.21855469)(10.78632813,254.74589844)
\curveto(11.35273438,255.27324219)(12.14960938,255.53691406)(13.17695313,255.53691406)
\curveto(13.23945313,255.53691406)(13.33320313,255.53496094)(13.45820313,255.53105469)
\lineto(13.45820313,250.89042969)
\curveto(14.14179688,250.92949219)(14.66523438,251.12285156)(15.02851563,251.47050781)
\curveto(15.39179688,251.81816406)(15.5734375,252.25175781)(15.5734375,252.77128906)
\curveto(15.5734375,253.15800781)(15.471875,253.48808594)(15.26875,253.76152344)
\curveto(15.065625,254.03496094)(14.74140625,254.25175781)(14.29609375,254.41191406)
\closepath
\moveto(12.59101563,250.94902344)
\lineto(12.59101563,254.42363281)
\curveto(12.06757813,254.37675781)(11.675,254.24394531)(11.41328125,254.02519531)
\curveto(11.00703125,253.68925781)(10.80390625,253.25371094)(10.80390625,252.71855469)
\curveto(10.80390625,252.23417969)(10.96601563,251.82597656)(11.29023438,251.49394531)
\curveto(11.61445313,251.16582031)(12.04804688,250.98417969)(12.59101563,250.94902344)
\closepath
}
}
{
\newrgbcolor{curcolor}{0 0 0}
\pscustom[linestyle=none,fillstyle=solid,fillcolor=curcolor]
{
\newpath
\moveto(14.02070313,260.88652344)
\lineto(14.15546875,261.92363281)
\curveto(14.8703125,261.81035156)(15.43085938,261.51933594)(15.83710938,261.05058594)
\curveto(16.23945313,260.58574219)(16.440625,260.01347656)(16.440625,259.33378906)
\curveto(16.440625,258.48222656)(16.16328125,257.79667969)(15.60859375,257.27714844)
\curveto(15.05,256.76152344)(14.25117188,256.50371094)(13.21210938,256.50371094)
\curveto(12.54023438,256.50371094)(11.95234375,256.61503906)(11.4484375,256.83769531)
\curveto(10.94453125,257.06035156)(10.56757813,257.39824219)(10.31757813,257.85136719)
\curveto(10.06367188,258.30839844)(9.93671875,258.80449219)(9.93671875,259.33964844)
\curveto(9.93671875,260.01542969)(10.10859375,260.56816406)(10.45234375,260.99785156)
\curveto(10.7921875,261.42753906)(11.2765625,261.70292969)(11.90546875,261.82402344)
\lineto(12.06367188,260.79863281)
\curveto(11.64570313,260.70097656)(11.33125,260.52714844)(11.1203125,260.27714844)
\curveto(10.909375,260.03105469)(10.80390625,259.73222656)(10.80390625,259.38066406)
\curveto(10.80390625,258.84941406)(10.9953125,258.41777344)(11.378125,258.08574219)
\curveto(11.75703125,257.75371094)(12.35859375,257.58769531)(13.1828125,257.58769531)
\curveto(14.01875,257.58769531)(14.62617188,257.74785156)(15.00507813,258.06816406)
\curveto(15.38398438,258.38847656)(15.5734375,258.80644531)(15.5734375,259.32207031)
\curveto(15.5734375,259.73613281)(15.44648438,260.08183594)(15.19257813,260.35917969)
\curveto(14.93867188,260.63652344)(14.54804688,260.81230469)(14.02070313,260.88652344)
\closepath
}
}
{
\newrgbcolor{curcolor}{0 0 0}
\pscustom[linestyle=none,fillstyle=solid,fillcolor=curcolor]
{
\newpath
\moveto(13.18867188,262.43339844)
\curveto(12.03632813,262.43339844)(11.1828125,262.75371094)(10.628125,263.39433594)
\curveto(10.1671875,263.92949219)(9.93671875,264.58183594)(9.93671875,265.35136719)
\curveto(9.93671875,266.20683594)(10.21796875,266.90605469)(10.78046875,267.44902344)
\curveto(11.3390625,267.99199219)(12.1125,268.26347656)(13.10078125,268.26347656)
\curveto(13.9015625,268.26347656)(14.53242188,268.14238281)(14.99335938,267.90019531)
\curveto(15.45039063,267.66191406)(15.80585938,267.31230469)(16.05976563,266.85136719)
\curveto(16.31367188,266.39433594)(16.440625,265.89433594)(16.440625,265.35136719)
\curveto(16.440625,264.48027344)(16.16132813,263.77519531)(15.60273438,263.23613281)
\curveto(15.04414063,262.70097656)(14.23945313,262.43339844)(13.18867188,262.43339844)
\closepath
\moveto(13.18867188,263.51738281)
\curveto(13.98554688,263.51738281)(14.58320313,263.69121094)(14.98164063,264.03886719)
\curveto(15.37617188,264.38652344)(15.5734375,264.82402344)(15.5734375,265.35136719)
\curveto(15.5734375,265.87480469)(15.37421875,266.31035156)(14.97578125,266.65800781)
\curveto(14.57734375,267.00566406)(13.96992188,267.17949219)(13.15351563,267.17949219)
\curveto(12.38398438,267.17949219)(11.80195313,267.00371094)(11.40742188,266.65214844)
\curveto(11.00898438,266.30449219)(10.80976563,265.87089844)(10.80976563,265.35136719)
\curveto(10.80976563,264.82402344)(11.00703125,264.38652344)(11.4015625,264.03886719)
\curveto(11.79609375,263.69121094)(12.39179688,263.51738281)(13.18867188,263.51738281)
\closepath
}
}
{
\newrgbcolor{curcolor}{0 0 0}
\pscustom[linestyle=none,fillstyle=solid,fillcolor=curcolor]
{
\newpath
\moveto(16.3,269.49980469)
\lineto(10.07734375,269.49980469)
\lineto(10.07734375,270.44902344)
\lineto(10.96210938,270.44902344)
\curveto(10.27851563,270.90605469)(9.93671875,271.56621094)(9.93671875,272.42949219)
\curveto(9.93671875,272.80449219)(10.00507813,273.14824219)(10.14179688,273.46074219)
\curveto(10.27460938,273.77714844)(10.45039063,274.01347656)(10.66914063,274.16972656)
\curveto(10.88789063,274.32597656)(11.14765625,274.43535156)(11.4484375,274.49785156)
\curveto(11.64375,274.53691406)(11.98554688,274.55644531)(12.47382813,274.55644531)
\lineto(16.3,274.55644531)
\lineto(16.3,273.50175781)
\lineto(12.51484375,273.50175781)
\curveto(12.08515625,273.50175781)(11.76484375,273.46074219)(11.55390625,273.37871094)
\curveto(11.3390625,273.29667969)(11.16914063,273.15019531)(11.04414063,272.93925781)
\curveto(10.91523438,272.73222656)(10.85078125,272.48808594)(10.85078125,272.20683594)
\curveto(10.85078125,271.75761719)(10.99335938,271.36894531)(11.27851563,271.04082031)
\curveto(11.56367188,270.71660156)(12.1046875,270.55449219)(12.9015625,270.55449219)
\lineto(16.3,270.55449219)
\closepath
}
}
{
\newrgbcolor{curcolor}{0 0 0}
\pscustom[linestyle=none,fillstyle=solid,fillcolor=curcolor]
{
\newpath
\moveto(16.3,280.21074219)
\lineto(15.51484375,280.21074219)
\curveto(16.13203125,279.81621094)(16.440625,279.23613281)(16.440625,278.47050781)
\curveto(16.440625,277.97441406)(16.30390625,277.51738281)(16.03046875,277.09941406)
\curveto(15.75703125,276.68535156)(15.37617188,276.36308594)(14.88789063,276.13261719)
\curveto(14.39570313,275.90605469)(13.83125,275.79277344)(13.19453125,275.79277344)
\curveto(12.5734375,275.79277344)(12.0109375,275.89628906)(11.50703125,276.10332031)
\curveto(10.99921875,276.31035156)(10.61054688,276.62089844)(10.34101563,277.03496094)
\curveto(10.07148438,277.44902344)(9.93671875,277.91191406)(9.93671875,278.42363281)
\curveto(9.93671875,278.79863281)(10.01679688,279.13261719)(10.17695313,279.42558594)
\curveto(10.33320313,279.71855469)(10.53828125,279.95683594)(10.7921875,280.14042969)
\lineto(7.71015625,280.14042969)
\lineto(7.71015625,281.18925781)
\lineto(16.3,281.18925781)
\closepath
\moveto(13.19453125,276.87675781)
\curveto(13.99140625,276.87675781)(14.58710938,277.04472656)(14.98164063,277.38066406)
\curveto(15.37617188,277.71660156)(15.5734375,278.11308594)(15.5734375,278.57011719)
\curveto(15.5734375,279.03105469)(15.3859375,279.42167969)(15.0109375,279.74199219)
\curveto(14.63203125,280.06621094)(14.05585938,280.22832031)(13.28242188,280.22832031)
\curveto(12.43085938,280.22832031)(11.80585938,280.06425781)(11.40742188,279.73613281)
\curveto(11.00898438,279.40800781)(10.80976563,279.00371094)(10.80976563,278.52324219)
\curveto(10.80976563,278.05449219)(11.00117188,277.66191406)(11.38398438,277.34550781)
\curveto(11.76679688,277.03300781)(12.3703125,276.87675781)(13.19453125,276.87675781)
\closepath
}
}
{
\newrgbcolor{curcolor}{0 0 0}
\pscustom[linestyle=none,fillstyle=solid,fillcolor=curcolor]
{
\newpath
\moveto(14.44257813,282.42558594)
\lineto(14.27851563,283.46855469)
\curveto(14.69648438,283.52714844)(15.01679688,283.68925781)(15.23945313,283.95488281)
\curveto(15.46210938,284.22441406)(15.5734375,284.59941406)(15.5734375,285.07988281)
\curveto(15.5734375,285.56425781)(15.47578125,285.92363281)(15.28046875,286.15800781)
\curveto(15.08125,286.39238281)(14.84882813,286.50957031)(14.58320313,286.50957031)
\curveto(14.34492188,286.50957031)(14.15742188,286.40605469)(14.02070313,286.19902344)
\curveto(13.92695313,286.05449219)(13.8078125,285.69511719)(13.66328125,285.12089844)
\curveto(13.46796875,284.34746094)(13.3,283.81035156)(13.159375,283.50957031)
\curveto(13.01484375,283.21269531)(12.81757813,282.98613281)(12.56757813,282.82988281)
\curveto(12.31367188,282.67753906)(12.034375,282.60136719)(11.7296875,282.60136719)
\curveto(11.45234375,282.60136719)(11.19648438,282.66386719)(10.96210938,282.78886719)
\curveto(10.72382813,282.91777344)(10.5265625,283.09160156)(10.3703125,283.31035156)
\curveto(10.24921875,283.47441406)(10.14765625,283.69707031)(10.065625,283.97832031)
\curveto(9.9796875,284.26347656)(9.93671875,284.56816406)(9.93671875,284.89238281)
\curveto(9.93671875,285.38066406)(10.00703125,285.80839844)(10.14765625,286.17558594)
\curveto(10.28828125,286.54667969)(10.4796875,286.82011719)(10.721875,286.99589844)
\curveto(10.96015625,287.17167969)(11.28046875,287.29277344)(11.6828125,287.35917969)
\lineto(11.8234375,286.32792969)
\curveto(11.503125,286.28105469)(11.253125,286.14433594)(11.0734375,285.91777344)
\curveto(10.89375,285.69511719)(10.80390625,285.37871094)(10.80390625,284.96855469)
\curveto(10.80390625,284.48417969)(10.88398438,284.13847656)(11.04414063,283.93144531)
\curveto(11.20429688,283.72441406)(11.39179688,283.62089844)(11.60664063,283.62089844)
\curveto(11.74335938,283.62089844)(11.86640625,283.66386719)(11.97578125,283.74980469)
\curveto(12.0890625,283.83574219)(12.1828125,283.97050781)(12.25703125,284.15410156)
\curveto(12.29609375,284.25957031)(12.3859375,284.57011719)(12.5265625,285.08574219)
\curveto(12.72578125,285.83183594)(12.88984375,286.35136719)(13.01875,286.64433594)
\curveto(13.14375,286.94121094)(13.32734375,287.17363281)(13.56953125,287.34160156)
\curveto(13.81171875,287.50957031)(14.1125,287.59355469)(14.471875,287.59355469)
\curveto(14.8234375,287.59355469)(15.15546875,287.49003906)(15.46796875,287.28300781)
\curveto(15.7765625,287.07988281)(16.01679688,286.78496094)(16.18867188,286.39824219)
\curveto(16.35664063,286.01152344)(16.440625,285.57402344)(16.440625,285.08574219)
\curveto(16.440625,284.27714844)(16.27265625,283.65996094)(15.93671875,283.23417969)
\curveto(15.60078125,282.81230469)(15.10273438,282.54277344)(14.44257813,282.42558594)
\closepath
}
}
{
\newrgbcolor{curcolor}{0 0 0}
\pscustom[linestyle=none,fillstyle=solid,fillcolor=curcolor]
{
\newpath
\moveto(18.82539063,289.53886719)
\lineto(18.82539063,288.78300781)
\curveto(16.95039063,289.95097656)(15.0734375,290.53496094)(13.19453125,290.53496094)
\curveto(12.46015625,290.53496094)(11.73164063,290.45097656)(11.00898438,290.28300781)
\curveto(10.42304688,290.15019531)(9.86054688,289.96464844)(9.32148438,289.72636719)
\curveto(8.96992188,289.57402344)(8.38398438,289.25957031)(7.56367188,288.78300781)
\lineto(7.56367188,289.53886719)
\curveto(8.54414063,290.27324219)(9.52851563,290.81621094)(10.51679688,291.16777344)
\curveto(11.36835938,291.46855469)(12.25898438,291.61894531)(13.18867188,291.61894531)
\curveto(14.24335938,291.61894531)(15.26289063,291.41582031)(16.24726563,291.00957031)
\curveto(17.23164063,290.60722656)(18.09101563,290.11699219)(18.82539063,289.53886719)
\closepath
}
}
{
\newrgbcolor{curcolor}{0 0 0}
\pscustom[linestyle=none,fillstyle=solid,fillcolor=curcolor]
{
\newpath
\moveto(317.97753906,8.7)
\lineto(317.97753906,17.28984375)
\lineto(323.77246094,17.28984375)
\lineto(323.77246094,16.27617187)
\lineto(319.11425781,16.27617187)
\lineto(319.11425781,13.61601562)
\lineto(323.14550781,13.61601562)
\lineto(323.14550781,12.60234375)
\lineto(319.11425781,12.60234375)
\lineto(319.11425781,8.7)
\closepath
}
}
{
\newrgbcolor{curcolor}{0 0 0}
\pscustom[linestyle=none,fillstyle=solid,fillcolor=curcolor]
{
\newpath
\moveto(325.12011719,16.07695312)
\lineto(325.12011719,17.28984375)
\lineto(326.17480469,17.28984375)
\lineto(326.17480469,16.07695312)
\closepath
\moveto(325.12011719,8.7)
\lineto(325.12011719,14.92265625)
\lineto(326.17480469,14.92265625)
\lineto(326.17480469,8.7)
\closepath
}
}
{
\newrgbcolor{curcolor}{0 0 0}
\pscustom[linestyle=none,fillstyle=solid,fillcolor=curcolor]
{
\newpath
\moveto(327.75683594,8.7)
\lineto(327.75683594,17.28984375)
\lineto(328.81152344,17.28984375)
\lineto(328.81152344,8.7)
\closepath
}
}
{
\newrgbcolor{curcolor}{0 0 0}
\pscustom[linestyle=none,fillstyle=solid,fillcolor=curcolor]
{
\newpath
\moveto(334.70605469,10.70390625)
\lineto(335.79589844,10.56914062)
\curveto(335.62402344,9.93242187)(335.30566406,9.43828125)(334.84082031,9.08671875)
\curveto(334.37597656,8.73515625)(333.78222656,8.559375)(333.05957031,8.559375)
\curveto(332.14941406,8.559375)(331.42675781,8.83867187)(330.89160156,9.39726562)
\curveto(330.36035156,9.95976562)(330.09472656,10.746875)(330.09472656,11.75859375)
\curveto(330.09472656,12.80546875)(330.36425781,13.61796875)(330.90332031,14.19609375)
\curveto(331.44238281,14.77421875)(332.14160156,15.06328125)(333.00097656,15.06328125)
\curveto(333.83300781,15.06328125)(334.51269531,14.78007812)(335.04003906,14.21367187)
\curveto(335.56738281,13.64726562)(335.83105469,12.85039062)(335.83105469,11.82304687)
\curveto(335.83105469,11.76054687)(335.82910156,11.66679687)(335.82519531,11.54179687)
\lineto(331.18457031,11.54179687)
\curveto(331.22363281,10.85820312)(331.41699219,10.33476562)(331.76464844,9.97148437)
\curveto(332.11230469,9.60820312)(332.54589844,9.4265625)(333.06542969,9.4265625)
\curveto(333.45214844,9.4265625)(333.78222656,9.528125)(334.05566406,9.73125)
\curveto(334.32910156,9.934375)(334.54589844,10.25859375)(334.70605469,10.70390625)
\closepath
\moveto(331.24316406,12.40898437)
\lineto(334.71777344,12.40898437)
\curveto(334.67089844,12.93242187)(334.53808594,13.325)(334.31933594,13.58671875)
\curveto(333.98339844,13.99296875)(333.54785156,14.19609375)(333.01269531,14.19609375)
\curveto(332.52832031,14.19609375)(332.12011719,14.03398437)(331.78808594,13.70976562)
\curveto(331.45996094,13.38554687)(331.27832031,12.95195312)(331.24316406,12.40898437)
\closepath
}
}
{
\newrgbcolor{curcolor}{0 0 0}
\pscustom[linestyle=none,fillstyle=solid,fillcolor=curcolor]
{
\newpath
\moveto(340.20214844,11.45976562)
\lineto(341.27441406,11.55351562)
\curveto(341.32519531,11.12382812)(341.44238281,10.7703125)(341.62597656,10.49296875)
\curveto(341.81347656,10.21953125)(342.10253906,9.996875)(342.49316406,9.825)
\curveto(342.88378906,9.65703125)(343.32324219,9.57304687)(343.81152344,9.57304687)
\curveto(344.24511719,9.57304687)(344.62792969,9.6375)(344.95996094,9.76640625)
\curveto(345.29199219,9.8953125)(345.53808594,10.07109375)(345.69824219,10.29375)
\curveto(345.86230469,10.5203125)(345.94433594,10.76640625)(345.94433594,11.03203125)
\curveto(345.94433594,11.3015625)(345.86621094,11.5359375)(345.70996094,11.73515625)
\curveto(345.55371094,11.93828125)(345.29589844,12.10820312)(344.93652344,12.24492187)
\curveto(344.70605469,12.33476562)(344.19628906,12.4734375)(343.40722656,12.6609375)
\curveto(342.61816406,12.85234375)(342.06542969,13.03203125)(341.74902344,13.2)
\curveto(341.33886719,13.41484375)(341.03222656,13.68046875)(340.82910156,13.996875)
\curveto(340.62988281,14.3171875)(340.53027344,14.67460937)(340.53027344,15.06914062)
\curveto(340.53027344,15.50273437)(340.65332031,15.90703125)(340.89941406,16.28203125)
\curveto(341.14550781,16.6609375)(341.50488281,16.94804687)(341.97753906,17.14335937)
\curveto(342.45019531,17.33867187)(342.97558594,17.43632812)(343.55371094,17.43632812)
\curveto(344.19042969,17.43632812)(344.75097656,17.3328125)(345.23535156,17.12578125)
\curveto(345.72363281,16.92265625)(346.09863281,16.621875)(346.36035156,16.2234375)
\curveto(346.62207031,15.825)(346.76269531,15.37382812)(346.78222656,14.86992187)
\lineto(345.69238281,14.78789062)
\curveto(345.63378906,15.33085937)(345.43457031,15.74101562)(345.09472656,16.01835937)
\curveto(344.75878906,16.29570312)(344.26074219,16.434375)(343.60058594,16.434375)
\curveto(342.91308594,16.434375)(342.41113281,16.30742187)(342.09472656,16.05351562)
\curveto(341.78222656,15.80351562)(341.62597656,15.50078125)(341.62597656,15.1453125)
\curveto(341.62597656,14.83671875)(341.73730469,14.5828125)(341.95996094,14.38359375)
\curveto(342.17871094,14.184375)(342.74902344,13.97929687)(343.67089844,13.76835937)
\curveto(344.59667969,13.56132812)(345.23144531,13.3796875)(345.57519531,13.2234375)
\curveto(346.07519531,12.99296875)(346.44433594,12.7)(346.68261719,12.34453125)
\curveto(346.92089844,11.99296875)(347.04003906,11.58671875)(347.04003906,11.12578125)
\curveto(347.04003906,10.66875)(346.90917969,10.23710937)(346.64746094,9.83085937)
\curveto(346.38574219,9.42851562)(346.00878906,9.1140625)(345.51660156,8.8875)
\curveto(345.02832031,8.66484375)(344.47753906,8.55351562)(343.86425781,8.55351562)
\curveto(343.08691406,8.55351562)(342.43457031,8.66679687)(341.90722656,8.89335937)
\curveto(341.38378906,9.11992187)(340.97167969,9.45976562)(340.67089844,9.91289062)
\curveto(340.37402344,10.36992187)(340.21777344,10.88554687)(340.20214844,11.45976562)
\closepath
}
}
{
\newrgbcolor{curcolor}{0 0 0}
\pscustom[linestyle=none,fillstyle=solid,fillcolor=curcolor]
{
\newpath
\moveto(348.46386719,16.07695312)
\lineto(348.46386719,17.28984375)
\lineto(349.51855469,17.28984375)
\lineto(349.51855469,16.07695312)
\closepath
\moveto(348.46386719,8.7)
\lineto(348.46386719,14.92265625)
\lineto(349.51855469,14.92265625)
\lineto(349.51855469,8.7)
\closepath
}
}
{
\newrgbcolor{curcolor}{0 0 0}
\pscustom[linestyle=none,fillstyle=solid,fillcolor=curcolor]
{
\newpath
\moveto(350.56738281,8.7)
\lineto(350.56738281,9.55546875)
\lineto(354.52832031,14.10234375)
\curveto(354.07910156,14.07890625)(353.68261719,14.0671875)(353.33886719,14.0671875)
\lineto(350.80175781,14.0671875)
\lineto(350.80175781,14.92265625)
\lineto(355.88769531,14.92265625)
\lineto(355.88769531,14.22539062)
\lineto(352.51855469,10.27617187)
\lineto(351.86816406,9.55546875)
\curveto(352.34082031,9.590625)(352.78417969,9.60820312)(353.19824219,9.60820312)
\lineto(356.07519531,9.60820312)
\lineto(356.07519531,8.7)
\closepath
}
}
{
\newrgbcolor{curcolor}{0 0 0}
\pscustom[linestyle=none,fillstyle=solid,fillcolor=curcolor]
{
\newpath
\moveto(361.38378906,10.70390625)
\lineto(362.47363281,10.56914062)
\curveto(362.30175781,9.93242187)(361.98339844,9.43828125)(361.51855469,9.08671875)
\curveto(361.05371094,8.73515625)(360.45996094,8.559375)(359.73730469,8.559375)
\curveto(358.82714844,8.559375)(358.10449219,8.83867187)(357.56933594,9.39726562)
\curveto(357.03808594,9.95976562)(356.77246094,10.746875)(356.77246094,11.75859375)
\curveto(356.77246094,12.80546875)(357.04199219,13.61796875)(357.58105469,14.19609375)
\curveto(358.12011719,14.77421875)(358.81933594,15.06328125)(359.67871094,15.06328125)
\curveto(360.51074219,15.06328125)(361.19042969,14.78007812)(361.71777344,14.21367187)
\curveto(362.24511719,13.64726562)(362.50878906,12.85039062)(362.50878906,11.82304687)
\curveto(362.50878906,11.76054687)(362.50683594,11.66679687)(362.50292969,11.54179687)
\lineto(357.86230469,11.54179687)
\curveto(357.90136719,10.85820312)(358.09472656,10.33476562)(358.44238281,9.97148437)
\curveto(358.79003906,9.60820312)(359.22363281,9.4265625)(359.74316406,9.4265625)
\curveto(360.12988281,9.4265625)(360.45996094,9.528125)(360.73339844,9.73125)
\curveto(361.00683594,9.934375)(361.22363281,10.25859375)(361.38378906,10.70390625)
\closepath
\moveto(357.92089844,12.40898437)
\lineto(361.39550781,12.40898437)
\curveto(361.34863281,12.93242187)(361.21582031,13.325)(360.99707031,13.58671875)
\curveto(360.66113281,13.99296875)(360.22558594,14.19609375)(359.69042969,14.19609375)
\curveto(359.20605469,14.19609375)(358.79785156,14.03398437)(358.46582031,13.70976562)
\curveto(358.13769531,13.38554687)(357.95605469,12.95195312)(357.92089844,12.40898437)
\closepath
}
}
{
\newrgbcolor{curcolor}{0 0 0}
\pscustom[linestyle=none,fillstyle=solid,fillcolor=curcolor]
{
\newpath
\moveto(186.22363281,449)
\lineto(186.22363281,457.58984375)
\lineto(190.03222656,457.58984375)
\curveto(190.79785156,457.58984375)(191.37988281,457.51171875)(191.77832031,457.35546875)
\curveto(192.17675781,457.203125)(192.49511719,456.93164062)(192.73339844,456.54101562)
\curveto(192.97167969,456.15039062)(193.09082031,455.71875)(193.09082031,455.24609375)
\curveto(193.09082031,454.63671875)(192.89355469,454.12304688)(192.49902344,453.70507812)
\curveto(192.10449219,453.28710938)(191.49511719,453.02148438)(190.67089844,452.90820312)
\curveto(190.97167969,452.76367188)(191.20019531,452.62109375)(191.35644531,452.48046875)
\curveto(191.68847656,452.17578125)(192.00292969,451.79492188)(192.29980469,451.33789062)
\lineto(193.79394531,449)
\lineto(192.36425781,449)
\lineto(191.22753906,450.78710938)
\curveto(190.89550781,451.30273438)(190.62207031,451.69726562)(190.40722656,451.97070312)
\curveto(190.19238281,452.24414062)(189.99902344,452.43554688)(189.82714844,452.54492188)
\curveto(189.65917969,452.65429688)(189.48730469,452.73046875)(189.31152344,452.7734375)
\curveto(189.18261719,452.80078125)(188.97167969,452.81445312)(188.67871094,452.81445312)
\lineto(187.36035156,452.81445312)
\lineto(187.36035156,449)
\closepath
\moveto(187.36035156,453.79882812)
\lineto(189.80371094,453.79882812)
\curveto(190.32324219,453.79882812)(190.72949219,453.8515625)(191.02246094,453.95703125)
\curveto(191.31542969,454.06640625)(191.53808594,454.23828125)(191.69042969,454.47265625)
\curveto(191.84277344,454.7109375)(191.91894531,454.96875)(191.91894531,455.24609375)
\curveto(191.91894531,455.65234375)(191.77050781,455.98632812)(191.47363281,456.24804688)
\curveto(191.18066406,456.50976562)(190.71582031,456.640625)(190.07910156,456.640625)
\lineto(187.36035156,456.640625)
\closepath
}
}
{
\newrgbcolor{curcolor}{0 0 0}
\pscustom[linestyle=none,fillstyle=solid,fillcolor=curcolor]
{
\newpath
\moveto(198.81542969,449)
\lineto(198.81542969,449.9140625)
\curveto(198.33105469,449.2109375)(197.67285156,448.859375)(196.84082031,448.859375)
\curveto(196.47363281,448.859375)(196.12988281,448.9296875)(195.80957031,449.0703125)
\curveto(195.49316406,449.2109375)(195.25683594,449.38671875)(195.10058594,449.59765625)
\curveto(194.94824219,449.8125)(194.84082031,450.07421875)(194.77832031,450.3828125)
\curveto(194.73535156,450.58984375)(194.71386719,450.91796875)(194.71386719,451.3671875)
\lineto(194.71386719,455.22265625)
\lineto(195.76855469,455.22265625)
\lineto(195.76855469,451.77148438)
\curveto(195.76855469,451.22070312)(195.79003906,450.84960938)(195.83300781,450.65820312)
\curveto(195.89941406,450.38085938)(196.04003906,450.16210938)(196.25488281,450.00195312)
\curveto(196.46972656,449.84570312)(196.73535156,449.76757812)(197.05175781,449.76757812)
\curveto(197.36816406,449.76757812)(197.66503906,449.84765625)(197.94238281,450.0078125)
\curveto(198.21972656,450.171875)(198.41503906,450.39257812)(198.52832031,450.66992188)
\curveto(198.64550781,450.95117188)(198.70410156,451.35742188)(198.70410156,451.88867188)
\lineto(198.70410156,455.22265625)
\lineto(199.75878906,455.22265625)
\lineto(199.75878906,449)
\closepath
}
}
{
\newrgbcolor{curcolor}{0 0 0}
\pscustom[linestyle=none,fillstyle=solid,fillcolor=curcolor]
{
\newpath
\moveto(201.41113281,449)
\lineto(201.41113281,455.22265625)
\lineto(202.36035156,455.22265625)
\lineto(202.36035156,454.33789062)
\curveto(202.81738281,455.02148438)(203.47753906,455.36328125)(204.34082031,455.36328125)
\curveto(204.71582031,455.36328125)(205.05957031,455.29492188)(205.37207031,455.15820312)
\curveto(205.68847656,455.02539062)(205.92480469,454.84960938)(206.08105469,454.63085938)
\curveto(206.23730469,454.41210938)(206.34667969,454.15234375)(206.40917969,453.8515625)
\curveto(206.44824219,453.65625)(206.46777344,453.31445312)(206.46777344,452.82617188)
\lineto(206.46777344,449)
\lineto(205.41308594,449)
\lineto(205.41308594,452.78515625)
\curveto(205.41308594,453.21484375)(205.37207031,453.53515625)(205.29003906,453.74609375)
\curveto(205.20800781,453.9609375)(205.06152344,454.13085938)(204.85058594,454.25585938)
\curveto(204.64355469,454.38476562)(204.39941406,454.44921875)(204.11816406,454.44921875)
\curveto(203.66894531,454.44921875)(203.28027344,454.30664062)(202.95214844,454.02148438)
\curveto(202.62792969,453.73632812)(202.46582031,453.1953125)(202.46582031,452.3984375)
\lineto(202.46582031,449)
\closepath
}
}
{
\newrgbcolor{curcolor}{0 0 0}
\pscustom[linestyle=none,fillstyle=solid,fillcolor=curcolor]
{
\newpath
\moveto(210.38769531,449.94335938)
\lineto(210.54003906,449.01171875)
\curveto(210.24316406,448.94921875)(209.97753906,448.91796875)(209.74316406,448.91796875)
\curveto(209.36035156,448.91796875)(209.06347656,448.97851562)(208.85253906,449.09960938)
\curveto(208.64160156,449.22070312)(208.49316406,449.37890625)(208.40722656,449.57421875)
\curveto(208.32128906,449.7734375)(208.27832031,450.18945312)(208.27832031,450.82226562)
\lineto(208.27832031,454.40234375)
\lineto(207.50488281,454.40234375)
\lineto(207.50488281,455.22265625)
\lineto(208.27832031,455.22265625)
\lineto(208.27832031,456.76367188)
\lineto(209.32714844,457.39648438)
\lineto(209.32714844,455.22265625)
\lineto(210.38769531,455.22265625)
\lineto(210.38769531,454.40234375)
\lineto(209.32714844,454.40234375)
\lineto(209.32714844,450.76367188)
\curveto(209.32714844,450.46289062)(209.34472656,450.26953125)(209.37988281,450.18359375)
\curveto(209.41894531,450.09765625)(209.47949219,450.02929688)(209.56152344,449.97851562)
\curveto(209.64746094,449.92773438)(209.76855469,449.90234375)(209.92480469,449.90234375)
\curveto(210.04199219,449.90234375)(210.19628906,449.91601562)(210.38769531,449.94335938)
\closepath
}
}
{
\newrgbcolor{curcolor}{0 0 0}
\pscustom[linestyle=none,fillstyle=solid,fillcolor=curcolor]
{
\newpath
\moveto(211.42480469,456.37695312)
\lineto(211.42480469,457.58984375)
\lineto(212.47949219,457.58984375)
\lineto(212.47949219,456.37695312)
\closepath
\moveto(211.42480469,449)
\lineto(211.42480469,455.22265625)
\lineto(212.47949219,455.22265625)
\lineto(212.47949219,449)
\closepath
}
}
{
\newrgbcolor{curcolor}{0 0 0}
\pscustom[linestyle=none,fillstyle=solid,fillcolor=curcolor]
{
\newpath
\moveto(214.08496094,449)
\lineto(214.08496094,455.22265625)
\lineto(215.02832031,455.22265625)
\lineto(215.02832031,454.34960938)
\curveto(215.22363281,454.65429688)(215.48339844,454.8984375)(215.80761719,455.08203125)
\curveto(216.13183594,455.26953125)(216.50097656,455.36328125)(216.91503906,455.36328125)
\curveto(217.37597656,455.36328125)(217.75292969,455.26757812)(218.04589844,455.07617188)
\curveto(218.34277344,454.88476562)(218.55175781,454.6171875)(218.67285156,454.2734375)
\curveto(219.16503906,455)(219.80566406,455.36328125)(220.59472656,455.36328125)
\curveto(221.21191406,455.36328125)(221.68652344,455.19140625)(222.01855469,454.84765625)
\curveto(222.35058594,454.5078125)(222.51660156,453.98242188)(222.51660156,453.27148438)
\lineto(222.51660156,449)
\lineto(221.46777344,449)
\lineto(221.46777344,452.91992188)
\curveto(221.46777344,453.34179688)(221.43261719,453.64453125)(221.36230469,453.828125)
\curveto(221.29589844,454.015625)(221.17285156,454.16601562)(220.99316406,454.27929688)
\curveto(220.81347656,454.39257812)(220.60253906,454.44921875)(220.36035156,454.44921875)
\curveto(219.92285156,454.44921875)(219.55957031,454.30273438)(219.27050781,454.00976562)
\curveto(218.98144531,453.72070312)(218.83691406,453.25585938)(218.83691406,452.61523438)
\lineto(218.83691406,449)
\lineto(217.78222656,449)
\lineto(217.78222656,453.04296875)
\curveto(217.78222656,453.51171875)(217.69628906,453.86328125)(217.52441406,454.09765625)
\curveto(217.35253906,454.33203125)(217.07128906,454.44921875)(216.68066406,454.44921875)
\curveto(216.38378906,454.44921875)(216.10839844,454.37109375)(215.85449219,454.21484375)
\curveto(215.60449219,454.05859375)(215.42285156,453.83007812)(215.30957031,453.52929688)
\curveto(215.19628906,453.22851562)(215.13964844,452.79492188)(215.13964844,452.22851562)
\lineto(215.13964844,449)
\closepath
}
}
{
\newrgbcolor{curcolor}{0 0 0}
\pscustom[linestyle=none,fillstyle=solid,fillcolor=curcolor]
{
\newpath
\moveto(228.34082031,451.00390625)
\lineto(229.43066406,450.86914062)
\curveto(229.25878906,450.23242188)(228.94042969,449.73828125)(228.47558594,449.38671875)
\curveto(228.01074219,449.03515625)(227.41699219,448.859375)(226.69433594,448.859375)
\curveto(225.78417969,448.859375)(225.06152344,449.13867188)(224.52636719,449.69726562)
\curveto(223.99511719,450.25976562)(223.72949219,451.046875)(223.72949219,452.05859375)
\curveto(223.72949219,453.10546875)(223.99902344,453.91796875)(224.53808594,454.49609375)
\curveto(225.07714844,455.07421875)(225.77636719,455.36328125)(226.63574219,455.36328125)
\curveto(227.46777344,455.36328125)(228.14746094,455.08007812)(228.67480469,454.51367188)
\curveto(229.20214844,453.94726562)(229.46582031,453.15039062)(229.46582031,452.12304688)
\curveto(229.46582031,452.06054688)(229.46386719,451.96679688)(229.45996094,451.84179688)
\lineto(224.81933594,451.84179688)
\curveto(224.85839844,451.15820312)(225.05175781,450.63476562)(225.39941406,450.27148438)
\curveto(225.74707031,449.90820312)(226.18066406,449.7265625)(226.70019531,449.7265625)
\curveto(227.08691406,449.7265625)(227.41699219,449.828125)(227.69042969,450.03125)
\curveto(227.96386719,450.234375)(228.18066406,450.55859375)(228.34082031,451.00390625)
\closepath
\moveto(224.87792969,452.70898438)
\lineto(228.35253906,452.70898438)
\curveto(228.30566406,453.23242188)(228.17285156,453.625)(227.95410156,453.88671875)
\curveto(227.61816406,454.29296875)(227.18261719,454.49609375)(226.64746094,454.49609375)
\curveto(226.16308594,454.49609375)(225.75488281,454.33398438)(225.42285156,454.00976562)
\curveto(225.09472656,453.68554688)(224.91308594,453.25195312)(224.87792969,452.70898438)
\closepath
}
}
{
\newrgbcolor{curcolor}{0 0 0}
\pscustom[linestyle=none,fillstyle=solid,fillcolor=curcolor]
{
\newpath
\moveto(233.87792969,453.18359375)
\curveto(233.87792969,454.609375)(234.26074219,455.72460938)(235.02636719,456.52929688)
\curveto(235.79199219,457.33789062)(236.78027344,457.7421875)(237.99121094,457.7421875)
\curveto(238.78417969,457.7421875)(239.49902344,457.55273438)(240.13574219,457.17382812)
\curveto(240.77246094,456.79492188)(241.25683594,456.265625)(241.58886719,455.5859375)
\curveto(241.92480469,454.91015625)(242.09277344,454.14257812)(242.09277344,453.28320312)
\curveto(242.09277344,452.41210938)(241.91699219,451.6328125)(241.56542969,450.9453125)
\curveto(241.21386719,450.2578125)(240.71582031,449.73632812)(240.07128906,449.38085938)
\curveto(239.42675781,449.02929688)(238.73144531,448.85351562)(237.98535156,448.85351562)
\curveto(237.17675781,448.85351562)(236.45410156,449.04882812)(235.81738281,449.43945312)
\curveto(235.18066406,449.83007812)(234.69824219,450.36328125)(234.37011719,451.0390625)
\curveto(234.04199219,451.71484375)(233.87792969,452.4296875)(233.87792969,453.18359375)
\closepath
\moveto(235.04980469,453.16601562)
\curveto(235.04980469,452.13085938)(235.32714844,451.31445312)(235.88183594,450.71679688)
\curveto(236.44042969,450.12304688)(237.13964844,449.82617188)(237.97949219,449.82617188)
\curveto(238.83496094,449.82617188)(239.53808594,450.12695312)(240.08886719,450.72851562)
\curveto(240.64355469,451.33007812)(240.92089844,452.18359375)(240.92089844,453.2890625)
\curveto(240.92089844,453.98828125)(240.80175781,454.59765625)(240.56347656,455.1171875)
\curveto(240.32910156,455.640625)(239.98339844,456.04492188)(239.52636719,456.33007812)
\curveto(239.07324219,456.61914062)(238.56347656,456.76367188)(237.99707031,456.76367188)
\curveto(237.19238281,456.76367188)(236.49902344,456.48632812)(235.91699219,455.93164062)
\curveto(235.33886719,455.38085938)(235.04980469,454.45898438)(235.04980469,453.16601562)
\closepath
}
}
{
\newrgbcolor{curcolor}{0 0 0}
\pscustom[linestyle=none,fillstyle=solid,fillcolor=curcolor]
{
\newpath
\moveto(243.39941406,449)
\lineto(243.39941406,457.58984375)
\lineto(244.45410156,457.58984375)
\lineto(244.45410156,449)
\closepath
}
}
{
\newrgbcolor{curcolor}{0 0 0}
\pscustom[linestyle=none,fillstyle=solid,fillcolor=curcolor]
{
\newpath
\moveto(250.12597656,449)
\lineto(250.12597656,449.78515625)
\curveto(249.73144531,449.16796875)(249.15136719,448.859375)(248.38574219,448.859375)
\curveto(247.88964844,448.859375)(247.43261719,448.99609375)(247.01464844,449.26953125)
\curveto(246.60058594,449.54296875)(246.27832031,449.92382812)(246.04785156,450.41210938)
\curveto(245.82128906,450.90429688)(245.70800781,451.46875)(245.70800781,452.10546875)
\curveto(245.70800781,452.7265625)(245.81152344,453.2890625)(246.01855469,453.79296875)
\curveto(246.22558594,454.30078125)(246.53613281,454.68945312)(246.95019531,454.95898438)
\curveto(247.36425781,455.22851562)(247.82714844,455.36328125)(248.33886719,455.36328125)
\curveto(248.71386719,455.36328125)(249.04785156,455.28320312)(249.34082031,455.12304688)
\curveto(249.63378906,454.96679688)(249.87207031,454.76171875)(250.05566406,454.5078125)
\lineto(250.05566406,457.58984375)
\lineto(251.10449219,457.58984375)
\lineto(251.10449219,449)
\closepath
\moveto(246.79199219,452.10546875)
\curveto(246.79199219,451.30859375)(246.95996094,450.71289062)(247.29589844,450.31835938)
\curveto(247.63183594,449.92382812)(248.02832031,449.7265625)(248.48535156,449.7265625)
\curveto(248.94628906,449.7265625)(249.33691406,449.9140625)(249.65722656,450.2890625)
\curveto(249.98144531,450.66796875)(250.14355469,451.24414062)(250.14355469,452.01757812)
\curveto(250.14355469,452.86914062)(249.97949219,453.49414062)(249.65136719,453.89257812)
\curveto(249.32324219,454.29101562)(248.91894531,454.49023438)(248.43847656,454.49023438)
\curveto(247.96972656,454.49023438)(247.57714844,454.29882812)(247.26074219,453.91601562)
\curveto(246.94824219,453.53320312)(246.79199219,452.9296875)(246.79199219,452.10546875)
\closepath
}
}
{
\newrgbcolor{curcolor}{0 0 0}
\pscustom[linestyle=none,fillstyle=solid,fillcolor=curcolor]
{
\newpath
\moveto(255.54589844,449)
\lineto(255.54589844,450.0546875)
\lineto(259.94628906,455.55664062)
\curveto(260.25878906,455.94726562)(260.55566406,456.28710938)(260.83691406,456.57617188)
\lineto(256.04394531,456.57617188)
\lineto(256.04394531,457.58984375)
\lineto(262.19628906,457.58984375)
\lineto(262.19628906,456.57617188)
\lineto(257.37402344,450.6171875)
\lineto(256.85253906,450.01367188)
\lineto(262.33691406,450.01367188)
\lineto(262.33691406,449)
\closepath
}
}
{
\newrgbcolor{curcolor}{0 0 0}
\pscustom[linestyle=none,fillstyle=solid,fillcolor=curcolor]
{
\newpath
\moveto(263.62011719,449)
\lineto(263.62011719,457.58984375)
\lineto(269.41503906,457.58984375)
\lineto(269.41503906,456.57617188)
\lineto(264.75683594,456.57617188)
\lineto(264.75683594,453.91601562)
\lineto(268.78808594,453.91601562)
\lineto(268.78808594,452.90234375)
\lineto(264.75683594,452.90234375)
\lineto(264.75683594,449)
\closepath
}
}
{
\newrgbcolor{curcolor}{0 0 0}
\pscustom[linestyle=none,fillstyle=solid,fillcolor=curcolor]
{
\newpath
\moveto(270.50488281,451.75976562)
\lineto(271.57714844,451.85351562)
\curveto(271.62792969,451.42382812)(271.74511719,451.0703125)(271.92871094,450.79296875)
\curveto(272.11621094,450.51953125)(272.40527344,450.296875)(272.79589844,450.125)
\curveto(273.18652344,449.95703125)(273.62597656,449.87304688)(274.11425781,449.87304688)
\curveto(274.54785156,449.87304688)(274.93066406,449.9375)(275.26269531,450.06640625)
\curveto(275.59472656,450.1953125)(275.84082031,450.37109375)(276.00097656,450.59375)
\curveto(276.16503906,450.8203125)(276.24707031,451.06640625)(276.24707031,451.33203125)
\curveto(276.24707031,451.6015625)(276.16894531,451.8359375)(276.01269531,452.03515625)
\curveto(275.85644531,452.23828125)(275.59863281,452.40820312)(275.23925781,452.54492188)
\curveto(275.00878906,452.63476562)(274.49902344,452.7734375)(273.70996094,452.9609375)
\curveto(272.92089844,453.15234375)(272.36816406,453.33203125)(272.05175781,453.5)
\curveto(271.64160156,453.71484375)(271.33496094,453.98046875)(271.13183594,454.296875)
\curveto(270.93261719,454.6171875)(270.83300781,454.97460938)(270.83300781,455.36914062)
\curveto(270.83300781,455.80273438)(270.95605469,456.20703125)(271.20214844,456.58203125)
\curveto(271.44824219,456.9609375)(271.80761719,457.24804688)(272.28027344,457.44335938)
\curveto(272.75292969,457.63867188)(273.27832031,457.73632812)(273.85644531,457.73632812)
\curveto(274.49316406,457.73632812)(275.05371094,457.6328125)(275.53808594,457.42578125)
\curveto(276.02636719,457.22265625)(276.40136719,456.921875)(276.66308594,456.5234375)
\curveto(276.92480469,456.125)(277.06542969,455.67382812)(277.08496094,455.16992188)
\lineto(275.99511719,455.08789062)
\curveto(275.93652344,455.63085938)(275.73730469,456.04101562)(275.39746094,456.31835938)
\curveto(275.06152344,456.59570312)(274.56347656,456.734375)(273.90332031,456.734375)
\curveto(273.21582031,456.734375)(272.71386719,456.60742188)(272.39746094,456.35351562)
\curveto(272.08496094,456.10351562)(271.92871094,455.80078125)(271.92871094,455.4453125)
\curveto(271.92871094,455.13671875)(272.04003906,454.8828125)(272.26269531,454.68359375)
\curveto(272.48144531,454.484375)(273.05175781,454.27929688)(273.97363281,454.06835938)
\curveto(274.89941406,453.86132812)(275.53417969,453.6796875)(275.87792969,453.5234375)
\curveto(276.37792969,453.29296875)(276.74707031,453)(276.98535156,452.64453125)
\curveto(277.22363281,452.29296875)(277.34277344,451.88671875)(277.34277344,451.42578125)
\curveto(277.34277344,450.96875)(277.21191406,450.53710938)(276.95019531,450.13085938)
\curveto(276.68847656,449.72851562)(276.31152344,449.4140625)(275.81933594,449.1875)
\curveto(275.33105469,448.96484375)(274.78027344,448.85351562)(274.16699219,448.85351562)
\curveto(273.38964844,448.85351562)(272.73730469,448.96679688)(272.20996094,449.19335938)
\curveto(271.68652344,449.41992188)(271.27441406,449.75976562)(270.97363281,450.21289062)
\curveto(270.67675781,450.66992188)(270.52050781,451.18554688)(270.50488281,451.75976562)
\closepath
}
}
{
\newrgbcolor{curcolor}{0 0 0}
\pscustom[linestyle=none,fillstyle=solid,fillcolor=curcolor]
{
\newpath
\moveto(285.77441406,449)
\lineto(284.71972656,449)
\lineto(284.71972656,455.72070312)
\curveto(284.46582031,455.47851562)(284.13183594,455.23632812)(283.71777344,454.99414062)
\curveto(283.30761719,454.75195312)(282.93847656,454.5703125)(282.61035156,454.44921875)
\lineto(282.61035156,455.46875)
\curveto(283.20019531,455.74609375)(283.71582031,456.08203125)(284.15722656,456.4765625)
\curveto(284.59863281,456.87109375)(284.91113281,457.25390625)(285.09472656,457.625)
\lineto(285.77441406,457.625)
\closepath
}
}
{
\newrgbcolor{curcolor}{0 0 0}
\pscustom[linestyle=none,fillstyle=solid,fillcolor=curcolor]
{
\newpath
\moveto(288.86816406,449)
\lineto(288.86816406,457.58984375)
\lineto(290.57910156,457.58984375)
\lineto(292.61230469,451.5078125)
\curveto(292.79980469,450.94140625)(292.93652344,450.51757812)(293.02246094,450.23632812)
\curveto(293.12011719,450.54882812)(293.27246094,451.0078125)(293.47949219,451.61328125)
\lineto(295.53613281,457.58984375)
\lineto(297.06542969,457.58984375)
\lineto(297.06542969,449)
\lineto(295.96972656,449)
\lineto(295.96972656,456.18945312)
\lineto(293.47363281,449)
\lineto(292.44824219,449)
\lineto(289.96386719,456.3125)
\lineto(289.96386719,449)
\closepath
}
}
{
\newrgbcolor{curcolor}{0 0 0}
\pscustom[linestyle=none,fillstyle=solid,fillcolor=curcolor]
{
\newpath
\moveto(302.25097656,449)
\lineto(302.25097656,457.58984375)
\lineto(306.05957031,457.58984375)
\curveto(306.82519531,457.58984375)(307.40722656,457.51171875)(307.80566406,457.35546875)
\curveto(308.20410156,457.203125)(308.52246094,456.93164062)(308.76074219,456.54101562)
\curveto(308.99902344,456.15039062)(309.11816406,455.71875)(309.11816406,455.24609375)
\curveto(309.11816406,454.63671875)(308.92089844,454.12304688)(308.52636719,453.70507812)
\curveto(308.13183594,453.28710938)(307.52246094,453.02148438)(306.69824219,452.90820312)
\curveto(306.99902344,452.76367188)(307.22753906,452.62109375)(307.38378906,452.48046875)
\curveto(307.71582031,452.17578125)(308.03027344,451.79492188)(308.32714844,451.33789062)
\lineto(309.82128906,449)
\lineto(308.39160156,449)
\lineto(307.25488281,450.78710938)
\curveto(306.92285156,451.30273438)(306.64941406,451.69726562)(306.43457031,451.97070312)
\curveto(306.21972656,452.24414062)(306.02636719,452.43554688)(305.85449219,452.54492188)
\curveto(305.68652344,452.65429688)(305.51464844,452.73046875)(305.33886719,452.7734375)
\curveto(305.20996094,452.80078125)(304.99902344,452.81445312)(304.70605469,452.81445312)
\lineto(303.38769531,452.81445312)
\lineto(303.38769531,449)
\closepath
\moveto(303.38769531,453.79882812)
\lineto(305.83105469,453.79882812)
\curveto(306.35058594,453.79882812)(306.75683594,453.8515625)(307.04980469,453.95703125)
\curveto(307.34277344,454.06640625)(307.56542969,454.23828125)(307.71777344,454.47265625)
\curveto(307.87011719,454.7109375)(307.94628906,454.96875)(307.94628906,455.24609375)
\curveto(307.94628906,455.65234375)(307.79785156,455.98632812)(307.50097656,456.24804688)
\curveto(307.20800781,456.50976562)(306.74316406,456.640625)(306.10644531,456.640625)
\lineto(303.38769531,456.640625)
\closepath
}
}
{
\newrgbcolor{curcolor}{0 0 0}
\pscustom[linestyle=none,fillstyle=solid,fillcolor=curcolor]
{
\newpath
\moveto(315.02441406,451.00390625)
\lineto(316.11425781,450.86914062)
\curveto(315.94238281,450.23242188)(315.62402344,449.73828125)(315.15917969,449.38671875)
\curveto(314.69433594,449.03515625)(314.10058594,448.859375)(313.37792969,448.859375)
\curveto(312.46777344,448.859375)(311.74511719,449.13867188)(311.20996094,449.69726562)
\curveto(310.67871094,450.25976562)(310.41308594,451.046875)(310.41308594,452.05859375)
\curveto(310.41308594,453.10546875)(310.68261719,453.91796875)(311.22167969,454.49609375)
\curveto(311.76074219,455.07421875)(312.45996094,455.36328125)(313.31933594,455.36328125)
\curveto(314.15136719,455.36328125)(314.83105469,455.08007812)(315.35839844,454.51367188)
\curveto(315.88574219,453.94726562)(316.14941406,453.15039062)(316.14941406,452.12304688)
\curveto(316.14941406,452.06054688)(316.14746094,451.96679688)(316.14355469,451.84179688)
\lineto(311.50292969,451.84179688)
\curveto(311.54199219,451.15820312)(311.73535156,450.63476562)(312.08300781,450.27148438)
\curveto(312.43066406,449.90820312)(312.86425781,449.7265625)(313.38378906,449.7265625)
\curveto(313.77050781,449.7265625)(314.10058594,449.828125)(314.37402344,450.03125)
\curveto(314.64746094,450.234375)(314.86425781,450.55859375)(315.02441406,451.00390625)
\closepath
\moveto(311.56152344,452.70898438)
\lineto(315.03613281,452.70898438)
\curveto(314.98925781,453.23242188)(314.85644531,453.625)(314.63769531,453.88671875)
\curveto(314.30175781,454.29296875)(313.86621094,454.49609375)(313.33105469,454.49609375)
\curveto(312.84667969,454.49609375)(312.43847656,454.33398438)(312.10644531,454.00976562)
\curveto(311.77832031,453.68554688)(311.59667969,453.25195312)(311.56152344,452.70898438)
\closepath
}
}
{
\newrgbcolor{curcolor}{0 0 0}
\pscustom[linestyle=none,fillstyle=solid,fillcolor=curcolor]
{
\newpath
\moveto(321.49902344,449.76757812)
\curveto(321.10839844,449.43554688)(320.73144531,449.20117188)(320.36816406,449.06445312)
\curveto(320.00878906,448.92773438)(319.62207031,448.859375)(319.20800781,448.859375)
\curveto(318.52441406,448.859375)(317.99902344,449.02539062)(317.63183594,449.35742188)
\curveto(317.26464844,449.69335938)(317.08105469,450.12109375)(317.08105469,450.640625)
\curveto(317.08105469,450.9453125)(317.14941406,451.22265625)(317.28613281,451.47265625)
\curveto(317.42675781,451.7265625)(317.60839844,451.9296875)(317.83105469,452.08203125)
\curveto(318.05761719,452.234375)(318.31152344,452.34960938)(318.59277344,452.42773438)
\curveto(318.79980469,452.48242188)(319.11230469,452.53515625)(319.53027344,452.5859375)
\curveto(320.38183594,452.6875)(321.00878906,452.80859375)(321.41113281,452.94921875)
\curveto(321.41503906,453.09375)(321.41699219,453.18554688)(321.41699219,453.22460938)
\curveto(321.41699219,453.65429688)(321.31738281,453.95703125)(321.11816406,454.1328125)
\curveto(320.84863281,454.37109375)(320.44824219,454.49023438)(319.91699219,454.49023438)
\curveto(319.42089844,454.49023438)(319.05371094,454.40234375)(318.81542969,454.2265625)
\curveto(318.58105469,454.0546875)(318.40722656,453.74804688)(318.29394531,453.30664062)
\lineto(317.26269531,453.44726562)
\curveto(317.35644531,453.88867188)(317.51074219,454.24414062)(317.72558594,454.51367188)
\curveto(317.94042969,454.78710938)(318.25097656,454.99609375)(318.65722656,455.140625)
\curveto(319.06347656,455.2890625)(319.53417969,455.36328125)(320.06933594,455.36328125)
\curveto(320.60058594,455.36328125)(321.03222656,455.30078125)(321.36425781,455.17578125)
\curveto(321.69628906,455.05078125)(321.94042969,454.89257812)(322.09667969,454.70117188)
\curveto(322.25292969,454.51367188)(322.36230469,454.27539062)(322.42480469,453.98632812)
\curveto(322.45996094,453.80664062)(322.47753906,453.48242188)(322.47753906,453.01367188)
\lineto(322.47753906,451.60742188)
\curveto(322.47753906,450.62695312)(322.49902344,450.00585938)(322.54199219,449.74414062)
\curveto(322.58886719,449.48632812)(322.67871094,449.23828125)(322.81152344,449)
\lineto(321.70996094,449)
\curveto(321.60058594,449.21875)(321.53027344,449.47460938)(321.49902344,449.76757812)
\closepath
\moveto(321.41113281,452.12304688)
\curveto(321.02832031,451.96679688)(320.45410156,451.83398438)(319.68847656,451.72460938)
\curveto(319.25488281,451.66210938)(318.94824219,451.59179688)(318.76855469,451.51367188)
\curveto(318.58886719,451.43554688)(318.45019531,451.3203125)(318.35253906,451.16796875)
\curveto(318.25488281,451.01953125)(318.20605469,450.85351562)(318.20605469,450.66992188)
\curveto(318.20605469,450.38867188)(318.31152344,450.15429688)(318.52246094,449.96679688)
\curveto(318.73730469,449.77929688)(319.04980469,449.68554688)(319.45996094,449.68554688)
\curveto(319.86621094,449.68554688)(320.22753906,449.7734375)(320.54394531,449.94921875)
\curveto(320.86035156,450.12890625)(321.09277344,450.37304688)(321.24121094,450.68164062)
\curveto(321.35449219,450.91992188)(321.41113281,451.27148438)(321.41113281,451.73632812)
\closepath
}
}
{
\newrgbcolor{curcolor}{0 0 0}
\pscustom[linestyle=none,fillstyle=solid,fillcolor=curcolor]
{
\newpath
\moveto(328.14941406,449)
\lineto(328.14941406,449.78515625)
\curveto(327.75488281,449.16796875)(327.17480469,448.859375)(326.40917969,448.859375)
\curveto(325.91308594,448.859375)(325.45605469,448.99609375)(325.03808594,449.26953125)
\curveto(324.62402344,449.54296875)(324.30175781,449.92382812)(324.07128906,450.41210938)
\curveto(323.84472656,450.90429688)(323.73144531,451.46875)(323.73144531,452.10546875)
\curveto(323.73144531,452.7265625)(323.83496094,453.2890625)(324.04199219,453.79296875)
\curveto(324.24902344,454.30078125)(324.55957031,454.68945312)(324.97363281,454.95898438)
\curveto(325.38769531,455.22851562)(325.85058594,455.36328125)(326.36230469,455.36328125)
\curveto(326.73730469,455.36328125)(327.07128906,455.28320312)(327.36425781,455.12304688)
\curveto(327.65722656,454.96679688)(327.89550781,454.76171875)(328.07910156,454.5078125)
\lineto(328.07910156,457.58984375)
\lineto(329.12792969,457.58984375)
\lineto(329.12792969,449)
\closepath
\moveto(324.81542969,452.10546875)
\curveto(324.81542969,451.30859375)(324.98339844,450.71289062)(325.31933594,450.31835938)
\curveto(325.65527344,449.92382812)(326.05175781,449.7265625)(326.50878906,449.7265625)
\curveto(326.96972656,449.7265625)(327.36035156,449.9140625)(327.68066406,450.2890625)
\curveto(328.00488281,450.66796875)(328.16699219,451.24414062)(328.16699219,452.01757812)
\curveto(328.16699219,452.86914062)(328.00292969,453.49414062)(327.67480469,453.89257812)
\curveto(327.34667969,454.29101562)(326.94238281,454.49023438)(326.46191406,454.49023438)
\curveto(325.99316406,454.49023438)(325.60058594,454.29882812)(325.28417969,453.91601562)
\curveto(324.97167969,453.53320312)(324.81542969,452.9296875)(324.81542969,452.10546875)
\closepath
}
}
{
\newrgbcolor{curcolor}{0 0 0}
\pscustom[linestyle=none,fillstyle=solid,fillcolor=curcolor]
{
\newpath
\moveto(330.36425781,450.85742188)
\lineto(331.40722656,451.02148438)
\curveto(331.46582031,450.60351562)(331.62792969,450.28320312)(331.89355469,450.06054688)
\curveto(332.16308594,449.83789062)(332.53808594,449.7265625)(333.01855469,449.7265625)
\curveto(333.50292969,449.7265625)(333.86230469,449.82421875)(334.09667969,450.01953125)
\curveto(334.33105469,450.21875)(334.44824219,450.45117188)(334.44824219,450.71679688)
\curveto(334.44824219,450.95507812)(334.34472656,451.14257812)(334.13769531,451.27929688)
\curveto(333.99316406,451.37304688)(333.63378906,451.4921875)(333.05957031,451.63671875)
\curveto(332.28613281,451.83203125)(331.74902344,452)(331.44824219,452.140625)
\curveto(331.15136719,452.28515625)(330.92480469,452.48242188)(330.76855469,452.73242188)
\curveto(330.61621094,452.98632812)(330.54003906,453.265625)(330.54003906,453.5703125)
\curveto(330.54003906,453.84765625)(330.60253906,454.10351562)(330.72753906,454.33789062)
\curveto(330.85644531,454.57617188)(331.03027344,454.7734375)(331.24902344,454.9296875)
\curveto(331.41308594,455.05078125)(331.63574219,455.15234375)(331.91699219,455.234375)
\curveto(332.20214844,455.3203125)(332.50683594,455.36328125)(332.83105469,455.36328125)
\curveto(333.31933594,455.36328125)(333.74707031,455.29296875)(334.11425781,455.15234375)
\curveto(334.48535156,455.01171875)(334.75878906,454.8203125)(334.93457031,454.578125)
\curveto(335.11035156,454.33984375)(335.23144531,454.01953125)(335.29785156,453.6171875)
\lineto(334.26660156,453.4765625)
\curveto(334.21972656,453.796875)(334.08300781,454.046875)(333.85644531,454.2265625)
\curveto(333.63378906,454.40625)(333.31738281,454.49609375)(332.90722656,454.49609375)
\curveto(332.42285156,454.49609375)(332.07714844,454.41601562)(331.87011719,454.25585938)
\curveto(331.66308594,454.09570312)(331.55957031,453.90820312)(331.55957031,453.69335938)
\curveto(331.55957031,453.55664062)(331.60253906,453.43359375)(331.68847656,453.32421875)
\curveto(331.77441406,453.2109375)(331.90917969,453.1171875)(332.09277344,453.04296875)
\curveto(332.19824219,453.00390625)(332.50878906,452.9140625)(333.02441406,452.7734375)
\curveto(333.77050781,452.57421875)(334.29003906,452.41015625)(334.58300781,452.28125)
\curveto(334.87988281,452.15625)(335.11230469,451.97265625)(335.28027344,451.73046875)
\curveto(335.44824219,451.48828125)(335.53222656,451.1875)(335.53222656,450.828125)
\curveto(335.53222656,450.4765625)(335.42871094,450.14453125)(335.22167969,449.83203125)
\curveto(335.01855469,449.5234375)(334.72363281,449.28320312)(334.33691406,449.11132812)
\curveto(333.95019531,448.94335938)(333.51269531,448.859375)(333.02441406,448.859375)
\curveto(332.21582031,448.859375)(331.59863281,449.02734375)(331.17285156,449.36328125)
\curveto(330.75097656,449.69921875)(330.48144531,450.19726562)(330.36425781,450.85742188)
\closepath
}
}
{
\newrgbcolor{curcolor}{0 0 0}
\pscustom[linestyle=none,fillstyle=solid,fillcolor=curcolor]
{
\newpath
\moveto(339.86816406,451.75976562)
\lineto(340.94042969,451.85351562)
\curveto(340.99121094,451.42382812)(341.10839844,451.0703125)(341.29199219,450.79296875)
\curveto(341.47949219,450.51953125)(341.76855469,450.296875)(342.15917969,450.125)
\curveto(342.54980469,449.95703125)(342.98925781,449.87304688)(343.47753906,449.87304688)
\curveto(343.91113281,449.87304688)(344.29394531,449.9375)(344.62597656,450.06640625)
\curveto(344.95800781,450.1953125)(345.20410156,450.37109375)(345.36425781,450.59375)
\curveto(345.52832031,450.8203125)(345.61035156,451.06640625)(345.61035156,451.33203125)
\curveto(345.61035156,451.6015625)(345.53222656,451.8359375)(345.37597656,452.03515625)
\curveto(345.21972656,452.23828125)(344.96191406,452.40820312)(344.60253906,452.54492188)
\curveto(344.37207031,452.63476562)(343.86230469,452.7734375)(343.07324219,452.9609375)
\curveto(342.28417969,453.15234375)(341.73144531,453.33203125)(341.41503906,453.5)
\curveto(341.00488281,453.71484375)(340.69824219,453.98046875)(340.49511719,454.296875)
\curveto(340.29589844,454.6171875)(340.19628906,454.97460938)(340.19628906,455.36914062)
\curveto(340.19628906,455.80273438)(340.31933594,456.20703125)(340.56542969,456.58203125)
\curveto(340.81152344,456.9609375)(341.17089844,457.24804688)(341.64355469,457.44335938)
\curveto(342.11621094,457.63867188)(342.64160156,457.73632812)(343.21972656,457.73632812)
\curveto(343.85644531,457.73632812)(344.41699219,457.6328125)(344.90136719,457.42578125)
\curveto(345.38964844,457.22265625)(345.76464844,456.921875)(346.02636719,456.5234375)
\curveto(346.28808594,456.125)(346.42871094,455.67382812)(346.44824219,455.16992188)
\lineto(345.35839844,455.08789062)
\curveto(345.29980469,455.63085938)(345.10058594,456.04101562)(344.76074219,456.31835938)
\curveto(344.42480469,456.59570312)(343.92675781,456.734375)(343.26660156,456.734375)
\curveto(342.57910156,456.734375)(342.07714844,456.60742188)(341.76074219,456.35351562)
\curveto(341.44824219,456.10351562)(341.29199219,455.80078125)(341.29199219,455.4453125)
\curveto(341.29199219,455.13671875)(341.40332031,454.8828125)(341.62597656,454.68359375)
\curveto(341.84472656,454.484375)(342.41503906,454.27929688)(343.33691406,454.06835938)
\curveto(344.26269531,453.86132812)(344.89746094,453.6796875)(345.24121094,453.5234375)
\curveto(345.74121094,453.29296875)(346.11035156,453)(346.34863281,452.64453125)
\curveto(346.58691406,452.29296875)(346.70605469,451.88671875)(346.70605469,451.42578125)
\curveto(346.70605469,450.96875)(346.57519531,450.53710938)(346.31347656,450.13085938)
\curveto(346.05175781,449.72851562)(345.67480469,449.4140625)(345.18261719,449.1875)
\curveto(344.69433594,448.96484375)(344.14355469,448.85351562)(343.53027344,448.85351562)
\curveto(342.75292969,448.85351562)(342.10058594,448.96679688)(341.57324219,449.19335938)
\curveto(341.04980469,449.41992188)(340.63769531,449.75976562)(340.33691406,450.21289062)
\curveto(340.04003906,450.66992188)(339.88378906,451.18554688)(339.86816406,451.75976562)
\closepath
}
}
{
\newrgbcolor{curcolor}{0 0 0}
\pscustom[linestyle=none,fillstyle=solid,fillcolor=curcolor]
{
\newpath
\moveto(348.12402344,449)
\lineto(348.12402344,455.22265625)
\lineto(349.06738281,455.22265625)
\lineto(349.06738281,454.34960938)
\curveto(349.26269531,454.65429688)(349.52246094,454.8984375)(349.84667969,455.08203125)
\curveto(350.17089844,455.26953125)(350.54003906,455.36328125)(350.95410156,455.36328125)
\curveto(351.41503906,455.36328125)(351.79199219,455.26757812)(352.08496094,455.07617188)
\curveto(352.38183594,454.88476562)(352.59082031,454.6171875)(352.71191406,454.2734375)
\curveto(353.20410156,455)(353.84472656,455.36328125)(354.63378906,455.36328125)
\curveto(355.25097656,455.36328125)(355.72558594,455.19140625)(356.05761719,454.84765625)
\curveto(356.38964844,454.5078125)(356.55566406,453.98242188)(356.55566406,453.27148438)
\lineto(356.55566406,449)
\lineto(355.50683594,449)
\lineto(355.50683594,452.91992188)
\curveto(355.50683594,453.34179688)(355.47167969,453.64453125)(355.40136719,453.828125)
\curveto(355.33496094,454.015625)(355.21191406,454.16601562)(355.03222656,454.27929688)
\curveto(354.85253906,454.39257812)(354.64160156,454.44921875)(354.39941406,454.44921875)
\curveto(353.96191406,454.44921875)(353.59863281,454.30273438)(353.30957031,454.00976562)
\curveto(353.02050781,453.72070312)(352.87597656,453.25585938)(352.87597656,452.61523438)
\lineto(352.87597656,449)
\lineto(351.82128906,449)
\lineto(351.82128906,453.04296875)
\curveto(351.82128906,453.51171875)(351.73535156,453.86328125)(351.56347656,454.09765625)
\curveto(351.39160156,454.33203125)(351.11035156,454.44921875)(350.71972656,454.44921875)
\curveto(350.42285156,454.44921875)(350.14746094,454.37109375)(349.89355469,454.21484375)
\curveto(349.64355469,454.05859375)(349.46191406,453.83007812)(349.34863281,453.52929688)
\curveto(349.23535156,453.22851562)(349.17871094,452.79492188)(349.17871094,452.22851562)
\lineto(349.17871094,449)
\closepath
}
}
{
\newrgbcolor{curcolor}{0 0 0}
\pscustom[linestyle=none,fillstyle=solid,fillcolor=curcolor]
{
\newpath
\moveto(362.18066406,449.76757812)
\curveto(361.79003906,449.43554688)(361.41308594,449.20117188)(361.04980469,449.06445312)
\curveto(360.69042969,448.92773438)(360.30371094,448.859375)(359.88964844,448.859375)
\curveto(359.20605469,448.859375)(358.68066406,449.02539062)(358.31347656,449.35742188)
\curveto(357.94628906,449.69335938)(357.76269531,450.12109375)(357.76269531,450.640625)
\curveto(357.76269531,450.9453125)(357.83105469,451.22265625)(357.96777344,451.47265625)
\curveto(358.10839844,451.7265625)(358.29003906,451.9296875)(358.51269531,452.08203125)
\curveto(358.73925781,452.234375)(358.99316406,452.34960938)(359.27441406,452.42773438)
\curveto(359.48144531,452.48242188)(359.79394531,452.53515625)(360.21191406,452.5859375)
\curveto(361.06347656,452.6875)(361.69042969,452.80859375)(362.09277344,452.94921875)
\curveto(362.09667969,453.09375)(362.09863281,453.18554688)(362.09863281,453.22460938)
\curveto(362.09863281,453.65429688)(361.99902344,453.95703125)(361.79980469,454.1328125)
\curveto(361.53027344,454.37109375)(361.12988281,454.49023438)(360.59863281,454.49023438)
\curveto(360.10253906,454.49023438)(359.73535156,454.40234375)(359.49707031,454.2265625)
\curveto(359.26269531,454.0546875)(359.08886719,453.74804688)(358.97558594,453.30664062)
\lineto(357.94433594,453.44726562)
\curveto(358.03808594,453.88867188)(358.19238281,454.24414062)(358.40722656,454.51367188)
\curveto(358.62207031,454.78710938)(358.93261719,454.99609375)(359.33886719,455.140625)
\curveto(359.74511719,455.2890625)(360.21582031,455.36328125)(360.75097656,455.36328125)
\curveto(361.28222656,455.36328125)(361.71386719,455.30078125)(362.04589844,455.17578125)
\curveto(362.37792969,455.05078125)(362.62207031,454.89257812)(362.77832031,454.70117188)
\curveto(362.93457031,454.51367188)(363.04394531,454.27539062)(363.10644531,453.98632812)
\curveto(363.14160156,453.80664062)(363.15917969,453.48242188)(363.15917969,453.01367188)
\lineto(363.15917969,451.60742188)
\curveto(363.15917969,450.62695312)(363.18066406,450.00585938)(363.22363281,449.74414062)
\curveto(363.27050781,449.48632812)(363.36035156,449.23828125)(363.49316406,449)
\lineto(362.39160156,449)
\curveto(362.28222656,449.21875)(362.21191406,449.47460938)(362.18066406,449.76757812)
\closepath
\moveto(362.09277344,452.12304688)
\curveto(361.70996094,451.96679688)(361.13574219,451.83398438)(360.37011719,451.72460938)
\curveto(359.93652344,451.66210938)(359.62988281,451.59179688)(359.45019531,451.51367188)
\curveto(359.27050781,451.43554688)(359.13183594,451.3203125)(359.03417969,451.16796875)
\curveto(358.93652344,451.01953125)(358.88769531,450.85351562)(358.88769531,450.66992188)
\curveto(358.88769531,450.38867188)(358.99316406,450.15429688)(359.20410156,449.96679688)
\curveto(359.41894531,449.77929688)(359.73144531,449.68554688)(360.14160156,449.68554688)
\curveto(360.54785156,449.68554688)(360.90917969,449.7734375)(361.22558594,449.94921875)
\curveto(361.54199219,450.12890625)(361.77441406,450.37304688)(361.92285156,450.68164062)
\curveto(362.03613281,450.91992188)(362.09277344,451.27148438)(362.09277344,451.73632812)
\closepath
}
}
{
\newrgbcolor{curcolor}{0 0 0}
\pscustom[linestyle=none,fillstyle=solid,fillcolor=curcolor]
{
\newpath
\moveto(364.77050781,449)
\lineto(364.77050781,457.58984375)
\lineto(365.82519531,457.58984375)
\lineto(365.82519531,449)
\closepath
}
}
{
\newrgbcolor{curcolor}{0 0 0}
\pscustom[linestyle=none,fillstyle=solid,fillcolor=curcolor]
{
\newpath
\moveto(367.43652344,449)
\lineto(367.43652344,457.58984375)
\lineto(368.49121094,457.58984375)
\lineto(368.49121094,449)
\closepath
}
}
{
\newrgbcolor{curcolor}{0 0 0}
\pscustom[linestyle=none,fillstyle=solid,fillcolor=curcolor]
{
\newpath
\moveto(373.59472656,449)
\lineto(373.59472656,457.58984375)
\lineto(376.83496094,457.58984375)
\curveto(377.40527344,457.58984375)(377.84082031,457.5625)(378.14160156,457.5078125)
\curveto(378.56347656,457.4375)(378.91699219,457.30273438)(379.20214844,457.10351562)
\curveto(379.48730469,456.90820312)(379.71582031,456.6328125)(379.88769531,456.27734375)
\curveto(380.06347656,455.921875)(380.15136719,455.53125)(380.15136719,455.10546875)
\curveto(380.15136719,454.375)(379.91894531,453.75585938)(379.45410156,453.24804688)
\curveto(378.98925781,452.74414062)(378.14941406,452.4921875)(376.93457031,452.4921875)
\lineto(374.73144531,452.4921875)
\lineto(374.73144531,449)
\closepath
\moveto(374.73144531,453.50585938)
\lineto(376.95214844,453.50585938)
\curveto(377.68652344,453.50585938)(378.20800781,453.64257812)(378.51660156,453.91601562)
\curveto(378.82519531,454.18945312)(378.97949219,454.57421875)(378.97949219,455.0703125)
\curveto(378.97949219,455.4296875)(378.88769531,455.73632812)(378.70410156,455.99023438)
\curveto(378.52441406,456.24804688)(378.28613281,456.41796875)(377.98925781,456.5)
\curveto(377.79785156,456.55078125)(377.44433594,456.57617188)(376.92871094,456.57617188)
\lineto(374.73144531,456.57617188)
\closepath
}
}
{
\newrgbcolor{curcolor}{0 0 0}
\pscustom[linestyle=none,fillstyle=solid,fillcolor=curcolor]
{
\newpath
\moveto(381.45214844,449)
\lineto(381.45214844,455.22265625)
\lineto(382.40136719,455.22265625)
\lineto(382.40136719,454.27929688)
\curveto(382.64355469,454.72070312)(382.86621094,455.01171875)(383.06933594,455.15234375)
\curveto(383.27636719,455.29296875)(383.50292969,455.36328125)(383.74902344,455.36328125)
\curveto(384.10449219,455.36328125)(384.46582031,455.25)(384.83300781,455.0234375)
\lineto(384.46972656,454.04492188)
\curveto(384.21191406,454.19726562)(383.95410156,454.2734375)(383.69628906,454.2734375)
\curveto(383.46582031,454.2734375)(383.25878906,454.203125)(383.07519531,454.0625)
\curveto(382.89160156,453.92578125)(382.76074219,453.734375)(382.68261719,453.48828125)
\curveto(382.56542969,453.11328125)(382.50683594,452.703125)(382.50683594,452.2578125)
\lineto(382.50683594,449)
\closepath
}
}
{
\newrgbcolor{curcolor}{0 0 0}
\pscustom[linestyle=none,fillstyle=solid,fillcolor=curcolor]
{
\newpath
\moveto(385.06738281,452.11132812)
\curveto(385.06738281,453.26367188)(385.38769531,454.1171875)(386.02832031,454.671875)
\curveto(386.56347656,455.1328125)(387.21582031,455.36328125)(387.98535156,455.36328125)
\curveto(388.84082031,455.36328125)(389.54003906,455.08203125)(390.08300781,454.51953125)
\curveto(390.62597656,453.9609375)(390.89746094,453.1875)(390.89746094,452.19921875)
\curveto(390.89746094,451.3984375)(390.77636719,450.76757812)(390.53417969,450.30664062)
\curveto(390.29589844,449.84960938)(389.94628906,449.49414062)(389.48535156,449.24023438)
\curveto(389.02832031,448.98632812)(388.52832031,448.859375)(387.98535156,448.859375)
\curveto(387.11425781,448.859375)(386.40917969,449.13867188)(385.87011719,449.69726562)
\curveto(385.33496094,450.25585938)(385.06738281,451.06054688)(385.06738281,452.11132812)
\closepath
\moveto(386.15136719,452.11132812)
\curveto(386.15136719,451.31445312)(386.32519531,450.71679688)(386.67285156,450.31835938)
\curveto(387.02050781,449.92382812)(387.45800781,449.7265625)(387.98535156,449.7265625)
\curveto(388.50878906,449.7265625)(388.94433594,449.92578125)(389.29199219,450.32421875)
\curveto(389.63964844,450.72265625)(389.81347656,451.33007812)(389.81347656,452.14648438)
\curveto(389.81347656,452.91601562)(389.63769531,453.49804688)(389.28613281,453.89257812)
\curveto(388.93847656,454.29101562)(388.50488281,454.49023438)(387.98535156,454.49023438)
\curveto(387.45800781,454.49023438)(387.02050781,454.29296875)(386.67285156,453.8984375)
\curveto(386.32519531,453.50390625)(386.15136719,452.90820312)(386.15136719,452.11132812)
\closepath
}
}
{
\newrgbcolor{curcolor}{0 0 0}
\pscustom[linestyle=none,fillstyle=solid,fillcolor=curcolor]
{
\newpath
\moveto(396.19433594,451.27929688)
\lineto(397.23144531,451.14453125)
\curveto(397.11816406,450.4296875)(396.82714844,449.86914062)(396.35839844,449.46289062)
\curveto(395.89355469,449.06054688)(395.32128906,448.859375)(394.64160156,448.859375)
\curveto(393.79003906,448.859375)(393.10449219,449.13671875)(392.58496094,449.69140625)
\curveto(392.06933594,450.25)(391.81152344,451.04882812)(391.81152344,452.08789062)
\curveto(391.81152344,452.75976562)(391.92285156,453.34765625)(392.14550781,453.8515625)
\curveto(392.36816406,454.35546875)(392.70605469,454.73242188)(393.15917969,454.98242188)
\curveto(393.61621094,455.23632812)(394.11230469,455.36328125)(394.64746094,455.36328125)
\curveto(395.32324219,455.36328125)(395.87597656,455.19140625)(396.30566406,454.84765625)
\curveto(396.73535156,454.5078125)(397.01074219,454.0234375)(397.13183594,453.39453125)
\lineto(396.10644531,453.23632812)
\curveto(396.00878906,453.65429688)(395.83496094,453.96875)(395.58496094,454.1796875)
\curveto(395.33886719,454.390625)(395.04003906,454.49609375)(394.68847656,454.49609375)
\curveto(394.15722656,454.49609375)(393.72558594,454.3046875)(393.39355469,453.921875)
\curveto(393.06152344,453.54296875)(392.89550781,452.94140625)(392.89550781,452.1171875)
\curveto(392.89550781,451.28125)(393.05566406,450.67382812)(393.37597656,450.29492188)
\curveto(393.69628906,449.91601562)(394.11425781,449.7265625)(394.62988281,449.7265625)
\curveto(395.04394531,449.7265625)(395.38964844,449.85351562)(395.66699219,450.10742188)
\curveto(395.94433594,450.36132812)(396.12011719,450.75195312)(396.19433594,451.27929688)
\closepath
}
}
{
\newrgbcolor{curcolor}{0 0 0}
\pscustom[linestyle=none,fillstyle=solid,fillcolor=curcolor]
{
\newpath
\moveto(402.39355469,451.00390625)
\lineto(403.48339844,450.86914062)
\curveto(403.31152344,450.23242188)(402.99316406,449.73828125)(402.52832031,449.38671875)
\curveto(402.06347656,449.03515625)(401.46972656,448.859375)(400.74707031,448.859375)
\curveto(399.83691406,448.859375)(399.11425781,449.13867188)(398.57910156,449.69726562)
\curveto(398.04785156,450.25976562)(397.78222656,451.046875)(397.78222656,452.05859375)
\curveto(397.78222656,453.10546875)(398.05175781,453.91796875)(398.59082031,454.49609375)
\curveto(399.12988281,455.07421875)(399.82910156,455.36328125)(400.68847656,455.36328125)
\curveto(401.52050781,455.36328125)(402.20019531,455.08007812)(402.72753906,454.51367188)
\curveto(403.25488281,453.94726562)(403.51855469,453.15039062)(403.51855469,452.12304688)
\curveto(403.51855469,452.06054688)(403.51660156,451.96679688)(403.51269531,451.84179688)
\lineto(398.87207031,451.84179688)
\curveto(398.91113281,451.15820312)(399.10449219,450.63476562)(399.45214844,450.27148438)
\curveto(399.79980469,449.90820312)(400.23339844,449.7265625)(400.75292969,449.7265625)
\curveto(401.13964844,449.7265625)(401.46972656,449.828125)(401.74316406,450.03125)
\curveto(402.01660156,450.234375)(402.23339844,450.55859375)(402.39355469,451.00390625)
\closepath
\moveto(398.93066406,452.70898438)
\lineto(402.40527344,452.70898438)
\curveto(402.35839844,453.23242188)(402.22558594,453.625)(402.00683594,453.88671875)
\curveto(401.67089844,454.29296875)(401.23535156,454.49609375)(400.70019531,454.49609375)
\curveto(400.21582031,454.49609375)(399.80761719,454.33398438)(399.47558594,454.00976562)
\curveto(399.14746094,453.68554688)(398.96582031,453.25195312)(398.93066406,452.70898438)
\closepath
}
}
{
\newrgbcolor{curcolor}{0 0 0}
\pscustom[linestyle=none,fillstyle=solid,fillcolor=curcolor]
{
\newpath
\moveto(404.38574219,450.85742188)
\lineto(405.42871094,451.02148438)
\curveto(405.48730469,450.60351562)(405.64941406,450.28320312)(405.91503906,450.06054688)
\curveto(406.18457031,449.83789062)(406.55957031,449.7265625)(407.04003906,449.7265625)
\curveto(407.52441406,449.7265625)(407.88378906,449.82421875)(408.11816406,450.01953125)
\curveto(408.35253906,450.21875)(408.46972656,450.45117188)(408.46972656,450.71679688)
\curveto(408.46972656,450.95507812)(408.36621094,451.14257812)(408.15917969,451.27929688)
\curveto(408.01464844,451.37304688)(407.65527344,451.4921875)(407.08105469,451.63671875)
\curveto(406.30761719,451.83203125)(405.77050781,452)(405.46972656,452.140625)
\curveto(405.17285156,452.28515625)(404.94628906,452.48242188)(404.79003906,452.73242188)
\curveto(404.63769531,452.98632812)(404.56152344,453.265625)(404.56152344,453.5703125)
\curveto(404.56152344,453.84765625)(404.62402344,454.10351562)(404.74902344,454.33789062)
\curveto(404.87792969,454.57617188)(405.05175781,454.7734375)(405.27050781,454.9296875)
\curveto(405.43457031,455.05078125)(405.65722656,455.15234375)(405.93847656,455.234375)
\curveto(406.22363281,455.3203125)(406.52832031,455.36328125)(406.85253906,455.36328125)
\curveto(407.34082031,455.36328125)(407.76855469,455.29296875)(408.13574219,455.15234375)
\curveto(408.50683594,455.01171875)(408.78027344,454.8203125)(408.95605469,454.578125)
\curveto(409.13183594,454.33984375)(409.25292969,454.01953125)(409.31933594,453.6171875)
\lineto(408.28808594,453.4765625)
\curveto(408.24121094,453.796875)(408.10449219,454.046875)(407.87792969,454.2265625)
\curveto(407.65527344,454.40625)(407.33886719,454.49609375)(406.92871094,454.49609375)
\curveto(406.44433594,454.49609375)(406.09863281,454.41601562)(405.89160156,454.25585938)
\curveto(405.68457031,454.09570312)(405.58105469,453.90820312)(405.58105469,453.69335938)
\curveto(405.58105469,453.55664062)(405.62402344,453.43359375)(405.70996094,453.32421875)
\curveto(405.79589844,453.2109375)(405.93066406,453.1171875)(406.11425781,453.04296875)
\curveto(406.21972656,453.00390625)(406.53027344,452.9140625)(407.04589844,452.7734375)
\curveto(407.79199219,452.57421875)(408.31152344,452.41015625)(408.60449219,452.28125)
\curveto(408.90136719,452.15625)(409.13378906,451.97265625)(409.30175781,451.73046875)
\curveto(409.46972656,451.48828125)(409.55371094,451.1875)(409.55371094,450.828125)
\curveto(409.55371094,450.4765625)(409.45019531,450.14453125)(409.24316406,449.83203125)
\curveto(409.04003906,449.5234375)(408.74511719,449.28320312)(408.35839844,449.11132812)
\curveto(407.97167969,448.94335938)(407.53417969,448.859375)(407.04589844,448.859375)
\curveto(406.23730469,448.859375)(405.62011719,449.02734375)(405.19433594,449.36328125)
\curveto(404.77246094,449.69921875)(404.50292969,450.19726562)(404.38574219,450.85742188)
\closepath
}
}
{
\newrgbcolor{curcolor}{0 0 0}
\pscustom[linestyle=none,fillstyle=solid,fillcolor=curcolor]
{
\newpath
\moveto(410.38574219,450.85742188)
\lineto(411.42871094,451.02148438)
\curveto(411.48730469,450.60351562)(411.64941406,450.28320312)(411.91503906,450.06054688)
\curveto(412.18457031,449.83789062)(412.55957031,449.7265625)(413.04003906,449.7265625)
\curveto(413.52441406,449.7265625)(413.88378906,449.82421875)(414.11816406,450.01953125)
\curveto(414.35253906,450.21875)(414.46972656,450.45117188)(414.46972656,450.71679688)
\curveto(414.46972656,450.95507812)(414.36621094,451.14257812)(414.15917969,451.27929688)
\curveto(414.01464844,451.37304688)(413.65527344,451.4921875)(413.08105469,451.63671875)
\curveto(412.30761719,451.83203125)(411.77050781,452)(411.46972656,452.140625)
\curveto(411.17285156,452.28515625)(410.94628906,452.48242188)(410.79003906,452.73242188)
\curveto(410.63769531,452.98632812)(410.56152344,453.265625)(410.56152344,453.5703125)
\curveto(410.56152344,453.84765625)(410.62402344,454.10351562)(410.74902344,454.33789062)
\curveto(410.87792969,454.57617188)(411.05175781,454.7734375)(411.27050781,454.9296875)
\curveto(411.43457031,455.05078125)(411.65722656,455.15234375)(411.93847656,455.234375)
\curveto(412.22363281,455.3203125)(412.52832031,455.36328125)(412.85253906,455.36328125)
\curveto(413.34082031,455.36328125)(413.76855469,455.29296875)(414.13574219,455.15234375)
\curveto(414.50683594,455.01171875)(414.78027344,454.8203125)(414.95605469,454.578125)
\curveto(415.13183594,454.33984375)(415.25292969,454.01953125)(415.31933594,453.6171875)
\lineto(414.28808594,453.4765625)
\curveto(414.24121094,453.796875)(414.10449219,454.046875)(413.87792969,454.2265625)
\curveto(413.65527344,454.40625)(413.33886719,454.49609375)(412.92871094,454.49609375)
\curveto(412.44433594,454.49609375)(412.09863281,454.41601562)(411.89160156,454.25585938)
\curveto(411.68457031,454.09570312)(411.58105469,453.90820312)(411.58105469,453.69335938)
\curveto(411.58105469,453.55664062)(411.62402344,453.43359375)(411.70996094,453.32421875)
\curveto(411.79589844,453.2109375)(411.93066406,453.1171875)(412.11425781,453.04296875)
\curveto(412.21972656,453.00390625)(412.53027344,452.9140625)(413.04589844,452.7734375)
\curveto(413.79199219,452.57421875)(414.31152344,452.41015625)(414.60449219,452.28125)
\curveto(414.90136719,452.15625)(415.13378906,451.97265625)(415.30175781,451.73046875)
\curveto(415.46972656,451.48828125)(415.55371094,451.1875)(415.55371094,450.828125)
\curveto(415.55371094,450.4765625)(415.45019531,450.14453125)(415.24316406,449.83203125)
\curveto(415.04003906,449.5234375)(414.74511719,449.28320312)(414.35839844,449.11132812)
\curveto(413.97167969,448.94335938)(413.53417969,448.859375)(413.04589844,448.859375)
\curveto(412.23730469,448.859375)(411.62011719,449.02734375)(411.19433594,449.36328125)
\curveto(410.77246094,449.69921875)(410.50292969,450.19726562)(410.38574219,450.85742188)
\closepath
}
}
{
\newrgbcolor{curcolor}{0 0 0}
\pscustom[linestyle=none,fillstyle=solid,fillcolor=curcolor]
{
\newpath
\moveto(422.15722656,446.47460938)
\curveto(421.57519531,447.20898438)(421.08300781,448.06835938)(420.68066406,449.05273438)
\curveto(420.27832031,450.03710938)(420.07714844,451.05664062)(420.07714844,452.11132812)
\curveto(420.07714844,453.04101562)(420.22753906,453.93164062)(420.52832031,454.78320312)
\curveto(420.87988281,455.77148438)(421.42285156,456.75585938)(422.15722656,457.73632812)
\lineto(422.91308594,457.73632812)
\curveto(422.44042969,456.92382812)(422.12792969,456.34375)(421.97558594,455.99609375)
\curveto(421.73730469,455.45703125)(421.54980469,454.89453125)(421.41308594,454.30859375)
\curveto(421.24511719,453.578125)(421.16113281,452.84375)(421.16113281,452.10546875)
\curveto(421.16113281,450.2265625)(421.74511719,448.34960938)(422.91308594,446.47460938)
\closepath
}
}
{
\newrgbcolor{curcolor}{0 0 0}
\pscustom[linestyle=none,fillstyle=solid,fillcolor=curcolor]
{
\newpath
\moveto(424.27246094,449)
\lineto(424.27246094,457.58984375)
\lineto(427.23144531,457.58984375)
\curveto(427.89941406,457.58984375)(428.40917969,457.54882812)(428.76074219,457.46679688)
\curveto(429.25292969,457.35351562)(429.67285156,457.1484375)(430.02050781,456.8515625)
\curveto(430.47363281,456.46875)(430.81152344,455.97851562)(431.03417969,455.38085938)
\curveto(431.26074219,454.78710938)(431.37402344,454.10742188)(431.37402344,453.34179688)
\curveto(431.37402344,452.68945312)(431.29785156,452.11132812)(431.14550781,451.60742188)
\curveto(430.99316406,451.10351562)(430.79785156,450.68554688)(430.55957031,450.35351562)
\curveto(430.32128906,450.02539062)(430.05957031,449.765625)(429.77441406,449.57421875)
\curveto(429.49316406,449.38671875)(429.15136719,449.24414062)(428.74902344,449.14648438)
\curveto(428.35058594,449.04882812)(427.89160156,449)(427.37207031,449)
\closepath
\moveto(425.40917969,450.01367188)
\lineto(427.24316406,450.01367188)
\curveto(427.80957031,450.01367188)(428.25292969,450.06640625)(428.57324219,450.171875)
\curveto(428.89746094,450.27734375)(429.15527344,450.42578125)(429.34667969,450.6171875)
\curveto(429.61621094,450.88671875)(429.82519531,451.24804688)(429.97363281,451.70117188)
\curveto(430.12597656,452.15820312)(430.20214844,452.7109375)(430.20214844,453.359375)
\curveto(430.20214844,454.2578125)(430.05371094,454.94726562)(429.75683594,455.42773438)
\curveto(429.46386719,455.91210938)(429.10644531,456.23632812)(428.68457031,456.40039062)
\curveto(428.37988281,456.51757812)(427.88964844,456.57617188)(427.21386719,456.57617188)
\lineto(425.40917969,456.57617188)
\closepath
}
}
{
\newrgbcolor{curcolor}{0 0 0}
\pscustom[linestyle=none,fillstyle=solid,fillcolor=curcolor]
{
\newpath
\moveto(436.86425781,449.76757812)
\curveto(436.47363281,449.43554688)(436.09667969,449.20117188)(435.73339844,449.06445312)
\curveto(435.37402344,448.92773438)(434.98730469,448.859375)(434.57324219,448.859375)
\curveto(433.88964844,448.859375)(433.36425781,449.02539062)(432.99707031,449.35742188)
\curveto(432.62988281,449.69335938)(432.44628906,450.12109375)(432.44628906,450.640625)
\curveto(432.44628906,450.9453125)(432.51464844,451.22265625)(432.65136719,451.47265625)
\curveto(432.79199219,451.7265625)(432.97363281,451.9296875)(433.19628906,452.08203125)
\curveto(433.42285156,452.234375)(433.67675781,452.34960938)(433.95800781,452.42773438)
\curveto(434.16503906,452.48242188)(434.47753906,452.53515625)(434.89550781,452.5859375)
\curveto(435.74707031,452.6875)(436.37402344,452.80859375)(436.77636719,452.94921875)
\curveto(436.78027344,453.09375)(436.78222656,453.18554688)(436.78222656,453.22460938)
\curveto(436.78222656,453.65429688)(436.68261719,453.95703125)(436.48339844,454.1328125)
\curveto(436.21386719,454.37109375)(435.81347656,454.49023438)(435.28222656,454.49023438)
\curveto(434.78613281,454.49023438)(434.41894531,454.40234375)(434.18066406,454.2265625)
\curveto(433.94628906,454.0546875)(433.77246094,453.74804688)(433.65917969,453.30664062)
\lineto(432.62792969,453.44726562)
\curveto(432.72167969,453.88867188)(432.87597656,454.24414062)(433.09082031,454.51367188)
\curveto(433.30566406,454.78710938)(433.61621094,454.99609375)(434.02246094,455.140625)
\curveto(434.42871094,455.2890625)(434.89941406,455.36328125)(435.43457031,455.36328125)
\curveto(435.96582031,455.36328125)(436.39746094,455.30078125)(436.72949219,455.17578125)
\curveto(437.06152344,455.05078125)(437.30566406,454.89257812)(437.46191406,454.70117188)
\curveto(437.61816406,454.51367188)(437.72753906,454.27539062)(437.79003906,453.98632812)
\curveto(437.82519531,453.80664062)(437.84277344,453.48242188)(437.84277344,453.01367188)
\lineto(437.84277344,451.60742188)
\curveto(437.84277344,450.62695312)(437.86425781,450.00585938)(437.90722656,449.74414062)
\curveto(437.95410156,449.48632812)(438.04394531,449.23828125)(438.17675781,449)
\lineto(437.07519531,449)
\curveto(436.96582031,449.21875)(436.89550781,449.47460938)(436.86425781,449.76757812)
\closepath
\moveto(436.77636719,452.12304688)
\curveto(436.39355469,451.96679688)(435.81933594,451.83398438)(435.05371094,451.72460938)
\curveto(434.62011719,451.66210938)(434.31347656,451.59179688)(434.13378906,451.51367188)
\curveto(433.95410156,451.43554688)(433.81542969,451.3203125)(433.71777344,451.16796875)
\curveto(433.62011719,451.01953125)(433.57128906,450.85351562)(433.57128906,450.66992188)
\curveto(433.57128906,450.38867188)(433.67675781,450.15429688)(433.88769531,449.96679688)
\curveto(434.10253906,449.77929688)(434.41503906,449.68554688)(434.82519531,449.68554688)
\curveto(435.23144531,449.68554688)(435.59277344,449.7734375)(435.90917969,449.94921875)
\curveto(436.22558594,450.12890625)(436.45800781,450.37304688)(436.60644531,450.68164062)
\curveto(436.71972656,450.91992188)(436.77636719,451.27148438)(436.77636719,451.73632812)
\closepath
}
}
{
\newrgbcolor{curcolor}{0 0 0}
\pscustom[linestyle=none,fillstyle=solid,fillcolor=curcolor]
{
\newpath
\moveto(441.78027344,449.94335938)
\lineto(441.93261719,449.01171875)
\curveto(441.63574219,448.94921875)(441.37011719,448.91796875)(441.13574219,448.91796875)
\curveto(440.75292969,448.91796875)(440.45605469,448.97851562)(440.24511719,449.09960938)
\curveto(440.03417969,449.22070312)(439.88574219,449.37890625)(439.79980469,449.57421875)
\curveto(439.71386719,449.7734375)(439.67089844,450.18945312)(439.67089844,450.82226562)
\lineto(439.67089844,454.40234375)
\lineto(438.89746094,454.40234375)
\lineto(438.89746094,455.22265625)
\lineto(439.67089844,455.22265625)
\lineto(439.67089844,456.76367188)
\lineto(440.71972656,457.39648438)
\lineto(440.71972656,455.22265625)
\lineto(441.78027344,455.22265625)
\lineto(441.78027344,454.40234375)
\lineto(440.71972656,454.40234375)
\lineto(440.71972656,450.76367188)
\curveto(440.71972656,450.46289062)(440.73730469,450.26953125)(440.77246094,450.18359375)
\curveto(440.81152344,450.09765625)(440.87207031,450.02929688)(440.95410156,449.97851562)
\curveto(441.04003906,449.92773438)(441.16113281,449.90234375)(441.31738281,449.90234375)
\curveto(441.43457031,449.90234375)(441.58886719,449.91601562)(441.78027344,449.94335938)
\closepath
}
}
{
\newrgbcolor{curcolor}{0 0 0}
\pscustom[linestyle=none,fillstyle=solid,fillcolor=curcolor]
{
\newpath
\moveto(446.87207031,449.76757812)
\curveto(446.48144531,449.43554688)(446.10449219,449.20117188)(445.74121094,449.06445312)
\curveto(445.38183594,448.92773438)(444.99511719,448.859375)(444.58105469,448.859375)
\curveto(443.89746094,448.859375)(443.37207031,449.02539062)(443.00488281,449.35742188)
\curveto(442.63769531,449.69335938)(442.45410156,450.12109375)(442.45410156,450.640625)
\curveto(442.45410156,450.9453125)(442.52246094,451.22265625)(442.65917969,451.47265625)
\curveto(442.79980469,451.7265625)(442.98144531,451.9296875)(443.20410156,452.08203125)
\curveto(443.43066406,452.234375)(443.68457031,452.34960938)(443.96582031,452.42773438)
\curveto(444.17285156,452.48242188)(444.48535156,452.53515625)(444.90332031,452.5859375)
\curveto(445.75488281,452.6875)(446.38183594,452.80859375)(446.78417969,452.94921875)
\curveto(446.78808594,453.09375)(446.79003906,453.18554688)(446.79003906,453.22460938)
\curveto(446.79003906,453.65429688)(446.69042969,453.95703125)(446.49121094,454.1328125)
\curveto(446.22167969,454.37109375)(445.82128906,454.49023438)(445.29003906,454.49023438)
\curveto(444.79394531,454.49023438)(444.42675781,454.40234375)(444.18847656,454.2265625)
\curveto(443.95410156,454.0546875)(443.78027344,453.74804688)(443.66699219,453.30664062)
\lineto(442.63574219,453.44726562)
\curveto(442.72949219,453.88867188)(442.88378906,454.24414062)(443.09863281,454.51367188)
\curveto(443.31347656,454.78710938)(443.62402344,454.99609375)(444.03027344,455.140625)
\curveto(444.43652344,455.2890625)(444.90722656,455.36328125)(445.44238281,455.36328125)
\curveto(445.97363281,455.36328125)(446.40527344,455.30078125)(446.73730469,455.17578125)
\curveto(447.06933594,455.05078125)(447.31347656,454.89257812)(447.46972656,454.70117188)
\curveto(447.62597656,454.51367188)(447.73535156,454.27539062)(447.79785156,453.98632812)
\curveto(447.83300781,453.80664062)(447.85058594,453.48242188)(447.85058594,453.01367188)
\lineto(447.85058594,451.60742188)
\curveto(447.85058594,450.62695312)(447.87207031,450.00585938)(447.91503906,449.74414062)
\curveto(447.96191406,449.48632812)(448.05175781,449.23828125)(448.18457031,449)
\lineto(447.08300781,449)
\curveto(446.97363281,449.21875)(446.90332031,449.47460938)(446.87207031,449.76757812)
\closepath
\moveto(446.78417969,452.12304688)
\curveto(446.40136719,451.96679688)(445.82714844,451.83398438)(445.06152344,451.72460938)
\curveto(444.62792969,451.66210938)(444.32128906,451.59179688)(444.14160156,451.51367188)
\curveto(443.96191406,451.43554688)(443.82324219,451.3203125)(443.72558594,451.16796875)
\curveto(443.62792969,451.01953125)(443.57910156,450.85351562)(443.57910156,450.66992188)
\curveto(443.57910156,450.38867188)(443.68457031,450.15429688)(443.89550781,449.96679688)
\curveto(444.11035156,449.77929688)(444.42285156,449.68554688)(444.83300781,449.68554688)
\curveto(445.23925781,449.68554688)(445.60058594,449.7734375)(445.91699219,449.94921875)
\curveto(446.23339844,450.12890625)(446.46582031,450.37304688)(446.61425781,450.68164062)
\curveto(446.72753906,450.91992188)(446.78417969,451.27148438)(446.78417969,451.73632812)
\closepath
}
}
{
\newrgbcolor{curcolor}{0 0 0}
\pscustom[linestyle=none,fillstyle=solid,fillcolor=curcolor]
{
\newpath
\moveto(452.94238281,449)
\lineto(452.94238281,457.58984375)
\lineto(454.10839844,457.58984375)
\lineto(458.62011719,450.84570312)
\lineto(458.62011719,457.58984375)
\lineto(459.70996094,457.58984375)
\lineto(459.70996094,449)
\lineto(458.54394531,449)
\lineto(454.03222656,455.75)
\lineto(454.03222656,449)
\closepath
}
}
{
\newrgbcolor{curcolor}{0 0 0}
\pscustom[linestyle=none,fillstyle=solid,fillcolor=curcolor]
{
\newpath
\moveto(461.09277344,452.11132812)
\curveto(461.09277344,453.26367188)(461.41308594,454.1171875)(462.05371094,454.671875)
\curveto(462.58886719,455.1328125)(463.24121094,455.36328125)(464.01074219,455.36328125)
\curveto(464.86621094,455.36328125)(465.56542969,455.08203125)(466.10839844,454.51953125)
\curveto(466.65136719,453.9609375)(466.92285156,453.1875)(466.92285156,452.19921875)
\curveto(466.92285156,451.3984375)(466.80175781,450.76757812)(466.55957031,450.30664062)
\curveto(466.32128906,449.84960938)(465.97167969,449.49414062)(465.51074219,449.24023438)
\curveto(465.05371094,448.98632812)(464.55371094,448.859375)(464.01074219,448.859375)
\curveto(463.13964844,448.859375)(462.43457031,449.13867188)(461.89550781,449.69726562)
\curveto(461.36035156,450.25585938)(461.09277344,451.06054688)(461.09277344,452.11132812)
\closepath
\moveto(462.17675781,452.11132812)
\curveto(462.17675781,451.31445312)(462.35058594,450.71679688)(462.69824219,450.31835938)
\curveto(463.04589844,449.92382812)(463.48339844,449.7265625)(464.01074219,449.7265625)
\curveto(464.53417969,449.7265625)(464.96972656,449.92578125)(465.31738281,450.32421875)
\curveto(465.66503906,450.72265625)(465.83886719,451.33007812)(465.83886719,452.14648438)
\curveto(465.83886719,452.91601562)(465.66308594,453.49804688)(465.31152344,453.89257812)
\curveto(464.96386719,454.29101562)(464.53027344,454.49023438)(464.01074219,454.49023438)
\curveto(463.48339844,454.49023438)(463.04589844,454.29296875)(462.69824219,453.8984375)
\curveto(462.35058594,453.50390625)(462.17675781,452.90820312)(462.17675781,452.11132812)
\closepath
}
}
{
\newrgbcolor{curcolor}{0 0 0}
\pscustom[linestyle=none,fillstyle=solid,fillcolor=curcolor]
{
\newpath
\moveto(472.19628906,449)
\lineto(472.19628906,449.78515625)
\curveto(471.80175781,449.16796875)(471.22167969,448.859375)(470.45605469,448.859375)
\curveto(469.95996094,448.859375)(469.50292969,448.99609375)(469.08496094,449.26953125)
\curveto(468.67089844,449.54296875)(468.34863281,449.92382812)(468.11816406,450.41210938)
\curveto(467.89160156,450.90429688)(467.77832031,451.46875)(467.77832031,452.10546875)
\curveto(467.77832031,452.7265625)(467.88183594,453.2890625)(468.08886719,453.79296875)
\curveto(468.29589844,454.30078125)(468.60644531,454.68945312)(469.02050781,454.95898438)
\curveto(469.43457031,455.22851562)(469.89746094,455.36328125)(470.40917969,455.36328125)
\curveto(470.78417969,455.36328125)(471.11816406,455.28320312)(471.41113281,455.12304688)
\curveto(471.70410156,454.96679688)(471.94238281,454.76171875)(472.12597656,454.5078125)
\lineto(472.12597656,457.58984375)
\lineto(473.17480469,457.58984375)
\lineto(473.17480469,449)
\closepath
\moveto(468.86230469,452.10546875)
\curveto(468.86230469,451.30859375)(469.03027344,450.71289062)(469.36621094,450.31835938)
\curveto(469.70214844,449.92382812)(470.09863281,449.7265625)(470.55566406,449.7265625)
\curveto(471.01660156,449.7265625)(471.40722656,449.9140625)(471.72753906,450.2890625)
\curveto(472.05175781,450.66796875)(472.21386719,451.24414062)(472.21386719,452.01757812)
\curveto(472.21386719,452.86914062)(472.04980469,453.49414062)(471.72167969,453.89257812)
\curveto(471.39355469,454.29101562)(470.98925781,454.49023438)(470.50878906,454.49023438)
\curveto(470.04003906,454.49023438)(469.64746094,454.29882812)(469.33105469,453.91601562)
\curveto(469.01855469,453.53320312)(468.86230469,452.9296875)(468.86230469,452.10546875)
\closepath
}
}
{
\newrgbcolor{curcolor}{0 0 0}
\pscustom[linestyle=none,fillstyle=solid,fillcolor=curcolor]
{
\newpath
\moveto(479.09277344,451.00390625)
\lineto(480.18261719,450.86914062)
\curveto(480.01074219,450.23242188)(479.69238281,449.73828125)(479.22753906,449.38671875)
\curveto(478.76269531,449.03515625)(478.16894531,448.859375)(477.44628906,448.859375)
\curveto(476.53613281,448.859375)(475.81347656,449.13867188)(475.27832031,449.69726562)
\curveto(474.74707031,450.25976562)(474.48144531,451.046875)(474.48144531,452.05859375)
\curveto(474.48144531,453.10546875)(474.75097656,453.91796875)(475.29003906,454.49609375)
\curveto(475.82910156,455.07421875)(476.52832031,455.36328125)(477.38769531,455.36328125)
\curveto(478.21972656,455.36328125)(478.89941406,455.08007812)(479.42675781,454.51367188)
\curveto(479.95410156,453.94726562)(480.21777344,453.15039062)(480.21777344,452.12304688)
\curveto(480.21777344,452.06054688)(480.21582031,451.96679688)(480.21191406,451.84179688)
\lineto(475.57128906,451.84179688)
\curveto(475.61035156,451.15820312)(475.80371094,450.63476562)(476.15136719,450.27148438)
\curveto(476.49902344,449.90820312)(476.93261719,449.7265625)(477.45214844,449.7265625)
\curveto(477.83886719,449.7265625)(478.16894531,449.828125)(478.44238281,450.03125)
\curveto(478.71582031,450.234375)(478.93261719,450.55859375)(479.09277344,451.00390625)
\closepath
\moveto(475.62988281,452.70898438)
\lineto(479.10449219,452.70898438)
\curveto(479.05761719,453.23242188)(478.92480469,453.625)(478.70605469,453.88671875)
\curveto(478.37011719,454.29296875)(477.93457031,454.49609375)(477.39941406,454.49609375)
\curveto(476.91503906,454.49609375)(476.50683594,454.33398438)(476.17480469,454.00976562)
\curveto(475.84667969,453.68554688)(475.66503906,453.25195312)(475.62988281,452.70898438)
\closepath
}
}
{
\newrgbcolor{curcolor}{0 0 0}
\pscustom[linestyle=none,fillstyle=solid,fillcolor=curcolor]
{
\newpath
\moveto(484.54785156,453.23632812)
\curveto(484.54785156,454.25195312)(484.65136719,455.06835938)(484.85839844,455.68554688)
\curveto(485.06933594,456.30664062)(485.37988281,456.78515625)(485.79003906,457.12109375)
\curveto(486.20410156,457.45703125)(486.72363281,457.625)(487.34863281,457.625)
\curveto(487.80957031,457.625)(488.21386719,457.53125)(488.56152344,457.34375)
\curveto(488.90917969,457.16015625)(489.19628906,456.89257812)(489.42285156,456.54101562)
\curveto(489.64941406,456.19335938)(489.82714844,455.76757812)(489.95605469,455.26367188)
\curveto(490.08496094,454.76367188)(490.14941406,454.08789062)(490.14941406,453.23632812)
\curveto(490.14941406,452.22851562)(490.04589844,451.4140625)(489.83886719,450.79296875)
\curveto(489.63183594,450.17578125)(489.32128906,449.69726562)(488.90722656,449.35742188)
\curveto(488.49707031,449.02148438)(487.97753906,448.85351562)(487.34863281,448.85351562)
\curveto(486.52050781,448.85351562)(485.87011719,449.15039062)(485.39746094,449.74414062)
\curveto(484.83105469,450.45898438)(484.54785156,451.62304688)(484.54785156,453.23632812)
\closepath
\moveto(485.63183594,453.23632812)
\curveto(485.63183594,451.82617188)(485.79589844,450.88671875)(486.12402344,450.41796875)
\curveto(486.45605469,449.953125)(486.86425781,449.72070312)(487.34863281,449.72070312)
\curveto(487.83300781,449.72070312)(488.23925781,449.95507812)(488.56738281,450.42382812)
\curveto(488.89941406,450.89257812)(489.06542969,451.83007812)(489.06542969,453.23632812)
\curveto(489.06542969,454.65039062)(488.89941406,455.58984375)(488.56738281,456.0546875)
\curveto(488.23925781,456.51953125)(487.82910156,456.75195312)(487.33691406,456.75195312)
\curveto(486.85253906,456.75195312)(486.46582031,456.546875)(486.17675781,456.13671875)
\curveto(485.81347656,455.61328125)(485.63183594,454.64648438)(485.63183594,453.23632812)
\closepath
}
}
{
\newrgbcolor{curcolor}{0 0 0}
\pscustom[linestyle=none,fillstyle=solid,fillcolor=curcolor]
{
\newpath
\moveto(492.20605469,446.47460938)
\lineto(491.45019531,446.47460938)
\curveto(492.61816406,448.34960938)(493.20214844,450.2265625)(493.20214844,452.10546875)
\curveto(493.20214844,452.83984375)(493.11816406,453.56835938)(492.95019531,454.29101562)
\curveto(492.81738281,454.87695312)(492.63183594,455.43945312)(492.39355469,455.97851562)
\curveto(492.24121094,456.33007812)(491.92675781,456.91601562)(491.45019531,457.73632812)
\lineto(492.20605469,457.73632812)
\curveto(492.94042969,456.75585938)(493.48339844,455.77148438)(493.83496094,454.78320312)
\curveto(494.13574219,453.93164062)(494.28613281,453.04101562)(494.28613281,452.11132812)
\curveto(494.28613281,451.05664062)(494.08300781,450.03710938)(493.67675781,449.05273438)
\curveto(493.27441406,448.06835938)(492.78417969,447.20898438)(492.20605469,446.47460938)
\closepath
}
}
{
\newrgbcolor{curcolor}{0 0 0}
\pscustom[linestyle=none,fillstyle=solid,fillcolor=curcolor]
{
\newpath
\moveto(166.76992187,406.75976562)
\lineto(167.8421875,406.85351562)
\curveto(167.89296875,406.42382812)(168.01015625,406.0703125)(168.19375,405.79296875)
\curveto(168.38125,405.51953125)(168.6703125,405.296875)(169.0609375,405.125)
\curveto(169.4515625,404.95703125)(169.89101562,404.87304688)(170.37929687,404.87304688)
\curveto(170.81289062,404.87304688)(171.19570312,404.9375)(171.52773437,405.06640625)
\curveto(171.85976562,405.1953125)(172.10585937,405.37109375)(172.26601562,405.59375)
\curveto(172.43007812,405.8203125)(172.51210937,406.06640625)(172.51210937,406.33203125)
\curveto(172.51210937,406.6015625)(172.43398437,406.8359375)(172.27773437,407.03515625)
\curveto(172.12148437,407.23828125)(171.86367187,407.40820312)(171.50429687,407.54492188)
\curveto(171.27382812,407.63476562)(170.7640625,407.7734375)(169.975,407.9609375)
\curveto(169.1859375,408.15234375)(168.63320312,408.33203125)(168.31679687,408.5)
\curveto(167.90664062,408.71484375)(167.6,408.98046875)(167.396875,409.296875)
\curveto(167.19765625,409.6171875)(167.09804687,409.97460938)(167.09804687,410.36914062)
\curveto(167.09804687,410.80273438)(167.22109375,411.20703125)(167.4671875,411.58203125)
\curveto(167.71328125,411.9609375)(168.07265625,412.24804688)(168.5453125,412.44335938)
\curveto(169.01796875,412.63867188)(169.54335937,412.73632812)(170.12148437,412.73632812)
\curveto(170.75820312,412.73632812)(171.31875,412.6328125)(171.803125,412.42578125)
\curveto(172.29140625,412.22265625)(172.66640625,411.921875)(172.928125,411.5234375)
\curveto(173.18984375,411.125)(173.33046875,410.67382812)(173.35,410.16992188)
\lineto(172.26015625,410.08789062)
\curveto(172.2015625,410.63085938)(172.00234375,411.04101562)(171.6625,411.31835938)
\curveto(171.3265625,411.59570312)(170.82851562,411.734375)(170.16835937,411.734375)
\curveto(169.48085937,411.734375)(168.97890625,411.60742188)(168.6625,411.35351562)
\curveto(168.35,411.10351562)(168.19375,410.80078125)(168.19375,410.4453125)
\curveto(168.19375,410.13671875)(168.30507812,409.8828125)(168.52773437,409.68359375)
\curveto(168.74648437,409.484375)(169.31679687,409.27929688)(170.23867187,409.06835938)
\curveto(171.16445312,408.86132812)(171.79921875,408.6796875)(172.14296875,408.5234375)
\curveto(172.64296875,408.29296875)(173.01210937,408)(173.25039062,407.64453125)
\curveto(173.48867187,407.29296875)(173.6078125,406.88671875)(173.6078125,406.42578125)
\curveto(173.6078125,405.96875)(173.47695312,405.53710938)(173.21523437,405.13085938)
\curveto(172.95351562,404.72851562)(172.5765625,404.4140625)(172.084375,404.1875)
\curveto(171.59609375,403.96484375)(171.0453125,403.85351562)(170.43203125,403.85351562)
\curveto(169.6546875,403.85351562)(169.00234375,403.96679688)(168.475,404.19335938)
\curveto(167.9515625,404.41992188)(167.53945312,404.75976562)(167.23867187,405.21289062)
\curveto(166.94179687,405.66992188)(166.78554687,406.18554688)(166.76992187,406.75976562)
\closepath
}
}
{
\newrgbcolor{curcolor}{0 0 0}
\pscustom[linestyle=none,fillstyle=solid,fillcolor=curcolor]
{
\newpath
\moveto(179.08632812,404.76757812)
\curveto(178.69570312,404.43554688)(178.31875,404.20117188)(177.95546875,404.06445312)
\curveto(177.59609375,403.92773438)(177.209375,403.859375)(176.7953125,403.859375)
\curveto(176.11171875,403.859375)(175.58632812,404.02539062)(175.21914062,404.35742188)
\curveto(174.85195312,404.69335938)(174.66835937,405.12109375)(174.66835937,405.640625)
\curveto(174.66835937,405.9453125)(174.73671875,406.22265625)(174.8734375,406.47265625)
\curveto(175.0140625,406.7265625)(175.19570312,406.9296875)(175.41835937,407.08203125)
\curveto(175.64492187,407.234375)(175.89882812,407.34960938)(176.18007812,407.42773438)
\curveto(176.38710937,407.48242188)(176.69960937,407.53515625)(177.11757812,407.5859375)
\curveto(177.96914062,407.6875)(178.59609375,407.80859375)(178.9984375,407.94921875)
\curveto(179.00234375,408.09375)(179.00429687,408.18554688)(179.00429687,408.22460938)
\curveto(179.00429687,408.65429688)(178.9046875,408.95703125)(178.70546875,409.1328125)
\curveto(178.4359375,409.37109375)(178.03554687,409.49023438)(177.50429687,409.49023438)
\curveto(177.00820312,409.49023438)(176.64101562,409.40234375)(176.40273437,409.2265625)
\curveto(176.16835937,409.0546875)(175.99453125,408.74804688)(175.88125,408.30664062)
\lineto(174.85,408.44726562)
\curveto(174.94375,408.88867188)(175.09804687,409.24414062)(175.31289062,409.51367188)
\curveto(175.52773437,409.78710938)(175.83828125,409.99609375)(176.24453125,410.140625)
\curveto(176.65078125,410.2890625)(177.12148437,410.36328125)(177.65664062,410.36328125)
\curveto(178.18789062,410.36328125)(178.61953125,410.30078125)(178.9515625,410.17578125)
\curveto(179.28359375,410.05078125)(179.52773437,409.89257812)(179.68398437,409.70117188)
\curveto(179.84023437,409.51367188)(179.94960937,409.27539062)(180.01210937,408.98632812)
\curveto(180.04726562,408.80664062)(180.06484375,408.48242188)(180.06484375,408.01367188)
\lineto(180.06484375,406.60742188)
\curveto(180.06484375,405.62695312)(180.08632812,405.00585938)(180.12929687,404.74414062)
\curveto(180.17617187,404.48632812)(180.26601562,404.23828125)(180.39882812,404)
\lineto(179.29726562,404)
\curveto(179.18789062,404.21875)(179.11757812,404.47460938)(179.08632812,404.76757812)
\closepath
\moveto(178.9984375,407.12304688)
\curveto(178.615625,406.96679688)(178.04140625,406.83398438)(177.27578125,406.72460938)
\curveto(176.8421875,406.66210938)(176.53554687,406.59179688)(176.35585937,406.51367188)
\curveto(176.17617187,406.43554688)(176.0375,406.3203125)(175.93984375,406.16796875)
\curveto(175.8421875,406.01953125)(175.79335937,405.85351562)(175.79335937,405.66992188)
\curveto(175.79335937,405.38867188)(175.89882812,405.15429688)(176.10976562,404.96679688)
\curveto(176.32460937,404.77929688)(176.63710937,404.68554688)(177.04726562,404.68554688)
\curveto(177.45351562,404.68554688)(177.81484375,404.7734375)(178.13125,404.94921875)
\curveto(178.44765625,405.12890625)(178.68007812,405.37304688)(178.82851562,405.68164062)
\curveto(178.94179687,405.91992188)(178.9984375,406.27148438)(178.9984375,406.73632812)
\closepath
}
}
{
\newrgbcolor{curcolor}{0 0 0}
\pscustom[linestyle=none,fillstyle=solid,fillcolor=curcolor]
{
\newpath
\moveto(181.69960937,404)
\lineto(181.69960937,410.22265625)
\lineto(182.64296875,410.22265625)
\lineto(182.64296875,409.34960938)
\curveto(182.83828125,409.65429688)(183.09804687,409.8984375)(183.42226562,410.08203125)
\curveto(183.74648437,410.26953125)(184.115625,410.36328125)(184.5296875,410.36328125)
\curveto(184.990625,410.36328125)(185.36757812,410.26757812)(185.66054687,410.07617188)
\curveto(185.95742187,409.88476562)(186.16640625,409.6171875)(186.2875,409.2734375)
\curveto(186.7796875,410)(187.4203125,410.36328125)(188.209375,410.36328125)
\curveto(188.8265625,410.36328125)(189.30117187,410.19140625)(189.63320312,409.84765625)
\curveto(189.96523437,409.5078125)(190.13125,408.98242188)(190.13125,408.27148438)
\lineto(190.13125,404)
\lineto(189.08242187,404)
\lineto(189.08242187,407.91992188)
\curveto(189.08242187,408.34179688)(189.04726562,408.64453125)(188.97695312,408.828125)
\curveto(188.91054687,409.015625)(188.7875,409.16601562)(188.6078125,409.27929688)
\curveto(188.428125,409.39257812)(188.2171875,409.44921875)(187.975,409.44921875)
\curveto(187.5375,409.44921875)(187.17421875,409.30273438)(186.88515625,409.00976562)
\curveto(186.59609375,408.72070312)(186.4515625,408.25585938)(186.4515625,407.61523438)
\lineto(186.4515625,404)
\lineto(185.396875,404)
\lineto(185.396875,408.04296875)
\curveto(185.396875,408.51171875)(185.3109375,408.86328125)(185.1390625,409.09765625)
\curveto(184.9671875,409.33203125)(184.6859375,409.44921875)(184.2953125,409.44921875)
\curveto(183.9984375,409.44921875)(183.72304687,409.37109375)(183.46914062,409.21484375)
\curveto(183.21914062,409.05859375)(183.0375,408.83007812)(182.92421875,408.52929688)
\curveto(182.8109375,408.22851562)(182.75429687,407.79492188)(182.75429687,407.22851562)
\lineto(182.75429687,404)
\closepath
}
}
{
\newrgbcolor{curcolor}{0 0 0}
\pscustom[linestyle=none,fillstyle=solid,fillcolor=curcolor]
{
\newpath
\moveto(195.95546875,406.00390625)
\lineto(197.0453125,405.86914062)
\curveto(196.8734375,405.23242188)(196.55507812,404.73828125)(196.09023437,404.38671875)
\curveto(195.62539062,404.03515625)(195.03164062,403.859375)(194.30898437,403.859375)
\curveto(193.39882812,403.859375)(192.67617187,404.13867188)(192.14101562,404.69726562)
\curveto(191.60976562,405.25976562)(191.34414062,406.046875)(191.34414062,407.05859375)
\curveto(191.34414062,408.10546875)(191.61367187,408.91796875)(192.15273437,409.49609375)
\curveto(192.69179687,410.07421875)(193.39101562,410.36328125)(194.25039062,410.36328125)
\curveto(195.08242187,410.36328125)(195.76210937,410.08007812)(196.28945312,409.51367188)
\curveto(196.81679687,408.94726562)(197.08046875,408.15039062)(197.08046875,407.12304688)
\curveto(197.08046875,407.06054688)(197.07851562,406.96679688)(197.07460937,406.84179688)
\lineto(192.43398437,406.84179688)
\curveto(192.47304687,406.15820312)(192.66640625,405.63476562)(193.0140625,405.27148438)
\curveto(193.36171875,404.90820312)(193.7953125,404.7265625)(194.31484375,404.7265625)
\curveto(194.7015625,404.7265625)(195.03164062,404.828125)(195.30507812,405.03125)
\curveto(195.57851562,405.234375)(195.7953125,405.55859375)(195.95546875,406.00390625)
\closepath
\moveto(192.49257812,407.70898438)
\lineto(195.9671875,407.70898438)
\curveto(195.9203125,408.23242188)(195.7875,408.625)(195.56875,408.88671875)
\curveto(195.2328125,409.29296875)(194.79726562,409.49609375)(194.26210937,409.49609375)
\curveto(193.77773437,409.49609375)(193.36953125,409.33398438)(193.0375,409.00976562)
\curveto(192.709375,408.68554688)(192.52773437,408.25195312)(192.49257812,407.70898438)
\closepath
}
}
{
\newrgbcolor{curcolor}{0 0 0}
\pscustom[linestyle=none,fillstyle=solid,fillcolor=curcolor]
{
\newpath
\moveto(201.8265625,404)
\lineto(201.8265625,412.58984375)
\lineto(202.99257812,412.58984375)
\lineto(207.50429687,405.84570312)
\lineto(207.50429687,412.58984375)
\lineto(208.59414062,412.58984375)
\lineto(208.59414062,404)
\lineto(207.428125,404)
\lineto(202.91640625,410.75)
\lineto(202.91640625,404)
\closepath
}
}
{
\newrgbcolor{curcolor}{0 0 0}
\pscustom[linestyle=none,fillstyle=solid,fillcolor=curcolor]
{
\newpath
\moveto(209.97695312,407.11132812)
\curveto(209.97695312,408.26367188)(210.29726562,409.1171875)(210.93789062,409.671875)
\curveto(211.47304687,410.1328125)(212.12539062,410.36328125)(212.89492187,410.36328125)
\curveto(213.75039062,410.36328125)(214.44960937,410.08203125)(214.99257812,409.51953125)
\curveto(215.53554687,408.9609375)(215.80703125,408.1875)(215.80703125,407.19921875)
\curveto(215.80703125,406.3984375)(215.6859375,405.76757812)(215.44375,405.30664062)
\curveto(215.20546875,404.84960938)(214.85585937,404.49414062)(214.39492187,404.24023438)
\curveto(213.93789062,403.98632812)(213.43789062,403.859375)(212.89492187,403.859375)
\curveto(212.02382812,403.859375)(211.31875,404.13867188)(210.7796875,404.69726562)
\curveto(210.24453125,405.25585938)(209.97695312,406.06054688)(209.97695312,407.11132812)
\closepath
\moveto(211.0609375,407.11132812)
\curveto(211.0609375,406.31445312)(211.23476562,405.71679688)(211.58242187,405.31835938)
\curveto(211.93007812,404.92382812)(212.36757812,404.7265625)(212.89492187,404.7265625)
\curveto(213.41835937,404.7265625)(213.85390625,404.92578125)(214.2015625,405.32421875)
\curveto(214.54921875,405.72265625)(214.72304687,406.33007812)(214.72304687,407.14648438)
\curveto(214.72304687,407.91601562)(214.54726562,408.49804688)(214.19570312,408.89257812)
\curveto(213.84804687,409.29101562)(213.41445312,409.49023438)(212.89492187,409.49023438)
\curveto(212.36757812,409.49023438)(211.93007812,409.29296875)(211.58242187,408.8984375)
\curveto(211.23476562,408.50390625)(211.0609375,407.90820312)(211.0609375,407.11132812)
\closepath
}
}
{
\newrgbcolor{curcolor}{0 0 0}
\pscustom[linestyle=none,fillstyle=solid,fillcolor=curcolor]
{
\newpath
\moveto(221.08046875,404)
\lineto(221.08046875,404.78515625)
\curveto(220.6859375,404.16796875)(220.10585937,403.859375)(219.34023437,403.859375)
\curveto(218.84414062,403.859375)(218.38710937,403.99609375)(217.96914062,404.26953125)
\curveto(217.55507812,404.54296875)(217.2328125,404.92382812)(217.00234375,405.41210938)
\curveto(216.77578125,405.90429688)(216.6625,406.46875)(216.6625,407.10546875)
\curveto(216.6625,407.7265625)(216.76601562,408.2890625)(216.97304687,408.79296875)
\curveto(217.18007812,409.30078125)(217.490625,409.68945312)(217.9046875,409.95898438)
\curveto(218.31875,410.22851562)(218.78164062,410.36328125)(219.29335937,410.36328125)
\curveto(219.66835937,410.36328125)(220.00234375,410.28320312)(220.2953125,410.12304688)
\curveto(220.58828125,409.96679688)(220.8265625,409.76171875)(221.01015625,409.5078125)
\lineto(221.01015625,412.58984375)
\lineto(222.05898437,412.58984375)
\lineto(222.05898437,404)
\closepath
\moveto(217.74648437,407.10546875)
\curveto(217.74648437,406.30859375)(217.91445312,405.71289062)(218.25039062,405.31835938)
\curveto(218.58632812,404.92382812)(218.9828125,404.7265625)(219.43984375,404.7265625)
\curveto(219.90078125,404.7265625)(220.29140625,404.9140625)(220.61171875,405.2890625)
\curveto(220.9359375,405.66796875)(221.09804687,406.24414062)(221.09804687,407.01757812)
\curveto(221.09804687,407.86914062)(220.93398437,408.49414062)(220.60585937,408.89257812)
\curveto(220.27773437,409.29101562)(219.8734375,409.49023438)(219.39296875,409.49023438)
\curveto(218.92421875,409.49023438)(218.53164062,409.29882812)(218.21523437,408.91601562)
\curveto(217.90273437,408.53320312)(217.74648437,407.9296875)(217.74648437,407.10546875)
\closepath
}
}
{
\newrgbcolor{curcolor}{0 0 0}
\pscustom[linestyle=none,fillstyle=solid,fillcolor=curcolor]
{
\newpath
\moveto(227.97695312,406.00390625)
\lineto(229.06679687,405.86914062)
\curveto(228.89492187,405.23242188)(228.5765625,404.73828125)(228.11171875,404.38671875)
\curveto(227.646875,404.03515625)(227.053125,403.859375)(226.33046875,403.859375)
\curveto(225.4203125,403.859375)(224.69765625,404.13867188)(224.1625,404.69726562)
\curveto(223.63125,405.25976562)(223.365625,406.046875)(223.365625,407.05859375)
\curveto(223.365625,408.10546875)(223.63515625,408.91796875)(224.17421875,409.49609375)
\curveto(224.71328125,410.07421875)(225.4125,410.36328125)(226.271875,410.36328125)
\curveto(227.10390625,410.36328125)(227.78359375,410.08007812)(228.3109375,409.51367188)
\curveto(228.83828125,408.94726562)(229.10195312,408.15039062)(229.10195312,407.12304688)
\curveto(229.10195312,407.06054688)(229.1,406.96679688)(229.09609375,406.84179688)
\lineto(224.45546875,406.84179688)
\curveto(224.49453125,406.15820312)(224.68789062,405.63476562)(225.03554687,405.27148438)
\curveto(225.38320312,404.90820312)(225.81679687,404.7265625)(226.33632812,404.7265625)
\curveto(226.72304687,404.7265625)(227.053125,404.828125)(227.3265625,405.03125)
\curveto(227.6,405.234375)(227.81679687,405.55859375)(227.97695312,406.00390625)
\closepath
\moveto(224.5140625,407.70898438)
\lineto(227.98867187,407.70898438)
\curveto(227.94179687,408.23242188)(227.80898437,408.625)(227.59023437,408.88671875)
\curveto(227.25429687,409.29296875)(226.81875,409.49609375)(226.28359375,409.49609375)
\curveto(225.79921875,409.49609375)(225.39101562,409.33398438)(225.05898437,409.00976562)
\curveto(224.73085937,408.68554688)(224.54921875,408.25195312)(224.5140625,407.70898438)
\closepath
}
}
{
\newrgbcolor{curcolor}{0 0 1}
\pscustom[linewidth=1,linecolor=curcolor]
{
\newpath
\moveto(237.9,407.9)
\lineto(280.1,407.9)
\moveto(105.1,83.8)
\lineto(183.4,130)
\lineto(261.7,222.6)
\lineto(340.1,264.3)
\lineto(418.4,307.8)
\lineto(496.7,353.8)
\lineto(575,399.7)
}
}
{
\newrgbcolor{curcolor}{0 0 0}
\pscustom[linestyle=none,fillstyle=solid,fillcolor=curcolor]
{
\newpath
\moveto(153.36953125,386)
\lineto(153.36953125,394.58984375)
\lineto(156.32851562,394.58984375)
\curveto(156.99648437,394.58984375)(157.50625,394.54882812)(157.8578125,394.46679688)
\curveto(158.35,394.35351562)(158.76992187,394.1484375)(159.11757812,393.8515625)
\curveto(159.57070312,393.46875)(159.90859375,392.97851562)(160.13125,392.38085938)
\curveto(160.3578125,391.78710938)(160.47109375,391.10742188)(160.47109375,390.34179688)
\curveto(160.47109375,389.68945312)(160.39492187,389.11132812)(160.24257812,388.60742188)
\curveto(160.09023437,388.10351562)(159.89492187,387.68554688)(159.65664062,387.35351562)
\curveto(159.41835937,387.02539062)(159.15664062,386.765625)(158.87148437,386.57421875)
\curveto(158.59023437,386.38671875)(158.2484375,386.24414062)(157.84609375,386.14648438)
\curveto(157.44765625,386.04882812)(156.98867187,386)(156.46914062,386)
\closepath
\moveto(154.50625,387.01367188)
\lineto(156.34023437,387.01367188)
\curveto(156.90664062,387.01367188)(157.35,387.06640625)(157.6703125,387.171875)
\curveto(157.99453125,387.27734375)(158.25234375,387.42578125)(158.44375,387.6171875)
\curveto(158.71328125,387.88671875)(158.92226562,388.24804688)(159.07070312,388.70117188)
\curveto(159.22304687,389.15820312)(159.29921875,389.7109375)(159.29921875,390.359375)
\curveto(159.29921875,391.2578125)(159.15078125,391.94726562)(158.85390625,392.42773438)
\curveto(158.5609375,392.91210938)(158.20351562,393.23632812)(157.78164062,393.40039062)
\curveto(157.47695312,393.51757812)(156.98671875,393.57617188)(156.3109375,393.57617188)
\lineto(154.50625,393.57617188)
\closepath
}
}
{
\newrgbcolor{curcolor}{0 0 0}
\pscustom[linestyle=none,fillstyle=solid,fillcolor=curcolor]
{
\newpath
\moveto(161.90664062,393.37695312)
\lineto(161.90664062,394.58984375)
\lineto(162.96132812,394.58984375)
\lineto(162.96132812,393.37695312)
\closepath
\moveto(161.90664062,386)
\lineto(161.90664062,392.22265625)
\lineto(162.96132812,392.22265625)
\lineto(162.96132812,386)
\closepath
}
}
{
\newrgbcolor{curcolor}{0 0 0}
\pscustom[linestyle=none,fillstyle=solid,fillcolor=curcolor]
{
\newpath
\moveto(164.81875,386)
\lineto(164.81875,391.40234375)
\lineto(163.88710937,391.40234375)
\lineto(163.88710937,392.22265625)
\lineto(164.81875,392.22265625)
\lineto(164.81875,392.88476562)
\curveto(164.81875,393.30273438)(164.85585937,393.61328125)(164.93007812,393.81640625)
\curveto(165.03164062,394.08984375)(165.209375,394.31054688)(165.46328125,394.47851562)
\curveto(165.72109375,394.65039062)(166.08046875,394.73632812)(166.54140625,394.73632812)
\curveto(166.83828125,394.73632812)(167.16640625,394.70117188)(167.52578125,394.63085938)
\lineto(167.36757812,393.7109375)
\curveto(167.14882812,393.75)(166.94179687,393.76953125)(166.74648437,393.76953125)
\curveto(166.42617187,393.76953125)(166.19960937,393.70117188)(166.06679687,393.56445312)
\curveto(165.93398437,393.42773438)(165.86757812,393.171875)(165.86757812,392.796875)
\lineto(165.86757812,392.22265625)
\lineto(167.08046875,392.22265625)
\lineto(167.08046875,391.40234375)
\lineto(165.86757812,391.40234375)
\lineto(165.86757812,386)
\closepath
}
}
{
\newrgbcolor{curcolor}{0 0 0}
\pscustom[linestyle=none,fillstyle=solid,fillcolor=curcolor]
{
\newpath
\moveto(167.9359375,386)
\lineto(167.9359375,391.40234375)
\lineto(167.00429687,391.40234375)
\lineto(167.00429687,392.22265625)
\lineto(167.9359375,392.22265625)
\lineto(167.9359375,392.88476562)
\curveto(167.9359375,393.30273438)(167.97304687,393.61328125)(168.04726562,393.81640625)
\curveto(168.14882812,394.08984375)(168.3265625,394.31054688)(168.58046875,394.47851562)
\curveto(168.83828125,394.65039062)(169.19765625,394.73632812)(169.65859375,394.73632812)
\curveto(169.95546875,394.73632812)(170.28359375,394.70117188)(170.64296875,394.63085938)
\lineto(170.48476562,393.7109375)
\curveto(170.26601562,393.75)(170.05898437,393.76953125)(169.86367187,393.76953125)
\curveto(169.54335937,393.76953125)(169.31679687,393.70117188)(169.18398437,393.56445312)
\curveto(169.05117187,393.42773438)(168.98476562,393.171875)(168.98476562,392.796875)
\lineto(168.98476562,392.22265625)
\lineto(170.19765625,392.22265625)
\lineto(170.19765625,391.40234375)
\lineto(168.98476562,391.40234375)
\lineto(168.98476562,386)
\closepath
}
}
{
\newrgbcolor{curcolor}{0 0 0}
\pscustom[linestyle=none,fillstyle=solid,fillcolor=curcolor]
{
\newpath
\moveto(175.27773437,388.00390625)
\lineto(176.36757812,387.86914062)
\curveto(176.19570312,387.23242188)(175.87734375,386.73828125)(175.4125,386.38671875)
\curveto(174.94765625,386.03515625)(174.35390625,385.859375)(173.63125,385.859375)
\curveto(172.72109375,385.859375)(171.9984375,386.13867188)(171.46328125,386.69726562)
\curveto(170.93203125,387.25976562)(170.66640625,388.046875)(170.66640625,389.05859375)
\curveto(170.66640625,390.10546875)(170.9359375,390.91796875)(171.475,391.49609375)
\curveto(172.0140625,392.07421875)(172.71328125,392.36328125)(173.57265625,392.36328125)
\curveto(174.4046875,392.36328125)(175.084375,392.08007812)(175.61171875,391.51367188)
\curveto(176.1390625,390.94726562)(176.40273437,390.15039062)(176.40273437,389.12304688)
\curveto(176.40273437,389.06054688)(176.40078125,388.96679688)(176.396875,388.84179688)
\lineto(171.75625,388.84179688)
\curveto(171.7953125,388.15820312)(171.98867187,387.63476562)(172.33632812,387.27148438)
\curveto(172.68398437,386.90820312)(173.11757812,386.7265625)(173.63710937,386.7265625)
\curveto(174.02382812,386.7265625)(174.35390625,386.828125)(174.62734375,387.03125)
\curveto(174.90078125,387.234375)(175.11757812,387.55859375)(175.27773437,388.00390625)
\closepath
\moveto(171.81484375,389.70898438)
\lineto(175.28945312,389.70898438)
\curveto(175.24257812,390.23242188)(175.10976562,390.625)(174.89101562,390.88671875)
\curveto(174.55507812,391.29296875)(174.11953125,391.49609375)(173.584375,391.49609375)
\curveto(173.1,391.49609375)(172.69179687,391.33398438)(172.35976562,391.00976562)
\curveto(172.03164062,390.68554688)(171.85,390.25195312)(171.81484375,389.70898438)
\closepath
}
}
{
\newrgbcolor{curcolor}{0 0 0}
\pscustom[linestyle=none,fillstyle=solid,fillcolor=curcolor]
{
\newpath
\moveto(177.68007812,386)
\lineto(177.68007812,392.22265625)
\lineto(178.62929687,392.22265625)
\lineto(178.62929687,391.27929688)
\curveto(178.87148437,391.72070312)(179.09414062,392.01171875)(179.29726562,392.15234375)
\curveto(179.50429687,392.29296875)(179.73085937,392.36328125)(179.97695312,392.36328125)
\curveto(180.33242187,392.36328125)(180.69375,392.25)(181.0609375,392.0234375)
\lineto(180.69765625,391.04492188)
\curveto(180.43984375,391.19726562)(180.18203125,391.2734375)(179.92421875,391.2734375)
\curveto(179.69375,391.2734375)(179.48671875,391.203125)(179.303125,391.0625)
\curveto(179.11953125,390.92578125)(178.98867187,390.734375)(178.91054687,390.48828125)
\curveto(178.79335937,390.11328125)(178.73476562,389.703125)(178.73476562,389.2578125)
\lineto(178.73476562,386)
\closepath
}
}
{
\newrgbcolor{curcolor}{0 0 0}
\pscustom[linestyle=none,fillstyle=solid,fillcolor=curcolor]
{
\newpath
\moveto(185.94765625,388.00390625)
\lineto(187.0375,387.86914062)
\curveto(186.865625,387.23242188)(186.54726562,386.73828125)(186.08242187,386.38671875)
\curveto(185.61757812,386.03515625)(185.02382812,385.859375)(184.30117187,385.859375)
\curveto(183.39101562,385.859375)(182.66835937,386.13867188)(182.13320312,386.69726562)
\curveto(181.60195312,387.25976562)(181.33632812,388.046875)(181.33632812,389.05859375)
\curveto(181.33632812,390.10546875)(181.60585937,390.91796875)(182.14492187,391.49609375)
\curveto(182.68398437,392.07421875)(183.38320312,392.36328125)(184.24257812,392.36328125)
\curveto(185.07460937,392.36328125)(185.75429687,392.08007812)(186.28164062,391.51367188)
\curveto(186.80898437,390.94726562)(187.07265625,390.15039062)(187.07265625,389.12304688)
\curveto(187.07265625,389.06054688)(187.07070312,388.96679688)(187.06679687,388.84179688)
\lineto(182.42617187,388.84179688)
\curveto(182.46523437,388.15820312)(182.65859375,387.63476562)(183.00625,387.27148438)
\curveto(183.35390625,386.90820312)(183.7875,386.7265625)(184.30703125,386.7265625)
\curveto(184.69375,386.7265625)(185.02382812,386.828125)(185.29726562,387.03125)
\curveto(185.57070312,387.234375)(185.7875,387.55859375)(185.94765625,388.00390625)
\closepath
\moveto(182.48476562,389.70898438)
\lineto(185.959375,389.70898438)
\curveto(185.9125,390.23242188)(185.7796875,390.625)(185.5609375,390.88671875)
\curveto(185.225,391.29296875)(184.78945312,391.49609375)(184.25429687,391.49609375)
\curveto(183.76992187,391.49609375)(183.36171875,391.33398438)(183.0296875,391.00976562)
\curveto(182.7015625,390.68554688)(182.51992187,390.25195312)(182.48476562,389.70898438)
\closepath
}
}
{
\newrgbcolor{curcolor}{0 0 0}
\pscustom[linestyle=none,fillstyle=solid,fillcolor=curcolor]
{
\newpath
\moveto(188.36171875,386)
\lineto(188.36171875,392.22265625)
\lineto(189.3109375,392.22265625)
\lineto(189.3109375,391.33789062)
\curveto(189.76796875,392.02148438)(190.428125,392.36328125)(191.29140625,392.36328125)
\curveto(191.66640625,392.36328125)(192.01015625,392.29492188)(192.32265625,392.15820312)
\curveto(192.6390625,392.02539062)(192.87539062,391.84960938)(193.03164062,391.63085938)
\curveto(193.18789062,391.41210938)(193.29726562,391.15234375)(193.35976562,390.8515625)
\curveto(193.39882812,390.65625)(193.41835937,390.31445312)(193.41835937,389.82617188)
\lineto(193.41835937,386)
\lineto(192.36367187,386)
\lineto(192.36367187,389.78515625)
\curveto(192.36367187,390.21484375)(192.32265625,390.53515625)(192.240625,390.74609375)
\curveto(192.15859375,390.9609375)(192.01210937,391.13085938)(191.80117187,391.25585938)
\curveto(191.59414062,391.38476562)(191.35,391.44921875)(191.06875,391.44921875)
\curveto(190.61953125,391.44921875)(190.23085937,391.30664062)(189.90273437,391.02148438)
\curveto(189.57851562,390.73632812)(189.41640625,390.1953125)(189.41640625,389.3984375)
\lineto(189.41640625,386)
\closepath
}
}
{
\newrgbcolor{curcolor}{0 0 0}
\pscustom[linestyle=none,fillstyle=solid,fillcolor=curcolor]
{
\newpath
\moveto(197.33828125,386.94335938)
\lineto(197.490625,386.01171875)
\curveto(197.19375,385.94921875)(196.928125,385.91796875)(196.69375,385.91796875)
\curveto(196.3109375,385.91796875)(196.0140625,385.97851562)(195.803125,386.09960938)
\curveto(195.5921875,386.22070312)(195.44375,386.37890625)(195.3578125,386.57421875)
\curveto(195.271875,386.7734375)(195.22890625,387.18945312)(195.22890625,387.82226562)
\lineto(195.22890625,391.40234375)
\lineto(194.45546875,391.40234375)
\lineto(194.45546875,392.22265625)
\lineto(195.22890625,392.22265625)
\lineto(195.22890625,393.76367188)
\lineto(196.27773437,394.39648438)
\lineto(196.27773437,392.22265625)
\lineto(197.33828125,392.22265625)
\lineto(197.33828125,391.40234375)
\lineto(196.27773437,391.40234375)
\lineto(196.27773437,387.76367188)
\curveto(196.27773437,387.46289062)(196.2953125,387.26953125)(196.33046875,387.18359375)
\curveto(196.36953125,387.09765625)(196.43007812,387.02929688)(196.51210937,386.97851562)
\curveto(196.59804687,386.92773438)(196.71914062,386.90234375)(196.87539062,386.90234375)
\curveto(196.99257812,386.90234375)(197.146875,386.91601562)(197.33828125,386.94335938)
\closepath
}
}
{
\newrgbcolor{curcolor}{0 0 0}
\pscustom[linestyle=none,fillstyle=solid,fillcolor=curcolor]
{
\newpath
\moveto(201.8265625,386)
\lineto(201.8265625,394.58984375)
\lineto(202.99257812,394.58984375)
\lineto(207.50429687,387.84570312)
\lineto(207.50429687,394.58984375)
\lineto(208.59414062,394.58984375)
\lineto(208.59414062,386)
\lineto(207.428125,386)
\lineto(202.91640625,392.75)
\lineto(202.91640625,386)
\closepath
}
}
{
\newrgbcolor{curcolor}{0 0 0}
\pscustom[linestyle=none,fillstyle=solid,fillcolor=curcolor]
{
\newpath
\moveto(209.97695312,389.11132812)
\curveto(209.97695312,390.26367188)(210.29726562,391.1171875)(210.93789062,391.671875)
\curveto(211.47304687,392.1328125)(212.12539062,392.36328125)(212.89492187,392.36328125)
\curveto(213.75039062,392.36328125)(214.44960937,392.08203125)(214.99257812,391.51953125)
\curveto(215.53554687,390.9609375)(215.80703125,390.1875)(215.80703125,389.19921875)
\curveto(215.80703125,388.3984375)(215.6859375,387.76757812)(215.44375,387.30664062)
\curveto(215.20546875,386.84960938)(214.85585937,386.49414062)(214.39492187,386.24023438)
\curveto(213.93789062,385.98632812)(213.43789062,385.859375)(212.89492187,385.859375)
\curveto(212.02382812,385.859375)(211.31875,386.13867188)(210.7796875,386.69726562)
\curveto(210.24453125,387.25585938)(209.97695312,388.06054688)(209.97695312,389.11132812)
\closepath
\moveto(211.0609375,389.11132812)
\curveto(211.0609375,388.31445312)(211.23476562,387.71679688)(211.58242187,387.31835938)
\curveto(211.93007812,386.92382812)(212.36757812,386.7265625)(212.89492187,386.7265625)
\curveto(213.41835937,386.7265625)(213.85390625,386.92578125)(214.2015625,387.32421875)
\curveto(214.54921875,387.72265625)(214.72304687,388.33007812)(214.72304687,389.14648438)
\curveto(214.72304687,389.91601562)(214.54726562,390.49804688)(214.19570312,390.89257812)
\curveto(213.84804687,391.29101562)(213.41445312,391.49023438)(212.89492187,391.49023438)
\curveto(212.36757812,391.49023438)(211.93007812,391.29296875)(211.58242187,390.8984375)
\curveto(211.23476562,390.50390625)(211.0609375,389.90820312)(211.0609375,389.11132812)
\closepath
}
}
{
\newrgbcolor{curcolor}{0 0 0}
\pscustom[linestyle=none,fillstyle=solid,fillcolor=curcolor]
{
\newpath
\moveto(221.08046875,386)
\lineto(221.08046875,386.78515625)
\curveto(220.6859375,386.16796875)(220.10585937,385.859375)(219.34023437,385.859375)
\curveto(218.84414062,385.859375)(218.38710937,385.99609375)(217.96914062,386.26953125)
\curveto(217.55507812,386.54296875)(217.2328125,386.92382812)(217.00234375,387.41210938)
\curveto(216.77578125,387.90429688)(216.6625,388.46875)(216.6625,389.10546875)
\curveto(216.6625,389.7265625)(216.76601562,390.2890625)(216.97304687,390.79296875)
\curveto(217.18007812,391.30078125)(217.490625,391.68945312)(217.9046875,391.95898438)
\curveto(218.31875,392.22851562)(218.78164062,392.36328125)(219.29335937,392.36328125)
\curveto(219.66835937,392.36328125)(220.00234375,392.28320312)(220.2953125,392.12304688)
\curveto(220.58828125,391.96679688)(220.8265625,391.76171875)(221.01015625,391.5078125)
\lineto(221.01015625,394.58984375)
\lineto(222.05898437,394.58984375)
\lineto(222.05898437,386)
\closepath
\moveto(217.74648437,389.10546875)
\curveto(217.74648437,388.30859375)(217.91445312,387.71289062)(218.25039062,387.31835938)
\curveto(218.58632812,386.92382812)(218.9828125,386.7265625)(219.43984375,386.7265625)
\curveto(219.90078125,386.7265625)(220.29140625,386.9140625)(220.61171875,387.2890625)
\curveto(220.9359375,387.66796875)(221.09804687,388.24414062)(221.09804687,389.01757812)
\curveto(221.09804687,389.86914062)(220.93398437,390.49414062)(220.60585937,390.89257812)
\curveto(220.27773437,391.29101562)(219.8734375,391.49023438)(219.39296875,391.49023438)
\curveto(218.92421875,391.49023438)(218.53164062,391.29882812)(218.21523437,390.91601562)
\curveto(217.90273437,390.53320312)(217.74648437,389.9296875)(217.74648437,389.10546875)
\closepath
}
}
{
\newrgbcolor{curcolor}{0 0 0}
\pscustom[linestyle=none,fillstyle=solid,fillcolor=curcolor]
{
\newpath
\moveto(227.97695312,388.00390625)
\lineto(229.06679687,387.86914062)
\curveto(228.89492187,387.23242188)(228.5765625,386.73828125)(228.11171875,386.38671875)
\curveto(227.646875,386.03515625)(227.053125,385.859375)(226.33046875,385.859375)
\curveto(225.4203125,385.859375)(224.69765625,386.13867188)(224.1625,386.69726562)
\curveto(223.63125,387.25976562)(223.365625,388.046875)(223.365625,389.05859375)
\curveto(223.365625,390.10546875)(223.63515625,390.91796875)(224.17421875,391.49609375)
\curveto(224.71328125,392.07421875)(225.4125,392.36328125)(226.271875,392.36328125)
\curveto(227.10390625,392.36328125)(227.78359375,392.08007812)(228.3109375,391.51367188)
\curveto(228.83828125,390.94726562)(229.10195312,390.15039062)(229.10195312,389.12304688)
\curveto(229.10195312,389.06054688)(229.1,388.96679688)(229.09609375,388.84179688)
\lineto(224.45546875,388.84179688)
\curveto(224.49453125,388.15820312)(224.68789062,387.63476562)(225.03554687,387.27148438)
\curveto(225.38320312,386.90820312)(225.81679687,386.7265625)(226.33632812,386.7265625)
\curveto(226.72304687,386.7265625)(227.053125,386.828125)(227.3265625,387.03125)
\curveto(227.6,387.234375)(227.81679687,387.55859375)(227.97695312,388.00390625)
\closepath
\moveto(224.5140625,389.70898438)
\lineto(227.98867187,389.70898438)
\curveto(227.94179687,390.23242188)(227.80898437,390.625)(227.59023437,390.88671875)
\curveto(227.25429687,391.29296875)(226.81875,391.49609375)(226.28359375,391.49609375)
\curveto(225.79921875,391.49609375)(225.39101562,391.33398438)(225.05898437,391.00976562)
\curveto(224.73085937,390.68554688)(224.54921875,390.25195312)(224.5140625,389.70898438)
\closepath
}
}
{
\newrgbcolor{curcolor}{1 0 0}
\pscustom[linewidth=1,linecolor=curcolor]
{
\newpath
\moveto(237.9,389.9)
\lineto(280.1,389.9)
\moveto(105.1,87.9)
\lineto(183.4,125.9)
\lineto(261.7,223.6)
\lineto(340.1,269.1)
\lineto(418.4,316.4)
\lineto(496.7,362.4)
\lineto(575,409.3)
}
}
{
\newrgbcolor{curcolor}{0 0 0}
\pscustom[linewidth=1,linecolor=curcolor]
{
\newpath
\moveto(105.1,425.9)
\lineto(105.1,57.6)
\lineto(575,57.6)
\lineto(575,425.9)
\closepath
}
}
\end{pspicture}
}
    \caption{Reading a file from a different node than the one its ARC data is stored on causes a 10-15\% loss of 
        performance, proving that ZFS, specifically its ARC, is affected by NUMA.}
    \label{fig:OldZFS}
\end{figure}

\section{How to Mitigate the Effects}
Fundamentally, the performance loss observed comes from the need to pass data across the interconnect between NUMA nodes.
The data needs to be close to the process that is reading it in order to get the best performance.

When we find a process in this situation, reading data from another node, there are two ways we can fix it.
Either all the data the process needs can be moved closer to the process, or we can move the process closer to its data.
Regardless, some data will need to move across the interconnect, as the process's other memory allocations should continue to be local to it, 
and not have to be accessed via the interconnect, as otherwise this will incur the same performance penalty we are trying to avoid.

\chapter{NUMA Balancing}
Linux attempts to prevent this kind of performance loss by tracking remote node memory accesses, and automatically moving memory to the node
that uses it most \cite[{Documentation/sysctl/kernel.txt}]{linux}.
It tracks this by periodically marking pages inaccessible in the page table and trapping the eventual page fault that occurs when 
a process tries to access it.
It then keeps statistics on which NUMA node the process is running on  when it accesses memory.
A periodic scanning thread looks at these statistics for each process and decides if a page needs to be moved
based on how often it was accessed from a different node.
However, this operates only at the page level, meaning that by the time it figures out that a page needs to be moved,
it is often too late to help speed up the memory accesses of a particular process.
It also only scans once every second by default, so if your program takes less time than that the memory it accesses
is not even considered for migration\cite[{kernel/sched/fair.c}]{linux}.

Enabling NUMA Balancing and using the same simple read program as before, at best it manages to regain
3\% of the lost performance (Figure \ref{fig:NUMABalance}).
This is likely within the margin of error for these tests, as due to the way that the ARC allocates memory (Section \ref{chapter:ARC}),
it is given physical pages by the Linux kernel and thus the kernel is unable to move them around as it would be if the ARC
asked for virtual memory.

\begin{figure}[H]
    \centering
    \resizebox{0.75\linewidth}{!}{%LaTeX with PSTricks extensions
%%Creator: Inkscape 1.0.2-2 (e86c870879, 2021-01-15)
%%Please note this file requires PSTricks extensions
\psset{xunit=.5pt,yunit=.5pt,runit=.5pt}
\begin{pspicture}(600,480)
{
\newrgbcolor{curcolor}{0 0 0}
\pscustom[linewidth=1,linecolor=curcolor]
{
\newpath
\moveto(105.1,57.6)
\lineto(114.1,57.6)
\moveto(575,57.6)
\lineto(566,57.6)
}
}
{
\newrgbcolor{curcolor}{0 0 0}
\pscustom[linestyle=none,fillstyle=solid,fillcolor=curcolor]
{
\newpath
\moveto(53.92109375,57.93632812)
\curveto(53.92109375,58.95195312)(54.02460937,59.76835937)(54.23164062,60.38554687)
\curveto(54.44257812,61.00664062)(54.753125,61.48515625)(55.16328125,61.82109375)
\curveto(55.57734375,62.15703125)(56.096875,62.325)(56.721875,62.325)
\curveto(57.1828125,62.325)(57.58710937,62.23125)(57.93476562,62.04375)
\curveto(58.28242187,61.86015625)(58.56953125,61.59257812)(58.79609375,61.24101562)
\curveto(59.02265625,60.89335937)(59.20039062,60.46757812)(59.32929687,59.96367187)
\curveto(59.45820312,59.46367187)(59.52265625,58.78789062)(59.52265625,57.93632812)
\curveto(59.52265625,56.92851562)(59.41914062,56.1140625)(59.21210937,55.49296875)
\curveto(59.00507812,54.87578125)(58.69453125,54.39726562)(58.28046875,54.05742187)
\curveto(57.8703125,53.72148437)(57.35078125,53.55351562)(56.721875,53.55351562)
\curveto(55.89375,53.55351562)(55.24335937,53.85039062)(54.77070312,54.44414062)
\curveto(54.20429687,55.15898437)(53.92109375,56.32304687)(53.92109375,57.93632812)
\closepath
\moveto(55.00507812,57.93632812)
\curveto(55.00507812,56.52617187)(55.16914062,55.58671875)(55.49726562,55.11796875)
\curveto(55.82929687,54.653125)(56.2375,54.42070312)(56.721875,54.42070312)
\curveto(57.20625,54.42070312)(57.6125,54.65507812)(57.940625,55.12382812)
\curveto(58.27265625,55.59257812)(58.43867187,56.53007812)(58.43867187,57.93632812)
\curveto(58.43867187,59.35039062)(58.27265625,60.28984375)(57.940625,60.7546875)
\curveto(57.6125,61.21953125)(57.20234375,61.45195312)(56.71015625,61.45195312)
\curveto(56.22578125,61.45195312)(55.8390625,61.246875)(55.55,60.83671875)
\curveto(55.18671875,60.31328125)(55.00507812,59.34648437)(55.00507812,57.93632812)
\closepath
}
}
{
\newrgbcolor{curcolor}{0 0 0}
\pscustom[linestyle=none,fillstyle=solid,fillcolor=curcolor]
{
\newpath
\moveto(61.18671875,53.7)
\lineto(61.18671875,54.90117187)
\lineto(62.38789062,54.90117187)
\lineto(62.38789062,53.7)
\closepath
}
}
{
\newrgbcolor{curcolor}{0 0 0}
\pscustom[linestyle=none,fillstyle=solid,fillcolor=curcolor]
{
\newpath
\moveto(63.92890625,57.93632812)
\curveto(63.92890625,58.95195312)(64.03242187,59.76835937)(64.23945312,60.38554687)
\curveto(64.45039062,61.00664062)(64.7609375,61.48515625)(65.17109375,61.82109375)
\curveto(65.58515625,62.15703125)(66.1046875,62.325)(66.7296875,62.325)
\curveto(67.190625,62.325)(67.59492187,62.23125)(67.94257812,62.04375)
\curveto(68.29023437,61.86015625)(68.57734375,61.59257812)(68.80390625,61.24101562)
\curveto(69.03046875,60.89335937)(69.20820312,60.46757812)(69.33710937,59.96367187)
\curveto(69.46601562,59.46367187)(69.53046875,58.78789062)(69.53046875,57.93632812)
\curveto(69.53046875,56.92851562)(69.42695312,56.1140625)(69.21992187,55.49296875)
\curveto(69.01289062,54.87578125)(68.70234375,54.39726562)(68.28828125,54.05742187)
\curveto(67.878125,53.72148437)(67.35859375,53.55351562)(66.7296875,53.55351562)
\curveto(65.9015625,53.55351562)(65.25117187,53.85039062)(64.77851562,54.44414062)
\curveto(64.21210937,55.15898437)(63.92890625,56.32304687)(63.92890625,57.93632812)
\closepath
\moveto(65.01289062,57.93632812)
\curveto(65.01289062,56.52617187)(65.17695312,55.58671875)(65.50507812,55.11796875)
\curveto(65.83710937,54.653125)(66.2453125,54.42070312)(66.7296875,54.42070312)
\curveto(67.2140625,54.42070312)(67.6203125,54.65507812)(67.9484375,55.12382812)
\curveto(68.28046875,55.59257812)(68.44648437,56.53007812)(68.44648437,57.93632812)
\curveto(68.44648437,59.35039062)(68.28046875,60.28984375)(67.9484375,60.7546875)
\curveto(67.6203125,61.21953125)(67.21015625,61.45195312)(66.71796875,61.45195312)
\curveto(66.23359375,61.45195312)(65.846875,61.246875)(65.5578125,60.83671875)
\curveto(65.19453125,60.31328125)(65.01289062,59.34648437)(65.01289062,57.93632812)
\closepath
}
}
{
\newrgbcolor{curcolor}{0 0 0}
\pscustom[linestyle=none,fillstyle=solid,fillcolor=curcolor]
{
\newpath
\moveto(70.60859375,55.96757812)
\lineto(71.66328125,56.10820312)
\curveto(71.784375,55.51054687)(71.98945312,55.07890625)(72.27851562,54.81328125)
\curveto(72.57148437,54.5515625)(72.92695312,54.42070312)(73.34492187,54.42070312)
\curveto(73.84101562,54.42070312)(74.25898437,54.59257812)(74.59882812,54.93632812)
\curveto(74.94257812,55.28007812)(75.11445312,55.70585937)(75.11445312,56.21367187)
\curveto(75.11445312,56.69804687)(74.95625,57.09648437)(74.63984375,57.40898437)
\curveto(74.3234375,57.72539062)(73.92109375,57.88359375)(73.4328125,57.88359375)
\curveto(73.23359375,57.88359375)(72.98554687,57.84453125)(72.68867187,57.76640625)
\lineto(72.80585937,58.6921875)
\curveto(72.87617187,58.684375)(72.9328125,58.68046875)(72.97578125,58.68046875)
\curveto(73.425,58.68046875)(73.82929687,58.79765625)(74.18867187,59.03203125)
\curveto(74.54804687,59.26640625)(74.72773437,59.62773437)(74.72773437,60.11601562)
\curveto(74.72773437,60.50273437)(74.596875,60.82304687)(74.33515625,61.07695312)
\curveto(74.0734375,61.33085937)(73.73554687,61.4578125)(73.32148437,61.4578125)
\curveto(72.91132812,61.4578125)(72.56953125,61.32890625)(72.29609375,61.07109375)
\curveto(72.02265625,60.81328125)(71.846875,60.4265625)(71.76875,59.9109375)
\lineto(70.7140625,60.0984375)
\curveto(70.84296875,60.80546875)(71.1359375,61.35234375)(71.59296875,61.7390625)
\curveto(72.05,62.1296875)(72.61835937,62.325)(73.29804687,62.325)
\curveto(73.76679687,62.325)(74.1984375,62.2234375)(74.59296875,62.0203125)
\curveto(74.9875,61.82109375)(75.28828125,61.54765625)(75.4953125,61.2)
\curveto(75.70625,60.85234375)(75.81171875,60.48320312)(75.81171875,60.09257812)
\curveto(75.81171875,59.72148437)(75.71210937,59.38359375)(75.51289062,59.07890625)
\curveto(75.31367187,58.77421875)(75.01875,58.53203125)(74.628125,58.35234375)
\curveto(75.1359375,58.23515625)(75.53046875,57.99101562)(75.81171875,57.61992187)
\curveto(76.09296875,57.25273437)(76.23359375,56.79179687)(76.23359375,56.23710937)
\curveto(76.23359375,55.48710937)(75.96015625,54.85039062)(75.41328125,54.32695312)
\curveto(74.86640625,53.80742187)(74.175,53.54765625)(73.3390625,53.54765625)
\curveto(72.58515625,53.54765625)(71.95820312,53.77226562)(71.45820312,54.22148437)
\curveto(70.96210937,54.67070312)(70.67890625,55.25273437)(70.60859375,55.96757812)
\closepath
}
}
{
\newrgbcolor{curcolor}{0 0 0}
\pscustom[linestyle=none,fillstyle=solid,fillcolor=curcolor]
{
\newpath
\moveto(81.24921875,53.7)
\lineto(80.19453125,53.7)
\lineto(80.19453125,60.42070312)
\curveto(79.940625,60.17851562)(79.60664062,59.93632812)(79.19257812,59.69414062)
\curveto(78.78242187,59.45195312)(78.41328125,59.2703125)(78.08515625,59.14921875)
\lineto(78.08515625,60.16875)
\curveto(78.675,60.44609375)(79.190625,60.78203125)(79.63203125,61.1765625)
\curveto(80.0734375,61.57109375)(80.3859375,61.95390625)(80.56953125,62.325)
\lineto(81.24921875,62.325)
\closepath
}
}
{
\newrgbcolor{curcolor}{0 0 0}
\pscustom[linestyle=none,fillstyle=solid,fillcolor=curcolor]
{
\newpath
\moveto(89.49335937,54.71367187)
\lineto(89.49335937,53.7)
\lineto(83.815625,53.7)
\curveto(83.8078125,53.95390625)(83.84882812,54.19804687)(83.93867187,54.43242187)
\curveto(84.08320312,54.81914062)(84.31367187,55.2)(84.63007812,55.575)
\curveto(84.95039062,55.95)(85.41132812,56.38359375)(86.01289062,56.87578125)
\curveto(86.94648437,57.64140625)(87.57734375,58.246875)(87.90546875,58.6921875)
\curveto(88.23359375,59.14140625)(88.39765625,59.56523437)(88.39765625,59.96367187)
\curveto(88.39765625,60.38164062)(88.24726562,60.73320312)(87.94648437,61.01835937)
\curveto(87.64960937,61.30742187)(87.2609375,61.45195312)(86.78046875,61.45195312)
\curveto(86.27265625,61.45195312)(85.86640625,61.29960937)(85.56171875,60.99492187)
\curveto(85.25703125,60.69023437)(85.10273437,60.26835937)(85.09882812,59.72929687)
\lineto(84.01484375,59.840625)
\curveto(84.0890625,60.64921875)(84.36835937,61.26445312)(84.85273437,61.68632812)
\curveto(85.33710937,62.11210937)(85.9875,62.325)(86.80390625,62.325)
\curveto(87.628125,62.325)(88.28046875,62.09648437)(88.7609375,61.63945312)
\curveto(89.24140625,61.18242187)(89.48164062,60.61601562)(89.48164062,59.94023437)
\curveto(89.48164062,59.59648437)(89.41132812,59.25859375)(89.27070312,58.9265625)
\curveto(89.13007812,58.59453125)(88.89570312,58.24492187)(88.56757812,57.87773437)
\curveto(88.24335937,57.51054687)(87.70234375,57.00664062)(86.94453125,56.36601562)
\curveto(86.31171875,55.83476562)(85.90546875,55.4734375)(85.72578125,55.28203125)
\curveto(85.54609375,55.09453125)(85.39765625,54.90507812)(85.28046875,54.71367187)
\closepath
}
}
{
\newrgbcolor{curcolor}{0 0 0}
\pscustom[linestyle=none,fillstyle=solid,fillcolor=curcolor]
{
\newpath
\moveto(90.62421875,55.95)
\lineto(91.73164062,56.04375)
\curveto(91.81367187,55.5046875)(92.003125,55.0984375)(92.3,54.825)
\curveto(92.60078125,54.55546875)(92.96210937,54.42070312)(93.38398437,54.42070312)
\curveto(93.89179687,54.42070312)(94.32148437,54.61210937)(94.67304687,54.99492187)
\curveto(95.02460937,55.37773437)(95.20039062,55.88554687)(95.20039062,56.51835937)
\curveto(95.20039062,57.11992187)(95.03046875,57.59453125)(94.690625,57.9421875)
\curveto(94.3546875,58.28984375)(93.91328125,58.46367187)(93.36640625,58.46367187)
\curveto(93.0265625,58.46367187)(92.71992187,58.38554687)(92.44648437,58.22929687)
\curveto(92.17304687,58.07695312)(91.95820312,57.87773437)(91.80195312,57.63164062)
\lineto(90.81171875,57.76054687)
\lineto(91.64375,62.17265625)
\lineto(95.91523437,62.17265625)
\lineto(95.91523437,61.16484375)
\lineto(92.4875,61.16484375)
\lineto(92.02460937,58.85625)
\curveto(92.54023437,59.215625)(93.08125,59.3953125)(93.64765625,59.3953125)
\curveto(94.39765625,59.3953125)(95.03046875,59.13554687)(95.54609375,58.61601562)
\curveto(96.06171875,58.09648437)(96.31953125,57.42851562)(96.31953125,56.61210937)
\curveto(96.31953125,55.83476562)(96.09296875,55.16289062)(95.63984375,54.59648437)
\curveto(95.0890625,53.90117187)(94.33710937,53.55351562)(93.38398437,53.55351562)
\curveto(92.60273437,53.55351562)(91.9640625,53.77226562)(91.46796875,54.20976562)
\curveto(90.97578125,54.64726562)(90.69453125,55.22734375)(90.62421875,55.95)
\closepath
}
}
{
\newrgbcolor{curcolor}{0 0 0}
\pscustom[linewidth=1,linecolor=curcolor]
{
\newpath
\moveto(105.1,103.6)
\lineto(114.1,103.6)
\moveto(575,103.6)
\lineto(566,103.6)
}
}
{
\newrgbcolor{curcolor}{0 0 0}
\pscustom[linestyle=none,fillstyle=solid,fillcolor=curcolor]
{
\newpath
\moveto(60.59492187,103.93632812)
\curveto(60.59492187,104.95195312)(60.6984375,105.76835937)(60.90546875,106.38554687)
\curveto(61.11640625,107.00664062)(61.42695312,107.48515625)(61.83710937,107.82109375)
\curveto(62.25117187,108.15703125)(62.77070312,108.325)(63.39570312,108.325)
\curveto(63.85664062,108.325)(64.2609375,108.23125)(64.60859375,108.04375)
\curveto(64.95625,107.86015625)(65.24335937,107.59257812)(65.46992187,107.24101562)
\curveto(65.69648437,106.89335937)(65.87421875,106.46757812)(66.003125,105.96367187)
\curveto(66.13203125,105.46367187)(66.19648437,104.78789062)(66.19648437,103.93632812)
\curveto(66.19648437,102.92851562)(66.09296875,102.1140625)(65.8859375,101.49296875)
\curveto(65.67890625,100.87578125)(65.36835937,100.39726562)(64.95429687,100.05742187)
\curveto(64.54414062,99.72148437)(64.02460937,99.55351562)(63.39570312,99.55351562)
\curveto(62.56757812,99.55351562)(61.9171875,99.85039062)(61.44453125,100.44414062)
\curveto(60.878125,101.15898437)(60.59492187,102.32304687)(60.59492187,103.93632812)
\closepath
\moveto(61.67890625,103.93632812)
\curveto(61.67890625,102.52617187)(61.84296875,101.58671875)(62.17109375,101.11796875)
\curveto(62.503125,100.653125)(62.91132812,100.42070312)(63.39570312,100.42070312)
\curveto(63.88007812,100.42070312)(64.28632812,100.65507812)(64.61445312,101.12382812)
\curveto(64.94648437,101.59257812)(65.1125,102.53007812)(65.1125,103.93632812)
\curveto(65.1125,105.35039062)(64.94648437,106.28984375)(64.61445312,106.7546875)
\curveto(64.28632812,107.21953125)(63.87617187,107.45195312)(63.38398437,107.45195312)
\curveto(62.89960937,107.45195312)(62.51289062,107.246875)(62.22382812,106.83671875)
\curveto(61.86054687,106.31328125)(61.67890625,105.34648437)(61.67890625,103.93632812)
\closepath
}
}
{
\newrgbcolor{curcolor}{0 0 0}
\pscustom[linestyle=none,fillstyle=solid,fillcolor=curcolor]
{
\newpath
\moveto(67.86054687,99.7)
\lineto(67.86054687,100.90117187)
\lineto(69.06171875,100.90117187)
\lineto(69.06171875,99.7)
\closepath
}
}
{
\newrgbcolor{curcolor}{0 0 0}
\pscustom[linestyle=none,fillstyle=solid,fillcolor=curcolor]
{
\newpath
\moveto(70.60273437,103.93632812)
\curveto(70.60273437,104.95195312)(70.70625,105.76835937)(70.91328125,106.38554687)
\curveto(71.12421875,107.00664062)(71.43476562,107.48515625)(71.84492187,107.82109375)
\curveto(72.25898437,108.15703125)(72.77851562,108.325)(73.40351562,108.325)
\curveto(73.86445312,108.325)(74.26875,108.23125)(74.61640625,108.04375)
\curveto(74.9640625,107.86015625)(75.25117187,107.59257812)(75.47773437,107.24101562)
\curveto(75.70429687,106.89335937)(75.88203125,106.46757812)(76.0109375,105.96367187)
\curveto(76.13984375,105.46367187)(76.20429687,104.78789062)(76.20429687,103.93632812)
\curveto(76.20429687,102.92851562)(76.10078125,102.1140625)(75.89375,101.49296875)
\curveto(75.68671875,100.87578125)(75.37617187,100.39726562)(74.96210937,100.05742187)
\curveto(74.55195312,99.72148437)(74.03242187,99.55351562)(73.40351562,99.55351562)
\curveto(72.57539062,99.55351562)(71.925,99.85039062)(71.45234375,100.44414062)
\curveto(70.8859375,101.15898437)(70.60273437,102.32304687)(70.60273437,103.93632812)
\closepath
\moveto(71.68671875,103.93632812)
\curveto(71.68671875,102.52617187)(71.85078125,101.58671875)(72.17890625,101.11796875)
\curveto(72.5109375,100.653125)(72.91914062,100.42070312)(73.40351562,100.42070312)
\curveto(73.88789062,100.42070312)(74.29414062,100.65507812)(74.62226562,101.12382812)
\curveto(74.95429687,101.59257812)(75.1203125,102.53007812)(75.1203125,103.93632812)
\curveto(75.1203125,105.35039062)(74.95429687,106.28984375)(74.62226562,106.7546875)
\curveto(74.29414062,107.21953125)(73.88398437,107.45195312)(73.39179687,107.45195312)
\curveto(72.90742187,107.45195312)(72.52070312,107.246875)(72.23164062,106.83671875)
\curveto(71.86835937,106.31328125)(71.68671875,105.34648437)(71.68671875,103.93632812)
\closepath
}
}
{
\newrgbcolor{curcolor}{0 0 0}
\pscustom[linestyle=none,fillstyle=solid,fillcolor=curcolor]
{
\newpath
\moveto(82.74921875,106.18632812)
\lineto(81.70039062,106.10429687)
\curveto(81.60664062,106.51835937)(81.47382812,106.81914062)(81.30195312,107.00664062)
\curveto(81.01679687,107.30742187)(80.66523437,107.4578125)(80.24726562,107.4578125)
\curveto(79.91132812,107.4578125)(79.61640625,107.3640625)(79.3625,107.1765625)
\curveto(79.03046875,106.934375)(78.76875,106.58085937)(78.57734375,106.11601562)
\curveto(78.3859375,105.65117187)(78.28632812,104.9890625)(78.27851562,104.1296875)
\curveto(78.53242187,104.51640625)(78.84296875,104.80351562)(79.21015625,104.99101562)
\curveto(79.57734375,105.17851562)(79.96210937,105.27226562)(80.36445312,105.27226562)
\curveto(81.06757812,105.27226562)(81.66523437,105.0125)(82.15742187,104.49296875)
\curveto(82.65351562,103.97734375)(82.9015625,103.309375)(82.9015625,102.4890625)
\curveto(82.9015625,101.95)(82.784375,101.44804687)(82.55,100.98320312)
\curveto(82.31953125,100.52226562)(82.00117187,100.16875)(81.59492187,99.92265625)
\curveto(81.18867187,99.6765625)(80.72773437,99.55351562)(80.21210937,99.55351562)
\curveto(79.33320312,99.55351562)(78.61640625,99.87578125)(78.06171875,100.5203125)
\curveto(77.50703125,101.16875)(77.2296875,102.23515625)(77.2296875,103.71953125)
\curveto(77.2296875,105.3796875)(77.53632812,106.58671875)(78.14960937,107.340625)
\curveto(78.68476562,107.996875)(79.40546875,108.325)(80.31171875,108.325)
\curveto(80.9875,108.325)(81.54023437,108.13554687)(81.96992187,107.75664062)
\curveto(82.40351562,107.37773437)(82.66328125,106.85429687)(82.74921875,106.18632812)
\closepath
\moveto(78.44257812,102.48320312)
\curveto(78.44257812,102.11992187)(78.51875,101.77226562)(78.67109375,101.44023437)
\curveto(78.82734375,101.10820312)(79.04414062,100.85429687)(79.32148437,100.67851562)
\curveto(79.59882812,100.50664062)(79.88984375,100.42070312)(80.19453125,100.42070312)
\curveto(80.63984375,100.42070312)(81.02265625,100.60039062)(81.34296875,100.95976562)
\curveto(81.66328125,101.31914062)(81.8234375,101.80742187)(81.8234375,102.42460937)
\curveto(81.8234375,103.01835937)(81.66523437,103.48515625)(81.34882812,103.825)
\curveto(81.03242187,104.16875)(80.63398437,104.340625)(80.15351562,104.340625)
\curveto(79.67695312,104.340625)(79.27265625,104.16875)(78.940625,103.825)
\curveto(78.60859375,103.48515625)(78.44257812,103.03789062)(78.44257812,102.48320312)
\closepath
}
}
{
\newrgbcolor{curcolor}{0 0 0}
\pscustom[linestyle=none,fillstyle=solid,fillcolor=curcolor]
{
\newpath
\moveto(89.49335937,100.71367187)
\lineto(89.49335937,99.7)
\lineto(83.815625,99.7)
\curveto(83.8078125,99.95390625)(83.84882812,100.19804687)(83.93867187,100.43242187)
\curveto(84.08320312,100.81914062)(84.31367187,101.2)(84.63007812,101.575)
\curveto(84.95039062,101.95)(85.41132812,102.38359375)(86.01289062,102.87578125)
\curveto(86.94648437,103.64140625)(87.57734375,104.246875)(87.90546875,104.6921875)
\curveto(88.23359375,105.14140625)(88.39765625,105.56523437)(88.39765625,105.96367187)
\curveto(88.39765625,106.38164062)(88.24726562,106.73320312)(87.94648437,107.01835937)
\curveto(87.64960937,107.30742187)(87.2609375,107.45195312)(86.78046875,107.45195312)
\curveto(86.27265625,107.45195312)(85.86640625,107.29960937)(85.56171875,106.99492187)
\curveto(85.25703125,106.69023437)(85.10273437,106.26835937)(85.09882812,105.72929687)
\lineto(84.01484375,105.840625)
\curveto(84.0890625,106.64921875)(84.36835937,107.26445312)(84.85273437,107.68632812)
\curveto(85.33710937,108.11210937)(85.9875,108.325)(86.80390625,108.325)
\curveto(87.628125,108.325)(88.28046875,108.09648437)(88.7609375,107.63945312)
\curveto(89.24140625,107.18242187)(89.48164062,106.61601562)(89.48164062,105.94023437)
\curveto(89.48164062,105.59648437)(89.41132812,105.25859375)(89.27070312,104.9265625)
\curveto(89.13007812,104.59453125)(88.89570312,104.24492187)(88.56757812,103.87773437)
\curveto(88.24335937,103.51054687)(87.70234375,103.00664062)(86.94453125,102.36601562)
\curveto(86.31171875,101.83476562)(85.90546875,101.4734375)(85.72578125,101.28203125)
\curveto(85.54609375,101.09453125)(85.39765625,100.90507812)(85.28046875,100.71367187)
\closepath
}
}
{
\newrgbcolor{curcolor}{0 0 0}
\pscustom[linestyle=none,fillstyle=solid,fillcolor=curcolor]
{
\newpath
\moveto(90.62421875,101.95)
\lineto(91.73164062,102.04375)
\curveto(91.81367187,101.5046875)(92.003125,101.0984375)(92.3,100.825)
\curveto(92.60078125,100.55546875)(92.96210937,100.42070312)(93.38398437,100.42070312)
\curveto(93.89179687,100.42070312)(94.32148437,100.61210937)(94.67304687,100.99492187)
\curveto(95.02460937,101.37773437)(95.20039062,101.88554687)(95.20039062,102.51835937)
\curveto(95.20039062,103.11992187)(95.03046875,103.59453125)(94.690625,103.9421875)
\curveto(94.3546875,104.28984375)(93.91328125,104.46367187)(93.36640625,104.46367187)
\curveto(93.0265625,104.46367187)(92.71992187,104.38554687)(92.44648437,104.22929687)
\curveto(92.17304687,104.07695312)(91.95820312,103.87773437)(91.80195312,103.63164062)
\lineto(90.81171875,103.76054687)
\lineto(91.64375,108.17265625)
\lineto(95.91523437,108.17265625)
\lineto(95.91523437,107.16484375)
\lineto(92.4875,107.16484375)
\lineto(92.02460937,104.85625)
\curveto(92.54023437,105.215625)(93.08125,105.3953125)(93.64765625,105.3953125)
\curveto(94.39765625,105.3953125)(95.03046875,105.13554687)(95.54609375,104.61601562)
\curveto(96.06171875,104.09648437)(96.31953125,103.42851562)(96.31953125,102.61210937)
\curveto(96.31953125,101.83476562)(96.09296875,101.16289062)(95.63984375,100.59648437)
\curveto(95.0890625,99.90117187)(94.33710937,99.55351562)(93.38398437,99.55351562)
\curveto(92.60273437,99.55351562)(91.9640625,99.77226562)(91.46796875,100.20976562)
\curveto(90.97578125,100.64726562)(90.69453125,101.22734375)(90.62421875,101.95)
\closepath
}
}
{
\newrgbcolor{curcolor}{0 0 0}
\pscustom[linewidth=1,linecolor=curcolor]
{
\newpath
\moveto(105.1,149.7)
\lineto(114.1,149.7)
\moveto(575,149.7)
\lineto(566,149.7)
}
}
{
\newrgbcolor{curcolor}{0 0 0}
\pscustom[linestyle=none,fillstyle=solid,fillcolor=curcolor]
{
\newpath
\moveto(67.26875,150.03632813)
\curveto(67.26875,151.05195313)(67.37226562,151.86835938)(67.57929687,152.48554688)
\curveto(67.79023437,153.10664063)(68.10078125,153.58515625)(68.5109375,153.92109375)
\curveto(68.925,154.25703125)(69.44453125,154.425)(70.06953125,154.425)
\curveto(70.53046875,154.425)(70.93476562,154.33125)(71.28242187,154.14375)
\curveto(71.63007812,153.96015625)(71.9171875,153.69257813)(72.14375,153.34101563)
\curveto(72.3703125,152.99335938)(72.54804687,152.56757813)(72.67695312,152.06367188)
\curveto(72.80585937,151.56367188)(72.8703125,150.88789063)(72.8703125,150.03632813)
\curveto(72.8703125,149.02851563)(72.76679687,148.2140625)(72.55976562,147.59296875)
\curveto(72.35273437,146.97578125)(72.0421875,146.49726563)(71.628125,146.15742188)
\curveto(71.21796875,145.82148438)(70.6984375,145.65351563)(70.06953125,145.65351563)
\curveto(69.24140625,145.65351563)(68.59101562,145.95039063)(68.11835937,146.54414063)
\curveto(67.55195312,147.25898438)(67.26875,148.42304688)(67.26875,150.03632813)
\closepath
\moveto(68.35273437,150.03632813)
\curveto(68.35273437,148.62617188)(68.51679687,147.68671875)(68.84492187,147.21796875)
\curveto(69.17695312,146.753125)(69.58515625,146.52070313)(70.06953125,146.52070313)
\curveto(70.55390625,146.52070313)(70.96015625,146.75507813)(71.28828125,147.22382813)
\curveto(71.6203125,147.69257813)(71.78632812,148.63007813)(71.78632812,150.03632813)
\curveto(71.78632812,151.45039063)(71.6203125,152.38984375)(71.28828125,152.8546875)
\curveto(70.96015625,153.31953125)(70.55,153.55195313)(70.0578125,153.55195313)
\curveto(69.5734375,153.55195313)(69.18671875,153.346875)(68.89765625,152.93671875)
\curveto(68.534375,152.41328125)(68.35273437,151.44648438)(68.35273437,150.03632813)
\closepath
}
}
{
\newrgbcolor{curcolor}{0 0 0}
\pscustom[linestyle=none,fillstyle=solid,fillcolor=curcolor]
{
\newpath
\moveto(74.534375,145.8)
\lineto(74.534375,147.00117188)
\lineto(75.73554687,147.00117188)
\lineto(75.73554687,145.8)
\closepath
}
}
{
\newrgbcolor{curcolor}{0 0 0}
\pscustom[linestyle=none,fillstyle=solid,fillcolor=curcolor]
{
\newpath
\moveto(81.24921875,145.8)
\lineto(80.19453125,145.8)
\lineto(80.19453125,152.52070313)
\curveto(79.940625,152.27851563)(79.60664062,152.03632813)(79.19257812,151.79414063)
\curveto(78.78242187,151.55195313)(78.41328125,151.3703125)(78.08515625,151.24921875)
\lineto(78.08515625,152.26875)
\curveto(78.675,152.54609375)(79.190625,152.88203125)(79.63203125,153.2765625)
\curveto(80.0734375,153.67109375)(80.3859375,154.05390625)(80.56953125,154.425)
\lineto(81.24921875,154.425)
\closepath
}
}
{
\newrgbcolor{curcolor}{0 0 0}
\pscustom[linestyle=none,fillstyle=solid,fillcolor=curcolor]
{
\newpath
\moveto(89.49335937,146.81367188)
\lineto(89.49335937,145.8)
\lineto(83.815625,145.8)
\curveto(83.8078125,146.05390625)(83.84882812,146.29804688)(83.93867187,146.53242188)
\curveto(84.08320312,146.91914063)(84.31367187,147.3)(84.63007812,147.675)
\curveto(84.95039062,148.05)(85.41132812,148.48359375)(86.01289062,148.97578125)
\curveto(86.94648437,149.74140625)(87.57734375,150.346875)(87.90546875,150.7921875)
\curveto(88.23359375,151.24140625)(88.39765625,151.66523438)(88.39765625,152.06367188)
\curveto(88.39765625,152.48164063)(88.24726562,152.83320313)(87.94648437,153.11835938)
\curveto(87.64960937,153.40742188)(87.2609375,153.55195313)(86.78046875,153.55195313)
\curveto(86.27265625,153.55195313)(85.86640625,153.39960938)(85.56171875,153.09492188)
\curveto(85.25703125,152.79023438)(85.10273437,152.36835938)(85.09882812,151.82929688)
\lineto(84.01484375,151.940625)
\curveto(84.0890625,152.74921875)(84.36835937,153.36445313)(84.85273437,153.78632813)
\curveto(85.33710937,154.21210938)(85.9875,154.425)(86.80390625,154.425)
\curveto(87.628125,154.425)(88.28046875,154.19648438)(88.7609375,153.73945313)
\curveto(89.24140625,153.28242188)(89.48164062,152.71601563)(89.48164062,152.04023438)
\curveto(89.48164062,151.69648438)(89.41132812,151.35859375)(89.27070312,151.0265625)
\curveto(89.13007812,150.69453125)(88.89570312,150.34492188)(88.56757812,149.97773438)
\curveto(88.24335937,149.61054688)(87.70234375,149.10664063)(86.94453125,148.46601563)
\curveto(86.31171875,147.93476563)(85.90546875,147.5734375)(85.72578125,147.38203125)
\curveto(85.54609375,147.19453125)(85.39765625,147.00507813)(85.28046875,146.81367188)
\closepath
}
}
{
\newrgbcolor{curcolor}{0 0 0}
\pscustom[linestyle=none,fillstyle=solid,fillcolor=curcolor]
{
\newpath
\moveto(90.62421875,148.05)
\lineto(91.73164062,148.14375)
\curveto(91.81367187,147.6046875)(92.003125,147.1984375)(92.3,146.925)
\curveto(92.60078125,146.65546875)(92.96210937,146.52070313)(93.38398437,146.52070313)
\curveto(93.89179687,146.52070313)(94.32148437,146.71210938)(94.67304687,147.09492188)
\curveto(95.02460937,147.47773438)(95.20039062,147.98554688)(95.20039062,148.61835938)
\curveto(95.20039062,149.21992188)(95.03046875,149.69453125)(94.690625,150.0421875)
\curveto(94.3546875,150.38984375)(93.91328125,150.56367188)(93.36640625,150.56367188)
\curveto(93.0265625,150.56367188)(92.71992187,150.48554688)(92.44648437,150.32929688)
\curveto(92.17304687,150.17695313)(91.95820312,149.97773438)(91.80195312,149.73164063)
\lineto(90.81171875,149.86054688)
\lineto(91.64375,154.27265625)
\lineto(95.91523437,154.27265625)
\lineto(95.91523437,153.26484375)
\lineto(92.4875,153.26484375)
\lineto(92.02460937,150.95625)
\curveto(92.54023437,151.315625)(93.08125,151.4953125)(93.64765625,151.4953125)
\curveto(94.39765625,151.4953125)(95.03046875,151.23554688)(95.54609375,150.71601563)
\curveto(96.06171875,150.19648438)(96.31953125,149.52851563)(96.31953125,148.71210938)
\curveto(96.31953125,147.93476563)(96.09296875,147.26289063)(95.63984375,146.69648438)
\curveto(95.0890625,146.00117188)(94.33710937,145.65351563)(93.38398437,145.65351563)
\curveto(92.60273437,145.65351563)(91.9640625,145.87226563)(91.46796875,146.30976563)
\curveto(90.97578125,146.74726563)(90.69453125,147.32734375)(90.62421875,148.05)
\closepath
}
}
{
\newrgbcolor{curcolor}{0 0 0}
\pscustom[linewidth=1,linecolor=curcolor]
{
\newpath
\moveto(105.1,195.7)
\lineto(114.1,195.7)
\moveto(575,195.7)
\lineto(566,195.7)
}
}
{
\newrgbcolor{curcolor}{0 0 0}
\pscustom[linestyle=none,fillstyle=solid,fillcolor=curcolor]
{
\newpath
\moveto(73.94257812,196.03632813)
\curveto(73.94257812,197.05195313)(74.04609375,197.86835938)(74.253125,198.48554688)
\curveto(74.4640625,199.10664063)(74.77460937,199.58515625)(75.18476562,199.92109375)
\curveto(75.59882812,200.25703125)(76.11835937,200.425)(76.74335937,200.425)
\curveto(77.20429687,200.425)(77.60859375,200.33125)(77.95625,200.14375)
\curveto(78.30390625,199.96015625)(78.59101562,199.69257813)(78.81757812,199.34101563)
\curveto(79.04414062,198.99335938)(79.221875,198.56757813)(79.35078125,198.06367188)
\curveto(79.4796875,197.56367188)(79.54414062,196.88789063)(79.54414062,196.03632813)
\curveto(79.54414062,195.02851563)(79.440625,194.2140625)(79.23359375,193.59296875)
\curveto(79.0265625,192.97578125)(78.71601562,192.49726563)(78.30195312,192.15742188)
\curveto(77.89179687,191.82148438)(77.37226562,191.65351563)(76.74335937,191.65351563)
\curveto(75.91523437,191.65351563)(75.26484375,191.95039063)(74.7921875,192.54414063)
\curveto(74.22578125,193.25898438)(73.94257812,194.42304688)(73.94257812,196.03632813)
\closepath
\moveto(75.0265625,196.03632813)
\curveto(75.0265625,194.62617188)(75.190625,193.68671875)(75.51875,193.21796875)
\curveto(75.85078125,192.753125)(76.25898437,192.52070313)(76.74335937,192.52070313)
\curveto(77.22773437,192.52070313)(77.63398437,192.75507813)(77.96210937,193.22382813)
\curveto(78.29414062,193.69257813)(78.46015625,194.63007813)(78.46015625,196.03632813)
\curveto(78.46015625,197.45039063)(78.29414062,198.38984375)(77.96210937,198.8546875)
\curveto(77.63398437,199.31953125)(77.22382812,199.55195313)(76.73164062,199.55195313)
\curveto(76.24726562,199.55195313)(75.86054687,199.346875)(75.57148437,198.93671875)
\curveto(75.20820312,198.41328125)(75.0265625,197.44648438)(75.0265625,196.03632813)
\closepath
}
}
{
\newrgbcolor{curcolor}{0 0 0}
\pscustom[linestyle=none,fillstyle=solid,fillcolor=curcolor]
{
\newpath
\moveto(81.20820312,191.8)
\lineto(81.20820312,193.00117188)
\lineto(82.409375,193.00117188)
\lineto(82.409375,191.8)
\closepath
}
}
{
\newrgbcolor{curcolor}{0 0 0}
\pscustom[linestyle=none,fillstyle=solid,fillcolor=curcolor]
{
\newpath
\moveto(89.49335937,192.81367188)
\lineto(89.49335937,191.8)
\lineto(83.815625,191.8)
\curveto(83.8078125,192.05390625)(83.84882812,192.29804688)(83.93867187,192.53242188)
\curveto(84.08320312,192.91914063)(84.31367187,193.3)(84.63007812,193.675)
\curveto(84.95039062,194.05)(85.41132812,194.48359375)(86.01289062,194.97578125)
\curveto(86.94648437,195.74140625)(87.57734375,196.346875)(87.90546875,196.7921875)
\curveto(88.23359375,197.24140625)(88.39765625,197.66523438)(88.39765625,198.06367188)
\curveto(88.39765625,198.48164063)(88.24726562,198.83320313)(87.94648437,199.11835938)
\curveto(87.64960937,199.40742188)(87.2609375,199.55195313)(86.78046875,199.55195313)
\curveto(86.27265625,199.55195313)(85.86640625,199.39960938)(85.56171875,199.09492188)
\curveto(85.25703125,198.79023438)(85.10273437,198.36835938)(85.09882812,197.82929688)
\lineto(84.01484375,197.940625)
\curveto(84.0890625,198.74921875)(84.36835937,199.36445313)(84.85273437,199.78632813)
\curveto(85.33710937,200.21210938)(85.9875,200.425)(86.80390625,200.425)
\curveto(87.628125,200.425)(88.28046875,200.19648438)(88.7609375,199.73945313)
\curveto(89.24140625,199.28242188)(89.48164062,198.71601563)(89.48164062,198.04023438)
\curveto(89.48164062,197.69648438)(89.41132812,197.35859375)(89.27070312,197.0265625)
\curveto(89.13007812,196.69453125)(88.89570312,196.34492188)(88.56757812,195.97773438)
\curveto(88.24335937,195.61054688)(87.70234375,195.10664063)(86.94453125,194.46601563)
\curveto(86.31171875,193.93476563)(85.90546875,193.5734375)(85.72578125,193.38203125)
\curveto(85.54609375,193.19453125)(85.39765625,193.00507813)(85.28046875,192.81367188)
\closepath
}
}
{
\newrgbcolor{curcolor}{0 0 0}
\pscustom[linestyle=none,fillstyle=solid,fillcolor=curcolor]
{
\newpath
\moveto(90.62421875,194.05)
\lineto(91.73164062,194.14375)
\curveto(91.81367187,193.6046875)(92.003125,193.1984375)(92.3,192.925)
\curveto(92.60078125,192.65546875)(92.96210937,192.52070313)(93.38398437,192.52070313)
\curveto(93.89179687,192.52070313)(94.32148437,192.71210938)(94.67304687,193.09492188)
\curveto(95.02460937,193.47773438)(95.20039062,193.98554688)(95.20039062,194.61835938)
\curveto(95.20039062,195.21992188)(95.03046875,195.69453125)(94.690625,196.0421875)
\curveto(94.3546875,196.38984375)(93.91328125,196.56367188)(93.36640625,196.56367188)
\curveto(93.0265625,196.56367188)(92.71992187,196.48554688)(92.44648437,196.32929688)
\curveto(92.17304687,196.17695313)(91.95820312,195.97773438)(91.80195312,195.73164063)
\lineto(90.81171875,195.86054688)
\lineto(91.64375,200.27265625)
\lineto(95.91523437,200.27265625)
\lineto(95.91523437,199.26484375)
\lineto(92.4875,199.26484375)
\lineto(92.02460937,196.95625)
\curveto(92.54023437,197.315625)(93.08125,197.4953125)(93.64765625,197.4953125)
\curveto(94.39765625,197.4953125)(95.03046875,197.23554688)(95.54609375,196.71601563)
\curveto(96.06171875,196.19648438)(96.31953125,195.52851563)(96.31953125,194.71210938)
\curveto(96.31953125,193.93476563)(96.09296875,193.26289063)(95.63984375,192.69648438)
\curveto(95.0890625,192.00117188)(94.33710937,191.65351563)(93.38398437,191.65351563)
\curveto(92.60273437,191.65351563)(91.9640625,191.87226563)(91.46796875,192.30976563)
\curveto(90.97578125,192.74726563)(90.69453125,193.32734375)(90.62421875,194.05)
\closepath
}
}
{
\newrgbcolor{curcolor}{0 0 0}
\pscustom[linewidth=1,linecolor=curcolor]
{
\newpath
\moveto(105.1,241.8)
\lineto(114.1,241.8)
\moveto(575,241.8)
\lineto(566,241.8)
}
}
{
\newrgbcolor{curcolor}{0 0 0}
\pscustom[linestyle=none,fillstyle=solid,fillcolor=curcolor]
{
\newpath
\moveto(80.61640625,242.13632813)
\curveto(80.61640625,243.15195313)(80.71992187,243.96835938)(80.92695312,244.58554688)
\curveto(81.13789062,245.20664063)(81.4484375,245.68515625)(81.85859375,246.02109375)
\curveto(82.27265625,246.35703125)(82.7921875,246.525)(83.4171875,246.525)
\curveto(83.878125,246.525)(84.28242187,246.43125)(84.63007812,246.24375)
\curveto(84.97773437,246.06015625)(85.26484375,245.79257813)(85.49140625,245.44101563)
\curveto(85.71796875,245.09335938)(85.89570312,244.66757813)(86.02460937,244.16367188)
\curveto(86.15351562,243.66367188)(86.21796875,242.98789063)(86.21796875,242.13632813)
\curveto(86.21796875,241.12851563)(86.11445312,240.3140625)(85.90742187,239.69296875)
\curveto(85.70039062,239.07578125)(85.38984375,238.59726563)(84.97578125,238.25742188)
\curveto(84.565625,237.92148438)(84.04609375,237.75351563)(83.4171875,237.75351563)
\curveto(82.5890625,237.75351563)(81.93867187,238.05039063)(81.46601562,238.64414063)
\curveto(80.89960937,239.35898438)(80.61640625,240.52304688)(80.61640625,242.13632813)
\closepath
\moveto(81.70039062,242.13632813)
\curveto(81.70039062,240.72617188)(81.86445312,239.78671875)(82.19257812,239.31796875)
\curveto(82.52460937,238.853125)(82.9328125,238.62070313)(83.4171875,238.62070313)
\curveto(83.9015625,238.62070313)(84.3078125,238.85507813)(84.6359375,239.32382813)
\curveto(84.96796875,239.79257813)(85.13398437,240.73007813)(85.13398437,242.13632813)
\curveto(85.13398437,243.55039063)(84.96796875,244.48984375)(84.6359375,244.9546875)
\curveto(84.3078125,245.41953125)(83.89765625,245.65195313)(83.40546875,245.65195313)
\curveto(82.92109375,245.65195313)(82.534375,245.446875)(82.2453125,245.03671875)
\curveto(81.88203125,244.51328125)(81.70039062,243.54648438)(81.70039062,242.13632813)
\closepath
}
}
{
\newrgbcolor{curcolor}{0 0 0}
\pscustom[linestyle=none,fillstyle=solid,fillcolor=curcolor]
{
\newpath
\moveto(87.88203125,237.9)
\lineto(87.88203125,239.10117188)
\lineto(89.08320312,239.10117188)
\lineto(89.08320312,237.9)
\closepath
}
}
{
\newrgbcolor{curcolor}{0 0 0}
\pscustom[linestyle=none,fillstyle=solid,fillcolor=curcolor]
{
\newpath
\moveto(90.62421875,240.15)
\lineto(91.73164062,240.24375)
\curveto(91.81367187,239.7046875)(92.003125,239.2984375)(92.3,239.025)
\curveto(92.60078125,238.75546875)(92.96210937,238.62070313)(93.38398437,238.62070313)
\curveto(93.89179687,238.62070313)(94.32148437,238.81210938)(94.67304687,239.19492188)
\curveto(95.02460937,239.57773438)(95.20039062,240.08554688)(95.20039062,240.71835938)
\curveto(95.20039062,241.31992188)(95.03046875,241.79453125)(94.690625,242.1421875)
\curveto(94.3546875,242.48984375)(93.91328125,242.66367188)(93.36640625,242.66367188)
\curveto(93.0265625,242.66367188)(92.71992187,242.58554688)(92.44648437,242.42929688)
\curveto(92.17304687,242.27695313)(91.95820312,242.07773438)(91.80195312,241.83164063)
\lineto(90.81171875,241.96054688)
\lineto(91.64375,246.37265625)
\lineto(95.91523437,246.37265625)
\lineto(95.91523437,245.36484375)
\lineto(92.4875,245.36484375)
\lineto(92.02460937,243.05625)
\curveto(92.54023437,243.415625)(93.08125,243.5953125)(93.64765625,243.5953125)
\curveto(94.39765625,243.5953125)(95.03046875,243.33554688)(95.54609375,242.81601563)
\curveto(96.06171875,242.29648438)(96.31953125,241.62851563)(96.31953125,240.81210938)
\curveto(96.31953125,240.03476563)(96.09296875,239.36289063)(95.63984375,238.79648438)
\curveto(95.0890625,238.10117188)(94.33710937,237.75351563)(93.38398437,237.75351563)
\curveto(92.60273437,237.75351563)(91.9640625,237.97226563)(91.46796875,238.40976563)
\curveto(90.97578125,238.84726563)(90.69453125,239.42734375)(90.62421875,240.15)
\closepath
}
}
{
\newrgbcolor{curcolor}{0 0 0}
\pscustom[linewidth=1,linecolor=curcolor]
{
\newpath
\moveto(105.1,287.8)
\lineto(114.1,287.8)
\moveto(575,287.8)
\lineto(566,287.8)
}
}
{
\newrgbcolor{curcolor}{0 0 0}
\pscustom[linestyle=none,fillstyle=solid,fillcolor=curcolor]
{
\newpath
\moveto(94.596875,283.9)
\lineto(93.5421875,283.9)
\lineto(93.5421875,290.62070312)
\curveto(93.28828125,290.37851562)(92.95429687,290.13632812)(92.54023437,289.89414062)
\curveto(92.13007812,289.65195312)(91.7609375,289.4703125)(91.4328125,289.34921875)
\lineto(91.4328125,290.36875)
\curveto(92.02265625,290.64609375)(92.53828125,290.98203125)(92.9796875,291.3765625)
\curveto(93.42109375,291.77109375)(93.73359375,292.15390625)(93.9171875,292.525)
\lineto(94.596875,292.525)
\closepath
}
}
{
\newrgbcolor{curcolor}{0 0 0}
\pscustom[linewidth=1,linecolor=curcolor]
{
\newpath
\moveto(105.1,333.8)
\lineto(114.1,333.8)
\moveto(575,333.8)
\lineto(566,333.8)
}
}
{
\newrgbcolor{curcolor}{0 0 0}
\pscustom[linestyle=none,fillstyle=solid,fillcolor=curcolor]
{
\newpath
\moveto(96.1671875,330.91367187)
\lineto(96.1671875,329.9)
\lineto(90.48945312,329.9)
\curveto(90.48164062,330.15390625)(90.52265625,330.39804687)(90.6125,330.63242187)
\curveto(90.75703125,331.01914062)(90.9875,331.4)(91.30390625,331.775)
\curveto(91.62421875,332.15)(92.08515625,332.58359375)(92.68671875,333.07578125)
\curveto(93.6203125,333.84140625)(94.25117187,334.446875)(94.57929687,334.8921875)
\curveto(94.90742187,335.34140625)(95.07148437,335.76523437)(95.07148437,336.16367187)
\curveto(95.07148437,336.58164062)(94.92109375,336.93320312)(94.6203125,337.21835937)
\curveto(94.3234375,337.50742187)(93.93476562,337.65195312)(93.45429687,337.65195312)
\curveto(92.94648437,337.65195312)(92.54023437,337.49960937)(92.23554687,337.19492187)
\curveto(91.93085937,336.89023437)(91.7765625,336.46835937)(91.77265625,335.92929687)
\lineto(90.68867187,336.040625)
\curveto(90.76289062,336.84921875)(91.0421875,337.46445312)(91.5265625,337.88632812)
\curveto(92.0109375,338.31210937)(92.66132812,338.525)(93.47773437,338.525)
\curveto(94.30195312,338.525)(94.95429687,338.29648437)(95.43476562,337.83945312)
\curveto(95.91523437,337.38242187)(96.15546875,336.81601562)(96.15546875,336.14023437)
\curveto(96.15546875,335.79648437)(96.08515625,335.45859375)(95.94453125,335.1265625)
\curveto(95.80390625,334.79453125)(95.56953125,334.44492187)(95.24140625,334.07773437)
\curveto(94.9171875,333.71054687)(94.37617187,333.20664062)(93.61835937,332.56601562)
\curveto(92.98554687,332.03476562)(92.57929687,331.6734375)(92.39960937,331.48203125)
\curveto(92.21992187,331.29453125)(92.07148437,331.10507812)(91.95429687,330.91367187)
\closepath
}
}
{
\newrgbcolor{curcolor}{0 0 0}
\pscustom[linewidth=1,linecolor=curcolor]
{
\newpath
\moveto(105.1,379.9)
\lineto(114.1,379.9)
\moveto(575,379.9)
\lineto(566,379.9)
}
}
{
\newrgbcolor{curcolor}{0 0 0}
\pscustom[linestyle=none,fillstyle=solid,fillcolor=curcolor]
{
\newpath
\moveto(94.00507812,376)
\lineto(94.00507812,378.05664062)
\lineto(90.27851562,378.05664062)
\lineto(90.27851562,379.0234375)
\lineto(94.1984375,384.58984375)
\lineto(95.05976562,384.58984375)
\lineto(95.05976562,379.0234375)
\lineto(96.21992187,379.0234375)
\lineto(96.21992187,378.05664062)
\lineto(95.05976562,378.05664062)
\lineto(95.05976562,376)
\closepath
\moveto(94.00507812,379.0234375)
\lineto(94.00507812,382.89648438)
\lineto(91.315625,379.0234375)
\closepath
}
}
{
\newrgbcolor{curcolor}{0 0 0}
\pscustom[linewidth=1,linecolor=curcolor]
{
\newpath
\moveto(105.1,425.9)
\lineto(114.1,425.9)
\moveto(575,425.9)
\lineto(566,425.9)
}
}
{
\newrgbcolor{curcolor}{0 0 0}
\pscustom[linestyle=none,fillstyle=solid,fillcolor=curcolor]
{
\newpath
\moveto(92.24726562,426.65820312)
\curveto(91.80976562,426.81835938)(91.48554687,427.046875)(91.27460937,427.34375)
\curveto(91.06367187,427.640625)(90.95820312,427.99609375)(90.95820312,428.41015625)
\curveto(90.95820312,429.03515625)(91.1828125,429.56054688)(91.63203125,429.98632812)
\curveto(92.08125,430.41210938)(92.67890625,430.625)(93.425,430.625)
\curveto(94.175,430.625)(94.77851562,430.40625)(95.23554687,429.96875)
\curveto(95.69257812,429.53515625)(95.92109375,429.00585938)(95.92109375,428.38085938)
\curveto(95.92109375,427.98242188)(95.815625,427.63476562)(95.6046875,427.33789062)
\curveto(95.39765625,427.04492188)(95.08125,426.81835938)(94.65546875,426.65820312)
\curveto(95.1828125,426.48632812)(95.58320312,426.20898438)(95.85664062,425.82617188)
\curveto(96.13398437,425.44335938)(96.27265625,424.98632812)(96.27265625,424.45507812)
\curveto(96.27265625,423.72070312)(96.01289062,423.10351562)(95.49335937,422.60351562)
\curveto(94.97382812,422.10351562)(94.29023437,421.85351562)(93.44257812,421.85351562)
\curveto(92.59492187,421.85351562)(91.91132812,422.10351562)(91.39179687,422.60351562)
\curveto(90.87226562,423.10742188)(90.6125,423.734375)(90.6125,424.484375)
\curveto(90.6125,425.04296875)(90.753125,425.50976562)(91.034375,425.88476562)
\curveto(91.31953125,426.26367188)(91.72382812,426.52148438)(92.24726562,426.65820312)
\closepath
\moveto(92.03632812,428.4453125)
\curveto(92.03632812,428.0390625)(92.1671875,427.70703125)(92.42890625,427.44921875)
\curveto(92.690625,427.19140625)(93.03046875,427.0625)(93.4484375,427.0625)
\curveto(93.8546875,427.0625)(94.18671875,427.18945312)(94.44453125,427.44335938)
\curveto(94.70625,427.70117188)(94.83710937,428.015625)(94.83710937,428.38671875)
\curveto(94.83710937,428.7734375)(94.70234375,429.09765625)(94.4328125,429.359375)
\curveto(94.1671875,429.625)(93.83515625,429.7578125)(93.43671875,429.7578125)
\curveto(93.034375,429.7578125)(92.70039062,429.62890625)(92.43476562,429.37109375)
\curveto(92.16914062,429.11328125)(92.03632812,428.8046875)(92.03632812,428.4453125)
\closepath
\moveto(91.69648437,424.47851562)
\curveto(91.69648437,424.17773438)(91.76679687,423.88671875)(91.90742187,423.60546875)
\curveto(92.05195312,423.32421875)(92.26484375,423.10546875)(92.54609375,422.94921875)
\curveto(92.82734375,422.796875)(93.13007812,422.72070312)(93.45429687,422.72070312)
\curveto(93.95820312,422.72070312)(94.37421875,422.8828125)(94.70234375,423.20703125)
\curveto(95.03046875,423.53125)(95.19453125,423.94335938)(95.19453125,424.44335938)
\curveto(95.19453125,424.95117188)(95.02460937,425.37109375)(94.68476562,425.703125)
\curveto(94.34882812,426.03515625)(93.92695312,426.20117188)(93.41914062,426.20117188)
\curveto(92.92304687,426.20117188)(92.5109375,426.03710938)(92.1828125,425.70898438)
\curveto(91.85859375,425.38085938)(91.69648437,424.97070312)(91.69648437,424.47851562)
\closepath
}
}
{
\newrgbcolor{curcolor}{0 0 0}
\pscustom[linewidth=1,linecolor=curcolor]
{
\newpath
\moveto(105.1,57.6)
\lineto(105.1,66.6)
\moveto(105.1,425.9)
\lineto(105.1,416.9)
}
}
{
\newrgbcolor{curcolor}{0 0 0}
\pscustom[linestyle=none,fillstyle=solid,fillcolor=curcolor]
{
\newpath
\moveto(96.13222656,36.71367187)
\lineto(96.13222656,35.7)
\lineto(90.45449219,35.7)
\curveto(90.44667969,35.95390625)(90.48769531,36.19804687)(90.57753906,36.43242187)
\curveto(90.72207031,36.81914062)(90.95253906,37.2)(91.26894531,37.575)
\curveto(91.58925781,37.95)(92.05019531,38.38359375)(92.65175781,38.87578125)
\curveto(93.58535156,39.64140625)(94.21621094,40.246875)(94.54433594,40.6921875)
\curveto(94.87246094,41.14140625)(95.03652344,41.56523437)(95.03652344,41.96367187)
\curveto(95.03652344,42.38164062)(94.88613281,42.73320312)(94.58535156,43.01835937)
\curveto(94.28847656,43.30742187)(93.89980469,43.45195312)(93.41933594,43.45195312)
\curveto(92.91152344,43.45195312)(92.50527344,43.29960937)(92.20058594,42.99492187)
\curveto(91.89589844,42.69023437)(91.74160156,42.26835937)(91.73769531,41.72929687)
\lineto(90.65371094,41.840625)
\curveto(90.72792969,42.64921875)(91.00722656,43.26445312)(91.49160156,43.68632812)
\curveto(91.97597656,44.11210937)(92.62636719,44.325)(93.44277344,44.325)
\curveto(94.26699219,44.325)(94.91933594,44.09648437)(95.39980469,43.63945312)
\curveto(95.88027344,43.18242187)(96.12050781,42.61601562)(96.12050781,41.94023437)
\curveto(96.12050781,41.59648437)(96.05019531,41.25859375)(95.90957031,40.9265625)
\curveto(95.76894531,40.59453125)(95.53457031,40.24492187)(95.20644531,39.87773437)
\curveto(94.88222656,39.51054687)(94.34121094,39.00664062)(93.58339844,38.36601562)
\curveto(92.95058594,37.83476562)(92.54433594,37.4734375)(92.36464844,37.28203125)
\curveto(92.18496094,37.09453125)(92.03652344,36.90507812)(91.91933594,36.71367187)
\closepath
}
}
{
\newrgbcolor{curcolor}{0 0 0}
\pscustom[linestyle=none,fillstyle=solid,fillcolor=curcolor]
{
\newpath
\moveto(97.26308594,37.95)
\lineto(98.37050781,38.04375)
\curveto(98.45253906,37.5046875)(98.64199219,37.0984375)(98.93886719,36.825)
\curveto(99.23964844,36.55546875)(99.60097656,36.42070312)(100.02285156,36.42070312)
\curveto(100.53066406,36.42070312)(100.96035156,36.61210937)(101.31191406,36.99492187)
\curveto(101.66347656,37.37773437)(101.83925781,37.88554687)(101.83925781,38.51835937)
\curveto(101.83925781,39.11992187)(101.66933594,39.59453125)(101.32949219,39.9421875)
\curveto(100.99355469,40.28984375)(100.55214844,40.46367187)(100.00527344,40.46367187)
\curveto(99.66542969,40.46367187)(99.35878906,40.38554687)(99.08535156,40.22929687)
\curveto(98.81191406,40.07695312)(98.59707031,39.87773437)(98.44082031,39.63164062)
\lineto(97.45058594,39.76054687)
\lineto(98.28261719,44.17265625)
\lineto(102.55410156,44.17265625)
\lineto(102.55410156,43.16484375)
\lineto(99.12636719,43.16484375)
\lineto(98.66347656,40.85625)
\curveto(99.17910156,41.215625)(99.72011719,41.3953125)(100.28652344,41.3953125)
\curveto(101.03652344,41.3953125)(101.66933594,41.13554687)(102.18496094,40.61601562)
\curveto(102.70058594,40.09648437)(102.95839844,39.42851562)(102.95839844,38.61210937)
\curveto(102.95839844,37.83476562)(102.73183594,37.16289062)(102.27871094,36.59648437)
\curveto(101.72792969,35.90117187)(100.97597656,35.55351562)(100.02285156,35.55351562)
\curveto(99.24160156,35.55351562)(98.60292969,35.77226562)(98.10683594,36.20976562)
\curveto(97.61464844,36.64726562)(97.33339844,37.22734375)(97.26308594,37.95)
\closepath
}
}
{
\newrgbcolor{curcolor}{0 0 0}
\pscustom[linestyle=none,fillstyle=solid,fillcolor=curcolor]
{
\newpath
\moveto(103.93691406,39.93632812)
\curveto(103.93691406,40.95195312)(104.04042969,41.76835937)(104.24746094,42.38554687)
\curveto(104.45839844,43.00664062)(104.76894531,43.48515625)(105.17910156,43.82109375)
\curveto(105.59316406,44.15703125)(106.11269531,44.325)(106.73769531,44.325)
\curveto(107.19863281,44.325)(107.60292969,44.23125)(107.95058594,44.04375)
\curveto(108.29824219,43.86015625)(108.58535156,43.59257812)(108.81191406,43.24101562)
\curveto(109.03847656,42.89335937)(109.21621094,42.46757812)(109.34511719,41.96367187)
\curveto(109.47402344,41.46367187)(109.53847656,40.78789062)(109.53847656,39.93632812)
\curveto(109.53847656,38.92851562)(109.43496094,38.1140625)(109.22792969,37.49296875)
\curveto(109.02089844,36.87578125)(108.71035156,36.39726562)(108.29628906,36.05742187)
\curveto(107.88613281,35.72148437)(107.36660156,35.55351562)(106.73769531,35.55351562)
\curveto(105.90957031,35.55351562)(105.25917969,35.85039062)(104.78652344,36.44414062)
\curveto(104.22011719,37.15898437)(103.93691406,38.32304687)(103.93691406,39.93632812)
\closepath
\moveto(105.02089844,39.93632812)
\curveto(105.02089844,38.52617187)(105.18496094,37.58671875)(105.51308594,37.11796875)
\curveto(105.84511719,36.653125)(106.25332031,36.42070312)(106.73769531,36.42070312)
\curveto(107.22207031,36.42070312)(107.62832031,36.65507812)(107.95644531,37.12382812)
\curveto(108.28847656,37.59257812)(108.45449219,38.53007812)(108.45449219,39.93632812)
\curveto(108.45449219,41.35039062)(108.28847656,42.28984375)(107.95644531,42.7546875)
\curveto(107.62832031,43.21953125)(107.21816406,43.45195312)(106.72597656,43.45195312)
\curveto(106.24160156,43.45195312)(105.85488281,43.246875)(105.56582031,42.83671875)
\curveto(105.20253906,42.31328125)(105.02089844,41.34648437)(105.02089844,39.93632812)
\closepath
}
}
{
\newrgbcolor{curcolor}{0 0 0}
\pscustom[linestyle=none,fillstyle=solid,fillcolor=curcolor]
{
\newpath
\moveto(111.00332031,35.7)
\lineto(111.00332031,44.28984375)
\lineto(112.71425781,44.28984375)
\lineto(114.74746094,38.2078125)
\curveto(114.93496094,37.64140625)(115.07167969,37.21757812)(115.15761719,36.93632812)
\curveto(115.25527344,37.24882812)(115.40761719,37.7078125)(115.61464844,38.31328125)
\lineto(117.67128906,44.28984375)
\lineto(119.20058594,44.28984375)
\lineto(119.20058594,35.7)
\lineto(118.10488281,35.7)
\lineto(118.10488281,42.88945312)
\lineto(115.60878906,35.7)
\lineto(114.58339844,35.7)
\lineto(112.09902344,43.0125)
\lineto(112.09902344,35.7)
\closepath
}
}
{
\newrgbcolor{curcolor}{0 0 0}
\pscustom[linewidth=1,linecolor=curcolor]
{
\newpath
\moveto(183.4,57.6)
\lineto(183.4,66.6)
\moveto(183.4,425.9)
\lineto(183.4,416.9)
}
}
{
\newrgbcolor{curcolor}{0 0 0}
\pscustom[linestyle=none,fillstyle=solid,fillcolor=curcolor]
{
\newpath
\moveto(168.88925781,37.95)
\lineto(169.99667969,38.04375)
\curveto(170.07871094,37.5046875)(170.26816406,37.0984375)(170.56503906,36.825)
\curveto(170.86582031,36.55546875)(171.22714844,36.42070312)(171.64902344,36.42070312)
\curveto(172.15683594,36.42070312)(172.58652344,36.61210937)(172.93808594,36.99492187)
\curveto(173.28964844,37.37773437)(173.46542969,37.88554687)(173.46542969,38.51835937)
\curveto(173.46542969,39.11992187)(173.29550781,39.59453125)(172.95566406,39.9421875)
\curveto(172.61972656,40.28984375)(172.17832031,40.46367187)(171.63144531,40.46367187)
\curveto(171.29160156,40.46367187)(170.98496094,40.38554687)(170.71152344,40.22929687)
\curveto(170.43808594,40.07695312)(170.22324219,39.87773437)(170.06699219,39.63164062)
\lineto(169.07675781,39.76054687)
\lineto(169.90878906,44.17265625)
\lineto(174.18027344,44.17265625)
\lineto(174.18027344,43.16484375)
\lineto(170.75253906,43.16484375)
\lineto(170.28964844,40.85625)
\curveto(170.80527344,41.215625)(171.34628906,41.3953125)(171.91269531,41.3953125)
\curveto(172.66269531,41.3953125)(173.29550781,41.13554687)(173.81113281,40.61601562)
\curveto(174.32675781,40.09648437)(174.58457031,39.42851562)(174.58457031,38.61210937)
\curveto(174.58457031,37.83476562)(174.35800781,37.16289062)(173.90488281,36.59648437)
\curveto(173.35410156,35.90117187)(172.60214844,35.55351562)(171.64902344,35.55351562)
\curveto(170.86777344,35.55351562)(170.22910156,35.77226562)(169.73300781,36.20976562)
\curveto(169.24082031,36.64726562)(168.95957031,37.22734375)(168.88925781,37.95)
\closepath
}
}
{
\newrgbcolor{curcolor}{0 0 0}
\pscustom[linestyle=none,fillstyle=solid,fillcolor=curcolor]
{
\newpath
\moveto(175.56308594,39.93632812)
\curveto(175.56308594,40.95195312)(175.66660156,41.76835937)(175.87363281,42.38554687)
\curveto(176.08457031,43.00664062)(176.39511719,43.48515625)(176.80527344,43.82109375)
\curveto(177.21933594,44.15703125)(177.73886719,44.325)(178.36386719,44.325)
\curveto(178.82480469,44.325)(179.22910156,44.23125)(179.57675781,44.04375)
\curveto(179.92441406,43.86015625)(180.21152344,43.59257812)(180.43808594,43.24101562)
\curveto(180.66464844,42.89335937)(180.84238281,42.46757812)(180.97128906,41.96367187)
\curveto(181.10019531,41.46367187)(181.16464844,40.78789062)(181.16464844,39.93632812)
\curveto(181.16464844,38.92851562)(181.06113281,38.1140625)(180.85410156,37.49296875)
\curveto(180.64707031,36.87578125)(180.33652344,36.39726562)(179.92246094,36.05742187)
\curveto(179.51230469,35.72148437)(178.99277344,35.55351562)(178.36386719,35.55351562)
\curveto(177.53574219,35.55351562)(176.88535156,35.85039062)(176.41269531,36.44414062)
\curveto(175.84628906,37.15898437)(175.56308594,38.32304687)(175.56308594,39.93632812)
\closepath
\moveto(176.64707031,39.93632812)
\curveto(176.64707031,38.52617187)(176.81113281,37.58671875)(177.13925781,37.11796875)
\curveto(177.47128906,36.653125)(177.87949219,36.42070312)(178.36386719,36.42070312)
\curveto(178.84824219,36.42070312)(179.25449219,36.65507812)(179.58261719,37.12382812)
\curveto(179.91464844,37.59257812)(180.08066406,38.53007812)(180.08066406,39.93632812)
\curveto(180.08066406,41.35039062)(179.91464844,42.28984375)(179.58261719,42.7546875)
\curveto(179.25449219,43.21953125)(178.84433594,43.45195312)(178.35214844,43.45195312)
\curveto(177.86777344,43.45195312)(177.48105469,43.246875)(177.19199219,42.83671875)
\curveto(176.82871094,42.31328125)(176.64707031,41.34648437)(176.64707031,39.93632812)
\closepath
}
}
{
\newrgbcolor{curcolor}{0 0 0}
\pscustom[linestyle=none,fillstyle=solid,fillcolor=curcolor]
{
\newpath
\moveto(182.23691406,39.93632812)
\curveto(182.23691406,40.95195312)(182.34042969,41.76835937)(182.54746094,42.38554687)
\curveto(182.75839844,43.00664062)(183.06894531,43.48515625)(183.47910156,43.82109375)
\curveto(183.89316406,44.15703125)(184.41269531,44.325)(185.03769531,44.325)
\curveto(185.49863281,44.325)(185.90292969,44.23125)(186.25058594,44.04375)
\curveto(186.59824219,43.86015625)(186.88535156,43.59257812)(187.11191406,43.24101562)
\curveto(187.33847656,42.89335937)(187.51621094,42.46757812)(187.64511719,41.96367187)
\curveto(187.77402344,41.46367187)(187.83847656,40.78789062)(187.83847656,39.93632812)
\curveto(187.83847656,38.92851562)(187.73496094,38.1140625)(187.52792969,37.49296875)
\curveto(187.32089844,36.87578125)(187.01035156,36.39726562)(186.59628906,36.05742187)
\curveto(186.18613281,35.72148437)(185.66660156,35.55351562)(185.03769531,35.55351562)
\curveto(184.20957031,35.55351562)(183.55917969,35.85039062)(183.08652344,36.44414062)
\curveto(182.52011719,37.15898437)(182.23691406,38.32304687)(182.23691406,39.93632812)
\closepath
\moveto(183.32089844,39.93632812)
\curveto(183.32089844,38.52617187)(183.48496094,37.58671875)(183.81308594,37.11796875)
\curveto(184.14511719,36.653125)(184.55332031,36.42070312)(185.03769531,36.42070312)
\curveto(185.52207031,36.42070312)(185.92832031,36.65507812)(186.25644531,37.12382812)
\curveto(186.58847656,37.59257812)(186.75449219,38.53007812)(186.75449219,39.93632812)
\curveto(186.75449219,41.35039062)(186.58847656,42.28984375)(186.25644531,42.7546875)
\curveto(185.92832031,43.21953125)(185.51816406,43.45195312)(185.02597656,43.45195312)
\curveto(184.54160156,43.45195312)(184.15488281,43.246875)(183.86582031,42.83671875)
\curveto(183.50253906,42.31328125)(183.32089844,41.34648437)(183.32089844,39.93632812)
\closepath
}
}
{
\newrgbcolor{curcolor}{0 0 0}
\pscustom[linestyle=none,fillstyle=solid,fillcolor=curcolor]
{
\newpath
\moveto(189.30332031,35.7)
\lineto(189.30332031,44.28984375)
\lineto(191.01425781,44.28984375)
\lineto(193.04746094,38.2078125)
\curveto(193.23496094,37.64140625)(193.37167969,37.21757812)(193.45761719,36.93632812)
\curveto(193.55527344,37.24882812)(193.70761719,37.7078125)(193.91464844,38.31328125)
\lineto(195.97128906,44.28984375)
\lineto(197.50058594,44.28984375)
\lineto(197.50058594,35.7)
\lineto(196.40488281,35.7)
\lineto(196.40488281,42.88945312)
\lineto(193.90878906,35.7)
\lineto(192.88339844,35.7)
\lineto(190.39902344,43.0125)
\lineto(190.39902344,35.7)
\closepath
}
}
{
\newrgbcolor{curcolor}{0 0 0}
\pscustom[linewidth=1,linecolor=curcolor]
{
\newpath
\moveto(261.7,57.6)
\lineto(261.7,66.6)
\moveto(261.7,425.9)
\lineto(261.7,416.9)
}
}
{
\newrgbcolor{curcolor}{0 0 0}
\pscustom[linestyle=none,fillstyle=solid,fillcolor=curcolor]
{
\newpath
\moveto(258.16679687,35.7)
\lineto(257.11210937,35.7)
\lineto(257.11210937,42.42070312)
\curveto(256.85820312,42.17851562)(256.52421875,41.93632812)(256.11015625,41.69414062)
\curveto(255.7,41.45195312)(255.33085937,41.2703125)(255.00273437,41.14921875)
\lineto(255.00273437,42.16875)
\curveto(255.59257812,42.44609375)(256.10820312,42.78203125)(256.54960937,43.1765625)
\curveto(256.99101562,43.57109375)(257.30351562,43.95390625)(257.48710937,44.325)
\lineto(258.16679687,44.325)
\closepath
}
}
{
\newrgbcolor{curcolor}{0 0 0}
\pscustom[linestyle=none,fillstyle=solid,fillcolor=curcolor]
{
\newpath
\moveto(265.31523437,39.06914062)
\lineto(265.31523437,40.07695312)
\lineto(268.95390625,40.0828125)
\lineto(268.95390625,36.8953125)
\curveto(268.3953125,36.45)(267.81914062,36.1140625)(267.22539062,35.8875)
\curveto(266.63164062,35.66484375)(266.02226562,35.55351562)(265.39726562,35.55351562)
\curveto(264.55351562,35.55351562)(263.7859375,35.73320312)(263.09453125,36.09257812)
\curveto(262.40703125,36.45585937)(261.8875,36.97929687)(261.5359375,37.66289062)
\curveto(261.184375,38.34648437)(261.00859375,39.11015625)(261.00859375,39.95390625)
\curveto(261.00859375,40.78984375)(261.18242187,41.56914062)(261.53007812,42.29179687)
\curveto(261.88164062,43.01835937)(262.38554687,43.55742187)(263.04179687,43.90898437)
\curveto(263.69804687,44.26054687)(264.45390625,44.43632812)(265.309375,44.43632812)
\curveto(265.93046875,44.43632812)(266.49101562,44.33476562)(266.99101562,44.13164062)
\curveto(267.49492187,43.93242187)(267.88945312,43.653125)(268.17460937,43.29375)
\curveto(268.45976562,42.934375)(268.6765625,42.465625)(268.825,41.8875)
\lineto(267.79960937,41.60625)
\curveto(267.67070312,42.04375)(267.51054687,42.3875)(267.31914062,42.6375)
\curveto(267.12773437,42.8875)(266.85429687,43.08671875)(266.49882812,43.23515625)
\curveto(266.14335937,43.3875)(265.74882812,43.46367187)(265.31523437,43.46367187)
\curveto(264.79570312,43.46367187)(264.34648437,43.38359375)(263.96757812,43.2234375)
\curveto(263.58867187,43.0671875)(263.28203125,42.86015625)(263.04765625,42.60234375)
\curveto(262.8171875,42.34453125)(262.6375,42.06132812)(262.50859375,41.75273437)
\curveto(262.28984375,41.22148437)(262.18046875,40.6453125)(262.18046875,40.02421875)
\curveto(262.18046875,39.25859375)(262.31132812,38.61796875)(262.57304687,38.10234375)
\curveto(262.83867187,37.58671875)(263.2234375,37.20390625)(263.72734375,36.95390625)
\curveto(264.23125,36.70390625)(264.76640625,36.57890625)(265.3328125,36.57890625)
\curveto(265.825,36.57890625)(266.30546875,36.67265625)(266.77421875,36.86015625)
\curveto(267.24296875,37.0515625)(267.5984375,37.2546875)(267.840625,37.46953125)
\lineto(267.840625,39.06914062)
\closepath
}
}
{
\newrgbcolor{curcolor}{0 0 0}
\pscustom[linewidth=1,linecolor=curcolor]
{
\newpath
\moveto(340.1,57.6)
\lineto(340.1,66.6)
\moveto(340.1,425.9)
\lineto(340.1,416.9)
}
}
{
\newrgbcolor{curcolor}{0 0 0}
\pscustom[linestyle=none,fillstyle=solid,fillcolor=curcolor]
{
\newpath
\moveto(338.13710938,36.71367187)
\lineto(338.13710938,35.7)
\lineto(332.459375,35.7)
\curveto(332.4515625,35.95390625)(332.49257813,36.19804687)(332.58242188,36.43242187)
\curveto(332.72695313,36.81914062)(332.95742188,37.2)(333.27382813,37.575)
\curveto(333.59414063,37.95)(334.05507813,38.38359375)(334.65664063,38.87578125)
\curveto(335.59023438,39.64140625)(336.22109375,40.246875)(336.54921875,40.6921875)
\curveto(336.87734375,41.14140625)(337.04140625,41.56523437)(337.04140625,41.96367187)
\curveto(337.04140625,42.38164062)(336.89101563,42.73320312)(336.59023438,43.01835937)
\curveto(336.29335938,43.30742187)(335.9046875,43.45195312)(335.42421875,43.45195312)
\curveto(334.91640625,43.45195312)(334.51015625,43.29960937)(334.20546875,42.99492187)
\curveto(333.90078125,42.69023437)(333.74648438,42.26835937)(333.74257813,41.72929687)
\lineto(332.65859375,41.840625)
\curveto(332.7328125,42.64921875)(333.01210938,43.26445312)(333.49648438,43.68632812)
\curveto(333.98085938,44.11210937)(334.63125,44.325)(335.44765625,44.325)
\curveto(336.271875,44.325)(336.92421875,44.09648437)(337.4046875,43.63945312)
\curveto(337.88515625,43.18242187)(338.12539063,42.61601562)(338.12539063,41.94023437)
\curveto(338.12539063,41.59648437)(338.05507813,41.25859375)(337.91445313,40.9265625)
\curveto(337.77382813,40.59453125)(337.53945313,40.24492187)(337.21132813,39.87773437)
\curveto(336.88710938,39.51054687)(336.34609375,39.00664062)(335.58828125,38.36601562)
\curveto(334.95546875,37.83476562)(334.54921875,37.4734375)(334.36953125,37.28203125)
\curveto(334.18984375,37.09453125)(334.04140625,36.90507812)(333.92421875,36.71367187)
\closepath
}
}
{
\newrgbcolor{curcolor}{0 0 0}
\pscustom[linestyle=none,fillstyle=solid,fillcolor=curcolor]
{
\newpath
\moveto(343.71523438,39.06914062)
\lineto(343.71523438,40.07695312)
\lineto(347.35390625,40.0828125)
\lineto(347.35390625,36.8953125)
\curveto(346.7953125,36.45)(346.21914063,36.1140625)(345.62539063,35.8875)
\curveto(345.03164063,35.66484375)(344.42226563,35.55351562)(343.79726563,35.55351562)
\curveto(342.95351563,35.55351562)(342.1859375,35.73320312)(341.49453125,36.09257812)
\curveto(340.80703125,36.45585937)(340.2875,36.97929687)(339.9359375,37.66289062)
\curveto(339.584375,38.34648437)(339.40859375,39.11015625)(339.40859375,39.95390625)
\curveto(339.40859375,40.78984375)(339.58242188,41.56914062)(339.93007813,42.29179687)
\curveto(340.28164063,43.01835937)(340.78554688,43.55742187)(341.44179688,43.90898437)
\curveto(342.09804688,44.26054687)(342.85390625,44.43632812)(343.709375,44.43632812)
\curveto(344.33046875,44.43632812)(344.89101563,44.33476562)(345.39101563,44.13164062)
\curveto(345.89492188,43.93242187)(346.28945313,43.653125)(346.57460938,43.29375)
\curveto(346.85976563,42.934375)(347.0765625,42.465625)(347.225,41.8875)
\lineto(346.19960938,41.60625)
\curveto(346.07070313,42.04375)(345.91054688,42.3875)(345.71914063,42.6375)
\curveto(345.52773438,42.8875)(345.25429688,43.08671875)(344.89882813,43.23515625)
\curveto(344.54335938,43.3875)(344.14882813,43.46367187)(343.71523438,43.46367187)
\curveto(343.19570313,43.46367187)(342.74648438,43.38359375)(342.36757813,43.2234375)
\curveto(341.98867188,43.0671875)(341.68203125,42.86015625)(341.44765625,42.60234375)
\curveto(341.2171875,42.34453125)(341.0375,42.06132812)(340.90859375,41.75273437)
\curveto(340.68984375,41.22148437)(340.58046875,40.6453125)(340.58046875,40.02421875)
\curveto(340.58046875,39.25859375)(340.71132813,38.61796875)(340.97304688,38.10234375)
\curveto(341.23867188,37.58671875)(341.6234375,37.20390625)(342.12734375,36.95390625)
\curveto(342.63125,36.70390625)(343.16640625,36.57890625)(343.7328125,36.57890625)
\curveto(344.225,36.57890625)(344.70546875,36.67265625)(345.17421875,36.86015625)
\curveto(345.64296875,37.0515625)(345.9984375,37.2546875)(346.240625,37.46953125)
\lineto(346.240625,39.06914062)
\closepath
}
}
{
\newrgbcolor{curcolor}{0 0 0}
\pscustom[linewidth=1,linecolor=curcolor]
{
\newpath
\moveto(418.4,57.6)
\lineto(418.4,66.6)
\moveto(418.4,425.9)
\lineto(418.4,416.9)
}
}
{
\newrgbcolor{curcolor}{0 0 0}
\pscustom[linestyle=none,fillstyle=solid,fillcolor=curcolor]
{
\newpath
\moveto(414.275,35.7)
\lineto(414.275,37.75664062)
\lineto(410.5484375,37.75664062)
\lineto(410.5484375,38.7234375)
\lineto(414.46835937,44.28984375)
\lineto(415.3296875,44.28984375)
\lineto(415.3296875,38.7234375)
\lineto(416.48984375,38.7234375)
\lineto(416.48984375,37.75664062)
\lineto(415.3296875,37.75664062)
\lineto(415.3296875,35.7)
\closepath
\moveto(414.275,38.7234375)
\lineto(414.275,42.59648437)
\lineto(411.58554687,38.7234375)
\closepath
}
}
{
\newrgbcolor{curcolor}{0 0 0}
\pscustom[linestyle=none,fillstyle=solid,fillcolor=curcolor]
{
\newpath
\moveto(422.01523437,39.06914062)
\lineto(422.01523437,40.07695312)
\lineto(425.65390625,40.0828125)
\lineto(425.65390625,36.8953125)
\curveto(425.0953125,36.45)(424.51914062,36.1140625)(423.92539062,35.8875)
\curveto(423.33164062,35.66484375)(422.72226562,35.55351562)(422.09726562,35.55351562)
\curveto(421.25351562,35.55351562)(420.4859375,35.73320312)(419.79453125,36.09257812)
\curveto(419.10703125,36.45585937)(418.5875,36.97929687)(418.2359375,37.66289062)
\curveto(417.884375,38.34648437)(417.70859375,39.11015625)(417.70859375,39.95390625)
\curveto(417.70859375,40.78984375)(417.88242187,41.56914062)(418.23007812,42.29179687)
\curveto(418.58164062,43.01835937)(419.08554687,43.55742187)(419.74179687,43.90898437)
\curveto(420.39804687,44.26054687)(421.15390625,44.43632812)(422.009375,44.43632812)
\curveto(422.63046875,44.43632812)(423.19101562,44.33476562)(423.69101562,44.13164062)
\curveto(424.19492187,43.93242187)(424.58945312,43.653125)(424.87460937,43.29375)
\curveto(425.15976562,42.934375)(425.3765625,42.465625)(425.525,41.8875)
\lineto(424.49960937,41.60625)
\curveto(424.37070312,42.04375)(424.21054687,42.3875)(424.01914062,42.6375)
\curveto(423.82773437,42.8875)(423.55429687,43.08671875)(423.19882812,43.23515625)
\curveto(422.84335937,43.3875)(422.44882812,43.46367187)(422.01523437,43.46367187)
\curveto(421.49570312,43.46367187)(421.04648437,43.38359375)(420.66757812,43.2234375)
\curveto(420.28867187,43.0671875)(419.98203125,42.86015625)(419.74765625,42.60234375)
\curveto(419.5171875,42.34453125)(419.3375,42.06132812)(419.20859375,41.75273437)
\curveto(418.98984375,41.22148437)(418.88046875,40.6453125)(418.88046875,40.02421875)
\curveto(418.88046875,39.25859375)(419.01132812,38.61796875)(419.27304687,38.10234375)
\curveto(419.53867187,37.58671875)(419.9234375,37.20390625)(420.42734375,36.95390625)
\curveto(420.93125,36.70390625)(421.46640625,36.57890625)(422.0328125,36.57890625)
\curveto(422.525,36.57890625)(423.00546875,36.67265625)(423.47421875,36.86015625)
\curveto(423.94296875,37.0515625)(424.2984375,37.2546875)(424.540625,37.46953125)
\lineto(424.540625,39.06914062)
\closepath
}
}
{
\newrgbcolor{curcolor}{0 0 0}
\pscustom[linewidth=1,linecolor=curcolor]
{
\newpath
\moveto(496.7,57.6)
\lineto(496.7,66.6)
\moveto(496.7,425.9)
\lineto(496.7,416.9)
}
}
{
\newrgbcolor{curcolor}{0 0 0}
\pscustom[linestyle=none,fillstyle=solid,fillcolor=curcolor]
{
\newpath
\moveto(490.8171875,40.35820312)
\curveto(490.3796875,40.51835937)(490.05546875,40.746875)(489.84453125,41.04375)
\curveto(489.63359375,41.340625)(489.528125,41.69609375)(489.528125,42.11015625)
\curveto(489.528125,42.73515625)(489.75273437,43.26054687)(490.20195312,43.68632812)
\curveto(490.65117187,44.11210937)(491.24882812,44.325)(491.99492187,44.325)
\curveto(492.74492187,44.325)(493.3484375,44.10625)(493.80546875,43.66875)
\curveto(494.2625,43.23515625)(494.49101562,42.70585937)(494.49101562,42.08085937)
\curveto(494.49101562,41.68242187)(494.38554687,41.33476562)(494.17460937,41.03789062)
\curveto(493.96757812,40.74492187)(493.65117187,40.51835937)(493.22539062,40.35820312)
\curveto(493.75273437,40.18632812)(494.153125,39.90898437)(494.4265625,39.52617187)
\curveto(494.70390625,39.14335937)(494.84257812,38.68632812)(494.84257812,38.15507812)
\curveto(494.84257812,37.42070312)(494.5828125,36.80351562)(494.06328125,36.30351562)
\curveto(493.54375,35.80351562)(492.86015625,35.55351562)(492.0125,35.55351562)
\curveto(491.16484375,35.55351562)(490.48125,35.80351562)(489.96171875,36.30351562)
\curveto(489.4421875,36.80742187)(489.18242187,37.434375)(489.18242187,38.184375)
\curveto(489.18242187,38.74296875)(489.32304687,39.20976562)(489.60429687,39.58476562)
\curveto(489.88945312,39.96367187)(490.29375,40.22148437)(490.8171875,40.35820312)
\closepath
\moveto(490.60625,42.1453125)
\curveto(490.60625,41.7390625)(490.73710937,41.40703125)(490.99882812,41.14921875)
\curveto(491.26054687,40.89140625)(491.60039062,40.7625)(492.01835937,40.7625)
\curveto(492.42460937,40.7625)(492.75664062,40.88945312)(493.01445312,41.14335937)
\curveto(493.27617187,41.40117187)(493.40703125,41.715625)(493.40703125,42.08671875)
\curveto(493.40703125,42.4734375)(493.27226562,42.79765625)(493.00273437,43.059375)
\curveto(492.73710937,43.325)(492.40507812,43.4578125)(492.00664062,43.4578125)
\curveto(491.60429687,43.4578125)(491.2703125,43.32890625)(491.0046875,43.07109375)
\curveto(490.7390625,42.81328125)(490.60625,42.5046875)(490.60625,42.1453125)
\closepath
\moveto(490.26640625,38.17851562)
\curveto(490.26640625,37.87773437)(490.33671875,37.58671875)(490.47734375,37.30546875)
\curveto(490.621875,37.02421875)(490.83476562,36.80546875)(491.11601562,36.64921875)
\curveto(491.39726562,36.496875)(491.7,36.42070312)(492.02421875,36.42070312)
\curveto(492.528125,36.42070312)(492.94414062,36.5828125)(493.27226562,36.90703125)
\curveto(493.60039062,37.23125)(493.76445312,37.64335937)(493.76445312,38.14335937)
\curveto(493.76445312,38.65117187)(493.59453125,39.07109375)(493.2546875,39.403125)
\curveto(492.91875,39.73515625)(492.496875,39.90117187)(491.9890625,39.90117187)
\curveto(491.49296875,39.90117187)(491.08085937,39.73710937)(490.75273437,39.40898437)
\curveto(490.42851562,39.08085937)(490.26640625,38.67070312)(490.26640625,38.17851562)
\closepath
}
}
{
\newrgbcolor{curcolor}{0 0 0}
\pscustom[linestyle=none,fillstyle=solid,fillcolor=curcolor]
{
\newpath
\moveto(500.31523437,39.06914062)
\lineto(500.31523437,40.07695312)
\lineto(503.95390625,40.0828125)
\lineto(503.95390625,36.8953125)
\curveto(503.3953125,36.45)(502.81914062,36.1140625)(502.22539062,35.8875)
\curveto(501.63164062,35.66484375)(501.02226562,35.55351562)(500.39726562,35.55351562)
\curveto(499.55351562,35.55351562)(498.7859375,35.73320312)(498.09453125,36.09257812)
\curveto(497.40703125,36.45585937)(496.8875,36.97929687)(496.5359375,37.66289062)
\curveto(496.184375,38.34648437)(496.00859375,39.11015625)(496.00859375,39.95390625)
\curveto(496.00859375,40.78984375)(496.18242187,41.56914062)(496.53007812,42.29179687)
\curveto(496.88164062,43.01835937)(497.38554687,43.55742187)(498.04179687,43.90898437)
\curveto(498.69804687,44.26054687)(499.45390625,44.43632812)(500.309375,44.43632812)
\curveto(500.93046875,44.43632812)(501.49101562,44.33476562)(501.99101562,44.13164062)
\curveto(502.49492187,43.93242187)(502.88945312,43.653125)(503.17460937,43.29375)
\curveto(503.45976562,42.934375)(503.6765625,42.465625)(503.825,41.8875)
\lineto(502.79960937,41.60625)
\curveto(502.67070312,42.04375)(502.51054687,42.3875)(502.31914062,42.6375)
\curveto(502.12773437,42.8875)(501.85429687,43.08671875)(501.49882812,43.23515625)
\curveto(501.14335937,43.3875)(500.74882812,43.46367187)(500.31523437,43.46367187)
\curveto(499.79570312,43.46367187)(499.34648437,43.38359375)(498.96757812,43.2234375)
\curveto(498.58867187,43.0671875)(498.28203125,42.86015625)(498.04765625,42.60234375)
\curveto(497.8171875,42.34453125)(497.6375,42.06132812)(497.50859375,41.75273437)
\curveto(497.28984375,41.22148437)(497.18046875,40.6453125)(497.18046875,40.02421875)
\curveto(497.18046875,39.25859375)(497.31132812,38.61796875)(497.57304687,38.10234375)
\curveto(497.83867187,37.58671875)(498.2234375,37.20390625)(498.72734375,36.95390625)
\curveto(499.23125,36.70390625)(499.76640625,36.57890625)(500.3328125,36.57890625)
\curveto(500.825,36.57890625)(501.30546875,36.67265625)(501.77421875,36.86015625)
\curveto(502.24296875,37.0515625)(502.5984375,37.2546875)(502.840625,37.46953125)
\lineto(502.840625,39.06914062)
\closepath
}
}
{
\newrgbcolor{curcolor}{0 0 0}
\pscustom[linewidth=1,linecolor=curcolor]
{
\newpath
\moveto(575,57.6)
\lineto(575,66.6)
\moveto(575,425.9)
\lineto(575,416.9)
}
}
{
\newrgbcolor{curcolor}{0 0 0}
\pscustom[linestyle=none,fillstyle=solid,fillcolor=curcolor]
{
\newpath
\moveto(568.12988281,35.7)
\lineto(567.07519531,35.7)
\lineto(567.07519531,42.42070312)
\curveto(566.82128906,42.17851562)(566.48730469,41.93632812)(566.07324219,41.69414062)
\curveto(565.66308594,41.45195312)(565.29394531,41.2703125)(564.96582031,41.14921875)
\lineto(564.96582031,42.16875)
\curveto(565.55566406,42.44609375)(566.07128906,42.78203125)(566.51269531,43.1765625)
\curveto(566.95410156,43.57109375)(567.26660156,43.95390625)(567.45019531,44.325)
\lineto(568.12988281,44.325)
\closepath
}
}
{
\newrgbcolor{curcolor}{0 0 0}
\pscustom[linestyle=none,fillstyle=solid,fillcolor=curcolor]
{
\newpath
\moveto(576.30371094,42.18632812)
\lineto(575.25488281,42.10429687)
\curveto(575.16113281,42.51835937)(575.02832031,42.81914062)(574.85644531,43.00664062)
\curveto(574.57128906,43.30742187)(574.21972656,43.4578125)(573.80175781,43.4578125)
\curveto(573.46582031,43.4578125)(573.17089844,43.3640625)(572.91699219,43.1765625)
\curveto(572.58496094,42.934375)(572.32324219,42.58085937)(572.13183594,42.11601562)
\curveto(571.94042969,41.65117187)(571.84082031,40.9890625)(571.83300781,40.1296875)
\curveto(572.08691406,40.51640625)(572.39746094,40.80351562)(572.76464844,40.99101562)
\curveto(573.13183594,41.17851562)(573.51660156,41.27226562)(573.91894531,41.27226562)
\curveto(574.62207031,41.27226562)(575.21972656,41.0125)(575.71191406,40.49296875)
\curveto(576.20800781,39.97734375)(576.45605469,39.309375)(576.45605469,38.4890625)
\curveto(576.45605469,37.95)(576.33886719,37.44804687)(576.10449219,36.98320312)
\curveto(575.87402344,36.52226562)(575.55566406,36.16875)(575.14941406,35.92265625)
\curveto(574.74316406,35.6765625)(574.28222656,35.55351562)(573.76660156,35.55351562)
\curveto(572.88769531,35.55351562)(572.17089844,35.87578125)(571.61621094,36.5203125)
\curveto(571.06152344,37.16875)(570.78417969,38.23515625)(570.78417969,39.71953125)
\curveto(570.78417969,41.3796875)(571.09082031,42.58671875)(571.70410156,43.340625)
\curveto(572.23925781,43.996875)(572.95996094,44.325)(573.86621094,44.325)
\curveto(574.54199219,44.325)(575.09472656,44.13554687)(575.52441406,43.75664062)
\curveto(575.95800781,43.37773437)(576.21777344,42.85429687)(576.30371094,42.18632812)
\closepath
\moveto(571.99707031,38.48320312)
\curveto(571.99707031,38.11992187)(572.07324219,37.77226562)(572.22558594,37.44023437)
\curveto(572.38183594,37.10820312)(572.59863281,36.85429687)(572.87597656,36.67851562)
\curveto(573.15332031,36.50664062)(573.44433594,36.42070312)(573.74902344,36.42070312)
\curveto(574.19433594,36.42070312)(574.57714844,36.60039062)(574.89746094,36.95976562)
\curveto(575.21777344,37.31914062)(575.37792969,37.80742187)(575.37792969,38.42460937)
\curveto(575.37792969,39.01835937)(575.21972656,39.48515625)(574.90332031,39.825)
\curveto(574.58691406,40.16875)(574.18847656,40.340625)(573.70800781,40.340625)
\curveto(573.23144531,40.340625)(572.82714844,40.16875)(572.49511719,39.825)
\curveto(572.16308594,39.48515625)(571.99707031,39.03789062)(571.99707031,38.48320312)
\closepath
}
}
{
\newrgbcolor{curcolor}{0 0 0}
\pscustom[linestyle=none,fillstyle=solid,fillcolor=curcolor]
{
\newpath
\moveto(581.95214844,39.06914062)
\lineto(581.95214844,40.07695312)
\lineto(585.59082031,40.0828125)
\lineto(585.59082031,36.8953125)
\curveto(585.03222656,36.45)(584.45605469,36.1140625)(583.86230469,35.8875)
\curveto(583.26855469,35.66484375)(582.65917969,35.55351562)(582.03417969,35.55351562)
\curveto(581.19042969,35.55351562)(580.42285156,35.73320312)(579.73144531,36.09257812)
\curveto(579.04394531,36.45585937)(578.52441406,36.97929687)(578.17285156,37.66289062)
\curveto(577.82128906,38.34648437)(577.64550781,39.11015625)(577.64550781,39.95390625)
\curveto(577.64550781,40.78984375)(577.81933594,41.56914062)(578.16699219,42.29179687)
\curveto(578.51855469,43.01835937)(579.02246094,43.55742187)(579.67871094,43.90898437)
\curveto(580.33496094,44.26054687)(581.09082031,44.43632812)(581.94628906,44.43632812)
\curveto(582.56738281,44.43632812)(583.12792969,44.33476562)(583.62792969,44.13164062)
\curveto(584.13183594,43.93242187)(584.52636719,43.653125)(584.81152344,43.29375)
\curveto(585.09667969,42.934375)(585.31347656,42.465625)(585.46191406,41.8875)
\lineto(584.43652344,41.60625)
\curveto(584.30761719,42.04375)(584.14746094,42.3875)(583.95605469,42.6375)
\curveto(583.76464844,42.8875)(583.49121094,43.08671875)(583.13574219,43.23515625)
\curveto(582.78027344,43.3875)(582.38574219,43.46367187)(581.95214844,43.46367187)
\curveto(581.43261719,43.46367187)(580.98339844,43.38359375)(580.60449219,43.2234375)
\curveto(580.22558594,43.0671875)(579.91894531,42.86015625)(579.68457031,42.60234375)
\curveto(579.45410156,42.34453125)(579.27441406,42.06132812)(579.14550781,41.75273437)
\curveto(578.92675781,41.22148437)(578.81738281,40.6453125)(578.81738281,40.02421875)
\curveto(578.81738281,39.25859375)(578.94824219,38.61796875)(579.20996094,38.10234375)
\curveto(579.47558594,37.58671875)(579.86035156,37.20390625)(580.36425781,36.95390625)
\curveto(580.86816406,36.70390625)(581.40332031,36.57890625)(581.96972656,36.57890625)
\curveto(582.46191406,36.57890625)(582.94238281,36.67265625)(583.41113281,36.86015625)
\curveto(583.87988281,37.0515625)(584.23535156,37.2546875)(584.47753906,37.46953125)
\lineto(584.47753906,39.06914062)
\closepath
}
}
{
\newrgbcolor{curcolor}{0 0 0}
\pscustom[linewidth=1,linecolor=curcolor]
{
\newpath
\moveto(105.1,425.9)
\lineto(105.1,57.6)
\lineto(575,57.6)
\lineto(575,425.9)
\closepath
}
}
{
\newrgbcolor{curcolor}{0 0 0}
\pscustom[linestyle=none,fillstyle=solid,fillcolor=curcolor]
{
\newpath
\moveto(16.3,192.29082031)
\lineto(7.71015625,192.29082031)
\lineto(7.71015625,196.09941406)
\curveto(7.71015625,196.86503906)(7.78828125,197.44707031)(7.94453125,197.84550781)
\curveto(8.096875,198.24394531)(8.36835938,198.56230469)(8.75898438,198.80058594)
\curveto(9.14960938,199.03886719)(9.58125,199.15800781)(10.05390625,199.15800781)
\curveto(10.66328125,199.15800781)(11.17695313,198.96074219)(11.59492188,198.56621094)
\curveto(12.01289063,198.17167969)(12.27851563,197.56230469)(12.39179688,196.73808594)
\curveto(12.53632813,197.03886719)(12.67890625,197.26738281)(12.81953125,197.42363281)
\curveto(13.12421875,197.75566406)(13.50507813,198.07011719)(13.96210938,198.36699219)
\lineto(16.3,199.86113281)
\lineto(16.3,198.43144531)
\lineto(14.51289063,197.29472656)
\curveto(13.99726563,196.96269531)(13.60273438,196.68925781)(13.32929688,196.47441406)
\curveto(13.05585938,196.25957031)(12.86445313,196.06621094)(12.75507813,195.89433594)
\curveto(12.64570313,195.72636719)(12.56953125,195.55449219)(12.5265625,195.37871094)
\curveto(12.49921875,195.24980469)(12.48554688,195.03886719)(12.48554688,194.74589844)
\lineto(12.48554688,193.42753906)
\lineto(16.3,193.42753906)
\closepath
\moveto(11.50117188,193.42753906)
\lineto(11.50117188,195.87089844)
\curveto(11.50117188,196.39042969)(11.4484375,196.79667969)(11.34296875,197.08964844)
\curveto(11.23359375,197.38261719)(11.06171875,197.60527344)(10.82734375,197.75761719)
\curveto(10.5890625,197.90996094)(10.33125,197.98613281)(10.05390625,197.98613281)
\curveto(9.64765625,197.98613281)(9.31367188,197.83769531)(9.05195313,197.54082031)
\curveto(8.79023438,197.24785156)(8.659375,196.78300781)(8.659375,196.14628906)
\lineto(8.659375,193.42753906)
\closepath
}
}
{
\newrgbcolor{curcolor}{0 0 0}
\pscustom[linestyle=none,fillstyle=solid,fillcolor=curcolor]
{
\newpath
\moveto(16.3,204.88261719)
\lineto(15.3859375,204.88261719)
\curveto(16.0890625,204.39824219)(16.440625,203.74003906)(16.440625,202.90800781)
\curveto(16.440625,202.54082031)(16.3703125,202.19707031)(16.2296875,201.87675781)
\curveto(16.0890625,201.56035156)(15.91328125,201.32402344)(15.70234375,201.16777344)
\curveto(15.4875,201.01542969)(15.22578125,200.90800781)(14.9171875,200.84550781)
\curveto(14.71015625,200.80253906)(14.38203125,200.78105469)(13.9328125,200.78105469)
\lineto(10.07734375,200.78105469)
\lineto(10.07734375,201.83574219)
\lineto(13.52851563,201.83574219)
\curveto(14.07929688,201.83574219)(14.45039063,201.85722656)(14.64179688,201.90019531)
\curveto(14.91914063,201.96660156)(15.13789063,202.10722656)(15.29804688,202.32207031)
\curveto(15.45429688,202.53691406)(15.53242188,202.80253906)(15.53242188,203.11894531)
\curveto(15.53242188,203.43535156)(15.45234375,203.73222656)(15.2921875,204.00957031)
\curveto(15.128125,204.28691406)(14.90742188,204.48222656)(14.63007813,204.59550781)
\curveto(14.34882813,204.71269531)(13.94257813,204.77128906)(13.41132813,204.77128906)
\lineto(10.07734375,204.77128906)
\lineto(10.07734375,205.82597656)
\lineto(16.3,205.82597656)
\closepath
}
}
{
\newrgbcolor{curcolor}{0 0 0}
\pscustom[linestyle=none,fillstyle=solid,fillcolor=curcolor]
{
\newpath
\moveto(16.3,207.47832031)
\lineto(10.07734375,207.47832031)
\lineto(10.07734375,208.42753906)
\lineto(10.96210938,208.42753906)
\curveto(10.27851563,208.88457031)(9.93671875,209.54472656)(9.93671875,210.40800781)
\curveto(9.93671875,210.78300781)(10.00507813,211.12675781)(10.14179688,211.43925781)
\curveto(10.27460938,211.75566406)(10.45039063,211.99199219)(10.66914063,212.14824219)
\curveto(10.88789063,212.30449219)(11.14765625,212.41386719)(11.4484375,212.47636719)
\curveto(11.64375,212.51542969)(11.98554688,212.53496094)(12.47382813,212.53496094)
\lineto(16.3,212.53496094)
\lineto(16.3,211.48027344)
\lineto(12.51484375,211.48027344)
\curveto(12.08515625,211.48027344)(11.76484375,211.43925781)(11.55390625,211.35722656)
\curveto(11.3390625,211.27519531)(11.16914063,211.12871094)(11.04414063,210.91777344)
\curveto(10.91523438,210.71074219)(10.85078125,210.46660156)(10.85078125,210.18535156)
\curveto(10.85078125,209.73613281)(10.99335938,209.34746094)(11.27851563,209.01933594)
\curveto(11.56367188,208.69511719)(12.1046875,208.53300781)(12.9015625,208.53300781)
\lineto(16.3,208.53300781)
\closepath
}
}
{
\newrgbcolor{curcolor}{0 0 0}
\pscustom[linestyle=none,fillstyle=solid,fillcolor=curcolor]
{
\newpath
\moveto(15.35664063,216.45488281)
\lineto(16.28828125,216.60722656)
\curveto(16.35078125,216.31035156)(16.38203125,216.04472656)(16.38203125,215.81035156)
\curveto(16.38203125,215.42753906)(16.32148438,215.13066406)(16.20039063,214.91972656)
\curveto(16.07929688,214.70878906)(15.92109375,214.56035156)(15.72578125,214.47441406)
\curveto(15.5265625,214.38847656)(15.11054688,214.34550781)(14.47773438,214.34550781)
\lineto(10.89765625,214.34550781)
\lineto(10.89765625,213.57207031)
\lineto(10.07734375,213.57207031)
\lineto(10.07734375,214.34550781)
\lineto(8.53632813,214.34550781)
\lineto(7.90351563,215.39433594)
\lineto(10.07734375,215.39433594)
\lineto(10.07734375,216.45488281)
\lineto(10.89765625,216.45488281)
\lineto(10.89765625,215.39433594)
\lineto(14.53632813,215.39433594)
\curveto(14.83710938,215.39433594)(15.03046875,215.41191406)(15.11640625,215.44707031)
\curveto(15.20234375,215.48613281)(15.27070313,215.54667969)(15.32148438,215.62871094)
\curveto(15.37226563,215.71464844)(15.39765625,215.83574219)(15.39765625,215.99199219)
\curveto(15.39765625,216.10917969)(15.38398438,216.26347656)(15.35664063,216.45488281)
\closepath
}
}
{
\newrgbcolor{curcolor}{0 0 0}
\pscustom[linestyle=none,fillstyle=solid,fillcolor=curcolor]
{
\newpath
\moveto(8.92304688,217.49199219)
\lineto(7.71015625,217.49199219)
\lineto(7.71015625,218.54667969)
\lineto(8.92304688,218.54667969)
\closepath
\moveto(16.3,217.49199219)
\lineto(10.07734375,217.49199219)
\lineto(10.07734375,218.54667969)
\lineto(16.3,218.54667969)
\closepath
}
}
{
\newrgbcolor{curcolor}{0 0 0}
\pscustom[linestyle=none,fillstyle=solid,fillcolor=curcolor]
{
\newpath
\moveto(16.3,220.15214844)
\lineto(10.07734375,220.15214844)
\lineto(10.07734375,221.09550781)
\lineto(10.95039063,221.09550781)
\curveto(10.64570313,221.29082031)(10.4015625,221.55058594)(10.21796875,221.87480469)
\curveto(10.03046875,222.19902344)(9.93671875,222.56816406)(9.93671875,222.98222656)
\curveto(9.93671875,223.44316406)(10.03242188,223.82011719)(10.22382813,224.11308594)
\curveto(10.41523438,224.40996094)(10.6828125,224.61894531)(11.0265625,224.74003906)
\curveto(10.3,225.23222656)(9.93671875,225.87285156)(9.93671875,226.66191406)
\curveto(9.93671875,227.27910156)(10.10859375,227.75371094)(10.45234375,228.08574219)
\curveto(10.7921875,228.41777344)(11.31757813,228.58378906)(12.02851563,228.58378906)
\lineto(16.3,228.58378906)
\lineto(16.3,227.53496094)
\lineto(12.38007813,227.53496094)
\curveto(11.95820313,227.53496094)(11.65546875,227.49980469)(11.471875,227.42949219)
\curveto(11.284375,227.36308594)(11.13398438,227.24003906)(11.02070313,227.06035156)
\curveto(10.90742188,226.88066406)(10.85078125,226.66972656)(10.85078125,226.42753906)
\curveto(10.85078125,225.99003906)(10.99726563,225.62675781)(11.29023438,225.33769531)
\curveto(11.57929688,225.04863281)(12.04414063,224.90410156)(12.68476563,224.90410156)
\lineto(16.3,224.90410156)
\lineto(16.3,223.84941406)
\lineto(12.25703125,223.84941406)
\curveto(11.78828125,223.84941406)(11.43671875,223.76347656)(11.20234375,223.59160156)
\curveto(10.96796875,223.41972656)(10.85078125,223.13847656)(10.85078125,222.74785156)
\curveto(10.85078125,222.45097656)(10.92890625,222.17558594)(11.08515625,221.92167969)
\curveto(11.24140625,221.67167969)(11.46992188,221.49003906)(11.77070313,221.37675781)
\curveto(12.07148438,221.26347656)(12.50507813,221.20683594)(13.07148438,221.20683594)
\lineto(16.3,221.20683594)
\closepath
}
}
{
\newrgbcolor{curcolor}{0 0 0}
\pscustom[linestyle=none,fillstyle=solid,fillcolor=curcolor]
{
\newpath
\moveto(14.29609375,234.40800781)
\lineto(14.43085938,235.49785156)
\curveto(15.06757813,235.32597656)(15.56171875,235.00761719)(15.91328125,234.54277344)
\curveto(16.26484375,234.07792969)(16.440625,233.48417969)(16.440625,232.76152344)
\curveto(16.440625,231.85136719)(16.16132813,231.12871094)(15.60273438,230.59355469)
\curveto(15.04023438,230.06230469)(14.253125,229.79667969)(13.24140625,229.79667969)
\curveto(12.19453125,229.79667969)(11.38203125,230.06621094)(10.80390625,230.60527344)
\curveto(10.22578125,231.14433594)(9.93671875,231.84355469)(9.93671875,232.70292969)
\curveto(9.93671875,233.53496094)(10.21992188,234.21464844)(10.78632813,234.74199219)
\curveto(11.35273438,235.26933594)(12.14960938,235.53300781)(13.17695313,235.53300781)
\curveto(13.23945313,235.53300781)(13.33320313,235.53105469)(13.45820313,235.52714844)
\lineto(13.45820313,230.88652344)
\curveto(14.14179688,230.92558594)(14.66523438,231.11894531)(15.02851563,231.46660156)
\curveto(15.39179688,231.81425781)(15.5734375,232.24785156)(15.5734375,232.76738281)
\curveto(15.5734375,233.15410156)(15.471875,233.48417969)(15.26875,233.75761719)
\curveto(15.065625,234.03105469)(14.74140625,234.24785156)(14.29609375,234.40800781)
\closepath
\moveto(12.59101563,230.94511719)
\lineto(12.59101563,234.41972656)
\curveto(12.06757813,234.37285156)(11.675,234.24003906)(11.41328125,234.02128906)
\curveto(11.00703125,233.68535156)(10.80390625,233.24980469)(10.80390625,232.71464844)
\curveto(10.80390625,232.23027344)(10.96601563,231.82207031)(11.29023438,231.49003906)
\curveto(11.61445313,231.16191406)(12.04804688,230.98027344)(12.59101563,230.94511719)
\closepath
}
}
{
\newrgbcolor{curcolor}{0 0 0}
\pscustom[linestyle=none,fillstyle=solid,fillcolor=curcolor]
{
\newpath
\moveto(18.82539063,242.17167969)
\curveto(18.09101563,241.58964844)(17.23164063,241.09746094)(16.24726563,240.69511719)
\curveto(15.26289063,240.29277344)(14.24335938,240.09160156)(13.18867188,240.09160156)
\curveto(12.25898438,240.09160156)(11.36835938,240.24199219)(10.51679688,240.54277344)
\curveto(9.52851563,240.89433594)(8.54414063,241.43730469)(7.56367188,242.17167969)
\lineto(7.56367188,242.92753906)
\curveto(8.37617188,242.45488281)(8.95625,242.14238281)(9.30390625,241.99003906)
\curveto(9.84296875,241.75175781)(10.40546875,241.56425781)(10.99140625,241.42753906)
\curveto(11.721875,241.25957031)(12.45625,241.17558594)(13.19453125,241.17558594)
\curveto(15.0734375,241.17558594)(16.95039063,241.75957031)(18.82539063,242.92753906)
\closepath
}
}
{
\newrgbcolor{curcolor}{0 0 0}
\pscustom[linestyle=none,fillstyle=solid,fillcolor=curcolor]
{
\newpath
\moveto(14.44257813,243.73027344)
\lineto(14.27851563,244.77324219)
\curveto(14.69648438,244.83183594)(15.01679688,244.99394531)(15.23945313,245.25957031)
\curveto(15.46210938,245.52910156)(15.5734375,245.90410156)(15.5734375,246.38457031)
\curveto(15.5734375,246.86894531)(15.47578125,247.22832031)(15.28046875,247.46269531)
\curveto(15.08125,247.69707031)(14.84882813,247.81425781)(14.58320313,247.81425781)
\curveto(14.34492188,247.81425781)(14.15742188,247.71074219)(14.02070313,247.50371094)
\curveto(13.92695313,247.35917969)(13.8078125,246.99980469)(13.66328125,246.42558594)
\curveto(13.46796875,245.65214844)(13.3,245.11503906)(13.159375,244.81425781)
\curveto(13.01484375,244.51738281)(12.81757813,244.29082031)(12.56757813,244.13457031)
\curveto(12.31367188,243.98222656)(12.034375,243.90605469)(11.7296875,243.90605469)
\curveto(11.45234375,243.90605469)(11.19648438,243.96855469)(10.96210938,244.09355469)
\curveto(10.72382813,244.22246094)(10.5265625,244.39628906)(10.3703125,244.61503906)
\curveto(10.24921875,244.77910156)(10.14765625,245.00175781)(10.065625,245.28300781)
\curveto(9.9796875,245.56816406)(9.93671875,245.87285156)(9.93671875,246.19707031)
\curveto(9.93671875,246.68535156)(10.00703125,247.11308594)(10.14765625,247.48027344)
\curveto(10.28828125,247.85136719)(10.4796875,248.12480469)(10.721875,248.30058594)
\curveto(10.96015625,248.47636719)(11.28046875,248.59746094)(11.6828125,248.66386719)
\lineto(11.8234375,247.63261719)
\curveto(11.503125,247.58574219)(11.253125,247.44902344)(11.0734375,247.22246094)
\curveto(10.89375,246.99980469)(10.80390625,246.68339844)(10.80390625,246.27324219)
\curveto(10.80390625,245.78886719)(10.88398438,245.44316406)(11.04414063,245.23613281)
\curveto(11.20429688,245.02910156)(11.39179688,244.92558594)(11.60664063,244.92558594)
\curveto(11.74335938,244.92558594)(11.86640625,244.96855469)(11.97578125,245.05449219)
\curveto(12.0890625,245.14042969)(12.1828125,245.27519531)(12.25703125,245.45878906)
\curveto(12.29609375,245.56425781)(12.3859375,245.87480469)(12.5265625,246.39042969)
\curveto(12.72578125,247.13652344)(12.88984375,247.65605469)(13.01875,247.94902344)
\curveto(13.14375,248.24589844)(13.32734375,248.47832031)(13.56953125,248.64628906)
\curveto(13.81171875,248.81425781)(14.1125,248.89824219)(14.471875,248.89824219)
\curveto(14.8234375,248.89824219)(15.15546875,248.79472656)(15.46796875,248.58769531)
\curveto(15.7765625,248.38457031)(16.01679688,248.08964844)(16.18867188,247.70292969)
\curveto(16.35664063,247.31621094)(16.440625,246.87871094)(16.440625,246.39042969)
\curveto(16.440625,245.58183594)(16.27265625,244.96464844)(15.93671875,244.53886719)
\curveto(15.60078125,244.11699219)(15.10273438,243.84746094)(14.44257813,243.73027344)
\closepath
}
}
{
\newrgbcolor{curcolor}{0 0 0}
\pscustom[linestyle=none,fillstyle=solid,fillcolor=curcolor]
{
\newpath
\moveto(14.29609375,254.41191406)
\lineto(14.43085938,255.50175781)
\curveto(15.06757813,255.32988281)(15.56171875,255.01152344)(15.91328125,254.54667969)
\curveto(16.26484375,254.08183594)(16.440625,253.48808594)(16.440625,252.76542969)
\curveto(16.440625,251.85527344)(16.16132813,251.13261719)(15.60273438,250.59746094)
\curveto(15.04023438,250.06621094)(14.253125,249.80058594)(13.24140625,249.80058594)
\curveto(12.19453125,249.80058594)(11.38203125,250.07011719)(10.80390625,250.60917969)
\curveto(10.22578125,251.14824219)(9.93671875,251.84746094)(9.93671875,252.70683594)
\curveto(9.93671875,253.53886719)(10.21992188,254.21855469)(10.78632813,254.74589844)
\curveto(11.35273438,255.27324219)(12.14960938,255.53691406)(13.17695313,255.53691406)
\curveto(13.23945313,255.53691406)(13.33320313,255.53496094)(13.45820313,255.53105469)
\lineto(13.45820313,250.89042969)
\curveto(14.14179688,250.92949219)(14.66523438,251.12285156)(15.02851563,251.47050781)
\curveto(15.39179688,251.81816406)(15.5734375,252.25175781)(15.5734375,252.77128906)
\curveto(15.5734375,253.15800781)(15.471875,253.48808594)(15.26875,253.76152344)
\curveto(15.065625,254.03496094)(14.74140625,254.25175781)(14.29609375,254.41191406)
\closepath
\moveto(12.59101563,250.94902344)
\lineto(12.59101563,254.42363281)
\curveto(12.06757813,254.37675781)(11.675,254.24394531)(11.41328125,254.02519531)
\curveto(11.00703125,253.68925781)(10.80390625,253.25371094)(10.80390625,252.71855469)
\curveto(10.80390625,252.23417969)(10.96601563,251.82597656)(11.29023438,251.49394531)
\curveto(11.61445313,251.16582031)(12.04804688,250.98417969)(12.59101563,250.94902344)
\closepath
}
}
{
\newrgbcolor{curcolor}{0 0 0}
\pscustom[linestyle=none,fillstyle=solid,fillcolor=curcolor]
{
\newpath
\moveto(14.02070313,260.88652344)
\lineto(14.15546875,261.92363281)
\curveto(14.8703125,261.81035156)(15.43085938,261.51933594)(15.83710938,261.05058594)
\curveto(16.23945313,260.58574219)(16.440625,260.01347656)(16.440625,259.33378906)
\curveto(16.440625,258.48222656)(16.16328125,257.79667969)(15.60859375,257.27714844)
\curveto(15.05,256.76152344)(14.25117188,256.50371094)(13.21210938,256.50371094)
\curveto(12.54023438,256.50371094)(11.95234375,256.61503906)(11.4484375,256.83769531)
\curveto(10.94453125,257.06035156)(10.56757813,257.39824219)(10.31757813,257.85136719)
\curveto(10.06367188,258.30839844)(9.93671875,258.80449219)(9.93671875,259.33964844)
\curveto(9.93671875,260.01542969)(10.10859375,260.56816406)(10.45234375,260.99785156)
\curveto(10.7921875,261.42753906)(11.2765625,261.70292969)(11.90546875,261.82402344)
\lineto(12.06367188,260.79863281)
\curveto(11.64570313,260.70097656)(11.33125,260.52714844)(11.1203125,260.27714844)
\curveto(10.909375,260.03105469)(10.80390625,259.73222656)(10.80390625,259.38066406)
\curveto(10.80390625,258.84941406)(10.9953125,258.41777344)(11.378125,258.08574219)
\curveto(11.75703125,257.75371094)(12.35859375,257.58769531)(13.1828125,257.58769531)
\curveto(14.01875,257.58769531)(14.62617188,257.74785156)(15.00507813,258.06816406)
\curveto(15.38398438,258.38847656)(15.5734375,258.80644531)(15.5734375,259.32207031)
\curveto(15.5734375,259.73613281)(15.44648438,260.08183594)(15.19257813,260.35917969)
\curveto(14.93867188,260.63652344)(14.54804688,260.81230469)(14.02070313,260.88652344)
\closepath
}
}
{
\newrgbcolor{curcolor}{0 0 0}
\pscustom[linestyle=none,fillstyle=solid,fillcolor=curcolor]
{
\newpath
\moveto(13.18867188,262.43339844)
\curveto(12.03632813,262.43339844)(11.1828125,262.75371094)(10.628125,263.39433594)
\curveto(10.1671875,263.92949219)(9.93671875,264.58183594)(9.93671875,265.35136719)
\curveto(9.93671875,266.20683594)(10.21796875,266.90605469)(10.78046875,267.44902344)
\curveto(11.3390625,267.99199219)(12.1125,268.26347656)(13.10078125,268.26347656)
\curveto(13.9015625,268.26347656)(14.53242188,268.14238281)(14.99335938,267.90019531)
\curveto(15.45039063,267.66191406)(15.80585938,267.31230469)(16.05976563,266.85136719)
\curveto(16.31367188,266.39433594)(16.440625,265.89433594)(16.440625,265.35136719)
\curveto(16.440625,264.48027344)(16.16132813,263.77519531)(15.60273438,263.23613281)
\curveto(15.04414063,262.70097656)(14.23945313,262.43339844)(13.18867188,262.43339844)
\closepath
\moveto(13.18867188,263.51738281)
\curveto(13.98554688,263.51738281)(14.58320313,263.69121094)(14.98164063,264.03886719)
\curveto(15.37617188,264.38652344)(15.5734375,264.82402344)(15.5734375,265.35136719)
\curveto(15.5734375,265.87480469)(15.37421875,266.31035156)(14.97578125,266.65800781)
\curveto(14.57734375,267.00566406)(13.96992188,267.17949219)(13.15351563,267.17949219)
\curveto(12.38398438,267.17949219)(11.80195313,267.00371094)(11.40742188,266.65214844)
\curveto(11.00898438,266.30449219)(10.80976563,265.87089844)(10.80976563,265.35136719)
\curveto(10.80976563,264.82402344)(11.00703125,264.38652344)(11.4015625,264.03886719)
\curveto(11.79609375,263.69121094)(12.39179688,263.51738281)(13.18867188,263.51738281)
\closepath
}
}
{
\newrgbcolor{curcolor}{0 0 0}
\pscustom[linestyle=none,fillstyle=solid,fillcolor=curcolor]
{
\newpath
\moveto(16.3,269.49980469)
\lineto(10.07734375,269.49980469)
\lineto(10.07734375,270.44902344)
\lineto(10.96210938,270.44902344)
\curveto(10.27851563,270.90605469)(9.93671875,271.56621094)(9.93671875,272.42949219)
\curveto(9.93671875,272.80449219)(10.00507813,273.14824219)(10.14179688,273.46074219)
\curveto(10.27460938,273.77714844)(10.45039063,274.01347656)(10.66914063,274.16972656)
\curveto(10.88789063,274.32597656)(11.14765625,274.43535156)(11.4484375,274.49785156)
\curveto(11.64375,274.53691406)(11.98554688,274.55644531)(12.47382813,274.55644531)
\lineto(16.3,274.55644531)
\lineto(16.3,273.50175781)
\lineto(12.51484375,273.50175781)
\curveto(12.08515625,273.50175781)(11.76484375,273.46074219)(11.55390625,273.37871094)
\curveto(11.3390625,273.29667969)(11.16914063,273.15019531)(11.04414063,272.93925781)
\curveto(10.91523438,272.73222656)(10.85078125,272.48808594)(10.85078125,272.20683594)
\curveto(10.85078125,271.75761719)(10.99335938,271.36894531)(11.27851563,271.04082031)
\curveto(11.56367188,270.71660156)(12.1046875,270.55449219)(12.9015625,270.55449219)
\lineto(16.3,270.55449219)
\closepath
}
}
{
\newrgbcolor{curcolor}{0 0 0}
\pscustom[linestyle=none,fillstyle=solid,fillcolor=curcolor]
{
\newpath
\moveto(16.3,280.21074219)
\lineto(15.51484375,280.21074219)
\curveto(16.13203125,279.81621094)(16.440625,279.23613281)(16.440625,278.47050781)
\curveto(16.440625,277.97441406)(16.30390625,277.51738281)(16.03046875,277.09941406)
\curveto(15.75703125,276.68535156)(15.37617188,276.36308594)(14.88789063,276.13261719)
\curveto(14.39570313,275.90605469)(13.83125,275.79277344)(13.19453125,275.79277344)
\curveto(12.5734375,275.79277344)(12.0109375,275.89628906)(11.50703125,276.10332031)
\curveto(10.99921875,276.31035156)(10.61054688,276.62089844)(10.34101563,277.03496094)
\curveto(10.07148438,277.44902344)(9.93671875,277.91191406)(9.93671875,278.42363281)
\curveto(9.93671875,278.79863281)(10.01679688,279.13261719)(10.17695313,279.42558594)
\curveto(10.33320313,279.71855469)(10.53828125,279.95683594)(10.7921875,280.14042969)
\lineto(7.71015625,280.14042969)
\lineto(7.71015625,281.18925781)
\lineto(16.3,281.18925781)
\closepath
\moveto(13.19453125,276.87675781)
\curveto(13.99140625,276.87675781)(14.58710938,277.04472656)(14.98164063,277.38066406)
\curveto(15.37617188,277.71660156)(15.5734375,278.11308594)(15.5734375,278.57011719)
\curveto(15.5734375,279.03105469)(15.3859375,279.42167969)(15.0109375,279.74199219)
\curveto(14.63203125,280.06621094)(14.05585938,280.22832031)(13.28242188,280.22832031)
\curveto(12.43085938,280.22832031)(11.80585938,280.06425781)(11.40742188,279.73613281)
\curveto(11.00898438,279.40800781)(10.80976563,279.00371094)(10.80976563,278.52324219)
\curveto(10.80976563,278.05449219)(11.00117188,277.66191406)(11.38398438,277.34550781)
\curveto(11.76679688,277.03300781)(12.3703125,276.87675781)(13.19453125,276.87675781)
\closepath
}
}
{
\newrgbcolor{curcolor}{0 0 0}
\pscustom[linestyle=none,fillstyle=solid,fillcolor=curcolor]
{
\newpath
\moveto(14.44257813,282.42558594)
\lineto(14.27851563,283.46855469)
\curveto(14.69648438,283.52714844)(15.01679688,283.68925781)(15.23945313,283.95488281)
\curveto(15.46210938,284.22441406)(15.5734375,284.59941406)(15.5734375,285.07988281)
\curveto(15.5734375,285.56425781)(15.47578125,285.92363281)(15.28046875,286.15800781)
\curveto(15.08125,286.39238281)(14.84882813,286.50957031)(14.58320313,286.50957031)
\curveto(14.34492188,286.50957031)(14.15742188,286.40605469)(14.02070313,286.19902344)
\curveto(13.92695313,286.05449219)(13.8078125,285.69511719)(13.66328125,285.12089844)
\curveto(13.46796875,284.34746094)(13.3,283.81035156)(13.159375,283.50957031)
\curveto(13.01484375,283.21269531)(12.81757813,282.98613281)(12.56757813,282.82988281)
\curveto(12.31367188,282.67753906)(12.034375,282.60136719)(11.7296875,282.60136719)
\curveto(11.45234375,282.60136719)(11.19648438,282.66386719)(10.96210938,282.78886719)
\curveto(10.72382813,282.91777344)(10.5265625,283.09160156)(10.3703125,283.31035156)
\curveto(10.24921875,283.47441406)(10.14765625,283.69707031)(10.065625,283.97832031)
\curveto(9.9796875,284.26347656)(9.93671875,284.56816406)(9.93671875,284.89238281)
\curveto(9.93671875,285.38066406)(10.00703125,285.80839844)(10.14765625,286.17558594)
\curveto(10.28828125,286.54667969)(10.4796875,286.82011719)(10.721875,286.99589844)
\curveto(10.96015625,287.17167969)(11.28046875,287.29277344)(11.6828125,287.35917969)
\lineto(11.8234375,286.32792969)
\curveto(11.503125,286.28105469)(11.253125,286.14433594)(11.0734375,285.91777344)
\curveto(10.89375,285.69511719)(10.80390625,285.37871094)(10.80390625,284.96855469)
\curveto(10.80390625,284.48417969)(10.88398438,284.13847656)(11.04414063,283.93144531)
\curveto(11.20429688,283.72441406)(11.39179688,283.62089844)(11.60664063,283.62089844)
\curveto(11.74335938,283.62089844)(11.86640625,283.66386719)(11.97578125,283.74980469)
\curveto(12.0890625,283.83574219)(12.1828125,283.97050781)(12.25703125,284.15410156)
\curveto(12.29609375,284.25957031)(12.3859375,284.57011719)(12.5265625,285.08574219)
\curveto(12.72578125,285.83183594)(12.88984375,286.35136719)(13.01875,286.64433594)
\curveto(13.14375,286.94121094)(13.32734375,287.17363281)(13.56953125,287.34160156)
\curveto(13.81171875,287.50957031)(14.1125,287.59355469)(14.471875,287.59355469)
\curveto(14.8234375,287.59355469)(15.15546875,287.49003906)(15.46796875,287.28300781)
\curveto(15.7765625,287.07988281)(16.01679688,286.78496094)(16.18867188,286.39824219)
\curveto(16.35664063,286.01152344)(16.440625,285.57402344)(16.440625,285.08574219)
\curveto(16.440625,284.27714844)(16.27265625,283.65996094)(15.93671875,283.23417969)
\curveto(15.60078125,282.81230469)(15.10273438,282.54277344)(14.44257813,282.42558594)
\closepath
}
}
{
\newrgbcolor{curcolor}{0 0 0}
\pscustom[linestyle=none,fillstyle=solid,fillcolor=curcolor]
{
\newpath
\moveto(18.82539063,289.53886719)
\lineto(18.82539063,288.78300781)
\curveto(16.95039063,289.95097656)(15.0734375,290.53496094)(13.19453125,290.53496094)
\curveto(12.46015625,290.53496094)(11.73164063,290.45097656)(11.00898438,290.28300781)
\curveto(10.42304688,290.15019531)(9.86054688,289.96464844)(9.32148438,289.72636719)
\curveto(8.96992188,289.57402344)(8.38398438,289.25957031)(7.56367188,288.78300781)
\lineto(7.56367188,289.53886719)
\curveto(8.54414063,290.27324219)(9.52851563,290.81621094)(10.51679688,291.16777344)
\curveto(11.36835938,291.46855469)(12.25898438,291.61894531)(13.18867188,291.61894531)
\curveto(14.24335938,291.61894531)(15.26289063,291.41582031)(16.24726563,291.00957031)
\curveto(17.23164063,290.60722656)(18.09101563,290.11699219)(18.82539063,289.53886719)
\closepath
}
}
{
\newrgbcolor{curcolor}{0 0 0}
\pscustom[linestyle=none,fillstyle=solid,fillcolor=curcolor]
{
\newpath
\moveto(317.97753906,8.7)
\lineto(317.97753906,17.28984375)
\lineto(323.77246094,17.28984375)
\lineto(323.77246094,16.27617187)
\lineto(319.11425781,16.27617187)
\lineto(319.11425781,13.61601562)
\lineto(323.14550781,13.61601562)
\lineto(323.14550781,12.60234375)
\lineto(319.11425781,12.60234375)
\lineto(319.11425781,8.7)
\closepath
}
}
{
\newrgbcolor{curcolor}{0 0 0}
\pscustom[linestyle=none,fillstyle=solid,fillcolor=curcolor]
{
\newpath
\moveto(325.12011719,16.07695312)
\lineto(325.12011719,17.28984375)
\lineto(326.17480469,17.28984375)
\lineto(326.17480469,16.07695312)
\closepath
\moveto(325.12011719,8.7)
\lineto(325.12011719,14.92265625)
\lineto(326.17480469,14.92265625)
\lineto(326.17480469,8.7)
\closepath
}
}
{
\newrgbcolor{curcolor}{0 0 0}
\pscustom[linestyle=none,fillstyle=solid,fillcolor=curcolor]
{
\newpath
\moveto(327.75683594,8.7)
\lineto(327.75683594,17.28984375)
\lineto(328.81152344,17.28984375)
\lineto(328.81152344,8.7)
\closepath
}
}
{
\newrgbcolor{curcolor}{0 0 0}
\pscustom[linestyle=none,fillstyle=solid,fillcolor=curcolor]
{
\newpath
\moveto(334.70605469,10.70390625)
\lineto(335.79589844,10.56914062)
\curveto(335.62402344,9.93242187)(335.30566406,9.43828125)(334.84082031,9.08671875)
\curveto(334.37597656,8.73515625)(333.78222656,8.559375)(333.05957031,8.559375)
\curveto(332.14941406,8.559375)(331.42675781,8.83867187)(330.89160156,9.39726562)
\curveto(330.36035156,9.95976562)(330.09472656,10.746875)(330.09472656,11.75859375)
\curveto(330.09472656,12.80546875)(330.36425781,13.61796875)(330.90332031,14.19609375)
\curveto(331.44238281,14.77421875)(332.14160156,15.06328125)(333.00097656,15.06328125)
\curveto(333.83300781,15.06328125)(334.51269531,14.78007812)(335.04003906,14.21367187)
\curveto(335.56738281,13.64726562)(335.83105469,12.85039062)(335.83105469,11.82304687)
\curveto(335.83105469,11.76054687)(335.82910156,11.66679687)(335.82519531,11.54179687)
\lineto(331.18457031,11.54179687)
\curveto(331.22363281,10.85820312)(331.41699219,10.33476562)(331.76464844,9.97148437)
\curveto(332.11230469,9.60820312)(332.54589844,9.4265625)(333.06542969,9.4265625)
\curveto(333.45214844,9.4265625)(333.78222656,9.528125)(334.05566406,9.73125)
\curveto(334.32910156,9.934375)(334.54589844,10.25859375)(334.70605469,10.70390625)
\closepath
\moveto(331.24316406,12.40898437)
\lineto(334.71777344,12.40898437)
\curveto(334.67089844,12.93242187)(334.53808594,13.325)(334.31933594,13.58671875)
\curveto(333.98339844,13.99296875)(333.54785156,14.19609375)(333.01269531,14.19609375)
\curveto(332.52832031,14.19609375)(332.12011719,14.03398437)(331.78808594,13.70976562)
\curveto(331.45996094,13.38554687)(331.27832031,12.95195312)(331.24316406,12.40898437)
\closepath
}
}
{
\newrgbcolor{curcolor}{0 0 0}
\pscustom[linestyle=none,fillstyle=solid,fillcolor=curcolor]
{
\newpath
\moveto(340.20214844,11.45976562)
\lineto(341.27441406,11.55351562)
\curveto(341.32519531,11.12382812)(341.44238281,10.7703125)(341.62597656,10.49296875)
\curveto(341.81347656,10.21953125)(342.10253906,9.996875)(342.49316406,9.825)
\curveto(342.88378906,9.65703125)(343.32324219,9.57304687)(343.81152344,9.57304687)
\curveto(344.24511719,9.57304687)(344.62792969,9.6375)(344.95996094,9.76640625)
\curveto(345.29199219,9.8953125)(345.53808594,10.07109375)(345.69824219,10.29375)
\curveto(345.86230469,10.5203125)(345.94433594,10.76640625)(345.94433594,11.03203125)
\curveto(345.94433594,11.3015625)(345.86621094,11.5359375)(345.70996094,11.73515625)
\curveto(345.55371094,11.93828125)(345.29589844,12.10820312)(344.93652344,12.24492187)
\curveto(344.70605469,12.33476562)(344.19628906,12.4734375)(343.40722656,12.6609375)
\curveto(342.61816406,12.85234375)(342.06542969,13.03203125)(341.74902344,13.2)
\curveto(341.33886719,13.41484375)(341.03222656,13.68046875)(340.82910156,13.996875)
\curveto(340.62988281,14.3171875)(340.53027344,14.67460937)(340.53027344,15.06914062)
\curveto(340.53027344,15.50273437)(340.65332031,15.90703125)(340.89941406,16.28203125)
\curveto(341.14550781,16.6609375)(341.50488281,16.94804687)(341.97753906,17.14335937)
\curveto(342.45019531,17.33867187)(342.97558594,17.43632812)(343.55371094,17.43632812)
\curveto(344.19042969,17.43632812)(344.75097656,17.3328125)(345.23535156,17.12578125)
\curveto(345.72363281,16.92265625)(346.09863281,16.621875)(346.36035156,16.2234375)
\curveto(346.62207031,15.825)(346.76269531,15.37382812)(346.78222656,14.86992187)
\lineto(345.69238281,14.78789062)
\curveto(345.63378906,15.33085937)(345.43457031,15.74101562)(345.09472656,16.01835937)
\curveto(344.75878906,16.29570312)(344.26074219,16.434375)(343.60058594,16.434375)
\curveto(342.91308594,16.434375)(342.41113281,16.30742187)(342.09472656,16.05351562)
\curveto(341.78222656,15.80351562)(341.62597656,15.50078125)(341.62597656,15.1453125)
\curveto(341.62597656,14.83671875)(341.73730469,14.5828125)(341.95996094,14.38359375)
\curveto(342.17871094,14.184375)(342.74902344,13.97929687)(343.67089844,13.76835937)
\curveto(344.59667969,13.56132812)(345.23144531,13.3796875)(345.57519531,13.2234375)
\curveto(346.07519531,12.99296875)(346.44433594,12.7)(346.68261719,12.34453125)
\curveto(346.92089844,11.99296875)(347.04003906,11.58671875)(347.04003906,11.12578125)
\curveto(347.04003906,10.66875)(346.90917969,10.23710937)(346.64746094,9.83085937)
\curveto(346.38574219,9.42851562)(346.00878906,9.1140625)(345.51660156,8.8875)
\curveto(345.02832031,8.66484375)(344.47753906,8.55351562)(343.86425781,8.55351562)
\curveto(343.08691406,8.55351562)(342.43457031,8.66679687)(341.90722656,8.89335937)
\curveto(341.38378906,9.11992187)(340.97167969,9.45976562)(340.67089844,9.91289062)
\curveto(340.37402344,10.36992187)(340.21777344,10.88554687)(340.20214844,11.45976562)
\closepath
}
}
{
\newrgbcolor{curcolor}{0 0 0}
\pscustom[linestyle=none,fillstyle=solid,fillcolor=curcolor]
{
\newpath
\moveto(348.46386719,16.07695312)
\lineto(348.46386719,17.28984375)
\lineto(349.51855469,17.28984375)
\lineto(349.51855469,16.07695312)
\closepath
\moveto(348.46386719,8.7)
\lineto(348.46386719,14.92265625)
\lineto(349.51855469,14.92265625)
\lineto(349.51855469,8.7)
\closepath
}
}
{
\newrgbcolor{curcolor}{0 0 0}
\pscustom[linestyle=none,fillstyle=solid,fillcolor=curcolor]
{
\newpath
\moveto(350.56738281,8.7)
\lineto(350.56738281,9.55546875)
\lineto(354.52832031,14.10234375)
\curveto(354.07910156,14.07890625)(353.68261719,14.0671875)(353.33886719,14.0671875)
\lineto(350.80175781,14.0671875)
\lineto(350.80175781,14.92265625)
\lineto(355.88769531,14.92265625)
\lineto(355.88769531,14.22539062)
\lineto(352.51855469,10.27617187)
\lineto(351.86816406,9.55546875)
\curveto(352.34082031,9.590625)(352.78417969,9.60820312)(353.19824219,9.60820312)
\lineto(356.07519531,9.60820312)
\lineto(356.07519531,8.7)
\closepath
}
}
{
\newrgbcolor{curcolor}{0 0 0}
\pscustom[linestyle=none,fillstyle=solid,fillcolor=curcolor]
{
\newpath
\moveto(361.38378906,10.70390625)
\lineto(362.47363281,10.56914062)
\curveto(362.30175781,9.93242187)(361.98339844,9.43828125)(361.51855469,9.08671875)
\curveto(361.05371094,8.73515625)(360.45996094,8.559375)(359.73730469,8.559375)
\curveto(358.82714844,8.559375)(358.10449219,8.83867187)(357.56933594,9.39726562)
\curveto(357.03808594,9.95976562)(356.77246094,10.746875)(356.77246094,11.75859375)
\curveto(356.77246094,12.80546875)(357.04199219,13.61796875)(357.58105469,14.19609375)
\curveto(358.12011719,14.77421875)(358.81933594,15.06328125)(359.67871094,15.06328125)
\curveto(360.51074219,15.06328125)(361.19042969,14.78007812)(361.71777344,14.21367187)
\curveto(362.24511719,13.64726562)(362.50878906,12.85039062)(362.50878906,11.82304687)
\curveto(362.50878906,11.76054687)(362.50683594,11.66679687)(362.50292969,11.54179687)
\lineto(357.86230469,11.54179687)
\curveto(357.90136719,10.85820312)(358.09472656,10.33476562)(358.44238281,9.97148437)
\curveto(358.79003906,9.60820312)(359.22363281,9.4265625)(359.74316406,9.4265625)
\curveto(360.12988281,9.4265625)(360.45996094,9.528125)(360.73339844,9.73125)
\curveto(361.00683594,9.934375)(361.22363281,10.25859375)(361.38378906,10.70390625)
\closepath
\moveto(357.92089844,12.40898437)
\lineto(361.39550781,12.40898437)
\curveto(361.34863281,12.93242187)(361.21582031,13.325)(360.99707031,13.58671875)
\curveto(360.66113281,13.99296875)(360.22558594,14.19609375)(359.69042969,14.19609375)
\curveto(359.20605469,14.19609375)(358.79785156,14.03398437)(358.46582031,13.70976562)
\curveto(358.13769531,13.38554687)(357.95605469,12.95195312)(357.92089844,12.40898437)
\closepath
}
}
{
\newrgbcolor{curcolor}{0 0 0}
\pscustom[linestyle=none,fillstyle=solid,fillcolor=curcolor]
{
\newpath
\moveto(139.20214844,449)
\lineto(139.20214844,457.58984375)
\lineto(143.01074219,457.58984375)
\curveto(143.77636719,457.58984375)(144.35839844,457.51171875)(144.75683594,457.35546875)
\curveto(145.15527344,457.203125)(145.47363281,456.93164062)(145.71191406,456.54101562)
\curveto(145.95019531,456.15039062)(146.06933594,455.71875)(146.06933594,455.24609375)
\curveto(146.06933594,454.63671875)(145.87207031,454.12304688)(145.47753906,453.70507812)
\curveto(145.08300781,453.28710938)(144.47363281,453.02148438)(143.64941406,452.90820312)
\curveto(143.95019531,452.76367188)(144.17871094,452.62109375)(144.33496094,452.48046875)
\curveto(144.66699219,452.17578125)(144.98144531,451.79492188)(145.27832031,451.33789062)
\lineto(146.77246094,449)
\lineto(145.34277344,449)
\lineto(144.20605469,450.78710938)
\curveto(143.87402344,451.30273438)(143.60058594,451.69726562)(143.38574219,451.97070312)
\curveto(143.17089844,452.24414062)(142.97753906,452.43554688)(142.80566406,452.54492188)
\curveto(142.63769531,452.65429688)(142.46582031,452.73046875)(142.29003906,452.7734375)
\curveto(142.16113281,452.80078125)(141.95019531,452.81445312)(141.65722656,452.81445312)
\lineto(140.33886719,452.81445312)
\lineto(140.33886719,449)
\closepath
\moveto(140.33886719,453.79882812)
\lineto(142.78222656,453.79882812)
\curveto(143.30175781,453.79882812)(143.70800781,453.8515625)(144.00097656,453.95703125)
\curveto(144.29394531,454.06640625)(144.51660156,454.23828125)(144.66894531,454.47265625)
\curveto(144.82128906,454.7109375)(144.89746094,454.96875)(144.89746094,455.24609375)
\curveto(144.89746094,455.65234375)(144.74902344,455.98632812)(144.45214844,456.24804688)
\curveto(144.15917969,456.50976562)(143.69433594,456.640625)(143.05761719,456.640625)
\lineto(140.33886719,456.640625)
\closepath
}
}
{
\newrgbcolor{curcolor}{0 0 0}
\pscustom[linestyle=none,fillstyle=solid,fillcolor=curcolor]
{
\newpath
\moveto(151.79394531,449)
\lineto(151.79394531,449.9140625)
\curveto(151.30957031,449.2109375)(150.65136719,448.859375)(149.81933594,448.859375)
\curveto(149.45214844,448.859375)(149.10839844,448.9296875)(148.78808594,449.0703125)
\curveto(148.47167969,449.2109375)(148.23535156,449.38671875)(148.07910156,449.59765625)
\curveto(147.92675781,449.8125)(147.81933594,450.07421875)(147.75683594,450.3828125)
\curveto(147.71386719,450.58984375)(147.69238281,450.91796875)(147.69238281,451.3671875)
\lineto(147.69238281,455.22265625)
\lineto(148.74707031,455.22265625)
\lineto(148.74707031,451.77148438)
\curveto(148.74707031,451.22070312)(148.76855469,450.84960938)(148.81152344,450.65820312)
\curveto(148.87792969,450.38085938)(149.01855469,450.16210938)(149.23339844,450.00195312)
\curveto(149.44824219,449.84570312)(149.71386719,449.76757812)(150.03027344,449.76757812)
\curveto(150.34667969,449.76757812)(150.64355469,449.84765625)(150.92089844,450.0078125)
\curveto(151.19824219,450.171875)(151.39355469,450.39257812)(151.50683594,450.66992188)
\curveto(151.62402344,450.95117188)(151.68261719,451.35742188)(151.68261719,451.88867188)
\lineto(151.68261719,455.22265625)
\lineto(152.73730469,455.22265625)
\lineto(152.73730469,449)
\closepath
}
}
{
\newrgbcolor{curcolor}{0 0 0}
\pscustom[linestyle=none,fillstyle=solid,fillcolor=curcolor]
{
\newpath
\moveto(154.38964844,449)
\lineto(154.38964844,455.22265625)
\lineto(155.33886719,455.22265625)
\lineto(155.33886719,454.33789062)
\curveto(155.79589844,455.02148438)(156.45605469,455.36328125)(157.31933594,455.36328125)
\curveto(157.69433594,455.36328125)(158.03808594,455.29492188)(158.35058594,455.15820312)
\curveto(158.66699219,455.02539062)(158.90332031,454.84960938)(159.05957031,454.63085938)
\curveto(159.21582031,454.41210938)(159.32519531,454.15234375)(159.38769531,453.8515625)
\curveto(159.42675781,453.65625)(159.44628906,453.31445312)(159.44628906,452.82617188)
\lineto(159.44628906,449)
\lineto(158.39160156,449)
\lineto(158.39160156,452.78515625)
\curveto(158.39160156,453.21484375)(158.35058594,453.53515625)(158.26855469,453.74609375)
\curveto(158.18652344,453.9609375)(158.04003906,454.13085938)(157.82910156,454.25585938)
\curveto(157.62207031,454.38476562)(157.37792969,454.44921875)(157.09667969,454.44921875)
\curveto(156.64746094,454.44921875)(156.25878906,454.30664062)(155.93066406,454.02148438)
\curveto(155.60644531,453.73632812)(155.44433594,453.1953125)(155.44433594,452.3984375)
\lineto(155.44433594,449)
\closepath
}
}
{
\newrgbcolor{curcolor}{0 0 0}
\pscustom[linestyle=none,fillstyle=solid,fillcolor=curcolor]
{
\newpath
\moveto(163.36621094,449.94335938)
\lineto(163.51855469,449.01171875)
\curveto(163.22167969,448.94921875)(162.95605469,448.91796875)(162.72167969,448.91796875)
\curveto(162.33886719,448.91796875)(162.04199219,448.97851562)(161.83105469,449.09960938)
\curveto(161.62011719,449.22070312)(161.47167969,449.37890625)(161.38574219,449.57421875)
\curveto(161.29980469,449.7734375)(161.25683594,450.18945312)(161.25683594,450.82226562)
\lineto(161.25683594,454.40234375)
\lineto(160.48339844,454.40234375)
\lineto(160.48339844,455.22265625)
\lineto(161.25683594,455.22265625)
\lineto(161.25683594,456.76367188)
\lineto(162.30566406,457.39648438)
\lineto(162.30566406,455.22265625)
\lineto(163.36621094,455.22265625)
\lineto(163.36621094,454.40234375)
\lineto(162.30566406,454.40234375)
\lineto(162.30566406,450.76367188)
\curveto(162.30566406,450.46289062)(162.32324219,450.26953125)(162.35839844,450.18359375)
\curveto(162.39746094,450.09765625)(162.45800781,450.02929688)(162.54003906,449.97851562)
\curveto(162.62597656,449.92773438)(162.74707031,449.90234375)(162.90332031,449.90234375)
\curveto(163.02050781,449.90234375)(163.17480469,449.91601562)(163.36621094,449.94335938)
\closepath
}
}
{
\newrgbcolor{curcolor}{0 0 0}
\pscustom[linestyle=none,fillstyle=solid,fillcolor=curcolor]
{
\newpath
\moveto(164.40332031,456.37695312)
\lineto(164.40332031,457.58984375)
\lineto(165.45800781,457.58984375)
\lineto(165.45800781,456.37695312)
\closepath
\moveto(164.40332031,449)
\lineto(164.40332031,455.22265625)
\lineto(165.45800781,455.22265625)
\lineto(165.45800781,449)
\closepath
}
}
{
\newrgbcolor{curcolor}{0 0 0}
\pscustom[linestyle=none,fillstyle=solid,fillcolor=curcolor]
{
\newpath
\moveto(167.06347656,449)
\lineto(167.06347656,455.22265625)
\lineto(168.00683594,455.22265625)
\lineto(168.00683594,454.34960938)
\curveto(168.20214844,454.65429688)(168.46191406,454.8984375)(168.78613281,455.08203125)
\curveto(169.11035156,455.26953125)(169.47949219,455.36328125)(169.89355469,455.36328125)
\curveto(170.35449219,455.36328125)(170.73144531,455.26757812)(171.02441406,455.07617188)
\curveto(171.32128906,454.88476562)(171.53027344,454.6171875)(171.65136719,454.2734375)
\curveto(172.14355469,455)(172.78417969,455.36328125)(173.57324219,455.36328125)
\curveto(174.19042969,455.36328125)(174.66503906,455.19140625)(174.99707031,454.84765625)
\curveto(175.32910156,454.5078125)(175.49511719,453.98242188)(175.49511719,453.27148438)
\lineto(175.49511719,449)
\lineto(174.44628906,449)
\lineto(174.44628906,452.91992188)
\curveto(174.44628906,453.34179688)(174.41113281,453.64453125)(174.34082031,453.828125)
\curveto(174.27441406,454.015625)(174.15136719,454.16601562)(173.97167969,454.27929688)
\curveto(173.79199219,454.39257812)(173.58105469,454.44921875)(173.33886719,454.44921875)
\curveto(172.90136719,454.44921875)(172.53808594,454.30273438)(172.24902344,454.00976562)
\curveto(171.95996094,453.72070312)(171.81542969,453.25585938)(171.81542969,452.61523438)
\lineto(171.81542969,449)
\lineto(170.76074219,449)
\lineto(170.76074219,453.04296875)
\curveto(170.76074219,453.51171875)(170.67480469,453.86328125)(170.50292969,454.09765625)
\curveto(170.33105469,454.33203125)(170.04980469,454.44921875)(169.65917969,454.44921875)
\curveto(169.36230469,454.44921875)(169.08691406,454.37109375)(168.83300781,454.21484375)
\curveto(168.58300781,454.05859375)(168.40136719,453.83007812)(168.28808594,453.52929688)
\curveto(168.17480469,453.22851562)(168.11816406,452.79492188)(168.11816406,452.22851562)
\lineto(168.11816406,449)
\closepath
}
}
{
\newrgbcolor{curcolor}{0 0 0}
\pscustom[linestyle=none,fillstyle=solid,fillcolor=curcolor]
{
\newpath
\moveto(181.31933594,451.00390625)
\lineto(182.40917969,450.86914062)
\curveto(182.23730469,450.23242188)(181.91894531,449.73828125)(181.45410156,449.38671875)
\curveto(180.98925781,449.03515625)(180.39550781,448.859375)(179.67285156,448.859375)
\curveto(178.76269531,448.859375)(178.04003906,449.13867188)(177.50488281,449.69726562)
\curveto(176.97363281,450.25976562)(176.70800781,451.046875)(176.70800781,452.05859375)
\curveto(176.70800781,453.10546875)(176.97753906,453.91796875)(177.51660156,454.49609375)
\curveto(178.05566406,455.07421875)(178.75488281,455.36328125)(179.61425781,455.36328125)
\curveto(180.44628906,455.36328125)(181.12597656,455.08007812)(181.65332031,454.51367188)
\curveto(182.18066406,453.94726562)(182.44433594,453.15039062)(182.44433594,452.12304688)
\curveto(182.44433594,452.06054688)(182.44238281,451.96679688)(182.43847656,451.84179688)
\lineto(177.79785156,451.84179688)
\curveto(177.83691406,451.15820312)(178.03027344,450.63476562)(178.37792969,450.27148438)
\curveto(178.72558594,449.90820312)(179.15917969,449.7265625)(179.67871094,449.7265625)
\curveto(180.06542969,449.7265625)(180.39550781,449.828125)(180.66894531,450.03125)
\curveto(180.94238281,450.234375)(181.15917969,450.55859375)(181.31933594,451.00390625)
\closepath
\moveto(177.85644531,452.70898438)
\lineto(181.33105469,452.70898438)
\curveto(181.28417969,453.23242188)(181.15136719,453.625)(180.93261719,453.88671875)
\curveto(180.59667969,454.29296875)(180.16113281,454.49609375)(179.62597656,454.49609375)
\curveto(179.14160156,454.49609375)(178.73339844,454.33398438)(178.40136719,454.00976562)
\curveto(178.07324219,453.68554688)(177.89160156,453.25195312)(177.85644531,452.70898438)
\closepath
}
}
{
\newrgbcolor{curcolor}{0 0 0}
\pscustom[linestyle=none,fillstyle=solid,fillcolor=curcolor]
{
\newpath
\moveto(186.85644531,453.18359375)
\curveto(186.85644531,454.609375)(187.23925781,455.72460938)(188.00488281,456.52929688)
\curveto(188.77050781,457.33789062)(189.75878906,457.7421875)(190.96972656,457.7421875)
\curveto(191.76269531,457.7421875)(192.47753906,457.55273438)(193.11425781,457.17382812)
\curveto(193.75097656,456.79492188)(194.23535156,456.265625)(194.56738281,455.5859375)
\curveto(194.90332031,454.91015625)(195.07128906,454.14257812)(195.07128906,453.28320312)
\curveto(195.07128906,452.41210938)(194.89550781,451.6328125)(194.54394531,450.9453125)
\curveto(194.19238281,450.2578125)(193.69433594,449.73632812)(193.04980469,449.38085938)
\curveto(192.40527344,449.02929688)(191.70996094,448.85351562)(190.96386719,448.85351562)
\curveto(190.15527344,448.85351562)(189.43261719,449.04882812)(188.79589844,449.43945312)
\curveto(188.15917969,449.83007812)(187.67675781,450.36328125)(187.34863281,451.0390625)
\curveto(187.02050781,451.71484375)(186.85644531,452.4296875)(186.85644531,453.18359375)
\closepath
\moveto(188.02832031,453.16601562)
\curveto(188.02832031,452.13085938)(188.30566406,451.31445312)(188.86035156,450.71679688)
\curveto(189.41894531,450.12304688)(190.11816406,449.82617188)(190.95800781,449.82617188)
\curveto(191.81347656,449.82617188)(192.51660156,450.12695312)(193.06738281,450.72851562)
\curveto(193.62207031,451.33007812)(193.89941406,452.18359375)(193.89941406,453.2890625)
\curveto(193.89941406,453.98828125)(193.78027344,454.59765625)(193.54199219,455.1171875)
\curveto(193.30761719,455.640625)(192.96191406,456.04492188)(192.50488281,456.33007812)
\curveto(192.05175781,456.61914062)(191.54199219,456.76367188)(190.97558594,456.76367188)
\curveto(190.17089844,456.76367188)(189.47753906,456.48632812)(188.89550781,455.93164062)
\curveto(188.31738281,455.38085938)(188.02832031,454.45898438)(188.02832031,453.16601562)
\closepath
}
}
{
\newrgbcolor{curcolor}{0 0 0}
\pscustom[linestyle=none,fillstyle=solid,fillcolor=curcolor]
{
\newpath
\moveto(196.37792969,449)
\lineto(196.37792969,457.58984375)
\lineto(197.43261719,457.58984375)
\lineto(197.43261719,449)
\closepath
}
}
{
\newrgbcolor{curcolor}{0 0 0}
\pscustom[linestyle=none,fillstyle=solid,fillcolor=curcolor]
{
\newpath
\moveto(203.10449219,449)
\lineto(203.10449219,449.78515625)
\curveto(202.70996094,449.16796875)(202.12988281,448.859375)(201.36425781,448.859375)
\curveto(200.86816406,448.859375)(200.41113281,448.99609375)(199.99316406,449.26953125)
\curveto(199.57910156,449.54296875)(199.25683594,449.92382812)(199.02636719,450.41210938)
\curveto(198.79980469,450.90429688)(198.68652344,451.46875)(198.68652344,452.10546875)
\curveto(198.68652344,452.7265625)(198.79003906,453.2890625)(198.99707031,453.79296875)
\curveto(199.20410156,454.30078125)(199.51464844,454.68945312)(199.92871094,454.95898438)
\curveto(200.34277344,455.22851562)(200.80566406,455.36328125)(201.31738281,455.36328125)
\curveto(201.69238281,455.36328125)(202.02636719,455.28320312)(202.31933594,455.12304688)
\curveto(202.61230469,454.96679688)(202.85058594,454.76171875)(203.03417969,454.5078125)
\lineto(203.03417969,457.58984375)
\lineto(204.08300781,457.58984375)
\lineto(204.08300781,449)
\closepath
\moveto(199.77050781,452.10546875)
\curveto(199.77050781,451.30859375)(199.93847656,450.71289062)(200.27441406,450.31835938)
\curveto(200.61035156,449.92382812)(201.00683594,449.7265625)(201.46386719,449.7265625)
\curveto(201.92480469,449.7265625)(202.31542969,449.9140625)(202.63574219,450.2890625)
\curveto(202.95996094,450.66796875)(203.12207031,451.24414062)(203.12207031,452.01757812)
\curveto(203.12207031,452.86914062)(202.95800781,453.49414062)(202.62988281,453.89257812)
\curveto(202.30175781,454.29101562)(201.89746094,454.49023438)(201.41699219,454.49023438)
\curveto(200.94824219,454.49023438)(200.55566406,454.29882812)(200.23925781,453.91601562)
\curveto(199.92675781,453.53320312)(199.77050781,452.9296875)(199.77050781,452.10546875)
\closepath
}
}
{
\newrgbcolor{curcolor}{0 0 0}
\pscustom[linestyle=none,fillstyle=solid,fillcolor=curcolor]
{
\newpath
\moveto(208.52441406,449)
\lineto(208.52441406,450.0546875)
\lineto(212.92480469,455.55664062)
\curveto(213.23730469,455.94726562)(213.53417969,456.28710938)(213.81542969,456.57617188)
\lineto(209.02246094,456.57617188)
\lineto(209.02246094,457.58984375)
\lineto(215.17480469,457.58984375)
\lineto(215.17480469,456.57617188)
\lineto(210.35253906,450.6171875)
\lineto(209.83105469,450.01367188)
\lineto(215.31542969,450.01367188)
\lineto(215.31542969,449)
\closepath
}
}
{
\newrgbcolor{curcolor}{0 0 0}
\pscustom[linestyle=none,fillstyle=solid,fillcolor=curcolor]
{
\newpath
\moveto(216.59863281,449)
\lineto(216.59863281,457.58984375)
\lineto(222.39355469,457.58984375)
\lineto(222.39355469,456.57617188)
\lineto(217.73535156,456.57617188)
\lineto(217.73535156,453.91601562)
\lineto(221.76660156,453.91601562)
\lineto(221.76660156,452.90234375)
\lineto(217.73535156,452.90234375)
\lineto(217.73535156,449)
\closepath
}
}
{
\newrgbcolor{curcolor}{0 0 0}
\pscustom[linestyle=none,fillstyle=solid,fillcolor=curcolor]
{
\newpath
\moveto(223.48339844,451.75976562)
\lineto(224.55566406,451.85351562)
\curveto(224.60644531,451.42382812)(224.72363281,451.0703125)(224.90722656,450.79296875)
\curveto(225.09472656,450.51953125)(225.38378906,450.296875)(225.77441406,450.125)
\curveto(226.16503906,449.95703125)(226.60449219,449.87304688)(227.09277344,449.87304688)
\curveto(227.52636719,449.87304688)(227.90917969,449.9375)(228.24121094,450.06640625)
\curveto(228.57324219,450.1953125)(228.81933594,450.37109375)(228.97949219,450.59375)
\curveto(229.14355469,450.8203125)(229.22558594,451.06640625)(229.22558594,451.33203125)
\curveto(229.22558594,451.6015625)(229.14746094,451.8359375)(228.99121094,452.03515625)
\curveto(228.83496094,452.23828125)(228.57714844,452.40820312)(228.21777344,452.54492188)
\curveto(227.98730469,452.63476562)(227.47753906,452.7734375)(226.68847656,452.9609375)
\curveto(225.89941406,453.15234375)(225.34667969,453.33203125)(225.03027344,453.5)
\curveto(224.62011719,453.71484375)(224.31347656,453.98046875)(224.11035156,454.296875)
\curveto(223.91113281,454.6171875)(223.81152344,454.97460938)(223.81152344,455.36914062)
\curveto(223.81152344,455.80273438)(223.93457031,456.20703125)(224.18066406,456.58203125)
\curveto(224.42675781,456.9609375)(224.78613281,457.24804688)(225.25878906,457.44335938)
\curveto(225.73144531,457.63867188)(226.25683594,457.73632812)(226.83496094,457.73632812)
\curveto(227.47167969,457.73632812)(228.03222656,457.6328125)(228.51660156,457.42578125)
\curveto(229.00488281,457.22265625)(229.37988281,456.921875)(229.64160156,456.5234375)
\curveto(229.90332031,456.125)(230.04394531,455.67382812)(230.06347656,455.16992188)
\lineto(228.97363281,455.08789062)
\curveto(228.91503906,455.63085938)(228.71582031,456.04101562)(228.37597656,456.31835938)
\curveto(228.04003906,456.59570312)(227.54199219,456.734375)(226.88183594,456.734375)
\curveto(226.19433594,456.734375)(225.69238281,456.60742188)(225.37597656,456.35351562)
\curveto(225.06347656,456.10351562)(224.90722656,455.80078125)(224.90722656,455.4453125)
\curveto(224.90722656,455.13671875)(225.01855469,454.8828125)(225.24121094,454.68359375)
\curveto(225.45996094,454.484375)(226.03027344,454.27929688)(226.95214844,454.06835938)
\curveto(227.87792969,453.86132812)(228.51269531,453.6796875)(228.85644531,453.5234375)
\curveto(229.35644531,453.29296875)(229.72558594,453)(229.96386719,452.64453125)
\curveto(230.20214844,452.29296875)(230.32128906,451.88671875)(230.32128906,451.42578125)
\curveto(230.32128906,450.96875)(230.19042969,450.53710938)(229.92871094,450.13085938)
\curveto(229.66699219,449.72851562)(229.29003906,449.4140625)(228.79785156,449.1875)
\curveto(228.30957031,448.96484375)(227.75878906,448.85351562)(227.14550781,448.85351562)
\curveto(226.36816406,448.85351562)(225.71582031,448.96679688)(225.18847656,449.19335938)
\curveto(224.66503906,449.41992188)(224.25292969,449.75976562)(223.95214844,450.21289062)
\curveto(223.65527344,450.66992188)(223.49902344,451.18554688)(223.48339844,451.75976562)
\closepath
}
}
{
\newrgbcolor{curcolor}{0 0 0}
\pscustom[linestyle=none,fillstyle=solid,fillcolor=curcolor]
{
\newpath
\moveto(238.75292969,449)
\lineto(237.69824219,449)
\lineto(237.69824219,455.72070312)
\curveto(237.44433594,455.47851562)(237.11035156,455.23632812)(236.69628906,454.99414062)
\curveto(236.28613281,454.75195312)(235.91699219,454.5703125)(235.58886719,454.44921875)
\lineto(235.58886719,455.46875)
\curveto(236.17871094,455.74609375)(236.69433594,456.08203125)(237.13574219,456.4765625)
\curveto(237.57714844,456.87109375)(237.88964844,457.25390625)(238.07324219,457.625)
\lineto(238.75292969,457.625)
\closepath
}
}
{
\newrgbcolor{curcolor}{0 0 0}
\pscustom[linestyle=none,fillstyle=solid,fillcolor=curcolor]
{
\newpath
\moveto(241.84667969,449)
\lineto(241.84667969,457.58984375)
\lineto(243.55761719,457.58984375)
\lineto(245.59082031,451.5078125)
\curveto(245.77832031,450.94140625)(245.91503906,450.51757812)(246.00097656,450.23632812)
\curveto(246.09863281,450.54882812)(246.25097656,451.0078125)(246.45800781,451.61328125)
\lineto(248.51464844,457.58984375)
\lineto(250.04394531,457.58984375)
\lineto(250.04394531,449)
\lineto(248.94824219,449)
\lineto(248.94824219,456.18945312)
\lineto(246.45214844,449)
\lineto(245.42675781,449)
\lineto(242.94238281,456.3125)
\lineto(242.94238281,449)
\closepath
}
}
{
\newrgbcolor{curcolor}{0 0 0}
\pscustom[linestyle=none,fillstyle=solid,fillcolor=curcolor]
{
\newpath
\moveto(255.22949219,449)
\lineto(255.22949219,457.58984375)
\lineto(259.03808594,457.58984375)
\curveto(259.80371094,457.58984375)(260.38574219,457.51171875)(260.78417969,457.35546875)
\curveto(261.18261719,457.203125)(261.50097656,456.93164062)(261.73925781,456.54101562)
\curveto(261.97753906,456.15039062)(262.09667969,455.71875)(262.09667969,455.24609375)
\curveto(262.09667969,454.63671875)(261.89941406,454.12304688)(261.50488281,453.70507812)
\curveto(261.11035156,453.28710938)(260.50097656,453.02148438)(259.67675781,452.90820312)
\curveto(259.97753906,452.76367188)(260.20605469,452.62109375)(260.36230469,452.48046875)
\curveto(260.69433594,452.17578125)(261.00878906,451.79492188)(261.30566406,451.33789062)
\lineto(262.79980469,449)
\lineto(261.37011719,449)
\lineto(260.23339844,450.78710938)
\curveto(259.90136719,451.30273438)(259.62792969,451.69726562)(259.41308594,451.97070312)
\curveto(259.19824219,452.24414062)(259.00488281,452.43554688)(258.83300781,452.54492188)
\curveto(258.66503906,452.65429688)(258.49316406,452.73046875)(258.31738281,452.7734375)
\curveto(258.18847656,452.80078125)(257.97753906,452.81445312)(257.68457031,452.81445312)
\lineto(256.36621094,452.81445312)
\lineto(256.36621094,449)
\closepath
\moveto(256.36621094,453.79882812)
\lineto(258.80957031,453.79882812)
\curveto(259.32910156,453.79882812)(259.73535156,453.8515625)(260.02832031,453.95703125)
\curveto(260.32128906,454.06640625)(260.54394531,454.23828125)(260.69628906,454.47265625)
\curveto(260.84863281,454.7109375)(260.92480469,454.96875)(260.92480469,455.24609375)
\curveto(260.92480469,455.65234375)(260.77636719,455.98632812)(260.47949219,456.24804688)
\curveto(260.18652344,456.50976562)(259.72167969,456.640625)(259.08496094,456.640625)
\lineto(256.36621094,456.640625)
\closepath
}
}
{
\newrgbcolor{curcolor}{0 0 0}
\pscustom[linestyle=none,fillstyle=solid,fillcolor=curcolor]
{
\newpath
\moveto(268.00292969,451.00390625)
\lineto(269.09277344,450.86914062)
\curveto(268.92089844,450.23242188)(268.60253906,449.73828125)(268.13769531,449.38671875)
\curveto(267.67285156,449.03515625)(267.07910156,448.859375)(266.35644531,448.859375)
\curveto(265.44628906,448.859375)(264.72363281,449.13867188)(264.18847656,449.69726562)
\curveto(263.65722656,450.25976562)(263.39160156,451.046875)(263.39160156,452.05859375)
\curveto(263.39160156,453.10546875)(263.66113281,453.91796875)(264.20019531,454.49609375)
\curveto(264.73925781,455.07421875)(265.43847656,455.36328125)(266.29785156,455.36328125)
\curveto(267.12988281,455.36328125)(267.80957031,455.08007812)(268.33691406,454.51367188)
\curveto(268.86425781,453.94726562)(269.12792969,453.15039062)(269.12792969,452.12304688)
\curveto(269.12792969,452.06054688)(269.12597656,451.96679688)(269.12207031,451.84179688)
\lineto(264.48144531,451.84179688)
\curveto(264.52050781,451.15820312)(264.71386719,450.63476562)(265.06152344,450.27148438)
\curveto(265.40917969,449.90820312)(265.84277344,449.7265625)(266.36230469,449.7265625)
\curveto(266.74902344,449.7265625)(267.07910156,449.828125)(267.35253906,450.03125)
\curveto(267.62597656,450.234375)(267.84277344,450.55859375)(268.00292969,451.00390625)
\closepath
\moveto(264.54003906,452.70898438)
\lineto(268.01464844,452.70898438)
\curveto(267.96777344,453.23242188)(267.83496094,453.625)(267.61621094,453.88671875)
\curveto(267.28027344,454.29296875)(266.84472656,454.49609375)(266.30957031,454.49609375)
\curveto(265.82519531,454.49609375)(265.41699219,454.33398438)(265.08496094,454.00976562)
\curveto(264.75683594,453.68554688)(264.57519531,453.25195312)(264.54003906,452.70898438)
\closepath
}
}
{
\newrgbcolor{curcolor}{0 0 0}
\pscustom[linestyle=none,fillstyle=solid,fillcolor=curcolor]
{
\newpath
\moveto(274.47753906,449.76757812)
\curveto(274.08691406,449.43554688)(273.70996094,449.20117188)(273.34667969,449.06445312)
\curveto(272.98730469,448.92773438)(272.60058594,448.859375)(272.18652344,448.859375)
\curveto(271.50292969,448.859375)(270.97753906,449.02539062)(270.61035156,449.35742188)
\curveto(270.24316406,449.69335938)(270.05957031,450.12109375)(270.05957031,450.640625)
\curveto(270.05957031,450.9453125)(270.12792969,451.22265625)(270.26464844,451.47265625)
\curveto(270.40527344,451.7265625)(270.58691406,451.9296875)(270.80957031,452.08203125)
\curveto(271.03613281,452.234375)(271.29003906,452.34960938)(271.57128906,452.42773438)
\curveto(271.77832031,452.48242188)(272.09082031,452.53515625)(272.50878906,452.5859375)
\curveto(273.36035156,452.6875)(273.98730469,452.80859375)(274.38964844,452.94921875)
\curveto(274.39355469,453.09375)(274.39550781,453.18554688)(274.39550781,453.22460938)
\curveto(274.39550781,453.65429688)(274.29589844,453.95703125)(274.09667969,454.1328125)
\curveto(273.82714844,454.37109375)(273.42675781,454.49023438)(272.89550781,454.49023438)
\curveto(272.39941406,454.49023438)(272.03222656,454.40234375)(271.79394531,454.2265625)
\curveto(271.55957031,454.0546875)(271.38574219,453.74804688)(271.27246094,453.30664062)
\lineto(270.24121094,453.44726562)
\curveto(270.33496094,453.88867188)(270.48925781,454.24414062)(270.70410156,454.51367188)
\curveto(270.91894531,454.78710938)(271.22949219,454.99609375)(271.63574219,455.140625)
\curveto(272.04199219,455.2890625)(272.51269531,455.36328125)(273.04785156,455.36328125)
\curveto(273.57910156,455.36328125)(274.01074219,455.30078125)(274.34277344,455.17578125)
\curveto(274.67480469,455.05078125)(274.91894531,454.89257812)(275.07519531,454.70117188)
\curveto(275.23144531,454.51367188)(275.34082031,454.27539062)(275.40332031,453.98632812)
\curveto(275.43847656,453.80664062)(275.45605469,453.48242188)(275.45605469,453.01367188)
\lineto(275.45605469,451.60742188)
\curveto(275.45605469,450.62695312)(275.47753906,450.00585938)(275.52050781,449.74414062)
\curveto(275.56738281,449.48632812)(275.65722656,449.23828125)(275.79003906,449)
\lineto(274.68847656,449)
\curveto(274.57910156,449.21875)(274.50878906,449.47460938)(274.47753906,449.76757812)
\closepath
\moveto(274.38964844,452.12304688)
\curveto(274.00683594,451.96679688)(273.43261719,451.83398438)(272.66699219,451.72460938)
\curveto(272.23339844,451.66210938)(271.92675781,451.59179688)(271.74707031,451.51367188)
\curveto(271.56738281,451.43554688)(271.42871094,451.3203125)(271.33105469,451.16796875)
\curveto(271.23339844,451.01953125)(271.18457031,450.85351562)(271.18457031,450.66992188)
\curveto(271.18457031,450.38867188)(271.29003906,450.15429688)(271.50097656,449.96679688)
\curveto(271.71582031,449.77929688)(272.02832031,449.68554688)(272.43847656,449.68554688)
\curveto(272.84472656,449.68554688)(273.20605469,449.7734375)(273.52246094,449.94921875)
\curveto(273.83886719,450.12890625)(274.07128906,450.37304688)(274.21972656,450.68164062)
\curveto(274.33300781,450.91992188)(274.38964844,451.27148438)(274.38964844,451.73632812)
\closepath
}
}
{
\newrgbcolor{curcolor}{0 0 0}
\pscustom[linestyle=none,fillstyle=solid,fillcolor=curcolor]
{
\newpath
\moveto(281.12792969,449)
\lineto(281.12792969,449.78515625)
\curveto(280.73339844,449.16796875)(280.15332031,448.859375)(279.38769531,448.859375)
\curveto(278.89160156,448.859375)(278.43457031,448.99609375)(278.01660156,449.26953125)
\curveto(277.60253906,449.54296875)(277.28027344,449.92382812)(277.04980469,450.41210938)
\curveto(276.82324219,450.90429688)(276.70996094,451.46875)(276.70996094,452.10546875)
\curveto(276.70996094,452.7265625)(276.81347656,453.2890625)(277.02050781,453.79296875)
\curveto(277.22753906,454.30078125)(277.53808594,454.68945312)(277.95214844,454.95898438)
\curveto(278.36621094,455.22851562)(278.82910156,455.36328125)(279.34082031,455.36328125)
\curveto(279.71582031,455.36328125)(280.04980469,455.28320312)(280.34277344,455.12304688)
\curveto(280.63574219,454.96679688)(280.87402344,454.76171875)(281.05761719,454.5078125)
\lineto(281.05761719,457.58984375)
\lineto(282.10644531,457.58984375)
\lineto(282.10644531,449)
\closepath
\moveto(277.79394531,452.10546875)
\curveto(277.79394531,451.30859375)(277.96191406,450.71289062)(278.29785156,450.31835938)
\curveto(278.63378906,449.92382812)(279.03027344,449.7265625)(279.48730469,449.7265625)
\curveto(279.94824219,449.7265625)(280.33886719,449.9140625)(280.65917969,450.2890625)
\curveto(280.98339844,450.66796875)(281.14550781,451.24414062)(281.14550781,452.01757812)
\curveto(281.14550781,452.86914062)(280.98144531,453.49414062)(280.65332031,453.89257812)
\curveto(280.32519531,454.29101562)(279.92089844,454.49023438)(279.44042969,454.49023438)
\curveto(278.97167969,454.49023438)(278.57910156,454.29882812)(278.26269531,453.91601562)
\curveto(277.95019531,453.53320312)(277.79394531,452.9296875)(277.79394531,452.10546875)
\closepath
}
}
{
\newrgbcolor{curcolor}{0 0 0}
\pscustom[linestyle=none,fillstyle=solid,fillcolor=curcolor]
{
\newpath
\moveto(283.34277344,450.85742188)
\lineto(284.38574219,451.02148438)
\curveto(284.44433594,450.60351562)(284.60644531,450.28320312)(284.87207031,450.06054688)
\curveto(285.14160156,449.83789062)(285.51660156,449.7265625)(285.99707031,449.7265625)
\curveto(286.48144531,449.7265625)(286.84082031,449.82421875)(287.07519531,450.01953125)
\curveto(287.30957031,450.21875)(287.42675781,450.45117188)(287.42675781,450.71679688)
\curveto(287.42675781,450.95507812)(287.32324219,451.14257812)(287.11621094,451.27929688)
\curveto(286.97167969,451.37304688)(286.61230469,451.4921875)(286.03808594,451.63671875)
\curveto(285.26464844,451.83203125)(284.72753906,452)(284.42675781,452.140625)
\curveto(284.12988281,452.28515625)(283.90332031,452.48242188)(283.74707031,452.73242188)
\curveto(283.59472656,452.98632812)(283.51855469,453.265625)(283.51855469,453.5703125)
\curveto(283.51855469,453.84765625)(283.58105469,454.10351562)(283.70605469,454.33789062)
\curveto(283.83496094,454.57617188)(284.00878906,454.7734375)(284.22753906,454.9296875)
\curveto(284.39160156,455.05078125)(284.61425781,455.15234375)(284.89550781,455.234375)
\curveto(285.18066406,455.3203125)(285.48535156,455.36328125)(285.80957031,455.36328125)
\curveto(286.29785156,455.36328125)(286.72558594,455.29296875)(287.09277344,455.15234375)
\curveto(287.46386719,455.01171875)(287.73730469,454.8203125)(287.91308594,454.578125)
\curveto(288.08886719,454.33984375)(288.20996094,454.01953125)(288.27636719,453.6171875)
\lineto(287.24511719,453.4765625)
\curveto(287.19824219,453.796875)(287.06152344,454.046875)(286.83496094,454.2265625)
\curveto(286.61230469,454.40625)(286.29589844,454.49609375)(285.88574219,454.49609375)
\curveto(285.40136719,454.49609375)(285.05566406,454.41601562)(284.84863281,454.25585938)
\curveto(284.64160156,454.09570312)(284.53808594,453.90820312)(284.53808594,453.69335938)
\curveto(284.53808594,453.55664062)(284.58105469,453.43359375)(284.66699219,453.32421875)
\curveto(284.75292969,453.2109375)(284.88769531,453.1171875)(285.07128906,453.04296875)
\curveto(285.17675781,453.00390625)(285.48730469,452.9140625)(286.00292969,452.7734375)
\curveto(286.74902344,452.57421875)(287.26855469,452.41015625)(287.56152344,452.28125)
\curveto(287.85839844,452.15625)(288.09082031,451.97265625)(288.25878906,451.73046875)
\curveto(288.42675781,451.48828125)(288.51074219,451.1875)(288.51074219,450.828125)
\curveto(288.51074219,450.4765625)(288.40722656,450.14453125)(288.20019531,449.83203125)
\curveto(287.99707031,449.5234375)(287.70214844,449.28320312)(287.31542969,449.11132812)
\curveto(286.92871094,448.94335938)(286.49121094,448.859375)(286.00292969,448.859375)
\curveto(285.19433594,448.859375)(284.57714844,449.02734375)(284.15136719,449.36328125)
\curveto(283.72949219,449.69921875)(283.45996094,450.19726562)(283.34277344,450.85742188)
\closepath
}
}
{
\newrgbcolor{curcolor}{0 0 0}
\pscustom[linestyle=none,fillstyle=solid,fillcolor=curcolor]
{
\newpath
\moveto(293.22167969,449)
\lineto(293.22167969,457.58984375)
\lineto(294.38769531,457.58984375)
\lineto(298.89941406,450.84570312)
\lineto(298.89941406,457.58984375)
\lineto(299.98925781,457.58984375)
\lineto(299.98925781,449)
\lineto(298.82324219,449)
\lineto(294.31152344,455.75)
\lineto(294.31152344,449)
\closepath
}
}
{
\newrgbcolor{curcolor}{0 0 0}
\pscustom[linestyle=none,fillstyle=solid,fillcolor=curcolor]
{
\newpath
\moveto(307.53613281,457.58984375)
\lineto(308.67285156,457.58984375)
\lineto(308.67285156,452.62695312)
\curveto(308.67285156,451.76367188)(308.57519531,451.078125)(308.37988281,450.5703125)
\curveto(308.18457031,450.0625)(307.83105469,449.6484375)(307.31933594,449.328125)
\curveto(306.81152344,449.01171875)(306.14355469,448.85351562)(305.31542969,448.85351562)
\curveto(304.51074219,448.85351562)(303.85253906,448.9921875)(303.34082031,449.26953125)
\curveto(302.82910156,449.546875)(302.46386719,449.94726562)(302.24511719,450.47070312)
\curveto(302.02636719,450.99804688)(301.91699219,451.71679688)(301.91699219,452.62695312)
\lineto(301.91699219,457.58984375)
\lineto(303.05371094,457.58984375)
\lineto(303.05371094,452.6328125)
\curveto(303.05371094,451.88671875)(303.12207031,451.3359375)(303.25878906,450.98046875)
\curveto(303.39941406,450.62890625)(303.63769531,450.35742188)(303.97363281,450.16601562)
\curveto(304.31347656,449.97460938)(304.72753906,449.87890625)(305.21582031,449.87890625)
\curveto(306.05175781,449.87890625)(306.64746094,450.06835938)(307.00292969,450.44726562)
\curveto(307.35839844,450.82617188)(307.53613281,451.5546875)(307.53613281,452.6328125)
\closepath
}
}
{
\newrgbcolor{curcolor}{0 0 0}
\pscustom[linestyle=none,fillstyle=solid,fillcolor=curcolor]
{
\newpath
\moveto(310.53027344,449)
\lineto(310.53027344,457.58984375)
\lineto(312.24121094,457.58984375)
\lineto(314.27441406,451.5078125)
\curveto(314.46191406,450.94140625)(314.59863281,450.51757812)(314.68457031,450.23632812)
\curveto(314.78222656,450.54882812)(314.93457031,451.0078125)(315.14160156,451.61328125)
\lineto(317.19824219,457.58984375)
\lineto(318.72753906,457.58984375)
\lineto(318.72753906,449)
\lineto(317.63183594,449)
\lineto(317.63183594,456.18945312)
\lineto(315.13574219,449)
\lineto(314.11035156,449)
\lineto(311.62597656,456.3125)
\lineto(311.62597656,449)
\closepath
}
}
{
\newrgbcolor{curcolor}{0 0 0}
\pscustom[linestyle=none,fillstyle=solid,fillcolor=curcolor]
{
\newpath
\moveto(319.61816406,449)
\lineto(322.91699219,457.58984375)
\lineto(324.14160156,457.58984375)
\lineto(327.65722656,449)
\lineto(326.36230469,449)
\lineto(325.36035156,451.6015625)
\lineto(321.76855469,451.6015625)
\lineto(320.82519531,449)
\closepath
\moveto(322.09667969,452.52734375)
\lineto(325.00878906,452.52734375)
\lineto(324.11230469,454.90625)
\curveto(323.83886719,455.62890625)(323.63574219,456.22265625)(323.50292969,456.6875)
\curveto(323.39355469,456.13671875)(323.23925781,455.58984375)(323.04003906,455.046875)
\closepath
}
}
{
\newrgbcolor{curcolor}{0 0 0}
\pscustom[linestyle=none,fillstyle=solid,fillcolor=curcolor]
{
\newpath
\moveto(331.19042969,449)
\lineto(331.19042969,457.58984375)
\lineto(334.41308594,457.58984375)
\curveto(335.06933594,457.58984375)(335.59472656,457.50195312)(335.98925781,457.32617188)
\curveto(336.38769531,457.15429688)(336.69824219,456.88671875)(336.92089844,456.5234375)
\curveto(337.14746094,456.1640625)(337.26074219,455.78710938)(337.26074219,455.39257812)
\curveto(337.26074219,455.02539062)(337.16113281,454.6796875)(336.96191406,454.35546875)
\curveto(336.76269531,454.03125)(336.46191406,453.76953125)(336.05957031,453.5703125)
\curveto(336.57910156,453.41796875)(336.97753906,453.15820312)(337.25488281,452.79101562)
\curveto(337.53613281,452.42382812)(337.67675781,451.99023438)(337.67675781,451.49023438)
\curveto(337.67675781,451.08789062)(337.59082031,450.71289062)(337.41894531,450.36523438)
\curveto(337.25097656,450.02148438)(337.04199219,449.75585938)(336.79199219,449.56835938)
\curveto(336.54199219,449.38085938)(336.22753906,449.23828125)(335.84863281,449.140625)
\curveto(335.47363281,449.046875)(335.01269531,449)(334.46582031,449)
\closepath
\moveto(332.32714844,453.98046875)
\lineto(334.18457031,453.98046875)
\curveto(334.68847656,453.98046875)(335.04980469,454.01367188)(335.26855469,454.08007812)
\curveto(335.55761719,454.16601562)(335.77441406,454.30859375)(335.91894531,454.5078125)
\curveto(336.06738281,454.70703125)(336.14160156,454.95703125)(336.14160156,455.2578125)
\curveto(336.14160156,455.54296875)(336.07324219,455.79296875)(335.93652344,456.0078125)
\curveto(335.79980469,456.2265625)(335.60449219,456.375)(335.35058594,456.453125)
\curveto(335.09667969,456.53515625)(334.66113281,456.57617188)(334.04394531,456.57617188)
\lineto(332.32714844,456.57617188)
\closepath
\moveto(332.32714844,450.01367188)
\lineto(334.46582031,450.01367188)
\curveto(334.83300781,450.01367188)(335.09082031,450.02734375)(335.23925781,450.0546875)
\curveto(335.50097656,450.1015625)(335.71972656,450.1796875)(335.89550781,450.2890625)
\curveto(336.07128906,450.3984375)(336.21582031,450.55664062)(336.32910156,450.76367188)
\curveto(336.44238281,450.97460938)(336.49902344,451.21679688)(336.49902344,451.49023438)
\curveto(336.49902344,451.81054688)(336.41699219,452.08789062)(336.25292969,452.32226562)
\curveto(336.08886719,452.56054688)(335.86035156,452.7265625)(335.56738281,452.8203125)
\curveto(335.27832031,452.91796875)(334.86035156,452.96679688)(334.31347656,452.96679688)
\lineto(332.32714844,452.96679688)
\closepath
}
}
{
\newrgbcolor{curcolor}{0 0 0}
\pscustom[linestyle=none,fillstyle=solid,fillcolor=curcolor]
{
\newpath
\moveto(343.16699219,449.76757812)
\curveto(342.77636719,449.43554688)(342.39941406,449.20117188)(342.03613281,449.06445312)
\curveto(341.67675781,448.92773438)(341.29003906,448.859375)(340.87597656,448.859375)
\curveto(340.19238281,448.859375)(339.66699219,449.02539062)(339.29980469,449.35742188)
\curveto(338.93261719,449.69335938)(338.74902344,450.12109375)(338.74902344,450.640625)
\curveto(338.74902344,450.9453125)(338.81738281,451.22265625)(338.95410156,451.47265625)
\curveto(339.09472656,451.7265625)(339.27636719,451.9296875)(339.49902344,452.08203125)
\curveto(339.72558594,452.234375)(339.97949219,452.34960938)(340.26074219,452.42773438)
\curveto(340.46777344,452.48242188)(340.78027344,452.53515625)(341.19824219,452.5859375)
\curveto(342.04980469,452.6875)(342.67675781,452.80859375)(343.07910156,452.94921875)
\curveto(343.08300781,453.09375)(343.08496094,453.18554688)(343.08496094,453.22460938)
\curveto(343.08496094,453.65429688)(342.98535156,453.95703125)(342.78613281,454.1328125)
\curveto(342.51660156,454.37109375)(342.11621094,454.49023438)(341.58496094,454.49023438)
\curveto(341.08886719,454.49023438)(340.72167969,454.40234375)(340.48339844,454.2265625)
\curveto(340.24902344,454.0546875)(340.07519531,453.74804688)(339.96191406,453.30664062)
\lineto(338.93066406,453.44726562)
\curveto(339.02441406,453.88867188)(339.17871094,454.24414062)(339.39355469,454.51367188)
\curveto(339.60839844,454.78710938)(339.91894531,454.99609375)(340.32519531,455.140625)
\curveto(340.73144531,455.2890625)(341.20214844,455.36328125)(341.73730469,455.36328125)
\curveto(342.26855469,455.36328125)(342.70019531,455.30078125)(343.03222656,455.17578125)
\curveto(343.36425781,455.05078125)(343.60839844,454.89257812)(343.76464844,454.70117188)
\curveto(343.92089844,454.51367188)(344.03027344,454.27539062)(344.09277344,453.98632812)
\curveto(344.12792969,453.80664062)(344.14550781,453.48242188)(344.14550781,453.01367188)
\lineto(344.14550781,451.60742188)
\curveto(344.14550781,450.62695312)(344.16699219,450.00585938)(344.20996094,449.74414062)
\curveto(344.25683594,449.48632812)(344.34667969,449.23828125)(344.47949219,449)
\lineto(343.37792969,449)
\curveto(343.26855469,449.21875)(343.19824219,449.47460938)(343.16699219,449.76757812)
\closepath
\moveto(343.07910156,452.12304688)
\curveto(342.69628906,451.96679688)(342.12207031,451.83398438)(341.35644531,451.72460938)
\curveto(340.92285156,451.66210938)(340.61621094,451.59179688)(340.43652344,451.51367188)
\curveto(340.25683594,451.43554688)(340.11816406,451.3203125)(340.02050781,451.16796875)
\curveto(339.92285156,451.01953125)(339.87402344,450.85351562)(339.87402344,450.66992188)
\curveto(339.87402344,450.38867188)(339.97949219,450.15429688)(340.19042969,449.96679688)
\curveto(340.40527344,449.77929688)(340.71777344,449.68554688)(341.12792969,449.68554688)
\curveto(341.53417969,449.68554688)(341.89550781,449.7734375)(342.21191406,449.94921875)
\curveto(342.52832031,450.12890625)(342.76074219,450.37304688)(342.90917969,450.68164062)
\curveto(343.02246094,450.91992188)(343.07910156,451.27148438)(343.07910156,451.73632812)
\closepath
}
}
{
\newrgbcolor{curcolor}{0 0 0}
\pscustom[linestyle=none,fillstyle=solid,fillcolor=curcolor]
{
\newpath
\moveto(345.75683594,449)
\lineto(345.75683594,457.58984375)
\lineto(346.81152344,457.58984375)
\lineto(346.81152344,449)
\closepath
}
}
{
\newrgbcolor{curcolor}{0 0 0}
\pscustom[linestyle=none,fillstyle=solid,fillcolor=curcolor]
{
\newpath
\moveto(352.50683594,449.76757812)
\curveto(352.11621094,449.43554688)(351.73925781,449.20117188)(351.37597656,449.06445312)
\curveto(351.01660156,448.92773438)(350.62988281,448.859375)(350.21582031,448.859375)
\curveto(349.53222656,448.859375)(349.00683594,449.02539062)(348.63964844,449.35742188)
\curveto(348.27246094,449.69335938)(348.08886719,450.12109375)(348.08886719,450.640625)
\curveto(348.08886719,450.9453125)(348.15722656,451.22265625)(348.29394531,451.47265625)
\curveto(348.43457031,451.7265625)(348.61621094,451.9296875)(348.83886719,452.08203125)
\curveto(349.06542969,452.234375)(349.31933594,452.34960938)(349.60058594,452.42773438)
\curveto(349.80761719,452.48242188)(350.12011719,452.53515625)(350.53808594,452.5859375)
\curveto(351.38964844,452.6875)(352.01660156,452.80859375)(352.41894531,452.94921875)
\curveto(352.42285156,453.09375)(352.42480469,453.18554688)(352.42480469,453.22460938)
\curveto(352.42480469,453.65429688)(352.32519531,453.95703125)(352.12597656,454.1328125)
\curveto(351.85644531,454.37109375)(351.45605469,454.49023438)(350.92480469,454.49023438)
\curveto(350.42871094,454.49023438)(350.06152344,454.40234375)(349.82324219,454.2265625)
\curveto(349.58886719,454.0546875)(349.41503906,453.74804688)(349.30175781,453.30664062)
\lineto(348.27050781,453.44726562)
\curveto(348.36425781,453.88867188)(348.51855469,454.24414062)(348.73339844,454.51367188)
\curveto(348.94824219,454.78710938)(349.25878906,454.99609375)(349.66503906,455.140625)
\curveto(350.07128906,455.2890625)(350.54199219,455.36328125)(351.07714844,455.36328125)
\curveto(351.60839844,455.36328125)(352.04003906,455.30078125)(352.37207031,455.17578125)
\curveto(352.70410156,455.05078125)(352.94824219,454.89257812)(353.10449219,454.70117188)
\curveto(353.26074219,454.51367188)(353.37011719,454.27539062)(353.43261719,453.98632812)
\curveto(353.46777344,453.80664062)(353.48535156,453.48242188)(353.48535156,453.01367188)
\lineto(353.48535156,451.60742188)
\curveto(353.48535156,450.62695312)(353.50683594,450.00585938)(353.54980469,449.74414062)
\curveto(353.59667969,449.48632812)(353.68652344,449.23828125)(353.81933594,449)
\lineto(352.71777344,449)
\curveto(352.60839844,449.21875)(352.53808594,449.47460938)(352.50683594,449.76757812)
\closepath
\moveto(352.41894531,452.12304688)
\curveto(352.03613281,451.96679688)(351.46191406,451.83398438)(350.69628906,451.72460938)
\curveto(350.26269531,451.66210938)(349.95605469,451.59179688)(349.77636719,451.51367188)
\curveto(349.59667969,451.43554688)(349.45800781,451.3203125)(349.36035156,451.16796875)
\curveto(349.26269531,451.01953125)(349.21386719,450.85351562)(349.21386719,450.66992188)
\curveto(349.21386719,450.38867188)(349.31933594,450.15429688)(349.53027344,449.96679688)
\curveto(349.74511719,449.77929688)(350.05761719,449.68554688)(350.46777344,449.68554688)
\curveto(350.87402344,449.68554688)(351.23535156,449.7734375)(351.55175781,449.94921875)
\curveto(351.86816406,450.12890625)(352.10058594,450.37304688)(352.24902344,450.68164062)
\curveto(352.36230469,450.91992188)(352.41894531,451.27148438)(352.41894531,451.73632812)
\closepath
}
}
{
\newrgbcolor{curcolor}{0 0 0}
\pscustom[linestyle=none,fillstyle=solid,fillcolor=curcolor]
{
\newpath
\moveto(355.12011719,449)
\lineto(355.12011719,455.22265625)
\lineto(356.06933594,455.22265625)
\lineto(356.06933594,454.33789062)
\curveto(356.52636719,455.02148438)(357.18652344,455.36328125)(358.04980469,455.36328125)
\curveto(358.42480469,455.36328125)(358.76855469,455.29492188)(359.08105469,455.15820312)
\curveto(359.39746094,455.02539062)(359.63378906,454.84960938)(359.79003906,454.63085938)
\curveto(359.94628906,454.41210938)(360.05566406,454.15234375)(360.11816406,453.8515625)
\curveto(360.15722656,453.65625)(360.17675781,453.31445312)(360.17675781,452.82617188)
\lineto(360.17675781,449)
\lineto(359.12207031,449)
\lineto(359.12207031,452.78515625)
\curveto(359.12207031,453.21484375)(359.08105469,453.53515625)(358.99902344,453.74609375)
\curveto(358.91699219,453.9609375)(358.77050781,454.13085938)(358.55957031,454.25585938)
\curveto(358.35253906,454.38476562)(358.10839844,454.44921875)(357.82714844,454.44921875)
\curveto(357.37792969,454.44921875)(356.98925781,454.30664062)(356.66113281,454.02148438)
\curveto(356.33691406,453.73632812)(356.17480469,453.1953125)(356.17480469,452.3984375)
\lineto(356.17480469,449)
\closepath
}
}
{
\newrgbcolor{curcolor}{0 0 0}
\pscustom[linestyle=none,fillstyle=solid,fillcolor=curcolor]
{
\newpath
\moveto(365.85449219,451.27929688)
\lineto(366.89160156,451.14453125)
\curveto(366.77832031,450.4296875)(366.48730469,449.86914062)(366.01855469,449.46289062)
\curveto(365.55371094,449.06054688)(364.98144531,448.859375)(364.30175781,448.859375)
\curveto(363.45019531,448.859375)(362.76464844,449.13671875)(362.24511719,449.69140625)
\curveto(361.72949219,450.25)(361.47167969,451.04882812)(361.47167969,452.08789062)
\curveto(361.47167969,452.75976562)(361.58300781,453.34765625)(361.80566406,453.8515625)
\curveto(362.02832031,454.35546875)(362.36621094,454.73242188)(362.81933594,454.98242188)
\curveto(363.27636719,455.23632812)(363.77246094,455.36328125)(364.30761719,455.36328125)
\curveto(364.98339844,455.36328125)(365.53613281,455.19140625)(365.96582031,454.84765625)
\curveto(366.39550781,454.5078125)(366.67089844,454.0234375)(366.79199219,453.39453125)
\lineto(365.76660156,453.23632812)
\curveto(365.66894531,453.65429688)(365.49511719,453.96875)(365.24511719,454.1796875)
\curveto(364.99902344,454.390625)(364.70019531,454.49609375)(364.34863281,454.49609375)
\curveto(363.81738281,454.49609375)(363.38574219,454.3046875)(363.05371094,453.921875)
\curveto(362.72167969,453.54296875)(362.55566406,452.94140625)(362.55566406,452.1171875)
\curveto(362.55566406,451.28125)(362.71582031,450.67382812)(363.03613281,450.29492188)
\curveto(363.35644531,449.91601562)(363.77441406,449.7265625)(364.29003906,449.7265625)
\curveto(364.70410156,449.7265625)(365.04980469,449.85351562)(365.32714844,450.10742188)
\curveto(365.60449219,450.36132812)(365.78027344,450.75195312)(365.85449219,451.27929688)
\closepath
}
}
{
\newrgbcolor{curcolor}{0 0 0}
\pscustom[linestyle=none,fillstyle=solid,fillcolor=curcolor]
{
\newpath
\moveto(367.79980469,456.37695312)
\lineto(367.79980469,457.58984375)
\lineto(368.85449219,457.58984375)
\lineto(368.85449219,456.37695312)
\closepath
\moveto(367.79980469,449)
\lineto(367.79980469,455.22265625)
\lineto(368.85449219,455.22265625)
\lineto(368.85449219,449)
\closepath
}
}
{
\newrgbcolor{curcolor}{0 0 0}
\pscustom[linestyle=none,fillstyle=solid,fillcolor=curcolor]
{
\newpath
\moveto(370.45996094,449)
\lineto(370.45996094,455.22265625)
\lineto(371.40917969,455.22265625)
\lineto(371.40917969,454.33789062)
\curveto(371.86621094,455.02148438)(372.52636719,455.36328125)(373.38964844,455.36328125)
\curveto(373.76464844,455.36328125)(374.10839844,455.29492188)(374.42089844,455.15820312)
\curveto(374.73730469,455.02539062)(374.97363281,454.84960938)(375.12988281,454.63085938)
\curveto(375.28613281,454.41210938)(375.39550781,454.15234375)(375.45800781,453.8515625)
\curveto(375.49707031,453.65625)(375.51660156,453.31445312)(375.51660156,452.82617188)
\lineto(375.51660156,449)
\lineto(374.46191406,449)
\lineto(374.46191406,452.78515625)
\curveto(374.46191406,453.21484375)(374.42089844,453.53515625)(374.33886719,453.74609375)
\curveto(374.25683594,453.9609375)(374.11035156,454.13085938)(373.89941406,454.25585938)
\curveto(373.69238281,454.38476562)(373.44824219,454.44921875)(373.16699219,454.44921875)
\curveto(372.71777344,454.44921875)(372.32910156,454.30664062)(372.00097656,454.02148438)
\curveto(371.67675781,453.73632812)(371.51464844,453.1953125)(371.51464844,452.3984375)
\lineto(371.51464844,449)
\closepath
}
}
{
\newrgbcolor{curcolor}{0 0 0}
\pscustom[linestyle=none,fillstyle=solid,fillcolor=curcolor]
{
\newpath
\moveto(376.94042969,448.484375)
\lineto(377.96582031,448.33203125)
\curveto(378.00878906,448.015625)(378.12792969,447.78515625)(378.32324219,447.640625)
\curveto(378.58496094,447.4453125)(378.94238281,447.34765625)(379.39550781,447.34765625)
\curveto(379.88378906,447.34765625)(380.26074219,447.4453125)(380.52636719,447.640625)
\curveto(380.79199219,447.8359375)(380.97167969,448.109375)(381.06542969,448.4609375)
\curveto(381.12011719,448.67578125)(381.14550781,449.12695312)(381.14160156,449.81445312)
\curveto(380.68066406,449.27148438)(380.10644531,449)(379.41894531,449)
\curveto(378.56347656,449)(377.90136719,449.30859375)(377.43261719,449.92578125)
\curveto(376.96386719,450.54296875)(376.72949219,451.28320312)(376.72949219,452.14648438)
\curveto(376.72949219,452.74023438)(376.83691406,453.28710938)(377.05175781,453.78710938)
\curveto(377.26660156,454.29101562)(377.57714844,454.6796875)(377.98339844,454.953125)
\curveto(378.39355469,455.2265625)(378.87402344,455.36328125)(379.42480469,455.36328125)
\curveto(380.15917969,455.36328125)(380.76464844,455.06640625)(381.24121094,454.47265625)
\lineto(381.24121094,455.22265625)
\lineto(382.21386719,455.22265625)
\lineto(382.21386719,449.84375)
\curveto(382.21386719,448.875)(382.11425781,448.18945312)(381.91503906,447.78710938)
\curveto(381.71972656,447.38085938)(381.40722656,447.06054688)(380.97753906,446.82617188)
\curveto(380.55175781,446.59179688)(380.02636719,446.47460938)(379.40136719,446.47460938)
\curveto(378.65917969,446.47460938)(378.05957031,446.64257812)(377.60253906,446.97851562)
\curveto(377.14550781,447.31054688)(376.92480469,447.8125)(376.94042969,448.484375)
\closepath
\moveto(377.81347656,452.22265625)
\curveto(377.81347656,451.40625)(377.97558594,450.81054688)(378.29980469,450.43554688)
\curveto(378.62402344,450.06054688)(379.03027344,449.87304688)(379.51855469,449.87304688)
\curveto(380.00292969,449.87304688)(380.40917969,450.05859375)(380.73730469,450.4296875)
\curveto(381.06542969,450.8046875)(381.22949219,451.390625)(381.22949219,452.1875)
\curveto(381.22949219,452.94921875)(381.05957031,453.5234375)(380.71972656,453.91015625)
\curveto(380.38378906,454.296875)(379.97753906,454.49023438)(379.50097656,454.49023438)
\curveto(379.03222656,454.49023438)(378.63378906,454.29882812)(378.30566406,453.91601562)
\curveto(377.97753906,453.53710938)(377.81347656,452.97265625)(377.81347656,452.22265625)
\closepath
}
}
{
\newrgbcolor{curcolor}{0 0 0}
\pscustom[linestyle=none,fillstyle=solid,fillcolor=curcolor]
{
\newpath
\moveto(386.88964844,451.75976562)
\lineto(387.96191406,451.85351562)
\curveto(388.01269531,451.42382812)(388.12988281,451.0703125)(388.31347656,450.79296875)
\curveto(388.50097656,450.51953125)(388.79003906,450.296875)(389.18066406,450.125)
\curveto(389.57128906,449.95703125)(390.01074219,449.87304688)(390.49902344,449.87304688)
\curveto(390.93261719,449.87304688)(391.31542969,449.9375)(391.64746094,450.06640625)
\curveto(391.97949219,450.1953125)(392.22558594,450.37109375)(392.38574219,450.59375)
\curveto(392.54980469,450.8203125)(392.63183594,451.06640625)(392.63183594,451.33203125)
\curveto(392.63183594,451.6015625)(392.55371094,451.8359375)(392.39746094,452.03515625)
\curveto(392.24121094,452.23828125)(391.98339844,452.40820312)(391.62402344,452.54492188)
\curveto(391.39355469,452.63476562)(390.88378906,452.7734375)(390.09472656,452.9609375)
\curveto(389.30566406,453.15234375)(388.75292969,453.33203125)(388.43652344,453.5)
\curveto(388.02636719,453.71484375)(387.71972656,453.98046875)(387.51660156,454.296875)
\curveto(387.31738281,454.6171875)(387.21777344,454.97460938)(387.21777344,455.36914062)
\curveto(387.21777344,455.80273438)(387.34082031,456.20703125)(387.58691406,456.58203125)
\curveto(387.83300781,456.9609375)(388.19238281,457.24804688)(388.66503906,457.44335938)
\curveto(389.13769531,457.63867188)(389.66308594,457.73632812)(390.24121094,457.73632812)
\curveto(390.87792969,457.73632812)(391.43847656,457.6328125)(391.92285156,457.42578125)
\curveto(392.41113281,457.22265625)(392.78613281,456.921875)(393.04785156,456.5234375)
\curveto(393.30957031,456.125)(393.45019531,455.67382812)(393.46972656,455.16992188)
\lineto(392.37988281,455.08789062)
\curveto(392.32128906,455.63085938)(392.12207031,456.04101562)(391.78222656,456.31835938)
\curveto(391.44628906,456.59570312)(390.94824219,456.734375)(390.28808594,456.734375)
\curveto(389.60058594,456.734375)(389.09863281,456.60742188)(388.78222656,456.35351562)
\curveto(388.46972656,456.10351562)(388.31347656,455.80078125)(388.31347656,455.4453125)
\curveto(388.31347656,455.13671875)(388.42480469,454.8828125)(388.64746094,454.68359375)
\curveto(388.86621094,454.484375)(389.43652344,454.27929688)(390.35839844,454.06835938)
\curveto(391.28417969,453.86132812)(391.91894531,453.6796875)(392.26269531,453.5234375)
\curveto(392.76269531,453.29296875)(393.13183594,453)(393.37011719,452.64453125)
\curveto(393.60839844,452.29296875)(393.72753906,451.88671875)(393.72753906,451.42578125)
\curveto(393.72753906,450.96875)(393.59667969,450.53710938)(393.33496094,450.13085938)
\curveto(393.07324219,449.72851562)(392.69628906,449.4140625)(392.20410156,449.1875)
\curveto(391.71582031,448.96484375)(391.16503906,448.85351562)(390.55175781,448.85351562)
\curveto(389.77441406,448.85351562)(389.12207031,448.96679688)(388.59472656,449.19335938)
\curveto(388.07128906,449.41992188)(387.65917969,449.75976562)(387.35839844,450.21289062)
\curveto(387.06152344,450.66992188)(386.90527344,451.18554688)(386.88964844,451.75976562)
\closepath
}
}
{
\newrgbcolor{curcolor}{0 0 0}
\pscustom[linestyle=none,fillstyle=solid,fillcolor=curcolor]
{
\newpath
\moveto(395.14550781,449)
\lineto(395.14550781,455.22265625)
\lineto(396.08886719,455.22265625)
\lineto(396.08886719,454.34960938)
\curveto(396.28417969,454.65429688)(396.54394531,454.8984375)(396.86816406,455.08203125)
\curveto(397.19238281,455.26953125)(397.56152344,455.36328125)(397.97558594,455.36328125)
\curveto(398.43652344,455.36328125)(398.81347656,455.26757812)(399.10644531,455.07617188)
\curveto(399.40332031,454.88476562)(399.61230469,454.6171875)(399.73339844,454.2734375)
\curveto(400.22558594,455)(400.86621094,455.36328125)(401.65527344,455.36328125)
\curveto(402.27246094,455.36328125)(402.74707031,455.19140625)(403.07910156,454.84765625)
\curveto(403.41113281,454.5078125)(403.57714844,453.98242188)(403.57714844,453.27148438)
\lineto(403.57714844,449)
\lineto(402.52832031,449)
\lineto(402.52832031,452.91992188)
\curveto(402.52832031,453.34179688)(402.49316406,453.64453125)(402.42285156,453.828125)
\curveto(402.35644531,454.015625)(402.23339844,454.16601562)(402.05371094,454.27929688)
\curveto(401.87402344,454.39257812)(401.66308594,454.44921875)(401.42089844,454.44921875)
\curveto(400.98339844,454.44921875)(400.62011719,454.30273438)(400.33105469,454.00976562)
\curveto(400.04199219,453.72070312)(399.89746094,453.25585938)(399.89746094,452.61523438)
\lineto(399.89746094,449)
\lineto(398.84277344,449)
\lineto(398.84277344,453.04296875)
\curveto(398.84277344,453.51171875)(398.75683594,453.86328125)(398.58496094,454.09765625)
\curveto(398.41308594,454.33203125)(398.13183594,454.44921875)(397.74121094,454.44921875)
\curveto(397.44433594,454.44921875)(397.16894531,454.37109375)(396.91503906,454.21484375)
\curveto(396.66503906,454.05859375)(396.48339844,453.83007812)(396.37011719,453.52929688)
\curveto(396.25683594,453.22851562)(396.20019531,452.79492188)(396.20019531,452.22851562)
\lineto(396.20019531,449)
\closepath
}
}
{
\newrgbcolor{curcolor}{0 0 0}
\pscustom[linestyle=none,fillstyle=solid,fillcolor=curcolor]
{
\newpath
\moveto(409.20214844,449.76757812)
\curveto(408.81152344,449.43554688)(408.43457031,449.20117188)(408.07128906,449.06445312)
\curveto(407.71191406,448.92773438)(407.32519531,448.859375)(406.91113281,448.859375)
\curveto(406.22753906,448.859375)(405.70214844,449.02539062)(405.33496094,449.35742188)
\curveto(404.96777344,449.69335938)(404.78417969,450.12109375)(404.78417969,450.640625)
\curveto(404.78417969,450.9453125)(404.85253906,451.22265625)(404.98925781,451.47265625)
\curveto(405.12988281,451.7265625)(405.31152344,451.9296875)(405.53417969,452.08203125)
\curveto(405.76074219,452.234375)(406.01464844,452.34960938)(406.29589844,452.42773438)
\curveto(406.50292969,452.48242188)(406.81542969,452.53515625)(407.23339844,452.5859375)
\curveto(408.08496094,452.6875)(408.71191406,452.80859375)(409.11425781,452.94921875)
\curveto(409.11816406,453.09375)(409.12011719,453.18554688)(409.12011719,453.22460938)
\curveto(409.12011719,453.65429688)(409.02050781,453.95703125)(408.82128906,454.1328125)
\curveto(408.55175781,454.37109375)(408.15136719,454.49023438)(407.62011719,454.49023438)
\curveto(407.12402344,454.49023438)(406.75683594,454.40234375)(406.51855469,454.2265625)
\curveto(406.28417969,454.0546875)(406.11035156,453.74804688)(405.99707031,453.30664062)
\lineto(404.96582031,453.44726562)
\curveto(405.05957031,453.88867188)(405.21386719,454.24414062)(405.42871094,454.51367188)
\curveto(405.64355469,454.78710938)(405.95410156,454.99609375)(406.36035156,455.140625)
\curveto(406.76660156,455.2890625)(407.23730469,455.36328125)(407.77246094,455.36328125)
\curveto(408.30371094,455.36328125)(408.73535156,455.30078125)(409.06738281,455.17578125)
\curveto(409.39941406,455.05078125)(409.64355469,454.89257812)(409.79980469,454.70117188)
\curveto(409.95605469,454.51367188)(410.06542969,454.27539062)(410.12792969,453.98632812)
\curveto(410.16308594,453.80664062)(410.18066406,453.48242188)(410.18066406,453.01367188)
\lineto(410.18066406,451.60742188)
\curveto(410.18066406,450.62695312)(410.20214844,450.00585938)(410.24511719,449.74414062)
\curveto(410.29199219,449.48632812)(410.38183594,449.23828125)(410.51464844,449)
\lineto(409.41308594,449)
\curveto(409.30371094,449.21875)(409.23339844,449.47460938)(409.20214844,449.76757812)
\closepath
\moveto(409.11425781,452.12304688)
\curveto(408.73144531,451.96679688)(408.15722656,451.83398438)(407.39160156,451.72460938)
\curveto(406.95800781,451.66210938)(406.65136719,451.59179688)(406.47167969,451.51367188)
\curveto(406.29199219,451.43554688)(406.15332031,451.3203125)(406.05566406,451.16796875)
\curveto(405.95800781,451.01953125)(405.90917969,450.85351562)(405.90917969,450.66992188)
\curveto(405.90917969,450.38867188)(406.01464844,450.15429688)(406.22558594,449.96679688)
\curveto(406.44042969,449.77929688)(406.75292969,449.68554688)(407.16308594,449.68554688)
\curveto(407.56933594,449.68554688)(407.93066406,449.7734375)(408.24707031,449.94921875)
\curveto(408.56347656,450.12890625)(408.79589844,450.37304688)(408.94433594,450.68164062)
\curveto(409.05761719,450.91992188)(409.11425781,451.27148438)(409.11425781,451.73632812)
\closepath
}
}
{
\newrgbcolor{curcolor}{0 0 0}
\pscustom[linestyle=none,fillstyle=solid,fillcolor=curcolor]
{
\newpath
\moveto(411.79199219,449)
\lineto(411.79199219,457.58984375)
\lineto(412.84667969,457.58984375)
\lineto(412.84667969,449)
\closepath
}
}
{
\newrgbcolor{curcolor}{0 0 0}
\pscustom[linestyle=none,fillstyle=solid,fillcolor=curcolor]
{
\newpath
\moveto(414.45800781,449)
\lineto(414.45800781,457.58984375)
\lineto(415.51269531,457.58984375)
\lineto(415.51269531,449)
\closepath
}
}
{
\newrgbcolor{curcolor}{0 0 0}
\pscustom[linestyle=none,fillstyle=solid,fillcolor=curcolor]
{
\newpath
\moveto(420.61621094,449)
\lineto(420.61621094,457.58984375)
\lineto(423.85644531,457.58984375)
\curveto(424.42675781,457.58984375)(424.86230469,457.5625)(425.16308594,457.5078125)
\curveto(425.58496094,457.4375)(425.93847656,457.30273438)(426.22363281,457.10351562)
\curveto(426.50878906,456.90820312)(426.73730469,456.6328125)(426.90917969,456.27734375)
\curveto(427.08496094,455.921875)(427.17285156,455.53125)(427.17285156,455.10546875)
\curveto(427.17285156,454.375)(426.94042969,453.75585938)(426.47558594,453.24804688)
\curveto(426.01074219,452.74414062)(425.17089844,452.4921875)(423.95605469,452.4921875)
\lineto(421.75292969,452.4921875)
\lineto(421.75292969,449)
\closepath
\moveto(421.75292969,453.50585938)
\lineto(423.97363281,453.50585938)
\curveto(424.70800781,453.50585938)(425.22949219,453.64257812)(425.53808594,453.91601562)
\curveto(425.84667969,454.18945312)(426.00097656,454.57421875)(426.00097656,455.0703125)
\curveto(426.00097656,455.4296875)(425.90917969,455.73632812)(425.72558594,455.99023438)
\curveto(425.54589844,456.24804688)(425.30761719,456.41796875)(425.01074219,456.5)
\curveto(424.81933594,456.55078125)(424.46582031,456.57617188)(423.95019531,456.57617188)
\lineto(421.75292969,456.57617188)
\closepath
}
}
{
\newrgbcolor{curcolor}{0 0 0}
\pscustom[linestyle=none,fillstyle=solid,fillcolor=curcolor]
{
\newpath
\moveto(428.47363281,449)
\lineto(428.47363281,455.22265625)
\lineto(429.42285156,455.22265625)
\lineto(429.42285156,454.27929688)
\curveto(429.66503906,454.72070312)(429.88769531,455.01171875)(430.09082031,455.15234375)
\curveto(430.29785156,455.29296875)(430.52441406,455.36328125)(430.77050781,455.36328125)
\curveto(431.12597656,455.36328125)(431.48730469,455.25)(431.85449219,455.0234375)
\lineto(431.49121094,454.04492188)
\curveto(431.23339844,454.19726562)(430.97558594,454.2734375)(430.71777344,454.2734375)
\curveto(430.48730469,454.2734375)(430.28027344,454.203125)(430.09667969,454.0625)
\curveto(429.91308594,453.92578125)(429.78222656,453.734375)(429.70410156,453.48828125)
\curveto(429.58691406,453.11328125)(429.52832031,452.703125)(429.52832031,452.2578125)
\lineto(429.52832031,449)
\closepath
}
}
{
\newrgbcolor{curcolor}{0 0 0}
\pscustom[linestyle=none,fillstyle=solid,fillcolor=curcolor]
{
\newpath
\moveto(432.08886719,452.11132812)
\curveto(432.08886719,453.26367188)(432.40917969,454.1171875)(433.04980469,454.671875)
\curveto(433.58496094,455.1328125)(434.23730469,455.36328125)(435.00683594,455.36328125)
\curveto(435.86230469,455.36328125)(436.56152344,455.08203125)(437.10449219,454.51953125)
\curveto(437.64746094,453.9609375)(437.91894531,453.1875)(437.91894531,452.19921875)
\curveto(437.91894531,451.3984375)(437.79785156,450.76757812)(437.55566406,450.30664062)
\curveto(437.31738281,449.84960938)(436.96777344,449.49414062)(436.50683594,449.24023438)
\curveto(436.04980469,448.98632812)(435.54980469,448.859375)(435.00683594,448.859375)
\curveto(434.13574219,448.859375)(433.43066406,449.13867188)(432.89160156,449.69726562)
\curveto(432.35644531,450.25585938)(432.08886719,451.06054688)(432.08886719,452.11132812)
\closepath
\moveto(433.17285156,452.11132812)
\curveto(433.17285156,451.31445312)(433.34667969,450.71679688)(433.69433594,450.31835938)
\curveto(434.04199219,449.92382812)(434.47949219,449.7265625)(435.00683594,449.7265625)
\curveto(435.53027344,449.7265625)(435.96582031,449.92578125)(436.31347656,450.32421875)
\curveto(436.66113281,450.72265625)(436.83496094,451.33007812)(436.83496094,452.14648438)
\curveto(436.83496094,452.91601562)(436.65917969,453.49804688)(436.30761719,453.89257812)
\curveto(435.95996094,454.29101562)(435.52636719,454.49023438)(435.00683594,454.49023438)
\curveto(434.47949219,454.49023438)(434.04199219,454.29296875)(433.69433594,453.8984375)
\curveto(433.34667969,453.50390625)(433.17285156,452.90820312)(433.17285156,452.11132812)
\closepath
}
}
{
\newrgbcolor{curcolor}{0 0 0}
\pscustom[linestyle=none,fillstyle=solid,fillcolor=curcolor]
{
\newpath
\moveto(443.21582031,451.27929688)
\lineto(444.25292969,451.14453125)
\curveto(444.13964844,450.4296875)(443.84863281,449.86914062)(443.37988281,449.46289062)
\curveto(442.91503906,449.06054688)(442.34277344,448.859375)(441.66308594,448.859375)
\curveto(440.81152344,448.859375)(440.12597656,449.13671875)(439.60644531,449.69140625)
\curveto(439.09082031,450.25)(438.83300781,451.04882812)(438.83300781,452.08789062)
\curveto(438.83300781,452.75976562)(438.94433594,453.34765625)(439.16699219,453.8515625)
\curveto(439.38964844,454.35546875)(439.72753906,454.73242188)(440.18066406,454.98242188)
\curveto(440.63769531,455.23632812)(441.13378906,455.36328125)(441.66894531,455.36328125)
\curveto(442.34472656,455.36328125)(442.89746094,455.19140625)(443.32714844,454.84765625)
\curveto(443.75683594,454.5078125)(444.03222656,454.0234375)(444.15332031,453.39453125)
\lineto(443.12792969,453.23632812)
\curveto(443.03027344,453.65429688)(442.85644531,453.96875)(442.60644531,454.1796875)
\curveto(442.36035156,454.390625)(442.06152344,454.49609375)(441.70996094,454.49609375)
\curveto(441.17871094,454.49609375)(440.74707031,454.3046875)(440.41503906,453.921875)
\curveto(440.08300781,453.54296875)(439.91699219,452.94140625)(439.91699219,452.1171875)
\curveto(439.91699219,451.28125)(440.07714844,450.67382812)(440.39746094,450.29492188)
\curveto(440.71777344,449.91601562)(441.13574219,449.7265625)(441.65136719,449.7265625)
\curveto(442.06542969,449.7265625)(442.41113281,449.85351562)(442.68847656,450.10742188)
\curveto(442.96582031,450.36132812)(443.14160156,450.75195312)(443.21582031,451.27929688)
\closepath
}
}
{
\newrgbcolor{curcolor}{0 0 0}
\pscustom[linestyle=none,fillstyle=solid,fillcolor=curcolor]
{
\newpath
\moveto(449.41503906,451.00390625)
\lineto(450.50488281,450.86914062)
\curveto(450.33300781,450.23242188)(450.01464844,449.73828125)(449.54980469,449.38671875)
\curveto(449.08496094,449.03515625)(448.49121094,448.859375)(447.76855469,448.859375)
\curveto(446.85839844,448.859375)(446.13574219,449.13867188)(445.60058594,449.69726562)
\curveto(445.06933594,450.25976562)(444.80371094,451.046875)(444.80371094,452.05859375)
\curveto(444.80371094,453.10546875)(445.07324219,453.91796875)(445.61230469,454.49609375)
\curveto(446.15136719,455.07421875)(446.85058594,455.36328125)(447.70996094,455.36328125)
\curveto(448.54199219,455.36328125)(449.22167969,455.08007812)(449.74902344,454.51367188)
\curveto(450.27636719,453.94726562)(450.54003906,453.15039062)(450.54003906,452.12304688)
\curveto(450.54003906,452.06054688)(450.53808594,451.96679688)(450.53417969,451.84179688)
\lineto(445.89355469,451.84179688)
\curveto(445.93261719,451.15820312)(446.12597656,450.63476562)(446.47363281,450.27148438)
\curveto(446.82128906,449.90820312)(447.25488281,449.7265625)(447.77441406,449.7265625)
\curveto(448.16113281,449.7265625)(448.49121094,449.828125)(448.76464844,450.03125)
\curveto(449.03808594,450.234375)(449.25488281,450.55859375)(449.41503906,451.00390625)
\closepath
\moveto(445.95214844,452.70898438)
\lineto(449.42675781,452.70898438)
\curveto(449.37988281,453.23242188)(449.24707031,453.625)(449.02832031,453.88671875)
\curveto(448.69238281,454.29296875)(448.25683594,454.49609375)(447.72167969,454.49609375)
\curveto(447.23730469,454.49609375)(446.82910156,454.33398438)(446.49707031,454.00976562)
\curveto(446.16894531,453.68554688)(445.98730469,453.25195312)(445.95214844,452.70898438)
\closepath
}
}
{
\newrgbcolor{curcolor}{0 0 0}
\pscustom[linestyle=none,fillstyle=solid,fillcolor=curcolor]
{
\newpath
\moveto(451.40722656,450.85742188)
\lineto(452.45019531,451.02148438)
\curveto(452.50878906,450.60351562)(452.67089844,450.28320312)(452.93652344,450.06054688)
\curveto(453.20605469,449.83789062)(453.58105469,449.7265625)(454.06152344,449.7265625)
\curveto(454.54589844,449.7265625)(454.90527344,449.82421875)(455.13964844,450.01953125)
\curveto(455.37402344,450.21875)(455.49121094,450.45117188)(455.49121094,450.71679688)
\curveto(455.49121094,450.95507812)(455.38769531,451.14257812)(455.18066406,451.27929688)
\curveto(455.03613281,451.37304688)(454.67675781,451.4921875)(454.10253906,451.63671875)
\curveto(453.32910156,451.83203125)(452.79199219,452)(452.49121094,452.140625)
\curveto(452.19433594,452.28515625)(451.96777344,452.48242188)(451.81152344,452.73242188)
\curveto(451.65917969,452.98632812)(451.58300781,453.265625)(451.58300781,453.5703125)
\curveto(451.58300781,453.84765625)(451.64550781,454.10351562)(451.77050781,454.33789062)
\curveto(451.89941406,454.57617188)(452.07324219,454.7734375)(452.29199219,454.9296875)
\curveto(452.45605469,455.05078125)(452.67871094,455.15234375)(452.95996094,455.234375)
\curveto(453.24511719,455.3203125)(453.54980469,455.36328125)(453.87402344,455.36328125)
\curveto(454.36230469,455.36328125)(454.79003906,455.29296875)(455.15722656,455.15234375)
\curveto(455.52832031,455.01171875)(455.80175781,454.8203125)(455.97753906,454.578125)
\curveto(456.15332031,454.33984375)(456.27441406,454.01953125)(456.34082031,453.6171875)
\lineto(455.30957031,453.4765625)
\curveto(455.26269531,453.796875)(455.12597656,454.046875)(454.89941406,454.2265625)
\curveto(454.67675781,454.40625)(454.36035156,454.49609375)(453.95019531,454.49609375)
\curveto(453.46582031,454.49609375)(453.12011719,454.41601562)(452.91308594,454.25585938)
\curveto(452.70605469,454.09570312)(452.60253906,453.90820312)(452.60253906,453.69335938)
\curveto(452.60253906,453.55664062)(452.64550781,453.43359375)(452.73144531,453.32421875)
\curveto(452.81738281,453.2109375)(452.95214844,453.1171875)(453.13574219,453.04296875)
\curveto(453.24121094,453.00390625)(453.55175781,452.9140625)(454.06738281,452.7734375)
\curveto(454.81347656,452.57421875)(455.33300781,452.41015625)(455.62597656,452.28125)
\curveto(455.92285156,452.15625)(456.15527344,451.97265625)(456.32324219,451.73046875)
\curveto(456.49121094,451.48828125)(456.57519531,451.1875)(456.57519531,450.828125)
\curveto(456.57519531,450.4765625)(456.47167969,450.14453125)(456.26464844,449.83203125)
\curveto(456.06152344,449.5234375)(455.76660156,449.28320312)(455.37988281,449.11132812)
\curveto(454.99316406,448.94335938)(454.55566406,448.859375)(454.06738281,448.859375)
\curveto(453.25878906,448.859375)(452.64160156,449.02734375)(452.21582031,449.36328125)
\curveto(451.79394531,449.69921875)(451.52441406,450.19726562)(451.40722656,450.85742188)
\closepath
}
}
{
\newrgbcolor{curcolor}{0 0 0}
\pscustom[linestyle=none,fillstyle=solid,fillcolor=curcolor]
{
\newpath
\moveto(457.40722656,450.85742188)
\lineto(458.45019531,451.02148438)
\curveto(458.50878906,450.60351562)(458.67089844,450.28320312)(458.93652344,450.06054688)
\curveto(459.20605469,449.83789062)(459.58105469,449.7265625)(460.06152344,449.7265625)
\curveto(460.54589844,449.7265625)(460.90527344,449.82421875)(461.13964844,450.01953125)
\curveto(461.37402344,450.21875)(461.49121094,450.45117188)(461.49121094,450.71679688)
\curveto(461.49121094,450.95507812)(461.38769531,451.14257812)(461.18066406,451.27929688)
\curveto(461.03613281,451.37304688)(460.67675781,451.4921875)(460.10253906,451.63671875)
\curveto(459.32910156,451.83203125)(458.79199219,452)(458.49121094,452.140625)
\curveto(458.19433594,452.28515625)(457.96777344,452.48242188)(457.81152344,452.73242188)
\curveto(457.65917969,452.98632812)(457.58300781,453.265625)(457.58300781,453.5703125)
\curveto(457.58300781,453.84765625)(457.64550781,454.10351562)(457.77050781,454.33789062)
\curveto(457.89941406,454.57617188)(458.07324219,454.7734375)(458.29199219,454.9296875)
\curveto(458.45605469,455.05078125)(458.67871094,455.15234375)(458.95996094,455.234375)
\curveto(459.24511719,455.3203125)(459.54980469,455.36328125)(459.87402344,455.36328125)
\curveto(460.36230469,455.36328125)(460.79003906,455.29296875)(461.15722656,455.15234375)
\curveto(461.52832031,455.01171875)(461.80175781,454.8203125)(461.97753906,454.578125)
\curveto(462.15332031,454.33984375)(462.27441406,454.01953125)(462.34082031,453.6171875)
\lineto(461.30957031,453.4765625)
\curveto(461.26269531,453.796875)(461.12597656,454.046875)(460.89941406,454.2265625)
\curveto(460.67675781,454.40625)(460.36035156,454.49609375)(459.95019531,454.49609375)
\curveto(459.46582031,454.49609375)(459.12011719,454.41601562)(458.91308594,454.25585938)
\curveto(458.70605469,454.09570312)(458.60253906,453.90820312)(458.60253906,453.69335938)
\curveto(458.60253906,453.55664062)(458.64550781,453.43359375)(458.73144531,453.32421875)
\curveto(458.81738281,453.2109375)(458.95214844,453.1171875)(459.13574219,453.04296875)
\curveto(459.24121094,453.00390625)(459.55175781,452.9140625)(460.06738281,452.7734375)
\curveto(460.81347656,452.57421875)(461.33300781,452.41015625)(461.62597656,452.28125)
\curveto(461.92285156,452.15625)(462.15527344,451.97265625)(462.32324219,451.73046875)
\curveto(462.49121094,451.48828125)(462.57519531,451.1875)(462.57519531,450.828125)
\curveto(462.57519531,450.4765625)(462.47167969,450.14453125)(462.26464844,449.83203125)
\curveto(462.06152344,449.5234375)(461.76660156,449.28320312)(461.37988281,449.11132812)
\curveto(460.99316406,448.94335938)(460.55566406,448.859375)(460.06738281,448.859375)
\curveto(459.25878906,448.859375)(458.64160156,449.02734375)(458.21582031,449.36328125)
\curveto(457.79394531,449.69921875)(457.52441406,450.19726562)(457.40722656,450.85742188)
\closepath
}
}
{
\newrgbcolor{curcolor}{0 0 0}
\pscustom[linestyle=none,fillstyle=solid,fillcolor=curcolor]
{
\newpath
\moveto(469.17871094,446.47460938)
\curveto(468.59667969,447.20898438)(468.10449219,448.06835938)(467.70214844,449.05273438)
\curveto(467.29980469,450.03710938)(467.09863281,451.05664062)(467.09863281,452.11132812)
\curveto(467.09863281,453.04101562)(467.24902344,453.93164062)(467.54980469,454.78320312)
\curveto(467.90136719,455.77148438)(468.44433594,456.75585938)(469.17871094,457.73632812)
\lineto(469.93457031,457.73632812)
\curveto(469.46191406,456.92382812)(469.14941406,456.34375)(468.99707031,455.99609375)
\curveto(468.75878906,455.45703125)(468.57128906,454.89453125)(468.43457031,454.30859375)
\curveto(468.26660156,453.578125)(468.18261719,452.84375)(468.18261719,452.10546875)
\curveto(468.18261719,450.2265625)(468.76660156,448.34960938)(469.93457031,446.47460938)
\closepath
}
}
{
\newrgbcolor{curcolor}{0 0 0}
\pscustom[linestyle=none,fillstyle=solid,fillcolor=curcolor]
{
\newpath
\moveto(471.29394531,449)
\lineto(471.29394531,457.58984375)
\lineto(474.25292969,457.58984375)
\curveto(474.92089844,457.58984375)(475.43066406,457.54882812)(475.78222656,457.46679688)
\curveto(476.27441406,457.35351562)(476.69433594,457.1484375)(477.04199219,456.8515625)
\curveto(477.49511719,456.46875)(477.83300781,455.97851562)(478.05566406,455.38085938)
\curveto(478.28222656,454.78710938)(478.39550781,454.10742188)(478.39550781,453.34179688)
\curveto(478.39550781,452.68945312)(478.31933594,452.11132812)(478.16699219,451.60742188)
\curveto(478.01464844,451.10351562)(477.81933594,450.68554688)(477.58105469,450.35351562)
\curveto(477.34277344,450.02539062)(477.08105469,449.765625)(476.79589844,449.57421875)
\curveto(476.51464844,449.38671875)(476.17285156,449.24414062)(475.77050781,449.14648438)
\curveto(475.37207031,449.04882812)(474.91308594,449)(474.39355469,449)
\closepath
\moveto(472.43066406,450.01367188)
\lineto(474.26464844,450.01367188)
\curveto(474.83105469,450.01367188)(475.27441406,450.06640625)(475.59472656,450.171875)
\curveto(475.91894531,450.27734375)(476.17675781,450.42578125)(476.36816406,450.6171875)
\curveto(476.63769531,450.88671875)(476.84667969,451.24804688)(476.99511719,451.70117188)
\curveto(477.14746094,452.15820312)(477.22363281,452.7109375)(477.22363281,453.359375)
\curveto(477.22363281,454.2578125)(477.07519531,454.94726562)(476.77832031,455.42773438)
\curveto(476.48535156,455.91210938)(476.12792969,456.23632812)(475.70605469,456.40039062)
\curveto(475.40136719,456.51757812)(474.91113281,456.57617188)(474.23535156,456.57617188)
\lineto(472.43066406,456.57617188)
\closepath
}
}
{
\newrgbcolor{curcolor}{0 0 0}
\pscustom[linestyle=none,fillstyle=solid,fillcolor=curcolor]
{
\newpath
\moveto(483.88574219,449.76757812)
\curveto(483.49511719,449.43554688)(483.11816406,449.20117188)(482.75488281,449.06445312)
\curveto(482.39550781,448.92773438)(482.00878906,448.859375)(481.59472656,448.859375)
\curveto(480.91113281,448.859375)(480.38574219,449.02539062)(480.01855469,449.35742188)
\curveto(479.65136719,449.69335938)(479.46777344,450.12109375)(479.46777344,450.640625)
\curveto(479.46777344,450.9453125)(479.53613281,451.22265625)(479.67285156,451.47265625)
\curveto(479.81347656,451.7265625)(479.99511719,451.9296875)(480.21777344,452.08203125)
\curveto(480.44433594,452.234375)(480.69824219,452.34960938)(480.97949219,452.42773438)
\curveto(481.18652344,452.48242188)(481.49902344,452.53515625)(481.91699219,452.5859375)
\curveto(482.76855469,452.6875)(483.39550781,452.80859375)(483.79785156,452.94921875)
\curveto(483.80175781,453.09375)(483.80371094,453.18554688)(483.80371094,453.22460938)
\curveto(483.80371094,453.65429688)(483.70410156,453.95703125)(483.50488281,454.1328125)
\curveto(483.23535156,454.37109375)(482.83496094,454.49023438)(482.30371094,454.49023438)
\curveto(481.80761719,454.49023438)(481.44042969,454.40234375)(481.20214844,454.2265625)
\curveto(480.96777344,454.0546875)(480.79394531,453.74804688)(480.68066406,453.30664062)
\lineto(479.64941406,453.44726562)
\curveto(479.74316406,453.88867188)(479.89746094,454.24414062)(480.11230469,454.51367188)
\curveto(480.32714844,454.78710938)(480.63769531,454.99609375)(481.04394531,455.140625)
\curveto(481.45019531,455.2890625)(481.92089844,455.36328125)(482.45605469,455.36328125)
\curveto(482.98730469,455.36328125)(483.41894531,455.30078125)(483.75097656,455.17578125)
\curveto(484.08300781,455.05078125)(484.32714844,454.89257812)(484.48339844,454.70117188)
\curveto(484.63964844,454.51367188)(484.74902344,454.27539062)(484.81152344,453.98632812)
\curveto(484.84667969,453.80664062)(484.86425781,453.48242188)(484.86425781,453.01367188)
\lineto(484.86425781,451.60742188)
\curveto(484.86425781,450.62695312)(484.88574219,450.00585938)(484.92871094,449.74414062)
\curveto(484.97558594,449.48632812)(485.06542969,449.23828125)(485.19824219,449)
\lineto(484.09667969,449)
\curveto(483.98730469,449.21875)(483.91699219,449.47460938)(483.88574219,449.76757812)
\closepath
\moveto(483.79785156,452.12304688)
\curveto(483.41503906,451.96679688)(482.84082031,451.83398438)(482.07519531,451.72460938)
\curveto(481.64160156,451.66210938)(481.33496094,451.59179688)(481.15527344,451.51367188)
\curveto(480.97558594,451.43554688)(480.83691406,451.3203125)(480.73925781,451.16796875)
\curveto(480.64160156,451.01953125)(480.59277344,450.85351562)(480.59277344,450.66992188)
\curveto(480.59277344,450.38867188)(480.69824219,450.15429688)(480.90917969,449.96679688)
\curveto(481.12402344,449.77929688)(481.43652344,449.68554688)(481.84667969,449.68554688)
\curveto(482.25292969,449.68554688)(482.61425781,449.7734375)(482.93066406,449.94921875)
\curveto(483.24707031,450.12890625)(483.47949219,450.37304688)(483.62792969,450.68164062)
\curveto(483.74121094,450.91992188)(483.79785156,451.27148438)(483.79785156,451.73632812)
\closepath
}
}
{
\newrgbcolor{curcolor}{0 0 0}
\pscustom[linestyle=none,fillstyle=solid,fillcolor=curcolor]
{
\newpath
\moveto(488.80175781,449.94335938)
\lineto(488.95410156,449.01171875)
\curveto(488.65722656,448.94921875)(488.39160156,448.91796875)(488.15722656,448.91796875)
\curveto(487.77441406,448.91796875)(487.47753906,448.97851562)(487.26660156,449.09960938)
\curveto(487.05566406,449.22070312)(486.90722656,449.37890625)(486.82128906,449.57421875)
\curveto(486.73535156,449.7734375)(486.69238281,450.18945312)(486.69238281,450.82226562)
\lineto(486.69238281,454.40234375)
\lineto(485.91894531,454.40234375)
\lineto(485.91894531,455.22265625)
\lineto(486.69238281,455.22265625)
\lineto(486.69238281,456.76367188)
\lineto(487.74121094,457.39648438)
\lineto(487.74121094,455.22265625)
\lineto(488.80175781,455.22265625)
\lineto(488.80175781,454.40234375)
\lineto(487.74121094,454.40234375)
\lineto(487.74121094,450.76367188)
\curveto(487.74121094,450.46289062)(487.75878906,450.26953125)(487.79394531,450.18359375)
\curveto(487.83300781,450.09765625)(487.89355469,450.02929688)(487.97558594,449.97851562)
\curveto(488.06152344,449.92773438)(488.18261719,449.90234375)(488.33886719,449.90234375)
\curveto(488.45605469,449.90234375)(488.61035156,449.91601562)(488.80175781,449.94335938)
\closepath
}
}
{
\newrgbcolor{curcolor}{0 0 0}
\pscustom[linestyle=none,fillstyle=solid,fillcolor=curcolor]
{
\newpath
\moveto(493.89355469,449.76757812)
\curveto(493.50292969,449.43554688)(493.12597656,449.20117188)(492.76269531,449.06445312)
\curveto(492.40332031,448.92773438)(492.01660156,448.859375)(491.60253906,448.859375)
\curveto(490.91894531,448.859375)(490.39355469,449.02539062)(490.02636719,449.35742188)
\curveto(489.65917969,449.69335938)(489.47558594,450.12109375)(489.47558594,450.640625)
\curveto(489.47558594,450.9453125)(489.54394531,451.22265625)(489.68066406,451.47265625)
\curveto(489.82128906,451.7265625)(490.00292969,451.9296875)(490.22558594,452.08203125)
\curveto(490.45214844,452.234375)(490.70605469,452.34960938)(490.98730469,452.42773438)
\curveto(491.19433594,452.48242188)(491.50683594,452.53515625)(491.92480469,452.5859375)
\curveto(492.77636719,452.6875)(493.40332031,452.80859375)(493.80566406,452.94921875)
\curveto(493.80957031,453.09375)(493.81152344,453.18554688)(493.81152344,453.22460938)
\curveto(493.81152344,453.65429688)(493.71191406,453.95703125)(493.51269531,454.1328125)
\curveto(493.24316406,454.37109375)(492.84277344,454.49023438)(492.31152344,454.49023438)
\curveto(491.81542969,454.49023438)(491.44824219,454.40234375)(491.20996094,454.2265625)
\curveto(490.97558594,454.0546875)(490.80175781,453.74804688)(490.68847656,453.30664062)
\lineto(489.65722656,453.44726562)
\curveto(489.75097656,453.88867188)(489.90527344,454.24414062)(490.12011719,454.51367188)
\curveto(490.33496094,454.78710938)(490.64550781,454.99609375)(491.05175781,455.140625)
\curveto(491.45800781,455.2890625)(491.92871094,455.36328125)(492.46386719,455.36328125)
\curveto(492.99511719,455.36328125)(493.42675781,455.30078125)(493.75878906,455.17578125)
\curveto(494.09082031,455.05078125)(494.33496094,454.89257812)(494.49121094,454.70117188)
\curveto(494.64746094,454.51367188)(494.75683594,454.27539062)(494.81933594,453.98632812)
\curveto(494.85449219,453.80664062)(494.87207031,453.48242188)(494.87207031,453.01367188)
\lineto(494.87207031,451.60742188)
\curveto(494.87207031,450.62695312)(494.89355469,450.00585938)(494.93652344,449.74414062)
\curveto(494.98339844,449.48632812)(495.07324219,449.23828125)(495.20605469,449)
\lineto(494.10449219,449)
\curveto(493.99511719,449.21875)(493.92480469,449.47460938)(493.89355469,449.76757812)
\closepath
\moveto(493.80566406,452.12304688)
\curveto(493.42285156,451.96679688)(492.84863281,451.83398438)(492.08300781,451.72460938)
\curveto(491.64941406,451.66210938)(491.34277344,451.59179688)(491.16308594,451.51367188)
\curveto(490.98339844,451.43554688)(490.84472656,451.3203125)(490.74707031,451.16796875)
\curveto(490.64941406,451.01953125)(490.60058594,450.85351562)(490.60058594,450.66992188)
\curveto(490.60058594,450.38867188)(490.70605469,450.15429688)(490.91699219,449.96679688)
\curveto(491.13183594,449.77929688)(491.44433594,449.68554688)(491.85449219,449.68554688)
\curveto(492.26074219,449.68554688)(492.62207031,449.7734375)(492.93847656,449.94921875)
\curveto(493.25488281,450.12890625)(493.48730469,450.37304688)(493.63574219,450.68164062)
\curveto(493.74902344,450.91992188)(493.80566406,451.27148438)(493.80566406,451.73632812)
\closepath
}
}
{
\newrgbcolor{curcolor}{0 0 0}
\pscustom[linestyle=none,fillstyle=solid,fillcolor=curcolor]
{
\newpath
\moveto(499.96386719,449)
\lineto(499.96386719,457.58984375)
\lineto(501.12988281,457.58984375)
\lineto(505.64160156,450.84570312)
\lineto(505.64160156,457.58984375)
\lineto(506.73144531,457.58984375)
\lineto(506.73144531,449)
\lineto(505.56542969,449)
\lineto(501.05371094,455.75)
\lineto(501.05371094,449)
\closepath
}
}
{
\newrgbcolor{curcolor}{0 0 0}
\pscustom[linestyle=none,fillstyle=solid,fillcolor=curcolor]
{
\newpath
\moveto(508.11425781,452.11132812)
\curveto(508.11425781,453.26367188)(508.43457031,454.1171875)(509.07519531,454.671875)
\curveto(509.61035156,455.1328125)(510.26269531,455.36328125)(511.03222656,455.36328125)
\curveto(511.88769531,455.36328125)(512.58691406,455.08203125)(513.12988281,454.51953125)
\curveto(513.67285156,453.9609375)(513.94433594,453.1875)(513.94433594,452.19921875)
\curveto(513.94433594,451.3984375)(513.82324219,450.76757812)(513.58105469,450.30664062)
\curveto(513.34277344,449.84960938)(512.99316406,449.49414062)(512.53222656,449.24023438)
\curveto(512.07519531,448.98632812)(511.57519531,448.859375)(511.03222656,448.859375)
\curveto(510.16113281,448.859375)(509.45605469,449.13867188)(508.91699219,449.69726562)
\curveto(508.38183594,450.25585938)(508.11425781,451.06054688)(508.11425781,452.11132812)
\closepath
\moveto(509.19824219,452.11132812)
\curveto(509.19824219,451.31445312)(509.37207031,450.71679688)(509.71972656,450.31835938)
\curveto(510.06738281,449.92382812)(510.50488281,449.7265625)(511.03222656,449.7265625)
\curveto(511.55566406,449.7265625)(511.99121094,449.92578125)(512.33886719,450.32421875)
\curveto(512.68652344,450.72265625)(512.86035156,451.33007812)(512.86035156,452.14648438)
\curveto(512.86035156,452.91601562)(512.68457031,453.49804688)(512.33300781,453.89257812)
\curveto(511.98535156,454.29101562)(511.55175781,454.49023438)(511.03222656,454.49023438)
\curveto(510.50488281,454.49023438)(510.06738281,454.29296875)(509.71972656,453.8984375)
\curveto(509.37207031,453.50390625)(509.19824219,452.90820312)(509.19824219,452.11132812)
\closepath
}
}
{
\newrgbcolor{curcolor}{0 0 0}
\pscustom[linestyle=none,fillstyle=solid,fillcolor=curcolor]
{
\newpath
\moveto(519.21777344,449)
\lineto(519.21777344,449.78515625)
\curveto(518.82324219,449.16796875)(518.24316406,448.859375)(517.47753906,448.859375)
\curveto(516.98144531,448.859375)(516.52441406,448.99609375)(516.10644531,449.26953125)
\curveto(515.69238281,449.54296875)(515.37011719,449.92382812)(515.13964844,450.41210938)
\curveto(514.91308594,450.90429688)(514.79980469,451.46875)(514.79980469,452.10546875)
\curveto(514.79980469,452.7265625)(514.90332031,453.2890625)(515.11035156,453.79296875)
\curveto(515.31738281,454.30078125)(515.62792969,454.68945312)(516.04199219,454.95898438)
\curveto(516.45605469,455.22851562)(516.91894531,455.36328125)(517.43066406,455.36328125)
\curveto(517.80566406,455.36328125)(518.13964844,455.28320312)(518.43261719,455.12304688)
\curveto(518.72558594,454.96679688)(518.96386719,454.76171875)(519.14746094,454.5078125)
\lineto(519.14746094,457.58984375)
\lineto(520.19628906,457.58984375)
\lineto(520.19628906,449)
\closepath
\moveto(515.88378906,452.10546875)
\curveto(515.88378906,451.30859375)(516.05175781,450.71289062)(516.38769531,450.31835938)
\curveto(516.72363281,449.92382812)(517.12011719,449.7265625)(517.57714844,449.7265625)
\curveto(518.03808594,449.7265625)(518.42871094,449.9140625)(518.74902344,450.2890625)
\curveto(519.07324219,450.66796875)(519.23535156,451.24414062)(519.23535156,452.01757812)
\curveto(519.23535156,452.86914062)(519.07128906,453.49414062)(518.74316406,453.89257812)
\curveto(518.41503906,454.29101562)(518.01074219,454.49023438)(517.53027344,454.49023438)
\curveto(517.06152344,454.49023438)(516.66894531,454.29882812)(516.35253906,453.91601562)
\curveto(516.04003906,453.53320312)(515.88378906,452.9296875)(515.88378906,452.10546875)
\closepath
}
}
{
\newrgbcolor{curcolor}{0 0 0}
\pscustom[linestyle=none,fillstyle=solid,fillcolor=curcolor]
{
\newpath
\moveto(526.11425781,451.00390625)
\lineto(527.20410156,450.86914062)
\curveto(527.03222656,450.23242188)(526.71386719,449.73828125)(526.24902344,449.38671875)
\curveto(525.78417969,449.03515625)(525.19042969,448.859375)(524.46777344,448.859375)
\curveto(523.55761719,448.859375)(522.83496094,449.13867188)(522.29980469,449.69726562)
\curveto(521.76855469,450.25976562)(521.50292969,451.046875)(521.50292969,452.05859375)
\curveto(521.50292969,453.10546875)(521.77246094,453.91796875)(522.31152344,454.49609375)
\curveto(522.85058594,455.07421875)(523.54980469,455.36328125)(524.40917969,455.36328125)
\curveto(525.24121094,455.36328125)(525.92089844,455.08007812)(526.44824219,454.51367188)
\curveto(526.97558594,453.94726562)(527.23925781,453.15039062)(527.23925781,452.12304688)
\curveto(527.23925781,452.06054688)(527.23730469,451.96679688)(527.23339844,451.84179688)
\lineto(522.59277344,451.84179688)
\curveto(522.63183594,451.15820312)(522.82519531,450.63476562)(523.17285156,450.27148438)
\curveto(523.52050781,449.90820312)(523.95410156,449.7265625)(524.47363281,449.7265625)
\curveto(524.86035156,449.7265625)(525.19042969,449.828125)(525.46386719,450.03125)
\curveto(525.73730469,450.234375)(525.95410156,450.55859375)(526.11425781,451.00390625)
\closepath
\moveto(522.65136719,452.70898438)
\lineto(526.12597656,452.70898438)
\curveto(526.07910156,453.23242188)(525.94628906,453.625)(525.72753906,453.88671875)
\curveto(525.39160156,454.29296875)(524.95605469,454.49609375)(524.42089844,454.49609375)
\curveto(523.93652344,454.49609375)(523.52832031,454.33398438)(523.19628906,454.00976562)
\curveto(522.86816406,453.68554688)(522.68652344,453.25195312)(522.65136719,452.70898438)
\closepath
}
}
{
\newrgbcolor{curcolor}{0 0 0}
\pscustom[linestyle=none,fillstyle=solid,fillcolor=curcolor]
{
\newpath
\moveto(531.56933594,453.23632812)
\curveto(531.56933594,454.25195312)(531.67285156,455.06835938)(531.87988281,455.68554688)
\curveto(532.09082031,456.30664062)(532.40136719,456.78515625)(532.81152344,457.12109375)
\curveto(533.22558594,457.45703125)(533.74511719,457.625)(534.37011719,457.625)
\curveto(534.83105469,457.625)(535.23535156,457.53125)(535.58300781,457.34375)
\curveto(535.93066406,457.16015625)(536.21777344,456.89257812)(536.44433594,456.54101562)
\curveto(536.67089844,456.19335938)(536.84863281,455.76757812)(536.97753906,455.26367188)
\curveto(537.10644531,454.76367188)(537.17089844,454.08789062)(537.17089844,453.23632812)
\curveto(537.17089844,452.22851562)(537.06738281,451.4140625)(536.86035156,450.79296875)
\curveto(536.65332031,450.17578125)(536.34277344,449.69726562)(535.92871094,449.35742188)
\curveto(535.51855469,449.02148438)(534.99902344,448.85351562)(534.37011719,448.85351562)
\curveto(533.54199219,448.85351562)(532.89160156,449.15039062)(532.41894531,449.74414062)
\curveto(531.85253906,450.45898438)(531.56933594,451.62304688)(531.56933594,453.23632812)
\closepath
\moveto(532.65332031,453.23632812)
\curveto(532.65332031,451.82617188)(532.81738281,450.88671875)(533.14550781,450.41796875)
\curveto(533.47753906,449.953125)(533.88574219,449.72070312)(534.37011719,449.72070312)
\curveto(534.85449219,449.72070312)(535.26074219,449.95507812)(535.58886719,450.42382812)
\curveto(535.92089844,450.89257812)(536.08691406,451.83007812)(536.08691406,453.23632812)
\curveto(536.08691406,454.65039062)(535.92089844,455.58984375)(535.58886719,456.0546875)
\curveto(535.26074219,456.51953125)(534.85058594,456.75195312)(534.35839844,456.75195312)
\curveto(533.87402344,456.75195312)(533.48730469,456.546875)(533.19824219,456.13671875)
\curveto(532.83496094,455.61328125)(532.65332031,454.64648438)(532.65332031,453.23632812)
\closepath
}
}
{
\newrgbcolor{curcolor}{0 0 0}
\pscustom[linestyle=none,fillstyle=solid,fillcolor=curcolor]
{
\newpath
\moveto(539.22753906,446.47460938)
\lineto(538.47167969,446.47460938)
\curveto(539.63964844,448.34960938)(540.22363281,450.2265625)(540.22363281,452.10546875)
\curveto(540.22363281,452.83984375)(540.13964844,453.56835938)(539.97167969,454.29101562)
\curveto(539.83886719,454.87695312)(539.65332031,455.43945312)(539.41503906,455.97851562)
\curveto(539.26269531,456.33007812)(538.94824219,456.91601562)(538.47167969,457.73632812)
\lineto(539.22753906,457.73632812)
\curveto(539.96191406,456.75585938)(540.50488281,455.77148438)(540.85644531,454.78320312)
\curveto(541.15722656,453.93164062)(541.30761719,453.04101562)(541.30761719,452.11132812)
\curveto(541.30761719,451.05664062)(541.10449219,450.03710938)(540.69824219,449.05273438)
\curveto(540.29589844,448.06835938)(539.80566406,447.20898438)(539.22753906,446.47460938)
\closepath
}
}
{
\newrgbcolor{curcolor}{0 0 0}
\pscustom[linestyle=none,fillstyle=solid,fillcolor=curcolor]
{
\newpath
\moveto(166.76992187,406.75976562)
\lineto(167.8421875,406.85351562)
\curveto(167.89296875,406.42382812)(168.01015625,406.0703125)(168.19375,405.79296875)
\curveto(168.38125,405.51953125)(168.6703125,405.296875)(169.0609375,405.125)
\curveto(169.4515625,404.95703125)(169.89101562,404.87304688)(170.37929687,404.87304688)
\curveto(170.81289062,404.87304688)(171.19570312,404.9375)(171.52773437,405.06640625)
\curveto(171.85976562,405.1953125)(172.10585937,405.37109375)(172.26601562,405.59375)
\curveto(172.43007812,405.8203125)(172.51210937,406.06640625)(172.51210937,406.33203125)
\curveto(172.51210937,406.6015625)(172.43398437,406.8359375)(172.27773437,407.03515625)
\curveto(172.12148437,407.23828125)(171.86367187,407.40820312)(171.50429687,407.54492188)
\curveto(171.27382812,407.63476562)(170.7640625,407.7734375)(169.975,407.9609375)
\curveto(169.1859375,408.15234375)(168.63320312,408.33203125)(168.31679687,408.5)
\curveto(167.90664062,408.71484375)(167.6,408.98046875)(167.396875,409.296875)
\curveto(167.19765625,409.6171875)(167.09804687,409.97460938)(167.09804687,410.36914062)
\curveto(167.09804687,410.80273438)(167.22109375,411.20703125)(167.4671875,411.58203125)
\curveto(167.71328125,411.9609375)(168.07265625,412.24804688)(168.5453125,412.44335938)
\curveto(169.01796875,412.63867188)(169.54335937,412.73632812)(170.12148437,412.73632812)
\curveto(170.75820312,412.73632812)(171.31875,412.6328125)(171.803125,412.42578125)
\curveto(172.29140625,412.22265625)(172.66640625,411.921875)(172.928125,411.5234375)
\curveto(173.18984375,411.125)(173.33046875,410.67382812)(173.35,410.16992188)
\lineto(172.26015625,410.08789062)
\curveto(172.2015625,410.63085938)(172.00234375,411.04101562)(171.6625,411.31835938)
\curveto(171.3265625,411.59570312)(170.82851562,411.734375)(170.16835937,411.734375)
\curveto(169.48085937,411.734375)(168.97890625,411.60742188)(168.6625,411.35351562)
\curveto(168.35,411.10351562)(168.19375,410.80078125)(168.19375,410.4453125)
\curveto(168.19375,410.13671875)(168.30507812,409.8828125)(168.52773437,409.68359375)
\curveto(168.74648437,409.484375)(169.31679687,409.27929688)(170.23867187,409.06835938)
\curveto(171.16445312,408.86132812)(171.79921875,408.6796875)(172.14296875,408.5234375)
\curveto(172.64296875,408.29296875)(173.01210937,408)(173.25039062,407.64453125)
\curveto(173.48867187,407.29296875)(173.6078125,406.88671875)(173.6078125,406.42578125)
\curveto(173.6078125,405.96875)(173.47695312,405.53710938)(173.21523437,405.13085938)
\curveto(172.95351562,404.72851562)(172.5765625,404.4140625)(172.084375,404.1875)
\curveto(171.59609375,403.96484375)(171.0453125,403.85351562)(170.43203125,403.85351562)
\curveto(169.6546875,403.85351562)(169.00234375,403.96679688)(168.475,404.19335938)
\curveto(167.9515625,404.41992188)(167.53945312,404.75976562)(167.23867187,405.21289062)
\curveto(166.94179687,405.66992188)(166.78554687,406.18554688)(166.76992187,406.75976562)
\closepath
}
}
{
\newrgbcolor{curcolor}{0 0 0}
\pscustom[linestyle=none,fillstyle=solid,fillcolor=curcolor]
{
\newpath
\moveto(179.08632812,404.76757812)
\curveto(178.69570312,404.43554688)(178.31875,404.20117188)(177.95546875,404.06445312)
\curveto(177.59609375,403.92773438)(177.209375,403.859375)(176.7953125,403.859375)
\curveto(176.11171875,403.859375)(175.58632812,404.02539062)(175.21914062,404.35742188)
\curveto(174.85195312,404.69335938)(174.66835937,405.12109375)(174.66835937,405.640625)
\curveto(174.66835937,405.9453125)(174.73671875,406.22265625)(174.8734375,406.47265625)
\curveto(175.0140625,406.7265625)(175.19570312,406.9296875)(175.41835937,407.08203125)
\curveto(175.64492187,407.234375)(175.89882812,407.34960938)(176.18007812,407.42773438)
\curveto(176.38710937,407.48242188)(176.69960937,407.53515625)(177.11757812,407.5859375)
\curveto(177.96914062,407.6875)(178.59609375,407.80859375)(178.9984375,407.94921875)
\curveto(179.00234375,408.09375)(179.00429687,408.18554688)(179.00429687,408.22460938)
\curveto(179.00429687,408.65429688)(178.9046875,408.95703125)(178.70546875,409.1328125)
\curveto(178.4359375,409.37109375)(178.03554687,409.49023438)(177.50429687,409.49023438)
\curveto(177.00820312,409.49023438)(176.64101562,409.40234375)(176.40273437,409.2265625)
\curveto(176.16835937,409.0546875)(175.99453125,408.74804688)(175.88125,408.30664062)
\lineto(174.85,408.44726562)
\curveto(174.94375,408.88867188)(175.09804687,409.24414062)(175.31289062,409.51367188)
\curveto(175.52773437,409.78710938)(175.83828125,409.99609375)(176.24453125,410.140625)
\curveto(176.65078125,410.2890625)(177.12148437,410.36328125)(177.65664062,410.36328125)
\curveto(178.18789062,410.36328125)(178.61953125,410.30078125)(178.9515625,410.17578125)
\curveto(179.28359375,410.05078125)(179.52773437,409.89257812)(179.68398437,409.70117188)
\curveto(179.84023437,409.51367188)(179.94960937,409.27539062)(180.01210937,408.98632812)
\curveto(180.04726562,408.80664062)(180.06484375,408.48242188)(180.06484375,408.01367188)
\lineto(180.06484375,406.60742188)
\curveto(180.06484375,405.62695312)(180.08632812,405.00585938)(180.12929687,404.74414062)
\curveto(180.17617187,404.48632812)(180.26601562,404.23828125)(180.39882812,404)
\lineto(179.29726562,404)
\curveto(179.18789062,404.21875)(179.11757812,404.47460938)(179.08632812,404.76757812)
\closepath
\moveto(178.9984375,407.12304688)
\curveto(178.615625,406.96679688)(178.04140625,406.83398438)(177.27578125,406.72460938)
\curveto(176.8421875,406.66210938)(176.53554687,406.59179688)(176.35585937,406.51367188)
\curveto(176.17617187,406.43554688)(176.0375,406.3203125)(175.93984375,406.16796875)
\curveto(175.8421875,406.01953125)(175.79335937,405.85351562)(175.79335937,405.66992188)
\curveto(175.79335937,405.38867188)(175.89882812,405.15429688)(176.10976562,404.96679688)
\curveto(176.32460937,404.77929688)(176.63710937,404.68554688)(177.04726562,404.68554688)
\curveto(177.45351562,404.68554688)(177.81484375,404.7734375)(178.13125,404.94921875)
\curveto(178.44765625,405.12890625)(178.68007812,405.37304688)(178.82851562,405.68164062)
\curveto(178.94179687,405.91992188)(178.9984375,406.27148438)(178.9984375,406.73632812)
\closepath
}
}
{
\newrgbcolor{curcolor}{0 0 0}
\pscustom[linestyle=none,fillstyle=solid,fillcolor=curcolor]
{
\newpath
\moveto(181.69960937,404)
\lineto(181.69960937,410.22265625)
\lineto(182.64296875,410.22265625)
\lineto(182.64296875,409.34960938)
\curveto(182.83828125,409.65429688)(183.09804687,409.8984375)(183.42226562,410.08203125)
\curveto(183.74648437,410.26953125)(184.115625,410.36328125)(184.5296875,410.36328125)
\curveto(184.990625,410.36328125)(185.36757812,410.26757812)(185.66054687,410.07617188)
\curveto(185.95742187,409.88476562)(186.16640625,409.6171875)(186.2875,409.2734375)
\curveto(186.7796875,410)(187.4203125,410.36328125)(188.209375,410.36328125)
\curveto(188.8265625,410.36328125)(189.30117187,410.19140625)(189.63320312,409.84765625)
\curveto(189.96523437,409.5078125)(190.13125,408.98242188)(190.13125,408.27148438)
\lineto(190.13125,404)
\lineto(189.08242187,404)
\lineto(189.08242187,407.91992188)
\curveto(189.08242187,408.34179688)(189.04726562,408.64453125)(188.97695312,408.828125)
\curveto(188.91054687,409.015625)(188.7875,409.16601562)(188.6078125,409.27929688)
\curveto(188.428125,409.39257812)(188.2171875,409.44921875)(187.975,409.44921875)
\curveto(187.5375,409.44921875)(187.17421875,409.30273438)(186.88515625,409.00976562)
\curveto(186.59609375,408.72070312)(186.4515625,408.25585938)(186.4515625,407.61523438)
\lineto(186.4515625,404)
\lineto(185.396875,404)
\lineto(185.396875,408.04296875)
\curveto(185.396875,408.51171875)(185.3109375,408.86328125)(185.1390625,409.09765625)
\curveto(184.9671875,409.33203125)(184.6859375,409.44921875)(184.2953125,409.44921875)
\curveto(183.9984375,409.44921875)(183.72304687,409.37109375)(183.46914062,409.21484375)
\curveto(183.21914062,409.05859375)(183.0375,408.83007812)(182.92421875,408.52929688)
\curveto(182.8109375,408.22851562)(182.75429687,407.79492188)(182.75429687,407.22851562)
\lineto(182.75429687,404)
\closepath
}
}
{
\newrgbcolor{curcolor}{0 0 0}
\pscustom[linestyle=none,fillstyle=solid,fillcolor=curcolor]
{
\newpath
\moveto(195.95546875,406.00390625)
\lineto(197.0453125,405.86914062)
\curveto(196.8734375,405.23242188)(196.55507812,404.73828125)(196.09023437,404.38671875)
\curveto(195.62539062,404.03515625)(195.03164062,403.859375)(194.30898437,403.859375)
\curveto(193.39882812,403.859375)(192.67617187,404.13867188)(192.14101562,404.69726562)
\curveto(191.60976562,405.25976562)(191.34414062,406.046875)(191.34414062,407.05859375)
\curveto(191.34414062,408.10546875)(191.61367187,408.91796875)(192.15273437,409.49609375)
\curveto(192.69179687,410.07421875)(193.39101562,410.36328125)(194.25039062,410.36328125)
\curveto(195.08242187,410.36328125)(195.76210937,410.08007812)(196.28945312,409.51367188)
\curveto(196.81679687,408.94726562)(197.08046875,408.15039062)(197.08046875,407.12304688)
\curveto(197.08046875,407.06054688)(197.07851562,406.96679688)(197.07460937,406.84179688)
\lineto(192.43398437,406.84179688)
\curveto(192.47304687,406.15820312)(192.66640625,405.63476562)(193.0140625,405.27148438)
\curveto(193.36171875,404.90820312)(193.7953125,404.7265625)(194.31484375,404.7265625)
\curveto(194.7015625,404.7265625)(195.03164062,404.828125)(195.30507812,405.03125)
\curveto(195.57851562,405.234375)(195.7953125,405.55859375)(195.95546875,406.00390625)
\closepath
\moveto(192.49257812,407.70898438)
\lineto(195.9671875,407.70898438)
\curveto(195.9203125,408.23242188)(195.7875,408.625)(195.56875,408.88671875)
\curveto(195.2328125,409.29296875)(194.79726562,409.49609375)(194.26210937,409.49609375)
\curveto(193.77773437,409.49609375)(193.36953125,409.33398438)(193.0375,409.00976562)
\curveto(192.709375,408.68554688)(192.52773437,408.25195312)(192.49257812,407.70898438)
\closepath
}
}
{
\newrgbcolor{curcolor}{0 0 0}
\pscustom[linestyle=none,fillstyle=solid,fillcolor=curcolor]
{
\newpath
\moveto(201.8265625,404)
\lineto(201.8265625,412.58984375)
\lineto(202.99257812,412.58984375)
\lineto(207.50429687,405.84570312)
\lineto(207.50429687,412.58984375)
\lineto(208.59414062,412.58984375)
\lineto(208.59414062,404)
\lineto(207.428125,404)
\lineto(202.91640625,410.75)
\lineto(202.91640625,404)
\closepath
}
}
{
\newrgbcolor{curcolor}{0 0 0}
\pscustom[linestyle=none,fillstyle=solid,fillcolor=curcolor]
{
\newpath
\moveto(209.97695312,407.11132812)
\curveto(209.97695312,408.26367188)(210.29726562,409.1171875)(210.93789062,409.671875)
\curveto(211.47304687,410.1328125)(212.12539062,410.36328125)(212.89492187,410.36328125)
\curveto(213.75039062,410.36328125)(214.44960937,410.08203125)(214.99257812,409.51953125)
\curveto(215.53554687,408.9609375)(215.80703125,408.1875)(215.80703125,407.19921875)
\curveto(215.80703125,406.3984375)(215.6859375,405.76757812)(215.44375,405.30664062)
\curveto(215.20546875,404.84960938)(214.85585937,404.49414062)(214.39492187,404.24023438)
\curveto(213.93789062,403.98632812)(213.43789062,403.859375)(212.89492187,403.859375)
\curveto(212.02382812,403.859375)(211.31875,404.13867188)(210.7796875,404.69726562)
\curveto(210.24453125,405.25585938)(209.97695312,406.06054688)(209.97695312,407.11132812)
\closepath
\moveto(211.0609375,407.11132812)
\curveto(211.0609375,406.31445312)(211.23476562,405.71679688)(211.58242187,405.31835938)
\curveto(211.93007812,404.92382812)(212.36757812,404.7265625)(212.89492187,404.7265625)
\curveto(213.41835937,404.7265625)(213.85390625,404.92578125)(214.2015625,405.32421875)
\curveto(214.54921875,405.72265625)(214.72304687,406.33007812)(214.72304687,407.14648438)
\curveto(214.72304687,407.91601562)(214.54726562,408.49804688)(214.19570312,408.89257812)
\curveto(213.84804687,409.29101562)(213.41445312,409.49023438)(212.89492187,409.49023438)
\curveto(212.36757812,409.49023438)(211.93007812,409.29296875)(211.58242187,408.8984375)
\curveto(211.23476562,408.50390625)(211.0609375,407.90820312)(211.0609375,407.11132812)
\closepath
}
}
{
\newrgbcolor{curcolor}{0 0 0}
\pscustom[linestyle=none,fillstyle=solid,fillcolor=curcolor]
{
\newpath
\moveto(221.08046875,404)
\lineto(221.08046875,404.78515625)
\curveto(220.6859375,404.16796875)(220.10585937,403.859375)(219.34023437,403.859375)
\curveto(218.84414062,403.859375)(218.38710937,403.99609375)(217.96914062,404.26953125)
\curveto(217.55507812,404.54296875)(217.2328125,404.92382812)(217.00234375,405.41210938)
\curveto(216.77578125,405.90429688)(216.6625,406.46875)(216.6625,407.10546875)
\curveto(216.6625,407.7265625)(216.76601562,408.2890625)(216.97304687,408.79296875)
\curveto(217.18007812,409.30078125)(217.490625,409.68945312)(217.9046875,409.95898438)
\curveto(218.31875,410.22851562)(218.78164062,410.36328125)(219.29335937,410.36328125)
\curveto(219.66835937,410.36328125)(220.00234375,410.28320312)(220.2953125,410.12304688)
\curveto(220.58828125,409.96679688)(220.8265625,409.76171875)(221.01015625,409.5078125)
\lineto(221.01015625,412.58984375)
\lineto(222.05898437,412.58984375)
\lineto(222.05898437,404)
\closepath
\moveto(217.74648437,407.10546875)
\curveto(217.74648437,406.30859375)(217.91445312,405.71289062)(218.25039062,405.31835938)
\curveto(218.58632812,404.92382812)(218.9828125,404.7265625)(219.43984375,404.7265625)
\curveto(219.90078125,404.7265625)(220.29140625,404.9140625)(220.61171875,405.2890625)
\curveto(220.9359375,405.66796875)(221.09804687,406.24414062)(221.09804687,407.01757812)
\curveto(221.09804687,407.86914062)(220.93398437,408.49414062)(220.60585937,408.89257812)
\curveto(220.27773437,409.29101562)(219.8734375,409.49023438)(219.39296875,409.49023438)
\curveto(218.92421875,409.49023438)(218.53164062,409.29882812)(218.21523437,408.91601562)
\curveto(217.90273437,408.53320312)(217.74648437,407.9296875)(217.74648437,407.10546875)
\closepath
}
}
{
\newrgbcolor{curcolor}{0 0 0}
\pscustom[linestyle=none,fillstyle=solid,fillcolor=curcolor]
{
\newpath
\moveto(227.97695312,406.00390625)
\lineto(229.06679687,405.86914062)
\curveto(228.89492187,405.23242188)(228.5765625,404.73828125)(228.11171875,404.38671875)
\curveto(227.646875,404.03515625)(227.053125,403.859375)(226.33046875,403.859375)
\curveto(225.4203125,403.859375)(224.69765625,404.13867188)(224.1625,404.69726562)
\curveto(223.63125,405.25976562)(223.365625,406.046875)(223.365625,407.05859375)
\curveto(223.365625,408.10546875)(223.63515625,408.91796875)(224.17421875,409.49609375)
\curveto(224.71328125,410.07421875)(225.4125,410.36328125)(226.271875,410.36328125)
\curveto(227.10390625,410.36328125)(227.78359375,410.08007812)(228.3109375,409.51367188)
\curveto(228.83828125,408.94726562)(229.10195312,408.15039062)(229.10195312,407.12304688)
\curveto(229.10195312,407.06054688)(229.1,406.96679688)(229.09609375,406.84179688)
\lineto(224.45546875,406.84179688)
\curveto(224.49453125,406.15820312)(224.68789062,405.63476562)(225.03554687,405.27148438)
\curveto(225.38320312,404.90820312)(225.81679687,404.7265625)(226.33632812,404.7265625)
\curveto(226.72304687,404.7265625)(227.053125,404.828125)(227.3265625,405.03125)
\curveto(227.6,405.234375)(227.81679687,405.55859375)(227.97695312,406.00390625)
\closepath
\moveto(224.5140625,407.70898438)
\lineto(227.98867187,407.70898438)
\curveto(227.94179687,408.23242188)(227.80898437,408.625)(227.59023437,408.88671875)
\curveto(227.25429687,409.29296875)(226.81875,409.49609375)(226.28359375,409.49609375)
\curveto(225.79921875,409.49609375)(225.39101562,409.33398438)(225.05898437,409.00976562)
\curveto(224.73085937,408.68554688)(224.54921875,408.25195312)(224.5140625,407.70898438)
\closepath
}
}
{
\newrgbcolor{curcolor}{0 0 1}
\pscustom[linewidth=1,linecolor=curcolor]
{
\newpath
\moveto(237.9,407.9)
\lineto(280.1,407.9)
\moveto(105.1,79.8)
\lineto(183.4,130.3)
\lineto(261.7,223.4)
\lineto(340.1,263.9)
\lineto(418.4,308.7)
\lineto(496.7,353.6)
\lineto(575,399.7)
}
}
{
\newrgbcolor{curcolor}{0 0 0}
\pscustom[linestyle=none,fillstyle=solid,fillcolor=curcolor]
{
\newpath
\moveto(153.36953125,386)
\lineto(153.36953125,394.58984375)
\lineto(156.32851562,394.58984375)
\curveto(156.99648437,394.58984375)(157.50625,394.54882812)(157.8578125,394.46679688)
\curveto(158.35,394.35351562)(158.76992187,394.1484375)(159.11757812,393.8515625)
\curveto(159.57070312,393.46875)(159.90859375,392.97851562)(160.13125,392.38085938)
\curveto(160.3578125,391.78710938)(160.47109375,391.10742188)(160.47109375,390.34179688)
\curveto(160.47109375,389.68945312)(160.39492187,389.11132812)(160.24257812,388.60742188)
\curveto(160.09023437,388.10351562)(159.89492187,387.68554688)(159.65664062,387.35351562)
\curveto(159.41835937,387.02539062)(159.15664062,386.765625)(158.87148437,386.57421875)
\curveto(158.59023437,386.38671875)(158.2484375,386.24414062)(157.84609375,386.14648438)
\curveto(157.44765625,386.04882812)(156.98867187,386)(156.46914062,386)
\closepath
\moveto(154.50625,387.01367188)
\lineto(156.34023437,387.01367188)
\curveto(156.90664062,387.01367188)(157.35,387.06640625)(157.6703125,387.171875)
\curveto(157.99453125,387.27734375)(158.25234375,387.42578125)(158.44375,387.6171875)
\curveto(158.71328125,387.88671875)(158.92226562,388.24804688)(159.07070312,388.70117188)
\curveto(159.22304687,389.15820312)(159.29921875,389.7109375)(159.29921875,390.359375)
\curveto(159.29921875,391.2578125)(159.15078125,391.94726562)(158.85390625,392.42773438)
\curveto(158.5609375,392.91210938)(158.20351562,393.23632812)(157.78164062,393.40039062)
\curveto(157.47695312,393.51757812)(156.98671875,393.57617188)(156.3109375,393.57617188)
\lineto(154.50625,393.57617188)
\closepath
}
}
{
\newrgbcolor{curcolor}{0 0 0}
\pscustom[linestyle=none,fillstyle=solid,fillcolor=curcolor]
{
\newpath
\moveto(161.90664062,393.37695312)
\lineto(161.90664062,394.58984375)
\lineto(162.96132812,394.58984375)
\lineto(162.96132812,393.37695312)
\closepath
\moveto(161.90664062,386)
\lineto(161.90664062,392.22265625)
\lineto(162.96132812,392.22265625)
\lineto(162.96132812,386)
\closepath
}
}
{
\newrgbcolor{curcolor}{0 0 0}
\pscustom[linestyle=none,fillstyle=solid,fillcolor=curcolor]
{
\newpath
\moveto(164.81875,386)
\lineto(164.81875,391.40234375)
\lineto(163.88710937,391.40234375)
\lineto(163.88710937,392.22265625)
\lineto(164.81875,392.22265625)
\lineto(164.81875,392.88476562)
\curveto(164.81875,393.30273438)(164.85585937,393.61328125)(164.93007812,393.81640625)
\curveto(165.03164062,394.08984375)(165.209375,394.31054688)(165.46328125,394.47851562)
\curveto(165.72109375,394.65039062)(166.08046875,394.73632812)(166.54140625,394.73632812)
\curveto(166.83828125,394.73632812)(167.16640625,394.70117188)(167.52578125,394.63085938)
\lineto(167.36757812,393.7109375)
\curveto(167.14882812,393.75)(166.94179687,393.76953125)(166.74648437,393.76953125)
\curveto(166.42617187,393.76953125)(166.19960937,393.70117188)(166.06679687,393.56445312)
\curveto(165.93398437,393.42773438)(165.86757812,393.171875)(165.86757812,392.796875)
\lineto(165.86757812,392.22265625)
\lineto(167.08046875,392.22265625)
\lineto(167.08046875,391.40234375)
\lineto(165.86757812,391.40234375)
\lineto(165.86757812,386)
\closepath
}
}
{
\newrgbcolor{curcolor}{0 0 0}
\pscustom[linestyle=none,fillstyle=solid,fillcolor=curcolor]
{
\newpath
\moveto(167.9359375,386)
\lineto(167.9359375,391.40234375)
\lineto(167.00429687,391.40234375)
\lineto(167.00429687,392.22265625)
\lineto(167.9359375,392.22265625)
\lineto(167.9359375,392.88476562)
\curveto(167.9359375,393.30273438)(167.97304687,393.61328125)(168.04726562,393.81640625)
\curveto(168.14882812,394.08984375)(168.3265625,394.31054688)(168.58046875,394.47851562)
\curveto(168.83828125,394.65039062)(169.19765625,394.73632812)(169.65859375,394.73632812)
\curveto(169.95546875,394.73632812)(170.28359375,394.70117188)(170.64296875,394.63085938)
\lineto(170.48476562,393.7109375)
\curveto(170.26601562,393.75)(170.05898437,393.76953125)(169.86367187,393.76953125)
\curveto(169.54335937,393.76953125)(169.31679687,393.70117188)(169.18398437,393.56445312)
\curveto(169.05117187,393.42773438)(168.98476562,393.171875)(168.98476562,392.796875)
\lineto(168.98476562,392.22265625)
\lineto(170.19765625,392.22265625)
\lineto(170.19765625,391.40234375)
\lineto(168.98476562,391.40234375)
\lineto(168.98476562,386)
\closepath
}
}
{
\newrgbcolor{curcolor}{0 0 0}
\pscustom[linestyle=none,fillstyle=solid,fillcolor=curcolor]
{
\newpath
\moveto(175.27773437,388.00390625)
\lineto(176.36757812,387.86914062)
\curveto(176.19570312,387.23242188)(175.87734375,386.73828125)(175.4125,386.38671875)
\curveto(174.94765625,386.03515625)(174.35390625,385.859375)(173.63125,385.859375)
\curveto(172.72109375,385.859375)(171.9984375,386.13867188)(171.46328125,386.69726562)
\curveto(170.93203125,387.25976562)(170.66640625,388.046875)(170.66640625,389.05859375)
\curveto(170.66640625,390.10546875)(170.9359375,390.91796875)(171.475,391.49609375)
\curveto(172.0140625,392.07421875)(172.71328125,392.36328125)(173.57265625,392.36328125)
\curveto(174.4046875,392.36328125)(175.084375,392.08007812)(175.61171875,391.51367188)
\curveto(176.1390625,390.94726562)(176.40273437,390.15039062)(176.40273437,389.12304688)
\curveto(176.40273437,389.06054688)(176.40078125,388.96679688)(176.396875,388.84179688)
\lineto(171.75625,388.84179688)
\curveto(171.7953125,388.15820312)(171.98867187,387.63476562)(172.33632812,387.27148438)
\curveto(172.68398437,386.90820312)(173.11757812,386.7265625)(173.63710937,386.7265625)
\curveto(174.02382812,386.7265625)(174.35390625,386.828125)(174.62734375,387.03125)
\curveto(174.90078125,387.234375)(175.11757812,387.55859375)(175.27773437,388.00390625)
\closepath
\moveto(171.81484375,389.70898438)
\lineto(175.28945312,389.70898438)
\curveto(175.24257812,390.23242188)(175.10976562,390.625)(174.89101562,390.88671875)
\curveto(174.55507812,391.29296875)(174.11953125,391.49609375)(173.584375,391.49609375)
\curveto(173.1,391.49609375)(172.69179687,391.33398438)(172.35976562,391.00976562)
\curveto(172.03164062,390.68554688)(171.85,390.25195312)(171.81484375,389.70898438)
\closepath
}
}
{
\newrgbcolor{curcolor}{0 0 0}
\pscustom[linestyle=none,fillstyle=solid,fillcolor=curcolor]
{
\newpath
\moveto(177.68007812,386)
\lineto(177.68007812,392.22265625)
\lineto(178.62929687,392.22265625)
\lineto(178.62929687,391.27929688)
\curveto(178.87148437,391.72070312)(179.09414062,392.01171875)(179.29726562,392.15234375)
\curveto(179.50429687,392.29296875)(179.73085937,392.36328125)(179.97695312,392.36328125)
\curveto(180.33242187,392.36328125)(180.69375,392.25)(181.0609375,392.0234375)
\lineto(180.69765625,391.04492188)
\curveto(180.43984375,391.19726562)(180.18203125,391.2734375)(179.92421875,391.2734375)
\curveto(179.69375,391.2734375)(179.48671875,391.203125)(179.303125,391.0625)
\curveto(179.11953125,390.92578125)(178.98867187,390.734375)(178.91054687,390.48828125)
\curveto(178.79335937,390.11328125)(178.73476562,389.703125)(178.73476562,389.2578125)
\lineto(178.73476562,386)
\closepath
}
}
{
\newrgbcolor{curcolor}{0 0 0}
\pscustom[linestyle=none,fillstyle=solid,fillcolor=curcolor]
{
\newpath
\moveto(185.94765625,388.00390625)
\lineto(187.0375,387.86914062)
\curveto(186.865625,387.23242188)(186.54726562,386.73828125)(186.08242187,386.38671875)
\curveto(185.61757812,386.03515625)(185.02382812,385.859375)(184.30117187,385.859375)
\curveto(183.39101562,385.859375)(182.66835937,386.13867188)(182.13320312,386.69726562)
\curveto(181.60195312,387.25976562)(181.33632812,388.046875)(181.33632812,389.05859375)
\curveto(181.33632812,390.10546875)(181.60585937,390.91796875)(182.14492187,391.49609375)
\curveto(182.68398437,392.07421875)(183.38320312,392.36328125)(184.24257812,392.36328125)
\curveto(185.07460937,392.36328125)(185.75429687,392.08007812)(186.28164062,391.51367188)
\curveto(186.80898437,390.94726562)(187.07265625,390.15039062)(187.07265625,389.12304688)
\curveto(187.07265625,389.06054688)(187.07070312,388.96679688)(187.06679687,388.84179688)
\lineto(182.42617187,388.84179688)
\curveto(182.46523437,388.15820312)(182.65859375,387.63476562)(183.00625,387.27148438)
\curveto(183.35390625,386.90820312)(183.7875,386.7265625)(184.30703125,386.7265625)
\curveto(184.69375,386.7265625)(185.02382812,386.828125)(185.29726562,387.03125)
\curveto(185.57070312,387.234375)(185.7875,387.55859375)(185.94765625,388.00390625)
\closepath
\moveto(182.48476562,389.70898438)
\lineto(185.959375,389.70898438)
\curveto(185.9125,390.23242188)(185.7796875,390.625)(185.5609375,390.88671875)
\curveto(185.225,391.29296875)(184.78945312,391.49609375)(184.25429687,391.49609375)
\curveto(183.76992187,391.49609375)(183.36171875,391.33398438)(183.0296875,391.00976562)
\curveto(182.7015625,390.68554688)(182.51992187,390.25195312)(182.48476562,389.70898438)
\closepath
}
}
{
\newrgbcolor{curcolor}{0 0 0}
\pscustom[linestyle=none,fillstyle=solid,fillcolor=curcolor]
{
\newpath
\moveto(188.36171875,386)
\lineto(188.36171875,392.22265625)
\lineto(189.3109375,392.22265625)
\lineto(189.3109375,391.33789062)
\curveto(189.76796875,392.02148438)(190.428125,392.36328125)(191.29140625,392.36328125)
\curveto(191.66640625,392.36328125)(192.01015625,392.29492188)(192.32265625,392.15820312)
\curveto(192.6390625,392.02539062)(192.87539062,391.84960938)(193.03164062,391.63085938)
\curveto(193.18789062,391.41210938)(193.29726562,391.15234375)(193.35976562,390.8515625)
\curveto(193.39882812,390.65625)(193.41835937,390.31445312)(193.41835937,389.82617188)
\lineto(193.41835937,386)
\lineto(192.36367187,386)
\lineto(192.36367187,389.78515625)
\curveto(192.36367187,390.21484375)(192.32265625,390.53515625)(192.240625,390.74609375)
\curveto(192.15859375,390.9609375)(192.01210937,391.13085938)(191.80117187,391.25585938)
\curveto(191.59414062,391.38476562)(191.35,391.44921875)(191.06875,391.44921875)
\curveto(190.61953125,391.44921875)(190.23085937,391.30664062)(189.90273437,391.02148438)
\curveto(189.57851562,390.73632812)(189.41640625,390.1953125)(189.41640625,389.3984375)
\lineto(189.41640625,386)
\closepath
}
}
{
\newrgbcolor{curcolor}{0 0 0}
\pscustom[linestyle=none,fillstyle=solid,fillcolor=curcolor]
{
\newpath
\moveto(197.33828125,386.94335938)
\lineto(197.490625,386.01171875)
\curveto(197.19375,385.94921875)(196.928125,385.91796875)(196.69375,385.91796875)
\curveto(196.3109375,385.91796875)(196.0140625,385.97851562)(195.803125,386.09960938)
\curveto(195.5921875,386.22070312)(195.44375,386.37890625)(195.3578125,386.57421875)
\curveto(195.271875,386.7734375)(195.22890625,387.18945312)(195.22890625,387.82226562)
\lineto(195.22890625,391.40234375)
\lineto(194.45546875,391.40234375)
\lineto(194.45546875,392.22265625)
\lineto(195.22890625,392.22265625)
\lineto(195.22890625,393.76367188)
\lineto(196.27773437,394.39648438)
\lineto(196.27773437,392.22265625)
\lineto(197.33828125,392.22265625)
\lineto(197.33828125,391.40234375)
\lineto(196.27773437,391.40234375)
\lineto(196.27773437,387.76367188)
\curveto(196.27773437,387.46289062)(196.2953125,387.26953125)(196.33046875,387.18359375)
\curveto(196.36953125,387.09765625)(196.43007812,387.02929688)(196.51210937,386.97851562)
\curveto(196.59804687,386.92773438)(196.71914062,386.90234375)(196.87539062,386.90234375)
\curveto(196.99257812,386.90234375)(197.146875,386.91601562)(197.33828125,386.94335938)
\closepath
}
}
{
\newrgbcolor{curcolor}{0 0 0}
\pscustom[linestyle=none,fillstyle=solid,fillcolor=curcolor]
{
\newpath
\moveto(201.8265625,386)
\lineto(201.8265625,394.58984375)
\lineto(202.99257812,394.58984375)
\lineto(207.50429687,387.84570312)
\lineto(207.50429687,394.58984375)
\lineto(208.59414062,394.58984375)
\lineto(208.59414062,386)
\lineto(207.428125,386)
\lineto(202.91640625,392.75)
\lineto(202.91640625,386)
\closepath
}
}
{
\newrgbcolor{curcolor}{0 0 0}
\pscustom[linestyle=none,fillstyle=solid,fillcolor=curcolor]
{
\newpath
\moveto(209.97695312,389.11132812)
\curveto(209.97695312,390.26367188)(210.29726562,391.1171875)(210.93789062,391.671875)
\curveto(211.47304687,392.1328125)(212.12539062,392.36328125)(212.89492187,392.36328125)
\curveto(213.75039062,392.36328125)(214.44960937,392.08203125)(214.99257812,391.51953125)
\curveto(215.53554687,390.9609375)(215.80703125,390.1875)(215.80703125,389.19921875)
\curveto(215.80703125,388.3984375)(215.6859375,387.76757812)(215.44375,387.30664062)
\curveto(215.20546875,386.84960938)(214.85585937,386.49414062)(214.39492187,386.24023438)
\curveto(213.93789062,385.98632812)(213.43789062,385.859375)(212.89492187,385.859375)
\curveto(212.02382812,385.859375)(211.31875,386.13867188)(210.7796875,386.69726562)
\curveto(210.24453125,387.25585938)(209.97695312,388.06054688)(209.97695312,389.11132812)
\closepath
\moveto(211.0609375,389.11132812)
\curveto(211.0609375,388.31445312)(211.23476562,387.71679688)(211.58242187,387.31835938)
\curveto(211.93007812,386.92382812)(212.36757812,386.7265625)(212.89492187,386.7265625)
\curveto(213.41835937,386.7265625)(213.85390625,386.92578125)(214.2015625,387.32421875)
\curveto(214.54921875,387.72265625)(214.72304687,388.33007812)(214.72304687,389.14648438)
\curveto(214.72304687,389.91601562)(214.54726562,390.49804688)(214.19570312,390.89257812)
\curveto(213.84804687,391.29101562)(213.41445312,391.49023438)(212.89492187,391.49023438)
\curveto(212.36757812,391.49023438)(211.93007812,391.29296875)(211.58242187,390.8984375)
\curveto(211.23476562,390.50390625)(211.0609375,389.90820312)(211.0609375,389.11132812)
\closepath
}
}
{
\newrgbcolor{curcolor}{0 0 0}
\pscustom[linestyle=none,fillstyle=solid,fillcolor=curcolor]
{
\newpath
\moveto(221.08046875,386)
\lineto(221.08046875,386.78515625)
\curveto(220.6859375,386.16796875)(220.10585937,385.859375)(219.34023437,385.859375)
\curveto(218.84414062,385.859375)(218.38710937,385.99609375)(217.96914062,386.26953125)
\curveto(217.55507812,386.54296875)(217.2328125,386.92382812)(217.00234375,387.41210938)
\curveto(216.77578125,387.90429688)(216.6625,388.46875)(216.6625,389.10546875)
\curveto(216.6625,389.7265625)(216.76601562,390.2890625)(216.97304687,390.79296875)
\curveto(217.18007812,391.30078125)(217.490625,391.68945312)(217.9046875,391.95898438)
\curveto(218.31875,392.22851562)(218.78164062,392.36328125)(219.29335937,392.36328125)
\curveto(219.66835937,392.36328125)(220.00234375,392.28320312)(220.2953125,392.12304688)
\curveto(220.58828125,391.96679688)(220.8265625,391.76171875)(221.01015625,391.5078125)
\lineto(221.01015625,394.58984375)
\lineto(222.05898437,394.58984375)
\lineto(222.05898437,386)
\closepath
\moveto(217.74648437,389.10546875)
\curveto(217.74648437,388.30859375)(217.91445312,387.71289062)(218.25039062,387.31835938)
\curveto(218.58632812,386.92382812)(218.9828125,386.7265625)(219.43984375,386.7265625)
\curveto(219.90078125,386.7265625)(220.29140625,386.9140625)(220.61171875,387.2890625)
\curveto(220.9359375,387.66796875)(221.09804687,388.24414062)(221.09804687,389.01757812)
\curveto(221.09804687,389.86914062)(220.93398437,390.49414062)(220.60585937,390.89257812)
\curveto(220.27773437,391.29101562)(219.8734375,391.49023438)(219.39296875,391.49023438)
\curveto(218.92421875,391.49023438)(218.53164062,391.29882812)(218.21523437,390.91601562)
\curveto(217.90273437,390.53320312)(217.74648437,389.9296875)(217.74648437,389.10546875)
\closepath
}
}
{
\newrgbcolor{curcolor}{0 0 0}
\pscustom[linestyle=none,fillstyle=solid,fillcolor=curcolor]
{
\newpath
\moveto(227.97695312,388.00390625)
\lineto(229.06679687,387.86914062)
\curveto(228.89492187,387.23242188)(228.5765625,386.73828125)(228.11171875,386.38671875)
\curveto(227.646875,386.03515625)(227.053125,385.859375)(226.33046875,385.859375)
\curveto(225.4203125,385.859375)(224.69765625,386.13867188)(224.1625,386.69726562)
\curveto(223.63125,387.25976562)(223.365625,388.046875)(223.365625,389.05859375)
\curveto(223.365625,390.10546875)(223.63515625,390.91796875)(224.17421875,391.49609375)
\curveto(224.71328125,392.07421875)(225.4125,392.36328125)(226.271875,392.36328125)
\curveto(227.10390625,392.36328125)(227.78359375,392.08007812)(228.3109375,391.51367188)
\curveto(228.83828125,390.94726562)(229.10195312,390.15039062)(229.10195312,389.12304688)
\curveto(229.10195312,389.06054688)(229.1,388.96679688)(229.09609375,388.84179688)
\lineto(224.45546875,388.84179688)
\curveto(224.49453125,388.15820312)(224.68789062,387.63476562)(225.03554687,387.27148438)
\curveto(225.38320312,386.90820312)(225.81679687,386.7265625)(226.33632812,386.7265625)
\curveto(226.72304687,386.7265625)(227.053125,386.828125)(227.3265625,387.03125)
\curveto(227.6,387.234375)(227.81679687,387.55859375)(227.97695312,388.00390625)
\closepath
\moveto(224.5140625,389.70898438)
\lineto(227.98867187,389.70898438)
\curveto(227.94179687,390.23242188)(227.80898437,390.625)(227.59023437,390.88671875)
\curveto(227.25429687,391.29296875)(226.81875,391.49609375)(226.28359375,391.49609375)
\curveto(225.79921875,391.49609375)(225.39101562,391.33398438)(225.05898437,391.00976562)
\curveto(224.73085937,390.68554688)(224.54921875,390.25195312)(224.5140625,389.70898438)
\closepath
}
}
{
\newrgbcolor{curcolor}{1 0 0}
\pscustom[linewidth=1,linecolor=curcolor]
{
\newpath
\moveto(237.9,389.9)
\lineto(280.1,389.9)
\moveto(105.1,90.1)
\lineto(183.4,127.1)
\lineto(261.7,226.7)
\lineto(340.1,270.1)
\lineto(418.4,315.3)
\lineto(496.7,362.2)
\lineto(575,408.3)
}
}
{
\newrgbcolor{curcolor}{0 0 0}
\pscustom[linewidth=1,linecolor=curcolor]
{
\newpath
\moveto(105.1,425.9)
\lineto(105.1,57.6)
\lineto(575,57.6)
\lineto(575,425.9)
\closepath
}
}
\end{pspicture}
}
    \captionsetup{width=0.75\linewidth}
    \caption{Repeating the previous testing with NUMA Balancing enabled does not improve performance when accessing the ARC from a different node.}
    \label{fig:NUMABalance}
\end{figure}

Other testing repeatedly reading a file from a different node with NUMA Balancing enabled confirms this,
as the read latency never improves and is always higher than a read from the same node as the ARC data.

\begin{figure}[H]
    \centering
    \begin{subfigure}{0.4\linewidth}
        \resizebox{0.9\linewidth}{!}{%LaTeX with PSTricks extensions
%%Creator: Inkscape 1.0.2-2 (e86c870879, 2021-01-15)
%%Please note this file requires PSTricks extensions
\psset{xunit=.5pt,yunit=.5pt,runit=.5pt}
\begin{pspicture}(600,480)
{
\newrgbcolor{curcolor}{0 0 0}
\pscustom[linewidth=1,linecolor=curcolor]
{
\newpath
\moveto(96.8,57.6)
\lineto(105.8,57.6)
\moveto(575,57.6)
\lineto(566,57.6)
}
}
{
\newrgbcolor{curcolor}{0 0 0}
\pscustom[linestyle=none,fillstyle=solid,fillcolor=curcolor]
{
\newpath
\moveto(54.49804688,54.71367187)
\lineto(54.49804688,53.7)
\lineto(48.8203125,53.7)
\curveto(48.8125,53.95390625)(48.85351562,54.19804687)(48.94335938,54.43242187)
\curveto(49.08789062,54.81914062)(49.31835938,55.2)(49.63476562,55.575)
\curveto(49.95507812,55.95)(50.41601562,56.38359375)(51.01757812,56.87578125)
\curveto(51.95117188,57.64140625)(52.58203125,58.246875)(52.91015625,58.6921875)
\curveto(53.23828125,59.14140625)(53.40234375,59.56523437)(53.40234375,59.96367187)
\curveto(53.40234375,60.38164062)(53.25195312,60.73320312)(52.95117188,61.01835937)
\curveto(52.65429688,61.30742187)(52.265625,61.45195312)(51.78515625,61.45195312)
\curveto(51.27734375,61.45195312)(50.87109375,61.29960937)(50.56640625,60.99492187)
\curveto(50.26171875,60.69023437)(50.10742188,60.26835937)(50.10351562,59.72929687)
\lineto(49.01953125,59.840625)
\curveto(49.09375,60.64921875)(49.37304688,61.26445312)(49.85742188,61.68632812)
\curveto(50.34179688,62.11210937)(50.9921875,62.325)(51.80859375,62.325)
\curveto(52.6328125,62.325)(53.28515625,62.09648437)(53.765625,61.63945312)
\curveto(54.24609375,61.18242187)(54.48632812,60.61601562)(54.48632812,59.94023437)
\curveto(54.48632812,59.59648437)(54.41601562,59.25859375)(54.27539062,58.9265625)
\curveto(54.13476562,58.59453125)(53.90039062,58.24492187)(53.57226562,57.87773437)
\curveto(53.24804688,57.51054687)(52.70703125,57.00664062)(51.94921875,56.36601562)
\curveto(51.31640625,55.83476562)(50.91015625,55.4734375)(50.73046875,55.28203125)
\curveto(50.55078125,55.09453125)(50.40234375,54.90507812)(50.28515625,54.71367187)
\closepath
}
}
{
\newrgbcolor{curcolor}{0 0 0}
\pscustom[linestyle=none,fillstyle=solid,fillcolor=curcolor]
{
\newpath
\moveto(61.1015625,60.18632812)
\lineto(60.05273438,60.10429687)
\curveto(59.95898438,60.51835937)(59.82617188,60.81914062)(59.65429688,61.00664062)
\curveto(59.36914062,61.30742187)(59.01757812,61.4578125)(58.59960938,61.4578125)
\curveto(58.26367188,61.4578125)(57.96875,61.3640625)(57.71484375,61.1765625)
\curveto(57.3828125,60.934375)(57.12109375,60.58085937)(56.9296875,60.11601562)
\curveto(56.73828125,59.65117187)(56.63867188,58.9890625)(56.63085938,58.1296875)
\curveto(56.88476562,58.51640625)(57.1953125,58.80351562)(57.5625,58.99101562)
\curveto(57.9296875,59.17851562)(58.31445312,59.27226562)(58.71679688,59.27226562)
\curveto(59.41992188,59.27226562)(60.01757812,59.0125)(60.50976562,58.49296875)
\curveto(61.00585938,57.97734375)(61.25390625,57.309375)(61.25390625,56.4890625)
\curveto(61.25390625,55.95)(61.13671875,55.44804687)(60.90234375,54.98320312)
\curveto(60.671875,54.52226562)(60.35351562,54.16875)(59.94726562,53.92265625)
\curveto(59.54101562,53.6765625)(59.08007812,53.55351562)(58.56445312,53.55351562)
\curveto(57.68554688,53.55351562)(56.96875,53.87578125)(56.4140625,54.5203125)
\curveto(55.859375,55.16875)(55.58203125,56.23515625)(55.58203125,57.71953125)
\curveto(55.58203125,59.3796875)(55.88867188,60.58671875)(56.50195312,61.340625)
\curveto(57.03710938,61.996875)(57.7578125,62.325)(58.6640625,62.325)
\curveto(59.33984375,62.325)(59.89257812,62.13554687)(60.32226562,61.75664062)
\curveto(60.75585938,61.37773437)(61.015625,60.85429687)(61.1015625,60.18632812)
\closepath
\moveto(56.79492188,56.48320312)
\curveto(56.79492188,56.11992187)(56.87109375,55.77226562)(57.0234375,55.44023437)
\curveto(57.1796875,55.10820312)(57.39648438,54.85429687)(57.67382812,54.67851562)
\curveto(57.95117188,54.50664062)(58.2421875,54.42070312)(58.546875,54.42070312)
\curveto(58.9921875,54.42070312)(59.375,54.60039062)(59.6953125,54.95976562)
\curveto(60.015625,55.31914062)(60.17578125,55.80742187)(60.17578125,56.42460937)
\curveto(60.17578125,57.01835937)(60.01757812,57.48515625)(59.70117188,57.825)
\curveto(59.38476562,58.16875)(58.98632812,58.340625)(58.50585938,58.340625)
\curveto(58.02929688,58.340625)(57.625,58.16875)(57.29296875,57.825)
\curveto(56.9609375,57.48515625)(56.79492188,57.03789062)(56.79492188,56.48320312)
\closepath
}
}
{
\newrgbcolor{curcolor}{0 0 0}
\pscustom[linestyle=none,fillstyle=solid,fillcolor=curcolor]
{
\newpath
\moveto(67.84570312,54.71367187)
\lineto(67.84570312,53.7)
\lineto(62.16796875,53.7)
\curveto(62.16015625,53.95390625)(62.20117188,54.19804687)(62.29101562,54.43242187)
\curveto(62.43554688,54.81914062)(62.66601562,55.2)(62.98242188,55.575)
\curveto(63.30273438,55.95)(63.76367188,56.38359375)(64.36523438,56.87578125)
\curveto(65.29882812,57.64140625)(65.9296875,58.246875)(66.2578125,58.6921875)
\curveto(66.5859375,59.14140625)(66.75,59.56523437)(66.75,59.96367187)
\curveto(66.75,60.38164062)(66.59960938,60.73320312)(66.29882812,61.01835937)
\curveto(66.00195312,61.30742187)(65.61328125,61.45195312)(65.1328125,61.45195312)
\curveto(64.625,61.45195312)(64.21875,61.29960937)(63.9140625,60.99492187)
\curveto(63.609375,60.69023437)(63.45507812,60.26835937)(63.45117188,59.72929687)
\lineto(62.3671875,59.840625)
\curveto(62.44140625,60.64921875)(62.72070312,61.26445312)(63.20507812,61.68632812)
\curveto(63.68945312,62.11210937)(64.33984375,62.325)(65.15625,62.325)
\curveto(65.98046875,62.325)(66.6328125,62.09648437)(67.11328125,61.63945312)
\curveto(67.59375,61.18242187)(67.83398438,60.61601562)(67.83398438,59.94023437)
\curveto(67.83398438,59.59648437)(67.76367188,59.25859375)(67.62304688,58.9265625)
\curveto(67.48242188,58.59453125)(67.24804688,58.24492187)(66.91992188,57.87773437)
\curveto(66.59570312,57.51054687)(66.0546875,57.00664062)(65.296875,56.36601562)
\curveto(64.6640625,55.83476562)(64.2578125,55.4734375)(64.078125,55.28203125)
\curveto(63.8984375,55.09453125)(63.75,54.90507812)(63.6328125,54.71367187)
\closepath
}
}
{
\newrgbcolor{curcolor}{0 0 0}
\pscustom[linestyle=none,fillstyle=solid,fillcolor=curcolor]
{
\newpath
\moveto(72.94921875,53.7)
\lineto(71.89453125,53.7)
\lineto(71.89453125,60.42070312)
\curveto(71.640625,60.17851562)(71.30664062,59.93632812)(70.89257812,59.69414062)
\curveto(70.48242188,59.45195312)(70.11328125,59.2703125)(69.78515625,59.14921875)
\lineto(69.78515625,60.16875)
\curveto(70.375,60.44609375)(70.890625,60.78203125)(71.33203125,61.1765625)
\curveto(71.7734375,61.57109375)(72.0859375,61.95390625)(72.26953125,62.325)
\lineto(72.94921875,62.325)
\closepath
}
}
{
\newrgbcolor{curcolor}{0 0 0}
\pscustom[linestyle=none,fillstyle=solid,fillcolor=curcolor]
{
\newpath
\moveto(79.03125,53.7)
\lineto(79.03125,55.75664062)
\lineto(75.3046875,55.75664062)
\lineto(75.3046875,56.7234375)
\lineto(79.22460938,62.28984375)
\lineto(80.0859375,62.28984375)
\lineto(80.0859375,56.7234375)
\lineto(81.24609375,56.7234375)
\lineto(81.24609375,55.75664062)
\lineto(80.0859375,55.75664062)
\lineto(80.0859375,53.7)
\closepath
\moveto(79.03125,56.7234375)
\lineto(79.03125,60.59648437)
\lineto(76.34179688,56.7234375)
\closepath
}
}
{
\newrgbcolor{curcolor}{0 0 0}
\pscustom[linestyle=none,fillstyle=solid,fillcolor=curcolor]
{
\newpath
\moveto(85.70507812,53.7)
\lineto(85.70507812,55.75664062)
\lineto(81.97851562,55.75664062)
\lineto(81.97851562,56.7234375)
\lineto(85.8984375,62.28984375)
\lineto(86.75976562,62.28984375)
\lineto(86.75976562,56.7234375)
\lineto(87.91992188,56.7234375)
\lineto(87.91992188,55.75664062)
\lineto(86.75976562,55.75664062)
\lineto(86.75976562,53.7)
\closepath
\moveto(85.70507812,56.7234375)
\lineto(85.70507812,60.59648437)
\lineto(83.015625,56.7234375)
\closepath
}
}
{
\newrgbcolor{curcolor}{0 0 0}
\pscustom[linewidth=1,linecolor=curcolor]
{
\newpath
\moveto(96.8,425.9)
\lineto(105.8,425.9)
\moveto(575,425.9)
\lineto(566,425.9)
}
}
{
\newrgbcolor{curcolor}{0 0 0}
\pscustom[linestyle=none,fillstyle=solid,fillcolor=curcolor]
{
\newpath
\moveto(48.95507812,424.25)
\lineto(50.0625,424.34375)
\curveto(50.14453125,423.8046875)(50.33398438,423.3984375)(50.63085938,423.125)
\curveto(50.93164062,422.85546875)(51.29296875,422.72070312)(51.71484375,422.72070312)
\curveto(52.22265625,422.72070312)(52.65234375,422.91210938)(53.00390625,423.29492188)
\curveto(53.35546875,423.67773438)(53.53125,424.18554688)(53.53125,424.81835938)
\curveto(53.53125,425.41992188)(53.36132812,425.89453125)(53.02148438,426.2421875)
\curveto(52.68554688,426.58984375)(52.24414062,426.76367188)(51.69726562,426.76367188)
\curveto(51.35742188,426.76367188)(51.05078125,426.68554688)(50.77734375,426.52929688)
\curveto(50.50390625,426.37695312)(50.2890625,426.17773438)(50.1328125,425.93164062)
\lineto(49.14257812,426.06054688)
\lineto(49.97460938,430.47265625)
\lineto(54.24609375,430.47265625)
\lineto(54.24609375,429.46484375)
\lineto(50.81835938,429.46484375)
\lineto(50.35546875,427.15625)
\curveto(50.87109375,427.515625)(51.41210938,427.6953125)(51.97851562,427.6953125)
\curveto(52.72851562,427.6953125)(53.36132812,427.43554688)(53.87695312,426.91601562)
\curveto(54.39257812,426.39648438)(54.65039062,425.72851562)(54.65039062,424.91210938)
\curveto(54.65039062,424.13476562)(54.42382812,423.46289062)(53.97070312,422.89648438)
\curveto(53.41992188,422.20117188)(52.66796875,421.85351562)(51.71484375,421.85351562)
\curveto(50.93359375,421.85351562)(50.29492188,422.07226562)(49.79882812,422.50976562)
\curveto(49.30664062,422.94726562)(49.02539062,423.52734375)(48.95507812,424.25)
\closepath
}
}
{
\newrgbcolor{curcolor}{0 0 0}
\pscustom[linestyle=none,fillstyle=solid,fillcolor=curcolor]
{
\newpath
\moveto(61.171875,423.01367188)
\lineto(61.171875,422)
\lineto(55.49414062,422)
\curveto(55.48632812,422.25390625)(55.52734375,422.49804688)(55.6171875,422.73242188)
\curveto(55.76171875,423.11914062)(55.9921875,423.5)(56.30859375,423.875)
\curveto(56.62890625,424.25)(57.08984375,424.68359375)(57.69140625,425.17578125)
\curveto(58.625,425.94140625)(59.25585938,426.546875)(59.58398438,426.9921875)
\curveto(59.91210938,427.44140625)(60.07617188,427.86523438)(60.07617188,428.26367188)
\curveto(60.07617188,428.68164062)(59.92578125,429.03320312)(59.625,429.31835938)
\curveto(59.328125,429.60742188)(58.93945312,429.75195312)(58.45898438,429.75195312)
\curveto(57.95117188,429.75195312)(57.54492188,429.59960938)(57.24023438,429.29492188)
\curveto(56.93554688,428.99023438)(56.78125,428.56835938)(56.77734375,428.02929688)
\lineto(55.69335938,428.140625)
\curveto(55.76757812,428.94921875)(56.046875,429.56445312)(56.53125,429.98632812)
\curveto(57.015625,430.41210938)(57.66601562,430.625)(58.48242188,430.625)
\curveto(59.30664062,430.625)(59.95898438,430.39648438)(60.43945312,429.93945312)
\curveto(60.91992188,429.48242188)(61.16015625,428.91601562)(61.16015625,428.24023438)
\curveto(61.16015625,427.89648438)(61.08984375,427.55859375)(60.94921875,427.2265625)
\curveto(60.80859375,426.89453125)(60.57421875,426.54492188)(60.24609375,426.17773438)
\curveto(59.921875,425.81054688)(59.38085938,425.30664062)(58.62304688,424.66601562)
\curveto(57.99023438,424.13476562)(57.58398438,423.7734375)(57.40429688,423.58203125)
\curveto(57.22460938,423.39453125)(57.07617188,423.20507812)(56.95898438,423.01367188)
\closepath
}
}
{
\newrgbcolor{curcolor}{0 0 0}
\pscustom[linestyle=none,fillstyle=solid,fillcolor=curcolor]
{
\newpath
\moveto(65.68359375,422)
\lineto(65.68359375,424.05664062)
\lineto(61.95703125,424.05664062)
\lineto(61.95703125,425.0234375)
\lineto(65.87695312,430.58984375)
\lineto(66.73828125,430.58984375)
\lineto(66.73828125,425.0234375)
\lineto(67.8984375,425.0234375)
\lineto(67.8984375,424.05664062)
\lineto(66.73828125,424.05664062)
\lineto(66.73828125,422)
\closepath
\moveto(65.68359375,425.0234375)
\lineto(65.68359375,428.89648438)
\lineto(62.99414062,425.0234375)
\closepath
}
}
{
\newrgbcolor{curcolor}{0 0 0}
\pscustom[linestyle=none,fillstyle=solid,fillcolor=curcolor]
{
\newpath
\moveto(74.51953125,423.01367188)
\lineto(74.51953125,422)
\lineto(68.84179688,422)
\curveto(68.83398438,422.25390625)(68.875,422.49804688)(68.96484375,422.73242188)
\curveto(69.109375,423.11914062)(69.33984375,423.5)(69.65625,423.875)
\curveto(69.9765625,424.25)(70.4375,424.68359375)(71.0390625,425.17578125)
\curveto(71.97265625,425.94140625)(72.60351562,426.546875)(72.93164062,426.9921875)
\curveto(73.25976562,427.44140625)(73.42382812,427.86523438)(73.42382812,428.26367188)
\curveto(73.42382812,428.68164062)(73.2734375,429.03320312)(72.97265625,429.31835938)
\curveto(72.67578125,429.60742188)(72.28710938,429.75195312)(71.80664062,429.75195312)
\curveto(71.29882812,429.75195312)(70.89257812,429.59960938)(70.58789062,429.29492188)
\curveto(70.28320312,428.99023438)(70.12890625,428.56835938)(70.125,428.02929688)
\lineto(69.04101562,428.140625)
\curveto(69.11523438,428.94921875)(69.39453125,429.56445312)(69.87890625,429.98632812)
\curveto(70.36328125,430.41210938)(71.01367188,430.625)(71.83007812,430.625)
\curveto(72.65429688,430.625)(73.30664062,430.39648438)(73.78710938,429.93945312)
\curveto(74.26757812,429.48242188)(74.5078125,428.91601562)(74.5078125,428.24023438)
\curveto(74.5078125,427.89648438)(74.4375,427.55859375)(74.296875,427.2265625)
\curveto(74.15625,426.89453125)(73.921875,426.54492188)(73.59375,426.17773438)
\curveto(73.26953125,425.81054688)(72.72851562,425.30664062)(71.97070312,424.66601562)
\curveto(71.33789062,424.13476562)(70.93164062,423.7734375)(70.75195312,423.58203125)
\curveto(70.57226562,423.39453125)(70.42382812,423.20507812)(70.30664062,423.01367188)
\closepath
}
}
{
\newrgbcolor{curcolor}{0 0 0}
\pscustom[linestyle=none,fillstyle=solid,fillcolor=curcolor]
{
\newpath
\moveto(77.2734375,426.65820312)
\curveto(76.8359375,426.81835938)(76.51171875,427.046875)(76.30078125,427.34375)
\curveto(76.08984375,427.640625)(75.984375,427.99609375)(75.984375,428.41015625)
\curveto(75.984375,429.03515625)(76.20898438,429.56054688)(76.65820312,429.98632812)
\curveto(77.10742188,430.41210938)(77.70507812,430.625)(78.45117188,430.625)
\curveto(79.20117188,430.625)(79.8046875,430.40625)(80.26171875,429.96875)
\curveto(80.71875,429.53515625)(80.94726562,429.00585938)(80.94726562,428.38085938)
\curveto(80.94726562,427.98242188)(80.84179688,427.63476562)(80.63085938,427.33789062)
\curveto(80.42382812,427.04492188)(80.10742188,426.81835938)(79.68164062,426.65820312)
\curveto(80.20898438,426.48632812)(80.609375,426.20898438)(80.8828125,425.82617188)
\curveto(81.16015625,425.44335938)(81.29882812,424.98632812)(81.29882812,424.45507812)
\curveto(81.29882812,423.72070312)(81.0390625,423.10351562)(80.51953125,422.60351562)
\curveto(80,422.10351562)(79.31640625,421.85351562)(78.46875,421.85351562)
\curveto(77.62109375,421.85351562)(76.9375,422.10351562)(76.41796875,422.60351562)
\curveto(75.8984375,423.10742188)(75.63867188,423.734375)(75.63867188,424.484375)
\curveto(75.63867188,425.04296875)(75.77929688,425.50976562)(76.06054688,425.88476562)
\curveto(76.34570312,426.26367188)(76.75,426.52148438)(77.2734375,426.65820312)
\closepath
\moveto(77.0625,428.4453125)
\curveto(77.0625,428.0390625)(77.19335938,427.70703125)(77.45507812,427.44921875)
\curveto(77.71679688,427.19140625)(78.05664062,427.0625)(78.47460938,427.0625)
\curveto(78.88085938,427.0625)(79.21289062,427.18945312)(79.47070312,427.44335938)
\curveto(79.73242188,427.70117188)(79.86328125,428.015625)(79.86328125,428.38671875)
\curveto(79.86328125,428.7734375)(79.72851562,429.09765625)(79.45898438,429.359375)
\curveto(79.19335938,429.625)(78.86132812,429.7578125)(78.46289062,429.7578125)
\curveto(78.06054688,429.7578125)(77.7265625,429.62890625)(77.4609375,429.37109375)
\curveto(77.1953125,429.11328125)(77.0625,428.8046875)(77.0625,428.4453125)
\closepath
\moveto(76.72265625,424.47851562)
\curveto(76.72265625,424.17773438)(76.79296875,423.88671875)(76.93359375,423.60546875)
\curveto(77.078125,423.32421875)(77.29101562,423.10546875)(77.57226562,422.94921875)
\curveto(77.85351562,422.796875)(78.15625,422.72070312)(78.48046875,422.72070312)
\curveto(78.984375,422.72070312)(79.40039062,422.8828125)(79.72851562,423.20703125)
\curveto(80.05664062,423.53125)(80.22070312,423.94335938)(80.22070312,424.44335938)
\curveto(80.22070312,424.95117188)(80.05078125,425.37109375)(79.7109375,425.703125)
\curveto(79.375,426.03515625)(78.953125,426.20117188)(78.4453125,426.20117188)
\curveto(77.94921875,426.20117188)(77.53710938,426.03710938)(77.20898438,425.70898438)
\curveto(76.88476562,425.38085938)(76.72265625,424.97070312)(76.72265625,424.47851562)
\closepath
}
}
{
\newrgbcolor{curcolor}{0 0 0}
\pscustom[linestyle=none,fillstyle=solid,fillcolor=curcolor]
{
\newpath
\moveto(83.94726562,426.65820312)
\curveto(83.50976562,426.81835938)(83.18554688,427.046875)(82.97460938,427.34375)
\curveto(82.76367188,427.640625)(82.65820312,427.99609375)(82.65820312,428.41015625)
\curveto(82.65820312,429.03515625)(82.8828125,429.56054688)(83.33203125,429.98632812)
\curveto(83.78125,430.41210938)(84.37890625,430.625)(85.125,430.625)
\curveto(85.875,430.625)(86.47851562,430.40625)(86.93554688,429.96875)
\curveto(87.39257812,429.53515625)(87.62109375,429.00585938)(87.62109375,428.38085938)
\curveto(87.62109375,427.98242188)(87.515625,427.63476562)(87.3046875,427.33789062)
\curveto(87.09765625,427.04492188)(86.78125,426.81835938)(86.35546875,426.65820312)
\curveto(86.8828125,426.48632812)(87.28320312,426.20898438)(87.55664062,425.82617188)
\curveto(87.83398438,425.44335938)(87.97265625,424.98632812)(87.97265625,424.45507812)
\curveto(87.97265625,423.72070312)(87.71289062,423.10351562)(87.19335938,422.60351562)
\curveto(86.67382812,422.10351562)(85.99023438,421.85351562)(85.14257812,421.85351562)
\curveto(84.29492188,421.85351562)(83.61132812,422.10351562)(83.09179688,422.60351562)
\curveto(82.57226562,423.10742188)(82.3125,423.734375)(82.3125,424.484375)
\curveto(82.3125,425.04296875)(82.453125,425.50976562)(82.734375,425.88476562)
\curveto(83.01953125,426.26367188)(83.42382812,426.52148438)(83.94726562,426.65820312)
\closepath
\moveto(83.73632812,428.4453125)
\curveto(83.73632812,428.0390625)(83.8671875,427.70703125)(84.12890625,427.44921875)
\curveto(84.390625,427.19140625)(84.73046875,427.0625)(85.1484375,427.0625)
\curveto(85.5546875,427.0625)(85.88671875,427.18945312)(86.14453125,427.44335938)
\curveto(86.40625,427.70117188)(86.53710938,428.015625)(86.53710938,428.38671875)
\curveto(86.53710938,428.7734375)(86.40234375,429.09765625)(86.1328125,429.359375)
\curveto(85.8671875,429.625)(85.53515625,429.7578125)(85.13671875,429.7578125)
\curveto(84.734375,429.7578125)(84.40039062,429.62890625)(84.13476562,429.37109375)
\curveto(83.86914062,429.11328125)(83.73632812,428.8046875)(83.73632812,428.4453125)
\closepath
\moveto(83.39648438,424.47851562)
\curveto(83.39648438,424.17773438)(83.46679688,423.88671875)(83.60742188,423.60546875)
\curveto(83.75195312,423.32421875)(83.96484375,423.10546875)(84.24609375,422.94921875)
\curveto(84.52734375,422.796875)(84.83007812,422.72070312)(85.15429688,422.72070312)
\curveto(85.65820312,422.72070312)(86.07421875,422.8828125)(86.40234375,423.20703125)
\curveto(86.73046875,423.53125)(86.89453125,423.94335938)(86.89453125,424.44335938)
\curveto(86.89453125,424.95117188)(86.72460938,425.37109375)(86.38476562,425.703125)
\curveto(86.04882812,426.03515625)(85.62695312,426.20117188)(85.11914062,426.20117188)
\curveto(84.62304688,426.20117188)(84.2109375,426.03710938)(83.8828125,425.70898438)
\curveto(83.55859375,425.38085938)(83.39648438,424.97070312)(83.39648438,424.47851562)
\closepath
}
}
{
\newrgbcolor{curcolor}{0 0 0}
\pscustom[linewidth=1,linecolor=curcolor]
{
\newpath
\moveto(96.8,57.6)
\lineto(96.8,66.6)
\moveto(96.8,425.9)
\lineto(96.8,416.9)
}
}
{
\newrgbcolor{curcolor}{0 0 0}
\pscustom[linestyle=none,fillstyle=solid,fillcolor=curcolor]
{
\newpath
\moveto(93.96113281,39.93632812)
\curveto(93.96113281,40.95195312)(94.06464844,41.76835937)(94.27167969,42.38554687)
\curveto(94.48261719,43.00664062)(94.79316406,43.48515625)(95.20332031,43.82109375)
\curveto(95.61738281,44.15703125)(96.13691406,44.325)(96.76191406,44.325)
\curveto(97.22285156,44.325)(97.62714844,44.23125)(97.97480469,44.04375)
\curveto(98.32246094,43.86015625)(98.60957031,43.59257812)(98.83613281,43.24101562)
\curveto(99.06269531,42.89335937)(99.24042969,42.46757812)(99.36933594,41.96367187)
\curveto(99.49824219,41.46367187)(99.56269531,40.78789062)(99.56269531,39.93632812)
\curveto(99.56269531,38.92851562)(99.45917969,38.1140625)(99.25214844,37.49296875)
\curveto(99.04511719,36.87578125)(98.73457031,36.39726562)(98.32050781,36.05742187)
\curveto(97.91035156,35.72148437)(97.39082031,35.55351562)(96.76191406,35.55351562)
\curveto(95.93378906,35.55351562)(95.28339844,35.85039062)(94.81074219,36.44414062)
\curveto(94.24433594,37.15898437)(93.96113281,38.32304687)(93.96113281,39.93632812)
\closepath
\moveto(95.04511719,39.93632812)
\curveto(95.04511719,38.52617187)(95.20917969,37.58671875)(95.53730469,37.11796875)
\curveto(95.86933594,36.653125)(96.27753906,36.42070312)(96.76191406,36.42070312)
\curveto(97.24628906,36.42070312)(97.65253906,36.65507812)(97.98066406,37.12382812)
\curveto(98.31269531,37.59257812)(98.47871094,38.53007812)(98.47871094,39.93632812)
\curveto(98.47871094,41.35039062)(98.31269531,42.28984375)(97.98066406,42.7546875)
\curveto(97.65253906,43.21953125)(97.24238281,43.45195312)(96.75019531,43.45195312)
\curveto(96.26582031,43.45195312)(95.87910156,43.246875)(95.59003906,42.83671875)
\curveto(95.22675781,42.31328125)(95.04511719,41.34648437)(95.04511719,39.93632812)
\closepath
}
}
{
\newrgbcolor{curcolor}{0 0 0}
\pscustom[linewidth=1,linecolor=curcolor]
{
\newpath
\moveto(144.6,57.6)
\lineto(144.6,66.6)
\moveto(144.6,425.9)
\lineto(144.6,416.9)
}
}
{
\newrgbcolor{curcolor}{0 0 0}
\pscustom[linestyle=none,fillstyle=solid,fillcolor=curcolor]
{
\newpath
\moveto(142.396875,35.7)
\lineto(141.3421875,35.7)
\lineto(141.3421875,42.42070312)
\curveto(141.08828125,42.17851562)(140.75429687,41.93632812)(140.34023437,41.69414062)
\curveto(139.93007812,41.45195312)(139.5609375,41.2703125)(139.2328125,41.14921875)
\lineto(139.2328125,42.16875)
\curveto(139.82265625,42.44609375)(140.33828125,42.78203125)(140.7796875,43.1765625)
\curveto(141.22109375,43.57109375)(141.53359375,43.95390625)(141.7171875,44.325)
\lineto(142.396875,44.325)
\closepath
}
}
{
\newrgbcolor{curcolor}{0 0 0}
\pscustom[linestyle=none,fillstyle=solid,fillcolor=curcolor]
{
\newpath
\moveto(145.09804687,39.93632812)
\curveto(145.09804687,40.95195312)(145.2015625,41.76835937)(145.40859375,42.38554687)
\curveto(145.61953125,43.00664062)(145.93007812,43.48515625)(146.34023437,43.82109375)
\curveto(146.75429687,44.15703125)(147.27382812,44.325)(147.89882812,44.325)
\curveto(148.35976562,44.325)(148.7640625,44.23125)(149.11171875,44.04375)
\curveto(149.459375,43.86015625)(149.74648437,43.59257812)(149.97304687,43.24101562)
\curveto(150.19960937,42.89335937)(150.37734375,42.46757812)(150.50625,41.96367187)
\curveto(150.63515625,41.46367187)(150.69960937,40.78789062)(150.69960937,39.93632812)
\curveto(150.69960937,38.92851562)(150.59609375,38.1140625)(150.3890625,37.49296875)
\curveto(150.18203125,36.87578125)(149.87148437,36.39726562)(149.45742187,36.05742187)
\curveto(149.04726562,35.72148437)(148.52773437,35.55351562)(147.89882812,35.55351562)
\curveto(147.07070312,35.55351562)(146.4203125,35.85039062)(145.94765625,36.44414062)
\curveto(145.38125,37.15898437)(145.09804687,38.32304687)(145.09804687,39.93632812)
\closepath
\moveto(146.18203125,39.93632812)
\curveto(146.18203125,38.52617187)(146.34609375,37.58671875)(146.67421875,37.11796875)
\curveto(147.00625,36.653125)(147.41445312,36.42070312)(147.89882812,36.42070312)
\curveto(148.38320312,36.42070312)(148.78945312,36.65507812)(149.11757812,37.12382812)
\curveto(149.44960937,37.59257812)(149.615625,38.53007812)(149.615625,39.93632812)
\curveto(149.615625,41.35039062)(149.44960937,42.28984375)(149.11757812,42.7546875)
\curveto(148.78945312,43.21953125)(148.37929687,43.45195312)(147.88710937,43.45195312)
\curveto(147.40273437,43.45195312)(147.01601562,43.246875)(146.72695312,42.83671875)
\curveto(146.36367187,42.31328125)(146.18203125,41.34648437)(146.18203125,39.93632812)
\closepath
}
}
{
\newrgbcolor{curcolor}{0 0 0}
\pscustom[linewidth=1,linecolor=curcolor]
{
\newpath
\moveto(192.4,57.6)
\lineto(192.4,66.6)
\moveto(192.4,425.9)
\lineto(192.4,416.9)
}
}
{
\newrgbcolor{curcolor}{0 0 0}
\pscustom[linestyle=none,fillstyle=solid,fillcolor=curcolor]
{
\newpath
\moveto(191.7671875,36.71367187)
\lineto(191.7671875,35.7)
\lineto(186.08945313,35.7)
\curveto(186.08164063,35.95390625)(186.12265625,36.19804687)(186.2125,36.43242187)
\curveto(186.35703125,36.81914062)(186.5875,37.2)(186.90390625,37.575)
\curveto(187.22421875,37.95)(187.68515625,38.38359375)(188.28671875,38.87578125)
\curveto(189.2203125,39.64140625)(189.85117188,40.246875)(190.17929688,40.6921875)
\curveto(190.50742188,41.14140625)(190.67148438,41.56523437)(190.67148438,41.96367187)
\curveto(190.67148438,42.38164062)(190.52109375,42.73320312)(190.2203125,43.01835937)
\curveto(189.9234375,43.30742187)(189.53476563,43.45195312)(189.05429688,43.45195312)
\curveto(188.54648438,43.45195312)(188.14023438,43.29960937)(187.83554688,42.99492187)
\curveto(187.53085938,42.69023437)(187.3765625,42.26835937)(187.37265625,41.72929687)
\lineto(186.28867188,41.840625)
\curveto(186.36289063,42.64921875)(186.6421875,43.26445312)(187.1265625,43.68632812)
\curveto(187.6109375,44.11210937)(188.26132813,44.325)(189.07773438,44.325)
\curveto(189.90195313,44.325)(190.55429688,44.09648437)(191.03476563,43.63945312)
\curveto(191.51523438,43.18242187)(191.75546875,42.61601562)(191.75546875,41.94023437)
\curveto(191.75546875,41.59648437)(191.68515625,41.25859375)(191.54453125,40.9265625)
\curveto(191.40390625,40.59453125)(191.16953125,40.24492187)(190.84140625,39.87773437)
\curveto(190.5171875,39.51054687)(189.97617188,39.00664062)(189.21835938,38.36601562)
\curveto(188.58554688,37.83476562)(188.17929688,37.4734375)(187.99960938,37.28203125)
\curveto(187.81992188,37.09453125)(187.67148438,36.90507812)(187.55429688,36.71367187)
\closepath
}
}
{
\newrgbcolor{curcolor}{0 0 0}
\pscustom[linestyle=none,fillstyle=solid,fillcolor=curcolor]
{
\newpath
\moveto(192.89804688,39.93632812)
\curveto(192.89804688,40.95195312)(193.0015625,41.76835937)(193.20859375,42.38554687)
\curveto(193.41953125,43.00664062)(193.73007813,43.48515625)(194.14023438,43.82109375)
\curveto(194.55429688,44.15703125)(195.07382813,44.325)(195.69882813,44.325)
\curveto(196.15976563,44.325)(196.5640625,44.23125)(196.91171875,44.04375)
\curveto(197.259375,43.86015625)(197.54648438,43.59257812)(197.77304688,43.24101562)
\curveto(197.99960938,42.89335937)(198.17734375,42.46757812)(198.30625,41.96367187)
\curveto(198.43515625,41.46367187)(198.49960938,40.78789062)(198.49960938,39.93632812)
\curveto(198.49960938,38.92851562)(198.39609375,38.1140625)(198.1890625,37.49296875)
\curveto(197.98203125,36.87578125)(197.67148438,36.39726562)(197.25742188,36.05742187)
\curveto(196.84726563,35.72148437)(196.32773438,35.55351562)(195.69882813,35.55351562)
\curveto(194.87070313,35.55351562)(194.2203125,35.85039062)(193.74765625,36.44414062)
\curveto(193.18125,37.15898437)(192.89804688,38.32304687)(192.89804688,39.93632812)
\closepath
\moveto(193.98203125,39.93632812)
\curveto(193.98203125,38.52617187)(194.14609375,37.58671875)(194.47421875,37.11796875)
\curveto(194.80625,36.653125)(195.21445313,36.42070312)(195.69882813,36.42070312)
\curveto(196.18320313,36.42070312)(196.58945313,36.65507812)(196.91757813,37.12382812)
\curveto(197.24960938,37.59257812)(197.415625,38.53007812)(197.415625,39.93632812)
\curveto(197.415625,41.35039062)(197.24960938,42.28984375)(196.91757813,42.7546875)
\curveto(196.58945313,43.21953125)(196.17929688,43.45195312)(195.68710938,43.45195312)
\curveto(195.20273438,43.45195312)(194.81601563,43.246875)(194.52695313,42.83671875)
\curveto(194.16367188,42.31328125)(193.98203125,41.34648437)(193.98203125,39.93632812)
\closepath
}
}
{
\newrgbcolor{curcolor}{0 0 0}
\pscustom[linewidth=1,linecolor=curcolor]
{
\newpath
\moveto(240.3,57.6)
\lineto(240.3,66.6)
\moveto(240.3,425.9)
\lineto(240.3,416.9)
}
}
{
\newrgbcolor{curcolor}{0 0 0}
\pscustom[linestyle=none,fillstyle=solid,fillcolor=curcolor]
{
\newpath
\moveto(234.13007813,37.96757812)
\lineto(235.18476563,38.10820312)
\curveto(235.30585938,37.51054687)(235.5109375,37.07890625)(235.8,36.81328125)
\curveto(236.09296875,36.5515625)(236.4484375,36.42070312)(236.86640625,36.42070312)
\curveto(237.3625,36.42070312)(237.78046875,36.59257812)(238.1203125,36.93632812)
\curveto(238.4640625,37.28007812)(238.6359375,37.70585937)(238.6359375,38.21367187)
\curveto(238.6359375,38.69804687)(238.47773438,39.09648437)(238.16132813,39.40898437)
\curveto(237.84492188,39.72539062)(237.44257813,39.88359375)(236.95429688,39.88359375)
\curveto(236.75507813,39.88359375)(236.50703125,39.84453125)(236.21015625,39.76640625)
\lineto(236.32734375,40.6921875)
\curveto(236.39765625,40.684375)(236.45429688,40.68046875)(236.49726563,40.68046875)
\curveto(236.94648438,40.68046875)(237.35078125,40.79765625)(237.71015625,41.03203125)
\curveto(238.06953125,41.26640625)(238.24921875,41.62773437)(238.24921875,42.11601562)
\curveto(238.24921875,42.50273437)(238.11835938,42.82304687)(237.85664063,43.07695312)
\curveto(237.59492188,43.33085937)(237.25703125,43.4578125)(236.84296875,43.4578125)
\curveto(236.4328125,43.4578125)(236.09101563,43.32890625)(235.81757813,43.07109375)
\curveto(235.54414063,42.81328125)(235.36835938,42.4265625)(235.29023438,41.9109375)
\lineto(234.23554688,42.0984375)
\curveto(234.36445313,42.80546875)(234.65742188,43.35234375)(235.11445313,43.7390625)
\curveto(235.57148438,44.1296875)(236.13984375,44.325)(236.81953125,44.325)
\curveto(237.28828125,44.325)(237.71992188,44.2234375)(238.11445313,44.0203125)
\curveto(238.50898438,43.82109375)(238.80976563,43.54765625)(239.01679688,43.2)
\curveto(239.22773438,42.85234375)(239.33320313,42.48320312)(239.33320313,42.09257812)
\curveto(239.33320313,41.72148437)(239.23359375,41.38359375)(239.034375,41.07890625)
\curveto(238.83515625,40.77421875)(238.54023438,40.53203125)(238.14960938,40.35234375)
\curveto(238.65742188,40.23515625)(239.05195313,39.99101562)(239.33320313,39.61992187)
\curveto(239.61445313,39.25273437)(239.75507813,38.79179687)(239.75507813,38.23710937)
\curveto(239.75507813,37.48710937)(239.48164063,36.85039062)(238.93476563,36.32695312)
\curveto(238.38789063,35.80742187)(237.69648438,35.54765625)(236.86054688,35.54765625)
\curveto(236.10664063,35.54765625)(235.4796875,35.77226562)(234.9796875,36.22148437)
\curveto(234.48359375,36.67070312)(234.20039063,37.25273437)(234.13007813,37.96757812)
\closepath
}
}
{
\newrgbcolor{curcolor}{0 0 0}
\pscustom[linestyle=none,fillstyle=solid,fillcolor=curcolor]
{
\newpath
\moveto(240.79804688,39.93632812)
\curveto(240.79804688,40.95195312)(240.9015625,41.76835937)(241.10859375,42.38554687)
\curveto(241.31953125,43.00664062)(241.63007813,43.48515625)(242.04023438,43.82109375)
\curveto(242.45429688,44.15703125)(242.97382813,44.325)(243.59882813,44.325)
\curveto(244.05976563,44.325)(244.4640625,44.23125)(244.81171875,44.04375)
\curveto(245.159375,43.86015625)(245.44648438,43.59257812)(245.67304688,43.24101562)
\curveto(245.89960938,42.89335937)(246.07734375,42.46757812)(246.20625,41.96367187)
\curveto(246.33515625,41.46367187)(246.39960938,40.78789062)(246.39960938,39.93632812)
\curveto(246.39960938,38.92851562)(246.29609375,38.1140625)(246.0890625,37.49296875)
\curveto(245.88203125,36.87578125)(245.57148438,36.39726562)(245.15742188,36.05742187)
\curveto(244.74726563,35.72148437)(244.22773438,35.55351562)(243.59882813,35.55351562)
\curveto(242.77070313,35.55351562)(242.1203125,35.85039062)(241.64765625,36.44414062)
\curveto(241.08125,37.15898437)(240.79804688,38.32304687)(240.79804688,39.93632812)
\closepath
\moveto(241.88203125,39.93632812)
\curveto(241.88203125,38.52617187)(242.04609375,37.58671875)(242.37421875,37.11796875)
\curveto(242.70625,36.653125)(243.11445313,36.42070312)(243.59882813,36.42070312)
\curveto(244.08320313,36.42070312)(244.48945313,36.65507812)(244.81757813,37.12382812)
\curveto(245.14960938,37.59257812)(245.315625,38.53007812)(245.315625,39.93632812)
\curveto(245.315625,41.35039062)(245.14960938,42.28984375)(244.81757813,42.7546875)
\curveto(244.48945313,43.21953125)(244.07929688,43.45195312)(243.58710938,43.45195312)
\curveto(243.10273438,43.45195312)(242.71601563,43.246875)(242.42695313,42.83671875)
\curveto(242.06367188,42.31328125)(241.88203125,41.34648437)(241.88203125,39.93632812)
\closepath
}
}
{
\newrgbcolor{curcolor}{0 0 0}
\pscustom[linewidth=1,linecolor=curcolor]
{
\newpath
\moveto(288.1,57.6)
\lineto(288.1,66.6)
\moveto(288.1,425.9)
\lineto(288.1,416.9)
}
}
{
\newrgbcolor{curcolor}{0 0 0}
\pscustom[linestyle=none,fillstyle=solid,fillcolor=curcolor]
{
\newpath
\moveto(285.30507813,35.7)
\lineto(285.30507813,37.75664062)
\lineto(281.57851563,37.75664062)
\lineto(281.57851563,38.7234375)
\lineto(285.4984375,44.28984375)
\lineto(286.35976563,44.28984375)
\lineto(286.35976563,38.7234375)
\lineto(287.51992188,38.7234375)
\lineto(287.51992188,37.75664062)
\lineto(286.35976563,37.75664062)
\lineto(286.35976563,35.7)
\closepath
\moveto(285.30507813,38.7234375)
\lineto(285.30507813,42.59648437)
\lineto(282.615625,38.7234375)
\closepath
}
}
{
\newrgbcolor{curcolor}{0 0 0}
\pscustom[linestyle=none,fillstyle=solid,fillcolor=curcolor]
{
\newpath
\moveto(288.59804688,39.93632812)
\curveto(288.59804688,40.95195312)(288.7015625,41.76835937)(288.90859375,42.38554687)
\curveto(289.11953125,43.00664062)(289.43007813,43.48515625)(289.84023438,43.82109375)
\curveto(290.25429688,44.15703125)(290.77382813,44.325)(291.39882813,44.325)
\curveto(291.85976563,44.325)(292.2640625,44.23125)(292.61171875,44.04375)
\curveto(292.959375,43.86015625)(293.24648438,43.59257812)(293.47304688,43.24101562)
\curveto(293.69960938,42.89335937)(293.87734375,42.46757812)(294.00625,41.96367187)
\curveto(294.13515625,41.46367187)(294.19960938,40.78789062)(294.19960938,39.93632812)
\curveto(294.19960938,38.92851562)(294.09609375,38.1140625)(293.8890625,37.49296875)
\curveto(293.68203125,36.87578125)(293.37148438,36.39726562)(292.95742188,36.05742187)
\curveto(292.54726563,35.72148437)(292.02773438,35.55351562)(291.39882813,35.55351562)
\curveto(290.57070313,35.55351562)(289.9203125,35.85039062)(289.44765625,36.44414062)
\curveto(288.88125,37.15898437)(288.59804688,38.32304687)(288.59804688,39.93632812)
\closepath
\moveto(289.68203125,39.93632812)
\curveto(289.68203125,38.52617187)(289.84609375,37.58671875)(290.17421875,37.11796875)
\curveto(290.50625,36.653125)(290.91445313,36.42070312)(291.39882813,36.42070312)
\curveto(291.88320313,36.42070312)(292.28945313,36.65507812)(292.61757813,37.12382812)
\curveto(292.94960938,37.59257812)(293.115625,38.53007812)(293.115625,39.93632812)
\curveto(293.115625,41.35039062)(292.94960938,42.28984375)(292.61757813,42.7546875)
\curveto(292.28945313,43.21953125)(291.87929688,43.45195312)(291.38710938,43.45195312)
\curveto(290.90273438,43.45195312)(290.51601563,43.246875)(290.22695313,42.83671875)
\curveto(289.86367188,42.31328125)(289.68203125,41.34648437)(289.68203125,39.93632812)
\closepath
}
}
{
\newrgbcolor{curcolor}{0 0 0}
\pscustom[linewidth=1,linecolor=curcolor]
{
\newpath
\moveto(335.9,57.6)
\lineto(335.9,66.6)
\moveto(335.9,425.9)
\lineto(335.9,416.9)
}
}
{
\newrgbcolor{curcolor}{0 0 0}
\pscustom[linestyle=none,fillstyle=solid,fillcolor=curcolor]
{
\newpath
\moveto(329.72421875,37.95)
\lineto(330.83164062,38.04375)
\curveto(330.91367187,37.5046875)(331.103125,37.0984375)(331.4,36.825)
\curveto(331.70078125,36.55546875)(332.06210937,36.42070312)(332.48398437,36.42070312)
\curveto(332.99179687,36.42070312)(333.42148437,36.61210937)(333.77304687,36.99492187)
\curveto(334.12460937,37.37773437)(334.30039062,37.88554687)(334.30039062,38.51835937)
\curveto(334.30039062,39.11992187)(334.13046875,39.59453125)(333.790625,39.9421875)
\curveto(333.4546875,40.28984375)(333.01328125,40.46367187)(332.46640625,40.46367187)
\curveto(332.1265625,40.46367187)(331.81992187,40.38554687)(331.54648437,40.22929687)
\curveto(331.27304687,40.07695312)(331.05820312,39.87773437)(330.90195312,39.63164062)
\lineto(329.91171875,39.76054687)
\lineto(330.74375,44.17265625)
\lineto(335.01523437,44.17265625)
\lineto(335.01523437,43.16484375)
\lineto(331.5875,43.16484375)
\lineto(331.12460937,40.85625)
\curveto(331.64023437,41.215625)(332.18125,41.3953125)(332.74765625,41.3953125)
\curveto(333.49765625,41.3953125)(334.13046875,41.13554687)(334.64609375,40.61601562)
\curveto(335.16171875,40.09648437)(335.41953125,39.42851562)(335.41953125,38.61210937)
\curveto(335.41953125,37.83476562)(335.19296875,37.16289062)(334.73984375,36.59648437)
\curveto(334.1890625,35.90117187)(333.43710937,35.55351562)(332.48398437,35.55351562)
\curveto(331.70273437,35.55351562)(331.0640625,35.77226562)(330.56796875,36.20976562)
\curveto(330.07578125,36.64726562)(329.79453125,37.22734375)(329.72421875,37.95)
\closepath
}
}
{
\newrgbcolor{curcolor}{0 0 0}
\pscustom[linestyle=none,fillstyle=solid,fillcolor=curcolor]
{
\newpath
\moveto(336.39804687,39.93632812)
\curveto(336.39804687,40.95195312)(336.5015625,41.76835937)(336.70859375,42.38554687)
\curveto(336.91953125,43.00664062)(337.23007812,43.48515625)(337.64023437,43.82109375)
\curveto(338.05429687,44.15703125)(338.57382812,44.325)(339.19882812,44.325)
\curveto(339.65976562,44.325)(340.0640625,44.23125)(340.41171875,44.04375)
\curveto(340.759375,43.86015625)(341.04648437,43.59257812)(341.27304687,43.24101562)
\curveto(341.49960937,42.89335937)(341.67734375,42.46757812)(341.80625,41.96367187)
\curveto(341.93515625,41.46367187)(341.99960937,40.78789062)(341.99960937,39.93632812)
\curveto(341.99960937,38.92851562)(341.89609375,38.1140625)(341.6890625,37.49296875)
\curveto(341.48203125,36.87578125)(341.17148437,36.39726562)(340.75742187,36.05742187)
\curveto(340.34726562,35.72148437)(339.82773437,35.55351562)(339.19882812,35.55351562)
\curveto(338.37070312,35.55351562)(337.7203125,35.85039062)(337.24765625,36.44414062)
\curveto(336.68125,37.15898437)(336.39804687,38.32304687)(336.39804687,39.93632812)
\closepath
\moveto(337.48203125,39.93632812)
\curveto(337.48203125,38.52617187)(337.64609375,37.58671875)(337.97421875,37.11796875)
\curveto(338.30625,36.653125)(338.71445312,36.42070312)(339.19882812,36.42070312)
\curveto(339.68320312,36.42070312)(340.08945312,36.65507812)(340.41757812,37.12382812)
\curveto(340.74960937,37.59257812)(340.915625,38.53007812)(340.915625,39.93632812)
\curveto(340.915625,41.35039062)(340.74960937,42.28984375)(340.41757812,42.7546875)
\curveto(340.08945312,43.21953125)(339.67929687,43.45195312)(339.18710937,43.45195312)
\curveto(338.70273437,43.45195312)(338.31601562,43.246875)(338.02695312,42.83671875)
\curveto(337.66367187,42.31328125)(337.48203125,41.34648437)(337.48203125,39.93632812)
\closepath
}
}
{
\newrgbcolor{curcolor}{0 0 0}
\pscustom[linewidth=1,linecolor=curcolor]
{
\newpath
\moveto(383.7,57.6)
\lineto(383.7,66.6)
\moveto(383.7,425.9)
\lineto(383.7,416.9)
}
}
{
\newrgbcolor{curcolor}{0 0 0}
\pscustom[linestyle=none,fillstyle=solid,fillcolor=curcolor]
{
\newpath
\moveto(382.996875,42.18632812)
\lineto(381.94804687,42.10429687)
\curveto(381.85429687,42.51835937)(381.72148437,42.81914062)(381.54960937,43.00664062)
\curveto(381.26445312,43.30742187)(380.91289062,43.4578125)(380.49492187,43.4578125)
\curveto(380.15898437,43.4578125)(379.8640625,43.3640625)(379.61015625,43.1765625)
\curveto(379.278125,42.934375)(379.01640625,42.58085937)(378.825,42.11601562)
\curveto(378.63359375,41.65117187)(378.53398437,40.9890625)(378.52617187,40.1296875)
\curveto(378.78007812,40.51640625)(379.090625,40.80351562)(379.4578125,40.99101562)
\curveto(379.825,41.17851562)(380.20976562,41.27226562)(380.61210937,41.27226562)
\curveto(381.31523437,41.27226562)(381.91289062,41.0125)(382.40507812,40.49296875)
\curveto(382.90117187,39.97734375)(383.14921875,39.309375)(383.14921875,38.4890625)
\curveto(383.14921875,37.95)(383.03203125,37.44804687)(382.79765625,36.98320312)
\curveto(382.5671875,36.52226562)(382.24882812,36.16875)(381.84257812,35.92265625)
\curveto(381.43632812,35.6765625)(380.97539062,35.55351562)(380.45976562,35.55351562)
\curveto(379.58085937,35.55351562)(378.8640625,35.87578125)(378.309375,36.5203125)
\curveto(377.7546875,37.16875)(377.47734375,38.23515625)(377.47734375,39.71953125)
\curveto(377.47734375,41.3796875)(377.78398437,42.58671875)(378.39726562,43.340625)
\curveto(378.93242187,43.996875)(379.653125,44.325)(380.559375,44.325)
\curveto(381.23515625,44.325)(381.78789062,44.13554687)(382.21757812,43.75664062)
\curveto(382.65117187,43.37773437)(382.9109375,42.85429687)(382.996875,42.18632812)
\closepath
\moveto(378.69023437,38.48320312)
\curveto(378.69023437,38.11992187)(378.76640625,37.77226562)(378.91875,37.44023437)
\curveto(379.075,37.10820312)(379.29179687,36.85429687)(379.56914062,36.67851562)
\curveto(379.84648437,36.50664062)(380.1375,36.42070312)(380.4421875,36.42070312)
\curveto(380.8875,36.42070312)(381.2703125,36.60039062)(381.590625,36.95976562)
\curveto(381.9109375,37.31914062)(382.07109375,37.80742187)(382.07109375,38.42460937)
\curveto(382.07109375,39.01835937)(381.91289062,39.48515625)(381.59648437,39.825)
\curveto(381.28007812,40.16875)(380.88164062,40.340625)(380.40117187,40.340625)
\curveto(379.92460937,40.340625)(379.5203125,40.16875)(379.18828125,39.825)
\curveto(378.85625,39.48515625)(378.69023437,39.03789062)(378.69023437,38.48320312)
\closepath
}
}
{
\newrgbcolor{curcolor}{0 0 0}
\pscustom[linestyle=none,fillstyle=solid,fillcolor=curcolor]
{
\newpath
\moveto(384.19804687,39.93632812)
\curveto(384.19804687,40.95195312)(384.3015625,41.76835937)(384.50859375,42.38554687)
\curveto(384.71953125,43.00664062)(385.03007812,43.48515625)(385.44023437,43.82109375)
\curveto(385.85429687,44.15703125)(386.37382812,44.325)(386.99882812,44.325)
\curveto(387.45976562,44.325)(387.8640625,44.23125)(388.21171875,44.04375)
\curveto(388.559375,43.86015625)(388.84648437,43.59257812)(389.07304687,43.24101562)
\curveto(389.29960937,42.89335937)(389.47734375,42.46757812)(389.60625,41.96367187)
\curveto(389.73515625,41.46367187)(389.79960937,40.78789062)(389.79960937,39.93632812)
\curveto(389.79960937,38.92851562)(389.69609375,38.1140625)(389.4890625,37.49296875)
\curveto(389.28203125,36.87578125)(388.97148437,36.39726562)(388.55742187,36.05742187)
\curveto(388.14726562,35.72148437)(387.62773437,35.55351562)(386.99882812,35.55351562)
\curveto(386.17070312,35.55351562)(385.5203125,35.85039062)(385.04765625,36.44414062)
\curveto(384.48125,37.15898437)(384.19804687,38.32304687)(384.19804687,39.93632812)
\closepath
\moveto(385.28203125,39.93632812)
\curveto(385.28203125,38.52617187)(385.44609375,37.58671875)(385.77421875,37.11796875)
\curveto(386.10625,36.653125)(386.51445312,36.42070312)(386.99882812,36.42070312)
\curveto(387.48320312,36.42070312)(387.88945312,36.65507812)(388.21757812,37.12382812)
\curveto(388.54960937,37.59257812)(388.715625,38.53007812)(388.715625,39.93632812)
\curveto(388.715625,41.35039062)(388.54960937,42.28984375)(388.21757812,42.7546875)
\curveto(387.88945312,43.21953125)(387.47929687,43.45195312)(386.98710937,43.45195312)
\curveto(386.50273437,43.45195312)(386.11601562,43.246875)(385.82695312,42.83671875)
\curveto(385.46367187,42.31328125)(385.28203125,41.34648437)(385.28203125,39.93632812)
\closepath
}
}
{
\newrgbcolor{curcolor}{0 0 0}
\pscustom[linewidth=1,linecolor=curcolor]
{
\newpath
\moveto(431.5,57.6)
\lineto(431.5,66.6)
\moveto(431.5,425.9)
\lineto(431.5,416.9)
}
}
{
\newrgbcolor{curcolor}{0 0 0}
\pscustom[linestyle=none,fillstyle=solid,fillcolor=curcolor]
{
\newpath
\moveto(425.39453125,43.16484375)
\lineto(425.39453125,44.17851562)
\lineto(430.95507812,44.17851562)
\lineto(430.95507812,43.35820312)
\curveto(430.40820312,42.77617187)(429.86523438,42.00273437)(429.32617188,41.03789062)
\curveto(428.79101562,40.07304687)(428.37695312,39.08085937)(428.08398438,38.06132812)
\curveto(427.87304688,37.34257812)(427.73828125,36.55546875)(427.6796875,35.7)
\lineto(426.59570312,35.7)
\curveto(426.60742188,36.37578125)(426.74023438,37.1921875)(426.99414062,38.14921875)
\curveto(427.24804688,39.10625)(427.61132812,40.028125)(428.08398438,40.91484375)
\curveto(428.56054688,41.80546875)(429.06640625,42.55546875)(429.6015625,43.16484375)
\closepath
}
}
{
\newrgbcolor{curcolor}{0 0 0}
\pscustom[linestyle=none,fillstyle=solid,fillcolor=curcolor]
{
\newpath
\moveto(431.99804688,39.93632812)
\curveto(431.99804688,40.95195312)(432.1015625,41.76835937)(432.30859375,42.38554687)
\curveto(432.51953125,43.00664062)(432.83007812,43.48515625)(433.24023438,43.82109375)
\curveto(433.65429688,44.15703125)(434.17382812,44.325)(434.79882812,44.325)
\curveto(435.25976562,44.325)(435.6640625,44.23125)(436.01171875,44.04375)
\curveto(436.359375,43.86015625)(436.64648438,43.59257812)(436.87304688,43.24101562)
\curveto(437.09960938,42.89335937)(437.27734375,42.46757812)(437.40625,41.96367187)
\curveto(437.53515625,41.46367187)(437.59960938,40.78789062)(437.59960938,39.93632812)
\curveto(437.59960938,38.92851562)(437.49609375,38.1140625)(437.2890625,37.49296875)
\curveto(437.08203125,36.87578125)(436.77148438,36.39726562)(436.35742188,36.05742187)
\curveto(435.94726562,35.72148437)(435.42773438,35.55351562)(434.79882812,35.55351562)
\curveto(433.97070312,35.55351562)(433.3203125,35.85039062)(432.84765625,36.44414062)
\curveto(432.28125,37.15898437)(431.99804688,38.32304687)(431.99804688,39.93632812)
\closepath
\moveto(433.08203125,39.93632812)
\curveto(433.08203125,38.52617187)(433.24609375,37.58671875)(433.57421875,37.11796875)
\curveto(433.90625,36.653125)(434.31445312,36.42070312)(434.79882812,36.42070312)
\curveto(435.28320312,36.42070312)(435.68945312,36.65507812)(436.01757812,37.12382812)
\curveto(436.34960938,37.59257812)(436.515625,38.53007812)(436.515625,39.93632812)
\curveto(436.515625,41.35039062)(436.34960938,42.28984375)(436.01757812,42.7546875)
\curveto(435.68945312,43.21953125)(435.27929688,43.45195312)(434.78710938,43.45195312)
\curveto(434.30273438,43.45195312)(433.91601562,43.246875)(433.62695312,42.83671875)
\curveto(433.26367188,42.31328125)(433.08203125,41.34648437)(433.08203125,39.93632812)
\closepath
}
}
{
\newrgbcolor{curcolor}{0 0 0}
\pscustom[linewidth=1,linecolor=curcolor]
{
\newpath
\moveto(479.4,57.6)
\lineto(479.4,66.6)
\moveto(479.4,425.9)
\lineto(479.4,416.9)
}
}
{
\newrgbcolor{curcolor}{0 0 0}
\pscustom[linestyle=none,fillstyle=solid,fillcolor=curcolor]
{
\newpath
\moveto(474.84726562,40.35820312)
\curveto(474.40976562,40.51835937)(474.08554687,40.746875)(473.87460937,41.04375)
\curveto(473.66367187,41.340625)(473.55820312,41.69609375)(473.55820312,42.11015625)
\curveto(473.55820312,42.73515625)(473.7828125,43.26054687)(474.23203125,43.68632812)
\curveto(474.68125,44.11210937)(475.27890625,44.325)(476.025,44.325)
\curveto(476.775,44.325)(477.37851562,44.10625)(477.83554687,43.66875)
\curveto(478.29257812,43.23515625)(478.52109375,42.70585937)(478.52109375,42.08085937)
\curveto(478.52109375,41.68242187)(478.415625,41.33476562)(478.2046875,41.03789062)
\curveto(477.99765625,40.74492187)(477.68125,40.51835937)(477.25546875,40.35820312)
\curveto(477.7828125,40.18632812)(478.18320312,39.90898437)(478.45664062,39.52617187)
\curveto(478.73398437,39.14335937)(478.87265625,38.68632812)(478.87265625,38.15507812)
\curveto(478.87265625,37.42070312)(478.61289062,36.80351562)(478.09335937,36.30351562)
\curveto(477.57382812,35.80351562)(476.89023437,35.55351562)(476.04257812,35.55351562)
\curveto(475.19492187,35.55351562)(474.51132812,35.80351562)(473.99179687,36.30351562)
\curveto(473.47226562,36.80742187)(473.2125,37.434375)(473.2125,38.184375)
\curveto(473.2125,38.74296875)(473.353125,39.20976562)(473.634375,39.58476562)
\curveto(473.91953125,39.96367187)(474.32382812,40.22148437)(474.84726562,40.35820312)
\closepath
\moveto(474.63632812,42.1453125)
\curveto(474.63632812,41.7390625)(474.7671875,41.40703125)(475.02890625,41.14921875)
\curveto(475.290625,40.89140625)(475.63046875,40.7625)(476.0484375,40.7625)
\curveto(476.4546875,40.7625)(476.78671875,40.88945312)(477.04453125,41.14335937)
\curveto(477.30625,41.40117187)(477.43710937,41.715625)(477.43710937,42.08671875)
\curveto(477.43710937,42.4734375)(477.30234375,42.79765625)(477.0328125,43.059375)
\curveto(476.7671875,43.325)(476.43515625,43.4578125)(476.03671875,43.4578125)
\curveto(475.634375,43.4578125)(475.30039062,43.32890625)(475.03476562,43.07109375)
\curveto(474.76914062,42.81328125)(474.63632812,42.5046875)(474.63632812,42.1453125)
\closepath
\moveto(474.29648437,38.17851562)
\curveto(474.29648437,37.87773437)(474.36679687,37.58671875)(474.50742187,37.30546875)
\curveto(474.65195312,37.02421875)(474.86484375,36.80546875)(475.14609375,36.64921875)
\curveto(475.42734375,36.496875)(475.73007812,36.42070312)(476.05429687,36.42070312)
\curveto(476.55820312,36.42070312)(476.97421875,36.5828125)(477.30234375,36.90703125)
\curveto(477.63046875,37.23125)(477.79453125,37.64335937)(477.79453125,38.14335937)
\curveto(477.79453125,38.65117187)(477.62460937,39.07109375)(477.28476562,39.403125)
\curveto(476.94882812,39.73515625)(476.52695312,39.90117187)(476.01914062,39.90117187)
\curveto(475.52304687,39.90117187)(475.1109375,39.73710937)(474.7828125,39.40898437)
\curveto(474.45859375,39.08085937)(474.29648437,38.67070312)(474.29648437,38.17851562)
\closepath
}
}
{
\newrgbcolor{curcolor}{0 0 0}
\pscustom[linestyle=none,fillstyle=solid,fillcolor=curcolor]
{
\newpath
\moveto(479.89804687,39.93632812)
\curveto(479.89804687,40.95195312)(480.0015625,41.76835937)(480.20859375,42.38554687)
\curveto(480.41953125,43.00664062)(480.73007812,43.48515625)(481.14023437,43.82109375)
\curveto(481.55429687,44.15703125)(482.07382812,44.325)(482.69882812,44.325)
\curveto(483.15976562,44.325)(483.5640625,44.23125)(483.91171875,44.04375)
\curveto(484.259375,43.86015625)(484.54648437,43.59257812)(484.77304687,43.24101562)
\curveto(484.99960937,42.89335937)(485.17734375,42.46757812)(485.30625,41.96367187)
\curveto(485.43515625,41.46367187)(485.49960937,40.78789062)(485.49960937,39.93632812)
\curveto(485.49960937,38.92851562)(485.39609375,38.1140625)(485.1890625,37.49296875)
\curveto(484.98203125,36.87578125)(484.67148437,36.39726562)(484.25742187,36.05742187)
\curveto(483.84726562,35.72148437)(483.32773437,35.55351562)(482.69882812,35.55351562)
\curveto(481.87070312,35.55351562)(481.2203125,35.85039062)(480.74765625,36.44414062)
\curveto(480.18125,37.15898437)(479.89804687,38.32304687)(479.89804687,39.93632812)
\closepath
\moveto(480.98203125,39.93632812)
\curveto(480.98203125,38.52617187)(481.14609375,37.58671875)(481.47421875,37.11796875)
\curveto(481.80625,36.653125)(482.21445312,36.42070312)(482.69882812,36.42070312)
\curveto(483.18320312,36.42070312)(483.58945312,36.65507812)(483.91757812,37.12382812)
\curveto(484.24960937,37.59257812)(484.415625,38.53007812)(484.415625,39.93632812)
\curveto(484.415625,41.35039062)(484.24960937,42.28984375)(483.91757812,42.7546875)
\curveto(483.58945312,43.21953125)(483.17929687,43.45195312)(482.68710937,43.45195312)
\curveto(482.20273437,43.45195312)(481.81601562,43.246875)(481.52695312,42.83671875)
\curveto(481.16367187,42.31328125)(480.98203125,41.34648437)(480.98203125,39.93632812)
\closepath
}
}
{
\newrgbcolor{curcolor}{0 0 0}
\pscustom[linewidth=1,linecolor=curcolor]
{
\newpath
\moveto(527.2,57.6)
\lineto(527.2,66.6)
\moveto(527.2,425.9)
\lineto(527.2,416.9)
}
}
{
\newrgbcolor{curcolor}{0 0 0}
\pscustom[linestyle=none,fillstyle=solid,fillcolor=curcolor]
{
\newpath
\moveto(521.18242188,37.68632812)
\lineto(522.19609375,37.78007812)
\curveto(522.28203125,37.30351562)(522.44609375,36.9578125)(522.68828125,36.74296875)
\curveto(522.93046875,36.528125)(523.24101563,36.42070312)(523.61992188,36.42070312)
\curveto(523.94414063,36.42070312)(524.22734375,36.49492187)(524.46953125,36.64335937)
\curveto(524.715625,36.79179687)(524.91679688,36.9890625)(525.07304688,37.23515625)
\curveto(525.22929688,37.48515625)(525.36015625,37.82109375)(525.465625,38.24296875)
\curveto(525.57109375,38.66484375)(525.62382813,39.09453125)(525.62382813,39.53203125)
\curveto(525.62382813,39.57890625)(525.621875,39.64921875)(525.61796875,39.74296875)
\curveto(525.40703125,39.40703125)(525.11796875,39.13359375)(524.75078125,38.92265625)
\curveto(524.3875,38.715625)(523.99296875,38.61210937)(523.5671875,38.61210937)
\curveto(522.85625,38.61210937)(522.2546875,38.86992187)(521.7625,39.38554687)
\curveto(521.2703125,39.90117187)(521.02421875,40.58085937)(521.02421875,41.42460937)
\curveto(521.02421875,42.29570312)(521.28007813,42.996875)(521.79179688,43.528125)
\curveto(522.30742188,44.059375)(522.95195313,44.325)(523.72539063,44.325)
\curveto(524.28398438,44.325)(524.79375,44.17460937)(525.2546875,43.87382812)
\curveto(525.71953125,43.57304687)(526.07109375,43.14335937)(526.309375,42.58476562)
\curveto(526.5515625,42.03007812)(526.67265625,41.22539062)(526.67265625,40.17070312)
\curveto(526.67265625,39.07304687)(526.55351563,38.19804687)(526.31523438,37.54570312)
\curveto(526.07695313,36.89726562)(525.72148438,36.403125)(525.24882813,36.06328125)
\curveto(524.78007813,35.7234375)(524.22929688,35.55351562)(523.59648438,35.55351562)
\curveto(522.92460938,35.55351562)(522.37578125,35.7390625)(521.95,36.11015625)
\curveto(521.52421875,36.48515625)(521.26835938,37.01054687)(521.18242188,37.68632812)
\closepath
\moveto(525.50078125,41.47734375)
\curveto(525.50078125,42.0828125)(525.33867188,42.56328125)(525.01445313,42.91875)
\curveto(524.69414063,43.27421875)(524.30742188,43.45195312)(523.85429688,43.45195312)
\curveto(523.38554688,43.45195312)(522.97734375,43.26054687)(522.6296875,42.87773437)
\curveto(522.28203125,42.49492187)(522.10820313,41.99882812)(522.10820313,41.38945312)
\curveto(522.10820313,40.84257812)(522.27226563,40.39726562)(522.60039063,40.05351562)
\curveto(522.93242188,39.71367187)(523.340625,39.54375)(523.825,39.54375)
\curveto(524.31328125,39.54375)(524.71367188,39.71367187)(525.02617188,40.05351562)
\curveto(525.34257813,40.39726562)(525.50078125,40.871875)(525.50078125,41.47734375)
\closepath
}
}
{
\newrgbcolor{curcolor}{0 0 0}
\pscustom[linestyle=none,fillstyle=solid,fillcolor=curcolor]
{
\newpath
\moveto(527.69804688,39.93632812)
\curveto(527.69804688,40.95195312)(527.8015625,41.76835937)(528.00859375,42.38554687)
\curveto(528.21953125,43.00664062)(528.53007813,43.48515625)(528.94023438,43.82109375)
\curveto(529.35429688,44.15703125)(529.87382813,44.325)(530.49882813,44.325)
\curveto(530.95976563,44.325)(531.3640625,44.23125)(531.71171875,44.04375)
\curveto(532.059375,43.86015625)(532.34648438,43.59257812)(532.57304688,43.24101562)
\curveto(532.79960938,42.89335937)(532.97734375,42.46757812)(533.10625,41.96367187)
\curveto(533.23515625,41.46367187)(533.29960938,40.78789062)(533.29960938,39.93632812)
\curveto(533.29960938,38.92851562)(533.19609375,38.1140625)(532.9890625,37.49296875)
\curveto(532.78203125,36.87578125)(532.47148438,36.39726562)(532.05742188,36.05742187)
\curveto(531.64726563,35.72148437)(531.12773438,35.55351562)(530.49882813,35.55351562)
\curveto(529.67070313,35.55351562)(529.0203125,35.85039062)(528.54765625,36.44414062)
\curveto(527.98125,37.15898437)(527.69804688,38.32304687)(527.69804688,39.93632812)
\closepath
\moveto(528.78203125,39.93632812)
\curveto(528.78203125,38.52617187)(528.94609375,37.58671875)(529.27421875,37.11796875)
\curveto(529.60625,36.653125)(530.01445313,36.42070312)(530.49882813,36.42070312)
\curveto(530.98320313,36.42070312)(531.38945313,36.65507812)(531.71757813,37.12382812)
\curveto(532.04960938,37.59257812)(532.215625,38.53007812)(532.215625,39.93632812)
\curveto(532.215625,41.35039062)(532.04960938,42.28984375)(531.71757813,42.7546875)
\curveto(531.38945313,43.21953125)(530.97929688,43.45195312)(530.48710938,43.45195312)
\curveto(530.00273438,43.45195312)(529.61601563,43.246875)(529.32695313,42.83671875)
\curveto(528.96367188,42.31328125)(528.78203125,41.34648437)(528.78203125,39.93632812)
\closepath
}
}
{
\newrgbcolor{curcolor}{0 0 0}
\pscustom[linewidth=1,linecolor=curcolor]
{
\newpath
\moveto(575,57.6)
\lineto(575,66.6)
\moveto(575,425.9)
\lineto(575,416.9)
}
}
{
\newrgbcolor{curcolor}{0 0 0}
\pscustom[linestyle=none,fillstyle=solid,fillcolor=curcolor]
{
\newpath
\moveto(569.45996094,35.7)
\lineto(568.40527344,35.7)
\lineto(568.40527344,42.42070312)
\curveto(568.15136719,42.17851562)(567.81738281,41.93632812)(567.40332031,41.69414062)
\curveto(566.99316406,41.45195312)(566.62402344,41.2703125)(566.29589844,41.14921875)
\lineto(566.29589844,42.16875)
\curveto(566.88574219,42.44609375)(567.40136719,42.78203125)(567.84277344,43.1765625)
\curveto(568.28417969,43.57109375)(568.59667969,43.95390625)(568.78027344,44.325)
\lineto(569.45996094,44.325)
\closepath
}
}
{
\newrgbcolor{curcolor}{0 0 0}
\pscustom[linestyle=none,fillstyle=solid,fillcolor=curcolor]
{
\newpath
\moveto(572.16113281,39.93632812)
\curveto(572.16113281,40.95195312)(572.26464844,41.76835937)(572.47167969,42.38554687)
\curveto(572.68261719,43.00664062)(572.99316406,43.48515625)(573.40332031,43.82109375)
\curveto(573.81738281,44.15703125)(574.33691406,44.325)(574.96191406,44.325)
\curveto(575.42285156,44.325)(575.82714844,44.23125)(576.17480469,44.04375)
\curveto(576.52246094,43.86015625)(576.80957031,43.59257812)(577.03613281,43.24101562)
\curveto(577.26269531,42.89335937)(577.44042969,42.46757812)(577.56933594,41.96367187)
\curveto(577.69824219,41.46367187)(577.76269531,40.78789062)(577.76269531,39.93632812)
\curveto(577.76269531,38.92851562)(577.65917969,38.1140625)(577.45214844,37.49296875)
\curveto(577.24511719,36.87578125)(576.93457031,36.39726562)(576.52050781,36.05742187)
\curveto(576.11035156,35.72148437)(575.59082031,35.55351562)(574.96191406,35.55351562)
\curveto(574.13378906,35.55351562)(573.48339844,35.85039062)(573.01074219,36.44414062)
\curveto(572.44433594,37.15898437)(572.16113281,38.32304687)(572.16113281,39.93632812)
\closepath
\moveto(573.24511719,39.93632812)
\curveto(573.24511719,38.52617187)(573.40917969,37.58671875)(573.73730469,37.11796875)
\curveto(574.06933594,36.653125)(574.47753906,36.42070312)(574.96191406,36.42070312)
\curveto(575.44628906,36.42070312)(575.85253906,36.65507812)(576.18066406,37.12382812)
\curveto(576.51269531,37.59257812)(576.67871094,38.53007812)(576.67871094,39.93632812)
\curveto(576.67871094,41.35039062)(576.51269531,42.28984375)(576.18066406,42.7546875)
\curveto(575.85253906,43.21953125)(575.44238281,43.45195312)(574.95019531,43.45195312)
\curveto(574.46582031,43.45195312)(574.07910156,43.246875)(573.79003906,42.83671875)
\curveto(573.42675781,42.31328125)(573.24511719,41.34648437)(573.24511719,39.93632812)
\closepath
}
}
{
\newrgbcolor{curcolor}{0 0 0}
\pscustom[linestyle=none,fillstyle=solid,fillcolor=curcolor]
{
\newpath
\moveto(578.83496094,39.93632812)
\curveto(578.83496094,40.95195312)(578.93847656,41.76835937)(579.14550781,42.38554687)
\curveto(579.35644531,43.00664062)(579.66699219,43.48515625)(580.07714844,43.82109375)
\curveto(580.49121094,44.15703125)(581.01074219,44.325)(581.63574219,44.325)
\curveto(582.09667969,44.325)(582.50097656,44.23125)(582.84863281,44.04375)
\curveto(583.19628906,43.86015625)(583.48339844,43.59257812)(583.70996094,43.24101562)
\curveto(583.93652344,42.89335937)(584.11425781,42.46757812)(584.24316406,41.96367187)
\curveto(584.37207031,41.46367187)(584.43652344,40.78789062)(584.43652344,39.93632812)
\curveto(584.43652344,38.92851562)(584.33300781,38.1140625)(584.12597656,37.49296875)
\curveto(583.91894531,36.87578125)(583.60839844,36.39726562)(583.19433594,36.05742187)
\curveto(582.78417969,35.72148437)(582.26464844,35.55351562)(581.63574219,35.55351562)
\curveto(580.80761719,35.55351562)(580.15722656,35.85039062)(579.68457031,36.44414062)
\curveto(579.11816406,37.15898437)(578.83496094,38.32304687)(578.83496094,39.93632812)
\closepath
\moveto(579.91894531,39.93632812)
\curveto(579.91894531,38.52617187)(580.08300781,37.58671875)(580.41113281,37.11796875)
\curveto(580.74316406,36.653125)(581.15136719,36.42070312)(581.63574219,36.42070312)
\curveto(582.12011719,36.42070312)(582.52636719,36.65507812)(582.85449219,37.12382812)
\curveto(583.18652344,37.59257812)(583.35253906,38.53007812)(583.35253906,39.93632812)
\curveto(583.35253906,41.35039062)(583.18652344,42.28984375)(582.85449219,42.7546875)
\curveto(582.52636719,43.21953125)(582.11621094,43.45195312)(581.62402344,43.45195312)
\curveto(581.13964844,43.45195312)(580.75292969,43.246875)(580.46386719,42.83671875)
\curveto(580.10058594,42.31328125)(579.91894531,41.34648437)(579.91894531,39.93632812)
\closepath
}
}
{
\newrgbcolor{curcolor}{0 0 0}
\pscustom[linewidth=1,linecolor=curcolor]
{
\newpath
\moveto(96.8,425.9)
\lineto(96.8,57.6)
\lineto(575,57.6)
\lineto(575,425.9)
\closepath
}
}
{
\newrgbcolor{curcolor}{0 0 0}
\pscustom[linestyle=none,fillstyle=solid,fillcolor=curcolor]
{
\newpath
\moveto(14.05,167.49101562)
\lineto(13.95625,168.5984375)
\curveto(14.4953125,168.68046875)(14.9015625,168.86992187)(15.175,169.16679687)
\curveto(15.44453125,169.46757812)(15.57929688,169.82890625)(15.57929688,170.25078125)
\curveto(15.57929688,170.75859375)(15.38789063,171.18828125)(15.00507813,171.53984375)
\curveto(14.62226563,171.89140625)(14.11445313,172.0671875)(13.48164063,172.0671875)
\curveto(12.88007813,172.0671875)(12.40546875,171.89726562)(12.0578125,171.55742187)
\curveto(11.71015625,171.22148437)(11.53632813,170.78007812)(11.53632813,170.23320312)
\curveto(11.53632813,169.89335937)(11.61445313,169.58671875)(11.77070313,169.31328125)
\curveto(11.92304688,169.03984375)(12.12226563,168.825)(12.36835938,168.66875)
\lineto(12.23945313,167.67851562)
\lineto(7.82734375,168.51054687)
\lineto(7.82734375,172.78203125)
\lineto(8.83515625,172.78203125)
\lineto(8.83515625,169.35429687)
\lineto(11.14375,168.89140625)
\curveto(10.784375,169.40703125)(10.6046875,169.94804687)(10.6046875,170.51445312)
\curveto(10.6046875,171.26445312)(10.86445313,171.89726562)(11.38398438,172.41289062)
\curveto(11.90351563,172.92851562)(12.57148438,173.18632812)(13.38789063,173.18632812)
\curveto(14.16523438,173.18632812)(14.83710938,172.95976562)(15.40351563,172.50664062)
\curveto(16.09882813,171.95585937)(16.44648438,171.20390625)(16.44648438,170.25078125)
\curveto(16.44648438,169.46953125)(16.22773438,168.83085937)(15.79023438,168.33476562)
\curveto(15.35273438,167.84257812)(14.77265625,167.56132812)(14.05,167.49101562)
\closepath
}
}
{
\newrgbcolor{curcolor}{0 0 0}
\pscustom[linestyle=none,fillstyle=solid,fillcolor=curcolor]
{
\newpath
\moveto(12.06367188,174.16484375)
\curveto(11.04804688,174.16484375)(10.23164063,174.26835937)(9.61445313,174.47539062)
\curveto(8.99335938,174.68632812)(8.51484375,174.996875)(8.17890625,175.40703125)
\curveto(7.84296875,175.82109375)(7.675,176.340625)(7.675,176.965625)
\curveto(7.675,177.4265625)(7.76875,177.83085937)(7.95625,178.17851562)
\curveto(8.13984375,178.52617187)(8.40742188,178.81328125)(8.75898438,179.03984375)
\curveto(9.10664063,179.26640625)(9.53242188,179.44414062)(10.03632813,179.57304687)
\curveto(10.53632813,179.70195312)(11.21210938,179.76640625)(12.06367188,179.76640625)
\curveto(13.07148438,179.76640625)(13.8859375,179.66289062)(14.50703125,179.45585937)
\curveto(15.12421875,179.24882812)(15.60273438,178.93828125)(15.94257813,178.52421875)
\curveto(16.27851563,178.1140625)(16.44648438,177.59453125)(16.44648438,176.965625)
\curveto(16.44648438,176.1375)(16.14960938,175.48710937)(15.55585938,175.01445312)
\curveto(14.84101563,174.44804687)(13.67695313,174.16484375)(12.06367188,174.16484375)
\closepath
\moveto(12.06367188,175.24882812)
\curveto(13.47382813,175.24882812)(14.41328125,175.41289062)(14.88203125,175.74101562)
\curveto(15.346875,176.07304687)(15.57929688,176.48125)(15.57929688,176.965625)
\curveto(15.57929688,177.45)(15.34492188,177.85625)(14.87617188,178.184375)
\curveto(14.40742188,178.51640625)(13.46992188,178.68242187)(12.06367188,178.68242187)
\curveto(10.64960938,178.68242187)(9.71015625,178.51640625)(9.2453125,178.184375)
\curveto(8.78046875,177.85625)(8.54804688,177.44609375)(8.54804688,176.95390625)
\curveto(8.54804688,176.46953125)(8.753125,176.0828125)(9.16328125,175.79375)
\curveto(9.68671875,175.43046875)(10.65351563,175.24882812)(12.06367188,175.24882812)
\closepath
}
}
{
\newrgbcolor{curcolor}{0 0 0}
\pscustom[linestyle=none,fillstyle=solid,fillcolor=curcolor]
{
\newpath
\moveto(15.35664063,183.434375)
\lineto(16.28828125,183.58671875)
\curveto(16.35078125,183.28984375)(16.38203125,183.02421875)(16.38203125,182.78984375)
\curveto(16.38203125,182.40703125)(16.32148438,182.11015625)(16.20039063,181.89921875)
\curveto(16.07929688,181.68828125)(15.92109375,181.53984375)(15.72578125,181.45390625)
\curveto(15.5265625,181.36796875)(15.11054688,181.325)(14.47773438,181.325)
\lineto(10.89765625,181.325)
\lineto(10.89765625,180.5515625)
\lineto(10.07734375,180.5515625)
\lineto(10.07734375,181.325)
\lineto(8.53632813,181.325)
\lineto(7.90351563,182.37382812)
\lineto(10.07734375,182.37382812)
\lineto(10.07734375,183.434375)
\lineto(10.89765625,183.434375)
\lineto(10.89765625,182.37382812)
\lineto(14.53632813,182.37382812)
\curveto(14.83710938,182.37382812)(15.03046875,182.39140625)(15.11640625,182.4265625)
\curveto(15.20234375,182.465625)(15.27070313,182.52617187)(15.32148438,182.60820312)
\curveto(15.37226563,182.69414062)(15.39765625,182.81523437)(15.39765625,182.97148437)
\curveto(15.39765625,183.08867187)(15.38398438,183.24296875)(15.35664063,183.434375)
\closepath
}
}
{
\newrgbcolor{curcolor}{0 0 0}
\pscustom[linestyle=none,fillstyle=solid,fillcolor=curcolor]
{
\newpath
\moveto(16.3,184.465625)
\lineto(7.71015625,184.465625)
\lineto(7.71015625,185.5203125)
\lineto(10.7921875,185.5203125)
\curveto(10.221875,186.0125)(9.93671875,186.63359375)(9.93671875,187.38359375)
\curveto(9.93671875,187.84453125)(10.02851563,188.24492187)(10.21210938,188.58476562)
\curveto(10.39179688,188.92460937)(10.64179688,189.16679687)(10.96210938,189.31132812)
\curveto(11.28242188,189.45976562)(11.74726563,189.53398437)(12.35664063,189.53398437)
\lineto(16.3,189.53398437)
\lineto(16.3,188.47929687)
\lineto(12.35664063,188.47929687)
\curveto(11.82929688,188.47929687)(11.44648438,188.3640625)(11.20820313,188.13359375)
\curveto(10.96601563,187.90703125)(10.84492188,187.58476562)(10.84492188,187.16679687)
\curveto(10.84492188,186.85429687)(10.92695313,186.559375)(11.09101563,186.28203125)
\curveto(11.25117188,186.00859375)(11.46992188,185.81328125)(11.74726563,185.69609375)
\curveto(12.02460938,185.57890625)(12.40742188,185.5203125)(12.89570313,185.5203125)
\lineto(16.3,185.5203125)
\closepath
}
}
{
\newrgbcolor{curcolor}{0 0 0}
\pscustom[linestyle=none,fillstyle=solid,fillcolor=curcolor]
{
\newpath
\moveto(16.3,194.60820312)
\lineto(7.71015625,194.60820312)
\lineto(7.71015625,197.8484375)
\curveto(7.71015625,198.41875)(7.7375,198.85429687)(7.7921875,199.15507812)
\curveto(7.8625,199.57695312)(7.99726563,199.93046875)(8.19648438,200.215625)
\curveto(8.39179688,200.50078125)(8.6671875,200.72929687)(9.02265625,200.90117187)
\curveto(9.378125,201.07695312)(9.76875,201.16484375)(10.19453125,201.16484375)
\curveto(10.925,201.16484375)(11.54414063,200.93242187)(12.05195313,200.46757812)
\curveto(12.55585938,200.00273437)(12.8078125,199.16289062)(12.8078125,197.94804687)
\lineto(12.8078125,195.74492187)
\lineto(16.3,195.74492187)
\closepath
\moveto(11.79414063,195.74492187)
\lineto(11.79414063,197.965625)
\curveto(11.79414063,198.7)(11.65742188,199.22148437)(11.38398438,199.53007812)
\curveto(11.11054688,199.83867187)(10.72578125,199.99296875)(10.2296875,199.99296875)
\curveto(9.8703125,199.99296875)(9.56367188,199.90117187)(9.30976563,199.71757812)
\curveto(9.05195313,199.53789062)(8.88203125,199.29960937)(8.8,199.00273437)
\curveto(8.74921875,198.81132812)(8.72382813,198.4578125)(8.72382813,197.9421875)
\lineto(8.72382813,195.74492187)
\closepath
}
}
{
\newrgbcolor{curcolor}{0 0 0}
\pscustom[linestyle=none,fillstyle=solid,fillcolor=curcolor]
{
\newpath
\moveto(14.29609375,206.73710937)
\lineto(14.43085938,207.82695312)
\curveto(15.06757813,207.65507812)(15.56171875,207.33671875)(15.91328125,206.871875)
\curveto(16.26484375,206.40703125)(16.440625,205.81328125)(16.440625,205.090625)
\curveto(16.440625,204.18046875)(16.16132813,203.4578125)(15.60273438,202.92265625)
\curveto(15.04023438,202.39140625)(14.253125,202.12578125)(13.24140625,202.12578125)
\curveto(12.19453125,202.12578125)(11.38203125,202.3953125)(10.80390625,202.934375)
\curveto(10.22578125,203.4734375)(9.93671875,204.17265625)(9.93671875,205.03203125)
\curveto(9.93671875,205.8640625)(10.21992188,206.54375)(10.78632813,207.07109375)
\curveto(11.35273438,207.5984375)(12.14960938,207.86210937)(13.17695313,207.86210937)
\curveto(13.23945313,207.86210937)(13.33320313,207.86015625)(13.45820313,207.85625)
\lineto(13.45820313,203.215625)
\curveto(14.14179688,203.2546875)(14.66523438,203.44804687)(15.02851563,203.79570312)
\curveto(15.39179688,204.14335937)(15.5734375,204.57695312)(15.5734375,205.09648437)
\curveto(15.5734375,205.48320312)(15.471875,205.81328125)(15.26875,206.08671875)
\curveto(15.065625,206.36015625)(14.74140625,206.57695312)(14.29609375,206.73710937)
\closepath
\moveto(12.59101563,203.27421875)
\lineto(12.59101563,206.74882812)
\curveto(12.06757813,206.70195312)(11.675,206.56914062)(11.41328125,206.35039062)
\curveto(11.00703125,206.01445312)(10.80390625,205.57890625)(10.80390625,205.04375)
\curveto(10.80390625,204.559375)(10.96601563,204.15117187)(11.29023438,203.81914062)
\curveto(11.61445313,203.49101562)(12.04804688,203.309375)(12.59101563,203.27421875)
\closepath
}
}
{
\newrgbcolor{curcolor}{0 0 0}
\pscustom[linestyle=none,fillstyle=solid,fillcolor=curcolor]
{
\newpath
\moveto(16.3,209.13945312)
\lineto(10.07734375,209.13945312)
\lineto(10.07734375,210.08867187)
\lineto(11.02070313,210.08867187)
\curveto(10.57929688,210.33085937)(10.28828125,210.55351562)(10.14765625,210.75664062)
\curveto(10.00703125,210.96367187)(9.93671875,211.19023437)(9.93671875,211.43632812)
\curveto(9.93671875,211.79179687)(10.05,212.153125)(10.2765625,212.5203125)
\lineto(11.25507813,212.15703125)
\curveto(11.10273438,211.89921875)(11.0265625,211.64140625)(11.0265625,211.38359375)
\curveto(11.0265625,211.153125)(11.096875,210.94609375)(11.2375,210.7625)
\curveto(11.37421875,210.57890625)(11.565625,210.44804687)(11.81171875,210.36992187)
\curveto(12.18671875,210.25273437)(12.596875,210.19414062)(13.0421875,210.19414062)
\lineto(16.3,210.19414062)
\closepath
}
}
{
\newrgbcolor{curcolor}{0 0 0}
\pscustom[linestyle=none,fillstyle=solid,fillcolor=curcolor]
{
\newpath
\moveto(14.02070313,217.2078125)
\lineto(14.15546875,218.24492187)
\curveto(14.8703125,218.13164062)(15.43085938,217.840625)(15.83710938,217.371875)
\curveto(16.23945313,216.90703125)(16.440625,216.33476562)(16.440625,215.65507812)
\curveto(16.440625,214.80351562)(16.16328125,214.11796875)(15.60859375,213.5984375)
\curveto(15.05,213.0828125)(14.25117188,212.825)(13.21210938,212.825)
\curveto(12.54023438,212.825)(11.95234375,212.93632812)(11.4484375,213.15898437)
\curveto(10.94453125,213.38164062)(10.56757813,213.71953125)(10.31757813,214.17265625)
\curveto(10.06367188,214.6296875)(9.93671875,215.12578125)(9.93671875,215.6609375)
\curveto(9.93671875,216.33671875)(10.10859375,216.88945312)(10.45234375,217.31914062)
\curveto(10.7921875,217.74882812)(11.2765625,218.02421875)(11.90546875,218.1453125)
\lineto(12.06367188,217.11992187)
\curveto(11.64570313,217.02226562)(11.33125,216.8484375)(11.1203125,216.5984375)
\curveto(10.909375,216.35234375)(10.80390625,216.05351562)(10.80390625,215.70195312)
\curveto(10.80390625,215.17070312)(10.9953125,214.7390625)(11.378125,214.40703125)
\curveto(11.75703125,214.075)(12.35859375,213.90898437)(13.1828125,213.90898437)
\curveto(14.01875,213.90898437)(14.62617188,214.06914062)(15.00507813,214.38945312)
\curveto(15.38398438,214.70976562)(15.5734375,215.12773437)(15.5734375,215.64335937)
\curveto(15.5734375,216.05742187)(15.44648438,216.403125)(15.19257813,216.68046875)
\curveto(14.93867188,216.9578125)(14.54804688,217.13359375)(14.02070313,217.2078125)
\closepath
}
}
{
\newrgbcolor{curcolor}{0 0 0}
\pscustom[linestyle=none,fillstyle=solid,fillcolor=curcolor]
{
\newpath
\moveto(14.29609375,223.40703125)
\lineto(14.43085938,224.496875)
\curveto(15.06757813,224.325)(15.56171875,224.00664062)(15.91328125,223.54179687)
\curveto(16.26484375,223.07695312)(16.440625,222.48320312)(16.440625,221.76054687)
\curveto(16.440625,220.85039062)(16.16132813,220.12773437)(15.60273438,219.59257812)
\curveto(15.04023438,219.06132812)(14.253125,218.79570312)(13.24140625,218.79570312)
\curveto(12.19453125,218.79570312)(11.38203125,219.06523437)(10.80390625,219.60429687)
\curveto(10.22578125,220.14335937)(9.93671875,220.84257812)(9.93671875,221.70195312)
\curveto(9.93671875,222.53398437)(10.21992188,223.21367187)(10.78632813,223.74101562)
\curveto(11.35273438,224.26835937)(12.14960938,224.53203125)(13.17695313,224.53203125)
\curveto(13.23945313,224.53203125)(13.33320313,224.53007812)(13.45820313,224.52617187)
\lineto(13.45820313,219.88554687)
\curveto(14.14179688,219.92460937)(14.66523438,220.11796875)(15.02851563,220.465625)
\curveto(15.39179688,220.81328125)(15.5734375,221.246875)(15.5734375,221.76640625)
\curveto(15.5734375,222.153125)(15.471875,222.48320312)(15.26875,222.75664062)
\curveto(15.065625,223.03007812)(14.74140625,223.246875)(14.29609375,223.40703125)
\closepath
\moveto(12.59101563,219.94414062)
\lineto(12.59101563,223.41875)
\curveto(12.06757813,223.371875)(11.675,223.2390625)(11.41328125,223.0203125)
\curveto(11.00703125,222.684375)(10.80390625,222.24882812)(10.80390625,221.71367187)
\curveto(10.80390625,221.22929687)(10.96601563,220.82109375)(11.29023438,220.4890625)
\curveto(11.61445313,220.1609375)(12.04804688,219.97929687)(12.59101563,219.94414062)
\closepath
}
}
{
\newrgbcolor{curcolor}{0 0 0}
\pscustom[linestyle=none,fillstyle=solid,fillcolor=curcolor]
{
\newpath
\moveto(16.3,225.82109375)
\lineto(10.07734375,225.82109375)
\lineto(10.07734375,226.7703125)
\lineto(10.96210938,226.7703125)
\curveto(10.27851563,227.22734375)(9.93671875,227.8875)(9.93671875,228.75078125)
\curveto(9.93671875,229.12578125)(10.00507813,229.46953125)(10.14179688,229.78203125)
\curveto(10.27460938,230.0984375)(10.45039063,230.33476562)(10.66914063,230.49101562)
\curveto(10.88789063,230.64726562)(11.14765625,230.75664062)(11.4484375,230.81914062)
\curveto(11.64375,230.85820312)(11.98554688,230.87773437)(12.47382813,230.87773437)
\lineto(16.3,230.87773437)
\lineto(16.3,229.82304687)
\lineto(12.51484375,229.82304687)
\curveto(12.08515625,229.82304687)(11.76484375,229.78203125)(11.55390625,229.7)
\curveto(11.3390625,229.61796875)(11.16914063,229.47148437)(11.04414063,229.26054687)
\curveto(10.91523438,229.05351562)(10.85078125,228.809375)(10.85078125,228.528125)
\curveto(10.85078125,228.07890625)(10.99335938,227.69023437)(11.27851563,227.36210937)
\curveto(11.56367188,227.03789062)(12.1046875,226.87578125)(12.9015625,226.87578125)
\lineto(16.3,226.87578125)
\closepath
}
}
{
\newrgbcolor{curcolor}{0 0 0}
\pscustom[linestyle=none,fillstyle=solid,fillcolor=curcolor]
{
\newpath
\moveto(15.35664063,234.79765625)
\lineto(16.28828125,234.95)
\curveto(16.35078125,234.653125)(16.38203125,234.3875)(16.38203125,234.153125)
\curveto(16.38203125,233.7703125)(16.32148438,233.4734375)(16.20039063,233.2625)
\curveto(16.07929688,233.0515625)(15.92109375,232.903125)(15.72578125,232.8171875)
\curveto(15.5265625,232.73125)(15.11054688,232.68828125)(14.47773438,232.68828125)
\lineto(10.89765625,232.68828125)
\lineto(10.89765625,231.91484375)
\lineto(10.07734375,231.91484375)
\lineto(10.07734375,232.68828125)
\lineto(8.53632813,232.68828125)
\lineto(7.90351563,233.73710937)
\lineto(10.07734375,233.73710937)
\lineto(10.07734375,234.79765625)
\lineto(10.89765625,234.79765625)
\lineto(10.89765625,233.73710937)
\lineto(14.53632813,233.73710937)
\curveto(14.83710938,233.73710937)(15.03046875,233.7546875)(15.11640625,233.78984375)
\curveto(15.20234375,233.82890625)(15.27070313,233.88945312)(15.32148438,233.97148437)
\curveto(15.37226563,234.05742187)(15.39765625,234.17851562)(15.39765625,234.33476562)
\curveto(15.39765625,234.45195312)(15.38398438,234.60625)(15.35664063,234.79765625)
\closepath
}
}
{
\newrgbcolor{curcolor}{0 0 0}
\pscustom[linestyle=none,fillstyle=solid,fillcolor=curcolor]
{
\newpath
\moveto(8.92304688,235.83476562)
\lineto(7.71015625,235.83476562)
\lineto(7.71015625,236.88945312)
\lineto(8.92304688,236.88945312)
\closepath
\moveto(16.3,235.83476562)
\lineto(10.07734375,235.83476562)
\lineto(10.07734375,236.88945312)
\lineto(16.3,236.88945312)
\closepath
}
}
{
\newrgbcolor{curcolor}{0 0 0}
\pscustom[linestyle=none,fillstyle=solid,fillcolor=curcolor]
{
\newpath
\moveto(16.3,238.47148437)
\lineto(7.71015625,238.47148437)
\lineto(7.71015625,239.52617187)
\lineto(16.3,239.52617187)
\closepath
}
}
{
\newrgbcolor{curcolor}{0 0 0}
\pscustom[linestyle=none,fillstyle=solid,fillcolor=curcolor]
{
\newpath
\moveto(14.29609375,245.42070312)
\lineto(14.43085938,246.51054687)
\curveto(15.06757813,246.33867187)(15.56171875,246.0203125)(15.91328125,245.55546875)
\curveto(16.26484375,245.090625)(16.440625,244.496875)(16.440625,243.77421875)
\curveto(16.440625,242.8640625)(16.16132813,242.14140625)(15.60273438,241.60625)
\curveto(15.04023438,241.075)(14.253125,240.809375)(13.24140625,240.809375)
\curveto(12.19453125,240.809375)(11.38203125,241.07890625)(10.80390625,241.61796875)
\curveto(10.22578125,242.15703125)(9.93671875,242.85625)(9.93671875,243.715625)
\curveto(9.93671875,244.54765625)(10.21992188,245.22734375)(10.78632813,245.7546875)
\curveto(11.35273438,246.28203125)(12.14960938,246.54570312)(13.17695313,246.54570312)
\curveto(13.23945313,246.54570312)(13.33320313,246.54375)(13.45820313,246.53984375)
\lineto(13.45820313,241.89921875)
\curveto(14.14179688,241.93828125)(14.66523438,242.13164062)(15.02851563,242.47929687)
\curveto(15.39179688,242.82695312)(15.5734375,243.26054687)(15.5734375,243.78007812)
\curveto(15.5734375,244.16679687)(15.471875,244.496875)(15.26875,244.7703125)
\curveto(15.065625,245.04375)(14.74140625,245.26054687)(14.29609375,245.42070312)
\closepath
\moveto(12.59101563,241.9578125)
\lineto(12.59101563,245.43242187)
\curveto(12.06757813,245.38554687)(11.675,245.25273437)(11.41328125,245.03398437)
\curveto(11.00703125,244.69804687)(10.80390625,244.2625)(10.80390625,243.72734375)
\curveto(10.80390625,243.24296875)(10.96601563,242.83476562)(11.29023438,242.50273437)
\curveto(11.61445313,242.17460937)(12.04804688,241.99296875)(12.59101563,241.9578125)
\closepath
}
}
{
\newrgbcolor{curcolor}{0 0 0}
\pscustom[linestyle=none,fillstyle=solid,fillcolor=curcolor]
{
\newpath
\moveto(16.3,251.25664062)
\lineto(7.71015625,251.25664062)
\lineto(7.71015625,252.39335937)
\lineto(15.28632813,252.39335937)
\lineto(15.28632813,256.62382812)
\lineto(16.3,256.62382812)
\closepath
}
}
{
\newrgbcolor{curcolor}{0 0 0}
\pscustom[linestyle=none,fillstyle=solid,fillcolor=curcolor]
{
\newpath
\moveto(15.53242188,261.903125)
\curveto(15.86445313,261.5125)(16.09882813,261.13554687)(16.23554688,260.77226562)
\curveto(16.37226563,260.41289062)(16.440625,260.02617187)(16.440625,259.61210937)
\curveto(16.440625,258.92851562)(16.27460938,258.403125)(15.94257813,258.0359375)
\curveto(15.60664063,257.66875)(15.17890625,257.48515625)(14.659375,257.48515625)
\curveto(14.3546875,257.48515625)(14.07734375,257.55351562)(13.82734375,257.69023437)
\curveto(13.5734375,257.83085937)(13.3703125,258.0125)(13.21796875,258.23515625)
\curveto(13.065625,258.46171875)(12.95039063,258.715625)(12.87226563,258.996875)
\curveto(12.81757813,259.20390625)(12.76484375,259.51640625)(12.7140625,259.934375)
\curveto(12.6125,260.7859375)(12.49140625,261.41289062)(12.35078125,261.81523437)
\curveto(12.20625,261.81914062)(12.11445313,261.82109375)(12.07539063,261.82109375)
\curveto(11.64570313,261.82109375)(11.34296875,261.72148437)(11.1671875,261.52226562)
\curveto(10.92890625,261.25273437)(10.80976563,260.85234375)(10.80976563,260.32109375)
\curveto(10.80976563,259.825)(10.89765625,259.4578125)(11.0734375,259.21953125)
\curveto(11.2453125,258.98515625)(11.55195313,258.81132812)(11.99335938,258.69804687)
\lineto(11.85273438,257.66679687)
\curveto(11.41132813,257.76054687)(11.05585938,257.91484375)(10.78632813,258.1296875)
\curveto(10.51289063,258.34453125)(10.30390625,258.65507812)(10.159375,259.06132812)
\curveto(10.0109375,259.46757812)(9.93671875,259.93828125)(9.93671875,260.4734375)
\curveto(9.93671875,261.0046875)(9.99921875,261.43632812)(10.12421875,261.76835937)
\curveto(10.24921875,262.10039062)(10.40742188,262.34453125)(10.59882813,262.50078125)
\curveto(10.78632813,262.65703125)(11.02460938,262.76640625)(11.31367188,262.82890625)
\curveto(11.49335938,262.8640625)(11.81757813,262.88164062)(12.28632813,262.88164062)
\lineto(13.69257813,262.88164062)
\curveto(14.67304688,262.88164062)(15.29414063,262.903125)(15.55585938,262.94609375)
\curveto(15.81367188,262.99296875)(16.06171875,263.0828125)(16.3,263.215625)
\lineto(16.3,262.1140625)
\curveto(16.08125,262.0046875)(15.82539063,261.934375)(15.53242188,261.903125)
\closepath
\moveto(13.17695313,261.81523437)
\curveto(13.33320313,261.43242187)(13.46601563,260.85820312)(13.57539063,260.09257812)
\curveto(13.63789063,259.65898437)(13.70820313,259.35234375)(13.78632813,259.17265625)
\curveto(13.86445313,258.99296875)(13.9796875,258.85429687)(14.13203125,258.75664062)
\curveto(14.28046875,258.65898437)(14.44648438,258.61015625)(14.63007813,258.61015625)
\curveto(14.91132813,258.61015625)(15.14570313,258.715625)(15.33320313,258.9265625)
\curveto(15.52070313,259.14140625)(15.61445313,259.45390625)(15.61445313,259.8640625)
\curveto(15.61445313,260.2703125)(15.5265625,260.63164062)(15.35078125,260.94804687)
\curveto(15.17109375,261.26445312)(14.92695313,261.496875)(14.61835938,261.6453125)
\curveto(14.38007813,261.75859375)(14.02851563,261.81523437)(13.56367188,261.81523437)
\closepath
}
}
{
\newrgbcolor{curcolor}{0 0 0}
\pscustom[linestyle=none,fillstyle=solid,fillcolor=curcolor]
{
\newpath
\moveto(15.35664063,266.81914062)
\lineto(16.28828125,266.97148437)
\curveto(16.35078125,266.67460937)(16.38203125,266.40898437)(16.38203125,266.17460937)
\curveto(16.38203125,265.79179687)(16.32148438,265.49492187)(16.20039063,265.28398437)
\curveto(16.07929688,265.07304687)(15.92109375,264.92460937)(15.72578125,264.83867187)
\curveto(15.5265625,264.75273437)(15.11054688,264.70976562)(14.47773438,264.70976562)
\lineto(10.89765625,264.70976562)
\lineto(10.89765625,263.93632812)
\lineto(10.07734375,263.93632812)
\lineto(10.07734375,264.70976562)
\lineto(8.53632813,264.70976562)
\lineto(7.90351563,265.75859375)
\lineto(10.07734375,265.75859375)
\lineto(10.07734375,266.81914062)
\lineto(10.89765625,266.81914062)
\lineto(10.89765625,265.75859375)
\lineto(14.53632813,265.75859375)
\curveto(14.83710938,265.75859375)(15.03046875,265.77617187)(15.11640625,265.81132812)
\curveto(15.20234375,265.85039062)(15.27070313,265.9109375)(15.32148438,265.99296875)
\curveto(15.37226563,266.07890625)(15.39765625,266.2)(15.39765625,266.35625)
\curveto(15.39765625,266.4734375)(15.38398438,266.62773437)(15.35664063,266.81914062)
\closepath
}
}
{
\newrgbcolor{curcolor}{0 0 0}
\pscustom[linestyle=none,fillstyle=solid,fillcolor=curcolor]
{
\newpath
\moveto(14.29609375,272.11015625)
\lineto(14.43085938,273.2)
\curveto(15.06757813,273.028125)(15.56171875,272.70976562)(15.91328125,272.24492187)
\curveto(16.26484375,271.78007812)(16.440625,271.18632812)(16.440625,270.46367187)
\curveto(16.440625,269.55351562)(16.16132813,268.83085937)(15.60273438,268.29570312)
\curveto(15.04023438,267.76445312)(14.253125,267.49882812)(13.24140625,267.49882812)
\curveto(12.19453125,267.49882812)(11.38203125,267.76835937)(10.80390625,268.30742187)
\curveto(10.22578125,268.84648437)(9.93671875,269.54570312)(9.93671875,270.40507812)
\curveto(9.93671875,271.23710937)(10.21992188,271.91679687)(10.78632813,272.44414062)
\curveto(11.35273438,272.97148437)(12.14960938,273.23515625)(13.17695313,273.23515625)
\curveto(13.23945313,273.23515625)(13.33320313,273.23320312)(13.45820313,273.22929687)
\lineto(13.45820313,268.58867187)
\curveto(14.14179688,268.62773437)(14.66523438,268.82109375)(15.02851563,269.16875)
\curveto(15.39179688,269.51640625)(15.5734375,269.95)(15.5734375,270.46953125)
\curveto(15.5734375,270.85625)(15.471875,271.18632812)(15.26875,271.45976562)
\curveto(15.065625,271.73320312)(14.74140625,271.95)(14.29609375,272.11015625)
\closepath
\moveto(12.59101563,268.64726562)
\lineto(12.59101563,272.121875)
\curveto(12.06757813,272.075)(11.675,271.9421875)(11.41328125,271.7234375)
\curveto(11.00703125,271.3875)(10.80390625,270.95195312)(10.80390625,270.41679687)
\curveto(10.80390625,269.93242187)(10.96601563,269.52421875)(11.29023438,269.1921875)
\curveto(11.61445313,268.8640625)(12.04804688,268.68242187)(12.59101563,268.64726562)
\closepath
}
}
{
\newrgbcolor{curcolor}{0 0 0}
\pscustom[linestyle=none,fillstyle=solid,fillcolor=curcolor]
{
\newpath
\moveto(16.3,274.52421875)
\lineto(10.07734375,274.52421875)
\lineto(10.07734375,275.4734375)
\lineto(10.96210938,275.4734375)
\curveto(10.27851563,275.93046875)(9.93671875,276.590625)(9.93671875,277.45390625)
\curveto(9.93671875,277.82890625)(10.00507813,278.17265625)(10.14179688,278.48515625)
\curveto(10.27460938,278.8015625)(10.45039063,279.03789062)(10.66914063,279.19414062)
\curveto(10.88789063,279.35039062)(11.14765625,279.45976562)(11.4484375,279.52226562)
\curveto(11.64375,279.56132812)(11.98554688,279.58085937)(12.47382813,279.58085937)
\lineto(16.3,279.58085937)
\lineto(16.3,278.52617187)
\lineto(12.51484375,278.52617187)
\curveto(12.08515625,278.52617187)(11.76484375,278.48515625)(11.55390625,278.403125)
\curveto(11.3390625,278.32109375)(11.16914063,278.17460937)(11.04414063,277.96367187)
\curveto(10.91523438,277.75664062)(10.85078125,277.5125)(10.85078125,277.23125)
\curveto(10.85078125,276.78203125)(10.99335938,276.39335937)(11.27851563,276.06523437)
\curveto(11.56367188,275.74101562)(12.1046875,275.57890625)(12.9015625,275.57890625)
\lineto(16.3,275.57890625)
\closepath
}
}
{
\newrgbcolor{curcolor}{0 0 0}
\pscustom[linestyle=none,fillstyle=solid,fillcolor=curcolor]
{
\newpath
\moveto(14.02070313,285.25859375)
\lineto(14.15546875,286.29570312)
\curveto(14.8703125,286.18242187)(15.43085938,285.89140625)(15.83710938,285.42265625)
\curveto(16.23945313,284.9578125)(16.440625,284.38554687)(16.440625,283.70585937)
\curveto(16.440625,282.85429687)(16.16328125,282.16875)(15.60859375,281.64921875)
\curveto(15.05,281.13359375)(14.25117188,280.87578125)(13.21210938,280.87578125)
\curveto(12.54023438,280.87578125)(11.95234375,280.98710937)(11.4484375,281.20976562)
\curveto(10.94453125,281.43242187)(10.56757813,281.7703125)(10.31757813,282.2234375)
\curveto(10.06367188,282.68046875)(9.93671875,283.1765625)(9.93671875,283.71171875)
\curveto(9.93671875,284.3875)(10.10859375,284.94023437)(10.45234375,285.36992187)
\curveto(10.7921875,285.79960937)(11.2765625,286.075)(11.90546875,286.19609375)
\lineto(12.06367188,285.17070312)
\curveto(11.64570313,285.07304687)(11.33125,284.89921875)(11.1203125,284.64921875)
\curveto(10.909375,284.403125)(10.80390625,284.10429687)(10.80390625,283.75273437)
\curveto(10.80390625,283.22148437)(10.9953125,282.78984375)(11.378125,282.4578125)
\curveto(11.75703125,282.12578125)(12.35859375,281.95976562)(13.1828125,281.95976562)
\curveto(14.01875,281.95976562)(14.62617188,282.11992187)(15.00507813,282.44023437)
\curveto(15.38398438,282.76054687)(15.5734375,283.17851562)(15.5734375,283.69414062)
\curveto(15.5734375,284.10820312)(15.44648438,284.45390625)(15.19257813,284.73125)
\curveto(14.93867188,285.00859375)(14.54804688,285.184375)(14.02070313,285.25859375)
\closepath
}
}
{
\newrgbcolor{curcolor}{0 0 0}
\pscustom[linestyle=none,fillstyle=solid,fillcolor=curcolor]
{
\newpath
\moveto(18.69648438,287.15117187)
\lineto(17.70625,287.03398437)
\curveto(17.76875,287.26445312)(17.8,287.465625)(17.8,287.6375)
\curveto(17.8,287.871875)(17.7609375,288.059375)(17.6828125,288.2)
\curveto(17.6046875,288.340625)(17.4953125,288.45585937)(17.3546875,288.54570312)
\curveto(17.24921875,288.61210937)(16.9875,288.71953125)(16.56953125,288.86796875)
\curveto(16.5109375,288.8875)(16.425,288.91875)(16.31171875,288.96171875)
\lineto(10.07734375,286.60039062)
\lineto(10.07734375,287.73710937)
\lineto(13.68085938,289.03203125)
\curveto(14.13789063,289.2)(14.61835938,289.35039062)(15.12226563,289.48320312)
\curveto(14.63789063,289.60429687)(14.16523438,289.74882812)(13.70429688,289.91679687)
\lineto(10.07734375,291.246875)
\lineto(10.07734375,292.3015625)
\lineto(16.40546875,289.934375)
\curveto(17.0890625,289.68046875)(17.55976563,289.48320312)(17.81757813,289.34257812)
\curveto(18.16523438,289.15507812)(18.41914063,288.94023437)(18.57929688,288.69804687)
\curveto(18.74335938,288.45585937)(18.82539063,288.16679687)(18.82539063,287.83085937)
\curveto(18.82539063,287.62773437)(18.78242188,287.40117187)(18.69648438,287.15117187)
\closepath
}
}
{
\newrgbcolor{curcolor}{0 0 0}
\pscustom[linestyle=none,fillstyle=solid,fillcolor=curcolor]
{
\newpath
\moveto(18.82539063,298.54765625)
\curveto(18.09101563,297.965625)(17.23164063,297.4734375)(16.24726563,297.07109375)
\curveto(15.26289063,296.66875)(14.24335938,296.46757812)(13.18867188,296.46757812)
\curveto(12.25898438,296.46757812)(11.36835938,296.61796875)(10.51679688,296.91875)
\curveto(9.52851563,297.2703125)(8.54414063,297.81328125)(7.56367188,298.54765625)
\lineto(7.56367188,299.30351562)
\curveto(8.37617188,298.83085937)(8.95625,298.51835937)(9.30390625,298.36601562)
\curveto(9.84296875,298.12773437)(10.40546875,297.94023437)(10.99140625,297.80351562)
\curveto(11.721875,297.63554687)(12.45625,297.5515625)(13.19453125,297.5515625)
\curveto(15.0734375,297.5515625)(16.95039063,298.13554687)(18.82539063,299.30351562)
\closepath
}
}
{
\newrgbcolor{curcolor}{0 0 0}
\pscustom[linestyle=none,fillstyle=solid,fillcolor=curcolor]
{
\newpath
\moveto(16.3,300.528125)
\lineto(10.07734375,300.528125)
\lineto(10.07734375,301.47734375)
\lineto(10.96210938,301.47734375)
\curveto(10.27851563,301.934375)(9.93671875,302.59453125)(9.93671875,303.4578125)
\curveto(9.93671875,303.8328125)(10.00507813,304.1765625)(10.14179688,304.4890625)
\curveto(10.27460938,304.80546875)(10.45039063,305.04179687)(10.66914063,305.19804687)
\curveto(10.88789063,305.35429687)(11.14765625,305.46367187)(11.4484375,305.52617187)
\curveto(11.64375,305.56523437)(11.98554688,305.58476562)(12.47382813,305.58476562)
\lineto(16.3,305.58476562)
\lineto(16.3,304.53007812)
\lineto(12.51484375,304.53007812)
\curveto(12.08515625,304.53007812)(11.76484375,304.4890625)(11.55390625,304.40703125)
\curveto(11.3390625,304.325)(11.16914063,304.17851562)(11.04414063,303.96757812)
\curveto(10.91523438,303.76054687)(10.85078125,303.51640625)(10.85078125,303.23515625)
\curveto(10.85078125,302.7859375)(10.99335938,302.39726562)(11.27851563,302.06914062)
\curveto(11.56367188,301.74492187)(12.1046875,301.5828125)(12.9015625,301.5828125)
\lineto(16.3,301.5828125)
\closepath
}
}
{
\newrgbcolor{curcolor}{0 0 0}
\pscustom[linestyle=none,fillstyle=solid,fillcolor=curcolor]
{
\newpath
\moveto(14.44257813,306.78007812)
\lineto(14.27851563,307.82304687)
\curveto(14.69648438,307.88164062)(15.01679688,308.04375)(15.23945313,308.309375)
\curveto(15.46210938,308.57890625)(15.5734375,308.95390625)(15.5734375,309.434375)
\curveto(15.5734375,309.91875)(15.47578125,310.278125)(15.28046875,310.5125)
\curveto(15.08125,310.746875)(14.84882813,310.8640625)(14.58320313,310.8640625)
\curveto(14.34492188,310.8640625)(14.15742188,310.76054687)(14.02070313,310.55351562)
\curveto(13.92695313,310.40898437)(13.8078125,310.04960937)(13.66328125,309.47539062)
\curveto(13.46796875,308.70195312)(13.3,308.16484375)(13.159375,307.8640625)
\curveto(13.01484375,307.5671875)(12.81757813,307.340625)(12.56757813,307.184375)
\curveto(12.31367188,307.03203125)(12.034375,306.95585937)(11.7296875,306.95585937)
\curveto(11.45234375,306.95585937)(11.19648438,307.01835937)(10.96210938,307.14335937)
\curveto(10.72382813,307.27226562)(10.5265625,307.44609375)(10.3703125,307.66484375)
\curveto(10.24921875,307.82890625)(10.14765625,308.0515625)(10.065625,308.3328125)
\curveto(9.9796875,308.61796875)(9.93671875,308.92265625)(9.93671875,309.246875)
\curveto(9.93671875,309.73515625)(10.00703125,310.16289062)(10.14765625,310.53007812)
\curveto(10.28828125,310.90117187)(10.4796875,311.17460937)(10.721875,311.35039062)
\curveto(10.96015625,311.52617187)(11.28046875,311.64726562)(11.6828125,311.71367187)
\lineto(11.8234375,310.68242187)
\curveto(11.503125,310.63554687)(11.253125,310.49882812)(11.0734375,310.27226562)
\curveto(10.89375,310.04960937)(10.80390625,309.73320312)(10.80390625,309.32304687)
\curveto(10.80390625,308.83867187)(10.88398438,308.49296875)(11.04414063,308.2859375)
\curveto(11.20429688,308.07890625)(11.39179688,307.97539062)(11.60664063,307.97539062)
\curveto(11.74335938,307.97539062)(11.86640625,308.01835937)(11.97578125,308.10429687)
\curveto(12.0890625,308.19023437)(12.1828125,308.325)(12.25703125,308.50859375)
\curveto(12.29609375,308.6140625)(12.3859375,308.92460937)(12.5265625,309.44023437)
\curveto(12.72578125,310.18632812)(12.88984375,310.70585937)(13.01875,310.99882812)
\curveto(13.14375,311.29570312)(13.32734375,311.528125)(13.56953125,311.69609375)
\curveto(13.81171875,311.8640625)(14.1125,311.94804687)(14.471875,311.94804687)
\curveto(14.8234375,311.94804687)(15.15546875,311.84453125)(15.46796875,311.6375)
\curveto(15.7765625,311.434375)(16.01679688,311.13945312)(16.18867188,310.75273437)
\curveto(16.35664063,310.36601562)(16.440625,309.92851562)(16.440625,309.44023437)
\curveto(16.440625,308.63164062)(16.27265625,308.01445312)(15.93671875,307.58867187)
\curveto(15.60078125,307.16679687)(15.10273438,306.89726562)(14.44257813,306.78007812)
\closepath
}
}
{
\newrgbcolor{curcolor}{0 0 0}
\pscustom[linestyle=none,fillstyle=solid,fillcolor=curcolor]
{
\newpath
\moveto(18.82539063,313.89335937)
\lineto(18.82539063,313.1375)
\curveto(16.95039063,314.30546875)(15.0734375,314.88945312)(13.19453125,314.88945312)
\curveto(12.46015625,314.88945312)(11.73164063,314.80546875)(11.00898438,314.6375)
\curveto(10.42304688,314.5046875)(9.86054688,314.31914062)(9.32148438,314.08085937)
\curveto(8.96992188,313.92851562)(8.38398438,313.6140625)(7.56367188,313.1375)
\lineto(7.56367188,313.89335937)
\curveto(8.54414063,314.62773437)(9.52851563,315.17070312)(10.51679688,315.52226562)
\curveto(11.36835938,315.82304687)(12.25898438,315.9734375)(13.18867188,315.9734375)
\curveto(14.24335938,315.9734375)(15.26289063,315.7703125)(16.24726563,315.3640625)
\curveto(17.23164063,314.96171875)(18.09101563,314.47148437)(18.82539063,313.89335937)
\closepath
}
}
{
\newrgbcolor{curcolor}{0 0 0}
\pscustom[linestyle=none,fillstyle=solid,fillcolor=curcolor]
{
\newpath
\moveto(322.83652344,8.7)
\lineto(322.83652344,17.28984375)
\lineto(326.64511719,17.28984375)
\curveto(327.41074219,17.28984375)(327.99277344,17.21171875)(328.39121094,17.05546875)
\curveto(328.78964844,16.903125)(329.10800781,16.63164062)(329.34628906,16.24101562)
\curveto(329.58457031,15.85039062)(329.70371094,15.41875)(329.70371094,14.94609375)
\curveto(329.70371094,14.33671875)(329.50644531,13.82304687)(329.11191406,13.40507812)
\curveto(328.71738281,12.98710937)(328.10800781,12.72148437)(327.28378906,12.60820312)
\curveto(327.58457031,12.46367187)(327.81308594,12.32109375)(327.96933594,12.18046875)
\curveto(328.30136719,11.87578125)(328.61582031,11.49492187)(328.91269531,11.03789062)
\lineto(330.40683594,8.7)
\lineto(328.97714844,8.7)
\lineto(327.84042969,10.48710937)
\curveto(327.50839844,11.00273437)(327.23496094,11.39726562)(327.02011719,11.67070312)
\curveto(326.80527344,11.94414062)(326.61191406,12.13554687)(326.44003906,12.24492187)
\curveto(326.27207031,12.35429687)(326.10019531,12.43046875)(325.92441406,12.4734375)
\curveto(325.79550781,12.50078125)(325.58457031,12.51445312)(325.29160156,12.51445312)
\lineto(323.97324219,12.51445312)
\lineto(323.97324219,8.7)
\closepath
\moveto(323.97324219,13.49882812)
\lineto(326.41660156,13.49882812)
\curveto(326.93613281,13.49882812)(327.34238281,13.5515625)(327.63535156,13.65703125)
\curveto(327.92832031,13.76640625)(328.15097656,13.93828125)(328.30332031,14.17265625)
\curveto(328.45566406,14.4109375)(328.53183594,14.66875)(328.53183594,14.94609375)
\curveto(328.53183594,15.35234375)(328.38339844,15.68632812)(328.08652344,15.94804687)
\curveto(327.79355469,16.20976562)(327.32871094,16.340625)(326.69199219,16.340625)
\lineto(323.97324219,16.340625)
\closepath
}
}
{
\newrgbcolor{curcolor}{0 0 0}
\pscustom[linestyle=none,fillstyle=solid,fillcolor=curcolor]
{
\newpath
\moveto(335.42832031,8.7)
\lineto(335.42832031,9.6140625)
\curveto(334.94394531,8.9109375)(334.28574219,8.559375)(333.45371094,8.559375)
\curveto(333.08652344,8.559375)(332.74277344,8.6296875)(332.42246094,8.7703125)
\curveto(332.10605469,8.9109375)(331.86972656,9.08671875)(331.71347656,9.29765625)
\curveto(331.56113281,9.5125)(331.45371094,9.77421875)(331.39121094,10.0828125)
\curveto(331.34824219,10.28984375)(331.32675781,10.61796875)(331.32675781,11.0671875)
\lineto(331.32675781,14.92265625)
\lineto(332.38144531,14.92265625)
\lineto(332.38144531,11.47148437)
\curveto(332.38144531,10.92070312)(332.40292969,10.54960937)(332.44589844,10.35820312)
\curveto(332.51230469,10.08085937)(332.65292969,9.86210937)(332.86777344,9.70195312)
\curveto(333.08261719,9.54570312)(333.34824219,9.46757812)(333.66464844,9.46757812)
\curveto(333.98105469,9.46757812)(334.27792969,9.54765625)(334.55527344,9.7078125)
\curveto(334.83261719,9.871875)(335.02792969,10.09257812)(335.14121094,10.36992187)
\curveto(335.25839844,10.65117187)(335.31699219,11.05742187)(335.31699219,11.58867187)
\lineto(335.31699219,14.92265625)
\lineto(336.37167969,14.92265625)
\lineto(336.37167969,8.7)
\closepath
}
}
{
\newrgbcolor{curcolor}{0 0 0}
\pscustom[linestyle=none,fillstyle=solid,fillcolor=curcolor]
{
\newpath
\moveto(338.02402344,8.7)
\lineto(338.02402344,14.92265625)
\lineto(338.97324219,14.92265625)
\lineto(338.97324219,14.03789062)
\curveto(339.43027344,14.72148437)(340.09042969,15.06328125)(340.95371094,15.06328125)
\curveto(341.32871094,15.06328125)(341.67246094,14.99492187)(341.98496094,14.85820312)
\curveto(342.30136719,14.72539062)(342.53769531,14.54960937)(342.69394531,14.33085937)
\curveto(342.85019531,14.11210937)(342.95957031,13.85234375)(343.02207031,13.5515625)
\curveto(343.06113281,13.35625)(343.08066406,13.01445312)(343.08066406,12.52617187)
\lineto(343.08066406,8.7)
\lineto(342.02597656,8.7)
\lineto(342.02597656,12.48515625)
\curveto(342.02597656,12.91484375)(341.98496094,13.23515625)(341.90292969,13.44609375)
\curveto(341.82089844,13.6609375)(341.67441406,13.83085937)(341.46347656,13.95585937)
\curveto(341.25644531,14.08476562)(341.01230469,14.14921875)(340.73105469,14.14921875)
\curveto(340.28183594,14.14921875)(339.89316406,14.00664062)(339.56503906,13.72148437)
\curveto(339.24082031,13.43632812)(339.07871094,12.8953125)(339.07871094,12.0984375)
\lineto(339.07871094,8.7)
\closepath
}
}
{
\newrgbcolor{curcolor}{0 0 0}
\pscustom[linestyle=none,fillstyle=solid,fillcolor=curcolor]
{
\newpath
\moveto(344.27597656,10.55742187)
\lineto(345.31894531,10.72148437)
\curveto(345.37753906,10.30351562)(345.53964844,9.98320312)(345.80527344,9.76054687)
\curveto(346.07480469,9.53789062)(346.44980469,9.4265625)(346.93027344,9.4265625)
\curveto(347.41464844,9.4265625)(347.77402344,9.52421875)(348.00839844,9.71953125)
\curveto(348.24277344,9.91875)(348.35996094,10.15117187)(348.35996094,10.41679687)
\curveto(348.35996094,10.65507812)(348.25644531,10.84257812)(348.04941406,10.97929687)
\curveto(347.90488281,11.07304687)(347.54550781,11.1921875)(346.97128906,11.33671875)
\curveto(346.19785156,11.53203125)(345.66074219,11.7)(345.35996094,11.840625)
\curveto(345.06308594,11.98515625)(344.83652344,12.18242187)(344.68027344,12.43242187)
\curveto(344.52792969,12.68632812)(344.45175781,12.965625)(344.45175781,13.2703125)
\curveto(344.45175781,13.54765625)(344.51425781,13.80351562)(344.63925781,14.03789062)
\curveto(344.76816406,14.27617187)(344.94199219,14.4734375)(345.16074219,14.6296875)
\curveto(345.32480469,14.75078125)(345.54746094,14.85234375)(345.82871094,14.934375)
\curveto(346.11386719,15.0203125)(346.41855469,15.06328125)(346.74277344,15.06328125)
\curveto(347.23105469,15.06328125)(347.65878906,14.99296875)(348.02597656,14.85234375)
\curveto(348.39707031,14.71171875)(348.67050781,14.5203125)(348.84628906,14.278125)
\curveto(349.02207031,14.03984375)(349.14316406,13.71953125)(349.20957031,13.3171875)
\lineto(348.17832031,13.1765625)
\curveto(348.13144531,13.496875)(347.99472656,13.746875)(347.76816406,13.9265625)
\curveto(347.54550781,14.10625)(347.22910156,14.19609375)(346.81894531,14.19609375)
\curveto(346.33457031,14.19609375)(345.98886719,14.11601562)(345.78183594,13.95585937)
\curveto(345.57480469,13.79570312)(345.47128906,13.60820312)(345.47128906,13.39335937)
\curveto(345.47128906,13.25664062)(345.51425781,13.13359375)(345.60019531,13.02421875)
\curveto(345.68613281,12.9109375)(345.82089844,12.8171875)(346.00449219,12.74296875)
\curveto(346.10996094,12.70390625)(346.42050781,12.6140625)(346.93613281,12.4734375)
\curveto(347.68222656,12.27421875)(348.20175781,12.11015625)(348.49472656,11.98125)
\curveto(348.79160156,11.85625)(349.02402344,11.67265625)(349.19199219,11.43046875)
\curveto(349.35996094,11.18828125)(349.44394531,10.8875)(349.44394531,10.528125)
\curveto(349.44394531,10.1765625)(349.34042969,9.84453125)(349.13339844,9.53203125)
\curveto(348.93027344,9.2234375)(348.63535156,8.98320312)(348.24863281,8.81132812)
\curveto(347.86191406,8.64335937)(347.42441406,8.559375)(346.93613281,8.559375)
\curveto(346.12753906,8.559375)(345.51035156,8.72734375)(345.08457031,9.06328125)
\curveto(344.66269531,9.39921875)(344.39316406,9.89726562)(344.27597656,10.55742187)
\closepath
}
}
{
\newrgbcolor{curcolor}{0 0 0}
\pscustom[linestyle=none,fillstyle=solid,fillcolor=curcolor]
{
\newpath
\moveto(98.946875,451.25)
\lineto(100.05429687,451.34375)
\curveto(100.13632812,450.8046875)(100.32578125,450.3984375)(100.62265625,450.125)
\curveto(100.9234375,449.85546875)(101.28476562,449.72070312)(101.70664062,449.72070312)
\curveto(102.21445312,449.72070312)(102.64414062,449.91210938)(102.99570312,450.29492188)
\curveto(103.34726562,450.67773438)(103.52304687,451.18554688)(103.52304687,451.81835938)
\curveto(103.52304687,452.41992188)(103.353125,452.89453125)(103.01328125,453.2421875)
\curveto(102.67734375,453.58984375)(102.2359375,453.76367188)(101.6890625,453.76367188)
\curveto(101.34921875,453.76367188)(101.04257812,453.68554688)(100.76914062,453.52929688)
\curveto(100.49570312,453.37695312)(100.28085937,453.17773438)(100.12460937,452.93164062)
\lineto(99.134375,453.06054688)
\lineto(99.96640625,457.47265625)
\lineto(104.23789062,457.47265625)
\lineto(104.23789062,456.46484375)
\lineto(100.81015625,456.46484375)
\lineto(100.34726562,454.15625)
\curveto(100.86289062,454.515625)(101.40390625,454.6953125)(101.9703125,454.6953125)
\curveto(102.7203125,454.6953125)(103.353125,454.43554688)(103.86875,453.91601562)
\curveto(104.384375,453.39648438)(104.6421875,452.72851562)(104.6421875,451.91210938)
\curveto(104.6421875,451.13476562)(104.415625,450.46289062)(103.9625,449.89648438)
\curveto(103.41171875,449.20117188)(102.65976562,448.85351562)(101.70664062,448.85351562)
\curveto(100.92539062,448.85351562)(100.28671875,449.07226562)(99.790625,449.50976562)
\curveto(99.2984375,449.94726562)(99.0171875,450.52734375)(98.946875,451.25)
\closepath
}
}
{
\newrgbcolor{curcolor}{0 0 0}
\pscustom[linestyle=none,fillstyle=solid,fillcolor=curcolor]
{
\newpath
\moveto(105.62070312,453.23632812)
\curveto(105.62070312,454.25195312)(105.72421875,455.06835938)(105.93125,455.68554688)
\curveto(106.1421875,456.30664062)(106.45273437,456.78515625)(106.86289062,457.12109375)
\curveto(107.27695312,457.45703125)(107.79648437,457.625)(108.42148437,457.625)
\curveto(108.88242187,457.625)(109.28671875,457.53125)(109.634375,457.34375)
\curveto(109.98203125,457.16015625)(110.26914062,456.89257812)(110.49570312,456.54101562)
\curveto(110.72226562,456.19335938)(110.9,455.76757812)(111.02890625,455.26367188)
\curveto(111.1578125,454.76367188)(111.22226562,454.08789062)(111.22226562,453.23632812)
\curveto(111.22226562,452.22851562)(111.11875,451.4140625)(110.91171875,450.79296875)
\curveto(110.7046875,450.17578125)(110.39414062,449.69726562)(109.98007812,449.35742188)
\curveto(109.56992187,449.02148438)(109.05039062,448.85351562)(108.42148437,448.85351562)
\curveto(107.59335937,448.85351562)(106.94296875,449.15039062)(106.4703125,449.74414062)
\curveto(105.90390625,450.45898438)(105.62070312,451.62304688)(105.62070312,453.23632812)
\closepath
\moveto(106.7046875,453.23632812)
\curveto(106.7046875,451.82617188)(106.86875,450.88671875)(107.196875,450.41796875)
\curveto(107.52890625,449.953125)(107.93710937,449.72070312)(108.42148437,449.72070312)
\curveto(108.90585937,449.72070312)(109.31210937,449.95507812)(109.64023437,450.42382812)
\curveto(109.97226562,450.89257812)(110.13828125,451.83007812)(110.13828125,453.23632812)
\curveto(110.13828125,454.65039062)(109.97226562,455.58984375)(109.64023437,456.0546875)
\curveto(109.31210937,456.51953125)(108.90195312,456.75195312)(108.40976562,456.75195312)
\curveto(107.92539062,456.75195312)(107.53867187,456.546875)(107.24960937,456.13671875)
\curveto(106.88632812,455.61328125)(106.7046875,454.64648438)(106.7046875,453.23632812)
\closepath
}
}
{
\newrgbcolor{curcolor}{0 0 0}
\pscustom[linestyle=none,fillstyle=solid,fillcolor=curcolor]
{
\newpath
\moveto(114.89023437,449.94335938)
\lineto(115.04257812,449.01171875)
\curveto(114.74570312,448.94921875)(114.48007812,448.91796875)(114.24570312,448.91796875)
\curveto(113.86289062,448.91796875)(113.56601562,448.97851562)(113.35507812,449.09960938)
\curveto(113.14414062,449.22070312)(112.99570312,449.37890625)(112.90976562,449.57421875)
\curveto(112.82382812,449.7734375)(112.78085937,450.18945312)(112.78085937,450.82226562)
\lineto(112.78085937,454.40234375)
\lineto(112.00742187,454.40234375)
\lineto(112.00742187,455.22265625)
\lineto(112.78085937,455.22265625)
\lineto(112.78085937,456.76367188)
\lineto(113.8296875,457.39648438)
\lineto(113.8296875,455.22265625)
\lineto(114.89023437,455.22265625)
\lineto(114.89023437,454.40234375)
\lineto(113.8296875,454.40234375)
\lineto(113.8296875,450.76367188)
\curveto(113.8296875,450.46289062)(113.84726562,450.26953125)(113.88242187,450.18359375)
\curveto(113.92148437,450.09765625)(113.98203125,450.02929688)(114.0640625,449.97851562)
\curveto(114.15,449.92773438)(114.27109375,449.90234375)(114.42734375,449.90234375)
\curveto(114.54453125,449.90234375)(114.69882812,449.91601562)(114.89023437,449.94335938)
\closepath
}
}
{
\newrgbcolor{curcolor}{0 0 0}
\pscustom[linestyle=none,fillstyle=solid,fillcolor=curcolor]
{
\newpath
\moveto(115.92148437,449)
\lineto(115.92148437,457.58984375)
\lineto(116.97617187,457.58984375)
\lineto(116.97617187,454.5078125)
\curveto(117.46835937,455.078125)(118.08945312,455.36328125)(118.83945312,455.36328125)
\curveto(119.30039062,455.36328125)(119.70078125,455.27148438)(120.040625,455.08789062)
\curveto(120.38046875,454.90820312)(120.62265625,454.65820312)(120.7671875,454.33789062)
\curveto(120.915625,454.01757812)(120.98984375,453.55273438)(120.98984375,452.94335938)
\lineto(120.98984375,449)
\lineto(119.93515625,449)
\lineto(119.93515625,452.94335938)
\curveto(119.93515625,453.47070312)(119.81992187,453.85351562)(119.58945312,454.09179688)
\curveto(119.36289062,454.33398438)(119.040625,454.45507812)(118.62265625,454.45507812)
\curveto(118.31015625,454.45507812)(118.01523437,454.37304688)(117.73789062,454.20898438)
\curveto(117.46445312,454.04882812)(117.26914062,453.83007812)(117.15195312,453.55273438)
\curveto(117.03476562,453.27539062)(116.97617187,452.89257812)(116.97617187,452.40429688)
\lineto(116.97617187,449)
\closepath
}
}
{
\newrgbcolor{curcolor}{0 0 0}
\pscustom[linestyle=none,fillstyle=solid,fillcolor=curcolor]
{
\newpath
\moveto(126.0640625,449)
\lineto(126.0640625,457.58984375)
\lineto(129.30429687,457.58984375)
\curveto(129.87460937,457.58984375)(130.31015625,457.5625)(130.6109375,457.5078125)
\curveto(131.0328125,457.4375)(131.38632812,457.30273438)(131.67148437,457.10351562)
\curveto(131.95664062,456.90820312)(132.18515625,456.6328125)(132.35703125,456.27734375)
\curveto(132.5328125,455.921875)(132.62070312,455.53125)(132.62070312,455.10546875)
\curveto(132.62070312,454.375)(132.38828125,453.75585938)(131.9234375,453.24804688)
\curveto(131.45859375,452.74414062)(130.61875,452.4921875)(129.40390625,452.4921875)
\lineto(127.20078125,452.4921875)
\lineto(127.20078125,449)
\closepath
\moveto(127.20078125,453.50585938)
\lineto(129.42148437,453.50585938)
\curveto(130.15585937,453.50585938)(130.67734375,453.64257812)(130.9859375,453.91601562)
\curveto(131.29453125,454.18945312)(131.44882812,454.57421875)(131.44882812,455.0703125)
\curveto(131.44882812,455.4296875)(131.35703125,455.73632812)(131.1734375,455.99023438)
\curveto(130.99375,456.24804688)(130.75546875,456.41796875)(130.45859375,456.5)
\curveto(130.2671875,456.55078125)(129.91367187,456.57617188)(129.39804687,456.57617188)
\lineto(127.20078125,456.57617188)
\closepath
}
}
{
\newrgbcolor{curcolor}{0 0 0}
\pscustom[linestyle=none,fillstyle=solid,fillcolor=curcolor]
{
\newpath
\moveto(138.19296875,451.00390625)
\lineto(139.2828125,450.86914062)
\curveto(139.1109375,450.23242188)(138.79257812,449.73828125)(138.32773437,449.38671875)
\curveto(137.86289062,449.03515625)(137.26914062,448.859375)(136.54648437,448.859375)
\curveto(135.63632812,448.859375)(134.91367187,449.13867188)(134.37851562,449.69726562)
\curveto(133.84726562,450.25976562)(133.58164062,451.046875)(133.58164062,452.05859375)
\curveto(133.58164062,453.10546875)(133.85117187,453.91796875)(134.39023437,454.49609375)
\curveto(134.92929687,455.07421875)(135.62851562,455.36328125)(136.48789062,455.36328125)
\curveto(137.31992187,455.36328125)(137.99960937,455.08007812)(138.52695312,454.51367188)
\curveto(139.05429687,453.94726562)(139.31796875,453.15039062)(139.31796875,452.12304688)
\curveto(139.31796875,452.06054688)(139.31601562,451.96679688)(139.31210937,451.84179688)
\lineto(134.67148437,451.84179688)
\curveto(134.71054687,451.15820312)(134.90390625,450.63476562)(135.2515625,450.27148438)
\curveto(135.59921875,449.90820312)(136.0328125,449.7265625)(136.55234375,449.7265625)
\curveto(136.9390625,449.7265625)(137.26914062,449.828125)(137.54257812,450.03125)
\curveto(137.81601562,450.234375)(138.0328125,450.55859375)(138.19296875,451.00390625)
\closepath
\moveto(134.73007812,452.70898438)
\lineto(138.2046875,452.70898438)
\curveto(138.1578125,453.23242188)(138.025,453.625)(137.80625,453.88671875)
\curveto(137.4703125,454.29296875)(137.03476562,454.49609375)(136.49960937,454.49609375)
\curveto(136.01523437,454.49609375)(135.60703125,454.33398438)(135.275,454.00976562)
\curveto(134.946875,453.68554688)(134.76523437,453.25195312)(134.73007812,452.70898438)
\closepath
}
}
{
\newrgbcolor{curcolor}{0 0 0}
\pscustom[linestyle=none,fillstyle=solid,fillcolor=curcolor]
{
\newpath
\moveto(140.5953125,449)
\lineto(140.5953125,455.22265625)
\lineto(141.54453125,455.22265625)
\lineto(141.54453125,454.27929688)
\curveto(141.78671875,454.72070312)(142.009375,455.01171875)(142.2125,455.15234375)
\curveto(142.41953125,455.29296875)(142.64609375,455.36328125)(142.8921875,455.36328125)
\curveto(143.24765625,455.36328125)(143.60898437,455.25)(143.97617187,455.0234375)
\lineto(143.61289062,454.04492188)
\curveto(143.35507812,454.19726562)(143.09726562,454.2734375)(142.83945312,454.2734375)
\curveto(142.60898437,454.2734375)(142.40195312,454.203125)(142.21835937,454.0625)
\curveto(142.03476562,453.92578125)(141.90390625,453.734375)(141.82578125,453.48828125)
\curveto(141.70859375,453.11328125)(141.65,452.703125)(141.65,452.2578125)
\lineto(141.65,449)
\closepath
}
}
{
\newrgbcolor{curcolor}{0 0 0}
\pscustom[linestyle=none,fillstyle=solid,fillcolor=curcolor]
{
\newpath
\moveto(148.66367187,451.27929688)
\lineto(149.70078125,451.14453125)
\curveto(149.5875,450.4296875)(149.29648437,449.86914062)(148.82773437,449.46289062)
\curveto(148.36289062,449.06054688)(147.790625,448.859375)(147.1109375,448.859375)
\curveto(146.259375,448.859375)(145.57382812,449.13671875)(145.05429687,449.69140625)
\curveto(144.53867187,450.25)(144.28085937,451.04882812)(144.28085937,452.08789062)
\curveto(144.28085937,452.75976562)(144.3921875,453.34765625)(144.61484375,453.8515625)
\curveto(144.8375,454.35546875)(145.17539062,454.73242188)(145.62851562,454.98242188)
\curveto(146.08554687,455.23632812)(146.58164062,455.36328125)(147.11679687,455.36328125)
\curveto(147.79257812,455.36328125)(148.3453125,455.19140625)(148.775,454.84765625)
\curveto(149.2046875,454.5078125)(149.48007812,454.0234375)(149.60117187,453.39453125)
\lineto(148.57578125,453.23632812)
\curveto(148.478125,453.65429688)(148.30429687,453.96875)(148.05429687,454.1796875)
\curveto(147.80820312,454.390625)(147.509375,454.49609375)(147.1578125,454.49609375)
\curveto(146.6265625,454.49609375)(146.19492187,454.3046875)(145.86289062,453.921875)
\curveto(145.53085937,453.54296875)(145.36484375,452.94140625)(145.36484375,452.1171875)
\curveto(145.36484375,451.28125)(145.525,450.67382812)(145.8453125,450.29492188)
\curveto(146.165625,449.91601562)(146.58359375,449.7265625)(147.09921875,449.7265625)
\curveto(147.51328125,449.7265625)(147.85898437,449.85351562)(148.13632812,450.10742188)
\curveto(148.41367187,450.36132812)(148.58945312,450.75195312)(148.66367187,451.27929688)
\closepath
}
}
{
\newrgbcolor{curcolor}{0 0 0}
\pscustom[linestyle=none,fillstyle=solid,fillcolor=curcolor]
{
\newpath
\moveto(154.86289062,451.00390625)
\lineto(155.95273437,450.86914062)
\curveto(155.78085937,450.23242188)(155.4625,449.73828125)(154.99765625,449.38671875)
\curveto(154.5328125,449.03515625)(153.9390625,448.859375)(153.21640625,448.859375)
\curveto(152.30625,448.859375)(151.58359375,449.13867188)(151.0484375,449.69726562)
\curveto(150.5171875,450.25976562)(150.2515625,451.046875)(150.2515625,452.05859375)
\curveto(150.2515625,453.10546875)(150.52109375,453.91796875)(151.06015625,454.49609375)
\curveto(151.59921875,455.07421875)(152.2984375,455.36328125)(153.1578125,455.36328125)
\curveto(153.98984375,455.36328125)(154.66953125,455.08007812)(155.196875,454.51367188)
\curveto(155.72421875,453.94726562)(155.98789062,453.15039062)(155.98789062,452.12304688)
\curveto(155.98789062,452.06054688)(155.9859375,451.96679688)(155.98203125,451.84179688)
\lineto(151.34140625,451.84179688)
\curveto(151.38046875,451.15820312)(151.57382812,450.63476562)(151.92148437,450.27148438)
\curveto(152.26914062,449.90820312)(152.70273437,449.7265625)(153.22226562,449.7265625)
\curveto(153.60898437,449.7265625)(153.9390625,449.828125)(154.2125,450.03125)
\curveto(154.4859375,450.234375)(154.70273437,450.55859375)(154.86289062,451.00390625)
\closepath
\moveto(151.4,452.70898438)
\lineto(154.87460937,452.70898438)
\curveto(154.82773437,453.23242188)(154.69492187,453.625)(154.47617187,453.88671875)
\curveto(154.14023437,454.29296875)(153.7046875,454.49609375)(153.16953125,454.49609375)
\curveto(152.68515625,454.49609375)(152.27695312,454.33398438)(151.94492187,454.00976562)
\curveto(151.61679687,453.68554688)(151.43515625,453.25195312)(151.4,452.70898438)
\closepath
}
}
{
\newrgbcolor{curcolor}{0 0 0}
\pscustom[linestyle=none,fillstyle=solid,fillcolor=curcolor]
{
\newpath
\moveto(157.27695312,449)
\lineto(157.27695312,455.22265625)
\lineto(158.22617187,455.22265625)
\lineto(158.22617187,454.33789062)
\curveto(158.68320312,455.02148438)(159.34335937,455.36328125)(160.20664062,455.36328125)
\curveto(160.58164062,455.36328125)(160.92539062,455.29492188)(161.23789062,455.15820312)
\curveto(161.55429687,455.02539062)(161.790625,454.84960938)(161.946875,454.63085938)
\curveto(162.103125,454.41210938)(162.2125,454.15234375)(162.275,453.8515625)
\curveto(162.3140625,453.65625)(162.33359375,453.31445312)(162.33359375,452.82617188)
\lineto(162.33359375,449)
\lineto(161.27890625,449)
\lineto(161.27890625,452.78515625)
\curveto(161.27890625,453.21484375)(161.23789062,453.53515625)(161.15585937,453.74609375)
\curveto(161.07382812,453.9609375)(160.92734375,454.13085938)(160.71640625,454.25585938)
\curveto(160.509375,454.38476562)(160.26523437,454.44921875)(159.98398437,454.44921875)
\curveto(159.53476562,454.44921875)(159.14609375,454.30664062)(158.81796875,454.02148438)
\curveto(158.49375,453.73632812)(158.33164062,453.1953125)(158.33164062,452.3984375)
\lineto(158.33164062,449)
\closepath
}
}
{
\newrgbcolor{curcolor}{0 0 0}
\pscustom[linestyle=none,fillstyle=solid,fillcolor=curcolor]
{
\newpath
\moveto(166.25351562,449.94335938)
\lineto(166.40585937,449.01171875)
\curveto(166.10898437,448.94921875)(165.84335937,448.91796875)(165.60898437,448.91796875)
\curveto(165.22617187,448.91796875)(164.92929687,448.97851562)(164.71835937,449.09960938)
\curveto(164.50742187,449.22070312)(164.35898437,449.37890625)(164.27304687,449.57421875)
\curveto(164.18710937,449.7734375)(164.14414062,450.18945312)(164.14414062,450.82226562)
\lineto(164.14414062,454.40234375)
\lineto(163.37070312,454.40234375)
\lineto(163.37070312,455.22265625)
\lineto(164.14414062,455.22265625)
\lineto(164.14414062,456.76367188)
\lineto(165.19296875,457.39648438)
\lineto(165.19296875,455.22265625)
\lineto(166.25351562,455.22265625)
\lineto(166.25351562,454.40234375)
\lineto(165.19296875,454.40234375)
\lineto(165.19296875,450.76367188)
\curveto(165.19296875,450.46289062)(165.21054687,450.26953125)(165.24570312,450.18359375)
\curveto(165.28476562,450.09765625)(165.3453125,450.02929688)(165.42734375,449.97851562)
\curveto(165.51328125,449.92773438)(165.634375,449.90234375)(165.790625,449.90234375)
\curveto(165.9078125,449.90234375)(166.06210937,449.91601562)(166.25351562,449.94335938)
\closepath
}
}
{
\newrgbcolor{curcolor}{0 0 0}
\pscustom[linestyle=none,fillstyle=solid,fillcolor=curcolor]
{
\newpath
\moveto(167.290625,456.37695312)
\lineto(167.290625,457.58984375)
\lineto(168.3453125,457.58984375)
\lineto(168.3453125,456.37695312)
\closepath
\moveto(167.290625,449)
\lineto(167.290625,455.22265625)
\lineto(168.3453125,455.22265625)
\lineto(168.3453125,449)
\closepath
}
}
{
\newrgbcolor{curcolor}{0 0 0}
\pscustom[linestyle=none,fillstyle=solid,fillcolor=curcolor]
{
\newpath
\moveto(169.92734375,449)
\lineto(169.92734375,457.58984375)
\lineto(170.98203125,457.58984375)
\lineto(170.98203125,449)
\closepath
}
}
{
\newrgbcolor{curcolor}{0 0 0}
\pscustom[linestyle=none,fillstyle=solid,fillcolor=curcolor]
{
\newpath
\moveto(176.8765625,451.00390625)
\lineto(177.96640625,450.86914062)
\curveto(177.79453125,450.23242188)(177.47617187,449.73828125)(177.01132812,449.38671875)
\curveto(176.54648437,449.03515625)(175.95273437,448.859375)(175.23007812,448.859375)
\curveto(174.31992187,448.859375)(173.59726562,449.13867188)(173.06210937,449.69726562)
\curveto(172.53085937,450.25976562)(172.26523437,451.046875)(172.26523437,452.05859375)
\curveto(172.26523437,453.10546875)(172.53476562,453.91796875)(173.07382812,454.49609375)
\curveto(173.61289062,455.07421875)(174.31210937,455.36328125)(175.17148437,455.36328125)
\curveto(176.00351562,455.36328125)(176.68320312,455.08007812)(177.21054687,454.51367188)
\curveto(177.73789062,453.94726562)(178.0015625,453.15039062)(178.0015625,452.12304688)
\curveto(178.0015625,452.06054688)(177.99960937,451.96679688)(177.99570312,451.84179688)
\lineto(173.35507812,451.84179688)
\curveto(173.39414062,451.15820312)(173.5875,450.63476562)(173.93515625,450.27148438)
\curveto(174.2828125,449.90820312)(174.71640625,449.7265625)(175.2359375,449.7265625)
\curveto(175.62265625,449.7265625)(175.95273437,449.828125)(176.22617187,450.03125)
\curveto(176.49960937,450.234375)(176.71640625,450.55859375)(176.8765625,451.00390625)
\closepath
\moveto(173.41367187,452.70898438)
\lineto(176.88828125,452.70898438)
\curveto(176.84140625,453.23242188)(176.70859375,453.625)(176.48984375,453.88671875)
\curveto(176.15390625,454.29296875)(175.71835937,454.49609375)(175.18320312,454.49609375)
\curveto(174.69882812,454.49609375)(174.290625,454.33398438)(173.95859375,454.00976562)
\curveto(173.63046875,453.68554688)(173.44882812,453.25195312)(173.41367187,452.70898438)
\closepath
}
}
{
\newrgbcolor{curcolor}{0 0 0}
\pscustom[linestyle=none,fillstyle=solid,fillcolor=curcolor]
{
\newpath
\moveto(182.7125,449)
\lineto(182.7125,457.58984375)
\lineto(183.84921875,457.58984375)
\lineto(183.84921875,450.01367188)
\lineto(188.0796875,450.01367188)
\lineto(188.0796875,449)
\closepath
}
}
{
\newrgbcolor{curcolor}{0 0 0}
\pscustom[linestyle=none,fillstyle=solid,fillcolor=curcolor]
{
\newpath
\moveto(193.35898437,449.76757812)
\curveto(192.96835937,449.43554688)(192.59140625,449.20117188)(192.228125,449.06445312)
\curveto(191.86875,448.92773438)(191.48203125,448.859375)(191.06796875,448.859375)
\curveto(190.384375,448.859375)(189.85898437,449.02539062)(189.49179687,449.35742188)
\curveto(189.12460937,449.69335938)(188.94101562,450.12109375)(188.94101562,450.640625)
\curveto(188.94101562,450.9453125)(189.009375,451.22265625)(189.14609375,451.47265625)
\curveto(189.28671875,451.7265625)(189.46835937,451.9296875)(189.69101562,452.08203125)
\curveto(189.91757812,452.234375)(190.17148437,452.34960938)(190.45273437,452.42773438)
\curveto(190.65976562,452.48242188)(190.97226562,452.53515625)(191.39023437,452.5859375)
\curveto(192.24179687,452.6875)(192.86875,452.80859375)(193.27109375,452.94921875)
\curveto(193.275,453.09375)(193.27695312,453.18554688)(193.27695312,453.22460938)
\curveto(193.27695312,453.65429688)(193.17734375,453.95703125)(192.978125,454.1328125)
\curveto(192.70859375,454.37109375)(192.30820312,454.49023438)(191.77695312,454.49023438)
\curveto(191.28085937,454.49023438)(190.91367187,454.40234375)(190.67539062,454.2265625)
\curveto(190.44101562,454.0546875)(190.2671875,453.74804688)(190.15390625,453.30664062)
\lineto(189.12265625,453.44726562)
\curveto(189.21640625,453.88867188)(189.37070312,454.24414062)(189.58554687,454.51367188)
\curveto(189.80039062,454.78710938)(190.1109375,454.99609375)(190.5171875,455.140625)
\curveto(190.9234375,455.2890625)(191.39414062,455.36328125)(191.92929687,455.36328125)
\curveto(192.46054687,455.36328125)(192.8921875,455.30078125)(193.22421875,455.17578125)
\curveto(193.55625,455.05078125)(193.80039062,454.89257812)(193.95664062,454.70117188)
\curveto(194.11289062,454.51367188)(194.22226562,454.27539062)(194.28476562,453.98632812)
\curveto(194.31992187,453.80664062)(194.3375,453.48242188)(194.3375,453.01367188)
\lineto(194.3375,451.60742188)
\curveto(194.3375,450.62695312)(194.35898437,450.00585938)(194.40195312,449.74414062)
\curveto(194.44882812,449.48632812)(194.53867187,449.23828125)(194.67148437,449)
\lineto(193.56992187,449)
\curveto(193.46054687,449.21875)(193.39023437,449.47460938)(193.35898437,449.76757812)
\closepath
\moveto(193.27109375,452.12304688)
\curveto(192.88828125,451.96679688)(192.3140625,451.83398438)(191.5484375,451.72460938)
\curveto(191.11484375,451.66210938)(190.80820312,451.59179688)(190.62851562,451.51367188)
\curveto(190.44882812,451.43554688)(190.31015625,451.3203125)(190.2125,451.16796875)
\curveto(190.11484375,451.01953125)(190.06601562,450.85351562)(190.06601562,450.66992188)
\curveto(190.06601562,450.38867188)(190.17148437,450.15429688)(190.38242187,449.96679688)
\curveto(190.59726562,449.77929688)(190.90976562,449.68554688)(191.31992187,449.68554688)
\curveto(191.72617187,449.68554688)(192.0875,449.7734375)(192.40390625,449.94921875)
\curveto(192.7203125,450.12890625)(192.95273437,450.37304688)(193.10117187,450.68164062)
\curveto(193.21445312,450.91992188)(193.27109375,451.27148438)(193.27109375,451.73632812)
\closepath
}
}
{
\newrgbcolor{curcolor}{0 0 0}
\pscustom[linestyle=none,fillstyle=solid,fillcolor=curcolor]
{
\newpath
\moveto(198.275,449.94335938)
\lineto(198.42734375,449.01171875)
\curveto(198.13046875,448.94921875)(197.86484375,448.91796875)(197.63046875,448.91796875)
\curveto(197.24765625,448.91796875)(196.95078125,448.97851562)(196.73984375,449.09960938)
\curveto(196.52890625,449.22070312)(196.38046875,449.37890625)(196.29453125,449.57421875)
\curveto(196.20859375,449.7734375)(196.165625,450.18945312)(196.165625,450.82226562)
\lineto(196.165625,454.40234375)
\lineto(195.3921875,454.40234375)
\lineto(195.3921875,455.22265625)
\lineto(196.165625,455.22265625)
\lineto(196.165625,456.76367188)
\lineto(197.21445312,457.39648438)
\lineto(197.21445312,455.22265625)
\lineto(198.275,455.22265625)
\lineto(198.275,454.40234375)
\lineto(197.21445312,454.40234375)
\lineto(197.21445312,450.76367188)
\curveto(197.21445312,450.46289062)(197.23203125,450.26953125)(197.2671875,450.18359375)
\curveto(197.30625,450.09765625)(197.36679687,450.02929688)(197.44882812,449.97851562)
\curveto(197.53476562,449.92773438)(197.65585937,449.90234375)(197.81210937,449.90234375)
\curveto(197.92929687,449.90234375)(198.08359375,449.91601562)(198.275,449.94335938)
\closepath
}
}
{
\newrgbcolor{curcolor}{0 0 0}
\pscustom[linestyle=none,fillstyle=solid,fillcolor=curcolor]
{
\newpath
\moveto(203.56601562,451.00390625)
\lineto(204.65585937,450.86914062)
\curveto(204.48398437,450.23242188)(204.165625,449.73828125)(203.70078125,449.38671875)
\curveto(203.2359375,449.03515625)(202.6421875,448.859375)(201.91953125,448.859375)
\curveto(201.009375,448.859375)(200.28671875,449.13867188)(199.7515625,449.69726562)
\curveto(199.2203125,450.25976562)(198.9546875,451.046875)(198.9546875,452.05859375)
\curveto(198.9546875,453.10546875)(199.22421875,453.91796875)(199.76328125,454.49609375)
\curveto(200.30234375,455.07421875)(201.0015625,455.36328125)(201.8609375,455.36328125)
\curveto(202.69296875,455.36328125)(203.37265625,455.08007812)(203.9,454.51367188)
\curveto(204.42734375,453.94726562)(204.69101562,453.15039062)(204.69101562,452.12304688)
\curveto(204.69101562,452.06054688)(204.6890625,451.96679688)(204.68515625,451.84179688)
\lineto(200.04453125,451.84179688)
\curveto(200.08359375,451.15820312)(200.27695312,450.63476562)(200.62460937,450.27148438)
\curveto(200.97226562,449.90820312)(201.40585937,449.7265625)(201.92539062,449.7265625)
\curveto(202.31210937,449.7265625)(202.6421875,449.828125)(202.915625,450.03125)
\curveto(203.1890625,450.234375)(203.40585937,450.55859375)(203.56601562,451.00390625)
\closepath
\moveto(200.103125,452.70898438)
\lineto(203.57773437,452.70898438)
\curveto(203.53085937,453.23242188)(203.39804687,453.625)(203.17929687,453.88671875)
\curveto(202.84335937,454.29296875)(202.4078125,454.49609375)(201.87265625,454.49609375)
\curveto(201.38828125,454.49609375)(200.98007812,454.33398438)(200.64804687,454.00976562)
\curveto(200.31992187,453.68554688)(200.13828125,453.25195312)(200.103125,452.70898438)
\closepath
}
}
{
\newrgbcolor{curcolor}{0 0 0}
\pscustom[linestyle=none,fillstyle=solid,fillcolor=curcolor]
{
\newpath
\moveto(205.98007812,449)
\lineto(205.98007812,455.22265625)
\lineto(206.92929687,455.22265625)
\lineto(206.92929687,454.33789062)
\curveto(207.38632812,455.02148438)(208.04648437,455.36328125)(208.90976562,455.36328125)
\curveto(209.28476562,455.36328125)(209.62851562,455.29492188)(209.94101562,455.15820312)
\curveto(210.25742187,455.02539062)(210.49375,454.84960938)(210.65,454.63085938)
\curveto(210.80625,454.41210938)(210.915625,454.15234375)(210.978125,453.8515625)
\curveto(211.0171875,453.65625)(211.03671875,453.31445312)(211.03671875,452.82617188)
\lineto(211.03671875,449)
\lineto(209.98203125,449)
\lineto(209.98203125,452.78515625)
\curveto(209.98203125,453.21484375)(209.94101562,453.53515625)(209.85898437,453.74609375)
\curveto(209.77695312,453.9609375)(209.63046875,454.13085938)(209.41953125,454.25585938)
\curveto(209.2125,454.38476562)(208.96835937,454.44921875)(208.68710937,454.44921875)
\curveto(208.23789062,454.44921875)(207.84921875,454.30664062)(207.52109375,454.02148438)
\curveto(207.196875,453.73632812)(207.03476562,453.1953125)(207.03476562,452.3984375)
\lineto(207.03476562,449)
\closepath
}
}
{
\newrgbcolor{curcolor}{0 0 0}
\pscustom[linestyle=none,fillstyle=solid,fillcolor=curcolor]
{
\newpath
\moveto(216.71445312,451.27929688)
\lineto(217.7515625,451.14453125)
\curveto(217.63828125,450.4296875)(217.34726562,449.86914062)(216.87851562,449.46289062)
\curveto(216.41367187,449.06054688)(215.84140625,448.859375)(215.16171875,448.859375)
\curveto(214.31015625,448.859375)(213.62460937,449.13671875)(213.10507812,449.69140625)
\curveto(212.58945312,450.25)(212.33164062,451.04882812)(212.33164062,452.08789062)
\curveto(212.33164062,452.75976562)(212.44296875,453.34765625)(212.665625,453.8515625)
\curveto(212.88828125,454.35546875)(213.22617187,454.73242188)(213.67929687,454.98242188)
\curveto(214.13632812,455.23632812)(214.63242187,455.36328125)(215.16757812,455.36328125)
\curveto(215.84335937,455.36328125)(216.39609375,455.19140625)(216.82578125,454.84765625)
\curveto(217.25546875,454.5078125)(217.53085937,454.0234375)(217.65195312,453.39453125)
\lineto(216.6265625,453.23632812)
\curveto(216.52890625,453.65429688)(216.35507812,453.96875)(216.10507812,454.1796875)
\curveto(215.85898437,454.390625)(215.56015625,454.49609375)(215.20859375,454.49609375)
\curveto(214.67734375,454.49609375)(214.24570312,454.3046875)(213.91367187,453.921875)
\curveto(213.58164062,453.54296875)(213.415625,452.94140625)(213.415625,452.1171875)
\curveto(213.415625,451.28125)(213.57578125,450.67382812)(213.89609375,450.29492188)
\curveto(214.21640625,449.91601562)(214.634375,449.7265625)(215.15,449.7265625)
\curveto(215.5640625,449.7265625)(215.90976562,449.85351562)(216.18710937,450.10742188)
\curveto(216.46445312,450.36132812)(216.64023437,450.75195312)(216.71445312,451.27929688)
\closepath
}
}
{
\newrgbcolor{curcolor}{0 0 0}
\pscustom[linestyle=none,fillstyle=solid,fillcolor=curcolor]
{
\newpath
\moveto(218.60703125,446.60351562)
\lineto(218.48984375,447.59375)
\curveto(218.7203125,447.53125)(218.92148437,447.5)(219.09335937,447.5)
\curveto(219.32773437,447.5)(219.51523437,447.5390625)(219.65585937,447.6171875)
\curveto(219.79648437,447.6953125)(219.91171875,447.8046875)(220.0015625,447.9453125)
\curveto(220.06796875,448.05078125)(220.17539062,448.3125)(220.32382812,448.73046875)
\curveto(220.34335937,448.7890625)(220.37460937,448.875)(220.41757812,448.98828125)
\lineto(218.05625,455.22265625)
\lineto(219.19296875,455.22265625)
\lineto(220.48789062,451.61914062)
\curveto(220.65585937,451.16210938)(220.80625,450.68164062)(220.9390625,450.17773438)
\curveto(221.06015625,450.66210938)(221.2046875,451.13476562)(221.37265625,451.59570312)
\lineto(222.70273437,455.22265625)
\lineto(223.75742187,455.22265625)
\lineto(221.39023437,448.89453125)
\curveto(221.13632812,448.2109375)(220.9390625,447.74023438)(220.7984375,447.48242188)
\curveto(220.6109375,447.13476562)(220.39609375,446.88085938)(220.15390625,446.72070312)
\curveto(219.91171875,446.55664062)(219.62265625,446.47460938)(219.28671875,446.47460938)
\curveto(219.08359375,446.47460938)(218.85703125,446.51757812)(218.60703125,446.60351562)
\closepath
}
}
{
\newrgbcolor{curcolor}{0 0 0}
\pscustom[linestyle=none,fillstyle=solid,fillcolor=curcolor]
{
\newpath
\moveto(227.77695312,453.18359375)
\curveto(227.77695312,454.609375)(228.15976562,455.72460938)(228.92539062,456.52929688)
\curveto(229.69101562,457.33789062)(230.67929687,457.7421875)(231.89023437,457.7421875)
\curveto(232.68320312,457.7421875)(233.39804687,457.55273438)(234.03476562,457.17382812)
\curveto(234.67148437,456.79492188)(235.15585937,456.265625)(235.48789062,455.5859375)
\curveto(235.82382812,454.91015625)(235.99179687,454.14257812)(235.99179687,453.28320312)
\curveto(235.99179687,452.41210938)(235.81601562,451.6328125)(235.46445312,450.9453125)
\curveto(235.11289062,450.2578125)(234.61484375,449.73632812)(233.9703125,449.38085938)
\curveto(233.32578125,449.02929688)(232.63046875,448.85351562)(231.884375,448.85351562)
\curveto(231.07578125,448.85351562)(230.353125,449.04882812)(229.71640625,449.43945312)
\curveto(229.0796875,449.83007812)(228.59726562,450.36328125)(228.26914062,451.0390625)
\curveto(227.94101562,451.71484375)(227.77695312,452.4296875)(227.77695312,453.18359375)
\closepath
\moveto(228.94882812,453.16601562)
\curveto(228.94882812,452.13085938)(229.22617187,451.31445312)(229.78085937,450.71679688)
\curveto(230.33945312,450.12304688)(231.03867187,449.82617188)(231.87851562,449.82617188)
\curveto(232.73398437,449.82617188)(233.43710937,450.12695312)(233.98789062,450.72851562)
\curveto(234.54257812,451.33007812)(234.81992187,452.18359375)(234.81992187,453.2890625)
\curveto(234.81992187,453.98828125)(234.70078125,454.59765625)(234.4625,455.1171875)
\curveto(234.228125,455.640625)(233.88242187,456.04492188)(233.42539062,456.33007812)
\curveto(232.97226562,456.61914062)(232.4625,456.76367188)(231.89609375,456.76367188)
\curveto(231.09140625,456.76367188)(230.39804687,456.48632812)(229.81601562,455.93164062)
\curveto(229.23789062,455.38085938)(228.94882812,454.45898438)(228.94882812,453.16601562)
\closepath
}
}
{
\newrgbcolor{curcolor}{0 0 0}
\pscustom[linestyle=none,fillstyle=solid,fillcolor=curcolor]
{
\newpath
\moveto(237.2984375,449)
\lineto(237.2984375,457.58984375)
\lineto(238.353125,457.58984375)
\lineto(238.353125,449)
\closepath
}
}
{
\newrgbcolor{curcolor}{0 0 0}
\pscustom[linestyle=none,fillstyle=solid,fillcolor=curcolor]
{
\newpath
\moveto(244.025,449)
\lineto(244.025,449.78515625)
\curveto(243.63046875,449.16796875)(243.05039062,448.859375)(242.28476562,448.859375)
\curveto(241.78867187,448.859375)(241.33164062,448.99609375)(240.91367187,449.26953125)
\curveto(240.49960937,449.54296875)(240.17734375,449.92382812)(239.946875,450.41210938)
\curveto(239.7203125,450.90429688)(239.60703125,451.46875)(239.60703125,452.10546875)
\curveto(239.60703125,452.7265625)(239.71054687,453.2890625)(239.91757812,453.79296875)
\curveto(240.12460937,454.30078125)(240.43515625,454.68945312)(240.84921875,454.95898438)
\curveto(241.26328125,455.22851562)(241.72617187,455.36328125)(242.23789062,455.36328125)
\curveto(242.61289062,455.36328125)(242.946875,455.28320312)(243.23984375,455.12304688)
\curveto(243.5328125,454.96679688)(243.77109375,454.76171875)(243.9546875,454.5078125)
\lineto(243.9546875,457.58984375)
\lineto(245.00351562,457.58984375)
\lineto(245.00351562,449)
\closepath
\moveto(240.69101562,452.10546875)
\curveto(240.69101562,451.30859375)(240.85898437,450.71289062)(241.19492187,450.31835938)
\curveto(241.53085937,449.92382812)(241.92734375,449.7265625)(242.384375,449.7265625)
\curveto(242.8453125,449.7265625)(243.2359375,449.9140625)(243.55625,450.2890625)
\curveto(243.88046875,450.66796875)(244.04257812,451.24414062)(244.04257812,452.01757812)
\curveto(244.04257812,452.86914062)(243.87851562,453.49414062)(243.55039062,453.89257812)
\curveto(243.22226562,454.29101562)(242.81796875,454.49023438)(242.3375,454.49023438)
\curveto(241.86875,454.49023438)(241.47617187,454.29882812)(241.15976562,453.91601562)
\curveto(240.84726562,453.53320312)(240.69101562,452.9296875)(240.69101562,452.10546875)
\closepath
}
}
{
\newrgbcolor{curcolor}{0 0 0}
\pscustom[linestyle=none,fillstyle=solid,fillcolor=curcolor]
{
\newpath
\moveto(249.44492187,449)
\lineto(249.44492187,450.0546875)
\lineto(253.8453125,455.55664062)
\curveto(254.1578125,455.94726562)(254.4546875,456.28710938)(254.7359375,456.57617188)
\lineto(249.94296875,456.57617188)
\lineto(249.94296875,457.58984375)
\lineto(256.0953125,457.58984375)
\lineto(256.0953125,456.57617188)
\lineto(251.27304687,450.6171875)
\lineto(250.7515625,450.01367188)
\lineto(256.2359375,450.01367188)
\lineto(256.2359375,449)
\closepath
}
}
{
\newrgbcolor{curcolor}{0 0 0}
\pscustom[linestyle=none,fillstyle=solid,fillcolor=curcolor]
{
\newpath
\moveto(257.51914062,449)
\lineto(257.51914062,457.58984375)
\lineto(263.3140625,457.58984375)
\lineto(263.3140625,456.57617188)
\lineto(258.65585937,456.57617188)
\lineto(258.65585937,453.91601562)
\lineto(262.68710937,453.91601562)
\lineto(262.68710937,452.90234375)
\lineto(258.65585937,452.90234375)
\lineto(258.65585937,449)
\closepath
}
}
{
\newrgbcolor{curcolor}{0 0 0}
\pscustom[linestyle=none,fillstyle=solid,fillcolor=curcolor]
{
\newpath
\moveto(264.40390625,451.75976562)
\lineto(265.47617187,451.85351562)
\curveto(265.52695312,451.42382812)(265.64414062,451.0703125)(265.82773437,450.79296875)
\curveto(266.01523437,450.51953125)(266.30429687,450.296875)(266.69492187,450.125)
\curveto(267.08554687,449.95703125)(267.525,449.87304688)(268.01328125,449.87304688)
\curveto(268.446875,449.87304688)(268.8296875,449.9375)(269.16171875,450.06640625)
\curveto(269.49375,450.1953125)(269.73984375,450.37109375)(269.9,450.59375)
\curveto(270.0640625,450.8203125)(270.14609375,451.06640625)(270.14609375,451.33203125)
\curveto(270.14609375,451.6015625)(270.06796875,451.8359375)(269.91171875,452.03515625)
\curveto(269.75546875,452.23828125)(269.49765625,452.40820312)(269.13828125,452.54492188)
\curveto(268.9078125,452.63476562)(268.39804687,452.7734375)(267.60898437,452.9609375)
\curveto(266.81992187,453.15234375)(266.2671875,453.33203125)(265.95078125,453.5)
\curveto(265.540625,453.71484375)(265.23398437,453.98046875)(265.03085937,454.296875)
\curveto(264.83164062,454.6171875)(264.73203125,454.97460938)(264.73203125,455.36914062)
\curveto(264.73203125,455.80273438)(264.85507812,456.20703125)(265.10117187,456.58203125)
\curveto(265.34726562,456.9609375)(265.70664062,457.24804688)(266.17929687,457.44335938)
\curveto(266.65195312,457.63867188)(267.17734375,457.73632812)(267.75546875,457.73632812)
\curveto(268.3921875,457.73632812)(268.95273437,457.6328125)(269.43710937,457.42578125)
\curveto(269.92539062,457.22265625)(270.30039062,456.921875)(270.56210937,456.5234375)
\curveto(270.82382812,456.125)(270.96445312,455.67382812)(270.98398437,455.16992188)
\lineto(269.89414062,455.08789062)
\curveto(269.83554687,455.63085938)(269.63632812,456.04101562)(269.29648437,456.31835938)
\curveto(268.96054687,456.59570312)(268.4625,456.734375)(267.80234375,456.734375)
\curveto(267.11484375,456.734375)(266.61289062,456.60742188)(266.29648437,456.35351562)
\curveto(265.98398437,456.10351562)(265.82773437,455.80078125)(265.82773437,455.4453125)
\curveto(265.82773437,455.13671875)(265.9390625,454.8828125)(266.16171875,454.68359375)
\curveto(266.38046875,454.484375)(266.95078125,454.27929688)(267.87265625,454.06835938)
\curveto(268.7984375,453.86132812)(269.43320312,453.6796875)(269.77695312,453.5234375)
\curveto(270.27695312,453.29296875)(270.64609375,453)(270.884375,452.64453125)
\curveto(271.12265625,452.29296875)(271.24179687,451.88671875)(271.24179687,451.42578125)
\curveto(271.24179687,450.96875)(271.1109375,450.53710938)(270.84921875,450.13085938)
\curveto(270.5875,449.72851562)(270.21054687,449.4140625)(269.71835937,449.1875)
\curveto(269.23007812,448.96484375)(268.67929687,448.85351562)(268.06601562,448.85351562)
\curveto(267.28867187,448.85351562)(266.63632812,448.96679688)(266.10898437,449.19335938)
\curveto(265.58554687,449.41992188)(265.1734375,449.75976562)(264.87265625,450.21289062)
\curveto(264.57578125,450.66992188)(264.41953125,451.18554688)(264.40390625,451.75976562)
\closepath
}
}
{
\newrgbcolor{curcolor}{0 0 0}
\pscustom[linestyle=none,fillstyle=solid,fillcolor=curcolor]
{
\newpath
\moveto(279.6734375,449)
\lineto(278.61875,449)
\lineto(278.61875,455.72070312)
\curveto(278.36484375,455.47851562)(278.03085937,455.23632812)(277.61679687,454.99414062)
\curveto(277.20664062,454.75195312)(276.8375,454.5703125)(276.509375,454.44921875)
\lineto(276.509375,455.46875)
\curveto(277.09921875,455.74609375)(277.61484375,456.08203125)(278.05625,456.4765625)
\curveto(278.49765625,456.87109375)(278.81015625,457.25390625)(278.99375,457.625)
\lineto(279.6734375,457.625)
\closepath
}
}
{
\newrgbcolor{curcolor}{0 0 0}
\pscustom[linestyle=none,fillstyle=solid,fillcolor=curcolor]
{
\newpath
\moveto(282.7671875,449)
\lineto(282.7671875,457.58984375)
\lineto(284.478125,457.58984375)
\lineto(286.51132812,451.5078125)
\curveto(286.69882812,450.94140625)(286.83554687,450.51757812)(286.92148437,450.23632812)
\curveto(287.01914062,450.54882812)(287.17148437,451.0078125)(287.37851562,451.61328125)
\lineto(289.43515625,457.58984375)
\lineto(290.96445312,457.58984375)
\lineto(290.96445312,449)
\lineto(289.86875,449)
\lineto(289.86875,456.18945312)
\lineto(287.37265625,449)
\lineto(286.34726562,449)
\lineto(283.86289062,456.3125)
\lineto(283.86289062,449)
\closepath
}
}
{
\newrgbcolor{curcolor}{0 0 0}
\pscustom[linestyle=none,fillstyle=solid,fillcolor=curcolor]
{
\newpath
\moveto(296.15,449)
\lineto(296.15,457.58984375)
\lineto(299.95859375,457.58984375)
\curveto(300.72421875,457.58984375)(301.30625,457.51171875)(301.7046875,457.35546875)
\curveto(302.103125,457.203125)(302.42148437,456.93164062)(302.65976562,456.54101562)
\curveto(302.89804687,456.15039062)(303.0171875,455.71875)(303.0171875,455.24609375)
\curveto(303.0171875,454.63671875)(302.81992187,454.12304688)(302.42539062,453.70507812)
\curveto(302.03085937,453.28710938)(301.42148437,453.02148438)(300.59726562,452.90820312)
\curveto(300.89804687,452.76367188)(301.1265625,452.62109375)(301.2828125,452.48046875)
\curveto(301.61484375,452.17578125)(301.92929687,451.79492188)(302.22617187,451.33789062)
\lineto(303.7203125,449)
\lineto(302.290625,449)
\lineto(301.15390625,450.78710938)
\curveto(300.821875,451.30273438)(300.5484375,451.69726562)(300.33359375,451.97070312)
\curveto(300.11875,452.24414062)(299.92539062,452.43554688)(299.75351562,452.54492188)
\curveto(299.58554687,452.65429688)(299.41367187,452.73046875)(299.23789062,452.7734375)
\curveto(299.10898437,452.80078125)(298.89804687,452.81445312)(298.60507812,452.81445312)
\lineto(297.28671875,452.81445312)
\lineto(297.28671875,449)
\closepath
\moveto(297.28671875,453.79882812)
\lineto(299.73007812,453.79882812)
\curveto(300.24960937,453.79882812)(300.65585937,453.8515625)(300.94882812,453.95703125)
\curveto(301.24179687,454.06640625)(301.46445312,454.23828125)(301.61679687,454.47265625)
\curveto(301.76914062,454.7109375)(301.8453125,454.96875)(301.8453125,455.24609375)
\curveto(301.8453125,455.65234375)(301.696875,455.98632812)(301.4,456.24804688)
\curveto(301.10703125,456.50976562)(300.6421875,456.640625)(300.00546875,456.640625)
\lineto(297.28671875,456.640625)
\closepath
}
}
{
\newrgbcolor{curcolor}{0 0 0}
\pscustom[linestyle=none,fillstyle=solid,fillcolor=curcolor]
{
\newpath
\moveto(308.9234375,451.00390625)
\lineto(310.01328125,450.86914062)
\curveto(309.84140625,450.23242188)(309.52304687,449.73828125)(309.05820312,449.38671875)
\curveto(308.59335937,449.03515625)(307.99960937,448.859375)(307.27695312,448.859375)
\curveto(306.36679687,448.859375)(305.64414062,449.13867188)(305.10898437,449.69726562)
\curveto(304.57773437,450.25976562)(304.31210937,451.046875)(304.31210937,452.05859375)
\curveto(304.31210937,453.10546875)(304.58164062,453.91796875)(305.12070312,454.49609375)
\curveto(305.65976562,455.07421875)(306.35898437,455.36328125)(307.21835937,455.36328125)
\curveto(308.05039062,455.36328125)(308.73007812,455.08007812)(309.25742187,454.51367188)
\curveto(309.78476562,453.94726562)(310.0484375,453.15039062)(310.0484375,452.12304688)
\curveto(310.0484375,452.06054688)(310.04648437,451.96679688)(310.04257812,451.84179688)
\lineto(305.40195312,451.84179688)
\curveto(305.44101562,451.15820312)(305.634375,450.63476562)(305.98203125,450.27148438)
\curveto(306.3296875,449.90820312)(306.76328125,449.7265625)(307.2828125,449.7265625)
\curveto(307.66953125,449.7265625)(307.99960937,449.828125)(308.27304687,450.03125)
\curveto(308.54648437,450.234375)(308.76328125,450.55859375)(308.9234375,451.00390625)
\closepath
\moveto(305.46054687,452.70898438)
\lineto(308.93515625,452.70898438)
\curveto(308.88828125,453.23242188)(308.75546875,453.625)(308.53671875,453.88671875)
\curveto(308.20078125,454.29296875)(307.76523437,454.49609375)(307.23007812,454.49609375)
\curveto(306.74570312,454.49609375)(306.3375,454.33398438)(306.00546875,454.00976562)
\curveto(305.67734375,453.68554688)(305.49570312,453.25195312)(305.46054687,452.70898438)
\closepath
}
}
{
\newrgbcolor{curcolor}{0 0 0}
\pscustom[linestyle=none,fillstyle=solid,fillcolor=curcolor]
{
\newpath
\moveto(315.39804687,449.76757812)
\curveto(315.00742187,449.43554688)(314.63046875,449.20117188)(314.2671875,449.06445312)
\curveto(313.9078125,448.92773438)(313.52109375,448.859375)(313.10703125,448.859375)
\curveto(312.4234375,448.859375)(311.89804687,449.02539062)(311.53085937,449.35742188)
\curveto(311.16367187,449.69335938)(310.98007812,450.12109375)(310.98007812,450.640625)
\curveto(310.98007812,450.9453125)(311.0484375,451.22265625)(311.18515625,451.47265625)
\curveto(311.32578125,451.7265625)(311.50742187,451.9296875)(311.73007812,452.08203125)
\curveto(311.95664062,452.234375)(312.21054687,452.34960938)(312.49179687,452.42773438)
\curveto(312.69882812,452.48242188)(313.01132812,452.53515625)(313.42929687,452.5859375)
\curveto(314.28085937,452.6875)(314.9078125,452.80859375)(315.31015625,452.94921875)
\curveto(315.3140625,453.09375)(315.31601562,453.18554688)(315.31601562,453.22460938)
\curveto(315.31601562,453.65429688)(315.21640625,453.95703125)(315.0171875,454.1328125)
\curveto(314.74765625,454.37109375)(314.34726562,454.49023438)(313.81601562,454.49023438)
\curveto(313.31992187,454.49023438)(312.95273437,454.40234375)(312.71445312,454.2265625)
\curveto(312.48007812,454.0546875)(312.30625,453.74804688)(312.19296875,453.30664062)
\lineto(311.16171875,453.44726562)
\curveto(311.25546875,453.88867188)(311.40976562,454.24414062)(311.62460937,454.51367188)
\curveto(311.83945312,454.78710938)(312.15,454.99609375)(312.55625,455.140625)
\curveto(312.9625,455.2890625)(313.43320312,455.36328125)(313.96835937,455.36328125)
\curveto(314.49960937,455.36328125)(314.93125,455.30078125)(315.26328125,455.17578125)
\curveto(315.5953125,455.05078125)(315.83945312,454.89257812)(315.99570312,454.70117188)
\curveto(316.15195312,454.51367188)(316.26132812,454.27539062)(316.32382812,453.98632812)
\curveto(316.35898437,453.80664062)(316.3765625,453.48242188)(316.3765625,453.01367188)
\lineto(316.3765625,451.60742188)
\curveto(316.3765625,450.62695312)(316.39804687,450.00585938)(316.44101562,449.74414062)
\curveto(316.48789062,449.48632812)(316.57773437,449.23828125)(316.71054687,449)
\lineto(315.60898437,449)
\curveto(315.49960937,449.21875)(315.42929687,449.47460938)(315.39804687,449.76757812)
\closepath
\moveto(315.31015625,452.12304688)
\curveto(314.92734375,451.96679688)(314.353125,451.83398438)(313.5875,451.72460938)
\curveto(313.15390625,451.66210938)(312.84726562,451.59179688)(312.66757812,451.51367188)
\curveto(312.48789062,451.43554688)(312.34921875,451.3203125)(312.2515625,451.16796875)
\curveto(312.15390625,451.01953125)(312.10507812,450.85351562)(312.10507812,450.66992188)
\curveto(312.10507812,450.38867188)(312.21054687,450.15429688)(312.42148437,449.96679688)
\curveto(312.63632812,449.77929688)(312.94882812,449.68554688)(313.35898437,449.68554688)
\curveto(313.76523437,449.68554688)(314.1265625,449.7734375)(314.44296875,449.94921875)
\curveto(314.759375,450.12890625)(314.99179687,450.37304688)(315.14023437,450.68164062)
\curveto(315.25351562,450.91992188)(315.31015625,451.27148438)(315.31015625,451.73632812)
\closepath
}
}
{
\newrgbcolor{curcolor}{0 0 0}
\pscustom[linestyle=none,fillstyle=solid,fillcolor=curcolor]
{
\newpath
\moveto(322.0484375,449)
\lineto(322.0484375,449.78515625)
\curveto(321.65390625,449.16796875)(321.07382812,448.859375)(320.30820312,448.859375)
\curveto(319.81210937,448.859375)(319.35507812,448.99609375)(318.93710937,449.26953125)
\curveto(318.52304687,449.54296875)(318.20078125,449.92382812)(317.9703125,450.41210938)
\curveto(317.74375,450.90429688)(317.63046875,451.46875)(317.63046875,452.10546875)
\curveto(317.63046875,452.7265625)(317.73398437,453.2890625)(317.94101562,453.79296875)
\curveto(318.14804687,454.30078125)(318.45859375,454.68945312)(318.87265625,454.95898438)
\curveto(319.28671875,455.22851562)(319.74960937,455.36328125)(320.26132812,455.36328125)
\curveto(320.63632812,455.36328125)(320.9703125,455.28320312)(321.26328125,455.12304688)
\curveto(321.55625,454.96679688)(321.79453125,454.76171875)(321.978125,454.5078125)
\lineto(321.978125,457.58984375)
\lineto(323.02695312,457.58984375)
\lineto(323.02695312,449)
\closepath
\moveto(318.71445312,452.10546875)
\curveto(318.71445312,451.30859375)(318.88242187,450.71289062)(319.21835937,450.31835938)
\curveto(319.55429687,449.92382812)(319.95078125,449.7265625)(320.4078125,449.7265625)
\curveto(320.86875,449.7265625)(321.259375,449.9140625)(321.5796875,450.2890625)
\curveto(321.90390625,450.66796875)(322.06601562,451.24414062)(322.06601562,452.01757812)
\curveto(322.06601562,452.86914062)(321.90195312,453.49414062)(321.57382812,453.89257812)
\curveto(321.24570312,454.29101562)(320.84140625,454.49023438)(320.3609375,454.49023438)
\curveto(319.8921875,454.49023438)(319.49960937,454.29882812)(319.18320312,453.91601562)
\curveto(318.87070312,453.53320312)(318.71445312,452.9296875)(318.71445312,452.10546875)
\closepath
}
}
{
\newrgbcolor{curcolor}{0 0 0}
\pscustom[linestyle=none,fillstyle=solid,fillcolor=curcolor]
{
\newpath
\moveto(324.26328125,450.85742188)
\lineto(325.30625,451.02148438)
\curveto(325.36484375,450.60351562)(325.52695312,450.28320312)(325.79257812,450.06054688)
\curveto(326.06210937,449.83789062)(326.43710937,449.7265625)(326.91757812,449.7265625)
\curveto(327.40195312,449.7265625)(327.76132812,449.82421875)(327.99570312,450.01953125)
\curveto(328.23007812,450.21875)(328.34726562,450.45117188)(328.34726562,450.71679688)
\curveto(328.34726562,450.95507812)(328.24375,451.14257812)(328.03671875,451.27929688)
\curveto(327.8921875,451.37304688)(327.5328125,451.4921875)(326.95859375,451.63671875)
\curveto(326.18515625,451.83203125)(325.64804687,452)(325.34726562,452.140625)
\curveto(325.05039062,452.28515625)(324.82382812,452.48242188)(324.66757812,452.73242188)
\curveto(324.51523437,452.98632812)(324.4390625,453.265625)(324.4390625,453.5703125)
\curveto(324.4390625,453.84765625)(324.5015625,454.10351562)(324.6265625,454.33789062)
\curveto(324.75546875,454.57617188)(324.92929687,454.7734375)(325.14804687,454.9296875)
\curveto(325.31210937,455.05078125)(325.53476562,455.15234375)(325.81601562,455.234375)
\curveto(326.10117187,455.3203125)(326.40585937,455.36328125)(326.73007812,455.36328125)
\curveto(327.21835937,455.36328125)(327.64609375,455.29296875)(328.01328125,455.15234375)
\curveto(328.384375,455.01171875)(328.6578125,454.8203125)(328.83359375,454.578125)
\curveto(329.009375,454.33984375)(329.13046875,454.01953125)(329.196875,453.6171875)
\lineto(328.165625,453.4765625)
\curveto(328.11875,453.796875)(327.98203125,454.046875)(327.75546875,454.2265625)
\curveto(327.5328125,454.40625)(327.21640625,454.49609375)(326.80625,454.49609375)
\curveto(326.321875,454.49609375)(325.97617187,454.41601562)(325.76914062,454.25585938)
\curveto(325.56210937,454.09570312)(325.45859375,453.90820312)(325.45859375,453.69335938)
\curveto(325.45859375,453.55664062)(325.5015625,453.43359375)(325.5875,453.32421875)
\curveto(325.6734375,453.2109375)(325.80820312,453.1171875)(325.99179687,453.04296875)
\curveto(326.09726562,453.00390625)(326.4078125,452.9140625)(326.9234375,452.7734375)
\curveto(327.66953125,452.57421875)(328.1890625,452.41015625)(328.48203125,452.28125)
\curveto(328.77890625,452.15625)(329.01132812,451.97265625)(329.17929687,451.73046875)
\curveto(329.34726562,451.48828125)(329.43125,451.1875)(329.43125,450.828125)
\curveto(329.43125,450.4765625)(329.32773437,450.14453125)(329.12070312,449.83203125)
\curveto(328.91757812,449.5234375)(328.62265625,449.28320312)(328.2359375,449.11132812)
\curveto(327.84921875,448.94335938)(327.41171875,448.859375)(326.9234375,448.859375)
\curveto(326.11484375,448.859375)(325.49765625,449.02734375)(325.071875,449.36328125)
\curveto(324.65,449.69921875)(324.38046875,450.19726562)(324.26328125,450.85742188)
\closepath
}
}
{
\newrgbcolor{curcolor}{0 0 0}
\pscustom[linestyle=none,fillstyle=solid,fillcolor=curcolor]
{
\newpath
\moveto(334.1421875,449)
\lineto(334.1421875,457.58984375)
\lineto(335.30820312,457.58984375)
\lineto(339.81992187,450.84570312)
\lineto(339.81992187,457.58984375)
\lineto(340.90976562,457.58984375)
\lineto(340.90976562,449)
\lineto(339.74375,449)
\lineto(335.23203125,455.75)
\lineto(335.23203125,449)
\closepath
}
}
{
\newrgbcolor{curcolor}{0 0 0}
\pscustom[linestyle=none,fillstyle=solid,fillcolor=curcolor]
{
\newpath
\moveto(348.45664062,457.58984375)
\lineto(349.59335937,457.58984375)
\lineto(349.59335937,452.62695312)
\curveto(349.59335937,451.76367188)(349.49570312,451.078125)(349.30039062,450.5703125)
\curveto(349.10507812,450.0625)(348.7515625,449.6484375)(348.23984375,449.328125)
\curveto(347.73203125,449.01171875)(347.0640625,448.85351562)(346.2359375,448.85351562)
\curveto(345.43125,448.85351562)(344.77304687,448.9921875)(344.26132812,449.26953125)
\curveto(343.74960937,449.546875)(343.384375,449.94726562)(343.165625,450.47070312)
\curveto(342.946875,450.99804688)(342.8375,451.71679688)(342.8375,452.62695312)
\lineto(342.8375,457.58984375)
\lineto(343.97421875,457.58984375)
\lineto(343.97421875,452.6328125)
\curveto(343.97421875,451.88671875)(344.04257812,451.3359375)(344.17929687,450.98046875)
\curveto(344.31992187,450.62890625)(344.55820312,450.35742188)(344.89414062,450.16601562)
\curveto(345.23398437,449.97460938)(345.64804687,449.87890625)(346.13632812,449.87890625)
\curveto(346.97226562,449.87890625)(347.56796875,450.06835938)(347.9234375,450.44726562)
\curveto(348.27890625,450.82617188)(348.45664062,451.5546875)(348.45664062,452.6328125)
\closepath
}
}
{
\newrgbcolor{curcolor}{0 0 0}
\pscustom[linestyle=none,fillstyle=solid,fillcolor=curcolor]
{
\newpath
\moveto(351.45078125,449)
\lineto(351.45078125,457.58984375)
\lineto(353.16171875,457.58984375)
\lineto(355.19492187,451.5078125)
\curveto(355.38242187,450.94140625)(355.51914062,450.51757812)(355.60507812,450.23632812)
\curveto(355.70273437,450.54882812)(355.85507812,451.0078125)(356.06210937,451.61328125)
\lineto(358.11875,457.58984375)
\lineto(359.64804687,457.58984375)
\lineto(359.64804687,449)
\lineto(358.55234375,449)
\lineto(358.55234375,456.18945312)
\lineto(356.05625,449)
\lineto(355.03085937,449)
\lineto(352.54648437,456.3125)
\lineto(352.54648437,449)
\closepath
}
}
{
\newrgbcolor{curcolor}{0 0 0}
\pscustom[linestyle=none,fillstyle=solid,fillcolor=curcolor]
{
\newpath
\moveto(360.53867187,449)
\lineto(363.8375,457.58984375)
\lineto(365.06210937,457.58984375)
\lineto(368.57773437,449)
\lineto(367.2828125,449)
\lineto(366.28085937,451.6015625)
\lineto(362.6890625,451.6015625)
\lineto(361.74570312,449)
\closepath
\moveto(363.0171875,452.52734375)
\lineto(365.92929687,452.52734375)
\lineto(365.0328125,454.90625)
\curveto(364.759375,455.62890625)(364.55625,456.22265625)(364.4234375,456.6875)
\curveto(364.3140625,456.13671875)(364.15976562,455.58984375)(363.96054687,455.046875)
\closepath
}
}
{
\newrgbcolor{curcolor}{0 0 0}
\pscustom[linestyle=none,fillstyle=solid,fillcolor=curcolor]
{
\newpath
\moveto(372.1109375,449)
\lineto(372.1109375,457.58984375)
\lineto(375.33359375,457.58984375)
\curveto(375.98984375,457.58984375)(376.51523437,457.50195312)(376.90976562,457.32617188)
\curveto(377.30820312,457.15429688)(377.61875,456.88671875)(377.84140625,456.5234375)
\curveto(378.06796875,456.1640625)(378.18125,455.78710938)(378.18125,455.39257812)
\curveto(378.18125,455.02539062)(378.08164062,454.6796875)(377.88242187,454.35546875)
\curveto(377.68320312,454.03125)(377.38242187,453.76953125)(376.98007812,453.5703125)
\curveto(377.49960937,453.41796875)(377.89804687,453.15820312)(378.17539062,452.79101562)
\curveto(378.45664062,452.42382812)(378.59726562,451.99023438)(378.59726562,451.49023438)
\curveto(378.59726562,451.08789062)(378.51132812,450.71289062)(378.33945312,450.36523438)
\curveto(378.17148437,450.02148438)(377.9625,449.75585938)(377.7125,449.56835938)
\curveto(377.4625,449.38085938)(377.14804687,449.23828125)(376.76914062,449.140625)
\curveto(376.39414062,449.046875)(375.93320312,449)(375.38632812,449)
\closepath
\moveto(373.24765625,453.98046875)
\lineto(375.10507812,453.98046875)
\curveto(375.60898437,453.98046875)(375.9703125,454.01367188)(376.1890625,454.08007812)
\curveto(376.478125,454.16601562)(376.69492187,454.30859375)(376.83945312,454.5078125)
\curveto(376.98789062,454.70703125)(377.06210937,454.95703125)(377.06210937,455.2578125)
\curveto(377.06210937,455.54296875)(376.99375,455.79296875)(376.85703125,456.0078125)
\curveto(376.7203125,456.2265625)(376.525,456.375)(376.27109375,456.453125)
\curveto(376.0171875,456.53515625)(375.58164062,456.57617188)(374.96445312,456.57617188)
\lineto(373.24765625,456.57617188)
\closepath
\moveto(373.24765625,450.01367188)
\lineto(375.38632812,450.01367188)
\curveto(375.75351562,450.01367188)(376.01132812,450.02734375)(376.15976562,450.0546875)
\curveto(376.42148437,450.1015625)(376.64023437,450.1796875)(376.81601562,450.2890625)
\curveto(376.99179687,450.3984375)(377.13632812,450.55664062)(377.24960937,450.76367188)
\curveto(377.36289062,450.97460938)(377.41953125,451.21679688)(377.41953125,451.49023438)
\curveto(377.41953125,451.81054688)(377.3375,452.08789062)(377.1734375,452.32226562)
\curveto(377.009375,452.56054688)(376.78085937,452.7265625)(376.48789062,452.8203125)
\curveto(376.19882812,452.91796875)(375.78085937,452.96679688)(375.23398437,452.96679688)
\lineto(373.24765625,452.96679688)
\closepath
}
}
{
\newrgbcolor{curcolor}{0 0 0}
\pscustom[linestyle=none,fillstyle=solid,fillcolor=curcolor]
{
\newpath
\moveto(384.0875,449.76757812)
\curveto(383.696875,449.43554688)(383.31992187,449.20117188)(382.95664062,449.06445312)
\curveto(382.59726562,448.92773438)(382.21054687,448.859375)(381.79648437,448.859375)
\curveto(381.11289062,448.859375)(380.5875,449.02539062)(380.2203125,449.35742188)
\curveto(379.853125,449.69335938)(379.66953125,450.12109375)(379.66953125,450.640625)
\curveto(379.66953125,450.9453125)(379.73789062,451.22265625)(379.87460937,451.47265625)
\curveto(380.01523437,451.7265625)(380.196875,451.9296875)(380.41953125,452.08203125)
\curveto(380.64609375,452.234375)(380.9,452.34960938)(381.18125,452.42773438)
\curveto(381.38828125,452.48242188)(381.70078125,452.53515625)(382.11875,452.5859375)
\curveto(382.9703125,452.6875)(383.59726562,452.80859375)(383.99960937,452.94921875)
\curveto(384.00351562,453.09375)(384.00546875,453.18554688)(384.00546875,453.22460938)
\curveto(384.00546875,453.65429688)(383.90585937,453.95703125)(383.70664062,454.1328125)
\curveto(383.43710937,454.37109375)(383.03671875,454.49023438)(382.50546875,454.49023438)
\curveto(382.009375,454.49023438)(381.6421875,454.40234375)(381.40390625,454.2265625)
\curveto(381.16953125,454.0546875)(380.99570312,453.74804688)(380.88242187,453.30664062)
\lineto(379.85117187,453.44726562)
\curveto(379.94492187,453.88867188)(380.09921875,454.24414062)(380.3140625,454.51367188)
\curveto(380.52890625,454.78710938)(380.83945312,454.99609375)(381.24570312,455.140625)
\curveto(381.65195312,455.2890625)(382.12265625,455.36328125)(382.6578125,455.36328125)
\curveto(383.1890625,455.36328125)(383.62070312,455.30078125)(383.95273437,455.17578125)
\curveto(384.28476562,455.05078125)(384.52890625,454.89257812)(384.68515625,454.70117188)
\curveto(384.84140625,454.51367188)(384.95078125,454.27539062)(385.01328125,453.98632812)
\curveto(385.0484375,453.80664062)(385.06601562,453.48242188)(385.06601562,453.01367188)
\lineto(385.06601562,451.60742188)
\curveto(385.06601562,450.62695312)(385.0875,450.00585938)(385.13046875,449.74414062)
\curveto(385.17734375,449.48632812)(385.2671875,449.23828125)(385.4,449)
\lineto(384.2984375,449)
\curveto(384.1890625,449.21875)(384.11875,449.47460938)(384.0875,449.76757812)
\closepath
\moveto(383.99960937,452.12304688)
\curveto(383.61679687,451.96679688)(383.04257812,451.83398438)(382.27695312,451.72460938)
\curveto(381.84335937,451.66210938)(381.53671875,451.59179688)(381.35703125,451.51367188)
\curveto(381.17734375,451.43554688)(381.03867187,451.3203125)(380.94101562,451.16796875)
\curveto(380.84335937,451.01953125)(380.79453125,450.85351562)(380.79453125,450.66992188)
\curveto(380.79453125,450.38867188)(380.9,450.15429688)(381.1109375,449.96679688)
\curveto(381.32578125,449.77929688)(381.63828125,449.68554688)(382.0484375,449.68554688)
\curveto(382.4546875,449.68554688)(382.81601562,449.7734375)(383.13242187,449.94921875)
\curveto(383.44882812,450.12890625)(383.68125,450.37304688)(383.8296875,450.68164062)
\curveto(383.94296875,450.91992188)(383.99960937,451.27148438)(383.99960937,451.73632812)
\closepath
}
}
{
\newrgbcolor{curcolor}{0 0 0}
\pscustom[linestyle=none,fillstyle=solid,fillcolor=curcolor]
{
\newpath
\moveto(386.67734375,449)
\lineto(386.67734375,457.58984375)
\lineto(387.73203125,457.58984375)
\lineto(387.73203125,449)
\closepath
}
}
{
\newrgbcolor{curcolor}{0 0 0}
\pscustom[linestyle=none,fillstyle=solid,fillcolor=curcolor]
{
\newpath
\moveto(393.42734375,449.76757812)
\curveto(393.03671875,449.43554688)(392.65976562,449.20117188)(392.29648437,449.06445312)
\curveto(391.93710937,448.92773438)(391.55039062,448.859375)(391.13632812,448.859375)
\curveto(390.45273437,448.859375)(389.92734375,449.02539062)(389.56015625,449.35742188)
\curveto(389.19296875,449.69335938)(389.009375,450.12109375)(389.009375,450.640625)
\curveto(389.009375,450.9453125)(389.07773437,451.22265625)(389.21445312,451.47265625)
\curveto(389.35507812,451.7265625)(389.53671875,451.9296875)(389.759375,452.08203125)
\curveto(389.9859375,452.234375)(390.23984375,452.34960938)(390.52109375,452.42773438)
\curveto(390.728125,452.48242188)(391.040625,452.53515625)(391.45859375,452.5859375)
\curveto(392.31015625,452.6875)(392.93710937,452.80859375)(393.33945312,452.94921875)
\curveto(393.34335937,453.09375)(393.3453125,453.18554688)(393.3453125,453.22460938)
\curveto(393.3453125,453.65429688)(393.24570312,453.95703125)(393.04648437,454.1328125)
\curveto(392.77695312,454.37109375)(392.3765625,454.49023438)(391.8453125,454.49023438)
\curveto(391.34921875,454.49023438)(390.98203125,454.40234375)(390.74375,454.2265625)
\curveto(390.509375,454.0546875)(390.33554687,453.74804688)(390.22226562,453.30664062)
\lineto(389.19101562,453.44726562)
\curveto(389.28476562,453.88867188)(389.4390625,454.24414062)(389.65390625,454.51367188)
\curveto(389.86875,454.78710938)(390.17929687,454.99609375)(390.58554687,455.140625)
\curveto(390.99179687,455.2890625)(391.4625,455.36328125)(391.99765625,455.36328125)
\curveto(392.52890625,455.36328125)(392.96054687,455.30078125)(393.29257812,455.17578125)
\curveto(393.62460937,455.05078125)(393.86875,454.89257812)(394.025,454.70117188)
\curveto(394.18125,454.51367188)(394.290625,454.27539062)(394.353125,453.98632812)
\curveto(394.38828125,453.80664062)(394.40585937,453.48242188)(394.40585937,453.01367188)
\lineto(394.40585937,451.60742188)
\curveto(394.40585937,450.62695312)(394.42734375,450.00585938)(394.4703125,449.74414062)
\curveto(394.5171875,449.48632812)(394.60703125,449.23828125)(394.73984375,449)
\lineto(393.63828125,449)
\curveto(393.52890625,449.21875)(393.45859375,449.47460938)(393.42734375,449.76757812)
\closepath
\moveto(393.33945312,452.12304688)
\curveto(392.95664062,451.96679688)(392.38242187,451.83398438)(391.61679687,451.72460938)
\curveto(391.18320312,451.66210938)(390.8765625,451.59179688)(390.696875,451.51367188)
\curveto(390.5171875,451.43554688)(390.37851562,451.3203125)(390.28085937,451.16796875)
\curveto(390.18320312,451.01953125)(390.134375,450.85351562)(390.134375,450.66992188)
\curveto(390.134375,450.38867188)(390.23984375,450.15429688)(390.45078125,449.96679688)
\curveto(390.665625,449.77929688)(390.978125,449.68554688)(391.38828125,449.68554688)
\curveto(391.79453125,449.68554688)(392.15585937,449.7734375)(392.47226562,449.94921875)
\curveto(392.78867187,450.12890625)(393.02109375,450.37304688)(393.16953125,450.68164062)
\curveto(393.2828125,450.91992188)(393.33945312,451.27148438)(393.33945312,451.73632812)
\closepath
}
}
{
\newrgbcolor{curcolor}{0 0 0}
\pscustom[linestyle=none,fillstyle=solid,fillcolor=curcolor]
{
\newpath
\moveto(396.040625,449)
\lineto(396.040625,455.22265625)
\lineto(396.98984375,455.22265625)
\lineto(396.98984375,454.33789062)
\curveto(397.446875,455.02148438)(398.10703125,455.36328125)(398.9703125,455.36328125)
\curveto(399.3453125,455.36328125)(399.6890625,455.29492188)(400.0015625,455.15820312)
\curveto(400.31796875,455.02539062)(400.55429687,454.84960938)(400.71054687,454.63085938)
\curveto(400.86679687,454.41210938)(400.97617187,454.15234375)(401.03867187,453.8515625)
\curveto(401.07773437,453.65625)(401.09726562,453.31445312)(401.09726562,452.82617188)
\lineto(401.09726562,449)
\lineto(400.04257812,449)
\lineto(400.04257812,452.78515625)
\curveto(400.04257812,453.21484375)(400.0015625,453.53515625)(399.91953125,453.74609375)
\curveto(399.8375,453.9609375)(399.69101562,454.13085938)(399.48007812,454.25585938)
\curveto(399.27304687,454.38476562)(399.02890625,454.44921875)(398.74765625,454.44921875)
\curveto(398.2984375,454.44921875)(397.90976562,454.30664062)(397.58164062,454.02148438)
\curveto(397.25742187,453.73632812)(397.0953125,453.1953125)(397.0953125,452.3984375)
\lineto(397.0953125,449)
\closepath
}
}
{
\newrgbcolor{curcolor}{0 0 0}
\pscustom[linestyle=none,fillstyle=solid,fillcolor=curcolor]
{
\newpath
\moveto(406.775,451.27929688)
\lineto(407.81210937,451.14453125)
\curveto(407.69882812,450.4296875)(407.4078125,449.86914062)(406.9390625,449.46289062)
\curveto(406.47421875,449.06054688)(405.90195312,448.859375)(405.22226562,448.859375)
\curveto(404.37070312,448.859375)(403.68515625,449.13671875)(403.165625,449.69140625)
\curveto(402.65,450.25)(402.3921875,451.04882812)(402.3921875,452.08789062)
\curveto(402.3921875,452.75976562)(402.50351562,453.34765625)(402.72617187,453.8515625)
\curveto(402.94882812,454.35546875)(403.28671875,454.73242188)(403.73984375,454.98242188)
\curveto(404.196875,455.23632812)(404.69296875,455.36328125)(405.228125,455.36328125)
\curveto(405.90390625,455.36328125)(406.45664062,455.19140625)(406.88632812,454.84765625)
\curveto(407.31601562,454.5078125)(407.59140625,454.0234375)(407.7125,453.39453125)
\lineto(406.68710937,453.23632812)
\curveto(406.58945312,453.65429688)(406.415625,453.96875)(406.165625,454.1796875)
\curveto(405.91953125,454.390625)(405.62070312,454.49609375)(405.26914062,454.49609375)
\curveto(404.73789062,454.49609375)(404.30625,454.3046875)(403.97421875,453.921875)
\curveto(403.6421875,453.54296875)(403.47617187,452.94140625)(403.47617187,452.1171875)
\curveto(403.47617187,451.28125)(403.63632812,450.67382812)(403.95664062,450.29492188)
\curveto(404.27695312,449.91601562)(404.69492187,449.7265625)(405.21054687,449.7265625)
\curveto(405.62460937,449.7265625)(405.9703125,449.85351562)(406.24765625,450.10742188)
\curveto(406.525,450.36132812)(406.70078125,450.75195312)(406.775,451.27929688)
\closepath
}
}
{
\newrgbcolor{curcolor}{0 0 0}
\pscustom[linestyle=none,fillstyle=solid,fillcolor=curcolor]
{
\newpath
\moveto(408.7203125,456.37695312)
\lineto(408.7203125,457.58984375)
\lineto(409.775,457.58984375)
\lineto(409.775,456.37695312)
\closepath
\moveto(408.7203125,449)
\lineto(408.7203125,455.22265625)
\lineto(409.775,455.22265625)
\lineto(409.775,449)
\closepath
}
}
{
\newrgbcolor{curcolor}{0 0 0}
\pscustom[linestyle=none,fillstyle=solid,fillcolor=curcolor]
{
\newpath
\moveto(411.38046875,449)
\lineto(411.38046875,455.22265625)
\lineto(412.3296875,455.22265625)
\lineto(412.3296875,454.33789062)
\curveto(412.78671875,455.02148438)(413.446875,455.36328125)(414.31015625,455.36328125)
\curveto(414.68515625,455.36328125)(415.02890625,455.29492188)(415.34140625,455.15820312)
\curveto(415.6578125,455.02539062)(415.89414062,454.84960938)(416.05039062,454.63085938)
\curveto(416.20664062,454.41210938)(416.31601562,454.15234375)(416.37851562,453.8515625)
\curveto(416.41757812,453.65625)(416.43710937,453.31445312)(416.43710937,452.82617188)
\lineto(416.43710937,449)
\lineto(415.38242187,449)
\lineto(415.38242187,452.78515625)
\curveto(415.38242187,453.21484375)(415.34140625,453.53515625)(415.259375,453.74609375)
\curveto(415.17734375,453.9609375)(415.03085937,454.13085938)(414.81992187,454.25585938)
\curveto(414.61289062,454.38476562)(414.36875,454.44921875)(414.0875,454.44921875)
\curveto(413.63828125,454.44921875)(413.24960937,454.30664062)(412.92148437,454.02148438)
\curveto(412.59726562,453.73632812)(412.43515625,453.1953125)(412.43515625,452.3984375)
\lineto(412.43515625,449)
\closepath
}
}
{
\newrgbcolor{curcolor}{0 0 0}
\pscustom[linestyle=none,fillstyle=solid,fillcolor=curcolor]
{
\newpath
\moveto(417.8609375,448.484375)
\lineto(418.88632812,448.33203125)
\curveto(418.92929687,448.015625)(419.0484375,447.78515625)(419.24375,447.640625)
\curveto(419.50546875,447.4453125)(419.86289062,447.34765625)(420.31601562,447.34765625)
\curveto(420.80429687,447.34765625)(421.18125,447.4453125)(421.446875,447.640625)
\curveto(421.7125,447.8359375)(421.8921875,448.109375)(421.9859375,448.4609375)
\curveto(422.040625,448.67578125)(422.06601562,449.12695312)(422.06210937,449.81445312)
\curveto(421.60117187,449.27148438)(421.02695312,449)(420.33945312,449)
\curveto(419.48398437,449)(418.821875,449.30859375)(418.353125,449.92578125)
\curveto(417.884375,450.54296875)(417.65,451.28320312)(417.65,452.14648438)
\curveto(417.65,452.74023438)(417.75742187,453.28710938)(417.97226562,453.78710938)
\curveto(418.18710937,454.29101562)(418.49765625,454.6796875)(418.90390625,454.953125)
\curveto(419.3140625,455.2265625)(419.79453125,455.36328125)(420.3453125,455.36328125)
\curveto(421.0796875,455.36328125)(421.68515625,455.06640625)(422.16171875,454.47265625)
\lineto(422.16171875,455.22265625)
\lineto(423.134375,455.22265625)
\lineto(423.134375,449.84375)
\curveto(423.134375,448.875)(423.03476562,448.18945312)(422.83554687,447.78710938)
\curveto(422.64023437,447.38085938)(422.32773437,447.06054688)(421.89804687,446.82617188)
\curveto(421.47226562,446.59179688)(420.946875,446.47460938)(420.321875,446.47460938)
\curveto(419.5796875,446.47460938)(418.98007812,446.64257812)(418.52304687,446.97851562)
\curveto(418.06601562,447.31054688)(417.8453125,447.8125)(417.8609375,448.484375)
\closepath
\moveto(418.73398437,452.22265625)
\curveto(418.73398437,451.40625)(418.89609375,450.81054688)(419.2203125,450.43554688)
\curveto(419.54453125,450.06054688)(419.95078125,449.87304688)(420.4390625,449.87304688)
\curveto(420.9234375,449.87304688)(421.3296875,450.05859375)(421.6578125,450.4296875)
\curveto(421.9859375,450.8046875)(422.15,451.390625)(422.15,452.1875)
\curveto(422.15,452.94921875)(421.98007812,453.5234375)(421.64023437,453.91015625)
\curveto(421.30429687,454.296875)(420.89804687,454.49023438)(420.42148437,454.49023438)
\curveto(419.95273437,454.49023438)(419.55429687,454.29882812)(419.22617187,453.91601562)
\curveto(418.89804687,453.53710938)(418.73398437,452.97265625)(418.73398437,452.22265625)
\closepath
}
}
{
\newrgbcolor{curcolor}{0 0 0}
\pscustom[linestyle=none,fillstyle=solid,fillcolor=curcolor]
{
\newpath
\moveto(431.74179687,449)
\lineto(430.68710937,449)
\lineto(430.68710937,455.72070312)
\curveto(430.43320312,455.47851562)(430.09921875,455.23632812)(429.68515625,454.99414062)
\curveto(429.275,454.75195312)(428.90585937,454.5703125)(428.57773437,454.44921875)
\lineto(428.57773437,455.46875)
\curveto(429.16757812,455.74609375)(429.68320312,456.08203125)(430.12460937,456.4765625)
\curveto(430.56601562,456.87109375)(430.87851562,457.25390625)(431.06210937,457.625)
\lineto(431.74179687,457.625)
\closepath
}
}
{
\newrgbcolor{curcolor}{0 0 0}
\pscustom[linestyle=none,fillstyle=solid,fillcolor=curcolor]
{
\newpath
\moveto(434.44296875,453.23632812)
\curveto(434.44296875,454.25195312)(434.54648437,455.06835938)(434.75351562,455.68554688)
\curveto(434.96445312,456.30664062)(435.275,456.78515625)(435.68515625,457.12109375)
\curveto(436.09921875,457.45703125)(436.61875,457.625)(437.24375,457.625)
\curveto(437.7046875,457.625)(438.10898437,457.53125)(438.45664062,457.34375)
\curveto(438.80429687,457.16015625)(439.09140625,456.89257812)(439.31796875,456.54101562)
\curveto(439.54453125,456.19335938)(439.72226562,455.76757812)(439.85117187,455.26367188)
\curveto(439.98007812,454.76367188)(440.04453125,454.08789062)(440.04453125,453.23632812)
\curveto(440.04453125,452.22851562)(439.94101562,451.4140625)(439.73398437,450.79296875)
\curveto(439.52695312,450.17578125)(439.21640625,449.69726562)(438.80234375,449.35742188)
\curveto(438.3921875,449.02148438)(437.87265625,448.85351562)(437.24375,448.85351562)
\curveto(436.415625,448.85351562)(435.76523437,449.15039062)(435.29257812,449.74414062)
\curveto(434.72617187,450.45898438)(434.44296875,451.62304688)(434.44296875,453.23632812)
\closepath
\moveto(435.52695312,453.23632812)
\curveto(435.52695312,451.82617188)(435.69101562,450.88671875)(436.01914062,450.41796875)
\curveto(436.35117187,449.953125)(436.759375,449.72070312)(437.24375,449.72070312)
\curveto(437.728125,449.72070312)(438.134375,449.95507812)(438.4625,450.42382812)
\curveto(438.79453125,450.89257812)(438.96054687,451.83007812)(438.96054687,453.23632812)
\curveto(438.96054687,454.65039062)(438.79453125,455.58984375)(438.4625,456.0546875)
\curveto(438.134375,456.51953125)(437.72421875,456.75195312)(437.23203125,456.75195312)
\curveto(436.74765625,456.75195312)(436.3609375,456.546875)(436.071875,456.13671875)
\curveto(435.70859375,455.61328125)(435.52695312,454.64648438)(435.52695312,453.23632812)
\closepath
}
}
{
\newrgbcolor{curcolor}{0 0 0}
\pscustom[linestyle=none,fillstyle=solid,fillcolor=curcolor]
{
\newpath
\moveto(441.11679687,453.23632812)
\curveto(441.11679687,454.25195312)(441.2203125,455.06835938)(441.42734375,455.68554688)
\curveto(441.63828125,456.30664062)(441.94882812,456.78515625)(442.35898437,457.12109375)
\curveto(442.77304687,457.45703125)(443.29257812,457.625)(443.91757812,457.625)
\curveto(444.37851562,457.625)(444.7828125,457.53125)(445.13046875,457.34375)
\curveto(445.478125,457.16015625)(445.76523437,456.89257812)(445.99179687,456.54101562)
\curveto(446.21835937,456.19335938)(446.39609375,455.76757812)(446.525,455.26367188)
\curveto(446.65390625,454.76367188)(446.71835937,454.08789062)(446.71835937,453.23632812)
\curveto(446.71835937,452.22851562)(446.61484375,451.4140625)(446.4078125,450.79296875)
\curveto(446.20078125,450.17578125)(445.89023437,449.69726562)(445.47617187,449.35742188)
\curveto(445.06601562,449.02148438)(444.54648437,448.85351562)(443.91757812,448.85351562)
\curveto(443.08945312,448.85351562)(442.4390625,449.15039062)(441.96640625,449.74414062)
\curveto(441.4,450.45898438)(441.11679687,451.62304688)(441.11679687,453.23632812)
\closepath
\moveto(442.20078125,453.23632812)
\curveto(442.20078125,451.82617188)(442.36484375,450.88671875)(442.69296875,450.41796875)
\curveto(443.025,449.953125)(443.43320312,449.72070312)(443.91757812,449.72070312)
\curveto(444.40195312,449.72070312)(444.80820312,449.95507812)(445.13632812,450.42382812)
\curveto(445.46835937,450.89257812)(445.634375,451.83007812)(445.634375,453.23632812)
\curveto(445.634375,454.65039062)(445.46835937,455.58984375)(445.13632812,456.0546875)
\curveto(444.80820312,456.51953125)(444.39804687,456.75195312)(443.90585937,456.75195312)
\curveto(443.42148437,456.75195312)(443.03476562,456.546875)(442.74570312,456.13671875)
\curveto(442.38242187,455.61328125)(442.20078125,454.64648438)(442.20078125,453.23632812)
\closepath
}
}
{
\newrgbcolor{curcolor}{0 0 0}
\pscustom[linestyle=none,fillstyle=solid,fillcolor=curcolor]
{
\newpath
\moveto(451.66953125,449)
\lineto(451.66953125,454.40234375)
\lineto(450.73789062,454.40234375)
\lineto(450.73789062,455.22265625)
\lineto(451.66953125,455.22265625)
\lineto(451.66953125,455.88476562)
\curveto(451.66953125,456.30273438)(451.70664062,456.61328125)(451.78085937,456.81640625)
\curveto(451.88242187,457.08984375)(452.06015625,457.31054688)(452.3140625,457.47851562)
\curveto(452.571875,457.65039062)(452.93125,457.73632812)(453.3921875,457.73632812)
\curveto(453.6890625,457.73632812)(454.0171875,457.70117188)(454.3765625,457.63085938)
\lineto(454.21835937,456.7109375)
\curveto(453.99960937,456.75)(453.79257812,456.76953125)(453.59726562,456.76953125)
\curveto(453.27695312,456.76953125)(453.05039062,456.70117188)(452.91757812,456.56445312)
\curveto(452.78476562,456.42773438)(452.71835937,456.171875)(452.71835937,455.796875)
\lineto(452.71835937,455.22265625)
\lineto(453.93125,455.22265625)
\lineto(453.93125,454.40234375)
\lineto(452.71835937,454.40234375)
\lineto(452.71835937,449)
\closepath
}
}
{
\newrgbcolor{curcolor}{0 0 0}
\pscustom[linestyle=none,fillstyle=solid,fillcolor=curcolor]
{
\newpath
\moveto(454.75742187,456.37695312)
\lineto(454.75742187,457.58984375)
\lineto(455.81210937,457.58984375)
\lineto(455.81210937,456.37695312)
\closepath
\moveto(454.75742187,449)
\lineto(454.75742187,455.22265625)
\lineto(455.81210937,455.22265625)
\lineto(455.81210937,449)
\closepath
}
}
{
\newrgbcolor{curcolor}{0 0 0}
\pscustom[linestyle=none,fillstyle=solid,fillcolor=curcolor]
{
\newpath
\moveto(457.025,452.11132812)
\curveto(457.025,453.26367188)(457.3453125,454.1171875)(457.9859375,454.671875)
\curveto(458.52109375,455.1328125)(459.1734375,455.36328125)(459.94296875,455.36328125)
\curveto(460.7984375,455.36328125)(461.49765625,455.08203125)(462.040625,454.51953125)
\curveto(462.58359375,453.9609375)(462.85507812,453.1875)(462.85507812,452.19921875)
\curveto(462.85507812,451.3984375)(462.73398437,450.76757812)(462.49179687,450.30664062)
\curveto(462.25351562,449.84960938)(461.90390625,449.49414062)(461.44296875,449.24023438)
\curveto(460.9859375,448.98632812)(460.4859375,448.859375)(459.94296875,448.859375)
\curveto(459.071875,448.859375)(458.36679687,449.13867188)(457.82773437,449.69726562)
\curveto(457.29257812,450.25585938)(457.025,451.06054688)(457.025,452.11132812)
\closepath
\moveto(458.10898437,452.11132812)
\curveto(458.10898437,451.31445312)(458.2828125,450.71679688)(458.63046875,450.31835938)
\curveto(458.978125,449.92382812)(459.415625,449.7265625)(459.94296875,449.7265625)
\curveto(460.46640625,449.7265625)(460.90195312,449.92578125)(461.24960937,450.32421875)
\curveto(461.59726562,450.72265625)(461.77109375,451.33007812)(461.77109375,452.14648438)
\curveto(461.77109375,452.91601562)(461.5953125,453.49804688)(461.24375,453.89257812)
\curveto(460.89609375,454.29101562)(460.4625,454.49023438)(459.94296875,454.49023438)
\curveto(459.415625,454.49023438)(458.978125,454.29296875)(458.63046875,453.8984375)
\curveto(458.2828125,453.50390625)(458.10898437,452.90820312)(458.10898437,452.11132812)
\closepath
}
}
{
\newrgbcolor{curcolor}{0 0 0}
\pscustom[linestyle=none,fillstyle=solid,fillcolor=curcolor]
{
\newpath
\moveto(467.57773437,449)
\lineto(467.57773437,457.58984375)
\lineto(471.38632812,457.58984375)
\curveto(472.15195312,457.58984375)(472.73398437,457.51171875)(473.13242187,457.35546875)
\curveto(473.53085937,457.203125)(473.84921875,456.93164062)(474.0875,456.54101562)
\curveto(474.32578125,456.15039062)(474.44492187,455.71875)(474.44492187,455.24609375)
\curveto(474.44492187,454.63671875)(474.24765625,454.12304688)(473.853125,453.70507812)
\curveto(473.45859375,453.28710938)(472.84921875,453.02148438)(472.025,452.90820312)
\curveto(472.32578125,452.76367188)(472.55429687,452.62109375)(472.71054687,452.48046875)
\curveto(473.04257812,452.17578125)(473.35703125,451.79492188)(473.65390625,451.33789062)
\lineto(475.14804687,449)
\lineto(473.71835937,449)
\lineto(472.58164062,450.78710938)
\curveto(472.24960937,451.30273438)(471.97617187,451.69726562)(471.76132812,451.97070312)
\curveto(471.54648437,452.24414062)(471.353125,452.43554688)(471.18125,452.54492188)
\curveto(471.01328125,452.65429688)(470.84140625,452.73046875)(470.665625,452.7734375)
\curveto(470.53671875,452.80078125)(470.32578125,452.81445312)(470.0328125,452.81445312)
\lineto(468.71445312,452.81445312)
\lineto(468.71445312,449)
\closepath
\moveto(468.71445312,453.79882812)
\lineto(471.1578125,453.79882812)
\curveto(471.67734375,453.79882812)(472.08359375,453.8515625)(472.3765625,453.95703125)
\curveto(472.66953125,454.06640625)(472.8921875,454.23828125)(473.04453125,454.47265625)
\curveto(473.196875,454.7109375)(473.27304687,454.96875)(473.27304687,455.24609375)
\curveto(473.27304687,455.65234375)(473.12460937,455.98632812)(472.82773437,456.24804688)
\curveto(472.53476562,456.50976562)(472.06992187,456.640625)(471.43320312,456.640625)
\lineto(468.71445312,456.640625)
\closepath
}
}
{
\newrgbcolor{curcolor}{0 0 0}
\pscustom[linestyle=none,fillstyle=solid,fillcolor=curcolor]
{
\newpath
\moveto(480.16953125,449)
\lineto(480.16953125,449.9140625)
\curveto(479.68515625,449.2109375)(479.02695312,448.859375)(478.19492187,448.859375)
\curveto(477.82773437,448.859375)(477.48398437,448.9296875)(477.16367187,449.0703125)
\curveto(476.84726562,449.2109375)(476.6109375,449.38671875)(476.4546875,449.59765625)
\curveto(476.30234375,449.8125)(476.19492187,450.07421875)(476.13242187,450.3828125)
\curveto(476.08945312,450.58984375)(476.06796875,450.91796875)(476.06796875,451.3671875)
\lineto(476.06796875,455.22265625)
\lineto(477.12265625,455.22265625)
\lineto(477.12265625,451.77148438)
\curveto(477.12265625,451.22070312)(477.14414062,450.84960938)(477.18710937,450.65820312)
\curveto(477.25351562,450.38085938)(477.39414062,450.16210938)(477.60898437,450.00195312)
\curveto(477.82382812,449.84570312)(478.08945312,449.76757812)(478.40585937,449.76757812)
\curveto(478.72226562,449.76757812)(479.01914062,449.84765625)(479.29648437,450.0078125)
\curveto(479.57382812,450.171875)(479.76914062,450.39257812)(479.88242187,450.66992188)
\curveto(479.99960937,450.95117188)(480.05820312,451.35742188)(480.05820312,451.88867188)
\lineto(480.05820312,455.22265625)
\lineto(481.11289062,455.22265625)
\lineto(481.11289062,449)
\closepath
}
}
{
\newrgbcolor{curcolor}{0 0 0}
\pscustom[linestyle=none,fillstyle=solid,fillcolor=curcolor]
{
\newpath
\moveto(482.76523437,449)
\lineto(482.76523437,455.22265625)
\lineto(483.71445312,455.22265625)
\lineto(483.71445312,454.33789062)
\curveto(484.17148437,455.02148438)(484.83164062,455.36328125)(485.69492187,455.36328125)
\curveto(486.06992187,455.36328125)(486.41367187,455.29492188)(486.72617187,455.15820312)
\curveto(487.04257812,455.02539062)(487.27890625,454.84960938)(487.43515625,454.63085938)
\curveto(487.59140625,454.41210938)(487.70078125,454.15234375)(487.76328125,453.8515625)
\curveto(487.80234375,453.65625)(487.821875,453.31445312)(487.821875,452.82617188)
\lineto(487.821875,449)
\lineto(486.7671875,449)
\lineto(486.7671875,452.78515625)
\curveto(486.7671875,453.21484375)(486.72617187,453.53515625)(486.64414062,453.74609375)
\curveto(486.56210937,453.9609375)(486.415625,454.13085938)(486.2046875,454.25585938)
\curveto(485.99765625,454.38476562)(485.75351562,454.44921875)(485.47226562,454.44921875)
\curveto(485.02304687,454.44921875)(484.634375,454.30664062)(484.30625,454.02148438)
\curveto(483.98203125,453.73632812)(483.81992187,453.1953125)(483.81992187,452.3984375)
\lineto(483.81992187,449)
\closepath
}
}
{
\newrgbcolor{curcolor}{0 0 0}
\pscustom[linestyle=none,fillstyle=solid,fillcolor=curcolor]
{
\newpath
\moveto(489.0171875,450.85742188)
\lineto(490.06015625,451.02148438)
\curveto(490.11875,450.60351562)(490.28085937,450.28320312)(490.54648437,450.06054688)
\curveto(490.81601562,449.83789062)(491.19101562,449.7265625)(491.67148437,449.7265625)
\curveto(492.15585937,449.7265625)(492.51523437,449.82421875)(492.74960937,450.01953125)
\curveto(492.98398437,450.21875)(493.10117187,450.45117188)(493.10117187,450.71679688)
\curveto(493.10117187,450.95507812)(492.99765625,451.14257812)(492.790625,451.27929688)
\curveto(492.64609375,451.37304688)(492.28671875,451.4921875)(491.7125,451.63671875)
\curveto(490.9390625,451.83203125)(490.40195312,452)(490.10117187,452.140625)
\curveto(489.80429687,452.28515625)(489.57773437,452.48242188)(489.42148437,452.73242188)
\curveto(489.26914062,452.98632812)(489.19296875,453.265625)(489.19296875,453.5703125)
\curveto(489.19296875,453.84765625)(489.25546875,454.10351562)(489.38046875,454.33789062)
\curveto(489.509375,454.57617188)(489.68320312,454.7734375)(489.90195312,454.9296875)
\curveto(490.06601562,455.05078125)(490.28867187,455.15234375)(490.56992187,455.234375)
\curveto(490.85507812,455.3203125)(491.15976562,455.36328125)(491.48398437,455.36328125)
\curveto(491.97226562,455.36328125)(492.4,455.29296875)(492.7671875,455.15234375)
\curveto(493.13828125,455.01171875)(493.41171875,454.8203125)(493.5875,454.578125)
\curveto(493.76328125,454.33984375)(493.884375,454.01953125)(493.95078125,453.6171875)
\lineto(492.91953125,453.4765625)
\curveto(492.87265625,453.796875)(492.7359375,454.046875)(492.509375,454.2265625)
\curveto(492.28671875,454.40625)(491.9703125,454.49609375)(491.56015625,454.49609375)
\curveto(491.07578125,454.49609375)(490.73007812,454.41601562)(490.52304687,454.25585938)
\curveto(490.31601562,454.09570312)(490.2125,453.90820312)(490.2125,453.69335938)
\curveto(490.2125,453.55664062)(490.25546875,453.43359375)(490.34140625,453.32421875)
\curveto(490.42734375,453.2109375)(490.56210937,453.1171875)(490.74570312,453.04296875)
\curveto(490.85117187,453.00390625)(491.16171875,452.9140625)(491.67734375,452.7734375)
\curveto(492.4234375,452.57421875)(492.94296875,452.41015625)(493.2359375,452.28125)
\curveto(493.5328125,452.15625)(493.76523437,451.97265625)(493.93320312,451.73046875)
\curveto(494.10117187,451.48828125)(494.18515625,451.1875)(494.18515625,450.828125)
\curveto(494.18515625,450.4765625)(494.08164062,450.14453125)(493.87460937,449.83203125)
\curveto(493.67148437,449.5234375)(493.3765625,449.28320312)(492.98984375,449.11132812)
\curveto(492.603125,448.94335938)(492.165625,448.859375)(491.67734375,448.859375)
\curveto(490.86875,448.859375)(490.2515625,449.02734375)(489.82578125,449.36328125)
\curveto(489.40390625,449.69921875)(489.134375,450.19726562)(489.0171875,450.85742188)
\closepath
}
}
{
\newrgbcolor{curcolor}{0 0 0}
\pscustom[linestyle=none,fillstyle=solid,fillcolor=curcolor]
{
\newpath
\moveto(500.78867187,446.47460938)
\curveto(500.20664062,447.20898438)(499.71445312,448.06835938)(499.31210937,449.05273438)
\curveto(498.90976562,450.03710938)(498.70859375,451.05664062)(498.70859375,452.11132812)
\curveto(498.70859375,453.04101562)(498.85898437,453.93164062)(499.15976562,454.78320312)
\curveto(499.51132812,455.77148438)(500.05429687,456.75585938)(500.78867187,457.73632812)
\lineto(501.54453125,457.73632812)
\curveto(501.071875,456.92382812)(500.759375,456.34375)(500.60703125,455.99609375)
\curveto(500.36875,455.45703125)(500.18125,454.89453125)(500.04453125,454.30859375)
\curveto(499.8765625,453.578125)(499.79257812,452.84375)(499.79257812,452.10546875)
\curveto(499.79257812,450.2265625)(500.3765625,448.34960938)(501.54453125,446.47460938)
\closepath
}
}
{
\newrgbcolor{curcolor}{0 0 0}
\pscustom[linestyle=none,fillstyle=solid,fillcolor=curcolor]
{
\newpath
\moveto(502.90390625,449)
\lineto(502.90390625,457.58984375)
\lineto(505.86289062,457.58984375)
\curveto(506.53085937,457.58984375)(507.040625,457.54882812)(507.3921875,457.46679688)
\curveto(507.884375,457.35351562)(508.30429687,457.1484375)(508.65195312,456.8515625)
\curveto(509.10507812,456.46875)(509.44296875,455.97851562)(509.665625,455.38085938)
\curveto(509.8921875,454.78710938)(510.00546875,454.10742188)(510.00546875,453.34179688)
\curveto(510.00546875,452.68945312)(509.92929687,452.11132812)(509.77695312,451.60742188)
\curveto(509.62460937,451.10351562)(509.42929687,450.68554688)(509.19101562,450.35351562)
\curveto(508.95273437,450.02539062)(508.69101562,449.765625)(508.40585937,449.57421875)
\curveto(508.12460937,449.38671875)(507.7828125,449.24414062)(507.38046875,449.14648438)
\curveto(506.98203125,449.04882812)(506.52304687,449)(506.00351562,449)
\closepath
\moveto(504.040625,450.01367188)
\lineto(505.87460937,450.01367188)
\curveto(506.44101562,450.01367188)(506.884375,450.06640625)(507.2046875,450.171875)
\curveto(507.52890625,450.27734375)(507.78671875,450.42578125)(507.978125,450.6171875)
\curveto(508.24765625,450.88671875)(508.45664062,451.24804688)(508.60507812,451.70117188)
\curveto(508.75742187,452.15820312)(508.83359375,452.7109375)(508.83359375,453.359375)
\curveto(508.83359375,454.2578125)(508.68515625,454.94726562)(508.38828125,455.42773438)
\curveto(508.0953125,455.91210938)(507.73789062,456.23632812)(507.31601562,456.40039062)
\curveto(507.01132812,456.51757812)(506.52109375,456.57617188)(505.8453125,456.57617188)
\lineto(504.040625,456.57617188)
\closepath
}
}
{
\newrgbcolor{curcolor}{0 0 0}
\pscustom[linestyle=none,fillstyle=solid,fillcolor=curcolor]
{
\newpath
\moveto(515.49570312,449.76757812)
\curveto(515.10507812,449.43554688)(514.728125,449.20117188)(514.36484375,449.06445312)
\curveto(514.00546875,448.92773438)(513.61875,448.859375)(513.2046875,448.859375)
\curveto(512.52109375,448.859375)(511.99570312,449.02539062)(511.62851562,449.35742188)
\curveto(511.26132812,449.69335938)(511.07773437,450.12109375)(511.07773437,450.640625)
\curveto(511.07773437,450.9453125)(511.14609375,451.22265625)(511.2828125,451.47265625)
\curveto(511.4234375,451.7265625)(511.60507812,451.9296875)(511.82773437,452.08203125)
\curveto(512.05429687,452.234375)(512.30820312,452.34960938)(512.58945312,452.42773438)
\curveto(512.79648437,452.48242188)(513.10898437,452.53515625)(513.52695312,452.5859375)
\curveto(514.37851562,452.6875)(515.00546875,452.80859375)(515.4078125,452.94921875)
\curveto(515.41171875,453.09375)(515.41367187,453.18554688)(515.41367187,453.22460938)
\curveto(515.41367187,453.65429688)(515.3140625,453.95703125)(515.11484375,454.1328125)
\curveto(514.8453125,454.37109375)(514.44492187,454.49023438)(513.91367187,454.49023438)
\curveto(513.41757812,454.49023438)(513.05039062,454.40234375)(512.81210937,454.2265625)
\curveto(512.57773437,454.0546875)(512.40390625,453.74804688)(512.290625,453.30664062)
\lineto(511.259375,453.44726562)
\curveto(511.353125,453.88867188)(511.50742187,454.24414062)(511.72226562,454.51367188)
\curveto(511.93710937,454.78710938)(512.24765625,454.99609375)(512.65390625,455.140625)
\curveto(513.06015625,455.2890625)(513.53085937,455.36328125)(514.06601562,455.36328125)
\curveto(514.59726562,455.36328125)(515.02890625,455.30078125)(515.3609375,455.17578125)
\curveto(515.69296875,455.05078125)(515.93710937,454.89257812)(516.09335937,454.70117188)
\curveto(516.24960937,454.51367188)(516.35898437,454.27539062)(516.42148437,453.98632812)
\curveto(516.45664062,453.80664062)(516.47421875,453.48242188)(516.47421875,453.01367188)
\lineto(516.47421875,451.60742188)
\curveto(516.47421875,450.62695312)(516.49570312,450.00585938)(516.53867187,449.74414062)
\curveto(516.58554687,449.48632812)(516.67539062,449.23828125)(516.80820312,449)
\lineto(515.70664062,449)
\curveto(515.59726562,449.21875)(515.52695312,449.47460938)(515.49570312,449.76757812)
\closepath
\moveto(515.4078125,452.12304688)
\curveto(515.025,451.96679688)(514.45078125,451.83398438)(513.68515625,451.72460938)
\curveto(513.2515625,451.66210938)(512.94492187,451.59179688)(512.76523437,451.51367188)
\curveto(512.58554687,451.43554688)(512.446875,451.3203125)(512.34921875,451.16796875)
\curveto(512.2515625,451.01953125)(512.20273437,450.85351562)(512.20273437,450.66992188)
\curveto(512.20273437,450.38867188)(512.30820312,450.15429688)(512.51914062,449.96679688)
\curveto(512.73398437,449.77929688)(513.04648437,449.68554688)(513.45664062,449.68554688)
\curveto(513.86289062,449.68554688)(514.22421875,449.7734375)(514.540625,449.94921875)
\curveto(514.85703125,450.12890625)(515.08945312,450.37304688)(515.23789062,450.68164062)
\curveto(515.35117187,450.91992188)(515.4078125,451.27148438)(515.4078125,451.73632812)
\closepath
}
}
{
\newrgbcolor{curcolor}{0 0 0}
\pscustom[linestyle=none,fillstyle=solid,fillcolor=curcolor]
{
\newpath
\moveto(520.41171875,449.94335938)
\lineto(520.5640625,449.01171875)
\curveto(520.2671875,448.94921875)(520.0015625,448.91796875)(519.7671875,448.91796875)
\curveto(519.384375,448.91796875)(519.0875,448.97851562)(518.8765625,449.09960938)
\curveto(518.665625,449.22070312)(518.5171875,449.37890625)(518.43125,449.57421875)
\curveto(518.3453125,449.7734375)(518.30234375,450.18945312)(518.30234375,450.82226562)
\lineto(518.30234375,454.40234375)
\lineto(517.52890625,454.40234375)
\lineto(517.52890625,455.22265625)
\lineto(518.30234375,455.22265625)
\lineto(518.30234375,456.76367188)
\lineto(519.35117187,457.39648438)
\lineto(519.35117187,455.22265625)
\lineto(520.41171875,455.22265625)
\lineto(520.41171875,454.40234375)
\lineto(519.35117187,454.40234375)
\lineto(519.35117187,450.76367188)
\curveto(519.35117187,450.46289062)(519.36875,450.26953125)(519.40390625,450.18359375)
\curveto(519.44296875,450.09765625)(519.50351562,450.02929688)(519.58554687,449.97851562)
\curveto(519.67148437,449.92773438)(519.79257812,449.90234375)(519.94882812,449.90234375)
\curveto(520.06601562,449.90234375)(520.2203125,449.91601562)(520.41171875,449.94335938)
\closepath
}
}
{
\newrgbcolor{curcolor}{0 0 0}
\pscustom[linestyle=none,fillstyle=solid,fillcolor=curcolor]
{
\newpath
\moveto(525.50351562,449.76757812)
\curveto(525.11289062,449.43554688)(524.7359375,449.20117188)(524.37265625,449.06445312)
\curveto(524.01328125,448.92773438)(523.6265625,448.859375)(523.2125,448.859375)
\curveto(522.52890625,448.859375)(522.00351562,449.02539062)(521.63632812,449.35742188)
\curveto(521.26914062,449.69335938)(521.08554687,450.12109375)(521.08554687,450.640625)
\curveto(521.08554687,450.9453125)(521.15390625,451.22265625)(521.290625,451.47265625)
\curveto(521.43125,451.7265625)(521.61289062,451.9296875)(521.83554687,452.08203125)
\curveto(522.06210937,452.234375)(522.31601562,452.34960938)(522.59726562,452.42773438)
\curveto(522.80429687,452.48242188)(523.11679687,452.53515625)(523.53476562,452.5859375)
\curveto(524.38632812,452.6875)(525.01328125,452.80859375)(525.415625,452.94921875)
\curveto(525.41953125,453.09375)(525.42148437,453.18554688)(525.42148437,453.22460938)
\curveto(525.42148437,453.65429688)(525.321875,453.95703125)(525.12265625,454.1328125)
\curveto(524.853125,454.37109375)(524.45273437,454.49023438)(523.92148437,454.49023438)
\curveto(523.42539062,454.49023438)(523.05820312,454.40234375)(522.81992187,454.2265625)
\curveto(522.58554687,454.0546875)(522.41171875,453.74804688)(522.2984375,453.30664062)
\lineto(521.2671875,453.44726562)
\curveto(521.3609375,453.88867188)(521.51523437,454.24414062)(521.73007812,454.51367188)
\curveto(521.94492187,454.78710938)(522.25546875,454.99609375)(522.66171875,455.140625)
\curveto(523.06796875,455.2890625)(523.53867187,455.36328125)(524.07382812,455.36328125)
\curveto(524.60507812,455.36328125)(525.03671875,455.30078125)(525.36875,455.17578125)
\curveto(525.70078125,455.05078125)(525.94492187,454.89257812)(526.10117187,454.70117188)
\curveto(526.25742187,454.51367188)(526.36679687,454.27539062)(526.42929687,453.98632812)
\curveto(526.46445312,453.80664062)(526.48203125,453.48242188)(526.48203125,453.01367188)
\lineto(526.48203125,451.60742188)
\curveto(526.48203125,450.62695312)(526.50351562,450.00585938)(526.54648437,449.74414062)
\curveto(526.59335937,449.48632812)(526.68320312,449.23828125)(526.81601562,449)
\lineto(525.71445312,449)
\curveto(525.60507812,449.21875)(525.53476562,449.47460938)(525.50351562,449.76757812)
\closepath
\moveto(525.415625,452.12304688)
\curveto(525.0328125,451.96679688)(524.45859375,451.83398438)(523.69296875,451.72460938)
\curveto(523.259375,451.66210938)(522.95273437,451.59179688)(522.77304687,451.51367188)
\curveto(522.59335937,451.43554688)(522.4546875,451.3203125)(522.35703125,451.16796875)
\curveto(522.259375,451.01953125)(522.21054687,450.85351562)(522.21054687,450.66992188)
\curveto(522.21054687,450.38867188)(522.31601562,450.15429688)(522.52695312,449.96679688)
\curveto(522.74179687,449.77929688)(523.05429687,449.68554688)(523.46445312,449.68554688)
\curveto(523.87070312,449.68554688)(524.23203125,449.7734375)(524.5484375,449.94921875)
\curveto(524.86484375,450.12890625)(525.09726562,450.37304688)(525.24570312,450.68164062)
\curveto(525.35898437,450.91992188)(525.415625,451.27148438)(525.415625,451.73632812)
\closepath
}
}
{
\newrgbcolor{curcolor}{0 0 0}
\pscustom[linestyle=none,fillstyle=solid,fillcolor=curcolor]
{
\newpath
\moveto(531.57382812,449)
\lineto(531.57382812,457.58984375)
\lineto(532.73984375,457.58984375)
\lineto(537.2515625,450.84570312)
\lineto(537.2515625,457.58984375)
\lineto(538.34140625,457.58984375)
\lineto(538.34140625,449)
\lineto(537.17539062,449)
\lineto(532.66367187,455.75)
\lineto(532.66367187,449)
\closepath
}
}
{
\newrgbcolor{curcolor}{0 0 0}
\pscustom[linestyle=none,fillstyle=solid,fillcolor=curcolor]
{
\newpath
\moveto(539.72421875,452.11132812)
\curveto(539.72421875,453.26367188)(540.04453125,454.1171875)(540.68515625,454.671875)
\curveto(541.2203125,455.1328125)(541.87265625,455.36328125)(542.6421875,455.36328125)
\curveto(543.49765625,455.36328125)(544.196875,455.08203125)(544.73984375,454.51953125)
\curveto(545.2828125,453.9609375)(545.55429687,453.1875)(545.55429687,452.19921875)
\curveto(545.55429687,451.3984375)(545.43320312,450.76757812)(545.19101562,450.30664062)
\curveto(544.95273437,449.84960938)(544.603125,449.49414062)(544.1421875,449.24023438)
\curveto(543.68515625,448.98632812)(543.18515625,448.859375)(542.6421875,448.859375)
\curveto(541.77109375,448.859375)(541.06601562,449.13867188)(540.52695312,449.69726562)
\curveto(539.99179687,450.25585938)(539.72421875,451.06054688)(539.72421875,452.11132812)
\closepath
\moveto(540.80820312,452.11132812)
\curveto(540.80820312,451.31445312)(540.98203125,450.71679688)(541.3296875,450.31835938)
\curveto(541.67734375,449.92382812)(542.11484375,449.7265625)(542.6421875,449.7265625)
\curveto(543.165625,449.7265625)(543.60117187,449.92578125)(543.94882812,450.32421875)
\curveto(544.29648437,450.72265625)(544.4703125,451.33007812)(544.4703125,452.14648438)
\curveto(544.4703125,452.91601562)(544.29453125,453.49804688)(543.94296875,453.89257812)
\curveto(543.5953125,454.29101562)(543.16171875,454.49023438)(542.6421875,454.49023438)
\curveto(542.11484375,454.49023438)(541.67734375,454.29296875)(541.3296875,453.8984375)
\curveto(540.98203125,453.50390625)(540.80820312,452.90820312)(540.80820312,452.11132812)
\closepath
}
}
{
\newrgbcolor{curcolor}{0 0 0}
\pscustom[linestyle=none,fillstyle=solid,fillcolor=curcolor]
{
\newpath
\moveto(550.82773437,449)
\lineto(550.82773437,449.78515625)
\curveto(550.43320312,449.16796875)(549.853125,448.859375)(549.0875,448.859375)
\curveto(548.59140625,448.859375)(548.134375,448.99609375)(547.71640625,449.26953125)
\curveto(547.30234375,449.54296875)(546.98007812,449.92382812)(546.74960937,450.41210938)
\curveto(546.52304687,450.90429688)(546.40976562,451.46875)(546.40976562,452.10546875)
\curveto(546.40976562,452.7265625)(546.51328125,453.2890625)(546.7203125,453.79296875)
\curveto(546.92734375,454.30078125)(547.23789062,454.68945312)(547.65195312,454.95898438)
\curveto(548.06601562,455.22851562)(548.52890625,455.36328125)(549.040625,455.36328125)
\curveto(549.415625,455.36328125)(549.74960937,455.28320312)(550.04257812,455.12304688)
\curveto(550.33554687,454.96679688)(550.57382812,454.76171875)(550.75742187,454.5078125)
\lineto(550.75742187,457.58984375)
\lineto(551.80625,457.58984375)
\lineto(551.80625,449)
\closepath
\moveto(547.49375,452.10546875)
\curveto(547.49375,451.30859375)(547.66171875,450.71289062)(547.99765625,450.31835938)
\curveto(548.33359375,449.92382812)(548.73007812,449.7265625)(549.18710937,449.7265625)
\curveto(549.64804687,449.7265625)(550.03867187,449.9140625)(550.35898437,450.2890625)
\curveto(550.68320312,450.66796875)(550.8453125,451.24414062)(550.8453125,452.01757812)
\curveto(550.8453125,452.86914062)(550.68125,453.49414062)(550.353125,453.89257812)
\curveto(550.025,454.29101562)(549.62070312,454.49023438)(549.14023437,454.49023438)
\curveto(548.67148437,454.49023438)(548.27890625,454.29882812)(547.9625,453.91601562)
\curveto(547.65,453.53320312)(547.49375,452.9296875)(547.49375,452.10546875)
\closepath
}
}
{
\newrgbcolor{curcolor}{0 0 0}
\pscustom[linestyle=none,fillstyle=solid,fillcolor=curcolor]
{
\newpath
\moveto(557.72421875,451.00390625)
\lineto(558.8140625,450.86914062)
\curveto(558.6421875,450.23242188)(558.32382812,449.73828125)(557.85898437,449.38671875)
\curveto(557.39414062,449.03515625)(556.80039062,448.859375)(556.07773437,448.859375)
\curveto(555.16757812,448.859375)(554.44492187,449.13867188)(553.90976562,449.69726562)
\curveto(553.37851562,450.25976562)(553.11289062,451.046875)(553.11289062,452.05859375)
\curveto(553.11289062,453.10546875)(553.38242187,453.91796875)(553.92148437,454.49609375)
\curveto(554.46054687,455.07421875)(555.15976562,455.36328125)(556.01914062,455.36328125)
\curveto(556.85117187,455.36328125)(557.53085937,455.08007812)(558.05820312,454.51367188)
\curveto(558.58554687,453.94726562)(558.84921875,453.15039062)(558.84921875,452.12304688)
\curveto(558.84921875,452.06054688)(558.84726562,451.96679688)(558.84335937,451.84179688)
\lineto(554.20273437,451.84179688)
\curveto(554.24179687,451.15820312)(554.43515625,450.63476562)(554.7828125,450.27148438)
\curveto(555.13046875,449.90820312)(555.5640625,449.7265625)(556.08359375,449.7265625)
\curveto(556.4703125,449.7265625)(556.80039062,449.828125)(557.07382812,450.03125)
\curveto(557.34726562,450.234375)(557.5640625,450.55859375)(557.72421875,451.00390625)
\closepath
\moveto(554.26132812,452.70898438)
\lineto(557.7359375,452.70898438)
\curveto(557.6890625,453.23242188)(557.55625,453.625)(557.3375,453.88671875)
\curveto(557.0015625,454.29296875)(556.56601562,454.49609375)(556.03085937,454.49609375)
\curveto(555.54648437,454.49609375)(555.13828125,454.33398438)(554.80625,454.00976562)
\curveto(554.478125,453.68554688)(554.29648437,453.25195312)(554.26132812,452.70898438)
\closepath
}
}
{
\newrgbcolor{curcolor}{0 0 0}
\pscustom[linestyle=none,fillstyle=solid,fillcolor=curcolor]
{
\newpath
\moveto(563.17929687,453.23632812)
\curveto(563.17929687,454.25195312)(563.2828125,455.06835938)(563.48984375,455.68554688)
\curveto(563.70078125,456.30664062)(564.01132812,456.78515625)(564.42148437,457.12109375)
\curveto(564.83554687,457.45703125)(565.35507812,457.625)(565.98007812,457.625)
\curveto(566.44101562,457.625)(566.8453125,457.53125)(567.19296875,457.34375)
\curveto(567.540625,457.16015625)(567.82773437,456.89257812)(568.05429687,456.54101562)
\curveto(568.28085937,456.19335938)(568.45859375,455.76757812)(568.5875,455.26367188)
\curveto(568.71640625,454.76367188)(568.78085937,454.08789062)(568.78085937,453.23632812)
\curveto(568.78085937,452.22851562)(568.67734375,451.4140625)(568.4703125,450.79296875)
\curveto(568.26328125,450.17578125)(567.95273437,449.69726562)(567.53867187,449.35742188)
\curveto(567.12851562,449.02148438)(566.60898437,448.85351562)(565.98007812,448.85351562)
\curveto(565.15195312,448.85351562)(564.5015625,449.15039062)(564.02890625,449.74414062)
\curveto(563.4625,450.45898438)(563.17929687,451.62304688)(563.17929687,453.23632812)
\closepath
\moveto(564.26328125,453.23632812)
\curveto(564.26328125,451.82617188)(564.42734375,450.88671875)(564.75546875,450.41796875)
\curveto(565.0875,449.953125)(565.49570312,449.72070312)(565.98007812,449.72070312)
\curveto(566.46445312,449.72070312)(566.87070312,449.95507812)(567.19882812,450.42382812)
\curveto(567.53085937,450.89257812)(567.696875,451.83007812)(567.696875,453.23632812)
\curveto(567.696875,454.65039062)(567.53085937,455.58984375)(567.19882812,456.0546875)
\curveto(566.87070312,456.51953125)(566.46054687,456.75195312)(565.96835937,456.75195312)
\curveto(565.48398437,456.75195312)(565.09726562,456.546875)(564.80820312,456.13671875)
\curveto(564.44492187,455.61328125)(564.26328125,454.64648438)(564.26328125,453.23632812)
\closepath
}
}
{
\newrgbcolor{curcolor}{0 0 0}
\pscustom[linestyle=none,fillstyle=solid,fillcolor=curcolor]
{
\newpath
\moveto(570.8375,446.47460938)
\lineto(570.08164062,446.47460938)
\curveto(571.24960937,448.34960938)(571.83359375,450.2265625)(571.83359375,452.10546875)
\curveto(571.83359375,452.83984375)(571.74960937,453.56835938)(571.58164062,454.29101562)
\curveto(571.44882812,454.87695312)(571.26328125,455.43945312)(571.025,455.97851562)
\curveto(570.87265625,456.33007812)(570.55820312,456.91601562)(570.08164062,457.73632812)
\lineto(570.8375,457.73632812)
\curveto(571.571875,456.75585938)(572.11484375,455.77148438)(572.46640625,454.78320312)
\curveto(572.7671875,453.93164062)(572.91757812,453.04101562)(572.91757812,452.11132812)
\curveto(572.91757812,451.05664062)(572.71445312,450.03710938)(572.30820312,449.05273438)
\curveto(571.90585937,448.06835938)(571.415625,447.20898438)(570.8375,446.47460938)
\closepath
}
}
{
\newrgbcolor{curcolor}{0 0 0}
\pscustom[linestyle=none,fillstyle=solid,fillcolor=curcolor]
{
\newpath
\moveto(158.46992188,406.75976562)
\lineto(159.5421875,406.85351562)
\curveto(159.59296875,406.42382812)(159.71015625,406.0703125)(159.89375,405.79296875)
\curveto(160.08125,405.51953125)(160.3703125,405.296875)(160.7609375,405.125)
\curveto(161.1515625,404.95703125)(161.59101563,404.87304688)(162.07929688,404.87304688)
\curveto(162.51289063,404.87304688)(162.89570313,404.9375)(163.22773438,405.06640625)
\curveto(163.55976563,405.1953125)(163.80585938,405.37109375)(163.96601563,405.59375)
\curveto(164.13007813,405.8203125)(164.21210938,406.06640625)(164.21210938,406.33203125)
\curveto(164.21210938,406.6015625)(164.13398438,406.8359375)(163.97773438,407.03515625)
\curveto(163.82148438,407.23828125)(163.56367188,407.40820312)(163.20429688,407.54492188)
\curveto(162.97382813,407.63476562)(162.4640625,407.7734375)(161.675,407.9609375)
\curveto(160.8859375,408.15234375)(160.33320313,408.33203125)(160.01679688,408.5)
\curveto(159.60664063,408.71484375)(159.3,408.98046875)(159.096875,409.296875)
\curveto(158.89765625,409.6171875)(158.79804688,409.97460938)(158.79804688,410.36914062)
\curveto(158.79804688,410.80273438)(158.92109375,411.20703125)(159.1671875,411.58203125)
\curveto(159.41328125,411.9609375)(159.77265625,412.24804688)(160.2453125,412.44335938)
\curveto(160.71796875,412.63867188)(161.24335938,412.73632812)(161.82148438,412.73632812)
\curveto(162.45820313,412.73632812)(163.01875,412.6328125)(163.503125,412.42578125)
\curveto(163.99140625,412.22265625)(164.36640625,411.921875)(164.628125,411.5234375)
\curveto(164.88984375,411.125)(165.03046875,410.67382812)(165.05,410.16992188)
\lineto(163.96015625,410.08789062)
\curveto(163.9015625,410.63085938)(163.70234375,411.04101562)(163.3625,411.31835938)
\curveto(163.0265625,411.59570312)(162.52851563,411.734375)(161.86835938,411.734375)
\curveto(161.18085938,411.734375)(160.67890625,411.60742188)(160.3625,411.35351562)
\curveto(160.05,411.10351562)(159.89375,410.80078125)(159.89375,410.4453125)
\curveto(159.89375,410.13671875)(160.00507813,409.8828125)(160.22773438,409.68359375)
\curveto(160.44648438,409.484375)(161.01679688,409.27929688)(161.93867188,409.06835938)
\curveto(162.86445313,408.86132812)(163.49921875,408.6796875)(163.84296875,408.5234375)
\curveto(164.34296875,408.29296875)(164.71210938,408)(164.95039063,407.64453125)
\curveto(165.18867188,407.29296875)(165.3078125,406.88671875)(165.3078125,406.42578125)
\curveto(165.3078125,405.96875)(165.17695313,405.53710938)(164.91523438,405.13085938)
\curveto(164.65351563,404.72851562)(164.2765625,404.4140625)(163.784375,404.1875)
\curveto(163.29609375,403.96484375)(162.7453125,403.85351562)(162.13203125,403.85351562)
\curveto(161.3546875,403.85351562)(160.70234375,403.96679688)(160.175,404.19335938)
\curveto(159.6515625,404.41992188)(159.23945313,404.75976562)(158.93867188,405.21289062)
\curveto(158.64179688,405.66992188)(158.48554688,406.18554688)(158.46992188,406.75976562)
\closepath
}
}
{
\newrgbcolor{curcolor}{0 0 0}
\pscustom[linestyle=none,fillstyle=solid,fillcolor=curcolor]
{
\newpath
\moveto(170.78632813,404.76757812)
\curveto(170.39570313,404.43554688)(170.01875,404.20117188)(169.65546875,404.06445312)
\curveto(169.29609375,403.92773438)(168.909375,403.859375)(168.4953125,403.859375)
\curveto(167.81171875,403.859375)(167.28632813,404.02539062)(166.91914063,404.35742188)
\curveto(166.55195313,404.69335938)(166.36835938,405.12109375)(166.36835938,405.640625)
\curveto(166.36835938,405.9453125)(166.43671875,406.22265625)(166.5734375,406.47265625)
\curveto(166.7140625,406.7265625)(166.89570313,406.9296875)(167.11835938,407.08203125)
\curveto(167.34492188,407.234375)(167.59882813,407.34960938)(167.88007813,407.42773438)
\curveto(168.08710938,407.48242188)(168.39960938,407.53515625)(168.81757813,407.5859375)
\curveto(169.66914063,407.6875)(170.29609375,407.80859375)(170.6984375,407.94921875)
\curveto(170.70234375,408.09375)(170.70429688,408.18554688)(170.70429688,408.22460938)
\curveto(170.70429688,408.65429688)(170.6046875,408.95703125)(170.40546875,409.1328125)
\curveto(170.1359375,409.37109375)(169.73554688,409.49023438)(169.20429688,409.49023438)
\curveto(168.70820313,409.49023438)(168.34101563,409.40234375)(168.10273438,409.2265625)
\curveto(167.86835938,409.0546875)(167.69453125,408.74804688)(167.58125,408.30664062)
\lineto(166.55,408.44726562)
\curveto(166.64375,408.88867188)(166.79804688,409.24414062)(167.01289063,409.51367188)
\curveto(167.22773438,409.78710938)(167.53828125,409.99609375)(167.94453125,410.140625)
\curveto(168.35078125,410.2890625)(168.82148438,410.36328125)(169.35664063,410.36328125)
\curveto(169.88789063,410.36328125)(170.31953125,410.30078125)(170.6515625,410.17578125)
\curveto(170.98359375,410.05078125)(171.22773438,409.89257812)(171.38398438,409.70117188)
\curveto(171.54023438,409.51367188)(171.64960938,409.27539062)(171.71210938,408.98632812)
\curveto(171.74726563,408.80664062)(171.76484375,408.48242188)(171.76484375,408.01367188)
\lineto(171.76484375,406.60742188)
\curveto(171.76484375,405.62695312)(171.78632813,405.00585938)(171.82929688,404.74414062)
\curveto(171.87617188,404.48632812)(171.96601563,404.23828125)(172.09882813,404)
\lineto(170.99726563,404)
\curveto(170.88789063,404.21875)(170.81757813,404.47460938)(170.78632813,404.76757812)
\closepath
\moveto(170.6984375,407.12304688)
\curveto(170.315625,406.96679688)(169.74140625,406.83398438)(168.97578125,406.72460938)
\curveto(168.5421875,406.66210938)(168.23554688,406.59179688)(168.05585938,406.51367188)
\curveto(167.87617188,406.43554688)(167.7375,406.3203125)(167.63984375,406.16796875)
\curveto(167.5421875,406.01953125)(167.49335938,405.85351562)(167.49335938,405.66992188)
\curveto(167.49335938,405.38867188)(167.59882813,405.15429688)(167.80976563,404.96679688)
\curveto(168.02460938,404.77929688)(168.33710938,404.68554688)(168.74726563,404.68554688)
\curveto(169.15351563,404.68554688)(169.51484375,404.7734375)(169.83125,404.94921875)
\curveto(170.14765625,405.12890625)(170.38007813,405.37304688)(170.52851563,405.68164062)
\curveto(170.64179688,405.91992188)(170.6984375,406.27148438)(170.6984375,406.73632812)
\closepath
}
}
{
\newrgbcolor{curcolor}{0 0 0}
\pscustom[linestyle=none,fillstyle=solid,fillcolor=curcolor]
{
\newpath
\moveto(173.39960938,404)
\lineto(173.39960938,410.22265625)
\lineto(174.34296875,410.22265625)
\lineto(174.34296875,409.34960938)
\curveto(174.53828125,409.65429688)(174.79804688,409.8984375)(175.12226563,410.08203125)
\curveto(175.44648438,410.26953125)(175.815625,410.36328125)(176.2296875,410.36328125)
\curveto(176.690625,410.36328125)(177.06757813,410.26757812)(177.36054688,410.07617188)
\curveto(177.65742188,409.88476562)(177.86640625,409.6171875)(177.9875,409.2734375)
\curveto(178.4796875,410)(179.1203125,410.36328125)(179.909375,410.36328125)
\curveto(180.5265625,410.36328125)(181.00117188,410.19140625)(181.33320313,409.84765625)
\curveto(181.66523438,409.5078125)(181.83125,408.98242188)(181.83125,408.27148438)
\lineto(181.83125,404)
\lineto(180.78242188,404)
\lineto(180.78242188,407.91992188)
\curveto(180.78242188,408.34179688)(180.74726563,408.64453125)(180.67695313,408.828125)
\curveto(180.61054688,409.015625)(180.4875,409.16601562)(180.3078125,409.27929688)
\curveto(180.128125,409.39257812)(179.9171875,409.44921875)(179.675,409.44921875)
\curveto(179.2375,409.44921875)(178.87421875,409.30273438)(178.58515625,409.00976562)
\curveto(178.29609375,408.72070312)(178.1515625,408.25585938)(178.1515625,407.61523438)
\lineto(178.1515625,404)
\lineto(177.096875,404)
\lineto(177.096875,408.04296875)
\curveto(177.096875,408.51171875)(177.0109375,408.86328125)(176.8390625,409.09765625)
\curveto(176.6671875,409.33203125)(176.3859375,409.44921875)(175.9953125,409.44921875)
\curveto(175.6984375,409.44921875)(175.42304688,409.37109375)(175.16914063,409.21484375)
\curveto(174.91914063,409.05859375)(174.7375,408.83007812)(174.62421875,408.52929688)
\curveto(174.5109375,408.22851562)(174.45429688,407.79492188)(174.45429688,407.22851562)
\lineto(174.45429688,404)
\closepath
}
}
{
\newrgbcolor{curcolor}{0 0 0}
\pscustom[linestyle=none,fillstyle=solid,fillcolor=curcolor]
{
\newpath
\moveto(187.65546875,406.00390625)
\lineto(188.7453125,405.86914062)
\curveto(188.5734375,405.23242188)(188.25507813,404.73828125)(187.79023438,404.38671875)
\curveto(187.32539063,404.03515625)(186.73164063,403.859375)(186.00898438,403.859375)
\curveto(185.09882813,403.859375)(184.37617188,404.13867188)(183.84101563,404.69726562)
\curveto(183.30976563,405.25976562)(183.04414063,406.046875)(183.04414063,407.05859375)
\curveto(183.04414063,408.10546875)(183.31367188,408.91796875)(183.85273438,409.49609375)
\curveto(184.39179688,410.07421875)(185.09101563,410.36328125)(185.95039063,410.36328125)
\curveto(186.78242188,410.36328125)(187.46210938,410.08007812)(187.98945313,409.51367188)
\curveto(188.51679688,408.94726562)(188.78046875,408.15039062)(188.78046875,407.12304688)
\curveto(188.78046875,407.06054688)(188.77851563,406.96679688)(188.77460938,406.84179688)
\lineto(184.13398438,406.84179688)
\curveto(184.17304688,406.15820312)(184.36640625,405.63476562)(184.7140625,405.27148438)
\curveto(185.06171875,404.90820312)(185.4953125,404.7265625)(186.01484375,404.7265625)
\curveto(186.4015625,404.7265625)(186.73164063,404.828125)(187.00507813,405.03125)
\curveto(187.27851563,405.234375)(187.4953125,405.55859375)(187.65546875,406.00390625)
\closepath
\moveto(184.19257813,407.70898438)
\lineto(187.6671875,407.70898438)
\curveto(187.6203125,408.23242188)(187.4875,408.625)(187.26875,408.88671875)
\curveto(186.9328125,409.29296875)(186.49726563,409.49609375)(185.96210938,409.49609375)
\curveto(185.47773438,409.49609375)(185.06953125,409.33398438)(184.7375,409.00976562)
\curveto(184.409375,408.68554688)(184.22773438,408.25195312)(184.19257813,407.70898438)
\closepath
}
}
{
\newrgbcolor{curcolor}{0 0 0}
\pscustom[linestyle=none,fillstyle=solid,fillcolor=curcolor]
{
\newpath
\moveto(193.5265625,404)
\lineto(193.5265625,412.58984375)
\lineto(194.69257813,412.58984375)
\lineto(199.20429688,405.84570312)
\lineto(199.20429688,412.58984375)
\lineto(200.29414063,412.58984375)
\lineto(200.29414063,404)
\lineto(199.128125,404)
\lineto(194.61640625,410.75)
\lineto(194.61640625,404)
\closepath
}
}
{
\newrgbcolor{curcolor}{0 0 0}
\pscustom[linestyle=none,fillstyle=solid,fillcolor=curcolor]
{
\newpath
\moveto(201.67695313,407.11132812)
\curveto(201.67695313,408.26367188)(201.99726563,409.1171875)(202.63789063,409.671875)
\curveto(203.17304688,410.1328125)(203.82539063,410.36328125)(204.59492188,410.36328125)
\curveto(205.45039063,410.36328125)(206.14960938,410.08203125)(206.69257813,409.51953125)
\curveto(207.23554688,408.9609375)(207.50703125,408.1875)(207.50703125,407.19921875)
\curveto(207.50703125,406.3984375)(207.3859375,405.76757812)(207.14375,405.30664062)
\curveto(206.90546875,404.84960938)(206.55585938,404.49414062)(206.09492188,404.24023438)
\curveto(205.63789063,403.98632812)(205.13789063,403.859375)(204.59492188,403.859375)
\curveto(203.72382813,403.859375)(203.01875,404.13867188)(202.4796875,404.69726562)
\curveto(201.94453125,405.25585938)(201.67695313,406.06054688)(201.67695313,407.11132812)
\closepath
\moveto(202.7609375,407.11132812)
\curveto(202.7609375,406.31445312)(202.93476563,405.71679688)(203.28242188,405.31835938)
\curveto(203.63007813,404.92382812)(204.06757813,404.7265625)(204.59492188,404.7265625)
\curveto(205.11835938,404.7265625)(205.55390625,404.92578125)(205.9015625,405.32421875)
\curveto(206.24921875,405.72265625)(206.42304688,406.33007812)(206.42304688,407.14648438)
\curveto(206.42304688,407.91601562)(206.24726563,408.49804688)(205.89570313,408.89257812)
\curveto(205.54804688,409.29101562)(205.11445313,409.49023438)(204.59492188,409.49023438)
\curveto(204.06757813,409.49023438)(203.63007813,409.29296875)(203.28242188,408.8984375)
\curveto(202.93476563,408.50390625)(202.7609375,407.90820312)(202.7609375,407.11132812)
\closepath
}
}
{
\newrgbcolor{curcolor}{0 0 0}
\pscustom[linestyle=none,fillstyle=solid,fillcolor=curcolor]
{
\newpath
\moveto(212.78046875,404)
\lineto(212.78046875,404.78515625)
\curveto(212.3859375,404.16796875)(211.80585938,403.859375)(211.04023438,403.859375)
\curveto(210.54414063,403.859375)(210.08710938,403.99609375)(209.66914063,404.26953125)
\curveto(209.25507813,404.54296875)(208.9328125,404.92382812)(208.70234375,405.41210938)
\curveto(208.47578125,405.90429688)(208.3625,406.46875)(208.3625,407.10546875)
\curveto(208.3625,407.7265625)(208.46601563,408.2890625)(208.67304688,408.79296875)
\curveto(208.88007813,409.30078125)(209.190625,409.68945312)(209.6046875,409.95898438)
\curveto(210.01875,410.22851562)(210.48164063,410.36328125)(210.99335938,410.36328125)
\curveto(211.36835938,410.36328125)(211.70234375,410.28320312)(211.9953125,410.12304688)
\curveto(212.28828125,409.96679688)(212.5265625,409.76171875)(212.71015625,409.5078125)
\lineto(212.71015625,412.58984375)
\lineto(213.75898438,412.58984375)
\lineto(213.75898438,404)
\closepath
\moveto(209.44648438,407.10546875)
\curveto(209.44648438,406.30859375)(209.61445313,405.71289062)(209.95039063,405.31835938)
\curveto(210.28632813,404.92382812)(210.6828125,404.7265625)(211.13984375,404.7265625)
\curveto(211.60078125,404.7265625)(211.99140625,404.9140625)(212.31171875,405.2890625)
\curveto(212.6359375,405.66796875)(212.79804688,406.24414062)(212.79804688,407.01757812)
\curveto(212.79804688,407.86914062)(212.63398438,408.49414062)(212.30585938,408.89257812)
\curveto(211.97773438,409.29101562)(211.5734375,409.49023438)(211.09296875,409.49023438)
\curveto(210.62421875,409.49023438)(210.23164063,409.29882812)(209.91523438,408.91601562)
\curveto(209.60273438,408.53320312)(209.44648438,407.9296875)(209.44648438,407.10546875)
\closepath
}
}
{
\newrgbcolor{curcolor}{0 0 0}
\pscustom[linestyle=none,fillstyle=solid,fillcolor=curcolor]
{
\newpath
\moveto(219.67695313,406.00390625)
\lineto(220.76679688,405.86914062)
\curveto(220.59492188,405.23242188)(220.2765625,404.73828125)(219.81171875,404.38671875)
\curveto(219.346875,404.03515625)(218.753125,403.859375)(218.03046875,403.859375)
\curveto(217.1203125,403.859375)(216.39765625,404.13867188)(215.8625,404.69726562)
\curveto(215.33125,405.25976562)(215.065625,406.046875)(215.065625,407.05859375)
\curveto(215.065625,408.10546875)(215.33515625,408.91796875)(215.87421875,409.49609375)
\curveto(216.41328125,410.07421875)(217.1125,410.36328125)(217.971875,410.36328125)
\curveto(218.80390625,410.36328125)(219.48359375,410.08007812)(220.0109375,409.51367188)
\curveto(220.53828125,408.94726562)(220.80195313,408.15039062)(220.80195313,407.12304688)
\curveto(220.80195313,407.06054688)(220.8,406.96679688)(220.79609375,406.84179688)
\lineto(216.15546875,406.84179688)
\curveto(216.19453125,406.15820312)(216.38789063,405.63476562)(216.73554688,405.27148438)
\curveto(217.08320313,404.90820312)(217.51679688,404.7265625)(218.03632813,404.7265625)
\curveto(218.42304688,404.7265625)(218.753125,404.828125)(219.0265625,405.03125)
\curveto(219.3,405.234375)(219.51679688,405.55859375)(219.67695313,406.00390625)
\closepath
\moveto(216.2140625,407.70898438)
\lineto(219.68867188,407.70898438)
\curveto(219.64179688,408.23242188)(219.50898438,408.625)(219.29023438,408.88671875)
\curveto(218.95429688,409.29296875)(218.51875,409.49609375)(217.98359375,409.49609375)
\curveto(217.49921875,409.49609375)(217.09101563,409.33398438)(216.75898438,409.00976562)
\curveto(216.43085938,408.68554688)(216.24921875,408.25195312)(216.2140625,407.70898438)
\closepath
}
}
{
\newrgbcolor{curcolor}{0 0 1}
\pscustom[linewidth=1,linecolor=curcolor]
{
\newpath
\moveto(229.6,407.9)
\lineto(271.8,407.9)
\moveto(96.8,223.8)
\lineto(101.6,223.8)
\lineto(106.4,223.8)
\lineto(111.1,223.8)
\lineto(115.9,223.8)
\lineto(120.7,223.8)
\lineto(125.5,223.8)
\lineto(130.3,223.8)
\lineto(135.1,223.8)
\lineto(139.8,223.8)
\lineto(144.6,223.8)
\lineto(149.4,223.8)
\lineto(154.2,223.8)
\lineto(159,223.8)
\lineto(163.7,223.8)
\lineto(168.5,223.8)
\lineto(173.3,223.8)
\lineto(178.1,223.8)
\lineto(182.9,223.8)
\lineto(187.7,223.8)
\lineto(192.4,223.8)
\lineto(197.2,223.8)
\lineto(202,223.8)
\lineto(206.8,223.8)
\lineto(211.6,223.8)
\lineto(216.4,223.8)
\lineto(221.1,223.8)
\lineto(225.9,223.8)
\lineto(230.7,223.8)
\lineto(235.5,223.8)
\lineto(240.3,223.8)
\lineto(245,223.8)
\lineto(249.8,223.8)
\lineto(254.6,223.8)
\lineto(259.4,223.8)
\lineto(264.2,223.8)
\lineto(269,223.8)
\lineto(273.7,223.8)
\lineto(278.5,223.8)
\lineto(283.3,223.8)
\lineto(288.1,223.8)
\lineto(292.9,223.8)
\lineto(297.6,223.8)
\lineto(302.4,223.8)
\lineto(307.2,223.8)
\lineto(312,223.8)
\lineto(316.8,223.8)
\lineto(321.6,223.8)
\lineto(326.3,223.8)
\lineto(331.1,223.8)
\lineto(335.9,223.8)
\lineto(340.7,223.8)
\lineto(345.5,223.8)
\lineto(350.2,223.8)
\lineto(355,223.8)
\lineto(359.8,223.8)
\lineto(364.6,223.8)
\lineto(369.4,223.8)
\lineto(374.2,223.8)
\lineto(378.9,223.8)
\lineto(383.7,223.8)
\lineto(388.5,223.8)
\lineto(393.3,223.8)
\lineto(398.1,223.8)
\lineto(402.8,223.8)
\lineto(407.6,223.8)
\lineto(412.4,223.8)
\lineto(417.2,223.8)
\lineto(422,223.8)
\lineto(426.8,223.8)
\lineto(431.5,223.8)
\lineto(436.3,223.8)
\lineto(441.1,223.8)
\lineto(445.9,223.8)
\lineto(450.7,223.8)
\lineto(455.5,223.8)
\lineto(460.2,223.8)
\lineto(465,223.8)
\lineto(469.8,223.8)
\lineto(474.6,223.8)
\lineto(479.4,223.8)
\lineto(484.1,223.8)
\lineto(488.9,223.8)
\lineto(493.7,223.8)
\lineto(498.5,223.8)
\lineto(503.3,223.8)
\lineto(508.1,223.8)
\lineto(512.8,223.8)
\lineto(517.6,223.8)
\lineto(522.4,223.8)
\lineto(527.2,223.8)
\lineto(532,223.8)
\lineto(536.7,223.8)
\lineto(541.5,223.8)
\lineto(546.3,223.8)
\lineto(551.1,223.8)
\lineto(555.9,223.8)
\lineto(560.7,223.8)
\lineto(565.4,223.8)
}
}
{
\newrgbcolor{curcolor}{0 0 0}
\pscustom[linestyle=none,fillstyle=solid,fillcolor=curcolor]
{
\newpath
\moveto(145.06953125,386)
\lineto(145.06953125,394.58984375)
\lineto(148.02851563,394.58984375)
\curveto(148.69648438,394.58984375)(149.20625,394.54882812)(149.5578125,394.46679688)
\curveto(150.05,394.35351562)(150.46992188,394.1484375)(150.81757813,393.8515625)
\curveto(151.27070313,393.46875)(151.60859375,392.97851562)(151.83125,392.38085938)
\curveto(152.0578125,391.78710938)(152.17109375,391.10742188)(152.17109375,390.34179688)
\curveto(152.17109375,389.68945312)(152.09492188,389.11132812)(151.94257813,388.60742188)
\curveto(151.79023438,388.10351562)(151.59492188,387.68554688)(151.35664063,387.35351562)
\curveto(151.11835938,387.02539062)(150.85664063,386.765625)(150.57148438,386.57421875)
\curveto(150.29023438,386.38671875)(149.9484375,386.24414062)(149.54609375,386.14648438)
\curveto(149.14765625,386.04882812)(148.68867188,386)(148.16914063,386)
\closepath
\moveto(146.20625,387.01367188)
\lineto(148.04023438,387.01367188)
\curveto(148.60664063,387.01367188)(149.05,387.06640625)(149.3703125,387.171875)
\curveto(149.69453125,387.27734375)(149.95234375,387.42578125)(150.14375,387.6171875)
\curveto(150.41328125,387.88671875)(150.62226563,388.24804688)(150.77070313,388.70117188)
\curveto(150.92304688,389.15820312)(150.99921875,389.7109375)(150.99921875,390.359375)
\curveto(150.99921875,391.2578125)(150.85078125,391.94726562)(150.55390625,392.42773438)
\curveto(150.2609375,392.91210938)(149.90351563,393.23632812)(149.48164063,393.40039062)
\curveto(149.17695313,393.51757812)(148.68671875,393.57617188)(148.0109375,393.57617188)
\lineto(146.20625,393.57617188)
\closepath
}
}
{
\newrgbcolor{curcolor}{0 0 0}
\pscustom[linestyle=none,fillstyle=solid,fillcolor=curcolor]
{
\newpath
\moveto(153.60664063,393.37695312)
\lineto(153.60664063,394.58984375)
\lineto(154.66132813,394.58984375)
\lineto(154.66132813,393.37695312)
\closepath
\moveto(153.60664063,386)
\lineto(153.60664063,392.22265625)
\lineto(154.66132813,392.22265625)
\lineto(154.66132813,386)
\closepath
}
}
{
\newrgbcolor{curcolor}{0 0 0}
\pscustom[linestyle=none,fillstyle=solid,fillcolor=curcolor]
{
\newpath
\moveto(156.51875,386)
\lineto(156.51875,391.40234375)
\lineto(155.58710938,391.40234375)
\lineto(155.58710938,392.22265625)
\lineto(156.51875,392.22265625)
\lineto(156.51875,392.88476562)
\curveto(156.51875,393.30273438)(156.55585938,393.61328125)(156.63007813,393.81640625)
\curveto(156.73164063,394.08984375)(156.909375,394.31054688)(157.16328125,394.47851562)
\curveto(157.42109375,394.65039062)(157.78046875,394.73632812)(158.24140625,394.73632812)
\curveto(158.53828125,394.73632812)(158.86640625,394.70117188)(159.22578125,394.63085938)
\lineto(159.06757813,393.7109375)
\curveto(158.84882813,393.75)(158.64179688,393.76953125)(158.44648438,393.76953125)
\curveto(158.12617188,393.76953125)(157.89960938,393.70117188)(157.76679688,393.56445312)
\curveto(157.63398438,393.42773438)(157.56757813,393.171875)(157.56757813,392.796875)
\lineto(157.56757813,392.22265625)
\lineto(158.78046875,392.22265625)
\lineto(158.78046875,391.40234375)
\lineto(157.56757813,391.40234375)
\lineto(157.56757813,386)
\closepath
}
}
{
\newrgbcolor{curcolor}{0 0 0}
\pscustom[linestyle=none,fillstyle=solid,fillcolor=curcolor]
{
\newpath
\moveto(159.6359375,386)
\lineto(159.6359375,391.40234375)
\lineto(158.70429688,391.40234375)
\lineto(158.70429688,392.22265625)
\lineto(159.6359375,392.22265625)
\lineto(159.6359375,392.88476562)
\curveto(159.6359375,393.30273438)(159.67304688,393.61328125)(159.74726563,393.81640625)
\curveto(159.84882813,394.08984375)(160.0265625,394.31054688)(160.28046875,394.47851562)
\curveto(160.53828125,394.65039062)(160.89765625,394.73632812)(161.35859375,394.73632812)
\curveto(161.65546875,394.73632812)(161.98359375,394.70117188)(162.34296875,394.63085938)
\lineto(162.18476563,393.7109375)
\curveto(161.96601563,393.75)(161.75898438,393.76953125)(161.56367188,393.76953125)
\curveto(161.24335938,393.76953125)(161.01679688,393.70117188)(160.88398438,393.56445312)
\curveto(160.75117188,393.42773438)(160.68476563,393.171875)(160.68476563,392.796875)
\lineto(160.68476563,392.22265625)
\lineto(161.89765625,392.22265625)
\lineto(161.89765625,391.40234375)
\lineto(160.68476563,391.40234375)
\lineto(160.68476563,386)
\closepath
}
}
{
\newrgbcolor{curcolor}{0 0 0}
\pscustom[linestyle=none,fillstyle=solid,fillcolor=curcolor]
{
\newpath
\moveto(166.97773438,388.00390625)
\lineto(168.06757813,387.86914062)
\curveto(167.89570313,387.23242188)(167.57734375,386.73828125)(167.1125,386.38671875)
\curveto(166.64765625,386.03515625)(166.05390625,385.859375)(165.33125,385.859375)
\curveto(164.42109375,385.859375)(163.6984375,386.13867188)(163.16328125,386.69726562)
\curveto(162.63203125,387.25976562)(162.36640625,388.046875)(162.36640625,389.05859375)
\curveto(162.36640625,390.10546875)(162.6359375,390.91796875)(163.175,391.49609375)
\curveto(163.7140625,392.07421875)(164.41328125,392.36328125)(165.27265625,392.36328125)
\curveto(166.1046875,392.36328125)(166.784375,392.08007812)(167.31171875,391.51367188)
\curveto(167.8390625,390.94726562)(168.10273438,390.15039062)(168.10273438,389.12304688)
\curveto(168.10273438,389.06054688)(168.10078125,388.96679688)(168.096875,388.84179688)
\lineto(163.45625,388.84179688)
\curveto(163.4953125,388.15820312)(163.68867188,387.63476562)(164.03632813,387.27148438)
\curveto(164.38398438,386.90820312)(164.81757813,386.7265625)(165.33710938,386.7265625)
\curveto(165.72382813,386.7265625)(166.05390625,386.828125)(166.32734375,387.03125)
\curveto(166.60078125,387.234375)(166.81757813,387.55859375)(166.97773438,388.00390625)
\closepath
\moveto(163.51484375,389.70898438)
\lineto(166.98945313,389.70898438)
\curveto(166.94257813,390.23242188)(166.80976563,390.625)(166.59101563,390.88671875)
\curveto(166.25507813,391.29296875)(165.81953125,391.49609375)(165.284375,391.49609375)
\curveto(164.8,391.49609375)(164.39179688,391.33398438)(164.05976563,391.00976562)
\curveto(163.73164063,390.68554688)(163.55,390.25195312)(163.51484375,389.70898438)
\closepath
}
}
{
\newrgbcolor{curcolor}{0 0 0}
\pscustom[linestyle=none,fillstyle=solid,fillcolor=curcolor]
{
\newpath
\moveto(169.38007813,386)
\lineto(169.38007813,392.22265625)
\lineto(170.32929688,392.22265625)
\lineto(170.32929688,391.27929688)
\curveto(170.57148438,391.72070312)(170.79414063,392.01171875)(170.99726563,392.15234375)
\curveto(171.20429688,392.29296875)(171.43085938,392.36328125)(171.67695313,392.36328125)
\curveto(172.03242188,392.36328125)(172.39375,392.25)(172.7609375,392.0234375)
\lineto(172.39765625,391.04492188)
\curveto(172.13984375,391.19726562)(171.88203125,391.2734375)(171.62421875,391.2734375)
\curveto(171.39375,391.2734375)(171.18671875,391.203125)(171.003125,391.0625)
\curveto(170.81953125,390.92578125)(170.68867188,390.734375)(170.61054688,390.48828125)
\curveto(170.49335938,390.11328125)(170.43476563,389.703125)(170.43476563,389.2578125)
\lineto(170.43476563,386)
\closepath
}
}
{
\newrgbcolor{curcolor}{0 0 0}
\pscustom[linestyle=none,fillstyle=solid,fillcolor=curcolor]
{
\newpath
\moveto(177.64765625,388.00390625)
\lineto(178.7375,387.86914062)
\curveto(178.565625,387.23242188)(178.24726563,386.73828125)(177.78242188,386.38671875)
\curveto(177.31757813,386.03515625)(176.72382813,385.859375)(176.00117188,385.859375)
\curveto(175.09101563,385.859375)(174.36835938,386.13867188)(173.83320313,386.69726562)
\curveto(173.30195313,387.25976562)(173.03632813,388.046875)(173.03632813,389.05859375)
\curveto(173.03632813,390.10546875)(173.30585938,390.91796875)(173.84492188,391.49609375)
\curveto(174.38398438,392.07421875)(175.08320313,392.36328125)(175.94257813,392.36328125)
\curveto(176.77460938,392.36328125)(177.45429688,392.08007812)(177.98164063,391.51367188)
\curveto(178.50898438,390.94726562)(178.77265625,390.15039062)(178.77265625,389.12304688)
\curveto(178.77265625,389.06054688)(178.77070313,388.96679688)(178.76679688,388.84179688)
\lineto(174.12617188,388.84179688)
\curveto(174.16523438,388.15820312)(174.35859375,387.63476562)(174.70625,387.27148438)
\curveto(175.05390625,386.90820312)(175.4875,386.7265625)(176.00703125,386.7265625)
\curveto(176.39375,386.7265625)(176.72382813,386.828125)(176.99726563,387.03125)
\curveto(177.27070313,387.234375)(177.4875,387.55859375)(177.64765625,388.00390625)
\closepath
\moveto(174.18476563,389.70898438)
\lineto(177.659375,389.70898438)
\curveto(177.6125,390.23242188)(177.4796875,390.625)(177.2609375,390.88671875)
\curveto(176.925,391.29296875)(176.48945313,391.49609375)(175.95429688,391.49609375)
\curveto(175.46992188,391.49609375)(175.06171875,391.33398438)(174.7296875,391.00976562)
\curveto(174.4015625,390.68554688)(174.21992188,390.25195312)(174.18476563,389.70898438)
\closepath
}
}
{
\newrgbcolor{curcolor}{0 0 0}
\pscustom[linestyle=none,fillstyle=solid,fillcolor=curcolor]
{
\newpath
\moveto(180.06171875,386)
\lineto(180.06171875,392.22265625)
\lineto(181.0109375,392.22265625)
\lineto(181.0109375,391.33789062)
\curveto(181.46796875,392.02148438)(182.128125,392.36328125)(182.99140625,392.36328125)
\curveto(183.36640625,392.36328125)(183.71015625,392.29492188)(184.02265625,392.15820312)
\curveto(184.3390625,392.02539062)(184.57539063,391.84960938)(184.73164063,391.63085938)
\curveto(184.88789063,391.41210938)(184.99726563,391.15234375)(185.05976563,390.8515625)
\curveto(185.09882813,390.65625)(185.11835938,390.31445312)(185.11835938,389.82617188)
\lineto(185.11835938,386)
\lineto(184.06367188,386)
\lineto(184.06367188,389.78515625)
\curveto(184.06367188,390.21484375)(184.02265625,390.53515625)(183.940625,390.74609375)
\curveto(183.85859375,390.9609375)(183.71210938,391.13085938)(183.50117188,391.25585938)
\curveto(183.29414063,391.38476562)(183.05,391.44921875)(182.76875,391.44921875)
\curveto(182.31953125,391.44921875)(181.93085938,391.30664062)(181.60273438,391.02148438)
\curveto(181.27851563,390.73632812)(181.11640625,390.1953125)(181.11640625,389.3984375)
\lineto(181.11640625,386)
\closepath
}
}
{
\newrgbcolor{curcolor}{0 0 0}
\pscustom[linestyle=none,fillstyle=solid,fillcolor=curcolor]
{
\newpath
\moveto(189.03828125,386.94335938)
\lineto(189.190625,386.01171875)
\curveto(188.89375,385.94921875)(188.628125,385.91796875)(188.39375,385.91796875)
\curveto(188.0109375,385.91796875)(187.7140625,385.97851562)(187.503125,386.09960938)
\curveto(187.2921875,386.22070312)(187.14375,386.37890625)(187.0578125,386.57421875)
\curveto(186.971875,386.7734375)(186.92890625,387.18945312)(186.92890625,387.82226562)
\lineto(186.92890625,391.40234375)
\lineto(186.15546875,391.40234375)
\lineto(186.15546875,392.22265625)
\lineto(186.92890625,392.22265625)
\lineto(186.92890625,393.76367188)
\lineto(187.97773438,394.39648438)
\lineto(187.97773438,392.22265625)
\lineto(189.03828125,392.22265625)
\lineto(189.03828125,391.40234375)
\lineto(187.97773438,391.40234375)
\lineto(187.97773438,387.76367188)
\curveto(187.97773438,387.46289062)(187.9953125,387.26953125)(188.03046875,387.18359375)
\curveto(188.06953125,387.09765625)(188.13007813,387.02929688)(188.21210938,386.97851562)
\curveto(188.29804688,386.92773438)(188.41914063,386.90234375)(188.57539063,386.90234375)
\curveto(188.69257813,386.90234375)(188.846875,386.91601562)(189.03828125,386.94335938)
\closepath
}
}
{
\newrgbcolor{curcolor}{0 0 0}
\pscustom[linestyle=none,fillstyle=solid,fillcolor=curcolor]
{
\newpath
\moveto(193.5265625,386)
\lineto(193.5265625,394.58984375)
\lineto(194.69257813,394.58984375)
\lineto(199.20429688,387.84570312)
\lineto(199.20429688,394.58984375)
\lineto(200.29414063,394.58984375)
\lineto(200.29414063,386)
\lineto(199.128125,386)
\lineto(194.61640625,392.75)
\lineto(194.61640625,386)
\closepath
}
}
{
\newrgbcolor{curcolor}{0 0 0}
\pscustom[linestyle=none,fillstyle=solid,fillcolor=curcolor]
{
\newpath
\moveto(201.67695313,389.11132812)
\curveto(201.67695313,390.26367188)(201.99726563,391.1171875)(202.63789063,391.671875)
\curveto(203.17304688,392.1328125)(203.82539063,392.36328125)(204.59492188,392.36328125)
\curveto(205.45039063,392.36328125)(206.14960938,392.08203125)(206.69257813,391.51953125)
\curveto(207.23554688,390.9609375)(207.50703125,390.1875)(207.50703125,389.19921875)
\curveto(207.50703125,388.3984375)(207.3859375,387.76757812)(207.14375,387.30664062)
\curveto(206.90546875,386.84960938)(206.55585938,386.49414062)(206.09492188,386.24023438)
\curveto(205.63789063,385.98632812)(205.13789063,385.859375)(204.59492188,385.859375)
\curveto(203.72382813,385.859375)(203.01875,386.13867188)(202.4796875,386.69726562)
\curveto(201.94453125,387.25585938)(201.67695313,388.06054688)(201.67695313,389.11132812)
\closepath
\moveto(202.7609375,389.11132812)
\curveto(202.7609375,388.31445312)(202.93476563,387.71679688)(203.28242188,387.31835938)
\curveto(203.63007813,386.92382812)(204.06757813,386.7265625)(204.59492188,386.7265625)
\curveto(205.11835938,386.7265625)(205.55390625,386.92578125)(205.9015625,387.32421875)
\curveto(206.24921875,387.72265625)(206.42304688,388.33007812)(206.42304688,389.14648438)
\curveto(206.42304688,389.91601562)(206.24726563,390.49804688)(205.89570313,390.89257812)
\curveto(205.54804688,391.29101562)(205.11445313,391.49023438)(204.59492188,391.49023438)
\curveto(204.06757813,391.49023438)(203.63007813,391.29296875)(203.28242188,390.8984375)
\curveto(202.93476563,390.50390625)(202.7609375,389.90820312)(202.7609375,389.11132812)
\closepath
}
}
{
\newrgbcolor{curcolor}{0 0 0}
\pscustom[linestyle=none,fillstyle=solid,fillcolor=curcolor]
{
\newpath
\moveto(212.78046875,386)
\lineto(212.78046875,386.78515625)
\curveto(212.3859375,386.16796875)(211.80585938,385.859375)(211.04023438,385.859375)
\curveto(210.54414063,385.859375)(210.08710938,385.99609375)(209.66914063,386.26953125)
\curveto(209.25507813,386.54296875)(208.9328125,386.92382812)(208.70234375,387.41210938)
\curveto(208.47578125,387.90429688)(208.3625,388.46875)(208.3625,389.10546875)
\curveto(208.3625,389.7265625)(208.46601563,390.2890625)(208.67304688,390.79296875)
\curveto(208.88007813,391.30078125)(209.190625,391.68945312)(209.6046875,391.95898438)
\curveto(210.01875,392.22851562)(210.48164063,392.36328125)(210.99335938,392.36328125)
\curveto(211.36835938,392.36328125)(211.70234375,392.28320312)(211.9953125,392.12304688)
\curveto(212.28828125,391.96679688)(212.5265625,391.76171875)(212.71015625,391.5078125)
\lineto(212.71015625,394.58984375)
\lineto(213.75898438,394.58984375)
\lineto(213.75898438,386)
\closepath
\moveto(209.44648438,389.10546875)
\curveto(209.44648438,388.30859375)(209.61445313,387.71289062)(209.95039063,387.31835938)
\curveto(210.28632813,386.92382812)(210.6828125,386.7265625)(211.13984375,386.7265625)
\curveto(211.60078125,386.7265625)(211.99140625,386.9140625)(212.31171875,387.2890625)
\curveto(212.6359375,387.66796875)(212.79804688,388.24414062)(212.79804688,389.01757812)
\curveto(212.79804688,389.86914062)(212.63398438,390.49414062)(212.30585938,390.89257812)
\curveto(211.97773438,391.29101562)(211.5734375,391.49023438)(211.09296875,391.49023438)
\curveto(210.62421875,391.49023438)(210.23164063,391.29882812)(209.91523438,390.91601562)
\curveto(209.60273438,390.53320312)(209.44648438,389.9296875)(209.44648438,389.10546875)
\closepath
}
}
{
\newrgbcolor{curcolor}{0 0 0}
\pscustom[linestyle=none,fillstyle=solid,fillcolor=curcolor]
{
\newpath
\moveto(219.67695313,388.00390625)
\lineto(220.76679688,387.86914062)
\curveto(220.59492188,387.23242188)(220.2765625,386.73828125)(219.81171875,386.38671875)
\curveto(219.346875,386.03515625)(218.753125,385.859375)(218.03046875,385.859375)
\curveto(217.1203125,385.859375)(216.39765625,386.13867188)(215.8625,386.69726562)
\curveto(215.33125,387.25976562)(215.065625,388.046875)(215.065625,389.05859375)
\curveto(215.065625,390.10546875)(215.33515625,390.91796875)(215.87421875,391.49609375)
\curveto(216.41328125,392.07421875)(217.1125,392.36328125)(217.971875,392.36328125)
\curveto(218.80390625,392.36328125)(219.48359375,392.08007812)(220.0109375,391.51367188)
\curveto(220.53828125,390.94726562)(220.80195313,390.15039062)(220.80195313,389.12304688)
\curveto(220.80195313,389.06054688)(220.8,388.96679688)(220.79609375,388.84179688)
\lineto(216.15546875,388.84179688)
\curveto(216.19453125,388.15820312)(216.38789063,387.63476562)(216.73554688,387.27148438)
\curveto(217.08320313,386.90820312)(217.51679688,386.7265625)(218.03632813,386.7265625)
\curveto(218.42304688,386.7265625)(218.753125,386.828125)(219.0265625,387.03125)
\curveto(219.3,387.234375)(219.51679688,387.55859375)(219.67695313,388.00390625)
\closepath
\moveto(216.2140625,389.70898438)
\lineto(219.68867188,389.70898438)
\curveto(219.64179688,390.23242188)(219.50898438,390.625)(219.29023438,390.88671875)
\curveto(218.95429688,391.29296875)(218.51875,391.49609375)(217.98359375,391.49609375)
\curveto(217.49921875,391.49609375)(217.09101563,391.33398438)(216.75898438,391.00976562)
\curveto(216.43085938,390.68554688)(216.24921875,390.25195312)(216.2140625,389.70898438)
\closepath
}
}
{
\newrgbcolor{curcolor}{1 0 0}
\pscustom[linewidth=1,linecolor=curcolor]
{
\newpath
\moveto(229.6,389.9)
\lineto(271.8,389.9)
\moveto(96.8,259)
\lineto(101.6,259)
\lineto(106.4,259)
\lineto(111.1,259)
\lineto(115.9,259)
\lineto(120.7,259)
\lineto(125.5,259)
\lineto(130.3,259)
\lineto(135.1,259)
\lineto(139.8,259)
\lineto(144.6,259)
\lineto(149.4,259)
\lineto(154.2,259)
\lineto(159,259)
\lineto(163.7,259)
\lineto(168.5,259)
\lineto(173.3,259)
\lineto(178.1,259)
\lineto(182.9,259)
\lineto(187.7,259)
\lineto(192.4,259)
\lineto(197.2,259)
\lineto(202,259)
\lineto(206.8,259)
\lineto(211.6,259)
\lineto(216.4,259)
\lineto(221.1,259)
\lineto(225.9,259)
\lineto(230.7,259)
\lineto(235.5,259)
\lineto(240.3,259)
\lineto(245,259)
\lineto(249.8,259)
\lineto(254.6,259)
\lineto(259.4,259)
\lineto(264.2,259)
\lineto(269,259)
\lineto(273.7,259)
\lineto(278.5,259)
\lineto(283.3,259)
\lineto(288.1,259)
\lineto(292.9,259)
\lineto(297.6,259)
\lineto(302.4,259)
\lineto(307.2,259)
\lineto(312,259)
\lineto(316.8,259)
\lineto(321.6,259)
\lineto(326.3,259)
\lineto(331.1,259)
\lineto(335.9,259)
\lineto(340.7,259)
\lineto(345.5,259)
\lineto(350.2,259)
\lineto(355,259)
\lineto(359.8,259)
\lineto(364.6,259)
\lineto(369.4,259)
\lineto(374.2,259)
\lineto(378.9,259)
\lineto(383.7,259)
\lineto(388.5,259)
\lineto(393.3,259)
\lineto(398.1,259)
\lineto(402.8,259)
\lineto(407.6,259)
\lineto(412.4,259)
\lineto(417.2,259)
\lineto(422,259)
\lineto(426.8,259)
\lineto(431.5,259)
\lineto(436.3,259)
\lineto(441.1,259)
\lineto(445.9,259)
\lineto(450.7,259)
\lineto(455.5,259)
\lineto(460.2,259)
\lineto(465,259)
\lineto(469.8,259)
\lineto(474.6,259)
\lineto(479.4,259)
\lineto(484.1,259)
\lineto(488.9,259)
\lineto(493.7,259)
\lineto(498.5,259)
\lineto(503.3,259)
\lineto(508.1,259)
\lineto(512.8,259)
\lineto(517.6,259)
\lineto(522.4,259)
\lineto(527.2,259)
\lineto(532,259)
\lineto(536.7,259)
\lineto(541.5,259)
\lineto(546.3,259)
\lineto(551.1,259)
\lineto(555.9,259)
\lineto(560.7,259)
\lineto(565.4,259)
}
}
{
\newrgbcolor{curcolor}{0 0 0}
\pscustom[linewidth=1,linecolor=curcolor]
{
\newpath
\moveto(96.8,425.9)
\lineto(96.8,57.6)
\lineto(575,57.6)
\lineto(575,425.9)
\closepath
}
}
\end{pspicture}
}
        \captionsetup{width=0.9\textwidth}
        \caption{50th percentile latency sees no change at all over 100 runs with NUMA Balancing enabled.}
    \end{subfigure}
    \begin{subfigure}{0.4\linewidth}
        \resizebox{0.9\linewidth}{!}{\input{graphs/90th Percentile Latency Old ZFS 1M Reads NUMA Balancing 100 fio Runs (Data Node 0)}}
        \captionsetup{width=0.9\textwidth}
        \caption{90th percentile latency sees no significant change at all over 100 runs with NUMA Balancing enabled.}
    \end{subfigure}
    \captionsetup{width=0.80\linewidth}
    \caption{Repeatedly reading data from the ARC does not trigger NUMA Balancing, likely because it does not consider kernel memory like the ARC.
    If NUMA Balancing was occurring the different node latency would converge towards the same
    node latency. Note that the same node latency is only from the initial test run on the same node
    after priming the ARC to ensure that it would not interfere with NUMA Balancing by adding more
    reads from the same node. It is extended across the graph to provide a point of reference.}
    \label{fig:100NUMABalance}
\end{figure}

Investigating the Linux NUMA balancing code confirms what this graph show by experimentation, that ARC memory is not considered
Only memory that is part of a specific process is considered and even then only after that process has been running for one full second
\cite[{kernel/sched/fair.c}]{linux}.

\chapter{Task Migration}
Task migration is moving a process automatically closer to the data that it is reading, if we find that the process is on a different node than that data.
This requires storing the current node number in the ARC header structure, 
and moving the current task and all of its memory to that same node if it is not running there already.
This will invalidate the CPU caching that has built up over the course of the process's current run time,
but will guarantee that the process will be local to the chapter of the ARC currently being accessed.

My expectation was that above a certain threshold of file size, 
the overhead from remote node accesses would be high enough that a significant performance improvement
would be visible in the results.
Ideally, this would mean a restoration of all lost performance from these remote node accesses.
Assuming the overhead from moving the process and its memory is not too high,
then a performance improvement should occur, as future memory accesses to that same part of the ARC become local node
accesses instead of remote node accesses.

\section{Implementing Task Migration}
% https://github.com/openzfs/zfs/compare/zfs-0.8.5...Tookmund:arcprocessmigrate-0.8.5
In order to implement task migration, I had to modify both the Linux Kernel and OpenZFS.
The initial step was to add code allowing ZFS to migrate the current process to another node, 
as such functionality is not currently available to kernel modules like ZFS in Linux.
Linux is rather conservative about the functionality it exposes to kernel modules,
and this allows one to take fairly direct control over the location of any process's memory,
which is more than any kernel module would typically need.
Something like this would usually be implemented in the scheduler or the page cache,
both of which are part of Linux itself, and thus would not need this functionality exported.
This required three steps, first moving the process to the correct NUMA node, then ensuring it will allocate new memory
only from that new node, and finally moving all of the process's current memory
allocations to the new node.

Every process in the Linux kernel has a task structure, which contains various pieces of information about that task.
By modifying this structure inside Linux or one of its modules, like ZFS, 
one can control how the process is scheduled, and where its memory is located.
This structure for the current process, in whose context this code is executing, 
can always be referred to by the \texttt{current} macro within the Linux kernel, or, 
within the SPL or ZFS as \texttt{curthread}, because that is what the structure was called on Solaris.

\begin{figure}[H]
    \centering
    \begin{subfigure}{0.4\textwidth}
        \centering
        \resizebox{0.9\linewidth}{!}{%LaTeX with PSTricks extensions
%%Creator: Inkscape 1.0.2-2 (e86c870879, 2021-01-15)
%%Please note this file requires PSTricks extensions
\psset{xunit=.5pt,yunit=.5pt,runit=.5pt}
\begin{pspicture}(612.31414555,725.00071128)
{
\newrgbcolor{curcolor}{0 0 0}
\pscustom[linestyle=none,fillstyle=solid,fillcolor=curcolor]
{
\newpath
\moveto(151.9007967,546.79375708)
\lineto(151.9007967,453.85970983)
}
}
{
\newrgbcolor{curcolor}{0 0 0}
\pscustom[linewidth=2.64566925,linecolor=curcolor]
{
\newpath
\moveto(151.9007967,546.79375708)
\lineto(151.9007967,453.85970983)
}
}
{
\newrgbcolor{curcolor}{0.80000001 0.80000001 0.80000001}
\pscustom[linestyle=none,fillstyle=solid,fillcolor=curcolor]
{
\newpath
\moveto(197.92706933,661.27663338)
\lineto(412.45029344,661.27663338)
\lineto(412.45029344,603.53934119)
\lineto(197.92706933,603.53934119)
\closepath
}
}
{
\newrgbcolor{curcolor}{0 0 0}
\pscustom[linewidth=0.4430022,linecolor=curcolor]
{
\newpath
\moveto(197.92706933,661.27663338)
\lineto(412.45029344,661.27663338)
\lineto(412.45029344,603.53934119)
\lineto(197.92706933,603.53934119)
\closepath
}
}
{
\newrgbcolor{curcolor}{0.80000001 0.80000001 0.80000001}
\pscustom[linestyle=none,fillstyle=solid,fillcolor=curcolor]
{
\newpath
\moveto(68.05813151,707.07728007)
\lineto(235.74345251,707.07728007)
\lineto(235.74345251,542.42242667)
\lineto(68.05813151,542.42242667)
\closepath
}
}
{
\newrgbcolor{curcolor}{0 0 0}
\pscustom[linewidth=0.66141731,linecolor=curcolor]
{
\newpath
\moveto(68.05813151,707.07728007)
\lineto(235.74345251,707.07728007)
\lineto(235.74345251,542.42242667)
\lineto(68.05813151,542.42242667)
\closepath
}
}
{
\newrgbcolor{curcolor}{0.80000001 0.80000001 0.80000001}
\pscustom[linestyle=none,fillstyle=solid,fillcolor=curcolor]
{
\newpath
\moveto(376.57113729,707.07728007)
\lineto(544.25645828,707.07728007)
\lineto(544.25645828,542.42242667)
\lineto(376.57113729,542.42242667)
\closepath
}
}
{
\newrgbcolor{curcolor}{0 0 0}
\pscustom[linewidth=0.66141731,linecolor=curcolor]
{
\newpath
\moveto(376.57113729,707.07728007)
\lineto(544.25645828,707.07728007)
\lineto(544.25645828,542.42242667)
\lineto(376.57113729,542.42242667)
\closepath
}
}
{
\newrgbcolor{curcolor}{0 0 0}
\pscustom[linestyle=none,fillstyle=solid,fillcolor=curcolor]
{
\newpath
\moveto(138.2192044,612.35736224)
\curveto(137.50306085,612.04486324)(136.85202125,611.75189542)(136.26608562,611.47845879)
\curveto(135.69317078,611.20502216)(134.93796485,610.91856474)(134.00046783,610.61908653)
\curveto(133.20619953,610.37169148)(132.34031687,610.16335881)(131.40281985,609.99408852)
\curveto(130.47834363,609.81179743)(129.45621147,609.72065189)(128.33642337,609.72065189)
\curveto(126.22705508,609.72065189)(124.30648828,610.0136197)(122.57472296,610.59955534)
\curveto(120.85597844,611.19851176)(119.35858737,612.12949838)(118.08254977,613.3925152)
\curveto(116.83255375,614.62949042)(115.85599435,616.19849584)(115.15287159,618.09953146)
\curveto(114.44974883,620.01358786)(114.09818745,622.23363288)(114.09818745,624.7596665)
\curveto(114.09818745,627.15549221)(114.43672804,629.29741247)(115.11380922,631.18542729)
\curveto(115.7908904,633.07344212)(116.76744979,634.66848912)(118.04348739,635.97056831)
\curveto(119.28046262,637.23358512)(120.77134329,638.19712372)(122.5161294,638.8611841)
\curveto(124.2739363,639.52524449)(126.22054469,639.85727468)(128.35595456,639.85727468)
\curveto(129.91844958,639.85727468)(131.47443421,639.6684732)(133.02390844,639.29087024)
\curveto(134.58640347,638.91326727)(136.31816879,638.24920689)(138.2192044,637.29868908)
\lineto(138.2192044,632.70885994)
\lineto(137.92623658,632.70885994)
\curveto(136.32467918,634.05000151)(134.73614258,635.0265609)(133.16062676,635.63853812)
\curveto(131.58511094,636.25051533)(129.89891839,636.55650394)(128.10204911,636.55650394)
\curveto(126.63069963,636.55650394)(125.30257886,636.31561929)(124.1176868,635.83384999)
\curveto(122.94581553,635.36510149)(121.89764179,634.62942675)(120.97316556,633.62682577)
\curveto(120.07473092,632.65026638)(119.37160816,631.41329115)(118.86379728,629.91590009)
\curveto(118.36900719,628.43152981)(118.12161214,626.71278528)(118.12161214,624.7596665)
\curveto(118.12161214,622.71540218)(118.39504877,620.95759528)(118.94192203,619.48624579)
\curveto(119.50181608,618.01489631)(120.21795963,616.81698346)(121.09035269,615.89250724)
\curveto(122.00180812,614.92896864)(123.06300266,614.21282508)(124.2739363,613.74407658)
\curveto(125.49789074,613.28834886)(126.78694913,613.060485)(128.14111149,613.060485)
\curveto(130.00308473,613.060485)(131.74787084,613.3794944)(133.37546982,614.01751321)
\curveto(135.00306881,614.65553201)(136.52650146,615.61256021)(137.94576777,616.88859781)
\lineto(138.2192044,616.88859781)
\closepath
}
}
{
\newrgbcolor{curcolor}{0 0 0}
\pscustom[linestyle=none,fillstyle=solid,fillcolor=curcolor]
{
\newpath
\moveto(162.71131334,630.5408981)
\curveto(162.71131334,629.2518397)(162.48344948,628.05392685)(162.02772176,626.94715954)
\curveto(161.58501484,625.85341302)(160.96001683,624.90289521)(160.15272773,624.09560612)
\curveto(159.15012676,623.09300514)(157.9652347,622.33779921)(156.59805155,621.82998833)
\curveto(155.2308684,621.33519824)(153.50561348,621.08780319)(151.42228678,621.08780319)
\lineto(147.55511159,621.08780319)
\lineto(147.55511159,610.24799396)
\lineto(143.68793641,610.24799396)
\lineto(143.68793641,639.32993261)
\lineto(151.57853628,639.32993261)
\curveto(153.32332239,639.32993261)(154.80118227,639.18019351)(156.01211592,638.88071529)
\curveto(157.22304956,638.59425787)(158.29726489,638.13853016)(159.23476191,637.51353215)
\curveto(160.34152921,636.77134701)(161.19439108,635.84687079)(161.79334751,634.74010348)
\curveto(162.40532473,633.63333617)(162.71131334,632.23360104)(162.71131334,630.5408981)
\closepath
\moveto(158.68788865,630.44324216)
\curveto(158.68788865,631.44584313)(158.51210796,632.31823619)(158.16054658,633.06042132)
\curveto(157.8089852,633.80260646)(157.27513273,634.40807328)(156.55898917,634.87682179)
\curveto(155.93399116,635.28046634)(155.21784761,635.56692376)(154.41055852,635.73619405)
\curveto(153.61629021,635.91848514)(152.60717884,636.00963068)(151.3832244,636.00963068)
\lineto(147.55511159,636.00963068)
\lineto(147.55511159,624.38857393)
\lineto(150.81681996,624.38857393)
\curveto(152.37931498,624.38857393)(153.64884219,624.52529225)(154.62540158,624.79872888)
\curveto(155.60196097,625.0851863)(156.39622928,625.53440362)(157.00820649,626.14638084)
\curveto(157.62018371,626.77137885)(158.04986984,627.42892884)(158.29726489,628.11903081)
\curveto(158.55768073,628.80913278)(158.68788865,629.58386989)(158.68788865,630.44324216)
\closepath
}
}
{
\newrgbcolor{curcolor}{0 0 0}
\pscustom[linestyle=none,fillstyle=solid,fillcolor=curcolor]
{
\newpath
\moveto(189.703413,621.92764427)
\curveto(189.703413,619.81827599)(189.46903874,617.97583394)(189.00029024,616.40031812)
\curveto(188.54456252,614.83782309)(187.78935659,613.53574391)(186.73467245,612.49408056)
\curveto(185.73207148,611.50450037)(184.56020021,610.78184642)(183.21905864,610.32611871)
\curveto(181.87791708,609.87039099)(180.31542206,609.64252714)(178.53157357,609.64252714)
\curveto(176.70866271,609.64252714)(175.1201261,609.88341178)(173.76596374,610.36518108)
\curveto(172.41180139,610.84695038)(171.2724821,611.55658354)(170.34800587,612.49408056)
\curveto(169.29332173,613.56178549)(168.53160541,614.85084389)(168.0628569,616.36125574)
\curveto(167.60712918,617.8716676)(167.37926533,619.72713044)(167.37926533,621.92764427)
\lineto(167.37926533,639.32993261)
\lineto(171.24644051,639.32993261)
\lineto(171.24644051,621.73233239)
\curveto(171.24644051,620.15681658)(171.35060685,618.91333095)(171.55893952,618.00187552)
\curveto(171.78029298,617.09042009)(172.14487515,616.2635998)(172.65268604,615.52141467)
\curveto(173.22560088,614.6750632)(174.000338,614.03704439)(174.97689739,613.60735826)
\curveto(175.96647757,613.17767213)(177.15136963,612.96282906)(178.53157357,612.96282906)
\curveto(179.9247983,612.96282906)(181.10969036,613.17116173)(182.08624975,613.58782707)
\curveto(183.06280914,614.01751321)(183.84405665,614.6620424)(184.42999229,615.52141467)
\curveto(184.93780317,616.2635998)(185.29587495,617.10995128)(185.50420762,618.06046908)
\curveto(185.72556108,619.02400768)(185.83623781,620.21541014)(185.83623781,621.63467645)
\lineto(185.83623781,639.32993261)
\lineto(189.703413,639.32993261)
\closepath
}
}
{
\newrgbcolor{curcolor}{0 0 0}
\pscustom[linestyle=none,fillstyle=solid,fillcolor=curcolor]
{
\newpath
\moveto(446.73219576,607.30656369)
\curveto(446.01605221,606.99406468)(445.36501261,606.70109687)(444.77907698,606.42766024)
\curveto(444.20616213,606.15422361)(443.45095621,605.86776619)(442.51345919,605.56828797)
\curveto(441.71919089,605.32089293)(440.85330823,605.11256026)(439.91581121,604.94328996)
\curveto(438.99133499,604.76099888)(437.96920283,604.66985334)(436.84941473,604.66985334)
\curveto(434.74004644,604.66985334)(432.81947964,604.96282115)(431.08771432,605.54875679)
\curveto(429.36896979,606.14771321)(427.87157873,607.07869983)(426.59554112,608.34171664)
\curveto(425.3455451,609.57869187)(424.36898571,611.14769729)(423.66586295,613.04873291)
\curveto(422.96274019,614.96278931)(422.61117881,617.18283433)(422.61117881,619.70886795)
\curveto(422.61117881,622.10469366)(422.9497194,624.24661392)(423.62680057,626.13462874)
\curveto(424.30388175,628.02264356)(425.28044114,629.61769057)(426.55647875,630.91976976)
\curveto(427.79345398,632.18278657)(429.28433465,633.14632517)(431.02912076,633.81038555)
\curveto(432.78692766,634.47444594)(434.73353605,634.80647613)(436.86894591,634.80647613)
\curveto(438.43144094,634.80647613)(439.98742557,634.61767465)(441.5368998,634.24007169)
\curveto(443.09939483,633.86246872)(444.83116015,633.19840834)(446.73219576,632.24789053)
\lineto(446.73219576,627.65806139)
\lineto(446.43922794,627.65806139)
\curveto(444.83767054,628.99920296)(443.24913393,629.97576235)(441.67361812,630.58773956)
\curveto(440.0981023,631.19971678)(438.41190975,631.50570539)(436.61504047,631.50570539)
\curveto(435.14369099,631.50570539)(433.81557022,631.26482074)(432.63067816,630.78305144)
\curveto(431.45880689,630.31430293)(430.41063314,629.57862819)(429.48615692,628.57602722)
\curveto(428.58772228,627.59946783)(427.88459952,626.3624926)(427.37678864,624.86510153)
\curveto(426.88199854,623.38073126)(426.6346035,621.66198673)(426.6346035,619.70886795)
\curveto(426.6346035,617.66460363)(426.90804013,615.90679672)(427.45491339,614.43544724)
\curveto(428.01480744,612.96409776)(428.73095099,611.76618491)(429.60334405,610.84170868)
\curveto(430.51479948,609.87817009)(431.57599402,609.16202653)(432.78692766,608.69327802)
\curveto(434.0108821,608.23755031)(435.29994049,608.00968645)(436.65410285,608.00968645)
\curveto(438.51607609,608.00968645)(440.2608622,608.32869585)(441.88846118,608.96671465)
\curveto(443.51606017,609.60473346)(445.03949282,610.56176166)(446.45875913,611.83779926)
\lineto(446.73219576,611.83779926)
\closepath
}
}
{
\newrgbcolor{curcolor}{0 0 0}
\pscustom[linestyle=none,fillstyle=solid,fillcolor=curcolor]
{
\newpath
\moveto(471.22430469,625.49009954)
\curveto(471.22430469,624.20104115)(470.99644084,623.0031283)(470.54071312,621.89636099)
\curveto(470.0980062,620.80261447)(469.47300819,619.85209666)(468.66571909,619.04480757)
\curveto(467.66311812,618.04220659)(466.47822605,617.28700066)(465.11104291,616.77918978)
\curveto(463.74385976,616.28439969)(462.01860484,616.03700464)(459.93527814,616.03700464)
\lineto(456.06810295,616.03700464)
\lineto(456.06810295,605.19719541)
\lineto(452.20092776,605.19719541)
\lineto(452.20092776,634.27913406)
\lineto(460.09152764,634.27913406)
\curveto(461.83631375,634.27913406)(463.31417363,634.12939495)(464.52510727,633.82991674)
\curveto(465.73604092,633.54345932)(466.81025625,633.0877316)(467.74775326,632.46273359)
\curveto(468.85452057,631.72054846)(469.70738244,630.79607223)(470.30633887,629.68930492)
\curveto(470.91831608,628.58253762)(471.22430469,627.18280249)(471.22430469,625.49009954)
\closepath
\moveto(467.20088,625.39244361)
\curveto(467.20088,626.39504458)(467.02509931,627.26743764)(466.67353793,628.00962277)
\curveto(466.32197655,628.75180791)(465.78812409,629.35727473)(465.07198053,629.82602324)
\curveto(464.44698252,630.22966779)(463.73083897,630.51612521)(462.92354987,630.6853955)
\curveto(462.12928157,630.86768659)(461.1201702,630.95883213)(459.89621576,630.95883213)
\lineto(456.06810295,630.95883213)
\lineto(456.06810295,619.33777538)
\lineto(459.32981131,619.33777538)
\curveto(460.89230634,619.33777538)(462.16183355,619.4744937)(463.13839294,619.74793033)
\curveto(464.11495233,620.03438775)(464.90922063,620.48360507)(465.52119785,621.09558229)
\curveto(466.13317507,621.7205803)(466.5628612,622.37813029)(466.81025625,623.06823226)
\curveto(467.07067208,623.75833423)(467.20088,624.53307134)(467.20088,625.39244361)
\closepath
}
}
{
\newrgbcolor{curcolor}{0 0 0}
\pscustom[linestyle=none,fillstyle=solid,fillcolor=curcolor]
{
\newpath
\moveto(498.21640436,616.87684572)
\curveto(498.21640436,614.76747743)(497.9820301,612.92503538)(497.51328159,611.34951957)
\curveto(497.05755388,609.78702454)(496.30234795,608.48494535)(495.24766381,607.443282)
\curveto(494.24506283,606.45370182)(493.07319156,605.73104787)(491.73205,605.27532016)
\curveto(490.39090844,604.81959244)(488.82841341,604.59172858)(487.04456493,604.59172858)
\curveto(485.22165406,604.59172858)(483.63311745,604.83261323)(482.2789551,605.31438253)
\curveto(480.92479274,605.79615183)(479.78547345,606.50578499)(478.86099723,607.443282)
\curveto(477.80631309,608.51098694)(477.04459676,609.80004533)(476.57584826,611.31045719)
\curveto(476.12012054,612.82086905)(475.89225668,614.67633189)(475.89225668,616.87684572)
\lineto(475.89225668,634.27913406)
\lineto(479.75943187,634.27913406)
\lineto(479.75943187,616.68153384)
\curveto(479.75943187,615.10601802)(479.86359821,613.8625324)(480.07193088,612.95107697)
\curveto(480.29328434,612.03962154)(480.65786651,611.21280125)(481.16567739,610.47061612)
\curveto(481.73859224,609.62426464)(482.51332935,608.98624584)(483.48988874,608.55655971)
\curveto(484.47946893,608.12687358)(485.66436099,607.91203051)(487.04456493,607.91203051)
\curveto(488.43778966,607.91203051)(489.62268172,608.12036318)(490.59924111,608.53702852)
\curveto(491.5758005,608.96671465)(492.35704801,609.61124385)(492.94298365,610.47061612)
\curveto(493.45079453,611.21280125)(493.80886631,612.05915272)(494.01719898,613.00967053)
\curveto(494.23855244,613.97320913)(494.34922917,615.16461159)(494.34922917,616.5838779)
\lineto(494.34922917,634.27913406)
\lineto(498.21640436,634.27913406)
\closepath
}
}
{
\newrgbcolor{curcolor}{0.80000001 0.80000001 0.80000001}
\pscustom[linestyle=none,fillstyle=solid,fillcolor=curcolor]
{
\newpath
\moveto(87.43154075,578.26067966)
\lineto(216.37004327,578.26067966)
\lineto(216.37004327,542.42247713)
\lineto(87.43154075,542.42247713)
\closepath
}
}
{
\newrgbcolor{curcolor}{0 0 0}
\pscustom[linewidth=0.27058431,linecolor=curcolor]
{
\newpath
\moveto(87.43154075,578.26067966)
\lineto(216.37004327,578.26067966)
\lineto(216.37004327,542.42247713)
\lineto(87.43154075,542.42247713)
\closepath
}
}
{
\newrgbcolor{curcolor}{0 0 0}
\pscustom[linestyle=none,fillstyle=solid,fillcolor=curcolor]
{
\newpath
\moveto(105.97956358,554.7622921)
\lineto(104.86077399,554.7622921)
\lineto(104.86077399,567.27653623)
\lineto(102.52148666,558.76138804)
\lineto(101.85473327,558.76138804)
\lineto(99.5323973,567.27653623)
\lineto(99.5323973,554.7622921)
\lineto(98.48706359,554.7622921)
\lineto(98.48706359,569.2858381)
\lineto(100.01268576,569.2858381)
\lineto(102.2559154,561.19986118)
\lineto(104.42568915,569.2858381)
\lineto(105.97956358,569.2858381)
\closepath
}
}
{
\newrgbcolor{curcolor}{0 0 0}
\pscustom[linestyle=none,fillstyle=solid,fillcolor=curcolor]
{
\newpath
\moveto(113.43816055,560.01964018)
\lineto(108.78783816,560.01964018)
\curveto(108.78783816,559.34987289)(108.84622617,558.76463934)(108.96300219,558.26393952)
\curveto(109.0797782,557.7697423)(109.23987436,557.36333011)(109.44329065,557.04470295)
\curveto(109.639173,556.73257839)(109.87084155,556.49848497)(110.1382963,556.34242269)
\curveto(110.40951802,556.18636041)(110.70710852,556.10832927)(111.03106779,556.10832927)
\curveto(111.46050218,556.10832927)(111.89182006,556.25463766)(112.32502141,556.54725443)
\curveto(112.76198974,556.8463738)(113.07276462,557.13899058)(113.25734607,557.42510476)
\lineto(113.3138506,557.42510476)
\lineto(113.3138506,555.42555679)
\curveto(112.95598861,555.16545299)(112.59059268,554.94761605)(112.21766281,554.77204599)
\curveto(111.84473295,554.59647592)(111.45296825,554.50869089)(111.0423687,554.50869089)
\curveto(109.99515151,554.50869089)(109.17771938,554.99638552)(108.59007232,555.97177477)
\curveto(108.00242526,556.95366662)(107.70860174,558.34522196)(107.70860174,560.14644078)
\curveto(107.70860174,561.92815182)(107.98924088,563.34246624)(108.55051916,564.38938404)
\curveto(109.1155644,565.43630184)(109.85765716,565.95976074)(110.77679743,565.95976074)
\curveto(111.62813227,565.95976074)(112.28358476,565.53058947)(112.7431549,564.67224692)
\curveto(113.206492,563.81390438)(113.43816055,562.59466781)(113.43816055,561.01453722)
\closepath
\moveto(112.40412775,561.42420071)
\curveto(112.40036078,562.38658477)(112.25909947,563.1311319)(111.98034381,563.6578421)
\curveto(111.70535512,564.1845523)(111.28533816,564.4479074)(110.72029291,564.4479074)
\curveto(110.15148069,564.4479074)(109.69756101,564.15854192)(109.35853386,563.57981096)
\curveto(109.02327368,563.00108)(108.83304178,562.28254325)(108.78783816,561.42420071)
\closepath
}
}
{
\newrgbcolor{curcolor}{0 0 0}
\pscustom[linestyle=none,fillstyle=solid,fillcolor=curcolor]
{
\newpath
\moveto(124.25877774,554.7622921)
\lineto(123.19649267,554.7622921)
\lineto(123.19649267,560.96576776)
\curveto(123.19649267,561.4339546)(123.18330829,561.88588495)(123.15693951,562.32155882)
\curveto(123.1343377,562.75723269)(123.08348362,563.10512152)(123.00437729,563.36522532)
\curveto(122.91773702,563.64483691)(122.79342706,563.85617125)(122.63144743,563.99922834)
\curveto(122.46946779,564.14228543)(122.23591575,564.21381397)(121.93079132,564.21381397)
\curveto(121.63320082,564.21381397)(121.33561032,564.08376207)(121.03801983,563.82365827)
\curveto(120.74042933,563.57005707)(120.44283883,563.24492731)(120.14524834,562.84826902)
\curveto(120.15654924,562.69870933)(120.16596666,562.52313927)(120.1735006,562.32155882)
\curveto(120.18103453,562.12648097)(120.1848015,561.93140312)(120.1848015,561.73632527)
\lineto(120.1848015,554.7622921)
\lineto(119.12251644,554.7622921)
\lineto(119.12251644,560.96576776)
\curveto(119.12251644,561.44695979)(119.10933205,561.90214144)(119.08296327,562.33131271)
\curveto(119.06036146,562.76698658)(119.00950739,563.11487541)(118.93040105,563.37497922)
\curveto(118.84376078,563.6545908)(118.71945083,563.86267384)(118.55747119,563.99922834)
\curveto(118.39549155,564.14228543)(118.16193951,564.21381397)(117.85681508,564.21381397)
\curveto(117.56675852,564.21381397)(117.27481848,564.09026467)(116.98099495,563.84316606)
\curveto(116.69093839,563.59606745)(116.40088183,563.28069159)(116.11082526,562.89703848)
\lineto(116.11082526,554.7622921)
\lineto(115.0485402,554.7622921)
\lineto(115.0485402,565.65739007)
\lineto(116.11082526,565.65739007)
\lineto(116.11082526,564.4479074)
\curveto(116.44231848,564.92259683)(116.7719282,565.29324475)(117.09965445,565.55985115)
\curveto(117.43114766,565.82645754)(117.7833592,565.95976074)(118.15628906,565.95976074)
\curveto(118.58572345,565.95976074)(118.94923589,565.80369846)(119.24682639,565.4915739)
\curveto(119.54818386,565.17944934)(119.77231847,564.74702677)(119.91923024,564.19430619)
\curveto(120.34866462,564.81855531)(120.74042933,565.26723437)(121.09452435,565.54034336)
\curveto(121.44861937,565.81995495)(121.82719969,565.95976074)(122.2302653,565.95976074)
\curveto(122.92338747,565.95976074)(123.43381168,565.59561542)(123.76153792,564.86732478)
\curveto(124.09303113,564.14553673)(124.25877774,563.1343832)(124.25877774,561.83386419)
\closepath
}
}
{
\newrgbcolor{curcolor}{0 0 0}
\pscustom[linestyle=none,fillstyle=solid,fillcolor=curcolor]
{
\newpath
\moveto(131.68347308,560.20496414)
\curveto(131.68347308,558.42975569)(131.4197853,557.02844646)(130.89240973,556.00103645)
\curveto(130.36503417,554.97362643)(129.65872761,554.45992143)(128.77349005,554.45992143)
\curveto(127.88071856,554.45992143)(127.17064503,554.97362643)(126.64326947,556.00103645)
\curveto(126.11966087,557.02844646)(125.85785657,558.42975569)(125.85785657,560.20496414)
\curveto(125.85785657,561.98017258)(126.11966087,563.38148181)(126.64326947,564.40889183)
\curveto(127.17064503,565.44280444)(127.88071856,565.95976074)(128.77349005,565.95976074)
\curveto(129.65872761,565.95976074)(130.36503417,565.44280444)(130.89240973,564.40889183)
\curveto(131.4197853,563.38148181)(131.68347308,561.98017258)(131.68347308,560.20496414)
\closepath
\moveto(130.5872853,560.20496414)
\curveto(130.5872853,561.61602726)(130.42718915,562.66294506)(130.10699684,563.34571754)
\curveto(129.78680453,564.03499261)(129.34230227,564.37963015)(128.77349005,564.37963015)
\curveto(128.1971439,564.37963015)(127.74887467,564.03499261)(127.42868236,563.34571754)
\curveto(127.11225703,562.66294506)(126.95404436,561.61602726)(126.95404436,560.20496414)
\curveto(126.95404436,558.83941918)(127.11414051,557.80225527)(127.43433282,557.09347241)
\curveto(127.75452512,556.39119215)(128.20091087,556.04005202)(128.77349005,556.04005202)
\curveto(129.3385353,556.04005202)(129.78115408,556.38794085)(130.10134639,557.08371852)
\curveto(130.42530566,557.78599879)(130.5872853,558.82641399)(130.5872853,560.20496414)
\closepath
}
}
{
\newrgbcolor{curcolor}{0 0 0}
\pscustom[linestyle=none,fillstyle=solid,fillcolor=curcolor]
{
\newpath
\moveto(137.26611903,563.6578421)
\lineto(137.20961451,563.6578421)
\curveto(137.05140184,563.72286805)(136.89695614,563.76838621)(136.7462774,563.79439659)
\curveto(136.59936564,563.82690957)(136.42420161,563.84316606)(136.22078532,563.84316606)
\curveto(135.89305908,563.84316606)(135.57663374,563.71636545)(135.27150931,563.46276425)
\curveto(134.96638487,563.21566564)(134.67256134,562.89378718)(134.39003872,562.49712889)
\lineto(134.39003872,554.7622921)
\lineto(133.32775365,554.7622921)
\lineto(133.32775365,565.65739007)
\lineto(134.39003872,565.65739007)
\lineto(134.39003872,564.0479978)
\curveto(134.81193917,564.63323135)(135.18298555,565.04614614)(135.50317786,565.28674215)
\curveto(135.82713713,565.53384077)(136.15674686,565.65739007)(136.49200704,565.65739007)
\curveto(136.67658849,565.65739007)(136.81031587,565.64763618)(136.89318917,565.62812839)
\curveto(136.97606247,565.6151232)(137.10037243,565.58586153)(137.26611903,565.54034336)
\closepath
}
}
{
\newrgbcolor{curcolor}{0 0 0}
\pscustom[linestyle=none,fillstyle=solid,fillcolor=curcolor]
{
\newpath
\moveto(143.72458537,565.65739007)
\lineto(140.04049036,550.74368837)
\lineto(138.90474941,550.74368837)
\lineto(140.08004352,555.28900229)
\lineto(137.56559217,565.65739007)
\lineto(138.71828448,565.65739007)
\lineto(140.65638968,557.58116704)
\lineto(142.61144623,565.65739007)
\closepath
}
}
{
\newrgbcolor{curcolor}{0 0 0}
\pscustom[linestyle=none,fillstyle=solid,fillcolor=curcolor]
{
\newpath
\moveto(155.76570005,555.81571249)
\curveto(155.55851679,555.65965021)(155.37016838,555.51334182)(155.2006548,555.37678733)
\curveto(155.0349082,555.24023283)(154.81642404,555.09717574)(154.54520232,554.94761605)
\curveto(154.31541725,554.82406675)(154.06491386,554.72002523)(153.79369214,554.63549149)
\curveto(153.52623739,554.54445516)(153.23053037,554.498937)(152.9065711,554.498937)
\curveto(152.29632223,554.498937)(151.7406944,554.64524538)(151.23968762,554.93786216)
\curveto(150.7424478,555.23698153)(150.30924644,555.70191708)(149.94008355,556.3326688)
\curveto(149.57845459,556.95041532)(149.29593196,557.73397802)(149.09251567,558.6833569)
\curveto(148.88909938,559.63923837)(148.78739124,560.74793082)(148.78739124,562.00943426)
\curveto(148.78739124,563.20591174)(148.88533242,564.27558863)(149.08121477,565.21846491)
\curveto(149.27709712,566.16134119)(149.55961974,566.95790908)(149.92878264,567.60816858)
\curveto(150.28664463,568.2389203)(150.7179625,568.72011233)(151.22273626,569.05174468)
\curveto(151.73127698,569.38337702)(152.29443874,569.5491932)(152.91222155,569.5491932)
\curveto(153.36425775,569.5491932)(153.81441046,569.45490557)(154.26267969,569.26633031)
\curveto(154.71471589,569.07775506)(155.21572268,568.74612271)(155.76570005,568.27143327)
\lineto(155.76570005,565.97926853)
\lineto(155.68094326,565.97926853)
\curveto(155.21760616,566.64903581)(154.75803603,567.13673044)(154.30223286,567.44235241)
\curveto(153.84642969,567.74797437)(153.3586073,567.90078536)(152.83876567,567.90078536)
\curveto(152.41309825,567.90078536)(152.02886748,567.78048735)(151.68607336,567.53989133)
\curveto(151.34704621,567.30579791)(151.04380526,566.93840129)(150.77635051,566.43770148)
\curveto(150.5164297,565.95000685)(150.31301341,565.33226032)(150.16610164,564.58446189)
\curveto(150.02295685,563.84316606)(149.95138445,562.98482351)(149.95138445,562.00943426)
\curveto(149.95138445,560.98852684)(150.03049078,560.11067651)(150.18870345,559.37588327)
\curveto(150.35068309,558.64109003)(150.55786635,558.04285129)(150.81025323,557.58116704)
\curveto(151.07394101,557.09997501)(151.38094893,556.74233228)(151.73127698,556.50823886)
\curveto(152.085372,556.28064804)(152.45830187,556.16685262)(152.85006657,556.16685262)
\curveto(153.38874304,556.16685262)(153.8935168,556.3261662)(154.36438784,556.64479336)
\curveto(154.83525888,556.96342051)(155.27599417,557.44136125)(155.68659372,558.07861556)
\lineto(155.76570005,558.07861556)
\closepath
}
}
{
\newrgbcolor{curcolor}{0 0 0}
\pscustom[linestyle=none,fillstyle=solid,fillcolor=curcolor]
{
\newpath
\moveto(162.64229916,560.20496414)
\curveto(162.64229916,558.42975569)(162.37861138,557.02844646)(161.85123582,556.00103645)
\curveto(161.32386025,554.97362643)(160.61755369,554.45992143)(159.73231614,554.45992143)
\curveto(158.83954464,554.45992143)(158.12947112,554.97362643)(157.60209555,556.00103645)
\curveto(157.07848695,557.02844646)(156.81668266,558.42975569)(156.81668266,560.20496414)
\curveto(156.81668266,561.98017258)(157.07848695,563.38148181)(157.60209555,564.40889183)
\curveto(158.12947112,565.44280444)(158.83954464,565.95976074)(159.73231614,565.95976074)
\curveto(160.61755369,565.95976074)(161.32386025,565.44280444)(161.85123582,564.40889183)
\curveto(162.37861138,563.38148181)(162.64229916,561.98017258)(162.64229916,560.20496414)
\closepath
\moveto(161.54611138,560.20496414)
\curveto(161.54611138,561.61602726)(161.38601523,562.66294506)(161.06582292,563.34571754)
\curveto(160.74563061,564.03499261)(160.30112835,564.37963015)(159.73231614,564.37963015)
\curveto(159.15596998,564.37963015)(158.70770075,564.03499261)(158.38750845,563.34571754)
\curveto(158.07108311,562.66294506)(157.91287044,561.61602726)(157.91287044,560.20496414)
\curveto(157.91287044,558.83941918)(158.07296659,557.80225527)(158.3931589,557.09347241)
\curveto(158.71335121,556.39119215)(159.15973695,556.04005202)(159.73231614,556.04005202)
\curveto(160.29736138,556.04005202)(160.73998016,556.38794085)(161.06017247,557.08371852)
\curveto(161.38413174,557.78599879)(161.54611138,558.82641399)(161.54611138,560.20496414)
\closepath
}
}
{
\newrgbcolor{curcolor}{0 0 0}
\pscustom[linestyle=none,fillstyle=solid,fillcolor=curcolor]
{
\newpath
\moveto(169.56410784,554.7622921)
\lineto(168.50182277,554.7622921)
\lineto(168.50182277,560.96576776)
\curveto(168.50182277,561.46646757)(168.48487141,561.93465442)(168.4509687,562.37032828)
\curveto(168.41706598,562.81250475)(168.35491101,563.15714228)(168.26450377,563.40424089)
\curveto(168.17032956,563.67734988)(168.0347187,563.87893033)(167.85767119,564.00898223)
\curveto(167.68062368,564.14553673)(167.45083861,564.21381397)(167.16831599,564.21381397)
\curveto(166.87825943,564.21381397)(166.57501848,564.09026467)(166.25859314,563.84316606)
\curveto(165.9421678,563.59606745)(165.63892685,563.28069159)(165.34887029,562.89703848)
\lineto(165.34887029,554.7622921)
\lineto(164.28658522,554.7622921)
\lineto(164.28658522,565.65739007)
\lineto(165.34887029,565.65739007)
\lineto(165.34887029,564.4479074)
\curveto(165.6803635,564.92259683)(166.02315762,565.29324475)(166.37725264,565.55985115)
\curveto(166.73134766,565.82645754)(167.0948601,565.95976074)(167.46778997,565.95976074)
\curveto(168.14961123,565.95976074)(168.66945286,565.60536931)(169.02731485,564.89658645)
\curveto(169.38517684,564.18780359)(169.56410784,563.16689617)(169.56410784,561.83386419)
\closepath
}
}
{
\newrgbcolor{curcolor}{0 0 0}
\pscustom[linestyle=none,fillstyle=solid,fillcolor=curcolor]
{
\newpath
\moveto(174.88682929,554.85983102)
\curveto(174.68717997,554.76879469)(174.46869581,554.69401485)(174.23137681,554.63549149)
\curveto(173.99782477,554.57696814)(173.78875803,554.54770646)(173.60417658,554.54770646)
\curveto(172.960025,554.54770646)(172.47031912,554.84682583)(172.13505894,555.44506457)
\curveto(171.79979876,556.04330332)(171.63216867,557.00243608)(171.63216867,558.32246288)
\lineto(171.63216867,564.11627505)
\lineto(170.9145612,564.11627505)
\lineto(170.9145612,565.65739007)
\lineto(171.63216867,565.65739007)
\lineto(171.63216867,568.78838958)
\lineto(172.69445373,568.78838958)
\lineto(172.69445373,565.65739007)
\lineto(174.88682929,565.65739007)
\lineto(174.88682929,564.11627505)
\lineto(172.69445373,564.11627505)
\lineto(172.69445373,559.15154374)
\curveto(172.69445373,558.57931538)(172.70198767,558.13063632)(172.71705554,557.80550657)
\curveto(172.73212342,557.48687941)(172.78486097,557.18776004)(172.87526821,556.90814846)
\curveto(172.95814151,556.64804465)(173.07115056,556.4562181)(173.21429536,556.3326688)
\curveto(173.36120713,556.21562208)(173.58345826,556.15709873)(173.88104875,556.15709873)
\curveto(174.0543293,556.15709873)(174.23514378,556.1993656)(174.42349219,556.28389933)
\curveto(174.61184061,556.37493566)(174.74745147,556.44971551)(174.83032477,556.50823886)
\lineto(174.88682929,556.50823886)
\closepath
}
}
{
\newrgbcolor{curcolor}{0 0 0}
\pscustom[linestyle=none,fillstyle=solid,fillcolor=curcolor]
{
\newpath
\moveto(180.10784767,563.6578421)
\lineto(180.05134315,563.6578421)
\curveto(179.89313048,563.72286805)(179.73868478,563.76838621)(179.58800605,563.79439659)
\curveto(179.44109428,563.82690957)(179.26593025,563.84316606)(179.06251397,563.84316606)
\curveto(178.73478772,563.84316606)(178.41836238,563.71636545)(178.11323795,563.46276425)
\curveto(177.80811351,563.21566564)(177.51428999,562.89378718)(177.23176736,562.49712889)
\lineto(177.23176736,554.7622921)
\lineto(176.1694823,554.7622921)
\lineto(176.1694823,565.65739007)
\lineto(177.23176736,565.65739007)
\lineto(177.23176736,564.0479978)
\curveto(177.65366781,564.63323135)(178.02471419,565.04614614)(178.3449065,565.28674215)
\curveto(178.66886578,565.53384077)(178.9984755,565.65739007)(179.33373568,565.65739007)
\curveto(179.51831713,565.65739007)(179.65204451,565.64763618)(179.73491781,565.62812839)
\curveto(179.81779111,565.6151232)(179.94210107,565.58586153)(180.10784767,565.54034336)
\closepath
}
}
{
\newrgbcolor{curcolor}{0 0 0}
\pscustom[linestyle=none,fillstyle=solid,fillcolor=curcolor]
{
\newpath
\moveto(186.48720768,560.20496414)
\curveto(186.48720768,558.42975569)(186.2235199,557.02844646)(185.69614433,556.00103645)
\curveto(185.16876877,554.97362643)(184.46246221,554.45992143)(183.57722465,554.45992143)
\curveto(182.68445316,554.45992143)(181.97437963,554.97362643)(181.44700407,556.00103645)
\curveto(180.92339547,557.02844646)(180.66159117,558.42975569)(180.66159117,560.20496414)
\curveto(180.66159117,561.98017258)(180.92339547,563.38148181)(181.44700407,564.40889183)
\curveto(181.97437963,565.44280444)(182.68445316,565.95976074)(183.57722465,565.95976074)
\curveto(184.46246221,565.95976074)(185.16876877,565.44280444)(185.69614433,564.40889183)
\curveto(186.2235199,563.38148181)(186.48720768,561.98017258)(186.48720768,560.20496414)
\closepath
\moveto(185.3910199,560.20496414)
\curveto(185.3910199,561.61602726)(185.23092375,562.66294506)(184.91073144,563.34571754)
\curveto(184.59053913,564.03499261)(184.14603687,564.37963015)(183.57722465,564.37963015)
\curveto(183.0008785,564.37963015)(182.55260927,564.03499261)(182.23241696,563.34571754)
\curveto(181.91599162,562.66294506)(181.75777895,561.61602726)(181.75777895,560.20496414)
\curveto(181.75777895,558.83941918)(181.91787511,557.80225527)(182.23806742,557.09347241)
\curveto(182.55825972,556.39119215)(183.00464547,556.04005202)(183.57722465,556.04005202)
\curveto(184.1422699,556.04005202)(184.58488868,556.38794085)(184.90508099,557.08371852)
\curveto(185.22904026,557.78599879)(185.3910199,558.82641399)(185.3910199,560.20496414)
\closepath
}
}
{
\newrgbcolor{curcolor}{0 0 0}
\pscustom[linestyle=none,fillstyle=solid,fillcolor=curcolor]
{
\newpath
\moveto(189.20507971,554.7622921)
\lineto(188.14279464,554.7622921)
\lineto(188.14279464,569.9393489)
\lineto(189.20507971,569.9393489)
\closepath
}
}
{
\newrgbcolor{curcolor}{0 0 0}
\pscustom[linestyle=none,fillstyle=solid,fillcolor=curcolor]
{
\newpath
\moveto(192.3806336,554.7622921)
\lineto(191.31834853,554.7622921)
\lineto(191.31834853,569.9393489)
\lineto(192.3806336,569.9393489)
\closepath
}
}
{
\newrgbcolor{curcolor}{0 0 0}
\pscustom[linestyle=none,fillstyle=solid,fillcolor=curcolor]
{
\newpath
\moveto(199.7657691,560.01964018)
\lineto(195.11544671,560.01964018)
\curveto(195.11544671,559.34987289)(195.17383472,558.76463934)(195.29061073,558.26393952)
\curveto(195.40738675,557.7697423)(195.56748291,557.36333011)(195.7708992,557.04470295)
\curveto(195.96678155,556.73257839)(196.1984501,556.49848497)(196.46590485,556.34242269)
\curveto(196.73712657,556.18636041)(197.03471707,556.10832927)(197.35867634,556.10832927)
\curveto(197.78811073,556.10832927)(198.2194286,556.25463766)(198.65262996,556.54725443)
\curveto(199.08959829,556.8463738)(199.40037317,557.13899058)(199.58495462,557.42510476)
\lineto(199.64145914,557.42510476)
\lineto(199.64145914,555.42555679)
\curveto(199.28359715,555.16545299)(198.91820123,554.94761605)(198.54527136,554.77204599)
\curveto(198.1723415,554.59647592)(197.78057679,554.50869089)(197.36997725,554.50869089)
\curveto(196.32276005,554.50869089)(195.50532793,554.99638552)(194.91768087,555.97177477)
\curveto(194.33003381,556.95366662)(194.03621028,558.34522196)(194.03621028,560.14644078)
\curveto(194.03621028,561.92815182)(194.31684942,563.34246624)(194.8781277,564.38938404)
\curveto(195.44317295,565.43630184)(196.18526571,565.95976074)(197.10440598,565.95976074)
\curveto(197.95574082,565.95976074)(198.61119331,565.53058947)(199.07076344,564.67224692)
\curveto(199.53410055,563.81390438)(199.7657691,562.59466781)(199.7657691,561.01453722)
\closepath
\moveto(198.73173629,561.42420071)
\curveto(198.72796933,562.38658477)(198.58670801,563.1311319)(198.30795236,563.6578421)
\curveto(198.03296367,564.1845523)(197.6129467,564.4479074)(197.04790146,564.4479074)
\curveto(196.47908924,564.4479074)(196.02516956,564.15854192)(195.68614241,563.57981096)
\curveto(195.35088223,563.00108)(195.16065033,562.28254325)(195.11544671,561.42420071)
\closepath
}
}
{
\newrgbcolor{curcolor}{0 0 0}
\pscustom[linestyle=none,fillstyle=solid,fillcolor=curcolor]
{
\newpath
\moveto(205.31451755,563.6578421)
\lineto(205.25801303,563.6578421)
\curveto(205.09980036,563.72286805)(204.94535466,563.76838621)(204.79467593,563.79439659)
\curveto(204.64776416,563.82690957)(204.47260013,563.84316606)(204.26918385,563.84316606)
\curveto(203.9414576,563.84316606)(203.62503226,563.71636545)(203.31990783,563.46276425)
\curveto(203.0147834,563.21566564)(202.72095987,562.89378718)(202.43843724,562.49712889)
\lineto(202.43843724,554.7622921)
\lineto(201.37615218,554.7622921)
\lineto(201.37615218,565.65739007)
\lineto(202.43843724,565.65739007)
\lineto(202.43843724,564.0479978)
\curveto(202.86033769,564.63323135)(203.23138407,565.04614614)(203.55157638,565.28674215)
\curveto(203.87553566,565.53384077)(204.20514538,565.65739007)(204.54040556,565.65739007)
\curveto(204.72498701,565.65739007)(204.85871439,565.64763618)(204.94158769,565.62812839)
\curveto(205.02446099,565.6151232)(205.14877095,565.58586153)(205.31451755,565.54034336)
\closepath
}
}
{
\newrgbcolor{curcolor}{0 0 0}
\pscustom[linestyle=none,fillstyle=solid,fillcolor=curcolor]
{
\newpath
\moveto(244.59399441,625.68560085)
\lineto(238.50345154,625.68560085)
\lineto(238.50345154,627.23619517)
\lineto(240.52327443,627.23619517)
\lineto(240.52327443,639.32471009)
\lineto(238.50345154,639.32471009)
\lineto(238.50345154,640.87530441)
\lineto(244.59399441,640.87530441)
\lineto(244.59399441,639.32471009)
\lineto(242.57417152,639.32471009)
\lineto(242.57417152,627.23619517)
\lineto(244.59399441,627.23619517)
\closepath
}
}
{
\newrgbcolor{curcolor}{0 0 0}
\pscustom[linestyle=none,fillstyle=solid,fillcolor=curcolor]
{
\newpath
\moveto(257.60372527,625.68560085)
\lineto(255.65640884,625.68560085)
\lineto(255.65640884,632.17361392)
\curveto(255.65640884,632.69727955)(255.62533465,633.18694091)(255.56318625,633.64259801)
\curveto(255.50103785,634.10505597)(255.38709913,634.46550114)(255.22137007,634.72393352)
\curveto(255.04873563,635.00956932)(254.80014205,635.22039574)(254.47558931,635.35641279)
\curveto(254.15103657,635.49923068)(253.72980855,635.57063963)(253.21190525,635.57063963)
\curveto(252.68019119,635.57063963)(252.12430831,635.44142344)(251.5442566,635.18299105)
\curveto(250.9642049,634.92455867)(250.40832202,634.59471733)(249.87660796,634.19346705)
\lineto(249.87660796,625.68560085)
\lineto(247.92929154,625.68560085)
\lineto(247.92929154,637.08042884)
\lineto(249.87660796,637.08042884)
\lineto(249.87660796,635.81547032)
\curveto(250.48428117,636.31193253)(251.11267052,636.69958111)(251.76177599,636.97841605)
\curveto(252.41088147,637.257251)(253.07725039,637.39666847)(253.76088275,637.39666847)
\curveto(255.01075606,637.39666847)(255.96369814,637.02602202)(256.61970899,636.28472912)
\curveto(257.27571984,635.54343622)(257.60372527,634.47570242)(257.60372527,633.0815277)
\closepath
}
}
{
\newrgbcolor{curcolor}{0 0 0}
\pscustom[linestyle=none,fillstyle=solid,fillcolor=curcolor]
{
\newpath
\moveto(267.36102477,625.78761363)
\curveto(266.99503977,625.6924017)(266.59452788,625.6141919)(266.1594891,625.55298423)
\curveto(265.7313557,625.49177656)(265.34810726,625.46117273)(265.00974377,625.46117273)
\curveto(263.82892423,625.46117273)(262.93122517,625.77401193)(262.31664658,626.39969034)
\curveto(261.70206799,627.02536875)(261.3947787,628.02849446)(261.3947787,629.40906747)
\lineto(261.3947787,635.46862685)
\lineto(260.0793043,635.46862685)
\lineto(260.0793043,637.08042884)
\lineto(261.3947787,637.08042884)
\lineto(261.3947787,640.35503921)
\lineto(263.34209512,640.35503921)
\lineto(263.34209512,637.08042884)
\lineto(267.36102477,637.08042884)
\lineto(267.36102477,635.46862685)
\lineto(263.34209512,635.46862685)
\lineto(263.34209512,630.27617614)
\curveto(263.34209512,629.67770114)(263.35590588,629.20844233)(263.38352739,628.86839972)
\curveto(263.4111489,628.53515795)(263.50782418,628.22231875)(263.67355324,627.9298821)
\curveto(263.82547154,627.65784801)(264.03263286,627.45722287)(264.2950372,627.32800668)
\curveto(264.56434692,627.20559133)(264.97176419,627.14438366)(265.517289,627.14438366)
\curveto(265.83493636,627.14438366)(266.16639448,627.1885892)(266.51166335,627.27700028)
\curveto(266.85693222,627.37221221)(267.10552581,627.45042202)(267.25744411,627.51162969)
\lineto(267.36102477,627.51162969)
\closepath
}
}
{
\newrgbcolor{curcolor}{0 0 0}
\pscustom[linestyle=none,fillstyle=solid,fillcolor=curcolor]
{
\newpath
\moveto(279.39709597,631.18408992)
\lineto(270.87240757,631.18408992)
\curveto(270.87240757,630.48360213)(270.97944092,629.87152543)(271.19350762,629.3478598)
\curveto(271.40757432,628.83099503)(271.70105286,628.40594176)(272.07394324,628.0727)
\curveto(272.43302287,627.74625909)(272.85770358,627.50142841)(273.34798537,627.33820795)
\curveto(273.84517254,627.1749875)(274.39069736,627.09337727)(274.98455982,627.09337727)
\curveto(275.77177284,627.09337727)(276.56243855,627.24639645)(277.35655695,627.5524348)
\curveto(278.15758073,627.865274)(278.72727437,628.17131236)(279.06563786,628.47054986)
\lineto(279.16921852,628.47054986)
\lineto(279.16921852,626.37928778)
\curveto(278.51320767,626.10725369)(277.84338606,625.87942514)(277.1597537,625.69580213)
\curveto(276.47612133,625.51217912)(275.75796208,625.42036761)(275.00527595,625.42036761)
\curveto(273.08558103,625.42036761)(271.58711414,625.93043153)(270.50987526,626.95055937)
\curveto(269.43263639,627.97748807)(268.89401695,629.43287045)(268.89401695,631.31670653)
\curveto(268.89401695,633.18014006)(269.40846757,634.65932543)(270.4373688,635.75426264)
\curveto(271.47317541,636.84919986)(272.83353476,637.39666847)(274.51844684,637.39666847)
\curveto(276.07906213,637.39666847)(277.2805978,636.94781222)(278.12305384,636.05009972)
\curveto(278.97241526,635.15238722)(279.39709597,633.87722742)(279.39709597,632.22462031)
\closepath
\moveto(277.50156988,632.65307401)
\curveto(277.4946645,633.65960014)(277.23571285,634.43829773)(276.72471492,634.98916676)
\curveto(276.22062237,635.5400358)(275.45067279,635.81547032)(274.41486618,635.81547032)
\curveto(273.37215419,635.81547032)(272.54005622,635.51283239)(271.91857225,634.90755654)
\curveto(271.30399366,634.30228068)(270.9552721,633.55078651)(270.87240757,632.65307401)
\closepath
}
}
{
\newrgbcolor{curcolor}{0 0 0}
\pscustom[linestyle=none,fillstyle=solid,fillcolor=curcolor]
{
\newpath
\moveto(289.5687173,634.98916676)
\lineto(289.46513664,634.98916676)
\curveto(289.17511079,635.05717529)(288.89199032,635.10478125)(288.61577522,635.13198466)
\curveto(288.3464655,635.16598892)(288.02536546,635.18299105)(287.65247508,635.18299105)
\curveto(287.05170724,635.18299105)(286.47165554,635.05037443)(285.91231997,634.7851412)
\curveto(285.3529844,634.52670881)(284.81436496,634.19006662)(284.29646166,633.77521463)
\lineto(284.29646166,625.68560085)
\lineto(282.34914523,625.68560085)
\lineto(282.34914523,637.08042884)
\lineto(284.29646166,637.08042884)
\lineto(284.29646166,635.3972179)
\curveto(285.06986393,636.00929461)(285.7500436,636.44114872)(286.33700068,636.69278026)
\curveto(286.93086314,636.95121265)(287.53508366,637.08042884)(288.14966225,637.08042884)
\curveto(288.48802574,637.08042884)(288.73316664,637.07022756)(288.88508494,637.049825)
\curveto(289.03700324,637.0362233)(289.2648807,637.00561946)(289.5687173,636.9580135)
\closepath
}
}
{
\newrgbcolor{curcolor}{0 0 0}
\pscustom[linestyle=none,fillstyle=solid,fillcolor=curcolor]
{
\newpath
\moveto(299.95785968,626.39969034)
\curveto(299.3087542,626.09365199)(298.69072292,625.85562216)(298.10376585,625.68560085)
\curveto(297.52371414,625.51557954)(296.90568287,625.43056889)(296.24967201,625.43056889)
\curveto(295.41412135,625.43056889)(294.64762446,625.5495838)(293.95018134,625.78761363)
\curveto(293.25273822,626.03244432)(292.65542308,626.39969034)(292.15823591,626.8893517)
\curveto(291.65414336,627.37901307)(291.26398953,627.99789062)(290.98777444,628.74598437)
\curveto(290.71155934,629.49407812)(290.57345179,630.36798764)(290.57345179,631.36771293)
\curveto(290.57345179,633.23114645)(291.0913551,634.69332969)(292.12716171,635.75426264)
\curveto(293.16987369,636.8151956)(294.5440438,637.34566208)(296.24967201,637.34566208)
\curveto(296.91258824,637.34566208)(297.56169372,637.25385057)(298.19698844,637.07022756)
\curveto(298.83918854,636.88660455)(299.42614562,636.66217642)(299.95785968,636.39694318)
\lineto(299.95785968,634.264876)
\lineto(299.85427902,634.264876)
\curveto(299.26041656,634.7205331)(298.64583797,635.07077699)(298.01054325,635.31560767)
\curveto(297.38215391,635.56043835)(296.76757532,635.6828537)(296.16680749,635.6828537)
\curveto(295.0619471,635.6828537)(294.18841686,635.31560767)(293.54621676,634.58111563)
\curveto(292.91092204,633.85342443)(292.59327468,632.7822902)(292.59327468,631.36771293)
\curveto(292.59327468,629.99394077)(292.90401666,628.93640824)(293.52550063,628.19511534)
\curveto(294.15388997,627.46062329)(295.03432559,627.09337727)(296.16680749,627.09337727)
\curveto(296.560414,627.09337727)(296.96092589,627.14438366)(297.36834315,627.24639645)
\curveto(297.77576042,627.34840923)(298.14174542,627.48102585)(298.46629816,627.64424631)
\curveto(298.74941863,627.7870642)(299.01527566,627.93668295)(299.26386925,628.09310256)
\curveto(299.51246283,628.25632301)(299.70926609,628.39574048)(299.85427902,628.51135497)
\lineto(299.95785968,628.51135497)
\closepath
}
}
{
\newrgbcolor{curcolor}{0 0 0}
\pscustom[linestyle=none,fillstyle=solid,fillcolor=curcolor]
{
\newpath
\moveto(312.31503084,631.37791421)
\curveto(312.31503084,629.52128153)(311.83165442,628.05569787)(310.86490159,626.98116321)
\curveto(309.89814875,625.90662855)(308.60339049,625.36936122)(306.9806268,625.36936122)
\curveto(305.34405236,625.36936122)(304.04238872,625.90662855)(303.07563588,626.98116321)
\curveto(302.11578842,628.05569787)(301.63586469,629.52128153)(301.63586469,631.37791421)
\curveto(301.63586469,633.23454688)(302.11578842,634.70013054)(303.07563588,635.7746652)
\curveto(304.04238872,636.85600071)(305.34405236,637.39666847)(306.9806268,637.39666847)
\curveto(308.60339049,637.39666847)(309.89814875,636.85600071)(310.86490159,635.7746652)
\curveto(311.83165442,634.70013054)(312.31503084,633.23454688)(312.31503084,631.37791421)
\closepath
\moveto(310.30556602,631.37791421)
\curveto(310.30556602,632.85369915)(310.01208748,633.94863637)(309.4251304,634.66272585)
\curveto(308.83817332,635.3836162)(308.02333879,635.74406137)(306.9806268,635.74406137)
\curveto(305.92410406,635.74406137)(305.10236415,635.3836162)(304.51540707,634.66272585)
\curveto(303.93535537,633.94863637)(303.64532952,632.85369915)(303.64532952,631.37791421)
\curveto(303.64532952,629.94973523)(303.93880806,628.86499929)(304.52576514,628.12370639)
\curveto(305.11272221,627.38921435)(305.93100944,627.02196832)(306.9806268,627.02196832)
\curveto(308.01643341,627.02196832)(308.82781525,627.38581392)(309.41477233,628.11350511)
\curveto(310.00863479,628.84799716)(310.30556602,629.93613352)(310.30556602,631.37791421)
\closepath
}
}
{
\newrgbcolor{curcolor}{0 0 0}
\pscustom[linestyle=none,fillstyle=solid,fillcolor=curcolor]
{
\newpath
\moveto(325.00365894,625.68560085)
\lineto(323.05634252,625.68560085)
\lineto(323.05634252,632.17361392)
\curveto(323.05634252,632.69727955)(323.02526832,633.18694091)(322.96311992,633.64259801)
\curveto(322.90097152,634.10505597)(322.7870328,634.46550114)(322.62130374,634.72393352)
\curveto(322.44866931,635.00956932)(322.20007572,635.22039574)(321.87552298,635.35641279)
\curveto(321.55097024,635.49923068)(321.12974222,635.57063963)(320.61183892,635.57063963)
\curveto(320.08012486,635.57063963)(319.52424198,635.44142344)(318.94419028,635.18299105)
\curveto(318.36413857,634.92455867)(317.80825569,634.59471733)(317.27654163,634.19346705)
\lineto(317.27654163,625.68560085)
\lineto(315.32922521,625.68560085)
\lineto(315.32922521,637.08042884)
\lineto(317.27654163,637.08042884)
\lineto(317.27654163,635.81547032)
\curveto(317.88421484,636.31193253)(318.51260419,636.69958111)(319.16170966,636.97841605)
\curveto(319.81081514,637.257251)(320.47718406,637.39666847)(321.16081642,637.39666847)
\curveto(322.41068973,637.39666847)(323.36363181,637.02602202)(324.01964266,636.28472912)
\curveto(324.67565352,635.54343622)(325.00365894,634.47570242)(325.00365894,633.0815277)
\closepath
}
}
{
\newrgbcolor{curcolor}{0 0 0}
\pscustom[linestyle=none,fillstyle=solid,fillcolor=curcolor]
{
\newpath
\moveto(338.42771384,625.68560085)
\lineto(336.48039741,625.68560085)
\lineto(336.48039741,632.17361392)
\curveto(336.48039741,632.69727955)(336.44932322,633.18694091)(336.38717482,633.64259801)
\curveto(336.32502642,634.10505597)(336.2110877,634.46550114)(336.04535864,634.72393352)
\curveto(335.8727242,635.00956932)(335.62413062,635.22039574)(335.29957788,635.35641279)
\curveto(334.97502514,635.49923068)(334.55379712,635.57063963)(334.03589381,635.57063963)
\curveto(333.50417976,635.57063963)(332.94829687,635.44142344)(332.36824517,635.18299105)
\curveto(331.78819347,634.92455867)(331.23231059,634.59471733)(330.70059653,634.19346705)
\lineto(330.70059653,625.68560085)
\lineto(328.7532801,625.68560085)
\lineto(328.7532801,637.08042884)
\lineto(330.70059653,637.08042884)
\lineto(330.70059653,635.81547032)
\curveto(331.30826974,636.31193253)(331.93665909,636.69958111)(332.58576456,636.97841605)
\curveto(333.23487004,637.257251)(333.90123896,637.39666847)(334.58487132,637.39666847)
\curveto(335.83474463,637.39666847)(336.78768671,637.02602202)(337.44369756,636.28472912)
\curveto(338.09970841,635.54343622)(338.42771384,634.47570242)(338.42771384,633.0815277)
\closepath
}
}
{
\newrgbcolor{curcolor}{0 0 0}
\pscustom[linestyle=none,fillstyle=solid,fillcolor=curcolor]
{
\newpath
\moveto(351.8621268,631.18408992)
\lineto(343.33743841,631.18408992)
\curveto(343.33743841,630.48360213)(343.44447175,629.87152543)(343.65853845,629.3478598)
\curveto(343.87260515,628.83099503)(344.16608369,628.40594176)(344.53897407,628.0727)
\curveto(344.8980537,627.74625909)(345.32273441,627.50142841)(345.8130162,627.33820795)
\curveto(346.31020338,627.1749875)(346.85572819,627.09337727)(347.44959065,627.09337727)
\curveto(348.23680367,627.09337727)(349.02746938,627.24639645)(349.82158778,627.5524348)
\curveto(350.62261156,627.865274)(351.1923052,628.17131236)(351.53066869,628.47054986)
\lineto(351.63424935,628.47054986)
\lineto(351.63424935,626.37928778)
\curveto(350.9782385,626.10725369)(350.30841689,625.87942514)(349.62478453,625.69580213)
\curveto(348.94115216,625.51217912)(348.22299291,625.42036761)(347.47030678,625.42036761)
\curveto(345.55061186,625.42036761)(344.05214497,625.93043153)(342.97490609,626.95055937)
\curveto(341.89766722,627.97748807)(341.35904778,629.43287045)(341.35904778,631.31670653)
\curveto(341.35904778,633.18014006)(341.8734984,634.65932543)(342.90239963,635.75426264)
\curveto(343.93820624,636.84919986)(345.29856559,637.39666847)(346.98347767,637.39666847)
\curveto(348.54409296,637.39666847)(349.74562863,636.94781222)(350.58808467,636.05009972)
\curveto(351.43744609,635.15238722)(351.8621268,633.87722742)(351.8621268,632.22462031)
\closepath
\moveto(349.96660071,632.65307401)
\curveto(349.95969533,633.65960014)(349.70074368,634.43829773)(349.18974575,634.98916676)
\curveto(348.6856532,635.5400358)(347.91570362,635.81547032)(346.87989701,635.81547032)
\curveto(345.83718502,635.81547032)(345.00508705,635.51283239)(344.38360308,634.90755654)
\curveto(343.76902449,634.30228068)(343.42030293,633.55078651)(343.33743841,632.65307401)
\closepath
}
}
{
\newrgbcolor{curcolor}{0 0 0}
\pscustom[linestyle=none,fillstyle=solid,fillcolor=curcolor]
{
\newpath
\moveto(363.36993866,626.39969034)
\curveto(362.72083319,626.09365199)(362.10280191,625.85562216)(361.51584483,625.68560085)
\curveto(360.93579313,625.51557954)(360.31776185,625.43056889)(359.661751,625.43056889)
\curveto(358.82620033,625.43056889)(358.05970344,625.5495838)(357.36226032,625.78761363)
\curveto(356.66481721,626.03244432)(356.06750206,626.39969034)(355.57031489,626.8893517)
\curveto(355.06622234,627.37901307)(354.67606852,627.99789062)(354.39985342,628.74598437)
\curveto(354.12363832,629.49407812)(353.98553078,630.36798764)(353.98553078,631.36771293)
\curveto(353.98553078,633.23114645)(354.50343408,634.69332969)(355.53924069,635.75426264)
\curveto(356.58195268,636.8151956)(357.95612278,637.34566208)(359.661751,637.34566208)
\curveto(360.32466723,637.34566208)(360.9737727,637.25385057)(361.60906742,637.07022756)
\curveto(362.25126752,636.88660455)(362.8382246,636.66217642)(363.36993866,636.39694318)
\lineto(363.36993866,634.264876)
\lineto(363.266358,634.264876)
\curveto(362.67249554,634.7205331)(362.05791696,635.07077699)(361.42262223,635.31560767)
\curveto(360.79423289,635.56043835)(360.1796543,635.6828537)(359.57888647,635.6828537)
\curveto(358.47402609,635.6828537)(357.60049584,635.31560767)(356.95829575,634.58111563)
\curveto(356.32300103,633.85342443)(356.00535367,632.7822902)(356.00535367,631.36771293)
\curveto(356.00535367,629.99394077)(356.31609565,628.93640824)(356.93757961,628.19511534)
\curveto(357.56596896,627.46062329)(358.44640458,627.09337727)(359.57888647,627.09337727)
\curveto(359.97249298,627.09337727)(360.37300487,627.14438366)(360.78042214,627.24639645)
\curveto(361.1878394,627.34840923)(361.55382441,627.48102585)(361.87837714,627.64424631)
\curveto(362.16149762,627.7870642)(362.42735465,627.93668295)(362.67594823,628.09310256)
\curveto(362.92454182,628.25632301)(363.12134507,628.39574048)(363.266358,628.51135497)
\lineto(363.36993866,628.51135497)
\closepath
}
}
{
\newrgbcolor{curcolor}{0 0 0}
\pscustom[linestyle=none,fillstyle=solid,fillcolor=curcolor]
{
\newpath
\moveto(371.87390924,625.78761363)
\curveto(371.50792423,625.6924017)(371.10741235,625.6141919)(370.67237357,625.55298423)
\curveto(370.24424017,625.49177656)(369.86099172,625.46117273)(369.52262823,625.46117273)
\curveto(368.3418087,625.46117273)(367.44410964,625.77401193)(366.82953105,626.39969034)
\curveto(366.21495246,627.02536875)(365.90766316,628.02849446)(365.90766316,629.40906747)
\lineto(365.90766316,635.46862685)
\lineto(364.59218877,635.46862685)
\lineto(364.59218877,637.08042884)
\lineto(365.90766316,637.08042884)
\lineto(365.90766316,640.35503921)
\lineto(367.85497959,640.35503921)
\lineto(367.85497959,637.08042884)
\lineto(371.87390924,637.08042884)
\lineto(371.87390924,635.46862685)
\lineto(367.85497959,635.46862685)
\lineto(367.85497959,630.27617614)
\curveto(367.85497959,629.67770114)(367.86879035,629.20844233)(367.89641185,628.86839972)
\curveto(367.92403336,628.53515795)(368.02070865,628.22231875)(368.18643771,627.9298821)
\curveto(368.33835601,627.65784801)(368.54551733,627.45722287)(368.80792167,627.32800668)
\curveto(369.07723139,627.20559133)(369.48464866,627.14438366)(370.03017347,627.14438366)
\curveto(370.34782083,627.14438366)(370.67927895,627.1885892)(371.02454782,627.27700028)
\curveto(371.36981669,627.37221221)(371.61841027,627.45042202)(371.77032858,627.51162969)
\lineto(371.87390924,627.51162969)
\closepath
}
}
{
\newrgbcolor{curcolor}{0.7019608 0.7019608 0.7019608}
\pscustom[linestyle=none,fillstyle=solid,fillcolor=curcolor]
{
\newpath
\moveto(460.41379324,546.46501377)
\lineto(460.41379324,453.53092873)
}
}
{
\newrgbcolor{curcolor}{0 0 0}
\pscustom[linewidth=2.64566925,linecolor=curcolor]
{
\newpath
\moveto(460.41379324,546.46501377)
\lineto(460.41379324,453.53092873)
}
}
{
\newrgbcolor{curcolor}{0.80000001 0.80000001 0.80000001}
\pscustom[linestyle=none,fillstyle=solid,fillcolor=curcolor]
{
\newpath
\moveto(395.94453211,578.26067966)
\lineto(524.88303462,578.26067966)
\lineto(524.88303462,542.42247713)
\lineto(395.94453211,542.42247713)
\closepath
}
}
{
\newrgbcolor{curcolor}{0 0 0}
\pscustom[linewidth=0.27058431,linecolor=curcolor]
{
\newpath
\moveto(395.94453211,578.26067966)
\lineto(524.88303462,578.26067966)
\lineto(524.88303462,542.42247713)
\lineto(395.94453211,542.42247713)
\closepath
}
}
{
\newrgbcolor{curcolor}{0 0 0}
\pscustom[linestyle=none,fillstyle=solid,fillcolor=curcolor]
{
\newpath
\moveto(414.49255153,554.7622921)
\lineto(413.37376194,554.7622921)
\lineto(413.37376194,567.27653623)
\lineto(411.03447462,558.76138804)
\lineto(410.36772122,558.76138804)
\lineto(408.04538525,567.27653623)
\lineto(408.04538525,554.7622921)
\lineto(407.00005155,554.7622921)
\lineto(407.00005155,569.2858381)
\lineto(408.52567371,569.2858381)
\lineto(410.76890335,561.19986118)
\lineto(412.9386771,569.2858381)
\lineto(414.49255153,569.2858381)
\closepath
}
}
{
\newrgbcolor{curcolor}{0 0 0}
\pscustom[linestyle=none,fillstyle=solid,fillcolor=curcolor]
{
\newpath
\moveto(421.9511485,560.01964018)
\lineto(417.30082611,560.01964018)
\curveto(417.30082611,559.34987289)(417.35921412,558.76463934)(417.47599014,558.26393952)
\curveto(417.59276616,557.7697423)(417.75286231,557.36333011)(417.9562786,557.04470295)
\curveto(418.15216095,556.73257839)(418.38382951,556.49848497)(418.65128426,556.34242269)
\curveto(418.92250597,556.18636041)(419.22009647,556.10832927)(419.54405575,556.10832927)
\curveto(419.97349014,556.10832927)(420.40480801,556.25463766)(420.83800937,556.54725443)
\curveto(421.27497769,556.8463738)(421.58575258,557.13899058)(421.77033402,557.42510476)
\lineto(421.82683855,557.42510476)
\lineto(421.82683855,555.42555679)
\curveto(421.46897656,555.16545299)(421.10358063,554.94761605)(420.73065077,554.77204599)
\curveto(420.3577209,554.59647592)(419.9659562,554.50869089)(419.55535665,554.50869089)
\curveto(418.50813946,554.50869089)(417.69070733,554.99638552)(417.10306028,555.97177477)
\curveto(416.51541322,556.95366662)(416.22158969,558.34522196)(416.22158969,560.14644078)
\curveto(416.22158969,561.92815182)(416.50222883,563.34246624)(417.06350711,564.38938404)
\curveto(417.62855236,565.43630184)(418.37064512,565.95976074)(419.28978539,565.95976074)
\curveto(420.14112023,565.95976074)(420.79657271,565.53058947)(421.25614285,564.67224692)
\curveto(421.71947995,563.81390438)(421.9511485,562.59466781)(421.9511485,561.01453722)
\closepath
\moveto(420.9171157,561.42420071)
\curveto(420.91334873,562.38658477)(420.77208742,563.1311319)(420.49333176,563.6578421)
\curveto(420.21834308,564.1845523)(419.79832611,564.4479074)(419.23328086,564.4479074)
\curveto(418.66446865,564.4479074)(418.21054896,564.15854192)(417.87152181,563.57981096)
\curveto(417.53626163,563.00108)(417.34602973,562.28254325)(417.30082611,561.42420071)
\closepath
}
}
{
\newrgbcolor{curcolor}{0 0 0}
\pscustom[linestyle=none,fillstyle=solid,fillcolor=curcolor]
{
\newpath
\moveto(432.77176569,554.7622921)
\lineto(431.70948063,554.7622921)
\lineto(431.70948063,560.96576776)
\curveto(431.70948063,561.4339546)(431.69629624,561.88588495)(431.66992746,562.32155882)
\curveto(431.64732565,562.75723269)(431.59647158,563.10512152)(431.51736524,563.36522532)
\curveto(431.43072497,563.64483691)(431.30641502,563.85617125)(431.14443538,563.99922834)
\curveto(430.98245574,564.14228543)(430.74890371,564.21381397)(430.44377927,564.21381397)
\curveto(430.14618878,564.21381397)(429.84859828,564.08376207)(429.55100778,563.82365827)
\curveto(429.25341728,563.57005707)(428.95582679,563.24492731)(428.65823629,562.84826902)
\curveto(428.66953719,562.69870933)(428.67895461,562.52313927)(428.68648855,562.32155882)
\curveto(428.69402249,562.12648097)(428.69778946,561.93140312)(428.69778946,561.73632527)
\lineto(428.69778946,554.7622921)
\lineto(427.63550439,554.7622921)
\lineto(427.63550439,560.96576776)
\curveto(427.63550439,561.44695979)(427.62232,561.90214144)(427.59595122,562.33131271)
\curveto(427.57334941,562.76698658)(427.52249534,563.11487541)(427.44338901,563.37497922)
\curveto(427.35674873,563.6545908)(427.23243878,563.86267384)(427.07045914,563.99922834)
\curveto(426.9084795,564.14228543)(426.67492747,564.21381397)(426.36980303,564.21381397)
\curveto(426.07974647,564.21381397)(425.78780643,564.09026467)(425.4939829,563.84316606)
\curveto(425.20392634,563.59606745)(424.91386978,563.28069159)(424.62381322,562.89703848)
\lineto(424.62381322,554.7622921)
\lineto(423.56152815,554.7622921)
\lineto(423.56152815,565.65739007)
\lineto(424.62381322,565.65739007)
\lineto(424.62381322,564.4479074)
\curveto(424.95530643,564.92259683)(425.28491616,565.29324475)(425.6126424,565.55985115)
\curveto(425.94413561,565.82645754)(426.29634715,565.95976074)(426.66927702,565.95976074)
\curveto(427.0987114,565.95976074)(427.46222385,565.80369846)(427.75981434,565.4915739)
\curveto(428.06117181,565.17944934)(428.28530643,564.74702677)(428.43221819,564.19430619)
\curveto(428.86165258,564.81855531)(429.25341728,565.26723437)(429.60751231,565.54034336)
\curveto(429.96160733,565.81995495)(430.34018764,565.95976074)(430.74325325,565.95976074)
\curveto(431.43637542,565.95976074)(431.94679963,565.59561542)(432.27452588,564.86732478)
\curveto(432.60601909,564.14553673)(432.77176569,563.1343832)(432.77176569,561.83386419)
\closepath
}
}
{
\newrgbcolor{curcolor}{0 0 0}
\pscustom[linestyle=none,fillstyle=solid,fillcolor=curcolor]
{
\newpath
\moveto(440.19646104,560.20496414)
\curveto(440.19646104,558.42975569)(439.93277325,557.02844646)(439.40539769,556.00103645)
\curveto(438.87802212,554.97362643)(438.17171556,554.45992143)(437.28647801,554.45992143)
\curveto(436.39370652,554.45992143)(435.68363299,554.97362643)(435.15625742,556.00103645)
\curveto(434.63264883,557.02844646)(434.37084453,558.42975569)(434.37084453,560.20496414)
\curveto(434.37084453,561.98017258)(434.63264883,563.38148181)(435.15625742,564.40889183)
\curveto(435.68363299,565.44280444)(436.39370652,565.95976074)(437.28647801,565.95976074)
\curveto(438.17171556,565.95976074)(438.87802212,565.44280444)(439.40539769,564.40889183)
\curveto(439.93277325,563.38148181)(440.19646104,561.98017258)(440.19646104,560.20496414)
\closepath
\moveto(439.10027325,560.20496414)
\curveto(439.10027325,561.61602726)(438.9401771,562.66294506)(438.61998479,563.34571754)
\curveto(438.29979249,564.03499261)(437.85529022,564.37963015)(437.28647801,564.37963015)
\curveto(436.71013186,564.37963015)(436.26186263,564.03499261)(435.94167032,563.34571754)
\curveto(435.62524498,562.66294506)(435.46703231,561.61602726)(435.46703231,560.20496414)
\curveto(435.46703231,558.83941918)(435.62712846,557.80225527)(435.94732077,557.09347241)
\curveto(436.26751308,556.39119215)(436.71389882,556.04005202)(437.28647801,556.04005202)
\curveto(437.85152326,556.04005202)(438.29414203,556.38794085)(438.61433434,557.08371852)
\curveto(438.93829362,557.78599879)(439.10027325,558.82641399)(439.10027325,560.20496414)
\closepath
}
}
{
\newrgbcolor{curcolor}{0 0 0}
\pscustom[linestyle=none,fillstyle=solid,fillcolor=curcolor]
{
\newpath
\moveto(445.77910699,563.6578421)
\lineto(445.72260246,563.6578421)
\curveto(445.56438979,563.72286805)(445.40994409,563.76838621)(445.25926536,563.79439659)
\curveto(445.11235359,563.82690957)(444.93718957,563.84316606)(444.73377328,563.84316606)
\curveto(444.40604703,563.84316606)(444.08962169,563.71636545)(443.78449726,563.46276425)
\curveto(443.47937283,563.21566564)(443.1855493,562.89378718)(442.90302667,562.49712889)
\lineto(442.90302667,554.7622921)
\lineto(441.84074161,554.7622921)
\lineto(441.84074161,565.65739007)
\lineto(442.90302667,565.65739007)
\lineto(442.90302667,564.0479978)
\curveto(443.32492713,564.63323135)(443.6959735,565.04614614)(444.01616581,565.28674215)
\curveto(444.34012509,565.53384077)(444.66973482,565.65739007)(445.004995,565.65739007)
\curveto(445.18957644,565.65739007)(445.32330382,565.64763618)(445.40617712,565.62812839)
\curveto(445.48905042,565.6151232)(445.61336038,565.58586153)(445.77910699,565.54034336)
\closepath
}
}
{
\newrgbcolor{curcolor}{0 0 0}
\pscustom[linestyle=none,fillstyle=solid,fillcolor=curcolor]
{
\newpath
\moveto(452.23757333,565.65739007)
\lineto(448.55347831,550.74368837)
\lineto(447.41773736,550.74368837)
\lineto(448.59303148,555.28900229)
\lineto(446.07858012,565.65739007)
\lineto(447.23127243,565.65739007)
\lineto(449.16937763,557.58116704)
\lineto(451.12443419,565.65739007)
\closepath
}
}
{
\newrgbcolor{curcolor}{0 0 0}
\pscustom[linestyle=none,fillstyle=solid,fillcolor=curcolor]
{
\newpath
\moveto(464.27868801,555.81571249)
\curveto(464.07150475,555.65965021)(463.88315633,555.51334182)(463.71364276,555.37678733)
\curveto(463.54789615,555.24023283)(463.32941199,555.09717574)(463.05819027,554.94761605)
\curveto(462.8284052,554.82406675)(462.57790181,554.72002523)(462.30668009,554.63549149)
\curveto(462.03922534,554.54445516)(461.74351833,554.498937)(461.41955905,554.498937)
\curveto(460.80931018,554.498937)(460.25368236,554.64524538)(459.75267557,554.93786216)
\curveto(459.25543575,555.23698153)(458.82223439,555.70191708)(458.4530715,556.3326688)
\curveto(458.09144254,556.95041532)(457.80891992,557.73397802)(457.60550363,558.6833569)
\curveto(457.40208734,559.63923837)(457.30037919,560.74793082)(457.30037919,562.00943426)
\curveto(457.30037919,563.20591174)(457.39832037,564.27558863)(457.59420272,565.21846491)
\curveto(457.79008507,566.16134119)(458.0726077,566.95790908)(458.44177059,567.60816858)
\curveto(458.79963258,568.2389203)(459.23095046,568.72011233)(459.73572421,569.05174468)
\curveto(460.24426493,569.38337702)(460.8074267,569.5491932)(461.4252095,569.5491932)
\curveto(461.8772457,569.5491932)(462.32739842,569.45490557)(462.77566765,569.26633031)
\curveto(463.22770384,569.07775506)(463.72871063,568.74612271)(464.27868801,568.27143327)
\lineto(464.27868801,565.97926853)
\lineto(464.19393122,565.97926853)
\curveto(463.73059411,566.64903581)(463.27102398,567.13673044)(462.81522081,567.44235241)
\curveto(462.35941765,567.74797437)(461.87159525,567.90078536)(461.35175362,567.90078536)
\curveto(460.9260862,567.90078536)(460.54185543,567.78048735)(460.19906131,567.53989133)
\curveto(459.86003417,567.30579791)(459.55679322,566.93840129)(459.28933847,566.43770148)
\curveto(459.02941765,565.95000685)(458.82600136,565.33226032)(458.6790896,564.58446189)
\curveto(458.5359448,563.84316606)(458.4643724,562.98482351)(458.4643724,562.00943426)
\curveto(458.4643724,560.98852684)(458.54347874,560.11067651)(458.70169141,559.37588327)
\curveto(458.86367105,558.64109003)(459.0708543,558.04285129)(459.32324118,557.58116704)
\curveto(459.58692896,557.09997501)(459.89393688,556.74233228)(460.24426493,556.50823886)
\curveto(460.59835996,556.28064804)(460.97128982,556.16685262)(461.36305453,556.16685262)
\curveto(461.901731,556.16685262)(462.40650475,556.3261662)(462.87737579,556.64479336)
\curveto(463.34824683,556.96342051)(463.78898212,557.44136125)(464.19958167,558.07861556)
\lineto(464.27868801,558.07861556)
\closepath
}
}
{
\newrgbcolor{curcolor}{0 0 0}
\pscustom[linestyle=none,fillstyle=solid,fillcolor=curcolor]
{
\newpath
\moveto(471.15528712,560.20496414)
\curveto(471.15528712,558.42975569)(470.89159933,557.02844646)(470.36422377,556.00103645)
\curveto(469.8368482,554.97362643)(469.13054164,554.45992143)(468.24530409,554.45992143)
\curveto(467.3525326,554.45992143)(466.64245907,554.97362643)(466.1150835,556.00103645)
\curveto(465.59147491,557.02844646)(465.32967061,558.42975569)(465.32967061,560.20496414)
\curveto(465.32967061,561.98017258)(465.59147491,563.38148181)(466.1150835,564.40889183)
\curveto(466.64245907,565.44280444)(467.3525326,565.95976074)(468.24530409,565.95976074)
\curveto(469.13054164,565.95976074)(469.8368482,565.44280444)(470.36422377,564.40889183)
\curveto(470.89159933,563.38148181)(471.15528712,561.98017258)(471.15528712,560.20496414)
\closepath
\moveto(470.05909934,560.20496414)
\curveto(470.05909934,561.61602726)(469.89900318,562.66294506)(469.57881088,563.34571754)
\curveto(469.25861857,564.03499261)(468.81411631,564.37963015)(468.24530409,564.37963015)
\curveto(467.66895794,564.37963015)(467.22068871,564.03499261)(466.9004964,563.34571754)
\curveto(466.58407106,562.66294506)(466.42585839,561.61602726)(466.42585839,560.20496414)
\curveto(466.42585839,558.83941918)(466.58595454,557.80225527)(466.90614685,557.09347241)
\curveto(467.22633916,556.39119215)(467.67272491,556.04005202)(468.24530409,556.04005202)
\curveto(468.81034934,556.04005202)(469.25296812,556.38794085)(469.57316042,557.08371852)
\curveto(469.8971197,557.78599879)(470.05909934,558.82641399)(470.05909934,560.20496414)
\closepath
}
}
{
\newrgbcolor{curcolor}{0 0 0}
\pscustom[linestyle=none,fillstyle=solid,fillcolor=curcolor]
{
\newpath
\moveto(478.07709579,554.7622921)
\lineto(477.01481073,554.7622921)
\lineto(477.01481073,560.96576776)
\curveto(477.01481073,561.46646757)(476.99785937,561.93465442)(476.96395665,562.37032828)
\curveto(476.93005394,562.81250475)(476.86789896,563.15714228)(476.77749172,563.40424089)
\curveto(476.68331751,563.67734988)(476.54770665,563.87893033)(476.37065914,564.00898223)
\curveto(476.19361163,564.14553673)(475.96382656,564.21381397)(475.68130394,564.21381397)
\curveto(475.39124738,564.21381397)(475.08800643,564.09026467)(474.77158109,563.84316606)
\curveto(474.45515575,563.59606745)(474.1519148,563.28069159)(473.86185824,562.89703848)
\lineto(473.86185824,554.7622921)
\lineto(472.79957318,554.7622921)
\lineto(472.79957318,565.65739007)
\lineto(473.86185824,565.65739007)
\lineto(473.86185824,564.4479074)
\curveto(474.19335145,564.92259683)(474.53614557,565.29324475)(474.89024059,565.55985115)
\curveto(475.24433562,565.82645754)(475.60784806,565.95976074)(475.98077792,565.95976074)
\curveto(476.66259919,565.95976074)(477.18244082,565.60536931)(477.54030281,564.89658645)
\curveto(477.8981648,564.18780359)(478.07709579,563.16689617)(478.07709579,561.83386419)
\closepath
}
}
{
\newrgbcolor{curcolor}{0 0 0}
\pscustom[linestyle=none,fillstyle=solid,fillcolor=curcolor]
{
\newpath
\moveto(483.39981725,554.85983102)
\curveto(483.20016793,554.76879469)(482.98168376,554.69401485)(482.74436476,554.63549149)
\curveto(482.51081272,554.57696814)(482.30174598,554.54770646)(482.11716454,554.54770646)
\curveto(481.47301295,554.54770646)(480.98330707,554.84682583)(480.64804689,555.44506457)
\curveto(480.31278671,556.04330332)(480.14515662,557.00243608)(480.14515662,558.32246288)
\lineto(480.14515662,564.11627505)
\lineto(479.42754915,564.11627505)
\lineto(479.42754915,565.65739007)
\lineto(480.14515662,565.65739007)
\lineto(480.14515662,568.78838958)
\lineto(481.20744169,568.78838958)
\lineto(481.20744169,565.65739007)
\lineto(483.39981725,565.65739007)
\lineto(483.39981725,564.11627505)
\lineto(481.20744169,564.11627505)
\lineto(481.20744169,559.15154374)
\curveto(481.20744169,558.57931538)(481.21497562,558.13063632)(481.2300435,557.80550657)
\curveto(481.24511137,557.48687941)(481.29784893,557.18776004)(481.38825617,556.90814846)
\curveto(481.47112947,556.64804465)(481.58413852,556.4562181)(481.72728331,556.3326688)
\curveto(481.87419508,556.21562208)(482.09644621,556.15709873)(482.39403671,556.15709873)
\curveto(482.56731725,556.15709873)(482.74813173,556.1993656)(482.93648014,556.28389933)
\curveto(483.12482856,556.37493566)(483.26043942,556.44971551)(483.34331272,556.50823886)
\lineto(483.39981725,556.50823886)
\closepath
}
}
{
\newrgbcolor{curcolor}{0 0 0}
\pscustom[linestyle=none,fillstyle=solid,fillcolor=curcolor]
{
\newpath
\moveto(488.62083563,563.6578421)
\lineto(488.5643311,563.6578421)
\curveto(488.40611843,563.72286805)(488.25167273,563.76838621)(488.100994,563.79439659)
\curveto(487.95408224,563.82690957)(487.77891821,563.84316606)(487.57550192,563.84316606)
\curveto(487.24777568,563.84316606)(486.93135034,563.71636545)(486.6262259,563.46276425)
\curveto(486.32110147,563.21566564)(486.02727794,562.89378718)(485.74475532,562.49712889)
\lineto(485.74475532,554.7622921)
\lineto(484.68247025,554.7622921)
\lineto(484.68247025,565.65739007)
\lineto(485.74475532,565.65739007)
\lineto(485.74475532,564.0479978)
\curveto(486.16665577,564.63323135)(486.53770215,565.04614614)(486.85789445,565.28674215)
\curveto(487.18185373,565.53384077)(487.51146346,565.65739007)(487.84672364,565.65739007)
\curveto(488.03130509,565.65739007)(488.16503246,565.64763618)(488.24790576,565.62812839)
\curveto(488.33077907,565.6151232)(488.45508902,565.58586153)(488.62083563,565.54034336)
\closepath
}
}
{
\newrgbcolor{curcolor}{0 0 0}
\pscustom[linestyle=none,fillstyle=solid,fillcolor=curcolor]
{
\newpath
\moveto(495.00019563,560.20496414)
\curveto(495.00019563,558.42975569)(494.73650785,557.02844646)(494.20913229,556.00103645)
\curveto(493.68175672,554.97362643)(492.97545016,554.45992143)(492.09021261,554.45992143)
\curveto(491.19744111,554.45992143)(490.48736759,554.97362643)(489.95999202,556.00103645)
\curveto(489.43638343,557.02844646)(489.17457913,558.42975569)(489.17457913,560.20496414)
\curveto(489.17457913,561.98017258)(489.43638343,563.38148181)(489.95999202,564.40889183)
\curveto(490.48736759,565.44280444)(491.19744111,565.95976074)(492.09021261,565.95976074)
\curveto(492.97545016,565.95976074)(493.68175672,565.44280444)(494.20913229,564.40889183)
\curveto(494.73650785,563.38148181)(495.00019563,561.98017258)(495.00019563,560.20496414)
\closepath
\moveto(493.90400785,560.20496414)
\curveto(493.90400785,561.61602726)(493.7439117,562.66294506)(493.42371939,563.34571754)
\curveto(493.10352708,564.03499261)(492.65902482,564.37963015)(492.09021261,564.37963015)
\curveto(491.51386645,564.37963015)(491.06559722,564.03499261)(490.74540492,563.34571754)
\curveto(490.42897958,562.66294506)(490.27076691,561.61602726)(490.27076691,560.20496414)
\curveto(490.27076691,558.83941918)(490.43086306,557.80225527)(490.75105537,557.09347241)
\curveto(491.07124768,556.39119215)(491.51763342,556.04005202)(492.09021261,556.04005202)
\curveto(492.65525785,556.04005202)(493.09787663,556.38794085)(493.41806894,557.08371852)
\curveto(493.74202821,557.78599879)(493.90400785,558.82641399)(493.90400785,560.20496414)
\closepath
}
}
{
\newrgbcolor{curcolor}{0 0 0}
\pscustom[linestyle=none,fillstyle=solid,fillcolor=curcolor]
{
\newpath
\moveto(497.71806766,554.7622921)
\lineto(496.6557826,554.7622921)
\lineto(496.6557826,569.9393489)
\lineto(497.71806766,569.9393489)
\closepath
}
}
{
\newrgbcolor{curcolor}{0 0 0}
\pscustom[linestyle=none,fillstyle=solid,fillcolor=curcolor]
{
\newpath
\moveto(500.89362155,554.7622921)
\lineto(499.83133649,554.7622921)
\lineto(499.83133649,569.9393489)
\lineto(500.89362155,569.9393489)
\closepath
}
}
{
\newrgbcolor{curcolor}{0 0 0}
\pscustom[linestyle=none,fillstyle=solid,fillcolor=curcolor]
{
\newpath
\moveto(508.27875705,560.01964018)
\lineto(503.62843466,560.01964018)
\curveto(503.62843466,559.34987289)(503.68682267,558.76463934)(503.80359869,558.26393952)
\curveto(503.92037471,557.7697423)(504.08047086,557.36333011)(504.28388715,557.04470295)
\curveto(504.4797695,556.73257839)(504.71143805,556.49848497)(504.9788928,556.34242269)
\curveto(505.25011452,556.18636041)(505.54770502,556.10832927)(505.8716643,556.10832927)
\curveto(506.30109868,556.10832927)(506.73241656,556.25463766)(507.16561791,556.54725443)
\curveto(507.60258624,556.8463738)(507.91336113,557.13899058)(508.09794257,557.42510476)
\lineto(508.1544471,557.42510476)
\lineto(508.1544471,555.42555679)
\curveto(507.79658511,555.16545299)(507.43118918,554.94761605)(507.05825932,554.77204599)
\curveto(506.68532945,554.59647592)(506.29356475,554.50869089)(505.8829652,554.50869089)
\curveto(504.83574801,554.50869089)(504.01831588,554.99638552)(503.43066882,555.97177477)
\curveto(502.84302177,556.95366662)(502.54919824,558.34522196)(502.54919824,560.14644078)
\curveto(502.54919824,561.92815182)(502.82983738,563.34246624)(503.39111566,564.38938404)
\curveto(503.95616091,565.43630184)(504.69825366,565.95976074)(505.61739393,565.95976074)
\curveto(506.46872877,565.95976074)(507.12418126,565.53058947)(507.5837514,564.67224692)
\curveto(508.0470885,563.81390438)(508.27875705,562.59466781)(508.27875705,561.01453722)
\closepath
\moveto(507.24472425,561.42420071)
\curveto(507.24095728,562.38658477)(507.09969597,563.1311319)(506.82094031,563.6578421)
\curveto(506.54595163,564.1845523)(506.12593466,564.4479074)(505.56088941,564.4479074)
\curveto(504.99207719,564.4479074)(504.53815751,564.15854192)(504.19913036,563.57981096)
\curveto(503.86387018,563.00108)(503.67363828,562.28254325)(503.62843466,561.42420071)
\closepath
}
}
{
\newrgbcolor{curcolor}{0 0 0}
\pscustom[linestyle=none,fillstyle=solid,fillcolor=curcolor]
{
\newpath
\moveto(513.82750551,563.6578421)
\lineto(513.77100098,563.6578421)
\curveto(513.61278831,563.72286805)(513.45834261,563.76838621)(513.30766388,563.79439659)
\curveto(513.16075212,563.82690957)(512.98558809,563.84316606)(512.7821718,563.84316606)
\curveto(512.45444556,563.84316606)(512.13802022,563.71636545)(511.83289578,563.46276425)
\curveto(511.52777135,563.21566564)(511.23394782,562.89378718)(510.9514252,562.49712889)
\lineto(510.9514252,554.7622921)
\lineto(509.88914013,554.7622921)
\lineto(509.88914013,565.65739007)
\lineto(510.9514252,565.65739007)
\lineto(510.9514252,564.0479978)
\curveto(511.37332565,564.63323135)(511.74437203,565.04614614)(512.06456433,565.28674215)
\curveto(512.38852361,565.53384077)(512.71813334,565.65739007)(513.05339352,565.65739007)
\curveto(513.23797497,565.65739007)(513.37170234,565.64763618)(513.45457564,565.62812839)
\curveto(513.53744895,565.6151232)(513.6617589,565.58586153)(513.82750551,565.54034336)
\closepath
}
}
{
\newrgbcolor{curcolor}{0.80000001 0.80000001 0.80000001}
\pscustom[linestyle=none,fillstyle=solid,fillcolor=curcolor]
{
\newpath
\moveto(27.11109176,467.44732613)
\lineto(276.69050667,467.44732613)
\lineto(276.69050667,284.70845759)
\lineto(27.11109176,284.70845759)
\closepath
}
}
{
\newrgbcolor{curcolor}{0 0 0}
\pscustom[linewidth=0.85007621,linecolor=curcolor]
{
\newpath
\moveto(27.11109176,467.44732613)
\lineto(276.69050667,467.44732613)
\lineto(276.69050667,284.70845759)
\lineto(27.11109176,284.70845759)
\closepath
}
}
{
\newrgbcolor{curcolor}{0 0 0}
\pscustom[linestyle=none,fillstyle=solid,fillcolor=curcolor]
{
\newpath
\moveto(103.14612544,365.81668391)
\lineto(99.54006489,365.81668391)
\lineto(99.54006489,391.88328572)
\lineto(92.00012011,374.14661902)
\lineto(89.85105372,374.14661902)
\lineto(82.36574621,391.88328572)
\lineto(82.36574621,365.81668391)
\lineto(78.99644721,365.81668391)
\lineto(78.99644721,396.06857019)
\lineto(83.91380251,396.06857019)
\lineto(91.14413604,379.22584774)
\lineto(98.13770801,396.06857019)
\lineto(103.14612544,396.06857019)
\closepath
}
}
{
\newrgbcolor{curcolor}{0 0 0}
\pscustom[linestyle=none,fillstyle=solid,fillcolor=curcolor]
{
\newpath
\moveto(127.18652876,376.76750104)
\lineto(112.19770132,376.76750104)
\curveto(112.19770132,375.37240622)(112.3858964,374.15339132)(112.76228656,373.11045636)
\curveto(113.13867672,372.081066)(113.65469548,371.23452788)(114.31034285,370.57084199)
\curveto(114.94170699,369.92070072)(115.6884165,369.43309476)(116.55047138,369.10802412)
\curveto(117.42466788,368.78295348)(118.3838557,368.62041816)(119.42803485,368.62041816)
\curveto(120.8121793,368.62041816)(122.20239456,368.92517189)(123.59868064,369.53467933)
\curveto(125.00710833,370.15773139)(126.00879181,370.76723884)(126.60373109,371.36320168)
\lineto(126.78585536,371.36320168)
\lineto(126.78585536,367.19823412)
\curveto(125.63240165,366.65644972)(124.45466471,366.20270529)(123.25264452,365.83700082)
\curveto(122.05062434,365.47129636)(120.78789607,365.28844412)(119.4644597,365.28844412)
\curveto(116.08908989,365.28844412)(113.45435878,366.30428987)(111.56026637,368.33598136)
\curveto(109.66617396,370.38121746)(108.71912776,373.27976398)(108.71912776,377.03162093)
\curveto(108.71912776,380.74284405)(109.6236783,383.68879671)(111.43277939,385.86947891)
\curveto(113.25402209,388.05016111)(115.64592084,389.14050221)(118.60847563,389.14050221)
\curveto(121.3524813,389.14050221)(123.46512284,388.24655796)(124.94640024,386.45866945)
\curveto(126.43981925,384.67078093)(127.18652876,382.13116657)(127.18652876,378.83982636)
\closepath
\moveto(123.85365461,379.69313678)
\curveto(123.841513,381.69773905)(123.38620232,383.24859689)(122.48772259,384.3457103)
\curveto(121.60138447,385.4428237)(120.24759406,385.9913804)(118.42635136,385.9913804)
\curveto(116.59296704,385.9913804)(115.12990207,385.38864526)(114.03715645,384.18317498)
\curveto(112.95655245,382.97770469)(112.34340074,381.4810253)(112.19770132,379.69313678)
\closepath
}
}
{
\newrgbcolor{curcolor}{0 0 0}
\pscustom[linestyle=none,fillstyle=solid,fillcolor=curcolor]
{
\newpath
\moveto(162.06332657,365.81668391)
\lineto(158.63939029,365.81668391)
\lineto(158.63939029,378.73824178)
\curveto(158.63939029,379.7134537)(158.59689463,380.65480409)(158.5119033,381.56229295)
\curveto(158.4390536,382.46978182)(158.27514175,383.19441845)(158.02016777,383.73620285)
\curveto(157.74091056,384.31862108)(157.34023717,384.7588209)(156.81814759,385.05680232)
\curveto(156.29605802,385.35478374)(155.5432777,385.50377445)(154.55980664,385.50377445)
\curveto(153.60061882,385.50377445)(152.64143099,385.23288225)(151.68224317,384.69109785)
\curveto(150.72305535,384.16285806)(149.76386752,383.48562757)(148.8046797,382.65940636)
\curveto(148.84110456,382.34788033)(148.8714586,381.98217586)(148.89574184,381.56229295)
\curveto(148.92002507,381.15595466)(148.93216669,380.74961636)(148.93216669,380.34327806)
\lineto(148.93216669,365.81668391)
\lineto(145.50823041,365.81668391)
\lineto(145.50823041,378.73824178)
\curveto(145.50823041,379.74054292)(145.46573475,380.68866561)(145.38074342,381.58260987)
\curveto(145.30789371,382.49009874)(145.14398187,383.21473537)(144.88900789,383.75651976)
\curveto(144.60975068,384.33893799)(144.20907728,384.77236551)(143.68698771,385.05680232)
\curveto(143.16489813,385.35478374)(142.41211782,385.50377445)(141.42864676,385.50377445)
\curveto(140.49374217,385.50377445)(139.55276677,385.24642686)(138.60572057,384.73173168)
\curveto(137.67081598,384.2170365)(136.73591139,383.56012292)(135.80100681,382.76099093)
\lineto(135.80100681,365.81668391)
\lineto(132.37707053,365.81668391)
\lineto(132.37707053,388.51067785)
\lineto(135.80100681,388.51067785)
\lineto(135.80100681,385.9913804)
\curveto(136.86946919,386.98013693)(137.93186077,387.75217969)(138.98818154,388.3075087)
\curveto(140.05664392,388.86283771)(141.19188521,389.14050221)(142.39390539,389.14050221)
\curveto(143.77804984,389.14050221)(144.94971598,388.81543157)(145.9089038,388.1652903)
\curveto(146.88023325,387.51514902)(147.60265952,386.61443246)(148.07618262,385.46314062)
\curveto(149.46032707,386.76342317)(150.72305535,387.69800125)(151.86436744,388.26687487)
\curveto(153.00567953,388.8492931)(154.22591214,389.14050221)(155.52506527,389.14050221)
\curveto(157.75912299,389.14050221)(159.40431223,388.38200406)(160.460633,386.86500774)
\curveto(161.52909538,385.36155604)(162.06332657,383.2553692)(162.06332657,380.54644721)
\closepath
}
}
{
\newrgbcolor{curcolor}{0 0 0}
\pscustom[linestyle=none,fillstyle=solid,fillcolor=curcolor]
{
\newpath
\moveto(185.99445517,377.15352242)
\curveto(185.99445517,373.45584391)(185.14454191,370.53698047)(183.44471539,368.3969321)
\curveto(181.74488887,366.25688373)(179.46833549,365.18685955)(176.61505526,365.18685955)
\curveto(173.73749179,365.18685955)(171.44879679,366.25688373)(169.74897027,368.3969321)
\curveto(168.06128537,370.53698047)(167.21744291,373.45584391)(167.21744291,377.15352242)
\curveto(167.21744291,380.85120093)(168.06128537,383.77006437)(169.74897027,385.91011274)
\curveto(171.44879679,388.06370572)(173.73749179,389.14050221)(176.61505526,389.14050221)
\curveto(179.46833549,389.14050221)(181.74488887,388.06370572)(183.44471539,385.91011274)
\curveto(185.14454191,383.77006437)(185.99445517,380.85120093)(185.99445517,377.15352242)
\closepath
\moveto(182.46124433,377.15352242)
\curveto(182.46124433,380.09270278)(181.94522557,382.27338498)(180.91318803,383.69556902)
\curveto(179.8811505,385.13129767)(178.44843958,385.849162)(176.61505526,385.849162)
\curveto(174.7573877,385.849162)(173.31253516,385.13129767)(172.28049763,383.69556902)
\curveto(171.26060171,382.27338498)(170.75065376,380.09270278)(170.75065376,377.15352242)
\curveto(170.75065376,374.30915434)(171.26667252,372.14878905)(172.29871005,370.67242657)
\curveto(173.33074758,369.2096087)(174.76952932,368.47819976)(176.61505526,368.47819976)
\curveto(178.43629796,368.47819976)(179.86293808,369.20283639)(180.89497561,370.65210965)
\curveto(181.93915476,372.11492753)(182.46124433,374.28206512)(182.46124433,377.15352242)
\closepath
}
}
{
\newrgbcolor{curcolor}{0 0 0}
\pscustom[linestyle=none,fillstyle=solid,fillcolor=curcolor]
{
\newpath
\moveto(203.98833048,384.3457103)
\lineto(203.80620621,384.3457103)
\curveto(203.29625825,384.4811564)(202.79845192,384.57596866)(202.3127872,384.6301471)
\curveto(201.83926409,384.69787015)(201.27467886,384.73173168)(200.61903148,384.73173168)
\curveto(199.56271072,384.73173168)(198.5428148,384.46761179)(197.55934374,383.939372)
\curveto(196.57587268,383.42467682)(195.62882648,382.75421863)(194.71820513,381.92799742)
\lineto(194.71820513,365.81668391)
\lineto(191.29426885,365.81668391)
\lineto(191.29426885,388.51067785)
\lineto(194.71820513,388.51067785)
\lineto(194.71820513,385.15838689)
\curveto(196.07806635,386.37740179)(197.27401572,387.23748452)(198.30605325,387.73863508)
\curveto(199.3502324,388.25333026)(200.41262398,388.51067785)(201.49322798,388.51067785)
\curveto(202.08816726,388.51067785)(202.5191947,388.49036094)(202.7863103,388.44972711)
\curveto(203.05342589,388.42263789)(203.45409929,388.36168714)(203.98833048,388.26687487)
\closepath
}
}
{
\newrgbcolor{curcolor}{0 0 0}
\pscustom[linestyle=none,fillstyle=solid,fillcolor=curcolor]
{
\newpath
\moveto(224.80514284,388.51067785)
\lineto(212.93064042,357.44611497)
\lineto(209.26994259,357.44611497)
\lineto(213.05812741,366.91379731)
\lineto(204.95359739,388.51067785)
\lineto(208.6689325,388.51067785)
\lineto(214.91579497,371.68827231)
\lineto(221.21729472,388.51067785)
\closepath
}
}
{
\newrgbcolor{curcolor}{0.80000001 0.80000001 0.80000001}
\pscustom[linestyle=none,fillstyle=solid,fillcolor=curcolor]
{
\newpath
\moveto(335.62408312,467.44732613)
\lineto(585.20349803,467.44732613)
\lineto(585.20349803,284.70845759)
\lineto(335.62408312,284.70845759)
\closepath
}
}
{
\newrgbcolor{curcolor}{0 0 0}
\pscustom[linewidth=0.85007621,linecolor=curcolor]
{
\newpath
\moveto(335.62408312,467.44732613)
\lineto(585.20349803,467.44732613)
\lineto(585.20349803,284.70845759)
\lineto(335.62408312,284.70845759)
\closepath
}
}
{
\newrgbcolor{curcolor}{0 0 0}
\pscustom[linestyle=none,fillstyle=solid,fillcolor=curcolor]
{
\newpath
\moveto(411.65911889,365.13721103)
\lineto(408.05305834,365.13721103)
\lineto(408.05305834,391.20381284)
\lineto(400.51311355,373.46714613)
\lineto(398.36404716,373.46714613)
\lineto(390.87873966,391.20381284)
\lineto(390.87873966,365.13721103)
\lineto(387.50944066,365.13721103)
\lineto(387.50944066,395.38909731)
\lineto(392.42679595,395.38909731)
\lineto(399.65712948,378.54637486)
\lineto(406.65070146,395.38909731)
\lineto(411.65911889,395.38909731)
\closepath
}
}
{
\newrgbcolor{curcolor}{0 0 0}
\pscustom[linestyle=none,fillstyle=solid,fillcolor=curcolor]
{
\newpath
\moveto(435.69952221,376.08802816)
\lineto(420.71069477,376.08802816)
\curveto(420.71069477,374.69293333)(420.89888985,373.47391844)(421.27528,372.43098347)
\curveto(421.65167016,371.40159312)(422.16768893,370.555055)(422.8233363,369.89136911)
\curveto(423.45470044,369.24122784)(424.20140995,368.75362188)(425.06346482,368.42855124)
\curveto(425.93766132,368.1034806)(426.89684914,367.94094528)(427.94102829,367.94094528)
\curveto(429.32517275,367.94094528)(430.71538801,368.24569901)(432.11167408,368.85520645)
\curveto(433.52010177,369.47825851)(434.52178526,370.08776596)(435.11672454,370.68372879)
\lineto(435.29884881,370.68372879)
\lineto(435.29884881,366.51876124)
\curveto(434.1453951,365.97697684)(432.96765815,365.52323241)(431.76563797,365.15752794)
\curveto(430.56361778,364.79182347)(429.30088951,364.60897124)(427.97745315,364.60897124)
\curveto(424.60208334,364.60897124)(421.96735223,365.62481698)(420.07325982,367.65650847)
\curveto(418.17916741,369.70174457)(417.2321212,372.6002911)(417.2321212,376.35214805)
\curveto(417.2321212,380.06337117)(418.13667175,383.00932383)(419.94577283,385.19000603)
\curveto(421.76701553,387.37068823)(424.15891428,388.46102933)(427.12146908,388.46102933)
\curveto(429.86547475,388.46102933)(431.97811628,387.56708507)(433.45939368,385.77919656)
\curveto(434.9528127,383.99130805)(435.69952221,381.45169369)(435.69952221,378.16035348)
\closepath
\moveto(432.36664806,379.0136639)
\curveto(432.35450644,381.01826617)(431.89919577,382.56912401)(431.00071603,383.66623741)
\curveto(430.11437792,384.76335082)(428.76058751,385.31190752)(426.93934481,385.31190752)
\curveto(425.10596049,385.31190752)(423.64289552,384.70917238)(422.5501499,383.50370209)
\curveto(421.46954589,382.29823181)(420.85639418,380.80155241)(420.71069477,379.0136639)
\closepath
}
}
{
\newrgbcolor{curcolor}{0 0 0}
\pscustom[linestyle=none,fillstyle=solid,fillcolor=curcolor]
{
\newpath
\moveto(470.57632002,365.13721103)
\lineto(467.15238374,365.13721103)
\lineto(467.15238374,378.0587689)
\curveto(467.15238374,379.03398082)(467.10988808,379.97533121)(467.02489675,380.88282007)
\curveto(466.95204704,381.79030894)(466.7881352,382.51494557)(466.53316122,383.05672997)
\curveto(466.25390401,383.63914819)(465.85323061,384.07934802)(465.33114104,384.37732943)
\curveto(464.80905146,384.67531085)(464.05627114,384.82430156)(463.07280009,384.82430156)
\curveto(462.11361226,384.82430156)(461.15442444,384.55340936)(460.19523662,384.01162497)
\curveto(459.23604879,383.48338518)(458.27686097,382.80615468)(457.31767315,381.97993348)
\curveto(457.354098,381.66840745)(457.38445205,381.30270298)(457.40873528,380.88282007)
\curveto(457.43301852,380.47648177)(457.44516014,380.07014348)(457.44516014,379.66380518)
\lineto(457.44516014,365.13721103)
\lineto(454.02122386,365.13721103)
\lineto(454.02122386,378.0587689)
\curveto(454.02122386,379.06107004)(453.97872819,380.00919273)(453.89373687,380.90313699)
\curveto(453.82088716,381.81062585)(453.65697532,382.53526248)(453.40200134,383.07704688)
\curveto(453.12274412,383.65946511)(452.72207073,384.09289263)(452.19998115,384.37732943)
\curveto(451.67789158,384.67531085)(450.92511126,384.82430156)(449.9416402,384.82430156)
\curveto(449.00673562,384.82430156)(448.06576022,384.56695397)(447.11871401,384.0522588)
\curveto(446.18380943,383.53756362)(445.24890484,382.88065004)(444.31400025,382.08151805)
\lineto(444.31400025,365.13721103)
\lineto(440.89006397,365.13721103)
\lineto(440.89006397,387.83120497)
\lineto(444.31400025,387.83120497)
\lineto(444.31400025,385.31190752)
\curveto(445.38246264,386.30066405)(446.44485421,387.07270681)(447.50117498,387.62803582)
\curveto(448.56963737,388.18336483)(449.70487865,388.46102933)(450.90689884,388.46102933)
\curveto(452.29104329,388.46102933)(453.46270943,388.13595869)(454.42189725,387.48581741)
\curveto(455.39322669,386.83567614)(456.11565296,385.93495958)(456.58917607,384.78366773)
\curveto(457.97332052,386.08395029)(459.23604879,387.01852837)(460.37736089,387.58740199)
\curveto(461.51867298,388.16982022)(462.73890559,388.46102933)(464.03805872,388.46102933)
\curveto(466.27211643,388.46102933)(467.91730567,387.70253117)(468.97362644,386.18553486)
\curveto(470.04208883,384.68208316)(470.57632002,382.57589631)(470.57632002,379.86697433)
\closepath
}
}
{
\newrgbcolor{curcolor}{0 0 0}
\pscustom[linestyle=none,fillstyle=solid,fillcolor=curcolor]
{
\newpath
\moveto(494.50744862,376.47404954)
\curveto(494.50744862,372.77637103)(493.65753536,369.85750759)(491.95770884,367.71745922)
\curveto(490.25788231,365.57741085)(487.98132894,364.50738666)(485.1280487,364.50738666)
\curveto(482.25048523,364.50738666)(479.96179024,365.57741085)(478.26196372,367.71745922)
\curveto(476.57427881,369.85750759)(475.73043636,372.77637103)(475.73043636,376.47404954)
\curveto(475.73043636,380.17172805)(476.57427881,383.09059149)(478.26196372,385.23063986)
\curveto(479.96179024,387.38423284)(482.25048523,388.46102933)(485.1280487,388.46102933)
\curveto(487.98132894,388.46102933)(490.25788231,387.38423284)(491.95770884,385.23063986)
\curveto(493.65753536,383.09059149)(494.50744862,380.17172805)(494.50744862,376.47404954)
\closepath
\moveto(490.97423778,376.47404954)
\curveto(490.97423778,379.41322989)(490.45821901,381.59391209)(489.42618148,383.01609614)
\curveto(488.39414395,384.45182479)(486.96143302,385.16968912)(485.1280487,385.16968912)
\curveto(483.27038115,385.16968912)(481.8255286,384.45182479)(480.79349107,383.01609614)
\curveto(479.77359516,381.59391209)(479.2636472,379.41322989)(479.2636472,376.47404954)
\curveto(479.2636472,373.62968145)(479.77966597,371.46931617)(480.8117035,369.99295369)
\curveto(481.84374103,368.53013581)(483.28252276,367.79872688)(485.1280487,367.79872688)
\curveto(486.9492914,367.79872688)(488.37593152,368.52336351)(489.40796905,369.97263677)
\curveto(490.4521482,371.43545464)(490.97423778,373.60259223)(490.97423778,376.47404954)
\closepath
}
}
{
\newrgbcolor{curcolor}{0 0 0}
\pscustom[linestyle=none,fillstyle=solid,fillcolor=curcolor]
{
\newpath
\moveto(512.50132393,383.66623741)
\lineto(512.31919966,383.66623741)
\curveto(511.8092517,383.80168351)(511.31144536,383.89649578)(510.82578064,383.95067422)
\curveto(510.35225754,384.01839727)(509.7876723,384.0522588)(509.13202493,384.0522588)
\curveto(508.07570416,384.0522588)(507.05580825,383.7881389)(506.07233719,383.25989912)
\curveto(505.08886613,382.74520394)(504.14181992,382.07474575)(503.23119857,381.24852454)
\lineto(503.23119857,365.13721103)
\lineto(499.80726229,365.13721103)
\lineto(499.80726229,387.83120497)
\lineto(503.23119857,387.83120497)
\lineto(503.23119857,384.47891401)
\curveto(504.59105979,385.6979289)(505.78700917,386.55801163)(506.8190467,387.0591622)
\curveto(507.86322585,387.57385738)(508.92561742,387.83120497)(510.00622143,387.83120497)
\curveto(510.60116071,387.83120497)(511.03218815,387.81088805)(511.29930374,387.77025422)
\curveto(511.56641934,387.743165)(511.96709273,387.68221426)(512.50132393,387.58740199)
\closepath
}
}
{
\newrgbcolor{curcolor}{0 0 0}
\pscustom[linestyle=none,fillstyle=solid,fillcolor=curcolor]
{
\newpath
\moveto(533.31813629,387.83120497)
\lineto(521.44363387,356.76664209)
\lineto(517.78293604,356.76664209)
\lineto(521.57112086,366.23432443)
\lineto(513.46659083,387.83120497)
\lineto(517.18192594,387.83120497)
\lineto(523.42878841,371.00879943)
\lineto(529.73028816,387.83120497)
\closepath
}
}
{
\newrgbcolor{curcolor}{0 1 0}
\pscustom[linestyle=none,fillstyle=solid,fillcolor=curcolor,opacity=0.50196099]
{
\newpath
\moveto(51.13457252,723.69498084)
\lineto(252.6670115,723.69498084)
\lineto(252.6670115,525.80472589)
\lineto(51.13457252,525.80472589)
\closepath
}
}
{
\newrgbcolor{curcolor}{0 1 0}
\pscustom[linewidth=2.61146087,linecolor=curcolor,strokeopacity=0.50196099]
{
\newpath
\moveto(51.13457252,723.69498084)
\lineto(252.6670115,723.69498084)
\lineto(252.6670115,525.80472589)
\lineto(51.13457252,525.80472589)
\closepath
}
}
{
\newrgbcolor{curcolor}{1 0.40000001 0}
\pscustom[linestyle=none,fillstyle=solid,fillcolor=curcolor,opacity=0.50196099]
{
\newpath
\moveto(1.68144752,486.06658594)
\lineto(302.12015092,486.06658594)
\lineto(302.12015092,266.08919778)
\lineto(1.68144752,266.08919778)
\closepath
}
}
{
\newrgbcolor{curcolor}{1 0.40000001 0}
\pscustom[linewidth=3.36289503,linecolor=curcolor,strokeopacity=0.50196099]
{
\newpath
\moveto(1.68144752,486.06658594)
\lineto(302.12015092,486.06658594)
\lineto(302.12015092,266.08919778)
\lineto(1.68144752,266.08919778)
\closepath
}
}
{
\newrgbcolor{curcolor}{1 0.40000001 0}
\pscustom[linestyle=none,fillstyle=solid,fillcolor=curcolor,opacity=0.50196099]
{
\newpath
\moveto(310.19436679,486.06664361)
\lineto(610.63327204,486.06664361)
\lineto(610.63327204,266.08914011)
\lineto(310.19436679,266.08914011)
\closepath
}
}
{
\newrgbcolor{curcolor}{1 0.40000001 0}
\pscustom[linewidth=3.36175378,linecolor=curcolor,strokeopacity=0.50196099]
{
\newpath
\moveto(310.19436679,486.06664361)
\lineto(610.63327204,486.06664361)
\lineto(610.63327204,266.08914011)
\lineto(310.19436679,266.08914011)
\closepath
}
}
{
\newrgbcolor{curcolor}{0 1 0}
\pscustom[linestyle=none,fillstyle=solid,fillcolor=curcolor,opacity=0.50196099]
{
\newpath
\moveto(359.64757829,723.69498084)
\lineto(561.17998844,723.69498084)
\lineto(561.17998844,525.80471148)
\lineto(359.64757829,525.80471148)
\closepath
}
}
{
\newrgbcolor{curcolor}{0 1 0}
\pscustom[linewidth=2.61146087,linecolor=curcolor,strokeopacity=0.50196099]
{
\newpath
\moveto(359.64757829,723.69498084)
\lineto(561.17998844,723.69498084)
\lineto(561.17998844,525.80471148)
\lineto(359.64757829,525.80471148)
\closepath
}
}
{
\newrgbcolor{curcolor}{0.7019608 0.7019608 0.7019608}
\pscustom[linestyle=none,fillstyle=solid,fillcolor=curcolor,opacity=0.92623001]
{
\newpath
\moveto(151.14285254,186.80061477)
\lineto(461.17129397,186.80061477)
\lineto(461.17129397,136.057883)
\lineto(151.14285254,136.057883)
\closepath
}
}
{
\newrgbcolor{curcolor}{0 0 0}
\pscustom[linewidth=1.33600254,linecolor=curcolor]
{
\newpath
\moveto(151.14285254,186.80061477)
\lineto(461.17129397,186.80061477)
\lineto(461.17129397,136.057883)
\lineto(151.14285254,136.057883)
\closepath
}
}
{
\newrgbcolor{curcolor}{0 0 0}
\pscustom[linestyle=none,fillstyle=solid,fillcolor=curcolor]
{
\newpath
\moveto(175.80593355,146.83944659)
\lineto(160.06379617,146.83944659)
\lineto(160.06379617,149.80818713)
\lineto(166.11846439,149.80818713)
\lineto(166.11846439,169.30031257)
\lineto(160.06379617,169.30031257)
\lineto(160.06379617,171.95655411)
\curveto(160.88410606,171.95655411)(161.76300951,172.02165807)(162.70050653,172.15186599)
\curveto(163.63800354,172.2950947)(164.3476367,172.49691698)(164.829406,172.75733282)
\curveto(165.42836242,173.08285261)(165.89711093,173.49300756)(166.23565152,173.98779765)
\curveto(166.5872129,174.49560853)(166.78903517,175.17268971)(166.84111834,176.01904118)
\lineto(169.86845245,176.01904118)
\lineto(169.86845245,149.80818713)
\lineto(175.80593355,149.80818713)
\closepath
}
}
{
\newrgbcolor{curcolor}{0 0 0}
\pscustom[linestyle=none,fillstyle=solid,fillcolor=curcolor]
{
\newpath
\moveto(215.29799489,146.83944659)
\lineto(199.55585752,146.83944659)
\lineto(199.55585752,149.80818713)
\lineto(205.61052574,149.80818713)
\lineto(205.61052574,169.30031257)
\lineto(199.55585752,169.30031257)
\lineto(199.55585752,171.95655411)
\curveto(200.3761674,171.95655411)(201.25507085,172.02165807)(202.19256787,172.15186599)
\curveto(203.13006489,172.2950947)(203.83969804,172.49691698)(204.32146734,172.75733282)
\curveto(204.92042377,173.08285261)(205.38917228,173.49300756)(205.72771286,173.98779765)
\curveto(206.07927424,174.49560853)(206.28109652,175.17268971)(206.33317969,176.01904118)
\lineto(209.3605138,176.01904118)
\lineto(209.3605138,149.80818713)
\lineto(215.29799489,149.80818713)
\closepath
}
}
{
\newrgbcolor{curcolor}{0 0 0}
\pscustom[linestyle=none,fillstyle=solid,fillcolor=curcolor]
{
\newpath
\moveto(254.79005624,146.83944659)
\lineto(239.04791886,146.83944659)
\lineto(239.04791886,149.80818713)
\lineto(245.10258708,149.80818713)
\lineto(245.10258708,169.30031257)
\lineto(239.04791886,169.30031257)
\lineto(239.04791886,171.95655411)
\curveto(239.86822875,171.95655411)(240.7471322,172.02165807)(241.68462921,172.15186599)
\curveto(242.62212623,172.2950947)(243.33175939,172.49691698)(243.81352869,172.75733282)
\curveto(244.41248511,173.08285261)(244.88123362,173.49300756)(245.21977421,173.98779765)
\curveto(245.57133559,174.49560853)(245.77315786,175.17268971)(245.82524103,176.01904118)
\lineto(248.85257514,176.01904118)
\lineto(248.85257514,149.80818713)
\lineto(254.79005624,149.80818713)
\closepath
}
}
{
\newrgbcolor{curcolor}{0 0 0}
\pscustom[linestyle=none,fillstyle=solid,fillcolor=curcolor]
{
\newpath
\moveto(294.28212119,146.83944659)
\lineto(278.53998381,146.83944659)
\lineto(278.53998381,149.80818713)
\lineto(284.59465203,149.80818713)
\lineto(284.59465203,169.30031257)
\lineto(278.53998381,169.30031257)
\lineto(278.53998381,171.95655411)
\curveto(279.3602937,171.95655411)(280.23919715,172.02165807)(281.17669416,172.15186599)
\curveto(282.11419118,172.2950947)(282.82382434,172.49691698)(283.30559364,172.75733282)
\curveto(283.90455006,173.08285261)(284.37329857,173.49300756)(284.71183916,173.98779765)
\curveto(285.06340054,174.49560853)(285.26522281,175.17268971)(285.31730598,176.01904118)
\lineto(288.34464009,176.01904118)
\lineto(288.34464009,149.80818713)
\lineto(294.28212119,149.80818713)
\closepath
}
}
{
\newrgbcolor{curcolor}{0 0 0}
\pscustom[linestyle=none,fillstyle=solid,fillcolor=curcolor]
{
\newpath
\moveto(333.77417893,146.83944659)
\lineto(318.03204155,146.83944659)
\lineto(318.03204155,149.80818713)
\lineto(324.08670977,149.80818713)
\lineto(324.08670977,169.30031257)
\lineto(318.03204155,169.30031257)
\lineto(318.03204155,171.95655411)
\curveto(318.85235144,171.95655411)(319.73125489,172.02165807)(320.6687519,172.15186599)
\curveto(321.60624892,172.2950947)(322.31588208,172.49691698)(322.79765138,172.75733282)
\curveto(323.3966078,173.08285261)(323.86535631,173.49300756)(324.2038969,173.98779765)
\curveto(324.55545828,174.49560853)(324.75728055,175.17268971)(324.80936372,176.01904118)
\lineto(327.83669783,176.01904118)
\lineto(327.83669783,149.80818713)
\lineto(333.77417893,149.80818713)
\closepath
}
}
{
\newrgbcolor{curcolor}{0 0 0}
\pscustom[linestyle=none,fillstyle=solid,fillcolor=curcolor]
{
\newpath
\moveto(373.26623667,146.83944659)
\lineto(357.52409929,146.83944659)
\lineto(357.52409929,149.80818713)
\lineto(363.57876751,149.80818713)
\lineto(363.57876751,169.30031257)
\lineto(357.52409929,169.30031257)
\lineto(357.52409929,171.95655411)
\curveto(358.34440918,171.95655411)(359.22331263,172.02165807)(360.16080964,172.15186599)
\curveto(361.09830666,172.2950947)(361.80793982,172.49691698)(362.28970912,172.75733282)
\curveto(362.88866554,173.08285261)(363.35741405,173.49300756)(363.69595464,173.98779765)
\curveto(364.04751602,174.49560853)(364.24933829,175.17268971)(364.30142146,176.01904118)
\lineto(367.32875557,176.01904118)
\lineto(367.32875557,149.80818713)
\lineto(373.26623667,149.80818713)
\closepath
}
}
{
\newrgbcolor{curcolor}{0 0 0}
\pscustom[linestyle=none,fillstyle=solid,fillcolor=curcolor]
{
\newpath
\moveto(412.75830883,146.83944659)
\lineto(397.01617145,146.83944659)
\lineto(397.01617145,149.80818713)
\lineto(403.07083967,149.80818713)
\lineto(403.07083967,169.30031257)
\lineto(397.01617145,169.30031257)
\lineto(397.01617145,171.95655411)
\curveto(397.83648134,171.95655411)(398.71538479,172.02165807)(399.6528818,172.15186599)
\curveto(400.59037882,172.2950947)(401.30001197,172.49691698)(401.78178127,172.75733282)
\curveto(402.3807377,173.08285261)(402.84948621,173.49300756)(403.1880268,173.98779765)
\curveto(403.53958818,174.49560853)(403.74141045,175.17268971)(403.79349362,176.01904118)
\lineto(406.82082773,176.01904118)
\lineto(406.82082773,149.80818713)
\lineto(412.75830883,149.80818713)
\closepath
}
}
{
\newrgbcolor{curcolor}{0 0 0}
\pscustom[linestyle=none,fillstyle=solid,fillcolor=curcolor]
{
\newpath
\moveto(452.25035215,146.83944659)
\lineto(436.50821477,146.83944659)
\lineto(436.50821477,149.80818713)
\lineto(442.56288299,149.80818713)
\lineto(442.56288299,169.30031257)
\lineto(436.50821477,169.30031257)
\lineto(436.50821477,171.95655411)
\curveto(437.32852466,171.95655411)(438.20742811,172.02165807)(439.14492512,172.15186599)
\curveto(440.08242214,172.2950947)(440.7920553,172.49691698)(441.2738246,172.75733282)
\curveto(441.87278102,173.08285261)(442.34152953,173.49300756)(442.68007012,173.98779765)
\curveto(443.0316315,174.49560853)(443.23345377,175.17268971)(443.28553694,176.01904118)
\lineto(446.31287105,176.01904118)
\lineto(446.31287105,149.80818713)
\lineto(452.25035215,149.80818713)
\closepath
}
}
{
\newrgbcolor{curcolor}{0 0 0}
\pscustom[linestyle=none,fillstyle=solid,fillcolor=curcolor]
{
\newpath
\moveto(235.74713825,209.22499233)
\curveto(234.52318382,208.63905669)(233.35782294,208.18332898)(232.25105563,207.85780918)
\curveto(231.15730912,207.53228938)(229.99194824,207.36952948)(228.75497302,207.36952948)
\curveto(227.1794572,207.36952948)(225.7341493,207.59739334)(224.41904932,208.05312106)
\curveto(223.10394934,208.52186957)(221.97765084,209.22499233)(221.04015383,210.16248934)
\curveto(220.08963602,211.09998636)(219.35396128,212.28487842)(218.83312961,213.71716552)
\curveto(218.31229793,215.14945263)(218.05188209,216.82262439)(218.05188209,218.73668079)
\curveto(218.05188209,222.30437777)(219.02844148,225.10384802)(220.98156027,227.13509155)
\curveto(222.94769984,229.16633508)(225.53883742,230.18195685)(228.75497302,230.18195685)
\curveto(230.00496904,230.18195685)(231.22892347,230.00617616)(232.42683633,229.65461478)
\curveto(233.63776997,229.3030534)(234.74453728,228.87336727)(235.74713825,228.36555638)
\lineto(235.74713825,224.28353813)
\lineto(235.55182638,224.28353813)
\curveto(234.43203827,225.15593119)(233.2731878,225.82650197)(232.07527494,226.29525048)
\curveto(230.89038288,226.76399898)(229.73153241,226.99837324)(228.59872351,226.99837324)
\curveto(226.51539681,226.99837324)(224.86826664,226.29525048)(223.657333,224.88900495)
\curveto(222.45942014,223.49578022)(221.86046372,221.4450055)(221.86046372,218.73668079)
\curveto(221.86046372,216.10648083)(222.44639935,214.0817477)(223.61827062,212.66248138)
\curveto(224.80316268,211.25623586)(226.46331365,210.5531131)(228.59872351,210.5531131)
\curveto(229.34090865,210.5531131)(230.09611458,210.65076904)(230.8643413,210.84608091)
\curveto(231.63256802,211.04139279)(232.32266999,211.29529823)(232.93464721,211.60779724)
\curveto(233.46849968,211.88123387)(233.96980016,212.16769129)(234.43854867,212.4671695)
\curveto(234.90729718,212.77966851)(235.27838975,213.04659474)(235.55182638,213.2679482)
\lineto(235.74713825,213.2679482)
\closepath
}
}
{
\newrgbcolor{curcolor}{0 0 0}
\pscustom[linestyle=none,fillstyle=solid,fillcolor=curcolor]
{
\newpath
\moveto(259.65331241,219.02964861)
\curveto(259.65331241,217.25882091)(259.39940697,215.63773233)(258.89159608,214.16638284)
\curveto(258.3837852,212.70805415)(257.66764165,211.47107893)(256.74316542,210.45545716)
\curveto(255.88379316,209.49191856)(254.86817139,208.74322303)(253.69630013,208.20937056)
\curveto(252.53744965,207.68853889)(251.30698482,207.42812305)(250.00490563,207.42812305)
\curveto(248.87209674,207.42812305)(247.84345418,207.55182057)(246.91897795,207.79921562)
\curveto(246.00752252,208.04661066)(245.0765359,208.43072402)(244.1260181,208.9515557)
\lineto(244.1260181,199.8109598)
\lineto(240.45415479,199.8109598)
\lineto(240.45415479,229.67414597)
\lineto(244.1260181,229.67414597)
\lineto(244.1260181,227.38899699)
\curveto(245.10257749,228.20930688)(246.19632401,228.89289846)(247.40725765,229.43977171)
\curveto(248.63121209,229.99966576)(249.93329127,230.27961279)(251.31349521,230.27961279)
\curveto(253.94369517,230.27961279)(255.9879595,229.28352221)(257.44628819,227.29134105)
\curveto(258.91763767,225.31218069)(259.65331241,222.55828321)(259.65331241,219.02964861)
\closepath
\moveto(255.86426197,218.93199267)
\curveto(255.86426197,221.56219263)(255.41504465,223.5283322)(254.51661001,224.83041139)
\curveto(253.61817537,226.13249058)(252.23797144,226.78353017)(250.3759982,226.78353017)
\curveto(249.32131406,226.78353017)(248.26011952,226.55566631)(247.19241458,226.0999386)
\curveto(246.12470965,225.64421088)(245.10257749,225.04525446)(244.1260181,224.30306932)
\lineto(244.1260181,211.93982743)
\curveto(245.16768145,211.47107893)(246.05960569,211.15206952)(246.80179083,210.98279923)
\curveto(247.55699676,210.81352894)(248.40985862,210.72889379)(249.36037643,210.72889379)
\curveto(251.40464076,210.72889379)(252.99968776,211.41899576)(254.14551745,212.7991997)
\curveto(255.29134713,214.17940364)(255.86426197,216.22366796)(255.86426197,218.93199267)
\closepath
}
}
{
\newrgbcolor{curcolor}{0 0 0}
\pscustom[linestyle=none,fillstyle=solid,fillcolor=curcolor]
{
\newpath
\moveto(283.46183161,207.85780918)
\lineto(279.7899683,207.85780918)
\lineto(279.7899683,210.27967647)
\curveto(278.55299307,209.30311708)(277.36810101,208.55442154)(276.23529212,208.03358987)
\curveto(275.10248322,207.51275819)(273.8524872,207.25234236)(272.48530406,207.25234236)
\curveto(270.19364469,207.25234236)(268.4097962,207.94895472)(267.13375859,209.34217945)
\curveto(265.85772099,210.74842498)(265.21970219,212.80571009)(265.21970219,215.5140348)
\lineto(265.21970219,229.67414597)
\lineto(268.8915655,229.67414597)
\lineto(268.8915655,217.25231052)
\curveto(268.8915655,216.14554321)(268.94364867,215.1950254)(269.047815,214.4007571)
\curveto(269.15198134,213.61950958)(269.3733348,212.9489388)(269.71187539,212.38904475)
\curveto(270.06343677,211.81612991)(270.51916448,211.39946457)(271.07905853,211.13904873)
\curveto(271.63895258,210.87863289)(272.45275208,210.74842498)(273.52045701,210.74842498)
\curveto(274.47097482,210.74842498)(275.50612777,210.99582002)(276.62591587,211.49061011)
\curveto(277.75872477,211.9854002)(278.81340891,212.61690861)(279.7899683,213.38513533)
\lineto(279.7899683,229.67414597)
\lineto(283.46183161,229.67414597)
\closepath
}
}
{
\newrgbcolor{curcolor}{0 0 0}
\pscustom[linestyle=none,fillstyle=solid,fillcolor=curcolor]
{
\newpath
\moveto(322.52420132,207.85780918)
\lineto(318.85233801,207.85780918)
\lineto(318.85233801,220.27964463)
\curveto(318.85233801,221.21714164)(318.80676524,222.12208668)(318.7156197,222.99447974)
\curveto(318.63749495,223.86687279)(318.46171426,224.56348516)(318.18827763,225.08431683)
\curveto(317.88879941,225.64421088)(317.45911328,226.06738662)(316.89921923,226.35384404)
\curveto(316.33932518,226.64030146)(315.53203608,226.78353017)(314.47735194,226.78353017)
\curveto(313.44870938,226.78353017)(312.42006683,226.52311433)(311.39142427,226.00228266)
\curveto(310.36278171,225.49447178)(309.33413915,224.84343218)(308.30549659,224.04916388)
\curveto(308.34455897,223.74968566)(308.37711095,223.39812428)(308.40315253,222.99447974)
\curveto(308.42919412,222.60385598)(308.44221491,222.21323222)(308.44221491,221.82260847)
\lineto(308.44221491,207.85780918)
\lineto(304.7703516,207.85780918)
\lineto(304.7703516,220.27964463)
\curveto(304.7703516,221.24318323)(304.72477883,222.15463866)(304.63363328,223.01401092)
\curveto(304.55550853,223.88640398)(304.37972784,224.58301634)(304.10629121,225.10384802)
\curveto(303.806813,225.66374207)(303.37712687,226.08040741)(302.81723282,226.35384404)
\curveto(302.25733877,226.64030146)(301.45004967,226.78353017)(300.39536553,226.78353017)
\curveto(299.39276455,226.78353017)(298.38365318,226.53613513)(297.36803142,226.04134503)
\curveto(296.36543044,225.54655494)(295.36282947,224.91504654)(294.36022849,224.14681982)
\lineto(294.36022849,207.85780918)
\lineto(290.68836519,207.85780918)
\lineto(290.68836519,229.67414597)
\lineto(294.36022849,229.67414597)
\lineto(294.36022849,227.25227868)
\curveto(295.50605818,228.20279649)(296.64537747,228.94498162)(297.77818636,229.47883409)
\curveto(298.92401605,230.01268656)(300.14146009,230.27961279)(301.43051848,230.27961279)
\curveto(302.91488876,230.27961279)(304.17139517,229.96711378)(305.20003773,229.34211577)
\curveto(306.24170108,228.71711776)(307.0164382,227.85123511)(307.52424908,226.7444678)
\curveto(309.00861935,227.99446382)(310.36278171,228.89289846)(311.58673615,229.43977171)
\curveto(312.81069058,229.99966576)(314.11928017,230.27961279)(315.5125049,230.27961279)
\curveto(317.9083306,230.27961279)(319.6726479,229.55044844)(320.80545679,228.09211975)
\curveto(321.95128648,226.64681186)(322.52420132,224.62207872)(322.52420132,222.01792034)
\closepath
}
}
{
\newrgbcolor{curcolor}{0 0 0}
\pscustom[linestyle=none,fillstyle=solid,fillcolor=curcolor]
{
\newpath
\moveto(346.54757087,207.85780918)
\lineto(342.89523874,207.85780918)
\lineto(342.89523874,210.18202053)
\curveto(342.56971895,209.96066707)(342.12701202,209.64816806)(341.56711797,209.24452351)
\curveto(341.02024471,208.85389976)(340.48639225,208.54140075)(339.96556057,208.3070265)
\curveto(339.35358335,208.00754829)(338.65046059,207.76015324)(337.85619229,207.56484136)
\curveto(337.06192398,207.35650869)(336.13093736,207.25234236)(335.06323243,207.25234236)
\curveto(333.09709286,207.25234236)(331.4304315,207.90338195)(330.06324835,209.20546114)
\curveto(328.6960652,210.50754033)(328.01247363,212.16769129)(328.01247363,214.18591403)
\curveto(328.01247363,215.8395546)(328.36403501,217.17418577)(329.06715777,218.18980753)
\curveto(329.78330133,219.21845009)(330.79892309,220.02573919)(332.11402307,220.61167482)
\curveto(333.44214384,221.19761046)(335.03719085,221.59474461)(336.89916409,221.80307728)
\curveto(338.76113732,222.01140995)(340.75982888,222.16765945)(342.89523874,222.27182579)
\lineto(342.89523874,222.83823023)
\curveto(342.89523874,223.67156091)(342.74549964,224.36166288)(342.44602142,224.90853614)
\curveto(342.159564,225.4554094)(341.74289866,225.88509553)(341.1960254,226.19759454)
\curveto(340.67519373,226.49707275)(340.05019572,226.69889502)(339.32103137,226.80306136)
\curveto(338.59186703,226.90722769)(337.8301507,226.95931086)(337.0358824,226.95931086)
\curveto(336.0723438,226.95931086)(334.99812847,226.82910294)(333.81323641,226.56868711)
\curveto(332.62834435,226.32129206)(331.40438991,225.95670989)(330.1413731,225.47494059)
\lineto(329.94606122,225.47494059)
\lineto(329.94606122,229.20539746)
\curveto(330.66220478,229.40070934)(331.69735773,229.6155524)(333.05152009,229.84992666)
\curveto(334.40568244,230.08430091)(335.74031361,230.20148804)(337.05541359,230.20148804)
\curveto(338.59186703,230.20148804)(339.9264982,230.07128012)(341.05930709,229.81086428)
\curveto(342.20513678,229.56346924)(343.19471696,229.1337831)(344.02804764,228.52180589)
\curveto(344.84835753,227.92284946)(345.47335554,227.14811234)(345.90304167,226.19759454)
\curveto(346.3327278,225.24707673)(346.54757087,224.06869507)(346.54757087,222.66244954)
\closepath
\moveto(342.89523874,213.22888583)
\lineto(342.89523874,219.30308524)
\curveto(341.77545064,219.23798128)(340.45384027,219.14032534)(338.93040762,219.01011742)
\curveto(337.41999576,218.8799095)(336.22208291,218.69110802)(335.33666906,218.44371297)
\curveto(334.28198492,218.14423476)(333.42912305,217.67548625)(332.77808346,217.03746745)
\curveto(332.12704386,216.41246944)(331.80152407,215.54658678)(331.80152407,214.43981947)
\curveto(331.80152407,213.18982345)(332.17912703,212.24581604)(332.93433296,211.60779724)
\curveto(333.68953889,210.98279923)(334.84187897,210.67030022)(336.3913532,210.67030022)
\curveto(337.6804116,210.67030022)(338.85879326,210.91769527)(339.9264982,211.41248536)
\curveto(340.99420313,211.92029624)(341.98378331,212.52576307)(342.89523874,213.22888583)
\closepath
}
}
{
\newrgbcolor{curcolor}{0 0 0}
\pscustom[linestyle=none,fillstyle=solid,fillcolor=curcolor]
{
\newpath
\moveto(369.20373871,214.14685166)
\curveto(369.20373871,212.1546705)(368.37691843,210.52056112)(366.72327786,209.24452351)
\curveto(365.08265808,207.96848591)(362.83657148,207.33046711)(359.98501806,207.33046711)
\curveto(358.37043987,207.33046711)(356.8860696,207.51926859)(355.53190724,207.89687156)
\curveto(354.19076568,208.28749531)(353.06446718,208.71067105)(352.15301175,209.16639876)
\lineto(352.15301175,213.28747939)
\lineto(352.34832363,213.28747939)
\curveto(353.50717411,212.41508634)(354.7962325,211.71847397)(356.21549882,211.1976423)
\curveto(357.63476513,210.68983141)(358.99543788,210.43592597)(360.29751707,210.43592597)
\curveto(361.91209526,210.43592597)(363.17511207,210.69634181)(364.0865675,211.21717348)
\curveto(364.99802294,211.73800516)(365.45375065,212.55831505)(365.45375065,213.67810315)
\curveto(365.45375065,214.53747541)(365.20635561,215.18851501)(364.71156551,215.63122193)
\curveto(364.21677542,216.07392885)(363.26625762,216.45153182)(361.86001209,216.76403082)
\curveto(361.33918042,216.88121795)(360.65558885,217.01793626)(359.80923737,217.17418577)
\curveto(358.97590669,217.33043527)(358.21419037,217.49970556)(357.5240884,217.68199665)
\curveto(355.61003199,218.18980753)(354.24935924,218.93199267)(353.44207015,219.90855206)
\curveto(352.64780184,220.89813224)(352.25066769,222.10906589)(352.25066769,223.54135299)
\curveto(352.25066769,224.43978763)(352.43295878,225.28613911)(352.79754095,226.08040741)
\curveto(353.17514391,226.87467571)(353.74154836,227.58430887)(354.49675429,228.20930688)
\curveto(355.22591863,228.8212841)(356.15039486,229.3030534)(357.27018296,229.65461478)
\curveto(358.40299185,230.01919695)(359.66600866,230.20148804)(361.05923339,230.20148804)
\curveto(362.36131258,230.20148804)(363.67641256,230.03872814)(365.00453333,229.71320834)
\curveto(366.3456749,229.40070934)(367.4589526,229.01659598)(368.34436645,228.56086826)
\lineto(368.34436645,224.63509951)
\lineto(368.14905457,224.63509951)
\curveto(367.21155756,225.32520148)(366.07223827,225.90462672)(364.7310967,226.37337523)
\curveto(363.38995514,226.85514453)(362.07485516,227.09602918)(360.78579676,227.09602918)
\curveto(359.4446552,227.09602918)(358.31184631,226.83561334)(357.38737008,226.31478166)
\curveto(356.46289386,225.80697078)(356.00065575,225.04525446)(356.00065575,224.02963269)
\curveto(356.00065575,223.13119805)(356.28060278,222.45411687)(356.84049683,221.99838916)
\curveto(357.38737008,221.54266144)(358.27278393,221.17156887)(359.49673837,220.88511145)
\curveto(360.17381955,220.72886195)(360.92902547,220.57261245)(361.76235616,220.41636294)
\curveto(362.60870763,220.26011344)(363.31183039,220.11688473)(363.87172444,219.98667681)
\curveto(365.57744817,219.59605306)(366.89254815,218.92548227)(367.81702438,217.97496447)
\curveto(368.7415006,217.01142587)(369.20373871,215.73538826)(369.20373871,214.14685166)
\closepath
}
}
{
\newrgbcolor{curcolor}{0 0 0}
\pscustom[linestyle=none,fillstyle=solid,fillcolor=curcolor]
{
\newpath
\moveto(394.26225655,207.85780918)
\lineto(389.41852198,207.85780918)
\lineto(380.66854983,217.40856002)
\lineto(378.28574492,215.14294223)
\lineto(378.28574492,207.85780918)
\lineto(374.61388161,207.85780918)
\lineto(374.61388161,238.24833742)
\lineto(378.28574492,238.24833742)
\lineto(378.28574492,218.75621198)
\lineto(388.8911799,229.67414597)
\lineto(393.52007142,229.67414597)
\lineto(383.38338494,219.59605306)
\closepath
}
}
{
\newrgbcolor{curcolor}{0.7019608 0.7019608 0.7019608}
\pscustom[linestyle=none,fillstyle=solid,fillcolor=curcolor,opacity=0.92623001]
{
\newpath
\moveto(267.93745972,47.47327808)
\lineto(344.37669754,47.47327808)
\lineto(344.37669754,0.31975232)
\lineto(267.93745972,0.31975232)
\closepath
}
}
{
\newrgbcolor{curcolor}{0 0 0}
\pscustom[linewidth=0.6394923,linecolor=curcolor]
{
\newpath
\moveto(267.93745972,47.47327808)
\lineto(344.37669754,47.47327808)
\lineto(344.37669754,0.31975232)
\lineto(267.93745972,0.31975232)
\closepath
}
}
{
\newrgbcolor{curcolor}{0 0 0}
\pscustom[linestyle=none,fillstyle=solid,fillcolor=curcolor]
{
\newpath
\moveto(294.28209773,9.30677418)
\lineto(278.53996035,9.30677418)
\lineto(278.53996035,12.27551472)
\lineto(284.59462857,12.27551472)
\lineto(284.59462857,31.76764016)
\lineto(278.53996035,31.76764016)
\lineto(278.53996035,34.42388171)
\curveto(279.36027024,34.42388171)(280.23917369,34.48898567)(281.1766707,34.61919358)
\curveto(282.11416772,34.76242229)(282.82380088,34.96424457)(283.30557018,35.22466041)
\curveto(283.9045266,35.5501802)(284.37327511,35.96033515)(284.7118157,36.45512524)
\curveto(285.06337708,36.96293612)(285.26519935,37.6400173)(285.31728252,38.48636877)
\lineto(288.34461663,38.48636877)
\lineto(288.34461663,12.27551472)
\lineto(294.28209773,12.27551472)
\closepath
}
}
{
\newrgbcolor{curcolor}{0 0 0}
\pscustom[linestyle=none,fillstyle=solid,fillcolor=curcolor]
{
\newpath
\moveto(333.77415907,9.30677418)
\lineto(318.03202169,9.30677418)
\lineto(318.03202169,12.27551472)
\lineto(324.08668992,12.27551472)
\lineto(324.08668992,31.76764016)
\lineto(318.03202169,31.76764016)
\lineto(318.03202169,34.42388171)
\curveto(318.85233158,34.42388171)(319.73123503,34.48898567)(320.66873205,34.61919358)
\curveto(321.60622906,34.76242229)(322.31586222,34.96424457)(322.79763152,35.22466041)
\curveto(323.39658795,35.5501802)(323.86533645,35.96033515)(324.20387704,36.45512524)
\curveto(324.55543842,36.96293612)(324.7572607,37.6400173)(324.80934387,38.48636877)
\lineto(327.83667798,38.48636877)
\lineto(327.83667798,12.27551472)
\lineto(333.77415907,12.27551472)
\closepath
}
}
{
\newrgbcolor{curcolor}{0 0 0}
\pscustom[linestyle=none,fillstyle=solid,fillcolor=curcolor]
{
\newpath
\moveto(223.44250171,70.32507127)
\lineto(219.7706384,70.32507127)
\lineto(219.7706384,82.74690672)
\curveto(219.7706384,83.74950769)(219.71204484,84.68700471)(219.59485771,85.55939776)
\curveto(219.47767058,86.44481161)(219.26282752,87.13491358)(218.95032851,87.62970367)
\curveto(218.62480871,88.17657693)(218.15606021,88.58022148)(217.54408299,88.84063731)
\curveto(216.93210577,89.11407394)(216.13783747,89.25079226)(215.16127808,89.25079226)
\curveto(214.1586771,89.25079226)(213.11050335,89.00339721)(212.01675684,88.50860712)
\curveto(210.92301032,88.01381703)(209.87483657,87.38230862)(208.8722356,86.6140819)
\lineto(208.8722356,70.32507127)
\lineto(205.20037229,70.32507127)
\lineto(205.20037229,92.14140806)
\lineto(208.8722356,92.14140806)
\lineto(208.8722356,89.71954077)
\curveto(210.01806528,90.67005857)(211.20295735,91.41224371)(212.42691178,91.94609618)
\curveto(213.65086622,92.47994864)(214.90737263,92.74687488)(216.19643103,92.74687488)
\curveto(218.55319436,92.74687488)(220.35006364,92.03724172)(221.58703887,90.61797541)
\curveto(222.82401409,89.19870909)(223.44250171,87.15444477)(223.44250171,84.48518243)
\closepath
}
}
{
\newrgbcolor{curcolor}{0 0 0}
\pscustom[linestyle=none,fillstyle=solid,fillcolor=curcolor]
{
\newpath
\moveto(249.10648199,81.22347407)
\curveto(249.10648199,77.66879789)(248.19502656,74.86281724)(246.3721157,72.80553212)
\curveto(244.54920483,70.748247)(242.10780636,69.71960445)(239.04792027,69.71960445)
\curveto(235.96199259,69.71960445)(233.50757332,70.748247)(231.68466246,72.80553212)
\curveto(229.87477239,74.86281724)(228.96982735,77.66879789)(228.96982735,81.22347407)
\curveto(228.96982735,84.77815025)(229.87477239,87.5841309)(231.68466246,89.64141602)
\curveto(233.50757332,91.71172192)(235.96199259,92.74687488)(239.04792027,92.74687488)
\curveto(242.10780636,92.74687488)(244.54920483,91.71172192)(246.3721157,89.64141602)
\curveto(248.19502656,87.5841309)(249.10648199,84.77815025)(249.10648199,81.22347407)
\closepath
\moveto(245.31743155,81.22347407)
\curveto(245.31743155,84.0489859)(244.7640479,86.1453334)(243.65728059,87.51251654)
\curveto(242.55051328,88.89272048)(241.01405984,89.58282245)(239.04792027,89.58282245)
\curveto(237.05573911,89.58282245)(235.50626487,88.89272048)(234.39949757,87.51251654)
\curveto(233.30575105,86.1453334)(232.75887779,84.0489859)(232.75887779,81.22347407)
\curveto(232.75887779,78.48910777)(233.31226144,76.41229147)(234.41902875,74.99302515)
\curveto(235.52579606,73.58677963)(237.0687599,72.88365687)(239.04792027,72.88365687)
\curveto(241.00103905,72.88365687)(242.53098209,73.58026924)(243.6377494,74.97349397)
\curveto(244.7575375,76.37973949)(245.31743155,78.46306619)(245.31743155,81.22347407)
\closepath
}
}
{
\newrgbcolor{curcolor}{0 0 0}
\pscustom[linestyle=none,fillstyle=solid,fillcolor=curcolor]
{
\newpath
\moveto(272.4853112,70.32507127)
\lineto(268.81344789,70.32507127)
\lineto(268.81344789,72.61022024)
\curveto(267.75876375,71.69876481)(266.65850684,70.98913165)(265.51267715,70.48132077)
\curveto(264.36684747,69.97350989)(263.12336184,69.71960445)(261.78222028,69.71960445)
\curveto(259.1780619,69.71960445)(257.107756,70.72220542)(255.57130255,72.72740737)
\curveto(254.04786991,74.73260932)(253.28615358,77.51254838)(253.28615358,81.06722456)
\curveto(253.28615358,82.91617701)(253.54656942,84.56330718)(254.06740109,86.00861508)
\curveto(254.60125356,87.45392298)(255.31739711,88.68438781)(256.21583175,89.70000958)
\curveto(257.1012456,90.68958976)(258.12988816,91.44479569)(259.30175943,91.96562737)
\curveto(260.48665149,92.48645904)(261.71060592,92.74687488)(262.97362274,92.74687488)
\curveto(264.11945242,92.74687488)(265.13507419,92.62317735)(266.02048804,92.37578231)
\curveto(266.90590188,92.14140806)(267.8368885,91.77031549)(268.81344789,91.2625046)
\lineto(268.81344789,100.71559951)
\lineto(272.4853112,100.71559951)
\closepath
\moveto(268.81344789,75.69614792)
\lineto(268.81344789,88.2156393)
\curveto(267.82386771,88.65834623)(266.93845386,88.96433484)(266.15720635,89.13360513)
\curveto(265.37595884,89.30287543)(264.52309697,89.38751057)(263.59862075,89.38751057)
\curveto(261.54133563,89.38751057)(259.93977823,88.67136702)(258.79394854,87.23907991)
\curveto(257.64811886,85.80679281)(257.07520402,83.77554928)(257.07520402,81.14534932)
\curveto(257.07520402,78.55421173)(257.51791094,76.58156176)(258.40332479,75.22739941)
\curveto(259.28873864,73.88625785)(260.70800495,73.21568706)(262.66112373,73.21568706)
\curveto(263.70278708,73.21568706)(264.75747122,73.44355092)(265.82517616,73.89927864)
\curveto(266.89288109,74.36802714)(267.88897167,74.96698357)(268.81344789,75.69614792)
\closepath
}
}
{
\newrgbcolor{curcolor}{0 0 0}
\pscustom[linestyle=none,fillstyle=solid,fillcolor=curcolor]
{
\newpath
\moveto(297.97351616,80.8523815)
\lineto(281.89934859,80.8523815)
\curveto(281.89934859,79.51123994)(282.10117086,78.33936867)(282.50481541,77.33676769)
\curveto(282.90845996,76.34718751)(283.46184361,75.53338802)(284.16496637,74.89536922)
\curveto(284.84204755,74.27037121)(285.64282625,73.8016227)(286.56730247,73.48912369)
\curveto(287.50479949,73.17662469)(288.53344205,73.02037519)(289.65323015,73.02037519)
\curveto(291.13760042,73.02037519)(292.62848109,73.313343)(294.12587216,73.89927864)
\curveto(295.63628401,74.49823506)(296.71049934,75.0841707)(297.34851815,75.65708554)
\lineto(297.54383002,75.65708554)
\lineto(297.54383002,71.65319204)
\curveto(296.3068548,71.13236036)(295.04383798,70.69616384)(293.75477959,70.34460246)
\curveto(292.46572119,69.99304107)(291.11155884,69.81726038)(289.69229252,69.81726038)
\curveto(286.07251238,69.81726038)(283.24700054,70.79381977)(281.21575701,72.74693856)
\curveto(279.18451348,74.71307813)(278.16889171,77.49952759)(278.16889171,81.10628694)
\curveto(278.16889171,84.67398391)(279.13894071,87.50600615)(281.0790387,89.60235364)
\curveto(283.03215748,91.69870113)(285.59725348,92.74687488)(288.7743267,92.74687488)
\curveto(291.71702566,92.74687488)(293.98264345,91.88750261)(295.57118006,90.16875809)
\curveto(297.17273746,88.45001356)(297.97351616,86.00861508)(297.97351616,82.84456266)
\closepath
\moveto(294.39930879,83.66487254)
\curveto(294.38628799,85.59194974)(293.8980083,87.08283041)(292.9344697,88.13751455)
\curveto(291.98395189,89.1921987)(290.5321336,89.71954077)(288.57901482,89.71954077)
\curveto(286.61287524,89.71954077)(285.04386982,89.14011553)(283.87199855,87.98126505)
\curveto(282.71314808,86.82241457)(282.05559809,85.38361707)(281.89934859,83.66487254)
\closepath
}
}
{
\newrgbcolor{curcolor}{0 0 0}
\pscustom[linestyle=none,fillstyle=solid,fillcolor=curcolor]
{
\newpath
\moveto(335.37573979,70.32507127)
\lineto(331.70387648,70.32507127)
\lineto(331.70387648,82.74690672)
\curveto(331.70387648,83.68440373)(331.65830371,84.58934877)(331.56715817,85.46174182)
\curveto(331.48903342,86.33413488)(331.31325273,87.03074724)(331.0398161,87.55157892)
\curveto(330.74033788,88.11147297)(330.31065175,88.53464871)(329.7507577,88.82110613)
\curveto(329.19086365,89.10756355)(328.38357456,89.25079226)(327.32889041,89.25079226)
\curveto(326.30024785,89.25079226)(325.2716053,88.99037642)(324.24296274,88.46954475)
\curveto(323.21432018,87.96173386)(322.18567762,87.31069427)(321.15703506,86.51642597)
\curveto(321.19609744,86.21694775)(321.22864942,85.86538637)(321.254691,85.46174182)
\curveto(321.28073259,85.07111807)(321.29375338,84.68049431)(321.29375338,84.28987055)
\lineto(321.29375338,70.32507127)
\lineto(317.62189007,70.32507127)
\lineto(317.62189007,82.74690672)
\curveto(317.62189007,83.71044532)(317.5763173,84.62190075)(317.48517175,85.48127301)
\curveto(317.407047,86.35366607)(317.23126631,87.05027843)(316.95782968,87.57111011)
\curveto(316.65835147,88.13100416)(316.22866534,88.5476695)(315.66877129,88.82110613)
\curveto(315.10887724,89.10756355)(314.30158814,89.25079226)(313.246904,89.25079226)
\curveto(312.24430302,89.25079226)(311.23519165,89.00339721)(310.21956989,88.50860712)
\curveto(309.21696891,88.01381703)(308.21436794,87.38230862)(307.21176696,86.6140819)
\lineto(307.21176696,70.32507127)
\lineto(303.53990366,70.32507127)
\lineto(303.53990366,92.14140806)
\lineto(307.21176696,92.14140806)
\lineto(307.21176696,89.71954077)
\curveto(308.35759665,90.67005857)(309.49691594,91.41224371)(310.62972483,91.94609618)
\curveto(311.77555452,92.47994864)(312.99299856,92.74687488)(314.28205695,92.74687488)
\curveto(315.76642723,92.74687488)(317.02293364,92.43437587)(318.0515762,91.80937786)
\curveto(319.09323955,91.18437985)(319.86797667,90.31849719)(320.37578755,89.21172988)
\curveto(321.86015783,90.4617259)(323.21432018,91.36016054)(324.43827462,91.9070338)
\curveto(325.66222905,92.46692785)(326.97081864,92.74687488)(328.36404337,92.74687488)
\curveto(330.75986907,92.74687488)(332.52418637,92.01771053)(333.65699526,90.55938184)
\curveto(334.80282495,89.11407394)(335.37573979,87.08934081)(335.37573979,84.48518243)
\closepath
}
}
{
\newrgbcolor{curcolor}{0 0 0}
\pscustom[linestyle=none,fillstyle=solid,fillcolor=curcolor]
{
\newpath
\moveto(359.39909492,70.32507127)
\lineto(355.7467628,70.32507127)
\lineto(355.7467628,72.64928262)
\curveto(355.421243,72.42792916)(354.97853608,72.11543015)(354.41864203,71.7117856)
\curveto(353.87176877,71.32116185)(353.3379163,71.00866284)(352.81708463,70.77428859)
\curveto(352.20510741,70.47481037)(351.50198465,70.22741533)(350.70771634,70.03210345)
\curveto(349.91344804,69.82377078)(348.98246142,69.71960445)(347.91475648,69.71960445)
\curveto(345.94861691,69.71960445)(344.28195555,70.37064404)(342.9147724,71.67272323)
\curveto(341.54758926,72.97480241)(340.86399768,74.63495338)(340.86399768,76.65317612)
\curveto(340.86399768,78.30681669)(341.21555906,79.64144785)(341.91868182,80.65706962)
\curveto(342.63482538,81.68571218)(343.65044714,82.49300128)(344.96554712,83.07893691)
\curveto(346.2936679,83.66487254)(347.8887149,84.0620067)(349.75068814,84.27033937)
\curveto(351.61266138,84.47867204)(353.61135293,84.63492154)(355.7467628,84.73908787)
\lineto(355.7467628,85.30549232)
\curveto(355.7467628,86.138823)(355.59702369,86.82892497)(355.29754548,87.37579823)
\curveto(355.01108806,87.92267149)(354.59442272,88.35235762)(354.04754946,88.66485662)
\curveto(353.52671778,88.96433484)(352.90171977,89.16615711)(352.17255543,89.27032345)
\curveto(351.44339108,89.37448978)(350.68167476,89.42657295)(349.88740645,89.42657295)
\curveto(348.92386785,89.42657295)(347.84965252,89.29636503)(346.66476046,89.03594919)
\curveto(345.4798684,88.78855415)(344.25591397,88.42397197)(342.99289715,87.94220268)
\lineto(342.79758528,87.94220268)
\lineto(342.79758528,91.67265955)
\curveto(343.51372883,91.86797143)(344.54888178,92.08281449)(345.90304414,92.31718875)
\curveto(347.25720649,92.551563)(348.59183766,92.66875013)(349.90693764,92.66875013)
\curveto(351.44339108,92.66875013)(352.77802225,92.53854221)(353.91083114,92.27812637)
\curveto(355.05666083,92.03073132)(356.04624101,91.60104519)(356.87957169,90.98906797)
\curveto(357.69988158,90.39011155)(358.32487959,89.61537443)(358.75456572,88.66485662)
\curveto(359.18425185,87.71433882)(359.39909492,86.53595715)(359.39909492,85.12971163)
\closepath
\moveto(355.7467628,75.69614792)
\lineto(355.7467628,81.77034733)
\curveto(354.6269747,81.70524337)(353.30536432,81.60758743)(351.78193167,81.47737951)
\curveto(350.27151981,81.34717159)(349.07360696,81.15837011)(348.18819311,80.91097506)
\curveto(347.13350897,80.61149685)(346.2806471,80.14274834)(345.62960751,79.50472954)
\curveto(344.97856792,78.87973153)(344.65304812,78.01384887)(344.65304812,76.90708156)
\curveto(344.65304812,75.65708554)(345.03065108,74.71307813)(345.78585701,74.07505933)
\curveto(346.54106294,73.45006132)(347.69340302,73.13756231)(349.24287726,73.13756231)
\curveto(350.53193565,73.13756231)(351.71031732,73.38495736)(352.77802225,73.87974745)
\curveto(353.84572718,74.38755833)(354.83530737,74.99302515)(355.7467628,75.69614792)
\closepath
}
}
{
\newrgbcolor{curcolor}{0 0 0}
\pscustom[linestyle=none,fillstyle=solid,fillcolor=curcolor]
{
\newpath
\moveto(382.05527718,76.61411374)
\curveto(382.05527718,74.62193259)(381.2284569,72.98782321)(379.57481633,71.7117856)
\curveto(377.93419655,70.435748)(375.68810996,69.7977292)(372.83655653,69.7977292)
\curveto(371.22197834,69.7977292)(369.73760807,69.98653068)(368.38344571,70.36413364)
\curveto(367.04230415,70.7547574)(365.91600565,71.17793314)(365.00455022,71.63366085)
\lineto(365.00455022,75.75474148)
\lineto(365.1998621,75.75474148)
\curveto(366.35871258,74.88234842)(367.64777097,74.18573606)(369.06703729,73.66490438)
\curveto(370.4863036,73.1570935)(371.84697635,72.90318806)(373.14905554,72.90318806)
\curveto(374.76363373,72.90318806)(376.02665054,73.1636039)(376.93810598,73.68443557)
\curveto(377.84956141,74.20526725)(378.30528912,75.02557713)(378.30528912,76.14536524)
\curveto(378.30528912,77.0047375)(378.05789408,77.65577709)(377.56310399,78.09848402)
\curveto(377.06831389,78.54119094)(376.11779609,78.91879391)(374.71155056,79.23129291)
\curveto(374.19071889,79.34848004)(373.50712732,79.48519835)(372.66077584,79.64144785)
\curveto(371.82744516,79.79769736)(371.06572884,79.96696765)(370.37562687,80.14925874)
\curveto(368.46157046,80.65706962)(367.10089771,81.39925476)(366.29360862,82.37581415)
\curveto(365.49934031,83.36539433)(365.10220616,84.57632798)(365.10220616,86.00861508)
\curveto(365.10220616,86.90704972)(365.28449725,87.75340119)(365.64907942,88.5476695)
\curveto(366.02668238,89.3419378)(366.59308683,90.05157096)(367.34829276,90.67656897)
\curveto(368.0774571,91.28854619)(369.00193333,91.77031549)(370.12172143,92.12187687)
\curveto(371.25453032,92.48645904)(372.51754713,92.66875013)(373.91077186,92.66875013)
\curveto(375.21285105,92.66875013)(376.52795103,92.50599023)(377.8560718,92.18047043)
\curveto(379.19721337,91.86797143)(380.31049107,91.48385807)(381.19590492,91.02813035)
\lineto(381.19590492,87.1023616)
\lineto(381.00059304,87.1023616)
\curveto(380.06309603,87.79246357)(378.92377674,88.37188881)(377.58263517,88.84063731)
\curveto(376.24149361,89.32240661)(374.92639363,89.56329126)(373.63733523,89.56329126)
\curveto(372.29619367,89.56329126)(371.16338478,89.30287543)(370.23890856,88.78204375)
\curveto(369.31443233,88.27423287)(368.85219422,87.51251654)(368.85219422,86.49689478)
\curveto(368.85219422,85.59846014)(369.13214125,84.92137896)(369.6920353,84.46565124)
\curveto(370.23890856,84.00992353)(371.1243224,83.63883096)(372.34827684,83.35237354)
\curveto(373.02535802,83.19612404)(373.78056395,83.03987453)(374.61389463,82.88362503)
\curveto(375.4602461,82.72737553)(376.16336886,82.58414682)(376.72326291,82.4539389)
\curveto(378.42898665,82.06331514)(379.74408662,81.39274436)(380.66856285,80.44222655)
\curveto(381.59303907,79.47868796)(382.05527718,78.20265035)(382.05527718,76.61411374)
\closepath
}
}
{
\newrgbcolor{curcolor}{0 0 0}
\pscustom[linestyle=none,fillstyle=solid,fillcolor=curcolor]
{
\newpath
\moveto(407.11379502,70.32507127)
\lineto(402.27006045,70.32507127)
\lineto(393.5200883,79.87582211)
\lineto(391.13728339,77.61020432)
\lineto(391.13728339,70.32507127)
\lineto(387.46542008,70.32507127)
\lineto(387.46542008,100.71559951)
\lineto(391.13728339,100.71559951)
\lineto(391.13728339,81.22347407)
\lineto(401.74271837,92.14140806)
\lineto(406.37160989,92.14140806)
\lineto(396.23492341,82.06331514)
\closepath
}
}
\end{pspicture}
}
        \captionsetup{width=0.9\textwidth}
        \caption{The default processors and memory a process is allowed to use is typically all of them available on the system.}
        \label{fig:defaultmasks}
    \end{subfigure}
    \begin{subfigure}{0.4\textwidth}
        \centering
        \resizebox{0.9\linewidth}{!}{\input{diagrams/Half cpumask and nodemask}}
        \captionsetup{width=0.9\textwidth}
        \caption{The default masks used in testing, in order to ensure that a process was running in a known location.
        This configuration was also mirrored to the other node to ensure that there were not differences between nodes, 
        but they performed identically throughout testing.}
        \label{fig:testingmasks}
    \end{subfigure}
    \captionsetup{width=0.8\textwidth}
    \caption{Each process has a specific set of processors it is allowed to run on and NUMA  nodes that it is allowed to allocate memory from.}
\end{figure}

The first step, changing where a process is running, was quite straightforward.
Linux provides a convenient interface to inform the scheduler where a process is allowed to run.
By default, a program is allowed to run anywhere (Figure \ref{fig:defaultmasks}), 
but for testing purposes programs were typically restricted to one NUMA node at the start, 
in order to simplify data collection (Figure \ref{fig:testingmasks}).
By restricting this set of processors to only those in a specific NUMA node,
we can ensure that the scheduler runs a process on that specific node.
This interface is a bitmask, a set of bits, one per each processor in the system, 
where each bit that is set to 1 is a processor that is allowed, and each bit that set to 0 is a processor that is not allowed.
Each NUMA node typically has multiple processors, in the case of my test system each node has four processors. 
Linux also provides a convenience function, \texttt{cpumask\_of\_node}, that can convert a simple node number to a bitmask
of processors, referred to internally in the Linux kernel as a cpumask.
This cpumask can then be passed, along with the task structure of the current process,
to the \texttt{set\_cpus\_allowed\_ptr} function, which will correctly set the cpumask of process whose
task structure is passed to it (Listing \ref{lst:linux_set_cpus_allowed}).

\begin{figure}[H]
    \centering
    \resizebox{0.5\linewidth}{!}{%LaTeX with PSTricks extensions
%%Creator: Inkscape 1.0.2-2 (e86c870879, 2021-01-15)
%%Please note this file requires PSTricks extensions
\psset{xunit=.5pt,yunit=.5pt,runit=.5pt}
\begin{pspicture}(612.31414555,460.59295414)
{
\newrgbcolor{curcolor}{0 0 0}
\pscustom[linestyle=none,fillstyle=solid,fillcolor=curcolor]
{
\newpath
\moveto(151.90079622,282.38599981)
\lineto(151.90079622,189.45195257)
}
}
{
\newrgbcolor{curcolor}{0 0 0}
\pscustom[linewidth=2.64566925,linecolor=curcolor]
{
\newpath
\moveto(151.90079622,282.38599981)
\lineto(151.90079622,189.45195257)
}
}
{
\newrgbcolor{curcolor}{0.80000001 0.80000001 0.80000001}
\pscustom[linestyle=none,fillstyle=solid,fillcolor=curcolor]
{
\newpath
\moveto(197.92706884,396.86887612)
\lineto(412.45029296,396.86887612)
\lineto(412.45029296,339.13158392)
\lineto(197.92706884,339.13158392)
\closepath
}
}
{
\newrgbcolor{curcolor}{0 0 0}
\pscustom[linewidth=0.4430022,linecolor=curcolor]
{
\newpath
\moveto(197.92706884,396.86887612)
\lineto(412.45029296,396.86887612)
\lineto(412.45029296,339.13158392)
\lineto(197.92706884,339.13158392)
\closepath
}
}
{
\newrgbcolor{curcolor}{0.80000001 0.80000001 0.80000001}
\pscustom[linestyle=none,fillstyle=solid,fillcolor=curcolor]
{
\newpath
\moveto(68.05813103,442.6695228)
\lineto(235.74345203,442.6695228)
\lineto(235.74345203,278.0146694)
\lineto(68.05813103,278.0146694)
\closepath
}
}
{
\newrgbcolor{curcolor}{0 0 0}
\pscustom[linewidth=0.66141731,linecolor=curcolor]
{
\newpath
\moveto(68.05813103,442.6695228)
\lineto(235.74345203,442.6695228)
\lineto(235.74345203,278.0146694)
\lineto(68.05813103,278.0146694)
\closepath
}
}
{
\newrgbcolor{curcolor}{0.80000001 0.80000001 0.80000001}
\pscustom[linestyle=none,fillstyle=solid,fillcolor=curcolor]
{
\newpath
\moveto(376.5711368,442.6695228)
\lineto(544.2564578,442.6695228)
\lineto(544.2564578,278.0146694)
\lineto(376.5711368,278.0146694)
\closepath
}
}
{
\newrgbcolor{curcolor}{0 0 0}
\pscustom[linewidth=0.66141731,linecolor=curcolor]
{
\newpath
\moveto(376.5711368,442.6695228)
\lineto(544.2564578,442.6695228)
\lineto(544.2564578,278.0146694)
\lineto(376.5711368,278.0146694)
\closepath
}
}
{
\newrgbcolor{curcolor}{0 0 0}
\pscustom[linestyle=none,fillstyle=solid,fillcolor=curcolor]
{
\newpath
\moveto(138.21920392,347.94960497)
\curveto(137.50306037,347.63710597)(136.85202077,347.34413815)(136.26608514,347.07070152)
\curveto(135.6931703,346.79726489)(134.93796437,346.51080747)(134.00046735,346.21132926)
\curveto(133.20619905,345.96393421)(132.34031639,345.75560154)(131.40281937,345.58633125)
\curveto(130.47834315,345.40404016)(129.45621099,345.31289462)(128.33642289,345.31289462)
\curveto(126.2270546,345.31289462)(124.3064878,345.60586244)(122.57472248,346.19179807)
\curveto(120.85597795,346.7907545)(119.35858689,347.72174112)(118.08254928,348.98475793)
\curveto(116.83255326,350.22173316)(115.85599387,351.79073858)(115.15287111,353.69177419)
\curveto(114.44974835,355.6058306)(114.09818697,357.82587561)(114.09818697,360.35190924)
\curveto(114.09818697,362.74773494)(114.43672756,364.88965521)(115.11380874,366.77767003)
\curveto(115.79088991,368.66568485)(116.7674493,370.26073185)(118.04348691,371.56281104)
\curveto(119.28046214,372.82582785)(120.77134281,373.78936645)(122.51612892,374.45342684)
\curveto(124.27393582,375.11748722)(126.22054421,375.44951742)(128.35595407,375.44951742)
\curveto(129.9184491,375.44951742)(131.47443373,375.26071593)(133.02390796,374.88311297)
\curveto(134.58640299,374.50551001)(136.31816831,373.84144962)(138.21920392,372.89093181)
\lineto(138.21920392,368.30110268)
\lineto(137.9262361,368.30110268)
\curveto(136.3246787,369.64224424)(134.73614209,370.61880363)(133.16062628,371.23078085)
\curveto(131.58511046,371.84275807)(129.89891791,372.14874668)(128.10204863,372.14874668)
\curveto(126.63069915,372.14874668)(125.30257838,371.90786203)(124.11768632,371.42609273)
\curveto(122.94581505,370.95734422)(121.8976413,370.22166948)(120.97316508,369.2190685)
\curveto(120.07473044,368.24250911)(119.37160768,367.00553389)(118.8637968,365.50814282)
\curveto(118.36900671,364.02377255)(118.12161166,362.30502802)(118.12161166,360.35190924)
\curveto(118.12161166,358.30764491)(118.39504829,356.54983801)(118.94192155,355.07848853)
\curveto(119.5018156,353.60713905)(120.21795915,352.40922619)(121.09035221,351.48474997)
\curveto(122.00180764,350.52121137)(123.06300218,349.80506782)(124.27393582,349.33631931)
\curveto(125.49789026,348.88059159)(126.78694865,348.65272774)(128.14111101,348.65272774)
\curveto(130.00308425,348.65272774)(131.74787036,348.97173714)(133.37546934,349.60975594)
\curveto(135.00306833,350.24777474)(136.52650098,351.20480294)(137.94576729,352.48084055)
\lineto(138.21920392,352.48084055)
\closepath
}
}
{
\newrgbcolor{curcolor}{0 0 0}
\pscustom[linestyle=none,fillstyle=solid,fillcolor=curcolor]
{
\newpath
\moveto(162.71131286,366.13314083)
\curveto(162.71131286,364.84408243)(162.483449,363.64616958)(162.02772128,362.53940227)
\curveto(161.58501436,361.44565575)(160.96001635,360.49513795)(160.15272725,359.68784885)
\curveto(159.15012628,358.68524788)(157.96523422,357.93004195)(156.59805107,357.42223106)
\curveto(155.23086792,356.92744097)(153.505613,356.68004593)(151.4222863,356.68004593)
\lineto(147.55511111,356.68004593)
\lineto(147.55511111,345.84023669)
\lineto(143.68793592,345.84023669)
\lineto(143.68793592,374.92217535)
\lineto(151.5785358,374.92217535)
\curveto(153.32332191,374.92217535)(154.80118179,374.77243624)(156.01211543,374.47295803)
\curveto(157.22304908,374.18650061)(158.29726441,373.73077289)(159.23476142,373.10577488)
\curveto(160.34152873,372.36358974)(161.1943906,371.43911352)(161.79334703,370.33234621)
\curveto(162.40532425,369.2255789)(162.71131286,367.82584377)(162.71131286,366.13314083)
\closepath
\moveto(158.68788817,366.03548489)
\curveto(158.68788817,367.03808587)(158.51210748,367.91047892)(158.16054609,368.65266406)
\curveto(157.80898471,369.39484919)(157.27513225,370.00031602)(156.55898869,370.46906452)
\curveto(155.93399068,370.87270907)(155.21784713,371.15916649)(154.41055803,371.32843679)
\curveto(153.61628973,371.51072787)(152.60717836,371.60187342)(151.38322392,371.60187342)
\lineto(147.55511111,371.60187342)
\lineto(147.55511111,359.98081667)
\lineto(150.81681948,359.98081667)
\curveto(152.3793145,359.98081667)(153.64884171,360.11753498)(154.6254011,360.39097161)
\curveto(155.60196049,360.67742903)(156.3962288,361.12664635)(157.00820601,361.73862357)
\curveto(157.62018323,362.36362158)(158.04986936,363.02117157)(158.29726441,363.71127354)
\curveto(158.55768025,364.40137551)(158.68788817,365.17611263)(158.68788817,366.03548489)
\closepath
}
}
{
\newrgbcolor{curcolor}{0 0 0}
\pscustom[linestyle=none,fillstyle=solid,fillcolor=curcolor]
{
\newpath
\moveto(189.70341252,357.519887)
\curveto(189.70341252,355.41051872)(189.46903826,353.56807667)(189.00028976,351.99256085)
\curveto(188.54456204,350.43006583)(187.78935611,349.12798664)(186.73467197,348.08632329)
\curveto(185.73207099,347.09674311)(184.56019973,346.37408916)(183.21905816,345.91836144)
\curveto(181.8779166,345.46263373)(180.31542157,345.23476987)(178.53157309,345.23476987)
\curveto(176.70866222,345.23476987)(175.12012562,345.47565452)(173.76596326,345.95742382)
\curveto(172.41180091,346.43919312)(171.27248162,347.14882627)(170.34800539,348.08632329)
\curveto(169.29332125,349.15402822)(168.53160493,350.44308662)(168.06285642,351.95349848)
\curveto(167.6071287,353.46391033)(167.37926485,355.31937318)(167.37926485,357.519887)
\lineto(167.37926485,374.92217535)
\lineto(171.24644003,374.92217535)
\lineto(171.24644003,357.32457513)
\curveto(171.24644003,355.74905931)(171.35060637,354.50557368)(171.55893904,353.59411825)
\curveto(171.7802925,352.68266282)(172.14487467,351.85584254)(172.65268556,351.1136574)
\curveto(173.2256004,350.26730593)(174.00033751,349.62928713)(174.9768969,349.199601)
\curveto(175.96647709,348.76991486)(177.15136915,348.5550718)(178.53157309,348.5550718)
\curveto(179.92479782,348.5550718)(181.10968988,348.76340447)(182.08624927,349.18006981)
\curveto(183.06280866,349.60975594)(183.84405617,350.25428514)(184.42999181,351.1136574)
\curveto(184.93780269,351.85584254)(185.29587447,352.70219401)(185.50420714,353.65271182)
\curveto(185.7255606,354.61625042)(185.83623733,355.80765287)(185.83623733,357.22691919)
\lineto(185.83623733,374.92217535)
\lineto(189.70341252,374.92217535)
\closepath
}
}
{
\newrgbcolor{curcolor}{0 0 0}
\pscustom[linestyle=none,fillstyle=solid,fillcolor=curcolor]
{
\newpath
\moveto(446.73219528,342.89880642)
\curveto(446.01605172,342.58630742)(445.36501213,342.2933396)(444.7790765,342.01990297)
\curveto(444.20616165,341.74646634)(443.45095572,341.46000892)(442.51345871,341.16053071)
\curveto(441.71919041,340.91313566)(440.85330775,340.70480299)(439.91581073,340.5355327)
\curveto(438.99133451,340.35324161)(437.96920234,340.26209607)(436.84941424,340.26209607)
\curveto(434.74004596,340.26209607)(432.81947916,340.55506389)(431.08771384,341.14099952)
\curveto(429.36896931,341.73995595)(427.87157825,342.67094257)(426.59554064,343.93395938)
\curveto(425.34554462,345.17093461)(424.36898523,346.73994003)(423.66586247,348.64097564)
\curveto(422.96273971,350.55503205)(422.61117833,352.77507706)(422.61117833,355.30111069)
\curveto(422.61117833,357.69693639)(422.94971892,359.83885665)(423.62680009,361.72687148)
\curveto(424.30388127,363.6148863)(425.28044066,365.2099333)(426.55647827,366.51201249)
\curveto(427.79345349,367.7750293)(429.28433416,368.7385679)(431.02912028,369.40262829)
\curveto(432.78692718,370.06668867)(434.73353556,370.39871887)(436.86894543,370.39871887)
\curveto(438.43144046,370.39871887)(439.98742509,370.20991738)(441.53689932,369.83231442)
\curveto(443.09939434,369.45471145)(444.83115966,368.79065107)(446.73219528,367.84013326)
\lineto(446.73219528,363.25030413)
\lineto(446.43922746,363.25030413)
\curveto(444.83767006,364.59144569)(443.24913345,365.56800508)(441.67361763,366.1799823)
\curveto(440.09810182,366.79195952)(438.41190927,367.09794813)(436.61503999,367.09794813)
\curveto(435.14369051,367.09794813)(433.81556974,366.85706348)(432.63067768,366.37529418)
\curveto(431.45880641,365.90654567)(430.41063266,365.17087093)(429.48615644,364.16826995)
\curveto(428.5877218,363.19171056)(427.88459904,361.95473533)(427.37678815,360.45734427)
\curveto(426.88199806,358.97297399)(426.63460302,357.25422947)(426.63460302,355.30111069)
\curveto(426.63460302,353.25684636)(426.90803965,351.49903946)(427.45491291,350.02768998)
\curveto(428.01480696,348.55634049)(428.73095051,347.35842764)(429.60334356,346.43395142)
\curveto(430.514799,345.47041282)(431.57599353,344.75426927)(432.78692718,344.28552076)
\curveto(434.01088161,343.82979304)(435.29994001,343.60192918)(436.65410237,343.60192918)
\curveto(438.5160756,343.60192918)(440.26086172,343.92093859)(441.8884607,344.55895739)
\curveto(443.51605968,345.19697619)(445.03949233,346.15400439)(446.45875865,347.430042)
\lineto(446.73219528,347.430042)
\closepath
}
}
{
\newrgbcolor{curcolor}{0 0 0}
\pscustom[linestyle=none,fillstyle=solid,fillcolor=curcolor]
{
\newpath
\moveto(471.22430421,361.08234228)
\curveto(471.22430421,359.79328388)(470.99644035,358.59537103)(470.54071264,357.48860372)
\curveto(470.09800571,356.3948572)(469.4730077,355.4443394)(468.66571861,354.6370503)
\curveto(467.66311763,353.63444933)(466.47822557,352.8792434)(465.11104243,352.37143251)
\curveto(463.74385928,351.87664242)(462.01860436,351.62924738)(459.93527766,351.62924738)
\lineto(456.06810247,351.62924738)
\lineto(456.06810247,340.78943814)
\lineto(452.20092728,340.78943814)
\lineto(452.20092728,369.87137679)
\lineto(460.09152716,369.87137679)
\curveto(461.83631327,369.87137679)(463.31417315,369.72163769)(464.52510679,369.42215948)
\curveto(465.73604044,369.13570205)(466.81025577,368.67997434)(467.74775278,368.05497633)
\curveto(468.85452009,367.31279119)(469.70738196,366.38831497)(470.30633838,365.28154766)
\curveto(470.9183156,364.17478035)(471.22430421,362.77504522)(471.22430421,361.08234228)
\closepath
\moveto(467.20087952,360.98468634)
\curveto(467.20087952,361.98728731)(467.02509883,362.85968037)(466.67353745,363.60186551)
\curveto(466.32197607,364.34405064)(465.7881236,364.94951747)(465.07198005,365.41826597)
\curveto(464.44698204,365.82191052)(463.73083849,366.10836794)(462.92354939,366.27763824)
\curveto(462.12928109,366.45992932)(461.12016972,366.55107487)(459.89621528,366.55107487)
\lineto(456.06810247,366.55107487)
\lineto(456.06810247,354.93001812)
\lineto(459.32981083,354.93001812)
\curveto(460.89230586,354.93001812)(462.16183307,355.06673643)(463.13839246,355.34017306)
\curveto(464.11495185,355.62663048)(464.90922015,356.0758478)(465.52119737,356.68782502)
\curveto(466.13317459,357.31282303)(466.56286072,357.97037302)(466.81025577,358.66047499)
\curveto(467.0706716,359.35057696)(467.20087952,360.12531408)(467.20087952,360.98468634)
\closepath
}
}
{
\newrgbcolor{curcolor}{0 0 0}
\pscustom[linestyle=none,fillstyle=solid,fillcolor=curcolor]
{
\newpath
\moveto(498.21640387,352.46908845)
\curveto(498.21640387,350.35972017)(497.98202962,348.51727812)(497.51328111,346.9417623)
\curveto(497.0575534,345.37926728)(496.30234747,344.07718809)(495.24766333,343.03552474)
\curveto(494.24506235,342.04594456)(493.07319108,341.32329061)(491.73204952,340.86756289)
\curveto(490.39090796,340.41183518)(488.82841293,340.18397132)(487.04456444,340.18397132)
\curveto(485.22165358,340.18397132)(483.63311697,340.42485597)(482.27895462,340.90662527)
\curveto(480.92479226,341.38839457)(479.78547297,342.09802772)(478.86099675,343.03552474)
\curveto(477.80631261,344.10322967)(477.04459628,345.39228807)(476.57584778,346.90269993)
\curveto(476.12012006,348.41311178)(475.8922562,350.26857463)(475.8922562,352.46908845)
\lineto(475.8922562,369.87137679)
\lineto(479.75943139,369.87137679)
\lineto(479.75943139,352.27377657)
\curveto(479.75943139,350.69826076)(479.86359772,349.45477513)(480.07193039,348.5433197)
\curveto(480.29328386,347.63186427)(480.65786603,346.80504399)(481.16567691,346.06285885)
\curveto(481.73859175,345.21650738)(482.51332887,344.57848858)(483.48988826,344.14880244)
\curveto(484.47946844,343.71911631)(485.66436051,343.50427325)(487.04456444,343.50427325)
\curveto(488.43778917,343.50427325)(489.62268124,343.71260592)(490.59924063,344.12927126)
\curveto(491.57580002,344.55895739)(492.35704753,345.20348659)(492.94298316,346.06285885)
\curveto(493.45079405,346.80504399)(493.80886582,347.65139546)(494.01719849,348.60191327)
\curveto(494.23855196,349.56545186)(494.34922869,350.75685432)(494.34922869,352.17612064)
\lineto(494.34922869,369.87137679)
\lineto(498.21640387,369.87137679)
\closepath
}
}
{
\newrgbcolor{curcolor}{0.80000001 0.80000001 0.80000001}
\pscustom[linestyle=none,fillstyle=solid,fillcolor=curcolor]
{
\newpath
\moveto(87.43154027,313.85292239)
\lineto(216.37004279,313.85292239)
\lineto(216.37004279,278.01471986)
\lineto(87.43154027,278.01471986)
\closepath
}
}
{
\newrgbcolor{curcolor}{0 0 0}
\pscustom[linewidth=0.27058431,linecolor=curcolor]
{
\newpath
\moveto(87.43154027,313.85292239)
\lineto(216.37004279,313.85292239)
\lineto(216.37004279,278.01471986)
\lineto(87.43154027,278.01471986)
\closepath
}
}
{
\newrgbcolor{curcolor}{0 0 0}
\pscustom[linestyle=none,fillstyle=solid,fillcolor=curcolor]
{
\newpath
\moveto(105.97956403,290.35453275)
\lineto(104.86077443,290.35453275)
\lineto(104.86077443,302.86877695)
\lineto(102.52148709,294.35362872)
\lineto(101.8547337,294.35362872)
\lineto(99.53239772,302.86877695)
\lineto(99.53239772,290.35453275)
\lineto(98.487064,290.35453275)
\lineto(98.487064,304.87807882)
\lineto(100.01268618,304.87807882)
\lineto(102.25591583,296.79210186)
\lineto(104.42568959,304.87807882)
\lineto(105.97956403,304.87807882)
\closepath
}
}
{
\newrgbcolor{curcolor}{0 0 0}
\pscustom[linestyle=none,fillstyle=solid,fillcolor=curcolor]
{
\newpath
\moveto(113.43816103,295.61188086)
\lineto(108.78783862,295.61188086)
\curveto(108.78783862,294.94211357)(108.84622663,294.35688001)(108.96300265,293.85618019)
\curveto(109.07977867,293.36198297)(109.23987482,292.95557078)(109.44329111,292.63694362)
\curveto(109.63917346,292.32481906)(109.87084202,292.09072563)(110.13829677,291.93466335)
\curveto(110.40951849,291.77860107)(110.70710899,291.70056993)(111.03106826,291.70056993)
\curveto(111.46050265,291.70056993)(111.89182053,291.84687832)(112.32502189,292.1394951)
\curveto(112.76199022,292.43861447)(113.0727651,292.73123125)(113.25734655,293.01734543)
\lineto(113.31385108,293.01734543)
\lineto(113.31385108,291.01779745)
\curveto(112.95598908,290.75769365)(112.59059316,290.53985671)(112.21766329,290.36428664)
\curveto(111.84473343,290.18871658)(111.45296872,290.10093154)(111.04236917,290.10093154)
\curveto(109.99515197,290.10093154)(109.17771984,290.58862617)(108.59007278,291.56401543)
\curveto(108.00242572,292.54590729)(107.70860219,293.93746263)(107.70860219,295.73868146)
\curveto(107.70860219,297.52039251)(107.98924133,298.93470694)(108.55051961,299.98162474)
\curveto(109.11556486,301.02854255)(109.85765763,301.55200145)(110.7767979,301.55200145)
\curveto(111.62813275,301.55200145)(112.28358524,301.12283017)(112.74315537,300.26448763)
\curveto(113.20649248,299.40614508)(113.43816103,298.1869085)(113.43816103,296.6067779)
\closepath
\moveto(112.40412822,297.01644139)
\curveto(112.40036125,297.97882546)(112.25909994,298.7233726)(111.98034429,299.2500828)
\curveto(111.7053556,299.776793)(111.28533863,300.0401481)(110.72029338,300.0401481)
\curveto(110.15148116,300.0401481)(109.69756147,299.75078262)(109.35853432,299.17205166)
\curveto(109.02327414,298.5933207)(108.83304224,297.87478394)(108.78783862,297.01644139)
\closepath
}
}
{
\newrgbcolor{curcolor}{0 0 0}
\pscustom[linestyle=none,fillstyle=solid,fillcolor=curcolor]
{
\newpath
\moveto(124.25877827,290.35453275)
\lineto(123.1964932,290.35453275)
\lineto(123.1964932,296.55800844)
\curveto(123.1964932,297.02619529)(123.18330881,297.47812564)(123.15694003,297.91379951)
\curveto(123.13433822,298.34947338)(123.08348415,298.69736222)(123.00437782,298.95746602)
\curveto(122.91773754,299.23707761)(122.79342759,299.44841195)(122.63144795,299.59146904)
\curveto(122.46946831,299.73452613)(122.23591627,299.80605467)(121.93079184,299.80605467)
\curveto(121.63320134,299.80605467)(121.33561084,299.67600277)(121.03802034,299.41589897)
\curveto(120.74042984,299.16229776)(120.44283935,298.83716801)(120.14524885,298.44050971)
\curveto(120.15654975,298.29095003)(120.16596717,298.11537996)(120.17350111,297.91379951)
\curveto(120.18103505,297.71872166)(120.18480201,297.52364381)(120.18480201,297.32856596)
\lineto(120.18480201,290.35453275)
\lineto(119.12251694,290.35453275)
\lineto(119.12251694,296.55800844)
\curveto(119.12251694,297.03920048)(119.10933255,297.49438213)(119.08296378,297.9235534)
\curveto(119.06036197,298.35922727)(119.00950789,298.70711611)(118.93040156,298.96721991)
\curveto(118.84376129,299.2468315)(118.71945133,299.45491454)(118.55747169,299.59146904)
\curveto(118.39549205,299.73452613)(118.16194002,299.80605467)(117.85681558,299.80605467)
\curveto(117.56675902,299.80605467)(117.27481897,299.68250537)(116.98099544,299.43540676)
\curveto(116.69093888,299.18830814)(116.40088232,298.87293228)(116.11082576,298.48927917)
\lineto(116.11082576,290.35453275)
\lineto(115.04854069,290.35453275)
\lineto(115.04854069,301.24963078)
\lineto(116.11082576,301.24963078)
\lineto(116.11082576,300.0401481)
\curveto(116.44231897,300.51483754)(116.7719287,300.88548545)(117.09965495,301.15209185)
\curveto(117.43114816,301.41869825)(117.7833597,301.55200145)(118.15628957,301.55200145)
\curveto(118.58572396,301.55200145)(118.9492364,301.39593917)(119.2468269,301.0838146)
\curveto(119.54818437,300.77169004)(119.77231898,300.33926747)(119.91923075,299.78654689)
\curveto(120.34866514,300.41079602)(120.74042984,300.85947507)(121.09452487,301.13258407)
\curveto(121.44861989,301.41219565)(121.82720021,301.55200145)(122.23026582,301.55200145)
\curveto(122.923388,301.55200145)(123.43381221,301.18785613)(123.76153845,300.45956548)
\curveto(124.09303166,299.73777743)(124.25877827,298.72662389)(124.25877827,297.42610488)
\closepath
}
}
{
\newrgbcolor{curcolor}{0 0 0}
\pscustom[linestyle=none,fillstyle=solid,fillcolor=curcolor]
{
\newpath
\moveto(131.68347365,295.79720482)
\curveto(131.68347365,294.02199637)(131.41978586,292.62068713)(130.8924103,291.59327711)
\curveto(130.36503473,290.56586709)(129.65872817,290.05216208)(128.77349061,290.05216208)
\curveto(127.88071911,290.05216208)(127.17064558,290.56586709)(126.64327001,291.59327711)
\curveto(126.11966141,292.62068713)(125.85785711,294.02199637)(125.85785711,295.79720482)
\curveto(125.85785711,297.57241327)(126.11966141,298.97372251)(126.64327001,300.00113253)
\curveto(127.17064558,301.03504514)(127.88071911,301.55200145)(128.77349061,301.55200145)
\curveto(129.65872817,301.55200145)(130.36503473,301.03504514)(130.8924103,300.00113253)
\curveto(131.41978586,298.97372251)(131.68347365,297.57241327)(131.68347365,295.79720482)
\closepath
\moveto(130.58728586,295.79720482)
\curveto(130.58728586,297.20826795)(130.42718971,298.25518575)(130.1069974,298.93795823)
\curveto(129.78680509,299.62723331)(129.34230283,299.97187085)(128.77349061,299.97187085)
\curveto(128.19714445,299.97187085)(127.74887522,299.62723331)(127.42868291,298.93795823)
\curveto(127.11225757,298.25518575)(126.9540449,297.20826795)(126.9540449,295.79720482)
\curveto(126.9540449,294.43165986)(127.11414105,293.39449594)(127.43433336,292.68571308)
\curveto(127.75452567,291.98343281)(128.20091142,291.63229268)(128.77349061,291.63229268)
\curveto(129.33853586,291.63229268)(129.78115464,291.98018152)(130.10134694,292.67595919)
\curveto(130.42530622,293.37823946)(130.58728586,294.41865467)(130.58728586,295.79720482)
\closepath
}
}
{
\newrgbcolor{curcolor}{0 0 0}
\pscustom[linestyle=none,fillstyle=solid,fillcolor=curcolor]
{
\newpath
\moveto(137.26611962,299.2500828)
\lineto(137.2096151,299.2500828)
\curveto(137.05140243,299.31510875)(136.89695673,299.36062691)(136.74627799,299.38663729)
\curveto(136.59936623,299.41915027)(136.4242022,299.43540676)(136.22078591,299.43540676)
\curveto(135.89305966,299.43540676)(135.57663432,299.30860615)(135.27150989,299.05500494)
\curveto(134.96638545,298.80790633)(134.67256192,298.48602788)(134.3900393,298.08936958)
\lineto(134.3900393,290.35453275)
\lineto(133.32775423,290.35453275)
\lineto(133.32775423,301.24963078)
\lineto(134.3900393,301.24963078)
\lineto(134.3900393,299.6402385)
\curveto(134.81193975,300.22547206)(135.18298613,300.63838684)(135.50317844,300.87898286)
\curveto(135.82713772,301.12608147)(136.15674745,301.24963078)(136.49200763,301.24963078)
\curveto(136.67658908,301.24963078)(136.81031645,301.23987689)(136.89318976,301.2203691)
\curveto(136.97606306,301.20736391)(137.10037302,301.17810223)(137.26611962,301.13258407)
\closepath
}
}
{
\newrgbcolor{curcolor}{0 0 0}
\pscustom[linestyle=none,fillstyle=solid,fillcolor=curcolor]
{
\newpath
\moveto(143.72458599,301.24963078)
\lineto(140.04049096,286.335929)
\lineto(138.90475001,286.335929)
\lineto(140.08004413,290.88124295)
\lineto(137.56559276,301.24963078)
\lineto(138.71828507,301.24963078)
\lineto(140.65639028,293.17340771)
\lineto(142.61144685,301.24963078)
\closepath
}
}
{
\newrgbcolor{curcolor}{0 0 0}
\pscustom[linestyle=none,fillstyle=solid,fillcolor=curcolor]
{
\newpath
\moveto(155.76570073,291.40795315)
\curveto(155.55851747,291.25189087)(155.37016905,291.10558248)(155.20065548,290.96902799)
\curveto(155.03490887,290.83247349)(154.81642471,290.6894164)(154.54520299,290.53985671)
\curveto(154.31541792,290.4163074)(154.06491452,290.31226588)(153.7936928,290.22773215)
\curveto(153.52623805,290.13669582)(153.23053104,290.09117765)(152.90657176,290.09117765)
\curveto(152.29632289,290.09117765)(151.74069506,290.23748604)(151.23968827,290.53010282)
\curveto(150.74244845,290.82922219)(150.30924709,291.29415774)(149.94008419,291.92490946)
\curveto(149.57845523,292.54265599)(149.29593261,293.3262187)(149.09251632,294.27559757)
\curveto(148.88910003,295.23147905)(148.78739188,296.34017151)(148.78739188,297.60167495)
\curveto(148.78739188,298.79815244)(148.88533306,299.86782933)(149.08121541,300.81070561)
\curveto(149.27709777,301.7535819)(149.55962039,302.55014979)(149.92878329,303.2004093)
\curveto(150.28664528,303.83116102)(150.71796316,304.31235305)(151.22273691,304.6439854)
\curveto(151.73127764,304.97561775)(152.2944394,305.14143392)(152.91222221,305.14143392)
\curveto(153.36425841,305.14143392)(153.81441113,305.04714629)(154.26268036,304.85857104)
\curveto(154.71471656,304.66999578)(155.21572335,304.33836343)(155.76570073,303.86367399)
\lineto(155.76570073,301.57150923)
\lineto(155.68094394,301.57150923)
\curveto(155.21760683,302.24127653)(154.7580367,302.72897115)(154.30223353,303.03459312)
\curveto(153.84643036,303.34021509)(153.35860796,303.49302607)(152.83876633,303.49302607)
\curveto(152.41309891,303.49302607)(152.02886814,303.37272807)(151.68607402,303.13213205)
\curveto(151.34704687,302.89803863)(151.04380592,302.53064201)(150.77635116,302.02994219)
\curveto(150.51643035,301.54224756)(150.31301406,300.92450103)(150.16610229,300.17670259)
\curveto(150.0229575,299.43540676)(149.9513851,298.57706421)(149.9513851,297.60167495)
\curveto(149.9513851,296.58076752)(150.03049143,295.70291719)(150.1887041,294.96812395)
\curveto(150.35068374,294.23333071)(150.557867,293.63509196)(150.81025388,293.17340771)
\curveto(151.07394166,292.69221568)(151.38094958,292.33457295)(151.73127764,292.10047953)
\curveto(152.08537266,291.8728887)(152.45830253,291.75909329)(152.85006723,291.75909329)
\curveto(153.38874371,291.75909329)(153.89351746,291.91840686)(154.36438851,292.23703402)
\curveto(154.83525955,292.55566118)(155.27599484,293.03360192)(155.68659439,293.67085623)
\lineto(155.76570073,293.67085623)
\closepath
}
}
{
\newrgbcolor{curcolor}{0 0 0}
\pscustom[linestyle=none,fillstyle=solid,fillcolor=curcolor]
{
\newpath
\moveto(162.64229987,295.79720482)
\curveto(162.64229987,294.02199637)(162.37861209,292.62068713)(161.85123652,291.59327711)
\curveto(161.32386095,290.56586709)(160.61755439,290.05216208)(159.73231683,290.05216208)
\curveto(158.83954533,290.05216208)(158.1294718,290.56586709)(157.60209624,291.59327711)
\curveto(157.07848764,292.62068713)(156.81668334,294.02199637)(156.81668334,295.79720482)
\curveto(156.81668334,297.57241327)(157.07848764,298.97372251)(157.60209624,300.00113253)
\curveto(158.1294718,301.03504514)(158.83954533,301.55200145)(159.73231683,301.55200145)
\curveto(160.61755439,301.55200145)(161.32386095,301.03504514)(161.85123652,300.00113253)
\curveto(162.37861209,298.97372251)(162.64229987,297.57241327)(162.64229987,295.79720482)
\closepath
\moveto(161.54611208,295.79720482)
\curveto(161.54611208,297.20826795)(161.38601593,298.25518575)(161.06582362,298.93795823)
\curveto(160.74563131,299.62723331)(160.30112905,299.97187085)(159.73231683,299.97187085)
\curveto(159.15597067,299.97187085)(158.70770144,299.62723331)(158.38750913,298.93795823)
\curveto(158.07108379,298.25518575)(157.91287112,297.20826795)(157.91287112,295.79720482)
\curveto(157.91287112,294.43165986)(158.07296728,293.39449594)(158.39315959,292.68571308)
\curveto(158.71335189,291.98343281)(159.15973764,291.63229268)(159.73231683,291.63229268)
\curveto(160.29736208,291.63229268)(160.73998086,291.98018152)(161.06017317,292.67595919)
\curveto(161.38413245,293.37823946)(161.54611208,294.41865467)(161.54611208,295.79720482)
\closepath
}
}
{
\newrgbcolor{curcolor}{0 0 0}
\pscustom[linestyle=none,fillstyle=solid,fillcolor=curcolor]
{
\newpath
\moveto(169.56410858,290.35453275)
\lineto(168.50182351,290.35453275)
\lineto(168.50182351,296.55800844)
\curveto(168.50182351,297.05870826)(168.48487215,297.52689511)(168.45096943,297.96256897)
\curveto(168.41706672,298.40474544)(168.35491174,298.74938298)(168.2645045,298.99648159)
\curveto(168.17033029,299.26959058)(168.03471943,299.47117103)(167.85767192,299.60122293)
\curveto(167.68062441,299.73777743)(167.45083934,299.80605467)(167.16831671,299.80605467)
\curveto(166.87826015,299.80605467)(166.5750192,299.68250537)(166.25859386,299.43540676)
\curveto(165.94216852,299.18830814)(165.63892757,298.87293228)(165.34887101,298.48927917)
\lineto(165.34887101,290.35453275)
\lineto(164.28658594,290.35453275)
\lineto(164.28658594,301.24963078)
\lineto(165.34887101,301.24963078)
\lineto(165.34887101,300.0401481)
\curveto(165.68036422,300.51483754)(166.02315834,300.88548545)(166.37725336,301.15209185)
\curveto(166.73134839,301.41869825)(167.09486083,301.55200145)(167.4677907,301.55200145)
\curveto(168.14961197,301.55200145)(168.6694536,301.19761002)(169.02731559,300.48882716)
\curveto(169.38517758,299.78004429)(169.56410858,298.75913687)(169.56410858,297.42610488)
\closepath
}
}
{
\newrgbcolor{curcolor}{0 0 0}
\pscustom[linestyle=none,fillstyle=solid,fillcolor=curcolor]
{
\newpath
\moveto(174.88683006,290.45207168)
\curveto(174.68718074,290.36103535)(174.46869657,290.2862555)(174.23137757,290.22773215)
\curveto(173.99782553,290.16920879)(173.78875879,290.13994711)(173.60417734,290.13994711)
\curveto(172.96002575,290.13994711)(172.47031987,290.43906649)(172.13505969,291.03730523)
\curveto(171.79979951,291.63554398)(171.63216941,292.59467675)(171.63216941,293.91470355)
\lineto(171.63216941,299.70851575)
\lineto(170.91456195,299.70851575)
\lineto(170.91456195,301.24963078)
\lineto(171.63216941,301.24963078)
\lineto(171.63216941,304.3806303)
\lineto(172.69445449,304.3806303)
\lineto(172.69445449,301.24963078)
\lineto(174.88683006,301.24963078)
\lineto(174.88683006,299.70851575)
\lineto(172.69445449,299.70851575)
\lineto(172.69445449,294.74378442)
\curveto(172.69445449,294.17155605)(172.70198842,293.72287699)(172.7170563,293.39774724)
\curveto(172.73212417,293.07912008)(172.78486173,292.78000071)(172.87526897,292.50038912)
\curveto(172.95814227,292.24028532)(173.07115132,292.04845877)(173.21429612,291.92490946)
\curveto(173.36120788,291.80786275)(173.58345901,291.74933939)(173.88104951,291.74933939)
\curveto(174.05433006,291.74933939)(174.23514454,291.79160626)(174.42349295,291.87614)
\curveto(174.61184137,291.96717633)(174.74745223,292.04195617)(174.83032553,292.10047953)
\lineto(174.88683006,292.10047953)
\closepath
}
}
{
\newrgbcolor{curcolor}{0 0 0}
\pscustom[linestyle=none,fillstyle=solid,fillcolor=curcolor]
{
\newpath
\moveto(180.10784846,299.2500828)
\lineto(180.05134394,299.2500828)
\curveto(179.89313127,299.31510875)(179.73868556,299.36062691)(179.58800683,299.38663729)
\curveto(179.44109507,299.41915027)(179.26593104,299.43540676)(179.06251475,299.43540676)
\curveto(178.7347885,299.43540676)(178.41836316,299.30860615)(178.11323873,299.05500494)
\curveto(177.80811429,298.80790633)(177.51429076,298.48602788)(177.23176814,298.08936958)
\lineto(177.23176814,290.35453275)
\lineto(176.16948306,290.35453275)
\lineto(176.16948306,301.24963078)
\lineto(177.23176814,301.24963078)
\lineto(177.23176814,299.6402385)
\curveto(177.65366859,300.22547206)(178.02471497,300.63838684)(178.34490728,300.87898286)
\curveto(178.66886656,301.12608147)(178.99847629,301.24963078)(179.33373647,301.24963078)
\curveto(179.51831792,301.24963078)(179.65204529,301.23987689)(179.7349186,301.2203691)
\curveto(179.8177919,301.20736391)(179.94210185,301.17810223)(180.10784846,301.13258407)
\closepath
}
}
{
\newrgbcolor{curcolor}{0 0 0}
\pscustom[linestyle=none,fillstyle=solid,fillcolor=curcolor]
{
\newpath
\moveto(186.4872085,295.79720482)
\curveto(186.4872085,294.02199637)(186.22352071,292.62068713)(185.69614515,291.59327711)
\curveto(185.16876958,290.56586709)(184.46246302,290.05216208)(183.57722546,290.05216208)
\curveto(182.68445396,290.05216208)(181.97438043,290.56586709)(181.44700486,291.59327711)
\curveto(180.92339626,292.62068713)(180.66159196,294.02199637)(180.66159196,295.79720482)
\curveto(180.66159196,297.57241327)(180.92339626,298.97372251)(181.44700486,300.00113253)
\curveto(181.97438043,301.03504514)(182.68445396,301.55200145)(183.57722546,301.55200145)
\curveto(184.46246302,301.55200145)(185.16876958,301.03504514)(185.69614515,300.00113253)
\curveto(186.22352071,298.97372251)(186.4872085,297.57241327)(186.4872085,295.79720482)
\closepath
\moveto(185.39102071,295.79720482)
\curveto(185.39102071,297.20826795)(185.23092456,298.25518575)(184.91073225,298.93795823)
\curveto(184.59053994,299.62723331)(184.14603768,299.97187085)(183.57722546,299.97187085)
\curveto(183.0008793,299.97187085)(182.55261007,299.62723331)(182.23241776,298.93795823)
\curveto(181.91599242,298.25518575)(181.75777975,297.20826795)(181.75777975,295.79720482)
\curveto(181.75777975,294.43165986)(181.9178759,293.39449594)(182.23806821,292.68571308)
\curveto(182.55826052,291.98343281)(183.00464627,291.63229268)(183.57722546,291.63229268)
\curveto(184.14227071,291.63229268)(184.58488949,291.98018152)(184.9050818,292.67595919)
\curveto(185.22904107,293.37823946)(185.39102071,294.41865467)(185.39102071,295.79720482)
\closepath
}
}
{
\newrgbcolor{curcolor}{0 0 0}
\pscustom[linestyle=none,fillstyle=solid,fillcolor=curcolor]
{
\newpath
\moveto(189.20508054,290.35453275)
\lineto(188.14279547,290.35453275)
\lineto(188.14279547,305.53158963)
\lineto(189.20508054,305.53158963)
\closepath
}
}
{
\newrgbcolor{curcolor}{0 0 0}
\pscustom[linestyle=none,fillstyle=solid,fillcolor=curcolor]
{
\newpath
\moveto(192.38063444,290.35453275)
\lineto(191.31834937,290.35453275)
\lineto(191.31834937,305.53158963)
\lineto(192.38063444,305.53158963)
\closepath
}
}
{
\newrgbcolor{curcolor}{0 0 0}
\pscustom[linestyle=none,fillstyle=solid,fillcolor=curcolor]
{
\newpath
\moveto(199.76576998,295.61188086)
\lineto(195.11544756,295.61188086)
\curveto(195.11544756,294.94211357)(195.17383557,294.35688001)(195.29061159,293.85618019)
\curveto(195.40738761,293.36198297)(195.56748376,292.95557078)(195.77090005,292.63694362)
\curveto(195.96678241,292.32481906)(196.19845096,292.09072563)(196.46590571,291.93466335)
\curveto(196.73712743,291.77860107)(197.03471793,291.70056993)(197.35867721,291.70056993)
\curveto(197.7881116,291.70056993)(198.21942947,291.84687832)(198.65263083,292.1394951)
\curveto(199.08959916,292.43861447)(199.40037405,292.73123125)(199.5849555,293.01734543)
\lineto(199.64146002,293.01734543)
\lineto(199.64146002,291.01779745)
\curveto(199.28359803,290.75769365)(198.9182021,290.53985671)(198.54527224,290.36428664)
\curveto(198.17234237,290.18871658)(197.78057766,290.10093154)(197.36997811,290.10093154)
\curveto(196.32276092,290.10093154)(195.50532879,290.58862617)(194.91768173,291.56401543)
\curveto(194.33003467,292.54590729)(194.03621114,293.93746263)(194.03621114,295.73868146)
\curveto(194.03621114,297.52039251)(194.31685028,298.93470694)(194.87812856,299.98162474)
\curveto(195.44317381,301.02854255)(196.18526657,301.55200145)(197.10440685,301.55200145)
\curveto(197.95574169,301.55200145)(198.61119418,301.12283017)(199.07076432,300.26448763)
\curveto(199.53410142,299.40614508)(199.76576998,298.1869085)(199.76576998,296.6067779)
\closepath
\moveto(198.73173717,297.01644139)
\curveto(198.7279702,297.97882546)(198.58670889,298.7233726)(198.30795323,299.2500828)
\curveto(198.03296454,299.776793)(197.61294757,300.0401481)(197.04790232,300.0401481)
\curveto(196.4790901,300.0401481)(196.02517042,299.75078262)(195.68614327,299.17205166)
\curveto(195.35088309,298.5933207)(195.16065118,297.87478394)(195.11544756,297.01644139)
\closepath
}
}
{
\newrgbcolor{curcolor}{0 0 0}
\pscustom[linestyle=none,fillstyle=solid,fillcolor=curcolor]
{
\newpath
\moveto(205.31451846,299.2500828)
\lineto(205.25801393,299.2500828)
\curveto(205.09980126,299.31510875)(204.94535556,299.36062691)(204.79467683,299.38663729)
\curveto(204.64776506,299.41915027)(204.47260103,299.43540676)(204.26918474,299.43540676)
\curveto(203.9414585,299.43540676)(203.62503316,299.30860615)(203.31990872,299.05500494)
\curveto(203.01478429,298.80790633)(202.72096076,298.48602788)(202.43843813,298.08936958)
\lineto(202.43843813,290.35453275)
\lineto(201.37615306,290.35453275)
\lineto(201.37615306,301.24963078)
\lineto(202.43843813,301.24963078)
\lineto(202.43843813,299.6402385)
\curveto(202.86033859,300.22547206)(203.23138497,300.63838684)(203.55157728,300.87898286)
\curveto(203.87553655,301.12608147)(204.20514628,301.24963078)(204.54040646,301.24963078)
\curveto(204.72498791,301.24963078)(204.85871529,301.23987689)(204.94158859,301.2203691)
\curveto(205.0244619,301.20736391)(205.14877185,301.17810223)(205.31451846,301.13258407)
\closepath
}
}
{
\newrgbcolor{curcolor}{0 0 0}
\pscustom[linestyle=none,fillstyle=solid,fillcolor=curcolor]
{
\newpath
\moveto(244.59399222,361.27784549)
\lineto(238.50344939,361.27784549)
\lineto(238.50344939,362.8284398)
\lineto(240.52327227,362.8284398)
\lineto(240.52327227,374.91695466)
\lineto(238.50344939,374.91695466)
\lineto(238.50344939,376.46754897)
\lineto(244.59399222,376.46754897)
\lineto(244.59399222,374.91695466)
\lineto(242.57416934,374.91695466)
\lineto(242.57416934,362.8284398)
\lineto(244.59399222,362.8284398)
\closepath
}
}
{
\newrgbcolor{curcolor}{0 0 0}
\pscustom[linestyle=none,fillstyle=solid,fillcolor=curcolor]
{
\newpath
\moveto(257.60372302,361.27784549)
\lineto(255.65640661,361.27784549)
\lineto(255.65640661,367.76585853)
\curveto(255.65640661,368.28952415)(255.62533241,368.77918551)(255.56318401,369.23484261)
\curveto(255.50103562,369.69730056)(255.38709689,370.05774573)(255.22136783,370.31617812)
\curveto(255.0487334,370.60181391)(254.80013981,370.81264033)(254.47558708,370.94865738)
\curveto(254.15103434,371.09147527)(253.72980632,371.16288422)(253.21190302,371.16288422)
\curveto(252.68018896,371.16288422)(252.12430608,371.03366803)(251.54425439,370.77523564)
\curveto(250.96420269,370.51680326)(250.40831981,370.18696193)(249.87660575,369.78571164)
\lineto(249.87660575,361.27784549)
\lineto(247.92928934,361.27784549)
\lineto(247.92928934,372.67267342)
\lineto(249.87660575,372.67267342)
\lineto(249.87660575,371.4077149)
\curveto(250.48427896,371.90417712)(251.1126683,372.29182569)(251.76177377,372.57066064)
\curveto(252.41087924,372.84949558)(253.07724816,372.98891305)(253.76088052,372.98891305)
\curveto(255.01075382,372.98891305)(255.9636959,372.6182666)(256.61970675,371.87697371)
\curveto(257.2757176,371.13568081)(257.60372302,370.06794701)(257.60372302,368.6737723)
\closepath
}
}
{
\newrgbcolor{curcolor}{0 0 0}
\pscustom[linestyle=none,fillstyle=solid,fillcolor=curcolor]
{
\newpath
\moveto(267.36102247,361.37985827)
\curveto(266.99503747,361.28464634)(266.59452559,361.20643654)(266.15948681,361.14522887)
\curveto(265.73135341,361.0840212)(265.34810497,361.05341736)(265.00974148,361.05341736)
\curveto(263.82892195,361.05341736)(262.93122289,361.36625657)(262.31664431,361.99193497)
\curveto(261.70206572,362.61761338)(261.39477643,363.62073909)(261.39477643,365.00131209)
\lineto(261.39477643,371.06087144)
\lineto(260.07930204,371.06087144)
\lineto(260.07930204,372.67267342)
\lineto(261.39477643,372.67267342)
\lineto(261.39477643,375.94728377)
\lineto(263.34209285,375.94728377)
\lineto(263.34209285,372.67267342)
\lineto(267.36102247,372.67267342)
\lineto(267.36102247,371.06087144)
\lineto(263.34209285,371.06087144)
\lineto(263.34209285,365.86842075)
\curveto(263.34209285,365.26994575)(263.3559036,364.80068695)(263.38352511,364.46064434)
\curveto(263.41114662,364.12740258)(263.5078219,363.81456338)(263.67355096,363.52212673)
\curveto(263.82546926,363.25009264)(264.03263058,363.0494675)(264.29503492,362.92025131)
\curveto(264.56434464,362.79783596)(264.9717619,362.73662829)(265.51728672,362.73662829)
\curveto(265.83493408,362.73662829)(266.16639219,362.78083383)(266.51166106,362.86924491)
\curveto(266.85692993,362.96445684)(267.10552351,363.04266665)(267.25744181,363.10387432)
\lineto(267.36102247,363.10387432)
\closepath
}
}
{
\newrgbcolor{curcolor}{0 0 0}
\pscustom[linestyle=none,fillstyle=solid,fillcolor=curcolor]
{
\newpath
\moveto(279.39709362,366.77633453)
\lineto(270.87240526,366.77633453)
\curveto(270.87240526,366.07584675)(270.97943861,365.46377004)(271.19350531,364.94010442)
\curveto(271.40757201,364.42323965)(271.70105054,363.99818639)(272.07394092,363.66494463)
\curveto(272.43302054,363.33850372)(272.85770125,363.09367304)(273.34798305,362.93045258)
\curveto(273.84517022,362.76723213)(274.39069503,362.6856219)(274.98455748,362.6856219)
\curveto(275.7717705,362.6856219)(276.56243621,362.83864108)(277.35655461,363.14467943)
\curveto(278.15757838,363.45751863)(278.72727201,363.76355698)(279.0656355,364.06279448)
\lineto(279.16921616,364.06279448)
\lineto(279.16921616,361.97153242)
\curveto(278.51320531,361.69949833)(277.84338371,361.47166978)(277.15975135,361.28804677)
\curveto(276.47611899,361.10442376)(275.75795975,361.01261225)(275.00527361,361.01261225)
\curveto(273.08557871,361.01261225)(271.58711182,361.52267617)(270.50987295,362.54280401)
\curveto(269.43263408,363.56973269)(268.89401465,365.02511507)(268.89401465,366.90895115)
\curveto(268.89401465,368.77238466)(269.40846526,370.25157002)(270.43736649,371.34650723)
\curveto(271.47317309,372.44144444)(272.83353243,372.98891305)(274.51844451,372.98891305)
\curveto(276.07905979,372.98891305)(277.28059545,372.5400568)(278.12305149,371.64234431)
\curveto(278.97241291,370.74463181)(279.39709362,369.46947201)(279.39709362,367.81686492)
\closepath
\moveto(277.50156753,368.24531861)
\curveto(277.49466215,369.25184474)(277.2357105,370.03054232)(276.72471258,370.58141136)
\curveto(276.22062003,371.13228039)(275.45067045,371.4077149)(274.41486385,371.4077149)
\curveto(273.37215187,371.4077149)(272.54005389,371.10507698)(271.91856993,370.49980113)
\curveto(271.30399135,369.89452528)(270.95526979,369.14303111)(270.87240526,368.24531861)
\closepath
}
}
{
\newrgbcolor{curcolor}{0 0 0}
\pscustom[linestyle=none,fillstyle=solid,fillcolor=curcolor]
{
\newpath
\moveto(289.5687149,370.58141136)
\lineto(289.46513424,370.58141136)
\curveto(289.17510839,370.64941988)(288.89198791,370.69702584)(288.61577282,370.72422925)
\curveto(288.3464631,370.75823351)(288.02536306,370.77523564)(287.65247268,370.77523564)
\curveto(287.05170485,370.77523564)(286.47165315,370.64261903)(285.91231758,370.37738579)
\curveto(285.35298202,370.1189534)(284.81436258,369.78231122)(284.29645928,369.36745923)
\lineto(284.29645928,361.27784549)
\lineto(282.34914286,361.27784549)
\lineto(282.34914286,372.67267342)
\lineto(284.29645928,372.67267342)
\lineto(284.29645928,370.98946249)
\curveto(285.06986154,371.60153919)(285.75004121,372.03339331)(286.33699829,372.28502484)
\curveto(286.93086074,372.54345723)(287.53508126,372.67267342)(288.14965985,372.67267342)
\curveto(288.48802334,372.67267342)(288.73316424,372.66247214)(288.88508254,372.64206958)
\curveto(289.03700084,372.62846788)(289.26487829,372.59786405)(289.5687149,372.55025808)
\closepath
}
}
{
\newrgbcolor{curcolor}{0 0 0}
\pscustom[linestyle=none,fillstyle=solid,fillcolor=curcolor]
{
\newpath
\moveto(299.95785722,361.99193497)
\curveto(299.30875175,361.68589662)(298.69072047,361.44786679)(298.1037634,361.27784549)
\curveto(297.5237117,361.10782418)(296.90568042,361.02281353)(296.24966957,361.02281353)
\curveto(295.41411891,361.02281353)(294.64762202,361.14182844)(293.95017891,361.37985827)
\curveto(293.2527358,361.62468895)(292.65542065,361.99193497)(292.15823348,362.48159634)
\curveto(291.65414094,362.9712577)(291.26398712,363.59013525)(290.98777202,364.338229)
\curveto(290.71155693,365.08632274)(290.57344938,365.96023226)(290.57344938,366.95995754)
\curveto(290.57344938,368.82339105)(291.09135268,370.28557428)(292.12715929,371.34650723)
\curveto(293.16987127,372.40744018)(294.54404136,372.93790666)(296.24966957,372.93790666)
\curveto(296.9125858,372.93790666)(297.56169127,372.84609515)(298.19698599,372.66247214)
\curveto(298.83918608,372.47884913)(299.42614316,372.25442101)(299.95785722,371.98918777)
\lineto(299.95785722,369.85712059)
\lineto(299.85427656,369.85712059)
\curveto(299.2604141,370.31277769)(298.64583552,370.66302158)(298.0105408,370.90785226)
\curveto(297.38215146,371.15268294)(296.76757288,371.27509828)(296.16680504,371.27509828)
\curveto(295.06194467,371.27509828)(294.18841443,370.90785226)(293.54621433,370.17336022)
\curveto(292.91091962,369.44566903)(292.59327226,368.3745348)(292.59327226,366.95995754)
\curveto(292.59327226,365.58618538)(292.90401424,364.52865286)(293.5254982,363.78735997)
\curveto(294.15388754,363.05286792)(295.03432316,362.6856219)(296.16680504,362.6856219)
\curveto(296.56041155,362.6856219)(296.96092344,362.73662829)(297.36834071,362.83864108)
\curveto(297.77575797,362.94065386)(298.14174297,363.07327048)(298.46629571,363.23649093)
\curveto(298.74941618,363.37930883)(299.01527321,363.52892758)(299.26386679,363.68534718)
\curveto(299.51246038,363.84856764)(299.70926363,363.98798511)(299.85427656,364.1035996)
\lineto(299.95785722,364.1035996)
\closepath
}
}
{
\newrgbcolor{curcolor}{0 0 0}
\pscustom[linestyle=none,fillstyle=solid,fillcolor=curcolor]
{
\newpath
\moveto(312.31502832,366.97015882)
\curveto(312.31502832,365.11352615)(311.8316519,363.6479425)(310.86489907,362.57340784)
\curveto(309.89814624,361.49887319)(308.60338799,360.96160586)(306.98062431,360.96160586)
\curveto(305.34404987,360.96160586)(304.04238624,361.49887319)(303.07563341,362.57340784)
\curveto(302.11578595,363.6479425)(301.63586223,365.11352615)(301.63586223,366.97015882)
\curveto(301.63586223,368.82679148)(302.11578595,370.29237514)(303.07563341,371.36690979)
\curveto(304.04238624,372.4482453)(305.34404987,372.98891305)(306.98062431,372.98891305)
\curveto(308.60338799,372.98891305)(309.89814624,372.4482453)(310.86489907,371.36690979)
\curveto(311.8316519,370.29237514)(312.31502832,368.82679148)(312.31502832,366.97015882)
\closepath
\moveto(310.30556351,366.97015882)
\curveto(310.30556351,368.44594375)(310.01208497,369.54088096)(309.42512789,370.25497045)
\curveto(308.83817082,370.97586079)(308.02333629,371.33630595)(306.98062431,371.33630595)
\curveto(305.92410157,371.33630595)(305.10236166,370.97586079)(304.51540459,370.25497045)
\curveto(303.93535289,369.54088096)(303.64532704,368.44594375)(303.64532704,366.97015882)
\curveto(303.64532704,365.54197984)(303.93880558,364.45724391)(304.52576265,363.71595102)
\curveto(305.11271973,362.98145898)(305.93100695,362.61421295)(306.98062431,362.61421295)
\curveto(308.01643091,362.61421295)(308.82781275,362.97805855)(309.41476983,363.70574974)
\curveto(310.00863228,364.44024178)(310.30556351,365.52837814)(310.30556351,366.97015882)
\closepath
}
}
{
\newrgbcolor{curcolor}{0 0 0}
\pscustom[linestyle=none,fillstyle=solid,fillcolor=curcolor]
{
\newpath
\moveto(325.00365636,361.27784549)
\lineto(323.05633994,361.27784549)
\lineto(323.05633994,367.76585853)
\curveto(323.05633994,368.28952415)(323.02526574,368.77918551)(322.96311735,369.23484261)
\curveto(322.90096895,369.69730056)(322.78703022,370.05774573)(322.62130117,370.31617812)
\curveto(322.44866673,370.60181391)(322.20007315,370.81264033)(321.87552041,370.94865738)
\curveto(321.55096768,371.09147527)(321.12973966,371.16288422)(320.61183635,371.16288422)
\curveto(320.0801223,371.16288422)(319.52423942,371.03366803)(318.94418772,370.77523564)
\curveto(318.36413602,370.51680326)(317.80825314,370.18696193)(317.27653909,369.78571164)
\lineto(317.27653909,361.27784549)
\lineto(315.32922267,361.27784549)
\lineto(315.32922267,372.67267342)
\lineto(317.27653909,372.67267342)
\lineto(317.27653909,371.4077149)
\curveto(317.8842123,371.90417712)(318.51260164,372.29182569)(319.16170711,372.57066064)
\curveto(319.81081258,372.84949558)(320.4771815,372.98891305)(321.16081385,372.98891305)
\curveto(322.41068716,372.98891305)(323.36362923,372.6182666)(324.01964008,371.87697371)
\curveto(324.67565093,371.13568081)(325.00365636,370.06794701)(325.00365636,368.6737723)
\closepath
}
}
{
\newrgbcolor{curcolor}{0 0 0}
\pscustom[linestyle=none,fillstyle=solid,fillcolor=curcolor]
{
\newpath
\moveto(338.42771119,361.27784549)
\lineto(336.48039477,361.27784549)
\lineto(336.48039477,367.76585853)
\curveto(336.48039477,368.28952415)(336.44932057,368.77918551)(336.38717218,369.23484261)
\curveto(336.32502378,369.69730056)(336.21108505,370.05774573)(336.045356,370.31617812)
\curveto(335.87272156,370.60181391)(335.62412798,370.81264033)(335.29957524,370.94865738)
\curveto(334.97502251,371.09147527)(334.55379449,371.16288422)(334.03589118,371.16288422)
\curveto(333.50417713,371.16288422)(332.94829425,371.03366803)(332.36824255,370.77523564)
\curveto(331.78819085,370.51680326)(331.23230797,370.18696193)(330.70059392,369.78571164)
\lineto(330.70059392,361.27784549)
\lineto(328.7532775,361.27784549)
\lineto(328.7532775,372.67267342)
\lineto(330.70059392,372.67267342)
\lineto(330.70059392,371.4077149)
\curveto(331.30826713,371.90417712)(331.93665647,372.29182569)(332.58576194,372.57066064)
\curveto(333.23486741,372.84949558)(333.90123633,372.98891305)(334.58486869,372.98891305)
\curveto(335.83474199,372.98891305)(336.78768406,372.6182666)(337.44369491,371.87697371)
\curveto(338.09970576,371.13568081)(338.42771119,370.06794701)(338.42771119,368.6737723)
\closepath
}
}
{
\newrgbcolor{curcolor}{0 0 0}
\pscustom[linestyle=none,fillstyle=solid,fillcolor=curcolor]
{
\newpath
\moveto(351.86212408,366.77633453)
\lineto(343.33743573,366.77633453)
\curveto(343.33743573,366.07584675)(343.44446908,365.46377004)(343.65853578,364.94010442)
\curveto(343.87260247,364.42323965)(344.16608101,363.99818639)(344.53897139,363.66494463)
\curveto(344.89805101,363.33850372)(345.32273172,363.09367304)(345.81301351,362.93045258)
\curveto(346.31020068,362.76723213)(346.8557255,362.6856219)(347.44958795,362.6856219)
\curveto(348.23680097,362.6856219)(349.02746668,362.83864108)(349.82158507,363.14467943)
\curveto(350.62260885,363.45751863)(351.19230248,363.76355698)(351.53066597,364.06279448)
\lineto(351.63424663,364.06279448)
\lineto(351.63424663,361.97153242)
\curveto(350.97823578,361.69949833)(350.30841418,361.47166978)(349.62478182,361.28804677)
\curveto(348.94114946,361.10442376)(348.22299021,361.01261225)(347.47030408,361.01261225)
\curveto(345.55060917,361.01261225)(344.05214229,361.52267617)(342.97490342,362.54280401)
\curveto(341.89766455,363.56973269)(341.35904511,365.02511507)(341.35904511,366.90895115)
\curveto(341.35904511,368.77238466)(341.87349573,370.25157002)(342.90239695,371.34650723)
\curveto(343.93820356,372.44144444)(345.2985629,372.98891305)(346.98347498,372.98891305)
\curveto(348.54409026,372.98891305)(349.74562592,372.5400568)(350.58808196,371.64234431)
\curveto(351.43744338,370.74463181)(351.86212408,369.46947201)(351.86212408,367.81686492)
\closepath
\moveto(349.966598,368.24531861)
\curveto(349.95969262,369.25184474)(349.70074097,370.03054232)(349.18974305,370.58141136)
\curveto(348.6856505,371.13228039)(347.91570092,371.4077149)(346.87989432,371.4077149)
\curveto(345.83718233,371.4077149)(345.00508436,371.10507698)(344.3836004,370.49980113)
\curveto(343.76902181,369.89452528)(343.42030026,369.14303111)(343.33743573,368.24531861)
\closepath
}
}
{
\newrgbcolor{curcolor}{0 0 0}
\pscustom[linestyle=none,fillstyle=solid,fillcolor=curcolor]
{
\newpath
\moveto(363.36993588,361.99193497)
\curveto(362.72083041,361.68589662)(362.10279914,361.44786679)(361.51584206,361.27784549)
\curveto(360.93579036,361.10782418)(360.31775909,361.02281353)(359.66174824,361.02281353)
\curveto(358.82619758,361.02281353)(358.05970069,361.14182844)(357.36225758,361.37985827)
\curveto(356.66481446,361.62468895)(356.06749932,361.99193497)(355.57031215,362.48159634)
\curveto(355.0662196,362.9712577)(354.67606578,363.59013525)(354.39985069,364.338229)
\curveto(354.12363559,365.08632274)(353.98552805,365.96023226)(353.98552805,366.95995754)
\curveto(353.98552805,368.82339105)(354.50343135,370.28557428)(355.53923795,371.34650723)
\curveto(356.58194994,372.40744018)(357.95612003,372.93790666)(359.66174824,372.93790666)
\curveto(360.32466447,372.93790666)(360.97376994,372.84609515)(361.60906466,372.66247214)
\curveto(362.25126475,372.47884913)(362.83822183,372.25442101)(363.36993588,371.98918777)
\lineto(363.36993588,369.85712059)
\lineto(363.26635522,369.85712059)
\curveto(362.67249277,370.31277769)(362.05791419,370.66302158)(361.42261947,370.90785226)
\curveto(360.79423013,371.15268294)(360.17965154,371.27509828)(359.57888371,371.27509828)
\curveto(358.47402333,371.27509828)(357.6004931,370.90785226)(356.958293,370.17336022)
\curveto(356.32299828,369.44566903)(356.00535093,368.3745348)(356.00535093,366.95995754)
\curveto(356.00535093,365.58618538)(356.31609291,364.52865286)(356.93757687,363.78735997)
\curveto(357.56596621,363.05286792)(358.44640182,362.6856219)(359.57888371,362.6856219)
\curveto(359.97249022,362.6856219)(360.37300211,362.73662829)(360.78041937,362.83864108)
\curveto(361.18783664,362.94065386)(361.55382164,363.07327048)(361.87837437,363.23649093)
\curveto(362.16149485,363.37930883)(362.42735187,363.52892758)(362.67594546,363.68534718)
\curveto(362.92453904,363.84856764)(363.1213423,363.98798511)(363.26635522,364.1035996)
\lineto(363.36993588,364.1035996)
\closepath
}
}
{
\newrgbcolor{curcolor}{0 0 0}
\pscustom[linestyle=none,fillstyle=solid,fillcolor=curcolor]
{
\newpath
\moveto(371.87390642,361.37985827)
\curveto(371.50792142,361.28464634)(371.10740953,361.20643654)(370.67237076,361.14522887)
\curveto(370.24423736,361.0840212)(369.86098892,361.05341736)(369.52262543,361.05341736)
\curveto(368.3418059,361.05341736)(367.44410684,361.36625657)(366.82952825,361.99193497)
\curveto(366.21494967,362.61761338)(365.90766038,363.62073909)(365.90766038,365.00131209)
\lineto(365.90766038,371.06087144)
\lineto(364.59218599,371.06087144)
\lineto(364.59218599,372.67267342)
\lineto(365.90766038,372.67267342)
\lineto(365.90766038,375.94728377)
\lineto(367.85497679,375.94728377)
\lineto(367.85497679,372.67267342)
\lineto(371.87390642,372.67267342)
\lineto(371.87390642,371.06087144)
\lineto(367.85497679,371.06087144)
\lineto(367.85497679,365.86842075)
\curveto(367.85497679,365.26994575)(367.86878755,364.80068695)(367.89640906,364.46064434)
\curveto(367.92403057,364.12740258)(368.02070585,363.81456338)(368.18643491,363.52212673)
\curveto(368.33835321,363.25009264)(368.54551453,363.0494675)(368.80791887,362.92025131)
\curveto(369.07722859,362.79783596)(369.48464585,362.73662829)(370.03017066,362.73662829)
\curveto(370.34781802,362.73662829)(370.67927613,362.78083383)(371.024545,362.86924491)
\curveto(371.36981387,362.96445684)(371.61840746,363.04266665)(371.77032576,363.10387432)
\lineto(371.87390642,363.10387432)
\closepath
}
}
{
\newrgbcolor{curcolor}{0.7019608 0.7019608 0.7019608}
\pscustom[linestyle=none,fillstyle=solid,fillcolor=curcolor]
{
\newpath
\moveto(460.41379276,282.0572565)
\lineto(460.41379276,189.12317146)
}
}
{
\newrgbcolor{curcolor}{0 0 0}
\pscustom[linewidth=2.64566925,linecolor=curcolor]
{
\newpath
\moveto(460.41379276,282.0572565)
\lineto(460.41379276,189.12317146)
}
}
{
\newrgbcolor{curcolor}{0.80000001 0.80000001 0.80000001}
\pscustom[linestyle=none,fillstyle=solid,fillcolor=curcolor]
{
\newpath
\moveto(395.94453163,313.85292239)
\lineto(524.88303414,313.85292239)
\lineto(524.88303414,278.01471986)
\lineto(395.94453163,278.01471986)
\closepath
}
}
{
\newrgbcolor{curcolor}{0 0 0}
\pscustom[linewidth=0.27058431,linecolor=curcolor]
{
\newpath
\moveto(395.94453163,313.85292239)
\lineto(524.88303414,313.85292239)
\lineto(524.88303414,278.01471986)
\lineto(395.94453163,278.01471986)
\closepath
}
}
{
\newrgbcolor{curcolor}{0 0 0}
\pscustom[linestyle=none,fillstyle=solid,fillcolor=curcolor]
{
\newpath
\moveto(414.4925534,290.35453275)
\lineto(413.3737638,290.35453275)
\lineto(413.3737638,302.86877695)
\lineto(411.03447646,294.35362872)
\lineto(410.36772307,294.35362872)
\lineto(408.04538709,302.86877695)
\lineto(408.04538709,290.35453275)
\lineto(407.00005337,290.35453275)
\lineto(407.00005337,304.87807882)
\lineto(408.52567555,304.87807882)
\lineto(410.7689052,296.79210186)
\lineto(412.93867896,304.87807882)
\lineto(414.4925534,304.87807882)
\closepath
}
}
{
\newrgbcolor{curcolor}{0 0 0}
\pscustom[linestyle=none,fillstyle=solid,fillcolor=curcolor]
{
\newpath
\moveto(421.9511504,295.61188086)
\lineto(417.30082799,295.61188086)
\curveto(417.30082799,294.94211357)(417.359216,294.35688001)(417.47599202,293.85618019)
\curveto(417.59276804,293.36198297)(417.75286419,292.95557078)(417.95628048,292.63694362)
\curveto(418.15216283,292.32481906)(418.38383139,292.09072563)(418.65128614,291.93466335)
\curveto(418.92250786,291.77860107)(419.22009836,291.70056993)(419.54405763,291.70056993)
\curveto(419.97349203,291.70056993)(420.4048099,291.84687832)(420.83801126,292.1394951)
\curveto(421.27497959,292.43861447)(421.58575447,292.73123125)(421.77033592,293.01734543)
\lineto(421.82684045,293.01734543)
\lineto(421.82684045,291.01779745)
\curveto(421.46897845,290.75769365)(421.10358253,290.53985671)(420.73065266,290.36428664)
\curveto(420.3577228,290.18871658)(419.96595809,290.10093154)(419.55535854,290.10093154)
\curveto(418.50814134,290.10093154)(417.69070921,290.58862617)(417.10306215,291.56401543)
\curveto(416.51541509,292.54590729)(416.22159156,293.93746263)(416.22159156,295.73868146)
\curveto(416.22159156,297.52039251)(416.5022307,298.93470694)(417.06350898,299.98162474)
\curveto(417.62855424,301.02854255)(418.370647,301.55200145)(419.28978727,301.55200145)
\curveto(420.14112212,301.55200145)(420.79657461,301.12283017)(421.25614474,300.26448763)
\curveto(421.71948185,299.40614508)(421.9511504,298.1869085)(421.9511504,296.6067779)
\closepath
\moveto(420.91711759,297.01644139)
\curveto(420.91335063,297.97882546)(420.77208931,298.7233726)(420.49333366,299.2500828)
\curveto(420.21834497,299.776793)(419.798328,300.0401481)(419.23328275,300.0401481)
\curveto(418.66447053,300.0401481)(418.21055084,299.75078262)(417.87152369,299.17205166)
\curveto(417.53626351,298.5933207)(417.34603161,297.87478394)(417.30082799,297.01644139)
\closepath
}
}
{
\newrgbcolor{curcolor}{0 0 0}
\pscustom[linestyle=none,fillstyle=solid,fillcolor=curcolor]
{
\newpath
\moveto(432.77176764,290.35453275)
\lineto(431.70948257,290.35453275)
\lineto(431.70948257,296.55800844)
\curveto(431.70948257,297.02619529)(431.69629818,297.47812564)(431.6699294,297.91379951)
\curveto(431.64732759,298.34947338)(431.59647352,298.69736222)(431.51736719,298.95746602)
\curveto(431.43072691,299.23707761)(431.30641696,299.44841195)(431.14443732,299.59146904)
\curveto(430.98245768,299.73452613)(430.74890564,299.80605467)(430.44378121,299.80605467)
\curveto(430.14619071,299.80605467)(429.84860021,299.67600277)(429.55100971,299.41589897)
\curveto(429.25341922,299.16229776)(428.95582872,298.83716801)(428.65823822,298.44050971)
\curveto(428.66953912,298.29095003)(428.67895654,298.11537996)(428.68649048,297.91379951)
\curveto(428.69402442,297.71872166)(428.69779139,297.52364381)(428.69779139,297.32856596)
\lineto(428.69779139,290.35453275)
\lineto(427.63550631,290.35453275)
\lineto(427.63550631,296.55800844)
\curveto(427.63550631,297.03920048)(427.62232193,297.49438213)(427.59595315,297.9235534)
\curveto(427.57335134,298.35922727)(427.52249726,298.70711611)(427.44339093,298.96721991)
\curveto(427.35675066,299.2468315)(427.2324407,299.45491454)(427.07046106,299.59146904)
\curveto(426.90848143,299.73452613)(426.67492939,299.80605467)(426.36980495,299.80605467)
\curveto(426.07974839,299.80605467)(425.78780834,299.68250537)(425.49398481,299.43540676)
\curveto(425.20392825,299.18830814)(424.91387169,298.87293228)(424.62381513,298.48927917)
\lineto(424.62381513,290.35453275)
\lineto(423.56153006,290.35453275)
\lineto(423.56153006,301.24963078)
\lineto(424.62381513,301.24963078)
\lineto(424.62381513,300.0401481)
\curveto(424.95530834,300.51483754)(425.28491807,300.88548545)(425.61264432,301.15209185)
\curveto(425.94413753,301.41869825)(426.29634907,301.55200145)(426.66927894,301.55200145)
\curveto(427.09871333,301.55200145)(427.46222577,301.39593917)(427.75981627,301.0838146)
\curveto(428.06117374,300.77169004)(428.28530835,300.33926747)(428.43222012,299.78654689)
\curveto(428.86165451,300.41079602)(429.25341922,300.85947507)(429.60751424,301.13258407)
\curveto(429.96160926,301.41219565)(430.34018958,301.55200145)(430.74325519,301.55200145)
\curveto(431.43637737,301.55200145)(431.94680158,301.18785613)(432.27452782,300.45956548)
\curveto(432.60602104,299.73777743)(432.77176764,298.72662389)(432.77176764,297.42610488)
\closepath
}
}
{
\newrgbcolor{curcolor}{0 0 0}
\pscustom[linestyle=none,fillstyle=solid,fillcolor=curcolor]
{
\newpath
\moveto(440.19646302,295.79720482)
\curveto(440.19646302,294.02199637)(439.93277523,292.62068713)(439.40539967,291.59327711)
\curveto(438.8780241,290.56586709)(438.17171754,290.05216208)(437.28647998,290.05216208)
\curveto(436.39370848,290.05216208)(435.68363495,290.56586709)(435.15625938,291.59327711)
\curveto(434.63265078,292.62068713)(434.37084648,294.02199637)(434.37084648,295.79720482)
\curveto(434.37084648,297.57241327)(434.63265078,298.97372251)(435.15625938,300.00113253)
\curveto(435.68363495,301.03504514)(436.39370848,301.55200145)(437.28647998,301.55200145)
\curveto(438.17171754,301.55200145)(438.8780241,301.03504514)(439.40539967,300.00113253)
\curveto(439.93277523,298.97372251)(440.19646302,297.57241327)(440.19646302,295.79720482)
\closepath
\moveto(439.10027523,295.79720482)
\curveto(439.10027523,297.20826795)(438.94017908,298.25518575)(438.61998677,298.93795823)
\curveto(438.29979446,299.62723331)(437.8552922,299.97187085)(437.28647998,299.97187085)
\curveto(436.71013382,299.97187085)(436.26186459,299.62723331)(435.94167228,298.93795823)
\curveto(435.62524694,298.25518575)(435.46703427,297.20826795)(435.46703427,295.79720482)
\curveto(435.46703427,294.43165986)(435.62713042,293.39449594)(435.94732273,292.68571308)
\curveto(436.26751504,291.98343281)(436.71390079,291.63229268)(437.28647998,291.63229268)
\curveto(437.85152523,291.63229268)(438.29414401,291.98018152)(438.61433632,292.67595919)
\curveto(438.93829559,293.37823946)(439.10027523,294.41865467)(439.10027523,295.79720482)
\closepath
}
}
{
\newrgbcolor{curcolor}{0 0 0}
\pscustom[linestyle=none,fillstyle=solid,fillcolor=curcolor]
{
\newpath
\moveto(445.77910899,299.2500828)
\lineto(445.72260447,299.2500828)
\curveto(445.5643918,299.31510875)(445.4099461,299.36062691)(445.25926736,299.38663729)
\curveto(445.1123556,299.41915027)(444.93719157,299.43540676)(444.73377528,299.43540676)
\curveto(444.40604903,299.43540676)(444.08962369,299.30860615)(443.78449926,299.05500494)
\curveto(443.47937482,298.80790633)(443.18555129,298.48602788)(442.90302867,298.08936958)
\lineto(442.90302867,290.35453275)
\lineto(441.8407436,290.35453275)
\lineto(441.8407436,301.24963078)
\lineto(442.90302867,301.24963078)
\lineto(442.90302867,299.6402385)
\curveto(443.32492912,300.22547206)(443.6959755,300.63838684)(444.01616781,300.87898286)
\curveto(444.34012709,301.12608147)(444.66973682,301.24963078)(445.004997,301.24963078)
\curveto(445.18957845,301.24963078)(445.32330582,301.23987689)(445.40617913,301.2203691)
\curveto(445.48905243,301.20736391)(445.61336239,301.17810223)(445.77910899,301.13258407)
\closepath
}
}
{
\newrgbcolor{curcolor}{0 0 0}
\pscustom[linestyle=none,fillstyle=solid,fillcolor=curcolor]
{
\newpath
\moveto(452.23757536,301.24963078)
\lineto(448.55348033,286.335929)
\lineto(447.41773938,286.335929)
\lineto(448.5930335,290.88124295)
\lineto(446.07858213,301.24963078)
\lineto(447.23127444,301.24963078)
\lineto(449.16937965,293.17340771)
\lineto(451.12443622,301.24963078)
\closepath
}
}
{
\newrgbcolor{curcolor}{0 0 0}
\pscustom[linestyle=none,fillstyle=solid,fillcolor=curcolor]
{
\newpath
\moveto(464.2786901,291.40795315)
\curveto(464.07150684,291.25189087)(463.88315842,291.10558248)(463.71364485,290.96902799)
\curveto(463.54789824,290.83247349)(463.32941408,290.6894164)(463.05819236,290.53985671)
\curveto(462.82840729,290.4163074)(462.57790389,290.31226588)(462.30668217,290.22773215)
\curveto(462.03922742,290.13669582)(461.74352041,290.09117765)(461.41956113,290.09117765)
\curveto(460.80931226,290.09117765)(460.25368443,290.23748604)(459.75267764,290.53010282)
\curveto(459.25543782,290.82922219)(458.82223646,291.29415774)(458.45307356,291.92490946)
\curveto(458.0914446,292.54265599)(457.80892198,293.3262187)(457.60550569,294.27559757)
\curveto(457.4020894,295.23147905)(457.30038125,296.34017151)(457.30038125,297.60167495)
\curveto(457.30038125,298.79815244)(457.39832243,299.86782933)(457.59420478,300.81070561)
\curveto(457.79008714,301.7535819)(458.07260976,302.55014979)(458.44177266,303.2004093)
\curveto(458.79963465,303.83116102)(459.23095253,304.31235305)(459.73572628,304.6439854)
\curveto(460.24426701,304.97561775)(460.80742878,305.14143392)(461.42521158,305.14143392)
\curveto(461.87724778,305.14143392)(462.3274005,305.04714629)(462.77566973,304.85857104)
\curveto(463.22770593,304.66999578)(463.72871272,304.33836343)(464.2786901,303.86367399)
\lineto(464.2786901,301.57150923)
\lineto(464.19393331,301.57150923)
\curveto(463.7305962,302.24127653)(463.27102607,302.72897115)(462.8152229,303.03459312)
\curveto(462.35941973,303.34021509)(461.87159733,303.49302607)(461.3517557,303.49302607)
\curveto(460.92608828,303.49302607)(460.54185751,303.37272807)(460.19906339,303.13213205)
\curveto(459.86003624,302.89803863)(459.55679529,302.53064201)(459.28934054,302.02994219)
\curveto(459.02941972,301.54224756)(458.82600343,300.92450103)(458.67909166,300.17670259)
\curveto(458.53594687,299.43540676)(458.46437447,298.57706421)(458.46437447,297.60167495)
\curveto(458.46437447,296.58076752)(458.5434808,295.70291719)(458.70169347,294.96812395)
\curveto(458.86367311,294.23333071)(459.07085637,293.63509196)(459.32324325,293.17340771)
\curveto(459.58693103,292.69221568)(459.89393895,292.33457295)(460.24426701,292.10047953)
\curveto(460.59836203,291.8728887)(460.9712919,291.75909329)(461.3630566,291.75909329)
\curveto(461.90173308,291.75909329)(462.40650683,291.91840686)(462.87737788,292.23703402)
\curveto(463.34824892,292.55566118)(463.78898421,293.03360192)(464.19958376,293.67085623)
\lineto(464.2786901,293.67085623)
\closepath
}
}
{
\newrgbcolor{curcolor}{0 0 0}
\pscustom[linestyle=none,fillstyle=solid,fillcolor=curcolor]
{
\newpath
\moveto(471.15528924,295.79720482)
\curveto(471.15528924,294.02199637)(470.89160146,292.62068713)(470.36422589,291.59327711)
\curveto(469.83685032,290.56586709)(469.13054376,290.05216208)(468.2453062,290.05216208)
\curveto(467.3525347,290.05216208)(466.64246117,290.56586709)(466.11508561,291.59327711)
\curveto(465.59147701,292.62068713)(465.32967271,294.02199637)(465.32967271,295.79720482)
\curveto(465.32967271,297.57241327)(465.59147701,298.97372251)(466.11508561,300.00113253)
\curveto(466.64246117,301.03504514)(467.3525347,301.55200145)(468.2453062,301.55200145)
\curveto(469.13054376,301.55200145)(469.83685032,301.03504514)(470.36422589,300.00113253)
\curveto(470.89160146,298.97372251)(471.15528924,297.57241327)(471.15528924,295.79720482)
\closepath
\moveto(470.05910145,295.79720482)
\curveto(470.05910145,297.20826795)(469.8990053,298.25518575)(469.57881299,298.93795823)
\curveto(469.25862068,299.62723331)(468.81411842,299.97187085)(468.2453062,299.97187085)
\curveto(467.66896004,299.97187085)(467.22069081,299.62723331)(466.9004985,298.93795823)
\curveto(466.58407316,298.25518575)(466.42586049,297.20826795)(466.42586049,295.79720482)
\curveto(466.42586049,294.43165986)(466.58595665,293.39449594)(466.90614896,292.68571308)
\curveto(467.22634127,291.98343281)(467.67272701,291.63229268)(468.2453062,291.63229268)
\curveto(468.81035145,291.63229268)(469.25297023,291.98018152)(469.57316254,292.67595919)
\curveto(469.89712182,293.37823946)(470.05910145,294.41865467)(470.05910145,295.79720482)
\closepath
}
}
{
\newrgbcolor{curcolor}{0 0 0}
\pscustom[linestyle=none,fillstyle=solid,fillcolor=curcolor]
{
\newpath
\moveto(478.07709795,290.35453275)
\lineto(477.01481288,290.35453275)
\lineto(477.01481288,296.55800844)
\curveto(477.01481288,297.05870826)(476.99786152,297.52689511)(476.9639588,297.96256897)
\curveto(476.93005609,298.40474544)(476.86790111,298.74938298)(476.77749387,298.99648159)
\curveto(476.68331966,299.26959058)(476.5477088,299.47117103)(476.37066129,299.60122293)
\curveto(476.19361378,299.73777743)(475.96382871,299.80605467)(475.68130609,299.80605467)
\curveto(475.39124952,299.80605467)(475.08800857,299.68250537)(474.77158323,299.43540676)
\curveto(474.45515789,299.18830814)(474.15191694,298.87293228)(473.86186038,298.48927917)
\lineto(473.86186038,290.35453275)
\lineto(472.79957531,290.35453275)
\lineto(472.79957531,301.24963078)
\lineto(473.86186038,301.24963078)
\lineto(473.86186038,300.0401481)
\curveto(474.19335359,300.51483754)(474.53614771,300.88548545)(474.89024273,301.15209185)
\curveto(475.24433776,301.41869825)(475.6078502,301.55200145)(475.98078007,301.55200145)
\curveto(476.66260134,301.55200145)(477.18244297,301.19761002)(477.54030496,300.48882716)
\curveto(477.89816695,299.78004429)(478.07709795,298.75913687)(478.07709795,297.42610488)
\closepath
}
}
{
\newrgbcolor{curcolor}{0 0 0}
\pscustom[linestyle=none,fillstyle=solid,fillcolor=curcolor]
{
\newpath
\moveto(483.39981943,290.45207168)
\curveto(483.20017011,290.36103535)(482.98168594,290.2862555)(482.74436694,290.22773215)
\curveto(482.5108149,290.16920879)(482.30174816,290.13994711)(482.11716671,290.13994711)
\curveto(481.47301512,290.13994711)(480.98330924,290.43906649)(480.64804906,291.03730523)
\curveto(480.31278888,291.63554398)(480.14515879,292.59467675)(480.14515879,293.91470355)
\lineto(480.14515879,299.70851575)
\lineto(479.42755132,299.70851575)
\lineto(479.42755132,301.24963078)
\lineto(480.14515879,301.24963078)
\lineto(480.14515879,304.3806303)
\lineto(481.20744386,304.3806303)
\lineto(481.20744386,301.24963078)
\lineto(483.39981943,301.24963078)
\lineto(483.39981943,299.70851575)
\lineto(481.20744386,299.70851575)
\lineto(481.20744386,294.74378442)
\curveto(481.20744386,294.17155605)(481.21497779,293.72287699)(481.23004567,293.39774724)
\curveto(481.24511354,293.07912008)(481.2978511,292.78000071)(481.38825834,292.50038912)
\curveto(481.47113164,292.24028532)(481.58414069,292.04845877)(481.72728549,291.92490946)
\curveto(481.87419725,291.80786275)(482.09644838,291.74933939)(482.39403888,291.74933939)
\curveto(482.56731943,291.74933939)(482.74813391,291.79160626)(482.93648232,291.87614)
\curveto(483.12483074,291.96717633)(483.2604416,292.04195617)(483.3433149,292.10047953)
\lineto(483.39981943,292.10047953)
\closepath
}
}
{
\newrgbcolor{curcolor}{0 0 0}
\pscustom[linestyle=none,fillstyle=solid,fillcolor=curcolor]
{
\newpath
\moveto(488.62083783,299.2500828)
\lineto(488.56433331,299.2500828)
\curveto(488.40612064,299.31510875)(488.25167493,299.36062691)(488.1009962,299.38663729)
\curveto(487.95408444,299.41915027)(487.77892041,299.43540676)(487.57550412,299.43540676)
\curveto(487.24777787,299.43540676)(486.93135253,299.30860615)(486.6262281,299.05500494)
\curveto(486.32110366,298.80790633)(486.02728013,298.48602788)(485.74475751,298.08936958)
\lineto(485.74475751,290.35453275)
\lineto(484.68247244,290.35453275)
\lineto(484.68247244,301.24963078)
\lineto(485.74475751,301.24963078)
\lineto(485.74475751,299.6402385)
\curveto(486.16665796,300.22547206)(486.53770434,300.63838684)(486.85789665,300.87898286)
\curveto(487.18185593,301.12608147)(487.51146566,301.24963078)(487.84672584,301.24963078)
\curveto(488.03130729,301.24963078)(488.16503466,301.23987689)(488.24790797,301.2203691)
\curveto(488.33078127,301.20736391)(488.45509123,301.17810223)(488.62083783,301.13258407)
\closepath
}
}
{
\newrgbcolor{curcolor}{0 0 0}
\pscustom[linestyle=none,fillstyle=solid,fillcolor=curcolor]
{
\newpath
\moveto(495.00019787,295.79720482)
\curveto(495.00019787,294.02199637)(494.73651008,292.62068713)(494.20913452,291.59327711)
\curveto(493.68175895,290.56586709)(492.97545239,290.05216208)(492.09021483,290.05216208)
\curveto(491.19744333,290.05216208)(490.4873698,290.56586709)(489.95999423,291.59327711)
\curveto(489.43638563,292.62068713)(489.17458133,294.02199637)(489.17458133,295.79720482)
\curveto(489.17458133,297.57241327)(489.43638563,298.97372251)(489.95999423,300.00113253)
\curveto(490.4873698,301.03504514)(491.19744333,301.55200145)(492.09021483,301.55200145)
\curveto(492.97545239,301.55200145)(493.68175895,301.03504514)(494.20913452,300.00113253)
\curveto(494.73651008,298.97372251)(495.00019787,297.57241327)(495.00019787,295.79720482)
\closepath
\moveto(493.90401008,295.79720482)
\curveto(493.90401008,297.20826795)(493.74391393,298.25518575)(493.42372162,298.93795823)
\curveto(493.10352931,299.62723331)(492.65902705,299.97187085)(492.09021483,299.97187085)
\curveto(491.51386867,299.97187085)(491.06559944,299.62723331)(490.74540713,298.93795823)
\curveto(490.42898179,298.25518575)(490.27076912,297.20826795)(490.27076912,295.79720482)
\curveto(490.27076912,294.43165986)(490.43086527,293.39449594)(490.75105758,292.68571308)
\curveto(491.07124989,291.98343281)(491.51763564,291.63229268)(492.09021483,291.63229268)
\curveto(492.65526008,291.63229268)(493.09787886,291.98018152)(493.41807117,292.67595919)
\curveto(493.74203044,293.37823946)(493.90401008,294.41865467)(493.90401008,295.79720482)
\closepath
}
}
{
\newrgbcolor{curcolor}{0 0 0}
\pscustom[linestyle=none,fillstyle=solid,fillcolor=curcolor]
{
\newpath
\moveto(497.71806991,290.35453275)
\lineto(496.65578484,290.35453275)
\lineto(496.65578484,305.53158963)
\lineto(497.71806991,305.53158963)
\closepath
}
}
{
\newrgbcolor{curcolor}{0 0 0}
\pscustom[linestyle=none,fillstyle=solid,fillcolor=curcolor]
{
\newpath
\moveto(500.89362381,290.35453275)
\lineto(499.83133874,290.35453275)
\lineto(499.83133874,305.53158963)
\lineto(500.89362381,305.53158963)
\closepath
}
}
{
\newrgbcolor{curcolor}{0 0 0}
\pscustom[linestyle=none,fillstyle=solid,fillcolor=curcolor]
{
\newpath
\moveto(508.27875935,295.61188086)
\lineto(503.62843693,295.61188086)
\curveto(503.62843693,294.94211357)(503.68682494,294.35688001)(503.80360096,293.85618019)
\curveto(503.92037698,293.36198297)(504.08047314,292.95557078)(504.28388943,292.63694362)
\curveto(504.47977178,292.32481906)(504.71144033,292.09072563)(504.97889508,291.93466335)
\curveto(505.2501168,291.77860107)(505.5477073,291.70056993)(505.87166658,291.70056993)
\curveto(506.30110097,291.70056993)(506.73241884,291.84687832)(507.1656202,292.1394951)
\curveto(507.60258853,292.43861447)(507.91336342,292.73123125)(508.09794487,293.01734543)
\lineto(508.15444939,293.01734543)
\lineto(508.15444939,291.01779745)
\curveto(507.7965874,290.75769365)(507.43119147,290.53985671)(507.05826161,290.36428664)
\curveto(506.68533174,290.18871658)(506.29356703,290.10093154)(505.88296748,290.10093154)
\curveto(504.83575029,290.10093154)(504.01831816,290.58862617)(503.4306711,291.56401543)
\curveto(502.84302404,292.54590729)(502.54920051,293.93746263)(502.54920051,295.73868146)
\curveto(502.54920051,297.52039251)(502.82983965,298.93470694)(503.39111793,299.98162474)
\curveto(503.95616318,301.02854255)(504.69825594,301.55200145)(505.61739622,301.55200145)
\curveto(506.46873106,301.55200145)(507.12418355,301.12283017)(507.58375369,300.26448763)
\curveto(508.04709079,299.40614508)(508.27875935,298.1869085)(508.27875935,296.6067779)
\closepath
\moveto(507.24472654,297.01644139)
\curveto(507.24095957,297.97882546)(507.09969826,298.7233726)(506.8209426,299.2500828)
\curveto(506.54595391,299.776793)(506.12593694,300.0401481)(505.56089169,300.0401481)
\curveto(504.99207947,300.0401481)(504.53815979,299.75078262)(504.19913264,299.17205166)
\curveto(503.86387246,298.5933207)(503.67364055,297.87478394)(503.62843693,297.01644139)
\closepath
}
}
{
\newrgbcolor{curcolor}{0 0 0}
\pscustom[linestyle=none,fillstyle=solid,fillcolor=curcolor]
{
\newpath
\moveto(513.82750783,299.2500828)
\lineto(513.7710033,299.2500828)
\curveto(513.61279063,299.31510875)(513.45834493,299.36062691)(513.3076662,299.38663729)
\curveto(513.16075443,299.41915027)(512.9855904,299.43540676)(512.78217411,299.43540676)
\curveto(512.45444787,299.43540676)(512.13802253,299.30860615)(511.83289809,299.05500494)
\curveto(511.52777366,298.80790633)(511.23395013,298.48602788)(510.9514275,298.08936958)
\lineto(510.9514275,290.35453275)
\lineto(509.88914243,290.35453275)
\lineto(509.88914243,301.24963078)
\lineto(510.9514275,301.24963078)
\lineto(510.9514275,299.6402385)
\curveto(511.37332796,300.22547206)(511.74437434,300.63838684)(512.06456665,300.87898286)
\curveto(512.38852592,301.12608147)(512.71813565,301.24963078)(513.05339583,301.24963078)
\curveto(513.23797728,301.24963078)(513.37170466,301.23987689)(513.45457796,301.2203691)
\curveto(513.53745127,301.20736391)(513.66176122,301.17810223)(513.82750783,301.13258407)
\closepath
}
}
{
\newrgbcolor{curcolor}{0.80000001 0.80000001 0.80000001}
\pscustom[linestyle=none,fillstyle=solid,fillcolor=curcolor]
{
\newpath
\moveto(27.11109128,203.03956886)
\lineto(276.69050619,203.03956886)
\lineto(276.69050619,20.30070033)
\lineto(27.11109128,20.30070033)
\closepath
}
}
{
\newrgbcolor{curcolor}{0 0 0}
\pscustom[linewidth=0.85007621,linecolor=curcolor]
{
\newpath
\moveto(27.11109128,203.03956886)
\lineto(276.69050619,203.03956886)
\lineto(276.69050619,20.30070033)
\lineto(27.11109128,20.30070033)
\closepath
}
}
{
\newrgbcolor{curcolor}{0 0 0}
\pscustom[linestyle=none,fillstyle=solid,fillcolor=curcolor]
{
\newpath
\moveto(103.14612599,101.40892333)
\lineto(99.54006542,101.40892333)
\lineto(99.54006542,127.47552528)
\lineto(92.0001206,109.73885848)
\lineto(89.8510542,109.73885848)
\lineto(82.36574665,127.47552528)
\lineto(82.36574665,101.40892333)
\lineto(78.99644764,101.40892333)
\lineto(78.99644764,131.66080977)
\lineto(83.91380296,131.66080977)
\lineto(91.14413652,114.81808723)
\lineto(98.13770854,131.66080977)
\lineto(103.14612599,131.66080977)
\closepath
}
}
{
\newrgbcolor{curcolor}{0 0 0}
\pscustom[linestyle=none,fillstyle=solid,fillcolor=curcolor]
{
\newpath
\moveto(127.18652943,112.35974051)
\lineto(112.19770192,112.35974051)
\curveto(112.19770192,110.96464568)(112.385897,109.74563078)(112.76228716,108.70269581)
\curveto(113.13867732,107.67330545)(113.65469609,106.82676733)(114.31034346,106.16308144)
\curveto(114.9417076,105.51294016)(115.68841711,105.0253342)(116.550472,104.70026356)
\curveto(117.4246685,104.37519292)(118.38385633,104.2126576)(119.42803548,104.2126576)
\curveto(120.81217994,104.2126576)(122.20239521,104.51741132)(123.59868129,105.12691877)
\curveto(125.00710899,105.74997083)(126.00879248,106.35947828)(126.60373177,106.95544112)
\lineto(126.78585604,106.95544112)
\lineto(126.78585604,102.79047355)
\curveto(125.63240232,102.24868915)(124.45466537,101.79494471)(123.25264518,101.42924024)
\curveto(122.05062499,101.06353577)(120.78789671,100.88068354)(119.46446034,100.88068354)
\curveto(116.08909051,100.88068354)(113.45435939,101.89652929)(111.56026697,103.92822079)
\curveto(109.66617455,105.9734569)(108.71912834,108.87200344)(108.71912834,112.62386041)
\curveto(108.71912834,116.33508355)(109.62367888,119.28103623)(111.43277998,121.46171844)
\curveto(113.25402269,123.64240065)(115.64592145,124.73274175)(118.60847626,124.73274175)
\curveto(121.35248195,124.73274175)(123.46512349,123.83879749)(124.9464009,122.05090897)
\curveto(126.43981992,120.26302045)(127.18652943,117.72340608)(127.18652943,114.43206584)
\closepath
\moveto(123.85365527,115.28537627)
\curveto(123.84151365,117.28997856)(123.38620297,118.8408364)(122.48772324,119.93794981)
\curveto(121.60138512,121.03506322)(120.2475947,121.58361993)(118.42635199,121.58361993)
\curveto(116.59296766,121.58361993)(115.12990268,120.98088478)(114.03715706,119.77541449)
\curveto(112.95655305,118.5699442)(112.34340133,117.0732648)(112.19770192,115.28537627)
\closepath
}
}
{
\newrgbcolor{curcolor}{0 0 0}
\pscustom[linestyle=none,fillstyle=solid,fillcolor=curcolor]
{
\newpath
\moveto(162.06332743,101.40892333)
\lineto(158.63939113,101.40892333)
\lineto(158.63939113,114.33048127)
\curveto(158.63939113,115.30569319)(158.59689547,116.24704358)(158.51190414,117.15453246)
\curveto(158.43905443,118.06202133)(158.27514259,118.78665796)(158.02016861,119.32844236)
\curveto(157.74091139,119.91086059)(157.34023799,120.35106042)(156.81814842,120.64904184)
\curveto(156.29605884,120.94702326)(155.54327852,121.09601397)(154.55980745,121.09601397)
\curveto(153.60061963,121.09601397)(152.6414318,120.82512177)(151.68224397,120.28333737)
\curveto(150.72305614,119.75509758)(149.76386831,119.07786708)(148.80468049,118.25164587)
\curveto(148.84110534,117.94011984)(148.87145939,117.57441537)(148.89574262,117.15453246)
\curveto(148.92002586,116.74819415)(148.93216748,116.34185585)(148.93216748,115.93551755)
\lineto(148.93216748,101.40892333)
\lineto(145.50823118,101.40892333)
\lineto(145.50823118,114.33048127)
\curveto(145.50823118,115.33278241)(145.46573552,116.28090511)(145.38074419,117.17484937)
\curveto(145.30789448,118.08233824)(145.14398264,118.80697488)(144.88900866,119.34875928)
\curveto(144.60975144,119.93117751)(144.20907804,120.36460503)(143.68698847,120.64904184)
\curveto(143.16489889,120.94702326)(142.41211857,121.09601397)(141.4286475,121.09601397)
\curveto(140.49374291,121.09601397)(139.55276751,120.83866638)(138.6057213,120.3239712)
\curveto(137.67081671,119.80927602)(136.73591212,119.15236243)(135.80100753,118.35323044)
\lineto(135.80100753,101.40892333)
\lineto(132.37707123,101.40892333)
\lineto(132.37707123,124.10291739)
\lineto(135.80100753,124.10291739)
\lineto(135.80100753,121.58361993)
\curveto(136.86946992,122.57237646)(137.9318615,123.34441923)(138.98818227,123.89974824)
\curveto(140.05664466,124.45507725)(141.19188595,124.73274175)(142.39390614,124.73274175)
\curveto(143.7780506,124.73274175)(144.94971675,124.40767111)(145.90890457,123.75752983)
\curveto(146.88023402,123.10738855)(147.6026603,122.20667199)(148.0761834,121.05538014)
\curveto(149.46032786,122.3556627)(150.72305614,123.29024079)(151.86436824,123.85911441)
\curveto(153.00568034,124.44153264)(154.22591296,124.73274175)(155.52506609,124.73274175)
\curveto(157.75912382,124.73274175)(159.40431307,123.97424359)(160.46063384,122.45724727)
\curveto(161.52909623,120.95379556)(162.06332743,118.84760871)(162.06332743,116.1386867)
\closepath
}
}
{
\newrgbcolor{curcolor}{0 0 0}
\pscustom[linestyle=none,fillstyle=solid,fillcolor=curcolor]
{
\newpath
\moveto(185.99445615,112.7457619)
\curveto(185.99445615,109.04808337)(185.14454289,106.12921991)(183.44471635,103.98917153)
\curveto(181.74488982,101.84912315)(179.46833643,100.77909896)(176.61505619,100.77909896)
\curveto(173.7374927,100.77909896)(171.44879769,101.84912315)(169.74897116,103.98917153)
\curveto(168.06128625,106.12921991)(167.21744379,109.04808337)(167.21744379,112.7457619)
\curveto(167.21744379,116.44344043)(168.06128625,119.36230389)(169.74897116,121.50235227)
\curveto(171.44879769,123.65594526)(173.7374927,124.73274175)(176.61505619,124.73274175)
\curveto(179.46833643,124.73274175)(181.74488982,123.65594526)(183.44471635,121.50235227)
\curveto(185.14454289,119.36230389)(185.99445615,116.44344043)(185.99445615,112.7457619)
\closepath
\moveto(182.46124529,112.7457619)
\curveto(182.46124529,115.68494227)(181.94522652,117.86562448)(180.91318899,119.28780853)
\curveto(179.88115145,120.72353719)(178.44844052,121.44140152)(176.61505619,121.44140152)
\curveto(174.75738862,121.44140152)(173.31253607,120.72353719)(172.28049853,119.28780853)
\curveto(171.26060261,117.86562448)(170.75065465,115.68494227)(170.75065465,112.7457619)
\curveto(170.75065465,109.9013938)(171.26667342,107.7410285)(172.29871096,106.26466601)
\curveto(173.3307485,104.80184813)(174.76953024,104.07043919)(176.61505619,104.07043919)
\curveto(178.4362989,104.07043919)(179.86293902,104.79507583)(180.89497656,106.2443491)
\curveto(181.93915571,107.70716698)(182.46124529,109.87430458)(182.46124529,112.7457619)
\closepath
}
}
{
\newrgbcolor{curcolor}{0 0 0}
\pscustom[linestyle=none,fillstyle=solid,fillcolor=curcolor]
{
\newpath
\moveto(203.98833155,119.93794981)
\lineto(203.80620728,119.93794981)
\curveto(203.29625932,120.07339591)(202.79845298,120.16820818)(202.31278826,120.22238662)
\curveto(201.83926515,120.29010967)(201.27467991,120.3239712)(200.61903254,120.3239712)
\curveto(199.56271176,120.3239712)(198.54281584,120.0598513)(197.55934478,119.53161151)
\curveto(196.57587372,119.01691633)(195.62882751,118.34645814)(194.71820615,117.52023693)
\lineto(194.71820615,101.40892333)
\lineto(191.29426985,101.40892333)
\lineto(191.29426985,124.10291739)
\lineto(194.71820615,124.10291739)
\lineto(194.71820615,120.75062641)
\curveto(196.07806737,121.96964131)(197.27401676,122.82972405)(198.30605429,123.33087462)
\curveto(199.35023345,123.8455698)(200.41262503,124.10291739)(201.49322904,124.10291739)
\curveto(202.08816832,124.10291739)(202.51919576,124.08260047)(202.78631136,124.04196664)
\curveto(203.05342696,124.01487742)(203.45410036,123.95392668)(203.98833155,123.85911441)
\closepath
}
}
{
\newrgbcolor{curcolor}{0 0 0}
\pscustom[linestyle=none,fillstyle=solid,fillcolor=curcolor]
{
\newpath
\moveto(224.80514402,124.10291739)
\lineto(212.93064154,93.03835435)
\lineto(209.26994369,93.03835435)
\lineto(213.05812853,102.50603674)
\lineto(204.95359846,124.10291739)
\lineto(208.66893359,124.10291739)
\lineto(214.91579609,107.28051176)
\lineto(221.21729588,124.10291739)
\closepath
}
}
{
\newrgbcolor{curcolor}{0.80000001 0.80000001 0.80000001}
\pscustom[linestyle=none,fillstyle=solid,fillcolor=curcolor]
{
\newpath
\moveto(335.62408264,203.03956886)
\lineto(585.20349755,203.03956886)
\lineto(585.20349755,20.30070033)
\lineto(335.62408264,20.30070033)
\closepath
}
}
{
\newrgbcolor{curcolor}{0 0 0}
\pscustom[linewidth=0.85007621,linecolor=curcolor]
{
\newpath
\moveto(335.62408264,203.03956886)
\lineto(585.20349755,203.03956886)
\lineto(585.20349755,20.30070033)
\lineto(335.62408264,20.30070033)
\closepath
}
}
{
\newrgbcolor{curcolor}{0 0 0}
\pscustom[linestyle=none,fillstyle=solid,fillcolor=curcolor]
{
\newpath
\moveto(411.65912103,100.72945044)
\lineto(408.05306046,100.72945044)
\lineto(408.05306046,126.79605239)
\lineto(400.51311564,109.05938559)
\lineto(398.36404924,109.05938559)
\lineto(390.87874169,126.79605239)
\lineto(390.87874169,100.72945044)
\lineto(387.50944268,100.72945044)
\lineto(387.50944268,130.98133688)
\lineto(392.426798,130.98133688)
\lineto(399.65713156,114.13861434)
\lineto(406.65070358,130.98133688)
\lineto(411.65912103,130.98133688)
\closepath
}
}
{
\newrgbcolor{curcolor}{0 0 0}
\pscustom[linestyle=none,fillstyle=solid,fillcolor=curcolor]
{
\newpath
\moveto(435.69952447,111.68026763)
\lineto(420.71069696,111.68026763)
\curveto(420.71069696,110.2851728)(420.89889204,109.0661579)(421.2752822,108.02322293)
\curveto(421.65167236,106.99383257)(422.16769113,106.14729444)(422.8233385,105.48360855)
\curveto(423.45470264,104.83346727)(424.20141215,104.34586131)(425.06346704,104.02079067)
\curveto(425.93766354,103.69572003)(426.89685137,103.53318471)(427.94103052,103.53318471)
\curveto(429.32517498,103.53318471)(430.71539025,103.83793844)(432.11167633,104.44744589)
\curveto(433.52010403,105.07049795)(434.52178752,105.6800054)(435.11672681,106.27596824)
\lineto(435.29885108,106.27596824)
\lineto(435.29885108,102.11100066)
\curveto(434.14539736,101.56921626)(432.96766041,101.11547183)(431.76564022,100.74976736)
\curveto(430.56362003,100.38406289)(429.30089175,100.20121065)(427.97745538,100.20121065)
\curveto(424.60208555,100.20121065)(421.96735443,101.2170564)(420.07326201,103.2487479)
\curveto(418.17916959,105.29398401)(417.23212338,108.19253055)(417.23212338,111.94438752)
\curveto(417.23212338,115.65561066)(418.13667392,118.60156334)(419.94577502,120.78224555)
\curveto(421.76701773,122.96292776)(424.15891649,124.05326887)(427.1214713,124.05326887)
\curveto(429.86547699,124.05326887)(431.97811853,123.15932461)(433.45939594,121.37143608)
\curveto(434.95281496,119.58354756)(435.69952447,117.04393319)(435.69952447,113.75259296)
\closepath
\moveto(432.36665031,114.60590339)
\curveto(432.35450869,116.61050567)(431.89919802,118.16136351)(431.00071828,119.25847692)
\curveto(430.11438016,120.35559033)(428.76058974,120.90414704)(426.93934703,120.90414704)
\curveto(425.1059627,120.90414704)(423.64289772,120.30141189)(422.5501521,119.0959416)
\curveto(421.46954809,117.89047131)(420.85639637,116.39379191)(420.71069696,114.60590339)
\closepath
}
}
{
\newrgbcolor{curcolor}{0 0 0}
\pscustom[linestyle=none,fillstyle=solid,fillcolor=curcolor]
{
\newpath
\moveto(470.57632247,100.72945044)
\lineto(467.15238617,100.72945044)
\lineto(467.15238617,113.65100838)
\curveto(467.15238617,114.6262203)(467.10989051,115.5675707)(467.02489918,116.47505957)
\curveto(466.95204947,117.38254844)(466.78813763,118.10718507)(466.53316365,118.64896947)
\curveto(466.25390643,119.2313877)(465.85323304,119.67158753)(465.33114346,119.96956895)
\curveto(464.80905388,120.26755037)(464.05627356,120.41654108)(463.0728025,120.41654108)
\curveto(462.11361467,120.41654108)(461.15442684,120.14564888)(460.19523901,119.60386448)
\curveto(459.23605118,119.07562469)(458.27686335,118.39839419)(457.31767553,117.57217298)
\curveto(457.35410038,117.26064695)(457.38445443,116.89494248)(457.40873766,116.47505957)
\curveto(457.4330209,116.06872127)(457.44516252,115.66238297)(457.44516252,115.25604467)
\lineto(457.44516252,100.72945044)
\lineto(454.02122622,100.72945044)
\lineto(454.02122622,113.65100838)
\curveto(454.02122622,114.65330952)(453.97873056,115.60143222)(453.89373923,116.49537648)
\curveto(453.82088952,117.40286535)(453.65697768,118.12750199)(453.4020037,118.66928639)
\curveto(453.12274648,119.25170462)(452.72207308,119.68513214)(452.19998351,119.96956895)
\curveto(451.67789393,120.26755037)(450.92511361,120.41654108)(449.94164254,120.41654108)
\curveto(449.00673795,120.41654108)(448.06576255,120.15919349)(447.11871634,119.64449831)
\curveto(446.18381175,119.12980313)(445.24890716,118.47288954)(444.31400257,117.67375755)
\lineto(444.31400257,100.72945044)
\lineto(440.89006627,100.72945044)
\lineto(440.89006627,123.4234445)
\lineto(444.31400257,123.4234445)
\lineto(444.31400257,120.90414704)
\curveto(445.38246496,121.89290357)(446.44485654,122.66494634)(447.50117731,123.22027535)
\curveto(448.5696397,123.77560436)(449.70488099,124.05326887)(450.90690118,124.05326887)
\curveto(452.29104564,124.05326887)(453.46271179,123.72819823)(454.42189962,123.07805695)
\curveto(455.39322906,122.42791567)(456.11565534,121.5271991)(456.58917844,120.37590725)
\curveto(457.9733229,121.67618981)(459.23605118,122.6107679)(460.37736328,123.17964152)
\curveto(461.51867538,123.76205975)(462.738908,124.05326887)(464.03806113,124.05326887)
\curveto(466.27211886,124.05326887)(467.91730811,123.29477071)(468.97362888,121.77777439)
\curveto(470.04209127,120.27432267)(470.57632247,118.16813582)(470.57632247,115.45921382)
\closepath
}
}
{
\newrgbcolor{curcolor}{0 0 0}
\pscustom[linestyle=none,fillstyle=solid,fillcolor=curcolor]
{
\newpath
\moveto(494.50745119,112.06628901)
\curveto(494.50745119,108.36861048)(493.65753793,105.44974703)(491.95771139,103.30969865)
\curveto(490.25788486,101.16965027)(487.98133147,100.09962608)(485.12805123,100.09962608)
\curveto(482.25048774,100.09962608)(479.96179273,101.16965027)(478.2619662,103.30969865)
\curveto(476.57428129,105.44974703)(475.73043883,108.36861048)(475.73043883,112.06628901)
\curveto(475.73043883,115.76396754)(476.57428129,118.682831)(478.2619662,120.82287938)
\curveto(479.96179273,122.97647237)(482.25048774,124.05326887)(485.12805123,124.05326887)
\curveto(487.98133147,124.05326887)(490.25788486,122.97647237)(491.95771139,120.82287938)
\curveto(493.65753793,118.682831)(494.50745119,115.76396754)(494.50745119,112.06628901)
\closepath
\moveto(490.97424033,112.06628901)
\curveto(490.97424033,115.00546938)(490.45822156,117.18615159)(489.42618403,118.60833564)
\curveto(488.39414649,120.0440643)(486.96143556,120.76192863)(485.12805123,120.76192863)
\curveto(483.27038366,120.76192863)(481.82553111,120.0440643)(480.79349357,118.60833564)
\curveto(479.77359765,117.18615159)(479.2636497,115.00546938)(479.2636497,112.06628901)
\curveto(479.2636497,109.22192091)(479.77966846,107.06155562)(480.811706,105.58519313)
\curveto(481.84374354,104.12237525)(483.28252528,103.39096631)(485.12805123,103.39096631)
\curveto(486.94929394,103.39096631)(488.37593406,104.11560294)(489.4079716,105.56487621)
\curveto(490.45215075,107.02769409)(490.97424033,109.19483169)(490.97424033,112.06628901)
\closepath
}
}
{
\newrgbcolor{curcolor}{0 0 0}
\pscustom[linestyle=none,fillstyle=solid,fillcolor=curcolor]
{
\newpath
\moveto(512.50132659,119.25847692)
\lineto(512.31920232,119.25847692)
\curveto(511.80925436,119.39392302)(511.31144802,119.48873529)(510.8257833,119.54291373)
\curveto(510.35226019,119.61063678)(509.78767495,119.64449831)(509.13202758,119.64449831)
\curveto(508.0757068,119.64449831)(507.05581088,119.38037841)(506.07233982,118.85213862)
\curveto(505.08886876,118.33744344)(504.14182255,117.66698525)(503.23120119,116.84076404)
\lineto(503.23120119,100.72945044)
\lineto(499.80726489,100.72945044)
\lineto(499.80726489,123.4234445)
\lineto(503.23120119,123.4234445)
\lineto(503.23120119,120.07115352)
\curveto(504.59106242,121.29016842)(505.7870118,122.15025116)(506.81904933,122.65140173)
\curveto(507.86322849,123.16609691)(508.92562007,123.4234445)(510.00622408,123.4234445)
\curveto(510.60116336,123.4234445)(511.03219081,123.40312759)(511.2993064,123.36249376)
\curveto(511.566422,123.33540454)(511.9670954,123.27445379)(512.50132659,123.17964152)
\closepath
}
}
{
\newrgbcolor{curcolor}{0 0 0}
\pscustom[linestyle=none,fillstyle=solid,fillcolor=curcolor]
{
\newpath
\moveto(533.31813906,123.4234445)
\lineto(521.44363658,92.35888146)
\lineto(517.78293873,92.35888146)
\lineto(521.57112357,101.82656385)
\lineto(513.4665935,123.4234445)
\lineto(517.18192863,123.4234445)
\lineto(523.42879113,106.60103888)
\lineto(529.73029092,123.4234445)
\closepath
}
}
{
\newrgbcolor{curcolor}{0 1 0}
\pscustom[linestyle=none,fillstyle=solid,fillcolor=curcolor,opacity=0.50196099]
{
\newpath
\moveto(359.64757781,459.28722357)
\lineto(561.17998796,459.28722357)
\lineto(561.17998796,261.39695421)
\lineto(359.64757781,261.39695421)
\closepath
}
}
{
\newrgbcolor{curcolor}{0 1 0}
\pscustom[linewidth=2.61146087,linecolor=curcolor,strokeopacity=0.50196099]
{
\newpath
\moveto(359.64757781,459.28722357)
\lineto(561.17998796,459.28722357)
\lineto(561.17998796,261.39695421)
\lineto(359.64757781,261.39695421)
\closepath
}
}
\end{pspicture}
}
    \captionsetup{width=0.6\linewidth}
    \caption{A visual representation of what calling \texttt{set\_cpus\_allowed\_ptr(curthread, cpumask\_of\_node(1))}
    does to the allowed processors of the current process.}
    \label{fig:setcpusallowed1}
\end{figure}

\singlespacing
\begin{lstlisting}[caption={The relevant chapters of Linux's scheduler code that handles a change in the cpumask indicating where a process is allowed to execute},label={lst:linux_set_cpus_allowed},language=C]
// include/linux/sched.h
struct task_struct {
// ...
	int				nr_cpus_allowed;
// ...
	cpumask_t			cpus_mask;
// ...
}
// kernel/sched/core.c
// set_cpus_allowed_ptr calls __set_cpus_allowed_ptr(p, new_mask, 0)
/* Change a given task's CPU affinity. Migrate the thread to a
 * proper CPU and schedule it away if the CPU it's executing on
 * is removed from the allowed bitmask.
 */
static int __set_cpus_allowed_ptr(struct task_struct *p, 
                                  const struct cpumask *new_mask, bool check)
{
	const struct cpumask *cpu_valid_mask = cpu_active_mask;
	unsigned int dest_cpu; struct rq_flags rf; struct rq *rq; int ret = 0;
	rq = task_rq_lock(p, &rf); update_rq_clock(rq);
// ...
	do_set_cpus_allowed(p, new_mask); // See Listing 3, new_mask is from cpumask_of_node
// ...
	/* Can the task run on the task's current CPU? If so, we're done */
	if (cpumask_test_cpu(task_cpu(p), new_mask)) goto out;
	if (task_running(rq, p) || p->state == TASK_WAKING) {
		struct migration_arg arg = { p, dest_cpu }; 
		/* Need help from migration thread: drop lock and wait. */
		task_rq_unlock(rq, p, &rf);
        stop_one_cpu(cpu_of(rq), migration_cpu_stop, &arg);
		return 0;
	} else if (task_on_rq_queued(p)) {	// dropping lock immediately anyways
	    rq = move_queued_task(rq, &rf, p, dest_cpu);
	}
out:
	task_rq_unlock(rq, p, &rf);
	return ret;
}
\end{lstlisting}
\begin{lstlisting}[caption={The relevant chapters of Linux's scheduler code that set a process's cpumask},label={lst:linux_set_cpus_allowed_common},language=C]
// do_set_cpus_allowed always calls this function
// A structure of function pointers is involved, 
// but for the normal scheduler it simply points to this function 
void set_cpus_allowed_common(struct task_struct *p, const struct cpumask *new_mask)
{
	cpumask_copy(&p->cpus_mask, new_mask);
	p->nr_cpus_allowed = cpumask_weight(new_mask);
}
\end{lstlisting}
\doublespacing

The second step, changing where the process was allowed to allocate new memory from, was quite similar.
As with the processors allowed, the task object also contains a bitmask indicating which memory it is allowed to use,
which also defaults to all nodes (Figure \ref{fig:defaultmasks}),
but was set to one node for testing purposes (Figure \ref{fig:testingmasks}).
This takes a slightly different kind of bitmask, a nodemask that indicates which nodes the process is allowed
to allocate memory from, and it also only operates on the current process.

\begin{figure}[H]
    \centering
    \resizebox{0.5\linewidth}{!}{\input{diagrams/setmemsallowed1}}
    \captionsetup{width=0.5\linewidth}
    \caption{A visual representation of what calling \texttt{set\_mems\_allowed(nodemask\_of\_node(1))}
    does to the allowed memory of the current process.}
    \label{fig:setmemsallowed1}
\end{figure}

\singlespacing
\begin{lstlisting}[caption={The relevant chapters of Linux's scheduler code that handle the nodemask
indicating where a process is allowed to allocate memory},label={lst:linux_set_mems_allowed},language=C]
// include/linux/sched.h
struct task_struct {
// ...
	/* Protection against (de-)allocation: mm, files, fs, tty, keyrings, mems_allowed, mempolicy: */
	spinlock_t      alloc_lock;
// ...
	nodemask_t      mems_allowed;
	/* Seqence number to catch updates: */
	seqcount_t      mems_allowed_seq;
// ...
}
// include/linux/cpuset.h
static inline void set_mems_allowed(nodemask_t nodemask)
{
	unsigned long flags;

	task_lock(current);
	local_irq_save(flags);
	write_seqcount_begin(&current->mems_allowed_seq);
	current->mems_allowed = nodemask;
	write_seqcount_end(&current->mems_allowed_seq);
	local_irq_restore(flags);
	task_unlock(current);
}
\end{lstlisting}
\doublespacing
The third step, however, was quite a bit more involved.
Linux does have a system call, \texttt{migrate\_pages}, that allows a user to move the memory of a program from one node
to another.
However, system calls like \texttt{migrate\_pages} cannot be called directly from kernel modules like ZFS 
for a variety of reasons,
but knowing that this functionality already existed enabled me to look into where in the Linux kernel 
this system call was implemented.

It turns out that this system call is just a wrapper around the \texttt{kernel\_migrate\_pages} function, but this did not help
as it still took arguments in the same way as \texttt{migrate\_pages}, so it could not be called from ZFS.
However, within this function was the \texttt{do\_migrate\_pages} function, which does all of the actual work to migrate
all of a process's memory from one node to another.
This structure of \texttt{function} calling \texttt{kernel\_function} calling \texttt{do\_function} is a common construct
in the Linux kernel.
This function could actually be called from ZFS, but was not exported to kernel modules, as it had no security checks, 
these all having been handled by \texttt{kernel\_migrate\_pages}.

\singlespacing
\lstinputlisting[caption={The entirety of my prototype patch to the Linux kernel, enabling the SPL to migrate a process's memory from one node to another. An actual patch to export this functionality could not just export this function. It would require another function with error checking to prevent misuse and would likely
still not be accepted, as kernel modules are generally not allowed to modify the scheduler or move around the memory of a user's process in this way.}, label={lst:linuxpatchinline},language=diff]{code/linux.patch}
\doublespacing

Once I could access this function from within ZFS, I added two new functions to the SPL, the Solaris Portability Layer
that is the interface between the Linux kernel and ZFS.
The first was \texttt{spl\_migrate\_pages}, which simply did the work of the \texttt{kernel\_migrate\_pages} function,
taking actual node numbers and a task structure and converting them to the arguments that \texttt{do\_migrate\_pages} actually
takes, that is a common structure in Linux kernel interfaces.

\singlespacing
\begin{lstlisting}[caption={The \texttt{spl\_migrate} function, used to move the current process
to another NUMA node},label={lst:splmigratefunction},language=C]]
void
spl_migrate(int node)
{
	if (node >= nr_node_ids || node < 0) {
		pr_warn("SPL: Can't migrate to node %d!\n", node);
		return;
	}
	set_cpus_allowed_ptr(curthread, cpumask_of_node(node));
	set_mems_allowed(nodemask_of_node(node));
	spl_migrate_pages(curthread, node);
	if (curnode != node) {
		pr_err(KERN_ERR "SPL: Failed to migrate task %s!\n", curthread->comm);
		dump_stack();
	}
}
EXPORT_SYMBOL(spl_migrate);
\end{lstlisting}
\doublespacing

Now that the migration function was in place, it needed to be called from within OpenZFS.
I first added a node number to the ARC header, the metadata structure for each block stored in the ARC discussed above
(Section \ref{chapter:ARC}).
Then, I modified the allocation method for ARC Buffer Data, to always prefer the current node when allocating memory (Section \ref{chapter:ABD}).
When the allocating large ABDs, ZFS attempts to allocate them in large chunks, as big as the
kernel will allow to fit all the data required. 
However, if the data required is larger than the kernel has available in contiguous physical pages in memory
then it will split up the buffer into a series of chunks, each of which points to the next,
but can be used as one big buffer within ZFS thanks to some abstractions built on top of this structure.

When allocating these chunks, ZFS will allocate from wherever is available, defaulting to the local node
but detecting when Linux is unable to get enough memory from the local node and instead allocating
from whatever remote node succeeded at allocating the previous large chunk.
This could result in ABDs not located on the same node as the current process at all,
and in the worst case could result in different parts of a buffer allocated on entirely different nodes,
which would be extremely difficult to handle.
To simplify this I removed this allowance, ensuring that this code would always try to allocate buffers
from the current node.
This would only have come into effect if one NUMA node ran out of memory.
Finally, I modified the \texttt{arc\_read} function, which is called by the DMU if it cannot find the data it is looking
for in its local cache.
In the code path for returning data already in the ARC,
I added code to check if the node number of this found ARC header was the same as the current node number that the process
was running on, and move the process if they were different 
For the specifics of this patch see Appendix \ref{sourcecode} Listing \ref{lst:arcpatch}.

\chapter{Results}
With task migration enabled in ZFS as implemented in the above prototype, we see significant improvements in runtime and latency
for sequential file-access in many cases.
Repeating the same tests used above to show a difference in local and remote node ARC access, we see a 10\%
improvement in runtime for files larger than 1 GB with a 1 MB read size that grows to 15\% as file size increases 
(Figure \ref{fig:smallresults}).

\begin{figure}[H]
    \centering
    \resizebox{0.45\linewidth}{!}{%LaTeX with PSTricks extensions
%%Creator: Inkscape 1.0.2-2 (e86c870879, 2021-01-15)
%%Please note this file requires PSTricks extensions
\psset{xunit=.5pt,yunit=.5pt,runit=.5pt}
\begin{pspicture}(600,480)
{
\newrgbcolor{curcolor}{0 0 0}
\pscustom[linewidth=1,linecolor=curcolor]
{
\newpath
\moveto(105.1,57.6)
\lineto(114.1,57.6)
\moveto(575,57.6)
\lineto(566,57.6)
}
}
{
\newrgbcolor{curcolor}{0 0 0}
\pscustom[linestyle=none,fillstyle=solid,fillcolor=curcolor]
{
\newpath
\moveto(53.92109375,57.93632812)
\curveto(53.92109375,58.95195312)(54.02460937,59.76835937)(54.23164062,60.38554687)
\curveto(54.44257812,61.00664062)(54.753125,61.48515625)(55.16328125,61.82109375)
\curveto(55.57734375,62.15703125)(56.096875,62.325)(56.721875,62.325)
\curveto(57.1828125,62.325)(57.58710937,62.23125)(57.93476562,62.04375)
\curveto(58.28242187,61.86015625)(58.56953125,61.59257812)(58.79609375,61.24101562)
\curveto(59.02265625,60.89335937)(59.20039062,60.46757812)(59.32929687,59.96367187)
\curveto(59.45820312,59.46367187)(59.52265625,58.78789062)(59.52265625,57.93632812)
\curveto(59.52265625,56.92851562)(59.41914062,56.1140625)(59.21210937,55.49296875)
\curveto(59.00507812,54.87578125)(58.69453125,54.39726562)(58.28046875,54.05742187)
\curveto(57.8703125,53.72148437)(57.35078125,53.55351562)(56.721875,53.55351562)
\curveto(55.89375,53.55351562)(55.24335937,53.85039062)(54.77070312,54.44414062)
\curveto(54.20429687,55.15898437)(53.92109375,56.32304687)(53.92109375,57.93632812)
\closepath
\moveto(55.00507812,57.93632812)
\curveto(55.00507812,56.52617187)(55.16914062,55.58671875)(55.49726562,55.11796875)
\curveto(55.82929687,54.653125)(56.2375,54.42070312)(56.721875,54.42070312)
\curveto(57.20625,54.42070312)(57.6125,54.65507812)(57.940625,55.12382812)
\curveto(58.27265625,55.59257812)(58.43867187,56.53007812)(58.43867187,57.93632812)
\curveto(58.43867187,59.35039062)(58.27265625,60.28984375)(57.940625,60.7546875)
\curveto(57.6125,61.21953125)(57.20234375,61.45195312)(56.71015625,61.45195312)
\curveto(56.22578125,61.45195312)(55.8390625,61.246875)(55.55,60.83671875)
\curveto(55.18671875,60.31328125)(55.00507812,59.34648437)(55.00507812,57.93632812)
\closepath
}
}
{
\newrgbcolor{curcolor}{0 0 0}
\pscustom[linestyle=none,fillstyle=solid,fillcolor=curcolor]
{
\newpath
\moveto(61.18671875,53.7)
\lineto(61.18671875,54.90117187)
\lineto(62.38789062,54.90117187)
\lineto(62.38789062,53.7)
\closepath
}
}
{
\newrgbcolor{curcolor}{0 0 0}
\pscustom[linestyle=none,fillstyle=solid,fillcolor=curcolor]
{
\newpath
\moveto(63.92890625,57.93632812)
\curveto(63.92890625,58.95195312)(64.03242187,59.76835937)(64.23945312,60.38554687)
\curveto(64.45039062,61.00664062)(64.7609375,61.48515625)(65.17109375,61.82109375)
\curveto(65.58515625,62.15703125)(66.1046875,62.325)(66.7296875,62.325)
\curveto(67.190625,62.325)(67.59492187,62.23125)(67.94257812,62.04375)
\curveto(68.29023437,61.86015625)(68.57734375,61.59257812)(68.80390625,61.24101562)
\curveto(69.03046875,60.89335937)(69.20820312,60.46757812)(69.33710937,59.96367187)
\curveto(69.46601562,59.46367187)(69.53046875,58.78789062)(69.53046875,57.93632812)
\curveto(69.53046875,56.92851562)(69.42695312,56.1140625)(69.21992187,55.49296875)
\curveto(69.01289062,54.87578125)(68.70234375,54.39726562)(68.28828125,54.05742187)
\curveto(67.878125,53.72148437)(67.35859375,53.55351562)(66.7296875,53.55351562)
\curveto(65.9015625,53.55351562)(65.25117187,53.85039062)(64.77851562,54.44414062)
\curveto(64.21210937,55.15898437)(63.92890625,56.32304687)(63.92890625,57.93632812)
\closepath
\moveto(65.01289062,57.93632812)
\curveto(65.01289062,56.52617187)(65.17695312,55.58671875)(65.50507812,55.11796875)
\curveto(65.83710937,54.653125)(66.2453125,54.42070312)(66.7296875,54.42070312)
\curveto(67.2140625,54.42070312)(67.6203125,54.65507812)(67.9484375,55.12382812)
\curveto(68.28046875,55.59257812)(68.44648437,56.53007812)(68.44648437,57.93632812)
\curveto(68.44648437,59.35039062)(68.28046875,60.28984375)(67.9484375,60.7546875)
\curveto(67.6203125,61.21953125)(67.21015625,61.45195312)(66.71796875,61.45195312)
\curveto(66.23359375,61.45195312)(65.846875,61.246875)(65.5578125,60.83671875)
\curveto(65.19453125,60.31328125)(65.01289062,59.34648437)(65.01289062,57.93632812)
\closepath
}
}
{
\newrgbcolor{curcolor}{0 0 0}
\pscustom[linestyle=none,fillstyle=solid,fillcolor=curcolor]
{
\newpath
\moveto(70.60859375,55.96757812)
\lineto(71.66328125,56.10820312)
\curveto(71.784375,55.51054687)(71.98945312,55.07890625)(72.27851562,54.81328125)
\curveto(72.57148437,54.5515625)(72.92695312,54.42070312)(73.34492187,54.42070312)
\curveto(73.84101562,54.42070312)(74.25898437,54.59257812)(74.59882812,54.93632812)
\curveto(74.94257812,55.28007812)(75.11445312,55.70585937)(75.11445312,56.21367187)
\curveto(75.11445312,56.69804687)(74.95625,57.09648437)(74.63984375,57.40898437)
\curveto(74.3234375,57.72539062)(73.92109375,57.88359375)(73.4328125,57.88359375)
\curveto(73.23359375,57.88359375)(72.98554687,57.84453125)(72.68867187,57.76640625)
\lineto(72.80585937,58.6921875)
\curveto(72.87617187,58.684375)(72.9328125,58.68046875)(72.97578125,58.68046875)
\curveto(73.425,58.68046875)(73.82929687,58.79765625)(74.18867187,59.03203125)
\curveto(74.54804687,59.26640625)(74.72773437,59.62773437)(74.72773437,60.11601562)
\curveto(74.72773437,60.50273437)(74.596875,60.82304687)(74.33515625,61.07695312)
\curveto(74.0734375,61.33085937)(73.73554687,61.4578125)(73.32148437,61.4578125)
\curveto(72.91132812,61.4578125)(72.56953125,61.32890625)(72.29609375,61.07109375)
\curveto(72.02265625,60.81328125)(71.846875,60.4265625)(71.76875,59.9109375)
\lineto(70.7140625,60.0984375)
\curveto(70.84296875,60.80546875)(71.1359375,61.35234375)(71.59296875,61.7390625)
\curveto(72.05,62.1296875)(72.61835937,62.325)(73.29804687,62.325)
\curveto(73.76679687,62.325)(74.1984375,62.2234375)(74.59296875,62.0203125)
\curveto(74.9875,61.82109375)(75.28828125,61.54765625)(75.4953125,61.2)
\curveto(75.70625,60.85234375)(75.81171875,60.48320312)(75.81171875,60.09257812)
\curveto(75.81171875,59.72148437)(75.71210937,59.38359375)(75.51289062,59.07890625)
\curveto(75.31367187,58.77421875)(75.01875,58.53203125)(74.628125,58.35234375)
\curveto(75.1359375,58.23515625)(75.53046875,57.99101562)(75.81171875,57.61992187)
\curveto(76.09296875,57.25273437)(76.23359375,56.79179687)(76.23359375,56.23710937)
\curveto(76.23359375,55.48710937)(75.96015625,54.85039062)(75.41328125,54.32695312)
\curveto(74.86640625,53.80742187)(74.175,53.54765625)(73.3390625,53.54765625)
\curveto(72.58515625,53.54765625)(71.95820312,53.77226562)(71.45820312,54.22148437)
\curveto(70.96210937,54.67070312)(70.67890625,55.25273437)(70.60859375,55.96757812)
\closepath
}
}
{
\newrgbcolor{curcolor}{0 0 0}
\pscustom[linestyle=none,fillstyle=solid,fillcolor=curcolor]
{
\newpath
\moveto(81.24921875,53.7)
\lineto(80.19453125,53.7)
\lineto(80.19453125,60.42070312)
\curveto(79.940625,60.17851562)(79.60664062,59.93632812)(79.19257812,59.69414062)
\curveto(78.78242187,59.45195312)(78.41328125,59.2703125)(78.08515625,59.14921875)
\lineto(78.08515625,60.16875)
\curveto(78.675,60.44609375)(79.190625,60.78203125)(79.63203125,61.1765625)
\curveto(80.0734375,61.57109375)(80.3859375,61.95390625)(80.56953125,62.325)
\lineto(81.24921875,62.325)
\closepath
}
}
{
\newrgbcolor{curcolor}{0 0 0}
\pscustom[linestyle=none,fillstyle=solid,fillcolor=curcolor]
{
\newpath
\moveto(89.49335937,54.71367187)
\lineto(89.49335937,53.7)
\lineto(83.815625,53.7)
\curveto(83.8078125,53.95390625)(83.84882812,54.19804687)(83.93867187,54.43242187)
\curveto(84.08320312,54.81914062)(84.31367187,55.2)(84.63007812,55.575)
\curveto(84.95039062,55.95)(85.41132812,56.38359375)(86.01289062,56.87578125)
\curveto(86.94648437,57.64140625)(87.57734375,58.246875)(87.90546875,58.6921875)
\curveto(88.23359375,59.14140625)(88.39765625,59.56523437)(88.39765625,59.96367187)
\curveto(88.39765625,60.38164062)(88.24726562,60.73320312)(87.94648437,61.01835937)
\curveto(87.64960937,61.30742187)(87.2609375,61.45195312)(86.78046875,61.45195312)
\curveto(86.27265625,61.45195312)(85.86640625,61.29960937)(85.56171875,60.99492187)
\curveto(85.25703125,60.69023437)(85.10273437,60.26835937)(85.09882812,59.72929687)
\lineto(84.01484375,59.840625)
\curveto(84.0890625,60.64921875)(84.36835937,61.26445312)(84.85273437,61.68632812)
\curveto(85.33710937,62.11210937)(85.9875,62.325)(86.80390625,62.325)
\curveto(87.628125,62.325)(88.28046875,62.09648437)(88.7609375,61.63945312)
\curveto(89.24140625,61.18242187)(89.48164062,60.61601562)(89.48164062,59.94023437)
\curveto(89.48164062,59.59648437)(89.41132812,59.25859375)(89.27070312,58.9265625)
\curveto(89.13007812,58.59453125)(88.89570312,58.24492187)(88.56757812,57.87773437)
\curveto(88.24335937,57.51054687)(87.70234375,57.00664062)(86.94453125,56.36601562)
\curveto(86.31171875,55.83476562)(85.90546875,55.4734375)(85.72578125,55.28203125)
\curveto(85.54609375,55.09453125)(85.39765625,54.90507812)(85.28046875,54.71367187)
\closepath
}
}
{
\newrgbcolor{curcolor}{0 0 0}
\pscustom[linestyle=none,fillstyle=solid,fillcolor=curcolor]
{
\newpath
\moveto(90.62421875,55.95)
\lineto(91.73164062,56.04375)
\curveto(91.81367187,55.5046875)(92.003125,55.0984375)(92.3,54.825)
\curveto(92.60078125,54.55546875)(92.96210937,54.42070312)(93.38398437,54.42070312)
\curveto(93.89179687,54.42070312)(94.32148437,54.61210937)(94.67304687,54.99492187)
\curveto(95.02460937,55.37773437)(95.20039062,55.88554687)(95.20039062,56.51835937)
\curveto(95.20039062,57.11992187)(95.03046875,57.59453125)(94.690625,57.9421875)
\curveto(94.3546875,58.28984375)(93.91328125,58.46367187)(93.36640625,58.46367187)
\curveto(93.0265625,58.46367187)(92.71992187,58.38554687)(92.44648437,58.22929687)
\curveto(92.17304687,58.07695312)(91.95820312,57.87773437)(91.80195312,57.63164062)
\lineto(90.81171875,57.76054687)
\lineto(91.64375,62.17265625)
\lineto(95.91523437,62.17265625)
\lineto(95.91523437,61.16484375)
\lineto(92.4875,61.16484375)
\lineto(92.02460937,58.85625)
\curveto(92.54023437,59.215625)(93.08125,59.3953125)(93.64765625,59.3953125)
\curveto(94.39765625,59.3953125)(95.03046875,59.13554687)(95.54609375,58.61601562)
\curveto(96.06171875,58.09648437)(96.31953125,57.42851562)(96.31953125,56.61210937)
\curveto(96.31953125,55.83476562)(96.09296875,55.16289062)(95.63984375,54.59648437)
\curveto(95.0890625,53.90117187)(94.33710937,53.55351562)(93.38398437,53.55351562)
\curveto(92.60273437,53.55351562)(91.9640625,53.77226562)(91.46796875,54.20976562)
\curveto(90.97578125,54.64726562)(90.69453125,55.22734375)(90.62421875,55.95)
\closepath
}
}
{
\newrgbcolor{curcolor}{0 0 0}
\pscustom[linewidth=1,linecolor=curcolor]
{
\newpath
\moveto(105.1,103.6)
\lineto(114.1,103.6)
\moveto(575,103.6)
\lineto(566,103.6)
}
}
{
\newrgbcolor{curcolor}{0 0 0}
\pscustom[linestyle=none,fillstyle=solid,fillcolor=curcolor]
{
\newpath
\moveto(60.59492187,103.93632812)
\curveto(60.59492187,104.95195312)(60.6984375,105.76835937)(60.90546875,106.38554687)
\curveto(61.11640625,107.00664062)(61.42695312,107.48515625)(61.83710937,107.82109375)
\curveto(62.25117187,108.15703125)(62.77070312,108.325)(63.39570312,108.325)
\curveto(63.85664062,108.325)(64.2609375,108.23125)(64.60859375,108.04375)
\curveto(64.95625,107.86015625)(65.24335937,107.59257812)(65.46992187,107.24101562)
\curveto(65.69648437,106.89335937)(65.87421875,106.46757812)(66.003125,105.96367187)
\curveto(66.13203125,105.46367187)(66.19648437,104.78789062)(66.19648437,103.93632812)
\curveto(66.19648437,102.92851562)(66.09296875,102.1140625)(65.8859375,101.49296875)
\curveto(65.67890625,100.87578125)(65.36835937,100.39726562)(64.95429687,100.05742187)
\curveto(64.54414062,99.72148437)(64.02460937,99.55351562)(63.39570312,99.55351562)
\curveto(62.56757812,99.55351562)(61.9171875,99.85039062)(61.44453125,100.44414062)
\curveto(60.878125,101.15898437)(60.59492187,102.32304687)(60.59492187,103.93632812)
\closepath
\moveto(61.67890625,103.93632812)
\curveto(61.67890625,102.52617187)(61.84296875,101.58671875)(62.17109375,101.11796875)
\curveto(62.503125,100.653125)(62.91132812,100.42070312)(63.39570312,100.42070312)
\curveto(63.88007812,100.42070312)(64.28632812,100.65507812)(64.61445312,101.12382812)
\curveto(64.94648437,101.59257812)(65.1125,102.53007812)(65.1125,103.93632812)
\curveto(65.1125,105.35039062)(64.94648437,106.28984375)(64.61445312,106.7546875)
\curveto(64.28632812,107.21953125)(63.87617187,107.45195312)(63.38398437,107.45195312)
\curveto(62.89960937,107.45195312)(62.51289062,107.246875)(62.22382812,106.83671875)
\curveto(61.86054687,106.31328125)(61.67890625,105.34648437)(61.67890625,103.93632812)
\closepath
}
}
{
\newrgbcolor{curcolor}{0 0 0}
\pscustom[linestyle=none,fillstyle=solid,fillcolor=curcolor]
{
\newpath
\moveto(67.86054687,99.7)
\lineto(67.86054687,100.90117187)
\lineto(69.06171875,100.90117187)
\lineto(69.06171875,99.7)
\closepath
}
}
{
\newrgbcolor{curcolor}{0 0 0}
\pscustom[linestyle=none,fillstyle=solid,fillcolor=curcolor]
{
\newpath
\moveto(70.60273437,103.93632812)
\curveto(70.60273437,104.95195312)(70.70625,105.76835937)(70.91328125,106.38554687)
\curveto(71.12421875,107.00664062)(71.43476562,107.48515625)(71.84492187,107.82109375)
\curveto(72.25898437,108.15703125)(72.77851562,108.325)(73.40351562,108.325)
\curveto(73.86445312,108.325)(74.26875,108.23125)(74.61640625,108.04375)
\curveto(74.9640625,107.86015625)(75.25117187,107.59257812)(75.47773437,107.24101562)
\curveto(75.70429687,106.89335937)(75.88203125,106.46757812)(76.0109375,105.96367187)
\curveto(76.13984375,105.46367187)(76.20429687,104.78789062)(76.20429687,103.93632812)
\curveto(76.20429687,102.92851562)(76.10078125,102.1140625)(75.89375,101.49296875)
\curveto(75.68671875,100.87578125)(75.37617187,100.39726562)(74.96210937,100.05742187)
\curveto(74.55195312,99.72148437)(74.03242187,99.55351562)(73.40351562,99.55351562)
\curveto(72.57539062,99.55351562)(71.925,99.85039062)(71.45234375,100.44414062)
\curveto(70.8859375,101.15898437)(70.60273437,102.32304687)(70.60273437,103.93632812)
\closepath
\moveto(71.68671875,103.93632812)
\curveto(71.68671875,102.52617187)(71.85078125,101.58671875)(72.17890625,101.11796875)
\curveto(72.5109375,100.653125)(72.91914062,100.42070312)(73.40351562,100.42070312)
\curveto(73.88789062,100.42070312)(74.29414062,100.65507812)(74.62226562,101.12382812)
\curveto(74.95429687,101.59257812)(75.1203125,102.53007812)(75.1203125,103.93632812)
\curveto(75.1203125,105.35039062)(74.95429687,106.28984375)(74.62226562,106.7546875)
\curveto(74.29414062,107.21953125)(73.88398437,107.45195312)(73.39179687,107.45195312)
\curveto(72.90742187,107.45195312)(72.52070312,107.246875)(72.23164062,106.83671875)
\curveto(71.86835937,106.31328125)(71.68671875,105.34648437)(71.68671875,103.93632812)
\closepath
}
}
{
\newrgbcolor{curcolor}{0 0 0}
\pscustom[linestyle=none,fillstyle=solid,fillcolor=curcolor]
{
\newpath
\moveto(82.74921875,106.18632812)
\lineto(81.70039062,106.10429687)
\curveto(81.60664062,106.51835937)(81.47382812,106.81914062)(81.30195312,107.00664062)
\curveto(81.01679687,107.30742187)(80.66523437,107.4578125)(80.24726562,107.4578125)
\curveto(79.91132812,107.4578125)(79.61640625,107.3640625)(79.3625,107.1765625)
\curveto(79.03046875,106.934375)(78.76875,106.58085937)(78.57734375,106.11601562)
\curveto(78.3859375,105.65117187)(78.28632812,104.9890625)(78.27851562,104.1296875)
\curveto(78.53242187,104.51640625)(78.84296875,104.80351562)(79.21015625,104.99101562)
\curveto(79.57734375,105.17851562)(79.96210937,105.27226562)(80.36445312,105.27226562)
\curveto(81.06757812,105.27226562)(81.66523437,105.0125)(82.15742187,104.49296875)
\curveto(82.65351562,103.97734375)(82.9015625,103.309375)(82.9015625,102.4890625)
\curveto(82.9015625,101.95)(82.784375,101.44804687)(82.55,100.98320312)
\curveto(82.31953125,100.52226562)(82.00117187,100.16875)(81.59492187,99.92265625)
\curveto(81.18867187,99.6765625)(80.72773437,99.55351562)(80.21210937,99.55351562)
\curveto(79.33320312,99.55351562)(78.61640625,99.87578125)(78.06171875,100.5203125)
\curveto(77.50703125,101.16875)(77.2296875,102.23515625)(77.2296875,103.71953125)
\curveto(77.2296875,105.3796875)(77.53632812,106.58671875)(78.14960937,107.340625)
\curveto(78.68476562,107.996875)(79.40546875,108.325)(80.31171875,108.325)
\curveto(80.9875,108.325)(81.54023437,108.13554687)(81.96992187,107.75664062)
\curveto(82.40351562,107.37773437)(82.66328125,106.85429687)(82.74921875,106.18632812)
\closepath
\moveto(78.44257812,102.48320312)
\curveto(78.44257812,102.11992187)(78.51875,101.77226562)(78.67109375,101.44023437)
\curveto(78.82734375,101.10820312)(79.04414062,100.85429687)(79.32148437,100.67851562)
\curveto(79.59882812,100.50664062)(79.88984375,100.42070312)(80.19453125,100.42070312)
\curveto(80.63984375,100.42070312)(81.02265625,100.60039062)(81.34296875,100.95976562)
\curveto(81.66328125,101.31914062)(81.8234375,101.80742187)(81.8234375,102.42460937)
\curveto(81.8234375,103.01835937)(81.66523437,103.48515625)(81.34882812,103.825)
\curveto(81.03242187,104.16875)(80.63398437,104.340625)(80.15351562,104.340625)
\curveto(79.67695312,104.340625)(79.27265625,104.16875)(78.940625,103.825)
\curveto(78.60859375,103.48515625)(78.44257812,103.03789062)(78.44257812,102.48320312)
\closepath
}
}
{
\newrgbcolor{curcolor}{0 0 0}
\pscustom[linestyle=none,fillstyle=solid,fillcolor=curcolor]
{
\newpath
\moveto(89.49335937,100.71367187)
\lineto(89.49335937,99.7)
\lineto(83.815625,99.7)
\curveto(83.8078125,99.95390625)(83.84882812,100.19804687)(83.93867187,100.43242187)
\curveto(84.08320312,100.81914062)(84.31367187,101.2)(84.63007812,101.575)
\curveto(84.95039062,101.95)(85.41132812,102.38359375)(86.01289062,102.87578125)
\curveto(86.94648437,103.64140625)(87.57734375,104.246875)(87.90546875,104.6921875)
\curveto(88.23359375,105.14140625)(88.39765625,105.56523437)(88.39765625,105.96367187)
\curveto(88.39765625,106.38164062)(88.24726562,106.73320312)(87.94648437,107.01835937)
\curveto(87.64960937,107.30742187)(87.2609375,107.45195312)(86.78046875,107.45195312)
\curveto(86.27265625,107.45195312)(85.86640625,107.29960937)(85.56171875,106.99492187)
\curveto(85.25703125,106.69023437)(85.10273437,106.26835937)(85.09882812,105.72929687)
\lineto(84.01484375,105.840625)
\curveto(84.0890625,106.64921875)(84.36835937,107.26445312)(84.85273437,107.68632812)
\curveto(85.33710937,108.11210937)(85.9875,108.325)(86.80390625,108.325)
\curveto(87.628125,108.325)(88.28046875,108.09648437)(88.7609375,107.63945312)
\curveto(89.24140625,107.18242187)(89.48164062,106.61601562)(89.48164062,105.94023437)
\curveto(89.48164062,105.59648437)(89.41132812,105.25859375)(89.27070312,104.9265625)
\curveto(89.13007812,104.59453125)(88.89570312,104.24492187)(88.56757812,103.87773437)
\curveto(88.24335937,103.51054687)(87.70234375,103.00664062)(86.94453125,102.36601562)
\curveto(86.31171875,101.83476562)(85.90546875,101.4734375)(85.72578125,101.28203125)
\curveto(85.54609375,101.09453125)(85.39765625,100.90507812)(85.28046875,100.71367187)
\closepath
}
}
{
\newrgbcolor{curcolor}{0 0 0}
\pscustom[linestyle=none,fillstyle=solid,fillcolor=curcolor]
{
\newpath
\moveto(90.62421875,101.95)
\lineto(91.73164062,102.04375)
\curveto(91.81367187,101.5046875)(92.003125,101.0984375)(92.3,100.825)
\curveto(92.60078125,100.55546875)(92.96210937,100.42070312)(93.38398437,100.42070312)
\curveto(93.89179687,100.42070312)(94.32148437,100.61210937)(94.67304687,100.99492187)
\curveto(95.02460937,101.37773437)(95.20039062,101.88554687)(95.20039062,102.51835937)
\curveto(95.20039062,103.11992187)(95.03046875,103.59453125)(94.690625,103.9421875)
\curveto(94.3546875,104.28984375)(93.91328125,104.46367187)(93.36640625,104.46367187)
\curveto(93.0265625,104.46367187)(92.71992187,104.38554687)(92.44648437,104.22929687)
\curveto(92.17304687,104.07695312)(91.95820312,103.87773437)(91.80195312,103.63164062)
\lineto(90.81171875,103.76054687)
\lineto(91.64375,108.17265625)
\lineto(95.91523437,108.17265625)
\lineto(95.91523437,107.16484375)
\lineto(92.4875,107.16484375)
\lineto(92.02460937,104.85625)
\curveto(92.54023437,105.215625)(93.08125,105.3953125)(93.64765625,105.3953125)
\curveto(94.39765625,105.3953125)(95.03046875,105.13554687)(95.54609375,104.61601562)
\curveto(96.06171875,104.09648437)(96.31953125,103.42851562)(96.31953125,102.61210937)
\curveto(96.31953125,101.83476562)(96.09296875,101.16289062)(95.63984375,100.59648437)
\curveto(95.0890625,99.90117187)(94.33710937,99.55351562)(93.38398437,99.55351562)
\curveto(92.60273437,99.55351562)(91.9640625,99.77226562)(91.46796875,100.20976562)
\curveto(90.97578125,100.64726562)(90.69453125,101.22734375)(90.62421875,101.95)
\closepath
}
}
{
\newrgbcolor{curcolor}{0 0 0}
\pscustom[linewidth=1,linecolor=curcolor]
{
\newpath
\moveto(105.1,149.7)
\lineto(114.1,149.7)
\moveto(575,149.7)
\lineto(566,149.7)
}
}
{
\newrgbcolor{curcolor}{0 0 0}
\pscustom[linestyle=none,fillstyle=solid,fillcolor=curcolor]
{
\newpath
\moveto(67.26875,150.03632813)
\curveto(67.26875,151.05195313)(67.37226562,151.86835938)(67.57929687,152.48554688)
\curveto(67.79023437,153.10664063)(68.10078125,153.58515625)(68.5109375,153.92109375)
\curveto(68.925,154.25703125)(69.44453125,154.425)(70.06953125,154.425)
\curveto(70.53046875,154.425)(70.93476562,154.33125)(71.28242187,154.14375)
\curveto(71.63007812,153.96015625)(71.9171875,153.69257813)(72.14375,153.34101563)
\curveto(72.3703125,152.99335938)(72.54804687,152.56757813)(72.67695312,152.06367188)
\curveto(72.80585937,151.56367188)(72.8703125,150.88789063)(72.8703125,150.03632813)
\curveto(72.8703125,149.02851563)(72.76679687,148.2140625)(72.55976562,147.59296875)
\curveto(72.35273437,146.97578125)(72.0421875,146.49726563)(71.628125,146.15742188)
\curveto(71.21796875,145.82148438)(70.6984375,145.65351563)(70.06953125,145.65351563)
\curveto(69.24140625,145.65351563)(68.59101562,145.95039063)(68.11835937,146.54414063)
\curveto(67.55195312,147.25898438)(67.26875,148.42304688)(67.26875,150.03632813)
\closepath
\moveto(68.35273437,150.03632813)
\curveto(68.35273437,148.62617188)(68.51679687,147.68671875)(68.84492187,147.21796875)
\curveto(69.17695312,146.753125)(69.58515625,146.52070313)(70.06953125,146.52070313)
\curveto(70.55390625,146.52070313)(70.96015625,146.75507813)(71.28828125,147.22382813)
\curveto(71.6203125,147.69257813)(71.78632812,148.63007813)(71.78632812,150.03632813)
\curveto(71.78632812,151.45039063)(71.6203125,152.38984375)(71.28828125,152.8546875)
\curveto(70.96015625,153.31953125)(70.55,153.55195313)(70.0578125,153.55195313)
\curveto(69.5734375,153.55195313)(69.18671875,153.346875)(68.89765625,152.93671875)
\curveto(68.534375,152.41328125)(68.35273437,151.44648438)(68.35273437,150.03632813)
\closepath
}
}
{
\newrgbcolor{curcolor}{0 0 0}
\pscustom[linestyle=none,fillstyle=solid,fillcolor=curcolor]
{
\newpath
\moveto(74.534375,145.8)
\lineto(74.534375,147.00117188)
\lineto(75.73554687,147.00117188)
\lineto(75.73554687,145.8)
\closepath
}
}
{
\newrgbcolor{curcolor}{0 0 0}
\pscustom[linestyle=none,fillstyle=solid,fillcolor=curcolor]
{
\newpath
\moveto(81.24921875,145.8)
\lineto(80.19453125,145.8)
\lineto(80.19453125,152.52070313)
\curveto(79.940625,152.27851563)(79.60664062,152.03632813)(79.19257812,151.79414063)
\curveto(78.78242187,151.55195313)(78.41328125,151.3703125)(78.08515625,151.24921875)
\lineto(78.08515625,152.26875)
\curveto(78.675,152.54609375)(79.190625,152.88203125)(79.63203125,153.2765625)
\curveto(80.0734375,153.67109375)(80.3859375,154.05390625)(80.56953125,154.425)
\lineto(81.24921875,154.425)
\closepath
}
}
{
\newrgbcolor{curcolor}{0 0 0}
\pscustom[linestyle=none,fillstyle=solid,fillcolor=curcolor]
{
\newpath
\moveto(89.49335937,146.81367188)
\lineto(89.49335937,145.8)
\lineto(83.815625,145.8)
\curveto(83.8078125,146.05390625)(83.84882812,146.29804688)(83.93867187,146.53242188)
\curveto(84.08320312,146.91914063)(84.31367187,147.3)(84.63007812,147.675)
\curveto(84.95039062,148.05)(85.41132812,148.48359375)(86.01289062,148.97578125)
\curveto(86.94648437,149.74140625)(87.57734375,150.346875)(87.90546875,150.7921875)
\curveto(88.23359375,151.24140625)(88.39765625,151.66523438)(88.39765625,152.06367188)
\curveto(88.39765625,152.48164063)(88.24726562,152.83320313)(87.94648437,153.11835938)
\curveto(87.64960937,153.40742188)(87.2609375,153.55195313)(86.78046875,153.55195313)
\curveto(86.27265625,153.55195313)(85.86640625,153.39960938)(85.56171875,153.09492188)
\curveto(85.25703125,152.79023438)(85.10273437,152.36835938)(85.09882812,151.82929688)
\lineto(84.01484375,151.940625)
\curveto(84.0890625,152.74921875)(84.36835937,153.36445313)(84.85273437,153.78632813)
\curveto(85.33710937,154.21210938)(85.9875,154.425)(86.80390625,154.425)
\curveto(87.628125,154.425)(88.28046875,154.19648438)(88.7609375,153.73945313)
\curveto(89.24140625,153.28242188)(89.48164062,152.71601563)(89.48164062,152.04023438)
\curveto(89.48164062,151.69648438)(89.41132812,151.35859375)(89.27070312,151.0265625)
\curveto(89.13007812,150.69453125)(88.89570312,150.34492188)(88.56757812,149.97773438)
\curveto(88.24335937,149.61054688)(87.70234375,149.10664063)(86.94453125,148.46601563)
\curveto(86.31171875,147.93476563)(85.90546875,147.5734375)(85.72578125,147.38203125)
\curveto(85.54609375,147.19453125)(85.39765625,147.00507813)(85.28046875,146.81367188)
\closepath
}
}
{
\newrgbcolor{curcolor}{0 0 0}
\pscustom[linestyle=none,fillstyle=solid,fillcolor=curcolor]
{
\newpath
\moveto(90.62421875,148.05)
\lineto(91.73164062,148.14375)
\curveto(91.81367187,147.6046875)(92.003125,147.1984375)(92.3,146.925)
\curveto(92.60078125,146.65546875)(92.96210937,146.52070313)(93.38398437,146.52070313)
\curveto(93.89179687,146.52070313)(94.32148437,146.71210938)(94.67304687,147.09492188)
\curveto(95.02460937,147.47773438)(95.20039062,147.98554688)(95.20039062,148.61835938)
\curveto(95.20039062,149.21992188)(95.03046875,149.69453125)(94.690625,150.0421875)
\curveto(94.3546875,150.38984375)(93.91328125,150.56367188)(93.36640625,150.56367188)
\curveto(93.0265625,150.56367188)(92.71992187,150.48554688)(92.44648437,150.32929688)
\curveto(92.17304687,150.17695313)(91.95820312,149.97773438)(91.80195312,149.73164063)
\lineto(90.81171875,149.86054688)
\lineto(91.64375,154.27265625)
\lineto(95.91523437,154.27265625)
\lineto(95.91523437,153.26484375)
\lineto(92.4875,153.26484375)
\lineto(92.02460937,150.95625)
\curveto(92.54023437,151.315625)(93.08125,151.4953125)(93.64765625,151.4953125)
\curveto(94.39765625,151.4953125)(95.03046875,151.23554688)(95.54609375,150.71601563)
\curveto(96.06171875,150.19648438)(96.31953125,149.52851563)(96.31953125,148.71210938)
\curveto(96.31953125,147.93476563)(96.09296875,147.26289063)(95.63984375,146.69648438)
\curveto(95.0890625,146.00117188)(94.33710937,145.65351563)(93.38398437,145.65351563)
\curveto(92.60273437,145.65351563)(91.9640625,145.87226563)(91.46796875,146.30976563)
\curveto(90.97578125,146.74726563)(90.69453125,147.32734375)(90.62421875,148.05)
\closepath
}
}
{
\newrgbcolor{curcolor}{0 0 0}
\pscustom[linewidth=1,linecolor=curcolor]
{
\newpath
\moveto(105.1,195.7)
\lineto(114.1,195.7)
\moveto(575,195.7)
\lineto(566,195.7)
}
}
{
\newrgbcolor{curcolor}{0 0 0}
\pscustom[linestyle=none,fillstyle=solid,fillcolor=curcolor]
{
\newpath
\moveto(73.94257812,196.03632813)
\curveto(73.94257812,197.05195313)(74.04609375,197.86835938)(74.253125,198.48554688)
\curveto(74.4640625,199.10664063)(74.77460937,199.58515625)(75.18476562,199.92109375)
\curveto(75.59882812,200.25703125)(76.11835937,200.425)(76.74335937,200.425)
\curveto(77.20429687,200.425)(77.60859375,200.33125)(77.95625,200.14375)
\curveto(78.30390625,199.96015625)(78.59101562,199.69257813)(78.81757812,199.34101563)
\curveto(79.04414062,198.99335938)(79.221875,198.56757813)(79.35078125,198.06367188)
\curveto(79.4796875,197.56367188)(79.54414062,196.88789063)(79.54414062,196.03632813)
\curveto(79.54414062,195.02851563)(79.440625,194.2140625)(79.23359375,193.59296875)
\curveto(79.0265625,192.97578125)(78.71601562,192.49726563)(78.30195312,192.15742188)
\curveto(77.89179687,191.82148438)(77.37226562,191.65351563)(76.74335937,191.65351563)
\curveto(75.91523437,191.65351563)(75.26484375,191.95039063)(74.7921875,192.54414063)
\curveto(74.22578125,193.25898438)(73.94257812,194.42304688)(73.94257812,196.03632813)
\closepath
\moveto(75.0265625,196.03632813)
\curveto(75.0265625,194.62617188)(75.190625,193.68671875)(75.51875,193.21796875)
\curveto(75.85078125,192.753125)(76.25898437,192.52070313)(76.74335937,192.52070313)
\curveto(77.22773437,192.52070313)(77.63398437,192.75507813)(77.96210937,193.22382813)
\curveto(78.29414062,193.69257813)(78.46015625,194.63007813)(78.46015625,196.03632813)
\curveto(78.46015625,197.45039063)(78.29414062,198.38984375)(77.96210937,198.8546875)
\curveto(77.63398437,199.31953125)(77.22382812,199.55195313)(76.73164062,199.55195313)
\curveto(76.24726562,199.55195313)(75.86054687,199.346875)(75.57148437,198.93671875)
\curveto(75.20820312,198.41328125)(75.0265625,197.44648438)(75.0265625,196.03632813)
\closepath
}
}
{
\newrgbcolor{curcolor}{0 0 0}
\pscustom[linestyle=none,fillstyle=solid,fillcolor=curcolor]
{
\newpath
\moveto(81.20820312,191.8)
\lineto(81.20820312,193.00117188)
\lineto(82.409375,193.00117188)
\lineto(82.409375,191.8)
\closepath
}
}
{
\newrgbcolor{curcolor}{0 0 0}
\pscustom[linestyle=none,fillstyle=solid,fillcolor=curcolor]
{
\newpath
\moveto(89.49335937,192.81367188)
\lineto(89.49335937,191.8)
\lineto(83.815625,191.8)
\curveto(83.8078125,192.05390625)(83.84882812,192.29804688)(83.93867187,192.53242188)
\curveto(84.08320312,192.91914063)(84.31367187,193.3)(84.63007812,193.675)
\curveto(84.95039062,194.05)(85.41132812,194.48359375)(86.01289062,194.97578125)
\curveto(86.94648437,195.74140625)(87.57734375,196.346875)(87.90546875,196.7921875)
\curveto(88.23359375,197.24140625)(88.39765625,197.66523438)(88.39765625,198.06367188)
\curveto(88.39765625,198.48164063)(88.24726562,198.83320313)(87.94648437,199.11835938)
\curveto(87.64960937,199.40742188)(87.2609375,199.55195313)(86.78046875,199.55195313)
\curveto(86.27265625,199.55195313)(85.86640625,199.39960938)(85.56171875,199.09492188)
\curveto(85.25703125,198.79023438)(85.10273437,198.36835938)(85.09882812,197.82929688)
\lineto(84.01484375,197.940625)
\curveto(84.0890625,198.74921875)(84.36835937,199.36445313)(84.85273437,199.78632813)
\curveto(85.33710937,200.21210938)(85.9875,200.425)(86.80390625,200.425)
\curveto(87.628125,200.425)(88.28046875,200.19648438)(88.7609375,199.73945313)
\curveto(89.24140625,199.28242188)(89.48164062,198.71601563)(89.48164062,198.04023438)
\curveto(89.48164062,197.69648438)(89.41132812,197.35859375)(89.27070312,197.0265625)
\curveto(89.13007812,196.69453125)(88.89570312,196.34492188)(88.56757812,195.97773438)
\curveto(88.24335937,195.61054688)(87.70234375,195.10664063)(86.94453125,194.46601563)
\curveto(86.31171875,193.93476563)(85.90546875,193.5734375)(85.72578125,193.38203125)
\curveto(85.54609375,193.19453125)(85.39765625,193.00507813)(85.28046875,192.81367188)
\closepath
}
}
{
\newrgbcolor{curcolor}{0 0 0}
\pscustom[linestyle=none,fillstyle=solid,fillcolor=curcolor]
{
\newpath
\moveto(90.62421875,194.05)
\lineto(91.73164062,194.14375)
\curveto(91.81367187,193.6046875)(92.003125,193.1984375)(92.3,192.925)
\curveto(92.60078125,192.65546875)(92.96210937,192.52070313)(93.38398437,192.52070313)
\curveto(93.89179687,192.52070313)(94.32148437,192.71210938)(94.67304687,193.09492188)
\curveto(95.02460937,193.47773438)(95.20039062,193.98554688)(95.20039062,194.61835938)
\curveto(95.20039062,195.21992188)(95.03046875,195.69453125)(94.690625,196.0421875)
\curveto(94.3546875,196.38984375)(93.91328125,196.56367188)(93.36640625,196.56367188)
\curveto(93.0265625,196.56367188)(92.71992187,196.48554688)(92.44648437,196.32929688)
\curveto(92.17304687,196.17695313)(91.95820312,195.97773438)(91.80195312,195.73164063)
\lineto(90.81171875,195.86054688)
\lineto(91.64375,200.27265625)
\lineto(95.91523437,200.27265625)
\lineto(95.91523437,199.26484375)
\lineto(92.4875,199.26484375)
\lineto(92.02460937,196.95625)
\curveto(92.54023437,197.315625)(93.08125,197.4953125)(93.64765625,197.4953125)
\curveto(94.39765625,197.4953125)(95.03046875,197.23554688)(95.54609375,196.71601563)
\curveto(96.06171875,196.19648438)(96.31953125,195.52851563)(96.31953125,194.71210938)
\curveto(96.31953125,193.93476563)(96.09296875,193.26289063)(95.63984375,192.69648438)
\curveto(95.0890625,192.00117188)(94.33710937,191.65351563)(93.38398437,191.65351563)
\curveto(92.60273437,191.65351563)(91.9640625,191.87226563)(91.46796875,192.30976563)
\curveto(90.97578125,192.74726563)(90.69453125,193.32734375)(90.62421875,194.05)
\closepath
}
}
{
\newrgbcolor{curcolor}{0 0 0}
\pscustom[linewidth=1,linecolor=curcolor]
{
\newpath
\moveto(105.1,241.8)
\lineto(114.1,241.8)
\moveto(575,241.8)
\lineto(566,241.8)
}
}
{
\newrgbcolor{curcolor}{0 0 0}
\pscustom[linestyle=none,fillstyle=solid,fillcolor=curcolor]
{
\newpath
\moveto(80.61640625,242.13632813)
\curveto(80.61640625,243.15195313)(80.71992187,243.96835938)(80.92695312,244.58554688)
\curveto(81.13789062,245.20664063)(81.4484375,245.68515625)(81.85859375,246.02109375)
\curveto(82.27265625,246.35703125)(82.7921875,246.525)(83.4171875,246.525)
\curveto(83.878125,246.525)(84.28242187,246.43125)(84.63007812,246.24375)
\curveto(84.97773437,246.06015625)(85.26484375,245.79257813)(85.49140625,245.44101563)
\curveto(85.71796875,245.09335938)(85.89570312,244.66757813)(86.02460937,244.16367188)
\curveto(86.15351562,243.66367188)(86.21796875,242.98789063)(86.21796875,242.13632813)
\curveto(86.21796875,241.12851563)(86.11445312,240.3140625)(85.90742187,239.69296875)
\curveto(85.70039062,239.07578125)(85.38984375,238.59726563)(84.97578125,238.25742188)
\curveto(84.565625,237.92148438)(84.04609375,237.75351563)(83.4171875,237.75351563)
\curveto(82.5890625,237.75351563)(81.93867187,238.05039063)(81.46601562,238.64414063)
\curveto(80.89960937,239.35898438)(80.61640625,240.52304688)(80.61640625,242.13632813)
\closepath
\moveto(81.70039062,242.13632813)
\curveto(81.70039062,240.72617188)(81.86445312,239.78671875)(82.19257812,239.31796875)
\curveto(82.52460937,238.853125)(82.9328125,238.62070313)(83.4171875,238.62070313)
\curveto(83.9015625,238.62070313)(84.3078125,238.85507813)(84.6359375,239.32382813)
\curveto(84.96796875,239.79257813)(85.13398437,240.73007813)(85.13398437,242.13632813)
\curveto(85.13398437,243.55039063)(84.96796875,244.48984375)(84.6359375,244.9546875)
\curveto(84.3078125,245.41953125)(83.89765625,245.65195313)(83.40546875,245.65195313)
\curveto(82.92109375,245.65195313)(82.534375,245.446875)(82.2453125,245.03671875)
\curveto(81.88203125,244.51328125)(81.70039062,243.54648438)(81.70039062,242.13632813)
\closepath
}
}
{
\newrgbcolor{curcolor}{0 0 0}
\pscustom[linestyle=none,fillstyle=solid,fillcolor=curcolor]
{
\newpath
\moveto(87.88203125,237.9)
\lineto(87.88203125,239.10117188)
\lineto(89.08320312,239.10117188)
\lineto(89.08320312,237.9)
\closepath
}
}
{
\newrgbcolor{curcolor}{0 0 0}
\pscustom[linestyle=none,fillstyle=solid,fillcolor=curcolor]
{
\newpath
\moveto(90.62421875,240.15)
\lineto(91.73164062,240.24375)
\curveto(91.81367187,239.7046875)(92.003125,239.2984375)(92.3,239.025)
\curveto(92.60078125,238.75546875)(92.96210937,238.62070313)(93.38398437,238.62070313)
\curveto(93.89179687,238.62070313)(94.32148437,238.81210938)(94.67304687,239.19492188)
\curveto(95.02460937,239.57773438)(95.20039062,240.08554688)(95.20039062,240.71835938)
\curveto(95.20039062,241.31992188)(95.03046875,241.79453125)(94.690625,242.1421875)
\curveto(94.3546875,242.48984375)(93.91328125,242.66367188)(93.36640625,242.66367188)
\curveto(93.0265625,242.66367188)(92.71992187,242.58554688)(92.44648437,242.42929688)
\curveto(92.17304687,242.27695313)(91.95820312,242.07773438)(91.80195312,241.83164063)
\lineto(90.81171875,241.96054688)
\lineto(91.64375,246.37265625)
\lineto(95.91523437,246.37265625)
\lineto(95.91523437,245.36484375)
\lineto(92.4875,245.36484375)
\lineto(92.02460937,243.05625)
\curveto(92.54023437,243.415625)(93.08125,243.5953125)(93.64765625,243.5953125)
\curveto(94.39765625,243.5953125)(95.03046875,243.33554688)(95.54609375,242.81601563)
\curveto(96.06171875,242.29648438)(96.31953125,241.62851563)(96.31953125,240.81210938)
\curveto(96.31953125,240.03476563)(96.09296875,239.36289063)(95.63984375,238.79648438)
\curveto(95.0890625,238.10117188)(94.33710937,237.75351563)(93.38398437,237.75351563)
\curveto(92.60273437,237.75351563)(91.9640625,237.97226563)(91.46796875,238.40976563)
\curveto(90.97578125,238.84726563)(90.69453125,239.42734375)(90.62421875,240.15)
\closepath
}
}
{
\newrgbcolor{curcolor}{0 0 0}
\pscustom[linewidth=1,linecolor=curcolor]
{
\newpath
\moveto(105.1,287.8)
\lineto(114.1,287.8)
\moveto(575,287.8)
\lineto(566,287.8)
}
}
{
\newrgbcolor{curcolor}{0 0 0}
\pscustom[linestyle=none,fillstyle=solid,fillcolor=curcolor]
{
\newpath
\moveto(94.596875,283.9)
\lineto(93.5421875,283.9)
\lineto(93.5421875,290.62070312)
\curveto(93.28828125,290.37851562)(92.95429687,290.13632812)(92.54023437,289.89414062)
\curveto(92.13007812,289.65195312)(91.7609375,289.4703125)(91.4328125,289.34921875)
\lineto(91.4328125,290.36875)
\curveto(92.02265625,290.64609375)(92.53828125,290.98203125)(92.9796875,291.3765625)
\curveto(93.42109375,291.77109375)(93.73359375,292.15390625)(93.9171875,292.525)
\lineto(94.596875,292.525)
\closepath
}
}
{
\newrgbcolor{curcolor}{0 0 0}
\pscustom[linewidth=1,linecolor=curcolor]
{
\newpath
\moveto(105.1,333.8)
\lineto(114.1,333.8)
\moveto(575,333.8)
\lineto(566,333.8)
}
}
{
\newrgbcolor{curcolor}{0 0 0}
\pscustom[linestyle=none,fillstyle=solid,fillcolor=curcolor]
{
\newpath
\moveto(96.1671875,330.91367187)
\lineto(96.1671875,329.9)
\lineto(90.48945312,329.9)
\curveto(90.48164062,330.15390625)(90.52265625,330.39804687)(90.6125,330.63242187)
\curveto(90.75703125,331.01914062)(90.9875,331.4)(91.30390625,331.775)
\curveto(91.62421875,332.15)(92.08515625,332.58359375)(92.68671875,333.07578125)
\curveto(93.6203125,333.84140625)(94.25117187,334.446875)(94.57929687,334.8921875)
\curveto(94.90742187,335.34140625)(95.07148437,335.76523437)(95.07148437,336.16367187)
\curveto(95.07148437,336.58164062)(94.92109375,336.93320312)(94.6203125,337.21835937)
\curveto(94.3234375,337.50742187)(93.93476562,337.65195312)(93.45429687,337.65195312)
\curveto(92.94648437,337.65195312)(92.54023437,337.49960937)(92.23554687,337.19492187)
\curveto(91.93085937,336.89023437)(91.7765625,336.46835937)(91.77265625,335.92929687)
\lineto(90.68867187,336.040625)
\curveto(90.76289062,336.84921875)(91.0421875,337.46445312)(91.5265625,337.88632812)
\curveto(92.0109375,338.31210937)(92.66132812,338.525)(93.47773437,338.525)
\curveto(94.30195312,338.525)(94.95429687,338.29648437)(95.43476562,337.83945312)
\curveto(95.91523437,337.38242187)(96.15546875,336.81601562)(96.15546875,336.14023437)
\curveto(96.15546875,335.79648437)(96.08515625,335.45859375)(95.94453125,335.1265625)
\curveto(95.80390625,334.79453125)(95.56953125,334.44492187)(95.24140625,334.07773437)
\curveto(94.9171875,333.71054687)(94.37617187,333.20664062)(93.61835937,332.56601562)
\curveto(92.98554687,332.03476562)(92.57929687,331.6734375)(92.39960937,331.48203125)
\curveto(92.21992187,331.29453125)(92.07148437,331.10507812)(91.95429687,330.91367187)
\closepath
}
}
{
\newrgbcolor{curcolor}{0 0 0}
\pscustom[linewidth=1,linecolor=curcolor]
{
\newpath
\moveto(105.1,379.9)
\lineto(114.1,379.9)
\moveto(575,379.9)
\lineto(566,379.9)
}
}
{
\newrgbcolor{curcolor}{0 0 0}
\pscustom[linestyle=none,fillstyle=solid,fillcolor=curcolor]
{
\newpath
\moveto(94.00507812,376)
\lineto(94.00507812,378.05664062)
\lineto(90.27851562,378.05664062)
\lineto(90.27851562,379.0234375)
\lineto(94.1984375,384.58984375)
\lineto(95.05976562,384.58984375)
\lineto(95.05976562,379.0234375)
\lineto(96.21992187,379.0234375)
\lineto(96.21992187,378.05664062)
\lineto(95.05976562,378.05664062)
\lineto(95.05976562,376)
\closepath
\moveto(94.00507812,379.0234375)
\lineto(94.00507812,382.89648438)
\lineto(91.315625,379.0234375)
\closepath
}
}
{
\newrgbcolor{curcolor}{0 0 0}
\pscustom[linewidth=1,linecolor=curcolor]
{
\newpath
\moveto(105.1,425.9)
\lineto(114.1,425.9)
\moveto(575,425.9)
\lineto(566,425.9)
}
}
{
\newrgbcolor{curcolor}{0 0 0}
\pscustom[linestyle=none,fillstyle=solid,fillcolor=curcolor]
{
\newpath
\moveto(92.24726562,426.65820312)
\curveto(91.80976562,426.81835938)(91.48554687,427.046875)(91.27460937,427.34375)
\curveto(91.06367187,427.640625)(90.95820312,427.99609375)(90.95820312,428.41015625)
\curveto(90.95820312,429.03515625)(91.1828125,429.56054688)(91.63203125,429.98632812)
\curveto(92.08125,430.41210938)(92.67890625,430.625)(93.425,430.625)
\curveto(94.175,430.625)(94.77851562,430.40625)(95.23554687,429.96875)
\curveto(95.69257812,429.53515625)(95.92109375,429.00585938)(95.92109375,428.38085938)
\curveto(95.92109375,427.98242188)(95.815625,427.63476562)(95.6046875,427.33789062)
\curveto(95.39765625,427.04492188)(95.08125,426.81835938)(94.65546875,426.65820312)
\curveto(95.1828125,426.48632812)(95.58320312,426.20898438)(95.85664062,425.82617188)
\curveto(96.13398437,425.44335938)(96.27265625,424.98632812)(96.27265625,424.45507812)
\curveto(96.27265625,423.72070312)(96.01289062,423.10351562)(95.49335937,422.60351562)
\curveto(94.97382812,422.10351562)(94.29023437,421.85351562)(93.44257812,421.85351562)
\curveto(92.59492187,421.85351562)(91.91132812,422.10351562)(91.39179687,422.60351562)
\curveto(90.87226562,423.10742188)(90.6125,423.734375)(90.6125,424.484375)
\curveto(90.6125,425.04296875)(90.753125,425.50976562)(91.034375,425.88476562)
\curveto(91.31953125,426.26367188)(91.72382812,426.52148438)(92.24726562,426.65820312)
\closepath
\moveto(92.03632812,428.4453125)
\curveto(92.03632812,428.0390625)(92.1671875,427.70703125)(92.42890625,427.44921875)
\curveto(92.690625,427.19140625)(93.03046875,427.0625)(93.4484375,427.0625)
\curveto(93.8546875,427.0625)(94.18671875,427.18945312)(94.44453125,427.44335938)
\curveto(94.70625,427.70117188)(94.83710937,428.015625)(94.83710937,428.38671875)
\curveto(94.83710937,428.7734375)(94.70234375,429.09765625)(94.4328125,429.359375)
\curveto(94.1671875,429.625)(93.83515625,429.7578125)(93.43671875,429.7578125)
\curveto(93.034375,429.7578125)(92.70039062,429.62890625)(92.43476562,429.37109375)
\curveto(92.16914062,429.11328125)(92.03632812,428.8046875)(92.03632812,428.4453125)
\closepath
\moveto(91.69648437,424.47851562)
\curveto(91.69648437,424.17773438)(91.76679687,423.88671875)(91.90742187,423.60546875)
\curveto(92.05195312,423.32421875)(92.26484375,423.10546875)(92.54609375,422.94921875)
\curveto(92.82734375,422.796875)(93.13007812,422.72070312)(93.45429687,422.72070312)
\curveto(93.95820312,422.72070312)(94.37421875,422.8828125)(94.70234375,423.20703125)
\curveto(95.03046875,423.53125)(95.19453125,423.94335938)(95.19453125,424.44335938)
\curveto(95.19453125,424.95117188)(95.02460937,425.37109375)(94.68476562,425.703125)
\curveto(94.34882812,426.03515625)(93.92695312,426.20117188)(93.41914062,426.20117188)
\curveto(92.92304687,426.20117188)(92.5109375,426.03710938)(92.1828125,425.70898438)
\curveto(91.85859375,425.38085938)(91.69648437,424.97070312)(91.69648437,424.47851562)
\closepath
}
}
{
\newrgbcolor{curcolor}{0 0 0}
\pscustom[linewidth=1,linecolor=curcolor]
{
\newpath
\moveto(105.1,57.6)
\lineto(105.1,66.6)
\moveto(105.1,425.9)
\lineto(105.1,416.9)
}
}
{
\newrgbcolor{curcolor}{0 0 0}
\pscustom[linestyle=none,fillstyle=solid,fillcolor=curcolor]
{
\newpath
\moveto(96.13222656,36.71367187)
\lineto(96.13222656,35.7)
\lineto(90.45449219,35.7)
\curveto(90.44667969,35.95390625)(90.48769531,36.19804687)(90.57753906,36.43242187)
\curveto(90.72207031,36.81914062)(90.95253906,37.2)(91.26894531,37.575)
\curveto(91.58925781,37.95)(92.05019531,38.38359375)(92.65175781,38.87578125)
\curveto(93.58535156,39.64140625)(94.21621094,40.246875)(94.54433594,40.6921875)
\curveto(94.87246094,41.14140625)(95.03652344,41.56523437)(95.03652344,41.96367187)
\curveto(95.03652344,42.38164062)(94.88613281,42.73320312)(94.58535156,43.01835937)
\curveto(94.28847656,43.30742187)(93.89980469,43.45195312)(93.41933594,43.45195312)
\curveto(92.91152344,43.45195312)(92.50527344,43.29960937)(92.20058594,42.99492187)
\curveto(91.89589844,42.69023437)(91.74160156,42.26835937)(91.73769531,41.72929687)
\lineto(90.65371094,41.840625)
\curveto(90.72792969,42.64921875)(91.00722656,43.26445312)(91.49160156,43.68632812)
\curveto(91.97597656,44.11210937)(92.62636719,44.325)(93.44277344,44.325)
\curveto(94.26699219,44.325)(94.91933594,44.09648437)(95.39980469,43.63945312)
\curveto(95.88027344,43.18242187)(96.12050781,42.61601562)(96.12050781,41.94023437)
\curveto(96.12050781,41.59648437)(96.05019531,41.25859375)(95.90957031,40.9265625)
\curveto(95.76894531,40.59453125)(95.53457031,40.24492187)(95.20644531,39.87773437)
\curveto(94.88222656,39.51054687)(94.34121094,39.00664062)(93.58339844,38.36601562)
\curveto(92.95058594,37.83476562)(92.54433594,37.4734375)(92.36464844,37.28203125)
\curveto(92.18496094,37.09453125)(92.03652344,36.90507812)(91.91933594,36.71367187)
\closepath
}
}
{
\newrgbcolor{curcolor}{0 0 0}
\pscustom[linestyle=none,fillstyle=solid,fillcolor=curcolor]
{
\newpath
\moveto(97.26308594,37.95)
\lineto(98.37050781,38.04375)
\curveto(98.45253906,37.5046875)(98.64199219,37.0984375)(98.93886719,36.825)
\curveto(99.23964844,36.55546875)(99.60097656,36.42070312)(100.02285156,36.42070312)
\curveto(100.53066406,36.42070312)(100.96035156,36.61210937)(101.31191406,36.99492187)
\curveto(101.66347656,37.37773437)(101.83925781,37.88554687)(101.83925781,38.51835937)
\curveto(101.83925781,39.11992187)(101.66933594,39.59453125)(101.32949219,39.9421875)
\curveto(100.99355469,40.28984375)(100.55214844,40.46367187)(100.00527344,40.46367187)
\curveto(99.66542969,40.46367187)(99.35878906,40.38554687)(99.08535156,40.22929687)
\curveto(98.81191406,40.07695312)(98.59707031,39.87773437)(98.44082031,39.63164062)
\lineto(97.45058594,39.76054687)
\lineto(98.28261719,44.17265625)
\lineto(102.55410156,44.17265625)
\lineto(102.55410156,43.16484375)
\lineto(99.12636719,43.16484375)
\lineto(98.66347656,40.85625)
\curveto(99.17910156,41.215625)(99.72011719,41.3953125)(100.28652344,41.3953125)
\curveto(101.03652344,41.3953125)(101.66933594,41.13554687)(102.18496094,40.61601562)
\curveto(102.70058594,40.09648437)(102.95839844,39.42851562)(102.95839844,38.61210937)
\curveto(102.95839844,37.83476562)(102.73183594,37.16289062)(102.27871094,36.59648437)
\curveto(101.72792969,35.90117187)(100.97597656,35.55351562)(100.02285156,35.55351562)
\curveto(99.24160156,35.55351562)(98.60292969,35.77226562)(98.10683594,36.20976562)
\curveto(97.61464844,36.64726562)(97.33339844,37.22734375)(97.26308594,37.95)
\closepath
}
}
{
\newrgbcolor{curcolor}{0 0 0}
\pscustom[linestyle=none,fillstyle=solid,fillcolor=curcolor]
{
\newpath
\moveto(103.93691406,39.93632812)
\curveto(103.93691406,40.95195312)(104.04042969,41.76835937)(104.24746094,42.38554687)
\curveto(104.45839844,43.00664062)(104.76894531,43.48515625)(105.17910156,43.82109375)
\curveto(105.59316406,44.15703125)(106.11269531,44.325)(106.73769531,44.325)
\curveto(107.19863281,44.325)(107.60292969,44.23125)(107.95058594,44.04375)
\curveto(108.29824219,43.86015625)(108.58535156,43.59257812)(108.81191406,43.24101562)
\curveto(109.03847656,42.89335937)(109.21621094,42.46757812)(109.34511719,41.96367187)
\curveto(109.47402344,41.46367187)(109.53847656,40.78789062)(109.53847656,39.93632812)
\curveto(109.53847656,38.92851562)(109.43496094,38.1140625)(109.22792969,37.49296875)
\curveto(109.02089844,36.87578125)(108.71035156,36.39726562)(108.29628906,36.05742187)
\curveto(107.88613281,35.72148437)(107.36660156,35.55351562)(106.73769531,35.55351562)
\curveto(105.90957031,35.55351562)(105.25917969,35.85039062)(104.78652344,36.44414062)
\curveto(104.22011719,37.15898437)(103.93691406,38.32304687)(103.93691406,39.93632812)
\closepath
\moveto(105.02089844,39.93632812)
\curveto(105.02089844,38.52617187)(105.18496094,37.58671875)(105.51308594,37.11796875)
\curveto(105.84511719,36.653125)(106.25332031,36.42070312)(106.73769531,36.42070312)
\curveto(107.22207031,36.42070312)(107.62832031,36.65507812)(107.95644531,37.12382812)
\curveto(108.28847656,37.59257812)(108.45449219,38.53007812)(108.45449219,39.93632812)
\curveto(108.45449219,41.35039062)(108.28847656,42.28984375)(107.95644531,42.7546875)
\curveto(107.62832031,43.21953125)(107.21816406,43.45195312)(106.72597656,43.45195312)
\curveto(106.24160156,43.45195312)(105.85488281,43.246875)(105.56582031,42.83671875)
\curveto(105.20253906,42.31328125)(105.02089844,41.34648437)(105.02089844,39.93632812)
\closepath
}
}
{
\newrgbcolor{curcolor}{0 0 0}
\pscustom[linestyle=none,fillstyle=solid,fillcolor=curcolor]
{
\newpath
\moveto(111.00332031,35.7)
\lineto(111.00332031,44.28984375)
\lineto(112.71425781,44.28984375)
\lineto(114.74746094,38.2078125)
\curveto(114.93496094,37.64140625)(115.07167969,37.21757812)(115.15761719,36.93632812)
\curveto(115.25527344,37.24882812)(115.40761719,37.7078125)(115.61464844,38.31328125)
\lineto(117.67128906,44.28984375)
\lineto(119.20058594,44.28984375)
\lineto(119.20058594,35.7)
\lineto(118.10488281,35.7)
\lineto(118.10488281,42.88945312)
\lineto(115.60878906,35.7)
\lineto(114.58339844,35.7)
\lineto(112.09902344,43.0125)
\lineto(112.09902344,35.7)
\closepath
}
}
{
\newrgbcolor{curcolor}{0 0 0}
\pscustom[linewidth=1,linecolor=curcolor]
{
\newpath
\moveto(183.4,57.6)
\lineto(183.4,66.6)
\moveto(183.4,425.9)
\lineto(183.4,416.9)
}
}
{
\newrgbcolor{curcolor}{0 0 0}
\pscustom[linestyle=none,fillstyle=solid,fillcolor=curcolor]
{
\newpath
\moveto(168.88925781,37.95)
\lineto(169.99667969,38.04375)
\curveto(170.07871094,37.5046875)(170.26816406,37.0984375)(170.56503906,36.825)
\curveto(170.86582031,36.55546875)(171.22714844,36.42070312)(171.64902344,36.42070312)
\curveto(172.15683594,36.42070312)(172.58652344,36.61210937)(172.93808594,36.99492187)
\curveto(173.28964844,37.37773437)(173.46542969,37.88554687)(173.46542969,38.51835937)
\curveto(173.46542969,39.11992187)(173.29550781,39.59453125)(172.95566406,39.9421875)
\curveto(172.61972656,40.28984375)(172.17832031,40.46367187)(171.63144531,40.46367187)
\curveto(171.29160156,40.46367187)(170.98496094,40.38554687)(170.71152344,40.22929687)
\curveto(170.43808594,40.07695312)(170.22324219,39.87773437)(170.06699219,39.63164062)
\lineto(169.07675781,39.76054687)
\lineto(169.90878906,44.17265625)
\lineto(174.18027344,44.17265625)
\lineto(174.18027344,43.16484375)
\lineto(170.75253906,43.16484375)
\lineto(170.28964844,40.85625)
\curveto(170.80527344,41.215625)(171.34628906,41.3953125)(171.91269531,41.3953125)
\curveto(172.66269531,41.3953125)(173.29550781,41.13554687)(173.81113281,40.61601562)
\curveto(174.32675781,40.09648437)(174.58457031,39.42851562)(174.58457031,38.61210937)
\curveto(174.58457031,37.83476562)(174.35800781,37.16289062)(173.90488281,36.59648437)
\curveto(173.35410156,35.90117187)(172.60214844,35.55351562)(171.64902344,35.55351562)
\curveto(170.86777344,35.55351562)(170.22910156,35.77226562)(169.73300781,36.20976562)
\curveto(169.24082031,36.64726562)(168.95957031,37.22734375)(168.88925781,37.95)
\closepath
}
}
{
\newrgbcolor{curcolor}{0 0 0}
\pscustom[linestyle=none,fillstyle=solid,fillcolor=curcolor]
{
\newpath
\moveto(175.56308594,39.93632812)
\curveto(175.56308594,40.95195312)(175.66660156,41.76835937)(175.87363281,42.38554687)
\curveto(176.08457031,43.00664062)(176.39511719,43.48515625)(176.80527344,43.82109375)
\curveto(177.21933594,44.15703125)(177.73886719,44.325)(178.36386719,44.325)
\curveto(178.82480469,44.325)(179.22910156,44.23125)(179.57675781,44.04375)
\curveto(179.92441406,43.86015625)(180.21152344,43.59257812)(180.43808594,43.24101562)
\curveto(180.66464844,42.89335937)(180.84238281,42.46757812)(180.97128906,41.96367187)
\curveto(181.10019531,41.46367187)(181.16464844,40.78789062)(181.16464844,39.93632812)
\curveto(181.16464844,38.92851562)(181.06113281,38.1140625)(180.85410156,37.49296875)
\curveto(180.64707031,36.87578125)(180.33652344,36.39726562)(179.92246094,36.05742187)
\curveto(179.51230469,35.72148437)(178.99277344,35.55351562)(178.36386719,35.55351562)
\curveto(177.53574219,35.55351562)(176.88535156,35.85039062)(176.41269531,36.44414062)
\curveto(175.84628906,37.15898437)(175.56308594,38.32304687)(175.56308594,39.93632812)
\closepath
\moveto(176.64707031,39.93632812)
\curveto(176.64707031,38.52617187)(176.81113281,37.58671875)(177.13925781,37.11796875)
\curveto(177.47128906,36.653125)(177.87949219,36.42070312)(178.36386719,36.42070312)
\curveto(178.84824219,36.42070312)(179.25449219,36.65507812)(179.58261719,37.12382812)
\curveto(179.91464844,37.59257812)(180.08066406,38.53007812)(180.08066406,39.93632812)
\curveto(180.08066406,41.35039062)(179.91464844,42.28984375)(179.58261719,42.7546875)
\curveto(179.25449219,43.21953125)(178.84433594,43.45195312)(178.35214844,43.45195312)
\curveto(177.86777344,43.45195312)(177.48105469,43.246875)(177.19199219,42.83671875)
\curveto(176.82871094,42.31328125)(176.64707031,41.34648437)(176.64707031,39.93632812)
\closepath
}
}
{
\newrgbcolor{curcolor}{0 0 0}
\pscustom[linestyle=none,fillstyle=solid,fillcolor=curcolor]
{
\newpath
\moveto(182.23691406,39.93632812)
\curveto(182.23691406,40.95195312)(182.34042969,41.76835937)(182.54746094,42.38554687)
\curveto(182.75839844,43.00664062)(183.06894531,43.48515625)(183.47910156,43.82109375)
\curveto(183.89316406,44.15703125)(184.41269531,44.325)(185.03769531,44.325)
\curveto(185.49863281,44.325)(185.90292969,44.23125)(186.25058594,44.04375)
\curveto(186.59824219,43.86015625)(186.88535156,43.59257812)(187.11191406,43.24101562)
\curveto(187.33847656,42.89335937)(187.51621094,42.46757812)(187.64511719,41.96367187)
\curveto(187.77402344,41.46367187)(187.83847656,40.78789062)(187.83847656,39.93632812)
\curveto(187.83847656,38.92851562)(187.73496094,38.1140625)(187.52792969,37.49296875)
\curveto(187.32089844,36.87578125)(187.01035156,36.39726562)(186.59628906,36.05742187)
\curveto(186.18613281,35.72148437)(185.66660156,35.55351562)(185.03769531,35.55351562)
\curveto(184.20957031,35.55351562)(183.55917969,35.85039062)(183.08652344,36.44414062)
\curveto(182.52011719,37.15898437)(182.23691406,38.32304687)(182.23691406,39.93632812)
\closepath
\moveto(183.32089844,39.93632812)
\curveto(183.32089844,38.52617187)(183.48496094,37.58671875)(183.81308594,37.11796875)
\curveto(184.14511719,36.653125)(184.55332031,36.42070312)(185.03769531,36.42070312)
\curveto(185.52207031,36.42070312)(185.92832031,36.65507812)(186.25644531,37.12382812)
\curveto(186.58847656,37.59257812)(186.75449219,38.53007812)(186.75449219,39.93632812)
\curveto(186.75449219,41.35039062)(186.58847656,42.28984375)(186.25644531,42.7546875)
\curveto(185.92832031,43.21953125)(185.51816406,43.45195312)(185.02597656,43.45195312)
\curveto(184.54160156,43.45195312)(184.15488281,43.246875)(183.86582031,42.83671875)
\curveto(183.50253906,42.31328125)(183.32089844,41.34648437)(183.32089844,39.93632812)
\closepath
}
}
{
\newrgbcolor{curcolor}{0 0 0}
\pscustom[linestyle=none,fillstyle=solid,fillcolor=curcolor]
{
\newpath
\moveto(189.30332031,35.7)
\lineto(189.30332031,44.28984375)
\lineto(191.01425781,44.28984375)
\lineto(193.04746094,38.2078125)
\curveto(193.23496094,37.64140625)(193.37167969,37.21757812)(193.45761719,36.93632812)
\curveto(193.55527344,37.24882812)(193.70761719,37.7078125)(193.91464844,38.31328125)
\lineto(195.97128906,44.28984375)
\lineto(197.50058594,44.28984375)
\lineto(197.50058594,35.7)
\lineto(196.40488281,35.7)
\lineto(196.40488281,42.88945312)
\lineto(193.90878906,35.7)
\lineto(192.88339844,35.7)
\lineto(190.39902344,43.0125)
\lineto(190.39902344,35.7)
\closepath
}
}
{
\newrgbcolor{curcolor}{0 0 0}
\pscustom[linewidth=1,linecolor=curcolor]
{
\newpath
\moveto(261.7,57.6)
\lineto(261.7,66.6)
\moveto(261.7,425.9)
\lineto(261.7,416.9)
}
}
{
\newrgbcolor{curcolor}{0 0 0}
\pscustom[linestyle=none,fillstyle=solid,fillcolor=curcolor]
{
\newpath
\moveto(258.16679687,35.7)
\lineto(257.11210937,35.7)
\lineto(257.11210937,42.42070312)
\curveto(256.85820312,42.17851562)(256.52421875,41.93632812)(256.11015625,41.69414062)
\curveto(255.7,41.45195312)(255.33085937,41.2703125)(255.00273437,41.14921875)
\lineto(255.00273437,42.16875)
\curveto(255.59257812,42.44609375)(256.10820312,42.78203125)(256.54960937,43.1765625)
\curveto(256.99101562,43.57109375)(257.30351562,43.95390625)(257.48710937,44.325)
\lineto(258.16679687,44.325)
\closepath
}
}
{
\newrgbcolor{curcolor}{0 0 0}
\pscustom[linestyle=none,fillstyle=solid,fillcolor=curcolor]
{
\newpath
\moveto(265.31523437,39.06914062)
\lineto(265.31523437,40.07695312)
\lineto(268.95390625,40.0828125)
\lineto(268.95390625,36.8953125)
\curveto(268.3953125,36.45)(267.81914062,36.1140625)(267.22539062,35.8875)
\curveto(266.63164062,35.66484375)(266.02226562,35.55351562)(265.39726562,35.55351562)
\curveto(264.55351562,35.55351562)(263.7859375,35.73320312)(263.09453125,36.09257812)
\curveto(262.40703125,36.45585937)(261.8875,36.97929687)(261.5359375,37.66289062)
\curveto(261.184375,38.34648437)(261.00859375,39.11015625)(261.00859375,39.95390625)
\curveto(261.00859375,40.78984375)(261.18242187,41.56914062)(261.53007812,42.29179687)
\curveto(261.88164062,43.01835937)(262.38554687,43.55742187)(263.04179687,43.90898437)
\curveto(263.69804687,44.26054687)(264.45390625,44.43632812)(265.309375,44.43632812)
\curveto(265.93046875,44.43632812)(266.49101562,44.33476562)(266.99101562,44.13164062)
\curveto(267.49492187,43.93242187)(267.88945312,43.653125)(268.17460937,43.29375)
\curveto(268.45976562,42.934375)(268.6765625,42.465625)(268.825,41.8875)
\lineto(267.79960937,41.60625)
\curveto(267.67070312,42.04375)(267.51054687,42.3875)(267.31914062,42.6375)
\curveto(267.12773437,42.8875)(266.85429687,43.08671875)(266.49882812,43.23515625)
\curveto(266.14335937,43.3875)(265.74882812,43.46367187)(265.31523437,43.46367187)
\curveto(264.79570312,43.46367187)(264.34648437,43.38359375)(263.96757812,43.2234375)
\curveto(263.58867187,43.0671875)(263.28203125,42.86015625)(263.04765625,42.60234375)
\curveto(262.8171875,42.34453125)(262.6375,42.06132812)(262.50859375,41.75273437)
\curveto(262.28984375,41.22148437)(262.18046875,40.6453125)(262.18046875,40.02421875)
\curveto(262.18046875,39.25859375)(262.31132812,38.61796875)(262.57304687,38.10234375)
\curveto(262.83867187,37.58671875)(263.2234375,37.20390625)(263.72734375,36.95390625)
\curveto(264.23125,36.70390625)(264.76640625,36.57890625)(265.3328125,36.57890625)
\curveto(265.825,36.57890625)(266.30546875,36.67265625)(266.77421875,36.86015625)
\curveto(267.24296875,37.0515625)(267.5984375,37.2546875)(267.840625,37.46953125)
\lineto(267.840625,39.06914062)
\closepath
}
}
{
\newrgbcolor{curcolor}{0 0 0}
\pscustom[linewidth=1,linecolor=curcolor]
{
\newpath
\moveto(340.1,57.6)
\lineto(340.1,66.6)
\moveto(340.1,425.9)
\lineto(340.1,416.9)
}
}
{
\newrgbcolor{curcolor}{0 0 0}
\pscustom[linestyle=none,fillstyle=solid,fillcolor=curcolor]
{
\newpath
\moveto(338.13710938,36.71367187)
\lineto(338.13710938,35.7)
\lineto(332.459375,35.7)
\curveto(332.4515625,35.95390625)(332.49257813,36.19804687)(332.58242188,36.43242187)
\curveto(332.72695313,36.81914062)(332.95742188,37.2)(333.27382813,37.575)
\curveto(333.59414063,37.95)(334.05507813,38.38359375)(334.65664063,38.87578125)
\curveto(335.59023438,39.64140625)(336.22109375,40.246875)(336.54921875,40.6921875)
\curveto(336.87734375,41.14140625)(337.04140625,41.56523437)(337.04140625,41.96367187)
\curveto(337.04140625,42.38164062)(336.89101563,42.73320312)(336.59023438,43.01835937)
\curveto(336.29335938,43.30742187)(335.9046875,43.45195312)(335.42421875,43.45195312)
\curveto(334.91640625,43.45195312)(334.51015625,43.29960937)(334.20546875,42.99492187)
\curveto(333.90078125,42.69023437)(333.74648438,42.26835937)(333.74257813,41.72929687)
\lineto(332.65859375,41.840625)
\curveto(332.7328125,42.64921875)(333.01210938,43.26445312)(333.49648438,43.68632812)
\curveto(333.98085938,44.11210937)(334.63125,44.325)(335.44765625,44.325)
\curveto(336.271875,44.325)(336.92421875,44.09648437)(337.4046875,43.63945312)
\curveto(337.88515625,43.18242187)(338.12539063,42.61601562)(338.12539063,41.94023437)
\curveto(338.12539063,41.59648437)(338.05507813,41.25859375)(337.91445313,40.9265625)
\curveto(337.77382813,40.59453125)(337.53945313,40.24492187)(337.21132813,39.87773437)
\curveto(336.88710938,39.51054687)(336.34609375,39.00664062)(335.58828125,38.36601562)
\curveto(334.95546875,37.83476562)(334.54921875,37.4734375)(334.36953125,37.28203125)
\curveto(334.18984375,37.09453125)(334.04140625,36.90507812)(333.92421875,36.71367187)
\closepath
}
}
{
\newrgbcolor{curcolor}{0 0 0}
\pscustom[linestyle=none,fillstyle=solid,fillcolor=curcolor]
{
\newpath
\moveto(343.71523438,39.06914062)
\lineto(343.71523438,40.07695312)
\lineto(347.35390625,40.0828125)
\lineto(347.35390625,36.8953125)
\curveto(346.7953125,36.45)(346.21914063,36.1140625)(345.62539063,35.8875)
\curveto(345.03164063,35.66484375)(344.42226563,35.55351562)(343.79726563,35.55351562)
\curveto(342.95351563,35.55351562)(342.1859375,35.73320312)(341.49453125,36.09257812)
\curveto(340.80703125,36.45585937)(340.2875,36.97929687)(339.9359375,37.66289062)
\curveto(339.584375,38.34648437)(339.40859375,39.11015625)(339.40859375,39.95390625)
\curveto(339.40859375,40.78984375)(339.58242188,41.56914062)(339.93007813,42.29179687)
\curveto(340.28164063,43.01835937)(340.78554688,43.55742187)(341.44179688,43.90898437)
\curveto(342.09804688,44.26054687)(342.85390625,44.43632812)(343.709375,44.43632812)
\curveto(344.33046875,44.43632812)(344.89101563,44.33476562)(345.39101563,44.13164062)
\curveto(345.89492188,43.93242187)(346.28945313,43.653125)(346.57460938,43.29375)
\curveto(346.85976563,42.934375)(347.0765625,42.465625)(347.225,41.8875)
\lineto(346.19960938,41.60625)
\curveto(346.07070313,42.04375)(345.91054688,42.3875)(345.71914063,42.6375)
\curveto(345.52773438,42.8875)(345.25429688,43.08671875)(344.89882813,43.23515625)
\curveto(344.54335938,43.3875)(344.14882813,43.46367187)(343.71523438,43.46367187)
\curveto(343.19570313,43.46367187)(342.74648438,43.38359375)(342.36757813,43.2234375)
\curveto(341.98867188,43.0671875)(341.68203125,42.86015625)(341.44765625,42.60234375)
\curveto(341.2171875,42.34453125)(341.0375,42.06132812)(340.90859375,41.75273437)
\curveto(340.68984375,41.22148437)(340.58046875,40.6453125)(340.58046875,40.02421875)
\curveto(340.58046875,39.25859375)(340.71132813,38.61796875)(340.97304688,38.10234375)
\curveto(341.23867188,37.58671875)(341.6234375,37.20390625)(342.12734375,36.95390625)
\curveto(342.63125,36.70390625)(343.16640625,36.57890625)(343.7328125,36.57890625)
\curveto(344.225,36.57890625)(344.70546875,36.67265625)(345.17421875,36.86015625)
\curveto(345.64296875,37.0515625)(345.9984375,37.2546875)(346.240625,37.46953125)
\lineto(346.240625,39.06914062)
\closepath
}
}
{
\newrgbcolor{curcolor}{0 0 0}
\pscustom[linewidth=1,linecolor=curcolor]
{
\newpath
\moveto(418.4,57.6)
\lineto(418.4,66.6)
\moveto(418.4,425.9)
\lineto(418.4,416.9)
}
}
{
\newrgbcolor{curcolor}{0 0 0}
\pscustom[linestyle=none,fillstyle=solid,fillcolor=curcolor]
{
\newpath
\moveto(414.275,35.7)
\lineto(414.275,37.75664062)
\lineto(410.5484375,37.75664062)
\lineto(410.5484375,38.7234375)
\lineto(414.46835937,44.28984375)
\lineto(415.3296875,44.28984375)
\lineto(415.3296875,38.7234375)
\lineto(416.48984375,38.7234375)
\lineto(416.48984375,37.75664062)
\lineto(415.3296875,37.75664062)
\lineto(415.3296875,35.7)
\closepath
\moveto(414.275,38.7234375)
\lineto(414.275,42.59648437)
\lineto(411.58554687,38.7234375)
\closepath
}
}
{
\newrgbcolor{curcolor}{0 0 0}
\pscustom[linestyle=none,fillstyle=solid,fillcolor=curcolor]
{
\newpath
\moveto(422.01523437,39.06914062)
\lineto(422.01523437,40.07695312)
\lineto(425.65390625,40.0828125)
\lineto(425.65390625,36.8953125)
\curveto(425.0953125,36.45)(424.51914062,36.1140625)(423.92539062,35.8875)
\curveto(423.33164062,35.66484375)(422.72226562,35.55351562)(422.09726562,35.55351562)
\curveto(421.25351562,35.55351562)(420.4859375,35.73320312)(419.79453125,36.09257812)
\curveto(419.10703125,36.45585937)(418.5875,36.97929687)(418.2359375,37.66289062)
\curveto(417.884375,38.34648437)(417.70859375,39.11015625)(417.70859375,39.95390625)
\curveto(417.70859375,40.78984375)(417.88242187,41.56914062)(418.23007812,42.29179687)
\curveto(418.58164062,43.01835937)(419.08554687,43.55742187)(419.74179687,43.90898437)
\curveto(420.39804687,44.26054687)(421.15390625,44.43632812)(422.009375,44.43632812)
\curveto(422.63046875,44.43632812)(423.19101562,44.33476562)(423.69101562,44.13164062)
\curveto(424.19492187,43.93242187)(424.58945312,43.653125)(424.87460937,43.29375)
\curveto(425.15976562,42.934375)(425.3765625,42.465625)(425.525,41.8875)
\lineto(424.49960937,41.60625)
\curveto(424.37070312,42.04375)(424.21054687,42.3875)(424.01914062,42.6375)
\curveto(423.82773437,42.8875)(423.55429687,43.08671875)(423.19882812,43.23515625)
\curveto(422.84335937,43.3875)(422.44882812,43.46367187)(422.01523437,43.46367187)
\curveto(421.49570312,43.46367187)(421.04648437,43.38359375)(420.66757812,43.2234375)
\curveto(420.28867187,43.0671875)(419.98203125,42.86015625)(419.74765625,42.60234375)
\curveto(419.5171875,42.34453125)(419.3375,42.06132812)(419.20859375,41.75273437)
\curveto(418.98984375,41.22148437)(418.88046875,40.6453125)(418.88046875,40.02421875)
\curveto(418.88046875,39.25859375)(419.01132812,38.61796875)(419.27304687,38.10234375)
\curveto(419.53867187,37.58671875)(419.9234375,37.20390625)(420.42734375,36.95390625)
\curveto(420.93125,36.70390625)(421.46640625,36.57890625)(422.0328125,36.57890625)
\curveto(422.525,36.57890625)(423.00546875,36.67265625)(423.47421875,36.86015625)
\curveto(423.94296875,37.0515625)(424.2984375,37.2546875)(424.540625,37.46953125)
\lineto(424.540625,39.06914062)
\closepath
}
}
{
\newrgbcolor{curcolor}{0 0 0}
\pscustom[linewidth=1,linecolor=curcolor]
{
\newpath
\moveto(496.7,57.6)
\lineto(496.7,66.6)
\moveto(496.7,425.9)
\lineto(496.7,416.9)
}
}
{
\newrgbcolor{curcolor}{0 0 0}
\pscustom[linestyle=none,fillstyle=solid,fillcolor=curcolor]
{
\newpath
\moveto(490.8171875,40.35820312)
\curveto(490.3796875,40.51835937)(490.05546875,40.746875)(489.84453125,41.04375)
\curveto(489.63359375,41.340625)(489.528125,41.69609375)(489.528125,42.11015625)
\curveto(489.528125,42.73515625)(489.75273437,43.26054687)(490.20195312,43.68632812)
\curveto(490.65117187,44.11210937)(491.24882812,44.325)(491.99492187,44.325)
\curveto(492.74492187,44.325)(493.3484375,44.10625)(493.80546875,43.66875)
\curveto(494.2625,43.23515625)(494.49101562,42.70585937)(494.49101562,42.08085937)
\curveto(494.49101562,41.68242187)(494.38554687,41.33476562)(494.17460937,41.03789062)
\curveto(493.96757812,40.74492187)(493.65117187,40.51835937)(493.22539062,40.35820312)
\curveto(493.75273437,40.18632812)(494.153125,39.90898437)(494.4265625,39.52617187)
\curveto(494.70390625,39.14335937)(494.84257812,38.68632812)(494.84257812,38.15507812)
\curveto(494.84257812,37.42070312)(494.5828125,36.80351562)(494.06328125,36.30351562)
\curveto(493.54375,35.80351562)(492.86015625,35.55351562)(492.0125,35.55351562)
\curveto(491.16484375,35.55351562)(490.48125,35.80351562)(489.96171875,36.30351562)
\curveto(489.4421875,36.80742187)(489.18242187,37.434375)(489.18242187,38.184375)
\curveto(489.18242187,38.74296875)(489.32304687,39.20976562)(489.60429687,39.58476562)
\curveto(489.88945312,39.96367187)(490.29375,40.22148437)(490.8171875,40.35820312)
\closepath
\moveto(490.60625,42.1453125)
\curveto(490.60625,41.7390625)(490.73710937,41.40703125)(490.99882812,41.14921875)
\curveto(491.26054687,40.89140625)(491.60039062,40.7625)(492.01835937,40.7625)
\curveto(492.42460937,40.7625)(492.75664062,40.88945312)(493.01445312,41.14335937)
\curveto(493.27617187,41.40117187)(493.40703125,41.715625)(493.40703125,42.08671875)
\curveto(493.40703125,42.4734375)(493.27226562,42.79765625)(493.00273437,43.059375)
\curveto(492.73710937,43.325)(492.40507812,43.4578125)(492.00664062,43.4578125)
\curveto(491.60429687,43.4578125)(491.2703125,43.32890625)(491.0046875,43.07109375)
\curveto(490.7390625,42.81328125)(490.60625,42.5046875)(490.60625,42.1453125)
\closepath
\moveto(490.26640625,38.17851562)
\curveto(490.26640625,37.87773437)(490.33671875,37.58671875)(490.47734375,37.30546875)
\curveto(490.621875,37.02421875)(490.83476562,36.80546875)(491.11601562,36.64921875)
\curveto(491.39726562,36.496875)(491.7,36.42070312)(492.02421875,36.42070312)
\curveto(492.528125,36.42070312)(492.94414062,36.5828125)(493.27226562,36.90703125)
\curveto(493.60039062,37.23125)(493.76445312,37.64335937)(493.76445312,38.14335937)
\curveto(493.76445312,38.65117187)(493.59453125,39.07109375)(493.2546875,39.403125)
\curveto(492.91875,39.73515625)(492.496875,39.90117187)(491.9890625,39.90117187)
\curveto(491.49296875,39.90117187)(491.08085937,39.73710937)(490.75273437,39.40898437)
\curveto(490.42851562,39.08085937)(490.26640625,38.67070312)(490.26640625,38.17851562)
\closepath
}
}
{
\newrgbcolor{curcolor}{0 0 0}
\pscustom[linestyle=none,fillstyle=solid,fillcolor=curcolor]
{
\newpath
\moveto(500.31523437,39.06914062)
\lineto(500.31523437,40.07695312)
\lineto(503.95390625,40.0828125)
\lineto(503.95390625,36.8953125)
\curveto(503.3953125,36.45)(502.81914062,36.1140625)(502.22539062,35.8875)
\curveto(501.63164062,35.66484375)(501.02226562,35.55351562)(500.39726562,35.55351562)
\curveto(499.55351562,35.55351562)(498.7859375,35.73320312)(498.09453125,36.09257812)
\curveto(497.40703125,36.45585937)(496.8875,36.97929687)(496.5359375,37.66289062)
\curveto(496.184375,38.34648437)(496.00859375,39.11015625)(496.00859375,39.95390625)
\curveto(496.00859375,40.78984375)(496.18242187,41.56914062)(496.53007812,42.29179687)
\curveto(496.88164062,43.01835937)(497.38554687,43.55742187)(498.04179687,43.90898437)
\curveto(498.69804687,44.26054687)(499.45390625,44.43632812)(500.309375,44.43632812)
\curveto(500.93046875,44.43632812)(501.49101562,44.33476562)(501.99101562,44.13164062)
\curveto(502.49492187,43.93242187)(502.88945312,43.653125)(503.17460937,43.29375)
\curveto(503.45976562,42.934375)(503.6765625,42.465625)(503.825,41.8875)
\lineto(502.79960937,41.60625)
\curveto(502.67070312,42.04375)(502.51054687,42.3875)(502.31914062,42.6375)
\curveto(502.12773437,42.8875)(501.85429687,43.08671875)(501.49882812,43.23515625)
\curveto(501.14335937,43.3875)(500.74882812,43.46367187)(500.31523437,43.46367187)
\curveto(499.79570312,43.46367187)(499.34648437,43.38359375)(498.96757812,43.2234375)
\curveto(498.58867187,43.0671875)(498.28203125,42.86015625)(498.04765625,42.60234375)
\curveto(497.8171875,42.34453125)(497.6375,42.06132812)(497.50859375,41.75273437)
\curveto(497.28984375,41.22148437)(497.18046875,40.6453125)(497.18046875,40.02421875)
\curveto(497.18046875,39.25859375)(497.31132812,38.61796875)(497.57304687,38.10234375)
\curveto(497.83867187,37.58671875)(498.2234375,37.20390625)(498.72734375,36.95390625)
\curveto(499.23125,36.70390625)(499.76640625,36.57890625)(500.3328125,36.57890625)
\curveto(500.825,36.57890625)(501.30546875,36.67265625)(501.77421875,36.86015625)
\curveto(502.24296875,37.0515625)(502.5984375,37.2546875)(502.840625,37.46953125)
\lineto(502.840625,39.06914062)
\closepath
}
}
{
\newrgbcolor{curcolor}{0 0 0}
\pscustom[linewidth=1,linecolor=curcolor]
{
\newpath
\moveto(575,57.6)
\lineto(575,66.6)
\moveto(575,425.9)
\lineto(575,416.9)
}
}
{
\newrgbcolor{curcolor}{0 0 0}
\pscustom[linestyle=none,fillstyle=solid,fillcolor=curcolor]
{
\newpath
\moveto(568.12988281,35.7)
\lineto(567.07519531,35.7)
\lineto(567.07519531,42.42070312)
\curveto(566.82128906,42.17851562)(566.48730469,41.93632812)(566.07324219,41.69414062)
\curveto(565.66308594,41.45195312)(565.29394531,41.2703125)(564.96582031,41.14921875)
\lineto(564.96582031,42.16875)
\curveto(565.55566406,42.44609375)(566.07128906,42.78203125)(566.51269531,43.1765625)
\curveto(566.95410156,43.57109375)(567.26660156,43.95390625)(567.45019531,44.325)
\lineto(568.12988281,44.325)
\closepath
}
}
{
\newrgbcolor{curcolor}{0 0 0}
\pscustom[linestyle=none,fillstyle=solid,fillcolor=curcolor]
{
\newpath
\moveto(576.30371094,42.18632812)
\lineto(575.25488281,42.10429687)
\curveto(575.16113281,42.51835937)(575.02832031,42.81914062)(574.85644531,43.00664062)
\curveto(574.57128906,43.30742187)(574.21972656,43.4578125)(573.80175781,43.4578125)
\curveto(573.46582031,43.4578125)(573.17089844,43.3640625)(572.91699219,43.1765625)
\curveto(572.58496094,42.934375)(572.32324219,42.58085937)(572.13183594,42.11601562)
\curveto(571.94042969,41.65117187)(571.84082031,40.9890625)(571.83300781,40.1296875)
\curveto(572.08691406,40.51640625)(572.39746094,40.80351562)(572.76464844,40.99101562)
\curveto(573.13183594,41.17851562)(573.51660156,41.27226562)(573.91894531,41.27226562)
\curveto(574.62207031,41.27226562)(575.21972656,41.0125)(575.71191406,40.49296875)
\curveto(576.20800781,39.97734375)(576.45605469,39.309375)(576.45605469,38.4890625)
\curveto(576.45605469,37.95)(576.33886719,37.44804687)(576.10449219,36.98320312)
\curveto(575.87402344,36.52226562)(575.55566406,36.16875)(575.14941406,35.92265625)
\curveto(574.74316406,35.6765625)(574.28222656,35.55351562)(573.76660156,35.55351562)
\curveto(572.88769531,35.55351562)(572.17089844,35.87578125)(571.61621094,36.5203125)
\curveto(571.06152344,37.16875)(570.78417969,38.23515625)(570.78417969,39.71953125)
\curveto(570.78417969,41.3796875)(571.09082031,42.58671875)(571.70410156,43.340625)
\curveto(572.23925781,43.996875)(572.95996094,44.325)(573.86621094,44.325)
\curveto(574.54199219,44.325)(575.09472656,44.13554687)(575.52441406,43.75664062)
\curveto(575.95800781,43.37773437)(576.21777344,42.85429687)(576.30371094,42.18632812)
\closepath
\moveto(571.99707031,38.48320312)
\curveto(571.99707031,38.11992187)(572.07324219,37.77226562)(572.22558594,37.44023437)
\curveto(572.38183594,37.10820312)(572.59863281,36.85429687)(572.87597656,36.67851562)
\curveto(573.15332031,36.50664062)(573.44433594,36.42070312)(573.74902344,36.42070312)
\curveto(574.19433594,36.42070312)(574.57714844,36.60039062)(574.89746094,36.95976562)
\curveto(575.21777344,37.31914062)(575.37792969,37.80742187)(575.37792969,38.42460937)
\curveto(575.37792969,39.01835937)(575.21972656,39.48515625)(574.90332031,39.825)
\curveto(574.58691406,40.16875)(574.18847656,40.340625)(573.70800781,40.340625)
\curveto(573.23144531,40.340625)(572.82714844,40.16875)(572.49511719,39.825)
\curveto(572.16308594,39.48515625)(571.99707031,39.03789062)(571.99707031,38.48320312)
\closepath
}
}
{
\newrgbcolor{curcolor}{0 0 0}
\pscustom[linestyle=none,fillstyle=solid,fillcolor=curcolor]
{
\newpath
\moveto(581.95214844,39.06914062)
\lineto(581.95214844,40.07695312)
\lineto(585.59082031,40.0828125)
\lineto(585.59082031,36.8953125)
\curveto(585.03222656,36.45)(584.45605469,36.1140625)(583.86230469,35.8875)
\curveto(583.26855469,35.66484375)(582.65917969,35.55351562)(582.03417969,35.55351562)
\curveto(581.19042969,35.55351562)(580.42285156,35.73320312)(579.73144531,36.09257812)
\curveto(579.04394531,36.45585937)(578.52441406,36.97929687)(578.17285156,37.66289062)
\curveto(577.82128906,38.34648437)(577.64550781,39.11015625)(577.64550781,39.95390625)
\curveto(577.64550781,40.78984375)(577.81933594,41.56914062)(578.16699219,42.29179687)
\curveto(578.51855469,43.01835937)(579.02246094,43.55742187)(579.67871094,43.90898437)
\curveto(580.33496094,44.26054687)(581.09082031,44.43632812)(581.94628906,44.43632812)
\curveto(582.56738281,44.43632812)(583.12792969,44.33476562)(583.62792969,44.13164062)
\curveto(584.13183594,43.93242187)(584.52636719,43.653125)(584.81152344,43.29375)
\curveto(585.09667969,42.934375)(585.31347656,42.465625)(585.46191406,41.8875)
\lineto(584.43652344,41.60625)
\curveto(584.30761719,42.04375)(584.14746094,42.3875)(583.95605469,42.6375)
\curveto(583.76464844,42.8875)(583.49121094,43.08671875)(583.13574219,43.23515625)
\curveto(582.78027344,43.3875)(582.38574219,43.46367187)(581.95214844,43.46367187)
\curveto(581.43261719,43.46367187)(580.98339844,43.38359375)(580.60449219,43.2234375)
\curveto(580.22558594,43.0671875)(579.91894531,42.86015625)(579.68457031,42.60234375)
\curveto(579.45410156,42.34453125)(579.27441406,42.06132812)(579.14550781,41.75273437)
\curveto(578.92675781,41.22148437)(578.81738281,40.6453125)(578.81738281,40.02421875)
\curveto(578.81738281,39.25859375)(578.94824219,38.61796875)(579.20996094,38.10234375)
\curveto(579.47558594,37.58671875)(579.86035156,37.20390625)(580.36425781,36.95390625)
\curveto(580.86816406,36.70390625)(581.40332031,36.57890625)(581.96972656,36.57890625)
\curveto(582.46191406,36.57890625)(582.94238281,36.67265625)(583.41113281,36.86015625)
\curveto(583.87988281,37.0515625)(584.23535156,37.2546875)(584.47753906,37.46953125)
\lineto(584.47753906,39.06914062)
\closepath
}
}
{
\newrgbcolor{curcolor}{0 0 0}
\pscustom[linewidth=1,linecolor=curcolor]
{
\newpath
\moveto(105.1,425.9)
\lineto(105.1,57.6)
\lineto(575,57.6)
\lineto(575,425.9)
\closepath
}
}
{
\newrgbcolor{curcolor}{0 0 0}
\pscustom[linestyle=none,fillstyle=solid,fillcolor=curcolor]
{
\newpath
\moveto(16.3,192.29082031)
\lineto(7.71015625,192.29082031)
\lineto(7.71015625,196.09941406)
\curveto(7.71015625,196.86503906)(7.78828125,197.44707031)(7.94453125,197.84550781)
\curveto(8.096875,198.24394531)(8.36835938,198.56230469)(8.75898438,198.80058594)
\curveto(9.14960938,199.03886719)(9.58125,199.15800781)(10.05390625,199.15800781)
\curveto(10.66328125,199.15800781)(11.17695313,198.96074219)(11.59492188,198.56621094)
\curveto(12.01289063,198.17167969)(12.27851563,197.56230469)(12.39179688,196.73808594)
\curveto(12.53632813,197.03886719)(12.67890625,197.26738281)(12.81953125,197.42363281)
\curveto(13.12421875,197.75566406)(13.50507813,198.07011719)(13.96210938,198.36699219)
\lineto(16.3,199.86113281)
\lineto(16.3,198.43144531)
\lineto(14.51289063,197.29472656)
\curveto(13.99726563,196.96269531)(13.60273438,196.68925781)(13.32929688,196.47441406)
\curveto(13.05585938,196.25957031)(12.86445313,196.06621094)(12.75507813,195.89433594)
\curveto(12.64570313,195.72636719)(12.56953125,195.55449219)(12.5265625,195.37871094)
\curveto(12.49921875,195.24980469)(12.48554688,195.03886719)(12.48554688,194.74589844)
\lineto(12.48554688,193.42753906)
\lineto(16.3,193.42753906)
\closepath
\moveto(11.50117188,193.42753906)
\lineto(11.50117188,195.87089844)
\curveto(11.50117188,196.39042969)(11.4484375,196.79667969)(11.34296875,197.08964844)
\curveto(11.23359375,197.38261719)(11.06171875,197.60527344)(10.82734375,197.75761719)
\curveto(10.5890625,197.90996094)(10.33125,197.98613281)(10.05390625,197.98613281)
\curveto(9.64765625,197.98613281)(9.31367188,197.83769531)(9.05195313,197.54082031)
\curveto(8.79023438,197.24785156)(8.659375,196.78300781)(8.659375,196.14628906)
\lineto(8.659375,193.42753906)
\closepath
}
}
{
\newrgbcolor{curcolor}{0 0 0}
\pscustom[linestyle=none,fillstyle=solid,fillcolor=curcolor]
{
\newpath
\moveto(16.3,204.88261719)
\lineto(15.3859375,204.88261719)
\curveto(16.0890625,204.39824219)(16.440625,203.74003906)(16.440625,202.90800781)
\curveto(16.440625,202.54082031)(16.3703125,202.19707031)(16.2296875,201.87675781)
\curveto(16.0890625,201.56035156)(15.91328125,201.32402344)(15.70234375,201.16777344)
\curveto(15.4875,201.01542969)(15.22578125,200.90800781)(14.9171875,200.84550781)
\curveto(14.71015625,200.80253906)(14.38203125,200.78105469)(13.9328125,200.78105469)
\lineto(10.07734375,200.78105469)
\lineto(10.07734375,201.83574219)
\lineto(13.52851563,201.83574219)
\curveto(14.07929688,201.83574219)(14.45039063,201.85722656)(14.64179688,201.90019531)
\curveto(14.91914063,201.96660156)(15.13789063,202.10722656)(15.29804688,202.32207031)
\curveto(15.45429688,202.53691406)(15.53242188,202.80253906)(15.53242188,203.11894531)
\curveto(15.53242188,203.43535156)(15.45234375,203.73222656)(15.2921875,204.00957031)
\curveto(15.128125,204.28691406)(14.90742188,204.48222656)(14.63007813,204.59550781)
\curveto(14.34882813,204.71269531)(13.94257813,204.77128906)(13.41132813,204.77128906)
\lineto(10.07734375,204.77128906)
\lineto(10.07734375,205.82597656)
\lineto(16.3,205.82597656)
\closepath
}
}
{
\newrgbcolor{curcolor}{0 0 0}
\pscustom[linestyle=none,fillstyle=solid,fillcolor=curcolor]
{
\newpath
\moveto(16.3,207.47832031)
\lineto(10.07734375,207.47832031)
\lineto(10.07734375,208.42753906)
\lineto(10.96210938,208.42753906)
\curveto(10.27851563,208.88457031)(9.93671875,209.54472656)(9.93671875,210.40800781)
\curveto(9.93671875,210.78300781)(10.00507813,211.12675781)(10.14179688,211.43925781)
\curveto(10.27460938,211.75566406)(10.45039063,211.99199219)(10.66914063,212.14824219)
\curveto(10.88789063,212.30449219)(11.14765625,212.41386719)(11.4484375,212.47636719)
\curveto(11.64375,212.51542969)(11.98554688,212.53496094)(12.47382813,212.53496094)
\lineto(16.3,212.53496094)
\lineto(16.3,211.48027344)
\lineto(12.51484375,211.48027344)
\curveto(12.08515625,211.48027344)(11.76484375,211.43925781)(11.55390625,211.35722656)
\curveto(11.3390625,211.27519531)(11.16914063,211.12871094)(11.04414063,210.91777344)
\curveto(10.91523438,210.71074219)(10.85078125,210.46660156)(10.85078125,210.18535156)
\curveto(10.85078125,209.73613281)(10.99335938,209.34746094)(11.27851563,209.01933594)
\curveto(11.56367188,208.69511719)(12.1046875,208.53300781)(12.9015625,208.53300781)
\lineto(16.3,208.53300781)
\closepath
}
}
{
\newrgbcolor{curcolor}{0 0 0}
\pscustom[linestyle=none,fillstyle=solid,fillcolor=curcolor]
{
\newpath
\moveto(15.35664063,216.45488281)
\lineto(16.28828125,216.60722656)
\curveto(16.35078125,216.31035156)(16.38203125,216.04472656)(16.38203125,215.81035156)
\curveto(16.38203125,215.42753906)(16.32148438,215.13066406)(16.20039063,214.91972656)
\curveto(16.07929688,214.70878906)(15.92109375,214.56035156)(15.72578125,214.47441406)
\curveto(15.5265625,214.38847656)(15.11054688,214.34550781)(14.47773438,214.34550781)
\lineto(10.89765625,214.34550781)
\lineto(10.89765625,213.57207031)
\lineto(10.07734375,213.57207031)
\lineto(10.07734375,214.34550781)
\lineto(8.53632813,214.34550781)
\lineto(7.90351563,215.39433594)
\lineto(10.07734375,215.39433594)
\lineto(10.07734375,216.45488281)
\lineto(10.89765625,216.45488281)
\lineto(10.89765625,215.39433594)
\lineto(14.53632813,215.39433594)
\curveto(14.83710938,215.39433594)(15.03046875,215.41191406)(15.11640625,215.44707031)
\curveto(15.20234375,215.48613281)(15.27070313,215.54667969)(15.32148438,215.62871094)
\curveto(15.37226563,215.71464844)(15.39765625,215.83574219)(15.39765625,215.99199219)
\curveto(15.39765625,216.10917969)(15.38398438,216.26347656)(15.35664063,216.45488281)
\closepath
}
}
{
\newrgbcolor{curcolor}{0 0 0}
\pscustom[linestyle=none,fillstyle=solid,fillcolor=curcolor]
{
\newpath
\moveto(8.92304688,217.49199219)
\lineto(7.71015625,217.49199219)
\lineto(7.71015625,218.54667969)
\lineto(8.92304688,218.54667969)
\closepath
\moveto(16.3,217.49199219)
\lineto(10.07734375,217.49199219)
\lineto(10.07734375,218.54667969)
\lineto(16.3,218.54667969)
\closepath
}
}
{
\newrgbcolor{curcolor}{0 0 0}
\pscustom[linestyle=none,fillstyle=solid,fillcolor=curcolor]
{
\newpath
\moveto(16.3,220.15214844)
\lineto(10.07734375,220.15214844)
\lineto(10.07734375,221.09550781)
\lineto(10.95039063,221.09550781)
\curveto(10.64570313,221.29082031)(10.4015625,221.55058594)(10.21796875,221.87480469)
\curveto(10.03046875,222.19902344)(9.93671875,222.56816406)(9.93671875,222.98222656)
\curveto(9.93671875,223.44316406)(10.03242188,223.82011719)(10.22382813,224.11308594)
\curveto(10.41523438,224.40996094)(10.6828125,224.61894531)(11.0265625,224.74003906)
\curveto(10.3,225.23222656)(9.93671875,225.87285156)(9.93671875,226.66191406)
\curveto(9.93671875,227.27910156)(10.10859375,227.75371094)(10.45234375,228.08574219)
\curveto(10.7921875,228.41777344)(11.31757813,228.58378906)(12.02851563,228.58378906)
\lineto(16.3,228.58378906)
\lineto(16.3,227.53496094)
\lineto(12.38007813,227.53496094)
\curveto(11.95820313,227.53496094)(11.65546875,227.49980469)(11.471875,227.42949219)
\curveto(11.284375,227.36308594)(11.13398438,227.24003906)(11.02070313,227.06035156)
\curveto(10.90742188,226.88066406)(10.85078125,226.66972656)(10.85078125,226.42753906)
\curveto(10.85078125,225.99003906)(10.99726563,225.62675781)(11.29023438,225.33769531)
\curveto(11.57929688,225.04863281)(12.04414063,224.90410156)(12.68476563,224.90410156)
\lineto(16.3,224.90410156)
\lineto(16.3,223.84941406)
\lineto(12.25703125,223.84941406)
\curveto(11.78828125,223.84941406)(11.43671875,223.76347656)(11.20234375,223.59160156)
\curveto(10.96796875,223.41972656)(10.85078125,223.13847656)(10.85078125,222.74785156)
\curveto(10.85078125,222.45097656)(10.92890625,222.17558594)(11.08515625,221.92167969)
\curveto(11.24140625,221.67167969)(11.46992188,221.49003906)(11.77070313,221.37675781)
\curveto(12.07148438,221.26347656)(12.50507813,221.20683594)(13.07148438,221.20683594)
\lineto(16.3,221.20683594)
\closepath
}
}
{
\newrgbcolor{curcolor}{0 0 0}
\pscustom[linestyle=none,fillstyle=solid,fillcolor=curcolor]
{
\newpath
\moveto(14.29609375,234.40800781)
\lineto(14.43085938,235.49785156)
\curveto(15.06757813,235.32597656)(15.56171875,235.00761719)(15.91328125,234.54277344)
\curveto(16.26484375,234.07792969)(16.440625,233.48417969)(16.440625,232.76152344)
\curveto(16.440625,231.85136719)(16.16132813,231.12871094)(15.60273438,230.59355469)
\curveto(15.04023438,230.06230469)(14.253125,229.79667969)(13.24140625,229.79667969)
\curveto(12.19453125,229.79667969)(11.38203125,230.06621094)(10.80390625,230.60527344)
\curveto(10.22578125,231.14433594)(9.93671875,231.84355469)(9.93671875,232.70292969)
\curveto(9.93671875,233.53496094)(10.21992188,234.21464844)(10.78632813,234.74199219)
\curveto(11.35273438,235.26933594)(12.14960938,235.53300781)(13.17695313,235.53300781)
\curveto(13.23945313,235.53300781)(13.33320313,235.53105469)(13.45820313,235.52714844)
\lineto(13.45820313,230.88652344)
\curveto(14.14179688,230.92558594)(14.66523438,231.11894531)(15.02851563,231.46660156)
\curveto(15.39179688,231.81425781)(15.5734375,232.24785156)(15.5734375,232.76738281)
\curveto(15.5734375,233.15410156)(15.471875,233.48417969)(15.26875,233.75761719)
\curveto(15.065625,234.03105469)(14.74140625,234.24785156)(14.29609375,234.40800781)
\closepath
\moveto(12.59101563,230.94511719)
\lineto(12.59101563,234.41972656)
\curveto(12.06757813,234.37285156)(11.675,234.24003906)(11.41328125,234.02128906)
\curveto(11.00703125,233.68535156)(10.80390625,233.24980469)(10.80390625,232.71464844)
\curveto(10.80390625,232.23027344)(10.96601563,231.82207031)(11.29023438,231.49003906)
\curveto(11.61445313,231.16191406)(12.04804688,230.98027344)(12.59101563,230.94511719)
\closepath
}
}
{
\newrgbcolor{curcolor}{0 0 0}
\pscustom[linestyle=none,fillstyle=solid,fillcolor=curcolor]
{
\newpath
\moveto(18.82539063,242.17167969)
\curveto(18.09101563,241.58964844)(17.23164063,241.09746094)(16.24726563,240.69511719)
\curveto(15.26289063,240.29277344)(14.24335938,240.09160156)(13.18867188,240.09160156)
\curveto(12.25898438,240.09160156)(11.36835938,240.24199219)(10.51679688,240.54277344)
\curveto(9.52851563,240.89433594)(8.54414063,241.43730469)(7.56367188,242.17167969)
\lineto(7.56367188,242.92753906)
\curveto(8.37617188,242.45488281)(8.95625,242.14238281)(9.30390625,241.99003906)
\curveto(9.84296875,241.75175781)(10.40546875,241.56425781)(10.99140625,241.42753906)
\curveto(11.721875,241.25957031)(12.45625,241.17558594)(13.19453125,241.17558594)
\curveto(15.0734375,241.17558594)(16.95039063,241.75957031)(18.82539063,242.92753906)
\closepath
}
}
{
\newrgbcolor{curcolor}{0 0 0}
\pscustom[linestyle=none,fillstyle=solid,fillcolor=curcolor]
{
\newpath
\moveto(14.44257813,243.73027344)
\lineto(14.27851563,244.77324219)
\curveto(14.69648438,244.83183594)(15.01679688,244.99394531)(15.23945313,245.25957031)
\curveto(15.46210938,245.52910156)(15.5734375,245.90410156)(15.5734375,246.38457031)
\curveto(15.5734375,246.86894531)(15.47578125,247.22832031)(15.28046875,247.46269531)
\curveto(15.08125,247.69707031)(14.84882813,247.81425781)(14.58320313,247.81425781)
\curveto(14.34492188,247.81425781)(14.15742188,247.71074219)(14.02070313,247.50371094)
\curveto(13.92695313,247.35917969)(13.8078125,246.99980469)(13.66328125,246.42558594)
\curveto(13.46796875,245.65214844)(13.3,245.11503906)(13.159375,244.81425781)
\curveto(13.01484375,244.51738281)(12.81757813,244.29082031)(12.56757813,244.13457031)
\curveto(12.31367188,243.98222656)(12.034375,243.90605469)(11.7296875,243.90605469)
\curveto(11.45234375,243.90605469)(11.19648438,243.96855469)(10.96210938,244.09355469)
\curveto(10.72382813,244.22246094)(10.5265625,244.39628906)(10.3703125,244.61503906)
\curveto(10.24921875,244.77910156)(10.14765625,245.00175781)(10.065625,245.28300781)
\curveto(9.9796875,245.56816406)(9.93671875,245.87285156)(9.93671875,246.19707031)
\curveto(9.93671875,246.68535156)(10.00703125,247.11308594)(10.14765625,247.48027344)
\curveto(10.28828125,247.85136719)(10.4796875,248.12480469)(10.721875,248.30058594)
\curveto(10.96015625,248.47636719)(11.28046875,248.59746094)(11.6828125,248.66386719)
\lineto(11.8234375,247.63261719)
\curveto(11.503125,247.58574219)(11.253125,247.44902344)(11.0734375,247.22246094)
\curveto(10.89375,246.99980469)(10.80390625,246.68339844)(10.80390625,246.27324219)
\curveto(10.80390625,245.78886719)(10.88398438,245.44316406)(11.04414063,245.23613281)
\curveto(11.20429688,245.02910156)(11.39179688,244.92558594)(11.60664063,244.92558594)
\curveto(11.74335938,244.92558594)(11.86640625,244.96855469)(11.97578125,245.05449219)
\curveto(12.0890625,245.14042969)(12.1828125,245.27519531)(12.25703125,245.45878906)
\curveto(12.29609375,245.56425781)(12.3859375,245.87480469)(12.5265625,246.39042969)
\curveto(12.72578125,247.13652344)(12.88984375,247.65605469)(13.01875,247.94902344)
\curveto(13.14375,248.24589844)(13.32734375,248.47832031)(13.56953125,248.64628906)
\curveto(13.81171875,248.81425781)(14.1125,248.89824219)(14.471875,248.89824219)
\curveto(14.8234375,248.89824219)(15.15546875,248.79472656)(15.46796875,248.58769531)
\curveto(15.7765625,248.38457031)(16.01679688,248.08964844)(16.18867188,247.70292969)
\curveto(16.35664063,247.31621094)(16.440625,246.87871094)(16.440625,246.39042969)
\curveto(16.440625,245.58183594)(16.27265625,244.96464844)(15.93671875,244.53886719)
\curveto(15.60078125,244.11699219)(15.10273438,243.84746094)(14.44257813,243.73027344)
\closepath
}
}
{
\newrgbcolor{curcolor}{0 0 0}
\pscustom[linestyle=none,fillstyle=solid,fillcolor=curcolor]
{
\newpath
\moveto(14.29609375,254.41191406)
\lineto(14.43085938,255.50175781)
\curveto(15.06757813,255.32988281)(15.56171875,255.01152344)(15.91328125,254.54667969)
\curveto(16.26484375,254.08183594)(16.440625,253.48808594)(16.440625,252.76542969)
\curveto(16.440625,251.85527344)(16.16132813,251.13261719)(15.60273438,250.59746094)
\curveto(15.04023438,250.06621094)(14.253125,249.80058594)(13.24140625,249.80058594)
\curveto(12.19453125,249.80058594)(11.38203125,250.07011719)(10.80390625,250.60917969)
\curveto(10.22578125,251.14824219)(9.93671875,251.84746094)(9.93671875,252.70683594)
\curveto(9.93671875,253.53886719)(10.21992188,254.21855469)(10.78632813,254.74589844)
\curveto(11.35273438,255.27324219)(12.14960938,255.53691406)(13.17695313,255.53691406)
\curveto(13.23945313,255.53691406)(13.33320313,255.53496094)(13.45820313,255.53105469)
\lineto(13.45820313,250.89042969)
\curveto(14.14179688,250.92949219)(14.66523438,251.12285156)(15.02851563,251.47050781)
\curveto(15.39179688,251.81816406)(15.5734375,252.25175781)(15.5734375,252.77128906)
\curveto(15.5734375,253.15800781)(15.471875,253.48808594)(15.26875,253.76152344)
\curveto(15.065625,254.03496094)(14.74140625,254.25175781)(14.29609375,254.41191406)
\closepath
\moveto(12.59101563,250.94902344)
\lineto(12.59101563,254.42363281)
\curveto(12.06757813,254.37675781)(11.675,254.24394531)(11.41328125,254.02519531)
\curveto(11.00703125,253.68925781)(10.80390625,253.25371094)(10.80390625,252.71855469)
\curveto(10.80390625,252.23417969)(10.96601563,251.82597656)(11.29023438,251.49394531)
\curveto(11.61445313,251.16582031)(12.04804688,250.98417969)(12.59101563,250.94902344)
\closepath
}
}
{
\newrgbcolor{curcolor}{0 0 0}
\pscustom[linestyle=none,fillstyle=solid,fillcolor=curcolor]
{
\newpath
\moveto(14.02070313,260.88652344)
\lineto(14.15546875,261.92363281)
\curveto(14.8703125,261.81035156)(15.43085938,261.51933594)(15.83710938,261.05058594)
\curveto(16.23945313,260.58574219)(16.440625,260.01347656)(16.440625,259.33378906)
\curveto(16.440625,258.48222656)(16.16328125,257.79667969)(15.60859375,257.27714844)
\curveto(15.05,256.76152344)(14.25117188,256.50371094)(13.21210938,256.50371094)
\curveto(12.54023438,256.50371094)(11.95234375,256.61503906)(11.4484375,256.83769531)
\curveto(10.94453125,257.06035156)(10.56757813,257.39824219)(10.31757813,257.85136719)
\curveto(10.06367188,258.30839844)(9.93671875,258.80449219)(9.93671875,259.33964844)
\curveto(9.93671875,260.01542969)(10.10859375,260.56816406)(10.45234375,260.99785156)
\curveto(10.7921875,261.42753906)(11.2765625,261.70292969)(11.90546875,261.82402344)
\lineto(12.06367188,260.79863281)
\curveto(11.64570313,260.70097656)(11.33125,260.52714844)(11.1203125,260.27714844)
\curveto(10.909375,260.03105469)(10.80390625,259.73222656)(10.80390625,259.38066406)
\curveto(10.80390625,258.84941406)(10.9953125,258.41777344)(11.378125,258.08574219)
\curveto(11.75703125,257.75371094)(12.35859375,257.58769531)(13.1828125,257.58769531)
\curveto(14.01875,257.58769531)(14.62617188,257.74785156)(15.00507813,258.06816406)
\curveto(15.38398438,258.38847656)(15.5734375,258.80644531)(15.5734375,259.32207031)
\curveto(15.5734375,259.73613281)(15.44648438,260.08183594)(15.19257813,260.35917969)
\curveto(14.93867188,260.63652344)(14.54804688,260.81230469)(14.02070313,260.88652344)
\closepath
}
}
{
\newrgbcolor{curcolor}{0 0 0}
\pscustom[linestyle=none,fillstyle=solid,fillcolor=curcolor]
{
\newpath
\moveto(13.18867188,262.43339844)
\curveto(12.03632813,262.43339844)(11.1828125,262.75371094)(10.628125,263.39433594)
\curveto(10.1671875,263.92949219)(9.93671875,264.58183594)(9.93671875,265.35136719)
\curveto(9.93671875,266.20683594)(10.21796875,266.90605469)(10.78046875,267.44902344)
\curveto(11.3390625,267.99199219)(12.1125,268.26347656)(13.10078125,268.26347656)
\curveto(13.9015625,268.26347656)(14.53242188,268.14238281)(14.99335938,267.90019531)
\curveto(15.45039063,267.66191406)(15.80585938,267.31230469)(16.05976563,266.85136719)
\curveto(16.31367188,266.39433594)(16.440625,265.89433594)(16.440625,265.35136719)
\curveto(16.440625,264.48027344)(16.16132813,263.77519531)(15.60273438,263.23613281)
\curveto(15.04414063,262.70097656)(14.23945313,262.43339844)(13.18867188,262.43339844)
\closepath
\moveto(13.18867188,263.51738281)
\curveto(13.98554688,263.51738281)(14.58320313,263.69121094)(14.98164063,264.03886719)
\curveto(15.37617188,264.38652344)(15.5734375,264.82402344)(15.5734375,265.35136719)
\curveto(15.5734375,265.87480469)(15.37421875,266.31035156)(14.97578125,266.65800781)
\curveto(14.57734375,267.00566406)(13.96992188,267.17949219)(13.15351563,267.17949219)
\curveto(12.38398438,267.17949219)(11.80195313,267.00371094)(11.40742188,266.65214844)
\curveto(11.00898438,266.30449219)(10.80976563,265.87089844)(10.80976563,265.35136719)
\curveto(10.80976563,264.82402344)(11.00703125,264.38652344)(11.4015625,264.03886719)
\curveto(11.79609375,263.69121094)(12.39179688,263.51738281)(13.18867188,263.51738281)
\closepath
}
}
{
\newrgbcolor{curcolor}{0 0 0}
\pscustom[linestyle=none,fillstyle=solid,fillcolor=curcolor]
{
\newpath
\moveto(16.3,269.49980469)
\lineto(10.07734375,269.49980469)
\lineto(10.07734375,270.44902344)
\lineto(10.96210938,270.44902344)
\curveto(10.27851563,270.90605469)(9.93671875,271.56621094)(9.93671875,272.42949219)
\curveto(9.93671875,272.80449219)(10.00507813,273.14824219)(10.14179688,273.46074219)
\curveto(10.27460938,273.77714844)(10.45039063,274.01347656)(10.66914063,274.16972656)
\curveto(10.88789063,274.32597656)(11.14765625,274.43535156)(11.4484375,274.49785156)
\curveto(11.64375,274.53691406)(11.98554688,274.55644531)(12.47382813,274.55644531)
\lineto(16.3,274.55644531)
\lineto(16.3,273.50175781)
\lineto(12.51484375,273.50175781)
\curveto(12.08515625,273.50175781)(11.76484375,273.46074219)(11.55390625,273.37871094)
\curveto(11.3390625,273.29667969)(11.16914063,273.15019531)(11.04414063,272.93925781)
\curveto(10.91523438,272.73222656)(10.85078125,272.48808594)(10.85078125,272.20683594)
\curveto(10.85078125,271.75761719)(10.99335938,271.36894531)(11.27851563,271.04082031)
\curveto(11.56367188,270.71660156)(12.1046875,270.55449219)(12.9015625,270.55449219)
\lineto(16.3,270.55449219)
\closepath
}
}
{
\newrgbcolor{curcolor}{0 0 0}
\pscustom[linestyle=none,fillstyle=solid,fillcolor=curcolor]
{
\newpath
\moveto(16.3,280.21074219)
\lineto(15.51484375,280.21074219)
\curveto(16.13203125,279.81621094)(16.440625,279.23613281)(16.440625,278.47050781)
\curveto(16.440625,277.97441406)(16.30390625,277.51738281)(16.03046875,277.09941406)
\curveto(15.75703125,276.68535156)(15.37617188,276.36308594)(14.88789063,276.13261719)
\curveto(14.39570313,275.90605469)(13.83125,275.79277344)(13.19453125,275.79277344)
\curveto(12.5734375,275.79277344)(12.0109375,275.89628906)(11.50703125,276.10332031)
\curveto(10.99921875,276.31035156)(10.61054688,276.62089844)(10.34101563,277.03496094)
\curveto(10.07148438,277.44902344)(9.93671875,277.91191406)(9.93671875,278.42363281)
\curveto(9.93671875,278.79863281)(10.01679688,279.13261719)(10.17695313,279.42558594)
\curveto(10.33320313,279.71855469)(10.53828125,279.95683594)(10.7921875,280.14042969)
\lineto(7.71015625,280.14042969)
\lineto(7.71015625,281.18925781)
\lineto(16.3,281.18925781)
\closepath
\moveto(13.19453125,276.87675781)
\curveto(13.99140625,276.87675781)(14.58710938,277.04472656)(14.98164063,277.38066406)
\curveto(15.37617188,277.71660156)(15.5734375,278.11308594)(15.5734375,278.57011719)
\curveto(15.5734375,279.03105469)(15.3859375,279.42167969)(15.0109375,279.74199219)
\curveto(14.63203125,280.06621094)(14.05585938,280.22832031)(13.28242188,280.22832031)
\curveto(12.43085938,280.22832031)(11.80585938,280.06425781)(11.40742188,279.73613281)
\curveto(11.00898438,279.40800781)(10.80976563,279.00371094)(10.80976563,278.52324219)
\curveto(10.80976563,278.05449219)(11.00117188,277.66191406)(11.38398438,277.34550781)
\curveto(11.76679688,277.03300781)(12.3703125,276.87675781)(13.19453125,276.87675781)
\closepath
}
}
{
\newrgbcolor{curcolor}{0 0 0}
\pscustom[linestyle=none,fillstyle=solid,fillcolor=curcolor]
{
\newpath
\moveto(14.44257813,282.42558594)
\lineto(14.27851563,283.46855469)
\curveto(14.69648438,283.52714844)(15.01679688,283.68925781)(15.23945313,283.95488281)
\curveto(15.46210938,284.22441406)(15.5734375,284.59941406)(15.5734375,285.07988281)
\curveto(15.5734375,285.56425781)(15.47578125,285.92363281)(15.28046875,286.15800781)
\curveto(15.08125,286.39238281)(14.84882813,286.50957031)(14.58320313,286.50957031)
\curveto(14.34492188,286.50957031)(14.15742188,286.40605469)(14.02070313,286.19902344)
\curveto(13.92695313,286.05449219)(13.8078125,285.69511719)(13.66328125,285.12089844)
\curveto(13.46796875,284.34746094)(13.3,283.81035156)(13.159375,283.50957031)
\curveto(13.01484375,283.21269531)(12.81757813,282.98613281)(12.56757813,282.82988281)
\curveto(12.31367188,282.67753906)(12.034375,282.60136719)(11.7296875,282.60136719)
\curveto(11.45234375,282.60136719)(11.19648438,282.66386719)(10.96210938,282.78886719)
\curveto(10.72382813,282.91777344)(10.5265625,283.09160156)(10.3703125,283.31035156)
\curveto(10.24921875,283.47441406)(10.14765625,283.69707031)(10.065625,283.97832031)
\curveto(9.9796875,284.26347656)(9.93671875,284.56816406)(9.93671875,284.89238281)
\curveto(9.93671875,285.38066406)(10.00703125,285.80839844)(10.14765625,286.17558594)
\curveto(10.28828125,286.54667969)(10.4796875,286.82011719)(10.721875,286.99589844)
\curveto(10.96015625,287.17167969)(11.28046875,287.29277344)(11.6828125,287.35917969)
\lineto(11.8234375,286.32792969)
\curveto(11.503125,286.28105469)(11.253125,286.14433594)(11.0734375,285.91777344)
\curveto(10.89375,285.69511719)(10.80390625,285.37871094)(10.80390625,284.96855469)
\curveto(10.80390625,284.48417969)(10.88398438,284.13847656)(11.04414063,283.93144531)
\curveto(11.20429688,283.72441406)(11.39179688,283.62089844)(11.60664063,283.62089844)
\curveto(11.74335938,283.62089844)(11.86640625,283.66386719)(11.97578125,283.74980469)
\curveto(12.0890625,283.83574219)(12.1828125,283.97050781)(12.25703125,284.15410156)
\curveto(12.29609375,284.25957031)(12.3859375,284.57011719)(12.5265625,285.08574219)
\curveto(12.72578125,285.83183594)(12.88984375,286.35136719)(13.01875,286.64433594)
\curveto(13.14375,286.94121094)(13.32734375,287.17363281)(13.56953125,287.34160156)
\curveto(13.81171875,287.50957031)(14.1125,287.59355469)(14.471875,287.59355469)
\curveto(14.8234375,287.59355469)(15.15546875,287.49003906)(15.46796875,287.28300781)
\curveto(15.7765625,287.07988281)(16.01679688,286.78496094)(16.18867188,286.39824219)
\curveto(16.35664063,286.01152344)(16.440625,285.57402344)(16.440625,285.08574219)
\curveto(16.440625,284.27714844)(16.27265625,283.65996094)(15.93671875,283.23417969)
\curveto(15.60078125,282.81230469)(15.10273438,282.54277344)(14.44257813,282.42558594)
\closepath
}
}
{
\newrgbcolor{curcolor}{0 0 0}
\pscustom[linestyle=none,fillstyle=solid,fillcolor=curcolor]
{
\newpath
\moveto(18.82539063,289.53886719)
\lineto(18.82539063,288.78300781)
\curveto(16.95039063,289.95097656)(15.0734375,290.53496094)(13.19453125,290.53496094)
\curveto(12.46015625,290.53496094)(11.73164063,290.45097656)(11.00898438,290.28300781)
\curveto(10.42304688,290.15019531)(9.86054688,289.96464844)(9.32148438,289.72636719)
\curveto(8.96992188,289.57402344)(8.38398438,289.25957031)(7.56367188,288.78300781)
\lineto(7.56367188,289.53886719)
\curveto(8.54414063,290.27324219)(9.52851563,290.81621094)(10.51679688,291.16777344)
\curveto(11.36835938,291.46855469)(12.25898438,291.61894531)(13.18867188,291.61894531)
\curveto(14.24335938,291.61894531)(15.26289063,291.41582031)(16.24726563,291.00957031)
\curveto(17.23164063,290.60722656)(18.09101563,290.11699219)(18.82539063,289.53886719)
\closepath
}
}
{
\newrgbcolor{curcolor}{0 0 0}
\pscustom[linestyle=none,fillstyle=solid,fillcolor=curcolor]
{
\newpath
\moveto(317.97753906,8.7)
\lineto(317.97753906,17.28984375)
\lineto(323.77246094,17.28984375)
\lineto(323.77246094,16.27617187)
\lineto(319.11425781,16.27617187)
\lineto(319.11425781,13.61601562)
\lineto(323.14550781,13.61601562)
\lineto(323.14550781,12.60234375)
\lineto(319.11425781,12.60234375)
\lineto(319.11425781,8.7)
\closepath
}
}
{
\newrgbcolor{curcolor}{0 0 0}
\pscustom[linestyle=none,fillstyle=solid,fillcolor=curcolor]
{
\newpath
\moveto(325.12011719,16.07695312)
\lineto(325.12011719,17.28984375)
\lineto(326.17480469,17.28984375)
\lineto(326.17480469,16.07695312)
\closepath
\moveto(325.12011719,8.7)
\lineto(325.12011719,14.92265625)
\lineto(326.17480469,14.92265625)
\lineto(326.17480469,8.7)
\closepath
}
}
{
\newrgbcolor{curcolor}{0 0 0}
\pscustom[linestyle=none,fillstyle=solid,fillcolor=curcolor]
{
\newpath
\moveto(327.75683594,8.7)
\lineto(327.75683594,17.28984375)
\lineto(328.81152344,17.28984375)
\lineto(328.81152344,8.7)
\closepath
}
}
{
\newrgbcolor{curcolor}{0 0 0}
\pscustom[linestyle=none,fillstyle=solid,fillcolor=curcolor]
{
\newpath
\moveto(334.70605469,10.70390625)
\lineto(335.79589844,10.56914062)
\curveto(335.62402344,9.93242187)(335.30566406,9.43828125)(334.84082031,9.08671875)
\curveto(334.37597656,8.73515625)(333.78222656,8.559375)(333.05957031,8.559375)
\curveto(332.14941406,8.559375)(331.42675781,8.83867187)(330.89160156,9.39726562)
\curveto(330.36035156,9.95976562)(330.09472656,10.746875)(330.09472656,11.75859375)
\curveto(330.09472656,12.80546875)(330.36425781,13.61796875)(330.90332031,14.19609375)
\curveto(331.44238281,14.77421875)(332.14160156,15.06328125)(333.00097656,15.06328125)
\curveto(333.83300781,15.06328125)(334.51269531,14.78007812)(335.04003906,14.21367187)
\curveto(335.56738281,13.64726562)(335.83105469,12.85039062)(335.83105469,11.82304687)
\curveto(335.83105469,11.76054687)(335.82910156,11.66679687)(335.82519531,11.54179687)
\lineto(331.18457031,11.54179687)
\curveto(331.22363281,10.85820312)(331.41699219,10.33476562)(331.76464844,9.97148437)
\curveto(332.11230469,9.60820312)(332.54589844,9.4265625)(333.06542969,9.4265625)
\curveto(333.45214844,9.4265625)(333.78222656,9.528125)(334.05566406,9.73125)
\curveto(334.32910156,9.934375)(334.54589844,10.25859375)(334.70605469,10.70390625)
\closepath
\moveto(331.24316406,12.40898437)
\lineto(334.71777344,12.40898437)
\curveto(334.67089844,12.93242187)(334.53808594,13.325)(334.31933594,13.58671875)
\curveto(333.98339844,13.99296875)(333.54785156,14.19609375)(333.01269531,14.19609375)
\curveto(332.52832031,14.19609375)(332.12011719,14.03398437)(331.78808594,13.70976562)
\curveto(331.45996094,13.38554687)(331.27832031,12.95195312)(331.24316406,12.40898437)
\closepath
}
}
{
\newrgbcolor{curcolor}{0 0 0}
\pscustom[linestyle=none,fillstyle=solid,fillcolor=curcolor]
{
\newpath
\moveto(340.20214844,11.45976562)
\lineto(341.27441406,11.55351562)
\curveto(341.32519531,11.12382812)(341.44238281,10.7703125)(341.62597656,10.49296875)
\curveto(341.81347656,10.21953125)(342.10253906,9.996875)(342.49316406,9.825)
\curveto(342.88378906,9.65703125)(343.32324219,9.57304687)(343.81152344,9.57304687)
\curveto(344.24511719,9.57304687)(344.62792969,9.6375)(344.95996094,9.76640625)
\curveto(345.29199219,9.8953125)(345.53808594,10.07109375)(345.69824219,10.29375)
\curveto(345.86230469,10.5203125)(345.94433594,10.76640625)(345.94433594,11.03203125)
\curveto(345.94433594,11.3015625)(345.86621094,11.5359375)(345.70996094,11.73515625)
\curveto(345.55371094,11.93828125)(345.29589844,12.10820312)(344.93652344,12.24492187)
\curveto(344.70605469,12.33476562)(344.19628906,12.4734375)(343.40722656,12.6609375)
\curveto(342.61816406,12.85234375)(342.06542969,13.03203125)(341.74902344,13.2)
\curveto(341.33886719,13.41484375)(341.03222656,13.68046875)(340.82910156,13.996875)
\curveto(340.62988281,14.3171875)(340.53027344,14.67460937)(340.53027344,15.06914062)
\curveto(340.53027344,15.50273437)(340.65332031,15.90703125)(340.89941406,16.28203125)
\curveto(341.14550781,16.6609375)(341.50488281,16.94804687)(341.97753906,17.14335937)
\curveto(342.45019531,17.33867187)(342.97558594,17.43632812)(343.55371094,17.43632812)
\curveto(344.19042969,17.43632812)(344.75097656,17.3328125)(345.23535156,17.12578125)
\curveto(345.72363281,16.92265625)(346.09863281,16.621875)(346.36035156,16.2234375)
\curveto(346.62207031,15.825)(346.76269531,15.37382812)(346.78222656,14.86992187)
\lineto(345.69238281,14.78789062)
\curveto(345.63378906,15.33085937)(345.43457031,15.74101562)(345.09472656,16.01835937)
\curveto(344.75878906,16.29570312)(344.26074219,16.434375)(343.60058594,16.434375)
\curveto(342.91308594,16.434375)(342.41113281,16.30742187)(342.09472656,16.05351562)
\curveto(341.78222656,15.80351562)(341.62597656,15.50078125)(341.62597656,15.1453125)
\curveto(341.62597656,14.83671875)(341.73730469,14.5828125)(341.95996094,14.38359375)
\curveto(342.17871094,14.184375)(342.74902344,13.97929687)(343.67089844,13.76835937)
\curveto(344.59667969,13.56132812)(345.23144531,13.3796875)(345.57519531,13.2234375)
\curveto(346.07519531,12.99296875)(346.44433594,12.7)(346.68261719,12.34453125)
\curveto(346.92089844,11.99296875)(347.04003906,11.58671875)(347.04003906,11.12578125)
\curveto(347.04003906,10.66875)(346.90917969,10.23710937)(346.64746094,9.83085937)
\curveto(346.38574219,9.42851562)(346.00878906,9.1140625)(345.51660156,8.8875)
\curveto(345.02832031,8.66484375)(344.47753906,8.55351562)(343.86425781,8.55351562)
\curveto(343.08691406,8.55351562)(342.43457031,8.66679687)(341.90722656,8.89335937)
\curveto(341.38378906,9.11992187)(340.97167969,9.45976562)(340.67089844,9.91289062)
\curveto(340.37402344,10.36992187)(340.21777344,10.88554687)(340.20214844,11.45976562)
\closepath
}
}
{
\newrgbcolor{curcolor}{0 0 0}
\pscustom[linestyle=none,fillstyle=solid,fillcolor=curcolor]
{
\newpath
\moveto(348.46386719,16.07695312)
\lineto(348.46386719,17.28984375)
\lineto(349.51855469,17.28984375)
\lineto(349.51855469,16.07695312)
\closepath
\moveto(348.46386719,8.7)
\lineto(348.46386719,14.92265625)
\lineto(349.51855469,14.92265625)
\lineto(349.51855469,8.7)
\closepath
}
}
{
\newrgbcolor{curcolor}{0 0 0}
\pscustom[linestyle=none,fillstyle=solid,fillcolor=curcolor]
{
\newpath
\moveto(350.56738281,8.7)
\lineto(350.56738281,9.55546875)
\lineto(354.52832031,14.10234375)
\curveto(354.07910156,14.07890625)(353.68261719,14.0671875)(353.33886719,14.0671875)
\lineto(350.80175781,14.0671875)
\lineto(350.80175781,14.92265625)
\lineto(355.88769531,14.92265625)
\lineto(355.88769531,14.22539062)
\lineto(352.51855469,10.27617187)
\lineto(351.86816406,9.55546875)
\curveto(352.34082031,9.590625)(352.78417969,9.60820312)(353.19824219,9.60820312)
\lineto(356.07519531,9.60820312)
\lineto(356.07519531,8.7)
\closepath
}
}
{
\newrgbcolor{curcolor}{0 0 0}
\pscustom[linestyle=none,fillstyle=solid,fillcolor=curcolor]
{
\newpath
\moveto(361.38378906,10.70390625)
\lineto(362.47363281,10.56914062)
\curveto(362.30175781,9.93242187)(361.98339844,9.43828125)(361.51855469,9.08671875)
\curveto(361.05371094,8.73515625)(360.45996094,8.559375)(359.73730469,8.559375)
\curveto(358.82714844,8.559375)(358.10449219,8.83867187)(357.56933594,9.39726562)
\curveto(357.03808594,9.95976562)(356.77246094,10.746875)(356.77246094,11.75859375)
\curveto(356.77246094,12.80546875)(357.04199219,13.61796875)(357.58105469,14.19609375)
\curveto(358.12011719,14.77421875)(358.81933594,15.06328125)(359.67871094,15.06328125)
\curveto(360.51074219,15.06328125)(361.19042969,14.78007812)(361.71777344,14.21367187)
\curveto(362.24511719,13.64726562)(362.50878906,12.85039062)(362.50878906,11.82304687)
\curveto(362.50878906,11.76054687)(362.50683594,11.66679687)(362.50292969,11.54179687)
\lineto(357.86230469,11.54179687)
\curveto(357.90136719,10.85820312)(358.09472656,10.33476562)(358.44238281,9.97148437)
\curveto(358.79003906,9.60820312)(359.22363281,9.4265625)(359.74316406,9.4265625)
\curveto(360.12988281,9.4265625)(360.45996094,9.528125)(360.73339844,9.73125)
\curveto(361.00683594,9.934375)(361.22363281,10.25859375)(361.38378906,10.70390625)
\closepath
\moveto(357.92089844,12.40898437)
\lineto(361.39550781,12.40898437)
\curveto(361.34863281,12.93242187)(361.21582031,13.325)(360.99707031,13.58671875)
\curveto(360.66113281,13.99296875)(360.22558594,14.19609375)(359.69042969,14.19609375)
\curveto(359.20605469,14.19609375)(358.79785156,14.03398437)(358.46582031,13.70976562)
\curveto(358.13769531,13.38554687)(357.95605469,12.95195312)(357.92089844,12.40898437)
\closepath
}
}
{
\newrgbcolor{curcolor}{0 0 0}
\pscustom[linestyle=none,fillstyle=solid,fillcolor=curcolor]
{
\newpath
\moveto(169.98144531,449)
\lineto(169.98144531,457.58984375)
\lineto(173.79003906,457.58984375)
\curveto(174.55566406,457.58984375)(175.13769531,457.51171875)(175.53613281,457.35546875)
\curveto(175.93457031,457.203125)(176.25292969,456.93164062)(176.49121094,456.54101562)
\curveto(176.72949219,456.15039062)(176.84863281,455.71875)(176.84863281,455.24609375)
\curveto(176.84863281,454.63671875)(176.65136719,454.12304688)(176.25683594,453.70507812)
\curveto(175.86230469,453.28710938)(175.25292969,453.02148438)(174.42871094,452.90820312)
\curveto(174.72949219,452.76367188)(174.95800781,452.62109375)(175.11425781,452.48046875)
\curveto(175.44628906,452.17578125)(175.76074219,451.79492188)(176.05761719,451.33789062)
\lineto(177.55175781,449)
\lineto(176.12207031,449)
\lineto(174.98535156,450.78710938)
\curveto(174.65332031,451.30273438)(174.37988281,451.69726562)(174.16503906,451.97070312)
\curveto(173.95019531,452.24414062)(173.75683594,452.43554688)(173.58496094,452.54492188)
\curveto(173.41699219,452.65429688)(173.24511719,452.73046875)(173.06933594,452.7734375)
\curveto(172.94042969,452.80078125)(172.72949219,452.81445312)(172.43652344,452.81445312)
\lineto(171.11816406,452.81445312)
\lineto(171.11816406,449)
\closepath
\moveto(171.11816406,453.79882812)
\lineto(173.56152344,453.79882812)
\curveto(174.08105469,453.79882812)(174.48730469,453.8515625)(174.78027344,453.95703125)
\curveto(175.07324219,454.06640625)(175.29589844,454.23828125)(175.44824219,454.47265625)
\curveto(175.60058594,454.7109375)(175.67675781,454.96875)(175.67675781,455.24609375)
\curveto(175.67675781,455.65234375)(175.52832031,455.98632812)(175.23144531,456.24804688)
\curveto(174.93847656,456.50976562)(174.47363281,456.640625)(173.83691406,456.640625)
\lineto(171.11816406,456.640625)
\closepath
}
}
{
\newrgbcolor{curcolor}{0 0 0}
\pscustom[linestyle=none,fillstyle=solid,fillcolor=curcolor]
{
\newpath
\moveto(182.57324219,449)
\lineto(182.57324219,449.9140625)
\curveto(182.08886719,449.2109375)(181.43066406,448.859375)(180.59863281,448.859375)
\curveto(180.23144531,448.859375)(179.88769531,448.9296875)(179.56738281,449.0703125)
\curveto(179.25097656,449.2109375)(179.01464844,449.38671875)(178.85839844,449.59765625)
\curveto(178.70605469,449.8125)(178.59863281,450.07421875)(178.53613281,450.3828125)
\curveto(178.49316406,450.58984375)(178.47167969,450.91796875)(178.47167969,451.3671875)
\lineto(178.47167969,455.22265625)
\lineto(179.52636719,455.22265625)
\lineto(179.52636719,451.77148438)
\curveto(179.52636719,451.22070312)(179.54785156,450.84960938)(179.59082031,450.65820312)
\curveto(179.65722656,450.38085938)(179.79785156,450.16210938)(180.01269531,450.00195312)
\curveto(180.22753906,449.84570312)(180.49316406,449.76757812)(180.80957031,449.76757812)
\curveto(181.12597656,449.76757812)(181.42285156,449.84765625)(181.70019531,450.0078125)
\curveto(181.97753906,450.171875)(182.17285156,450.39257812)(182.28613281,450.66992188)
\curveto(182.40332031,450.95117188)(182.46191406,451.35742188)(182.46191406,451.88867188)
\lineto(182.46191406,455.22265625)
\lineto(183.51660156,455.22265625)
\lineto(183.51660156,449)
\closepath
}
}
{
\newrgbcolor{curcolor}{0 0 0}
\pscustom[linestyle=none,fillstyle=solid,fillcolor=curcolor]
{
\newpath
\moveto(185.16894531,449)
\lineto(185.16894531,455.22265625)
\lineto(186.11816406,455.22265625)
\lineto(186.11816406,454.33789062)
\curveto(186.57519531,455.02148438)(187.23535156,455.36328125)(188.09863281,455.36328125)
\curveto(188.47363281,455.36328125)(188.81738281,455.29492188)(189.12988281,455.15820312)
\curveto(189.44628906,455.02539062)(189.68261719,454.84960938)(189.83886719,454.63085938)
\curveto(189.99511719,454.41210938)(190.10449219,454.15234375)(190.16699219,453.8515625)
\curveto(190.20605469,453.65625)(190.22558594,453.31445312)(190.22558594,452.82617188)
\lineto(190.22558594,449)
\lineto(189.17089844,449)
\lineto(189.17089844,452.78515625)
\curveto(189.17089844,453.21484375)(189.12988281,453.53515625)(189.04785156,453.74609375)
\curveto(188.96582031,453.9609375)(188.81933594,454.13085938)(188.60839844,454.25585938)
\curveto(188.40136719,454.38476562)(188.15722656,454.44921875)(187.87597656,454.44921875)
\curveto(187.42675781,454.44921875)(187.03808594,454.30664062)(186.70996094,454.02148438)
\curveto(186.38574219,453.73632812)(186.22363281,453.1953125)(186.22363281,452.3984375)
\lineto(186.22363281,449)
\closepath
}
}
{
\newrgbcolor{curcolor}{0 0 0}
\pscustom[linestyle=none,fillstyle=solid,fillcolor=curcolor]
{
\newpath
\moveto(194.14550781,449.94335938)
\lineto(194.29785156,449.01171875)
\curveto(194.00097656,448.94921875)(193.73535156,448.91796875)(193.50097656,448.91796875)
\curveto(193.11816406,448.91796875)(192.82128906,448.97851562)(192.61035156,449.09960938)
\curveto(192.39941406,449.22070312)(192.25097656,449.37890625)(192.16503906,449.57421875)
\curveto(192.07910156,449.7734375)(192.03613281,450.18945312)(192.03613281,450.82226562)
\lineto(192.03613281,454.40234375)
\lineto(191.26269531,454.40234375)
\lineto(191.26269531,455.22265625)
\lineto(192.03613281,455.22265625)
\lineto(192.03613281,456.76367188)
\lineto(193.08496094,457.39648438)
\lineto(193.08496094,455.22265625)
\lineto(194.14550781,455.22265625)
\lineto(194.14550781,454.40234375)
\lineto(193.08496094,454.40234375)
\lineto(193.08496094,450.76367188)
\curveto(193.08496094,450.46289062)(193.10253906,450.26953125)(193.13769531,450.18359375)
\curveto(193.17675781,450.09765625)(193.23730469,450.02929688)(193.31933594,449.97851562)
\curveto(193.40527344,449.92773438)(193.52636719,449.90234375)(193.68261719,449.90234375)
\curveto(193.79980469,449.90234375)(193.95410156,449.91601562)(194.14550781,449.94335938)
\closepath
}
}
{
\newrgbcolor{curcolor}{0 0 0}
\pscustom[linestyle=none,fillstyle=solid,fillcolor=curcolor]
{
\newpath
\moveto(195.18261719,456.37695312)
\lineto(195.18261719,457.58984375)
\lineto(196.23730469,457.58984375)
\lineto(196.23730469,456.37695312)
\closepath
\moveto(195.18261719,449)
\lineto(195.18261719,455.22265625)
\lineto(196.23730469,455.22265625)
\lineto(196.23730469,449)
\closepath
}
}
{
\newrgbcolor{curcolor}{0 0 0}
\pscustom[linestyle=none,fillstyle=solid,fillcolor=curcolor]
{
\newpath
\moveto(197.84277344,449)
\lineto(197.84277344,455.22265625)
\lineto(198.78613281,455.22265625)
\lineto(198.78613281,454.34960938)
\curveto(198.98144531,454.65429688)(199.24121094,454.8984375)(199.56542969,455.08203125)
\curveto(199.88964844,455.26953125)(200.25878906,455.36328125)(200.67285156,455.36328125)
\curveto(201.13378906,455.36328125)(201.51074219,455.26757812)(201.80371094,455.07617188)
\curveto(202.10058594,454.88476562)(202.30957031,454.6171875)(202.43066406,454.2734375)
\curveto(202.92285156,455)(203.56347656,455.36328125)(204.35253906,455.36328125)
\curveto(204.96972656,455.36328125)(205.44433594,455.19140625)(205.77636719,454.84765625)
\curveto(206.10839844,454.5078125)(206.27441406,453.98242188)(206.27441406,453.27148438)
\lineto(206.27441406,449)
\lineto(205.22558594,449)
\lineto(205.22558594,452.91992188)
\curveto(205.22558594,453.34179688)(205.19042969,453.64453125)(205.12011719,453.828125)
\curveto(205.05371094,454.015625)(204.93066406,454.16601562)(204.75097656,454.27929688)
\curveto(204.57128906,454.39257812)(204.36035156,454.44921875)(204.11816406,454.44921875)
\curveto(203.68066406,454.44921875)(203.31738281,454.30273438)(203.02832031,454.00976562)
\curveto(202.73925781,453.72070312)(202.59472656,453.25585938)(202.59472656,452.61523438)
\lineto(202.59472656,449)
\lineto(201.54003906,449)
\lineto(201.54003906,453.04296875)
\curveto(201.54003906,453.51171875)(201.45410156,453.86328125)(201.28222656,454.09765625)
\curveto(201.11035156,454.33203125)(200.82910156,454.44921875)(200.43847656,454.44921875)
\curveto(200.14160156,454.44921875)(199.86621094,454.37109375)(199.61230469,454.21484375)
\curveto(199.36230469,454.05859375)(199.18066406,453.83007812)(199.06738281,453.52929688)
\curveto(198.95410156,453.22851562)(198.89746094,452.79492188)(198.89746094,452.22851562)
\lineto(198.89746094,449)
\closepath
}
}
{
\newrgbcolor{curcolor}{0 0 0}
\pscustom[linestyle=none,fillstyle=solid,fillcolor=curcolor]
{
\newpath
\moveto(212.09863281,451.00390625)
\lineto(213.18847656,450.86914062)
\curveto(213.01660156,450.23242188)(212.69824219,449.73828125)(212.23339844,449.38671875)
\curveto(211.76855469,449.03515625)(211.17480469,448.859375)(210.45214844,448.859375)
\curveto(209.54199219,448.859375)(208.81933594,449.13867188)(208.28417969,449.69726562)
\curveto(207.75292969,450.25976562)(207.48730469,451.046875)(207.48730469,452.05859375)
\curveto(207.48730469,453.10546875)(207.75683594,453.91796875)(208.29589844,454.49609375)
\curveto(208.83496094,455.07421875)(209.53417969,455.36328125)(210.39355469,455.36328125)
\curveto(211.22558594,455.36328125)(211.90527344,455.08007812)(212.43261719,454.51367188)
\curveto(212.95996094,453.94726562)(213.22363281,453.15039062)(213.22363281,452.12304688)
\curveto(213.22363281,452.06054688)(213.22167969,451.96679688)(213.21777344,451.84179688)
\lineto(208.57714844,451.84179688)
\curveto(208.61621094,451.15820312)(208.80957031,450.63476562)(209.15722656,450.27148438)
\curveto(209.50488281,449.90820312)(209.93847656,449.7265625)(210.45800781,449.7265625)
\curveto(210.84472656,449.7265625)(211.17480469,449.828125)(211.44824219,450.03125)
\curveto(211.72167969,450.234375)(211.93847656,450.55859375)(212.09863281,451.00390625)
\closepath
\moveto(208.63574219,452.70898438)
\lineto(212.11035156,452.70898438)
\curveto(212.06347656,453.23242188)(211.93066406,453.625)(211.71191406,453.88671875)
\curveto(211.37597656,454.29296875)(210.94042969,454.49609375)(210.40527344,454.49609375)
\curveto(209.92089844,454.49609375)(209.51269531,454.33398438)(209.18066406,454.00976562)
\curveto(208.85253906,453.68554688)(208.67089844,453.25195312)(208.63574219,452.70898438)
\closepath
}
}
{
\newrgbcolor{curcolor}{0 0 0}
\pscustom[linestyle=none,fillstyle=solid,fillcolor=curcolor]
{
\newpath
\moveto(217.98144531,449)
\lineto(217.98144531,457.58984375)
\lineto(220.94042969,457.58984375)
\curveto(221.60839844,457.58984375)(222.11816406,457.54882812)(222.46972656,457.46679688)
\curveto(222.96191406,457.35351562)(223.38183594,457.1484375)(223.72949219,456.8515625)
\curveto(224.18261719,456.46875)(224.52050781,455.97851562)(224.74316406,455.38085938)
\curveto(224.96972656,454.78710938)(225.08300781,454.10742188)(225.08300781,453.34179688)
\curveto(225.08300781,452.68945312)(225.00683594,452.11132812)(224.85449219,451.60742188)
\curveto(224.70214844,451.10351562)(224.50683594,450.68554688)(224.26855469,450.35351562)
\curveto(224.03027344,450.02539062)(223.76855469,449.765625)(223.48339844,449.57421875)
\curveto(223.20214844,449.38671875)(222.86035156,449.24414062)(222.45800781,449.14648438)
\curveto(222.05957031,449.04882812)(221.60058594,449)(221.08105469,449)
\closepath
\moveto(219.11816406,450.01367188)
\lineto(220.95214844,450.01367188)
\curveto(221.51855469,450.01367188)(221.96191406,450.06640625)(222.28222656,450.171875)
\curveto(222.60644531,450.27734375)(222.86425781,450.42578125)(223.05566406,450.6171875)
\curveto(223.32519531,450.88671875)(223.53417969,451.24804688)(223.68261719,451.70117188)
\curveto(223.83496094,452.15820312)(223.91113281,452.7109375)(223.91113281,453.359375)
\curveto(223.91113281,454.2578125)(223.76269531,454.94726562)(223.46582031,455.42773438)
\curveto(223.17285156,455.91210938)(222.81542969,456.23632812)(222.39355469,456.40039062)
\curveto(222.08886719,456.51757812)(221.59863281,456.57617188)(220.92285156,456.57617188)
\lineto(219.11816406,456.57617188)
\closepath
}
}
{
\newrgbcolor{curcolor}{0 0 0}
\pscustom[linestyle=none,fillstyle=solid,fillcolor=curcolor]
{
\newpath
\moveto(226.51855469,456.37695312)
\lineto(226.51855469,457.58984375)
\lineto(227.57324219,457.58984375)
\lineto(227.57324219,456.37695312)
\closepath
\moveto(226.51855469,449)
\lineto(226.51855469,455.22265625)
\lineto(227.57324219,455.22265625)
\lineto(227.57324219,449)
\closepath
}
}
{
\newrgbcolor{curcolor}{0 0 0}
\pscustom[linestyle=none,fillstyle=solid,fillcolor=curcolor]
{
\newpath
\moveto(229.43066406,449)
\lineto(229.43066406,454.40234375)
\lineto(228.49902344,454.40234375)
\lineto(228.49902344,455.22265625)
\lineto(229.43066406,455.22265625)
\lineto(229.43066406,455.88476562)
\curveto(229.43066406,456.30273438)(229.46777344,456.61328125)(229.54199219,456.81640625)
\curveto(229.64355469,457.08984375)(229.82128906,457.31054688)(230.07519531,457.47851562)
\curveto(230.33300781,457.65039062)(230.69238281,457.73632812)(231.15332031,457.73632812)
\curveto(231.45019531,457.73632812)(231.77832031,457.70117188)(232.13769531,457.63085938)
\lineto(231.97949219,456.7109375)
\curveto(231.76074219,456.75)(231.55371094,456.76953125)(231.35839844,456.76953125)
\curveto(231.03808594,456.76953125)(230.81152344,456.70117188)(230.67871094,456.56445312)
\curveto(230.54589844,456.42773438)(230.47949219,456.171875)(230.47949219,455.796875)
\lineto(230.47949219,455.22265625)
\lineto(231.69238281,455.22265625)
\lineto(231.69238281,454.40234375)
\lineto(230.47949219,454.40234375)
\lineto(230.47949219,449)
\closepath
}
}
{
\newrgbcolor{curcolor}{0 0 0}
\pscustom[linestyle=none,fillstyle=solid,fillcolor=curcolor]
{
\newpath
\moveto(232.54785156,449)
\lineto(232.54785156,454.40234375)
\lineto(231.61621094,454.40234375)
\lineto(231.61621094,455.22265625)
\lineto(232.54785156,455.22265625)
\lineto(232.54785156,455.88476562)
\curveto(232.54785156,456.30273438)(232.58496094,456.61328125)(232.65917969,456.81640625)
\curveto(232.76074219,457.08984375)(232.93847656,457.31054688)(233.19238281,457.47851562)
\curveto(233.45019531,457.65039062)(233.80957031,457.73632812)(234.27050781,457.73632812)
\curveto(234.56738281,457.73632812)(234.89550781,457.70117188)(235.25488281,457.63085938)
\lineto(235.09667969,456.7109375)
\curveto(234.87792969,456.75)(234.67089844,456.76953125)(234.47558594,456.76953125)
\curveto(234.15527344,456.76953125)(233.92871094,456.70117188)(233.79589844,456.56445312)
\curveto(233.66308594,456.42773438)(233.59667969,456.171875)(233.59667969,455.796875)
\lineto(233.59667969,455.22265625)
\lineto(234.80957031,455.22265625)
\lineto(234.80957031,454.40234375)
\lineto(233.59667969,454.40234375)
\lineto(233.59667969,449)
\closepath
}
}
{
\newrgbcolor{curcolor}{0 0 0}
\pscustom[linestyle=none,fillstyle=solid,fillcolor=curcolor]
{
\newpath
\moveto(239.88964844,451.00390625)
\lineto(240.97949219,450.86914062)
\curveto(240.80761719,450.23242188)(240.48925781,449.73828125)(240.02441406,449.38671875)
\curveto(239.55957031,449.03515625)(238.96582031,448.859375)(238.24316406,448.859375)
\curveto(237.33300781,448.859375)(236.61035156,449.13867188)(236.07519531,449.69726562)
\curveto(235.54394531,450.25976562)(235.27832031,451.046875)(235.27832031,452.05859375)
\curveto(235.27832031,453.10546875)(235.54785156,453.91796875)(236.08691406,454.49609375)
\curveto(236.62597656,455.07421875)(237.32519531,455.36328125)(238.18457031,455.36328125)
\curveto(239.01660156,455.36328125)(239.69628906,455.08007812)(240.22363281,454.51367188)
\curveto(240.75097656,453.94726562)(241.01464844,453.15039062)(241.01464844,452.12304688)
\curveto(241.01464844,452.06054688)(241.01269531,451.96679688)(241.00878906,451.84179688)
\lineto(236.36816406,451.84179688)
\curveto(236.40722656,451.15820312)(236.60058594,450.63476562)(236.94824219,450.27148438)
\curveto(237.29589844,449.90820312)(237.72949219,449.7265625)(238.24902344,449.7265625)
\curveto(238.63574219,449.7265625)(238.96582031,449.828125)(239.23925781,450.03125)
\curveto(239.51269531,450.234375)(239.72949219,450.55859375)(239.88964844,451.00390625)
\closepath
\moveto(236.42675781,452.70898438)
\lineto(239.90136719,452.70898438)
\curveto(239.85449219,453.23242188)(239.72167969,453.625)(239.50292969,453.88671875)
\curveto(239.16699219,454.29296875)(238.73144531,454.49609375)(238.19628906,454.49609375)
\curveto(237.71191406,454.49609375)(237.30371094,454.33398438)(236.97167969,454.00976562)
\curveto(236.64355469,453.68554688)(236.46191406,453.25195312)(236.42675781,452.70898438)
\closepath
}
}
{
\newrgbcolor{curcolor}{0 0 0}
\pscustom[linestyle=none,fillstyle=solid,fillcolor=curcolor]
{
\newpath
\moveto(242.29199219,449)
\lineto(242.29199219,455.22265625)
\lineto(243.24121094,455.22265625)
\lineto(243.24121094,454.27929688)
\curveto(243.48339844,454.72070312)(243.70605469,455.01171875)(243.90917969,455.15234375)
\curveto(244.11621094,455.29296875)(244.34277344,455.36328125)(244.58886719,455.36328125)
\curveto(244.94433594,455.36328125)(245.30566406,455.25)(245.67285156,455.0234375)
\lineto(245.30957031,454.04492188)
\curveto(245.05175781,454.19726562)(244.79394531,454.2734375)(244.53613281,454.2734375)
\curveto(244.30566406,454.2734375)(244.09863281,454.203125)(243.91503906,454.0625)
\curveto(243.73144531,453.92578125)(243.60058594,453.734375)(243.52246094,453.48828125)
\curveto(243.40527344,453.11328125)(243.34667969,452.703125)(243.34667969,452.2578125)
\lineto(243.34667969,449)
\closepath
}
}
{
\newrgbcolor{curcolor}{0 0 0}
\pscustom[linestyle=none,fillstyle=solid,fillcolor=curcolor]
{
\newpath
\moveto(250.55957031,451.00390625)
\lineto(251.64941406,450.86914062)
\curveto(251.47753906,450.23242188)(251.15917969,449.73828125)(250.69433594,449.38671875)
\curveto(250.22949219,449.03515625)(249.63574219,448.859375)(248.91308594,448.859375)
\curveto(248.00292969,448.859375)(247.28027344,449.13867188)(246.74511719,449.69726562)
\curveto(246.21386719,450.25976562)(245.94824219,451.046875)(245.94824219,452.05859375)
\curveto(245.94824219,453.10546875)(246.21777344,453.91796875)(246.75683594,454.49609375)
\curveto(247.29589844,455.07421875)(247.99511719,455.36328125)(248.85449219,455.36328125)
\curveto(249.68652344,455.36328125)(250.36621094,455.08007812)(250.89355469,454.51367188)
\curveto(251.42089844,453.94726562)(251.68457031,453.15039062)(251.68457031,452.12304688)
\curveto(251.68457031,452.06054688)(251.68261719,451.96679688)(251.67871094,451.84179688)
\lineto(247.03808594,451.84179688)
\curveto(247.07714844,451.15820312)(247.27050781,450.63476562)(247.61816406,450.27148438)
\curveto(247.96582031,449.90820312)(248.39941406,449.7265625)(248.91894531,449.7265625)
\curveto(249.30566406,449.7265625)(249.63574219,449.828125)(249.90917969,450.03125)
\curveto(250.18261719,450.234375)(250.39941406,450.55859375)(250.55957031,451.00390625)
\closepath
\moveto(247.09667969,452.70898438)
\lineto(250.57128906,452.70898438)
\curveto(250.52441406,453.23242188)(250.39160156,453.625)(250.17285156,453.88671875)
\curveto(249.83691406,454.29296875)(249.40136719,454.49609375)(248.86621094,454.49609375)
\curveto(248.38183594,454.49609375)(247.97363281,454.33398438)(247.64160156,454.00976562)
\curveto(247.31347656,453.68554688)(247.13183594,453.25195312)(247.09667969,452.70898438)
\closepath
}
}
{
\newrgbcolor{curcolor}{0 0 0}
\pscustom[linestyle=none,fillstyle=solid,fillcolor=curcolor]
{
\newpath
\moveto(252.97363281,449)
\lineto(252.97363281,455.22265625)
\lineto(253.92285156,455.22265625)
\lineto(253.92285156,454.33789062)
\curveto(254.37988281,455.02148438)(255.04003906,455.36328125)(255.90332031,455.36328125)
\curveto(256.27832031,455.36328125)(256.62207031,455.29492188)(256.93457031,455.15820312)
\curveto(257.25097656,455.02539062)(257.48730469,454.84960938)(257.64355469,454.63085938)
\curveto(257.79980469,454.41210938)(257.90917969,454.15234375)(257.97167969,453.8515625)
\curveto(258.01074219,453.65625)(258.03027344,453.31445312)(258.03027344,452.82617188)
\lineto(258.03027344,449)
\lineto(256.97558594,449)
\lineto(256.97558594,452.78515625)
\curveto(256.97558594,453.21484375)(256.93457031,453.53515625)(256.85253906,453.74609375)
\curveto(256.77050781,453.9609375)(256.62402344,454.13085938)(256.41308594,454.25585938)
\curveto(256.20605469,454.38476562)(255.96191406,454.44921875)(255.68066406,454.44921875)
\curveto(255.23144531,454.44921875)(254.84277344,454.30664062)(254.51464844,454.02148438)
\curveto(254.19042969,453.73632812)(254.02832031,453.1953125)(254.02832031,452.3984375)
\lineto(254.02832031,449)
\closepath
}
}
{
\newrgbcolor{curcolor}{0 0 0}
\pscustom[linestyle=none,fillstyle=solid,fillcolor=curcolor]
{
\newpath
\moveto(261.95019531,449.94335938)
\lineto(262.10253906,449.01171875)
\curveto(261.80566406,448.94921875)(261.54003906,448.91796875)(261.30566406,448.91796875)
\curveto(260.92285156,448.91796875)(260.62597656,448.97851562)(260.41503906,449.09960938)
\curveto(260.20410156,449.22070312)(260.05566406,449.37890625)(259.96972656,449.57421875)
\curveto(259.88378906,449.7734375)(259.84082031,450.18945312)(259.84082031,450.82226562)
\lineto(259.84082031,454.40234375)
\lineto(259.06738281,454.40234375)
\lineto(259.06738281,455.22265625)
\lineto(259.84082031,455.22265625)
\lineto(259.84082031,456.76367188)
\lineto(260.88964844,457.39648438)
\lineto(260.88964844,455.22265625)
\lineto(261.95019531,455.22265625)
\lineto(261.95019531,454.40234375)
\lineto(260.88964844,454.40234375)
\lineto(260.88964844,450.76367188)
\curveto(260.88964844,450.46289062)(260.90722656,450.26953125)(260.94238281,450.18359375)
\curveto(260.98144531,450.09765625)(261.04199219,450.02929688)(261.12402344,449.97851562)
\curveto(261.20996094,449.92773438)(261.33105469,449.90234375)(261.48730469,449.90234375)
\curveto(261.60449219,449.90234375)(261.75878906,449.91601562)(261.95019531,449.94335938)
\closepath
}
}
{
\newrgbcolor{curcolor}{0 0 0}
\pscustom[linestyle=none,fillstyle=solid,fillcolor=curcolor]
{
\newpath
\moveto(266.43847656,449)
\lineto(266.43847656,457.58984375)
\lineto(267.60449219,457.58984375)
\lineto(272.11621094,450.84570312)
\lineto(272.11621094,457.58984375)
\lineto(273.20605469,457.58984375)
\lineto(273.20605469,449)
\lineto(272.04003906,449)
\lineto(267.52832031,455.75)
\lineto(267.52832031,449)
\closepath
}
}
{
\newrgbcolor{curcolor}{0 0 0}
\pscustom[linestyle=none,fillstyle=solid,fillcolor=curcolor]
{
\newpath
\moveto(274.58886719,452.11132812)
\curveto(274.58886719,453.26367188)(274.90917969,454.1171875)(275.54980469,454.671875)
\curveto(276.08496094,455.1328125)(276.73730469,455.36328125)(277.50683594,455.36328125)
\curveto(278.36230469,455.36328125)(279.06152344,455.08203125)(279.60449219,454.51953125)
\curveto(280.14746094,453.9609375)(280.41894531,453.1875)(280.41894531,452.19921875)
\curveto(280.41894531,451.3984375)(280.29785156,450.76757812)(280.05566406,450.30664062)
\curveto(279.81738281,449.84960938)(279.46777344,449.49414062)(279.00683594,449.24023438)
\curveto(278.54980469,448.98632812)(278.04980469,448.859375)(277.50683594,448.859375)
\curveto(276.63574219,448.859375)(275.93066406,449.13867188)(275.39160156,449.69726562)
\curveto(274.85644531,450.25585938)(274.58886719,451.06054688)(274.58886719,452.11132812)
\closepath
\moveto(275.67285156,452.11132812)
\curveto(275.67285156,451.31445312)(275.84667969,450.71679688)(276.19433594,450.31835938)
\curveto(276.54199219,449.92382812)(276.97949219,449.7265625)(277.50683594,449.7265625)
\curveto(278.03027344,449.7265625)(278.46582031,449.92578125)(278.81347656,450.32421875)
\curveto(279.16113281,450.72265625)(279.33496094,451.33007812)(279.33496094,452.14648438)
\curveto(279.33496094,452.91601562)(279.15917969,453.49804688)(278.80761719,453.89257812)
\curveto(278.45996094,454.29101562)(278.02636719,454.49023438)(277.50683594,454.49023438)
\curveto(276.97949219,454.49023438)(276.54199219,454.29296875)(276.19433594,453.8984375)
\curveto(275.84667969,453.50390625)(275.67285156,452.90820312)(275.67285156,452.11132812)
\closepath
}
}
{
\newrgbcolor{curcolor}{0 0 0}
\pscustom[linestyle=none,fillstyle=solid,fillcolor=curcolor]
{
\newpath
\moveto(285.69238281,449)
\lineto(285.69238281,449.78515625)
\curveto(285.29785156,449.16796875)(284.71777344,448.859375)(283.95214844,448.859375)
\curveto(283.45605469,448.859375)(282.99902344,448.99609375)(282.58105469,449.26953125)
\curveto(282.16699219,449.54296875)(281.84472656,449.92382812)(281.61425781,450.41210938)
\curveto(281.38769531,450.90429688)(281.27441406,451.46875)(281.27441406,452.10546875)
\curveto(281.27441406,452.7265625)(281.37792969,453.2890625)(281.58496094,453.79296875)
\curveto(281.79199219,454.30078125)(282.10253906,454.68945312)(282.51660156,454.95898438)
\curveto(282.93066406,455.22851562)(283.39355469,455.36328125)(283.90527344,455.36328125)
\curveto(284.28027344,455.36328125)(284.61425781,455.28320312)(284.90722656,455.12304688)
\curveto(285.20019531,454.96679688)(285.43847656,454.76171875)(285.62207031,454.5078125)
\lineto(285.62207031,457.58984375)
\lineto(286.67089844,457.58984375)
\lineto(286.67089844,449)
\closepath
\moveto(282.35839844,452.10546875)
\curveto(282.35839844,451.30859375)(282.52636719,450.71289062)(282.86230469,450.31835938)
\curveto(283.19824219,449.92382812)(283.59472656,449.7265625)(284.05175781,449.7265625)
\curveto(284.51269531,449.7265625)(284.90332031,449.9140625)(285.22363281,450.2890625)
\curveto(285.54785156,450.66796875)(285.70996094,451.24414062)(285.70996094,452.01757812)
\curveto(285.70996094,452.86914062)(285.54589844,453.49414062)(285.21777344,453.89257812)
\curveto(284.88964844,454.29101562)(284.48535156,454.49023438)(284.00488281,454.49023438)
\curveto(283.53613281,454.49023438)(283.14355469,454.29882812)(282.82714844,453.91601562)
\curveto(282.51464844,453.53320312)(282.35839844,452.9296875)(282.35839844,452.10546875)
\closepath
}
}
{
\newrgbcolor{curcolor}{0 0 0}
\pscustom[linestyle=none,fillstyle=solid,fillcolor=curcolor]
{
\newpath
\moveto(292.58886719,451.00390625)
\lineto(293.67871094,450.86914062)
\curveto(293.50683594,450.23242188)(293.18847656,449.73828125)(292.72363281,449.38671875)
\curveto(292.25878906,449.03515625)(291.66503906,448.859375)(290.94238281,448.859375)
\curveto(290.03222656,448.859375)(289.30957031,449.13867188)(288.77441406,449.69726562)
\curveto(288.24316406,450.25976562)(287.97753906,451.046875)(287.97753906,452.05859375)
\curveto(287.97753906,453.10546875)(288.24707031,453.91796875)(288.78613281,454.49609375)
\curveto(289.32519531,455.07421875)(290.02441406,455.36328125)(290.88378906,455.36328125)
\curveto(291.71582031,455.36328125)(292.39550781,455.08007812)(292.92285156,454.51367188)
\curveto(293.45019531,453.94726562)(293.71386719,453.15039062)(293.71386719,452.12304688)
\curveto(293.71386719,452.06054688)(293.71191406,451.96679688)(293.70800781,451.84179688)
\lineto(289.06738281,451.84179688)
\curveto(289.10644531,451.15820312)(289.29980469,450.63476562)(289.64746094,450.27148438)
\curveto(289.99511719,449.90820312)(290.42871094,449.7265625)(290.94824219,449.7265625)
\curveto(291.33496094,449.7265625)(291.66503906,449.828125)(291.93847656,450.03125)
\curveto(292.21191406,450.234375)(292.42871094,450.55859375)(292.58886719,451.00390625)
\closepath
\moveto(289.12597656,452.70898438)
\lineto(292.60058594,452.70898438)
\curveto(292.55371094,453.23242188)(292.42089844,453.625)(292.20214844,453.88671875)
\curveto(291.86621094,454.29296875)(291.43066406,454.49609375)(290.89550781,454.49609375)
\curveto(290.41113281,454.49609375)(290.00292969,454.33398438)(289.67089844,454.00976562)
\curveto(289.34277344,453.68554688)(289.16113281,453.25195312)(289.12597656,452.70898438)
\closepath
}
}
{
\newrgbcolor{curcolor}{0 0 0}
\pscustom[linestyle=none,fillstyle=solid,fillcolor=curcolor]
{
\newpath
\moveto(302.01660156,449)
\lineto(300.96191406,449)
\lineto(300.96191406,455.72070312)
\curveto(300.70800781,455.47851562)(300.37402344,455.23632812)(299.95996094,454.99414062)
\curveto(299.54980469,454.75195312)(299.18066406,454.5703125)(298.85253906,454.44921875)
\lineto(298.85253906,455.46875)
\curveto(299.44238281,455.74609375)(299.95800781,456.08203125)(300.39941406,456.4765625)
\curveto(300.84082031,456.87109375)(301.15332031,457.25390625)(301.33691406,457.625)
\lineto(302.01660156,457.625)
\closepath
}
}
{
\newrgbcolor{curcolor}{0 0 0}
\pscustom[linestyle=none,fillstyle=solid,fillcolor=curcolor]
{
\newpath
\moveto(305.11035156,449)
\lineto(305.11035156,457.58984375)
\lineto(306.82128906,457.58984375)
\lineto(308.85449219,451.5078125)
\curveto(309.04199219,450.94140625)(309.17871094,450.51757812)(309.26464844,450.23632812)
\curveto(309.36230469,450.54882812)(309.51464844,451.0078125)(309.72167969,451.61328125)
\lineto(311.77832031,457.58984375)
\lineto(313.30761719,457.58984375)
\lineto(313.30761719,449)
\lineto(312.21191406,449)
\lineto(312.21191406,456.18945312)
\lineto(309.71582031,449)
\lineto(308.69042969,449)
\lineto(306.20605469,456.3125)
\lineto(306.20605469,449)
\closepath
}
}
{
\newrgbcolor{curcolor}{0 0 0}
\pscustom[linestyle=none,fillstyle=solid,fillcolor=curcolor]
{
\newpath
\moveto(318.49316406,449)
\lineto(318.49316406,457.58984375)
\lineto(322.30175781,457.58984375)
\curveto(323.06738281,457.58984375)(323.64941406,457.51171875)(324.04785156,457.35546875)
\curveto(324.44628906,457.203125)(324.76464844,456.93164062)(325.00292969,456.54101562)
\curveto(325.24121094,456.15039062)(325.36035156,455.71875)(325.36035156,455.24609375)
\curveto(325.36035156,454.63671875)(325.16308594,454.12304688)(324.76855469,453.70507812)
\curveto(324.37402344,453.28710938)(323.76464844,453.02148438)(322.94042969,452.90820312)
\curveto(323.24121094,452.76367188)(323.46972656,452.62109375)(323.62597656,452.48046875)
\curveto(323.95800781,452.17578125)(324.27246094,451.79492188)(324.56933594,451.33789062)
\lineto(326.06347656,449)
\lineto(324.63378906,449)
\lineto(323.49707031,450.78710938)
\curveto(323.16503906,451.30273438)(322.89160156,451.69726562)(322.67675781,451.97070312)
\curveto(322.46191406,452.24414062)(322.26855469,452.43554688)(322.09667969,452.54492188)
\curveto(321.92871094,452.65429688)(321.75683594,452.73046875)(321.58105469,452.7734375)
\curveto(321.45214844,452.80078125)(321.24121094,452.81445312)(320.94824219,452.81445312)
\lineto(319.62988281,452.81445312)
\lineto(319.62988281,449)
\closepath
\moveto(319.62988281,453.79882812)
\lineto(322.07324219,453.79882812)
\curveto(322.59277344,453.79882812)(322.99902344,453.8515625)(323.29199219,453.95703125)
\curveto(323.58496094,454.06640625)(323.80761719,454.23828125)(323.95996094,454.47265625)
\curveto(324.11230469,454.7109375)(324.18847656,454.96875)(324.18847656,455.24609375)
\curveto(324.18847656,455.65234375)(324.04003906,455.98632812)(323.74316406,456.24804688)
\curveto(323.45019531,456.50976562)(322.98535156,456.640625)(322.34863281,456.640625)
\lineto(319.62988281,456.640625)
\closepath
}
}
{
\newrgbcolor{curcolor}{0 0 0}
\pscustom[linestyle=none,fillstyle=solid,fillcolor=curcolor]
{
\newpath
\moveto(331.26660156,451.00390625)
\lineto(332.35644531,450.86914062)
\curveto(332.18457031,450.23242188)(331.86621094,449.73828125)(331.40136719,449.38671875)
\curveto(330.93652344,449.03515625)(330.34277344,448.859375)(329.62011719,448.859375)
\curveto(328.70996094,448.859375)(327.98730469,449.13867188)(327.45214844,449.69726562)
\curveto(326.92089844,450.25976562)(326.65527344,451.046875)(326.65527344,452.05859375)
\curveto(326.65527344,453.10546875)(326.92480469,453.91796875)(327.46386719,454.49609375)
\curveto(328.00292969,455.07421875)(328.70214844,455.36328125)(329.56152344,455.36328125)
\curveto(330.39355469,455.36328125)(331.07324219,455.08007812)(331.60058594,454.51367188)
\curveto(332.12792969,453.94726562)(332.39160156,453.15039062)(332.39160156,452.12304688)
\curveto(332.39160156,452.06054688)(332.38964844,451.96679688)(332.38574219,451.84179688)
\lineto(327.74511719,451.84179688)
\curveto(327.78417969,451.15820312)(327.97753906,450.63476562)(328.32519531,450.27148438)
\curveto(328.67285156,449.90820312)(329.10644531,449.7265625)(329.62597656,449.7265625)
\curveto(330.01269531,449.7265625)(330.34277344,449.828125)(330.61621094,450.03125)
\curveto(330.88964844,450.234375)(331.10644531,450.55859375)(331.26660156,451.00390625)
\closepath
\moveto(327.80371094,452.70898438)
\lineto(331.27832031,452.70898438)
\curveto(331.23144531,453.23242188)(331.09863281,453.625)(330.87988281,453.88671875)
\curveto(330.54394531,454.29296875)(330.10839844,454.49609375)(329.57324219,454.49609375)
\curveto(329.08886719,454.49609375)(328.68066406,454.33398438)(328.34863281,454.00976562)
\curveto(328.02050781,453.68554688)(327.83886719,453.25195312)(327.80371094,452.70898438)
\closepath
}
}
{
\newrgbcolor{curcolor}{0 0 0}
\pscustom[linestyle=none,fillstyle=solid,fillcolor=curcolor]
{
\newpath
\moveto(337.74121094,449.76757812)
\curveto(337.35058594,449.43554688)(336.97363281,449.20117188)(336.61035156,449.06445312)
\curveto(336.25097656,448.92773438)(335.86425781,448.859375)(335.45019531,448.859375)
\curveto(334.76660156,448.859375)(334.24121094,449.02539062)(333.87402344,449.35742188)
\curveto(333.50683594,449.69335938)(333.32324219,450.12109375)(333.32324219,450.640625)
\curveto(333.32324219,450.9453125)(333.39160156,451.22265625)(333.52832031,451.47265625)
\curveto(333.66894531,451.7265625)(333.85058594,451.9296875)(334.07324219,452.08203125)
\curveto(334.29980469,452.234375)(334.55371094,452.34960938)(334.83496094,452.42773438)
\curveto(335.04199219,452.48242188)(335.35449219,452.53515625)(335.77246094,452.5859375)
\curveto(336.62402344,452.6875)(337.25097656,452.80859375)(337.65332031,452.94921875)
\curveto(337.65722656,453.09375)(337.65917969,453.18554688)(337.65917969,453.22460938)
\curveto(337.65917969,453.65429688)(337.55957031,453.95703125)(337.36035156,454.1328125)
\curveto(337.09082031,454.37109375)(336.69042969,454.49023438)(336.15917969,454.49023438)
\curveto(335.66308594,454.49023438)(335.29589844,454.40234375)(335.05761719,454.2265625)
\curveto(334.82324219,454.0546875)(334.64941406,453.74804688)(334.53613281,453.30664062)
\lineto(333.50488281,453.44726562)
\curveto(333.59863281,453.88867188)(333.75292969,454.24414062)(333.96777344,454.51367188)
\curveto(334.18261719,454.78710938)(334.49316406,454.99609375)(334.89941406,455.140625)
\curveto(335.30566406,455.2890625)(335.77636719,455.36328125)(336.31152344,455.36328125)
\curveto(336.84277344,455.36328125)(337.27441406,455.30078125)(337.60644531,455.17578125)
\curveto(337.93847656,455.05078125)(338.18261719,454.89257812)(338.33886719,454.70117188)
\curveto(338.49511719,454.51367188)(338.60449219,454.27539062)(338.66699219,453.98632812)
\curveto(338.70214844,453.80664062)(338.71972656,453.48242188)(338.71972656,453.01367188)
\lineto(338.71972656,451.60742188)
\curveto(338.71972656,450.62695312)(338.74121094,450.00585938)(338.78417969,449.74414062)
\curveto(338.83105469,449.48632812)(338.92089844,449.23828125)(339.05371094,449)
\lineto(337.95214844,449)
\curveto(337.84277344,449.21875)(337.77246094,449.47460938)(337.74121094,449.76757812)
\closepath
\moveto(337.65332031,452.12304688)
\curveto(337.27050781,451.96679688)(336.69628906,451.83398438)(335.93066406,451.72460938)
\curveto(335.49707031,451.66210938)(335.19042969,451.59179688)(335.01074219,451.51367188)
\curveto(334.83105469,451.43554688)(334.69238281,451.3203125)(334.59472656,451.16796875)
\curveto(334.49707031,451.01953125)(334.44824219,450.85351562)(334.44824219,450.66992188)
\curveto(334.44824219,450.38867188)(334.55371094,450.15429688)(334.76464844,449.96679688)
\curveto(334.97949219,449.77929688)(335.29199219,449.68554688)(335.70214844,449.68554688)
\curveto(336.10839844,449.68554688)(336.46972656,449.7734375)(336.78613281,449.94921875)
\curveto(337.10253906,450.12890625)(337.33496094,450.37304688)(337.48339844,450.68164062)
\curveto(337.59667969,450.91992188)(337.65332031,451.27148438)(337.65332031,451.73632812)
\closepath
}
}
{
\newrgbcolor{curcolor}{0 0 0}
\pscustom[linestyle=none,fillstyle=solid,fillcolor=curcolor]
{
\newpath
\moveto(344.39160156,449)
\lineto(344.39160156,449.78515625)
\curveto(343.99707031,449.16796875)(343.41699219,448.859375)(342.65136719,448.859375)
\curveto(342.15527344,448.859375)(341.69824219,448.99609375)(341.28027344,449.26953125)
\curveto(340.86621094,449.54296875)(340.54394531,449.92382812)(340.31347656,450.41210938)
\curveto(340.08691406,450.90429688)(339.97363281,451.46875)(339.97363281,452.10546875)
\curveto(339.97363281,452.7265625)(340.07714844,453.2890625)(340.28417969,453.79296875)
\curveto(340.49121094,454.30078125)(340.80175781,454.68945312)(341.21582031,454.95898438)
\curveto(341.62988281,455.22851562)(342.09277344,455.36328125)(342.60449219,455.36328125)
\curveto(342.97949219,455.36328125)(343.31347656,455.28320312)(343.60644531,455.12304688)
\curveto(343.89941406,454.96679688)(344.13769531,454.76171875)(344.32128906,454.5078125)
\lineto(344.32128906,457.58984375)
\lineto(345.37011719,457.58984375)
\lineto(345.37011719,449)
\closepath
\moveto(341.05761719,452.10546875)
\curveto(341.05761719,451.30859375)(341.22558594,450.71289062)(341.56152344,450.31835938)
\curveto(341.89746094,449.92382812)(342.29394531,449.7265625)(342.75097656,449.7265625)
\curveto(343.21191406,449.7265625)(343.60253906,449.9140625)(343.92285156,450.2890625)
\curveto(344.24707031,450.66796875)(344.40917969,451.24414062)(344.40917969,452.01757812)
\curveto(344.40917969,452.86914062)(344.24511719,453.49414062)(343.91699219,453.89257812)
\curveto(343.58886719,454.29101562)(343.18457031,454.49023438)(342.70410156,454.49023438)
\curveto(342.23535156,454.49023438)(341.84277344,454.29882812)(341.52636719,453.91601562)
\curveto(341.21386719,453.53320312)(341.05761719,452.9296875)(341.05761719,452.10546875)
\closepath
}
}
{
\newrgbcolor{curcolor}{0 0 0}
\pscustom[linestyle=none,fillstyle=solid,fillcolor=curcolor]
{
\newpath
\moveto(346.60644531,450.85742188)
\lineto(347.64941406,451.02148438)
\curveto(347.70800781,450.60351562)(347.87011719,450.28320312)(348.13574219,450.06054688)
\curveto(348.40527344,449.83789062)(348.78027344,449.7265625)(349.26074219,449.7265625)
\curveto(349.74511719,449.7265625)(350.10449219,449.82421875)(350.33886719,450.01953125)
\curveto(350.57324219,450.21875)(350.69042969,450.45117188)(350.69042969,450.71679688)
\curveto(350.69042969,450.95507812)(350.58691406,451.14257812)(350.37988281,451.27929688)
\curveto(350.23535156,451.37304688)(349.87597656,451.4921875)(349.30175781,451.63671875)
\curveto(348.52832031,451.83203125)(347.99121094,452)(347.69042969,452.140625)
\curveto(347.39355469,452.28515625)(347.16699219,452.48242188)(347.01074219,452.73242188)
\curveto(346.85839844,452.98632812)(346.78222656,453.265625)(346.78222656,453.5703125)
\curveto(346.78222656,453.84765625)(346.84472656,454.10351562)(346.96972656,454.33789062)
\curveto(347.09863281,454.57617188)(347.27246094,454.7734375)(347.49121094,454.9296875)
\curveto(347.65527344,455.05078125)(347.87792969,455.15234375)(348.15917969,455.234375)
\curveto(348.44433594,455.3203125)(348.74902344,455.36328125)(349.07324219,455.36328125)
\curveto(349.56152344,455.36328125)(349.98925781,455.29296875)(350.35644531,455.15234375)
\curveto(350.72753906,455.01171875)(351.00097656,454.8203125)(351.17675781,454.578125)
\curveto(351.35253906,454.33984375)(351.47363281,454.01953125)(351.54003906,453.6171875)
\lineto(350.50878906,453.4765625)
\curveto(350.46191406,453.796875)(350.32519531,454.046875)(350.09863281,454.2265625)
\curveto(349.87597656,454.40625)(349.55957031,454.49609375)(349.14941406,454.49609375)
\curveto(348.66503906,454.49609375)(348.31933594,454.41601562)(348.11230469,454.25585938)
\curveto(347.90527344,454.09570312)(347.80175781,453.90820312)(347.80175781,453.69335938)
\curveto(347.80175781,453.55664062)(347.84472656,453.43359375)(347.93066406,453.32421875)
\curveto(348.01660156,453.2109375)(348.15136719,453.1171875)(348.33496094,453.04296875)
\curveto(348.44042969,453.00390625)(348.75097656,452.9140625)(349.26660156,452.7734375)
\curveto(350.01269531,452.57421875)(350.53222656,452.41015625)(350.82519531,452.28125)
\curveto(351.12207031,452.15625)(351.35449219,451.97265625)(351.52246094,451.73046875)
\curveto(351.69042969,451.48828125)(351.77441406,451.1875)(351.77441406,450.828125)
\curveto(351.77441406,450.4765625)(351.67089844,450.14453125)(351.46386719,449.83203125)
\curveto(351.26074219,449.5234375)(350.96582031,449.28320312)(350.57910156,449.11132812)
\curveto(350.19238281,448.94335938)(349.75488281,448.859375)(349.26660156,448.859375)
\curveto(348.45800781,448.859375)(347.84082031,449.02734375)(347.41503906,449.36328125)
\curveto(346.99316406,449.69921875)(346.72363281,450.19726562)(346.60644531,450.85742188)
\closepath
}
}
{
\newrgbcolor{curcolor}{0 0 0}
\pscustom[linestyle=none,fillstyle=solid,fillcolor=curcolor]
{
\newpath
\moveto(356.11035156,451.75976562)
\lineto(357.18261719,451.85351562)
\curveto(357.23339844,451.42382812)(357.35058594,451.0703125)(357.53417969,450.79296875)
\curveto(357.72167969,450.51953125)(358.01074219,450.296875)(358.40136719,450.125)
\curveto(358.79199219,449.95703125)(359.23144531,449.87304688)(359.71972656,449.87304688)
\curveto(360.15332031,449.87304688)(360.53613281,449.9375)(360.86816406,450.06640625)
\curveto(361.20019531,450.1953125)(361.44628906,450.37109375)(361.60644531,450.59375)
\curveto(361.77050781,450.8203125)(361.85253906,451.06640625)(361.85253906,451.33203125)
\curveto(361.85253906,451.6015625)(361.77441406,451.8359375)(361.61816406,452.03515625)
\curveto(361.46191406,452.23828125)(361.20410156,452.40820312)(360.84472656,452.54492188)
\curveto(360.61425781,452.63476562)(360.10449219,452.7734375)(359.31542969,452.9609375)
\curveto(358.52636719,453.15234375)(357.97363281,453.33203125)(357.65722656,453.5)
\curveto(357.24707031,453.71484375)(356.94042969,453.98046875)(356.73730469,454.296875)
\curveto(356.53808594,454.6171875)(356.43847656,454.97460938)(356.43847656,455.36914062)
\curveto(356.43847656,455.80273438)(356.56152344,456.20703125)(356.80761719,456.58203125)
\curveto(357.05371094,456.9609375)(357.41308594,457.24804688)(357.88574219,457.44335938)
\curveto(358.35839844,457.63867188)(358.88378906,457.73632812)(359.46191406,457.73632812)
\curveto(360.09863281,457.73632812)(360.65917969,457.6328125)(361.14355469,457.42578125)
\curveto(361.63183594,457.22265625)(362.00683594,456.921875)(362.26855469,456.5234375)
\curveto(362.53027344,456.125)(362.67089844,455.67382812)(362.69042969,455.16992188)
\lineto(361.60058594,455.08789062)
\curveto(361.54199219,455.63085938)(361.34277344,456.04101562)(361.00292969,456.31835938)
\curveto(360.66699219,456.59570312)(360.16894531,456.734375)(359.50878906,456.734375)
\curveto(358.82128906,456.734375)(358.31933594,456.60742188)(358.00292969,456.35351562)
\curveto(357.69042969,456.10351562)(357.53417969,455.80078125)(357.53417969,455.4453125)
\curveto(357.53417969,455.13671875)(357.64550781,454.8828125)(357.86816406,454.68359375)
\curveto(358.08691406,454.484375)(358.65722656,454.27929688)(359.57910156,454.06835938)
\curveto(360.50488281,453.86132812)(361.13964844,453.6796875)(361.48339844,453.5234375)
\curveto(361.98339844,453.29296875)(362.35253906,453)(362.59082031,452.64453125)
\curveto(362.82910156,452.29296875)(362.94824219,451.88671875)(362.94824219,451.42578125)
\curveto(362.94824219,450.96875)(362.81738281,450.53710938)(362.55566406,450.13085938)
\curveto(362.29394531,449.72851562)(361.91699219,449.4140625)(361.42480469,449.1875)
\curveto(360.93652344,448.96484375)(360.38574219,448.85351562)(359.77246094,448.85351562)
\curveto(358.99511719,448.85351562)(358.34277344,448.96679688)(357.81542969,449.19335938)
\curveto(357.29199219,449.41992188)(356.87988281,449.75976562)(356.57910156,450.21289062)
\curveto(356.28222656,450.66992188)(356.12597656,451.18554688)(356.11035156,451.75976562)
\closepath
}
}
{
\newrgbcolor{curcolor}{0 0 0}
\pscustom[linestyle=none,fillstyle=solid,fillcolor=curcolor]
{
\newpath
\moveto(364.36621094,449)
\lineto(364.36621094,455.22265625)
\lineto(365.30957031,455.22265625)
\lineto(365.30957031,454.34960938)
\curveto(365.50488281,454.65429688)(365.76464844,454.8984375)(366.08886719,455.08203125)
\curveto(366.41308594,455.26953125)(366.78222656,455.36328125)(367.19628906,455.36328125)
\curveto(367.65722656,455.36328125)(368.03417969,455.26757812)(368.32714844,455.07617188)
\curveto(368.62402344,454.88476562)(368.83300781,454.6171875)(368.95410156,454.2734375)
\curveto(369.44628906,455)(370.08691406,455.36328125)(370.87597656,455.36328125)
\curveto(371.49316406,455.36328125)(371.96777344,455.19140625)(372.29980469,454.84765625)
\curveto(372.63183594,454.5078125)(372.79785156,453.98242188)(372.79785156,453.27148438)
\lineto(372.79785156,449)
\lineto(371.74902344,449)
\lineto(371.74902344,452.91992188)
\curveto(371.74902344,453.34179688)(371.71386719,453.64453125)(371.64355469,453.828125)
\curveto(371.57714844,454.015625)(371.45410156,454.16601562)(371.27441406,454.27929688)
\curveto(371.09472656,454.39257812)(370.88378906,454.44921875)(370.64160156,454.44921875)
\curveto(370.20410156,454.44921875)(369.84082031,454.30273438)(369.55175781,454.00976562)
\curveto(369.26269531,453.72070312)(369.11816406,453.25585938)(369.11816406,452.61523438)
\lineto(369.11816406,449)
\lineto(368.06347656,449)
\lineto(368.06347656,453.04296875)
\curveto(368.06347656,453.51171875)(367.97753906,453.86328125)(367.80566406,454.09765625)
\curveto(367.63378906,454.33203125)(367.35253906,454.44921875)(366.96191406,454.44921875)
\curveto(366.66503906,454.44921875)(366.38964844,454.37109375)(366.13574219,454.21484375)
\curveto(365.88574219,454.05859375)(365.70410156,453.83007812)(365.59082031,453.52929688)
\curveto(365.47753906,453.22851562)(365.42089844,452.79492188)(365.42089844,452.22851562)
\lineto(365.42089844,449)
\closepath
}
}
{
\newrgbcolor{curcolor}{0 0 0}
\pscustom[linestyle=none,fillstyle=solid,fillcolor=curcolor]
{
\newpath
\moveto(378.42285156,449.76757812)
\curveto(378.03222656,449.43554688)(377.65527344,449.20117188)(377.29199219,449.06445312)
\curveto(376.93261719,448.92773438)(376.54589844,448.859375)(376.13183594,448.859375)
\curveto(375.44824219,448.859375)(374.92285156,449.02539062)(374.55566406,449.35742188)
\curveto(374.18847656,449.69335938)(374.00488281,450.12109375)(374.00488281,450.640625)
\curveto(374.00488281,450.9453125)(374.07324219,451.22265625)(374.20996094,451.47265625)
\curveto(374.35058594,451.7265625)(374.53222656,451.9296875)(374.75488281,452.08203125)
\curveto(374.98144531,452.234375)(375.23535156,452.34960938)(375.51660156,452.42773438)
\curveto(375.72363281,452.48242188)(376.03613281,452.53515625)(376.45410156,452.5859375)
\curveto(377.30566406,452.6875)(377.93261719,452.80859375)(378.33496094,452.94921875)
\curveto(378.33886719,453.09375)(378.34082031,453.18554688)(378.34082031,453.22460938)
\curveto(378.34082031,453.65429688)(378.24121094,453.95703125)(378.04199219,454.1328125)
\curveto(377.77246094,454.37109375)(377.37207031,454.49023438)(376.84082031,454.49023438)
\curveto(376.34472656,454.49023438)(375.97753906,454.40234375)(375.73925781,454.2265625)
\curveto(375.50488281,454.0546875)(375.33105469,453.74804688)(375.21777344,453.30664062)
\lineto(374.18652344,453.44726562)
\curveto(374.28027344,453.88867188)(374.43457031,454.24414062)(374.64941406,454.51367188)
\curveto(374.86425781,454.78710938)(375.17480469,454.99609375)(375.58105469,455.140625)
\curveto(375.98730469,455.2890625)(376.45800781,455.36328125)(376.99316406,455.36328125)
\curveto(377.52441406,455.36328125)(377.95605469,455.30078125)(378.28808594,455.17578125)
\curveto(378.62011719,455.05078125)(378.86425781,454.89257812)(379.02050781,454.70117188)
\curveto(379.17675781,454.51367188)(379.28613281,454.27539062)(379.34863281,453.98632812)
\curveto(379.38378906,453.80664062)(379.40136719,453.48242188)(379.40136719,453.01367188)
\lineto(379.40136719,451.60742188)
\curveto(379.40136719,450.62695312)(379.42285156,450.00585938)(379.46582031,449.74414062)
\curveto(379.51269531,449.48632812)(379.60253906,449.23828125)(379.73535156,449)
\lineto(378.63378906,449)
\curveto(378.52441406,449.21875)(378.45410156,449.47460938)(378.42285156,449.76757812)
\closepath
\moveto(378.33496094,452.12304688)
\curveto(377.95214844,451.96679688)(377.37792969,451.83398438)(376.61230469,451.72460938)
\curveto(376.17871094,451.66210938)(375.87207031,451.59179688)(375.69238281,451.51367188)
\curveto(375.51269531,451.43554688)(375.37402344,451.3203125)(375.27636719,451.16796875)
\curveto(375.17871094,451.01953125)(375.12988281,450.85351562)(375.12988281,450.66992188)
\curveto(375.12988281,450.38867188)(375.23535156,450.15429688)(375.44628906,449.96679688)
\curveto(375.66113281,449.77929688)(375.97363281,449.68554688)(376.38378906,449.68554688)
\curveto(376.79003906,449.68554688)(377.15136719,449.7734375)(377.46777344,449.94921875)
\curveto(377.78417969,450.12890625)(378.01660156,450.37304688)(378.16503906,450.68164062)
\curveto(378.27832031,450.91992188)(378.33496094,451.27148438)(378.33496094,451.73632812)
\closepath
}
}
{
\newrgbcolor{curcolor}{0 0 0}
\pscustom[linestyle=none,fillstyle=solid,fillcolor=curcolor]
{
\newpath
\moveto(381.01269531,449)
\lineto(381.01269531,457.58984375)
\lineto(382.06738281,457.58984375)
\lineto(382.06738281,449)
\closepath
}
}
{
\newrgbcolor{curcolor}{0 0 0}
\pscustom[linestyle=none,fillstyle=solid,fillcolor=curcolor]
{
\newpath
\moveto(383.67871094,449)
\lineto(383.67871094,457.58984375)
\lineto(384.73339844,457.58984375)
\lineto(384.73339844,449)
\closepath
}
}
{
\newrgbcolor{curcolor}{0 0 0}
\pscustom[linestyle=none,fillstyle=solid,fillcolor=curcolor]
{
\newpath
\moveto(389.83691406,449)
\lineto(389.83691406,457.58984375)
\lineto(393.07714844,457.58984375)
\curveto(393.64746094,457.58984375)(394.08300781,457.5625)(394.38378906,457.5078125)
\curveto(394.80566406,457.4375)(395.15917969,457.30273438)(395.44433594,457.10351562)
\curveto(395.72949219,456.90820312)(395.95800781,456.6328125)(396.12988281,456.27734375)
\curveto(396.30566406,455.921875)(396.39355469,455.53125)(396.39355469,455.10546875)
\curveto(396.39355469,454.375)(396.16113281,453.75585938)(395.69628906,453.24804688)
\curveto(395.23144531,452.74414062)(394.39160156,452.4921875)(393.17675781,452.4921875)
\lineto(390.97363281,452.4921875)
\lineto(390.97363281,449)
\closepath
\moveto(390.97363281,453.50585938)
\lineto(393.19433594,453.50585938)
\curveto(393.92871094,453.50585938)(394.45019531,453.64257812)(394.75878906,453.91601562)
\curveto(395.06738281,454.18945312)(395.22167969,454.57421875)(395.22167969,455.0703125)
\curveto(395.22167969,455.4296875)(395.12988281,455.73632812)(394.94628906,455.99023438)
\curveto(394.76660156,456.24804688)(394.52832031,456.41796875)(394.23144531,456.5)
\curveto(394.04003906,456.55078125)(393.68652344,456.57617188)(393.17089844,456.57617188)
\lineto(390.97363281,456.57617188)
\closepath
}
}
{
\newrgbcolor{curcolor}{0 0 0}
\pscustom[linestyle=none,fillstyle=solid,fillcolor=curcolor]
{
\newpath
\moveto(397.69433594,449)
\lineto(397.69433594,455.22265625)
\lineto(398.64355469,455.22265625)
\lineto(398.64355469,454.27929688)
\curveto(398.88574219,454.72070312)(399.10839844,455.01171875)(399.31152344,455.15234375)
\curveto(399.51855469,455.29296875)(399.74511719,455.36328125)(399.99121094,455.36328125)
\curveto(400.34667969,455.36328125)(400.70800781,455.25)(401.07519531,455.0234375)
\lineto(400.71191406,454.04492188)
\curveto(400.45410156,454.19726562)(400.19628906,454.2734375)(399.93847656,454.2734375)
\curveto(399.70800781,454.2734375)(399.50097656,454.203125)(399.31738281,454.0625)
\curveto(399.13378906,453.92578125)(399.00292969,453.734375)(398.92480469,453.48828125)
\curveto(398.80761719,453.11328125)(398.74902344,452.703125)(398.74902344,452.2578125)
\lineto(398.74902344,449)
\closepath
}
}
{
\newrgbcolor{curcolor}{0 0 0}
\pscustom[linestyle=none,fillstyle=solid,fillcolor=curcolor]
{
\newpath
\moveto(401.30957031,452.11132812)
\curveto(401.30957031,453.26367188)(401.62988281,454.1171875)(402.27050781,454.671875)
\curveto(402.80566406,455.1328125)(403.45800781,455.36328125)(404.22753906,455.36328125)
\curveto(405.08300781,455.36328125)(405.78222656,455.08203125)(406.32519531,454.51953125)
\curveto(406.86816406,453.9609375)(407.13964844,453.1875)(407.13964844,452.19921875)
\curveto(407.13964844,451.3984375)(407.01855469,450.76757812)(406.77636719,450.30664062)
\curveto(406.53808594,449.84960938)(406.18847656,449.49414062)(405.72753906,449.24023438)
\curveto(405.27050781,448.98632812)(404.77050781,448.859375)(404.22753906,448.859375)
\curveto(403.35644531,448.859375)(402.65136719,449.13867188)(402.11230469,449.69726562)
\curveto(401.57714844,450.25585938)(401.30957031,451.06054688)(401.30957031,452.11132812)
\closepath
\moveto(402.39355469,452.11132812)
\curveto(402.39355469,451.31445312)(402.56738281,450.71679688)(402.91503906,450.31835938)
\curveto(403.26269531,449.92382812)(403.70019531,449.7265625)(404.22753906,449.7265625)
\curveto(404.75097656,449.7265625)(405.18652344,449.92578125)(405.53417969,450.32421875)
\curveto(405.88183594,450.72265625)(406.05566406,451.33007812)(406.05566406,452.14648438)
\curveto(406.05566406,452.91601562)(405.87988281,453.49804688)(405.52832031,453.89257812)
\curveto(405.18066406,454.29101562)(404.74707031,454.49023438)(404.22753906,454.49023438)
\curveto(403.70019531,454.49023438)(403.26269531,454.29296875)(402.91503906,453.8984375)
\curveto(402.56738281,453.50390625)(402.39355469,452.90820312)(402.39355469,452.11132812)
\closepath
}
}
{
\newrgbcolor{curcolor}{0 0 0}
\pscustom[linestyle=none,fillstyle=solid,fillcolor=curcolor]
{
\newpath
\moveto(412.43652344,451.27929688)
\lineto(413.47363281,451.14453125)
\curveto(413.36035156,450.4296875)(413.06933594,449.86914062)(412.60058594,449.46289062)
\curveto(412.13574219,449.06054688)(411.56347656,448.859375)(410.88378906,448.859375)
\curveto(410.03222656,448.859375)(409.34667969,449.13671875)(408.82714844,449.69140625)
\curveto(408.31152344,450.25)(408.05371094,451.04882812)(408.05371094,452.08789062)
\curveto(408.05371094,452.75976562)(408.16503906,453.34765625)(408.38769531,453.8515625)
\curveto(408.61035156,454.35546875)(408.94824219,454.73242188)(409.40136719,454.98242188)
\curveto(409.85839844,455.23632812)(410.35449219,455.36328125)(410.88964844,455.36328125)
\curveto(411.56542969,455.36328125)(412.11816406,455.19140625)(412.54785156,454.84765625)
\curveto(412.97753906,454.5078125)(413.25292969,454.0234375)(413.37402344,453.39453125)
\lineto(412.34863281,453.23632812)
\curveto(412.25097656,453.65429688)(412.07714844,453.96875)(411.82714844,454.1796875)
\curveto(411.58105469,454.390625)(411.28222656,454.49609375)(410.93066406,454.49609375)
\curveto(410.39941406,454.49609375)(409.96777344,454.3046875)(409.63574219,453.921875)
\curveto(409.30371094,453.54296875)(409.13769531,452.94140625)(409.13769531,452.1171875)
\curveto(409.13769531,451.28125)(409.29785156,450.67382812)(409.61816406,450.29492188)
\curveto(409.93847656,449.91601562)(410.35644531,449.7265625)(410.87207031,449.7265625)
\curveto(411.28613281,449.7265625)(411.63183594,449.85351562)(411.90917969,450.10742188)
\curveto(412.18652344,450.36132812)(412.36230469,450.75195312)(412.43652344,451.27929688)
\closepath
}
}
{
\newrgbcolor{curcolor}{0 0 0}
\pscustom[linestyle=none,fillstyle=solid,fillcolor=curcolor]
{
\newpath
\moveto(418.63574219,451.00390625)
\lineto(419.72558594,450.86914062)
\curveto(419.55371094,450.23242188)(419.23535156,449.73828125)(418.77050781,449.38671875)
\curveto(418.30566406,449.03515625)(417.71191406,448.859375)(416.98925781,448.859375)
\curveto(416.07910156,448.859375)(415.35644531,449.13867188)(414.82128906,449.69726562)
\curveto(414.29003906,450.25976562)(414.02441406,451.046875)(414.02441406,452.05859375)
\curveto(414.02441406,453.10546875)(414.29394531,453.91796875)(414.83300781,454.49609375)
\curveto(415.37207031,455.07421875)(416.07128906,455.36328125)(416.93066406,455.36328125)
\curveto(417.76269531,455.36328125)(418.44238281,455.08007812)(418.96972656,454.51367188)
\curveto(419.49707031,453.94726562)(419.76074219,453.15039062)(419.76074219,452.12304688)
\curveto(419.76074219,452.06054688)(419.75878906,451.96679688)(419.75488281,451.84179688)
\lineto(415.11425781,451.84179688)
\curveto(415.15332031,451.15820312)(415.34667969,450.63476562)(415.69433594,450.27148438)
\curveto(416.04199219,449.90820312)(416.47558594,449.7265625)(416.99511719,449.7265625)
\curveto(417.38183594,449.7265625)(417.71191406,449.828125)(417.98535156,450.03125)
\curveto(418.25878906,450.234375)(418.47558594,450.55859375)(418.63574219,451.00390625)
\closepath
\moveto(415.17285156,452.70898438)
\lineto(418.64746094,452.70898438)
\curveto(418.60058594,453.23242188)(418.46777344,453.625)(418.24902344,453.88671875)
\curveto(417.91308594,454.29296875)(417.47753906,454.49609375)(416.94238281,454.49609375)
\curveto(416.45800781,454.49609375)(416.04980469,454.33398438)(415.71777344,454.00976562)
\curveto(415.38964844,453.68554688)(415.20800781,453.25195312)(415.17285156,452.70898438)
\closepath
}
}
{
\newrgbcolor{curcolor}{0 0 0}
\pscustom[linestyle=none,fillstyle=solid,fillcolor=curcolor]
{
\newpath
\moveto(420.62792969,450.85742188)
\lineto(421.67089844,451.02148438)
\curveto(421.72949219,450.60351562)(421.89160156,450.28320312)(422.15722656,450.06054688)
\curveto(422.42675781,449.83789062)(422.80175781,449.7265625)(423.28222656,449.7265625)
\curveto(423.76660156,449.7265625)(424.12597656,449.82421875)(424.36035156,450.01953125)
\curveto(424.59472656,450.21875)(424.71191406,450.45117188)(424.71191406,450.71679688)
\curveto(424.71191406,450.95507812)(424.60839844,451.14257812)(424.40136719,451.27929688)
\curveto(424.25683594,451.37304688)(423.89746094,451.4921875)(423.32324219,451.63671875)
\curveto(422.54980469,451.83203125)(422.01269531,452)(421.71191406,452.140625)
\curveto(421.41503906,452.28515625)(421.18847656,452.48242188)(421.03222656,452.73242188)
\curveto(420.87988281,452.98632812)(420.80371094,453.265625)(420.80371094,453.5703125)
\curveto(420.80371094,453.84765625)(420.86621094,454.10351562)(420.99121094,454.33789062)
\curveto(421.12011719,454.57617188)(421.29394531,454.7734375)(421.51269531,454.9296875)
\curveto(421.67675781,455.05078125)(421.89941406,455.15234375)(422.18066406,455.234375)
\curveto(422.46582031,455.3203125)(422.77050781,455.36328125)(423.09472656,455.36328125)
\curveto(423.58300781,455.36328125)(424.01074219,455.29296875)(424.37792969,455.15234375)
\curveto(424.74902344,455.01171875)(425.02246094,454.8203125)(425.19824219,454.578125)
\curveto(425.37402344,454.33984375)(425.49511719,454.01953125)(425.56152344,453.6171875)
\lineto(424.53027344,453.4765625)
\curveto(424.48339844,453.796875)(424.34667969,454.046875)(424.12011719,454.2265625)
\curveto(423.89746094,454.40625)(423.58105469,454.49609375)(423.17089844,454.49609375)
\curveto(422.68652344,454.49609375)(422.34082031,454.41601562)(422.13378906,454.25585938)
\curveto(421.92675781,454.09570312)(421.82324219,453.90820312)(421.82324219,453.69335938)
\curveto(421.82324219,453.55664062)(421.86621094,453.43359375)(421.95214844,453.32421875)
\curveto(422.03808594,453.2109375)(422.17285156,453.1171875)(422.35644531,453.04296875)
\curveto(422.46191406,453.00390625)(422.77246094,452.9140625)(423.28808594,452.7734375)
\curveto(424.03417969,452.57421875)(424.55371094,452.41015625)(424.84667969,452.28125)
\curveto(425.14355469,452.15625)(425.37597656,451.97265625)(425.54394531,451.73046875)
\curveto(425.71191406,451.48828125)(425.79589844,451.1875)(425.79589844,450.828125)
\curveto(425.79589844,450.4765625)(425.69238281,450.14453125)(425.48535156,449.83203125)
\curveto(425.28222656,449.5234375)(424.98730469,449.28320312)(424.60058594,449.11132812)
\curveto(424.21386719,448.94335938)(423.77636719,448.859375)(423.28808594,448.859375)
\curveto(422.47949219,448.859375)(421.86230469,449.02734375)(421.43652344,449.36328125)
\curveto(421.01464844,449.69921875)(420.74511719,450.19726562)(420.62792969,450.85742188)
\closepath
}
}
{
\newrgbcolor{curcolor}{0 0 0}
\pscustom[linestyle=none,fillstyle=solid,fillcolor=curcolor]
{
\newpath
\moveto(426.62792969,450.85742188)
\lineto(427.67089844,451.02148438)
\curveto(427.72949219,450.60351562)(427.89160156,450.28320312)(428.15722656,450.06054688)
\curveto(428.42675781,449.83789062)(428.80175781,449.7265625)(429.28222656,449.7265625)
\curveto(429.76660156,449.7265625)(430.12597656,449.82421875)(430.36035156,450.01953125)
\curveto(430.59472656,450.21875)(430.71191406,450.45117188)(430.71191406,450.71679688)
\curveto(430.71191406,450.95507812)(430.60839844,451.14257812)(430.40136719,451.27929688)
\curveto(430.25683594,451.37304688)(429.89746094,451.4921875)(429.32324219,451.63671875)
\curveto(428.54980469,451.83203125)(428.01269531,452)(427.71191406,452.140625)
\curveto(427.41503906,452.28515625)(427.18847656,452.48242188)(427.03222656,452.73242188)
\curveto(426.87988281,452.98632812)(426.80371094,453.265625)(426.80371094,453.5703125)
\curveto(426.80371094,453.84765625)(426.86621094,454.10351562)(426.99121094,454.33789062)
\curveto(427.12011719,454.57617188)(427.29394531,454.7734375)(427.51269531,454.9296875)
\curveto(427.67675781,455.05078125)(427.89941406,455.15234375)(428.18066406,455.234375)
\curveto(428.46582031,455.3203125)(428.77050781,455.36328125)(429.09472656,455.36328125)
\curveto(429.58300781,455.36328125)(430.01074219,455.29296875)(430.37792969,455.15234375)
\curveto(430.74902344,455.01171875)(431.02246094,454.8203125)(431.19824219,454.578125)
\curveto(431.37402344,454.33984375)(431.49511719,454.01953125)(431.56152344,453.6171875)
\lineto(430.53027344,453.4765625)
\curveto(430.48339844,453.796875)(430.34667969,454.046875)(430.12011719,454.2265625)
\curveto(429.89746094,454.40625)(429.58105469,454.49609375)(429.17089844,454.49609375)
\curveto(428.68652344,454.49609375)(428.34082031,454.41601562)(428.13378906,454.25585938)
\curveto(427.92675781,454.09570312)(427.82324219,453.90820312)(427.82324219,453.69335938)
\curveto(427.82324219,453.55664062)(427.86621094,453.43359375)(427.95214844,453.32421875)
\curveto(428.03808594,453.2109375)(428.17285156,453.1171875)(428.35644531,453.04296875)
\curveto(428.46191406,453.00390625)(428.77246094,452.9140625)(429.28808594,452.7734375)
\curveto(430.03417969,452.57421875)(430.55371094,452.41015625)(430.84667969,452.28125)
\curveto(431.14355469,452.15625)(431.37597656,451.97265625)(431.54394531,451.73046875)
\curveto(431.71191406,451.48828125)(431.79589844,451.1875)(431.79589844,450.828125)
\curveto(431.79589844,450.4765625)(431.69238281,450.14453125)(431.48535156,449.83203125)
\curveto(431.28222656,449.5234375)(430.98730469,449.28320312)(430.60058594,449.11132812)
\curveto(430.21386719,448.94335938)(429.77636719,448.859375)(429.28808594,448.859375)
\curveto(428.47949219,448.859375)(427.86230469,449.02734375)(427.43652344,449.36328125)
\curveto(427.01464844,449.69921875)(426.74511719,450.19726562)(426.62792969,450.85742188)
\closepath
}
}
{
\newrgbcolor{curcolor}{0 0 0}
\pscustom[linestyle=none,fillstyle=solid,fillcolor=curcolor]
{
\newpath
\moveto(438.39941406,446.47460938)
\curveto(437.81738281,447.20898438)(437.32519531,448.06835938)(436.92285156,449.05273438)
\curveto(436.52050781,450.03710938)(436.31933594,451.05664062)(436.31933594,452.11132812)
\curveto(436.31933594,453.04101562)(436.46972656,453.93164062)(436.77050781,454.78320312)
\curveto(437.12207031,455.77148438)(437.66503906,456.75585938)(438.39941406,457.73632812)
\lineto(439.15527344,457.73632812)
\curveto(438.68261719,456.92382812)(438.37011719,456.34375)(438.21777344,455.99609375)
\curveto(437.97949219,455.45703125)(437.79199219,454.89453125)(437.65527344,454.30859375)
\curveto(437.48730469,453.578125)(437.40332031,452.84375)(437.40332031,452.10546875)
\curveto(437.40332031,450.2265625)(437.98730469,448.34960938)(439.15527344,446.47460938)
\closepath
}
}
{
\newrgbcolor{curcolor}{0 0 0}
\pscustom[linestyle=none,fillstyle=solid,fillcolor=curcolor]
{
\newpath
\moveto(440.51464844,449)
\lineto(440.51464844,457.58984375)
\lineto(443.47363281,457.58984375)
\curveto(444.14160156,457.58984375)(444.65136719,457.54882812)(445.00292969,457.46679688)
\curveto(445.49511719,457.35351562)(445.91503906,457.1484375)(446.26269531,456.8515625)
\curveto(446.71582031,456.46875)(447.05371094,455.97851562)(447.27636719,455.38085938)
\curveto(447.50292969,454.78710938)(447.61621094,454.10742188)(447.61621094,453.34179688)
\curveto(447.61621094,452.68945312)(447.54003906,452.11132812)(447.38769531,451.60742188)
\curveto(447.23535156,451.10351562)(447.04003906,450.68554688)(446.80175781,450.35351562)
\curveto(446.56347656,450.02539062)(446.30175781,449.765625)(446.01660156,449.57421875)
\curveto(445.73535156,449.38671875)(445.39355469,449.24414062)(444.99121094,449.14648438)
\curveto(444.59277344,449.04882812)(444.13378906,449)(443.61425781,449)
\closepath
\moveto(441.65136719,450.01367188)
\lineto(443.48535156,450.01367188)
\curveto(444.05175781,450.01367188)(444.49511719,450.06640625)(444.81542969,450.171875)
\curveto(445.13964844,450.27734375)(445.39746094,450.42578125)(445.58886719,450.6171875)
\curveto(445.85839844,450.88671875)(446.06738281,451.24804688)(446.21582031,451.70117188)
\curveto(446.36816406,452.15820312)(446.44433594,452.7109375)(446.44433594,453.359375)
\curveto(446.44433594,454.2578125)(446.29589844,454.94726562)(445.99902344,455.42773438)
\curveto(445.70605469,455.91210938)(445.34863281,456.23632812)(444.92675781,456.40039062)
\curveto(444.62207031,456.51757812)(444.13183594,456.57617188)(443.45605469,456.57617188)
\lineto(441.65136719,456.57617188)
\closepath
}
}
{
\newrgbcolor{curcolor}{0 0 0}
\pscustom[linestyle=none,fillstyle=solid,fillcolor=curcolor]
{
\newpath
\moveto(453.10644531,449.76757812)
\curveto(452.71582031,449.43554688)(452.33886719,449.20117188)(451.97558594,449.06445312)
\curveto(451.61621094,448.92773438)(451.22949219,448.859375)(450.81542969,448.859375)
\curveto(450.13183594,448.859375)(449.60644531,449.02539062)(449.23925781,449.35742188)
\curveto(448.87207031,449.69335938)(448.68847656,450.12109375)(448.68847656,450.640625)
\curveto(448.68847656,450.9453125)(448.75683594,451.22265625)(448.89355469,451.47265625)
\curveto(449.03417969,451.7265625)(449.21582031,451.9296875)(449.43847656,452.08203125)
\curveto(449.66503906,452.234375)(449.91894531,452.34960938)(450.20019531,452.42773438)
\curveto(450.40722656,452.48242188)(450.71972656,452.53515625)(451.13769531,452.5859375)
\curveto(451.98925781,452.6875)(452.61621094,452.80859375)(453.01855469,452.94921875)
\curveto(453.02246094,453.09375)(453.02441406,453.18554688)(453.02441406,453.22460938)
\curveto(453.02441406,453.65429688)(452.92480469,453.95703125)(452.72558594,454.1328125)
\curveto(452.45605469,454.37109375)(452.05566406,454.49023438)(451.52441406,454.49023438)
\curveto(451.02832031,454.49023438)(450.66113281,454.40234375)(450.42285156,454.2265625)
\curveto(450.18847656,454.0546875)(450.01464844,453.74804688)(449.90136719,453.30664062)
\lineto(448.87011719,453.44726562)
\curveto(448.96386719,453.88867188)(449.11816406,454.24414062)(449.33300781,454.51367188)
\curveto(449.54785156,454.78710938)(449.85839844,454.99609375)(450.26464844,455.140625)
\curveto(450.67089844,455.2890625)(451.14160156,455.36328125)(451.67675781,455.36328125)
\curveto(452.20800781,455.36328125)(452.63964844,455.30078125)(452.97167969,455.17578125)
\curveto(453.30371094,455.05078125)(453.54785156,454.89257812)(453.70410156,454.70117188)
\curveto(453.86035156,454.51367188)(453.96972656,454.27539062)(454.03222656,453.98632812)
\curveto(454.06738281,453.80664062)(454.08496094,453.48242188)(454.08496094,453.01367188)
\lineto(454.08496094,451.60742188)
\curveto(454.08496094,450.62695312)(454.10644531,450.00585938)(454.14941406,449.74414062)
\curveto(454.19628906,449.48632812)(454.28613281,449.23828125)(454.41894531,449)
\lineto(453.31738281,449)
\curveto(453.20800781,449.21875)(453.13769531,449.47460938)(453.10644531,449.76757812)
\closepath
\moveto(453.01855469,452.12304688)
\curveto(452.63574219,451.96679688)(452.06152344,451.83398438)(451.29589844,451.72460938)
\curveto(450.86230469,451.66210938)(450.55566406,451.59179688)(450.37597656,451.51367188)
\curveto(450.19628906,451.43554688)(450.05761719,451.3203125)(449.95996094,451.16796875)
\curveto(449.86230469,451.01953125)(449.81347656,450.85351562)(449.81347656,450.66992188)
\curveto(449.81347656,450.38867188)(449.91894531,450.15429688)(450.12988281,449.96679688)
\curveto(450.34472656,449.77929688)(450.65722656,449.68554688)(451.06738281,449.68554688)
\curveto(451.47363281,449.68554688)(451.83496094,449.7734375)(452.15136719,449.94921875)
\curveto(452.46777344,450.12890625)(452.70019531,450.37304688)(452.84863281,450.68164062)
\curveto(452.96191406,450.91992188)(453.01855469,451.27148438)(453.01855469,451.73632812)
\closepath
}
}
{
\newrgbcolor{curcolor}{0 0 0}
\pscustom[linestyle=none,fillstyle=solid,fillcolor=curcolor]
{
\newpath
\moveto(458.02246094,449.94335938)
\lineto(458.17480469,449.01171875)
\curveto(457.87792969,448.94921875)(457.61230469,448.91796875)(457.37792969,448.91796875)
\curveto(456.99511719,448.91796875)(456.69824219,448.97851562)(456.48730469,449.09960938)
\curveto(456.27636719,449.22070312)(456.12792969,449.37890625)(456.04199219,449.57421875)
\curveto(455.95605469,449.7734375)(455.91308594,450.18945312)(455.91308594,450.82226562)
\lineto(455.91308594,454.40234375)
\lineto(455.13964844,454.40234375)
\lineto(455.13964844,455.22265625)
\lineto(455.91308594,455.22265625)
\lineto(455.91308594,456.76367188)
\lineto(456.96191406,457.39648438)
\lineto(456.96191406,455.22265625)
\lineto(458.02246094,455.22265625)
\lineto(458.02246094,454.40234375)
\lineto(456.96191406,454.40234375)
\lineto(456.96191406,450.76367188)
\curveto(456.96191406,450.46289062)(456.97949219,450.26953125)(457.01464844,450.18359375)
\curveto(457.05371094,450.09765625)(457.11425781,450.02929688)(457.19628906,449.97851562)
\curveto(457.28222656,449.92773438)(457.40332031,449.90234375)(457.55957031,449.90234375)
\curveto(457.67675781,449.90234375)(457.83105469,449.91601562)(458.02246094,449.94335938)
\closepath
}
}
{
\newrgbcolor{curcolor}{0 0 0}
\pscustom[linestyle=none,fillstyle=solid,fillcolor=curcolor]
{
\newpath
\moveto(463.11425781,449.76757812)
\curveto(462.72363281,449.43554688)(462.34667969,449.20117188)(461.98339844,449.06445312)
\curveto(461.62402344,448.92773438)(461.23730469,448.859375)(460.82324219,448.859375)
\curveto(460.13964844,448.859375)(459.61425781,449.02539062)(459.24707031,449.35742188)
\curveto(458.87988281,449.69335938)(458.69628906,450.12109375)(458.69628906,450.640625)
\curveto(458.69628906,450.9453125)(458.76464844,451.22265625)(458.90136719,451.47265625)
\curveto(459.04199219,451.7265625)(459.22363281,451.9296875)(459.44628906,452.08203125)
\curveto(459.67285156,452.234375)(459.92675781,452.34960938)(460.20800781,452.42773438)
\curveto(460.41503906,452.48242188)(460.72753906,452.53515625)(461.14550781,452.5859375)
\curveto(461.99707031,452.6875)(462.62402344,452.80859375)(463.02636719,452.94921875)
\curveto(463.03027344,453.09375)(463.03222656,453.18554688)(463.03222656,453.22460938)
\curveto(463.03222656,453.65429688)(462.93261719,453.95703125)(462.73339844,454.1328125)
\curveto(462.46386719,454.37109375)(462.06347656,454.49023438)(461.53222656,454.49023438)
\curveto(461.03613281,454.49023438)(460.66894531,454.40234375)(460.43066406,454.2265625)
\curveto(460.19628906,454.0546875)(460.02246094,453.74804688)(459.90917969,453.30664062)
\lineto(458.87792969,453.44726562)
\curveto(458.97167969,453.88867188)(459.12597656,454.24414062)(459.34082031,454.51367188)
\curveto(459.55566406,454.78710938)(459.86621094,454.99609375)(460.27246094,455.140625)
\curveto(460.67871094,455.2890625)(461.14941406,455.36328125)(461.68457031,455.36328125)
\curveto(462.21582031,455.36328125)(462.64746094,455.30078125)(462.97949219,455.17578125)
\curveto(463.31152344,455.05078125)(463.55566406,454.89257812)(463.71191406,454.70117188)
\curveto(463.86816406,454.51367188)(463.97753906,454.27539062)(464.04003906,453.98632812)
\curveto(464.07519531,453.80664062)(464.09277344,453.48242188)(464.09277344,453.01367188)
\lineto(464.09277344,451.60742188)
\curveto(464.09277344,450.62695312)(464.11425781,450.00585938)(464.15722656,449.74414062)
\curveto(464.20410156,449.48632812)(464.29394531,449.23828125)(464.42675781,449)
\lineto(463.32519531,449)
\curveto(463.21582031,449.21875)(463.14550781,449.47460938)(463.11425781,449.76757812)
\closepath
\moveto(463.02636719,452.12304688)
\curveto(462.64355469,451.96679688)(462.06933594,451.83398438)(461.30371094,451.72460938)
\curveto(460.87011719,451.66210938)(460.56347656,451.59179688)(460.38378906,451.51367188)
\curveto(460.20410156,451.43554688)(460.06542969,451.3203125)(459.96777344,451.16796875)
\curveto(459.87011719,451.01953125)(459.82128906,450.85351562)(459.82128906,450.66992188)
\curveto(459.82128906,450.38867188)(459.92675781,450.15429688)(460.13769531,449.96679688)
\curveto(460.35253906,449.77929688)(460.66503906,449.68554688)(461.07519531,449.68554688)
\curveto(461.48144531,449.68554688)(461.84277344,449.7734375)(462.15917969,449.94921875)
\curveto(462.47558594,450.12890625)(462.70800781,450.37304688)(462.85644531,450.68164062)
\curveto(462.96972656,450.91992188)(463.02636719,451.27148438)(463.02636719,451.73632812)
\closepath
}
}
{
\newrgbcolor{curcolor}{0 0 0}
\pscustom[linestyle=none,fillstyle=solid,fillcolor=curcolor]
{
\newpath
\moveto(469.18457031,449)
\lineto(469.18457031,457.58984375)
\lineto(470.35058594,457.58984375)
\lineto(474.86230469,450.84570312)
\lineto(474.86230469,457.58984375)
\lineto(475.95214844,457.58984375)
\lineto(475.95214844,449)
\lineto(474.78613281,449)
\lineto(470.27441406,455.75)
\lineto(470.27441406,449)
\closepath
}
}
{
\newrgbcolor{curcolor}{0 0 0}
\pscustom[linestyle=none,fillstyle=solid,fillcolor=curcolor]
{
\newpath
\moveto(477.33496094,452.11132812)
\curveto(477.33496094,453.26367188)(477.65527344,454.1171875)(478.29589844,454.671875)
\curveto(478.83105469,455.1328125)(479.48339844,455.36328125)(480.25292969,455.36328125)
\curveto(481.10839844,455.36328125)(481.80761719,455.08203125)(482.35058594,454.51953125)
\curveto(482.89355469,453.9609375)(483.16503906,453.1875)(483.16503906,452.19921875)
\curveto(483.16503906,451.3984375)(483.04394531,450.76757812)(482.80175781,450.30664062)
\curveto(482.56347656,449.84960938)(482.21386719,449.49414062)(481.75292969,449.24023438)
\curveto(481.29589844,448.98632812)(480.79589844,448.859375)(480.25292969,448.859375)
\curveto(479.38183594,448.859375)(478.67675781,449.13867188)(478.13769531,449.69726562)
\curveto(477.60253906,450.25585938)(477.33496094,451.06054688)(477.33496094,452.11132812)
\closepath
\moveto(478.41894531,452.11132812)
\curveto(478.41894531,451.31445312)(478.59277344,450.71679688)(478.94042969,450.31835938)
\curveto(479.28808594,449.92382812)(479.72558594,449.7265625)(480.25292969,449.7265625)
\curveto(480.77636719,449.7265625)(481.21191406,449.92578125)(481.55957031,450.32421875)
\curveto(481.90722656,450.72265625)(482.08105469,451.33007812)(482.08105469,452.14648438)
\curveto(482.08105469,452.91601562)(481.90527344,453.49804688)(481.55371094,453.89257812)
\curveto(481.20605469,454.29101562)(480.77246094,454.49023438)(480.25292969,454.49023438)
\curveto(479.72558594,454.49023438)(479.28808594,454.29296875)(478.94042969,453.8984375)
\curveto(478.59277344,453.50390625)(478.41894531,452.90820312)(478.41894531,452.11132812)
\closepath
}
}
{
\newrgbcolor{curcolor}{0 0 0}
\pscustom[linestyle=none,fillstyle=solid,fillcolor=curcolor]
{
\newpath
\moveto(488.43847656,449)
\lineto(488.43847656,449.78515625)
\curveto(488.04394531,449.16796875)(487.46386719,448.859375)(486.69824219,448.859375)
\curveto(486.20214844,448.859375)(485.74511719,448.99609375)(485.32714844,449.26953125)
\curveto(484.91308594,449.54296875)(484.59082031,449.92382812)(484.36035156,450.41210938)
\curveto(484.13378906,450.90429688)(484.02050781,451.46875)(484.02050781,452.10546875)
\curveto(484.02050781,452.7265625)(484.12402344,453.2890625)(484.33105469,453.79296875)
\curveto(484.53808594,454.30078125)(484.84863281,454.68945312)(485.26269531,454.95898438)
\curveto(485.67675781,455.22851562)(486.13964844,455.36328125)(486.65136719,455.36328125)
\curveto(487.02636719,455.36328125)(487.36035156,455.28320312)(487.65332031,455.12304688)
\curveto(487.94628906,454.96679688)(488.18457031,454.76171875)(488.36816406,454.5078125)
\lineto(488.36816406,457.58984375)
\lineto(489.41699219,457.58984375)
\lineto(489.41699219,449)
\closepath
\moveto(485.10449219,452.10546875)
\curveto(485.10449219,451.30859375)(485.27246094,450.71289062)(485.60839844,450.31835938)
\curveto(485.94433594,449.92382812)(486.34082031,449.7265625)(486.79785156,449.7265625)
\curveto(487.25878906,449.7265625)(487.64941406,449.9140625)(487.96972656,450.2890625)
\curveto(488.29394531,450.66796875)(488.45605469,451.24414062)(488.45605469,452.01757812)
\curveto(488.45605469,452.86914062)(488.29199219,453.49414062)(487.96386719,453.89257812)
\curveto(487.63574219,454.29101562)(487.23144531,454.49023438)(486.75097656,454.49023438)
\curveto(486.28222656,454.49023438)(485.88964844,454.29882812)(485.57324219,453.91601562)
\curveto(485.26074219,453.53320312)(485.10449219,452.9296875)(485.10449219,452.10546875)
\closepath
}
}
{
\newrgbcolor{curcolor}{0 0 0}
\pscustom[linestyle=none,fillstyle=solid,fillcolor=curcolor]
{
\newpath
\moveto(495.33496094,451.00390625)
\lineto(496.42480469,450.86914062)
\curveto(496.25292969,450.23242188)(495.93457031,449.73828125)(495.46972656,449.38671875)
\curveto(495.00488281,449.03515625)(494.41113281,448.859375)(493.68847656,448.859375)
\curveto(492.77832031,448.859375)(492.05566406,449.13867188)(491.52050781,449.69726562)
\curveto(490.98925781,450.25976562)(490.72363281,451.046875)(490.72363281,452.05859375)
\curveto(490.72363281,453.10546875)(490.99316406,453.91796875)(491.53222656,454.49609375)
\curveto(492.07128906,455.07421875)(492.77050781,455.36328125)(493.62988281,455.36328125)
\curveto(494.46191406,455.36328125)(495.14160156,455.08007812)(495.66894531,454.51367188)
\curveto(496.19628906,453.94726562)(496.45996094,453.15039062)(496.45996094,452.12304688)
\curveto(496.45996094,452.06054688)(496.45800781,451.96679688)(496.45410156,451.84179688)
\lineto(491.81347656,451.84179688)
\curveto(491.85253906,451.15820312)(492.04589844,450.63476562)(492.39355469,450.27148438)
\curveto(492.74121094,449.90820312)(493.17480469,449.7265625)(493.69433594,449.7265625)
\curveto(494.08105469,449.7265625)(494.41113281,449.828125)(494.68457031,450.03125)
\curveto(494.95800781,450.234375)(495.17480469,450.55859375)(495.33496094,451.00390625)
\closepath
\moveto(491.87207031,452.70898438)
\lineto(495.34667969,452.70898438)
\curveto(495.29980469,453.23242188)(495.16699219,453.625)(494.94824219,453.88671875)
\curveto(494.61230469,454.29296875)(494.17675781,454.49609375)(493.64160156,454.49609375)
\curveto(493.15722656,454.49609375)(492.74902344,454.33398438)(492.41699219,454.00976562)
\curveto(492.08886719,453.68554688)(491.90722656,453.25195312)(491.87207031,452.70898438)
\closepath
}
}
{
\newrgbcolor{curcolor}{0 0 0}
\pscustom[linestyle=none,fillstyle=solid,fillcolor=curcolor]
{
\newpath
\moveto(500.79003906,453.23632812)
\curveto(500.79003906,454.25195312)(500.89355469,455.06835938)(501.10058594,455.68554688)
\curveto(501.31152344,456.30664062)(501.62207031,456.78515625)(502.03222656,457.12109375)
\curveto(502.44628906,457.45703125)(502.96582031,457.625)(503.59082031,457.625)
\curveto(504.05175781,457.625)(504.45605469,457.53125)(504.80371094,457.34375)
\curveto(505.15136719,457.16015625)(505.43847656,456.89257812)(505.66503906,456.54101562)
\curveto(505.89160156,456.19335938)(506.06933594,455.76757812)(506.19824219,455.26367188)
\curveto(506.32714844,454.76367188)(506.39160156,454.08789062)(506.39160156,453.23632812)
\curveto(506.39160156,452.22851562)(506.28808594,451.4140625)(506.08105469,450.79296875)
\curveto(505.87402344,450.17578125)(505.56347656,449.69726562)(505.14941406,449.35742188)
\curveto(504.73925781,449.02148438)(504.21972656,448.85351562)(503.59082031,448.85351562)
\curveto(502.76269531,448.85351562)(502.11230469,449.15039062)(501.63964844,449.74414062)
\curveto(501.07324219,450.45898438)(500.79003906,451.62304688)(500.79003906,453.23632812)
\closepath
\moveto(501.87402344,453.23632812)
\curveto(501.87402344,451.82617188)(502.03808594,450.88671875)(502.36621094,450.41796875)
\curveto(502.69824219,449.953125)(503.10644531,449.72070312)(503.59082031,449.72070312)
\curveto(504.07519531,449.72070312)(504.48144531,449.95507812)(504.80957031,450.42382812)
\curveto(505.14160156,450.89257812)(505.30761719,451.83007812)(505.30761719,453.23632812)
\curveto(505.30761719,454.65039062)(505.14160156,455.58984375)(504.80957031,456.0546875)
\curveto(504.48144531,456.51953125)(504.07128906,456.75195312)(503.57910156,456.75195312)
\curveto(503.09472656,456.75195312)(502.70800781,456.546875)(502.41894531,456.13671875)
\curveto(502.05566406,455.61328125)(501.87402344,454.64648438)(501.87402344,453.23632812)
\closepath
}
}
{
\newrgbcolor{curcolor}{0 0 0}
\pscustom[linestyle=none,fillstyle=solid,fillcolor=curcolor]
{
\newpath
\moveto(508.44824219,446.47460938)
\lineto(507.69238281,446.47460938)
\curveto(508.86035156,448.34960938)(509.44433594,450.2265625)(509.44433594,452.10546875)
\curveto(509.44433594,452.83984375)(509.36035156,453.56835938)(509.19238281,454.29101562)
\curveto(509.05957031,454.87695312)(508.87402344,455.43945312)(508.63574219,455.97851562)
\curveto(508.48339844,456.33007812)(508.16894531,456.91601562)(507.69238281,457.73632812)
\lineto(508.44824219,457.73632812)
\curveto(509.18261719,456.75585938)(509.72558594,455.77148438)(510.07714844,454.78320312)
\curveto(510.37792969,453.93164062)(510.52832031,453.04101562)(510.52832031,452.11132812)
\curveto(510.52832031,451.05664062)(510.32519531,450.03710938)(509.91894531,449.05273438)
\curveto(509.51660156,448.06835938)(509.02636719,447.20898438)(508.44824219,446.47460938)
\closepath
}
}
{
\newrgbcolor{curcolor}{0 0 0}
\pscustom[linestyle=none,fillstyle=solid,fillcolor=curcolor]
{
\newpath
\moveto(122.41015625,404)
\lineto(122.41015625,412.58984375)
\lineto(123.57617188,412.58984375)
\lineto(128.08789062,405.84570312)
\lineto(128.08789062,412.58984375)
\lineto(129.17773438,412.58984375)
\lineto(129.17773438,404)
\lineto(128.01171875,404)
\lineto(123.5,410.75)
\lineto(123.5,404)
\closepath
}
}
{
\newrgbcolor{curcolor}{0 0 0}
\pscustom[linestyle=none,fillstyle=solid,fillcolor=curcolor]
{
\newpath
\moveto(135.21289062,406.00390625)
\lineto(136.30273438,405.86914062)
\curveto(136.13085938,405.23242188)(135.8125,404.73828125)(135.34765625,404.38671875)
\curveto(134.8828125,404.03515625)(134.2890625,403.859375)(133.56640625,403.859375)
\curveto(132.65625,403.859375)(131.93359375,404.13867188)(131.3984375,404.69726562)
\curveto(130.8671875,405.25976562)(130.6015625,406.046875)(130.6015625,407.05859375)
\curveto(130.6015625,408.10546875)(130.87109375,408.91796875)(131.41015625,409.49609375)
\curveto(131.94921875,410.07421875)(132.6484375,410.36328125)(133.5078125,410.36328125)
\curveto(134.33984375,410.36328125)(135.01953125,410.08007812)(135.546875,409.51367188)
\curveto(136.07421875,408.94726562)(136.33789062,408.15039062)(136.33789062,407.12304688)
\curveto(136.33789062,407.06054688)(136.3359375,406.96679688)(136.33203125,406.84179688)
\lineto(131.69140625,406.84179688)
\curveto(131.73046875,406.15820312)(131.92382812,405.63476562)(132.27148438,405.27148438)
\curveto(132.61914062,404.90820312)(133.05273438,404.7265625)(133.57226562,404.7265625)
\curveto(133.95898438,404.7265625)(134.2890625,404.828125)(134.5625,405.03125)
\curveto(134.8359375,405.234375)(135.05273438,405.55859375)(135.21289062,406.00390625)
\closepath
\moveto(131.75,407.70898438)
\lineto(135.22460938,407.70898438)
\curveto(135.17773438,408.23242188)(135.04492188,408.625)(134.82617188,408.88671875)
\curveto(134.49023438,409.29296875)(134.0546875,409.49609375)(133.51953125,409.49609375)
\curveto(133.03515625,409.49609375)(132.62695312,409.33398438)(132.29492188,409.00976562)
\curveto(131.96679688,408.68554688)(131.78515625,408.25195312)(131.75,407.70898438)
\closepath
}
}
{
\newrgbcolor{curcolor}{0 0 0}
\pscustom[linestyle=none,fillstyle=solid,fillcolor=curcolor]
{
\newpath
\moveto(138.77539062,404)
\lineto(136.87109375,410.22265625)
\lineto(137.9609375,410.22265625)
\lineto(138.95117188,406.63085938)
\lineto(139.3203125,405.29492188)
\curveto(139.3359375,405.36132812)(139.44335938,405.7890625)(139.64257812,406.578125)
\lineto(140.6328125,410.22265625)
\lineto(141.71679688,410.22265625)
\lineto(142.6484375,406.61328125)
\lineto(142.95898438,405.42382812)
\lineto(143.31640625,406.625)
\lineto(144.3828125,410.22265625)
\lineto(145.40820312,410.22265625)
\lineto(143.46289062,404)
\lineto(142.3671875,404)
\lineto(141.37695312,407.7265625)
\lineto(141.13671875,408.78710938)
\lineto(139.87695312,404)
\closepath
}
}
{
\newrgbcolor{curcolor}{0 0 0}
\pscustom[linestyle=none,fillstyle=solid,fillcolor=curcolor]
{
\newpath
\moveto(149.07617188,404)
\lineto(149.07617188,405.0546875)
\lineto(153.4765625,410.55664062)
\curveto(153.7890625,410.94726562)(154.0859375,411.28710938)(154.3671875,411.57617188)
\lineto(149.57421875,411.57617188)
\lineto(149.57421875,412.58984375)
\lineto(155.7265625,412.58984375)
\lineto(155.7265625,411.57617188)
\lineto(150.90429688,405.6171875)
\lineto(150.3828125,405.01367188)
\lineto(155.8671875,405.01367188)
\lineto(155.8671875,404)
\closepath
}
}
{
\newrgbcolor{curcolor}{0 0 0}
\pscustom[linestyle=none,fillstyle=solid,fillcolor=curcolor]
{
\newpath
\moveto(157.15039062,404)
\lineto(157.15039062,412.58984375)
\lineto(162.9453125,412.58984375)
\lineto(162.9453125,411.57617188)
\lineto(158.28710938,411.57617188)
\lineto(158.28710938,408.91601562)
\lineto(162.31835938,408.91601562)
\lineto(162.31835938,407.90234375)
\lineto(158.28710938,407.90234375)
\lineto(158.28710938,404)
\closepath
}
}
{
\newrgbcolor{curcolor}{0 0 0}
\pscustom[linestyle=none,fillstyle=solid,fillcolor=curcolor]
{
\newpath
\moveto(164.03515625,406.75976562)
\lineto(165.10742188,406.85351562)
\curveto(165.15820312,406.42382812)(165.27539062,406.0703125)(165.45898438,405.79296875)
\curveto(165.64648438,405.51953125)(165.93554688,405.296875)(166.32617188,405.125)
\curveto(166.71679688,404.95703125)(167.15625,404.87304688)(167.64453125,404.87304688)
\curveto(168.078125,404.87304688)(168.4609375,404.9375)(168.79296875,405.06640625)
\curveto(169.125,405.1953125)(169.37109375,405.37109375)(169.53125,405.59375)
\curveto(169.6953125,405.8203125)(169.77734375,406.06640625)(169.77734375,406.33203125)
\curveto(169.77734375,406.6015625)(169.69921875,406.8359375)(169.54296875,407.03515625)
\curveto(169.38671875,407.23828125)(169.12890625,407.40820312)(168.76953125,407.54492188)
\curveto(168.5390625,407.63476562)(168.02929688,407.7734375)(167.24023438,407.9609375)
\curveto(166.45117188,408.15234375)(165.8984375,408.33203125)(165.58203125,408.5)
\curveto(165.171875,408.71484375)(164.86523438,408.98046875)(164.66210938,409.296875)
\curveto(164.46289062,409.6171875)(164.36328125,409.97460938)(164.36328125,410.36914062)
\curveto(164.36328125,410.80273438)(164.48632812,411.20703125)(164.73242188,411.58203125)
\curveto(164.97851562,411.9609375)(165.33789062,412.24804688)(165.81054688,412.44335938)
\curveto(166.28320312,412.63867188)(166.80859375,412.73632812)(167.38671875,412.73632812)
\curveto(168.0234375,412.73632812)(168.58398438,412.6328125)(169.06835938,412.42578125)
\curveto(169.55664062,412.22265625)(169.93164062,411.921875)(170.19335938,411.5234375)
\curveto(170.45507812,411.125)(170.59570312,410.67382812)(170.61523438,410.16992188)
\lineto(169.52539062,410.08789062)
\curveto(169.46679688,410.63085938)(169.26757812,411.04101562)(168.92773438,411.31835938)
\curveto(168.59179688,411.59570312)(168.09375,411.734375)(167.43359375,411.734375)
\curveto(166.74609375,411.734375)(166.24414062,411.60742188)(165.92773438,411.35351562)
\curveto(165.61523438,411.10351562)(165.45898438,410.80078125)(165.45898438,410.4453125)
\curveto(165.45898438,410.13671875)(165.5703125,409.8828125)(165.79296875,409.68359375)
\curveto(166.01171875,409.484375)(166.58203125,409.27929688)(167.50390625,409.06835938)
\curveto(168.4296875,408.86132812)(169.06445312,408.6796875)(169.40820312,408.5234375)
\curveto(169.90820312,408.29296875)(170.27734375,408)(170.515625,407.64453125)
\curveto(170.75390625,407.29296875)(170.87304688,406.88671875)(170.87304688,406.42578125)
\curveto(170.87304688,405.96875)(170.7421875,405.53710938)(170.48046875,405.13085938)
\curveto(170.21875,404.72851562)(169.84179688,404.4140625)(169.34960938,404.1875)
\curveto(168.86132812,403.96484375)(168.31054688,403.85351562)(167.69726562,403.85351562)
\curveto(166.91992188,403.85351562)(166.26757812,403.96679688)(165.74023438,404.19335938)
\curveto(165.21679688,404.41992188)(164.8046875,404.75976562)(164.50390625,405.21289062)
\curveto(164.20703125,405.66992188)(164.05078125,406.18554688)(164.03515625,406.75976562)
\closepath
}
}
{
\newrgbcolor{curcolor}{1 0 0}
\pscustom[linewidth=1,linecolor=curcolor]
{
\newpath
\moveto(179.8,407.9)
\lineto(222,407.9)
\moveto(105.1,89.7)
\lineto(183.4,129.6)
\lineto(261.7,217.3)
\lineto(340.1,262.8)
\lineto(418.4,308)
\lineto(496.7,354.3)
\lineto(575,400)
}
}
{
\newrgbcolor{curcolor}{0 0 0}
\pscustom[linestyle=none,fillstyle=solid,fillcolor=curcolor]
{
\newpath
\moveto(127.40820312,390.18359375)
\curveto(127.40820312,391.609375)(127.79101562,392.72460938)(128.55664062,393.52929688)
\curveto(129.32226562,394.33789062)(130.31054688,394.7421875)(131.52148438,394.7421875)
\curveto(132.31445312,394.7421875)(133.02929688,394.55273438)(133.66601562,394.17382812)
\curveto(134.30273438,393.79492188)(134.78710938,393.265625)(135.11914062,392.5859375)
\curveto(135.45507812,391.91015625)(135.62304688,391.14257812)(135.62304688,390.28320312)
\curveto(135.62304688,389.41210938)(135.44726562,388.6328125)(135.09570312,387.9453125)
\curveto(134.74414062,387.2578125)(134.24609375,386.73632812)(133.6015625,386.38085938)
\curveto(132.95703125,386.02929688)(132.26171875,385.85351562)(131.515625,385.85351562)
\curveto(130.70703125,385.85351562)(129.984375,386.04882812)(129.34765625,386.43945312)
\curveto(128.7109375,386.83007812)(128.22851562,387.36328125)(127.90039062,388.0390625)
\curveto(127.57226562,388.71484375)(127.40820312,389.4296875)(127.40820312,390.18359375)
\closepath
\moveto(128.58007812,390.16601562)
\curveto(128.58007812,389.13085938)(128.85742188,388.31445312)(129.41210938,387.71679688)
\curveto(129.97070312,387.12304688)(130.66992188,386.82617188)(131.50976562,386.82617188)
\curveto(132.36523438,386.82617188)(133.06835938,387.12695312)(133.61914062,387.72851562)
\curveto(134.17382812,388.33007812)(134.45117188,389.18359375)(134.45117188,390.2890625)
\curveto(134.45117188,390.98828125)(134.33203125,391.59765625)(134.09375,392.1171875)
\curveto(133.859375,392.640625)(133.51367188,393.04492188)(133.05664062,393.33007812)
\curveto(132.60351562,393.61914062)(132.09375,393.76367188)(131.52734375,393.76367188)
\curveto(130.72265625,393.76367188)(130.02929688,393.48632812)(129.44726562,392.93164062)
\curveto(128.86914062,392.38085938)(128.58007812,391.45898438)(128.58007812,390.16601562)
\closepath
}
}
{
\newrgbcolor{curcolor}{0 0 0}
\pscustom[linestyle=none,fillstyle=solid,fillcolor=curcolor]
{
\newpath
\moveto(136.9296875,386)
\lineto(136.9296875,394.58984375)
\lineto(137.984375,394.58984375)
\lineto(137.984375,386)
\closepath
}
}
{
\newrgbcolor{curcolor}{0 0 0}
\pscustom[linestyle=none,fillstyle=solid,fillcolor=curcolor]
{
\newpath
\moveto(143.65625,386)
\lineto(143.65625,386.78515625)
\curveto(143.26171875,386.16796875)(142.68164062,385.859375)(141.91601562,385.859375)
\curveto(141.41992188,385.859375)(140.96289062,385.99609375)(140.54492188,386.26953125)
\curveto(140.13085938,386.54296875)(139.80859375,386.92382812)(139.578125,387.41210938)
\curveto(139.3515625,387.90429688)(139.23828125,388.46875)(139.23828125,389.10546875)
\curveto(139.23828125,389.7265625)(139.34179688,390.2890625)(139.54882812,390.79296875)
\curveto(139.75585938,391.30078125)(140.06640625,391.68945312)(140.48046875,391.95898438)
\curveto(140.89453125,392.22851562)(141.35742188,392.36328125)(141.86914062,392.36328125)
\curveto(142.24414062,392.36328125)(142.578125,392.28320312)(142.87109375,392.12304688)
\curveto(143.1640625,391.96679688)(143.40234375,391.76171875)(143.5859375,391.5078125)
\lineto(143.5859375,394.58984375)
\lineto(144.63476562,394.58984375)
\lineto(144.63476562,386)
\closepath
\moveto(140.32226562,389.10546875)
\curveto(140.32226562,388.30859375)(140.49023438,387.71289062)(140.82617188,387.31835938)
\curveto(141.16210938,386.92382812)(141.55859375,386.7265625)(142.015625,386.7265625)
\curveto(142.4765625,386.7265625)(142.8671875,386.9140625)(143.1875,387.2890625)
\curveto(143.51171875,387.66796875)(143.67382812,388.24414062)(143.67382812,389.01757812)
\curveto(143.67382812,389.86914062)(143.50976562,390.49414062)(143.18164062,390.89257812)
\curveto(142.85351562,391.29101562)(142.44921875,391.49023438)(141.96875,391.49023438)
\curveto(141.5,391.49023438)(141.10742188,391.29882812)(140.79101562,390.91601562)
\curveto(140.47851562,390.53320312)(140.32226562,389.9296875)(140.32226562,389.10546875)
\closepath
}
}
{
\newrgbcolor{curcolor}{0 0 0}
\pscustom[linestyle=none,fillstyle=solid,fillcolor=curcolor]
{
\newpath
\moveto(149.07617188,386)
\lineto(149.07617188,387.0546875)
\lineto(153.4765625,392.55664062)
\curveto(153.7890625,392.94726562)(154.0859375,393.28710938)(154.3671875,393.57617188)
\lineto(149.57421875,393.57617188)
\lineto(149.57421875,394.58984375)
\lineto(155.7265625,394.58984375)
\lineto(155.7265625,393.57617188)
\lineto(150.90429688,387.6171875)
\lineto(150.3828125,387.01367188)
\lineto(155.8671875,387.01367188)
\lineto(155.8671875,386)
\closepath
}
}
{
\newrgbcolor{curcolor}{0 0 0}
\pscustom[linestyle=none,fillstyle=solid,fillcolor=curcolor]
{
\newpath
\moveto(157.15039062,386)
\lineto(157.15039062,394.58984375)
\lineto(162.9453125,394.58984375)
\lineto(162.9453125,393.57617188)
\lineto(158.28710938,393.57617188)
\lineto(158.28710938,390.91601562)
\lineto(162.31835938,390.91601562)
\lineto(162.31835938,389.90234375)
\lineto(158.28710938,389.90234375)
\lineto(158.28710938,386)
\closepath
}
}
{
\newrgbcolor{curcolor}{0 0 0}
\pscustom[linestyle=none,fillstyle=solid,fillcolor=curcolor]
{
\newpath
\moveto(164.03515625,388.75976562)
\lineto(165.10742188,388.85351562)
\curveto(165.15820312,388.42382812)(165.27539062,388.0703125)(165.45898438,387.79296875)
\curveto(165.64648438,387.51953125)(165.93554688,387.296875)(166.32617188,387.125)
\curveto(166.71679688,386.95703125)(167.15625,386.87304688)(167.64453125,386.87304688)
\curveto(168.078125,386.87304688)(168.4609375,386.9375)(168.79296875,387.06640625)
\curveto(169.125,387.1953125)(169.37109375,387.37109375)(169.53125,387.59375)
\curveto(169.6953125,387.8203125)(169.77734375,388.06640625)(169.77734375,388.33203125)
\curveto(169.77734375,388.6015625)(169.69921875,388.8359375)(169.54296875,389.03515625)
\curveto(169.38671875,389.23828125)(169.12890625,389.40820312)(168.76953125,389.54492188)
\curveto(168.5390625,389.63476562)(168.02929688,389.7734375)(167.24023438,389.9609375)
\curveto(166.45117188,390.15234375)(165.8984375,390.33203125)(165.58203125,390.5)
\curveto(165.171875,390.71484375)(164.86523438,390.98046875)(164.66210938,391.296875)
\curveto(164.46289062,391.6171875)(164.36328125,391.97460938)(164.36328125,392.36914062)
\curveto(164.36328125,392.80273438)(164.48632812,393.20703125)(164.73242188,393.58203125)
\curveto(164.97851562,393.9609375)(165.33789062,394.24804688)(165.81054688,394.44335938)
\curveto(166.28320312,394.63867188)(166.80859375,394.73632812)(167.38671875,394.73632812)
\curveto(168.0234375,394.73632812)(168.58398438,394.6328125)(169.06835938,394.42578125)
\curveto(169.55664062,394.22265625)(169.93164062,393.921875)(170.19335938,393.5234375)
\curveto(170.45507812,393.125)(170.59570312,392.67382812)(170.61523438,392.16992188)
\lineto(169.52539062,392.08789062)
\curveto(169.46679688,392.63085938)(169.26757812,393.04101562)(168.92773438,393.31835938)
\curveto(168.59179688,393.59570312)(168.09375,393.734375)(167.43359375,393.734375)
\curveto(166.74609375,393.734375)(166.24414062,393.60742188)(165.92773438,393.35351562)
\curveto(165.61523438,393.10351562)(165.45898438,392.80078125)(165.45898438,392.4453125)
\curveto(165.45898438,392.13671875)(165.5703125,391.8828125)(165.79296875,391.68359375)
\curveto(166.01171875,391.484375)(166.58203125,391.27929688)(167.50390625,391.06835938)
\curveto(168.4296875,390.86132812)(169.06445312,390.6796875)(169.40820312,390.5234375)
\curveto(169.90820312,390.29296875)(170.27734375,390)(170.515625,389.64453125)
\curveto(170.75390625,389.29296875)(170.87304688,388.88671875)(170.87304688,388.42578125)
\curveto(170.87304688,387.96875)(170.7421875,387.53710938)(170.48046875,387.13085938)
\curveto(170.21875,386.72851562)(169.84179688,386.4140625)(169.34960938,386.1875)
\curveto(168.86132812,385.96484375)(168.31054688,385.85351562)(167.69726562,385.85351562)
\curveto(166.91992188,385.85351562)(166.26757812,385.96679688)(165.74023438,386.19335938)
\curveto(165.21679688,386.41992188)(164.8046875,386.75976562)(164.50390625,387.21289062)
\curveto(164.20703125,387.66992188)(164.05078125,388.18554688)(164.03515625,388.75976562)
\closepath
}
}
{
\newrgbcolor{curcolor}{0 0 1}
\pscustom[linewidth=1,linecolor=curcolor]
{
\newpath
\moveto(179.8,389.9)
\lineto(222,389.9)
\moveto(105.1,87.9)
\lineto(183.4,125.9)
\lineto(261.7,223.6)
\lineto(340.1,269.1)
\lineto(418.4,316.4)
\lineto(496.7,362.4)
\lineto(575,409.3)
}
}
{
\newrgbcolor{curcolor}{0 0 0}
\pscustom[linewidth=1,linecolor=curcolor]
{
\newpath
\moveto(105.1,425.9)
\lineto(105.1,57.6)
\lineto(575,57.6)
\lineto(575,425.9)
\closepath
}
}
\end{pspicture}
}
    \resizebox{0.40\linewidth}{!}{%LaTeX with PSTricks extensions
%%Creator: Inkscape 1.0.2-2 (e86c870879, 2021-01-15)
%%Please note this file requires PSTricks extensions
\psset{xunit=.5pt,yunit=.5pt,runit=.5pt}
\begin{pspicture}(535.30073511,423.12458669)
{
\newrgbcolor{curcolor}{0.80000001 0.80000001 0.80000001}
\pscustom[linestyle=none,fillstyle=solid,fillcolor=curcolor]
{
\newpath
\moveto(410.08601575,262.18161094)
\lineto(410.08601575,169.2475259)
}
}
{
\newrgbcolor{curcolor}{0 0 0}
\pscustom[linewidth=2.64566925,linecolor=curcolor]
{
\newpath
\moveto(410.08601575,262.18161094)
\lineto(410.08601575,169.2475259)
}
}
{
\newrgbcolor{curcolor}{0 0 0}
\pscustom[linewidth=2.64566925,linecolor=curcolor]
{
\newpath
\moveto(125.21474646,262.51035425)
\lineto(125.21474646,169.57630701)
}
}
{
\newrgbcolor{curcolor}{0.80000001 0.80000001 0.80000001}
\pscustom[linestyle=none,fillstyle=solid,fillcolor=curcolor]
{
\newpath
\moveto(160.38875955,376.99323056)
\lineto(374.91198367,376.99323056)
\lineto(374.91198367,319.25593837)
\lineto(160.38875955,319.25593837)
\closepath
}
}
{
\newrgbcolor{curcolor}{0 0 0}
\pscustom[linewidth=0.4430022,linecolor=curcolor]
{
\newpath
\moveto(160.38875955,376.99323056)
\lineto(374.91198367,376.99323056)
\lineto(374.91198367,319.25593837)
\lineto(160.38875955,319.25593837)
\closepath
}
}
{
\newrgbcolor{curcolor}{0.80000001 0.80000001 0.80000001}
\pscustom[linestyle=none,fillstyle=solid,fillcolor=curcolor]
{
\newpath
\moveto(41.37207917,422.79387724)
\lineto(209.05740017,422.79387724)
\lineto(209.05740017,258.13902384)
\lineto(41.37207917,258.13902384)
\closepath
}
}
{
\newrgbcolor{curcolor}{0 0 0}
\pscustom[linewidth=0.66141731,linecolor=curcolor]
{
\newpath
\moveto(41.37207917,422.79387724)
\lineto(209.05740017,422.79387724)
\lineto(209.05740017,258.13902384)
\lineto(41.37207917,258.13902384)
\closepath
}
}
{
\newrgbcolor{curcolor}{0.80000001 0.80000001 0.80000001}
\pscustom[linestyle=none,fillstyle=solid,fillcolor=curcolor]
{
\newpath
\moveto(326.24335747,422.79387724)
\lineto(493.92867847,422.79387724)
\lineto(493.92867847,258.13902384)
\lineto(326.24335747,258.13902384)
\closepath
}
}
{
\newrgbcolor{curcolor}{0 0 0}
\pscustom[linewidth=0.66141731,linecolor=curcolor]
{
\newpath
\moveto(326.24335747,422.79387724)
\lineto(493.92867847,422.79387724)
\lineto(493.92867847,258.13902384)
\lineto(326.24335747,258.13902384)
\closepath
}
}
{
\newrgbcolor{curcolor}{0 0 0}
\pscustom[linestyle=none,fillstyle=solid,fillcolor=curcolor]
{
\newpath
\moveto(111.53315206,328.07395942)
\curveto(110.81700851,327.76146041)(110.16596892,327.46849259)(109.58003328,327.19505596)
\curveto(109.00711844,326.92161933)(108.25191251,326.63516191)(107.3144155,326.3356837)
\curveto(106.52014719,326.08828865)(105.65426453,325.87995598)(104.71676752,325.71068569)
\curveto(103.79229129,325.5283946)(102.77015913,325.43724906)(101.65037103,325.43724906)
\curveto(99.54100275,325.43724906)(97.62043595,325.73021688)(95.88867063,326.31615251)
\curveto(94.1699261,326.91510894)(92.67253503,327.84609556)(91.39649743,329.10911237)
\curveto(90.14650141,330.3460876)(89.16994202,331.91509302)(88.46681926,333.81612863)
\curveto(87.7636965,335.73018504)(87.41213511,337.95023005)(87.41213511,340.47626368)
\curveto(87.41213511,342.87208938)(87.7506757,345.01400965)(88.42775688,346.90202447)
\curveto(89.10483806,348.79003929)(90.08139745,350.3850863)(91.35743505,351.68716548)
\curveto(92.59441028,352.95018229)(94.08529095,353.91372089)(95.83007706,354.57778128)
\curveto(97.58788397,355.24184167)(99.53449235,355.57387186)(101.66990222,355.57387186)
\curveto(103.23239724,355.57387186)(104.78838187,355.38507038)(106.33785611,355.00746741)
\curveto(107.90035113,354.62986445)(109.63211645,353.96580406)(111.53315206,353.01528625)
\lineto(111.53315206,348.42545712)
\lineto(111.24018425,348.42545712)
\curveto(109.63862685,349.76659868)(108.05009024,350.74315807)(106.47457442,351.35513529)
\curveto(104.8990586,351.96711251)(103.21286606,352.27310112)(101.41599678,352.27310112)
\curveto(99.9446473,352.27310112)(98.61652652,352.03221647)(97.43163446,351.55044717)
\curveto(96.25976319,351.08169866)(95.21158945,350.34602392)(94.28711323,349.34342295)
\curveto(93.38867859,348.36686355)(92.68555582,347.12988833)(92.17774494,345.63249726)
\curveto(91.68295485,344.14812699)(91.4355598,342.42938246)(91.4355598,340.47626368)
\curveto(91.4355598,338.43199935)(91.70899643,336.67419245)(92.25586969,335.20284297)
\curveto(92.81576374,333.73149349)(93.5319073,332.53358063)(94.40430035,331.60910441)
\curveto(95.31575578,330.64556581)(96.37695032,329.92942226)(97.58788397,329.46067375)
\curveto(98.8118384,329.00494603)(100.1008968,328.77708218)(101.45505915,328.77708218)
\curveto(103.31703239,328.77708218)(105.0618185,329.09609158)(106.68941749,329.73411038)
\curveto(108.31701647,330.37212918)(109.84044912,331.32915738)(111.25971544,332.60519499)
\lineto(111.53315206,332.60519499)
\closepath
}
}
{
\newrgbcolor{curcolor}{0 0 0}
\pscustom[linestyle=none,fillstyle=solid,fillcolor=curcolor]
{
\newpath
\moveto(136.025261,346.25749527)
\curveto(136.025261,344.96843687)(135.79739714,343.77052402)(135.34166943,342.66375671)
\curveto(134.8989625,341.5700102)(134.27396449,340.61949239)(133.4666754,339.81220329)
\curveto(132.46407442,338.80960232)(131.27918236,338.05439639)(129.91199921,337.54658551)
\curveto(128.54481607,337.05179541)(126.81956114,336.80440037)(124.73623444,336.80440037)
\lineto(120.86905926,336.80440037)
\lineto(120.86905926,325.96459113)
\lineto(117.00188407,325.96459113)
\lineto(117.00188407,355.04652979)
\lineto(124.89248395,355.04652979)
\curveto(126.63727006,355.04652979)(128.11512993,354.89679068)(129.32606358,354.59731247)
\curveto(130.53699722,354.31085505)(131.61121255,353.85512733)(132.54870957,353.23012932)
\curveto(133.65547688,352.48794418)(134.50833875,351.56346796)(135.10729517,350.45670065)
\curveto(135.71927239,349.34993334)(136.025261,347.95019821)(136.025261,346.25749527)
\closepath
\moveto(132.00183631,346.15983933)
\curveto(132.00183631,347.16244031)(131.82605562,348.03483336)(131.47449424,348.7770185)
\curveto(131.12293286,349.51920364)(130.58908039,350.12467046)(129.87293684,350.59341897)
\curveto(129.24793883,350.99706351)(128.53179527,351.28352093)(127.72450618,351.45279123)
\curveto(126.93023787,351.63508232)(125.9211265,351.72622786)(124.69717207,351.72622786)
\lineto(120.86905926,351.72622786)
\lineto(120.86905926,340.10517111)
\lineto(124.13076762,340.10517111)
\curveto(125.69326265,340.10517111)(126.96278985,340.24188942)(127.93934924,340.51532605)
\curveto(128.91590864,340.80178347)(129.71017694,341.25100079)(130.32215416,341.86297801)
\curveto(130.93413138,342.48797602)(131.36381751,343.14552601)(131.61121255,343.83562798)
\curveto(131.87162839,344.52572995)(132.00183631,345.30046707)(132.00183631,346.15983933)
\closepath
}
}
{
\newrgbcolor{curcolor}{0 0 0}
\pscustom[linestyle=none,fillstyle=solid,fillcolor=curcolor]
{
\newpath
\moveto(163.01736066,337.64424144)
\curveto(163.01736066,335.53487316)(162.78298641,333.69243111)(162.3142379,332.11691529)
\curveto(161.85851018,330.55442027)(161.10330426,329.25234108)(160.04862011,328.21067773)
\curveto(159.04601914,327.22109755)(157.87414787,326.4984436)(156.53300631,326.04271588)
\curveto(155.19186474,325.58698817)(153.62936972,325.35912431)(151.84552123,325.35912431)
\curveto(150.02261037,325.35912431)(148.43407376,325.60000896)(147.0799114,326.08177826)
\curveto(145.72574905,326.56354756)(144.58642976,327.27318072)(143.66195354,328.21067773)
\curveto(142.6072694,329.27838266)(141.84555307,330.56744106)(141.37680456,332.07785292)
\curveto(140.92107685,333.58826478)(140.69321299,335.44372762)(140.69321299,337.64424144)
\lineto(140.69321299,355.04652979)
\lineto(144.56038818,355.04652979)
\lineto(144.56038818,337.44892957)
\curveto(144.56038818,335.87341375)(144.66455451,334.62992813)(144.87288718,333.71847269)
\curveto(145.09424064,332.80701726)(145.45882282,331.98019698)(145.9666337,331.23801184)
\curveto(146.53954854,330.39166037)(147.31428566,329.75364157)(148.29084505,329.32395544)
\curveto(149.28042523,328.8942693)(150.46531729,328.67942624)(151.84552123,328.67942624)
\curveto(153.23874596,328.67942624)(154.42363802,328.88775891)(155.40019741,329.30442425)
\curveto(156.3767568,329.73411038)(157.15800432,330.37863958)(157.74393995,331.23801184)
\curveto(158.25175083,331.98019698)(158.60982261,332.82654845)(158.81815528,333.77706626)
\curveto(159.03950874,334.74060486)(159.15018547,335.93200731)(159.15018547,337.35127363)
\lineto(159.15018547,355.04652979)
\lineto(163.01736066,355.04652979)
\closepath
}
}
{
\newrgbcolor{curcolor}{0 0 0}
\pscustom[linestyle=none,fillstyle=solid,fillcolor=curcolor]
{
\newpath
\moveto(396.40441594,323.02316086)
\curveto(395.68827239,322.71066186)(395.0372328,322.41769404)(394.45129716,322.14425741)
\curveto(393.87838232,321.87082078)(393.12317639,321.58436336)(392.18567938,321.28488515)
\curveto(391.39141107,321.0374901)(390.52552841,320.82915743)(389.5880314,320.65988714)
\curveto(388.66355517,320.47759605)(387.64142301,320.38645051)(386.52163491,320.38645051)
\curveto(384.41226663,320.38645051)(382.49169982,320.67941833)(380.75993451,321.26535396)
\curveto(379.04118998,321.86431039)(377.54379891,322.79529701)(376.26776131,324.05831382)
\curveto(375.01776529,325.29528905)(374.0412059,326.86429447)(373.33808314,328.76533008)
\curveto(372.63496037,330.67938649)(372.28339899,332.8994315)(372.28339899,335.42546513)
\curveto(372.28339899,337.82129083)(372.62193958,339.9632111)(373.29902076,341.85122592)
\curveto(373.97610194,343.73924074)(374.95266133,345.33428774)(376.22869893,346.63636693)
\curveto(377.46567416,347.89938374)(378.95655483,348.86292234)(380.70134094,349.52698273)
\curveto(382.45914785,350.19104311)(384.40575623,350.52307331)(386.5411661,350.52307331)
\curveto(388.10366112,350.52307331)(389.65964575,350.33427182)(391.20911999,349.95666886)
\curveto(392.77161501,349.5790659)(394.50338033,348.91500551)(396.40441594,347.9644877)
\lineto(396.40441594,343.37465857)
\lineto(396.11144813,343.37465857)
\curveto(394.50989073,344.71580013)(392.92135412,345.69235952)(391.3458383,346.30433674)
\curveto(389.77032248,346.91631396)(388.08412994,347.22230257)(386.28726066,347.22230257)
\curveto(384.81591117,347.22230257)(383.4877904,346.98141792)(382.30289834,346.49964862)
\curveto(381.13102707,346.03090011)(380.08285333,345.29522537)(379.1583771,344.29262439)
\curveto(378.25994247,343.316065)(377.5568197,342.07908978)(377.04900882,340.58169871)
\curveto(376.55421873,339.09732844)(376.30682368,337.37858391)(376.30682368,335.42546513)
\curveto(376.30682368,333.3812008)(376.58026031,331.6233939)(377.12713357,330.15204442)
\curveto(377.68702762,328.68069493)(378.40317118,327.48278208)(379.27556423,326.55830586)
\curveto(380.18701966,325.59476726)(381.2482142,324.87862371)(382.45914785,324.4098752)
\curveto(383.68310228,323.95414748)(384.97216068,323.72628363)(386.32632303,323.72628363)
\curveto(388.18829627,323.72628363)(389.93308238,324.04529303)(391.56068137,324.68331183)
\curveto(393.18828035,325.32133063)(394.711713,326.27835883)(396.13097931,327.55439644)
\lineto(396.40441594,327.55439644)
\closepath
}
}
{
\newrgbcolor{curcolor}{0 0 0}
\pscustom[linestyle=none,fillstyle=solid,fillcolor=curcolor]
{
\newpath
\moveto(420.89652488,341.20669672)
\curveto(420.89652488,339.91763832)(420.66866102,338.71972547)(420.21293331,337.61295816)
\curveto(419.77022638,336.51921164)(419.14522837,335.56869384)(418.33793928,334.76140474)
\curveto(417.3353383,333.75880377)(416.15044624,333.00359784)(414.78326309,332.49578695)
\curveto(413.41607995,332.00099686)(411.69082502,331.75360182)(409.60749832,331.75360182)
\lineto(405.74032314,331.75360182)
\lineto(405.74032314,320.91379258)
\lineto(401.87314795,320.91379258)
\lineto(401.87314795,349.99573124)
\lineto(409.76374782,349.99573124)
\curveto(411.50853394,349.99573124)(412.98639381,349.84599213)(414.19732746,349.54651392)
\curveto(415.4082611,349.26005649)(416.48247643,348.80432878)(417.41997345,348.17933077)
\curveto(418.52674076,347.43714563)(419.37960263,346.51266941)(419.97855905,345.4059021)
\curveto(420.59053627,344.29913479)(420.89652488,342.89939966)(420.89652488,341.20669672)
\closepath
\moveto(416.87310019,341.10904078)
\curveto(416.87310019,342.11164175)(416.6973195,342.98403481)(416.34575812,343.72621995)
\curveto(415.99419674,344.46840508)(415.46034427,345.07387191)(414.74420072,345.54262041)
\curveto(414.11920271,345.94626496)(413.40305915,346.23272238)(412.59577006,346.40199268)
\curveto(411.80150175,346.58428376)(410.79239038,346.67542931)(409.56843595,346.67542931)
\lineto(405.74032314,346.67542931)
\lineto(405.74032314,335.05437256)
\lineto(409.0020315,335.05437256)
\curveto(410.56452653,335.05437256)(411.83405373,335.19109087)(412.81061312,335.4645275)
\curveto(413.78717251,335.75098492)(414.58144082,336.20020224)(415.19341804,336.81217946)
\curveto(415.80539526,337.43717747)(416.23508139,338.09472746)(416.48247643,338.78482943)
\curveto(416.74289227,339.4749314)(416.87310019,340.24966852)(416.87310019,341.10904078)
\closepath
}
}
{
\newrgbcolor{curcolor}{0 0 0}
\pscustom[linestyle=none,fillstyle=solid,fillcolor=curcolor]
{
\newpath
\moveto(447.88862454,332.59344289)
\curveto(447.88862454,330.48407461)(447.65425029,328.64163256)(447.18550178,327.06611674)
\curveto(446.72977406,325.50362172)(445.97456813,324.20154253)(444.91988399,323.15987918)
\curveto(443.91728302,322.170299)(442.74541175,321.44764505)(441.40427019,320.99191733)
\curveto(440.06312862,320.53618962)(438.5006336,320.30832576)(436.71678511,320.30832576)
\curveto(434.89387425,320.30832576)(433.30533764,320.54921041)(431.95117528,321.03097971)
\curveto(430.59701293,321.51274901)(429.45769364,322.22238216)(428.53321742,323.15987918)
\curveto(427.47853327,324.22758411)(426.71681695,325.51664251)(426.24806844,327.02705437)
\curveto(425.79234073,328.53746622)(425.56447687,330.39292907)(425.56447687,332.59344289)
\lineto(425.56447687,349.99573124)
\lineto(429.43165206,349.99573124)
\lineto(429.43165206,332.39813102)
\curveto(429.43165206,330.8226152)(429.53581839,329.57912957)(429.74415106,328.66767414)
\curveto(429.96550452,327.75621871)(430.3300867,326.92939843)(430.83789758,326.18721329)
\curveto(431.41081242,325.34086182)(432.18554954,324.70284302)(433.16210893,324.27315688)
\curveto(434.15168911,323.84347075)(435.33658117,323.62862769)(436.71678511,323.62862769)
\curveto(438.11000984,323.62862769)(439.2949019,323.83696036)(440.27146129,324.2536257)
\curveto(441.24802068,324.68331183)(442.0292682,325.32784103)(442.61520383,326.18721329)
\curveto(443.12301471,326.92939843)(443.48108649,327.7757499)(443.68941916,328.72626771)
\curveto(443.91077262,329.68980631)(444.02144935,330.88120876)(444.02144935,332.30047508)
\lineto(444.02144935,349.99573124)
\lineto(447.88862454,349.99573124)
\closepath
}
}
{
\newrgbcolor{curcolor}{0 0 0}
\pscustom[linestyle=none,fillstyle=solid,fillcolor=curcolor]
{
\newpath
\moveto(224.90445649,338.63244012)
\lineto(220.60794414,338.63244012)
\lineto(220.60794414,340.61160632)
\lineto(222.03280793,340.61160632)
\lineto(222.03280793,356.0412902)
\lineto(220.60794414,356.0412902)
\lineto(220.60794414,358.0204564)
\lineto(224.90445649,358.0204564)
\lineto(224.90445649,356.0412902)
\lineto(223.4795927,356.0412902)
\lineto(223.4795927,340.61160632)
\lineto(224.90445649,340.61160632)
\closepath
}
}
{
\newrgbcolor{curcolor}{0 0 0}
\pscustom[linestyle=none,fillstyle=solid,fillcolor=curcolor]
{
\newpath
\moveto(234.08204078,338.63244012)
\lineto(232.70832595,338.63244012)
\lineto(232.70832595,346.91368818)
\curveto(232.70832595,347.5820908)(232.68640497,348.20709065)(232.642563,348.78868774)
\curveto(232.59872104,349.37896538)(232.51834411,349.83903471)(232.40143221,350.16889575)
\curveto(232.27964898,350.53347899)(232.10428113,350.80257615)(231.87532865,350.97618722)
\curveto(231.64637618,351.15847885)(231.3492251,351.24962466)(230.98387541,351.24962466)
\curveto(230.60878306,351.24962466)(230.21664106,351.08469414)(229.80744941,350.75483311)
\curveto(229.39825776,350.42497207)(229.00611576,350.00396523)(228.63102341,349.49181257)
\lineto(228.63102341,338.63244012)
\lineto(227.25730858,338.63244012)
\lineto(227.25730858,353.17670754)
\lineto(228.63102341,353.17670754)
\lineto(228.63102341,351.56212458)
\curveto(229.05970038,352.19580499)(229.50299134,352.69059654)(229.96089628,353.04649924)
\curveto(230.41880123,353.40240193)(230.88888449,353.58035328)(231.37114608,353.58035328)
\curveto(232.25285667,353.58035328)(232.9251001,353.10726311)(233.38787637,352.16108278)
\curveto(233.85065264,351.21490244)(234.08204078,349.85205554)(234.08204078,348.07254207)
\closepath
}
}
{
\newrgbcolor{curcolor}{0 0 0}
\pscustom[linestyle=none,fillstyle=solid,fillcolor=curcolor]
{
\newpath
\moveto(240.9652289,338.76264842)
\curveto(240.70704845,338.64112068)(240.42451136,338.54129431)(240.11761762,338.46316933)
\curveto(239.81559521,338.38504435)(239.54523644,338.34598186)(239.30654131,338.34598186)
\curveto(238.47354401,338.34598186)(237.84027122,338.74528732)(237.40672292,339.54389824)
\curveto(236.97317462,340.34250916)(236.75640047,341.62289081)(236.75640047,343.38504317)
\lineto(236.75640047,351.11941636)
\lineto(235.82841226,351.11941636)
\lineto(235.82841226,353.17670754)
\lineto(236.75640047,353.17670754)
\lineto(236.75640047,357.35639406)
\lineto(238.13011531,357.35639406)
\lineto(238.13011531,353.17670754)
\lineto(240.9652289,353.17670754)
\lineto(240.9652289,351.11941636)
\lineto(238.13011531,351.11941636)
\lineto(238.13011531,344.49181375)
\curveto(238.13011531,343.72792504)(238.13985797,343.12896684)(238.15934328,342.69493917)
\curveto(238.1788286,342.26959205)(238.24702721,341.87028658)(238.36393911,341.49702278)
\curveto(238.47110835,341.14980064)(238.61724823,340.89372431)(238.80235874,340.7287938)
\curveto(238.99234057,340.57254383)(239.279749,340.49441885)(239.664584,340.49441885)
\curveto(239.88866515,340.49441885)(240.12248895,340.55084245)(240.36605541,340.66368964)
\curveto(240.60962187,340.78521739)(240.78498972,340.88504376)(240.89215896,340.96316874)
\lineto(240.9652289,340.96316874)
\closepath
}
}
{
\newrgbcolor{curcolor}{0 0 0}
\pscustom[linestyle=none,fillstyle=solid,fillcolor=curcolor]
{
\newpath
\moveto(249.45595642,345.65066764)
\lineto(243.44230053,345.65066764)
\curveto(243.44230053,344.75657063)(243.51780613,343.97532081)(243.66881733,343.30691819)
\curveto(243.81982854,342.64719612)(244.02686003,342.10466153)(244.28991181,341.67931441)
\curveto(244.54322092,341.26264784)(244.84280767,340.95014791)(245.18867204,340.74181463)
\curveto(245.53940774,340.53348134)(245.92424275,340.4293147)(246.34317706,340.4293147)
\curveto(246.89850859,340.4293147)(247.45627578,340.62462715)(248.01647864,341.01525206)
\curveto(248.58155283,341.41455752)(248.98343748,341.80518243)(249.22213261,342.18712679)
\lineto(249.29520255,342.18712679)
\lineto(249.29520255,339.51785658)
\curveto(248.83242628,339.17063444)(248.35990735,338.8798359)(247.87764576,338.64546095)
\curveto(247.39538417,338.41108601)(246.88876593,338.29389853)(246.35779105,338.29389853)
\curveto(245.00356153,338.29389853)(243.9464831,338.94494005)(243.18655574,340.24702308)
\curveto(242.42662839,341.55778666)(242.04666471,343.41542511)(242.04666471,345.81993843)
\curveto(242.04666471,348.1984101)(242.40957874,350.08643049)(243.13540679,351.4839996)
\curveto(243.86610617,352.88156872)(244.82575802,353.58035328)(246.01436234,353.58035328)
\curveto(247.11528274,353.58035328)(247.96289402,353.00743675)(248.55719618,351.86160368)
\curveto(249.15636967,350.71577062)(249.45595642,349.08816683)(249.45595642,346.97879233)
\closepath
\moveto(248.11877655,347.5256672)
\curveto(248.11390522,348.81038912)(247.93123038,349.8043125)(247.57075202,350.50743733)
\curveto(247.21514499,351.21056217)(246.67199178,351.56212458)(245.9412924,351.56212458)
\curveto(245.20572169,351.56212458)(244.61872653,351.17583995)(244.1803069,350.40327069)
\curveto(243.7467586,349.63070143)(243.50075648,348.67150026)(243.44230053,347.5256672)
\closepath
}
}
{
\newrgbcolor{curcolor}{0 0 0}
\pscustom[linestyle=none,fillstyle=solid,fillcolor=curcolor]
{
\newpath
\moveto(256.63142372,350.50743733)
\lineto(256.55835378,350.50743733)
\curveto(256.35375795,350.59424287)(256.15403346,350.65500674)(255.95918029,350.68972896)
\curveto(255.76919845,350.73313172)(255.54268164,350.75483311)(255.27962987,350.75483311)
\curveto(254.85582423,350.75483311)(254.44663258,350.58556231)(254.05205491,350.24702073)
\curveto(253.65747725,349.91715969)(253.27751357,349.48747229)(252.91216388,348.95795853)
\lineto(252.91216388,338.63244012)
\lineto(251.53844905,338.63244012)
\lineto(251.53844905,353.17670754)
\lineto(252.91216388,353.17670754)
\lineto(252.91216388,351.02827054)
\curveto(253.45775275,351.80952036)(253.93757867,352.36073551)(254.35164166,352.68191599)
\curveto(254.77057597,353.01177702)(255.19681727,353.17670754)(255.63036557,353.17670754)
\curveto(255.8690607,353.17670754)(256.04199289,353.16368671)(256.14916213,353.13764505)
\curveto(256.25633137,353.12028394)(256.41708523,353.08122145)(256.63142372,353.02045758)
\closepath
}
}
{
\newrgbcolor{curcolor}{0 0 0}
\pscustom[linestyle=none,fillstyle=solid,fillcolor=curcolor]
{
\newpath
\moveto(263.9603386,339.54389824)
\curveto(263.50243365,339.15327333)(263.06644969,338.84945396)(262.65238671,338.63244012)
\curveto(262.24319506,338.41542628)(261.80721109,338.30691937)(261.34443482,338.30691937)
\curveto(260.75500399,338.30691937)(260.21428645,338.45882905)(259.7222822,338.76264842)
\curveto(259.23027795,339.07514835)(258.80890797,339.54389824)(258.45817227,340.16889809)
\curveto(258.10256524,340.79389795)(257.82733514,341.58382832)(257.63248197,342.5386892)
\curveto(257.43762881,343.49355009)(257.34020222,344.60900122)(257.34020222,345.88504259)
\curveto(257.34020222,348.26351425)(257.70555191,350.12983325)(258.43625129,351.4839996)
\curveto(259.171822,352.83816595)(260.14121651,353.51524913)(261.34443482,353.51524913)
\curveto(261.81208242,353.51524913)(262.26998737,353.39806165)(262.71814965,353.16368671)
\curveto(263.17118327,352.92931176)(263.58524625,352.6428535)(263.9603386,352.30431191)
\lineto(263.9603386,349.58295838)
\lineto(263.88726866,349.58295838)
\curveto(263.46833435,350.16455547)(263.03478605,350.61160397)(262.58662376,350.9241039)
\curveto(262.14333281,351.23660383)(261.70978451,351.39285379)(261.28597887,351.39285379)
\curveto(260.5065662,351.39285379)(259.89034306,350.9241039)(259.43730944,349.98660412)
\curveto(258.98914715,349.0577849)(258.76506601,347.69059772)(258.76506601,345.88504259)
\curveto(258.76506601,344.13157077)(258.98427583,342.7817447)(259.42269545,341.83556437)
\curveto(259.86598641,340.89806459)(260.48708088,340.4293147)(261.28597887,340.4293147)
\curveto(261.56364463,340.4293147)(261.84618173,340.49441885)(262.13359015,340.62462715)
\curveto(262.42099857,340.75483546)(262.67917902,340.92410625)(262.90813149,341.13243953)
\curveto(263.10785599,341.31473116)(263.29540216,341.50570334)(263.47077001,341.70535607)
\curveto(263.64613786,341.91368935)(263.78497075,342.0916407)(263.88726866,342.23921011)
\lineto(263.9603386,342.23921011)
\closepath
}
}
{
\newrgbcolor{curcolor}{0 0 0}
\pscustom[linestyle=none,fillstyle=solid,fillcolor=curcolor]
{
\newpath
\moveto(272.6775822,345.89806342)
\curveto(272.6775822,343.52827231)(272.33658915,341.65761302)(271.65460307,340.28608557)
\curveto(270.97261698,338.91455811)(270.05924276,338.22879438)(268.9144804,338.22879438)
\curveto(267.75997538,338.22879438)(266.84172982,338.91455811)(266.15974374,340.28608557)
\curveto(265.48262898,341.65761302)(265.1440716,343.52827231)(265.1440716,345.89806342)
\curveto(265.1440716,348.26785453)(265.48262898,350.13851381)(266.15974374,351.51004126)
\curveto(266.84172982,352.89024927)(267.75997538,353.58035328)(268.9144804,353.58035328)
\curveto(270.05924276,353.58035328)(270.97261698,352.89024927)(271.65460307,351.51004126)
\curveto(272.33658915,350.13851381)(272.6775822,348.26785453)(272.6775822,345.89806342)
\closepath
\moveto(271.2600254,345.89806342)
\curveto(271.2600254,347.78174353)(271.05299391,349.17931264)(270.63893093,350.09077076)
\curveto(270.22486795,351.01090944)(269.6500511,351.47097877)(268.9144804,351.47097877)
\curveto(268.16916703,351.47097877)(267.58947885,351.01090944)(267.17541587,350.09077076)
\curveto(266.76622422,349.17931264)(266.5616284,347.78174353)(266.5616284,345.89806342)
\curveto(266.5616284,344.07514718)(266.76865989,342.69059889)(267.18272287,341.74441856)
\curveto(267.59678585,340.80691878)(268.17403836,340.33816889)(268.9144804,340.33816889)
\curveto(269.64517977,340.33816889)(270.21756095,340.8025785)(270.63162394,341.73139773)
\curveto(271.05055825,342.66889751)(271.2600254,344.05778607)(271.2600254,345.89806342)
\closepath
}
}
{
\newrgbcolor{curcolor}{0 0 0}
\pscustom[linestyle=none,fillstyle=solid,fillcolor=curcolor]
{
\newpath
\moveto(281.62864807,338.63244012)
\lineto(280.25493324,338.63244012)
\lineto(280.25493324,346.91368818)
\curveto(280.25493324,347.5820908)(280.23301226,348.20709065)(280.18917029,348.78868774)
\curveto(280.14532833,349.37896538)(280.0649514,349.83903471)(279.9480395,350.16889575)
\curveto(279.82625627,350.53347899)(279.65088842,350.80257615)(279.42193595,350.97618722)
\curveto(279.19298347,351.15847885)(278.89583239,351.24962466)(278.5304827,351.24962466)
\curveto(278.15539036,351.24962466)(277.76324836,351.08469414)(277.3540567,350.75483311)
\curveto(276.94486505,350.42497207)(276.55272305,350.00396523)(276.1776307,349.49181257)
\lineto(276.1776307,338.63244012)
\lineto(274.80391587,338.63244012)
\lineto(274.80391587,353.17670754)
\lineto(276.1776307,353.17670754)
\lineto(276.1776307,351.56212458)
\curveto(276.60630767,352.19580499)(277.04959863,352.69059654)(277.50750357,353.04649924)
\curveto(277.96540852,353.40240193)(278.43549178,353.58035328)(278.91775337,353.58035328)
\curveto(279.79946396,353.58035328)(280.47170739,353.10726311)(280.93448366,352.16108278)
\curveto(281.39725993,351.21490244)(281.62864807,349.85205554)(281.62864807,348.07254207)
\closepath
}
}
{
\newrgbcolor{curcolor}{0 0 0}
\pscustom[linestyle=none,fillstyle=solid,fillcolor=curcolor]
{
\newpath
\moveto(291.09851131,338.63244012)
\lineto(289.72479648,338.63244012)
\lineto(289.72479648,346.91368818)
\curveto(289.72479648,347.5820908)(289.7028755,348.20709065)(289.65903354,348.78868774)
\curveto(289.61519158,349.37896538)(289.53481464,349.83903471)(289.41790274,350.16889575)
\curveto(289.29611951,350.53347899)(289.12075166,350.80257615)(288.89179919,350.97618722)
\curveto(288.66284672,351.15847885)(288.36569564,351.24962466)(288.00034595,351.24962466)
\curveto(287.6252536,351.24962466)(287.2331116,351.08469414)(286.82391995,350.75483311)
\curveto(286.4147283,350.42497207)(286.0225863,350.00396523)(285.64749395,349.49181257)
\lineto(285.64749395,338.63244012)
\lineto(284.27377911,338.63244012)
\lineto(284.27377911,353.17670754)
\lineto(285.64749395,353.17670754)
\lineto(285.64749395,351.56212458)
\curveto(286.07617092,352.19580499)(286.51946187,352.69059654)(286.97736682,353.04649924)
\curveto(287.43527176,353.40240193)(287.90535503,353.58035328)(288.38761662,353.58035328)
\curveto(289.2693272,353.58035328)(289.94157063,353.10726311)(290.4043469,352.16108278)
\curveto(290.86712318,351.21490244)(291.09851131,349.85205554)(291.09851131,348.07254207)
\closepath
}
}
{
\newrgbcolor{curcolor}{0 0 0}
\pscustom[linestyle=none,fillstyle=solid,fillcolor=curcolor]
{
\newpath
\moveto(300.57568155,345.65066764)
\lineto(294.56202566,345.65066764)
\curveto(294.56202566,344.75657063)(294.63753127,343.97532081)(294.78854247,343.30691819)
\curveto(294.93955368,342.64719612)(295.14658517,342.10466153)(295.40963694,341.67931441)
\curveto(295.66294606,341.26264784)(295.96253281,340.95014791)(296.30839718,340.74181463)
\curveto(296.65913288,340.53348134)(297.04396789,340.4293147)(297.4629022,340.4293147)
\curveto(298.01823373,340.4293147)(298.57600092,340.62462715)(299.13620378,341.01525206)
\curveto(299.70127796,341.41455752)(300.10316262,341.80518243)(300.34185775,342.18712679)
\lineto(300.41492769,342.18712679)
\lineto(300.41492769,339.51785658)
\curveto(299.95215142,339.17063444)(299.47963248,338.8798359)(298.99737089,338.64546095)
\curveto(298.5151093,338.41108601)(298.00849107,338.29389853)(297.47751619,338.29389853)
\curveto(296.12328667,338.29389853)(295.06620824,338.94494005)(294.30628088,340.24702308)
\curveto(293.54635353,341.55778666)(293.16638985,343.41542511)(293.16638985,345.81993843)
\curveto(293.16638985,348.1984101)(293.52930388,350.08643049)(294.25513192,351.4839996)
\curveto(294.9858313,352.88156872)(295.94548315,353.58035328)(297.13408748,353.58035328)
\curveto(298.23500788,353.58035328)(299.08261916,353.00743675)(299.67692132,351.86160368)
\curveto(300.27609481,350.71577062)(300.57568155,349.08816683)(300.57568155,346.97879233)
\closepath
\moveto(299.23850169,347.5256672)
\curveto(299.23363036,348.81038912)(299.05095552,349.8043125)(298.69047716,350.50743733)
\curveto(298.33487012,351.21056217)(297.79171692,351.56212458)(297.06101754,351.56212458)
\curveto(296.32544683,351.56212458)(295.73845166,351.17583995)(295.30003204,350.40327069)
\curveto(294.86648374,349.63070143)(294.62048161,348.67150026)(294.56202566,347.5256672)
\closepath
}
}
{
\newrgbcolor{curcolor}{0 0 0}
\pscustom[linestyle=none,fillstyle=solid,fillcolor=curcolor]
{
\newpath
\moveto(308.69375375,339.54389824)
\curveto(308.23584881,339.15327333)(307.79986485,338.84945396)(307.38580187,338.63244012)
\curveto(306.97661021,338.41542628)(306.54062625,338.30691937)(306.07784998,338.30691937)
\curveto(305.48841915,338.30691937)(304.94770161,338.45882905)(304.45569736,338.76264842)
\curveto(303.96369311,339.07514835)(303.54232313,339.54389824)(303.19158743,340.16889809)
\curveto(302.8359804,340.79389795)(302.5607503,341.58382832)(302.36589713,342.5386892)
\curveto(302.17104397,343.49355009)(302.07361738,344.60900122)(302.07361738,345.88504259)
\curveto(302.07361738,348.26351425)(302.43896707,350.12983325)(303.16966645,351.4839996)
\curveto(303.90523716,352.83816595)(304.87463167,353.51524913)(306.07784998,353.51524913)
\curveto(306.54549758,353.51524913)(307.00340252,353.39806165)(307.45156481,353.16368671)
\curveto(307.90459843,352.92931176)(308.31866141,352.6428535)(308.69375375,352.30431191)
\lineto(308.69375375,349.58295838)
\lineto(308.62068382,349.58295838)
\curveto(308.20174951,350.16455547)(307.76820121,350.61160397)(307.32003892,350.9241039)
\curveto(306.87674797,351.23660383)(306.44319967,351.39285379)(306.01939403,351.39285379)
\curveto(305.23998136,351.39285379)(304.62375821,350.9241039)(304.1707246,349.98660412)
\curveto(303.72256231,349.0577849)(303.49848117,347.69059772)(303.49848117,345.88504259)
\curveto(303.49848117,344.13157077)(303.71769098,342.7817447)(304.15611061,341.83556437)
\curveto(304.59940157,340.89806459)(305.22049604,340.4293147)(306.01939403,340.4293147)
\curveto(306.29705979,340.4293147)(306.57959688,340.49441885)(306.86700531,340.62462715)
\curveto(307.15441373,340.75483546)(307.41259418,340.92410625)(307.64154665,341.13243953)
\curveto(307.84127115,341.31473116)(308.02881732,341.50570334)(308.20418517,341.70535607)
\curveto(308.37955302,341.91368935)(308.5183859,342.0916407)(308.62068382,342.23921011)
\lineto(308.69375375,342.23921011)
\closepath
}
}
{
\newrgbcolor{curcolor}{0 0 0}
\pscustom[linestyle=none,fillstyle=solid,fillcolor=curcolor]
{
\newpath
\moveto(314.69279837,338.76264842)
\curveto(314.43461792,338.64112068)(314.15208083,338.54129431)(313.84518709,338.46316933)
\curveto(313.54316468,338.38504435)(313.27280591,338.34598186)(313.03411078,338.34598186)
\curveto(312.20111348,338.34598186)(311.56784069,338.74528732)(311.13429239,339.54389824)
\curveto(310.70074409,340.34250916)(310.48396994,341.62289081)(310.48396994,343.38504317)
\lineto(310.48396994,351.11941636)
\lineto(309.55598173,351.11941636)
\lineto(309.55598173,353.17670754)
\lineto(310.48396994,353.17670754)
\lineto(310.48396994,357.35639406)
\lineto(311.85768478,357.35639406)
\lineto(311.85768478,353.17670754)
\lineto(314.69279837,353.17670754)
\lineto(314.69279837,351.11941636)
\lineto(311.85768478,351.11941636)
\lineto(311.85768478,344.49181375)
\curveto(311.85768478,343.72792504)(311.86742743,343.12896684)(311.88691275,342.69493917)
\curveto(311.90639807,342.26959205)(311.97459668,341.87028658)(312.09150858,341.49702278)
\curveto(312.19867782,341.14980064)(312.34481769,340.89372431)(312.5299282,340.7287938)
\curveto(312.71991004,340.57254383)(313.00731847,340.49441885)(313.39215347,340.49441885)
\curveto(313.61623461,340.49441885)(313.85005842,340.55084245)(314.09362488,340.66368964)
\curveto(314.33719133,340.78521739)(314.51255919,340.88504376)(314.61972843,340.96316874)
\lineto(314.69279837,340.96316874)
\closepath
}
}
{
\newrgbcolor{curcolor}{0.80000001 0.80000001 0.80000001}
\pscustom[linestyle=none,fillstyle=solid,fillcolor=curcolor]
{
\newpath
\moveto(0.42503943,183.1639233)
\lineto(250.00445434,183.1639233)
\lineto(250.00445434,0.42505477)
\lineto(0.42503943,0.42505477)
\closepath
}
}
{
\newrgbcolor{curcolor}{0 0 0}
\pscustom[linewidth=0.85007621,linecolor=curcolor]
{
\newpath
\moveto(0.42503943,183.1639233)
\lineto(250.00445434,183.1639233)
\lineto(250.00445434,0.42505477)
\lineto(0.42503943,0.42505477)
\closepath
}
}
{
\newrgbcolor{curcolor}{0 0 0}
\pscustom[linestyle=none,fillstyle=solid,fillcolor=curcolor]
{
\newpath
\moveto(76.46007953,81.53327777)
\lineto(72.85401896,81.53327777)
\lineto(72.85401896,107.59987972)
\lineto(65.31407413,89.86321292)
\lineto(63.16500773,89.86321292)
\lineto(55.67970019,107.59987972)
\lineto(55.67970019,81.53327777)
\lineto(52.31040117,81.53327777)
\lineto(52.31040117,111.78516421)
\lineto(57.22775649,111.78516421)
\lineto(64.45809006,94.94244167)
\lineto(71.45166207,111.78516421)
\lineto(76.46007953,111.78516421)
\closepath
}
}
{
\newrgbcolor{curcolor}{0 0 0}
\pscustom[linestyle=none,fillstyle=solid,fillcolor=curcolor]
{
\newpath
\moveto(100.50048297,92.48409496)
\lineto(85.51165545,92.48409496)
\curveto(85.51165545,91.08900012)(85.69985053,89.86998522)(86.07624069,88.82705025)
\curveto(86.45263085,87.79765989)(86.96864962,86.95112177)(87.624297,86.28743588)
\curveto(88.25566114,85.6372946)(89.00237065,85.14968864)(89.86442553,84.824618)
\curveto(90.73862203,84.49954736)(91.69780986,84.33701204)(92.74198902,84.33701204)
\curveto(94.12613348,84.33701204)(95.51634875,84.64176576)(96.91263483,85.25127321)
\curveto(98.32106252,85.87432527)(99.32274601,86.48383272)(99.9176853,87.07979556)
\lineto(100.09980957,87.07979556)
\lineto(100.09980957,82.91482799)
\curveto(98.94635585,82.37304359)(97.7686189,81.91929915)(96.56659871,81.55359468)
\curveto(95.36457852,81.18789021)(94.10185024,81.00503798)(92.77841387,81.00503798)
\curveto(89.40304404,81.00503798)(86.76831292,82.02088373)(84.8742205,84.05257523)
\curveto(82.98012808,86.09781134)(82.03308187,88.99635788)(82.03308187,92.74821485)
\curveto(82.03308187,96.45943799)(82.93763242,99.40539067)(84.74673351,101.58607288)
\curveto(86.56797622,103.76675509)(88.95987498,104.85709619)(91.9224298,104.85709619)
\curveto(94.66643548,104.85709619)(96.77907703,103.96315193)(98.26035443,102.17526341)
\curveto(99.75377346,100.38737489)(100.50048297,97.84776052)(100.50048297,94.55642029)
\closepath
\moveto(97.1676088,95.40973072)
\curveto(97.15546719,97.414333)(96.70015651,98.96519084)(95.80167677,100.06230425)
\curveto(94.91533865,101.15941766)(93.56154824,101.70797437)(91.74030552,101.70797437)
\curveto(89.90692119,101.70797437)(88.44385622,101.10523922)(87.35111059,99.89976893)
\curveto(86.27050658,98.69429864)(85.65735487,97.19761924)(85.51165545,95.40973072)
\closepath
}
}
{
\newrgbcolor{curcolor}{0 0 0}
\pscustom[linestyle=none,fillstyle=solid,fillcolor=curcolor]
{
\newpath
\moveto(135.37728096,81.53327777)
\lineto(131.95334466,81.53327777)
\lineto(131.95334466,94.45483571)
\curveto(131.95334466,95.43004763)(131.910849,96.37139803)(131.82585767,97.2788869)
\curveto(131.75300797,98.18637577)(131.58909612,98.9110124)(131.33412214,99.4527968)
\curveto(131.05486493,100.03521503)(130.65419153,100.47541486)(130.13210195,100.77339628)
\curveto(129.61001237,101.0713777)(128.85723205,101.22036841)(127.87376099,101.22036841)
\curveto(126.91457316,101.22036841)(125.95538533,100.94947621)(124.9961975,100.40769181)
\curveto(124.03700968,99.87945202)(123.07782185,99.20222152)(122.11863402,98.37600031)
\curveto(122.15505887,98.06447428)(122.18541292,97.69876981)(122.20969616,97.2788869)
\curveto(122.23397939,96.8725486)(122.24612101,96.4662103)(122.24612101,96.059872)
\lineto(122.24612101,81.53327777)
\lineto(118.82218471,81.53327777)
\lineto(118.82218471,94.45483571)
\curveto(118.82218471,95.45713685)(118.77968905,96.40525955)(118.69469772,97.29920381)
\curveto(118.62184801,98.20669268)(118.45793617,98.93132932)(118.20296219,99.47311372)
\curveto(117.92370497,100.05553195)(117.52303158,100.48895947)(117.000942,100.77339628)
\curveto(116.47885242,101.0713777)(115.7260721,101.22036841)(114.74260104,101.22036841)
\curveto(113.80769645,101.22036841)(112.86672105,100.96302082)(111.91967484,100.44832564)
\curveto(110.98477024,99.93363046)(110.04986565,99.27671687)(109.11496106,98.47758488)
\lineto(109.11496106,81.53327777)
\lineto(105.69102476,81.53327777)
\lineto(105.69102476,104.22727183)
\lineto(109.11496106,104.22727183)
\lineto(109.11496106,101.70797437)
\curveto(110.18342345,102.6967309)(111.24581503,103.46877367)(112.30213581,104.02410268)
\curveto(113.3705982,104.57943169)(114.50583949,104.85709619)(115.70785968,104.85709619)
\curveto(117.09200414,104.85709619)(118.26367028,104.53202555)(119.22285811,103.88188427)
\curveto(120.19418756,103.23174299)(120.91661383,102.33102643)(121.39013694,101.17973458)
\curveto(122.7742814,102.48001714)(124.03700968,103.41459523)(125.17832178,103.98346885)
\curveto(126.31963388,104.56588708)(127.53986649,104.85709619)(128.83901963,104.85709619)
\curveto(131.07307735,104.85709619)(132.7182666,104.09859803)(133.77458737,102.58160171)
\curveto(134.84304977,101.07815)(135.37728096,98.97196315)(135.37728096,96.26304115)
\closepath
}
}
{
\newrgbcolor{curcolor}{0 0 0}
\pscustom[linestyle=none,fillstyle=solid,fillcolor=curcolor]
{
\newpath
\moveto(159.30840969,92.87011634)
\curveto(159.30840969,89.17243781)(158.45849642,86.25357435)(156.75866989,84.11352597)
\curveto(155.05884336,81.97347759)(152.78228997,80.9034534)(149.92900972,80.9034534)
\curveto(147.05144624,80.9034534)(144.76275123,81.97347759)(143.0629247,84.11352597)
\curveto(141.37523978,86.25357435)(140.53139733,89.17243781)(140.53139733,92.87011634)
\curveto(140.53139733,96.56779487)(141.37523978,99.48665833)(143.0629247,101.62670671)
\curveto(144.76275123,103.7802997)(147.05144624,104.85709619)(149.92900972,104.85709619)
\curveto(152.78228997,104.85709619)(155.05884336,103.7802997)(156.75866989,101.62670671)
\curveto(158.45849642,99.48665833)(159.30840969,96.56779487)(159.30840969,92.87011634)
\closepath
\moveto(155.77519882,92.87011634)
\curveto(155.77519882,95.80929671)(155.25918006,97.98997892)(154.22714252,99.41216297)
\curveto(153.19510498,100.84789163)(151.76239405,101.56575596)(149.92900972,101.56575596)
\curveto(148.07134215,101.56575596)(146.6264896,100.84789163)(145.59445207,99.41216297)
\curveto(144.57455615,97.98997892)(144.06460819,95.80929671)(144.06460819,92.87011634)
\curveto(144.06460819,90.02574824)(144.58062696,87.86538294)(145.61266449,86.38902045)
\curveto(146.64470203,84.92620257)(148.08348377,84.19479363)(149.92900972,84.19479363)
\curveto(151.75025243,84.19479363)(153.17689256,84.91943027)(154.20893009,86.36870354)
\curveto(155.25310925,87.83152142)(155.77519882,89.99865902)(155.77519882,92.87011634)
\closepath
}
}
{
\newrgbcolor{curcolor}{0 0 0}
\pscustom[linestyle=none,fillstyle=solid,fillcolor=curcolor]
{
\newpath
\moveto(177.30228509,100.06230425)
\lineto(177.12016082,100.06230425)
\curveto(176.61021286,100.19775035)(176.11240651,100.29256262)(175.62674179,100.34674106)
\curveto(175.15321869,100.41446411)(174.58863345,100.44832564)(173.93298607,100.44832564)
\curveto(172.8766653,100.44832564)(171.85676938,100.18420574)(170.87329831,99.65596595)
\curveto(169.88982725,99.14127077)(168.94278104,98.47081258)(168.03215968,97.64459137)
\lineto(168.03215968,81.53327777)
\lineto(164.60822339,81.53327777)
\lineto(164.60822339,104.22727183)
\lineto(168.03215968,104.22727183)
\lineto(168.03215968,100.87498085)
\curveto(169.39202091,102.09399575)(170.58797029,102.95407849)(171.62000783,103.45522906)
\curveto(172.66418698,103.96992424)(173.72657856,104.22727183)(174.80718257,104.22727183)
\curveto(175.40212186,104.22727183)(175.8331493,104.20695491)(176.1002649,104.16632108)
\curveto(176.36738049,104.13923186)(176.76805389,104.07828112)(177.30228509,103.98346885)
\closepath
}
}
{
\newrgbcolor{curcolor}{0 0 0}
\pscustom[linestyle=none,fillstyle=solid,fillcolor=curcolor]
{
\newpath
\moveto(198.11909755,104.22727183)
\lineto(186.24459507,73.16270879)
\lineto(182.58389722,73.16270879)
\lineto(186.37208206,82.63039118)
\lineto(178.267552,104.22727183)
\lineto(181.98288713,104.22727183)
\lineto(188.22974963,87.4048662)
\lineto(194.53124941,104.22727183)
\closepath
}
}
{
\newrgbcolor{curcolor}{0.80000001 0.80000001 0.80000001}
\pscustom[linestyle=none,fillstyle=solid,fillcolor=curcolor]
{
\newpath
\moveto(285.29630331,183.1639233)
\lineto(534.87571821,183.1639233)
\lineto(534.87571821,0.42505477)
\lineto(285.29630331,0.42505477)
\closepath
}
}
{
\newrgbcolor{curcolor}{0 0 0}
\pscustom[linewidth=0.85007621,linecolor=curcolor]
{
\newpath
\moveto(285.29630331,183.1639233)
\lineto(534.87571821,183.1639233)
\lineto(534.87571821,0.42505477)
\lineto(285.29630331,0.42505477)
\closepath
}
}
{
\newrgbcolor{curcolor}{0 0 0}
\pscustom[linestyle=none,fillstyle=solid,fillcolor=curcolor]
{
\newpath
\moveto(361.33132006,80.85380488)
\lineto(357.72525949,80.85380488)
\lineto(357.72525949,106.92040683)
\lineto(350.18531466,89.18374003)
\lineto(348.03624826,89.18374003)
\lineto(340.55094072,106.92040683)
\lineto(340.55094072,80.85380488)
\lineto(337.1816417,80.85380488)
\lineto(337.1816417,111.10569132)
\lineto(342.09899702,111.10569132)
\lineto(349.32933059,94.26296878)
\lineto(356.3229026,111.10569132)
\lineto(361.33132006,111.10569132)
\closepath
}
}
{
\newrgbcolor{curcolor}{0 0 0}
\pscustom[linestyle=none,fillstyle=solid,fillcolor=curcolor]
{
\newpath
\moveto(385.3717235,91.80462207)
\lineto(370.38289598,91.80462207)
\curveto(370.38289598,90.40952724)(370.57109106,89.19051234)(370.94748122,88.14757737)
\curveto(371.32387138,87.11818701)(371.83989015,86.27164888)(372.49553753,85.60796299)
\curveto(373.12690167,84.95782171)(373.87361118,84.47021575)(374.73566606,84.14514511)
\curveto(375.60986256,83.82007447)(376.56905039,83.65753915)(377.61322955,83.65753915)
\curveto(378.99737401,83.65753915)(380.38758928,83.96229288)(381.78387536,84.57180033)
\curveto(383.19230305,85.19485239)(384.19398655,85.80435984)(384.78892583,86.40032268)
\lineto(384.9710501,86.40032268)
\lineto(384.9710501,82.2353551)
\curveto(383.81759638,81.6935707)(382.63985943,81.23982627)(381.43783924,80.8741218)
\curveto(380.23581905,80.50841733)(378.97309077,80.32556509)(377.6496544,80.32556509)
\curveto(374.27428458,80.32556509)(371.63955345,81.34141084)(369.74546103,83.37310234)
\curveto(367.85136861,85.41833845)(366.9043224,88.31688499)(366.9043224,92.06874196)
\curveto(366.9043224,95.7799651)(367.80887295,98.72591778)(369.61797404,100.90659999)
\curveto(371.43921676,103.0872822)(373.83111552,104.17762331)(376.79367033,104.17762331)
\curveto(379.53767601,104.17762331)(381.65031756,103.28367905)(383.13159496,101.49579053)
\curveto(384.62501399,99.70790201)(385.3717235,97.16828763)(385.3717235,93.8769474)
\closepath
\moveto(382.03884934,94.73025783)
\curveto(382.02670772,96.73486011)(381.57139704,98.28571796)(380.6729173,99.38283137)
\curveto(379.78657918,100.47994478)(378.43278877,101.02850148)(376.61154606,101.02850148)
\curveto(374.77816173,101.02850148)(373.31509675,100.42576634)(372.22235112,99.22029605)
\curveto(371.14174711,98.01482576)(370.5285954,96.51814635)(370.38289598,94.73025783)
\closepath
}
}
{
\newrgbcolor{curcolor}{0 0 0}
\pscustom[linestyle=none,fillstyle=solid,fillcolor=curcolor]
{
\newpath
\moveto(420.24852149,80.85380488)
\lineto(416.82458519,80.85380488)
\lineto(416.82458519,93.77536282)
\curveto(416.82458519,94.75057474)(416.78208953,95.69192514)(416.69709821,96.59941401)
\curveto(416.6242485,97.50690288)(416.46033665,98.23153952)(416.20536267,98.77332392)
\curveto(415.92610546,99.35574215)(415.52543206,99.79594197)(415.00334248,100.09392339)
\curveto(414.48125291,100.39190481)(413.72847259,100.54089552)(412.74500152,100.54089552)
\curveto(411.78581369,100.54089552)(410.82662586,100.27000332)(409.86743804,99.72821892)
\curveto(408.90825021,99.19997913)(407.94906238,98.52274863)(406.98987455,97.69652742)
\curveto(407.02629941,97.38500139)(407.05665345,97.01929692)(407.08093669,96.59941401)
\curveto(407.10521992,96.19307571)(407.11736154,95.78673741)(407.11736154,95.38039911)
\lineto(407.11736154,80.85380488)
\lineto(403.69342524,80.85380488)
\lineto(403.69342524,93.77536282)
\curveto(403.69342524,94.77766396)(403.65092958,95.72578666)(403.56593825,96.61973092)
\curveto(403.49308855,97.5272198)(403.3291767,98.25185643)(403.07420272,98.79364083)
\curveto(402.79494551,99.37605906)(402.39427211,99.80948658)(401.87218253,100.09392339)
\curveto(401.35009296,100.39190481)(400.59731263,100.54089552)(399.61384157,100.54089552)
\curveto(398.67893698,100.54089552)(397.73796158,100.28354793)(396.79091537,99.76885275)
\curveto(395.85601078,99.25415757)(394.92110618,98.59724399)(393.98620159,97.798112)
\lineto(393.98620159,80.85380488)
\lineto(390.56226529,80.85380488)
\lineto(390.56226529,103.54779894)
\lineto(393.98620159,103.54779894)
\lineto(393.98620159,101.02850148)
\curveto(395.05466398,102.01725801)(396.11705556,102.78930078)(397.17337634,103.34462979)
\curveto(398.24183873,103.8999588)(399.37708002,104.17762331)(400.57910021,104.17762331)
\curveto(401.96324467,104.17762331)(403.13491081,103.85255267)(404.09409864,103.20241139)
\curveto(405.06542809,102.55227011)(405.78785436,101.65155354)(406.26137747,100.50026169)
\curveto(407.64552193,101.80054425)(408.90825021,102.73512234)(410.04956231,103.30399596)
\curveto(411.19087441,103.88641419)(412.41110702,104.17762331)(413.71026016,104.17762331)
\curveto(415.94431788,104.17762331)(417.58950713,103.41912515)(418.64582791,101.90212883)
\curveto(419.7142903,100.39867712)(420.24852149,98.29249026)(420.24852149,95.58356826)
\closepath
}
}
{
\newrgbcolor{curcolor}{0 0 0}
\pscustom[linestyle=none,fillstyle=solid,fillcolor=curcolor]
{
\newpath
\moveto(444.17965022,92.19064345)
\curveto(444.17965022,88.49296492)(443.32973695,85.57410147)(441.62991042,83.43405309)
\curveto(439.93008389,81.29400471)(437.6535305,80.22398052)(434.80025025,80.22398052)
\curveto(431.92268677,80.22398052)(429.63399176,81.29400471)(427.93416523,83.43405309)
\curveto(426.24648032,85.57410147)(425.40263786,88.49296492)(425.40263786,92.19064345)
\curveto(425.40263786,95.88832198)(426.24648032,98.80718544)(427.93416523,100.94723382)
\curveto(429.63399176,103.10082681)(431.92268677,104.17762331)(434.80025025,104.17762331)
\curveto(437.6535305,104.17762331)(439.93008389,103.10082681)(441.62991042,100.94723382)
\curveto(443.32973695,98.80718544)(444.17965022,95.88832198)(444.17965022,92.19064345)
\closepath
\moveto(440.64643936,92.19064345)
\curveto(440.64643936,95.12982382)(440.13042059,97.31050604)(439.09838305,98.73269009)
\curveto(438.06634551,100.16841875)(436.63363458,100.88628308)(434.80025025,100.88628308)
\curveto(432.94258269,100.88628308)(431.49773013,100.16841875)(430.4656926,98.73269009)
\curveto(429.44579668,97.31050604)(428.93584872,95.12982382)(428.93584872,92.19064345)
\curveto(428.93584872,89.34627535)(429.45186749,87.18591006)(430.48390503,85.70954757)
\curveto(431.51594256,84.24672969)(432.9547243,83.51532075)(434.80025025,83.51532075)
\curveto(436.62149296,83.51532075)(438.04813309,84.23995738)(439.08017062,85.68923065)
\curveto(440.12434978,87.15204853)(440.64643936,89.31918613)(440.64643936,92.19064345)
\closepath
}
}
{
\newrgbcolor{curcolor}{0 0 0}
\pscustom[linestyle=none,fillstyle=solid,fillcolor=curcolor]
{
\newpath
\moveto(462.17352562,99.38283137)
\lineto(461.99140135,99.38283137)
\curveto(461.48145339,99.51827747)(460.98364705,99.61308974)(460.49798232,99.66726818)
\curveto(460.02445922,99.73499123)(459.45987398,99.76885275)(458.8042266,99.76885275)
\curveto(457.74790583,99.76885275)(456.72800991,99.50473286)(455.74453885,98.97649307)
\curveto(454.76106778,98.46179789)(453.81402157,97.79133969)(452.90340022,96.96511848)
\lineto(452.90340022,80.85380488)
\lineto(449.47946392,80.85380488)
\lineto(449.47946392,103.54779894)
\lineto(452.90340022,103.54779894)
\lineto(452.90340022,100.19550797)
\curveto(454.26326144,101.41452287)(455.45921082,102.2746056)(456.49124836,102.77575617)
\curveto(457.53542751,103.29045135)(458.59781909,103.54779894)(459.6784231,103.54779894)
\curveto(460.27336239,103.54779894)(460.70438983,103.52748203)(460.97150543,103.4868482)
\curveto(461.23862103,103.45975898)(461.63929442,103.39880823)(462.17352562,103.30399596)
\closepath
}
}
{
\newrgbcolor{curcolor}{0 0 0}
\pscustom[linestyle=none,fillstyle=solid,fillcolor=curcolor]
{
\newpath
\moveto(482.99033808,103.54779894)
\lineto(471.1158356,72.4832359)
\lineto(467.45513775,72.4832359)
\lineto(471.24332259,81.95091829)
\lineto(463.13879253,103.54779894)
\lineto(466.85412766,103.54779894)
\lineto(473.10099016,86.72539332)
\lineto(479.40248994,103.54779894)
\closepath
}
}
{
\newrgbcolor{curcolor}{0 1 0}
\pscustom[linestyle=none,fillstyle=solid,fillcolor=curcolor]
{
\newpath
\moveto(335.02495007,315.09235633)
\lineto(485.14708286,315.09235633)
\lineto(485.14708286,269.97021272)
\lineto(335.02495007,269.97021272)
\closepath
}
}
{
\newrgbcolor{curcolor}{0.7019608 0.7019608 0.7019608}
\pscustom[linewidth=0.72824692,linecolor=curcolor]
{
\newpath
\moveto(335.02495007,315.09235633)
\lineto(485.14708286,315.09235633)
\lineto(485.14708286,269.97021272)
\lineto(335.02495007,269.97021272)
\closepath
}
}
{
\newrgbcolor{curcolor}{0 0 0}
\pscustom[linestyle=none,fillstyle=solid,fillcolor=curcolor]
{
\newpath
\moveto(356.97084153,301.46891206)
\curveto(356.97084153,300.30313541)(356.77606639,299.21978742)(356.38651611,298.21886808)
\curveto(356.00809583,297.22972426)(355.47385545,296.37011118)(354.78379495,295.64002884)
\curveto(353.92678433,294.73331367)(352.9139536,294.05033342)(351.74530275,293.59108807)
\curveto(350.57665191,293.14361825)(349.10192584,292.91988334)(347.32112455,292.91988334)
\lineto(344.01551217,292.91988334)
\lineto(344.01551217,283.11676156)
\lineto(340.70989978,283.11676156)
\lineto(340.70989978,309.41738918)
\lineto(347.45468465,309.41738918)
\curveto(348.94610573,309.41738918)(350.20936164,309.28197068)(351.24445239,309.01113368)
\curveto(352.27954314,308.7520722)(353.1977688,308.33992895)(353.99912938,307.77470391)
\curveto(354.94518006,307.10349917)(355.67419559,306.26743713)(356.18617596,305.26651779)
\curveto(356.70928634,304.26559845)(356.97084153,302.99972987)(356.97084153,301.46891206)
\closepath
\moveto(353.53166904,301.38059565)
\curveto(353.53166904,302.28731081)(353.38141393,303.07627076)(353.08090372,303.7474755)
\curveto(352.7803935,304.41868023)(352.32406317,304.96624199)(351.71191273,305.39016077)
\curveto(351.17767234,305.75520194)(350.5655219,306.01426342)(349.8754614,306.1673452)
\curveto(349.19653091,306.3322025)(348.33395529,306.41463115)(347.28773453,306.41463115)
\lineto(344.01551217,306.41463115)
\lineto(344.01551217,295.90497808)
\lineto(346.80357918,295.90497808)
\curveto(348.13918015,295.90497808)(349.22435593,296.02862105)(350.05910653,296.27590701)
\curveto(350.89385714,296.53496848)(351.57278763,296.94122398)(352.09589801,297.4946735)
\curveto(352.61900838,298.05989854)(352.98629865,298.65456238)(353.1977688,299.27866503)
\curveto(353.42036896,299.90276768)(353.53166904,300.60341122)(353.53166904,301.38059565)
\closepath
}
}
{
\newrgbcolor{curcolor}{0 0 0}
\pscustom[linestyle=none,fillstyle=solid,fillcolor=curcolor]
{
\newpath
\moveto(372.71423741,299.22567518)
\lineto(372.54728729,299.22567518)
\curveto(372.07982695,299.3434304)(371.62349662,299.42585905)(371.1782963,299.47296114)
\curveto(370.74422599,299.53183875)(370.22668061,299.56127755)(369.62566018,299.56127755)
\curveto(368.65734948,299.56127755)(367.7224288,299.33165488)(366.82089815,298.87240953)
\curveto(365.9193675,298.42493971)(365.05122687,297.84205139)(364.21647627,297.12374457)
\lineto(364.21647627,283.11676156)
\lineto(361.077814,283.11676156)
\lineto(361.077814,302.84664809)
\lineto(364.21647627,302.84664809)
\lineto(364.21647627,299.93220648)
\curveto(365.46303717,300.99200343)(366.55934296,301.73974906)(367.50539365,302.17544336)
\curveto(368.46257434,302.62291318)(369.43645004,302.84664809)(370.42702076,302.84664809)
\curveto(370.97239115,302.84664809)(371.36750644,302.82898481)(371.61236662,302.79365825)
\curveto(371.85722679,302.7701072)(372.22451706,302.71711735)(372.71423741,302.6346887)
\closepath
}
}
{
\newrgbcolor{curcolor}{0 0 0}
\pscustom[linestyle=none,fillstyle=solid,fillcolor=curcolor]
{
\newpath
\moveto(391.56290584,292.97287318)
\curveto(391.56290584,289.75815577)(390.78380528,287.22053085)(389.22560416,285.35999843)
\curveto(387.66740303,283.49946601)(385.58052652,282.5691998)(382.96497463,282.5691998)
\curveto(380.32716273,282.5691998)(378.22915621,283.49946601)(376.67095508,285.35999843)
\curveto(375.12388396,287.22053085)(374.35034841,289.75815577)(374.35034841,292.97287318)
\curveto(374.35034841,296.1875906)(375.12388396,298.72521551)(376.67095508,300.58574793)
\curveto(378.22915621,302.45805588)(380.32716273,303.39420985)(382.96497463,303.39420985)
\curveto(385.58052652,303.39420985)(387.66740303,302.45805588)(389.22560416,300.58574793)
\curveto(390.78380528,298.72521551)(391.56290584,296.1875906)(391.56290584,292.97287318)
\closepath
\moveto(388.3240735,292.97287318)
\curveto(388.3240735,295.52816138)(387.85104816,297.42402037)(386.90499748,298.66045014)
\curveto(385.9589468,299.90865544)(384.64560585,300.53275809)(382.96497463,300.53275809)
\curveto(381.2620834,300.53275809)(379.93761244,299.90865544)(378.99156176,298.66045014)
\curveto(378.05664108,297.42402037)(377.58918075,295.52816138)(377.58918075,292.97287318)
\curveto(377.58918075,290.50001364)(378.06220609,288.62181793)(379.00825677,287.33828607)
\curveto(379.95430746,286.06652973)(381.27321341,285.43065156)(382.96497463,285.43065156)
\curveto(384.63447584,285.43065156)(385.94225178,286.06064197)(386.88830247,287.32062279)
\curveto(387.84548316,288.59237913)(388.3240735,290.47646259)(388.3240735,292.97287318)
\closepath
}
}
{
\newrgbcolor{curcolor}{0 0 0}
\pscustom[linestyle=none,fillstyle=solid,fillcolor=curcolor]
{
\newpath
\moveto(411.54683574,285.35999843)
\curveto(411.54683574,282.01575028)(410.82895022,279.56055402)(409.39317918,277.99440964)
\curveto(407.95740814,276.42826526)(405.74810155,275.64519307)(402.76525939,275.64519307)
\curveto(401.77468867,275.64519307)(400.80637798,275.72173396)(399.86032729,275.87481574)
\curveto(398.92540662,276.016122)(398.00161595,276.22219363)(397.08895529,276.49303063)
\lineto(397.08895529,279.88438086)
\lineto(397.25590541,279.88438086)
\curveto(397.76788578,279.67242147)(398.58037637,279.41336)(399.69337717,279.10719643)
\curveto(400.80637798,278.78925735)(401.91937878,278.63028781)(403.03237958,278.63028781)
\curveto(404.10086036,278.63028781)(404.985696,278.7657063)(405.6868865,279.0365433)
\curveto(406.38807701,279.3073803)(406.9334474,279.68419699)(407.32299768,280.16699338)
\curveto(407.71254797,280.62623873)(407.99079817,281.17968824)(408.15774829,281.82734194)
\curveto(408.32469841,282.47499563)(408.40817347,283.19919021)(408.40817347,283.99992568)
\lineto(408.40817347,285.80158049)
\curveto(407.46212279,285.00084502)(406.55502713,284.40029342)(405.6868865,283.99992568)
\curveto(404.82987588,283.61133347)(403.73357009,283.41703736)(402.39796913,283.41703736)
\curveto(400.17196752,283.41703736)(398.40229624,284.26487492)(397.08895529,285.96055004)
\curveto(395.78674435,287.66800068)(395.13563888,290.0702071)(395.13563888,293.16716929)
\curveto(395.13563888,294.86284441)(395.35823904,296.32300909)(395.80343936,297.54766335)
\curveto(396.25976969,298.78409312)(396.87748514,299.84977783)(397.6565857,300.74471748)
\curveto(398.38003622,301.58077951)(399.25930686,302.22843321)(400.29439761,302.68767855)
\curveto(401.32948835,303.15869942)(402.3590141,303.39420985)(403.38297484,303.39420985)
\curveto(404.46258562,303.39420985)(405.36411627,303.27645463)(406.08756679,303.0409442)
\curveto(406.82214732,302.81720929)(407.59568288,302.4698314)(408.40817347,301.99881053)
\lineto(408.60851361,302.84664809)
\lineto(411.54683574,302.84664809)
\closepath
\moveto(408.40817347,288.53938928)
\lineto(408.40817347,299.29632831)
\curveto(407.57342287,299.69669605)(406.7943223,299.97930857)(406.07087178,300.14416587)
\curveto(405.35855126,300.3207987)(404.64623075,300.40911511)(403.93391024,300.40911511)
\curveto(402.20875899,300.40911511)(400.85089801,299.79678798)(399.86032729,298.57213373)
\curveto(398.86975658,297.34747948)(398.37447122,295.56937571)(398.37447122,293.23782242)
\curveto(398.37447122,291.02402435)(398.74176148,289.34601251)(399.47634201,288.20378691)
\curveto(400.21092254,287.06156131)(401.42965843,286.49044851)(403.13254966,286.49044851)
\curveto(404.04521032,286.49044851)(404.95787098,286.6729691)(405.87053163,287.03801027)
\curveto(406.7943223,287.41482696)(407.64020291,287.91528663)(408.40817347,288.53938928)
\closepath
}
}
{
\newrgbcolor{curcolor}{0 0 0}
\pscustom[linestyle=none,fillstyle=solid,fillcolor=curcolor]
{
\newpath
\moveto(429.36041007,299.22567518)
\lineto(429.19345995,299.22567518)
\curveto(428.72599961,299.3434304)(428.26966928,299.42585905)(427.82446896,299.47296114)
\curveto(427.39039865,299.53183875)(426.87285327,299.56127755)(426.27183284,299.56127755)
\curveto(425.30352214,299.56127755)(424.36860146,299.33165488)(423.46707081,298.87240953)
\curveto(422.56554016,298.42493971)(421.69739953,297.84205139)(420.86264893,297.12374457)
\lineto(420.86264893,283.11676156)
\lineto(417.72398666,283.11676156)
\lineto(417.72398666,302.84664809)
\lineto(420.86264893,302.84664809)
\lineto(420.86264893,299.93220648)
\curveto(422.10920983,300.99200343)(423.20551562,301.73974906)(424.15156631,302.17544336)
\curveto(425.108747,302.62291318)(426.0826227,302.84664809)(427.07319342,302.84664809)
\curveto(427.61856381,302.84664809)(428.0136791,302.82898481)(428.25853928,302.79365825)
\curveto(428.50339945,302.7701072)(428.87068972,302.71711735)(429.36041007,302.6346887)
\closepath
}
}
{
\newrgbcolor{curcolor}{0 0 0}
\pscustom[linestyle=none,fillstyle=solid,fillcolor=curcolor]
{
\newpath
\moveto(446.20567982,283.11676156)
\lineto(443.08371256,283.11676156)
\lineto(443.08371256,285.21869217)
\curveto(442.80546236,285.0185083)(442.42704209,284.73589578)(441.94845174,284.37085461)
\curveto(441.4809914,284.01758896)(441.02466107,283.73497644)(440.57946075,283.52301705)
\curveto(440.05635037,283.25218006)(439.45532994,283.02844514)(438.77639945,282.85181232)
\curveto(438.09746896,282.66340397)(437.30167338,282.5691998)(436.38901272,282.5691998)
\curveto(434.70838151,282.5691998)(433.28374048,283.15797588)(432.11508963,284.33552805)
\curveto(430.94643879,285.51308021)(430.36211337,287.01445923)(430.36211337,288.83966508)
\curveto(430.36211337,290.33515633)(430.66262358,291.5421473)(431.26364402,292.46063799)
\curveto(431.87579446,293.3909042)(432.74393509,294.12098655)(433.8680659,294.65088502)
\curveto(435.00332672,295.18078349)(436.36675271,295.5399369)(437.95834386,295.72834525)
\curveto(439.54993501,295.9167536)(441.25839124,296.05805986)(443.08371256,296.15226403)
\lineto(443.08371256,296.66449922)
\curveto(443.08371256,297.41813261)(442.95571747,298.04223526)(442.69972728,298.53680717)
\curveto(442.45486711,299.03137908)(442.09870685,299.41997129)(441.63124651,299.70258381)
\curveto(441.18604619,299.97342081)(440.6518058,300.15594139)(440.02852535,300.25014557)
\curveto(439.4052449,300.34434974)(438.75413943,300.39145183)(438.07520894,300.39145183)
\curveto(437.25158834,300.39145183)(436.33336268,300.27369661)(435.32053195,300.03818618)
\curveto(434.30770122,299.81445127)(433.26148046,299.48473666)(432.18186968,299.04904236)
\lineto(432.01491956,299.04904236)
\lineto(432.01491956,302.42272931)
\curveto(432.62707,302.59936214)(433.51190564,302.79365825)(434.66942648,303.00561764)
\curveto(435.82694732,303.21757702)(436.96777314,303.32355672)(438.09190395,303.32355672)
\curveto(439.4052449,303.32355672)(440.54607073,303.2058015)(441.51438143,302.97029107)
\curveto(442.49382213,302.74655616)(443.33970274,302.35796394)(444.05202326,301.80451443)
\curveto(444.75321377,301.26284043)(445.28745415,300.56219689)(445.65474442,299.70258381)
\curveto(446.02203468,298.84297073)(446.20567982,297.77728602)(446.20567982,296.50552968)
\closepath
\moveto(443.08371256,287.97416424)
\lineto(443.08371256,293.46744509)
\curveto(442.12653187,293.40856748)(440.99683605,293.32025107)(439.69462511,293.20249586)
\curveto(438.40354418,293.08474064)(437.37958344,292.91399558)(436.62274289,292.69026066)
\curveto(435.72121224,292.41942367)(434.99219671,291.99550489)(434.43569631,291.41850432)
\curveto(433.87919591,290.85327929)(433.60094571,290.0702071)(433.60094571,289.06928775)
\curveto(433.60094571,287.93883768)(433.92371594,287.08511236)(434.56925641,286.50811179)
\curveto(435.21479687,285.94288675)(436.19980258,285.66027423)(437.52427354,285.66027423)
\curveto(438.62614434,285.66027423)(439.63341007,285.88400915)(440.54607073,286.33147897)
\curveto(441.45873139,286.79072431)(442.304612,287.33828607)(443.08371256,287.97416424)
\closepath
}
}
{
\newrgbcolor{curcolor}{0 0 0}
\pscustom[linestyle=none,fillstyle=solid,fillcolor=curcolor]
{
\newpath
\moveto(479.46214364,283.11676156)
\lineto(476.32348137,283.11676156)
\lineto(476.32348137,294.35060922)
\curveto(476.32348137,295.19844678)(476.28452634,296.01684553)(476.20661629,296.80580548)
\curveto(476.13983624,297.59476543)(475.98958113,298.22475584)(475.75585096,298.69577671)
\curveto(475.49986078,299.20212414)(475.13257051,299.58482859)(474.65398017,299.84389007)
\curveto(474.17538982,300.10295155)(473.48532932,300.23248228)(472.58379867,300.23248228)
\curveto(471.70452803,300.23248228)(470.8252574,299.99697185)(469.94598676,299.52595099)
\curveto(469.06671613,299.06670564)(468.18744549,298.47792956)(467.30817486,297.75962274)
\curveto(467.34156488,297.48878574)(467.3693899,297.17084665)(467.39164992,296.80580548)
\curveto(467.41390993,296.45253983)(467.42503994,296.09927418)(467.42503994,295.74600853)
\lineto(467.42503994,283.11676156)
\lineto(464.28637767,283.11676156)
\lineto(464.28637767,294.35060922)
\curveto(464.28637767,295.22199782)(464.24742265,296.04628434)(464.16951259,296.82346877)
\curveto(464.10273254,297.61242872)(463.95247743,298.24241912)(463.71874726,298.71343999)
\curveto(463.46275708,299.21978742)(463.09546681,299.59660412)(462.61687647,299.84389007)
\curveto(462.13828612,300.10295155)(461.44822562,300.23248228)(460.54669497,300.23248228)
\curveto(459.68968435,300.23248228)(458.82710873,300.00874737)(457.9589681,299.56127755)
\curveto(457.10195748,299.11380773)(456.24494686,298.54269493)(455.38793624,297.84793915)
\lineto(455.38793624,283.11676156)
\lineto(452.24927397,283.11676156)
\lineto(452.24927397,302.84664809)
\lineto(455.38793624,302.84664809)
\lineto(455.38793624,300.65640106)
\curveto(456.36737695,301.51601415)(457.34125265,302.18721888)(458.30956335,302.67001527)
\curveto(459.28900406,303.15281166)(460.32965981,303.39420985)(461.43153061,303.39420985)
\curveto(462.70035153,303.39420985)(463.7743973,303.11159733)(464.65366794,302.54637229)
\curveto(465.54406858,301.98114725)(466.20630406,301.19807506)(466.64037438,300.19715572)
\curveto(467.90919529,301.3276058)(469.06671613,302.14011679)(470.11293688,302.6346887)
\curveto(471.15915764,303.14103613)(472.27772345,303.39420985)(473.46863431,303.39420985)
\curveto(475.51655579,303.39420985)(477.02467188,302.73478064)(477.99298258,301.41592221)
\curveto(478.97242329,300.10883931)(479.46214364,298.27774569)(479.46214364,295.92264136)
\closepath
}
}
{
\newrgbcolor{curcolor}{1 0.40000001 0}
\pscustom[linestyle=none,fillstyle=solid,fillcolor=curcolor]
{
\newpath
\moveto(52.3575999,178.83888562)
\lineto(198.07187944,178.83888562)
\lineto(198.07187944,124.55317477)
\lineto(52.3575999,124.55317477)
\closepath
}
}
{
\newrgbcolor{curcolor}{0.7019608 0.7019608 0.7019608}
\pscustom[linewidth=0.85039368,linecolor=curcolor]
{
\newpath
\moveto(52.3575999,178.83888562)
\lineto(198.07187944,178.83888562)
\lineto(198.07187944,124.55317477)
\lineto(52.3575999,124.55317477)
\closepath
}
}
{
\newrgbcolor{curcolor}{0 0 0}
\pscustom[linestyle=none,fillstyle=solid,fillcolor=curcolor]
{
\newpath
\moveto(106.15231,151.96947195)
\curveto(106.15231,149.3262512)(105.57288476,146.9304255)(104.41403429,144.78199484)
\curveto(103.2682046,142.63356418)(101.73826156,140.96690282)(99.82420515,139.78201076)
\curveto(98.49608438,138.96170087)(97.01171411,138.36925484)(95.37109433,138.00467267)
\curveto(93.74349535,137.6400905)(91.59506469,137.45779941)(88.92580235,137.45779941)
\lineto(81.58207573,137.45779941)
\lineto(81.58207573,166.53973806)
\lineto(88.8476776,166.53973806)
\curveto(91.68621023,166.53973806)(93.93880722,166.33140539)(95.60546858,165.91474005)
\curveto(97.28515074,165.51109551)(98.70441705,164.95120146)(99.86326753,164.2350579)
\curveto(101.84242789,162.99808267)(103.38539173,161.3509525)(104.49215904,159.29366738)
\curveto(105.59892635,157.23638227)(106.15231,154.79498379)(106.15231,151.96947195)
\closepath
\moveto(102.10935413,152.02806552)
\curveto(102.10935413,154.3067041)(101.71221997,156.2272709)(100.91795167,157.78976592)
\curveto(100.12368336,159.35226095)(98.9387913,160.58272578)(97.36327549,161.48116042)
\curveto(96.2174458,162.13220001)(95.00000176,162.58141733)(93.71094337,162.82881238)
\curveto(92.42188497,163.08922822)(90.87892113,163.21943614)(89.08205185,163.21943614)
\lineto(85.44925092,163.21943614)
\lineto(85.44925092,140.77810134)
\lineto(89.08205185,140.77810134)
\curveto(90.94402509,140.77810134)(92.56511368,140.91481965)(93.94531762,141.18825628)
\curveto(95.33854235,141.46169291)(96.61457995,141.96950379)(97.77343043,142.71168893)
\curveto(99.21873833,143.63616515)(100.29946406,144.85360919)(101.01560761,146.36402105)
\curveto(101.74477195,147.87443291)(102.10935413,149.76244773)(102.10935413,152.02806552)
\closepath
}
}
{
\newrgbcolor{curcolor}{0 0 0}
\pscustom[linestyle=none,fillstyle=solid,fillcolor=curcolor]
{
\newpath
\moveto(129.06239412,137.45779941)
\lineto(125.410062,137.45779941)
\lineto(125.410062,139.78201076)
\curveto(125.0845422,139.5606573)(124.64183528,139.24815829)(124.08194123,138.84451374)
\curveto(123.53506797,138.45388999)(123.0012155,138.14139098)(122.48038383,137.90701673)
\curveto(121.86840661,137.60753852)(121.16528385,137.36014347)(120.37101554,137.16483159)
\curveto(119.57674724,136.95649892)(118.64576062,136.85233259)(117.57805568,136.85233259)
\curveto(115.61191611,136.85233259)(113.94525475,137.50337218)(112.5780716,138.80545137)
\curveto(111.21088846,140.10753056)(110.52729688,141.76768152)(110.52729688,143.78590426)
\curveto(110.52729688,145.43954483)(110.87885826,146.774176)(111.58198103,147.78979776)
\curveto(112.29812458,148.81844032)(113.31374634,149.62572942)(114.62884632,150.21166505)
\curveto(115.9569671,150.79760069)(117.5520141,151.19473484)(119.41398734,151.40306751)
\curveto(121.27596058,151.61140018)(123.27465213,151.76764968)(125.410062,151.87181602)
\lineto(125.410062,152.43822046)
\curveto(125.410062,153.27155114)(125.26032289,153.96165311)(124.96084468,154.50852637)
\curveto(124.67438726,155.05539963)(124.25772192,155.48508576)(123.71084866,155.79758477)
\curveto(123.19001698,156.09706298)(122.56501897,156.29888525)(121.83585463,156.40305159)
\curveto(121.10669028,156.50721792)(120.34497396,156.55930109)(119.55070565,156.55930109)
\curveto(118.58716705,156.55930109)(117.51295172,156.42909317)(116.32805966,156.16867733)
\curveto(115.1431676,155.92128229)(113.91921317,155.55670012)(112.65619636,155.07493082)
\lineto(112.46088448,155.07493082)
\lineto(112.46088448,158.80538769)
\curveto(113.17702803,159.00069957)(114.21218098,159.21554263)(115.56634334,159.44991689)
\curveto(116.92050569,159.68429114)(118.25513686,159.80147827)(119.57023684,159.80147827)
\curveto(121.10669028,159.80147827)(122.44132145,159.67127035)(123.57413034,159.41085451)
\curveto(124.71996003,159.16345947)(125.70954021,158.73377333)(126.54287089,158.12179612)
\curveto(127.36318078,157.52283969)(127.98817879,156.74810257)(128.41786492,155.79758477)
\curveto(128.84755105,154.84706696)(129.06239412,153.66868529)(129.06239412,152.26243977)
\closepath
\moveto(125.410062,142.82887606)
\lineto(125.410062,148.90307547)
\curveto(124.2902739,148.83797151)(122.96866352,148.74031557)(121.44523087,148.61010765)
\curveto(119.93481901,148.47989973)(118.73690616,148.29109825)(117.85149231,148.0437032)
\curveto(116.79680817,147.74422499)(115.9439463,147.27547648)(115.29290671,146.63745768)
\curveto(114.64186712,146.01245967)(114.31634732,145.14657701)(114.31634732,144.0398097)
\curveto(114.31634732,142.78981368)(114.69395028,141.84580627)(115.44915621,141.20778747)
\curveto(116.20436214,140.58278946)(117.35670222,140.27029045)(118.90617646,140.27029045)
\curveto(120.19523485,140.27029045)(121.37361652,140.5176855)(122.44132145,141.01247559)
\curveto(123.50902638,141.52028647)(124.49860657,142.1257533)(125.410062,142.82887606)
\closepath
}
}
{
\newrgbcolor{curcolor}{0 0 0}
\pscustom[linestyle=none,fillstyle=solid,fillcolor=curcolor]
{
\newpath
\moveto(147.46077384,137.65311129)
\curveto(146.77067187,137.4708202)(146.01546594,137.32108109)(145.19515605,137.20389397)
\curveto(144.38786695,137.08670684)(143.665213,137.02811328)(143.0271942,137.02811328)
\curveto(140.80063879,137.02811328)(139.10793585,137.6270697)(137.94908537,138.82498256)
\curveto(136.79023489,140.02289541)(136.21080965,141.94346221)(136.21080965,144.58668296)
\lineto(136.21080965,156.18820852)
\lineto(133.7303488,156.18820852)
\lineto(133.7303488,159.2741362)
\lineto(136.21080965,159.2741362)
\lineto(136.21080965,165.54364749)
\lineto(139.88267296,165.54364749)
\lineto(139.88267296,159.2741362)
\lineto(147.46077384,159.2741362)
\lineto(147.46077384,156.18820852)
\lineto(139.88267296,156.18820852)
\lineto(139.88267296,146.24683393)
\curveto(139.88267296,145.10100424)(139.90871455,144.2025696)(139.96079771,143.55153001)
\curveto(140.01288088,142.9135112)(140.19517197,142.31455478)(140.50767097,141.75466073)
\curveto(140.79412839,141.23382905)(141.18475215,140.84971569)(141.67954224,140.60232065)
\curveto(142.18735313,140.36794639)(142.95557985,140.25075927)(143.9842224,140.25075927)
\curveto(144.58317883,140.25075927)(145.20817684,140.33539441)(145.85921643,140.50466471)
\curveto(146.51025603,140.68695579)(146.97900454,140.8366949)(147.26546196,140.95388203)
\lineto(147.46077384,140.95388203)
\closepath
}
}
{
\newrgbcolor{curcolor}{0 0 0}
\pscustom[linestyle=none,fillstyle=solid,fillcolor=curcolor]
{
\newpath
\moveto(168.84742421,137.45779941)
\lineto(165.19509209,137.45779941)
\lineto(165.19509209,139.78201076)
\curveto(164.8695723,139.5606573)(164.42686537,139.24815829)(163.86697132,138.84451374)
\curveto(163.32009806,138.45388999)(162.7862456,138.14139098)(162.26541392,137.90701673)
\curveto(161.6534367,137.60753852)(160.95031394,137.36014347)(160.15604564,137.16483159)
\curveto(159.36177733,136.95649892)(158.43079071,136.85233259)(157.36308578,136.85233259)
\curveto(155.39694621,136.85233259)(153.73028485,137.50337218)(152.3631017,138.80545137)
\curveto(150.99591855,140.10753056)(150.31232698,141.76768152)(150.31232698,143.78590426)
\curveto(150.31232698,145.43954483)(150.66388836,146.774176)(151.36701112,147.78979776)
\curveto(152.08315467,148.81844032)(153.09877644,149.62572942)(154.41387642,150.21166505)
\curveto(155.74199719,150.79760069)(157.3370442,151.19473484)(159.19901743,151.40306751)
\curveto(161.06099067,151.61140018)(163.05968222,151.76764968)(165.19509209,151.87181602)
\lineto(165.19509209,152.43822046)
\curveto(165.19509209,153.27155114)(165.04535299,153.96165311)(164.74587477,154.50852637)
\curveto(164.45941735,155.05539963)(164.04275201,155.48508576)(163.49587875,155.79758477)
\curveto(162.97504708,156.09706298)(162.35004907,156.29888525)(161.62088472,156.40305159)
\curveto(160.89172038,156.50721792)(160.13000405,156.55930109)(159.33573575,156.55930109)
\curveto(158.37219715,156.55930109)(157.29798182,156.42909317)(156.11308976,156.16867733)
\curveto(154.9281977,155.92128229)(153.70424326,155.55670012)(152.44122645,155.07493082)
\lineto(152.24591457,155.07493082)
\lineto(152.24591457,158.80538769)
\curveto(152.96205812,159.00069957)(153.99721108,159.21554263)(155.35137343,159.44991689)
\curveto(156.70553579,159.68429114)(158.04016696,159.80147827)(159.35526694,159.80147827)
\curveto(160.89172038,159.80147827)(162.22635154,159.67127035)(163.35916044,159.41085451)
\curveto(164.50499012,159.16345947)(165.49457031,158.73377333)(166.32790099,158.12179612)
\curveto(167.14821087,157.52283969)(167.77320888,156.74810257)(168.20289502,155.79758477)
\curveto(168.63258115,154.84706696)(168.84742421,153.66868529)(168.84742421,152.26243977)
\closepath
\moveto(165.19509209,142.82887606)
\lineto(165.19509209,148.90307547)
\curveto(164.07530399,148.83797151)(162.75369362,148.74031557)(161.23026097,148.61010765)
\curveto(159.71984911,148.47989973)(158.52193626,148.29109825)(157.63652241,148.0437032)
\curveto(156.58183827,147.74422499)(155.7289764,147.27547648)(155.0779368,146.63745768)
\curveto(154.42689721,146.01245967)(154.10137741,145.14657701)(154.10137741,144.0398097)
\curveto(154.10137741,142.78981368)(154.47898038,141.84580627)(155.23418631,141.20778747)
\curveto(155.98939224,140.58278946)(157.14173232,140.27029045)(158.69120655,140.27029045)
\curveto(159.98026495,140.27029045)(161.15864661,140.5176855)(162.22635154,141.01247559)
\curveto(163.29405648,141.52028647)(164.28363666,142.1257533)(165.19509209,142.82887606)
\closepath
}
}
{
\newrgbcolor{curcolor}{0 0 0}
\pscustom[linewidth=3.20881881,linecolor=curcolor]
{
\newpath
\moveto(125.21474646,176.00486764)
\lineto(125.21474646,312.64627866)
}
}
{
\newrgbcolor{curcolor}{0 0 0}
\pscustom[linestyle=none,fillstyle=solid,fillcolor=curcolor]
{
\newpath
\moveto(125.21474646,280.55809055)
\lineto(138.0500217,267.7228153)
\lineto(125.21474646,312.64627866)
\lineto(112.37947121,267.7228153)
\closepath
}
}
{
\newrgbcolor{curcolor}{0 0 0}
\pscustom[linewidth=3.42274017,linecolor=curcolor]
{
\newpath
\moveto(125.21474646,280.55809055)
\lineto(138.0500217,267.7228153)
\lineto(125.21474646,312.64627866)
\lineto(112.37947121,267.7228153)
\closepath
}
}
{
\newrgbcolor{curcolor}{0 0 0}
\pscustom[linewidth=3.43438118,linecolor=curcolor]
{
\newpath
\moveto(174.32469921,344.84808023)
\lineto(330.92751496,344.84808023)
}
}
{
\newrgbcolor{curcolor}{0 0 0}
\pscustom[linestyle=none,fillstyle=solid,fillcolor=curcolor]
{
\newpath
\moveto(296.58370311,344.84808023)
\lineto(282.84617838,331.1105555)
\lineto(330.92751496,344.84808023)
\lineto(282.84617838,358.58560497)
\closepath
}
}
{
\newrgbcolor{curcolor}{0 0 0}
\pscustom[linewidth=3.66334004,linecolor=curcolor]
{
\newpath
\moveto(296.58370311,344.84808023)
\lineto(282.84617838,331.1105555)
\lineto(330.92751496,344.84808023)
\lineto(282.84617838,358.58560497)
\closepath
}
}
\end{pspicture}
}
    \captionsetup{width=0.85\linewidth}
    \caption{Comparing ZFS with a modified version using task migration, a 10-15\% improvement is visible with task migration for files 1 GB and larger 
        when the ARC data being accessed is on another NUMA node.}
    \label{fig:smallresults}
\end{figure}

Most importantly, when comparing the results of remote node ARC access with task migration enabled to the optimal results
running on the same node, we see that the task migration has managed to regain all of the performance that would otherwise be lost when accessing memory from a different node (Figure \ref{fig:optimal}).

\begin{figure}[H]
    \centering
    \resizebox{0.75\linewidth}{!}{%LaTeX with PSTricks extensions
%%Creator: Inkscape 1.0.2-2 (e86c870879, 2021-01-15)
%%Please note this file requires PSTricks extensions
\psset{xunit=.5pt,yunit=.5pt,runit=.5pt}
\begin{pspicture}(600,480)
{
\newrgbcolor{curcolor}{0 0 0}
\pscustom[linewidth=1,linecolor=curcolor]
{
\newpath
\moveto(105.1,57.6)
\lineto(114.1,57.6)
\moveto(575,57.6)
\lineto(566,57.6)
}
}
{
\newrgbcolor{curcolor}{0 0 0}
\pscustom[linestyle=none,fillstyle=solid,fillcolor=curcolor]
{
\newpath
\moveto(53.92109375,57.93632812)
\curveto(53.92109375,58.95195312)(54.02460937,59.76835937)(54.23164062,60.38554687)
\curveto(54.44257812,61.00664062)(54.753125,61.48515625)(55.16328125,61.82109375)
\curveto(55.57734375,62.15703125)(56.096875,62.325)(56.721875,62.325)
\curveto(57.1828125,62.325)(57.58710937,62.23125)(57.93476562,62.04375)
\curveto(58.28242187,61.86015625)(58.56953125,61.59257812)(58.79609375,61.24101562)
\curveto(59.02265625,60.89335937)(59.20039062,60.46757812)(59.32929687,59.96367187)
\curveto(59.45820312,59.46367187)(59.52265625,58.78789062)(59.52265625,57.93632812)
\curveto(59.52265625,56.92851562)(59.41914062,56.1140625)(59.21210937,55.49296875)
\curveto(59.00507812,54.87578125)(58.69453125,54.39726562)(58.28046875,54.05742187)
\curveto(57.8703125,53.72148437)(57.35078125,53.55351562)(56.721875,53.55351562)
\curveto(55.89375,53.55351562)(55.24335937,53.85039062)(54.77070312,54.44414062)
\curveto(54.20429687,55.15898437)(53.92109375,56.32304687)(53.92109375,57.93632812)
\closepath
\moveto(55.00507812,57.93632812)
\curveto(55.00507812,56.52617187)(55.16914062,55.58671875)(55.49726562,55.11796875)
\curveto(55.82929687,54.653125)(56.2375,54.42070312)(56.721875,54.42070312)
\curveto(57.20625,54.42070312)(57.6125,54.65507812)(57.940625,55.12382812)
\curveto(58.27265625,55.59257812)(58.43867187,56.53007812)(58.43867187,57.93632812)
\curveto(58.43867187,59.35039062)(58.27265625,60.28984375)(57.940625,60.7546875)
\curveto(57.6125,61.21953125)(57.20234375,61.45195312)(56.71015625,61.45195312)
\curveto(56.22578125,61.45195312)(55.8390625,61.246875)(55.55,60.83671875)
\curveto(55.18671875,60.31328125)(55.00507812,59.34648437)(55.00507812,57.93632812)
\closepath
}
}
{
\newrgbcolor{curcolor}{0 0 0}
\pscustom[linestyle=none,fillstyle=solid,fillcolor=curcolor]
{
\newpath
\moveto(61.18671875,53.7)
\lineto(61.18671875,54.90117187)
\lineto(62.38789062,54.90117187)
\lineto(62.38789062,53.7)
\closepath
}
}
{
\newrgbcolor{curcolor}{0 0 0}
\pscustom[linestyle=none,fillstyle=solid,fillcolor=curcolor]
{
\newpath
\moveto(63.92890625,57.93632812)
\curveto(63.92890625,58.95195312)(64.03242187,59.76835937)(64.23945312,60.38554687)
\curveto(64.45039062,61.00664062)(64.7609375,61.48515625)(65.17109375,61.82109375)
\curveto(65.58515625,62.15703125)(66.1046875,62.325)(66.7296875,62.325)
\curveto(67.190625,62.325)(67.59492187,62.23125)(67.94257812,62.04375)
\curveto(68.29023437,61.86015625)(68.57734375,61.59257812)(68.80390625,61.24101562)
\curveto(69.03046875,60.89335937)(69.20820312,60.46757812)(69.33710937,59.96367187)
\curveto(69.46601562,59.46367187)(69.53046875,58.78789062)(69.53046875,57.93632812)
\curveto(69.53046875,56.92851562)(69.42695312,56.1140625)(69.21992187,55.49296875)
\curveto(69.01289062,54.87578125)(68.70234375,54.39726562)(68.28828125,54.05742187)
\curveto(67.878125,53.72148437)(67.35859375,53.55351562)(66.7296875,53.55351562)
\curveto(65.9015625,53.55351562)(65.25117187,53.85039062)(64.77851562,54.44414062)
\curveto(64.21210937,55.15898437)(63.92890625,56.32304687)(63.92890625,57.93632812)
\closepath
\moveto(65.01289062,57.93632812)
\curveto(65.01289062,56.52617187)(65.17695312,55.58671875)(65.50507812,55.11796875)
\curveto(65.83710937,54.653125)(66.2453125,54.42070312)(66.7296875,54.42070312)
\curveto(67.2140625,54.42070312)(67.6203125,54.65507812)(67.9484375,55.12382812)
\curveto(68.28046875,55.59257812)(68.44648437,56.53007812)(68.44648437,57.93632812)
\curveto(68.44648437,59.35039062)(68.28046875,60.28984375)(67.9484375,60.7546875)
\curveto(67.6203125,61.21953125)(67.21015625,61.45195312)(66.71796875,61.45195312)
\curveto(66.23359375,61.45195312)(65.846875,61.246875)(65.5578125,60.83671875)
\curveto(65.19453125,60.31328125)(65.01289062,59.34648437)(65.01289062,57.93632812)
\closepath
}
}
{
\newrgbcolor{curcolor}{0 0 0}
\pscustom[linestyle=none,fillstyle=solid,fillcolor=curcolor]
{
\newpath
\moveto(70.60859375,55.96757812)
\lineto(71.66328125,56.10820312)
\curveto(71.784375,55.51054687)(71.98945312,55.07890625)(72.27851562,54.81328125)
\curveto(72.57148437,54.5515625)(72.92695312,54.42070312)(73.34492187,54.42070312)
\curveto(73.84101562,54.42070312)(74.25898437,54.59257812)(74.59882812,54.93632812)
\curveto(74.94257812,55.28007812)(75.11445312,55.70585937)(75.11445312,56.21367187)
\curveto(75.11445312,56.69804687)(74.95625,57.09648437)(74.63984375,57.40898437)
\curveto(74.3234375,57.72539062)(73.92109375,57.88359375)(73.4328125,57.88359375)
\curveto(73.23359375,57.88359375)(72.98554687,57.84453125)(72.68867187,57.76640625)
\lineto(72.80585937,58.6921875)
\curveto(72.87617187,58.684375)(72.9328125,58.68046875)(72.97578125,58.68046875)
\curveto(73.425,58.68046875)(73.82929687,58.79765625)(74.18867187,59.03203125)
\curveto(74.54804687,59.26640625)(74.72773437,59.62773437)(74.72773437,60.11601562)
\curveto(74.72773437,60.50273437)(74.596875,60.82304687)(74.33515625,61.07695312)
\curveto(74.0734375,61.33085937)(73.73554687,61.4578125)(73.32148437,61.4578125)
\curveto(72.91132812,61.4578125)(72.56953125,61.32890625)(72.29609375,61.07109375)
\curveto(72.02265625,60.81328125)(71.846875,60.4265625)(71.76875,59.9109375)
\lineto(70.7140625,60.0984375)
\curveto(70.84296875,60.80546875)(71.1359375,61.35234375)(71.59296875,61.7390625)
\curveto(72.05,62.1296875)(72.61835937,62.325)(73.29804687,62.325)
\curveto(73.76679687,62.325)(74.1984375,62.2234375)(74.59296875,62.0203125)
\curveto(74.9875,61.82109375)(75.28828125,61.54765625)(75.4953125,61.2)
\curveto(75.70625,60.85234375)(75.81171875,60.48320312)(75.81171875,60.09257812)
\curveto(75.81171875,59.72148437)(75.71210937,59.38359375)(75.51289062,59.07890625)
\curveto(75.31367187,58.77421875)(75.01875,58.53203125)(74.628125,58.35234375)
\curveto(75.1359375,58.23515625)(75.53046875,57.99101562)(75.81171875,57.61992187)
\curveto(76.09296875,57.25273437)(76.23359375,56.79179687)(76.23359375,56.23710937)
\curveto(76.23359375,55.48710937)(75.96015625,54.85039062)(75.41328125,54.32695312)
\curveto(74.86640625,53.80742187)(74.175,53.54765625)(73.3390625,53.54765625)
\curveto(72.58515625,53.54765625)(71.95820312,53.77226562)(71.45820312,54.22148437)
\curveto(70.96210937,54.67070312)(70.67890625,55.25273437)(70.60859375,55.96757812)
\closepath
}
}
{
\newrgbcolor{curcolor}{0 0 0}
\pscustom[linestyle=none,fillstyle=solid,fillcolor=curcolor]
{
\newpath
\moveto(81.24921875,53.7)
\lineto(80.19453125,53.7)
\lineto(80.19453125,60.42070312)
\curveto(79.940625,60.17851562)(79.60664062,59.93632812)(79.19257812,59.69414062)
\curveto(78.78242187,59.45195312)(78.41328125,59.2703125)(78.08515625,59.14921875)
\lineto(78.08515625,60.16875)
\curveto(78.675,60.44609375)(79.190625,60.78203125)(79.63203125,61.1765625)
\curveto(80.0734375,61.57109375)(80.3859375,61.95390625)(80.56953125,62.325)
\lineto(81.24921875,62.325)
\closepath
}
}
{
\newrgbcolor{curcolor}{0 0 0}
\pscustom[linestyle=none,fillstyle=solid,fillcolor=curcolor]
{
\newpath
\moveto(89.49335937,54.71367187)
\lineto(89.49335937,53.7)
\lineto(83.815625,53.7)
\curveto(83.8078125,53.95390625)(83.84882812,54.19804687)(83.93867187,54.43242187)
\curveto(84.08320312,54.81914062)(84.31367187,55.2)(84.63007812,55.575)
\curveto(84.95039062,55.95)(85.41132812,56.38359375)(86.01289062,56.87578125)
\curveto(86.94648437,57.64140625)(87.57734375,58.246875)(87.90546875,58.6921875)
\curveto(88.23359375,59.14140625)(88.39765625,59.56523437)(88.39765625,59.96367187)
\curveto(88.39765625,60.38164062)(88.24726562,60.73320312)(87.94648437,61.01835937)
\curveto(87.64960937,61.30742187)(87.2609375,61.45195312)(86.78046875,61.45195312)
\curveto(86.27265625,61.45195312)(85.86640625,61.29960937)(85.56171875,60.99492187)
\curveto(85.25703125,60.69023437)(85.10273437,60.26835937)(85.09882812,59.72929687)
\lineto(84.01484375,59.840625)
\curveto(84.0890625,60.64921875)(84.36835937,61.26445312)(84.85273437,61.68632812)
\curveto(85.33710937,62.11210937)(85.9875,62.325)(86.80390625,62.325)
\curveto(87.628125,62.325)(88.28046875,62.09648437)(88.7609375,61.63945312)
\curveto(89.24140625,61.18242187)(89.48164062,60.61601562)(89.48164062,59.94023437)
\curveto(89.48164062,59.59648437)(89.41132812,59.25859375)(89.27070312,58.9265625)
\curveto(89.13007812,58.59453125)(88.89570312,58.24492187)(88.56757812,57.87773437)
\curveto(88.24335937,57.51054687)(87.70234375,57.00664062)(86.94453125,56.36601562)
\curveto(86.31171875,55.83476562)(85.90546875,55.4734375)(85.72578125,55.28203125)
\curveto(85.54609375,55.09453125)(85.39765625,54.90507812)(85.28046875,54.71367187)
\closepath
}
}
{
\newrgbcolor{curcolor}{0 0 0}
\pscustom[linestyle=none,fillstyle=solid,fillcolor=curcolor]
{
\newpath
\moveto(90.62421875,55.95)
\lineto(91.73164062,56.04375)
\curveto(91.81367187,55.5046875)(92.003125,55.0984375)(92.3,54.825)
\curveto(92.60078125,54.55546875)(92.96210937,54.42070312)(93.38398437,54.42070312)
\curveto(93.89179687,54.42070312)(94.32148437,54.61210937)(94.67304687,54.99492187)
\curveto(95.02460937,55.37773437)(95.20039062,55.88554687)(95.20039062,56.51835937)
\curveto(95.20039062,57.11992187)(95.03046875,57.59453125)(94.690625,57.9421875)
\curveto(94.3546875,58.28984375)(93.91328125,58.46367187)(93.36640625,58.46367187)
\curveto(93.0265625,58.46367187)(92.71992187,58.38554687)(92.44648437,58.22929687)
\curveto(92.17304687,58.07695312)(91.95820312,57.87773437)(91.80195312,57.63164062)
\lineto(90.81171875,57.76054687)
\lineto(91.64375,62.17265625)
\lineto(95.91523437,62.17265625)
\lineto(95.91523437,61.16484375)
\lineto(92.4875,61.16484375)
\lineto(92.02460937,58.85625)
\curveto(92.54023437,59.215625)(93.08125,59.3953125)(93.64765625,59.3953125)
\curveto(94.39765625,59.3953125)(95.03046875,59.13554687)(95.54609375,58.61601562)
\curveto(96.06171875,58.09648437)(96.31953125,57.42851562)(96.31953125,56.61210937)
\curveto(96.31953125,55.83476562)(96.09296875,55.16289062)(95.63984375,54.59648437)
\curveto(95.0890625,53.90117187)(94.33710937,53.55351562)(93.38398437,53.55351562)
\curveto(92.60273437,53.55351562)(91.9640625,53.77226562)(91.46796875,54.20976562)
\curveto(90.97578125,54.64726562)(90.69453125,55.22734375)(90.62421875,55.95)
\closepath
}
}
{
\newrgbcolor{curcolor}{0 0 0}
\pscustom[linewidth=1,linecolor=curcolor]
{
\newpath
\moveto(105.1,103.6)
\lineto(114.1,103.6)
\moveto(575,103.6)
\lineto(566,103.6)
}
}
{
\newrgbcolor{curcolor}{0 0 0}
\pscustom[linestyle=none,fillstyle=solid,fillcolor=curcolor]
{
\newpath
\moveto(60.59492187,103.93632812)
\curveto(60.59492187,104.95195312)(60.6984375,105.76835937)(60.90546875,106.38554687)
\curveto(61.11640625,107.00664062)(61.42695312,107.48515625)(61.83710937,107.82109375)
\curveto(62.25117187,108.15703125)(62.77070312,108.325)(63.39570312,108.325)
\curveto(63.85664062,108.325)(64.2609375,108.23125)(64.60859375,108.04375)
\curveto(64.95625,107.86015625)(65.24335937,107.59257812)(65.46992187,107.24101562)
\curveto(65.69648437,106.89335937)(65.87421875,106.46757812)(66.003125,105.96367187)
\curveto(66.13203125,105.46367187)(66.19648437,104.78789062)(66.19648437,103.93632812)
\curveto(66.19648437,102.92851562)(66.09296875,102.1140625)(65.8859375,101.49296875)
\curveto(65.67890625,100.87578125)(65.36835937,100.39726562)(64.95429687,100.05742187)
\curveto(64.54414062,99.72148437)(64.02460937,99.55351562)(63.39570312,99.55351562)
\curveto(62.56757812,99.55351562)(61.9171875,99.85039062)(61.44453125,100.44414062)
\curveto(60.878125,101.15898437)(60.59492187,102.32304687)(60.59492187,103.93632812)
\closepath
\moveto(61.67890625,103.93632812)
\curveto(61.67890625,102.52617187)(61.84296875,101.58671875)(62.17109375,101.11796875)
\curveto(62.503125,100.653125)(62.91132812,100.42070312)(63.39570312,100.42070312)
\curveto(63.88007812,100.42070312)(64.28632812,100.65507812)(64.61445312,101.12382812)
\curveto(64.94648437,101.59257812)(65.1125,102.53007812)(65.1125,103.93632812)
\curveto(65.1125,105.35039062)(64.94648437,106.28984375)(64.61445312,106.7546875)
\curveto(64.28632812,107.21953125)(63.87617187,107.45195312)(63.38398437,107.45195312)
\curveto(62.89960937,107.45195312)(62.51289062,107.246875)(62.22382812,106.83671875)
\curveto(61.86054687,106.31328125)(61.67890625,105.34648437)(61.67890625,103.93632812)
\closepath
}
}
{
\newrgbcolor{curcolor}{0 0 0}
\pscustom[linestyle=none,fillstyle=solid,fillcolor=curcolor]
{
\newpath
\moveto(67.86054687,99.7)
\lineto(67.86054687,100.90117187)
\lineto(69.06171875,100.90117187)
\lineto(69.06171875,99.7)
\closepath
}
}
{
\newrgbcolor{curcolor}{0 0 0}
\pscustom[linestyle=none,fillstyle=solid,fillcolor=curcolor]
{
\newpath
\moveto(70.60273437,103.93632812)
\curveto(70.60273437,104.95195312)(70.70625,105.76835937)(70.91328125,106.38554687)
\curveto(71.12421875,107.00664062)(71.43476562,107.48515625)(71.84492187,107.82109375)
\curveto(72.25898437,108.15703125)(72.77851562,108.325)(73.40351562,108.325)
\curveto(73.86445312,108.325)(74.26875,108.23125)(74.61640625,108.04375)
\curveto(74.9640625,107.86015625)(75.25117187,107.59257812)(75.47773437,107.24101562)
\curveto(75.70429687,106.89335937)(75.88203125,106.46757812)(76.0109375,105.96367187)
\curveto(76.13984375,105.46367187)(76.20429687,104.78789062)(76.20429687,103.93632812)
\curveto(76.20429687,102.92851562)(76.10078125,102.1140625)(75.89375,101.49296875)
\curveto(75.68671875,100.87578125)(75.37617187,100.39726562)(74.96210937,100.05742187)
\curveto(74.55195312,99.72148437)(74.03242187,99.55351562)(73.40351562,99.55351562)
\curveto(72.57539062,99.55351562)(71.925,99.85039062)(71.45234375,100.44414062)
\curveto(70.8859375,101.15898437)(70.60273437,102.32304687)(70.60273437,103.93632812)
\closepath
\moveto(71.68671875,103.93632812)
\curveto(71.68671875,102.52617187)(71.85078125,101.58671875)(72.17890625,101.11796875)
\curveto(72.5109375,100.653125)(72.91914062,100.42070312)(73.40351562,100.42070312)
\curveto(73.88789062,100.42070312)(74.29414062,100.65507812)(74.62226562,101.12382812)
\curveto(74.95429687,101.59257812)(75.1203125,102.53007812)(75.1203125,103.93632812)
\curveto(75.1203125,105.35039062)(74.95429687,106.28984375)(74.62226562,106.7546875)
\curveto(74.29414062,107.21953125)(73.88398437,107.45195312)(73.39179687,107.45195312)
\curveto(72.90742187,107.45195312)(72.52070312,107.246875)(72.23164062,106.83671875)
\curveto(71.86835937,106.31328125)(71.68671875,105.34648437)(71.68671875,103.93632812)
\closepath
}
}
{
\newrgbcolor{curcolor}{0 0 0}
\pscustom[linestyle=none,fillstyle=solid,fillcolor=curcolor]
{
\newpath
\moveto(82.74921875,106.18632812)
\lineto(81.70039062,106.10429687)
\curveto(81.60664062,106.51835937)(81.47382812,106.81914062)(81.30195312,107.00664062)
\curveto(81.01679687,107.30742187)(80.66523437,107.4578125)(80.24726562,107.4578125)
\curveto(79.91132812,107.4578125)(79.61640625,107.3640625)(79.3625,107.1765625)
\curveto(79.03046875,106.934375)(78.76875,106.58085937)(78.57734375,106.11601562)
\curveto(78.3859375,105.65117187)(78.28632812,104.9890625)(78.27851562,104.1296875)
\curveto(78.53242187,104.51640625)(78.84296875,104.80351562)(79.21015625,104.99101562)
\curveto(79.57734375,105.17851562)(79.96210937,105.27226562)(80.36445312,105.27226562)
\curveto(81.06757812,105.27226562)(81.66523437,105.0125)(82.15742187,104.49296875)
\curveto(82.65351562,103.97734375)(82.9015625,103.309375)(82.9015625,102.4890625)
\curveto(82.9015625,101.95)(82.784375,101.44804687)(82.55,100.98320312)
\curveto(82.31953125,100.52226562)(82.00117187,100.16875)(81.59492187,99.92265625)
\curveto(81.18867187,99.6765625)(80.72773437,99.55351562)(80.21210937,99.55351562)
\curveto(79.33320312,99.55351562)(78.61640625,99.87578125)(78.06171875,100.5203125)
\curveto(77.50703125,101.16875)(77.2296875,102.23515625)(77.2296875,103.71953125)
\curveto(77.2296875,105.3796875)(77.53632812,106.58671875)(78.14960937,107.340625)
\curveto(78.68476562,107.996875)(79.40546875,108.325)(80.31171875,108.325)
\curveto(80.9875,108.325)(81.54023437,108.13554687)(81.96992187,107.75664062)
\curveto(82.40351562,107.37773437)(82.66328125,106.85429687)(82.74921875,106.18632812)
\closepath
\moveto(78.44257812,102.48320312)
\curveto(78.44257812,102.11992187)(78.51875,101.77226562)(78.67109375,101.44023437)
\curveto(78.82734375,101.10820312)(79.04414062,100.85429687)(79.32148437,100.67851562)
\curveto(79.59882812,100.50664062)(79.88984375,100.42070312)(80.19453125,100.42070312)
\curveto(80.63984375,100.42070312)(81.02265625,100.60039062)(81.34296875,100.95976562)
\curveto(81.66328125,101.31914062)(81.8234375,101.80742187)(81.8234375,102.42460937)
\curveto(81.8234375,103.01835937)(81.66523437,103.48515625)(81.34882812,103.825)
\curveto(81.03242187,104.16875)(80.63398437,104.340625)(80.15351562,104.340625)
\curveto(79.67695312,104.340625)(79.27265625,104.16875)(78.940625,103.825)
\curveto(78.60859375,103.48515625)(78.44257812,103.03789062)(78.44257812,102.48320312)
\closepath
}
}
{
\newrgbcolor{curcolor}{0 0 0}
\pscustom[linestyle=none,fillstyle=solid,fillcolor=curcolor]
{
\newpath
\moveto(89.49335937,100.71367187)
\lineto(89.49335937,99.7)
\lineto(83.815625,99.7)
\curveto(83.8078125,99.95390625)(83.84882812,100.19804687)(83.93867187,100.43242187)
\curveto(84.08320312,100.81914062)(84.31367187,101.2)(84.63007812,101.575)
\curveto(84.95039062,101.95)(85.41132812,102.38359375)(86.01289062,102.87578125)
\curveto(86.94648437,103.64140625)(87.57734375,104.246875)(87.90546875,104.6921875)
\curveto(88.23359375,105.14140625)(88.39765625,105.56523437)(88.39765625,105.96367187)
\curveto(88.39765625,106.38164062)(88.24726562,106.73320312)(87.94648437,107.01835937)
\curveto(87.64960937,107.30742187)(87.2609375,107.45195312)(86.78046875,107.45195312)
\curveto(86.27265625,107.45195312)(85.86640625,107.29960937)(85.56171875,106.99492187)
\curveto(85.25703125,106.69023437)(85.10273437,106.26835937)(85.09882812,105.72929687)
\lineto(84.01484375,105.840625)
\curveto(84.0890625,106.64921875)(84.36835937,107.26445312)(84.85273437,107.68632812)
\curveto(85.33710937,108.11210937)(85.9875,108.325)(86.80390625,108.325)
\curveto(87.628125,108.325)(88.28046875,108.09648437)(88.7609375,107.63945312)
\curveto(89.24140625,107.18242187)(89.48164062,106.61601562)(89.48164062,105.94023437)
\curveto(89.48164062,105.59648437)(89.41132812,105.25859375)(89.27070312,104.9265625)
\curveto(89.13007812,104.59453125)(88.89570312,104.24492187)(88.56757812,103.87773437)
\curveto(88.24335937,103.51054687)(87.70234375,103.00664062)(86.94453125,102.36601562)
\curveto(86.31171875,101.83476562)(85.90546875,101.4734375)(85.72578125,101.28203125)
\curveto(85.54609375,101.09453125)(85.39765625,100.90507812)(85.28046875,100.71367187)
\closepath
}
}
{
\newrgbcolor{curcolor}{0 0 0}
\pscustom[linestyle=none,fillstyle=solid,fillcolor=curcolor]
{
\newpath
\moveto(90.62421875,101.95)
\lineto(91.73164062,102.04375)
\curveto(91.81367187,101.5046875)(92.003125,101.0984375)(92.3,100.825)
\curveto(92.60078125,100.55546875)(92.96210937,100.42070312)(93.38398437,100.42070312)
\curveto(93.89179687,100.42070312)(94.32148437,100.61210937)(94.67304687,100.99492187)
\curveto(95.02460937,101.37773437)(95.20039062,101.88554687)(95.20039062,102.51835937)
\curveto(95.20039062,103.11992187)(95.03046875,103.59453125)(94.690625,103.9421875)
\curveto(94.3546875,104.28984375)(93.91328125,104.46367187)(93.36640625,104.46367187)
\curveto(93.0265625,104.46367187)(92.71992187,104.38554687)(92.44648437,104.22929687)
\curveto(92.17304687,104.07695312)(91.95820312,103.87773437)(91.80195312,103.63164062)
\lineto(90.81171875,103.76054687)
\lineto(91.64375,108.17265625)
\lineto(95.91523437,108.17265625)
\lineto(95.91523437,107.16484375)
\lineto(92.4875,107.16484375)
\lineto(92.02460937,104.85625)
\curveto(92.54023437,105.215625)(93.08125,105.3953125)(93.64765625,105.3953125)
\curveto(94.39765625,105.3953125)(95.03046875,105.13554687)(95.54609375,104.61601562)
\curveto(96.06171875,104.09648437)(96.31953125,103.42851562)(96.31953125,102.61210937)
\curveto(96.31953125,101.83476562)(96.09296875,101.16289062)(95.63984375,100.59648437)
\curveto(95.0890625,99.90117187)(94.33710937,99.55351562)(93.38398437,99.55351562)
\curveto(92.60273437,99.55351562)(91.9640625,99.77226562)(91.46796875,100.20976562)
\curveto(90.97578125,100.64726562)(90.69453125,101.22734375)(90.62421875,101.95)
\closepath
}
}
{
\newrgbcolor{curcolor}{0 0 0}
\pscustom[linewidth=1,linecolor=curcolor]
{
\newpath
\moveto(105.1,149.7)
\lineto(114.1,149.7)
\moveto(575,149.7)
\lineto(566,149.7)
}
}
{
\newrgbcolor{curcolor}{0 0 0}
\pscustom[linestyle=none,fillstyle=solid,fillcolor=curcolor]
{
\newpath
\moveto(67.26875,150.03632813)
\curveto(67.26875,151.05195313)(67.37226562,151.86835938)(67.57929687,152.48554688)
\curveto(67.79023437,153.10664063)(68.10078125,153.58515625)(68.5109375,153.92109375)
\curveto(68.925,154.25703125)(69.44453125,154.425)(70.06953125,154.425)
\curveto(70.53046875,154.425)(70.93476562,154.33125)(71.28242187,154.14375)
\curveto(71.63007812,153.96015625)(71.9171875,153.69257813)(72.14375,153.34101563)
\curveto(72.3703125,152.99335938)(72.54804687,152.56757813)(72.67695312,152.06367188)
\curveto(72.80585937,151.56367188)(72.8703125,150.88789063)(72.8703125,150.03632813)
\curveto(72.8703125,149.02851563)(72.76679687,148.2140625)(72.55976562,147.59296875)
\curveto(72.35273437,146.97578125)(72.0421875,146.49726563)(71.628125,146.15742188)
\curveto(71.21796875,145.82148438)(70.6984375,145.65351563)(70.06953125,145.65351563)
\curveto(69.24140625,145.65351563)(68.59101562,145.95039063)(68.11835937,146.54414063)
\curveto(67.55195312,147.25898438)(67.26875,148.42304688)(67.26875,150.03632813)
\closepath
\moveto(68.35273437,150.03632813)
\curveto(68.35273437,148.62617188)(68.51679687,147.68671875)(68.84492187,147.21796875)
\curveto(69.17695312,146.753125)(69.58515625,146.52070313)(70.06953125,146.52070313)
\curveto(70.55390625,146.52070313)(70.96015625,146.75507813)(71.28828125,147.22382813)
\curveto(71.6203125,147.69257813)(71.78632812,148.63007813)(71.78632812,150.03632813)
\curveto(71.78632812,151.45039063)(71.6203125,152.38984375)(71.28828125,152.8546875)
\curveto(70.96015625,153.31953125)(70.55,153.55195313)(70.0578125,153.55195313)
\curveto(69.5734375,153.55195313)(69.18671875,153.346875)(68.89765625,152.93671875)
\curveto(68.534375,152.41328125)(68.35273437,151.44648438)(68.35273437,150.03632813)
\closepath
}
}
{
\newrgbcolor{curcolor}{0 0 0}
\pscustom[linestyle=none,fillstyle=solid,fillcolor=curcolor]
{
\newpath
\moveto(74.534375,145.8)
\lineto(74.534375,147.00117188)
\lineto(75.73554687,147.00117188)
\lineto(75.73554687,145.8)
\closepath
}
}
{
\newrgbcolor{curcolor}{0 0 0}
\pscustom[linestyle=none,fillstyle=solid,fillcolor=curcolor]
{
\newpath
\moveto(81.24921875,145.8)
\lineto(80.19453125,145.8)
\lineto(80.19453125,152.52070313)
\curveto(79.940625,152.27851563)(79.60664062,152.03632813)(79.19257812,151.79414063)
\curveto(78.78242187,151.55195313)(78.41328125,151.3703125)(78.08515625,151.24921875)
\lineto(78.08515625,152.26875)
\curveto(78.675,152.54609375)(79.190625,152.88203125)(79.63203125,153.2765625)
\curveto(80.0734375,153.67109375)(80.3859375,154.05390625)(80.56953125,154.425)
\lineto(81.24921875,154.425)
\closepath
}
}
{
\newrgbcolor{curcolor}{0 0 0}
\pscustom[linestyle=none,fillstyle=solid,fillcolor=curcolor]
{
\newpath
\moveto(89.49335937,146.81367188)
\lineto(89.49335937,145.8)
\lineto(83.815625,145.8)
\curveto(83.8078125,146.05390625)(83.84882812,146.29804688)(83.93867187,146.53242188)
\curveto(84.08320312,146.91914063)(84.31367187,147.3)(84.63007812,147.675)
\curveto(84.95039062,148.05)(85.41132812,148.48359375)(86.01289062,148.97578125)
\curveto(86.94648437,149.74140625)(87.57734375,150.346875)(87.90546875,150.7921875)
\curveto(88.23359375,151.24140625)(88.39765625,151.66523438)(88.39765625,152.06367188)
\curveto(88.39765625,152.48164063)(88.24726562,152.83320313)(87.94648437,153.11835938)
\curveto(87.64960937,153.40742188)(87.2609375,153.55195313)(86.78046875,153.55195313)
\curveto(86.27265625,153.55195313)(85.86640625,153.39960938)(85.56171875,153.09492188)
\curveto(85.25703125,152.79023438)(85.10273437,152.36835938)(85.09882812,151.82929688)
\lineto(84.01484375,151.940625)
\curveto(84.0890625,152.74921875)(84.36835937,153.36445313)(84.85273437,153.78632813)
\curveto(85.33710937,154.21210938)(85.9875,154.425)(86.80390625,154.425)
\curveto(87.628125,154.425)(88.28046875,154.19648438)(88.7609375,153.73945313)
\curveto(89.24140625,153.28242188)(89.48164062,152.71601563)(89.48164062,152.04023438)
\curveto(89.48164062,151.69648438)(89.41132812,151.35859375)(89.27070312,151.0265625)
\curveto(89.13007812,150.69453125)(88.89570312,150.34492188)(88.56757812,149.97773438)
\curveto(88.24335937,149.61054688)(87.70234375,149.10664063)(86.94453125,148.46601563)
\curveto(86.31171875,147.93476563)(85.90546875,147.5734375)(85.72578125,147.38203125)
\curveto(85.54609375,147.19453125)(85.39765625,147.00507813)(85.28046875,146.81367188)
\closepath
}
}
{
\newrgbcolor{curcolor}{0 0 0}
\pscustom[linestyle=none,fillstyle=solid,fillcolor=curcolor]
{
\newpath
\moveto(90.62421875,148.05)
\lineto(91.73164062,148.14375)
\curveto(91.81367187,147.6046875)(92.003125,147.1984375)(92.3,146.925)
\curveto(92.60078125,146.65546875)(92.96210937,146.52070313)(93.38398437,146.52070313)
\curveto(93.89179687,146.52070313)(94.32148437,146.71210938)(94.67304687,147.09492188)
\curveto(95.02460937,147.47773438)(95.20039062,147.98554688)(95.20039062,148.61835938)
\curveto(95.20039062,149.21992188)(95.03046875,149.69453125)(94.690625,150.0421875)
\curveto(94.3546875,150.38984375)(93.91328125,150.56367188)(93.36640625,150.56367188)
\curveto(93.0265625,150.56367188)(92.71992187,150.48554688)(92.44648437,150.32929688)
\curveto(92.17304687,150.17695313)(91.95820312,149.97773438)(91.80195312,149.73164063)
\lineto(90.81171875,149.86054688)
\lineto(91.64375,154.27265625)
\lineto(95.91523437,154.27265625)
\lineto(95.91523437,153.26484375)
\lineto(92.4875,153.26484375)
\lineto(92.02460937,150.95625)
\curveto(92.54023437,151.315625)(93.08125,151.4953125)(93.64765625,151.4953125)
\curveto(94.39765625,151.4953125)(95.03046875,151.23554688)(95.54609375,150.71601563)
\curveto(96.06171875,150.19648438)(96.31953125,149.52851563)(96.31953125,148.71210938)
\curveto(96.31953125,147.93476563)(96.09296875,147.26289063)(95.63984375,146.69648438)
\curveto(95.0890625,146.00117188)(94.33710937,145.65351563)(93.38398437,145.65351563)
\curveto(92.60273437,145.65351563)(91.9640625,145.87226563)(91.46796875,146.30976563)
\curveto(90.97578125,146.74726563)(90.69453125,147.32734375)(90.62421875,148.05)
\closepath
}
}
{
\newrgbcolor{curcolor}{0 0 0}
\pscustom[linewidth=1,linecolor=curcolor]
{
\newpath
\moveto(105.1,195.7)
\lineto(114.1,195.7)
\moveto(575,195.7)
\lineto(566,195.7)
}
}
{
\newrgbcolor{curcolor}{0 0 0}
\pscustom[linestyle=none,fillstyle=solid,fillcolor=curcolor]
{
\newpath
\moveto(73.94257812,196.03632813)
\curveto(73.94257812,197.05195313)(74.04609375,197.86835938)(74.253125,198.48554688)
\curveto(74.4640625,199.10664063)(74.77460937,199.58515625)(75.18476562,199.92109375)
\curveto(75.59882812,200.25703125)(76.11835937,200.425)(76.74335937,200.425)
\curveto(77.20429687,200.425)(77.60859375,200.33125)(77.95625,200.14375)
\curveto(78.30390625,199.96015625)(78.59101562,199.69257813)(78.81757812,199.34101563)
\curveto(79.04414062,198.99335938)(79.221875,198.56757813)(79.35078125,198.06367188)
\curveto(79.4796875,197.56367188)(79.54414062,196.88789063)(79.54414062,196.03632813)
\curveto(79.54414062,195.02851563)(79.440625,194.2140625)(79.23359375,193.59296875)
\curveto(79.0265625,192.97578125)(78.71601562,192.49726563)(78.30195312,192.15742188)
\curveto(77.89179687,191.82148438)(77.37226562,191.65351563)(76.74335937,191.65351563)
\curveto(75.91523437,191.65351563)(75.26484375,191.95039063)(74.7921875,192.54414063)
\curveto(74.22578125,193.25898438)(73.94257812,194.42304688)(73.94257812,196.03632813)
\closepath
\moveto(75.0265625,196.03632813)
\curveto(75.0265625,194.62617188)(75.190625,193.68671875)(75.51875,193.21796875)
\curveto(75.85078125,192.753125)(76.25898437,192.52070313)(76.74335937,192.52070313)
\curveto(77.22773437,192.52070313)(77.63398437,192.75507813)(77.96210937,193.22382813)
\curveto(78.29414062,193.69257813)(78.46015625,194.63007813)(78.46015625,196.03632813)
\curveto(78.46015625,197.45039063)(78.29414062,198.38984375)(77.96210937,198.8546875)
\curveto(77.63398437,199.31953125)(77.22382812,199.55195313)(76.73164062,199.55195313)
\curveto(76.24726562,199.55195313)(75.86054687,199.346875)(75.57148437,198.93671875)
\curveto(75.20820312,198.41328125)(75.0265625,197.44648438)(75.0265625,196.03632813)
\closepath
}
}
{
\newrgbcolor{curcolor}{0 0 0}
\pscustom[linestyle=none,fillstyle=solid,fillcolor=curcolor]
{
\newpath
\moveto(81.20820312,191.8)
\lineto(81.20820312,193.00117188)
\lineto(82.409375,193.00117188)
\lineto(82.409375,191.8)
\closepath
}
}
{
\newrgbcolor{curcolor}{0 0 0}
\pscustom[linestyle=none,fillstyle=solid,fillcolor=curcolor]
{
\newpath
\moveto(89.49335937,192.81367188)
\lineto(89.49335937,191.8)
\lineto(83.815625,191.8)
\curveto(83.8078125,192.05390625)(83.84882812,192.29804688)(83.93867187,192.53242188)
\curveto(84.08320312,192.91914063)(84.31367187,193.3)(84.63007812,193.675)
\curveto(84.95039062,194.05)(85.41132812,194.48359375)(86.01289062,194.97578125)
\curveto(86.94648437,195.74140625)(87.57734375,196.346875)(87.90546875,196.7921875)
\curveto(88.23359375,197.24140625)(88.39765625,197.66523438)(88.39765625,198.06367188)
\curveto(88.39765625,198.48164063)(88.24726562,198.83320313)(87.94648437,199.11835938)
\curveto(87.64960937,199.40742188)(87.2609375,199.55195313)(86.78046875,199.55195313)
\curveto(86.27265625,199.55195313)(85.86640625,199.39960938)(85.56171875,199.09492188)
\curveto(85.25703125,198.79023438)(85.10273437,198.36835938)(85.09882812,197.82929688)
\lineto(84.01484375,197.940625)
\curveto(84.0890625,198.74921875)(84.36835937,199.36445313)(84.85273437,199.78632813)
\curveto(85.33710937,200.21210938)(85.9875,200.425)(86.80390625,200.425)
\curveto(87.628125,200.425)(88.28046875,200.19648438)(88.7609375,199.73945313)
\curveto(89.24140625,199.28242188)(89.48164062,198.71601563)(89.48164062,198.04023438)
\curveto(89.48164062,197.69648438)(89.41132812,197.35859375)(89.27070312,197.0265625)
\curveto(89.13007812,196.69453125)(88.89570312,196.34492188)(88.56757812,195.97773438)
\curveto(88.24335937,195.61054688)(87.70234375,195.10664063)(86.94453125,194.46601563)
\curveto(86.31171875,193.93476563)(85.90546875,193.5734375)(85.72578125,193.38203125)
\curveto(85.54609375,193.19453125)(85.39765625,193.00507813)(85.28046875,192.81367188)
\closepath
}
}
{
\newrgbcolor{curcolor}{0 0 0}
\pscustom[linestyle=none,fillstyle=solid,fillcolor=curcolor]
{
\newpath
\moveto(90.62421875,194.05)
\lineto(91.73164062,194.14375)
\curveto(91.81367187,193.6046875)(92.003125,193.1984375)(92.3,192.925)
\curveto(92.60078125,192.65546875)(92.96210937,192.52070313)(93.38398437,192.52070313)
\curveto(93.89179687,192.52070313)(94.32148437,192.71210938)(94.67304687,193.09492188)
\curveto(95.02460937,193.47773438)(95.20039062,193.98554688)(95.20039062,194.61835938)
\curveto(95.20039062,195.21992188)(95.03046875,195.69453125)(94.690625,196.0421875)
\curveto(94.3546875,196.38984375)(93.91328125,196.56367188)(93.36640625,196.56367188)
\curveto(93.0265625,196.56367188)(92.71992187,196.48554688)(92.44648437,196.32929688)
\curveto(92.17304687,196.17695313)(91.95820312,195.97773438)(91.80195312,195.73164063)
\lineto(90.81171875,195.86054688)
\lineto(91.64375,200.27265625)
\lineto(95.91523437,200.27265625)
\lineto(95.91523437,199.26484375)
\lineto(92.4875,199.26484375)
\lineto(92.02460937,196.95625)
\curveto(92.54023437,197.315625)(93.08125,197.4953125)(93.64765625,197.4953125)
\curveto(94.39765625,197.4953125)(95.03046875,197.23554688)(95.54609375,196.71601563)
\curveto(96.06171875,196.19648438)(96.31953125,195.52851563)(96.31953125,194.71210938)
\curveto(96.31953125,193.93476563)(96.09296875,193.26289063)(95.63984375,192.69648438)
\curveto(95.0890625,192.00117188)(94.33710937,191.65351563)(93.38398437,191.65351563)
\curveto(92.60273437,191.65351563)(91.9640625,191.87226563)(91.46796875,192.30976563)
\curveto(90.97578125,192.74726563)(90.69453125,193.32734375)(90.62421875,194.05)
\closepath
}
}
{
\newrgbcolor{curcolor}{0 0 0}
\pscustom[linewidth=1,linecolor=curcolor]
{
\newpath
\moveto(105.1,241.8)
\lineto(114.1,241.8)
\moveto(575,241.8)
\lineto(566,241.8)
}
}
{
\newrgbcolor{curcolor}{0 0 0}
\pscustom[linestyle=none,fillstyle=solid,fillcolor=curcolor]
{
\newpath
\moveto(80.61640625,242.13632813)
\curveto(80.61640625,243.15195313)(80.71992187,243.96835938)(80.92695312,244.58554688)
\curveto(81.13789062,245.20664063)(81.4484375,245.68515625)(81.85859375,246.02109375)
\curveto(82.27265625,246.35703125)(82.7921875,246.525)(83.4171875,246.525)
\curveto(83.878125,246.525)(84.28242187,246.43125)(84.63007812,246.24375)
\curveto(84.97773437,246.06015625)(85.26484375,245.79257813)(85.49140625,245.44101563)
\curveto(85.71796875,245.09335938)(85.89570312,244.66757813)(86.02460937,244.16367188)
\curveto(86.15351562,243.66367188)(86.21796875,242.98789063)(86.21796875,242.13632813)
\curveto(86.21796875,241.12851563)(86.11445312,240.3140625)(85.90742187,239.69296875)
\curveto(85.70039062,239.07578125)(85.38984375,238.59726563)(84.97578125,238.25742188)
\curveto(84.565625,237.92148438)(84.04609375,237.75351563)(83.4171875,237.75351563)
\curveto(82.5890625,237.75351563)(81.93867187,238.05039063)(81.46601562,238.64414063)
\curveto(80.89960937,239.35898438)(80.61640625,240.52304688)(80.61640625,242.13632813)
\closepath
\moveto(81.70039062,242.13632813)
\curveto(81.70039062,240.72617188)(81.86445312,239.78671875)(82.19257812,239.31796875)
\curveto(82.52460937,238.853125)(82.9328125,238.62070313)(83.4171875,238.62070313)
\curveto(83.9015625,238.62070313)(84.3078125,238.85507813)(84.6359375,239.32382813)
\curveto(84.96796875,239.79257813)(85.13398437,240.73007813)(85.13398437,242.13632813)
\curveto(85.13398437,243.55039063)(84.96796875,244.48984375)(84.6359375,244.9546875)
\curveto(84.3078125,245.41953125)(83.89765625,245.65195313)(83.40546875,245.65195313)
\curveto(82.92109375,245.65195313)(82.534375,245.446875)(82.2453125,245.03671875)
\curveto(81.88203125,244.51328125)(81.70039062,243.54648438)(81.70039062,242.13632813)
\closepath
}
}
{
\newrgbcolor{curcolor}{0 0 0}
\pscustom[linestyle=none,fillstyle=solid,fillcolor=curcolor]
{
\newpath
\moveto(87.88203125,237.9)
\lineto(87.88203125,239.10117188)
\lineto(89.08320312,239.10117188)
\lineto(89.08320312,237.9)
\closepath
}
}
{
\newrgbcolor{curcolor}{0 0 0}
\pscustom[linestyle=none,fillstyle=solid,fillcolor=curcolor]
{
\newpath
\moveto(90.62421875,240.15)
\lineto(91.73164062,240.24375)
\curveto(91.81367187,239.7046875)(92.003125,239.2984375)(92.3,239.025)
\curveto(92.60078125,238.75546875)(92.96210937,238.62070313)(93.38398437,238.62070313)
\curveto(93.89179687,238.62070313)(94.32148437,238.81210938)(94.67304687,239.19492188)
\curveto(95.02460937,239.57773438)(95.20039062,240.08554688)(95.20039062,240.71835938)
\curveto(95.20039062,241.31992188)(95.03046875,241.79453125)(94.690625,242.1421875)
\curveto(94.3546875,242.48984375)(93.91328125,242.66367188)(93.36640625,242.66367188)
\curveto(93.0265625,242.66367188)(92.71992187,242.58554688)(92.44648437,242.42929688)
\curveto(92.17304687,242.27695313)(91.95820312,242.07773438)(91.80195312,241.83164063)
\lineto(90.81171875,241.96054688)
\lineto(91.64375,246.37265625)
\lineto(95.91523437,246.37265625)
\lineto(95.91523437,245.36484375)
\lineto(92.4875,245.36484375)
\lineto(92.02460937,243.05625)
\curveto(92.54023437,243.415625)(93.08125,243.5953125)(93.64765625,243.5953125)
\curveto(94.39765625,243.5953125)(95.03046875,243.33554688)(95.54609375,242.81601563)
\curveto(96.06171875,242.29648438)(96.31953125,241.62851563)(96.31953125,240.81210938)
\curveto(96.31953125,240.03476563)(96.09296875,239.36289063)(95.63984375,238.79648438)
\curveto(95.0890625,238.10117188)(94.33710937,237.75351563)(93.38398437,237.75351563)
\curveto(92.60273437,237.75351563)(91.9640625,237.97226563)(91.46796875,238.40976563)
\curveto(90.97578125,238.84726563)(90.69453125,239.42734375)(90.62421875,240.15)
\closepath
}
}
{
\newrgbcolor{curcolor}{0 0 0}
\pscustom[linewidth=1,linecolor=curcolor]
{
\newpath
\moveto(105.1,287.8)
\lineto(114.1,287.8)
\moveto(575,287.8)
\lineto(566,287.8)
}
}
{
\newrgbcolor{curcolor}{0 0 0}
\pscustom[linestyle=none,fillstyle=solid,fillcolor=curcolor]
{
\newpath
\moveto(94.596875,283.9)
\lineto(93.5421875,283.9)
\lineto(93.5421875,290.62070312)
\curveto(93.28828125,290.37851562)(92.95429687,290.13632812)(92.54023437,289.89414062)
\curveto(92.13007812,289.65195312)(91.7609375,289.4703125)(91.4328125,289.34921875)
\lineto(91.4328125,290.36875)
\curveto(92.02265625,290.64609375)(92.53828125,290.98203125)(92.9796875,291.3765625)
\curveto(93.42109375,291.77109375)(93.73359375,292.15390625)(93.9171875,292.525)
\lineto(94.596875,292.525)
\closepath
}
}
{
\newrgbcolor{curcolor}{0 0 0}
\pscustom[linewidth=1,linecolor=curcolor]
{
\newpath
\moveto(105.1,333.8)
\lineto(114.1,333.8)
\moveto(575,333.8)
\lineto(566,333.8)
}
}
{
\newrgbcolor{curcolor}{0 0 0}
\pscustom[linestyle=none,fillstyle=solid,fillcolor=curcolor]
{
\newpath
\moveto(96.1671875,330.91367187)
\lineto(96.1671875,329.9)
\lineto(90.48945312,329.9)
\curveto(90.48164062,330.15390625)(90.52265625,330.39804687)(90.6125,330.63242187)
\curveto(90.75703125,331.01914062)(90.9875,331.4)(91.30390625,331.775)
\curveto(91.62421875,332.15)(92.08515625,332.58359375)(92.68671875,333.07578125)
\curveto(93.6203125,333.84140625)(94.25117187,334.446875)(94.57929687,334.8921875)
\curveto(94.90742187,335.34140625)(95.07148437,335.76523437)(95.07148437,336.16367187)
\curveto(95.07148437,336.58164062)(94.92109375,336.93320312)(94.6203125,337.21835937)
\curveto(94.3234375,337.50742187)(93.93476562,337.65195312)(93.45429687,337.65195312)
\curveto(92.94648437,337.65195312)(92.54023437,337.49960937)(92.23554687,337.19492187)
\curveto(91.93085937,336.89023437)(91.7765625,336.46835937)(91.77265625,335.92929687)
\lineto(90.68867187,336.040625)
\curveto(90.76289062,336.84921875)(91.0421875,337.46445312)(91.5265625,337.88632812)
\curveto(92.0109375,338.31210937)(92.66132812,338.525)(93.47773437,338.525)
\curveto(94.30195312,338.525)(94.95429687,338.29648437)(95.43476562,337.83945312)
\curveto(95.91523437,337.38242187)(96.15546875,336.81601562)(96.15546875,336.14023437)
\curveto(96.15546875,335.79648437)(96.08515625,335.45859375)(95.94453125,335.1265625)
\curveto(95.80390625,334.79453125)(95.56953125,334.44492187)(95.24140625,334.07773437)
\curveto(94.9171875,333.71054687)(94.37617187,333.20664062)(93.61835937,332.56601562)
\curveto(92.98554687,332.03476562)(92.57929687,331.6734375)(92.39960937,331.48203125)
\curveto(92.21992187,331.29453125)(92.07148437,331.10507812)(91.95429687,330.91367187)
\closepath
}
}
{
\newrgbcolor{curcolor}{0 0 0}
\pscustom[linewidth=1,linecolor=curcolor]
{
\newpath
\moveto(105.1,379.9)
\lineto(114.1,379.9)
\moveto(575,379.9)
\lineto(566,379.9)
}
}
{
\newrgbcolor{curcolor}{0 0 0}
\pscustom[linestyle=none,fillstyle=solid,fillcolor=curcolor]
{
\newpath
\moveto(94.00507812,376)
\lineto(94.00507812,378.05664062)
\lineto(90.27851562,378.05664062)
\lineto(90.27851562,379.0234375)
\lineto(94.1984375,384.58984375)
\lineto(95.05976562,384.58984375)
\lineto(95.05976562,379.0234375)
\lineto(96.21992187,379.0234375)
\lineto(96.21992187,378.05664062)
\lineto(95.05976562,378.05664062)
\lineto(95.05976562,376)
\closepath
\moveto(94.00507812,379.0234375)
\lineto(94.00507812,382.89648438)
\lineto(91.315625,379.0234375)
\closepath
}
}
{
\newrgbcolor{curcolor}{0 0 0}
\pscustom[linewidth=1,linecolor=curcolor]
{
\newpath
\moveto(105.1,425.9)
\lineto(114.1,425.9)
\moveto(575,425.9)
\lineto(566,425.9)
}
}
{
\newrgbcolor{curcolor}{0 0 0}
\pscustom[linestyle=none,fillstyle=solid,fillcolor=curcolor]
{
\newpath
\moveto(92.24726562,426.65820312)
\curveto(91.80976562,426.81835938)(91.48554687,427.046875)(91.27460937,427.34375)
\curveto(91.06367187,427.640625)(90.95820312,427.99609375)(90.95820312,428.41015625)
\curveto(90.95820312,429.03515625)(91.1828125,429.56054688)(91.63203125,429.98632812)
\curveto(92.08125,430.41210938)(92.67890625,430.625)(93.425,430.625)
\curveto(94.175,430.625)(94.77851562,430.40625)(95.23554687,429.96875)
\curveto(95.69257812,429.53515625)(95.92109375,429.00585938)(95.92109375,428.38085938)
\curveto(95.92109375,427.98242188)(95.815625,427.63476562)(95.6046875,427.33789062)
\curveto(95.39765625,427.04492188)(95.08125,426.81835938)(94.65546875,426.65820312)
\curveto(95.1828125,426.48632812)(95.58320312,426.20898438)(95.85664062,425.82617188)
\curveto(96.13398437,425.44335938)(96.27265625,424.98632812)(96.27265625,424.45507812)
\curveto(96.27265625,423.72070312)(96.01289062,423.10351562)(95.49335937,422.60351562)
\curveto(94.97382812,422.10351562)(94.29023437,421.85351562)(93.44257812,421.85351562)
\curveto(92.59492187,421.85351562)(91.91132812,422.10351562)(91.39179687,422.60351562)
\curveto(90.87226562,423.10742188)(90.6125,423.734375)(90.6125,424.484375)
\curveto(90.6125,425.04296875)(90.753125,425.50976562)(91.034375,425.88476562)
\curveto(91.31953125,426.26367188)(91.72382812,426.52148438)(92.24726562,426.65820312)
\closepath
\moveto(92.03632812,428.4453125)
\curveto(92.03632812,428.0390625)(92.1671875,427.70703125)(92.42890625,427.44921875)
\curveto(92.690625,427.19140625)(93.03046875,427.0625)(93.4484375,427.0625)
\curveto(93.8546875,427.0625)(94.18671875,427.18945312)(94.44453125,427.44335938)
\curveto(94.70625,427.70117188)(94.83710937,428.015625)(94.83710937,428.38671875)
\curveto(94.83710937,428.7734375)(94.70234375,429.09765625)(94.4328125,429.359375)
\curveto(94.1671875,429.625)(93.83515625,429.7578125)(93.43671875,429.7578125)
\curveto(93.034375,429.7578125)(92.70039062,429.62890625)(92.43476562,429.37109375)
\curveto(92.16914062,429.11328125)(92.03632812,428.8046875)(92.03632812,428.4453125)
\closepath
\moveto(91.69648437,424.47851562)
\curveto(91.69648437,424.17773438)(91.76679687,423.88671875)(91.90742187,423.60546875)
\curveto(92.05195312,423.32421875)(92.26484375,423.10546875)(92.54609375,422.94921875)
\curveto(92.82734375,422.796875)(93.13007812,422.72070312)(93.45429687,422.72070312)
\curveto(93.95820312,422.72070312)(94.37421875,422.8828125)(94.70234375,423.20703125)
\curveto(95.03046875,423.53125)(95.19453125,423.94335938)(95.19453125,424.44335938)
\curveto(95.19453125,424.95117188)(95.02460937,425.37109375)(94.68476562,425.703125)
\curveto(94.34882812,426.03515625)(93.92695312,426.20117188)(93.41914062,426.20117188)
\curveto(92.92304687,426.20117188)(92.5109375,426.03710938)(92.1828125,425.70898438)
\curveto(91.85859375,425.38085938)(91.69648437,424.97070312)(91.69648437,424.47851562)
\closepath
}
}
{
\newrgbcolor{curcolor}{0 0 0}
\pscustom[linewidth=1,linecolor=curcolor]
{
\newpath
\moveto(105.1,57.6)
\lineto(105.1,66.6)
\moveto(105.1,425.9)
\lineto(105.1,416.9)
}
}
{
\newrgbcolor{curcolor}{0 0 0}
\pscustom[linestyle=none,fillstyle=solid,fillcolor=curcolor]
{
\newpath
\moveto(96.13222656,36.71367187)
\lineto(96.13222656,35.7)
\lineto(90.45449219,35.7)
\curveto(90.44667969,35.95390625)(90.48769531,36.19804687)(90.57753906,36.43242187)
\curveto(90.72207031,36.81914062)(90.95253906,37.2)(91.26894531,37.575)
\curveto(91.58925781,37.95)(92.05019531,38.38359375)(92.65175781,38.87578125)
\curveto(93.58535156,39.64140625)(94.21621094,40.246875)(94.54433594,40.6921875)
\curveto(94.87246094,41.14140625)(95.03652344,41.56523437)(95.03652344,41.96367187)
\curveto(95.03652344,42.38164062)(94.88613281,42.73320312)(94.58535156,43.01835937)
\curveto(94.28847656,43.30742187)(93.89980469,43.45195312)(93.41933594,43.45195312)
\curveto(92.91152344,43.45195312)(92.50527344,43.29960937)(92.20058594,42.99492187)
\curveto(91.89589844,42.69023437)(91.74160156,42.26835937)(91.73769531,41.72929687)
\lineto(90.65371094,41.840625)
\curveto(90.72792969,42.64921875)(91.00722656,43.26445312)(91.49160156,43.68632812)
\curveto(91.97597656,44.11210937)(92.62636719,44.325)(93.44277344,44.325)
\curveto(94.26699219,44.325)(94.91933594,44.09648437)(95.39980469,43.63945312)
\curveto(95.88027344,43.18242187)(96.12050781,42.61601562)(96.12050781,41.94023437)
\curveto(96.12050781,41.59648437)(96.05019531,41.25859375)(95.90957031,40.9265625)
\curveto(95.76894531,40.59453125)(95.53457031,40.24492187)(95.20644531,39.87773437)
\curveto(94.88222656,39.51054687)(94.34121094,39.00664062)(93.58339844,38.36601562)
\curveto(92.95058594,37.83476562)(92.54433594,37.4734375)(92.36464844,37.28203125)
\curveto(92.18496094,37.09453125)(92.03652344,36.90507812)(91.91933594,36.71367187)
\closepath
}
}
{
\newrgbcolor{curcolor}{0 0 0}
\pscustom[linestyle=none,fillstyle=solid,fillcolor=curcolor]
{
\newpath
\moveto(97.26308594,37.95)
\lineto(98.37050781,38.04375)
\curveto(98.45253906,37.5046875)(98.64199219,37.0984375)(98.93886719,36.825)
\curveto(99.23964844,36.55546875)(99.60097656,36.42070312)(100.02285156,36.42070312)
\curveto(100.53066406,36.42070312)(100.96035156,36.61210937)(101.31191406,36.99492187)
\curveto(101.66347656,37.37773437)(101.83925781,37.88554687)(101.83925781,38.51835937)
\curveto(101.83925781,39.11992187)(101.66933594,39.59453125)(101.32949219,39.9421875)
\curveto(100.99355469,40.28984375)(100.55214844,40.46367187)(100.00527344,40.46367187)
\curveto(99.66542969,40.46367187)(99.35878906,40.38554687)(99.08535156,40.22929687)
\curveto(98.81191406,40.07695312)(98.59707031,39.87773437)(98.44082031,39.63164062)
\lineto(97.45058594,39.76054687)
\lineto(98.28261719,44.17265625)
\lineto(102.55410156,44.17265625)
\lineto(102.55410156,43.16484375)
\lineto(99.12636719,43.16484375)
\lineto(98.66347656,40.85625)
\curveto(99.17910156,41.215625)(99.72011719,41.3953125)(100.28652344,41.3953125)
\curveto(101.03652344,41.3953125)(101.66933594,41.13554687)(102.18496094,40.61601562)
\curveto(102.70058594,40.09648437)(102.95839844,39.42851562)(102.95839844,38.61210937)
\curveto(102.95839844,37.83476562)(102.73183594,37.16289062)(102.27871094,36.59648437)
\curveto(101.72792969,35.90117187)(100.97597656,35.55351562)(100.02285156,35.55351562)
\curveto(99.24160156,35.55351562)(98.60292969,35.77226562)(98.10683594,36.20976562)
\curveto(97.61464844,36.64726562)(97.33339844,37.22734375)(97.26308594,37.95)
\closepath
}
}
{
\newrgbcolor{curcolor}{0 0 0}
\pscustom[linestyle=none,fillstyle=solid,fillcolor=curcolor]
{
\newpath
\moveto(103.93691406,39.93632812)
\curveto(103.93691406,40.95195312)(104.04042969,41.76835937)(104.24746094,42.38554687)
\curveto(104.45839844,43.00664062)(104.76894531,43.48515625)(105.17910156,43.82109375)
\curveto(105.59316406,44.15703125)(106.11269531,44.325)(106.73769531,44.325)
\curveto(107.19863281,44.325)(107.60292969,44.23125)(107.95058594,44.04375)
\curveto(108.29824219,43.86015625)(108.58535156,43.59257812)(108.81191406,43.24101562)
\curveto(109.03847656,42.89335937)(109.21621094,42.46757812)(109.34511719,41.96367187)
\curveto(109.47402344,41.46367187)(109.53847656,40.78789062)(109.53847656,39.93632812)
\curveto(109.53847656,38.92851562)(109.43496094,38.1140625)(109.22792969,37.49296875)
\curveto(109.02089844,36.87578125)(108.71035156,36.39726562)(108.29628906,36.05742187)
\curveto(107.88613281,35.72148437)(107.36660156,35.55351562)(106.73769531,35.55351562)
\curveto(105.90957031,35.55351562)(105.25917969,35.85039062)(104.78652344,36.44414062)
\curveto(104.22011719,37.15898437)(103.93691406,38.32304687)(103.93691406,39.93632812)
\closepath
\moveto(105.02089844,39.93632812)
\curveto(105.02089844,38.52617187)(105.18496094,37.58671875)(105.51308594,37.11796875)
\curveto(105.84511719,36.653125)(106.25332031,36.42070312)(106.73769531,36.42070312)
\curveto(107.22207031,36.42070312)(107.62832031,36.65507812)(107.95644531,37.12382812)
\curveto(108.28847656,37.59257812)(108.45449219,38.53007812)(108.45449219,39.93632812)
\curveto(108.45449219,41.35039062)(108.28847656,42.28984375)(107.95644531,42.7546875)
\curveto(107.62832031,43.21953125)(107.21816406,43.45195312)(106.72597656,43.45195312)
\curveto(106.24160156,43.45195312)(105.85488281,43.246875)(105.56582031,42.83671875)
\curveto(105.20253906,42.31328125)(105.02089844,41.34648437)(105.02089844,39.93632812)
\closepath
}
}
{
\newrgbcolor{curcolor}{0 0 0}
\pscustom[linestyle=none,fillstyle=solid,fillcolor=curcolor]
{
\newpath
\moveto(111.00332031,35.7)
\lineto(111.00332031,44.28984375)
\lineto(112.71425781,44.28984375)
\lineto(114.74746094,38.2078125)
\curveto(114.93496094,37.64140625)(115.07167969,37.21757812)(115.15761719,36.93632812)
\curveto(115.25527344,37.24882812)(115.40761719,37.7078125)(115.61464844,38.31328125)
\lineto(117.67128906,44.28984375)
\lineto(119.20058594,44.28984375)
\lineto(119.20058594,35.7)
\lineto(118.10488281,35.7)
\lineto(118.10488281,42.88945312)
\lineto(115.60878906,35.7)
\lineto(114.58339844,35.7)
\lineto(112.09902344,43.0125)
\lineto(112.09902344,35.7)
\closepath
}
}
{
\newrgbcolor{curcolor}{0 0 0}
\pscustom[linewidth=1,linecolor=curcolor]
{
\newpath
\moveto(183.4,57.6)
\lineto(183.4,66.6)
\moveto(183.4,425.9)
\lineto(183.4,416.9)
}
}
{
\newrgbcolor{curcolor}{0 0 0}
\pscustom[linestyle=none,fillstyle=solid,fillcolor=curcolor]
{
\newpath
\moveto(168.88925781,37.95)
\lineto(169.99667969,38.04375)
\curveto(170.07871094,37.5046875)(170.26816406,37.0984375)(170.56503906,36.825)
\curveto(170.86582031,36.55546875)(171.22714844,36.42070312)(171.64902344,36.42070312)
\curveto(172.15683594,36.42070312)(172.58652344,36.61210937)(172.93808594,36.99492187)
\curveto(173.28964844,37.37773437)(173.46542969,37.88554687)(173.46542969,38.51835937)
\curveto(173.46542969,39.11992187)(173.29550781,39.59453125)(172.95566406,39.9421875)
\curveto(172.61972656,40.28984375)(172.17832031,40.46367187)(171.63144531,40.46367187)
\curveto(171.29160156,40.46367187)(170.98496094,40.38554687)(170.71152344,40.22929687)
\curveto(170.43808594,40.07695312)(170.22324219,39.87773437)(170.06699219,39.63164062)
\lineto(169.07675781,39.76054687)
\lineto(169.90878906,44.17265625)
\lineto(174.18027344,44.17265625)
\lineto(174.18027344,43.16484375)
\lineto(170.75253906,43.16484375)
\lineto(170.28964844,40.85625)
\curveto(170.80527344,41.215625)(171.34628906,41.3953125)(171.91269531,41.3953125)
\curveto(172.66269531,41.3953125)(173.29550781,41.13554687)(173.81113281,40.61601562)
\curveto(174.32675781,40.09648437)(174.58457031,39.42851562)(174.58457031,38.61210937)
\curveto(174.58457031,37.83476562)(174.35800781,37.16289062)(173.90488281,36.59648437)
\curveto(173.35410156,35.90117187)(172.60214844,35.55351562)(171.64902344,35.55351562)
\curveto(170.86777344,35.55351562)(170.22910156,35.77226562)(169.73300781,36.20976562)
\curveto(169.24082031,36.64726562)(168.95957031,37.22734375)(168.88925781,37.95)
\closepath
}
}
{
\newrgbcolor{curcolor}{0 0 0}
\pscustom[linestyle=none,fillstyle=solid,fillcolor=curcolor]
{
\newpath
\moveto(175.56308594,39.93632812)
\curveto(175.56308594,40.95195312)(175.66660156,41.76835937)(175.87363281,42.38554687)
\curveto(176.08457031,43.00664062)(176.39511719,43.48515625)(176.80527344,43.82109375)
\curveto(177.21933594,44.15703125)(177.73886719,44.325)(178.36386719,44.325)
\curveto(178.82480469,44.325)(179.22910156,44.23125)(179.57675781,44.04375)
\curveto(179.92441406,43.86015625)(180.21152344,43.59257812)(180.43808594,43.24101562)
\curveto(180.66464844,42.89335937)(180.84238281,42.46757812)(180.97128906,41.96367187)
\curveto(181.10019531,41.46367187)(181.16464844,40.78789062)(181.16464844,39.93632812)
\curveto(181.16464844,38.92851562)(181.06113281,38.1140625)(180.85410156,37.49296875)
\curveto(180.64707031,36.87578125)(180.33652344,36.39726562)(179.92246094,36.05742187)
\curveto(179.51230469,35.72148437)(178.99277344,35.55351562)(178.36386719,35.55351562)
\curveto(177.53574219,35.55351562)(176.88535156,35.85039062)(176.41269531,36.44414062)
\curveto(175.84628906,37.15898437)(175.56308594,38.32304687)(175.56308594,39.93632812)
\closepath
\moveto(176.64707031,39.93632812)
\curveto(176.64707031,38.52617187)(176.81113281,37.58671875)(177.13925781,37.11796875)
\curveto(177.47128906,36.653125)(177.87949219,36.42070312)(178.36386719,36.42070312)
\curveto(178.84824219,36.42070312)(179.25449219,36.65507812)(179.58261719,37.12382812)
\curveto(179.91464844,37.59257812)(180.08066406,38.53007812)(180.08066406,39.93632812)
\curveto(180.08066406,41.35039062)(179.91464844,42.28984375)(179.58261719,42.7546875)
\curveto(179.25449219,43.21953125)(178.84433594,43.45195312)(178.35214844,43.45195312)
\curveto(177.86777344,43.45195312)(177.48105469,43.246875)(177.19199219,42.83671875)
\curveto(176.82871094,42.31328125)(176.64707031,41.34648437)(176.64707031,39.93632812)
\closepath
}
}
{
\newrgbcolor{curcolor}{0 0 0}
\pscustom[linestyle=none,fillstyle=solid,fillcolor=curcolor]
{
\newpath
\moveto(182.23691406,39.93632812)
\curveto(182.23691406,40.95195312)(182.34042969,41.76835937)(182.54746094,42.38554687)
\curveto(182.75839844,43.00664062)(183.06894531,43.48515625)(183.47910156,43.82109375)
\curveto(183.89316406,44.15703125)(184.41269531,44.325)(185.03769531,44.325)
\curveto(185.49863281,44.325)(185.90292969,44.23125)(186.25058594,44.04375)
\curveto(186.59824219,43.86015625)(186.88535156,43.59257812)(187.11191406,43.24101562)
\curveto(187.33847656,42.89335937)(187.51621094,42.46757812)(187.64511719,41.96367187)
\curveto(187.77402344,41.46367187)(187.83847656,40.78789062)(187.83847656,39.93632812)
\curveto(187.83847656,38.92851562)(187.73496094,38.1140625)(187.52792969,37.49296875)
\curveto(187.32089844,36.87578125)(187.01035156,36.39726562)(186.59628906,36.05742187)
\curveto(186.18613281,35.72148437)(185.66660156,35.55351562)(185.03769531,35.55351562)
\curveto(184.20957031,35.55351562)(183.55917969,35.85039062)(183.08652344,36.44414062)
\curveto(182.52011719,37.15898437)(182.23691406,38.32304687)(182.23691406,39.93632812)
\closepath
\moveto(183.32089844,39.93632812)
\curveto(183.32089844,38.52617187)(183.48496094,37.58671875)(183.81308594,37.11796875)
\curveto(184.14511719,36.653125)(184.55332031,36.42070312)(185.03769531,36.42070312)
\curveto(185.52207031,36.42070312)(185.92832031,36.65507812)(186.25644531,37.12382812)
\curveto(186.58847656,37.59257812)(186.75449219,38.53007812)(186.75449219,39.93632812)
\curveto(186.75449219,41.35039062)(186.58847656,42.28984375)(186.25644531,42.7546875)
\curveto(185.92832031,43.21953125)(185.51816406,43.45195312)(185.02597656,43.45195312)
\curveto(184.54160156,43.45195312)(184.15488281,43.246875)(183.86582031,42.83671875)
\curveto(183.50253906,42.31328125)(183.32089844,41.34648437)(183.32089844,39.93632812)
\closepath
}
}
{
\newrgbcolor{curcolor}{0 0 0}
\pscustom[linestyle=none,fillstyle=solid,fillcolor=curcolor]
{
\newpath
\moveto(189.30332031,35.7)
\lineto(189.30332031,44.28984375)
\lineto(191.01425781,44.28984375)
\lineto(193.04746094,38.2078125)
\curveto(193.23496094,37.64140625)(193.37167969,37.21757812)(193.45761719,36.93632812)
\curveto(193.55527344,37.24882812)(193.70761719,37.7078125)(193.91464844,38.31328125)
\lineto(195.97128906,44.28984375)
\lineto(197.50058594,44.28984375)
\lineto(197.50058594,35.7)
\lineto(196.40488281,35.7)
\lineto(196.40488281,42.88945312)
\lineto(193.90878906,35.7)
\lineto(192.88339844,35.7)
\lineto(190.39902344,43.0125)
\lineto(190.39902344,35.7)
\closepath
}
}
{
\newrgbcolor{curcolor}{0 0 0}
\pscustom[linewidth=1,linecolor=curcolor]
{
\newpath
\moveto(261.7,57.6)
\lineto(261.7,66.6)
\moveto(261.7,425.9)
\lineto(261.7,416.9)
}
}
{
\newrgbcolor{curcolor}{0 0 0}
\pscustom[linestyle=none,fillstyle=solid,fillcolor=curcolor]
{
\newpath
\moveto(258.16679687,35.7)
\lineto(257.11210937,35.7)
\lineto(257.11210937,42.42070312)
\curveto(256.85820312,42.17851562)(256.52421875,41.93632812)(256.11015625,41.69414062)
\curveto(255.7,41.45195312)(255.33085937,41.2703125)(255.00273437,41.14921875)
\lineto(255.00273437,42.16875)
\curveto(255.59257812,42.44609375)(256.10820312,42.78203125)(256.54960937,43.1765625)
\curveto(256.99101562,43.57109375)(257.30351562,43.95390625)(257.48710937,44.325)
\lineto(258.16679687,44.325)
\closepath
}
}
{
\newrgbcolor{curcolor}{0 0 0}
\pscustom[linestyle=none,fillstyle=solid,fillcolor=curcolor]
{
\newpath
\moveto(265.31523437,39.06914062)
\lineto(265.31523437,40.07695312)
\lineto(268.95390625,40.0828125)
\lineto(268.95390625,36.8953125)
\curveto(268.3953125,36.45)(267.81914062,36.1140625)(267.22539062,35.8875)
\curveto(266.63164062,35.66484375)(266.02226562,35.55351562)(265.39726562,35.55351562)
\curveto(264.55351562,35.55351562)(263.7859375,35.73320312)(263.09453125,36.09257812)
\curveto(262.40703125,36.45585937)(261.8875,36.97929687)(261.5359375,37.66289062)
\curveto(261.184375,38.34648437)(261.00859375,39.11015625)(261.00859375,39.95390625)
\curveto(261.00859375,40.78984375)(261.18242187,41.56914062)(261.53007812,42.29179687)
\curveto(261.88164062,43.01835937)(262.38554687,43.55742187)(263.04179687,43.90898437)
\curveto(263.69804687,44.26054687)(264.45390625,44.43632812)(265.309375,44.43632812)
\curveto(265.93046875,44.43632812)(266.49101562,44.33476562)(266.99101562,44.13164062)
\curveto(267.49492187,43.93242187)(267.88945312,43.653125)(268.17460937,43.29375)
\curveto(268.45976562,42.934375)(268.6765625,42.465625)(268.825,41.8875)
\lineto(267.79960937,41.60625)
\curveto(267.67070312,42.04375)(267.51054687,42.3875)(267.31914062,42.6375)
\curveto(267.12773437,42.8875)(266.85429687,43.08671875)(266.49882812,43.23515625)
\curveto(266.14335937,43.3875)(265.74882812,43.46367187)(265.31523437,43.46367187)
\curveto(264.79570312,43.46367187)(264.34648437,43.38359375)(263.96757812,43.2234375)
\curveto(263.58867187,43.0671875)(263.28203125,42.86015625)(263.04765625,42.60234375)
\curveto(262.8171875,42.34453125)(262.6375,42.06132812)(262.50859375,41.75273437)
\curveto(262.28984375,41.22148437)(262.18046875,40.6453125)(262.18046875,40.02421875)
\curveto(262.18046875,39.25859375)(262.31132812,38.61796875)(262.57304687,38.10234375)
\curveto(262.83867187,37.58671875)(263.2234375,37.20390625)(263.72734375,36.95390625)
\curveto(264.23125,36.70390625)(264.76640625,36.57890625)(265.3328125,36.57890625)
\curveto(265.825,36.57890625)(266.30546875,36.67265625)(266.77421875,36.86015625)
\curveto(267.24296875,37.0515625)(267.5984375,37.2546875)(267.840625,37.46953125)
\lineto(267.840625,39.06914062)
\closepath
}
}
{
\newrgbcolor{curcolor}{0 0 0}
\pscustom[linewidth=1,linecolor=curcolor]
{
\newpath
\moveto(340.1,57.6)
\lineto(340.1,66.6)
\moveto(340.1,425.9)
\lineto(340.1,416.9)
}
}
{
\newrgbcolor{curcolor}{0 0 0}
\pscustom[linestyle=none,fillstyle=solid,fillcolor=curcolor]
{
\newpath
\moveto(338.13710938,36.71367187)
\lineto(338.13710938,35.7)
\lineto(332.459375,35.7)
\curveto(332.4515625,35.95390625)(332.49257813,36.19804687)(332.58242188,36.43242187)
\curveto(332.72695313,36.81914062)(332.95742188,37.2)(333.27382813,37.575)
\curveto(333.59414063,37.95)(334.05507813,38.38359375)(334.65664063,38.87578125)
\curveto(335.59023438,39.64140625)(336.22109375,40.246875)(336.54921875,40.6921875)
\curveto(336.87734375,41.14140625)(337.04140625,41.56523437)(337.04140625,41.96367187)
\curveto(337.04140625,42.38164062)(336.89101563,42.73320312)(336.59023438,43.01835937)
\curveto(336.29335938,43.30742187)(335.9046875,43.45195312)(335.42421875,43.45195312)
\curveto(334.91640625,43.45195312)(334.51015625,43.29960937)(334.20546875,42.99492187)
\curveto(333.90078125,42.69023437)(333.74648438,42.26835937)(333.74257813,41.72929687)
\lineto(332.65859375,41.840625)
\curveto(332.7328125,42.64921875)(333.01210938,43.26445312)(333.49648438,43.68632812)
\curveto(333.98085938,44.11210937)(334.63125,44.325)(335.44765625,44.325)
\curveto(336.271875,44.325)(336.92421875,44.09648437)(337.4046875,43.63945312)
\curveto(337.88515625,43.18242187)(338.12539063,42.61601562)(338.12539063,41.94023437)
\curveto(338.12539063,41.59648437)(338.05507813,41.25859375)(337.91445313,40.9265625)
\curveto(337.77382813,40.59453125)(337.53945313,40.24492187)(337.21132813,39.87773437)
\curveto(336.88710938,39.51054687)(336.34609375,39.00664062)(335.58828125,38.36601562)
\curveto(334.95546875,37.83476562)(334.54921875,37.4734375)(334.36953125,37.28203125)
\curveto(334.18984375,37.09453125)(334.04140625,36.90507812)(333.92421875,36.71367187)
\closepath
}
}
{
\newrgbcolor{curcolor}{0 0 0}
\pscustom[linestyle=none,fillstyle=solid,fillcolor=curcolor]
{
\newpath
\moveto(343.71523438,39.06914062)
\lineto(343.71523438,40.07695312)
\lineto(347.35390625,40.0828125)
\lineto(347.35390625,36.8953125)
\curveto(346.7953125,36.45)(346.21914063,36.1140625)(345.62539063,35.8875)
\curveto(345.03164063,35.66484375)(344.42226563,35.55351562)(343.79726563,35.55351562)
\curveto(342.95351563,35.55351562)(342.1859375,35.73320312)(341.49453125,36.09257812)
\curveto(340.80703125,36.45585937)(340.2875,36.97929687)(339.9359375,37.66289062)
\curveto(339.584375,38.34648437)(339.40859375,39.11015625)(339.40859375,39.95390625)
\curveto(339.40859375,40.78984375)(339.58242188,41.56914062)(339.93007813,42.29179687)
\curveto(340.28164063,43.01835937)(340.78554688,43.55742187)(341.44179688,43.90898437)
\curveto(342.09804688,44.26054687)(342.85390625,44.43632812)(343.709375,44.43632812)
\curveto(344.33046875,44.43632812)(344.89101563,44.33476562)(345.39101563,44.13164062)
\curveto(345.89492188,43.93242187)(346.28945313,43.653125)(346.57460938,43.29375)
\curveto(346.85976563,42.934375)(347.0765625,42.465625)(347.225,41.8875)
\lineto(346.19960938,41.60625)
\curveto(346.07070313,42.04375)(345.91054688,42.3875)(345.71914063,42.6375)
\curveto(345.52773438,42.8875)(345.25429688,43.08671875)(344.89882813,43.23515625)
\curveto(344.54335938,43.3875)(344.14882813,43.46367187)(343.71523438,43.46367187)
\curveto(343.19570313,43.46367187)(342.74648438,43.38359375)(342.36757813,43.2234375)
\curveto(341.98867188,43.0671875)(341.68203125,42.86015625)(341.44765625,42.60234375)
\curveto(341.2171875,42.34453125)(341.0375,42.06132812)(340.90859375,41.75273437)
\curveto(340.68984375,41.22148437)(340.58046875,40.6453125)(340.58046875,40.02421875)
\curveto(340.58046875,39.25859375)(340.71132813,38.61796875)(340.97304688,38.10234375)
\curveto(341.23867188,37.58671875)(341.6234375,37.20390625)(342.12734375,36.95390625)
\curveto(342.63125,36.70390625)(343.16640625,36.57890625)(343.7328125,36.57890625)
\curveto(344.225,36.57890625)(344.70546875,36.67265625)(345.17421875,36.86015625)
\curveto(345.64296875,37.0515625)(345.9984375,37.2546875)(346.240625,37.46953125)
\lineto(346.240625,39.06914062)
\closepath
}
}
{
\newrgbcolor{curcolor}{0 0 0}
\pscustom[linewidth=1,linecolor=curcolor]
{
\newpath
\moveto(418.4,57.6)
\lineto(418.4,66.6)
\moveto(418.4,425.9)
\lineto(418.4,416.9)
}
}
{
\newrgbcolor{curcolor}{0 0 0}
\pscustom[linestyle=none,fillstyle=solid,fillcolor=curcolor]
{
\newpath
\moveto(414.275,35.7)
\lineto(414.275,37.75664062)
\lineto(410.5484375,37.75664062)
\lineto(410.5484375,38.7234375)
\lineto(414.46835937,44.28984375)
\lineto(415.3296875,44.28984375)
\lineto(415.3296875,38.7234375)
\lineto(416.48984375,38.7234375)
\lineto(416.48984375,37.75664062)
\lineto(415.3296875,37.75664062)
\lineto(415.3296875,35.7)
\closepath
\moveto(414.275,38.7234375)
\lineto(414.275,42.59648437)
\lineto(411.58554687,38.7234375)
\closepath
}
}
{
\newrgbcolor{curcolor}{0 0 0}
\pscustom[linestyle=none,fillstyle=solid,fillcolor=curcolor]
{
\newpath
\moveto(422.01523437,39.06914062)
\lineto(422.01523437,40.07695312)
\lineto(425.65390625,40.0828125)
\lineto(425.65390625,36.8953125)
\curveto(425.0953125,36.45)(424.51914062,36.1140625)(423.92539062,35.8875)
\curveto(423.33164062,35.66484375)(422.72226562,35.55351562)(422.09726562,35.55351562)
\curveto(421.25351562,35.55351562)(420.4859375,35.73320312)(419.79453125,36.09257812)
\curveto(419.10703125,36.45585937)(418.5875,36.97929687)(418.2359375,37.66289062)
\curveto(417.884375,38.34648437)(417.70859375,39.11015625)(417.70859375,39.95390625)
\curveto(417.70859375,40.78984375)(417.88242187,41.56914062)(418.23007812,42.29179687)
\curveto(418.58164062,43.01835937)(419.08554687,43.55742187)(419.74179687,43.90898437)
\curveto(420.39804687,44.26054687)(421.15390625,44.43632812)(422.009375,44.43632812)
\curveto(422.63046875,44.43632812)(423.19101562,44.33476562)(423.69101562,44.13164062)
\curveto(424.19492187,43.93242187)(424.58945312,43.653125)(424.87460937,43.29375)
\curveto(425.15976562,42.934375)(425.3765625,42.465625)(425.525,41.8875)
\lineto(424.49960937,41.60625)
\curveto(424.37070312,42.04375)(424.21054687,42.3875)(424.01914062,42.6375)
\curveto(423.82773437,42.8875)(423.55429687,43.08671875)(423.19882812,43.23515625)
\curveto(422.84335937,43.3875)(422.44882812,43.46367187)(422.01523437,43.46367187)
\curveto(421.49570312,43.46367187)(421.04648437,43.38359375)(420.66757812,43.2234375)
\curveto(420.28867187,43.0671875)(419.98203125,42.86015625)(419.74765625,42.60234375)
\curveto(419.5171875,42.34453125)(419.3375,42.06132812)(419.20859375,41.75273437)
\curveto(418.98984375,41.22148437)(418.88046875,40.6453125)(418.88046875,40.02421875)
\curveto(418.88046875,39.25859375)(419.01132812,38.61796875)(419.27304687,38.10234375)
\curveto(419.53867187,37.58671875)(419.9234375,37.20390625)(420.42734375,36.95390625)
\curveto(420.93125,36.70390625)(421.46640625,36.57890625)(422.0328125,36.57890625)
\curveto(422.525,36.57890625)(423.00546875,36.67265625)(423.47421875,36.86015625)
\curveto(423.94296875,37.0515625)(424.2984375,37.2546875)(424.540625,37.46953125)
\lineto(424.540625,39.06914062)
\closepath
}
}
{
\newrgbcolor{curcolor}{0 0 0}
\pscustom[linewidth=1,linecolor=curcolor]
{
\newpath
\moveto(496.7,57.6)
\lineto(496.7,66.6)
\moveto(496.7,425.9)
\lineto(496.7,416.9)
}
}
{
\newrgbcolor{curcolor}{0 0 0}
\pscustom[linestyle=none,fillstyle=solid,fillcolor=curcolor]
{
\newpath
\moveto(490.8171875,40.35820312)
\curveto(490.3796875,40.51835937)(490.05546875,40.746875)(489.84453125,41.04375)
\curveto(489.63359375,41.340625)(489.528125,41.69609375)(489.528125,42.11015625)
\curveto(489.528125,42.73515625)(489.75273437,43.26054687)(490.20195312,43.68632812)
\curveto(490.65117187,44.11210937)(491.24882812,44.325)(491.99492187,44.325)
\curveto(492.74492187,44.325)(493.3484375,44.10625)(493.80546875,43.66875)
\curveto(494.2625,43.23515625)(494.49101562,42.70585937)(494.49101562,42.08085937)
\curveto(494.49101562,41.68242187)(494.38554687,41.33476562)(494.17460937,41.03789062)
\curveto(493.96757812,40.74492187)(493.65117187,40.51835937)(493.22539062,40.35820312)
\curveto(493.75273437,40.18632812)(494.153125,39.90898437)(494.4265625,39.52617187)
\curveto(494.70390625,39.14335937)(494.84257812,38.68632812)(494.84257812,38.15507812)
\curveto(494.84257812,37.42070312)(494.5828125,36.80351562)(494.06328125,36.30351562)
\curveto(493.54375,35.80351562)(492.86015625,35.55351562)(492.0125,35.55351562)
\curveto(491.16484375,35.55351562)(490.48125,35.80351562)(489.96171875,36.30351562)
\curveto(489.4421875,36.80742187)(489.18242187,37.434375)(489.18242187,38.184375)
\curveto(489.18242187,38.74296875)(489.32304687,39.20976562)(489.60429687,39.58476562)
\curveto(489.88945312,39.96367187)(490.29375,40.22148437)(490.8171875,40.35820312)
\closepath
\moveto(490.60625,42.1453125)
\curveto(490.60625,41.7390625)(490.73710937,41.40703125)(490.99882812,41.14921875)
\curveto(491.26054687,40.89140625)(491.60039062,40.7625)(492.01835937,40.7625)
\curveto(492.42460937,40.7625)(492.75664062,40.88945312)(493.01445312,41.14335937)
\curveto(493.27617187,41.40117187)(493.40703125,41.715625)(493.40703125,42.08671875)
\curveto(493.40703125,42.4734375)(493.27226562,42.79765625)(493.00273437,43.059375)
\curveto(492.73710937,43.325)(492.40507812,43.4578125)(492.00664062,43.4578125)
\curveto(491.60429687,43.4578125)(491.2703125,43.32890625)(491.0046875,43.07109375)
\curveto(490.7390625,42.81328125)(490.60625,42.5046875)(490.60625,42.1453125)
\closepath
\moveto(490.26640625,38.17851562)
\curveto(490.26640625,37.87773437)(490.33671875,37.58671875)(490.47734375,37.30546875)
\curveto(490.621875,37.02421875)(490.83476562,36.80546875)(491.11601562,36.64921875)
\curveto(491.39726562,36.496875)(491.7,36.42070312)(492.02421875,36.42070312)
\curveto(492.528125,36.42070312)(492.94414062,36.5828125)(493.27226562,36.90703125)
\curveto(493.60039062,37.23125)(493.76445312,37.64335937)(493.76445312,38.14335937)
\curveto(493.76445312,38.65117187)(493.59453125,39.07109375)(493.2546875,39.403125)
\curveto(492.91875,39.73515625)(492.496875,39.90117187)(491.9890625,39.90117187)
\curveto(491.49296875,39.90117187)(491.08085937,39.73710937)(490.75273437,39.40898437)
\curveto(490.42851562,39.08085937)(490.26640625,38.67070312)(490.26640625,38.17851562)
\closepath
}
}
{
\newrgbcolor{curcolor}{0 0 0}
\pscustom[linestyle=none,fillstyle=solid,fillcolor=curcolor]
{
\newpath
\moveto(500.31523437,39.06914062)
\lineto(500.31523437,40.07695312)
\lineto(503.95390625,40.0828125)
\lineto(503.95390625,36.8953125)
\curveto(503.3953125,36.45)(502.81914062,36.1140625)(502.22539062,35.8875)
\curveto(501.63164062,35.66484375)(501.02226562,35.55351562)(500.39726562,35.55351562)
\curveto(499.55351562,35.55351562)(498.7859375,35.73320312)(498.09453125,36.09257812)
\curveto(497.40703125,36.45585937)(496.8875,36.97929687)(496.5359375,37.66289062)
\curveto(496.184375,38.34648437)(496.00859375,39.11015625)(496.00859375,39.95390625)
\curveto(496.00859375,40.78984375)(496.18242187,41.56914062)(496.53007812,42.29179687)
\curveto(496.88164062,43.01835937)(497.38554687,43.55742187)(498.04179687,43.90898437)
\curveto(498.69804687,44.26054687)(499.45390625,44.43632812)(500.309375,44.43632812)
\curveto(500.93046875,44.43632812)(501.49101562,44.33476562)(501.99101562,44.13164062)
\curveto(502.49492187,43.93242187)(502.88945312,43.653125)(503.17460937,43.29375)
\curveto(503.45976562,42.934375)(503.6765625,42.465625)(503.825,41.8875)
\lineto(502.79960937,41.60625)
\curveto(502.67070312,42.04375)(502.51054687,42.3875)(502.31914062,42.6375)
\curveto(502.12773437,42.8875)(501.85429687,43.08671875)(501.49882812,43.23515625)
\curveto(501.14335937,43.3875)(500.74882812,43.46367187)(500.31523437,43.46367187)
\curveto(499.79570312,43.46367187)(499.34648437,43.38359375)(498.96757812,43.2234375)
\curveto(498.58867187,43.0671875)(498.28203125,42.86015625)(498.04765625,42.60234375)
\curveto(497.8171875,42.34453125)(497.6375,42.06132812)(497.50859375,41.75273437)
\curveto(497.28984375,41.22148437)(497.18046875,40.6453125)(497.18046875,40.02421875)
\curveto(497.18046875,39.25859375)(497.31132812,38.61796875)(497.57304687,38.10234375)
\curveto(497.83867187,37.58671875)(498.2234375,37.20390625)(498.72734375,36.95390625)
\curveto(499.23125,36.70390625)(499.76640625,36.57890625)(500.3328125,36.57890625)
\curveto(500.825,36.57890625)(501.30546875,36.67265625)(501.77421875,36.86015625)
\curveto(502.24296875,37.0515625)(502.5984375,37.2546875)(502.840625,37.46953125)
\lineto(502.840625,39.06914062)
\closepath
}
}
{
\newrgbcolor{curcolor}{0 0 0}
\pscustom[linewidth=1,linecolor=curcolor]
{
\newpath
\moveto(575,57.6)
\lineto(575,66.6)
\moveto(575,425.9)
\lineto(575,416.9)
}
}
{
\newrgbcolor{curcolor}{0 0 0}
\pscustom[linestyle=none,fillstyle=solid,fillcolor=curcolor]
{
\newpath
\moveto(568.12988281,35.7)
\lineto(567.07519531,35.7)
\lineto(567.07519531,42.42070312)
\curveto(566.82128906,42.17851562)(566.48730469,41.93632812)(566.07324219,41.69414062)
\curveto(565.66308594,41.45195312)(565.29394531,41.2703125)(564.96582031,41.14921875)
\lineto(564.96582031,42.16875)
\curveto(565.55566406,42.44609375)(566.07128906,42.78203125)(566.51269531,43.1765625)
\curveto(566.95410156,43.57109375)(567.26660156,43.95390625)(567.45019531,44.325)
\lineto(568.12988281,44.325)
\closepath
}
}
{
\newrgbcolor{curcolor}{0 0 0}
\pscustom[linestyle=none,fillstyle=solid,fillcolor=curcolor]
{
\newpath
\moveto(576.30371094,42.18632812)
\lineto(575.25488281,42.10429687)
\curveto(575.16113281,42.51835937)(575.02832031,42.81914062)(574.85644531,43.00664062)
\curveto(574.57128906,43.30742187)(574.21972656,43.4578125)(573.80175781,43.4578125)
\curveto(573.46582031,43.4578125)(573.17089844,43.3640625)(572.91699219,43.1765625)
\curveto(572.58496094,42.934375)(572.32324219,42.58085937)(572.13183594,42.11601562)
\curveto(571.94042969,41.65117187)(571.84082031,40.9890625)(571.83300781,40.1296875)
\curveto(572.08691406,40.51640625)(572.39746094,40.80351562)(572.76464844,40.99101562)
\curveto(573.13183594,41.17851562)(573.51660156,41.27226562)(573.91894531,41.27226562)
\curveto(574.62207031,41.27226562)(575.21972656,41.0125)(575.71191406,40.49296875)
\curveto(576.20800781,39.97734375)(576.45605469,39.309375)(576.45605469,38.4890625)
\curveto(576.45605469,37.95)(576.33886719,37.44804687)(576.10449219,36.98320312)
\curveto(575.87402344,36.52226562)(575.55566406,36.16875)(575.14941406,35.92265625)
\curveto(574.74316406,35.6765625)(574.28222656,35.55351562)(573.76660156,35.55351562)
\curveto(572.88769531,35.55351562)(572.17089844,35.87578125)(571.61621094,36.5203125)
\curveto(571.06152344,37.16875)(570.78417969,38.23515625)(570.78417969,39.71953125)
\curveto(570.78417969,41.3796875)(571.09082031,42.58671875)(571.70410156,43.340625)
\curveto(572.23925781,43.996875)(572.95996094,44.325)(573.86621094,44.325)
\curveto(574.54199219,44.325)(575.09472656,44.13554687)(575.52441406,43.75664062)
\curveto(575.95800781,43.37773437)(576.21777344,42.85429687)(576.30371094,42.18632812)
\closepath
\moveto(571.99707031,38.48320312)
\curveto(571.99707031,38.11992187)(572.07324219,37.77226562)(572.22558594,37.44023437)
\curveto(572.38183594,37.10820312)(572.59863281,36.85429687)(572.87597656,36.67851562)
\curveto(573.15332031,36.50664062)(573.44433594,36.42070312)(573.74902344,36.42070312)
\curveto(574.19433594,36.42070312)(574.57714844,36.60039062)(574.89746094,36.95976562)
\curveto(575.21777344,37.31914062)(575.37792969,37.80742187)(575.37792969,38.42460937)
\curveto(575.37792969,39.01835937)(575.21972656,39.48515625)(574.90332031,39.825)
\curveto(574.58691406,40.16875)(574.18847656,40.340625)(573.70800781,40.340625)
\curveto(573.23144531,40.340625)(572.82714844,40.16875)(572.49511719,39.825)
\curveto(572.16308594,39.48515625)(571.99707031,39.03789062)(571.99707031,38.48320312)
\closepath
}
}
{
\newrgbcolor{curcolor}{0 0 0}
\pscustom[linestyle=none,fillstyle=solid,fillcolor=curcolor]
{
\newpath
\moveto(581.95214844,39.06914062)
\lineto(581.95214844,40.07695312)
\lineto(585.59082031,40.0828125)
\lineto(585.59082031,36.8953125)
\curveto(585.03222656,36.45)(584.45605469,36.1140625)(583.86230469,35.8875)
\curveto(583.26855469,35.66484375)(582.65917969,35.55351562)(582.03417969,35.55351562)
\curveto(581.19042969,35.55351562)(580.42285156,35.73320312)(579.73144531,36.09257812)
\curveto(579.04394531,36.45585937)(578.52441406,36.97929687)(578.17285156,37.66289062)
\curveto(577.82128906,38.34648437)(577.64550781,39.11015625)(577.64550781,39.95390625)
\curveto(577.64550781,40.78984375)(577.81933594,41.56914062)(578.16699219,42.29179687)
\curveto(578.51855469,43.01835937)(579.02246094,43.55742187)(579.67871094,43.90898437)
\curveto(580.33496094,44.26054687)(581.09082031,44.43632812)(581.94628906,44.43632812)
\curveto(582.56738281,44.43632812)(583.12792969,44.33476562)(583.62792969,44.13164062)
\curveto(584.13183594,43.93242187)(584.52636719,43.653125)(584.81152344,43.29375)
\curveto(585.09667969,42.934375)(585.31347656,42.465625)(585.46191406,41.8875)
\lineto(584.43652344,41.60625)
\curveto(584.30761719,42.04375)(584.14746094,42.3875)(583.95605469,42.6375)
\curveto(583.76464844,42.8875)(583.49121094,43.08671875)(583.13574219,43.23515625)
\curveto(582.78027344,43.3875)(582.38574219,43.46367187)(581.95214844,43.46367187)
\curveto(581.43261719,43.46367187)(580.98339844,43.38359375)(580.60449219,43.2234375)
\curveto(580.22558594,43.0671875)(579.91894531,42.86015625)(579.68457031,42.60234375)
\curveto(579.45410156,42.34453125)(579.27441406,42.06132812)(579.14550781,41.75273437)
\curveto(578.92675781,41.22148437)(578.81738281,40.6453125)(578.81738281,40.02421875)
\curveto(578.81738281,39.25859375)(578.94824219,38.61796875)(579.20996094,38.10234375)
\curveto(579.47558594,37.58671875)(579.86035156,37.20390625)(580.36425781,36.95390625)
\curveto(580.86816406,36.70390625)(581.40332031,36.57890625)(581.96972656,36.57890625)
\curveto(582.46191406,36.57890625)(582.94238281,36.67265625)(583.41113281,36.86015625)
\curveto(583.87988281,37.0515625)(584.23535156,37.2546875)(584.47753906,37.46953125)
\lineto(584.47753906,39.06914062)
\closepath
}
}
{
\newrgbcolor{curcolor}{0 0 0}
\pscustom[linewidth=1,linecolor=curcolor]
{
\newpath
\moveto(105.1,425.9)
\lineto(105.1,57.6)
\lineto(575,57.6)
\lineto(575,425.9)
\closepath
}
}
{
\newrgbcolor{curcolor}{0 0 0}
\pscustom[linestyle=none,fillstyle=solid,fillcolor=curcolor]
{
\newpath
\moveto(16.3,192.29082031)
\lineto(7.71015625,192.29082031)
\lineto(7.71015625,196.09941406)
\curveto(7.71015625,196.86503906)(7.78828125,197.44707031)(7.94453125,197.84550781)
\curveto(8.096875,198.24394531)(8.36835938,198.56230469)(8.75898438,198.80058594)
\curveto(9.14960938,199.03886719)(9.58125,199.15800781)(10.05390625,199.15800781)
\curveto(10.66328125,199.15800781)(11.17695313,198.96074219)(11.59492188,198.56621094)
\curveto(12.01289063,198.17167969)(12.27851563,197.56230469)(12.39179688,196.73808594)
\curveto(12.53632813,197.03886719)(12.67890625,197.26738281)(12.81953125,197.42363281)
\curveto(13.12421875,197.75566406)(13.50507813,198.07011719)(13.96210938,198.36699219)
\lineto(16.3,199.86113281)
\lineto(16.3,198.43144531)
\lineto(14.51289063,197.29472656)
\curveto(13.99726563,196.96269531)(13.60273438,196.68925781)(13.32929688,196.47441406)
\curveto(13.05585938,196.25957031)(12.86445313,196.06621094)(12.75507813,195.89433594)
\curveto(12.64570313,195.72636719)(12.56953125,195.55449219)(12.5265625,195.37871094)
\curveto(12.49921875,195.24980469)(12.48554688,195.03886719)(12.48554688,194.74589844)
\lineto(12.48554688,193.42753906)
\lineto(16.3,193.42753906)
\closepath
\moveto(11.50117188,193.42753906)
\lineto(11.50117188,195.87089844)
\curveto(11.50117188,196.39042969)(11.4484375,196.79667969)(11.34296875,197.08964844)
\curveto(11.23359375,197.38261719)(11.06171875,197.60527344)(10.82734375,197.75761719)
\curveto(10.5890625,197.90996094)(10.33125,197.98613281)(10.05390625,197.98613281)
\curveto(9.64765625,197.98613281)(9.31367188,197.83769531)(9.05195313,197.54082031)
\curveto(8.79023438,197.24785156)(8.659375,196.78300781)(8.659375,196.14628906)
\lineto(8.659375,193.42753906)
\closepath
}
}
{
\newrgbcolor{curcolor}{0 0 0}
\pscustom[linestyle=none,fillstyle=solid,fillcolor=curcolor]
{
\newpath
\moveto(16.3,204.88261719)
\lineto(15.3859375,204.88261719)
\curveto(16.0890625,204.39824219)(16.440625,203.74003906)(16.440625,202.90800781)
\curveto(16.440625,202.54082031)(16.3703125,202.19707031)(16.2296875,201.87675781)
\curveto(16.0890625,201.56035156)(15.91328125,201.32402344)(15.70234375,201.16777344)
\curveto(15.4875,201.01542969)(15.22578125,200.90800781)(14.9171875,200.84550781)
\curveto(14.71015625,200.80253906)(14.38203125,200.78105469)(13.9328125,200.78105469)
\lineto(10.07734375,200.78105469)
\lineto(10.07734375,201.83574219)
\lineto(13.52851563,201.83574219)
\curveto(14.07929688,201.83574219)(14.45039063,201.85722656)(14.64179688,201.90019531)
\curveto(14.91914063,201.96660156)(15.13789063,202.10722656)(15.29804688,202.32207031)
\curveto(15.45429688,202.53691406)(15.53242188,202.80253906)(15.53242188,203.11894531)
\curveto(15.53242188,203.43535156)(15.45234375,203.73222656)(15.2921875,204.00957031)
\curveto(15.128125,204.28691406)(14.90742188,204.48222656)(14.63007813,204.59550781)
\curveto(14.34882813,204.71269531)(13.94257813,204.77128906)(13.41132813,204.77128906)
\lineto(10.07734375,204.77128906)
\lineto(10.07734375,205.82597656)
\lineto(16.3,205.82597656)
\closepath
}
}
{
\newrgbcolor{curcolor}{0 0 0}
\pscustom[linestyle=none,fillstyle=solid,fillcolor=curcolor]
{
\newpath
\moveto(16.3,207.47832031)
\lineto(10.07734375,207.47832031)
\lineto(10.07734375,208.42753906)
\lineto(10.96210938,208.42753906)
\curveto(10.27851563,208.88457031)(9.93671875,209.54472656)(9.93671875,210.40800781)
\curveto(9.93671875,210.78300781)(10.00507813,211.12675781)(10.14179688,211.43925781)
\curveto(10.27460938,211.75566406)(10.45039063,211.99199219)(10.66914063,212.14824219)
\curveto(10.88789063,212.30449219)(11.14765625,212.41386719)(11.4484375,212.47636719)
\curveto(11.64375,212.51542969)(11.98554688,212.53496094)(12.47382813,212.53496094)
\lineto(16.3,212.53496094)
\lineto(16.3,211.48027344)
\lineto(12.51484375,211.48027344)
\curveto(12.08515625,211.48027344)(11.76484375,211.43925781)(11.55390625,211.35722656)
\curveto(11.3390625,211.27519531)(11.16914063,211.12871094)(11.04414063,210.91777344)
\curveto(10.91523438,210.71074219)(10.85078125,210.46660156)(10.85078125,210.18535156)
\curveto(10.85078125,209.73613281)(10.99335938,209.34746094)(11.27851563,209.01933594)
\curveto(11.56367188,208.69511719)(12.1046875,208.53300781)(12.9015625,208.53300781)
\lineto(16.3,208.53300781)
\closepath
}
}
{
\newrgbcolor{curcolor}{0 0 0}
\pscustom[linestyle=none,fillstyle=solid,fillcolor=curcolor]
{
\newpath
\moveto(15.35664063,216.45488281)
\lineto(16.28828125,216.60722656)
\curveto(16.35078125,216.31035156)(16.38203125,216.04472656)(16.38203125,215.81035156)
\curveto(16.38203125,215.42753906)(16.32148438,215.13066406)(16.20039063,214.91972656)
\curveto(16.07929688,214.70878906)(15.92109375,214.56035156)(15.72578125,214.47441406)
\curveto(15.5265625,214.38847656)(15.11054688,214.34550781)(14.47773438,214.34550781)
\lineto(10.89765625,214.34550781)
\lineto(10.89765625,213.57207031)
\lineto(10.07734375,213.57207031)
\lineto(10.07734375,214.34550781)
\lineto(8.53632813,214.34550781)
\lineto(7.90351563,215.39433594)
\lineto(10.07734375,215.39433594)
\lineto(10.07734375,216.45488281)
\lineto(10.89765625,216.45488281)
\lineto(10.89765625,215.39433594)
\lineto(14.53632813,215.39433594)
\curveto(14.83710938,215.39433594)(15.03046875,215.41191406)(15.11640625,215.44707031)
\curveto(15.20234375,215.48613281)(15.27070313,215.54667969)(15.32148438,215.62871094)
\curveto(15.37226563,215.71464844)(15.39765625,215.83574219)(15.39765625,215.99199219)
\curveto(15.39765625,216.10917969)(15.38398438,216.26347656)(15.35664063,216.45488281)
\closepath
}
}
{
\newrgbcolor{curcolor}{0 0 0}
\pscustom[linestyle=none,fillstyle=solid,fillcolor=curcolor]
{
\newpath
\moveto(8.92304688,217.49199219)
\lineto(7.71015625,217.49199219)
\lineto(7.71015625,218.54667969)
\lineto(8.92304688,218.54667969)
\closepath
\moveto(16.3,217.49199219)
\lineto(10.07734375,217.49199219)
\lineto(10.07734375,218.54667969)
\lineto(16.3,218.54667969)
\closepath
}
}
{
\newrgbcolor{curcolor}{0 0 0}
\pscustom[linestyle=none,fillstyle=solid,fillcolor=curcolor]
{
\newpath
\moveto(16.3,220.15214844)
\lineto(10.07734375,220.15214844)
\lineto(10.07734375,221.09550781)
\lineto(10.95039063,221.09550781)
\curveto(10.64570313,221.29082031)(10.4015625,221.55058594)(10.21796875,221.87480469)
\curveto(10.03046875,222.19902344)(9.93671875,222.56816406)(9.93671875,222.98222656)
\curveto(9.93671875,223.44316406)(10.03242188,223.82011719)(10.22382813,224.11308594)
\curveto(10.41523438,224.40996094)(10.6828125,224.61894531)(11.0265625,224.74003906)
\curveto(10.3,225.23222656)(9.93671875,225.87285156)(9.93671875,226.66191406)
\curveto(9.93671875,227.27910156)(10.10859375,227.75371094)(10.45234375,228.08574219)
\curveto(10.7921875,228.41777344)(11.31757813,228.58378906)(12.02851563,228.58378906)
\lineto(16.3,228.58378906)
\lineto(16.3,227.53496094)
\lineto(12.38007813,227.53496094)
\curveto(11.95820313,227.53496094)(11.65546875,227.49980469)(11.471875,227.42949219)
\curveto(11.284375,227.36308594)(11.13398438,227.24003906)(11.02070313,227.06035156)
\curveto(10.90742188,226.88066406)(10.85078125,226.66972656)(10.85078125,226.42753906)
\curveto(10.85078125,225.99003906)(10.99726563,225.62675781)(11.29023438,225.33769531)
\curveto(11.57929688,225.04863281)(12.04414063,224.90410156)(12.68476563,224.90410156)
\lineto(16.3,224.90410156)
\lineto(16.3,223.84941406)
\lineto(12.25703125,223.84941406)
\curveto(11.78828125,223.84941406)(11.43671875,223.76347656)(11.20234375,223.59160156)
\curveto(10.96796875,223.41972656)(10.85078125,223.13847656)(10.85078125,222.74785156)
\curveto(10.85078125,222.45097656)(10.92890625,222.17558594)(11.08515625,221.92167969)
\curveto(11.24140625,221.67167969)(11.46992188,221.49003906)(11.77070313,221.37675781)
\curveto(12.07148438,221.26347656)(12.50507813,221.20683594)(13.07148438,221.20683594)
\lineto(16.3,221.20683594)
\closepath
}
}
{
\newrgbcolor{curcolor}{0 0 0}
\pscustom[linestyle=none,fillstyle=solid,fillcolor=curcolor]
{
\newpath
\moveto(14.29609375,234.40800781)
\lineto(14.43085938,235.49785156)
\curveto(15.06757813,235.32597656)(15.56171875,235.00761719)(15.91328125,234.54277344)
\curveto(16.26484375,234.07792969)(16.440625,233.48417969)(16.440625,232.76152344)
\curveto(16.440625,231.85136719)(16.16132813,231.12871094)(15.60273438,230.59355469)
\curveto(15.04023438,230.06230469)(14.253125,229.79667969)(13.24140625,229.79667969)
\curveto(12.19453125,229.79667969)(11.38203125,230.06621094)(10.80390625,230.60527344)
\curveto(10.22578125,231.14433594)(9.93671875,231.84355469)(9.93671875,232.70292969)
\curveto(9.93671875,233.53496094)(10.21992188,234.21464844)(10.78632813,234.74199219)
\curveto(11.35273438,235.26933594)(12.14960938,235.53300781)(13.17695313,235.53300781)
\curveto(13.23945313,235.53300781)(13.33320313,235.53105469)(13.45820313,235.52714844)
\lineto(13.45820313,230.88652344)
\curveto(14.14179688,230.92558594)(14.66523438,231.11894531)(15.02851563,231.46660156)
\curveto(15.39179688,231.81425781)(15.5734375,232.24785156)(15.5734375,232.76738281)
\curveto(15.5734375,233.15410156)(15.471875,233.48417969)(15.26875,233.75761719)
\curveto(15.065625,234.03105469)(14.74140625,234.24785156)(14.29609375,234.40800781)
\closepath
\moveto(12.59101563,230.94511719)
\lineto(12.59101563,234.41972656)
\curveto(12.06757813,234.37285156)(11.675,234.24003906)(11.41328125,234.02128906)
\curveto(11.00703125,233.68535156)(10.80390625,233.24980469)(10.80390625,232.71464844)
\curveto(10.80390625,232.23027344)(10.96601563,231.82207031)(11.29023438,231.49003906)
\curveto(11.61445313,231.16191406)(12.04804688,230.98027344)(12.59101563,230.94511719)
\closepath
}
}
{
\newrgbcolor{curcolor}{0 0 0}
\pscustom[linestyle=none,fillstyle=solid,fillcolor=curcolor]
{
\newpath
\moveto(18.82539063,242.17167969)
\curveto(18.09101563,241.58964844)(17.23164063,241.09746094)(16.24726563,240.69511719)
\curveto(15.26289063,240.29277344)(14.24335938,240.09160156)(13.18867188,240.09160156)
\curveto(12.25898438,240.09160156)(11.36835938,240.24199219)(10.51679688,240.54277344)
\curveto(9.52851563,240.89433594)(8.54414063,241.43730469)(7.56367188,242.17167969)
\lineto(7.56367188,242.92753906)
\curveto(8.37617188,242.45488281)(8.95625,242.14238281)(9.30390625,241.99003906)
\curveto(9.84296875,241.75175781)(10.40546875,241.56425781)(10.99140625,241.42753906)
\curveto(11.721875,241.25957031)(12.45625,241.17558594)(13.19453125,241.17558594)
\curveto(15.0734375,241.17558594)(16.95039063,241.75957031)(18.82539063,242.92753906)
\closepath
}
}
{
\newrgbcolor{curcolor}{0 0 0}
\pscustom[linestyle=none,fillstyle=solid,fillcolor=curcolor]
{
\newpath
\moveto(14.44257813,243.73027344)
\lineto(14.27851563,244.77324219)
\curveto(14.69648438,244.83183594)(15.01679688,244.99394531)(15.23945313,245.25957031)
\curveto(15.46210938,245.52910156)(15.5734375,245.90410156)(15.5734375,246.38457031)
\curveto(15.5734375,246.86894531)(15.47578125,247.22832031)(15.28046875,247.46269531)
\curveto(15.08125,247.69707031)(14.84882813,247.81425781)(14.58320313,247.81425781)
\curveto(14.34492188,247.81425781)(14.15742188,247.71074219)(14.02070313,247.50371094)
\curveto(13.92695313,247.35917969)(13.8078125,246.99980469)(13.66328125,246.42558594)
\curveto(13.46796875,245.65214844)(13.3,245.11503906)(13.159375,244.81425781)
\curveto(13.01484375,244.51738281)(12.81757813,244.29082031)(12.56757813,244.13457031)
\curveto(12.31367188,243.98222656)(12.034375,243.90605469)(11.7296875,243.90605469)
\curveto(11.45234375,243.90605469)(11.19648438,243.96855469)(10.96210938,244.09355469)
\curveto(10.72382813,244.22246094)(10.5265625,244.39628906)(10.3703125,244.61503906)
\curveto(10.24921875,244.77910156)(10.14765625,245.00175781)(10.065625,245.28300781)
\curveto(9.9796875,245.56816406)(9.93671875,245.87285156)(9.93671875,246.19707031)
\curveto(9.93671875,246.68535156)(10.00703125,247.11308594)(10.14765625,247.48027344)
\curveto(10.28828125,247.85136719)(10.4796875,248.12480469)(10.721875,248.30058594)
\curveto(10.96015625,248.47636719)(11.28046875,248.59746094)(11.6828125,248.66386719)
\lineto(11.8234375,247.63261719)
\curveto(11.503125,247.58574219)(11.253125,247.44902344)(11.0734375,247.22246094)
\curveto(10.89375,246.99980469)(10.80390625,246.68339844)(10.80390625,246.27324219)
\curveto(10.80390625,245.78886719)(10.88398438,245.44316406)(11.04414063,245.23613281)
\curveto(11.20429688,245.02910156)(11.39179688,244.92558594)(11.60664063,244.92558594)
\curveto(11.74335938,244.92558594)(11.86640625,244.96855469)(11.97578125,245.05449219)
\curveto(12.0890625,245.14042969)(12.1828125,245.27519531)(12.25703125,245.45878906)
\curveto(12.29609375,245.56425781)(12.3859375,245.87480469)(12.5265625,246.39042969)
\curveto(12.72578125,247.13652344)(12.88984375,247.65605469)(13.01875,247.94902344)
\curveto(13.14375,248.24589844)(13.32734375,248.47832031)(13.56953125,248.64628906)
\curveto(13.81171875,248.81425781)(14.1125,248.89824219)(14.471875,248.89824219)
\curveto(14.8234375,248.89824219)(15.15546875,248.79472656)(15.46796875,248.58769531)
\curveto(15.7765625,248.38457031)(16.01679688,248.08964844)(16.18867188,247.70292969)
\curveto(16.35664063,247.31621094)(16.440625,246.87871094)(16.440625,246.39042969)
\curveto(16.440625,245.58183594)(16.27265625,244.96464844)(15.93671875,244.53886719)
\curveto(15.60078125,244.11699219)(15.10273438,243.84746094)(14.44257813,243.73027344)
\closepath
}
}
{
\newrgbcolor{curcolor}{0 0 0}
\pscustom[linestyle=none,fillstyle=solid,fillcolor=curcolor]
{
\newpath
\moveto(14.29609375,254.41191406)
\lineto(14.43085938,255.50175781)
\curveto(15.06757813,255.32988281)(15.56171875,255.01152344)(15.91328125,254.54667969)
\curveto(16.26484375,254.08183594)(16.440625,253.48808594)(16.440625,252.76542969)
\curveto(16.440625,251.85527344)(16.16132813,251.13261719)(15.60273438,250.59746094)
\curveto(15.04023438,250.06621094)(14.253125,249.80058594)(13.24140625,249.80058594)
\curveto(12.19453125,249.80058594)(11.38203125,250.07011719)(10.80390625,250.60917969)
\curveto(10.22578125,251.14824219)(9.93671875,251.84746094)(9.93671875,252.70683594)
\curveto(9.93671875,253.53886719)(10.21992188,254.21855469)(10.78632813,254.74589844)
\curveto(11.35273438,255.27324219)(12.14960938,255.53691406)(13.17695313,255.53691406)
\curveto(13.23945313,255.53691406)(13.33320313,255.53496094)(13.45820313,255.53105469)
\lineto(13.45820313,250.89042969)
\curveto(14.14179688,250.92949219)(14.66523438,251.12285156)(15.02851563,251.47050781)
\curveto(15.39179688,251.81816406)(15.5734375,252.25175781)(15.5734375,252.77128906)
\curveto(15.5734375,253.15800781)(15.471875,253.48808594)(15.26875,253.76152344)
\curveto(15.065625,254.03496094)(14.74140625,254.25175781)(14.29609375,254.41191406)
\closepath
\moveto(12.59101563,250.94902344)
\lineto(12.59101563,254.42363281)
\curveto(12.06757813,254.37675781)(11.675,254.24394531)(11.41328125,254.02519531)
\curveto(11.00703125,253.68925781)(10.80390625,253.25371094)(10.80390625,252.71855469)
\curveto(10.80390625,252.23417969)(10.96601563,251.82597656)(11.29023438,251.49394531)
\curveto(11.61445313,251.16582031)(12.04804688,250.98417969)(12.59101563,250.94902344)
\closepath
}
}
{
\newrgbcolor{curcolor}{0 0 0}
\pscustom[linestyle=none,fillstyle=solid,fillcolor=curcolor]
{
\newpath
\moveto(14.02070313,260.88652344)
\lineto(14.15546875,261.92363281)
\curveto(14.8703125,261.81035156)(15.43085938,261.51933594)(15.83710938,261.05058594)
\curveto(16.23945313,260.58574219)(16.440625,260.01347656)(16.440625,259.33378906)
\curveto(16.440625,258.48222656)(16.16328125,257.79667969)(15.60859375,257.27714844)
\curveto(15.05,256.76152344)(14.25117188,256.50371094)(13.21210938,256.50371094)
\curveto(12.54023438,256.50371094)(11.95234375,256.61503906)(11.4484375,256.83769531)
\curveto(10.94453125,257.06035156)(10.56757813,257.39824219)(10.31757813,257.85136719)
\curveto(10.06367188,258.30839844)(9.93671875,258.80449219)(9.93671875,259.33964844)
\curveto(9.93671875,260.01542969)(10.10859375,260.56816406)(10.45234375,260.99785156)
\curveto(10.7921875,261.42753906)(11.2765625,261.70292969)(11.90546875,261.82402344)
\lineto(12.06367188,260.79863281)
\curveto(11.64570313,260.70097656)(11.33125,260.52714844)(11.1203125,260.27714844)
\curveto(10.909375,260.03105469)(10.80390625,259.73222656)(10.80390625,259.38066406)
\curveto(10.80390625,258.84941406)(10.9953125,258.41777344)(11.378125,258.08574219)
\curveto(11.75703125,257.75371094)(12.35859375,257.58769531)(13.1828125,257.58769531)
\curveto(14.01875,257.58769531)(14.62617188,257.74785156)(15.00507813,258.06816406)
\curveto(15.38398438,258.38847656)(15.5734375,258.80644531)(15.5734375,259.32207031)
\curveto(15.5734375,259.73613281)(15.44648438,260.08183594)(15.19257813,260.35917969)
\curveto(14.93867188,260.63652344)(14.54804688,260.81230469)(14.02070313,260.88652344)
\closepath
}
}
{
\newrgbcolor{curcolor}{0 0 0}
\pscustom[linestyle=none,fillstyle=solid,fillcolor=curcolor]
{
\newpath
\moveto(13.18867188,262.43339844)
\curveto(12.03632813,262.43339844)(11.1828125,262.75371094)(10.628125,263.39433594)
\curveto(10.1671875,263.92949219)(9.93671875,264.58183594)(9.93671875,265.35136719)
\curveto(9.93671875,266.20683594)(10.21796875,266.90605469)(10.78046875,267.44902344)
\curveto(11.3390625,267.99199219)(12.1125,268.26347656)(13.10078125,268.26347656)
\curveto(13.9015625,268.26347656)(14.53242188,268.14238281)(14.99335938,267.90019531)
\curveto(15.45039063,267.66191406)(15.80585938,267.31230469)(16.05976563,266.85136719)
\curveto(16.31367188,266.39433594)(16.440625,265.89433594)(16.440625,265.35136719)
\curveto(16.440625,264.48027344)(16.16132813,263.77519531)(15.60273438,263.23613281)
\curveto(15.04414063,262.70097656)(14.23945313,262.43339844)(13.18867188,262.43339844)
\closepath
\moveto(13.18867188,263.51738281)
\curveto(13.98554688,263.51738281)(14.58320313,263.69121094)(14.98164063,264.03886719)
\curveto(15.37617188,264.38652344)(15.5734375,264.82402344)(15.5734375,265.35136719)
\curveto(15.5734375,265.87480469)(15.37421875,266.31035156)(14.97578125,266.65800781)
\curveto(14.57734375,267.00566406)(13.96992188,267.17949219)(13.15351563,267.17949219)
\curveto(12.38398438,267.17949219)(11.80195313,267.00371094)(11.40742188,266.65214844)
\curveto(11.00898438,266.30449219)(10.80976563,265.87089844)(10.80976563,265.35136719)
\curveto(10.80976563,264.82402344)(11.00703125,264.38652344)(11.4015625,264.03886719)
\curveto(11.79609375,263.69121094)(12.39179688,263.51738281)(13.18867188,263.51738281)
\closepath
}
}
{
\newrgbcolor{curcolor}{0 0 0}
\pscustom[linestyle=none,fillstyle=solid,fillcolor=curcolor]
{
\newpath
\moveto(16.3,269.49980469)
\lineto(10.07734375,269.49980469)
\lineto(10.07734375,270.44902344)
\lineto(10.96210938,270.44902344)
\curveto(10.27851563,270.90605469)(9.93671875,271.56621094)(9.93671875,272.42949219)
\curveto(9.93671875,272.80449219)(10.00507813,273.14824219)(10.14179688,273.46074219)
\curveto(10.27460938,273.77714844)(10.45039063,274.01347656)(10.66914063,274.16972656)
\curveto(10.88789063,274.32597656)(11.14765625,274.43535156)(11.4484375,274.49785156)
\curveto(11.64375,274.53691406)(11.98554688,274.55644531)(12.47382813,274.55644531)
\lineto(16.3,274.55644531)
\lineto(16.3,273.50175781)
\lineto(12.51484375,273.50175781)
\curveto(12.08515625,273.50175781)(11.76484375,273.46074219)(11.55390625,273.37871094)
\curveto(11.3390625,273.29667969)(11.16914063,273.15019531)(11.04414063,272.93925781)
\curveto(10.91523438,272.73222656)(10.85078125,272.48808594)(10.85078125,272.20683594)
\curveto(10.85078125,271.75761719)(10.99335938,271.36894531)(11.27851563,271.04082031)
\curveto(11.56367188,270.71660156)(12.1046875,270.55449219)(12.9015625,270.55449219)
\lineto(16.3,270.55449219)
\closepath
}
}
{
\newrgbcolor{curcolor}{0 0 0}
\pscustom[linestyle=none,fillstyle=solid,fillcolor=curcolor]
{
\newpath
\moveto(16.3,280.21074219)
\lineto(15.51484375,280.21074219)
\curveto(16.13203125,279.81621094)(16.440625,279.23613281)(16.440625,278.47050781)
\curveto(16.440625,277.97441406)(16.30390625,277.51738281)(16.03046875,277.09941406)
\curveto(15.75703125,276.68535156)(15.37617188,276.36308594)(14.88789063,276.13261719)
\curveto(14.39570313,275.90605469)(13.83125,275.79277344)(13.19453125,275.79277344)
\curveto(12.5734375,275.79277344)(12.0109375,275.89628906)(11.50703125,276.10332031)
\curveto(10.99921875,276.31035156)(10.61054688,276.62089844)(10.34101563,277.03496094)
\curveto(10.07148438,277.44902344)(9.93671875,277.91191406)(9.93671875,278.42363281)
\curveto(9.93671875,278.79863281)(10.01679688,279.13261719)(10.17695313,279.42558594)
\curveto(10.33320313,279.71855469)(10.53828125,279.95683594)(10.7921875,280.14042969)
\lineto(7.71015625,280.14042969)
\lineto(7.71015625,281.18925781)
\lineto(16.3,281.18925781)
\closepath
\moveto(13.19453125,276.87675781)
\curveto(13.99140625,276.87675781)(14.58710938,277.04472656)(14.98164063,277.38066406)
\curveto(15.37617188,277.71660156)(15.5734375,278.11308594)(15.5734375,278.57011719)
\curveto(15.5734375,279.03105469)(15.3859375,279.42167969)(15.0109375,279.74199219)
\curveto(14.63203125,280.06621094)(14.05585938,280.22832031)(13.28242188,280.22832031)
\curveto(12.43085938,280.22832031)(11.80585938,280.06425781)(11.40742188,279.73613281)
\curveto(11.00898438,279.40800781)(10.80976563,279.00371094)(10.80976563,278.52324219)
\curveto(10.80976563,278.05449219)(11.00117188,277.66191406)(11.38398438,277.34550781)
\curveto(11.76679688,277.03300781)(12.3703125,276.87675781)(13.19453125,276.87675781)
\closepath
}
}
{
\newrgbcolor{curcolor}{0 0 0}
\pscustom[linestyle=none,fillstyle=solid,fillcolor=curcolor]
{
\newpath
\moveto(14.44257813,282.42558594)
\lineto(14.27851563,283.46855469)
\curveto(14.69648438,283.52714844)(15.01679688,283.68925781)(15.23945313,283.95488281)
\curveto(15.46210938,284.22441406)(15.5734375,284.59941406)(15.5734375,285.07988281)
\curveto(15.5734375,285.56425781)(15.47578125,285.92363281)(15.28046875,286.15800781)
\curveto(15.08125,286.39238281)(14.84882813,286.50957031)(14.58320313,286.50957031)
\curveto(14.34492188,286.50957031)(14.15742188,286.40605469)(14.02070313,286.19902344)
\curveto(13.92695313,286.05449219)(13.8078125,285.69511719)(13.66328125,285.12089844)
\curveto(13.46796875,284.34746094)(13.3,283.81035156)(13.159375,283.50957031)
\curveto(13.01484375,283.21269531)(12.81757813,282.98613281)(12.56757813,282.82988281)
\curveto(12.31367188,282.67753906)(12.034375,282.60136719)(11.7296875,282.60136719)
\curveto(11.45234375,282.60136719)(11.19648438,282.66386719)(10.96210938,282.78886719)
\curveto(10.72382813,282.91777344)(10.5265625,283.09160156)(10.3703125,283.31035156)
\curveto(10.24921875,283.47441406)(10.14765625,283.69707031)(10.065625,283.97832031)
\curveto(9.9796875,284.26347656)(9.93671875,284.56816406)(9.93671875,284.89238281)
\curveto(9.93671875,285.38066406)(10.00703125,285.80839844)(10.14765625,286.17558594)
\curveto(10.28828125,286.54667969)(10.4796875,286.82011719)(10.721875,286.99589844)
\curveto(10.96015625,287.17167969)(11.28046875,287.29277344)(11.6828125,287.35917969)
\lineto(11.8234375,286.32792969)
\curveto(11.503125,286.28105469)(11.253125,286.14433594)(11.0734375,285.91777344)
\curveto(10.89375,285.69511719)(10.80390625,285.37871094)(10.80390625,284.96855469)
\curveto(10.80390625,284.48417969)(10.88398438,284.13847656)(11.04414063,283.93144531)
\curveto(11.20429688,283.72441406)(11.39179688,283.62089844)(11.60664063,283.62089844)
\curveto(11.74335938,283.62089844)(11.86640625,283.66386719)(11.97578125,283.74980469)
\curveto(12.0890625,283.83574219)(12.1828125,283.97050781)(12.25703125,284.15410156)
\curveto(12.29609375,284.25957031)(12.3859375,284.57011719)(12.5265625,285.08574219)
\curveto(12.72578125,285.83183594)(12.88984375,286.35136719)(13.01875,286.64433594)
\curveto(13.14375,286.94121094)(13.32734375,287.17363281)(13.56953125,287.34160156)
\curveto(13.81171875,287.50957031)(14.1125,287.59355469)(14.471875,287.59355469)
\curveto(14.8234375,287.59355469)(15.15546875,287.49003906)(15.46796875,287.28300781)
\curveto(15.7765625,287.07988281)(16.01679688,286.78496094)(16.18867188,286.39824219)
\curveto(16.35664063,286.01152344)(16.440625,285.57402344)(16.440625,285.08574219)
\curveto(16.440625,284.27714844)(16.27265625,283.65996094)(15.93671875,283.23417969)
\curveto(15.60078125,282.81230469)(15.10273438,282.54277344)(14.44257813,282.42558594)
\closepath
}
}
{
\newrgbcolor{curcolor}{0 0 0}
\pscustom[linestyle=none,fillstyle=solid,fillcolor=curcolor]
{
\newpath
\moveto(18.82539063,289.53886719)
\lineto(18.82539063,288.78300781)
\curveto(16.95039063,289.95097656)(15.0734375,290.53496094)(13.19453125,290.53496094)
\curveto(12.46015625,290.53496094)(11.73164063,290.45097656)(11.00898438,290.28300781)
\curveto(10.42304688,290.15019531)(9.86054688,289.96464844)(9.32148438,289.72636719)
\curveto(8.96992188,289.57402344)(8.38398438,289.25957031)(7.56367188,288.78300781)
\lineto(7.56367188,289.53886719)
\curveto(8.54414063,290.27324219)(9.52851563,290.81621094)(10.51679688,291.16777344)
\curveto(11.36835938,291.46855469)(12.25898438,291.61894531)(13.18867188,291.61894531)
\curveto(14.24335938,291.61894531)(15.26289063,291.41582031)(16.24726563,291.00957031)
\curveto(17.23164063,290.60722656)(18.09101563,290.11699219)(18.82539063,289.53886719)
\closepath
}
}
{
\newrgbcolor{curcolor}{0 0 0}
\pscustom[linestyle=none,fillstyle=solid,fillcolor=curcolor]
{
\newpath
\moveto(317.97753906,8.7)
\lineto(317.97753906,17.28984375)
\lineto(323.77246094,17.28984375)
\lineto(323.77246094,16.27617187)
\lineto(319.11425781,16.27617187)
\lineto(319.11425781,13.61601562)
\lineto(323.14550781,13.61601562)
\lineto(323.14550781,12.60234375)
\lineto(319.11425781,12.60234375)
\lineto(319.11425781,8.7)
\closepath
}
}
{
\newrgbcolor{curcolor}{0 0 0}
\pscustom[linestyle=none,fillstyle=solid,fillcolor=curcolor]
{
\newpath
\moveto(325.12011719,16.07695312)
\lineto(325.12011719,17.28984375)
\lineto(326.17480469,17.28984375)
\lineto(326.17480469,16.07695312)
\closepath
\moveto(325.12011719,8.7)
\lineto(325.12011719,14.92265625)
\lineto(326.17480469,14.92265625)
\lineto(326.17480469,8.7)
\closepath
}
}
{
\newrgbcolor{curcolor}{0 0 0}
\pscustom[linestyle=none,fillstyle=solid,fillcolor=curcolor]
{
\newpath
\moveto(327.75683594,8.7)
\lineto(327.75683594,17.28984375)
\lineto(328.81152344,17.28984375)
\lineto(328.81152344,8.7)
\closepath
}
}
{
\newrgbcolor{curcolor}{0 0 0}
\pscustom[linestyle=none,fillstyle=solid,fillcolor=curcolor]
{
\newpath
\moveto(334.70605469,10.70390625)
\lineto(335.79589844,10.56914062)
\curveto(335.62402344,9.93242187)(335.30566406,9.43828125)(334.84082031,9.08671875)
\curveto(334.37597656,8.73515625)(333.78222656,8.559375)(333.05957031,8.559375)
\curveto(332.14941406,8.559375)(331.42675781,8.83867187)(330.89160156,9.39726562)
\curveto(330.36035156,9.95976562)(330.09472656,10.746875)(330.09472656,11.75859375)
\curveto(330.09472656,12.80546875)(330.36425781,13.61796875)(330.90332031,14.19609375)
\curveto(331.44238281,14.77421875)(332.14160156,15.06328125)(333.00097656,15.06328125)
\curveto(333.83300781,15.06328125)(334.51269531,14.78007812)(335.04003906,14.21367187)
\curveto(335.56738281,13.64726562)(335.83105469,12.85039062)(335.83105469,11.82304687)
\curveto(335.83105469,11.76054687)(335.82910156,11.66679687)(335.82519531,11.54179687)
\lineto(331.18457031,11.54179687)
\curveto(331.22363281,10.85820312)(331.41699219,10.33476562)(331.76464844,9.97148437)
\curveto(332.11230469,9.60820312)(332.54589844,9.4265625)(333.06542969,9.4265625)
\curveto(333.45214844,9.4265625)(333.78222656,9.528125)(334.05566406,9.73125)
\curveto(334.32910156,9.934375)(334.54589844,10.25859375)(334.70605469,10.70390625)
\closepath
\moveto(331.24316406,12.40898437)
\lineto(334.71777344,12.40898437)
\curveto(334.67089844,12.93242187)(334.53808594,13.325)(334.31933594,13.58671875)
\curveto(333.98339844,13.99296875)(333.54785156,14.19609375)(333.01269531,14.19609375)
\curveto(332.52832031,14.19609375)(332.12011719,14.03398437)(331.78808594,13.70976562)
\curveto(331.45996094,13.38554687)(331.27832031,12.95195312)(331.24316406,12.40898437)
\closepath
}
}
{
\newrgbcolor{curcolor}{0 0 0}
\pscustom[linestyle=none,fillstyle=solid,fillcolor=curcolor]
{
\newpath
\moveto(340.20214844,11.45976562)
\lineto(341.27441406,11.55351562)
\curveto(341.32519531,11.12382812)(341.44238281,10.7703125)(341.62597656,10.49296875)
\curveto(341.81347656,10.21953125)(342.10253906,9.996875)(342.49316406,9.825)
\curveto(342.88378906,9.65703125)(343.32324219,9.57304687)(343.81152344,9.57304687)
\curveto(344.24511719,9.57304687)(344.62792969,9.6375)(344.95996094,9.76640625)
\curveto(345.29199219,9.8953125)(345.53808594,10.07109375)(345.69824219,10.29375)
\curveto(345.86230469,10.5203125)(345.94433594,10.76640625)(345.94433594,11.03203125)
\curveto(345.94433594,11.3015625)(345.86621094,11.5359375)(345.70996094,11.73515625)
\curveto(345.55371094,11.93828125)(345.29589844,12.10820312)(344.93652344,12.24492187)
\curveto(344.70605469,12.33476562)(344.19628906,12.4734375)(343.40722656,12.6609375)
\curveto(342.61816406,12.85234375)(342.06542969,13.03203125)(341.74902344,13.2)
\curveto(341.33886719,13.41484375)(341.03222656,13.68046875)(340.82910156,13.996875)
\curveto(340.62988281,14.3171875)(340.53027344,14.67460937)(340.53027344,15.06914062)
\curveto(340.53027344,15.50273437)(340.65332031,15.90703125)(340.89941406,16.28203125)
\curveto(341.14550781,16.6609375)(341.50488281,16.94804687)(341.97753906,17.14335937)
\curveto(342.45019531,17.33867187)(342.97558594,17.43632812)(343.55371094,17.43632812)
\curveto(344.19042969,17.43632812)(344.75097656,17.3328125)(345.23535156,17.12578125)
\curveto(345.72363281,16.92265625)(346.09863281,16.621875)(346.36035156,16.2234375)
\curveto(346.62207031,15.825)(346.76269531,15.37382812)(346.78222656,14.86992187)
\lineto(345.69238281,14.78789062)
\curveto(345.63378906,15.33085937)(345.43457031,15.74101562)(345.09472656,16.01835937)
\curveto(344.75878906,16.29570312)(344.26074219,16.434375)(343.60058594,16.434375)
\curveto(342.91308594,16.434375)(342.41113281,16.30742187)(342.09472656,16.05351562)
\curveto(341.78222656,15.80351562)(341.62597656,15.50078125)(341.62597656,15.1453125)
\curveto(341.62597656,14.83671875)(341.73730469,14.5828125)(341.95996094,14.38359375)
\curveto(342.17871094,14.184375)(342.74902344,13.97929687)(343.67089844,13.76835937)
\curveto(344.59667969,13.56132812)(345.23144531,13.3796875)(345.57519531,13.2234375)
\curveto(346.07519531,12.99296875)(346.44433594,12.7)(346.68261719,12.34453125)
\curveto(346.92089844,11.99296875)(347.04003906,11.58671875)(347.04003906,11.12578125)
\curveto(347.04003906,10.66875)(346.90917969,10.23710937)(346.64746094,9.83085937)
\curveto(346.38574219,9.42851562)(346.00878906,9.1140625)(345.51660156,8.8875)
\curveto(345.02832031,8.66484375)(344.47753906,8.55351562)(343.86425781,8.55351562)
\curveto(343.08691406,8.55351562)(342.43457031,8.66679687)(341.90722656,8.89335937)
\curveto(341.38378906,9.11992187)(340.97167969,9.45976562)(340.67089844,9.91289062)
\curveto(340.37402344,10.36992187)(340.21777344,10.88554687)(340.20214844,11.45976562)
\closepath
}
}
{
\newrgbcolor{curcolor}{0 0 0}
\pscustom[linestyle=none,fillstyle=solid,fillcolor=curcolor]
{
\newpath
\moveto(348.46386719,16.07695312)
\lineto(348.46386719,17.28984375)
\lineto(349.51855469,17.28984375)
\lineto(349.51855469,16.07695312)
\closepath
\moveto(348.46386719,8.7)
\lineto(348.46386719,14.92265625)
\lineto(349.51855469,14.92265625)
\lineto(349.51855469,8.7)
\closepath
}
}
{
\newrgbcolor{curcolor}{0 0 0}
\pscustom[linestyle=none,fillstyle=solid,fillcolor=curcolor]
{
\newpath
\moveto(350.56738281,8.7)
\lineto(350.56738281,9.55546875)
\lineto(354.52832031,14.10234375)
\curveto(354.07910156,14.07890625)(353.68261719,14.0671875)(353.33886719,14.0671875)
\lineto(350.80175781,14.0671875)
\lineto(350.80175781,14.92265625)
\lineto(355.88769531,14.92265625)
\lineto(355.88769531,14.22539062)
\lineto(352.51855469,10.27617187)
\lineto(351.86816406,9.55546875)
\curveto(352.34082031,9.590625)(352.78417969,9.60820312)(353.19824219,9.60820312)
\lineto(356.07519531,9.60820312)
\lineto(356.07519531,8.7)
\closepath
}
}
{
\newrgbcolor{curcolor}{0 0 0}
\pscustom[linestyle=none,fillstyle=solid,fillcolor=curcolor]
{
\newpath
\moveto(361.38378906,10.70390625)
\lineto(362.47363281,10.56914062)
\curveto(362.30175781,9.93242187)(361.98339844,9.43828125)(361.51855469,9.08671875)
\curveto(361.05371094,8.73515625)(360.45996094,8.559375)(359.73730469,8.559375)
\curveto(358.82714844,8.559375)(358.10449219,8.83867187)(357.56933594,9.39726562)
\curveto(357.03808594,9.95976562)(356.77246094,10.746875)(356.77246094,11.75859375)
\curveto(356.77246094,12.80546875)(357.04199219,13.61796875)(357.58105469,14.19609375)
\curveto(358.12011719,14.77421875)(358.81933594,15.06328125)(359.67871094,15.06328125)
\curveto(360.51074219,15.06328125)(361.19042969,14.78007812)(361.71777344,14.21367187)
\curveto(362.24511719,13.64726562)(362.50878906,12.85039062)(362.50878906,11.82304687)
\curveto(362.50878906,11.76054687)(362.50683594,11.66679687)(362.50292969,11.54179687)
\lineto(357.86230469,11.54179687)
\curveto(357.90136719,10.85820312)(358.09472656,10.33476562)(358.44238281,9.97148437)
\curveto(358.79003906,9.60820312)(359.22363281,9.4265625)(359.74316406,9.4265625)
\curveto(360.12988281,9.4265625)(360.45996094,9.528125)(360.73339844,9.73125)
\curveto(361.00683594,9.934375)(361.22363281,10.25859375)(361.38378906,10.70390625)
\closepath
\moveto(357.92089844,12.40898437)
\lineto(361.39550781,12.40898437)
\curveto(361.34863281,12.93242187)(361.21582031,13.325)(360.99707031,13.58671875)
\curveto(360.66113281,13.99296875)(360.22558594,14.19609375)(359.69042969,14.19609375)
\curveto(359.20605469,14.19609375)(358.79785156,14.03398437)(358.46582031,13.70976562)
\curveto(358.13769531,13.38554687)(357.95605469,12.95195312)(357.92089844,12.40898437)
\closepath
}
}
{
\newrgbcolor{curcolor}{0 0 0}
\pscustom[linestyle=none,fillstyle=solid,fillcolor=curcolor]
{
\newpath
\moveto(187.88769531,449)
\lineto(187.88769531,457.58984375)
\lineto(191.69628906,457.58984375)
\curveto(192.46191406,457.58984375)(193.04394531,457.51171875)(193.44238281,457.35546875)
\curveto(193.84082031,457.203125)(194.15917969,456.93164062)(194.39746094,456.54101562)
\curveto(194.63574219,456.15039062)(194.75488281,455.71875)(194.75488281,455.24609375)
\curveto(194.75488281,454.63671875)(194.55761719,454.12304688)(194.16308594,453.70507812)
\curveto(193.76855469,453.28710938)(193.15917969,453.02148438)(192.33496094,452.90820312)
\curveto(192.63574219,452.76367188)(192.86425781,452.62109375)(193.02050781,452.48046875)
\curveto(193.35253906,452.17578125)(193.66699219,451.79492188)(193.96386719,451.33789062)
\lineto(195.45800781,449)
\lineto(194.02832031,449)
\lineto(192.89160156,450.78710938)
\curveto(192.55957031,451.30273438)(192.28613281,451.69726562)(192.07128906,451.97070312)
\curveto(191.85644531,452.24414062)(191.66308594,452.43554688)(191.49121094,452.54492188)
\curveto(191.32324219,452.65429688)(191.15136719,452.73046875)(190.97558594,452.7734375)
\curveto(190.84667969,452.80078125)(190.63574219,452.81445312)(190.34277344,452.81445312)
\lineto(189.02441406,452.81445312)
\lineto(189.02441406,449)
\closepath
\moveto(189.02441406,453.79882812)
\lineto(191.46777344,453.79882812)
\curveto(191.98730469,453.79882812)(192.39355469,453.8515625)(192.68652344,453.95703125)
\curveto(192.97949219,454.06640625)(193.20214844,454.23828125)(193.35449219,454.47265625)
\curveto(193.50683594,454.7109375)(193.58300781,454.96875)(193.58300781,455.24609375)
\curveto(193.58300781,455.65234375)(193.43457031,455.98632812)(193.13769531,456.24804688)
\curveto(192.84472656,456.50976562)(192.37988281,456.640625)(191.74316406,456.640625)
\lineto(189.02441406,456.640625)
\closepath
}
}
{
\newrgbcolor{curcolor}{0 0 0}
\pscustom[linestyle=none,fillstyle=solid,fillcolor=curcolor]
{
\newpath
\moveto(200.47949219,449)
\lineto(200.47949219,449.9140625)
\curveto(199.99511719,449.2109375)(199.33691406,448.859375)(198.50488281,448.859375)
\curveto(198.13769531,448.859375)(197.79394531,448.9296875)(197.47363281,449.0703125)
\curveto(197.15722656,449.2109375)(196.92089844,449.38671875)(196.76464844,449.59765625)
\curveto(196.61230469,449.8125)(196.50488281,450.07421875)(196.44238281,450.3828125)
\curveto(196.39941406,450.58984375)(196.37792969,450.91796875)(196.37792969,451.3671875)
\lineto(196.37792969,455.22265625)
\lineto(197.43261719,455.22265625)
\lineto(197.43261719,451.77148438)
\curveto(197.43261719,451.22070312)(197.45410156,450.84960938)(197.49707031,450.65820312)
\curveto(197.56347656,450.38085938)(197.70410156,450.16210938)(197.91894531,450.00195312)
\curveto(198.13378906,449.84570312)(198.39941406,449.76757812)(198.71582031,449.76757812)
\curveto(199.03222656,449.76757812)(199.32910156,449.84765625)(199.60644531,450.0078125)
\curveto(199.88378906,450.171875)(200.07910156,450.39257812)(200.19238281,450.66992188)
\curveto(200.30957031,450.95117188)(200.36816406,451.35742188)(200.36816406,451.88867188)
\lineto(200.36816406,455.22265625)
\lineto(201.42285156,455.22265625)
\lineto(201.42285156,449)
\closepath
}
}
{
\newrgbcolor{curcolor}{0 0 0}
\pscustom[linestyle=none,fillstyle=solid,fillcolor=curcolor]
{
\newpath
\moveto(203.07519531,449)
\lineto(203.07519531,455.22265625)
\lineto(204.02441406,455.22265625)
\lineto(204.02441406,454.33789062)
\curveto(204.48144531,455.02148438)(205.14160156,455.36328125)(206.00488281,455.36328125)
\curveto(206.37988281,455.36328125)(206.72363281,455.29492188)(207.03613281,455.15820312)
\curveto(207.35253906,455.02539062)(207.58886719,454.84960938)(207.74511719,454.63085938)
\curveto(207.90136719,454.41210938)(208.01074219,454.15234375)(208.07324219,453.8515625)
\curveto(208.11230469,453.65625)(208.13183594,453.31445312)(208.13183594,452.82617188)
\lineto(208.13183594,449)
\lineto(207.07714844,449)
\lineto(207.07714844,452.78515625)
\curveto(207.07714844,453.21484375)(207.03613281,453.53515625)(206.95410156,453.74609375)
\curveto(206.87207031,453.9609375)(206.72558594,454.13085938)(206.51464844,454.25585938)
\curveto(206.30761719,454.38476562)(206.06347656,454.44921875)(205.78222656,454.44921875)
\curveto(205.33300781,454.44921875)(204.94433594,454.30664062)(204.61621094,454.02148438)
\curveto(204.29199219,453.73632812)(204.12988281,453.1953125)(204.12988281,452.3984375)
\lineto(204.12988281,449)
\closepath
}
}
{
\newrgbcolor{curcolor}{0 0 0}
\pscustom[linestyle=none,fillstyle=solid,fillcolor=curcolor]
{
\newpath
\moveto(212.05175781,449.94335938)
\lineto(212.20410156,449.01171875)
\curveto(211.90722656,448.94921875)(211.64160156,448.91796875)(211.40722656,448.91796875)
\curveto(211.02441406,448.91796875)(210.72753906,448.97851562)(210.51660156,449.09960938)
\curveto(210.30566406,449.22070312)(210.15722656,449.37890625)(210.07128906,449.57421875)
\curveto(209.98535156,449.7734375)(209.94238281,450.18945312)(209.94238281,450.82226562)
\lineto(209.94238281,454.40234375)
\lineto(209.16894531,454.40234375)
\lineto(209.16894531,455.22265625)
\lineto(209.94238281,455.22265625)
\lineto(209.94238281,456.76367188)
\lineto(210.99121094,457.39648438)
\lineto(210.99121094,455.22265625)
\lineto(212.05175781,455.22265625)
\lineto(212.05175781,454.40234375)
\lineto(210.99121094,454.40234375)
\lineto(210.99121094,450.76367188)
\curveto(210.99121094,450.46289062)(211.00878906,450.26953125)(211.04394531,450.18359375)
\curveto(211.08300781,450.09765625)(211.14355469,450.02929688)(211.22558594,449.97851562)
\curveto(211.31152344,449.92773438)(211.43261719,449.90234375)(211.58886719,449.90234375)
\curveto(211.70605469,449.90234375)(211.86035156,449.91601562)(212.05175781,449.94335938)
\closepath
}
}
{
\newrgbcolor{curcolor}{0 0 0}
\pscustom[linestyle=none,fillstyle=solid,fillcolor=curcolor]
{
\newpath
\moveto(213.08886719,456.37695312)
\lineto(213.08886719,457.58984375)
\lineto(214.14355469,457.58984375)
\lineto(214.14355469,456.37695312)
\closepath
\moveto(213.08886719,449)
\lineto(213.08886719,455.22265625)
\lineto(214.14355469,455.22265625)
\lineto(214.14355469,449)
\closepath
}
}
{
\newrgbcolor{curcolor}{0 0 0}
\pscustom[linestyle=none,fillstyle=solid,fillcolor=curcolor]
{
\newpath
\moveto(215.74902344,449)
\lineto(215.74902344,455.22265625)
\lineto(216.69238281,455.22265625)
\lineto(216.69238281,454.34960938)
\curveto(216.88769531,454.65429688)(217.14746094,454.8984375)(217.47167969,455.08203125)
\curveto(217.79589844,455.26953125)(218.16503906,455.36328125)(218.57910156,455.36328125)
\curveto(219.04003906,455.36328125)(219.41699219,455.26757812)(219.70996094,455.07617188)
\curveto(220.00683594,454.88476562)(220.21582031,454.6171875)(220.33691406,454.2734375)
\curveto(220.82910156,455)(221.46972656,455.36328125)(222.25878906,455.36328125)
\curveto(222.87597656,455.36328125)(223.35058594,455.19140625)(223.68261719,454.84765625)
\curveto(224.01464844,454.5078125)(224.18066406,453.98242188)(224.18066406,453.27148438)
\lineto(224.18066406,449)
\lineto(223.13183594,449)
\lineto(223.13183594,452.91992188)
\curveto(223.13183594,453.34179688)(223.09667969,453.64453125)(223.02636719,453.828125)
\curveto(222.95996094,454.015625)(222.83691406,454.16601562)(222.65722656,454.27929688)
\curveto(222.47753906,454.39257812)(222.26660156,454.44921875)(222.02441406,454.44921875)
\curveto(221.58691406,454.44921875)(221.22363281,454.30273438)(220.93457031,454.00976562)
\curveto(220.64550781,453.72070312)(220.50097656,453.25585938)(220.50097656,452.61523438)
\lineto(220.50097656,449)
\lineto(219.44628906,449)
\lineto(219.44628906,453.04296875)
\curveto(219.44628906,453.51171875)(219.36035156,453.86328125)(219.18847656,454.09765625)
\curveto(219.01660156,454.33203125)(218.73535156,454.44921875)(218.34472656,454.44921875)
\curveto(218.04785156,454.44921875)(217.77246094,454.37109375)(217.51855469,454.21484375)
\curveto(217.26855469,454.05859375)(217.08691406,453.83007812)(216.97363281,453.52929688)
\curveto(216.86035156,453.22851562)(216.80371094,452.79492188)(216.80371094,452.22851562)
\lineto(216.80371094,449)
\closepath
}
}
{
\newrgbcolor{curcolor}{0 0 0}
\pscustom[linestyle=none,fillstyle=solid,fillcolor=curcolor]
{
\newpath
\moveto(230.00488281,451.00390625)
\lineto(231.09472656,450.86914062)
\curveto(230.92285156,450.23242188)(230.60449219,449.73828125)(230.13964844,449.38671875)
\curveto(229.67480469,449.03515625)(229.08105469,448.859375)(228.35839844,448.859375)
\curveto(227.44824219,448.859375)(226.72558594,449.13867188)(226.19042969,449.69726562)
\curveto(225.65917969,450.25976562)(225.39355469,451.046875)(225.39355469,452.05859375)
\curveto(225.39355469,453.10546875)(225.66308594,453.91796875)(226.20214844,454.49609375)
\curveto(226.74121094,455.07421875)(227.44042969,455.36328125)(228.29980469,455.36328125)
\curveto(229.13183594,455.36328125)(229.81152344,455.08007812)(230.33886719,454.51367188)
\curveto(230.86621094,453.94726562)(231.12988281,453.15039062)(231.12988281,452.12304688)
\curveto(231.12988281,452.06054688)(231.12792969,451.96679688)(231.12402344,451.84179688)
\lineto(226.48339844,451.84179688)
\curveto(226.52246094,451.15820312)(226.71582031,450.63476562)(227.06347656,450.27148438)
\curveto(227.41113281,449.90820312)(227.84472656,449.7265625)(228.36425781,449.7265625)
\curveto(228.75097656,449.7265625)(229.08105469,449.828125)(229.35449219,450.03125)
\curveto(229.62792969,450.234375)(229.84472656,450.55859375)(230.00488281,451.00390625)
\closepath
\moveto(226.54199219,452.70898438)
\lineto(230.01660156,452.70898438)
\curveto(229.96972656,453.23242188)(229.83691406,453.625)(229.61816406,453.88671875)
\curveto(229.28222656,454.29296875)(228.84667969,454.49609375)(228.31152344,454.49609375)
\curveto(227.82714844,454.49609375)(227.41894531,454.33398438)(227.08691406,454.00976562)
\curveto(226.75878906,453.68554688)(226.57714844,453.25195312)(226.54199219,452.70898438)
\closepath
}
}
{
\newrgbcolor{curcolor}{0 0 0}
\pscustom[linestyle=none,fillstyle=solid,fillcolor=curcolor]
{
\newpath
\moveto(235.54199219,453.18359375)
\curveto(235.54199219,454.609375)(235.92480469,455.72460938)(236.69042969,456.52929688)
\curveto(237.45605469,457.33789062)(238.44433594,457.7421875)(239.65527344,457.7421875)
\curveto(240.44824219,457.7421875)(241.16308594,457.55273438)(241.79980469,457.17382812)
\curveto(242.43652344,456.79492188)(242.92089844,456.265625)(243.25292969,455.5859375)
\curveto(243.58886719,454.91015625)(243.75683594,454.14257812)(243.75683594,453.28320312)
\curveto(243.75683594,452.41210938)(243.58105469,451.6328125)(243.22949219,450.9453125)
\curveto(242.87792969,450.2578125)(242.37988281,449.73632812)(241.73535156,449.38085938)
\curveto(241.09082031,449.02929688)(240.39550781,448.85351562)(239.64941406,448.85351562)
\curveto(238.84082031,448.85351562)(238.11816406,449.04882812)(237.48144531,449.43945312)
\curveto(236.84472656,449.83007812)(236.36230469,450.36328125)(236.03417969,451.0390625)
\curveto(235.70605469,451.71484375)(235.54199219,452.4296875)(235.54199219,453.18359375)
\closepath
\moveto(236.71386719,453.16601562)
\curveto(236.71386719,452.13085938)(236.99121094,451.31445312)(237.54589844,450.71679688)
\curveto(238.10449219,450.12304688)(238.80371094,449.82617188)(239.64355469,449.82617188)
\curveto(240.49902344,449.82617188)(241.20214844,450.12695312)(241.75292969,450.72851562)
\curveto(242.30761719,451.33007812)(242.58496094,452.18359375)(242.58496094,453.2890625)
\curveto(242.58496094,453.98828125)(242.46582031,454.59765625)(242.22753906,455.1171875)
\curveto(241.99316406,455.640625)(241.64746094,456.04492188)(241.19042969,456.33007812)
\curveto(240.73730469,456.61914062)(240.22753906,456.76367188)(239.66113281,456.76367188)
\curveto(238.85644531,456.76367188)(238.16308594,456.48632812)(237.58105469,455.93164062)
\curveto(237.00292969,455.38085938)(236.71386719,454.45898438)(236.71386719,453.16601562)
\closepath
}
}
{
\newrgbcolor{curcolor}{0 0 0}
\pscustom[linestyle=none,fillstyle=solid,fillcolor=curcolor]
{
\newpath
\moveto(245.08691406,446.61523438)
\lineto(245.08691406,455.22265625)
\lineto(246.04785156,455.22265625)
\lineto(246.04785156,454.4140625)
\curveto(246.27441406,454.73046875)(246.53027344,454.96679688)(246.81542969,455.12304688)
\curveto(247.10058594,455.28320312)(247.44628906,455.36328125)(247.85253906,455.36328125)
\curveto(248.38378906,455.36328125)(248.85253906,455.2265625)(249.25878906,454.953125)
\curveto(249.66503906,454.6796875)(249.97167969,454.29296875)(250.17871094,453.79296875)
\curveto(250.38574219,453.296875)(250.48925781,452.75195312)(250.48925781,452.15820312)
\curveto(250.48925781,451.52148438)(250.37402344,450.94726562)(250.14355469,450.43554688)
\curveto(249.91699219,449.92773438)(249.58496094,449.53710938)(249.14746094,449.26367188)
\curveto(248.71386719,448.99414062)(248.25683594,448.859375)(247.77636719,448.859375)
\curveto(247.42480469,448.859375)(247.10839844,448.93359375)(246.82714844,449.08203125)
\curveto(246.54980469,449.23046875)(246.32128906,449.41796875)(246.14160156,449.64453125)
\lineto(246.14160156,446.61523438)
\closepath
\moveto(246.04199219,452.07617188)
\curveto(246.04199219,451.27539062)(246.20410156,450.68359375)(246.52832031,450.30078125)
\curveto(246.85253906,449.91796875)(247.24511719,449.7265625)(247.70605469,449.7265625)
\curveto(248.17480469,449.7265625)(248.57519531,449.92382812)(248.90722656,450.31835938)
\curveto(249.24316406,450.71679688)(249.41113281,451.33203125)(249.41113281,452.1640625)
\curveto(249.41113281,452.95703125)(249.24707031,453.55078125)(248.91894531,453.9453125)
\curveto(248.59472656,454.33984375)(248.20605469,454.53710938)(247.75292969,454.53710938)
\curveto(247.30371094,454.53710938)(246.90527344,454.32617188)(246.55761719,453.90429688)
\curveto(246.21386719,453.48632812)(246.04199219,452.87695312)(246.04199219,452.07617188)
\closepath
}
}
{
\newrgbcolor{curcolor}{0 0 0}
\pscustom[linestyle=none,fillstyle=solid,fillcolor=curcolor]
{
\newpath
\moveto(254.06347656,449.94335938)
\lineto(254.21582031,449.01171875)
\curveto(253.91894531,448.94921875)(253.65332031,448.91796875)(253.41894531,448.91796875)
\curveto(253.03613281,448.91796875)(252.73925781,448.97851562)(252.52832031,449.09960938)
\curveto(252.31738281,449.22070312)(252.16894531,449.37890625)(252.08300781,449.57421875)
\curveto(251.99707031,449.7734375)(251.95410156,450.18945312)(251.95410156,450.82226562)
\lineto(251.95410156,454.40234375)
\lineto(251.18066406,454.40234375)
\lineto(251.18066406,455.22265625)
\lineto(251.95410156,455.22265625)
\lineto(251.95410156,456.76367188)
\lineto(253.00292969,457.39648438)
\lineto(253.00292969,455.22265625)
\lineto(254.06347656,455.22265625)
\lineto(254.06347656,454.40234375)
\lineto(253.00292969,454.40234375)
\lineto(253.00292969,450.76367188)
\curveto(253.00292969,450.46289062)(253.02050781,450.26953125)(253.05566406,450.18359375)
\curveto(253.09472656,450.09765625)(253.15527344,450.02929688)(253.23730469,449.97851562)
\curveto(253.32324219,449.92773438)(253.44433594,449.90234375)(253.60058594,449.90234375)
\curveto(253.71777344,449.90234375)(253.87207031,449.91601562)(254.06347656,449.94335938)
\closepath
}
}
{
\newrgbcolor{curcolor}{0 0 0}
\pscustom[linestyle=none,fillstyle=solid,fillcolor=curcolor]
{
\newpath
\moveto(255.10058594,456.37695312)
\lineto(255.10058594,457.58984375)
\lineto(256.15527344,457.58984375)
\lineto(256.15527344,456.37695312)
\closepath
\moveto(255.10058594,449)
\lineto(255.10058594,455.22265625)
\lineto(256.15527344,455.22265625)
\lineto(256.15527344,449)
\closepath
}
}
{
\newrgbcolor{curcolor}{0 0 0}
\pscustom[linestyle=none,fillstyle=solid,fillcolor=curcolor]
{
\newpath
\moveto(257.76074219,449)
\lineto(257.76074219,455.22265625)
\lineto(258.70410156,455.22265625)
\lineto(258.70410156,454.34960938)
\curveto(258.89941406,454.65429688)(259.15917969,454.8984375)(259.48339844,455.08203125)
\curveto(259.80761719,455.26953125)(260.17675781,455.36328125)(260.59082031,455.36328125)
\curveto(261.05175781,455.36328125)(261.42871094,455.26757812)(261.72167969,455.07617188)
\curveto(262.01855469,454.88476562)(262.22753906,454.6171875)(262.34863281,454.2734375)
\curveto(262.84082031,455)(263.48144531,455.36328125)(264.27050781,455.36328125)
\curveto(264.88769531,455.36328125)(265.36230469,455.19140625)(265.69433594,454.84765625)
\curveto(266.02636719,454.5078125)(266.19238281,453.98242188)(266.19238281,453.27148438)
\lineto(266.19238281,449)
\lineto(265.14355469,449)
\lineto(265.14355469,452.91992188)
\curveto(265.14355469,453.34179688)(265.10839844,453.64453125)(265.03808594,453.828125)
\curveto(264.97167969,454.015625)(264.84863281,454.16601562)(264.66894531,454.27929688)
\curveto(264.48925781,454.39257812)(264.27832031,454.44921875)(264.03613281,454.44921875)
\curveto(263.59863281,454.44921875)(263.23535156,454.30273438)(262.94628906,454.00976562)
\curveto(262.65722656,453.72070312)(262.51269531,453.25585938)(262.51269531,452.61523438)
\lineto(262.51269531,449)
\lineto(261.45800781,449)
\lineto(261.45800781,453.04296875)
\curveto(261.45800781,453.51171875)(261.37207031,453.86328125)(261.20019531,454.09765625)
\curveto(261.02832031,454.33203125)(260.74707031,454.44921875)(260.35644531,454.44921875)
\curveto(260.05957031,454.44921875)(259.78417969,454.37109375)(259.53027344,454.21484375)
\curveto(259.28027344,454.05859375)(259.09863281,453.83007812)(258.98535156,453.52929688)
\curveto(258.87207031,453.22851562)(258.81542969,452.79492188)(258.81542969,452.22851562)
\lineto(258.81542969,449)
\closepath
}
}
{
\newrgbcolor{curcolor}{0 0 0}
\pscustom[linestyle=none,fillstyle=solid,fillcolor=curcolor]
{
\newpath
\moveto(271.81738281,449.76757812)
\curveto(271.42675781,449.43554688)(271.04980469,449.20117188)(270.68652344,449.06445312)
\curveto(270.32714844,448.92773438)(269.94042969,448.859375)(269.52636719,448.859375)
\curveto(268.84277344,448.859375)(268.31738281,449.02539062)(267.95019531,449.35742188)
\curveto(267.58300781,449.69335938)(267.39941406,450.12109375)(267.39941406,450.640625)
\curveto(267.39941406,450.9453125)(267.46777344,451.22265625)(267.60449219,451.47265625)
\curveto(267.74511719,451.7265625)(267.92675781,451.9296875)(268.14941406,452.08203125)
\curveto(268.37597656,452.234375)(268.62988281,452.34960938)(268.91113281,452.42773438)
\curveto(269.11816406,452.48242188)(269.43066406,452.53515625)(269.84863281,452.5859375)
\curveto(270.70019531,452.6875)(271.32714844,452.80859375)(271.72949219,452.94921875)
\curveto(271.73339844,453.09375)(271.73535156,453.18554688)(271.73535156,453.22460938)
\curveto(271.73535156,453.65429688)(271.63574219,453.95703125)(271.43652344,454.1328125)
\curveto(271.16699219,454.37109375)(270.76660156,454.49023438)(270.23535156,454.49023438)
\curveto(269.73925781,454.49023438)(269.37207031,454.40234375)(269.13378906,454.2265625)
\curveto(268.89941406,454.0546875)(268.72558594,453.74804688)(268.61230469,453.30664062)
\lineto(267.58105469,453.44726562)
\curveto(267.67480469,453.88867188)(267.82910156,454.24414062)(268.04394531,454.51367188)
\curveto(268.25878906,454.78710938)(268.56933594,454.99609375)(268.97558594,455.140625)
\curveto(269.38183594,455.2890625)(269.85253906,455.36328125)(270.38769531,455.36328125)
\curveto(270.91894531,455.36328125)(271.35058594,455.30078125)(271.68261719,455.17578125)
\curveto(272.01464844,455.05078125)(272.25878906,454.89257812)(272.41503906,454.70117188)
\curveto(272.57128906,454.51367188)(272.68066406,454.27539062)(272.74316406,453.98632812)
\curveto(272.77832031,453.80664062)(272.79589844,453.48242188)(272.79589844,453.01367188)
\lineto(272.79589844,451.60742188)
\curveto(272.79589844,450.62695312)(272.81738281,450.00585938)(272.86035156,449.74414062)
\curveto(272.90722656,449.48632812)(272.99707031,449.23828125)(273.12988281,449)
\lineto(272.02832031,449)
\curveto(271.91894531,449.21875)(271.84863281,449.47460938)(271.81738281,449.76757812)
\closepath
\moveto(271.72949219,452.12304688)
\curveto(271.34667969,451.96679688)(270.77246094,451.83398438)(270.00683594,451.72460938)
\curveto(269.57324219,451.66210938)(269.26660156,451.59179688)(269.08691406,451.51367188)
\curveto(268.90722656,451.43554688)(268.76855469,451.3203125)(268.67089844,451.16796875)
\curveto(268.57324219,451.01953125)(268.52441406,450.85351562)(268.52441406,450.66992188)
\curveto(268.52441406,450.38867188)(268.62988281,450.15429688)(268.84082031,449.96679688)
\curveto(269.05566406,449.77929688)(269.36816406,449.68554688)(269.77832031,449.68554688)
\curveto(270.18457031,449.68554688)(270.54589844,449.7734375)(270.86230469,449.94921875)
\curveto(271.17871094,450.12890625)(271.41113281,450.37304688)(271.55957031,450.68164062)
\curveto(271.67285156,450.91992188)(271.72949219,451.27148438)(271.72949219,451.73632812)
\closepath
}
}
{
\newrgbcolor{curcolor}{0 0 0}
\pscustom[linestyle=none,fillstyle=solid,fillcolor=curcolor]
{
\newpath
\moveto(274.40722656,449)
\lineto(274.40722656,457.58984375)
\lineto(275.46191406,457.58984375)
\lineto(275.46191406,449)
\closepath
}
}
{
\newrgbcolor{curcolor}{0 0 0}
\pscustom[linestyle=none,fillstyle=solid,fillcolor=curcolor]
{
\newpath
\moveto(284.11035156,449)
\lineto(283.05566406,449)
\lineto(283.05566406,455.72070312)
\curveto(282.80175781,455.47851562)(282.46777344,455.23632812)(282.05371094,454.99414062)
\curveto(281.64355469,454.75195312)(281.27441406,454.5703125)(280.94628906,454.44921875)
\lineto(280.94628906,455.46875)
\curveto(281.53613281,455.74609375)(282.05175781,456.08203125)(282.49316406,456.4765625)
\curveto(282.93457031,456.87109375)(283.24707031,457.25390625)(283.43066406,457.625)
\lineto(284.11035156,457.625)
\closepath
}
}
{
\newrgbcolor{curcolor}{0 0 0}
\pscustom[linestyle=none,fillstyle=solid,fillcolor=curcolor]
{
\newpath
\moveto(287.20410156,449)
\lineto(287.20410156,457.58984375)
\lineto(288.91503906,457.58984375)
\lineto(290.94824219,451.5078125)
\curveto(291.13574219,450.94140625)(291.27246094,450.51757812)(291.35839844,450.23632812)
\curveto(291.45605469,450.54882812)(291.60839844,451.0078125)(291.81542969,451.61328125)
\lineto(293.87207031,457.58984375)
\lineto(295.40136719,457.58984375)
\lineto(295.40136719,449)
\lineto(294.30566406,449)
\lineto(294.30566406,456.18945312)
\lineto(291.80957031,449)
\lineto(290.78417969,449)
\lineto(288.29980469,456.3125)
\lineto(288.29980469,449)
\closepath
}
}
{
\newrgbcolor{curcolor}{0 0 0}
\pscustom[linestyle=none,fillstyle=solid,fillcolor=curcolor]
{
\newpath
\moveto(300.58691406,449)
\lineto(300.58691406,457.58984375)
\lineto(304.39550781,457.58984375)
\curveto(305.16113281,457.58984375)(305.74316406,457.51171875)(306.14160156,457.35546875)
\curveto(306.54003906,457.203125)(306.85839844,456.93164062)(307.09667969,456.54101562)
\curveto(307.33496094,456.15039062)(307.45410156,455.71875)(307.45410156,455.24609375)
\curveto(307.45410156,454.63671875)(307.25683594,454.12304688)(306.86230469,453.70507812)
\curveto(306.46777344,453.28710938)(305.85839844,453.02148438)(305.03417969,452.90820312)
\curveto(305.33496094,452.76367188)(305.56347656,452.62109375)(305.71972656,452.48046875)
\curveto(306.05175781,452.17578125)(306.36621094,451.79492188)(306.66308594,451.33789062)
\lineto(308.15722656,449)
\lineto(306.72753906,449)
\lineto(305.59082031,450.78710938)
\curveto(305.25878906,451.30273438)(304.98535156,451.69726562)(304.77050781,451.97070312)
\curveto(304.55566406,452.24414062)(304.36230469,452.43554688)(304.19042969,452.54492188)
\curveto(304.02246094,452.65429688)(303.85058594,452.73046875)(303.67480469,452.7734375)
\curveto(303.54589844,452.80078125)(303.33496094,452.81445312)(303.04199219,452.81445312)
\lineto(301.72363281,452.81445312)
\lineto(301.72363281,449)
\closepath
\moveto(301.72363281,453.79882812)
\lineto(304.16699219,453.79882812)
\curveto(304.68652344,453.79882812)(305.09277344,453.8515625)(305.38574219,453.95703125)
\curveto(305.67871094,454.06640625)(305.90136719,454.23828125)(306.05371094,454.47265625)
\curveto(306.20605469,454.7109375)(306.28222656,454.96875)(306.28222656,455.24609375)
\curveto(306.28222656,455.65234375)(306.13378906,455.98632812)(305.83691406,456.24804688)
\curveto(305.54394531,456.50976562)(305.07910156,456.640625)(304.44238281,456.640625)
\lineto(301.72363281,456.640625)
\closepath
}
}
{
\newrgbcolor{curcolor}{0 0 0}
\pscustom[linestyle=none,fillstyle=solid,fillcolor=curcolor]
{
\newpath
\moveto(313.36035156,451.00390625)
\lineto(314.45019531,450.86914062)
\curveto(314.27832031,450.23242188)(313.95996094,449.73828125)(313.49511719,449.38671875)
\curveto(313.03027344,449.03515625)(312.43652344,448.859375)(311.71386719,448.859375)
\curveto(310.80371094,448.859375)(310.08105469,449.13867188)(309.54589844,449.69726562)
\curveto(309.01464844,450.25976562)(308.74902344,451.046875)(308.74902344,452.05859375)
\curveto(308.74902344,453.10546875)(309.01855469,453.91796875)(309.55761719,454.49609375)
\curveto(310.09667969,455.07421875)(310.79589844,455.36328125)(311.65527344,455.36328125)
\curveto(312.48730469,455.36328125)(313.16699219,455.08007812)(313.69433594,454.51367188)
\curveto(314.22167969,453.94726562)(314.48535156,453.15039062)(314.48535156,452.12304688)
\curveto(314.48535156,452.06054688)(314.48339844,451.96679688)(314.47949219,451.84179688)
\lineto(309.83886719,451.84179688)
\curveto(309.87792969,451.15820312)(310.07128906,450.63476562)(310.41894531,450.27148438)
\curveto(310.76660156,449.90820312)(311.20019531,449.7265625)(311.71972656,449.7265625)
\curveto(312.10644531,449.7265625)(312.43652344,449.828125)(312.70996094,450.03125)
\curveto(312.98339844,450.234375)(313.20019531,450.55859375)(313.36035156,451.00390625)
\closepath
\moveto(309.89746094,452.70898438)
\lineto(313.37207031,452.70898438)
\curveto(313.32519531,453.23242188)(313.19238281,453.625)(312.97363281,453.88671875)
\curveto(312.63769531,454.29296875)(312.20214844,454.49609375)(311.66699219,454.49609375)
\curveto(311.18261719,454.49609375)(310.77441406,454.33398438)(310.44238281,454.00976562)
\curveto(310.11425781,453.68554688)(309.93261719,453.25195312)(309.89746094,452.70898438)
\closepath
}
}
{
\newrgbcolor{curcolor}{0 0 0}
\pscustom[linestyle=none,fillstyle=solid,fillcolor=curcolor]
{
\newpath
\moveto(319.83496094,449.76757812)
\curveto(319.44433594,449.43554688)(319.06738281,449.20117188)(318.70410156,449.06445312)
\curveto(318.34472656,448.92773438)(317.95800781,448.859375)(317.54394531,448.859375)
\curveto(316.86035156,448.859375)(316.33496094,449.02539062)(315.96777344,449.35742188)
\curveto(315.60058594,449.69335938)(315.41699219,450.12109375)(315.41699219,450.640625)
\curveto(315.41699219,450.9453125)(315.48535156,451.22265625)(315.62207031,451.47265625)
\curveto(315.76269531,451.7265625)(315.94433594,451.9296875)(316.16699219,452.08203125)
\curveto(316.39355469,452.234375)(316.64746094,452.34960938)(316.92871094,452.42773438)
\curveto(317.13574219,452.48242188)(317.44824219,452.53515625)(317.86621094,452.5859375)
\curveto(318.71777344,452.6875)(319.34472656,452.80859375)(319.74707031,452.94921875)
\curveto(319.75097656,453.09375)(319.75292969,453.18554688)(319.75292969,453.22460938)
\curveto(319.75292969,453.65429688)(319.65332031,453.95703125)(319.45410156,454.1328125)
\curveto(319.18457031,454.37109375)(318.78417969,454.49023438)(318.25292969,454.49023438)
\curveto(317.75683594,454.49023438)(317.38964844,454.40234375)(317.15136719,454.2265625)
\curveto(316.91699219,454.0546875)(316.74316406,453.74804688)(316.62988281,453.30664062)
\lineto(315.59863281,453.44726562)
\curveto(315.69238281,453.88867188)(315.84667969,454.24414062)(316.06152344,454.51367188)
\curveto(316.27636719,454.78710938)(316.58691406,454.99609375)(316.99316406,455.140625)
\curveto(317.39941406,455.2890625)(317.87011719,455.36328125)(318.40527344,455.36328125)
\curveto(318.93652344,455.36328125)(319.36816406,455.30078125)(319.70019531,455.17578125)
\curveto(320.03222656,455.05078125)(320.27636719,454.89257812)(320.43261719,454.70117188)
\curveto(320.58886719,454.51367188)(320.69824219,454.27539062)(320.76074219,453.98632812)
\curveto(320.79589844,453.80664062)(320.81347656,453.48242188)(320.81347656,453.01367188)
\lineto(320.81347656,451.60742188)
\curveto(320.81347656,450.62695312)(320.83496094,450.00585938)(320.87792969,449.74414062)
\curveto(320.92480469,449.48632812)(321.01464844,449.23828125)(321.14746094,449)
\lineto(320.04589844,449)
\curveto(319.93652344,449.21875)(319.86621094,449.47460938)(319.83496094,449.76757812)
\closepath
\moveto(319.74707031,452.12304688)
\curveto(319.36425781,451.96679688)(318.79003906,451.83398438)(318.02441406,451.72460938)
\curveto(317.59082031,451.66210938)(317.28417969,451.59179688)(317.10449219,451.51367188)
\curveto(316.92480469,451.43554688)(316.78613281,451.3203125)(316.68847656,451.16796875)
\curveto(316.59082031,451.01953125)(316.54199219,450.85351562)(316.54199219,450.66992188)
\curveto(316.54199219,450.38867188)(316.64746094,450.15429688)(316.85839844,449.96679688)
\curveto(317.07324219,449.77929688)(317.38574219,449.68554688)(317.79589844,449.68554688)
\curveto(318.20214844,449.68554688)(318.56347656,449.7734375)(318.87988281,449.94921875)
\curveto(319.19628906,450.12890625)(319.42871094,450.37304688)(319.57714844,450.68164062)
\curveto(319.69042969,450.91992188)(319.74707031,451.27148438)(319.74707031,451.73632812)
\closepath
}
}
{
\newrgbcolor{curcolor}{0 0 0}
\pscustom[linestyle=none,fillstyle=solid,fillcolor=curcolor]
{
\newpath
\moveto(326.48535156,449)
\lineto(326.48535156,449.78515625)
\curveto(326.09082031,449.16796875)(325.51074219,448.859375)(324.74511719,448.859375)
\curveto(324.24902344,448.859375)(323.79199219,448.99609375)(323.37402344,449.26953125)
\curveto(322.95996094,449.54296875)(322.63769531,449.92382812)(322.40722656,450.41210938)
\curveto(322.18066406,450.90429688)(322.06738281,451.46875)(322.06738281,452.10546875)
\curveto(322.06738281,452.7265625)(322.17089844,453.2890625)(322.37792969,453.79296875)
\curveto(322.58496094,454.30078125)(322.89550781,454.68945312)(323.30957031,454.95898438)
\curveto(323.72363281,455.22851562)(324.18652344,455.36328125)(324.69824219,455.36328125)
\curveto(325.07324219,455.36328125)(325.40722656,455.28320312)(325.70019531,455.12304688)
\curveto(325.99316406,454.96679688)(326.23144531,454.76171875)(326.41503906,454.5078125)
\lineto(326.41503906,457.58984375)
\lineto(327.46386719,457.58984375)
\lineto(327.46386719,449)
\closepath
\moveto(323.15136719,452.10546875)
\curveto(323.15136719,451.30859375)(323.31933594,450.71289062)(323.65527344,450.31835938)
\curveto(323.99121094,449.92382812)(324.38769531,449.7265625)(324.84472656,449.7265625)
\curveto(325.30566406,449.7265625)(325.69628906,449.9140625)(326.01660156,450.2890625)
\curveto(326.34082031,450.66796875)(326.50292969,451.24414062)(326.50292969,452.01757812)
\curveto(326.50292969,452.86914062)(326.33886719,453.49414062)(326.01074219,453.89257812)
\curveto(325.68261719,454.29101562)(325.27832031,454.49023438)(324.79785156,454.49023438)
\curveto(324.32910156,454.49023438)(323.93652344,454.29882812)(323.62011719,453.91601562)
\curveto(323.30761719,453.53320312)(323.15136719,452.9296875)(323.15136719,452.10546875)
\closepath
}
}
{
\newrgbcolor{curcolor}{0 0 0}
\pscustom[linestyle=none,fillstyle=solid,fillcolor=curcolor]
{
\newpath
\moveto(328.70019531,450.85742188)
\lineto(329.74316406,451.02148438)
\curveto(329.80175781,450.60351562)(329.96386719,450.28320312)(330.22949219,450.06054688)
\curveto(330.49902344,449.83789062)(330.87402344,449.7265625)(331.35449219,449.7265625)
\curveto(331.83886719,449.7265625)(332.19824219,449.82421875)(332.43261719,450.01953125)
\curveto(332.66699219,450.21875)(332.78417969,450.45117188)(332.78417969,450.71679688)
\curveto(332.78417969,450.95507812)(332.68066406,451.14257812)(332.47363281,451.27929688)
\curveto(332.32910156,451.37304688)(331.96972656,451.4921875)(331.39550781,451.63671875)
\curveto(330.62207031,451.83203125)(330.08496094,452)(329.78417969,452.140625)
\curveto(329.48730469,452.28515625)(329.26074219,452.48242188)(329.10449219,452.73242188)
\curveto(328.95214844,452.98632812)(328.87597656,453.265625)(328.87597656,453.5703125)
\curveto(328.87597656,453.84765625)(328.93847656,454.10351562)(329.06347656,454.33789062)
\curveto(329.19238281,454.57617188)(329.36621094,454.7734375)(329.58496094,454.9296875)
\curveto(329.74902344,455.05078125)(329.97167969,455.15234375)(330.25292969,455.234375)
\curveto(330.53808594,455.3203125)(330.84277344,455.36328125)(331.16699219,455.36328125)
\curveto(331.65527344,455.36328125)(332.08300781,455.29296875)(332.45019531,455.15234375)
\curveto(332.82128906,455.01171875)(333.09472656,454.8203125)(333.27050781,454.578125)
\curveto(333.44628906,454.33984375)(333.56738281,454.01953125)(333.63378906,453.6171875)
\lineto(332.60253906,453.4765625)
\curveto(332.55566406,453.796875)(332.41894531,454.046875)(332.19238281,454.2265625)
\curveto(331.96972656,454.40625)(331.65332031,454.49609375)(331.24316406,454.49609375)
\curveto(330.75878906,454.49609375)(330.41308594,454.41601562)(330.20605469,454.25585938)
\curveto(329.99902344,454.09570312)(329.89550781,453.90820312)(329.89550781,453.69335938)
\curveto(329.89550781,453.55664062)(329.93847656,453.43359375)(330.02441406,453.32421875)
\curveto(330.11035156,453.2109375)(330.24511719,453.1171875)(330.42871094,453.04296875)
\curveto(330.53417969,453.00390625)(330.84472656,452.9140625)(331.36035156,452.7734375)
\curveto(332.10644531,452.57421875)(332.62597656,452.41015625)(332.91894531,452.28125)
\curveto(333.21582031,452.15625)(333.44824219,451.97265625)(333.61621094,451.73046875)
\curveto(333.78417969,451.48828125)(333.86816406,451.1875)(333.86816406,450.828125)
\curveto(333.86816406,450.4765625)(333.76464844,450.14453125)(333.55761719,449.83203125)
\curveto(333.35449219,449.5234375)(333.05957031,449.28320312)(332.67285156,449.11132812)
\curveto(332.28613281,448.94335938)(331.84863281,448.859375)(331.36035156,448.859375)
\curveto(330.55175781,448.859375)(329.93457031,449.02734375)(329.50878906,449.36328125)
\curveto(329.08691406,449.69921875)(328.81738281,450.19726562)(328.70019531,450.85742188)
\closepath
}
}
{
\newrgbcolor{curcolor}{0 0 0}
\pscustom[linestyle=none,fillstyle=solid,fillcolor=curcolor]
{
\newpath
\moveto(338.20410156,451.75976562)
\lineto(339.27636719,451.85351562)
\curveto(339.32714844,451.42382812)(339.44433594,451.0703125)(339.62792969,450.79296875)
\curveto(339.81542969,450.51953125)(340.10449219,450.296875)(340.49511719,450.125)
\curveto(340.88574219,449.95703125)(341.32519531,449.87304688)(341.81347656,449.87304688)
\curveto(342.24707031,449.87304688)(342.62988281,449.9375)(342.96191406,450.06640625)
\curveto(343.29394531,450.1953125)(343.54003906,450.37109375)(343.70019531,450.59375)
\curveto(343.86425781,450.8203125)(343.94628906,451.06640625)(343.94628906,451.33203125)
\curveto(343.94628906,451.6015625)(343.86816406,451.8359375)(343.71191406,452.03515625)
\curveto(343.55566406,452.23828125)(343.29785156,452.40820312)(342.93847656,452.54492188)
\curveto(342.70800781,452.63476562)(342.19824219,452.7734375)(341.40917969,452.9609375)
\curveto(340.62011719,453.15234375)(340.06738281,453.33203125)(339.75097656,453.5)
\curveto(339.34082031,453.71484375)(339.03417969,453.98046875)(338.83105469,454.296875)
\curveto(338.63183594,454.6171875)(338.53222656,454.97460938)(338.53222656,455.36914062)
\curveto(338.53222656,455.80273438)(338.65527344,456.20703125)(338.90136719,456.58203125)
\curveto(339.14746094,456.9609375)(339.50683594,457.24804688)(339.97949219,457.44335938)
\curveto(340.45214844,457.63867188)(340.97753906,457.73632812)(341.55566406,457.73632812)
\curveto(342.19238281,457.73632812)(342.75292969,457.6328125)(343.23730469,457.42578125)
\curveto(343.72558594,457.22265625)(344.10058594,456.921875)(344.36230469,456.5234375)
\curveto(344.62402344,456.125)(344.76464844,455.67382812)(344.78417969,455.16992188)
\lineto(343.69433594,455.08789062)
\curveto(343.63574219,455.63085938)(343.43652344,456.04101562)(343.09667969,456.31835938)
\curveto(342.76074219,456.59570312)(342.26269531,456.734375)(341.60253906,456.734375)
\curveto(340.91503906,456.734375)(340.41308594,456.60742188)(340.09667969,456.35351562)
\curveto(339.78417969,456.10351562)(339.62792969,455.80078125)(339.62792969,455.4453125)
\curveto(339.62792969,455.13671875)(339.73925781,454.8828125)(339.96191406,454.68359375)
\curveto(340.18066406,454.484375)(340.75097656,454.27929688)(341.67285156,454.06835938)
\curveto(342.59863281,453.86132812)(343.23339844,453.6796875)(343.57714844,453.5234375)
\curveto(344.07714844,453.29296875)(344.44628906,453)(344.68457031,452.64453125)
\curveto(344.92285156,452.29296875)(345.04199219,451.88671875)(345.04199219,451.42578125)
\curveto(345.04199219,450.96875)(344.91113281,450.53710938)(344.64941406,450.13085938)
\curveto(344.38769531,449.72851562)(344.01074219,449.4140625)(343.51855469,449.1875)
\curveto(343.03027344,448.96484375)(342.47949219,448.85351562)(341.86621094,448.85351562)
\curveto(341.08886719,448.85351562)(340.43652344,448.96679688)(339.90917969,449.19335938)
\curveto(339.38574219,449.41992188)(338.97363281,449.75976562)(338.67285156,450.21289062)
\curveto(338.37597656,450.66992188)(338.21972656,451.18554688)(338.20410156,451.75976562)
\closepath
}
}
{
\newrgbcolor{curcolor}{0 0 0}
\pscustom[linestyle=none,fillstyle=solid,fillcolor=curcolor]
{
\newpath
\moveto(346.45996094,449)
\lineto(346.45996094,455.22265625)
\lineto(347.40332031,455.22265625)
\lineto(347.40332031,454.34960938)
\curveto(347.59863281,454.65429688)(347.85839844,454.8984375)(348.18261719,455.08203125)
\curveto(348.50683594,455.26953125)(348.87597656,455.36328125)(349.29003906,455.36328125)
\curveto(349.75097656,455.36328125)(350.12792969,455.26757812)(350.42089844,455.07617188)
\curveto(350.71777344,454.88476562)(350.92675781,454.6171875)(351.04785156,454.2734375)
\curveto(351.54003906,455)(352.18066406,455.36328125)(352.96972656,455.36328125)
\curveto(353.58691406,455.36328125)(354.06152344,455.19140625)(354.39355469,454.84765625)
\curveto(354.72558594,454.5078125)(354.89160156,453.98242188)(354.89160156,453.27148438)
\lineto(354.89160156,449)
\lineto(353.84277344,449)
\lineto(353.84277344,452.91992188)
\curveto(353.84277344,453.34179688)(353.80761719,453.64453125)(353.73730469,453.828125)
\curveto(353.67089844,454.015625)(353.54785156,454.16601562)(353.36816406,454.27929688)
\curveto(353.18847656,454.39257812)(352.97753906,454.44921875)(352.73535156,454.44921875)
\curveto(352.29785156,454.44921875)(351.93457031,454.30273438)(351.64550781,454.00976562)
\curveto(351.35644531,453.72070312)(351.21191406,453.25585938)(351.21191406,452.61523438)
\lineto(351.21191406,449)
\lineto(350.15722656,449)
\lineto(350.15722656,453.04296875)
\curveto(350.15722656,453.51171875)(350.07128906,453.86328125)(349.89941406,454.09765625)
\curveto(349.72753906,454.33203125)(349.44628906,454.44921875)(349.05566406,454.44921875)
\curveto(348.75878906,454.44921875)(348.48339844,454.37109375)(348.22949219,454.21484375)
\curveto(347.97949219,454.05859375)(347.79785156,453.83007812)(347.68457031,453.52929688)
\curveto(347.57128906,453.22851562)(347.51464844,452.79492188)(347.51464844,452.22851562)
\lineto(347.51464844,449)
\closepath
}
}
{
\newrgbcolor{curcolor}{0 0 0}
\pscustom[linestyle=none,fillstyle=solid,fillcolor=curcolor]
{
\newpath
\moveto(360.51660156,449.76757812)
\curveto(360.12597656,449.43554688)(359.74902344,449.20117188)(359.38574219,449.06445312)
\curveto(359.02636719,448.92773438)(358.63964844,448.859375)(358.22558594,448.859375)
\curveto(357.54199219,448.859375)(357.01660156,449.02539062)(356.64941406,449.35742188)
\curveto(356.28222656,449.69335938)(356.09863281,450.12109375)(356.09863281,450.640625)
\curveto(356.09863281,450.9453125)(356.16699219,451.22265625)(356.30371094,451.47265625)
\curveto(356.44433594,451.7265625)(356.62597656,451.9296875)(356.84863281,452.08203125)
\curveto(357.07519531,452.234375)(357.32910156,452.34960938)(357.61035156,452.42773438)
\curveto(357.81738281,452.48242188)(358.12988281,452.53515625)(358.54785156,452.5859375)
\curveto(359.39941406,452.6875)(360.02636719,452.80859375)(360.42871094,452.94921875)
\curveto(360.43261719,453.09375)(360.43457031,453.18554688)(360.43457031,453.22460938)
\curveto(360.43457031,453.65429688)(360.33496094,453.95703125)(360.13574219,454.1328125)
\curveto(359.86621094,454.37109375)(359.46582031,454.49023438)(358.93457031,454.49023438)
\curveto(358.43847656,454.49023438)(358.07128906,454.40234375)(357.83300781,454.2265625)
\curveto(357.59863281,454.0546875)(357.42480469,453.74804688)(357.31152344,453.30664062)
\lineto(356.28027344,453.44726562)
\curveto(356.37402344,453.88867188)(356.52832031,454.24414062)(356.74316406,454.51367188)
\curveto(356.95800781,454.78710938)(357.26855469,454.99609375)(357.67480469,455.140625)
\curveto(358.08105469,455.2890625)(358.55175781,455.36328125)(359.08691406,455.36328125)
\curveto(359.61816406,455.36328125)(360.04980469,455.30078125)(360.38183594,455.17578125)
\curveto(360.71386719,455.05078125)(360.95800781,454.89257812)(361.11425781,454.70117188)
\curveto(361.27050781,454.51367188)(361.37988281,454.27539062)(361.44238281,453.98632812)
\curveto(361.47753906,453.80664062)(361.49511719,453.48242188)(361.49511719,453.01367188)
\lineto(361.49511719,451.60742188)
\curveto(361.49511719,450.62695312)(361.51660156,450.00585938)(361.55957031,449.74414062)
\curveto(361.60644531,449.48632812)(361.69628906,449.23828125)(361.82910156,449)
\lineto(360.72753906,449)
\curveto(360.61816406,449.21875)(360.54785156,449.47460938)(360.51660156,449.76757812)
\closepath
\moveto(360.42871094,452.12304688)
\curveto(360.04589844,451.96679688)(359.47167969,451.83398438)(358.70605469,451.72460938)
\curveto(358.27246094,451.66210938)(357.96582031,451.59179688)(357.78613281,451.51367188)
\curveto(357.60644531,451.43554688)(357.46777344,451.3203125)(357.37011719,451.16796875)
\curveto(357.27246094,451.01953125)(357.22363281,450.85351562)(357.22363281,450.66992188)
\curveto(357.22363281,450.38867188)(357.32910156,450.15429688)(357.54003906,449.96679688)
\curveto(357.75488281,449.77929688)(358.06738281,449.68554688)(358.47753906,449.68554688)
\curveto(358.88378906,449.68554688)(359.24511719,449.7734375)(359.56152344,449.94921875)
\curveto(359.87792969,450.12890625)(360.11035156,450.37304688)(360.25878906,450.68164062)
\curveto(360.37207031,450.91992188)(360.42871094,451.27148438)(360.42871094,451.73632812)
\closepath
}
}
{
\newrgbcolor{curcolor}{0 0 0}
\pscustom[linestyle=none,fillstyle=solid,fillcolor=curcolor]
{
\newpath
\moveto(363.10644531,449)
\lineto(363.10644531,457.58984375)
\lineto(364.16113281,457.58984375)
\lineto(364.16113281,449)
\closepath
}
}
{
\newrgbcolor{curcolor}{0 0 0}
\pscustom[linestyle=none,fillstyle=solid,fillcolor=curcolor]
{
\newpath
\moveto(365.77246094,449)
\lineto(365.77246094,457.58984375)
\lineto(366.82714844,457.58984375)
\lineto(366.82714844,449)
\closepath
}
}
{
\newrgbcolor{curcolor}{0 0 0}
\pscustom[linestyle=none,fillstyle=solid,fillcolor=curcolor]
{
\newpath
\moveto(371.93066406,449)
\lineto(371.93066406,457.58984375)
\lineto(375.17089844,457.58984375)
\curveto(375.74121094,457.58984375)(376.17675781,457.5625)(376.47753906,457.5078125)
\curveto(376.89941406,457.4375)(377.25292969,457.30273438)(377.53808594,457.10351562)
\curveto(377.82324219,456.90820312)(378.05175781,456.6328125)(378.22363281,456.27734375)
\curveto(378.39941406,455.921875)(378.48730469,455.53125)(378.48730469,455.10546875)
\curveto(378.48730469,454.375)(378.25488281,453.75585938)(377.79003906,453.24804688)
\curveto(377.32519531,452.74414062)(376.48535156,452.4921875)(375.27050781,452.4921875)
\lineto(373.06738281,452.4921875)
\lineto(373.06738281,449)
\closepath
\moveto(373.06738281,453.50585938)
\lineto(375.28808594,453.50585938)
\curveto(376.02246094,453.50585938)(376.54394531,453.64257812)(376.85253906,453.91601562)
\curveto(377.16113281,454.18945312)(377.31542969,454.57421875)(377.31542969,455.0703125)
\curveto(377.31542969,455.4296875)(377.22363281,455.73632812)(377.04003906,455.99023438)
\curveto(376.86035156,456.24804688)(376.62207031,456.41796875)(376.32519531,456.5)
\curveto(376.13378906,456.55078125)(375.78027344,456.57617188)(375.26464844,456.57617188)
\lineto(373.06738281,456.57617188)
\closepath
}
}
{
\newrgbcolor{curcolor}{0 0 0}
\pscustom[linestyle=none,fillstyle=solid,fillcolor=curcolor]
{
\newpath
\moveto(379.78808594,449)
\lineto(379.78808594,455.22265625)
\lineto(380.73730469,455.22265625)
\lineto(380.73730469,454.27929688)
\curveto(380.97949219,454.72070312)(381.20214844,455.01171875)(381.40527344,455.15234375)
\curveto(381.61230469,455.29296875)(381.83886719,455.36328125)(382.08496094,455.36328125)
\curveto(382.44042969,455.36328125)(382.80175781,455.25)(383.16894531,455.0234375)
\lineto(382.80566406,454.04492188)
\curveto(382.54785156,454.19726562)(382.29003906,454.2734375)(382.03222656,454.2734375)
\curveto(381.80175781,454.2734375)(381.59472656,454.203125)(381.41113281,454.0625)
\curveto(381.22753906,453.92578125)(381.09667969,453.734375)(381.01855469,453.48828125)
\curveto(380.90136719,453.11328125)(380.84277344,452.703125)(380.84277344,452.2578125)
\lineto(380.84277344,449)
\closepath
}
}
{
\newrgbcolor{curcolor}{0 0 0}
\pscustom[linestyle=none,fillstyle=solid,fillcolor=curcolor]
{
\newpath
\moveto(383.40332031,452.11132812)
\curveto(383.40332031,453.26367188)(383.72363281,454.1171875)(384.36425781,454.671875)
\curveto(384.89941406,455.1328125)(385.55175781,455.36328125)(386.32128906,455.36328125)
\curveto(387.17675781,455.36328125)(387.87597656,455.08203125)(388.41894531,454.51953125)
\curveto(388.96191406,453.9609375)(389.23339844,453.1875)(389.23339844,452.19921875)
\curveto(389.23339844,451.3984375)(389.11230469,450.76757812)(388.87011719,450.30664062)
\curveto(388.63183594,449.84960938)(388.28222656,449.49414062)(387.82128906,449.24023438)
\curveto(387.36425781,448.98632812)(386.86425781,448.859375)(386.32128906,448.859375)
\curveto(385.45019531,448.859375)(384.74511719,449.13867188)(384.20605469,449.69726562)
\curveto(383.67089844,450.25585938)(383.40332031,451.06054688)(383.40332031,452.11132812)
\closepath
\moveto(384.48730469,452.11132812)
\curveto(384.48730469,451.31445312)(384.66113281,450.71679688)(385.00878906,450.31835938)
\curveto(385.35644531,449.92382812)(385.79394531,449.7265625)(386.32128906,449.7265625)
\curveto(386.84472656,449.7265625)(387.28027344,449.92578125)(387.62792969,450.32421875)
\curveto(387.97558594,450.72265625)(388.14941406,451.33007812)(388.14941406,452.14648438)
\curveto(388.14941406,452.91601562)(387.97363281,453.49804688)(387.62207031,453.89257812)
\curveto(387.27441406,454.29101562)(386.84082031,454.49023438)(386.32128906,454.49023438)
\curveto(385.79394531,454.49023438)(385.35644531,454.29296875)(385.00878906,453.8984375)
\curveto(384.66113281,453.50390625)(384.48730469,452.90820312)(384.48730469,452.11132812)
\closepath
}
}
{
\newrgbcolor{curcolor}{0 0 0}
\pscustom[linestyle=none,fillstyle=solid,fillcolor=curcolor]
{
\newpath
\moveto(394.53027344,451.27929688)
\lineto(395.56738281,451.14453125)
\curveto(395.45410156,450.4296875)(395.16308594,449.86914062)(394.69433594,449.46289062)
\curveto(394.22949219,449.06054688)(393.65722656,448.859375)(392.97753906,448.859375)
\curveto(392.12597656,448.859375)(391.44042969,449.13671875)(390.92089844,449.69140625)
\curveto(390.40527344,450.25)(390.14746094,451.04882812)(390.14746094,452.08789062)
\curveto(390.14746094,452.75976562)(390.25878906,453.34765625)(390.48144531,453.8515625)
\curveto(390.70410156,454.35546875)(391.04199219,454.73242188)(391.49511719,454.98242188)
\curveto(391.95214844,455.23632812)(392.44824219,455.36328125)(392.98339844,455.36328125)
\curveto(393.65917969,455.36328125)(394.21191406,455.19140625)(394.64160156,454.84765625)
\curveto(395.07128906,454.5078125)(395.34667969,454.0234375)(395.46777344,453.39453125)
\lineto(394.44238281,453.23632812)
\curveto(394.34472656,453.65429688)(394.17089844,453.96875)(393.92089844,454.1796875)
\curveto(393.67480469,454.390625)(393.37597656,454.49609375)(393.02441406,454.49609375)
\curveto(392.49316406,454.49609375)(392.06152344,454.3046875)(391.72949219,453.921875)
\curveto(391.39746094,453.54296875)(391.23144531,452.94140625)(391.23144531,452.1171875)
\curveto(391.23144531,451.28125)(391.39160156,450.67382812)(391.71191406,450.29492188)
\curveto(392.03222656,449.91601562)(392.45019531,449.7265625)(392.96582031,449.7265625)
\curveto(393.37988281,449.7265625)(393.72558594,449.85351562)(394.00292969,450.10742188)
\curveto(394.28027344,450.36132812)(394.45605469,450.75195312)(394.53027344,451.27929688)
\closepath
}
}
{
\newrgbcolor{curcolor}{0 0 0}
\pscustom[linestyle=none,fillstyle=solid,fillcolor=curcolor]
{
\newpath
\moveto(400.72949219,451.00390625)
\lineto(401.81933594,450.86914062)
\curveto(401.64746094,450.23242188)(401.32910156,449.73828125)(400.86425781,449.38671875)
\curveto(400.39941406,449.03515625)(399.80566406,448.859375)(399.08300781,448.859375)
\curveto(398.17285156,448.859375)(397.45019531,449.13867188)(396.91503906,449.69726562)
\curveto(396.38378906,450.25976562)(396.11816406,451.046875)(396.11816406,452.05859375)
\curveto(396.11816406,453.10546875)(396.38769531,453.91796875)(396.92675781,454.49609375)
\curveto(397.46582031,455.07421875)(398.16503906,455.36328125)(399.02441406,455.36328125)
\curveto(399.85644531,455.36328125)(400.53613281,455.08007812)(401.06347656,454.51367188)
\curveto(401.59082031,453.94726562)(401.85449219,453.15039062)(401.85449219,452.12304688)
\curveto(401.85449219,452.06054688)(401.85253906,451.96679688)(401.84863281,451.84179688)
\lineto(397.20800781,451.84179688)
\curveto(397.24707031,451.15820312)(397.44042969,450.63476562)(397.78808594,450.27148438)
\curveto(398.13574219,449.90820312)(398.56933594,449.7265625)(399.08886719,449.7265625)
\curveto(399.47558594,449.7265625)(399.80566406,449.828125)(400.07910156,450.03125)
\curveto(400.35253906,450.234375)(400.56933594,450.55859375)(400.72949219,451.00390625)
\closepath
\moveto(397.26660156,452.70898438)
\lineto(400.74121094,452.70898438)
\curveto(400.69433594,453.23242188)(400.56152344,453.625)(400.34277344,453.88671875)
\curveto(400.00683594,454.29296875)(399.57128906,454.49609375)(399.03613281,454.49609375)
\curveto(398.55175781,454.49609375)(398.14355469,454.33398438)(397.81152344,454.00976562)
\curveto(397.48339844,453.68554688)(397.30175781,453.25195312)(397.26660156,452.70898438)
\closepath
}
}
{
\newrgbcolor{curcolor}{0 0 0}
\pscustom[linestyle=none,fillstyle=solid,fillcolor=curcolor]
{
\newpath
\moveto(402.72167969,450.85742188)
\lineto(403.76464844,451.02148438)
\curveto(403.82324219,450.60351562)(403.98535156,450.28320312)(404.25097656,450.06054688)
\curveto(404.52050781,449.83789062)(404.89550781,449.7265625)(405.37597656,449.7265625)
\curveto(405.86035156,449.7265625)(406.21972656,449.82421875)(406.45410156,450.01953125)
\curveto(406.68847656,450.21875)(406.80566406,450.45117188)(406.80566406,450.71679688)
\curveto(406.80566406,450.95507812)(406.70214844,451.14257812)(406.49511719,451.27929688)
\curveto(406.35058594,451.37304688)(405.99121094,451.4921875)(405.41699219,451.63671875)
\curveto(404.64355469,451.83203125)(404.10644531,452)(403.80566406,452.140625)
\curveto(403.50878906,452.28515625)(403.28222656,452.48242188)(403.12597656,452.73242188)
\curveto(402.97363281,452.98632812)(402.89746094,453.265625)(402.89746094,453.5703125)
\curveto(402.89746094,453.84765625)(402.95996094,454.10351562)(403.08496094,454.33789062)
\curveto(403.21386719,454.57617188)(403.38769531,454.7734375)(403.60644531,454.9296875)
\curveto(403.77050781,455.05078125)(403.99316406,455.15234375)(404.27441406,455.234375)
\curveto(404.55957031,455.3203125)(404.86425781,455.36328125)(405.18847656,455.36328125)
\curveto(405.67675781,455.36328125)(406.10449219,455.29296875)(406.47167969,455.15234375)
\curveto(406.84277344,455.01171875)(407.11621094,454.8203125)(407.29199219,454.578125)
\curveto(407.46777344,454.33984375)(407.58886719,454.01953125)(407.65527344,453.6171875)
\lineto(406.62402344,453.4765625)
\curveto(406.57714844,453.796875)(406.44042969,454.046875)(406.21386719,454.2265625)
\curveto(405.99121094,454.40625)(405.67480469,454.49609375)(405.26464844,454.49609375)
\curveto(404.78027344,454.49609375)(404.43457031,454.41601562)(404.22753906,454.25585938)
\curveto(404.02050781,454.09570312)(403.91699219,453.90820312)(403.91699219,453.69335938)
\curveto(403.91699219,453.55664062)(403.95996094,453.43359375)(404.04589844,453.32421875)
\curveto(404.13183594,453.2109375)(404.26660156,453.1171875)(404.45019531,453.04296875)
\curveto(404.55566406,453.00390625)(404.86621094,452.9140625)(405.38183594,452.7734375)
\curveto(406.12792969,452.57421875)(406.64746094,452.41015625)(406.94042969,452.28125)
\curveto(407.23730469,452.15625)(407.46972656,451.97265625)(407.63769531,451.73046875)
\curveto(407.80566406,451.48828125)(407.88964844,451.1875)(407.88964844,450.828125)
\curveto(407.88964844,450.4765625)(407.78613281,450.14453125)(407.57910156,449.83203125)
\curveto(407.37597656,449.5234375)(407.08105469,449.28320312)(406.69433594,449.11132812)
\curveto(406.30761719,448.94335938)(405.87011719,448.859375)(405.38183594,448.859375)
\curveto(404.57324219,448.859375)(403.95605469,449.02734375)(403.53027344,449.36328125)
\curveto(403.10839844,449.69921875)(402.83886719,450.19726562)(402.72167969,450.85742188)
\closepath
}
}
{
\newrgbcolor{curcolor}{0 0 0}
\pscustom[linestyle=none,fillstyle=solid,fillcolor=curcolor]
{
\newpath
\moveto(408.72167969,450.85742188)
\lineto(409.76464844,451.02148438)
\curveto(409.82324219,450.60351562)(409.98535156,450.28320312)(410.25097656,450.06054688)
\curveto(410.52050781,449.83789062)(410.89550781,449.7265625)(411.37597656,449.7265625)
\curveto(411.86035156,449.7265625)(412.21972656,449.82421875)(412.45410156,450.01953125)
\curveto(412.68847656,450.21875)(412.80566406,450.45117188)(412.80566406,450.71679688)
\curveto(412.80566406,450.95507812)(412.70214844,451.14257812)(412.49511719,451.27929688)
\curveto(412.35058594,451.37304688)(411.99121094,451.4921875)(411.41699219,451.63671875)
\curveto(410.64355469,451.83203125)(410.10644531,452)(409.80566406,452.140625)
\curveto(409.50878906,452.28515625)(409.28222656,452.48242188)(409.12597656,452.73242188)
\curveto(408.97363281,452.98632812)(408.89746094,453.265625)(408.89746094,453.5703125)
\curveto(408.89746094,453.84765625)(408.95996094,454.10351562)(409.08496094,454.33789062)
\curveto(409.21386719,454.57617188)(409.38769531,454.7734375)(409.60644531,454.9296875)
\curveto(409.77050781,455.05078125)(409.99316406,455.15234375)(410.27441406,455.234375)
\curveto(410.55957031,455.3203125)(410.86425781,455.36328125)(411.18847656,455.36328125)
\curveto(411.67675781,455.36328125)(412.10449219,455.29296875)(412.47167969,455.15234375)
\curveto(412.84277344,455.01171875)(413.11621094,454.8203125)(413.29199219,454.578125)
\curveto(413.46777344,454.33984375)(413.58886719,454.01953125)(413.65527344,453.6171875)
\lineto(412.62402344,453.4765625)
\curveto(412.57714844,453.796875)(412.44042969,454.046875)(412.21386719,454.2265625)
\curveto(411.99121094,454.40625)(411.67480469,454.49609375)(411.26464844,454.49609375)
\curveto(410.78027344,454.49609375)(410.43457031,454.41601562)(410.22753906,454.25585938)
\curveto(410.02050781,454.09570312)(409.91699219,453.90820312)(409.91699219,453.69335938)
\curveto(409.91699219,453.55664062)(409.95996094,453.43359375)(410.04589844,453.32421875)
\curveto(410.13183594,453.2109375)(410.26660156,453.1171875)(410.45019531,453.04296875)
\curveto(410.55566406,453.00390625)(410.86621094,452.9140625)(411.38183594,452.7734375)
\curveto(412.12792969,452.57421875)(412.64746094,452.41015625)(412.94042969,452.28125)
\curveto(413.23730469,452.15625)(413.46972656,451.97265625)(413.63769531,451.73046875)
\curveto(413.80566406,451.48828125)(413.88964844,451.1875)(413.88964844,450.828125)
\curveto(413.88964844,450.4765625)(413.78613281,450.14453125)(413.57910156,449.83203125)
\curveto(413.37597656,449.5234375)(413.08105469,449.28320312)(412.69433594,449.11132812)
\curveto(412.30761719,448.94335938)(411.87011719,448.859375)(411.38183594,448.859375)
\curveto(410.57324219,448.859375)(409.95605469,449.02734375)(409.53027344,449.36328125)
\curveto(409.10839844,449.69921875)(408.83886719,450.19726562)(408.72167969,450.85742188)
\closepath
}
}
{
\newrgbcolor{curcolor}{0 0 0}
\pscustom[linestyle=none,fillstyle=solid,fillcolor=curcolor]
{
\newpath
\moveto(420.49316406,446.47460938)
\curveto(419.91113281,447.20898438)(419.41894531,448.06835938)(419.01660156,449.05273438)
\curveto(418.61425781,450.03710938)(418.41308594,451.05664062)(418.41308594,452.11132812)
\curveto(418.41308594,453.04101562)(418.56347656,453.93164062)(418.86425781,454.78320312)
\curveto(419.21582031,455.77148438)(419.75878906,456.75585938)(420.49316406,457.73632812)
\lineto(421.24902344,457.73632812)
\curveto(420.77636719,456.92382812)(420.46386719,456.34375)(420.31152344,455.99609375)
\curveto(420.07324219,455.45703125)(419.88574219,454.89453125)(419.74902344,454.30859375)
\curveto(419.58105469,453.578125)(419.49707031,452.84375)(419.49707031,452.10546875)
\curveto(419.49707031,450.2265625)(420.08105469,448.34960938)(421.24902344,446.47460938)
\closepath
}
}
{
\newrgbcolor{curcolor}{0 0 0}
\pscustom[linestyle=none,fillstyle=solid,fillcolor=curcolor]
{
\newpath
\moveto(422.60839844,449)
\lineto(422.60839844,457.58984375)
\lineto(425.56738281,457.58984375)
\curveto(426.23535156,457.58984375)(426.74511719,457.54882812)(427.09667969,457.46679688)
\curveto(427.58886719,457.35351562)(428.00878906,457.1484375)(428.35644531,456.8515625)
\curveto(428.80957031,456.46875)(429.14746094,455.97851562)(429.37011719,455.38085938)
\curveto(429.59667969,454.78710938)(429.70996094,454.10742188)(429.70996094,453.34179688)
\curveto(429.70996094,452.68945312)(429.63378906,452.11132812)(429.48144531,451.60742188)
\curveto(429.32910156,451.10351562)(429.13378906,450.68554688)(428.89550781,450.35351562)
\curveto(428.65722656,450.02539062)(428.39550781,449.765625)(428.11035156,449.57421875)
\curveto(427.82910156,449.38671875)(427.48730469,449.24414062)(427.08496094,449.14648438)
\curveto(426.68652344,449.04882812)(426.22753906,449)(425.70800781,449)
\closepath
\moveto(423.74511719,450.01367188)
\lineto(425.57910156,450.01367188)
\curveto(426.14550781,450.01367188)(426.58886719,450.06640625)(426.90917969,450.171875)
\curveto(427.23339844,450.27734375)(427.49121094,450.42578125)(427.68261719,450.6171875)
\curveto(427.95214844,450.88671875)(428.16113281,451.24804688)(428.30957031,451.70117188)
\curveto(428.46191406,452.15820312)(428.53808594,452.7109375)(428.53808594,453.359375)
\curveto(428.53808594,454.2578125)(428.38964844,454.94726562)(428.09277344,455.42773438)
\curveto(427.79980469,455.91210938)(427.44238281,456.23632812)(427.02050781,456.40039062)
\curveto(426.71582031,456.51757812)(426.22558594,456.57617188)(425.54980469,456.57617188)
\lineto(423.74511719,456.57617188)
\closepath
}
}
{
\newrgbcolor{curcolor}{0 0 0}
\pscustom[linestyle=none,fillstyle=solid,fillcolor=curcolor]
{
\newpath
\moveto(435.20019531,449.76757812)
\curveto(434.80957031,449.43554688)(434.43261719,449.20117188)(434.06933594,449.06445312)
\curveto(433.70996094,448.92773438)(433.32324219,448.859375)(432.90917969,448.859375)
\curveto(432.22558594,448.859375)(431.70019531,449.02539062)(431.33300781,449.35742188)
\curveto(430.96582031,449.69335938)(430.78222656,450.12109375)(430.78222656,450.640625)
\curveto(430.78222656,450.9453125)(430.85058594,451.22265625)(430.98730469,451.47265625)
\curveto(431.12792969,451.7265625)(431.30957031,451.9296875)(431.53222656,452.08203125)
\curveto(431.75878906,452.234375)(432.01269531,452.34960938)(432.29394531,452.42773438)
\curveto(432.50097656,452.48242188)(432.81347656,452.53515625)(433.23144531,452.5859375)
\curveto(434.08300781,452.6875)(434.70996094,452.80859375)(435.11230469,452.94921875)
\curveto(435.11621094,453.09375)(435.11816406,453.18554688)(435.11816406,453.22460938)
\curveto(435.11816406,453.65429688)(435.01855469,453.95703125)(434.81933594,454.1328125)
\curveto(434.54980469,454.37109375)(434.14941406,454.49023438)(433.61816406,454.49023438)
\curveto(433.12207031,454.49023438)(432.75488281,454.40234375)(432.51660156,454.2265625)
\curveto(432.28222656,454.0546875)(432.10839844,453.74804688)(431.99511719,453.30664062)
\lineto(430.96386719,453.44726562)
\curveto(431.05761719,453.88867188)(431.21191406,454.24414062)(431.42675781,454.51367188)
\curveto(431.64160156,454.78710938)(431.95214844,454.99609375)(432.35839844,455.140625)
\curveto(432.76464844,455.2890625)(433.23535156,455.36328125)(433.77050781,455.36328125)
\curveto(434.30175781,455.36328125)(434.73339844,455.30078125)(435.06542969,455.17578125)
\curveto(435.39746094,455.05078125)(435.64160156,454.89257812)(435.79785156,454.70117188)
\curveto(435.95410156,454.51367188)(436.06347656,454.27539062)(436.12597656,453.98632812)
\curveto(436.16113281,453.80664062)(436.17871094,453.48242188)(436.17871094,453.01367188)
\lineto(436.17871094,451.60742188)
\curveto(436.17871094,450.62695312)(436.20019531,450.00585938)(436.24316406,449.74414062)
\curveto(436.29003906,449.48632812)(436.37988281,449.23828125)(436.51269531,449)
\lineto(435.41113281,449)
\curveto(435.30175781,449.21875)(435.23144531,449.47460938)(435.20019531,449.76757812)
\closepath
\moveto(435.11230469,452.12304688)
\curveto(434.72949219,451.96679688)(434.15527344,451.83398438)(433.38964844,451.72460938)
\curveto(432.95605469,451.66210938)(432.64941406,451.59179688)(432.46972656,451.51367188)
\curveto(432.29003906,451.43554688)(432.15136719,451.3203125)(432.05371094,451.16796875)
\curveto(431.95605469,451.01953125)(431.90722656,450.85351562)(431.90722656,450.66992188)
\curveto(431.90722656,450.38867188)(432.01269531,450.15429688)(432.22363281,449.96679688)
\curveto(432.43847656,449.77929688)(432.75097656,449.68554688)(433.16113281,449.68554688)
\curveto(433.56738281,449.68554688)(433.92871094,449.7734375)(434.24511719,449.94921875)
\curveto(434.56152344,450.12890625)(434.79394531,450.37304688)(434.94238281,450.68164062)
\curveto(435.05566406,450.91992188)(435.11230469,451.27148438)(435.11230469,451.73632812)
\closepath
}
}
{
\newrgbcolor{curcolor}{0 0 0}
\pscustom[linestyle=none,fillstyle=solid,fillcolor=curcolor]
{
\newpath
\moveto(440.11621094,449.94335938)
\lineto(440.26855469,449.01171875)
\curveto(439.97167969,448.94921875)(439.70605469,448.91796875)(439.47167969,448.91796875)
\curveto(439.08886719,448.91796875)(438.79199219,448.97851562)(438.58105469,449.09960938)
\curveto(438.37011719,449.22070312)(438.22167969,449.37890625)(438.13574219,449.57421875)
\curveto(438.04980469,449.7734375)(438.00683594,450.18945312)(438.00683594,450.82226562)
\lineto(438.00683594,454.40234375)
\lineto(437.23339844,454.40234375)
\lineto(437.23339844,455.22265625)
\lineto(438.00683594,455.22265625)
\lineto(438.00683594,456.76367188)
\lineto(439.05566406,457.39648438)
\lineto(439.05566406,455.22265625)
\lineto(440.11621094,455.22265625)
\lineto(440.11621094,454.40234375)
\lineto(439.05566406,454.40234375)
\lineto(439.05566406,450.76367188)
\curveto(439.05566406,450.46289062)(439.07324219,450.26953125)(439.10839844,450.18359375)
\curveto(439.14746094,450.09765625)(439.20800781,450.02929688)(439.29003906,449.97851562)
\curveto(439.37597656,449.92773438)(439.49707031,449.90234375)(439.65332031,449.90234375)
\curveto(439.77050781,449.90234375)(439.92480469,449.91601562)(440.11621094,449.94335938)
\closepath
}
}
{
\newrgbcolor{curcolor}{0 0 0}
\pscustom[linestyle=none,fillstyle=solid,fillcolor=curcolor]
{
\newpath
\moveto(445.20800781,449.76757812)
\curveto(444.81738281,449.43554688)(444.44042969,449.20117188)(444.07714844,449.06445312)
\curveto(443.71777344,448.92773438)(443.33105469,448.859375)(442.91699219,448.859375)
\curveto(442.23339844,448.859375)(441.70800781,449.02539062)(441.34082031,449.35742188)
\curveto(440.97363281,449.69335938)(440.79003906,450.12109375)(440.79003906,450.640625)
\curveto(440.79003906,450.9453125)(440.85839844,451.22265625)(440.99511719,451.47265625)
\curveto(441.13574219,451.7265625)(441.31738281,451.9296875)(441.54003906,452.08203125)
\curveto(441.76660156,452.234375)(442.02050781,452.34960938)(442.30175781,452.42773438)
\curveto(442.50878906,452.48242188)(442.82128906,452.53515625)(443.23925781,452.5859375)
\curveto(444.09082031,452.6875)(444.71777344,452.80859375)(445.12011719,452.94921875)
\curveto(445.12402344,453.09375)(445.12597656,453.18554688)(445.12597656,453.22460938)
\curveto(445.12597656,453.65429688)(445.02636719,453.95703125)(444.82714844,454.1328125)
\curveto(444.55761719,454.37109375)(444.15722656,454.49023438)(443.62597656,454.49023438)
\curveto(443.12988281,454.49023438)(442.76269531,454.40234375)(442.52441406,454.2265625)
\curveto(442.29003906,454.0546875)(442.11621094,453.74804688)(442.00292969,453.30664062)
\lineto(440.97167969,453.44726562)
\curveto(441.06542969,453.88867188)(441.21972656,454.24414062)(441.43457031,454.51367188)
\curveto(441.64941406,454.78710938)(441.95996094,454.99609375)(442.36621094,455.140625)
\curveto(442.77246094,455.2890625)(443.24316406,455.36328125)(443.77832031,455.36328125)
\curveto(444.30957031,455.36328125)(444.74121094,455.30078125)(445.07324219,455.17578125)
\curveto(445.40527344,455.05078125)(445.64941406,454.89257812)(445.80566406,454.70117188)
\curveto(445.96191406,454.51367188)(446.07128906,454.27539062)(446.13378906,453.98632812)
\curveto(446.16894531,453.80664062)(446.18652344,453.48242188)(446.18652344,453.01367188)
\lineto(446.18652344,451.60742188)
\curveto(446.18652344,450.62695312)(446.20800781,450.00585938)(446.25097656,449.74414062)
\curveto(446.29785156,449.48632812)(446.38769531,449.23828125)(446.52050781,449)
\lineto(445.41894531,449)
\curveto(445.30957031,449.21875)(445.23925781,449.47460938)(445.20800781,449.76757812)
\closepath
\moveto(445.12011719,452.12304688)
\curveto(444.73730469,451.96679688)(444.16308594,451.83398438)(443.39746094,451.72460938)
\curveto(442.96386719,451.66210938)(442.65722656,451.59179688)(442.47753906,451.51367188)
\curveto(442.29785156,451.43554688)(442.15917969,451.3203125)(442.06152344,451.16796875)
\curveto(441.96386719,451.01953125)(441.91503906,450.85351562)(441.91503906,450.66992188)
\curveto(441.91503906,450.38867188)(442.02050781,450.15429688)(442.23144531,449.96679688)
\curveto(442.44628906,449.77929688)(442.75878906,449.68554688)(443.16894531,449.68554688)
\curveto(443.57519531,449.68554688)(443.93652344,449.7734375)(444.25292969,449.94921875)
\curveto(444.56933594,450.12890625)(444.80175781,450.37304688)(444.95019531,450.68164062)
\curveto(445.06347656,450.91992188)(445.12011719,451.27148438)(445.12011719,451.73632812)
\closepath
}
}
{
\newrgbcolor{curcolor}{0 0 0}
\pscustom[linestyle=none,fillstyle=solid,fillcolor=curcolor]
{
\newpath
\moveto(451.27832031,449)
\lineto(451.27832031,457.58984375)
\lineto(452.44433594,457.58984375)
\lineto(456.95605469,450.84570312)
\lineto(456.95605469,457.58984375)
\lineto(458.04589844,457.58984375)
\lineto(458.04589844,449)
\lineto(456.87988281,449)
\lineto(452.36816406,455.75)
\lineto(452.36816406,449)
\closepath
}
}
{
\newrgbcolor{curcolor}{0 0 0}
\pscustom[linestyle=none,fillstyle=solid,fillcolor=curcolor]
{
\newpath
\moveto(459.42871094,452.11132812)
\curveto(459.42871094,453.26367188)(459.74902344,454.1171875)(460.38964844,454.671875)
\curveto(460.92480469,455.1328125)(461.57714844,455.36328125)(462.34667969,455.36328125)
\curveto(463.20214844,455.36328125)(463.90136719,455.08203125)(464.44433594,454.51953125)
\curveto(464.98730469,453.9609375)(465.25878906,453.1875)(465.25878906,452.19921875)
\curveto(465.25878906,451.3984375)(465.13769531,450.76757812)(464.89550781,450.30664062)
\curveto(464.65722656,449.84960938)(464.30761719,449.49414062)(463.84667969,449.24023438)
\curveto(463.38964844,448.98632812)(462.88964844,448.859375)(462.34667969,448.859375)
\curveto(461.47558594,448.859375)(460.77050781,449.13867188)(460.23144531,449.69726562)
\curveto(459.69628906,450.25585938)(459.42871094,451.06054688)(459.42871094,452.11132812)
\closepath
\moveto(460.51269531,452.11132812)
\curveto(460.51269531,451.31445312)(460.68652344,450.71679688)(461.03417969,450.31835938)
\curveto(461.38183594,449.92382812)(461.81933594,449.7265625)(462.34667969,449.7265625)
\curveto(462.87011719,449.7265625)(463.30566406,449.92578125)(463.65332031,450.32421875)
\curveto(464.00097656,450.72265625)(464.17480469,451.33007812)(464.17480469,452.14648438)
\curveto(464.17480469,452.91601562)(463.99902344,453.49804688)(463.64746094,453.89257812)
\curveto(463.29980469,454.29101562)(462.86621094,454.49023438)(462.34667969,454.49023438)
\curveto(461.81933594,454.49023438)(461.38183594,454.29296875)(461.03417969,453.8984375)
\curveto(460.68652344,453.50390625)(460.51269531,452.90820312)(460.51269531,452.11132812)
\closepath
}
}
{
\newrgbcolor{curcolor}{0 0 0}
\pscustom[linestyle=none,fillstyle=solid,fillcolor=curcolor]
{
\newpath
\moveto(470.53222656,449)
\lineto(470.53222656,449.78515625)
\curveto(470.13769531,449.16796875)(469.55761719,448.859375)(468.79199219,448.859375)
\curveto(468.29589844,448.859375)(467.83886719,448.99609375)(467.42089844,449.26953125)
\curveto(467.00683594,449.54296875)(466.68457031,449.92382812)(466.45410156,450.41210938)
\curveto(466.22753906,450.90429688)(466.11425781,451.46875)(466.11425781,452.10546875)
\curveto(466.11425781,452.7265625)(466.21777344,453.2890625)(466.42480469,453.79296875)
\curveto(466.63183594,454.30078125)(466.94238281,454.68945312)(467.35644531,454.95898438)
\curveto(467.77050781,455.22851562)(468.23339844,455.36328125)(468.74511719,455.36328125)
\curveto(469.12011719,455.36328125)(469.45410156,455.28320312)(469.74707031,455.12304688)
\curveto(470.04003906,454.96679688)(470.27832031,454.76171875)(470.46191406,454.5078125)
\lineto(470.46191406,457.58984375)
\lineto(471.51074219,457.58984375)
\lineto(471.51074219,449)
\closepath
\moveto(467.19824219,452.10546875)
\curveto(467.19824219,451.30859375)(467.36621094,450.71289062)(467.70214844,450.31835938)
\curveto(468.03808594,449.92382812)(468.43457031,449.7265625)(468.89160156,449.7265625)
\curveto(469.35253906,449.7265625)(469.74316406,449.9140625)(470.06347656,450.2890625)
\curveto(470.38769531,450.66796875)(470.54980469,451.24414062)(470.54980469,452.01757812)
\curveto(470.54980469,452.86914062)(470.38574219,453.49414062)(470.05761719,453.89257812)
\curveto(469.72949219,454.29101562)(469.32519531,454.49023438)(468.84472656,454.49023438)
\curveto(468.37597656,454.49023438)(467.98339844,454.29882812)(467.66699219,453.91601562)
\curveto(467.35449219,453.53320312)(467.19824219,452.9296875)(467.19824219,452.10546875)
\closepath
}
}
{
\newrgbcolor{curcolor}{0 0 0}
\pscustom[linestyle=none,fillstyle=solid,fillcolor=curcolor]
{
\newpath
\moveto(477.42871094,451.00390625)
\lineto(478.51855469,450.86914062)
\curveto(478.34667969,450.23242188)(478.02832031,449.73828125)(477.56347656,449.38671875)
\curveto(477.09863281,449.03515625)(476.50488281,448.859375)(475.78222656,448.859375)
\curveto(474.87207031,448.859375)(474.14941406,449.13867188)(473.61425781,449.69726562)
\curveto(473.08300781,450.25976562)(472.81738281,451.046875)(472.81738281,452.05859375)
\curveto(472.81738281,453.10546875)(473.08691406,453.91796875)(473.62597656,454.49609375)
\curveto(474.16503906,455.07421875)(474.86425781,455.36328125)(475.72363281,455.36328125)
\curveto(476.55566406,455.36328125)(477.23535156,455.08007812)(477.76269531,454.51367188)
\curveto(478.29003906,453.94726562)(478.55371094,453.15039062)(478.55371094,452.12304688)
\curveto(478.55371094,452.06054688)(478.55175781,451.96679688)(478.54785156,451.84179688)
\lineto(473.90722656,451.84179688)
\curveto(473.94628906,451.15820312)(474.13964844,450.63476562)(474.48730469,450.27148438)
\curveto(474.83496094,449.90820312)(475.26855469,449.7265625)(475.78808594,449.7265625)
\curveto(476.17480469,449.7265625)(476.50488281,449.828125)(476.77832031,450.03125)
\curveto(477.05175781,450.234375)(477.26855469,450.55859375)(477.42871094,451.00390625)
\closepath
\moveto(473.96582031,452.70898438)
\lineto(477.44042969,452.70898438)
\curveto(477.39355469,453.23242188)(477.26074219,453.625)(477.04199219,453.88671875)
\curveto(476.70605469,454.29296875)(476.27050781,454.49609375)(475.73535156,454.49609375)
\curveto(475.25097656,454.49609375)(474.84277344,454.33398438)(474.51074219,454.00976562)
\curveto(474.18261719,453.68554688)(474.00097656,453.25195312)(473.96582031,452.70898438)
\closepath
}
}
{
\newrgbcolor{curcolor}{0 0 0}
\pscustom[linestyle=none,fillstyle=solid,fillcolor=curcolor]
{
\newpath
\moveto(482.88378906,453.23632812)
\curveto(482.88378906,454.25195312)(482.98730469,455.06835938)(483.19433594,455.68554688)
\curveto(483.40527344,456.30664062)(483.71582031,456.78515625)(484.12597656,457.12109375)
\curveto(484.54003906,457.45703125)(485.05957031,457.625)(485.68457031,457.625)
\curveto(486.14550781,457.625)(486.54980469,457.53125)(486.89746094,457.34375)
\curveto(487.24511719,457.16015625)(487.53222656,456.89257812)(487.75878906,456.54101562)
\curveto(487.98535156,456.19335938)(488.16308594,455.76757812)(488.29199219,455.26367188)
\curveto(488.42089844,454.76367188)(488.48535156,454.08789062)(488.48535156,453.23632812)
\curveto(488.48535156,452.22851562)(488.38183594,451.4140625)(488.17480469,450.79296875)
\curveto(487.96777344,450.17578125)(487.65722656,449.69726562)(487.24316406,449.35742188)
\curveto(486.83300781,449.02148438)(486.31347656,448.85351562)(485.68457031,448.85351562)
\curveto(484.85644531,448.85351562)(484.20605469,449.15039062)(483.73339844,449.74414062)
\curveto(483.16699219,450.45898438)(482.88378906,451.62304688)(482.88378906,453.23632812)
\closepath
\moveto(483.96777344,453.23632812)
\curveto(483.96777344,451.82617188)(484.13183594,450.88671875)(484.45996094,450.41796875)
\curveto(484.79199219,449.953125)(485.20019531,449.72070312)(485.68457031,449.72070312)
\curveto(486.16894531,449.72070312)(486.57519531,449.95507812)(486.90332031,450.42382812)
\curveto(487.23535156,450.89257812)(487.40136719,451.83007812)(487.40136719,453.23632812)
\curveto(487.40136719,454.65039062)(487.23535156,455.58984375)(486.90332031,456.0546875)
\curveto(486.57519531,456.51953125)(486.16503906,456.75195312)(485.67285156,456.75195312)
\curveto(485.18847656,456.75195312)(484.80175781,456.546875)(484.51269531,456.13671875)
\curveto(484.14941406,455.61328125)(483.96777344,454.64648438)(483.96777344,453.23632812)
\closepath
}
}
{
\newrgbcolor{curcolor}{0 0 0}
\pscustom[linestyle=none,fillstyle=solid,fillcolor=curcolor]
{
\newpath
\moveto(490.54199219,446.47460938)
\lineto(489.78613281,446.47460938)
\curveto(490.95410156,448.34960938)(491.53808594,450.2265625)(491.53808594,452.10546875)
\curveto(491.53808594,452.83984375)(491.45410156,453.56835938)(491.28613281,454.29101562)
\curveto(491.15332031,454.87695312)(490.96777344,455.43945312)(490.72949219,455.97851562)
\curveto(490.57714844,456.33007812)(490.26269531,456.91601562)(489.78613281,457.73632812)
\lineto(490.54199219,457.73632812)
\curveto(491.27636719,456.75585938)(491.81933594,455.77148438)(492.17089844,454.78320312)
\curveto(492.47167969,453.93164062)(492.62207031,453.04101562)(492.62207031,452.11132812)
\curveto(492.62207031,451.05664062)(492.41894531,450.03710938)(492.01269531,449.05273438)
\curveto(491.61035156,448.06835938)(491.12011719,447.20898438)(490.54199219,446.47460938)
\closepath
}
}
{
\newrgbcolor{curcolor}{0 0 0}
\pscustom[linestyle=none,fillstyle=solid,fillcolor=curcolor]
{
\newpath
\moveto(166.41992188,404)
\lineto(166.41992188,412.58984375)
\lineto(167.5859375,412.58984375)
\lineto(172.09765625,405.84570312)
\lineto(172.09765625,412.58984375)
\lineto(173.1875,412.58984375)
\lineto(173.1875,404)
\lineto(172.02148438,404)
\lineto(167.50976562,410.75)
\lineto(167.50976562,404)
\closepath
}
}
{
\newrgbcolor{curcolor}{0 0 0}
\pscustom[linestyle=none,fillstyle=solid,fillcolor=curcolor]
{
\newpath
\moveto(179.22265625,406.00390625)
\lineto(180.3125,405.86914062)
\curveto(180.140625,405.23242188)(179.82226562,404.73828125)(179.35742188,404.38671875)
\curveto(178.89257812,404.03515625)(178.29882812,403.859375)(177.57617188,403.859375)
\curveto(176.66601562,403.859375)(175.94335938,404.13867188)(175.40820312,404.69726562)
\curveto(174.87695312,405.25976562)(174.61132812,406.046875)(174.61132812,407.05859375)
\curveto(174.61132812,408.10546875)(174.88085938,408.91796875)(175.41992188,409.49609375)
\curveto(175.95898438,410.07421875)(176.65820312,410.36328125)(177.51757812,410.36328125)
\curveto(178.34960938,410.36328125)(179.02929688,410.08007812)(179.55664062,409.51367188)
\curveto(180.08398438,408.94726562)(180.34765625,408.15039062)(180.34765625,407.12304688)
\curveto(180.34765625,407.06054688)(180.34570312,406.96679688)(180.34179688,406.84179688)
\lineto(175.70117188,406.84179688)
\curveto(175.74023438,406.15820312)(175.93359375,405.63476562)(176.28125,405.27148438)
\curveto(176.62890625,404.90820312)(177.0625,404.7265625)(177.58203125,404.7265625)
\curveto(177.96875,404.7265625)(178.29882812,404.828125)(178.57226562,405.03125)
\curveto(178.84570312,405.234375)(179.0625,405.55859375)(179.22265625,406.00390625)
\closepath
\moveto(175.75976562,407.70898438)
\lineto(179.234375,407.70898438)
\curveto(179.1875,408.23242188)(179.0546875,408.625)(178.8359375,408.88671875)
\curveto(178.5,409.29296875)(178.06445312,409.49609375)(177.52929688,409.49609375)
\curveto(177.04492188,409.49609375)(176.63671875,409.33398438)(176.3046875,409.00976562)
\curveto(175.9765625,408.68554688)(175.79492188,408.25195312)(175.75976562,407.70898438)
\closepath
}
}
{
\newrgbcolor{curcolor}{0 0 0}
\pscustom[linestyle=none,fillstyle=solid,fillcolor=curcolor]
{
\newpath
\moveto(182.78515625,404)
\lineto(180.88085938,410.22265625)
\lineto(181.97070312,410.22265625)
\lineto(182.9609375,406.63085938)
\lineto(183.33007812,405.29492188)
\curveto(183.34570312,405.36132812)(183.453125,405.7890625)(183.65234375,406.578125)
\lineto(184.64257812,410.22265625)
\lineto(185.7265625,410.22265625)
\lineto(186.65820312,406.61328125)
\lineto(186.96875,405.42382812)
\lineto(187.32617188,406.625)
\lineto(188.39257812,410.22265625)
\lineto(189.41796875,410.22265625)
\lineto(187.47265625,404)
\lineto(186.37695312,404)
\lineto(185.38671875,407.7265625)
\lineto(185.14648438,408.78710938)
\lineto(183.88671875,404)
\closepath
}
}
{
\newrgbcolor{curcolor}{0 0 0}
\pscustom[linestyle=none,fillstyle=solid,fillcolor=curcolor]
{
\newpath
\moveto(193.0859375,404)
\lineto(193.0859375,405.0546875)
\lineto(197.48632812,410.55664062)
\curveto(197.79882812,410.94726562)(198.09570312,411.28710938)(198.37695312,411.57617188)
\lineto(193.58398438,411.57617188)
\lineto(193.58398438,412.58984375)
\lineto(199.73632812,412.58984375)
\lineto(199.73632812,411.57617188)
\lineto(194.9140625,405.6171875)
\lineto(194.39257812,405.01367188)
\lineto(199.87695312,405.01367188)
\lineto(199.87695312,404)
\closepath
}
}
{
\newrgbcolor{curcolor}{0 0 0}
\pscustom[linestyle=none,fillstyle=solid,fillcolor=curcolor]
{
\newpath
\moveto(201.16015625,404)
\lineto(201.16015625,412.58984375)
\lineto(206.95507812,412.58984375)
\lineto(206.95507812,411.57617188)
\lineto(202.296875,411.57617188)
\lineto(202.296875,408.91601562)
\lineto(206.328125,408.91601562)
\lineto(206.328125,407.90234375)
\lineto(202.296875,407.90234375)
\lineto(202.296875,404)
\closepath
}
}
{
\newrgbcolor{curcolor}{0 0 0}
\pscustom[linestyle=none,fillstyle=solid,fillcolor=curcolor]
{
\newpath
\moveto(208.04492188,406.75976562)
\lineto(209.1171875,406.85351562)
\curveto(209.16796875,406.42382812)(209.28515625,406.0703125)(209.46875,405.79296875)
\curveto(209.65625,405.51953125)(209.9453125,405.296875)(210.3359375,405.125)
\curveto(210.7265625,404.95703125)(211.16601562,404.87304688)(211.65429688,404.87304688)
\curveto(212.08789062,404.87304688)(212.47070312,404.9375)(212.80273438,405.06640625)
\curveto(213.13476562,405.1953125)(213.38085938,405.37109375)(213.54101562,405.59375)
\curveto(213.70507812,405.8203125)(213.78710938,406.06640625)(213.78710938,406.33203125)
\curveto(213.78710938,406.6015625)(213.70898438,406.8359375)(213.55273438,407.03515625)
\curveto(213.39648438,407.23828125)(213.13867188,407.40820312)(212.77929688,407.54492188)
\curveto(212.54882812,407.63476562)(212.0390625,407.7734375)(211.25,407.9609375)
\curveto(210.4609375,408.15234375)(209.90820312,408.33203125)(209.59179688,408.5)
\curveto(209.18164062,408.71484375)(208.875,408.98046875)(208.671875,409.296875)
\curveto(208.47265625,409.6171875)(208.37304688,409.97460938)(208.37304688,410.36914062)
\curveto(208.37304688,410.80273438)(208.49609375,411.20703125)(208.7421875,411.58203125)
\curveto(208.98828125,411.9609375)(209.34765625,412.24804688)(209.8203125,412.44335938)
\curveto(210.29296875,412.63867188)(210.81835938,412.73632812)(211.39648438,412.73632812)
\curveto(212.03320312,412.73632812)(212.59375,412.6328125)(213.078125,412.42578125)
\curveto(213.56640625,412.22265625)(213.94140625,411.921875)(214.203125,411.5234375)
\curveto(214.46484375,411.125)(214.60546875,410.67382812)(214.625,410.16992188)
\lineto(213.53515625,410.08789062)
\curveto(213.4765625,410.63085938)(213.27734375,411.04101562)(212.9375,411.31835938)
\curveto(212.6015625,411.59570312)(212.10351562,411.734375)(211.44335938,411.734375)
\curveto(210.75585938,411.734375)(210.25390625,411.60742188)(209.9375,411.35351562)
\curveto(209.625,411.10351562)(209.46875,410.80078125)(209.46875,410.4453125)
\curveto(209.46875,410.13671875)(209.58007812,409.8828125)(209.80273438,409.68359375)
\curveto(210.02148438,409.484375)(210.59179688,409.27929688)(211.51367188,409.06835938)
\curveto(212.43945312,408.86132812)(213.07421875,408.6796875)(213.41796875,408.5234375)
\curveto(213.91796875,408.29296875)(214.28710938,408)(214.52539062,407.64453125)
\curveto(214.76367188,407.29296875)(214.8828125,406.88671875)(214.8828125,406.42578125)
\curveto(214.8828125,405.96875)(214.75195312,405.53710938)(214.49023438,405.13085938)
\curveto(214.22851562,404.72851562)(213.8515625,404.4140625)(213.359375,404.1875)
\curveto(212.87109375,403.96484375)(212.3203125,403.85351562)(211.70703125,403.85351562)
\curveto(210.9296875,403.85351562)(210.27734375,403.96679688)(209.75,404.19335938)
\curveto(209.2265625,404.41992188)(208.81445312,404.75976562)(208.51367188,405.21289062)
\curveto(208.21679688,405.66992188)(208.06054688,406.18554688)(208.04492188,406.75976562)
\closepath
}
}
{
\newrgbcolor{curcolor}{0 0 0}
\pscustom[linestyle=none,fillstyle=solid,fillcolor=curcolor]
{
\newpath
\moveto(219.76953125,404)
\lineto(219.76953125,412.58984375)
\lineto(222.72851562,412.58984375)
\curveto(223.39648438,412.58984375)(223.90625,412.54882812)(224.2578125,412.46679688)
\curveto(224.75,412.35351562)(225.16992188,412.1484375)(225.51757812,411.8515625)
\curveto(225.97070312,411.46875)(226.30859375,410.97851562)(226.53125,410.38085938)
\curveto(226.7578125,409.78710938)(226.87109375,409.10742188)(226.87109375,408.34179688)
\curveto(226.87109375,407.68945312)(226.79492188,407.11132812)(226.64257812,406.60742188)
\curveto(226.49023438,406.10351562)(226.29492188,405.68554688)(226.05664062,405.35351562)
\curveto(225.81835938,405.02539062)(225.55664062,404.765625)(225.27148438,404.57421875)
\curveto(224.99023438,404.38671875)(224.6484375,404.24414062)(224.24609375,404.14648438)
\curveto(223.84765625,404.04882812)(223.38867188,404)(222.86914062,404)
\closepath
\moveto(220.90625,405.01367188)
\lineto(222.74023438,405.01367188)
\curveto(223.30664062,405.01367188)(223.75,405.06640625)(224.0703125,405.171875)
\curveto(224.39453125,405.27734375)(224.65234375,405.42578125)(224.84375,405.6171875)
\curveto(225.11328125,405.88671875)(225.32226562,406.24804688)(225.47070312,406.70117188)
\curveto(225.62304688,407.15820312)(225.69921875,407.7109375)(225.69921875,408.359375)
\curveto(225.69921875,409.2578125)(225.55078125,409.94726562)(225.25390625,410.42773438)
\curveto(224.9609375,410.91210938)(224.60351562,411.23632812)(224.18164062,411.40039062)
\curveto(223.87695312,411.51757812)(223.38671875,411.57617188)(222.7109375,411.57617188)
\lineto(220.90625,411.57617188)
\closepath
}
}
{
\newrgbcolor{curcolor}{0 0 0}
\pscustom[linestyle=none,fillstyle=solid,fillcolor=curcolor]
{
\newpath
\moveto(228.30664062,411.37695312)
\lineto(228.30664062,412.58984375)
\lineto(229.36132812,412.58984375)
\lineto(229.36132812,411.37695312)
\closepath
\moveto(228.30664062,404)
\lineto(228.30664062,410.22265625)
\lineto(229.36132812,410.22265625)
\lineto(229.36132812,404)
\closepath
}
}
{
\newrgbcolor{curcolor}{0 0 0}
\pscustom[linestyle=none,fillstyle=solid,fillcolor=curcolor]
{
\newpath
\moveto(231.21875,404)
\lineto(231.21875,409.40234375)
\lineto(230.28710938,409.40234375)
\lineto(230.28710938,410.22265625)
\lineto(231.21875,410.22265625)
\lineto(231.21875,410.88476562)
\curveto(231.21875,411.30273438)(231.25585938,411.61328125)(231.33007812,411.81640625)
\curveto(231.43164062,412.08984375)(231.609375,412.31054688)(231.86328125,412.47851562)
\curveto(232.12109375,412.65039062)(232.48046875,412.73632812)(232.94140625,412.73632812)
\curveto(233.23828125,412.73632812)(233.56640625,412.70117188)(233.92578125,412.63085938)
\lineto(233.76757812,411.7109375)
\curveto(233.54882812,411.75)(233.34179688,411.76953125)(233.14648438,411.76953125)
\curveto(232.82617188,411.76953125)(232.59960938,411.70117188)(232.46679688,411.56445312)
\curveto(232.33398438,411.42773438)(232.26757812,411.171875)(232.26757812,410.796875)
\lineto(232.26757812,410.22265625)
\lineto(233.48046875,410.22265625)
\lineto(233.48046875,409.40234375)
\lineto(232.26757812,409.40234375)
\lineto(232.26757812,404)
\closepath
}
}
{
\newrgbcolor{curcolor}{0 0 0}
\pscustom[linestyle=none,fillstyle=solid,fillcolor=curcolor]
{
\newpath
\moveto(234.3359375,404)
\lineto(234.3359375,409.40234375)
\lineto(233.40429688,409.40234375)
\lineto(233.40429688,410.22265625)
\lineto(234.3359375,410.22265625)
\lineto(234.3359375,410.88476562)
\curveto(234.3359375,411.30273438)(234.37304688,411.61328125)(234.44726562,411.81640625)
\curveto(234.54882812,412.08984375)(234.7265625,412.31054688)(234.98046875,412.47851562)
\curveto(235.23828125,412.65039062)(235.59765625,412.73632812)(236.05859375,412.73632812)
\curveto(236.35546875,412.73632812)(236.68359375,412.70117188)(237.04296875,412.63085938)
\lineto(236.88476562,411.7109375)
\curveto(236.66601562,411.75)(236.45898438,411.76953125)(236.26367188,411.76953125)
\curveto(235.94335938,411.76953125)(235.71679688,411.70117188)(235.58398438,411.56445312)
\curveto(235.45117188,411.42773438)(235.38476562,411.171875)(235.38476562,410.796875)
\lineto(235.38476562,410.22265625)
\lineto(236.59765625,410.22265625)
\lineto(236.59765625,409.40234375)
\lineto(235.38476562,409.40234375)
\lineto(235.38476562,404)
\closepath
}
}
{
\newrgbcolor{curcolor}{0 0 0}
\pscustom[linestyle=none,fillstyle=solid,fillcolor=curcolor]
{
\newpath
\moveto(241.67773438,406.00390625)
\lineto(242.76757812,405.86914062)
\curveto(242.59570312,405.23242188)(242.27734375,404.73828125)(241.8125,404.38671875)
\curveto(241.34765625,404.03515625)(240.75390625,403.859375)(240.03125,403.859375)
\curveto(239.12109375,403.859375)(238.3984375,404.13867188)(237.86328125,404.69726562)
\curveto(237.33203125,405.25976562)(237.06640625,406.046875)(237.06640625,407.05859375)
\curveto(237.06640625,408.10546875)(237.3359375,408.91796875)(237.875,409.49609375)
\curveto(238.4140625,410.07421875)(239.11328125,410.36328125)(239.97265625,410.36328125)
\curveto(240.8046875,410.36328125)(241.484375,410.08007812)(242.01171875,409.51367188)
\curveto(242.5390625,408.94726562)(242.80273438,408.15039062)(242.80273438,407.12304688)
\curveto(242.80273438,407.06054688)(242.80078125,406.96679688)(242.796875,406.84179688)
\lineto(238.15625,406.84179688)
\curveto(238.1953125,406.15820312)(238.38867188,405.63476562)(238.73632812,405.27148438)
\curveto(239.08398438,404.90820312)(239.51757812,404.7265625)(240.03710938,404.7265625)
\curveto(240.42382812,404.7265625)(240.75390625,404.828125)(241.02734375,405.03125)
\curveto(241.30078125,405.234375)(241.51757812,405.55859375)(241.67773438,406.00390625)
\closepath
\moveto(238.21484375,407.70898438)
\lineto(241.68945312,407.70898438)
\curveto(241.64257812,408.23242188)(241.50976562,408.625)(241.29101562,408.88671875)
\curveto(240.95507812,409.29296875)(240.51953125,409.49609375)(239.984375,409.49609375)
\curveto(239.5,409.49609375)(239.09179688,409.33398438)(238.75976562,409.00976562)
\curveto(238.43164062,408.68554688)(238.25,408.25195312)(238.21484375,407.70898438)
\closepath
}
}
{
\newrgbcolor{curcolor}{0 0 0}
\pscustom[linestyle=none,fillstyle=solid,fillcolor=curcolor]
{
\newpath
\moveto(244.08007812,404)
\lineto(244.08007812,410.22265625)
\lineto(245.02929688,410.22265625)
\lineto(245.02929688,409.27929688)
\curveto(245.27148438,409.72070312)(245.49414062,410.01171875)(245.69726562,410.15234375)
\curveto(245.90429688,410.29296875)(246.13085938,410.36328125)(246.37695312,410.36328125)
\curveto(246.73242188,410.36328125)(247.09375,410.25)(247.4609375,410.0234375)
\lineto(247.09765625,409.04492188)
\curveto(246.83984375,409.19726562)(246.58203125,409.2734375)(246.32421875,409.2734375)
\curveto(246.09375,409.2734375)(245.88671875,409.203125)(245.703125,409.0625)
\curveto(245.51953125,408.92578125)(245.38867188,408.734375)(245.31054688,408.48828125)
\curveto(245.19335938,408.11328125)(245.13476562,407.703125)(245.13476562,407.2578125)
\lineto(245.13476562,404)
\closepath
}
}
{
\newrgbcolor{curcolor}{0 0 0}
\pscustom[linestyle=none,fillstyle=solid,fillcolor=curcolor]
{
\newpath
\moveto(252.34765625,406.00390625)
\lineto(253.4375,405.86914062)
\curveto(253.265625,405.23242188)(252.94726562,404.73828125)(252.48242188,404.38671875)
\curveto(252.01757812,404.03515625)(251.42382812,403.859375)(250.70117188,403.859375)
\curveto(249.79101562,403.859375)(249.06835938,404.13867188)(248.53320312,404.69726562)
\curveto(248.00195312,405.25976562)(247.73632812,406.046875)(247.73632812,407.05859375)
\curveto(247.73632812,408.10546875)(248.00585938,408.91796875)(248.54492188,409.49609375)
\curveto(249.08398438,410.07421875)(249.78320312,410.36328125)(250.64257812,410.36328125)
\curveto(251.47460938,410.36328125)(252.15429688,410.08007812)(252.68164062,409.51367188)
\curveto(253.20898438,408.94726562)(253.47265625,408.15039062)(253.47265625,407.12304688)
\curveto(253.47265625,407.06054688)(253.47070312,406.96679688)(253.46679688,406.84179688)
\lineto(248.82617188,406.84179688)
\curveto(248.86523438,406.15820312)(249.05859375,405.63476562)(249.40625,405.27148438)
\curveto(249.75390625,404.90820312)(250.1875,404.7265625)(250.70703125,404.7265625)
\curveto(251.09375,404.7265625)(251.42382812,404.828125)(251.69726562,405.03125)
\curveto(251.97070312,405.234375)(252.1875,405.55859375)(252.34765625,406.00390625)
\closepath
\moveto(248.88476562,407.70898438)
\lineto(252.359375,407.70898438)
\curveto(252.3125,408.23242188)(252.1796875,408.625)(251.9609375,408.88671875)
\curveto(251.625,409.29296875)(251.18945312,409.49609375)(250.65429688,409.49609375)
\curveto(250.16992188,409.49609375)(249.76171875,409.33398438)(249.4296875,409.00976562)
\curveto(249.1015625,408.68554688)(248.91992188,408.25195312)(248.88476562,407.70898438)
\closepath
}
}
{
\newrgbcolor{curcolor}{0 0 0}
\pscustom[linestyle=none,fillstyle=solid,fillcolor=curcolor]
{
\newpath
\moveto(254.76171875,404)
\lineto(254.76171875,410.22265625)
\lineto(255.7109375,410.22265625)
\lineto(255.7109375,409.33789062)
\curveto(256.16796875,410.02148438)(256.828125,410.36328125)(257.69140625,410.36328125)
\curveto(258.06640625,410.36328125)(258.41015625,410.29492188)(258.72265625,410.15820312)
\curveto(259.0390625,410.02539062)(259.27539062,409.84960938)(259.43164062,409.63085938)
\curveto(259.58789062,409.41210938)(259.69726562,409.15234375)(259.75976562,408.8515625)
\curveto(259.79882812,408.65625)(259.81835938,408.31445312)(259.81835938,407.82617188)
\lineto(259.81835938,404)
\lineto(258.76367188,404)
\lineto(258.76367188,407.78515625)
\curveto(258.76367188,408.21484375)(258.72265625,408.53515625)(258.640625,408.74609375)
\curveto(258.55859375,408.9609375)(258.41210938,409.13085938)(258.20117188,409.25585938)
\curveto(257.99414062,409.38476562)(257.75,409.44921875)(257.46875,409.44921875)
\curveto(257.01953125,409.44921875)(256.63085938,409.30664062)(256.30273438,409.02148438)
\curveto(255.97851562,408.73632812)(255.81640625,408.1953125)(255.81640625,407.3984375)
\lineto(255.81640625,404)
\closepath
}
}
{
\newrgbcolor{curcolor}{0 0 0}
\pscustom[linestyle=none,fillstyle=solid,fillcolor=curcolor]
{
\newpath
\moveto(263.73828125,404.94335938)
\lineto(263.890625,404.01171875)
\curveto(263.59375,403.94921875)(263.328125,403.91796875)(263.09375,403.91796875)
\curveto(262.7109375,403.91796875)(262.4140625,403.97851562)(262.203125,404.09960938)
\curveto(261.9921875,404.22070312)(261.84375,404.37890625)(261.7578125,404.57421875)
\curveto(261.671875,404.7734375)(261.62890625,405.18945312)(261.62890625,405.82226562)
\lineto(261.62890625,409.40234375)
\lineto(260.85546875,409.40234375)
\lineto(260.85546875,410.22265625)
\lineto(261.62890625,410.22265625)
\lineto(261.62890625,411.76367188)
\lineto(262.67773438,412.39648438)
\lineto(262.67773438,410.22265625)
\lineto(263.73828125,410.22265625)
\lineto(263.73828125,409.40234375)
\lineto(262.67773438,409.40234375)
\lineto(262.67773438,405.76367188)
\curveto(262.67773438,405.46289062)(262.6953125,405.26953125)(262.73046875,405.18359375)
\curveto(262.76953125,405.09765625)(262.83007812,405.02929688)(262.91210938,404.97851562)
\curveto(262.99804688,404.92773438)(263.11914062,404.90234375)(263.27539062,404.90234375)
\curveto(263.39257812,404.90234375)(263.546875,404.91601562)(263.73828125,404.94335938)
\closepath
}
}
{
\newrgbcolor{curcolor}{0 0 0}
\pscustom[linestyle=none,fillstyle=solid,fillcolor=curcolor]
{
\newpath
\moveto(268.2265625,404)
\lineto(268.2265625,412.58984375)
\lineto(269.39257812,412.58984375)
\lineto(273.90429688,405.84570312)
\lineto(273.90429688,412.58984375)
\lineto(274.99414062,412.58984375)
\lineto(274.99414062,404)
\lineto(273.828125,404)
\lineto(269.31640625,410.75)
\lineto(269.31640625,404)
\closepath
}
}
{
\newrgbcolor{curcolor}{0 0 0}
\pscustom[linestyle=none,fillstyle=solid,fillcolor=curcolor]
{
\newpath
\moveto(276.37695312,407.11132812)
\curveto(276.37695312,408.26367188)(276.69726562,409.1171875)(277.33789062,409.671875)
\curveto(277.87304688,410.1328125)(278.52539062,410.36328125)(279.29492188,410.36328125)
\curveto(280.15039062,410.36328125)(280.84960938,410.08203125)(281.39257812,409.51953125)
\curveto(281.93554688,408.9609375)(282.20703125,408.1875)(282.20703125,407.19921875)
\curveto(282.20703125,406.3984375)(282.0859375,405.76757812)(281.84375,405.30664062)
\curveto(281.60546875,404.84960938)(281.25585938,404.49414062)(280.79492188,404.24023438)
\curveto(280.33789062,403.98632812)(279.83789062,403.859375)(279.29492188,403.859375)
\curveto(278.42382812,403.859375)(277.71875,404.13867188)(277.1796875,404.69726562)
\curveto(276.64453125,405.25585938)(276.37695312,406.06054688)(276.37695312,407.11132812)
\closepath
\moveto(277.4609375,407.11132812)
\curveto(277.4609375,406.31445312)(277.63476562,405.71679688)(277.98242188,405.31835938)
\curveto(278.33007812,404.92382812)(278.76757812,404.7265625)(279.29492188,404.7265625)
\curveto(279.81835938,404.7265625)(280.25390625,404.92578125)(280.6015625,405.32421875)
\curveto(280.94921875,405.72265625)(281.12304688,406.33007812)(281.12304688,407.14648438)
\curveto(281.12304688,407.91601562)(280.94726562,408.49804688)(280.59570312,408.89257812)
\curveto(280.24804688,409.29101562)(279.81445312,409.49023438)(279.29492188,409.49023438)
\curveto(278.76757812,409.49023438)(278.33007812,409.29296875)(277.98242188,408.8984375)
\curveto(277.63476562,408.50390625)(277.4609375,407.90820312)(277.4609375,407.11132812)
\closepath
}
}
{
\newrgbcolor{curcolor}{0 0 0}
\pscustom[linestyle=none,fillstyle=solid,fillcolor=curcolor]
{
\newpath
\moveto(287.48046875,404)
\lineto(287.48046875,404.78515625)
\curveto(287.0859375,404.16796875)(286.50585938,403.859375)(285.74023438,403.859375)
\curveto(285.24414062,403.859375)(284.78710938,403.99609375)(284.36914062,404.26953125)
\curveto(283.95507812,404.54296875)(283.6328125,404.92382812)(283.40234375,405.41210938)
\curveto(283.17578125,405.90429688)(283.0625,406.46875)(283.0625,407.10546875)
\curveto(283.0625,407.7265625)(283.16601562,408.2890625)(283.37304688,408.79296875)
\curveto(283.58007812,409.30078125)(283.890625,409.68945312)(284.3046875,409.95898438)
\curveto(284.71875,410.22851562)(285.18164062,410.36328125)(285.69335938,410.36328125)
\curveto(286.06835938,410.36328125)(286.40234375,410.28320312)(286.6953125,410.12304688)
\curveto(286.98828125,409.96679688)(287.2265625,409.76171875)(287.41015625,409.5078125)
\lineto(287.41015625,412.58984375)
\lineto(288.45898438,412.58984375)
\lineto(288.45898438,404)
\closepath
\moveto(284.14648438,407.10546875)
\curveto(284.14648438,406.30859375)(284.31445312,405.71289062)(284.65039062,405.31835938)
\curveto(284.98632812,404.92382812)(285.3828125,404.7265625)(285.83984375,404.7265625)
\curveto(286.30078125,404.7265625)(286.69140625,404.9140625)(287.01171875,405.2890625)
\curveto(287.3359375,405.66796875)(287.49804688,406.24414062)(287.49804688,407.01757812)
\curveto(287.49804688,407.86914062)(287.33398438,408.49414062)(287.00585938,408.89257812)
\curveto(286.67773438,409.29101562)(286.2734375,409.49023438)(285.79296875,409.49023438)
\curveto(285.32421875,409.49023438)(284.93164062,409.29882812)(284.61523438,408.91601562)
\curveto(284.30273438,408.53320312)(284.14648438,407.9296875)(284.14648438,407.10546875)
\closepath
}
}
{
\newrgbcolor{curcolor}{0 0 0}
\pscustom[linestyle=none,fillstyle=solid,fillcolor=curcolor]
{
\newpath
\moveto(294.37695312,406.00390625)
\lineto(295.46679688,405.86914062)
\curveto(295.29492188,405.23242188)(294.9765625,404.73828125)(294.51171875,404.38671875)
\curveto(294.046875,404.03515625)(293.453125,403.859375)(292.73046875,403.859375)
\curveto(291.8203125,403.859375)(291.09765625,404.13867188)(290.5625,404.69726562)
\curveto(290.03125,405.25976562)(289.765625,406.046875)(289.765625,407.05859375)
\curveto(289.765625,408.10546875)(290.03515625,408.91796875)(290.57421875,409.49609375)
\curveto(291.11328125,410.07421875)(291.8125,410.36328125)(292.671875,410.36328125)
\curveto(293.50390625,410.36328125)(294.18359375,410.08007812)(294.7109375,409.51367188)
\curveto(295.23828125,408.94726562)(295.50195312,408.15039062)(295.50195312,407.12304688)
\curveto(295.50195312,407.06054688)(295.5,406.96679688)(295.49609375,406.84179688)
\lineto(290.85546875,406.84179688)
\curveto(290.89453125,406.15820312)(291.08789062,405.63476562)(291.43554688,405.27148438)
\curveto(291.78320312,404.90820312)(292.21679688,404.7265625)(292.73632812,404.7265625)
\curveto(293.12304688,404.7265625)(293.453125,404.828125)(293.7265625,405.03125)
\curveto(294,405.234375)(294.21679688,405.55859375)(294.37695312,406.00390625)
\closepath
\moveto(290.9140625,407.70898438)
\lineto(294.38867188,407.70898438)
\curveto(294.34179688,408.23242188)(294.20898438,408.625)(293.99023438,408.88671875)
\curveto(293.65429688,409.29296875)(293.21875,409.49609375)(292.68359375,409.49609375)
\curveto(292.19921875,409.49609375)(291.79101562,409.33398438)(291.45898438,409.00976562)
\curveto(291.13085938,408.68554688)(290.94921875,408.25195312)(290.9140625,407.70898438)
\closepath
}
}
{
\newrgbcolor{curcolor}{1 0 0}
\pscustom[linewidth=1,linecolor=curcolor]
{
\newpath
\moveto(304.3,407.9)
\lineto(346.5,407.9)
\moveto(105.1,89.7)
\lineto(183.4,129.6)
\lineto(261.7,217.3)
\lineto(340.1,262.8)
\lineto(418.4,308)
\lineto(496.7,354.3)
\lineto(575,400)
}
}
{
\newrgbcolor{curcolor}{0 0 0}
\pscustom[linestyle=none,fillstyle=solid,fillcolor=curcolor]
{
\newpath
\moveto(185.20507812,390.18359375)
\curveto(185.20507812,391.609375)(185.58789062,392.72460938)(186.35351562,393.52929688)
\curveto(187.11914062,394.33789062)(188.10742188,394.7421875)(189.31835938,394.7421875)
\curveto(190.11132812,394.7421875)(190.82617188,394.55273438)(191.46289062,394.17382812)
\curveto(192.09960938,393.79492188)(192.58398438,393.265625)(192.91601562,392.5859375)
\curveto(193.25195312,391.91015625)(193.41992188,391.14257812)(193.41992188,390.28320312)
\curveto(193.41992188,389.41210938)(193.24414062,388.6328125)(192.89257812,387.9453125)
\curveto(192.54101562,387.2578125)(192.04296875,386.73632812)(191.3984375,386.38085938)
\curveto(190.75390625,386.02929688)(190.05859375,385.85351562)(189.3125,385.85351562)
\curveto(188.50390625,385.85351562)(187.78125,386.04882812)(187.14453125,386.43945312)
\curveto(186.5078125,386.83007812)(186.02539062,387.36328125)(185.69726562,388.0390625)
\curveto(185.36914062,388.71484375)(185.20507812,389.4296875)(185.20507812,390.18359375)
\closepath
\moveto(186.37695312,390.16601562)
\curveto(186.37695312,389.13085938)(186.65429688,388.31445312)(187.20898438,387.71679688)
\curveto(187.76757812,387.12304688)(188.46679688,386.82617188)(189.30664062,386.82617188)
\curveto(190.16210938,386.82617188)(190.86523438,387.12695312)(191.41601562,387.72851562)
\curveto(191.97070312,388.33007812)(192.24804688,389.18359375)(192.24804688,390.2890625)
\curveto(192.24804688,390.98828125)(192.12890625,391.59765625)(191.890625,392.1171875)
\curveto(191.65625,392.640625)(191.31054688,393.04492188)(190.85351562,393.33007812)
\curveto(190.40039062,393.61914062)(189.890625,393.76367188)(189.32421875,393.76367188)
\curveto(188.51953125,393.76367188)(187.82617188,393.48632812)(187.24414062,392.93164062)
\curveto(186.66601562,392.38085938)(186.37695312,391.45898438)(186.37695312,390.16601562)
\closepath
}
}
{
\newrgbcolor{curcolor}{0 0 0}
\pscustom[linestyle=none,fillstyle=solid,fillcolor=curcolor]
{
\newpath
\moveto(194.7265625,386)
\lineto(194.7265625,394.58984375)
\lineto(195.78125,394.58984375)
\lineto(195.78125,386)
\closepath
}
}
{
\newrgbcolor{curcolor}{0 0 0}
\pscustom[linestyle=none,fillstyle=solid,fillcolor=curcolor]
{
\newpath
\moveto(201.453125,386)
\lineto(201.453125,386.78515625)
\curveto(201.05859375,386.16796875)(200.47851562,385.859375)(199.71289062,385.859375)
\curveto(199.21679688,385.859375)(198.75976562,385.99609375)(198.34179688,386.26953125)
\curveto(197.92773438,386.54296875)(197.60546875,386.92382812)(197.375,387.41210938)
\curveto(197.1484375,387.90429688)(197.03515625,388.46875)(197.03515625,389.10546875)
\curveto(197.03515625,389.7265625)(197.13867188,390.2890625)(197.34570312,390.79296875)
\curveto(197.55273438,391.30078125)(197.86328125,391.68945312)(198.27734375,391.95898438)
\curveto(198.69140625,392.22851562)(199.15429688,392.36328125)(199.66601562,392.36328125)
\curveto(200.04101562,392.36328125)(200.375,392.28320312)(200.66796875,392.12304688)
\curveto(200.9609375,391.96679688)(201.19921875,391.76171875)(201.3828125,391.5078125)
\lineto(201.3828125,394.58984375)
\lineto(202.43164062,394.58984375)
\lineto(202.43164062,386)
\closepath
\moveto(198.11914062,389.10546875)
\curveto(198.11914062,388.30859375)(198.28710938,387.71289062)(198.62304688,387.31835938)
\curveto(198.95898438,386.92382812)(199.35546875,386.7265625)(199.8125,386.7265625)
\curveto(200.2734375,386.7265625)(200.6640625,386.9140625)(200.984375,387.2890625)
\curveto(201.30859375,387.66796875)(201.47070312,388.24414062)(201.47070312,389.01757812)
\curveto(201.47070312,389.86914062)(201.30664062,390.49414062)(200.97851562,390.89257812)
\curveto(200.65039062,391.29101562)(200.24609375,391.49023438)(199.765625,391.49023438)
\curveto(199.296875,391.49023438)(198.90429688,391.29882812)(198.58789062,390.91601562)
\curveto(198.27539062,390.53320312)(198.11914062,389.9296875)(198.11914062,389.10546875)
\closepath
}
}
{
\newrgbcolor{curcolor}{0 0 0}
\pscustom[linestyle=none,fillstyle=solid,fillcolor=curcolor]
{
\newpath
\moveto(206.87304688,386)
\lineto(206.87304688,387.0546875)
\lineto(211.2734375,392.55664062)
\curveto(211.5859375,392.94726562)(211.8828125,393.28710938)(212.1640625,393.57617188)
\lineto(207.37109375,393.57617188)
\lineto(207.37109375,394.58984375)
\lineto(213.5234375,394.58984375)
\lineto(213.5234375,393.57617188)
\lineto(208.70117188,387.6171875)
\lineto(208.1796875,387.01367188)
\lineto(213.6640625,387.01367188)
\lineto(213.6640625,386)
\closepath
}
}
{
\newrgbcolor{curcolor}{0 0 0}
\pscustom[linestyle=none,fillstyle=solid,fillcolor=curcolor]
{
\newpath
\moveto(214.94726562,386)
\lineto(214.94726562,394.58984375)
\lineto(220.7421875,394.58984375)
\lineto(220.7421875,393.57617188)
\lineto(216.08398438,393.57617188)
\lineto(216.08398438,390.91601562)
\lineto(220.11523438,390.91601562)
\lineto(220.11523438,389.90234375)
\lineto(216.08398438,389.90234375)
\lineto(216.08398438,386)
\closepath
}
}
{
\newrgbcolor{curcolor}{0 0 0}
\pscustom[linestyle=none,fillstyle=solid,fillcolor=curcolor]
{
\newpath
\moveto(221.83203125,388.75976562)
\lineto(222.90429688,388.85351562)
\curveto(222.95507812,388.42382812)(223.07226562,388.0703125)(223.25585938,387.79296875)
\curveto(223.44335938,387.51953125)(223.73242188,387.296875)(224.12304688,387.125)
\curveto(224.51367188,386.95703125)(224.953125,386.87304688)(225.44140625,386.87304688)
\curveto(225.875,386.87304688)(226.2578125,386.9375)(226.58984375,387.06640625)
\curveto(226.921875,387.1953125)(227.16796875,387.37109375)(227.328125,387.59375)
\curveto(227.4921875,387.8203125)(227.57421875,388.06640625)(227.57421875,388.33203125)
\curveto(227.57421875,388.6015625)(227.49609375,388.8359375)(227.33984375,389.03515625)
\curveto(227.18359375,389.23828125)(226.92578125,389.40820312)(226.56640625,389.54492188)
\curveto(226.3359375,389.63476562)(225.82617188,389.7734375)(225.03710938,389.9609375)
\curveto(224.24804688,390.15234375)(223.6953125,390.33203125)(223.37890625,390.5)
\curveto(222.96875,390.71484375)(222.66210938,390.98046875)(222.45898438,391.296875)
\curveto(222.25976562,391.6171875)(222.16015625,391.97460938)(222.16015625,392.36914062)
\curveto(222.16015625,392.80273438)(222.28320312,393.20703125)(222.52929688,393.58203125)
\curveto(222.77539062,393.9609375)(223.13476562,394.24804688)(223.60742188,394.44335938)
\curveto(224.08007812,394.63867188)(224.60546875,394.73632812)(225.18359375,394.73632812)
\curveto(225.8203125,394.73632812)(226.38085938,394.6328125)(226.86523438,394.42578125)
\curveto(227.35351562,394.22265625)(227.72851562,393.921875)(227.99023438,393.5234375)
\curveto(228.25195312,393.125)(228.39257812,392.67382812)(228.41210938,392.16992188)
\lineto(227.32226562,392.08789062)
\curveto(227.26367188,392.63085938)(227.06445312,393.04101562)(226.72460938,393.31835938)
\curveto(226.38867188,393.59570312)(225.890625,393.734375)(225.23046875,393.734375)
\curveto(224.54296875,393.734375)(224.04101562,393.60742188)(223.72460938,393.35351562)
\curveto(223.41210938,393.10351562)(223.25585938,392.80078125)(223.25585938,392.4453125)
\curveto(223.25585938,392.13671875)(223.3671875,391.8828125)(223.58984375,391.68359375)
\curveto(223.80859375,391.484375)(224.37890625,391.27929688)(225.30078125,391.06835938)
\curveto(226.2265625,390.86132812)(226.86132812,390.6796875)(227.20507812,390.5234375)
\curveto(227.70507812,390.29296875)(228.07421875,390)(228.3125,389.64453125)
\curveto(228.55078125,389.29296875)(228.66992188,388.88671875)(228.66992188,388.42578125)
\curveto(228.66992188,387.96875)(228.5390625,387.53710938)(228.27734375,387.13085938)
\curveto(228.015625,386.72851562)(227.63867188,386.4140625)(227.14648438,386.1875)
\curveto(226.65820312,385.96484375)(226.10742188,385.85351562)(225.49414062,385.85351562)
\curveto(224.71679688,385.85351562)(224.06445312,385.96679688)(223.53710938,386.19335938)
\curveto(223.01367188,386.41992188)(222.6015625,386.75976562)(222.30078125,387.21289062)
\curveto(222.00390625,387.66992188)(221.84765625,388.18554688)(221.83203125,388.75976562)
\closepath
}
}
{
\newrgbcolor{curcolor}{0 0 0}
\pscustom[linestyle=none,fillstyle=solid,fillcolor=curcolor]
{
\newpath
\moveto(233.16992188,388.75976562)
\lineto(234.2421875,388.85351562)
\curveto(234.29296875,388.42382812)(234.41015625,388.0703125)(234.59375,387.79296875)
\curveto(234.78125,387.51953125)(235.0703125,387.296875)(235.4609375,387.125)
\curveto(235.8515625,386.95703125)(236.29101562,386.87304688)(236.77929688,386.87304688)
\curveto(237.21289062,386.87304688)(237.59570312,386.9375)(237.92773438,387.06640625)
\curveto(238.25976562,387.1953125)(238.50585938,387.37109375)(238.66601562,387.59375)
\curveto(238.83007812,387.8203125)(238.91210938,388.06640625)(238.91210938,388.33203125)
\curveto(238.91210938,388.6015625)(238.83398438,388.8359375)(238.67773438,389.03515625)
\curveto(238.52148438,389.23828125)(238.26367188,389.40820312)(237.90429688,389.54492188)
\curveto(237.67382812,389.63476562)(237.1640625,389.7734375)(236.375,389.9609375)
\curveto(235.5859375,390.15234375)(235.03320312,390.33203125)(234.71679688,390.5)
\curveto(234.30664062,390.71484375)(234,390.98046875)(233.796875,391.296875)
\curveto(233.59765625,391.6171875)(233.49804688,391.97460938)(233.49804688,392.36914062)
\curveto(233.49804688,392.80273438)(233.62109375,393.20703125)(233.8671875,393.58203125)
\curveto(234.11328125,393.9609375)(234.47265625,394.24804688)(234.9453125,394.44335938)
\curveto(235.41796875,394.63867188)(235.94335938,394.73632812)(236.52148438,394.73632812)
\curveto(237.15820312,394.73632812)(237.71875,394.6328125)(238.203125,394.42578125)
\curveto(238.69140625,394.22265625)(239.06640625,393.921875)(239.328125,393.5234375)
\curveto(239.58984375,393.125)(239.73046875,392.67382812)(239.75,392.16992188)
\lineto(238.66015625,392.08789062)
\curveto(238.6015625,392.63085938)(238.40234375,393.04101562)(238.0625,393.31835938)
\curveto(237.7265625,393.59570312)(237.22851562,393.734375)(236.56835938,393.734375)
\curveto(235.88085938,393.734375)(235.37890625,393.60742188)(235.0625,393.35351562)
\curveto(234.75,393.10351562)(234.59375,392.80078125)(234.59375,392.4453125)
\curveto(234.59375,392.13671875)(234.70507812,391.8828125)(234.92773438,391.68359375)
\curveto(235.14648438,391.484375)(235.71679688,391.27929688)(236.63867188,391.06835938)
\curveto(237.56445312,390.86132812)(238.19921875,390.6796875)(238.54296875,390.5234375)
\curveto(239.04296875,390.29296875)(239.41210938,390)(239.65039062,389.64453125)
\curveto(239.88867188,389.29296875)(240.0078125,388.88671875)(240.0078125,388.42578125)
\curveto(240.0078125,387.96875)(239.87695312,387.53710938)(239.61523438,387.13085938)
\curveto(239.35351562,386.72851562)(238.9765625,386.4140625)(238.484375,386.1875)
\curveto(237.99609375,385.96484375)(237.4453125,385.85351562)(236.83203125,385.85351562)
\curveto(236.0546875,385.85351562)(235.40234375,385.96679688)(234.875,386.19335938)
\curveto(234.3515625,386.41992188)(233.93945312,386.75976562)(233.63867188,387.21289062)
\curveto(233.34179688,387.66992188)(233.18554688,388.18554688)(233.16992188,388.75976562)
\closepath
}
}
{
\newrgbcolor{curcolor}{0 0 0}
\pscustom[linestyle=none,fillstyle=solid,fillcolor=curcolor]
{
\newpath
\moveto(245.48632812,386.76757812)
\curveto(245.09570312,386.43554688)(244.71875,386.20117188)(244.35546875,386.06445312)
\curveto(243.99609375,385.92773438)(243.609375,385.859375)(243.1953125,385.859375)
\curveto(242.51171875,385.859375)(241.98632812,386.02539062)(241.61914062,386.35742188)
\curveto(241.25195312,386.69335938)(241.06835938,387.12109375)(241.06835938,387.640625)
\curveto(241.06835938,387.9453125)(241.13671875,388.22265625)(241.2734375,388.47265625)
\curveto(241.4140625,388.7265625)(241.59570312,388.9296875)(241.81835938,389.08203125)
\curveto(242.04492188,389.234375)(242.29882812,389.34960938)(242.58007812,389.42773438)
\curveto(242.78710938,389.48242188)(243.09960938,389.53515625)(243.51757812,389.5859375)
\curveto(244.36914062,389.6875)(244.99609375,389.80859375)(245.3984375,389.94921875)
\curveto(245.40234375,390.09375)(245.40429688,390.18554688)(245.40429688,390.22460938)
\curveto(245.40429688,390.65429688)(245.3046875,390.95703125)(245.10546875,391.1328125)
\curveto(244.8359375,391.37109375)(244.43554688,391.49023438)(243.90429688,391.49023438)
\curveto(243.40820312,391.49023438)(243.04101562,391.40234375)(242.80273438,391.2265625)
\curveto(242.56835938,391.0546875)(242.39453125,390.74804688)(242.28125,390.30664062)
\lineto(241.25,390.44726562)
\curveto(241.34375,390.88867188)(241.49804688,391.24414062)(241.71289062,391.51367188)
\curveto(241.92773438,391.78710938)(242.23828125,391.99609375)(242.64453125,392.140625)
\curveto(243.05078125,392.2890625)(243.52148438,392.36328125)(244.05664062,392.36328125)
\curveto(244.58789062,392.36328125)(245.01953125,392.30078125)(245.3515625,392.17578125)
\curveto(245.68359375,392.05078125)(245.92773438,391.89257812)(246.08398438,391.70117188)
\curveto(246.24023438,391.51367188)(246.34960938,391.27539062)(246.41210938,390.98632812)
\curveto(246.44726562,390.80664062)(246.46484375,390.48242188)(246.46484375,390.01367188)
\lineto(246.46484375,388.60742188)
\curveto(246.46484375,387.62695312)(246.48632812,387.00585938)(246.52929688,386.74414062)
\curveto(246.57617188,386.48632812)(246.66601562,386.23828125)(246.79882812,386)
\lineto(245.69726562,386)
\curveto(245.58789062,386.21875)(245.51757812,386.47460938)(245.48632812,386.76757812)
\closepath
\moveto(245.3984375,389.12304688)
\curveto(245.015625,388.96679688)(244.44140625,388.83398438)(243.67578125,388.72460938)
\curveto(243.2421875,388.66210938)(242.93554688,388.59179688)(242.75585938,388.51367188)
\curveto(242.57617188,388.43554688)(242.4375,388.3203125)(242.33984375,388.16796875)
\curveto(242.2421875,388.01953125)(242.19335938,387.85351562)(242.19335938,387.66992188)
\curveto(242.19335938,387.38867188)(242.29882812,387.15429688)(242.50976562,386.96679688)
\curveto(242.72460938,386.77929688)(243.03710938,386.68554688)(243.44726562,386.68554688)
\curveto(243.85351562,386.68554688)(244.21484375,386.7734375)(244.53125,386.94921875)
\curveto(244.84765625,387.12890625)(245.08007812,387.37304688)(245.22851562,387.68164062)
\curveto(245.34179688,387.91992188)(245.3984375,388.27148438)(245.3984375,388.73632812)
\closepath
}
}
{
\newrgbcolor{curcolor}{0 0 0}
\pscustom[linestyle=none,fillstyle=solid,fillcolor=curcolor]
{
\newpath
\moveto(248.09960938,386)
\lineto(248.09960938,392.22265625)
\lineto(249.04296875,392.22265625)
\lineto(249.04296875,391.34960938)
\curveto(249.23828125,391.65429688)(249.49804688,391.8984375)(249.82226562,392.08203125)
\curveto(250.14648438,392.26953125)(250.515625,392.36328125)(250.9296875,392.36328125)
\curveto(251.390625,392.36328125)(251.76757812,392.26757812)(252.06054688,392.07617188)
\curveto(252.35742188,391.88476562)(252.56640625,391.6171875)(252.6875,391.2734375)
\curveto(253.1796875,392)(253.8203125,392.36328125)(254.609375,392.36328125)
\curveto(255.2265625,392.36328125)(255.70117188,392.19140625)(256.03320312,391.84765625)
\curveto(256.36523438,391.5078125)(256.53125,390.98242188)(256.53125,390.27148438)
\lineto(256.53125,386)
\lineto(255.48242188,386)
\lineto(255.48242188,389.91992188)
\curveto(255.48242188,390.34179688)(255.44726562,390.64453125)(255.37695312,390.828125)
\curveto(255.31054688,391.015625)(255.1875,391.16601562)(255.0078125,391.27929688)
\curveto(254.828125,391.39257812)(254.6171875,391.44921875)(254.375,391.44921875)
\curveto(253.9375,391.44921875)(253.57421875,391.30273438)(253.28515625,391.00976562)
\curveto(252.99609375,390.72070312)(252.8515625,390.25585938)(252.8515625,389.61523438)
\lineto(252.8515625,386)
\lineto(251.796875,386)
\lineto(251.796875,390.04296875)
\curveto(251.796875,390.51171875)(251.7109375,390.86328125)(251.5390625,391.09765625)
\curveto(251.3671875,391.33203125)(251.0859375,391.44921875)(250.6953125,391.44921875)
\curveto(250.3984375,391.44921875)(250.12304688,391.37109375)(249.86914062,391.21484375)
\curveto(249.61914062,391.05859375)(249.4375,390.83007812)(249.32421875,390.52929688)
\curveto(249.2109375,390.22851562)(249.15429688,389.79492188)(249.15429688,389.22851562)
\lineto(249.15429688,386)
\closepath
}
}
{
\newrgbcolor{curcolor}{0 0 0}
\pscustom[linestyle=none,fillstyle=solid,fillcolor=curcolor]
{
\newpath
\moveto(262.35546875,388.00390625)
\lineto(263.4453125,387.86914062)
\curveto(263.2734375,387.23242188)(262.95507812,386.73828125)(262.49023438,386.38671875)
\curveto(262.02539062,386.03515625)(261.43164062,385.859375)(260.70898438,385.859375)
\curveto(259.79882812,385.859375)(259.07617188,386.13867188)(258.54101562,386.69726562)
\curveto(258.00976562,387.25976562)(257.74414062,388.046875)(257.74414062,389.05859375)
\curveto(257.74414062,390.10546875)(258.01367188,390.91796875)(258.55273438,391.49609375)
\curveto(259.09179688,392.07421875)(259.79101562,392.36328125)(260.65039062,392.36328125)
\curveto(261.48242188,392.36328125)(262.16210938,392.08007812)(262.68945312,391.51367188)
\curveto(263.21679688,390.94726562)(263.48046875,390.15039062)(263.48046875,389.12304688)
\curveto(263.48046875,389.06054688)(263.47851562,388.96679688)(263.47460938,388.84179688)
\lineto(258.83398438,388.84179688)
\curveto(258.87304688,388.15820312)(259.06640625,387.63476562)(259.4140625,387.27148438)
\curveto(259.76171875,386.90820312)(260.1953125,386.7265625)(260.71484375,386.7265625)
\curveto(261.1015625,386.7265625)(261.43164062,386.828125)(261.70507812,387.03125)
\curveto(261.97851562,387.234375)(262.1953125,387.55859375)(262.35546875,388.00390625)
\closepath
\moveto(258.89257812,389.70898438)
\lineto(262.3671875,389.70898438)
\curveto(262.3203125,390.23242188)(262.1875,390.625)(261.96875,390.88671875)
\curveto(261.6328125,391.29296875)(261.19726562,391.49609375)(260.66210938,391.49609375)
\curveto(260.17773438,391.49609375)(259.76953125,391.33398438)(259.4375,391.00976562)
\curveto(259.109375,390.68554688)(258.92773438,390.25195312)(258.89257812,389.70898438)
\closepath
}
}
{
\newrgbcolor{curcolor}{0 0 0}
\pscustom[linestyle=none,fillstyle=solid,fillcolor=curcolor]
{
\newpath
\moveto(268.2265625,386)
\lineto(268.2265625,394.58984375)
\lineto(269.39257812,394.58984375)
\lineto(273.90429688,387.84570312)
\lineto(273.90429688,394.58984375)
\lineto(274.99414062,394.58984375)
\lineto(274.99414062,386)
\lineto(273.828125,386)
\lineto(269.31640625,392.75)
\lineto(269.31640625,386)
\closepath
}
}
{
\newrgbcolor{curcolor}{0 0 0}
\pscustom[linestyle=none,fillstyle=solid,fillcolor=curcolor]
{
\newpath
\moveto(276.37695312,389.11132812)
\curveto(276.37695312,390.26367188)(276.69726562,391.1171875)(277.33789062,391.671875)
\curveto(277.87304688,392.1328125)(278.52539062,392.36328125)(279.29492188,392.36328125)
\curveto(280.15039062,392.36328125)(280.84960938,392.08203125)(281.39257812,391.51953125)
\curveto(281.93554688,390.9609375)(282.20703125,390.1875)(282.20703125,389.19921875)
\curveto(282.20703125,388.3984375)(282.0859375,387.76757812)(281.84375,387.30664062)
\curveto(281.60546875,386.84960938)(281.25585938,386.49414062)(280.79492188,386.24023438)
\curveto(280.33789062,385.98632812)(279.83789062,385.859375)(279.29492188,385.859375)
\curveto(278.42382812,385.859375)(277.71875,386.13867188)(277.1796875,386.69726562)
\curveto(276.64453125,387.25585938)(276.37695312,388.06054688)(276.37695312,389.11132812)
\closepath
\moveto(277.4609375,389.11132812)
\curveto(277.4609375,388.31445312)(277.63476562,387.71679688)(277.98242188,387.31835938)
\curveto(278.33007812,386.92382812)(278.76757812,386.7265625)(279.29492188,386.7265625)
\curveto(279.81835938,386.7265625)(280.25390625,386.92578125)(280.6015625,387.32421875)
\curveto(280.94921875,387.72265625)(281.12304688,388.33007812)(281.12304688,389.14648438)
\curveto(281.12304688,389.91601562)(280.94726562,390.49804688)(280.59570312,390.89257812)
\curveto(280.24804688,391.29101562)(279.81445312,391.49023438)(279.29492188,391.49023438)
\curveto(278.76757812,391.49023438)(278.33007812,391.29296875)(277.98242188,390.8984375)
\curveto(277.63476562,390.50390625)(277.4609375,389.90820312)(277.4609375,389.11132812)
\closepath
}
}
{
\newrgbcolor{curcolor}{0 0 0}
\pscustom[linestyle=none,fillstyle=solid,fillcolor=curcolor]
{
\newpath
\moveto(287.48046875,386)
\lineto(287.48046875,386.78515625)
\curveto(287.0859375,386.16796875)(286.50585938,385.859375)(285.74023438,385.859375)
\curveto(285.24414062,385.859375)(284.78710938,385.99609375)(284.36914062,386.26953125)
\curveto(283.95507812,386.54296875)(283.6328125,386.92382812)(283.40234375,387.41210938)
\curveto(283.17578125,387.90429688)(283.0625,388.46875)(283.0625,389.10546875)
\curveto(283.0625,389.7265625)(283.16601562,390.2890625)(283.37304688,390.79296875)
\curveto(283.58007812,391.30078125)(283.890625,391.68945312)(284.3046875,391.95898438)
\curveto(284.71875,392.22851562)(285.18164062,392.36328125)(285.69335938,392.36328125)
\curveto(286.06835938,392.36328125)(286.40234375,392.28320312)(286.6953125,392.12304688)
\curveto(286.98828125,391.96679688)(287.2265625,391.76171875)(287.41015625,391.5078125)
\lineto(287.41015625,394.58984375)
\lineto(288.45898438,394.58984375)
\lineto(288.45898438,386)
\closepath
\moveto(284.14648438,389.10546875)
\curveto(284.14648438,388.30859375)(284.31445312,387.71289062)(284.65039062,387.31835938)
\curveto(284.98632812,386.92382812)(285.3828125,386.7265625)(285.83984375,386.7265625)
\curveto(286.30078125,386.7265625)(286.69140625,386.9140625)(287.01171875,387.2890625)
\curveto(287.3359375,387.66796875)(287.49804688,388.24414062)(287.49804688,389.01757812)
\curveto(287.49804688,389.86914062)(287.33398438,390.49414062)(287.00585938,390.89257812)
\curveto(286.67773438,391.29101562)(286.2734375,391.49023438)(285.79296875,391.49023438)
\curveto(285.32421875,391.49023438)(284.93164062,391.29882812)(284.61523438,390.91601562)
\curveto(284.30273438,390.53320312)(284.14648438,389.9296875)(284.14648438,389.10546875)
\closepath
}
}
{
\newrgbcolor{curcolor}{0 0 0}
\pscustom[linestyle=none,fillstyle=solid,fillcolor=curcolor]
{
\newpath
\moveto(294.37695312,388.00390625)
\lineto(295.46679688,387.86914062)
\curveto(295.29492188,387.23242188)(294.9765625,386.73828125)(294.51171875,386.38671875)
\curveto(294.046875,386.03515625)(293.453125,385.859375)(292.73046875,385.859375)
\curveto(291.8203125,385.859375)(291.09765625,386.13867188)(290.5625,386.69726562)
\curveto(290.03125,387.25976562)(289.765625,388.046875)(289.765625,389.05859375)
\curveto(289.765625,390.10546875)(290.03515625,390.91796875)(290.57421875,391.49609375)
\curveto(291.11328125,392.07421875)(291.8125,392.36328125)(292.671875,392.36328125)
\curveto(293.50390625,392.36328125)(294.18359375,392.08007812)(294.7109375,391.51367188)
\curveto(295.23828125,390.94726562)(295.50195312,390.15039062)(295.50195312,389.12304688)
\curveto(295.50195312,389.06054688)(295.5,388.96679688)(295.49609375,388.84179688)
\lineto(290.85546875,388.84179688)
\curveto(290.89453125,388.15820312)(291.08789062,387.63476562)(291.43554688,387.27148438)
\curveto(291.78320312,386.90820312)(292.21679688,386.7265625)(292.73632812,386.7265625)
\curveto(293.12304688,386.7265625)(293.453125,386.828125)(293.7265625,387.03125)
\curveto(294,387.234375)(294.21679688,387.55859375)(294.37695312,388.00390625)
\closepath
\moveto(290.9140625,389.70898438)
\lineto(294.38867188,389.70898438)
\curveto(294.34179688,390.23242188)(294.20898438,390.625)(293.99023438,390.88671875)
\curveto(293.65429688,391.29296875)(293.21875,391.49609375)(292.68359375,391.49609375)
\curveto(292.19921875,391.49609375)(291.79101562,391.33398438)(291.45898438,391.00976562)
\curveto(291.13085938,390.68554688)(290.94921875,390.25195312)(290.9140625,389.70898438)
\closepath
}
}
{
\newrgbcolor{curcolor}{0 0 1}
\pscustom[linewidth=1,linecolor=curcolor]
{
\newpath
\moveto(304.3,389.9)
\lineto(346.5,389.9)
\moveto(105.1,83.8)
\lineto(183.4,130)
\lineto(261.7,222.6)
\lineto(340.1,264.3)
\lineto(418.4,307.8)
\lineto(496.7,353.8)
\lineto(575,399.7)
}
}
{
\newrgbcolor{curcolor}{0 0 0}
\pscustom[linewidth=1,linecolor=curcolor]
{
\newpath
\moveto(105.1,425.9)
\lineto(105.1,57.6)
\lineto(575,57.6)
\lineto(575,425.9)
\closepath
}
}
\end{pspicture}
}
    \captionsetup{width=0.75\linewidth}
    \caption{Task migration with the ARC on a different node compared to starting the process on the same node.
        The results are practically identical, showing that task migration can significantly improve performance without substantial
        overhead in this scenario.}
    \label{fig:optimal}
\end{figure}

The bandwidth of the reads, as measured with fio, also substantially improves (Figure \ref{fig:bandwidth}).
Given that the memory and interconnect bandwidth on the test system are essentially the same,
31.99 GB/s for the 48GB of DDR3 1066 RAM and 25.60 GB/s for the QPI, one would not expect use of the interconnect to affect bandwidth very much, with only about a 6\% difference between the two\cite{kochhar_optimal_2009}.
However, testing shows that bandwidth is affected in a similar way as latency, with a fairly consistent 10\% improvement.

\begin{figure}[H]
    \centering
    \resizebox{0.75\linewidth}{!}{%LaTeX with PSTricks extensions
%%Creator: Inkscape 1.0.2-2 (e86c870879, 2021-01-15)
%%Please note this file requires PSTricks extensions
\psset{xunit=.5pt,yunit=.5pt,runit=.5pt}
\begin{pspicture}(600,480)
{
\newrgbcolor{curcolor}{0 0 0}
\pscustom[linewidth=1,linecolor=curcolor]
{
\newpath
\moveto(55.3,57.6)
\lineto(64.3,57.6)
\moveto(575,57.6)
\lineto(566,57.6)
}
}
{
\newrgbcolor{curcolor}{0 0 0}
\pscustom[linestyle=none,fillstyle=solid,fillcolor=curcolor]
{
\newpath
\moveto(46.3671875,54.71367187)
\lineto(46.3671875,53.7)
\lineto(40.68945312,53.7)
\curveto(40.68164062,53.95390625)(40.72265625,54.19804687)(40.8125,54.43242187)
\curveto(40.95703125,54.81914062)(41.1875,55.2)(41.50390625,55.575)
\curveto(41.82421875,55.95)(42.28515625,56.38359375)(42.88671875,56.87578125)
\curveto(43.8203125,57.64140625)(44.45117188,58.246875)(44.77929688,58.6921875)
\curveto(45.10742188,59.14140625)(45.27148438,59.56523437)(45.27148438,59.96367187)
\curveto(45.27148438,60.38164062)(45.12109375,60.73320312)(44.8203125,61.01835937)
\curveto(44.5234375,61.30742187)(44.13476562,61.45195312)(43.65429688,61.45195312)
\curveto(43.14648438,61.45195312)(42.74023438,61.29960937)(42.43554688,60.99492187)
\curveto(42.13085938,60.69023437)(41.9765625,60.26835937)(41.97265625,59.72929687)
\lineto(40.88867188,59.840625)
\curveto(40.96289062,60.64921875)(41.2421875,61.26445312)(41.7265625,61.68632812)
\curveto(42.2109375,62.11210937)(42.86132812,62.325)(43.67773438,62.325)
\curveto(44.50195312,62.325)(45.15429688,62.09648437)(45.63476562,61.63945312)
\curveto(46.11523438,61.18242187)(46.35546875,60.61601562)(46.35546875,59.94023437)
\curveto(46.35546875,59.59648437)(46.28515625,59.25859375)(46.14453125,58.9265625)
\curveto(46.00390625,58.59453125)(45.76953125,58.24492187)(45.44140625,57.87773437)
\curveto(45.1171875,57.51054687)(44.57617188,57.00664062)(43.81835938,56.36601562)
\curveto(43.18554688,55.83476562)(42.77929688,55.4734375)(42.59960938,55.28203125)
\curveto(42.41992188,55.09453125)(42.27148438,54.90507812)(42.15429688,54.71367187)
\closepath
}
}
{
\newrgbcolor{curcolor}{0 0 0}
\pscustom[linewidth=1,linecolor=curcolor]
{
\newpath
\moveto(55.3,241.8)
\lineto(64.3,241.8)
\moveto(575,241.8)
\lineto(566,241.8)
}
}
{
\newrgbcolor{curcolor}{0 0 0}
\pscustom[linestyle=none,fillstyle=solid,fillcolor=curcolor]
{
\newpath
\moveto(44.20507812,237.9)
\lineto(44.20507812,239.95664063)
\lineto(40.47851562,239.95664063)
\lineto(40.47851562,240.9234375)
\lineto(44.3984375,246.48984375)
\lineto(45.25976562,246.48984375)
\lineto(45.25976562,240.9234375)
\lineto(46.41992188,240.9234375)
\lineto(46.41992188,239.95664063)
\lineto(45.25976562,239.95664063)
\lineto(45.25976562,237.9)
\closepath
\moveto(44.20507812,240.9234375)
\lineto(44.20507812,244.79648438)
\lineto(41.515625,240.9234375)
\closepath
}
}
{
\newrgbcolor{curcolor}{0 0 0}
\pscustom[linewidth=1,linecolor=curcolor]
{
\newpath
\moveto(55.3,425.9)
\lineto(64.3,425.9)
\moveto(575,425.9)
\lineto(566,425.9)
}
}
{
\newrgbcolor{curcolor}{0 0 0}
\pscustom[linestyle=none,fillstyle=solid,fillcolor=curcolor]
{
\newpath
\moveto(42.44726562,426.65820312)
\curveto(42.00976562,426.81835938)(41.68554688,427.046875)(41.47460938,427.34375)
\curveto(41.26367188,427.640625)(41.15820312,427.99609375)(41.15820312,428.41015625)
\curveto(41.15820312,429.03515625)(41.3828125,429.56054688)(41.83203125,429.98632812)
\curveto(42.28125,430.41210938)(42.87890625,430.625)(43.625,430.625)
\curveto(44.375,430.625)(44.97851562,430.40625)(45.43554688,429.96875)
\curveto(45.89257812,429.53515625)(46.12109375,429.00585938)(46.12109375,428.38085938)
\curveto(46.12109375,427.98242188)(46.015625,427.63476562)(45.8046875,427.33789062)
\curveto(45.59765625,427.04492188)(45.28125,426.81835938)(44.85546875,426.65820312)
\curveto(45.3828125,426.48632812)(45.78320312,426.20898438)(46.05664062,425.82617188)
\curveto(46.33398438,425.44335938)(46.47265625,424.98632812)(46.47265625,424.45507812)
\curveto(46.47265625,423.72070312)(46.21289062,423.10351562)(45.69335938,422.60351562)
\curveto(45.17382812,422.10351562)(44.49023438,421.85351562)(43.64257812,421.85351562)
\curveto(42.79492188,421.85351562)(42.11132812,422.10351562)(41.59179688,422.60351562)
\curveto(41.07226562,423.10742188)(40.8125,423.734375)(40.8125,424.484375)
\curveto(40.8125,425.04296875)(40.953125,425.50976562)(41.234375,425.88476562)
\curveto(41.51953125,426.26367188)(41.92382812,426.52148438)(42.44726562,426.65820312)
\closepath
\moveto(42.23632812,428.4453125)
\curveto(42.23632812,428.0390625)(42.3671875,427.70703125)(42.62890625,427.44921875)
\curveto(42.890625,427.19140625)(43.23046875,427.0625)(43.6484375,427.0625)
\curveto(44.0546875,427.0625)(44.38671875,427.18945312)(44.64453125,427.44335938)
\curveto(44.90625,427.70117188)(45.03710938,428.015625)(45.03710938,428.38671875)
\curveto(45.03710938,428.7734375)(44.90234375,429.09765625)(44.6328125,429.359375)
\curveto(44.3671875,429.625)(44.03515625,429.7578125)(43.63671875,429.7578125)
\curveto(43.234375,429.7578125)(42.90039062,429.62890625)(42.63476562,429.37109375)
\curveto(42.36914062,429.11328125)(42.23632812,428.8046875)(42.23632812,428.4453125)
\closepath
\moveto(41.89648438,424.47851562)
\curveto(41.89648438,424.17773438)(41.96679688,423.88671875)(42.10742188,423.60546875)
\curveto(42.25195312,423.32421875)(42.46484375,423.10546875)(42.74609375,422.94921875)
\curveto(43.02734375,422.796875)(43.33007812,422.72070312)(43.65429688,422.72070312)
\curveto(44.15820312,422.72070312)(44.57421875,422.8828125)(44.90234375,423.20703125)
\curveto(45.23046875,423.53125)(45.39453125,423.94335938)(45.39453125,424.44335938)
\curveto(45.39453125,424.95117188)(45.22460938,425.37109375)(44.88476562,425.703125)
\curveto(44.54882812,426.03515625)(44.12695312,426.20117188)(43.61914062,426.20117188)
\curveto(43.12304688,426.20117188)(42.7109375,426.03710938)(42.3828125,425.70898438)
\curveto(42.05859375,425.38085938)(41.89648438,424.97070312)(41.89648438,424.47851562)
\closepath
}
}
{
\newrgbcolor{curcolor}{0 0 0}
\pscustom[linewidth=1,linecolor=curcolor]
{
\newpath
\moveto(55.3,57.6)
\lineto(55.3,66.6)
\moveto(55.3,425.9)
\lineto(55.3,416.9)
}
}
{
\newrgbcolor{curcolor}{0 0 0}
\pscustom[linestyle=none,fillstyle=solid,fillcolor=curcolor]
{
\newpath
\moveto(46.33222656,36.71367187)
\lineto(46.33222656,35.7)
\lineto(40.65449219,35.7)
\curveto(40.64667969,35.95390625)(40.68769531,36.19804687)(40.77753906,36.43242187)
\curveto(40.92207031,36.81914062)(41.15253906,37.2)(41.46894531,37.575)
\curveto(41.78925781,37.95)(42.25019531,38.38359375)(42.85175781,38.87578125)
\curveto(43.78535156,39.64140625)(44.41621094,40.246875)(44.74433594,40.6921875)
\curveto(45.07246094,41.14140625)(45.23652344,41.56523437)(45.23652344,41.96367187)
\curveto(45.23652344,42.38164062)(45.08613281,42.73320312)(44.78535156,43.01835937)
\curveto(44.48847656,43.30742187)(44.09980469,43.45195312)(43.61933594,43.45195312)
\curveto(43.11152344,43.45195312)(42.70527344,43.29960937)(42.40058594,42.99492187)
\curveto(42.09589844,42.69023437)(41.94160156,42.26835937)(41.93769531,41.72929687)
\lineto(40.85371094,41.840625)
\curveto(40.92792969,42.64921875)(41.20722656,43.26445312)(41.69160156,43.68632812)
\curveto(42.17597656,44.11210937)(42.82636719,44.325)(43.64277344,44.325)
\curveto(44.46699219,44.325)(45.11933594,44.09648437)(45.59980469,43.63945312)
\curveto(46.08027344,43.18242187)(46.32050781,42.61601562)(46.32050781,41.94023437)
\curveto(46.32050781,41.59648437)(46.25019531,41.25859375)(46.10957031,40.9265625)
\curveto(45.96894531,40.59453125)(45.73457031,40.24492187)(45.40644531,39.87773437)
\curveto(45.08222656,39.51054687)(44.54121094,39.00664062)(43.78339844,38.36601562)
\curveto(43.15058594,37.83476562)(42.74433594,37.4734375)(42.56464844,37.28203125)
\curveto(42.38496094,37.09453125)(42.23652344,36.90507812)(42.11933594,36.71367187)
\closepath
}
}
{
\newrgbcolor{curcolor}{0 0 0}
\pscustom[linestyle=none,fillstyle=solid,fillcolor=curcolor]
{
\newpath
\moveto(47.46308594,37.95)
\lineto(48.57050781,38.04375)
\curveto(48.65253906,37.5046875)(48.84199219,37.0984375)(49.13886719,36.825)
\curveto(49.43964844,36.55546875)(49.80097656,36.42070312)(50.22285156,36.42070312)
\curveto(50.73066406,36.42070312)(51.16035156,36.61210937)(51.51191406,36.99492187)
\curveto(51.86347656,37.37773437)(52.03925781,37.88554687)(52.03925781,38.51835937)
\curveto(52.03925781,39.11992187)(51.86933594,39.59453125)(51.52949219,39.9421875)
\curveto(51.19355469,40.28984375)(50.75214844,40.46367187)(50.20527344,40.46367187)
\curveto(49.86542969,40.46367187)(49.55878906,40.38554687)(49.28535156,40.22929687)
\curveto(49.01191406,40.07695312)(48.79707031,39.87773437)(48.64082031,39.63164062)
\lineto(47.65058594,39.76054687)
\lineto(48.48261719,44.17265625)
\lineto(52.75410156,44.17265625)
\lineto(52.75410156,43.16484375)
\lineto(49.32636719,43.16484375)
\lineto(48.86347656,40.85625)
\curveto(49.37910156,41.215625)(49.92011719,41.3953125)(50.48652344,41.3953125)
\curveto(51.23652344,41.3953125)(51.86933594,41.13554687)(52.38496094,40.61601562)
\curveto(52.90058594,40.09648437)(53.15839844,39.42851562)(53.15839844,38.61210937)
\curveto(53.15839844,37.83476562)(52.93183594,37.16289062)(52.47871094,36.59648437)
\curveto(51.92792969,35.90117187)(51.17597656,35.55351562)(50.22285156,35.55351562)
\curveto(49.44160156,35.55351562)(48.80292969,35.77226562)(48.30683594,36.20976562)
\curveto(47.81464844,36.64726562)(47.53339844,37.22734375)(47.46308594,37.95)
\closepath
}
}
{
\newrgbcolor{curcolor}{0 0 0}
\pscustom[linestyle=none,fillstyle=solid,fillcolor=curcolor]
{
\newpath
\moveto(54.13691406,39.93632812)
\curveto(54.13691406,40.95195312)(54.24042969,41.76835937)(54.44746094,42.38554687)
\curveto(54.65839844,43.00664062)(54.96894531,43.48515625)(55.37910156,43.82109375)
\curveto(55.79316406,44.15703125)(56.31269531,44.325)(56.93769531,44.325)
\curveto(57.39863281,44.325)(57.80292969,44.23125)(58.15058594,44.04375)
\curveto(58.49824219,43.86015625)(58.78535156,43.59257812)(59.01191406,43.24101562)
\curveto(59.23847656,42.89335937)(59.41621094,42.46757812)(59.54511719,41.96367187)
\curveto(59.67402344,41.46367187)(59.73847656,40.78789062)(59.73847656,39.93632812)
\curveto(59.73847656,38.92851562)(59.63496094,38.1140625)(59.42792969,37.49296875)
\curveto(59.22089844,36.87578125)(58.91035156,36.39726562)(58.49628906,36.05742187)
\curveto(58.08613281,35.72148437)(57.56660156,35.55351562)(56.93769531,35.55351562)
\curveto(56.10957031,35.55351562)(55.45917969,35.85039062)(54.98652344,36.44414062)
\curveto(54.42011719,37.15898437)(54.13691406,38.32304687)(54.13691406,39.93632812)
\closepath
\moveto(55.22089844,39.93632812)
\curveto(55.22089844,38.52617187)(55.38496094,37.58671875)(55.71308594,37.11796875)
\curveto(56.04511719,36.653125)(56.45332031,36.42070312)(56.93769531,36.42070312)
\curveto(57.42207031,36.42070312)(57.82832031,36.65507812)(58.15644531,37.12382812)
\curveto(58.48847656,37.59257812)(58.65449219,38.53007812)(58.65449219,39.93632812)
\curveto(58.65449219,41.35039062)(58.48847656,42.28984375)(58.15644531,42.7546875)
\curveto(57.82832031,43.21953125)(57.41816406,43.45195312)(56.92597656,43.45195312)
\curveto(56.44160156,43.45195312)(56.05488281,43.246875)(55.76582031,42.83671875)
\curveto(55.40253906,42.31328125)(55.22089844,41.34648437)(55.22089844,39.93632812)
\closepath
}
}
{
\newrgbcolor{curcolor}{0 0 0}
\pscustom[linestyle=none,fillstyle=solid,fillcolor=curcolor]
{
\newpath
\moveto(61.20332031,35.7)
\lineto(61.20332031,44.28984375)
\lineto(62.91425781,44.28984375)
\lineto(64.94746094,38.2078125)
\curveto(65.13496094,37.64140625)(65.27167969,37.21757812)(65.35761719,36.93632812)
\curveto(65.45527344,37.24882812)(65.60761719,37.7078125)(65.81464844,38.31328125)
\lineto(67.87128906,44.28984375)
\lineto(69.40058594,44.28984375)
\lineto(69.40058594,35.7)
\lineto(68.30488281,35.7)
\lineto(68.30488281,42.88945312)
\lineto(65.80878906,35.7)
\lineto(64.78339844,35.7)
\lineto(62.29902344,43.0125)
\lineto(62.29902344,35.7)
\closepath
}
}
{
\newrgbcolor{curcolor}{0 0 0}
\pscustom[linewidth=1,linecolor=curcolor]
{
\newpath
\moveto(141.9,57.6)
\lineto(141.9,66.6)
\moveto(141.9,425.9)
\lineto(141.9,416.9)
}
}
{
\newrgbcolor{curcolor}{0 0 0}
\pscustom[linestyle=none,fillstyle=solid,fillcolor=curcolor]
{
\newpath
\moveto(127.38925781,37.95)
\lineto(128.49667969,38.04375)
\curveto(128.57871094,37.5046875)(128.76816406,37.0984375)(129.06503906,36.825)
\curveto(129.36582031,36.55546875)(129.72714844,36.42070312)(130.14902344,36.42070312)
\curveto(130.65683594,36.42070312)(131.08652344,36.61210937)(131.43808594,36.99492187)
\curveto(131.78964844,37.37773437)(131.96542969,37.88554687)(131.96542969,38.51835937)
\curveto(131.96542969,39.11992187)(131.79550781,39.59453125)(131.45566406,39.9421875)
\curveto(131.11972656,40.28984375)(130.67832031,40.46367187)(130.13144531,40.46367187)
\curveto(129.79160156,40.46367187)(129.48496094,40.38554687)(129.21152344,40.22929687)
\curveto(128.93808594,40.07695312)(128.72324219,39.87773437)(128.56699219,39.63164062)
\lineto(127.57675781,39.76054687)
\lineto(128.40878906,44.17265625)
\lineto(132.68027344,44.17265625)
\lineto(132.68027344,43.16484375)
\lineto(129.25253906,43.16484375)
\lineto(128.78964844,40.85625)
\curveto(129.30527344,41.215625)(129.84628906,41.3953125)(130.41269531,41.3953125)
\curveto(131.16269531,41.3953125)(131.79550781,41.13554687)(132.31113281,40.61601562)
\curveto(132.82675781,40.09648437)(133.08457031,39.42851562)(133.08457031,38.61210937)
\curveto(133.08457031,37.83476562)(132.85800781,37.16289062)(132.40488281,36.59648437)
\curveto(131.85410156,35.90117187)(131.10214844,35.55351562)(130.14902344,35.55351562)
\curveto(129.36777344,35.55351562)(128.72910156,35.77226562)(128.23300781,36.20976562)
\curveto(127.74082031,36.64726562)(127.45957031,37.22734375)(127.38925781,37.95)
\closepath
}
}
{
\newrgbcolor{curcolor}{0 0 0}
\pscustom[linestyle=none,fillstyle=solid,fillcolor=curcolor]
{
\newpath
\moveto(134.06308594,39.93632812)
\curveto(134.06308594,40.95195312)(134.16660156,41.76835937)(134.37363281,42.38554687)
\curveto(134.58457031,43.00664062)(134.89511719,43.48515625)(135.30527344,43.82109375)
\curveto(135.71933594,44.15703125)(136.23886719,44.325)(136.86386719,44.325)
\curveto(137.32480469,44.325)(137.72910156,44.23125)(138.07675781,44.04375)
\curveto(138.42441406,43.86015625)(138.71152344,43.59257812)(138.93808594,43.24101562)
\curveto(139.16464844,42.89335937)(139.34238281,42.46757812)(139.47128906,41.96367187)
\curveto(139.60019531,41.46367187)(139.66464844,40.78789062)(139.66464844,39.93632812)
\curveto(139.66464844,38.92851562)(139.56113281,38.1140625)(139.35410156,37.49296875)
\curveto(139.14707031,36.87578125)(138.83652344,36.39726562)(138.42246094,36.05742187)
\curveto(138.01230469,35.72148437)(137.49277344,35.55351562)(136.86386719,35.55351562)
\curveto(136.03574219,35.55351562)(135.38535156,35.85039062)(134.91269531,36.44414062)
\curveto(134.34628906,37.15898437)(134.06308594,38.32304687)(134.06308594,39.93632812)
\closepath
\moveto(135.14707031,39.93632812)
\curveto(135.14707031,38.52617187)(135.31113281,37.58671875)(135.63925781,37.11796875)
\curveto(135.97128906,36.653125)(136.37949219,36.42070312)(136.86386719,36.42070312)
\curveto(137.34824219,36.42070312)(137.75449219,36.65507812)(138.08261719,37.12382812)
\curveto(138.41464844,37.59257812)(138.58066406,38.53007812)(138.58066406,39.93632812)
\curveto(138.58066406,41.35039062)(138.41464844,42.28984375)(138.08261719,42.7546875)
\curveto(137.75449219,43.21953125)(137.34433594,43.45195312)(136.85214844,43.45195312)
\curveto(136.36777344,43.45195312)(135.98105469,43.246875)(135.69199219,42.83671875)
\curveto(135.32871094,42.31328125)(135.14707031,41.34648437)(135.14707031,39.93632812)
\closepath
}
}
{
\newrgbcolor{curcolor}{0 0 0}
\pscustom[linestyle=none,fillstyle=solid,fillcolor=curcolor]
{
\newpath
\moveto(140.73691406,39.93632812)
\curveto(140.73691406,40.95195312)(140.84042969,41.76835937)(141.04746094,42.38554687)
\curveto(141.25839844,43.00664062)(141.56894531,43.48515625)(141.97910156,43.82109375)
\curveto(142.39316406,44.15703125)(142.91269531,44.325)(143.53769531,44.325)
\curveto(143.99863281,44.325)(144.40292969,44.23125)(144.75058594,44.04375)
\curveto(145.09824219,43.86015625)(145.38535156,43.59257812)(145.61191406,43.24101562)
\curveto(145.83847656,42.89335937)(146.01621094,42.46757812)(146.14511719,41.96367187)
\curveto(146.27402344,41.46367187)(146.33847656,40.78789062)(146.33847656,39.93632812)
\curveto(146.33847656,38.92851562)(146.23496094,38.1140625)(146.02792969,37.49296875)
\curveto(145.82089844,36.87578125)(145.51035156,36.39726562)(145.09628906,36.05742187)
\curveto(144.68613281,35.72148437)(144.16660156,35.55351562)(143.53769531,35.55351562)
\curveto(142.70957031,35.55351562)(142.05917969,35.85039062)(141.58652344,36.44414062)
\curveto(141.02011719,37.15898437)(140.73691406,38.32304687)(140.73691406,39.93632812)
\closepath
\moveto(141.82089844,39.93632812)
\curveto(141.82089844,38.52617187)(141.98496094,37.58671875)(142.31308594,37.11796875)
\curveto(142.64511719,36.653125)(143.05332031,36.42070312)(143.53769531,36.42070312)
\curveto(144.02207031,36.42070312)(144.42832031,36.65507812)(144.75644531,37.12382812)
\curveto(145.08847656,37.59257812)(145.25449219,38.53007812)(145.25449219,39.93632812)
\curveto(145.25449219,41.35039062)(145.08847656,42.28984375)(144.75644531,42.7546875)
\curveto(144.42832031,43.21953125)(144.01816406,43.45195312)(143.52597656,43.45195312)
\curveto(143.04160156,43.45195312)(142.65488281,43.246875)(142.36582031,42.83671875)
\curveto(142.00253906,42.31328125)(141.82089844,41.34648437)(141.82089844,39.93632812)
\closepath
}
}
{
\newrgbcolor{curcolor}{0 0 0}
\pscustom[linestyle=none,fillstyle=solid,fillcolor=curcolor]
{
\newpath
\moveto(147.80332031,35.7)
\lineto(147.80332031,44.28984375)
\lineto(149.51425781,44.28984375)
\lineto(151.54746094,38.2078125)
\curveto(151.73496094,37.64140625)(151.87167969,37.21757812)(151.95761719,36.93632812)
\curveto(152.05527344,37.24882812)(152.20761719,37.7078125)(152.41464844,38.31328125)
\lineto(154.47128906,44.28984375)
\lineto(156.00058594,44.28984375)
\lineto(156.00058594,35.7)
\lineto(154.90488281,35.7)
\lineto(154.90488281,42.88945312)
\lineto(152.40878906,35.7)
\lineto(151.38339844,35.7)
\lineto(148.89902344,43.0125)
\lineto(148.89902344,35.7)
\closepath
}
}
{
\newrgbcolor{curcolor}{0 0 0}
\pscustom[linewidth=1,linecolor=curcolor]
{
\newpath
\moveto(228.5,57.6)
\lineto(228.5,66.6)
\moveto(228.5,425.9)
\lineto(228.5,416.9)
}
}
{
\newrgbcolor{curcolor}{0 0 0}
\pscustom[linestyle=none,fillstyle=solid,fillcolor=curcolor]
{
\newpath
\moveto(224.96679688,35.7)
\lineto(223.91210938,35.7)
\lineto(223.91210938,42.42070312)
\curveto(223.65820312,42.17851562)(223.32421875,41.93632812)(222.91015625,41.69414062)
\curveto(222.5,41.45195312)(222.13085938,41.2703125)(221.80273438,41.14921875)
\lineto(221.80273438,42.16875)
\curveto(222.39257812,42.44609375)(222.90820312,42.78203125)(223.34960938,43.1765625)
\curveto(223.79101562,43.57109375)(224.10351562,43.95390625)(224.28710938,44.325)
\lineto(224.96679688,44.325)
\closepath
}
}
{
\newrgbcolor{curcolor}{0 0 0}
\pscustom[linestyle=none,fillstyle=solid,fillcolor=curcolor]
{
\newpath
\moveto(232.11523438,39.06914062)
\lineto(232.11523438,40.07695312)
\lineto(235.75390625,40.0828125)
\lineto(235.75390625,36.8953125)
\curveto(235.1953125,36.45)(234.61914062,36.1140625)(234.02539062,35.8875)
\curveto(233.43164062,35.66484375)(232.82226562,35.55351562)(232.19726562,35.55351562)
\curveto(231.35351562,35.55351562)(230.5859375,35.73320312)(229.89453125,36.09257812)
\curveto(229.20703125,36.45585937)(228.6875,36.97929687)(228.3359375,37.66289062)
\curveto(227.984375,38.34648437)(227.80859375,39.11015625)(227.80859375,39.95390625)
\curveto(227.80859375,40.78984375)(227.98242188,41.56914062)(228.33007812,42.29179687)
\curveto(228.68164062,43.01835937)(229.18554688,43.55742187)(229.84179688,43.90898437)
\curveto(230.49804688,44.26054687)(231.25390625,44.43632812)(232.109375,44.43632812)
\curveto(232.73046875,44.43632812)(233.29101562,44.33476562)(233.79101562,44.13164062)
\curveto(234.29492188,43.93242187)(234.68945312,43.653125)(234.97460938,43.29375)
\curveto(235.25976562,42.934375)(235.4765625,42.465625)(235.625,41.8875)
\lineto(234.59960938,41.60625)
\curveto(234.47070312,42.04375)(234.31054688,42.3875)(234.11914062,42.6375)
\curveto(233.92773438,42.8875)(233.65429688,43.08671875)(233.29882812,43.23515625)
\curveto(232.94335938,43.3875)(232.54882812,43.46367187)(232.11523438,43.46367187)
\curveto(231.59570312,43.46367187)(231.14648438,43.38359375)(230.76757812,43.2234375)
\curveto(230.38867188,43.0671875)(230.08203125,42.86015625)(229.84765625,42.60234375)
\curveto(229.6171875,42.34453125)(229.4375,42.06132812)(229.30859375,41.75273437)
\curveto(229.08984375,41.22148437)(228.98046875,40.6453125)(228.98046875,40.02421875)
\curveto(228.98046875,39.25859375)(229.11132812,38.61796875)(229.37304688,38.10234375)
\curveto(229.63867188,37.58671875)(230.0234375,37.20390625)(230.52734375,36.95390625)
\curveto(231.03125,36.70390625)(231.56640625,36.57890625)(232.1328125,36.57890625)
\curveto(232.625,36.57890625)(233.10546875,36.67265625)(233.57421875,36.86015625)
\curveto(234.04296875,37.0515625)(234.3984375,37.2546875)(234.640625,37.46953125)
\lineto(234.640625,39.06914062)
\closepath
}
}
{
\newrgbcolor{curcolor}{0 0 0}
\pscustom[linewidth=1,linecolor=curcolor]
{
\newpath
\moveto(315.2,57.6)
\lineto(315.2,66.6)
\moveto(315.2,425.9)
\lineto(315.2,416.9)
}
}
{
\newrgbcolor{curcolor}{0 0 0}
\pscustom[linestyle=none,fillstyle=solid,fillcolor=curcolor]
{
\newpath
\moveto(313.23710937,36.71367187)
\lineto(313.23710937,35.7)
\lineto(307.559375,35.7)
\curveto(307.5515625,35.95390625)(307.59257812,36.19804687)(307.68242187,36.43242187)
\curveto(307.82695312,36.81914062)(308.05742187,37.2)(308.37382812,37.575)
\curveto(308.69414062,37.95)(309.15507812,38.38359375)(309.75664062,38.87578125)
\curveto(310.69023437,39.64140625)(311.32109375,40.246875)(311.64921875,40.6921875)
\curveto(311.97734375,41.14140625)(312.14140625,41.56523437)(312.14140625,41.96367187)
\curveto(312.14140625,42.38164062)(311.99101562,42.73320312)(311.69023437,43.01835937)
\curveto(311.39335937,43.30742187)(311.0046875,43.45195312)(310.52421875,43.45195312)
\curveto(310.01640625,43.45195312)(309.61015625,43.29960937)(309.30546875,42.99492187)
\curveto(309.00078125,42.69023437)(308.84648437,42.26835937)(308.84257812,41.72929687)
\lineto(307.75859375,41.840625)
\curveto(307.8328125,42.64921875)(308.11210937,43.26445312)(308.59648437,43.68632812)
\curveto(309.08085937,44.11210937)(309.73125,44.325)(310.54765625,44.325)
\curveto(311.371875,44.325)(312.02421875,44.09648437)(312.5046875,43.63945312)
\curveto(312.98515625,43.18242187)(313.22539062,42.61601562)(313.22539062,41.94023437)
\curveto(313.22539062,41.59648437)(313.15507812,41.25859375)(313.01445312,40.9265625)
\curveto(312.87382812,40.59453125)(312.63945312,40.24492187)(312.31132812,39.87773437)
\curveto(311.98710937,39.51054687)(311.44609375,39.00664062)(310.68828125,38.36601562)
\curveto(310.05546875,37.83476562)(309.64921875,37.4734375)(309.46953125,37.28203125)
\curveto(309.28984375,37.09453125)(309.14140625,36.90507812)(309.02421875,36.71367187)
\closepath
}
}
{
\newrgbcolor{curcolor}{0 0 0}
\pscustom[linestyle=none,fillstyle=solid,fillcolor=curcolor]
{
\newpath
\moveto(318.81523437,39.06914062)
\lineto(318.81523437,40.07695312)
\lineto(322.45390625,40.0828125)
\lineto(322.45390625,36.8953125)
\curveto(321.8953125,36.45)(321.31914062,36.1140625)(320.72539062,35.8875)
\curveto(320.13164062,35.66484375)(319.52226562,35.55351562)(318.89726562,35.55351562)
\curveto(318.05351562,35.55351562)(317.2859375,35.73320312)(316.59453125,36.09257812)
\curveto(315.90703125,36.45585937)(315.3875,36.97929687)(315.0359375,37.66289062)
\curveto(314.684375,38.34648437)(314.50859375,39.11015625)(314.50859375,39.95390625)
\curveto(314.50859375,40.78984375)(314.68242187,41.56914062)(315.03007812,42.29179687)
\curveto(315.38164062,43.01835937)(315.88554687,43.55742187)(316.54179687,43.90898437)
\curveto(317.19804687,44.26054687)(317.95390625,44.43632812)(318.809375,44.43632812)
\curveto(319.43046875,44.43632812)(319.99101562,44.33476562)(320.49101562,44.13164062)
\curveto(320.99492187,43.93242187)(321.38945312,43.653125)(321.67460937,43.29375)
\curveto(321.95976562,42.934375)(322.1765625,42.465625)(322.325,41.8875)
\lineto(321.29960937,41.60625)
\curveto(321.17070312,42.04375)(321.01054687,42.3875)(320.81914062,42.6375)
\curveto(320.62773437,42.8875)(320.35429687,43.08671875)(319.99882812,43.23515625)
\curveto(319.64335937,43.3875)(319.24882812,43.46367187)(318.81523437,43.46367187)
\curveto(318.29570312,43.46367187)(317.84648437,43.38359375)(317.46757812,43.2234375)
\curveto(317.08867187,43.0671875)(316.78203125,42.86015625)(316.54765625,42.60234375)
\curveto(316.3171875,42.34453125)(316.1375,42.06132812)(316.00859375,41.75273437)
\curveto(315.78984375,41.22148437)(315.68046875,40.6453125)(315.68046875,40.02421875)
\curveto(315.68046875,39.25859375)(315.81132812,38.61796875)(316.07304687,38.10234375)
\curveto(316.33867187,37.58671875)(316.7234375,37.20390625)(317.22734375,36.95390625)
\curveto(317.73125,36.70390625)(318.26640625,36.57890625)(318.8328125,36.57890625)
\curveto(319.325,36.57890625)(319.80546875,36.67265625)(320.27421875,36.86015625)
\curveto(320.74296875,37.0515625)(321.0984375,37.2546875)(321.340625,37.46953125)
\lineto(321.340625,39.06914062)
\closepath
}
}
{
\newrgbcolor{curcolor}{0 0 0}
\pscustom[linewidth=1,linecolor=curcolor]
{
\newpath
\moveto(401.8,57.6)
\lineto(401.8,66.6)
\moveto(401.8,425.9)
\lineto(401.8,416.9)
}
}
{
\newrgbcolor{curcolor}{0 0 0}
\pscustom[linestyle=none,fillstyle=solid,fillcolor=curcolor]
{
\newpath
\moveto(397.675,35.7)
\lineto(397.675,37.75664062)
\lineto(393.9484375,37.75664062)
\lineto(393.9484375,38.7234375)
\lineto(397.86835938,44.28984375)
\lineto(398.7296875,44.28984375)
\lineto(398.7296875,38.7234375)
\lineto(399.88984375,38.7234375)
\lineto(399.88984375,37.75664062)
\lineto(398.7296875,37.75664062)
\lineto(398.7296875,35.7)
\closepath
\moveto(397.675,38.7234375)
\lineto(397.675,42.59648437)
\lineto(394.98554688,38.7234375)
\closepath
}
}
{
\newrgbcolor{curcolor}{0 0 0}
\pscustom[linestyle=none,fillstyle=solid,fillcolor=curcolor]
{
\newpath
\moveto(405.41523438,39.06914062)
\lineto(405.41523438,40.07695312)
\lineto(409.05390625,40.0828125)
\lineto(409.05390625,36.8953125)
\curveto(408.4953125,36.45)(407.91914063,36.1140625)(407.32539063,35.8875)
\curveto(406.73164063,35.66484375)(406.12226563,35.55351562)(405.49726563,35.55351562)
\curveto(404.65351563,35.55351562)(403.8859375,35.73320312)(403.19453125,36.09257812)
\curveto(402.50703125,36.45585937)(401.9875,36.97929687)(401.6359375,37.66289062)
\curveto(401.284375,38.34648437)(401.10859375,39.11015625)(401.10859375,39.95390625)
\curveto(401.10859375,40.78984375)(401.28242188,41.56914062)(401.63007813,42.29179687)
\curveto(401.98164063,43.01835937)(402.48554688,43.55742187)(403.14179688,43.90898437)
\curveto(403.79804688,44.26054687)(404.55390625,44.43632812)(405.409375,44.43632812)
\curveto(406.03046875,44.43632812)(406.59101563,44.33476562)(407.09101563,44.13164062)
\curveto(407.59492188,43.93242187)(407.98945313,43.653125)(408.27460938,43.29375)
\curveto(408.55976563,42.934375)(408.7765625,42.465625)(408.925,41.8875)
\lineto(407.89960938,41.60625)
\curveto(407.77070313,42.04375)(407.61054688,42.3875)(407.41914063,42.6375)
\curveto(407.22773438,42.8875)(406.95429688,43.08671875)(406.59882813,43.23515625)
\curveto(406.24335938,43.3875)(405.84882813,43.46367187)(405.41523438,43.46367187)
\curveto(404.89570313,43.46367187)(404.44648438,43.38359375)(404.06757813,43.2234375)
\curveto(403.68867188,43.0671875)(403.38203125,42.86015625)(403.14765625,42.60234375)
\curveto(402.9171875,42.34453125)(402.7375,42.06132812)(402.60859375,41.75273437)
\curveto(402.38984375,41.22148437)(402.28046875,40.6453125)(402.28046875,40.02421875)
\curveto(402.28046875,39.25859375)(402.41132813,38.61796875)(402.67304688,38.10234375)
\curveto(402.93867188,37.58671875)(403.3234375,37.20390625)(403.82734375,36.95390625)
\curveto(404.33125,36.70390625)(404.86640625,36.57890625)(405.4328125,36.57890625)
\curveto(405.925,36.57890625)(406.40546875,36.67265625)(406.87421875,36.86015625)
\curveto(407.34296875,37.0515625)(407.6984375,37.2546875)(407.940625,37.46953125)
\lineto(407.940625,39.06914062)
\closepath
}
}
{
\newrgbcolor{curcolor}{0 0 0}
\pscustom[linewidth=1,linecolor=curcolor]
{
\newpath
\moveto(488.4,57.6)
\lineto(488.4,66.6)
\moveto(488.4,425.9)
\lineto(488.4,416.9)
}
}
{
\newrgbcolor{curcolor}{0 0 0}
\pscustom[linestyle=none,fillstyle=solid,fillcolor=curcolor]
{
\newpath
\moveto(482.5171875,40.35820312)
\curveto(482.0796875,40.51835937)(481.75546875,40.746875)(481.54453125,41.04375)
\curveto(481.33359375,41.340625)(481.228125,41.69609375)(481.228125,42.11015625)
\curveto(481.228125,42.73515625)(481.45273437,43.26054687)(481.90195312,43.68632812)
\curveto(482.35117187,44.11210937)(482.94882812,44.325)(483.69492187,44.325)
\curveto(484.44492187,44.325)(485.0484375,44.10625)(485.50546875,43.66875)
\curveto(485.9625,43.23515625)(486.19101562,42.70585937)(486.19101562,42.08085937)
\curveto(486.19101562,41.68242187)(486.08554687,41.33476562)(485.87460937,41.03789062)
\curveto(485.66757812,40.74492187)(485.35117187,40.51835937)(484.92539062,40.35820312)
\curveto(485.45273437,40.18632812)(485.853125,39.90898437)(486.1265625,39.52617187)
\curveto(486.40390625,39.14335937)(486.54257812,38.68632812)(486.54257812,38.15507812)
\curveto(486.54257812,37.42070312)(486.2828125,36.80351562)(485.76328125,36.30351562)
\curveto(485.24375,35.80351562)(484.56015625,35.55351562)(483.7125,35.55351562)
\curveto(482.86484375,35.55351562)(482.18125,35.80351562)(481.66171875,36.30351562)
\curveto(481.1421875,36.80742187)(480.88242187,37.434375)(480.88242187,38.184375)
\curveto(480.88242187,38.74296875)(481.02304687,39.20976562)(481.30429687,39.58476562)
\curveto(481.58945312,39.96367187)(481.99375,40.22148437)(482.5171875,40.35820312)
\closepath
\moveto(482.30625,42.1453125)
\curveto(482.30625,41.7390625)(482.43710937,41.40703125)(482.69882812,41.14921875)
\curveto(482.96054687,40.89140625)(483.30039062,40.7625)(483.71835937,40.7625)
\curveto(484.12460937,40.7625)(484.45664062,40.88945312)(484.71445312,41.14335937)
\curveto(484.97617187,41.40117187)(485.10703125,41.715625)(485.10703125,42.08671875)
\curveto(485.10703125,42.4734375)(484.97226562,42.79765625)(484.70273437,43.059375)
\curveto(484.43710937,43.325)(484.10507812,43.4578125)(483.70664062,43.4578125)
\curveto(483.30429687,43.4578125)(482.9703125,43.32890625)(482.7046875,43.07109375)
\curveto(482.4390625,42.81328125)(482.30625,42.5046875)(482.30625,42.1453125)
\closepath
\moveto(481.96640625,38.17851562)
\curveto(481.96640625,37.87773437)(482.03671875,37.58671875)(482.17734375,37.30546875)
\curveto(482.321875,37.02421875)(482.53476562,36.80546875)(482.81601562,36.64921875)
\curveto(483.09726562,36.496875)(483.4,36.42070312)(483.72421875,36.42070312)
\curveto(484.228125,36.42070312)(484.64414062,36.5828125)(484.97226562,36.90703125)
\curveto(485.30039062,37.23125)(485.46445312,37.64335937)(485.46445312,38.14335937)
\curveto(485.46445312,38.65117187)(485.29453125,39.07109375)(484.9546875,39.403125)
\curveto(484.61875,39.73515625)(484.196875,39.90117187)(483.6890625,39.90117187)
\curveto(483.19296875,39.90117187)(482.78085937,39.73710937)(482.45273437,39.40898437)
\curveto(482.12851562,39.08085937)(481.96640625,38.67070312)(481.96640625,38.17851562)
\closepath
}
}
{
\newrgbcolor{curcolor}{0 0 0}
\pscustom[linestyle=none,fillstyle=solid,fillcolor=curcolor]
{
\newpath
\moveto(492.01523437,39.06914062)
\lineto(492.01523437,40.07695312)
\lineto(495.65390625,40.0828125)
\lineto(495.65390625,36.8953125)
\curveto(495.0953125,36.45)(494.51914062,36.1140625)(493.92539062,35.8875)
\curveto(493.33164062,35.66484375)(492.72226562,35.55351562)(492.09726562,35.55351562)
\curveto(491.25351562,35.55351562)(490.4859375,35.73320312)(489.79453125,36.09257812)
\curveto(489.10703125,36.45585937)(488.5875,36.97929687)(488.2359375,37.66289062)
\curveto(487.884375,38.34648437)(487.70859375,39.11015625)(487.70859375,39.95390625)
\curveto(487.70859375,40.78984375)(487.88242187,41.56914062)(488.23007812,42.29179687)
\curveto(488.58164062,43.01835937)(489.08554687,43.55742187)(489.74179687,43.90898437)
\curveto(490.39804687,44.26054687)(491.15390625,44.43632812)(492.009375,44.43632812)
\curveto(492.63046875,44.43632812)(493.19101562,44.33476562)(493.69101562,44.13164062)
\curveto(494.19492187,43.93242187)(494.58945312,43.653125)(494.87460937,43.29375)
\curveto(495.15976562,42.934375)(495.3765625,42.465625)(495.525,41.8875)
\lineto(494.49960937,41.60625)
\curveto(494.37070312,42.04375)(494.21054687,42.3875)(494.01914062,42.6375)
\curveto(493.82773437,42.8875)(493.55429687,43.08671875)(493.19882812,43.23515625)
\curveto(492.84335937,43.3875)(492.44882812,43.46367187)(492.01523437,43.46367187)
\curveto(491.49570312,43.46367187)(491.04648437,43.38359375)(490.66757812,43.2234375)
\curveto(490.28867187,43.0671875)(489.98203125,42.86015625)(489.74765625,42.60234375)
\curveto(489.5171875,42.34453125)(489.3375,42.06132812)(489.20859375,41.75273437)
\curveto(488.98984375,41.22148437)(488.88046875,40.6453125)(488.88046875,40.02421875)
\curveto(488.88046875,39.25859375)(489.01132812,38.61796875)(489.27304687,38.10234375)
\curveto(489.53867187,37.58671875)(489.9234375,37.20390625)(490.42734375,36.95390625)
\curveto(490.93125,36.70390625)(491.46640625,36.57890625)(492.0328125,36.57890625)
\curveto(492.525,36.57890625)(493.00546875,36.67265625)(493.47421875,36.86015625)
\curveto(493.94296875,37.0515625)(494.2984375,37.2546875)(494.540625,37.46953125)
\lineto(494.540625,39.06914062)
\closepath
}
}
{
\newrgbcolor{curcolor}{0 0 0}
\pscustom[linewidth=1,linecolor=curcolor]
{
\newpath
\moveto(575,57.6)
\lineto(575,66.6)
\moveto(575,425.9)
\lineto(575,416.9)
}
}
{
\newrgbcolor{curcolor}{0 0 0}
\pscustom[linestyle=none,fillstyle=solid,fillcolor=curcolor]
{
\newpath
\moveto(568.12988281,35.7)
\lineto(567.07519531,35.7)
\lineto(567.07519531,42.42070312)
\curveto(566.82128906,42.17851562)(566.48730469,41.93632812)(566.07324219,41.69414062)
\curveto(565.66308594,41.45195312)(565.29394531,41.2703125)(564.96582031,41.14921875)
\lineto(564.96582031,42.16875)
\curveto(565.55566406,42.44609375)(566.07128906,42.78203125)(566.51269531,43.1765625)
\curveto(566.95410156,43.57109375)(567.26660156,43.95390625)(567.45019531,44.325)
\lineto(568.12988281,44.325)
\closepath
}
}
{
\newrgbcolor{curcolor}{0 0 0}
\pscustom[linestyle=none,fillstyle=solid,fillcolor=curcolor]
{
\newpath
\moveto(576.30371094,42.18632812)
\lineto(575.25488281,42.10429687)
\curveto(575.16113281,42.51835937)(575.02832031,42.81914062)(574.85644531,43.00664062)
\curveto(574.57128906,43.30742187)(574.21972656,43.4578125)(573.80175781,43.4578125)
\curveto(573.46582031,43.4578125)(573.17089844,43.3640625)(572.91699219,43.1765625)
\curveto(572.58496094,42.934375)(572.32324219,42.58085937)(572.13183594,42.11601562)
\curveto(571.94042969,41.65117187)(571.84082031,40.9890625)(571.83300781,40.1296875)
\curveto(572.08691406,40.51640625)(572.39746094,40.80351562)(572.76464844,40.99101562)
\curveto(573.13183594,41.17851562)(573.51660156,41.27226562)(573.91894531,41.27226562)
\curveto(574.62207031,41.27226562)(575.21972656,41.0125)(575.71191406,40.49296875)
\curveto(576.20800781,39.97734375)(576.45605469,39.309375)(576.45605469,38.4890625)
\curveto(576.45605469,37.95)(576.33886719,37.44804687)(576.10449219,36.98320312)
\curveto(575.87402344,36.52226562)(575.55566406,36.16875)(575.14941406,35.92265625)
\curveto(574.74316406,35.6765625)(574.28222656,35.55351562)(573.76660156,35.55351562)
\curveto(572.88769531,35.55351562)(572.17089844,35.87578125)(571.61621094,36.5203125)
\curveto(571.06152344,37.16875)(570.78417969,38.23515625)(570.78417969,39.71953125)
\curveto(570.78417969,41.3796875)(571.09082031,42.58671875)(571.70410156,43.340625)
\curveto(572.23925781,43.996875)(572.95996094,44.325)(573.86621094,44.325)
\curveto(574.54199219,44.325)(575.09472656,44.13554687)(575.52441406,43.75664062)
\curveto(575.95800781,43.37773437)(576.21777344,42.85429687)(576.30371094,42.18632812)
\closepath
\moveto(571.99707031,38.48320312)
\curveto(571.99707031,38.11992187)(572.07324219,37.77226562)(572.22558594,37.44023437)
\curveto(572.38183594,37.10820312)(572.59863281,36.85429687)(572.87597656,36.67851562)
\curveto(573.15332031,36.50664062)(573.44433594,36.42070312)(573.74902344,36.42070312)
\curveto(574.19433594,36.42070312)(574.57714844,36.60039062)(574.89746094,36.95976562)
\curveto(575.21777344,37.31914062)(575.37792969,37.80742187)(575.37792969,38.42460937)
\curveto(575.37792969,39.01835937)(575.21972656,39.48515625)(574.90332031,39.825)
\curveto(574.58691406,40.16875)(574.18847656,40.340625)(573.70800781,40.340625)
\curveto(573.23144531,40.340625)(572.82714844,40.16875)(572.49511719,39.825)
\curveto(572.16308594,39.48515625)(571.99707031,39.03789062)(571.99707031,38.48320312)
\closepath
}
}
{
\newrgbcolor{curcolor}{0 0 0}
\pscustom[linestyle=none,fillstyle=solid,fillcolor=curcolor]
{
\newpath
\moveto(581.95214844,39.06914062)
\lineto(581.95214844,40.07695312)
\lineto(585.59082031,40.0828125)
\lineto(585.59082031,36.8953125)
\curveto(585.03222656,36.45)(584.45605469,36.1140625)(583.86230469,35.8875)
\curveto(583.26855469,35.66484375)(582.65917969,35.55351562)(582.03417969,35.55351562)
\curveto(581.19042969,35.55351562)(580.42285156,35.73320312)(579.73144531,36.09257812)
\curveto(579.04394531,36.45585937)(578.52441406,36.97929687)(578.17285156,37.66289062)
\curveto(577.82128906,38.34648437)(577.64550781,39.11015625)(577.64550781,39.95390625)
\curveto(577.64550781,40.78984375)(577.81933594,41.56914062)(578.16699219,42.29179687)
\curveto(578.51855469,43.01835937)(579.02246094,43.55742187)(579.67871094,43.90898437)
\curveto(580.33496094,44.26054687)(581.09082031,44.43632812)(581.94628906,44.43632812)
\curveto(582.56738281,44.43632812)(583.12792969,44.33476562)(583.62792969,44.13164062)
\curveto(584.13183594,43.93242187)(584.52636719,43.653125)(584.81152344,43.29375)
\curveto(585.09667969,42.934375)(585.31347656,42.465625)(585.46191406,41.8875)
\lineto(584.43652344,41.60625)
\curveto(584.30761719,42.04375)(584.14746094,42.3875)(583.95605469,42.6375)
\curveto(583.76464844,42.8875)(583.49121094,43.08671875)(583.13574219,43.23515625)
\curveto(582.78027344,43.3875)(582.38574219,43.46367187)(581.95214844,43.46367187)
\curveto(581.43261719,43.46367187)(580.98339844,43.38359375)(580.60449219,43.2234375)
\curveto(580.22558594,43.0671875)(579.91894531,42.86015625)(579.68457031,42.60234375)
\curveto(579.45410156,42.34453125)(579.27441406,42.06132812)(579.14550781,41.75273437)
\curveto(578.92675781,41.22148437)(578.81738281,40.6453125)(578.81738281,40.02421875)
\curveto(578.81738281,39.25859375)(578.94824219,38.61796875)(579.20996094,38.10234375)
\curveto(579.47558594,37.58671875)(579.86035156,37.20390625)(580.36425781,36.95390625)
\curveto(580.86816406,36.70390625)(581.40332031,36.57890625)(581.96972656,36.57890625)
\curveto(582.46191406,36.57890625)(582.94238281,36.67265625)(583.41113281,36.86015625)
\curveto(583.87988281,37.0515625)(584.23535156,37.2546875)(584.47753906,37.46953125)
\lineto(584.47753906,39.06914062)
\closepath
}
}
{
\newrgbcolor{curcolor}{0 0 0}
\pscustom[linewidth=1,linecolor=curcolor]
{
\newpath
\moveto(55.3,425.9)
\lineto(55.3,57.6)
\lineto(575,57.6)
\lineto(575,425.9)
\closepath
}
}
{
\newrgbcolor{curcolor}{0 0 0}
\pscustom[linestyle=none,fillstyle=solid,fillcolor=curcolor]
{
\newpath
\moveto(16.3,195.56035156)
\lineto(7.71015625,195.56035156)
\lineto(7.71015625,198.78300781)
\curveto(7.71015625,199.43925781)(7.79804688,199.96464844)(7.97382813,200.35917969)
\curveto(8.14570313,200.75761719)(8.41328125,201.06816406)(8.7765625,201.29082031)
\curveto(9.1359375,201.51738281)(9.51289063,201.63066406)(9.90742188,201.63066406)
\curveto(10.27460938,201.63066406)(10.6203125,201.53105469)(10.94453125,201.33183594)
\curveto(11.26875,201.13261719)(11.53046875,200.83183594)(11.7296875,200.42949219)
\curveto(11.88203125,200.94902344)(12.14179688,201.34746094)(12.50898438,201.62480469)
\curveto(12.87617188,201.90605469)(13.30976563,202.04667969)(13.80976563,202.04667969)
\curveto(14.21210938,202.04667969)(14.58710938,201.96074219)(14.93476563,201.78886719)
\curveto(15.27851563,201.62089844)(15.54414063,201.41191406)(15.73164063,201.16191406)
\curveto(15.91914063,200.91191406)(16.06171875,200.59746094)(16.159375,200.21855469)
\curveto(16.253125,199.84355469)(16.3,199.38261719)(16.3,198.83574219)
\closepath
\moveto(11.31953125,196.69707031)
\lineto(11.31953125,198.55449219)
\curveto(11.31953125,199.05839844)(11.28632813,199.41972656)(11.21992188,199.63847656)
\curveto(11.13398438,199.92753906)(10.99140625,200.14433594)(10.7921875,200.28886719)
\curveto(10.59296875,200.43730469)(10.34296875,200.51152344)(10.0421875,200.51152344)
\curveto(9.75703125,200.51152344)(9.50703125,200.44316406)(9.2921875,200.30644531)
\curveto(9.0734375,200.16972656)(8.925,199.97441406)(8.846875,199.72050781)
\curveto(8.76484375,199.46660156)(8.72382813,199.03105469)(8.72382813,198.41386719)
\lineto(8.72382813,196.69707031)
\closepath
\moveto(15.28632813,196.69707031)
\lineto(15.28632813,198.83574219)
\curveto(15.28632813,199.20292969)(15.27265625,199.46074219)(15.2453125,199.60917969)
\curveto(15.1984375,199.87089844)(15.1203125,200.08964844)(15.0109375,200.26542969)
\curveto(14.9015625,200.44121094)(14.74335938,200.58574219)(14.53632813,200.69902344)
\curveto(14.32539063,200.81230469)(14.08320313,200.86894531)(13.80976563,200.86894531)
\curveto(13.48945313,200.86894531)(13.21210938,200.78691406)(12.97773438,200.62285156)
\curveto(12.73945313,200.45878906)(12.5734375,200.23027344)(12.4796875,199.93730469)
\curveto(12.38203125,199.64824219)(12.33320313,199.23027344)(12.33320313,198.68339844)
\lineto(12.33320313,196.69707031)
\closepath
}
}
{
\newrgbcolor{curcolor}{0 0 0}
\pscustom[linestyle=none,fillstyle=solid,fillcolor=curcolor]
{
\newpath
\moveto(15.53242188,207.53691406)
\curveto(15.86445313,207.14628906)(16.09882813,206.76933594)(16.23554688,206.40605469)
\curveto(16.37226563,206.04667969)(16.440625,205.65996094)(16.440625,205.24589844)
\curveto(16.440625,204.56230469)(16.27460938,204.03691406)(15.94257813,203.66972656)
\curveto(15.60664063,203.30253906)(15.17890625,203.11894531)(14.659375,203.11894531)
\curveto(14.3546875,203.11894531)(14.07734375,203.18730469)(13.82734375,203.32402344)
\curveto(13.5734375,203.46464844)(13.3703125,203.64628906)(13.21796875,203.86894531)
\curveto(13.065625,204.09550781)(12.95039063,204.34941406)(12.87226563,204.63066406)
\curveto(12.81757813,204.83769531)(12.76484375,205.15019531)(12.7140625,205.56816406)
\curveto(12.6125,206.41972656)(12.49140625,207.04667969)(12.35078125,207.44902344)
\curveto(12.20625,207.45292969)(12.11445313,207.45488281)(12.07539063,207.45488281)
\curveto(11.64570313,207.45488281)(11.34296875,207.35527344)(11.1671875,207.15605469)
\curveto(10.92890625,206.88652344)(10.80976563,206.48613281)(10.80976563,205.95488281)
\curveto(10.80976563,205.45878906)(10.89765625,205.09160156)(11.0734375,204.85332031)
\curveto(11.2453125,204.61894531)(11.55195313,204.44511719)(11.99335938,204.33183594)
\lineto(11.85273438,203.30058594)
\curveto(11.41132813,203.39433594)(11.05585938,203.54863281)(10.78632813,203.76347656)
\curveto(10.51289063,203.97832031)(10.30390625,204.28886719)(10.159375,204.69511719)
\curveto(10.0109375,205.10136719)(9.93671875,205.57207031)(9.93671875,206.10722656)
\curveto(9.93671875,206.63847656)(9.99921875,207.07011719)(10.12421875,207.40214844)
\curveto(10.24921875,207.73417969)(10.40742188,207.97832031)(10.59882813,208.13457031)
\curveto(10.78632813,208.29082031)(11.02460938,208.40019531)(11.31367188,208.46269531)
\curveto(11.49335938,208.49785156)(11.81757813,208.51542969)(12.28632813,208.51542969)
\lineto(13.69257813,208.51542969)
\curveto(14.67304688,208.51542969)(15.29414063,208.53691406)(15.55585938,208.57988281)
\curveto(15.81367188,208.62675781)(16.06171875,208.71660156)(16.3,208.84941406)
\lineto(16.3,207.74785156)
\curveto(16.08125,207.63847656)(15.82539063,207.56816406)(15.53242188,207.53691406)
\closepath
\moveto(13.17695313,207.44902344)
\curveto(13.33320313,207.06621094)(13.46601563,206.49199219)(13.57539063,205.72636719)
\curveto(13.63789063,205.29277344)(13.70820313,204.98613281)(13.78632813,204.80644531)
\curveto(13.86445313,204.62675781)(13.9796875,204.48808594)(14.13203125,204.39042969)
\curveto(14.28046875,204.29277344)(14.44648438,204.24394531)(14.63007813,204.24394531)
\curveto(14.91132813,204.24394531)(15.14570313,204.34941406)(15.33320313,204.56035156)
\curveto(15.52070313,204.77519531)(15.61445313,205.08769531)(15.61445313,205.49785156)
\curveto(15.61445313,205.90410156)(15.5265625,206.26542969)(15.35078125,206.58183594)
\curveto(15.17109375,206.89824219)(14.92695313,207.13066406)(14.61835938,207.27910156)
\curveto(14.38007813,207.39238281)(14.02851563,207.44902344)(13.56367188,207.44902344)
\closepath
}
}
{
\newrgbcolor{curcolor}{0 0 0}
\pscustom[linestyle=none,fillstyle=solid,fillcolor=curcolor]
{
\newpath
\moveto(16.3,210.15019531)
\lineto(10.07734375,210.15019531)
\lineto(10.07734375,211.09941406)
\lineto(10.96210938,211.09941406)
\curveto(10.27851563,211.55644531)(9.93671875,212.21660156)(9.93671875,213.07988281)
\curveto(9.93671875,213.45488281)(10.00507813,213.79863281)(10.14179688,214.11113281)
\curveto(10.27460938,214.42753906)(10.45039063,214.66386719)(10.66914063,214.82011719)
\curveto(10.88789063,214.97636719)(11.14765625,215.08574219)(11.4484375,215.14824219)
\curveto(11.64375,215.18730469)(11.98554688,215.20683594)(12.47382813,215.20683594)
\lineto(16.3,215.20683594)
\lineto(16.3,214.15214844)
\lineto(12.51484375,214.15214844)
\curveto(12.08515625,214.15214844)(11.76484375,214.11113281)(11.55390625,214.02910156)
\curveto(11.3390625,213.94707031)(11.16914063,213.80058594)(11.04414063,213.58964844)
\curveto(10.91523438,213.38261719)(10.85078125,213.13847656)(10.85078125,212.85722656)
\curveto(10.85078125,212.40800781)(10.99335938,212.01933594)(11.27851563,211.69121094)
\curveto(11.56367188,211.36699219)(12.1046875,211.20488281)(12.9015625,211.20488281)
\lineto(16.3,211.20488281)
\closepath
}
}
{
\newrgbcolor{curcolor}{0 0 0}
\pscustom[linestyle=none,fillstyle=solid,fillcolor=curcolor]
{
\newpath
\moveto(16.3,220.86113281)
\lineto(15.51484375,220.86113281)
\curveto(16.13203125,220.46660156)(16.440625,219.88652344)(16.440625,219.12089844)
\curveto(16.440625,218.62480469)(16.30390625,218.16777344)(16.03046875,217.74980469)
\curveto(15.75703125,217.33574219)(15.37617188,217.01347656)(14.88789063,216.78300781)
\curveto(14.39570313,216.55644531)(13.83125,216.44316406)(13.19453125,216.44316406)
\curveto(12.5734375,216.44316406)(12.0109375,216.54667969)(11.50703125,216.75371094)
\curveto(10.99921875,216.96074219)(10.61054688,217.27128906)(10.34101563,217.68535156)
\curveto(10.07148438,218.09941406)(9.93671875,218.56230469)(9.93671875,219.07402344)
\curveto(9.93671875,219.44902344)(10.01679688,219.78300781)(10.17695313,220.07597656)
\curveto(10.33320313,220.36894531)(10.53828125,220.60722656)(10.7921875,220.79082031)
\lineto(7.71015625,220.79082031)
\lineto(7.71015625,221.83964844)
\lineto(16.3,221.83964844)
\closepath
\moveto(13.19453125,217.52714844)
\curveto(13.99140625,217.52714844)(14.58710938,217.69511719)(14.98164063,218.03105469)
\curveto(15.37617188,218.36699219)(15.5734375,218.76347656)(15.5734375,219.22050781)
\curveto(15.5734375,219.68144531)(15.3859375,220.07207031)(15.0109375,220.39238281)
\curveto(14.63203125,220.71660156)(14.05585938,220.87871094)(13.28242188,220.87871094)
\curveto(12.43085938,220.87871094)(11.80585938,220.71464844)(11.40742188,220.38652344)
\curveto(11.00898438,220.05839844)(10.80976563,219.65410156)(10.80976563,219.17363281)
\curveto(10.80976563,218.70488281)(11.00117188,218.31230469)(11.38398438,217.99589844)
\curveto(11.76679688,217.68339844)(12.3703125,217.52714844)(13.19453125,217.52714844)
\closepath
}
}
{
\newrgbcolor{curcolor}{0 0 0}
\pscustom[linestyle=none,fillstyle=solid,fillcolor=curcolor]
{
\newpath
\moveto(16.3,224.64628906)
\lineto(10.07734375,222.74199219)
\lineto(10.07734375,223.83183594)
\lineto(13.66914063,224.82207031)
\lineto(15.00507813,225.19121094)
\curveto(14.93867188,225.20683594)(14.5109375,225.31425781)(13.721875,225.51347656)
\lineto(10.07734375,226.50371094)
\lineto(10.07734375,227.58769531)
\lineto(13.68671875,228.51933594)
\lineto(14.87617188,228.82988281)
\lineto(13.675,229.18730469)
\lineto(10.07734375,230.25371094)
\lineto(10.07734375,231.27910156)
\lineto(16.3,229.33378906)
\lineto(16.3,228.23808594)
\lineto(12.5734375,227.24785156)
\lineto(11.51289063,227.00761719)
\lineto(16.3,225.74785156)
\closepath
}
}
{
\newrgbcolor{curcolor}{0 0 0}
\pscustom[linestyle=none,fillstyle=solid,fillcolor=curcolor]
{
\newpath
\moveto(8.92304688,232.16972656)
\lineto(7.71015625,232.16972656)
\lineto(7.71015625,233.22441406)
\lineto(8.92304688,233.22441406)
\closepath
\moveto(16.3,232.16972656)
\lineto(10.07734375,232.16972656)
\lineto(10.07734375,233.22441406)
\lineto(16.3,233.22441406)
\closepath
}
}
{
\newrgbcolor{curcolor}{0 0 0}
\pscustom[linestyle=none,fillstyle=solid,fillcolor=curcolor]
{
\newpath
\moveto(16.3,238.86699219)
\lineto(15.51484375,238.86699219)
\curveto(16.13203125,238.47246094)(16.440625,237.89238281)(16.440625,237.12675781)
\curveto(16.440625,236.63066406)(16.30390625,236.17363281)(16.03046875,235.75566406)
\curveto(15.75703125,235.34160156)(15.37617188,235.01933594)(14.88789063,234.78886719)
\curveto(14.39570313,234.56230469)(13.83125,234.44902344)(13.19453125,234.44902344)
\curveto(12.5734375,234.44902344)(12.0109375,234.55253906)(11.50703125,234.75957031)
\curveto(10.99921875,234.96660156)(10.61054688,235.27714844)(10.34101563,235.69121094)
\curveto(10.07148438,236.10527344)(9.93671875,236.56816406)(9.93671875,237.07988281)
\curveto(9.93671875,237.45488281)(10.01679688,237.78886719)(10.17695313,238.08183594)
\curveto(10.33320313,238.37480469)(10.53828125,238.61308594)(10.7921875,238.79667969)
\lineto(7.71015625,238.79667969)
\lineto(7.71015625,239.84550781)
\lineto(16.3,239.84550781)
\closepath
\moveto(13.19453125,235.53300781)
\curveto(13.99140625,235.53300781)(14.58710938,235.70097656)(14.98164063,236.03691406)
\curveto(15.37617188,236.37285156)(15.5734375,236.76933594)(15.5734375,237.22636719)
\curveto(15.5734375,237.68730469)(15.3859375,238.07792969)(15.0109375,238.39824219)
\curveto(14.63203125,238.72246094)(14.05585938,238.88457031)(13.28242188,238.88457031)
\curveto(12.43085938,238.88457031)(11.80585938,238.72050781)(11.40742188,238.39238281)
\curveto(11.00898438,238.06425781)(10.80976563,237.65996094)(10.80976563,237.17949219)
\curveto(10.80976563,236.71074219)(11.00117188,236.31816406)(11.38398438,236.00175781)
\curveto(11.76679688,235.68925781)(12.3703125,235.53300781)(13.19453125,235.53300781)
\closepath
}
}
{
\newrgbcolor{curcolor}{0 0 0}
\pscustom[linestyle=none,fillstyle=solid,fillcolor=curcolor]
{
\newpath
\moveto(15.35664063,243.80644531)
\lineto(16.28828125,243.95878906)
\curveto(16.35078125,243.66191406)(16.38203125,243.39628906)(16.38203125,243.16191406)
\curveto(16.38203125,242.77910156)(16.32148438,242.48222656)(16.20039063,242.27128906)
\curveto(16.07929688,242.06035156)(15.92109375,241.91191406)(15.72578125,241.82597656)
\curveto(15.5265625,241.74003906)(15.11054688,241.69707031)(14.47773438,241.69707031)
\lineto(10.89765625,241.69707031)
\lineto(10.89765625,240.92363281)
\lineto(10.07734375,240.92363281)
\lineto(10.07734375,241.69707031)
\lineto(8.53632813,241.69707031)
\lineto(7.90351563,242.74589844)
\lineto(10.07734375,242.74589844)
\lineto(10.07734375,243.80644531)
\lineto(10.89765625,243.80644531)
\lineto(10.89765625,242.74589844)
\lineto(14.53632813,242.74589844)
\curveto(14.83710938,242.74589844)(15.03046875,242.76347656)(15.11640625,242.79863281)
\curveto(15.20234375,242.83769531)(15.27070313,242.89824219)(15.32148438,242.98027344)
\curveto(15.37226563,243.06621094)(15.39765625,243.18730469)(15.39765625,243.34355469)
\curveto(15.39765625,243.46074219)(15.38398438,243.61503906)(15.35664063,243.80644531)
\closepath
}
}
{
\newrgbcolor{curcolor}{0 0 0}
\pscustom[linestyle=none,fillstyle=solid,fillcolor=curcolor]
{
\newpath
\moveto(16.3,244.83769531)
\lineto(7.71015625,244.83769531)
\lineto(7.71015625,245.89238281)
\lineto(10.7921875,245.89238281)
\curveto(10.221875,246.38457031)(9.93671875,247.00566406)(9.93671875,247.75566406)
\curveto(9.93671875,248.21660156)(10.02851563,248.61699219)(10.21210938,248.95683594)
\curveto(10.39179688,249.29667969)(10.64179688,249.53886719)(10.96210938,249.68339844)
\curveto(11.28242188,249.83183594)(11.74726563,249.90605469)(12.35664063,249.90605469)
\lineto(16.3,249.90605469)
\lineto(16.3,248.85136719)
\lineto(12.35664063,248.85136719)
\curveto(11.82929688,248.85136719)(11.44648438,248.73613281)(11.20820313,248.50566406)
\curveto(10.96601563,248.27910156)(10.84492188,247.95683594)(10.84492188,247.53886719)
\curveto(10.84492188,247.22636719)(10.92695313,246.93144531)(11.09101563,246.65410156)
\curveto(11.25117188,246.38066406)(11.46992188,246.18535156)(11.74726563,246.06816406)
\curveto(12.02460938,245.95097656)(12.40742188,245.89238281)(12.89570313,245.89238281)
\lineto(16.3,245.89238281)
\closepath
}
}
{
\newrgbcolor{curcolor}{0 0 0}
\pscustom[linestyle=none,fillstyle=solid,fillcolor=curcolor]
{
\newpath
\moveto(18.82539063,256.86113281)
\curveto(18.09101563,256.27910156)(17.23164063,255.78691406)(16.24726563,255.38457031)
\curveto(15.26289063,254.98222656)(14.24335938,254.78105469)(13.18867188,254.78105469)
\curveto(12.25898438,254.78105469)(11.36835938,254.93144531)(10.51679688,255.23222656)
\curveto(9.52851563,255.58378906)(8.54414063,256.12675781)(7.56367188,256.86113281)
\lineto(7.56367188,257.61699219)
\curveto(8.37617188,257.14433594)(8.95625,256.83183594)(9.30390625,256.67949219)
\curveto(9.84296875,256.44121094)(10.40546875,256.25371094)(10.99140625,256.11699219)
\curveto(11.721875,255.94902344)(12.45625,255.86503906)(13.19453125,255.86503906)
\curveto(15.0734375,255.86503906)(16.95039063,256.44902344)(18.82539063,257.61699219)
\closepath
}
}
{
\newrgbcolor{curcolor}{0 0 0}
\pscustom[linestyle=none,fillstyle=solid,fillcolor=curcolor]
{
\newpath
\moveto(12.93085938,262.99589844)
\lineto(11.92304688,262.99589844)
\lineto(11.9171875,266.63457031)
\lineto(15.1046875,266.63457031)
\curveto(15.55,266.07597656)(15.8859375,265.49980469)(16.1125,264.90605469)
\curveto(16.33515625,264.31230469)(16.44648438,263.70292969)(16.44648438,263.07792969)
\curveto(16.44648438,262.23417969)(16.26679688,261.46660156)(15.90742188,260.77519531)
\curveto(15.54414063,260.08769531)(15.02070313,259.56816406)(14.33710938,259.21660156)
\curveto(13.65351563,258.86503906)(12.88984375,258.68925781)(12.04609375,258.68925781)
\curveto(11.21015625,258.68925781)(10.43085938,258.86308594)(9.70820313,259.21074219)
\curveto(8.98164063,259.56230469)(8.44257813,260.06621094)(8.09101563,260.72246094)
\curveto(7.73945313,261.37871094)(7.56367188,262.13457031)(7.56367188,262.99003906)
\curveto(7.56367188,263.61113281)(7.66523438,264.17167969)(7.86835938,264.67167969)
\curveto(8.06757813,265.17558594)(8.346875,265.57011719)(8.70625,265.85527344)
\curveto(9.065625,266.14042969)(9.534375,266.35722656)(10.1125,266.50566406)
\lineto(10.39375,265.48027344)
\curveto(9.95625,265.35136719)(9.6125,265.19121094)(9.3625,264.99980469)
\curveto(9.1125,264.80839844)(8.91328125,264.53496094)(8.76484375,264.17949219)
\curveto(8.6125,263.82402344)(8.53632813,263.42949219)(8.53632813,262.99589844)
\curveto(8.53632813,262.47636719)(8.61640625,262.02714844)(8.7765625,261.64824219)
\curveto(8.9328125,261.26933594)(9.13984375,260.96269531)(9.39765625,260.72832031)
\curveto(9.65546875,260.49785156)(9.93867188,260.31816406)(10.24726563,260.18925781)
\curveto(10.77851563,259.97050781)(11.3546875,259.86113281)(11.97578125,259.86113281)
\curveto(12.74140625,259.86113281)(13.38203125,259.99199219)(13.89765625,260.25371094)
\curveto(14.41328125,260.51933594)(14.79609375,260.90410156)(15.04609375,261.40800781)
\curveto(15.29609375,261.91191406)(15.42109375,262.44707031)(15.42109375,263.01347656)
\curveto(15.42109375,263.50566406)(15.32734375,263.98613281)(15.13984375,264.45488281)
\curveto(14.9484375,264.92363281)(14.7453125,265.27910156)(14.53046875,265.52128906)
\lineto(12.93085938,265.52128906)
\closepath
}
}
{
\newrgbcolor{curcolor}{0 0 0}
\pscustom[linestyle=none,fillstyle=solid,fillcolor=curcolor]
{
\newpath
\moveto(16.3,268.26347656)
\lineto(7.71015625,268.26347656)
\lineto(7.71015625,271.48613281)
\curveto(7.71015625,272.14238281)(7.79804688,272.66777344)(7.97382813,273.06230469)
\curveto(8.14570313,273.46074219)(8.41328125,273.77128906)(8.7765625,273.99394531)
\curveto(9.1359375,274.22050781)(9.51289063,274.33378906)(9.90742188,274.33378906)
\curveto(10.27460938,274.33378906)(10.6203125,274.23417969)(10.94453125,274.03496094)
\curveto(11.26875,273.83574219)(11.53046875,273.53496094)(11.7296875,273.13261719)
\curveto(11.88203125,273.65214844)(12.14179688,274.05058594)(12.50898438,274.32792969)
\curveto(12.87617188,274.60917969)(13.30976563,274.74980469)(13.80976563,274.74980469)
\curveto(14.21210938,274.74980469)(14.58710938,274.66386719)(14.93476563,274.49199219)
\curveto(15.27851563,274.32402344)(15.54414063,274.11503906)(15.73164063,273.86503906)
\curveto(15.91914063,273.61503906)(16.06171875,273.30058594)(16.159375,272.92167969)
\curveto(16.253125,272.54667969)(16.3,272.08574219)(16.3,271.53886719)
\closepath
\moveto(11.31953125,269.40019531)
\lineto(11.31953125,271.25761719)
\curveto(11.31953125,271.76152344)(11.28632813,272.12285156)(11.21992188,272.34160156)
\curveto(11.13398438,272.63066406)(10.99140625,272.84746094)(10.7921875,272.99199219)
\curveto(10.59296875,273.14042969)(10.34296875,273.21464844)(10.0421875,273.21464844)
\curveto(9.75703125,273.21464844)(9.50703125,273.14628906)(9.2921875,273.00957031)
\curveto(9.0734375,272.87285156)(8.925,272.67753906)(8.846875,272.42363281)
\curveto(8.76484375,272.16972656)(8.72382813,271.73417969)(8.72382813,271.11699219)
\lineto(8.72382813,269.40019531)
\closepath
\moveto(15.28632813,269.40019531)
\lineto(15.28632813,271.53886719)
\curveto(15.28632813,271.90605469)(15.27265625,272.16386719)(15.2453125,272.31230469)
\curveto(15.1984375,272.57402344)(15.1203125,272.79277344)(15.0109375,272.96855469)
\curveto(14.9015625,273.14433594)(14.74335938,273.28886719)(14.53632813,273.40214844)
\curveto(14.32539063,273.51542969)(14.08320313,273.57207031)(13.80976563,273.57207031)
\curveto(13.48945313,273.57207031)(13.21210938,273.49003906)(12.97773438,273.32597656)
\curveto(12.73945313,273.16191406)(12.5734375,272.93339844)(12.4796875,272.64042969)
\curveto(12.38203125,272.35136719)(12.33320313,271.93339844)(12.33320313,271.38652344)
\lineto(12.33320313,269.40019531)
\closepath
}
}
{
\newrgbcolor{curcolor}{0 0 0}
\pscustom[linestyle=none,fillstyle=solid,fillcolor=curcolor]
{
\newpath
\moveto(16.44648438,275.38847656)
\lineto(7.56367188,277.87871094)
\lineto(7.56367188,278.72246094)
\lineto(16.44648438,276.23808594)
\closepath
}
}
{
\newrgbcolor{curcolor}{0 0 0}
\pscustom[linestyle=none,fillstyle=solid,fillcolor=curcolor]
{
\newpath
\moveto(14.44257813,279.09160156)
\lineto(14.27851563,280.13457031)
\curveto(14.69648438,280.19316406)(15.01679688,280.35527344)(15.23945313,280.62089844)
\curveto(15.46210938,280.89042969)(15.5734375,281.26542969)(15.5734375,281.74589844)
\curveto(15.5734375,282.23027344)(15.47578125,282.58964844)(15.28046875,282.82402344)
\curveto(15.08125,283.05839844)(14.84882813,283.17558594)(14.58320313,283.17558594)
\curveto(14.34492188,283.17558594)(14.15742188,283.07207031)(14.02070313,282.86503906)
\curveto(13.92695313,282.72050781)(13.8078125,282.36113281)(13.66328125,281.78691406)
\curveto(13.46796875,281.01347656)(13.3,280.47636719)(13.159375,280.17558594)
\curveto(13.01484375,279.87871094)(12.81757813,279.65214844)(12.56757813,279.49589844)
\curveto(12.31367188,279.34355469)(12.034375,279.26738281)(11.7296875,279.26738281)
\curveto(11.45234375,279.26738281)(11.19648438,279.32988281)(10.96210938,279.45488281)
\curveto(10.72382813,279.58378906)(10.5265625,279.75761719)(10.3703125,279.97636719)
\curveto(10.24921875,280.14042969)(10.14765625,280.36308594)(10.065625,280.64433594)
\curveto(9.9796875,280.92949219)(9.93671875,281.23417969)(9.93671875,281.55839844)
\curveto(9.93671875,282.04667969)(10.00703125,282.47441406)(10.14765625,282.84160156)
\curveto(10.28828125,283.21269531)(10.4796875,283.48613281)(10.721875,283.66191406)
\curveto(10.96015625,283.83769531)(11.28046875,283.95878906)(11.6828125,284.02519531)
\lineto(11.8234375,282.99394531)
\curveto(11.503125,282.94707031)(11.253125,282.81035156)(11.0734375,282.58378906)
\curveto(10.89375,282.36113281)(10.80390625,282.04472656)(10.80390625,281.63457031)
\curveto(10.80390625,281.15019531)(10.88398438,280.80449219)(11.04414063,280.59746094)
\curveto(11.20429688,280.39042969)(11.39179688,280.28691406)(11.60664063,280.28691406)
\curveto(11.74335938,280.28691406)(11.86640625,280.32988281)(11.97578125,280.41582031)
\curveto(12.0890625,280.50175781)(12.1828125,280.63652344)(12.25703125,280.82011719)
\curveto(12.29609375,280.92558594)(12.3859375,281.23613281)(12.5265625,281.75175781)
\curveto(12.72578125,282.49785156)(12.88984375,283.01738281)(13.01875,283.31035156)
\curveto(13.14375,283.60722656)(13.32734375,283.83964844)(13.56953125,284.00761719)
\curveto(13.81171875,284.17558594)(14.1125,284.25957031)(14.471875,284.25957031)
\curveto(14.8234375,284.25957031)(15.15546875,284.15605469)(15.46796875,283.94902344)
\curveto(15.7765625,283.74589844)(16.01679688,283.45097656)(16.18867188,283.06425781)
\curveto(16.35664063,282.67753906)(16.440625,282.24003906)(16.440625,281.75175781)
\curveto(16.440625,280.94316406)(16.27265625,280.32597656)(15.93671875,279.90019531)
\curveto(15.60078125,279.47832031)(15.10273438,279.20878906)(14.44257813,279.09160156)
\closepath
}
}
{
\newrgbcolor{curcolor}{0 0 0}
\pscustom[linestyle=none,fillstyle=solid,fillcolor=curcolor]
{
\newpath
\moveto(18.82539063,286.20488281)
\lineto(18.82539063,285.44902344)
\curveto(16.95039063,286.61699219)(15.0734375,287.20097656)(13.19453125,287.20097656)
\curveto(12.46015625,287.20097656)(11.73164063,287.11699219)(11.00898438,286.94902344)
\curveto(10.42304688,286.81621094)(9.86054688,286.63066406)(9.32148438,286.39238281)
\curveto(8.96992188,286.24003906)(8.38398438,285.92558594)(7.56367188,285.44902344)
\lineto(7.56367188,286.20488281)
\curveto(8.54414063,286.93925781)(9.52851563,287.48222656)(10.51679688,287.83378906)
\curveto(11.36835938,288.13457031)(12.25898438,288.28496094)(13.18867188,288.28496094)
\curveto(14.24335938,288.28496094)(15.26289063,288.08183594)(16.24726563,287.67558594)
\curveto(17.23164063,287.27324219)(18.09101563,286.78300781)(18.82539063,286.20488281)
\closepath
}
}
{
\newrgbcolor{curcolor}{0 0 0}
\pscustom[linestyle=none,fillstyle=solid,fillcolor=curcolor]
{
\newpath
\moveto(293.07753906,8.7)
\lineto(293.07753906,17.28984375)
\lineto(298.87246094,17.28984375)
\lineto(298.87246094,16.27617187)
\lineto(294.21425781,16.27617187)
\lineto(294.21425781,13.61601562)
\lineto(298.24550781,13.61601562)
\lineto(298.24550781,12.60234375)
\lineto(294.21425781,12.60234375)
\lineto(294.21425781,8.7)
\closepath
}
}
{
\newrgbcolor{curcolor}{0 0 0}
\pscustom[linestyle=none,fillstyle=solid,fillcolor=curcolor]
{
\newpath
\moveto(300.22011719,16.07695312)
\lineto(300.22011719,17.28984375)
\lineto(301.27480469,17.28984375)
\lineto(301.27480469,16.07695312)
\closepath
\moveto(300.22011719,8.7)
\lineto(300.22011719,14.92265625)
\lineto(301.27480469,14.92265625)
\lineto(301.27480469,8.7)
\closepath
}
}
{
\newrgbcolor{curcolor}{0 0 0}
\pscustom[linestyle=none,fillstyle=solid,fillcolor=curcolor]
{
\newpath
\moveto(302.85683594,8.7)
\lineto(302.85683594,17.28984375)
\lineto(303.91152344,17.28984375)
\lineto(303.91152344,8.7)
\closepath
}
}
{
\newrgbcolor{curcolor}{0 0 0}
\pscustom[linestyle=none,fillstyle=solid,fillcolor=curcolor]
{
\newpath
\moveto(309.80605469,10.70390625)
\lineto(310.89589844,10.56914062)
\curveto(310.72402344,9.93242187)(310.40566406,9.43828125)(309.94082031,9.08671875)
\curveto(309.47597656,8.73515625)(308.88222656,8.559375)(308.15957031,8.559375)
\curveto(307.24941406,8.559375)(306.52675781,8.83867187)(305.99160156,9.39726562)
\curveto(305.46035156,9.95976562)(305.19472656,10.746875)(305.19472656,11.75859375)
\curveto(305.19472656,12.80546875)(305.46425781,13.61796875)(306.00332031,14.19609375)
\curveto(306.54238281,14.77421875)(307.24160156,15.06328125)(308.10097656,15.06328125)
\curveto(308.93300781,15.06328125)(309.61269531,14.78007812)(310.14003906,14.21367187)
\curveto(310.66738281,13.64726562)(310.93105469,12.85039062)(310.93105469,11.82304687)
\curveto(310.93105469,11.76054687)(310.92910156,11.66679687)(310.92519531,11.54179687)
\lineto(306.28457031,11.54179687)
\curveto(306.32363281,10.85820312)(306.51699219,10.33476562)(306.86464844,9.97148437)
\curveto(307.21230469,9.60820312)(307.64589844,9.4265625)(308.16542969,9.4265625)
\curveto(308.55214844,9.4265625)(308.88222656,9.528125)(309.15566406,9.73125)
\curveto(309.42910156,9.934375)(309.64589844,10.25859375)(309.80605469,10.70390625)
\closepath
\moveto(306.34316406,12.40898437)
\lineto(309.81777344,12.40898437)
\curveto(309.77089844,12.93242187)(309.63808594,13.325)(309.41933594,13.58671875)
\curveto(309.08339844,13.99296875)(308.64785156,14.19609375)(308.11269531,14.19609375)
\curveto(307.62832031,14.19609375)(307.22011719,14.03398437)(306.88808594,13.70976562)
\curveto(306.55996094,13.38554687)(306.37832031,12.95195312)(306.34316406,12.40898437)
\closepath
}
}
{
\newrgbcolor{curcolor}{0 0 0}
\pscustom[linestyle=none,fillstyle=solid,fillcolor=curcolor]
{
\newpath
\moveto(315.30214844,11.45976562)
\lineto(316.37441406,11.55351562)
\curveto(316.42519531,11.12382812)(316.54238281,10.7703125)(316.72597656,10.49296875)
\curveto(316.91347656,10.21953125)(317.20253906,9.996875)(317.59316406,9.825)
\curveto(317.98378906,9.65703125)(318.42324219,9.57304687)(318.91152344,9.57304687)
\curveto(319.34511719,9.57304687)(319.72792969,9.6375)(320.05996094,9.76640625)
\curveto(320.39199219,9.8953125)(320.63808594,10.07109375)(320.79824219,10.29375)
\curveto(320.96230469,10.5203125)(321.04433594,10.76640625)(321.04433594,11.03203125)
\curveto(321.04433594,11.3015625)(320.96621094,11.5359375)(320.80996094,11.73515625)
\curveto(320.65371094,11.93828125)(320.39589844,12.10820312)(320.03652344,12.24492187)
\curveto(319.80605469,12.33476562)(319.29628906,12.4734375)(318.50722656,12.6609375)
\curveto(317.71816406,12.85234375)(317.16542969,13.03203125)(316.84902344,13.2)
\curveto(316.43886719,13.41484375)(316.13222656,13.68046875)(315.92910156,13.996875)
\curveto(315.72988281,14.3171875)(315.63027344,14.67460937)(315.63027344,15.06914062)
\curveto(315.63027344,15.50273437)(315.75332031,15.90703125)(315.99941406,16.28203125)
\curveto(316.24550781,16.6609375)(316.60488281,16.94804687)(317.07753906,17.14335937)
\curveto(317.55019531,17.33867187)(318.07558594,17.43632812)(318.65371094,17.43632812)
\curveto(319.29042969,17.43632812)(319.85097656,17.3328125)(320.33535156,17.12578125)
\curveto(320.82363281,16.92265625)(321.19863281,16.621875)(321.46035156,16.2234375)
\curveto(321.72207031,15.825)(321.86269531,15.37382812)(321.88222656,14.86992187)
\lineto(320.79238281,14.78789062)
\curveto(320.73378906,15.33085937)(320.53457031,15.74101562)(320.19472656,16.01835937)
\curveto(319.85878906,16.29570312)(319.36074219,16.434375)(318.70058594,16.434375)
\curveto(318.01308594,16.434375)(317.51113281,16.30742187)(317.19472656,16.05351562)
\curveto(316.88222656,15.80351562)(316.72597656,15.50078125)(316.72597656,15.1453125)
\curveto(316.72597656,14.83671875)(316.83730469,14.5828125)(317.05996094,14.38359375)
\curveto(317.27871094,14.184375)(317.84902344,13.97929687)(318.77089844,13.76835937)
\curveto(319.69667969,13.56132812)(320.33144531,13.3796875)(320.67519531,13.2234375)
\curveto(321.17519531,12.99296875)(321.54433594,12.7)(321.78261719,12.34453125)
\curveto(322.02089844,11.99296875)(322.14003906,11.58671875)(322.14003906,11.12578125)
\curveto(322.14003906,10.66875)(322.00917969,10.23710937)(321.74746094,9.83085937)
\curveto(321.48574219,9.42851562)(321.10878906,9.1140625)(320.61660156,8.8875)
\curveto(320.12832031,8.66484375)(319.57753906,8.55351562)(318.96425781,8.55351562)
\curveto(318.18691406,8.55351562)(317.53457031,8.66679687)(317.00722656,8.89335937)
\curveto(316.48378906,9.11992187)(316.07167969,9.45976562)(315.77089844,9.91289062)
\curveto(315.47402344,10.36992187)(315.31777344,10.88554687)(315.30214844,11.45976562)
\closepath
}
}
{
\newrgbcolor{curcolor}{0 0 0}
\pscustom[linestyle=none,fillstyle=solid,fillcolor=curcolor]
{
\newpath
\moveto(323.56386719,16.07695312)
\lineto(323.56386719,17.28984375)
\lineto(324.61855469,17.28984375)
\lineto(324.61855469,16.07695312)
\closepath
\moveto(323.56386719,8.7)
\lineto(323.56386719,14.92265625)
\lineto(324.61855469,14.92265625)
\lineto(324.61855469,8.7)
\closepath
}
}
{
\newrgbcolor{curcolor}{0 0 0}
\pscustom[linestyle=none,fillstyle=solid,fillcolor=curcolor]
{
\newpath
\moveto(325.66738281,8.7)
\lineto(325.66738281,9.55546875)
\lineto(329.62832031,14.10234375)
\curveto(329.17910156,14.07890625)(328.78261719,14.0671875)(328.43886719,14.0671875)
\lineto(325.90175781,14.0671875)
\lineto(325.90175781,14.92265625)
\lineto(330.98769531,14.92265625)
\lineto(330.98769531,14.22539062)
\lineto(327.61855469,10.27617187)
\lineto(326.96816406,9.55546875)
\curveto(327.44082031,9.590625)(327.88417969,9.60820312)(328.29824219,9.60820312)
\lineto(331.17519531,9.60820312)
\lineto(331.17519531,8.7)
\closepath
}
}
{
\newrgbcolor{curcolor}{0 0 0}
\pscustom[linestyle=none,fillstyle=solid,fillcolor=curcolor]
{
\newpath
\moveto(336.48378906,10.70390625)
\lineto(337.57363281,10.56914062)
\curveto(337.40175781,9.93242187)(337.08339844,9.43828125)(336.61855469,9.08671875)
\curveto(336.15371094,8.73515625)(335.55996094,8.559375)(334.83730469,8.559375)
\curveto(333.92714844,8.559375)(333.20449219,8.83867187)(332.66933594,9.39726562)
\curveto(332.13808594,9.95976562)(331.87246094,10.746875)(331.87246094,11.75859375)
\curveto(331.87246094,12.80546875)(332.14199219,13.61796875)(332.68105469,14.19609375)
\curveto(333.22011719,14.77421875)(333.91933594,15.06328125)(334.77871094,15.06328125)
\curveto(335.61074219,15.06328125)(336.29042969,14.78007812)(336.81777344,14.21367187)
\curveto(337.34511719,13.64726562)(337.60878906,12.85039062)(337.60878906,11.82304687)
\curveto(337.60878906,11.76054687)(337.60683594,11.66679687)(337.60292969,11.54179687)
\lineto(332.96230469,11.54179687)
\curveto(333.00136719,10.85820312)(333.19472656,10.33476562)(333.54238281,9.97148437)
\curveto(333.89003906,9.60820312)(334.32363281,9.4265625)(334.84316406,9.4265625)
\curveto(335.22988281,9.4265625)(335.55996094,9.528125)(335.83339844,9.73125)
\curveto(336.10683594,9.934375)(336.32363281,10.25859375)(336.48378906,10.70390625)
\closepath
\moveto(333.02089844,12.40898437)
\lineto(336.49550781,12.40898437)
\curveto(336.44863281,12.93242187)(336.31582031,13.325)(336.09707031,13.58671875)
\curveto(335.76113281,13.99296875)(335.32558594,14.19609375)(334.79042969,14.19609375)
\curveto(334.30605469,14.19609375)(333.89785156,14.03398437)(333.56582031,13.70976562)
\curveto(333.23769531,13.38554687)(333.05605469,12.95195312)(333.02089844,12.40898437)
\closepath
}
}
{
\newrgbcolor{curcolor}{0 0 0}
\pscustom[linestyle=none,fillstyle=solid,fillcolor=curcolor]
{
\newpath
\moveto(171.34609375,449)
\lineto(171.34609375,457.58984375)
\lineto(174.56875,457.58984375)
\curveto(175.225,457.58984375)(175.75039063,457.50195312)(176.14492188,457.32617188)
\curveto(176.54335938,457.15429688)(176.85390625,456.88671875)(177.0765625,456.5234375)
\curveto(177.303125,456.1640625)(177.41640625,455.78710938)(177.41640625,455.39257812)
\curveto(177.41640625,455.02539062)(177.31679688,454.6796875)(177.11757813,454.35546875)
\curveto(176.91835938,454.03125)(176.61757813,453.76953125)(176.21523438,453.5703125)
\curveto(176.73476563,453.41796875)(177.13320313,453.15820312)(177.41054688,452.79101562)
\curveto(177.69179688,452.42382812)(177.83242188,451.99023438)(177.83242188,451.49023438)
\curveto(177.83242188,451.08789062)(177.74648438,450.71289062)(177.57460938,450.36523438)
\curveto(177.40664063,450.02148438)(177.19765625,449.75585938)(176.94765625,449.56835938)
\curveto(176.69765625,449.38085938)(176.38320313,449.23828125)(176.00429688,449.140625)
\curveto(175.62929688,449.046875)(175.16835938,449)(174.62148438,449)
\closepath
\moveto(172.4828125,453.98046875)
\lineto(174.34023438,453.98046875)
\curveto(174.84414063,453.98046875)(175.20546875,454.01367188)(175.42421875,454.08007812)
\curveto(175.71328125,454.16601562)(175.93007813,454.30859375)(176.07460938,454.5078125)
\curveto(176.22304688,454.70703125)(176.29726563,454.95703125)(176.29726563,455.2578125)
\curveto(176.29726563,455.54296875)(176.22890625,455.79296875)(176.0921875,456.0078125)
\curveto(175.95546875,456.2265625)(175.76015625,456.375)(175.50625,456.453125)
\curveto(175.25234375,456.53515625)(174.81679688,456.57617188)(174.19960938,456.57617188)
\lineto(172.4828125,456.57617188)
\closepath
\moveto(172.4828125,450.01367188)
\lineto(174.62148438,450.01367188)
\curveto(174.98867188,450.01367188)(175.24648438,450.02734375)(175.39492188,450.0546875)
\curveto(175.65664063,450.1015625)(175.87539063,450.1796875)(176.05117188,450.2890625)
\curveto(176.22695313,450.3984375)(176.37148438,450.55664062)(176.48476563,450.76367188)
\curveto(176.59804688,450.97460938)(176.6546875,451.21679688)(176.6546875,451.49023438)
\curveto(176.6546875,451.81054688)(176.57265625,452.08789062)(176.40859375,452.32226562)
\curveto(176.24453125,452.56054688)(176.01601563,452.7265625)(175.72304688,452.8203125)
\curveto(175.43398438,452.91796875)(175.01601563,452.96679688)(174.46914063,452.96679688)
\lineto(172.4828125,452.96679688)
\closepath
}
}
{
\newrgbcolor{curcolor}{0 0 0}
\pscustom[linestyle=none,fillstyle=solid,fillcolor=curcolor]
{
\newpath
\moveto(183.32265625,449.76757812)
\curveto(182.93203125,449.43554688)(182.55507813,449.20117188)(182.19179688,449.06445312)
\curveto(181.83242188,448.92773438)(181.44570313,448.859375)(181.03164063,448.859375)
\curveto(180.34804688,448.859375)(179.82265625,449.02539062)(179.45546875,449.35742188)
\curveto(179.08828125,449.69335938)(178.9046875,450.12109375)(178.9046875,450.640625)
\curveto(178.9046875,450.9453125)(178.97304688,451.22265625)(179.10976563,451.47265625)
\curveto(179.25039063,451.7265625)(179.43203125,451.9296875)(179.6546875,452.08203125)
\curveto(179.88125,452.234375)(180.13515625,452.34960938)(180.41640625,452.42773438)
\curveto(180.6234375,452.48242188)(180.9359375,452.53515625)(181.35390625,452.5859375)
\curveto(182.20546875,452.6875)(182.83242188,452.80859375)(183.23476563,452.94921875)
\curveto(183.23867188,453.09375)(183.240625,453.18554688)(183.240625,453.22460938)
\curveto(183.240625,453.65429688)(183.14101563,453.95703125)(182.94179688,454.1328125)
\curveto(182.67226563,454.37109375)(182.271875,454.49023438)(181.740625,454.49023438)
\curveto(181.24453125,454.49023438)(180.87734375,454.40234375)(180.6390625,454.2265625)
\curveto(180.4046875,454.0546875)(180.23085938,453.74804688)(180.11757813,453.30664062)
\lineto(179.08632813,453.44726562)
\curveto(179.18007813,453.88867188)(179.334375,454.24414062)(179.54921875,454.51367188)
\curveto(179.7640625,454.78710938)(180.07460938,454.99609375)(180.48085938,455.140625)
\curveto(180.88710938,455.2890625)(181.3578125,455.36328125)(181.89296875,455.36328125)
\curveto(182.42421875,455.36328125)(182.85585938,455.30078125)(183.18789063,455.17578125)
\curveto(183.51992188,455.05078125)(183.7640625,454.89257812)(183.9203125,454.70117188)
\curveto(184.0765625,454.51367188)(184.1859375,454.27539062)(184.2484375,453.98632812)
\curveto(184.28359375,453.80664062)(184.30117188,453.48242188)(184.30117188,453.01367188)
\lineto(184.30117188,451.60742188)
\curveto(184.30117188,450.62695312)(184.32265625,450.00585938)(184.365625,449.74414062)
\curveto(184.4125,449.48632812)(184.50234375,449.23828125)(184.63515625,449)
\lineto(183.53359375,449)
\curveto(183.42421875,449.21875)(183.35390625,449.47460938)(183.32265625,449.76757812)
\closepath
\moveto(183.23476563,452.12304688)
\curveto(182.85195313,451.96679688)(182.27773438,451.83398438)(181.51210938,451.72460938)
\curveto(181.07851563,451.66210938)(180.771875,451.59179688)(180.5921875,451.51367188)
\curveto(180.4125,451.43554688)(180.27382813,451.3203125)(180.17617188,451.16796875)
\curveto(180.07851563,451.01953125)(180.0296875,450.85351562)(180.0296875,450.66992188)
\curveto(180.0296875,450.38867188)(180.13515625,450.15429688)(180.34609375,449.96679688)
\curveto(180.5609375,449.77929688)(180.8734375,449.68554688)(181.28359375,449.68554688)
\curveto(181.68984375,449.68554688)(182.05117188,449.7734375)(182.36757813,449.94921875)
\curveto(182.68398438,450.12890625)(182.91640625,450.37304688)(183.06484375,450.68164062)
\curveto(183.178125,450.91992188)(183.23476563,451.27148438)(183.23476563,451.73632812)
\closepath
}
}
{
\newrgbcolor{curcolor}{0 0 0}
\pscustom[linestyle=none,fillstyle=solid,fillcolor=curcolor]
{
\newpath
\moveto(185.9359375,449)
\lineto(185.9359375,455.22265625)
\lineto(186.88515625,455.22265625)
\lineto(186.88515625,454.33789062)
\curveto(187.3421875,455.02148438)(188.00234375,455.36328125)(188.865625,455.36328125)
\curveto(189.240625,455.36328125)(189.584375,455.29492188)(189.896875,455.15820312)
\curveto(190.21328125,455.02539062)(190.44960938,454.84960938)(190.60585938,454.63085938)
\curveto(190.76210938,454.41210938)(190.87148438,454.15234375)(190.93398438,453.8515625)
\curveto(190.97304688,453.65625)(190.99257813,453.31445312)(190.99257813,452.82617188)
\lineto(190.99257813,449)
\lineto(189.93789063,449)
\lineto(189.93789063,452.78515625)
\curveto(189.93789063,453.21484375)(189.896875,453.53515625)(189.81484375,453.74609375)
\curveto(189.7328125,453.9609375)(189.58632813,454.13085938)(189.37539063,454.25585938)
\curveto(189.16835938,454.38476562)(188.92421875,454.44921875)(188.64296875,454.44921875)
\curveto(188.19375,454.44921875)(187.80507813,454.30664062)(187.47695313,454.02148438)
\curveto(187.15273438,453.73632812)(186.990625,453.1953125)(186.990625,452.3984375)
\lineto(186.990625,449)
\closepath
}
}
{
\newrgbcolor{curcolor}{0 0 0}
\pscustom[linestyle=none,fillstyle=solid,fillcolor=curcolor]
{
\newpath
\moveto(196.646875,449)
\lineto(196.646875,449.78515625)
\curveto(196.25234375,449.16796875)(195.67226563,448.859375)(194.90664063,448.859375)
\curveto(194.41054688,448.859375)(193.95351563,448.99609375)(193.53554688,449.26953125)
\curveto(193.12148438,449.54296875)(192.79921875,449.92382812)(192.56875,450.41210938)
\curveto(192.3421875,450.90429688)(192.22890625,451.46875)(192.22890625,452.10546875)
\curveto(192.22890625,452.7265625)(192.33242188,453.2890625)(192.53945313,453.79296875)
\curveto(192.74648438,454.30078125)(193.05703125,454.68945312)(193.47109375,454.95898438)
\curveto(193.88515625,455.22851562)(194.34804688,455.36328125)(194.85976563,455.36328125)
\curveto(195.23476563,455.36328125)(195.56875,455.28320312)(195.86171875,455.12304688)
\curveto(196.1546875,454.96679688)(196.39296875,454.76171875)(196.5765625,454.5078125)
\lineto(196.5765625,457.58984375)
\lineto(197.62539063,457.58984375)
\lineto(197.62539063,449)
\closepath
\moveto(193.31289063,452.10546875)
\curveto(193.31289063,451.30859375)(193.48085938,450.71289062)(193.81679688,450.31835938)
\curveto(194.15273438,449.92382812)(194.54921875,449.7265625)(195.00625,449.7265625)
\curveto(195.4671875,449.7265625)(195.8578125,449.9140625)(196.178125,450.2890625)
\curveto(196.50234375,450.66796875)(196.66445313,451.24414062)(196.66445313,452.01757812)
\curveto(196.66445313,452.86914062)(196.50039063,453.49414062)(196.17226563,453.89257812)
\curveto(195.84414063,454.29101562)(195.43984375,454.49023438)(194.959375,454.49023438)
\curveto(194.490625,454.49023438)(194.09804688,454.29882812)(193.78164063,453.91601562)
\curveto(193.46914063,453.53320312)(193.31289063,452.9296875)(193.31289063,452.10546875)
\closepath
}
}
{
\newrgbcolor{curcolor}{0 0 0}
\pscustom[linestyle=none,fillstyle=solid,fillcolor=curcolor]
{
\newpath
\moveto(200.43203125,449)
\lineto(198.52773438,455.22265625)
\lineto(199.61757813,455.22265625)
\lineto(200.6078125,451.63085938)
\lineto(200.97695313,450.29492188)
\curveto(200.99257813,450.36132812)(201.1,450.7890625)(201.29921875,451.578125)
\lineto(202.28945313,455.22265625)
\lineto(203.3734375,455.22265625)
\lineto(204.30507813,451.61328125)
\lineto(204.615625,450.42382812)
\lineto(204.97304688,451.625)
\lineto(206.03945313,455.22265625)
\lineto(207.06484375,455.22265625)
\lineto(205.11953125,449)
\lineto(204.02382813,449)
\lineto(203.03359375,452.7265625)
\lineto(202.79335938,453.78710938)
\lineto(201.53359375,449)
\closepath
}
}
{
\newrgbcolor{curcolor}{0 0 0}
\pscustom[linestyle=none,fillstyle=solid,fillcolor=curcolor]
{
\newpath
\moveto(207.95546875,456.37695312)
\lineto(207.95546875,457.58984375)
\lineto(209.01015625,457.58984375)
\lineto(209.01015625,456.37695312)
\closepath
\moveto(207.95546875,449)
\lineto(207.95546875,455.22265625)
\lineto(209.01015625,455.22265625)
\lineto(209.01015625,449)
\closepath
}
}
{
\newrgbcolor{curcolor}{0 0 0}
\pscustom[linestyle=none,fillstyle=solid,fillcolor=curcolor]
{
\newpath
\moveto(214.65273438,449)
\lineto(214.65273438,449.78515625)
\curveto(214.25820313,449.16796875)(213.678125,448.859375)(212.9125,448.859375)
\curveto(212.41640625,448.859375)(211.959375,448.99609375)(211.54140625,449.26953125)
\curveto(211.12734375,449.54296875)(210.80507813,449.92382812)(210.57460938,450.41210938)
\curveto(210.34804688,450.90429688)(210.23476563,451.46875)(210.23476563,452.10546875)
\curveto(210.23476563,452.7265625)(210.33828125,453.2890625)(210.5453125,453.79296875)
\curveto(210.75234375,454.30078125)(211.06289063,454.68945312)(211.47695313,454.95898438)
\curveto(211.89101563,455.22851562)(212.35390625,455.36328125)(212.865625,455.36328125)
\curveto(213.240625,455.36328125)(213.57460938,455.28320312)(213.86757813,455.12304688)
\curveto(214.16054688,454.96679688)(214.39882813,454.76171875)(214.58242188,454.5078125)
\lineto(214.58242188,457.58984375)
\lineto(215.63125,457.58984375)
\lineto(215.63125,449)
\closepath
\moveto(211.31875,452.10546875)
\curveto(211.31875,451.30859375)(211.48671875,450.71289062)(211.82265625,450.31835938)
\curveto(212.15859375,449.92382812)(212.55507813,449.7265625)(213.01210938,449.7265625)
\curveto(213.47304688,449.7265625)(213.86367188,449.9140625)(214.18398438,450.2890625)
\curveto(214.50820313,450.66796875)(214.6703125,451.24414062)(214.6703125,452.01757812)
\curveto(214.6703125,452.86914062)(214.50625,453.49414062)(214.178125,453.89257812)
\curveto(213.85,454.29101562)(213.44570313,454.49023438)(212.96523438,454.49023438)
\curveto(212.49648438,454.49023438)(212.10390625,454.29882812)(211.7875,453.91601562)
\curveto(211.475,453.53320312)(211.31875,452.9296875)(211.31875,452.10546875)
\closepath
}
}
{
\newrgbcolor{curcolor}{0 0 0}
\pscustom[linestyle=none,fillstyle=solid,fillcolor=curcolor]
{
\newpath
\moveto(219.5921875,449.94335938)
\lineto(219.74453125,449.01171875)
\curveto(219.44765625,448.94921875)(219.18203125,448.91796875)(218.94765625,448.91796875)
\curveto(218.56484375,448.91796875)(218.26796875,448.97851562)(218.05703125,449.09960938)
\curveto(217.84609375,449.22070312)(217.69765625,449.37890625)(217.61171875,449.57421875)
\curveto(217.52578125,449.7734375)(217.4828125,450.18945312)(217.4828125,450.82226562)
\lineto(217.4828125,454.40234375)
\lineto(216.709375,454.40234375)
\lineto(216.709375,455.22265625)
\lineto(217.4828125,455.22265625)
\lineto(217.4828125,456.76367188)
\lineto(218.53164063,457.39648438)
\lineto(218.53164063,455.22265625)
\lineto(219.5921875,455.22265625)
\lineto(219.5921875,454.40234375)
\lineto(218.53164063,454.40234375)
\lineto(218.53164063,450.76367188)
\curveto(218.53164063,450.46289062)(218.54921875,450.26953125)(218.584375,450.18359375)
\curveto(218.6234375,450.09765625)(218.68398438,450.02929688)(218.76601563,449.97851562)
\curveto(218.85195313,449.92773438)(218.97304688,449.90234375)(219.12929688,449.90234375)
\curveto(219.24648438,449.90234375)(219.40078125,449.91601562)(219.5921875,449.94335938)
\closepath
}
}
{
\newrgbcolor{curcolor}{0 0 0}
\pscustom[linestyle=none,fillstyle=solid,fillcolor=curcolor]
{
\newpath
\moveto(220.6234375,449)
\lineto(220.6234375,457.58984375)
\lineto(221.678125,457.58984375)
\lineto(221.678125,454.5078125)
\curveto(222.1703125,455.078125)(222.79140625,455.36328125)(223.54140625,455.36328125)
\curveto(224.00234375,455.36328125)(224.40273438,455.27148438)(224.74257813,455.08789062)
\curveto(225.08242188,454.90820312)(225.32460938,454.65820312)(225.46914063,454.33789062)
\curveto(225.61757813,454.01757812)(225.69179688,453.55273438)(225.69179688,452.94335938)
\lineto(225.69179688,449)
\lineto(224.63710938,449)
\lineto(224.63710938,452.94335938)
\curveto(224.63710938,453.47070312)(224.521875,453.85351562)(224.29140625,454.09179688)
\curveto(224.06484375,454.33398438)(223.74257813,454.45507812)(223.32460938,454.45507812)
\curveto(223.01210938,454.45507812)(222.7171875,454.37304688)(222.43984375,454.20898438)
\curveto(222.16640625,454.04882812)(221.97109375,453.83007812)(221.85390625,453.55273438)
\curveto(221.73671875,453.27539062)(221.678125,452.89257812)(221.678125,452.40429688)
\lineto(221.678125,449)
\closepath
}
}
{
\newrgbcolor{curcolor}{0 0 0}
\pscustom[linestyle=none,fillstyle=solid,fillcolor=curcolor]
{
\newpath
\moveto(230.76601563,449)
\lineto(230.76601563,457.58984375)
\lineto(233.725,457.58984375)
\curveto(234.39296875,457.58984375)(234.90273438,457.54882812)(235.25429688,457.46679688)
\curveto(235.74648438,457.35351562)(236.16640625,457.1484375)(236.5140625,456.8515625)
\curveto(236.9671875,456.46875)(237.30507813,455.97851562)(237.52773438,455.38085938)
\curveto(237.75429688,454.78710938)(237.86757813,454.10742188)(237.86757813,453.34179688)
\curveto(237.86757813,452.68945312)(237.79140625,452.11132812)(237.6390625,451.60742188)
\curveto(237.48671875,451.10351562)(237.29140625,450.68554688)(237.053125,450.35351562)
\curveto(236.81484375,450.02539062)(236.553125,449.765625)(236.26796875,449.57421875)
\curveto(235.98671875,449.38671875)(235.64492188,449.24414062)(235.24257813,449.14648438)
\curveto(234.84414063,449.04882812)(234.38515625,449)(233.865625,449)
\closepath
\moveto(231.90273438,450.01367188)
\lineto(233.73671875,450.01367188)
\curveto(234.303125,450.01367188)(234.74648438,450.06640625)(235.06679688,450.171875)
\curveto(235.39101563,450.27734375)(235.64882813,450.42578125)(235.84023438,450.6171875)
\curveto(236.10976563,450.88671875)(236.31875,451.24804688)(236.4671875,451.70117188)
\curveto(236.61953125,452.15820312)(236.69570313,452.7109375)(236.69570313,453.359375)
\curveto(236.69570313,454.2578125)(236.54726563,454.94726562)(236.25039063,455.42773438)
\curveto(235.95742188,455.91210938)(235.6,456.23632812)(235.178125,456.40039062)
\curveto(234.8734375,456.51757812)(234.38320313,456.57617188)(233.70742188,456.57617188)
\lineto(231.90273438,456.57617188)
\closepath
}
}
{
\newrgbcolor{curcolor}{0 0 0}
\pscustom[linestyle=none,fillstyle=solid,fillcolor=curcolor]
{
\newpath
\moveto(239.303125,456.37695312)
\lineto(239.303125,457.58984375)
\lineto(240.3578125,457.58984375)
\lineto(240.3578125,456.37695312)
\closepath
\moveto(239.303125,449)
\lineto(239.303125,455.22265625)
\lineto(240.3578125,455.22265625)
\lineto(240.3578125,449)
\closepath
}
}
{
\newrgbcolor{curcolor}{0 0 0}
\pscustom[linestyle=none,fillstyle=solid,fillcolor=curcolor]
{
\newpath
\moveto(242.21523438,449)
\lineto(242.21523438,454.40234375)
\lineto(241.28359375,454.40234375)
\lineto(241.28359375,455.22265625)
\lineto(242.21523438,455.22265625)
\lineto(242.21523438,455.88476562)
\curveto(242.21523438,456.30273438)(242.25234375,456.61328125)(242.3265625,456.81640625)
\curveto(242.428125,457.08984375)(242.60585938,457.31054688)(242.85976563,457.47851562)
\curveto(243.11757813,457.65039062)(243.47695313,457.73632812)(243.93789063,457.73632812)
\curveto(244.23476563,457.73632812)(244.56289063,457.70117188)(244.92226563,457.63085938)
\lineto(244.7640625,456.7109375)
\curveto(244.5453125,456.75)(244.33828125,456.76953125)(244.14296875,456.76953125)
\curveto(243.82265625,456.76953125)(243.59609375,456.70117188)(243.46328125,456.56445312)
\curveto(243.33046875,456.42773438)(243.2640625,456.171875)(243.2640625,455.796875)
\lineto(243.2640625,455.22265625)
\lineto(244.47695313,455.22265625)
\lineto(244.47695313,454.40234375)
\lineto(243.2640625,454.40234375)
\lineto(243.2640625,449)
\closepath
}
}
{
\newrgbcolor{curcolor}{0 0 0}
\pscustom[linestyle=none,fillstyle=solid,fillcolor=curcolor]
{
\newpath
\moveto(245.33242188,449)
\lineto(245.33242188,454.40234375)
\lineto(244.40078125,454.40234375)
\lineto(244.40078125,455.22265625)
\lineto(245.33242188,455.22265625)
\lineto(245.33242188,455.88476562)
\curveto(245.33242188,456.30273438)(245.36953125,456.61328125)(245.44375,456.81640625)
\curveto(245.5453125,457.08984375)(245.72304688,457.31054688)(245.97695313,457.47851562)
\curveto(246.23476563,457.65039062)(246.59414063,457.73632812)(247.05507813,457.73632812)
\curveto(247.35195313,457.73632812)(247.68007813,457.70117188)(248.03945313,457.63085938)
\lineto(247.88125,456.7109375)
\curveto(247.6625,456.75)(247.45546875,456.76953125)(247.26015625,456.76953125)
\curveto(246.93984375,456.76953125)(246.71328125,456.70117188)(246.58046875,456.56445312)
\curveto(246.44765625,456.42773438)(246.38125,456.171875)(246.38125,455.796875)
\lineto(246.38125,455.22265625)
\lineto(247.59414063,455.22265625)
\lineto(247.59414063,454.40234375)
\lineto(246.38125,454.40234375)
\lineto(246.38125,449)
\closepath
}
}
{
\newrgbcolor{curcolor}{0 0 0}
\pscustom[linestyle=none,fillstyle=solid,fillcolor=curcolor]
{
\newpath
\moveto(252.67421875,451.00390625)
\lineto(253.7640625,450.86914062)
\curveto(253.5921875,450.23242188)(253.27382813,449.73828125)(252.80898438,449.38671875)
\curveto(252.34414063,449.03515625)(251.75039063,448.859375)(251.02773438,448.859375)
\curveto(250.11757813,448.859375)(249.39492188,449.13867188)(248.85976563,449.69726562)
\curveto(248.32851563,450.25976562)(248.06289063,451.046875)(248.06289063,452.05859375)
\curveto(248.06289063,453.10546875)(248.33242188,453.91796875)(248.87148438,454.49609375)
\curveto(249.41054688,455.07421875)(250.10976563,455.36328125)(250.96914063,455.36328125)
\curveto(251.80117188,455.36328125)(252.48085938,455.08007812)(253.00820313,454.51367188)
\curveto(253.53554688,453.94726562)(253.79921875,453.15039062)(253.79921875,452.12304688)
\curveto(253.79921875,452.06054688)(253.79726563,451.96679688)(253.79335938,451.84179688)
\lineto(249.15273438,451.84179688)
\curveto(249.19179688,451.15820312)(249.38515625,450.63476562)(249.7328125,450.27148438)
\curveto(250.08046875,449.90820312)(250.5140625,449.7265625)(251.03359375,449.7265625)
\curveto(251.4203125,449.7265625)(251.75039063,449.828125)(252.02382813,450.03125)
\curveto(252.29726563,450.234375)(252.5140625,450.55859375)(252.67421875,451.00390625)
\closepath
\moveto(249.21132813,452.70898438)
\lineto(252.6859375,452.70898438)
\curveto(252.6390625,453.23242188)(252.50625,453.625)(252.2875,453.88671875)
\curveto(251.9515625,454.29296875)(251.51601563,454.49609375)(250.98085938,454.49609375)
\curveto(250.49648438,454.49609375)(250.08828125,454.33398438)(249.75625,454.00976562)
\curveto(249.428125,453.68554688)(249.24648438,453.25195312)(249.21132813,452.70898438)
\closepath
}
}
{
\newrgbcolor{curcolor}{0 0 0}
\pscustom[linestyle=none,fillstyle=solid,fillcolor=curcolor]
{
\newpath
\moveto(255.0765625,449)
\lineto(255.0765625,455.22265625)
\lineto(256.02578125,455.22265625)
\lineto(256.02578125,454.27929688)
\curveto(256.26796875,454.72070312)(256.490625,455.01171875)(256.69375,455.15234375)
\curveto(256.90078125,455.29296875)(257.12734375,455.36328125)(257.3734375,455.36328125)
\curveto(257.72890625,455.36328125)(258.09023438,455.25)(258.45742188,455.0234375)
\lineto(258.09414063,454.04492188)
\curveto(257.83632813,454.19726562)(257.57851563,454.2734375)(257.32070313,454.2734375)
\curveto(257.09023438,454.2734375)(256.88320313,454.203125)(256.69960938,454.0625)
\curveto(256.51601563,453.92578125)(256.38515625,453.734375)(256.30703125,453.48828125)
\curveto(256.18984375,453.11328125)(256.13125,452.703125)(256.13125,452.2578125)
\lineto(256.13125,449)
\closepath
}
}
{
\newrgbcolor{curcolor}{0 0 0}
\pscustom[linestyle=none,fillstyle=solid,fillcolor=curcolor]
{
\newpath
\moveto(263.34414063,451.00390625)
\lineto(264.43398438,450.86914062)
\curveto(264.26210938,450.23242188)(263.94375,449.73828125)(263.47890625,449.38671875)
\curveto(263.0140625,449.03515625)(262.4203125,448.859375)(261.69765625,448.859375)
\curveto(260.7875,448.859375)(260.06484375,449.13867188)(259.5296875,449.69726562)
\curveto(258.9984375,450.25976562)(258.7328125,451.046875)(258.7328125,452.05859375)
\curveto(258.7328125,453.10546875)(259.00234375,453.91796875)(259.54140625,454.49609375)
\curveto(260.08046875,455.07421875)(260.7796875,455.36328125)(261.6390625,455.36328125)
\curveto(262.47109375,455.36328125)(263.15078125,455.08007812)(263.678125,454.51367188)
\curveto(264.20546875,453.94726562)(264.46914063,453.15039062)(264.46914063,452.12304688)
\curveto(264.46914063,452.06054688)(264.4671875,451.96679688)(264.46328125,451.84179688)
\lineto(259.82265625,451.84179688)
\curveto(259.86171875,451.15820312)(260.05507813,450.63476562)(260.40273438,450.27148438)
\curveto(260.75039063,449.90820312)(261.18398438,449.7265625)(261.70351563,449.7265625)
\curveto(262.09023438,449.7265625)(262.4203125,449.828125)(262.69375,450.03125)
\curveto(262.9671875,450.234375)(263.18398438,450.55859375)(263.34414063,451.00390625)
\closepath
\moveto(259.88125,452.70898438)
\lineto(263.35585938,452.70898438)
\curveto(263.30898438,453.23242188)(263.17617188,453.625)(262.95742188,453.88671875)
\curveto(262.62148438,454.29296875)(262.1859375,454.49609375)(261.65078125,454.49609375)
\curveto(261.16640625,454.49609375)(260.75820313,454.33398438)(260.42617188,454.00976562)
\curveto(260.09804688,453.68554688)(259.91640625,453.25195312)(259.88125,452.70898438)
\closepath
}
}
{
\newrgbcolor{curcolor}{0 0 0}
\pscustom[linestyle=none,fillstyle=solid,fillcolor=curcolor]
{
\newpath
\moveto(265.75820313,449)
\lineto(265.75820313,455.22265625)
\lineto(266.70742188,455.22265625)
\lineto(266.70742188,454.33789062)
\curveto(267.16445313,455.02148438)(267.82460938,455.36328125)(268.68789063,455.36328125)
\curveto(269.06289063,455.36328125)(269.40664063,455.29492188)(269.71914063,455.15820312)
\curveto(270.03554688,455.02539062)(270.271875,454.84960938)(270.428125,454.63085938)
\curveto(270.584375,454.41210938)(270.69375,454.15234375)(270.75625,453.8515625)
\curveto(270.7953125,453.65625)(270.81484375,453.31445312)(270.81484375,452.82617188)
\lineto(270.81484375,449)
\lineto(269.76015625,449)
\lineto(269.76015625,452.78515625)
\curveto(269.76015625,453.21484375)(269.71914063,453.53515625)(269.63710938,453.74609375)
\curveto(269.55507813,453.9609375)(269.40859375,454.13085938)(269.19765625,454.25585938)
\curveto(268.990625,454.38476562)(268.74648438,454.44921875)(268.46523438,454.44921875)
\curveto(268.01601563,454.44921875)(267.62734375,454.30664062)(267.29921875,454.02148438)
\curveto(266.975,453.73632812)(266.81289063,453.1953125)(266.81289063,452.3984375)
\lineto(266.81289063,449)
\closepath
}
}
{
\newrgbcolor{curcolor}{0 0 0}
\pscustom[linestyle=none,fillstyle=solid,fillcolor=curcolor]
{
\newpath
\moveto(274.73476563,449.94335938)
\lineto(274.88710938,449.01171875)
\curveto(274.59023438,448.94921875)(274.32460938,448.91796875)(274.09023438,448.91796875)
\curveto(273.70742188,448.91796875)(273.41054688,448.97851562)(273.19960938,449.09960938)
\curveto(272.98867188,449.22070312)(272.84023438,449.37890625)(272.75429688,449.57421875)
\curveto(272.66835938,449.7734375)(272.62539063,450.18945312)(272.62539063,450.82226562)
\lineto(272.62539063,454.40234375)
\lineto(271.85195313,454.40234375)
\lineto(271.85195313,455.22265625)
\lineto(272.62539063,455.22265625)
\lineto(272.62539063,456.76367188)
\lineto(273.67421875,457.39648438)
\lineto(273.67421875,455.22265625)
\lineto(274.73476563,455.22265625)
\lineto(274.73476563,454.40234375)
\lineto(273.67421875,454.40234375)
\lineto(273.67421875,450.76367188)
\curveto(273.67421875,450.46289062)(273.69179688,450.26953125)(273.72695313,450.18359375)
\curveto(273.76601563,450.09765625)(273.8265625,450.02929688)(273.90859375,449.97851562)
\curveto(273.99453125,449.92773438)(274.115625,449.90234375)(274.271875,449.90234375)
\curveto(274.3890625,449.90234375)(274.54335938,449.91601562)(274.73476563,449.94335938)
\closepath
}
}
{
\newrgbcolor{curcolor}{0 0 0}
\pscustom[linestyle=none,fillstyle=solid,fillcolor=curcolor]
{
\newpath
\moveto(279.22304688,449)
\lineto(279.22304688,457.58984375)
\lineto(280.3890625,457.58984375)
\lineto(284.90078125,450.84570312)
\lineto(284.90078125,457.58984375)
\lineto(285.990625,457.58984375)
\lineto(285.990625,449)
\lineto(284.82460938,449)
\lineto(280.31289063,455.75)
\lineto(280.31289063,449)
\closepath
}
}
{
\newrgbcolor{curcolor}{0 0 0}
\pscustom[linestyle=none,fillstyle=solid,fillcolor=curcolor]
{
\newpath
\moveto(287.3734375,452.11132812)
\curveto(287.3734375,453.26367188)(287.69375,454.1171875)(288.334375,454.671875)
\curveto(288.86953125,455.1328125)(289.521875,455.36328125)(290.29140625,455.36328125)
\curveto(291.146875,455.36328125)(291.84609375,455.08203125)(292.3890625,454.51953125)
\curveto(292.93203125,453.9609375)(293.20351563,453.1875)(293.20351563,452.19921875)
\curveto(293.20351563,451.3984375)(293.08242188,450.76757812)(292.84023438,450.30664062)
\curveto(292.60195313,449.84960938)(292.25234375,449.49414062)(291.79140625,449.24023438)
\curveto(291.334375,448.98632812)(290.834375,448.859375)(290.29140625,448.859375)
\curveto(289.4203125,448.859375)(288.71523438,449.13867188)(288.17617188,449.69726562)
\curveto(287.64101563,450.25585938)(287.3734375,451.06054688)(287.3734375,452.11132812)
\closepath
\moveto(288.45742188,452.11132812)
\curveto(288.45742188,451.31445312)(288.63125,450.71679688)(288.97890625,450.31835938)
\curveto(289.3265625,449.92382812)(289.7640625,449.7265625)(290.29140625,449.7265625)
\curveto(290.81484375,449.7265625)(291.25039063,449.92578125)(291.59804688,450.32421875)
\curveto(291.94570313,450.72265625)(292.11953125,451.33007812)(292.11953125,452.14648438)
\curveto(292.11953125,452.91601562)(291.94375,453.49804688)(291.5921875,453.89257812)
\curveto(291.24453125,454.29101562)(290.8109375,454.49023438)(290.29140625,454.49023438)
\curveto(289.7640625,454.49023438)(289.3265625,454.29296875)(288.97890625,453.8984375)
\curveto(288.63125,453.50390625)(288.45742188,452.90820312)(288.45742188,452.11132812)
\closepath
}
}
{
\newrgbcolor{curcolor}{0 0 0}
\pscustom[linestyle=none,fillstyle=solid,fillcolor=curcolor]
{
\newpath
\moveto(298.47695313,449)
\lineto(298.47695313,449.78515625)
\curveto(298.08242188,449.16796875)(297.50234375,448.859375)(296.73671875,448.859375)
\curveto(296.240625,448.859375)(295.78359375,448.99609375)(295.365625,449.26953125)
\curveto(294.9515625,449.54296875)(294.62929688,449.92382812)(294.39882813,450.41210938)
\curveto(294.17226563,450.90429688)(294.05898438,451.46875)(294.05898438,452.10546875)
\curveto(294.05898438,452.7265625)(294.1625,453.2890625)(294.36953125,453.79296875)
\curveto(294.5765625,454.30078125)(294.88710938,454.68945312)(295.30117188,454.95898438)
\curveto(295.71523438,455.22851562)(296.178125,455.36328125)(296.68984375,455.36328125)
\curveto(297.06484375,455.36328125)(297.39882813,455.28320312)(297.69179688,455.12304688)
\curveto(297.98476563,454.96679688)(298.22304688,454.76171875)(298.40664063,454.5078125)
\lineto(298.40664063,457.58984375)
\lineto(299.45546875,457.58984375)
\lineto(299.45546875,449)
\closepath
\moveto(295.14296875,452.10546875)
\curveto(295.14296875,451.30859375)(295.3109375,450.71289062)(295.646875,450.31835938)
\curveto(295.9828125,449.92382812)(296.37929688,449.7265625)(296.83632813,449.7265625)
\curveto(297.29726563,449.7265625)(297.68789063,449.9140625)(298.00820313,450.2890625)
\curveto(298.33242188,450.66796875)(298.49453125,451.24414062)(298.49453125,452.01757812)
\curveto(298.49453125,452.86914062)(298.33046875,453.49414062)(298.00234375,453.89257812)
\curveto(297.67421875,454.29101562)(297.26992188,454.49023438)(296.78945313,454.49023438)
\curveto(296.32070313,454.49023438)(295.928125,454.29882812)(295.61171875,453.91601562)
\curveto(295.29921875,453.53320312)(295.14296875,452.9296875)(295.14296875,452.10546875)
\closepath
}
}
{
\newrgbcolor{curcolor}{0 0 0}
\pscustom[linestyle=none,fillstyle=solid,fillcolor=curcolor]
{
\newpath
\moveto(305.3734375,451.00390625)
\lineto(306.46328125,450.86914062)
\curveto(306.29140625,450.23242188)(305.97304688,449.73828125)(305.50820313,449.38671875)
\curveto(305.04335938,449.03515625)(304.44960938,448.859375)(303.72695313,448.859375)
\curveto(302.81679688,448.859375)(302.09414063,449.13867188)(301.55898438,449.69726562)
\curveto(301.02773438,450.25976562)(300.76210938,451.046875)(300.76210938,452.05859375)
\curveto(300.76210938,453.10546875)(301.03164063,453.91796875)(301.57070313,454.49609375)
\curveto(302.10976563,455.07421875)(302.80898438,455.36328125)(303.66835938,455.36328125)
\curveto(304.50039063,455.36328125)(305.18007813,455.08007812)(305.70742188,454.51367188)
\curveto(306.23476563,453.94726562)(306.4984375,453.15039062)(306.4984375,452.12304688)
\curveto(306.4984375,452.06054688)(306.49648438,451.96679688)(306.49257813,451.84179688)
\lineto(301.85195313,451.84179688)
\curveto(301.89101563,451.15820312)(302.084375,450.63476562)(302.43203125,450.27148438)
\curveto(302.7796875,449.90820312)(303.21328125,449.7265625)(303.7328125,449.7265625)
\curveto(304.11953125,449.7265625)(304.44960938,449.828125)(304.72304688,450.03125)
\curveto(304.99648438,450.234375)(305.21328125,450.55859375)(305.3734375,451.00390625)
\closepath
\moveto(301.91054688,452.70898438)
\lineto(305.38515625,452.70898438)
\curveto(305.33828125,453.23242188)(305.20546875,453.625)(304.98671875,453.88671875)
\curveto(304.65078125,454.29296875)(304.21523438,454.49609375)(303.68007813,454.49609375)
\curveto(303.19570313,454.49609375)(302.7875,454.33398438)(302.45546875,454.00976562)
\curveto(302.12734375,453.68554688)(301.94570313,453.25195312)(301.91054688,452.70898438)
\closepath
}
}
{
\newrgbcolor{curcolor}{0 0 0}
\pscustom[linestyle=none,fillstyle=solid,fillcolor=curcolor]
{
\newpath
\moveto(316.37148438,450.01367188)
\lineto(316.37148438,449)
\lineto(310.69375,449)
\curveto(310.6859375,449.25390625)(310.72695313,449.49804688)(310.81679688,449.73242188)
\curveto(310.96132813,450.11914062)(311.19179688,450.5)(311.50820313,450.875)
\curveto(311.82851563,451.25)(312.28945313,451.68359375)(312.89101563,452.17578125)
\curveto(313.82460938,452.94140625)(314.45546875,453.546875)(314.78359375,453.9921875)
\curveto(315.11171875,454.44140625)(315.27578125,454.86523438)(315.27578125,455.26367188)
\curveto(315.27578125,455.68164062)(315.12539063,456.03320312)(314.82460938,456.31835938)
\curveto(314.52773438,456.60742188)(314.1390625,456.75195312)(313.65859375,456.75195312)
\curveto(313.15078125,456.75195312)(312.74453125,456.59960938)(312.43984375,456.29492188)
\curveto(312.13515625,455.99023438)(311.98085938,455.56835938)(311.97695313,455.02929688)
\lineto(310.89296875,455.140625)
\curveto(310.9671875,455.94921875)(311.24648438,456.56445312)(311.73085938,456.98632812)
\curveto(312.21523438,457.41210938)(312.865625,457.625)(313.68203125,457.625)
\curveto(314.50625,457.625)(315.15859375,457.39648438)(315.6390625,456.93945312)
\curveto(316.11953125,456.48242188)(316.35976563,455.91601562)(316.35976563,455.24023438)
\curveto(316.35976563,454.89648438)(316.28945313,454.55859375)(316.14882813,454.2265625)
\curveto(316.00820313,453.89453125)(315.77382813,453.54492188)(315.44570313,453.17773438)
\curveto(315.12148438,452.81054688)(314.58046875,452.30664062)(313.82265625,451.66601562)
\curveto(313.18984375,451.13476562)(312.78359375,450.7734375)(312.60390625,450.58203125)
\curveto(312.42421875,450.39453125)(312.27578125,450.20507812)(312.15859375,450.01367188)
\closepath
}
}
{
\newrgbcolor{curcolor}{0 0 0}
\pscustom[linestyle=none,fillstyle=solid,fillcolor=curcolor]
{
\newpath
\moveto(317.89492188,449)
\lineto(317.89492188,457.58984375)
\lineto(319.60585938,457.58984375)
\lineto(321.6390625,451.5078125)
\curveto(321.8265625,450.94140625)(321.96328125,450.51757812)(322.04921875,450.23632812)
\curveto(322.146875,450.54882812)(322.29921875,451.0078125)(322.50625,451.61328125)
\lineto(324.56289063,457.58984375)
\lineto(326.0921875,457.58984375)
\lineto(326.0921875,449)
\lineto(324.99648438,449)
\lineto(324.99648438,456.18945312)
\lineto(322.50039063,449)
\lineto(321.475,449)
\lineto(318.990625,456.3125)
\lineto(318.990625,449)
\closepath
}
}
{
\newrgbcolor{curcolor}{0 0 0}
\pscustom[linestyle=none,fillstyle=solid,fillcolor=curcolor]
{
\newpath
\moveto(331.27773438,449)
\lineto(331.27773438,457.58984375)
\lineto(335.08632813,457.58984375)
\curveto(335.85195313,457.58984375)(336.43398438,457.51171875)(336.83242188,457.35546875)
\curveto(337.23085938,457.203125)(337.54921875,456.93164062)(337.7875,456.54101562)
\curveto(338.02578125,456.15039062)(338.14492188,455.71875)(338.14492188,455.24609375)
\curveto(338.14492188,454.63671875)(337.94765625,454.12304688)(337.553125,453.70507812)
\curveto(337.15859375,453.28710938)(336.54921875,453.02148438)(335.725,452.90820312)
\curveto(336.02578125,452.76367188)(336.25429688,452.62109375)(336.41054688,452.48046875)
\curveto(336.74257813,452.17578125)(337.05703125,451.79492188)(337.35390625,451.33789062)
\lineto(338.84804688,449)
\lineto(337.41835938,449)
\lineto(336.28164063,450.78710938)
\curveto(335.94960938,451.30273438)(335.67617188,451.69726562)(335.46132813,451.97070312)
\curveto(335.24648438,452.24414062)(335.053125,452.43554688)(334.88125,452.54492188)
\curveto(334.71328125,452.65429688)(334.54140625,452.73046875)(334.365625,452.7734375)
\curveto(334.23671875,452.80078125)(334.02578125,452.81445312)(333.7328125,452.81445312)
\lineto(332.41445313,452.81445312)
\lineto(332.41445313,449)
\closepath
\moveto(332.41445313,453.79882812)
\lineto(334.8578125,453.79882812)
\curveto(335.37734375,453.79882812)(335.78359375,453.8515625)(336.0765625,453.95703125)
\curveto(336.36953125,454.06640625)(336.5921875,454.23828125)(336.74453125,454.47265625)
\curveto(336.896875,454.7109375)(336.97304688,454.96875)(336.97304688,455.24609375)
\curveto(336.97304688,455.65234375)(336.82460938,455.98632812)(336.52773438,456.24804688)
\curveto(336.23476563,456.50976562)(335.76992188,456.640625)(335.13320313,456.640625)
\lineto(332.41445313,456.640625)
\closepath
}
}
{
\newrgbcolor{curcolor}{0 0 0}
\pscustom[linestyle=none,fillstyle=solid,fillcolor=curcolor]
{
\newpath
\moveto(344.05117188,451.00390625)
\lineto(345.14101563,450.86914062)
\curveto(344.96914063,450.23242188)(344.65078125,449.73828125)(344.1859375,449.38671875)
\curveto(343.72109375,449.03515625)(343.12734375,448.859375)(342.4046875,448.859375)
\curveto(341.49453125,448.859375)(340.771875,449.13867188)(340.23671875,449.69726562)
\curveto(339.70546875,450.25976562)(339.43984375,451.046875)(339.43984375,452.05859375)
\curveto(339.43984375,453.10546875)(339.709375,453.91796875)(340.2484375,454.49609375)
\curveto(340.7875,455.07421875)(341.48671875,455.36328125)(342.34609375,455.36328125)
\curveto(343.178125,455.36328125)(343.8578125,455.08007812)(344.38515625,454.51367188)
\curveto(344.9125,453.94726562)(345.17617188,453.15039062)(345.17617188,452.12304688)
\curveto(345.17617188,452.06054688)(345.17421875,451.96679688)(345.1703125,451.84179688)
\lineto(340.5296875,451.84179688)
\curveto(340.56875,451.15820312)(340.76210938,450.63476562)(341.10976563,450.27148438)
\curveto(341.45742188,449.90820312)(341.89101563,449.7265625)(342.41054688,449.7265625)
\curveto(342.79726563,449.7265625)(343.12734375,449.828125)(343.40078125,450.03125)
\curveto(343.67421875,450.234375)(343.89101563,450.55859375)(344.05117188,451.00390625)
\closepath
\moveto(340.58828125,452.70898438)
\lineto(344.06289063,452.70898438)
\curveto(344.01601563,453.23242188)(343.88320313,453.625)(343.66445313,453.88671875)
\curveto(343.32851563,454.29296875)(342.89296875,454.49609375)(342.3578125,454.49609375)
\curveto(341.8734375,454.49609375)(341.46523438,454.33398438)(341.13320313,454.00976562)
\curveto(340.80507813,453.68554688)(340.6234375,453.25195312)(340.58828125,452.70898438)
\closepath
}
}
{
\newrgbcolor{curcolor}{0 0 0}
\pscustom[linestyle=none,fillstyle=solid,fillcolor=curcolor]
{
\newpath
\moveto(350.52578125,449.76757812)
\curveto(350.13515625,449.43554688)(349.75820313,449.20117188)(349.39492188,449.06445312)
\curveto(349.03554688,448.92773438)(348.64882813,448.859375)(348.23476563,448.859375)
\curveto(347.55117188,448.859375)(347.02578125,449.02539062)(346.65859375,449.35742188)
\curveto(346.29140625,449.69335938)(346.1078125,450.12109375)(346.1078125,450.640625)
\curveto(346.1078125,450.9453125)(346.17617188,451.22265625)(346.31289063,451.47265625)
\curveto(346.45351563,451.7265625)(346.63515625,451.9296875)(346.8578125,452.08203125)
\curveto(347.084375,452.234375)(347.33828125,452.34960938)(347.61953125,452.42773438)
\curveto(347.8265625,452.48242188)(348.1390625,452.53515625)(348.55703125,452.5859375)
\curveto(349.40859375,452.6875)(350.03554688,452.80859375)(350.43789063,452.94921875)
\curveto(350.44179688,453.09375)(350.44375,453.18554688)(350.44375,453.22460938)
\curveto(350.44375,453.65429688)(350.34414063,453.95703125)(350.14492188,454.1328125)
\curveto(349.87539063,454.37109375)(349.475,454.49023438)(348.94375,454.49023438)
\curveto(348.44765625,454.49023438)(348.08046875,454.40234375)(347.8421875,454.2265625)
\curveto(347.6078125,454.0546875)(347.43398438,453.74804688)(347.32070313,453.30664062)
\lineto(346.28945313,453.44726562)
\curveto(346.38320313,453.88867188)(346.5375,454.24414062)(346.75234375,454.51367188)
\curveto(346.9671875,454.78710938)(347.27773438,454.99609375)(347.68398438,455.140625)
\curveto(348.09023438,455.2890625)(348.5609375,455.36328125)(349.09609375,455.36328125)
\curveto(349.62734375,455.36328125)(350.05898438,455.30078125)(350.39101563,455.17578125)
\curveto(350.72304688,455.05078125)(350.9671875,454.89257812)(351.1234375,454.70117188)
\curveto(351.2796875,454.51367188)(351.3890625,454.27539062)(351.4515625,453.98632812)
\curveto(351.48671875,453.80664062)(351.50429688,453.48242188)(351.50429688,453.01367188)
\lineto(351.50429688,451.60742188)
\curveto(351.50429688,450.62695312)(351.52578125,450.00585938)(351.56875,449.74414062)
\curveto(351.615625,449.48632812)(351.70546875,449.23828125)(351.83828125,449)
\lineto(350.73671875,449)
\curveto(350.62734375,449.21875)(350.55703125,449.47460938)(350.52578125,449.76757812)
\closepath
\moveto(350.43789063,452.12304688)
\curveto(350.05507813,451.96679688)(349.48085938,451.83398438)(348.71523438,451.72460938)
\curveto(348.28164063,451.66210938)(347.975,451.59179688)(347.7953125,451.51367188)
\curveto(347.615625,451.43554688)(347.47695313,451.3203125)(347.37929688,451.16796875)
\curveto(347.28164063,451.01953125)(347.2328125,450.85351562)(347.2328125,450.66992188)
\curveto(347.2328125,450.38867188)(347.33828125,450.15429688)(347.54921875,449.96679688)
\curveto(347.7640625,449.77929688)(348.0765625,449.68554688)(348.48671875,449.68554688)
\curveto(348.89296875,449.68554688)(349.25429688,449.7734375)(349.57070313,449.94921875)
\curveto(349.88710938,450.12890625)(350.11953125,450.37304688)(350.26796875,450.68164062)
\curveto(350.38125,450.91992188)(350.43789063,451.27148438)(350.43789063,451.73632812)
\closepath
}
}
{
\newrgbcolor{curcolor}{0 0 0}
\pscustom[linestyle=none,fillstyle=solid,fillcolor=curcolor]
{
\newpath
\moveto(357.17617188,449)
\lineto(357.17617188,449.78515625)
\curveto(356.78164063,449.16796875)(356.2015625,448.859375)(355.4359375,448.859375)
\curveto(354.93984375,448.859375)(354.4828125,448.99609375)(354.06484375,449.26953125)
\curveto(353.65078125,449.54296875)(353.32851563,449.92382812)(353.09804688,450.41210938)
\curveto(352.87148438,450.90429688)(352.75820313,451.46875)(352.75820313,452.10546875)
\curveto(352.75820313,452.7265625)(352.86171875,453.2890625)(353.06875,453.79296875)
\curveto(353.27578125,454.30078125)(353.58632813,454.68945312)(354.00039063,454.95898438)
\curveto(354.41445313,455.22851562)(354.87734375,455.36328125)(355.3890625,455.36328125)
\curveto(355.7640625,455.36328125)(356.09804688,455.28320312)(356.39101563,455.12304688)
\curveto(356.68398438,454.96679688)(356.92226563,454.76171875)(357.10585938,454.5078125)
\lineto(357.10585938,457.58984375)
\lineto(358.1546875,457.58984375)
\lineto(358.1546875,449)
\closepath
\moveto(353.8421875,452.10546875)
\curveto(353.8421875,451.30859375)(354.01015625,450.71289062)(354.34609375,450.31835938)
\curveto(354.68203125,449.92382812)(355.07851563,449.7265625)(355.53554688,449.7265625)
\curveto(355.99648438,449.7265625)(356.38710938,449.9140625)(356.70742188,450.2890625)
\curveto(357.03164063,450.66796875)(357.19375,451.24414062)(357.19375,452.01757812)
\curveto(357.19375,452.86914062)(357.0296875,453.49414062)(356.7015625,453.89257812)
\curveto(356.3734375,454.29101562)(355.96914063,454.49023438)(355.48867188,454.49023438)
\curveto(355.01992188,454.49023438)(354.62734375,454.29882812)(354.3109375,453.91601562)
\curveto(353.9984375,453.53320312)(353.8421875,452.9296875)(353.8421875,452.10546875)
\closepath
}
}
{
\newrgbcolor{curcolor}{0 0 0}
\pscustom[linestyle=none,fillstyle=solid,fillcolor=curcolor]
{
\newpath
\moveto(359.39101563,450.85742188)
\lineto(360.43398438,451.02148438)
\curveto(360.49257813,450.60351562)(360.6546875,450.28320312)(360.9203125,450.06054688)
\curveto(361.18984375,449.83789062)(361.56484375,449.7265625)(362.0453125,449.7265625)
\curveto(362.5296875,449.7265625)(362.8890625,449.82421875)(363.1234375,450.01953125)
\curveto(363.3578125,450.21875)(363.475,450.45117188)(363.475,450.71679688)
\curveto(363.475,450.95507812)(363.37148438,451.14257812)(363.16445313,451.27929688)
\curveto(363.01992188,451.37304688)(362.66054688,451.4921875)(362.08632813,451.63671875)
\curveto(361.31289063,451.83203125)(360.77578125,452)(360.475,452.140625)
\curveto(360.178125,452.28515625)(359.9515625,452.48242188)(359.7953125,452.73242188)
\curveto(359.64296875,452.98632812)(359.56679688,453.265625)(359.56679688,453.5703125)
\curveto(359.56679688,453.84765625)(359.62929688,454.10351562)(359.75429688,454.33789062)
\curveto(359.88320313,454.57617188)(360.05703125,454.7734375)(360.27578125,454.9296875)
\curveto(360.43984375,455.05078125)(360.6625,455.15234375)(360.94375,455.234375)
\curveto(361.22890625,455.3203125)(361.53359375,455.36328125)(361.8578125,455.36328125)
\curveto(362.34609375,455.36328125)(362.77382813,455.29296875)(363.14101563,455.15234375)
\curveto(363.51210938,455.01171875)(363.78554688,454.8203125)(363.96132813,454.578125)
\curveto(364.13710938,454.33984375)(364.25820313,454.01953125)(364.32460938,453.6171875)
\lineto(363.29335938,453.4765625)
\curveto(363.24648438,453.796875)(363.10976563,454.046875)(362.88320313,454.2265625)
\curveto(362.66054688,454.40625)(362.34414063,454.49609375)(361.93398438,454.49609375)
\curveto(361.44960938,454.49609375)(361.10390625,454.41601562)(360.896875,454.25585938)
\curveto(360.68984375,454.09570312)(360.58632813,453.90820312)(360.58632813,453.69335938)
\curveto(360.58632813,453.55664062)(360.62929688,453.43359375)(360.71523438,453.32421875)
\curveto(360.80117188,453.2109375)(360.9359375,453.1171875)(361.11953125,453.04296875)
\curveto(361.225,453.00390625)(361.53554688,452.9140625)(362.05117188,452.7734375)
\curveto(362.79726563,452.57421875)(363.31679688,452.41015625)(363.60976563,452.28125)
\curveto(363.90664063,452.15625)(364.1390625,451.97265625)(364.30703125,451.73046875)
\curveto(364.475,451.48828125)(364.55898438,451.1875)(364.55898438,450.828125)
\curveto(364.55898438,450.4765625)(364.45546875,450.14453125)(364.2484375,449.83203125)
\curveto(364.0453125,449.5234375)(363.75039063,449.28320312)(363.36367188,449.11132812)
\curveto(362.97695313,448.94335938)(362.53945313,448.859375)(362.05117188,448.859375)
\curveto(361.24257813,448.859375)(360.62539063,449.02734375)(360.19960938,449.36328125)
\curveto(359.77773438,449.69921875)(359.50820313,450.19726562)(359.39101563,450.85742188)
\closepath
}
}
{
\newrgbcolor{curcolor}{0 0 0}
\pscustom[linestyle=none,fillstyle=solid,fillcolor=curcolor]
{
\newpath
\moveto(369.39882813,449)
\lineto(369.39882813,454.40234375)
\lineto(368.4671875,454.40234375)
\lineto(368.4671875,455.22265625)
\lineto(369.39882813,455.22265625)
\lineto(369.39882813,455.88476562)
\curveto(369.39882813,456.30273438)(369.4359375,456.61328125)(369.51015625,456.81640625)
\curveto(369.61171875,457.08984375)(369.78945313,457.31054688)(370.04335938,457.47851562)
\curveto(370.30117188,457.65039062)(370.66054688,457.73632812)(371.12148438,457.73632812)
\curveto(371.41835938,457.73632812)(371.74648438,457.70117188)(372.10585938,457.63085938)
\lineto(371.94765625,456.7109375)
\curveto(371.72890625,456.75)(371.521875,456.76953125)(371.3265625,456.76953125)
\curveto(371.00625,456.76953125)(370.7796875,456.70117188)(370.646875,456.56445312)
\curveto(370.5140625,456.42773438)(370.44765625,456.171875)(370.44765625,455.796875)
\lineto(370.44765625,455.22265625)
\lineto(371.66054688,455.22265625)
\lineto(371.66054688,454.40234375)
\lineto(370.44765625,454.40234375)
\lineto(370.44765625,449)
\closepath
}
}
{
\newrgbcolor{curcolor}{0 0 0}
\pscustom[linestyle=none,fillstyle=solid,fillcolor=curcolor]
{
\newpath
\moveto(372.48671875,456.37695312)
\lineto(372.48671875,457.58984375)
\lineto(373.54140625,457.58984375)
\lineto(373.54140625,456.37695312)
\closepath
\moveto(372.48671875,449)
\lineto(372.48671875,455.22265625)
\lineto(373.54140625,455.22265625)
\lineto(373.54140625,449)
\closepath
}
}
{
\newrgbcolor{curcolor}{0 0 0}
\pscustom[linestyle=none,fillstyle=solid,fillcolor=curcolor]
{
\newpath
\moveto(374.75429688,452.11132812)
\curveto(374.75429688,453.26367188)(375.07460938,454.1171875)(375.71523438,454.671875)
\curveto(376.25039063,455.1328125)(376.90273438,455.36328125)(377.67226563,455.36328125)
\curveto(378.52773438,455.36328125)(379.22695313,455.08203125)(379.76992188,454.51953125)
\curveto(380.31289063,453.9609375)(380.584375,453.1875)(380.584375,452.19921875)
\curveto(380.584375,451.3984375)(380.46328125,450.76757812)(380.22109375,450.30664062)
\curveto(379.9828125,449.84960938)(379.63320313,449.49414062)(379.17226563,449.24023438)
\curveto(378.71523438,448.98632812)(378.21523438,448.859375)(377.67226563,448.859375)
\curveto(376.80117188,448.859375)(376.09609375,449.13867188)(375.55703125,449.69726562)
\curveto(375.021875,450.25585938)(374.75429688,451.06054688)(374.75429688,452.11132812)
\closepath
\moveto(375.83828125,452.11132812)
\curveto(375.83828125,451.31445312)(376.01210938,450.71679688)(376.35976563,450.31835938)
\curveto(376.70742188,449.92382812)(377.14492188,449.7265625)(377.67226563,449.7265625)
\curveto(378.19570313,449.7265625)(378.63125,449.92578125)(378.97890625,450.32421875)
\curveto(379.3265625,450.72265625)(379.50039063,451.33007812)(379.50039063,452.14648438)
\curveto(379.50039063,452.91601562)(379.32460938,453.49804688)(378.97304688,453.89257812)
\curveto(378.62539063,454.29101562)(378.19179688,454.49023438)(377.67226563,454.49023438)
\curveto(377.14492188,454.49023438)(376.70742188,454.29296875)(376.35976563,453.8984375)
\curveto(376.01210938,453.50390625)(375.83828125,452.90820312)(375.83828125,452.11132812)
\closepath
}
}
{
\newrgbcolor{curcolor}{0 0 0}
\pscustom[linestyle=none,fillstyle=solid,fillcolor=curcolor]
{
\newpath
\moveto(387.1703125,446.47460938)
\curveto(386.58828125,447.20898438)(386.09609375,448.06835938)(385.69375,449.05273438)
\curveto(385.29140625,450.03710938)(385.09023438,451.05664062)(385.09023438,452.11132812)
\curveto(385.09023438,453.04101562)(385.240625,453.93164062)(385.54140625,454.78320312)
\curveto(385.89296875,455.77148438)(386.4359375,456.75585938)(387.1703125,457.73632812)
\lineto(387.92617188,457.73632812)
\curveto(387.45351563,456.92382812)(387.14101563,456.34375)(386.98867188,455.99609375)
\curveto(386.75039063,455.45703125)(386.56289063,454.89453125)(386.42617188,454.30859375)
\curveto(386.25820313,453.578125)(386.17421875,452.84375)(386.17421875,452.10546875)
\curveto(386.17421875,450.2265625)(386.75820313,448.34960938)(387.92617188,446.47460938)
\closepath
}
}
{
\newrgbcolor{curcolor}{0 0 0}
\pscustom[linestyle=none,fillstyle=solid,fillcolor=curcolor]
{
\newpath
\moveto(389.28554688,449)
\lineto(389.28554688,457.58984375)
\lineto(392.24453125,457.58984375)
\curveto(392.9125,457.58984375)(393.42226563,457.54882812)(393.77382813,457.46679688)
\curveto(394.26601563,457.35351562)(394.6859375,457.1484375)(395.03359375,456.8515625)
\curveto(395.48671875,456.46875)(395.82460938,455.97851562)(396.04726563,455.38085938)
\curveto(396.27382813,454.78710938)(396.38710938,454.10742188)(396.38710938,453.34179688)
\curveto(396.38710938,452.68945312)(396.3109375,452.11132812)(396.15859375,451.60742188)
\curveto(396.00625,451.10351562)(395.8109375,450.68554688)(395.57265625,450.35351562)
\curveto(395.334375,450.02539062)(395.07265625,449.765625)(394.7875,449.57421875)
\curveto(394.50625,449.38671875)(394.16445313,449.24414062)(393.76210938,449.14648438)
\curveto(393.36367188,449.04882812)(392.9046875,449)(392.38515625,449)
\closepath
\moveto(390.42226563,450.01367188)
\lineto(392.25625,450.01367188)
\curveto(392.82265625,450.01367188)(393.26601563,450.06640625)(393.58632813,450.171875)
\curveto(393.91054688,450.27734375)(394.16835938,450.42578125)(394.35976563,450.6171875)
\curveto(394.62929688,450.88671875)(394.83828125,451.24804688)(394.98671875,451.70117188)
\curveto(395.1390625,452.15820312)(395.21523438,452.7109375)(395.21523438,453.359375)
\curveto(395.21523438,454.2578125)(395.06679688,454.94726562)(394.76992188,455.42773438)
\curveto(394.47695313,455.91210938)(394.11953125,456.23632812)(393.69765625,456.40039062)
\curveto(393.39296875,456.51757812)(392.90273438,456.57617188)(392.22695313,456.57617188)
\lineto(390.42226563,456.57617188)
\closepath
}
}
{
\newrgbcolor{curcolor}{0 0 0}
\pscustom[linestyle=none,fillstyle=solid,fillcolor=curcolor]
{
\newpath
\moveto(401.87734375,449.76757812)
\curveto(401.48671875,449.43554688)(401.10976563,449.20117188)(400.74648438,449.06445312)
\curveto(400.38710938,448.92773438)(400.00039063,448.859375)(399.58632813,448.859375)
\curveto(398.90273438,448.859375)(398.37734375,449.02539062)(398.01015625,449.35742188)
\curveto(397.64296875,449.69335938)(397.459375,450.12109375)(397.459375,450.640625)
\curveto(397.459375,450.9453125)(397.52773438,451.22265625)(397.66445313,451.47265625)
\curveto(397.80507813,451.7265625)(397.98671875,451.9296875)(398.209375,452.08203125)
\curveto(398.4359375,452.234375)(398.68984375,452.34960938)(398.97109375,452.42773438)
\curveto(399.178125,452.48242188)(399.490625,452.53515625)(399.90859375,452.5859375)
\curveto(400.76015625,452.6875)(401.38710938,452.80859375)(401.78945313,452.94921875)
\curveto(401.79335938,453.09375)(401.7953125,453.18554688)(401.7953125,453.22460938)
\curveto(401.7953125,453.65429688)(401.69570313,453.95703125)(401.49648438,454.1328125)
\curveto(401.22695313,454.37109375)(400.8265625,454.49023438)(400.2953125,454.49023438)
\curveto(399.79921875,454.49023438)(399.43203125,454.40234375)(399.19375,454.2265625)
\curveto(398.959375,454.0546875)(398.78554688,453.74804688)(398.67226563,453.30664062)
\lineto(397.64101563,453.44726562)
\curveto(397.73476563,453.88867188)(397.8890625,454.24414062)(398.10390625,454.51367188)
\curveto(398.31875,454.78710938)(398.62929688,454.99609375)(399.03554688,455.140625)
\curveto(399.44179688,455.2890625)(399.9125,455.36328125)(400.44765625,455.36328125)
\curveto(400.97890625,455.36328125)(401.41054688,455.30078125)(401.74257813,455.17578125)
\curveto(402.07460938,455.05078125)(402.31875,454.89257812)(402.475,454.70117188)
\curveto(402.63125,454.51367188)(402.740625,454.27539062)(402.803125,453.98632812)
\curveto(402.83828125,453.80664062)(402.85585938,453.48242188)(402.85585938,453.01367188)
\lineto(402.85585938,451.60742188)
\curveto(402.85585938,450.62695312)(402.87734375,450.00585938)(402.9203125,449.74414062)
\curveto(402.9671875,449.48632812)(403.05703125,449.23828125)(403.18984375,449)
\lineto(402.08828125,449)
\curveto(401.97890625,449.21875)(401.90859375,449.47460938)(401.87734375,449.76757812)
\closepath
\moveto(401.78945313,452.12304688)
\curveto(401.40664063,451.96679688)(400.83242188,451.83398438)(400.06679688,451.72460938)
\curveto(399.63320313,451.66210938)(399.3265625,451.59179688)(399.146875,451.51367188)
\curveto(398.9671875,451.43554688)(398.82851563,451.3203125)(398.73085938,451.16796875)
\curveto(398.63320313,451.01953125)(398.584375,450.85351562)(398.584375,450.66992188)
\curveto(398.584375,450.38867188)(398.68984375,450.15429688)(398.90078125,449.96679688)
\curveto(399.115625,449.77929688)(399.428125,449.68554688)(399.83828125,449.68554688)
\curveto(400.24453125,449.68554688)(400.60585938,449.7734375)(400.92226563,449.94921875)
\curveto(401.23867188,450.12890625)(401.47109375,450.37304688)(401.61953125,450.68164062)
\curveto(401.7328125,450.91992188)(401.78945313,451.27148438)(401.78945313,451.73632812)
\closepath
}
}
{
\newrgbcolor{curcolor}{0 0 0}
\pscustom[linestyle=none,fillstyle=solid,fillcolor=curcolor]
{
\newpath
\moveto(406.79335938,449.94335938)
\lineto(406.94570313,449.01171875)
\curveto(406.64882813,448.94921875)(406.38320313,448.91796875)(406.14882813,448.91796875)
\curveto(405.76601563,448.91796875)(405.46914063,448.97851562)(405.25820313,449.09960938)
\curveto(405.04726563,449.22070312)(404.89882813,449.37890625)(404.81289063,449.57421875)
\curveto(404.72695313,449.7734375)(404.68398438,450.18945312)(404.68398438,450.82226562)
\lineto(404.68398438,454.40234375)
\lineto(403.91054688,454.40234375)
\lineto(403.91054688,455.22265625)
\lineto(404.68398438,455.22265625)
\lineto(404.68398438,456.76367188)
\lineto(405.7328125,457.39648438)
\lineto(405.7328125,455.22265625)
\lineto(406.79335938,455.22265625)
\lineto(406.79335938,454.40234375)
\lineto(405.7328125,454.40234375)
\lineto(405.7328125,450.76367188)
\curveto(405.7328125,450.46289062)(405.75039063,450.26953125)(405.78554688,450.18359375)
\curveto(405.82460938,450.09765625)(405.88515625,450.02929688)(405.9671875,449.97851562)
\curveto(406.053125,449.92773438)(406.17421875,449.90234375)(406.33046875,449.90234375)
\curveto(406.44765625,449.90234375)(406.60195313,449.91601562)(406.79335938,449.94335938)
\closepath
}
}
{
\newrgbcolor{curcolor}{0 0 0}
\pscustom[linestyle=none,fillstyle=solid,fillcolor=curcolor]
{
\newpath
\moveto(411.88515625,449.76757812)
\curveto(411.49453125,449.43554688)(411.11757813,449.20117188)(410.75429688,449.06445312)
\curveto(410.39492188,448.92773438)(410.00820313,448.859375)(409.59414063,448.859375)
\curveto(408.91054688,448.859375)(408.38515625,449.02539062)(408.01796875,449.35742188)
\curveto(407.65078125,449.69335938)(407.4671875,450.12109375)(407.4671875,450.640625)
\curveto(407.4671875,450.9453125)(407.53554688,451.22265625)(407.67226563,451.47265625)
\curveto(407.81289063,451.7265625)(407.99453125,451.9296875)(408.2171875,452.08203125)
\curveto(408.44375,452.234375)(408.69765625,452.34960938)(408.97890625,452.42773438)
\curveto(409.1859375,452.48242188)(409.4984375,452.53515625)(409.91640625,452.5859375)
\curveto(410.76796875,452.6875)(411.39492188,452.80859375)(411.79726563,452.94921875)
\curveto(411.80117188,453.09375)(411.803125,453.18554688)(411.803125,453.22460938)
\curveto(411.803125,453.65429688)(411.70351563,453.95703125)(411.50429688,454.1328125)
\curveto(411.23476563,454.37109375)(410.834375,454.49023438)(410.303125,454.49023438)
\curveto(409.80703125,454.49023438)(409.43984375,454.40234375)(409.2015625,454.2265625)
\curveto(408.9671875,454.0546875)(408.79335938,453.74804688)(408.68007813,453.30664062)
\lineto(407.64882813,453.44726562)
\curveto(407.74257813,453.88867188)(407.896875,454.24414062)(408.11171875,454.51367188)
\curveto(408.3265625,454.78710938)(408.63710938,454.99609375)(409.04335938,455.140625)
\curveto(409.44960938,455.2890625)(409.9203125,455.36328125)(410.45546875,455.36328125)
\curveto(410.98671875,455.36328125)(411.41835938,455.30078125)(411.75039063,455.17578125)
\curveto(412.08242188,455.05078125)(412.3265625,454.89257812)(412.4828125,454.70117188)
\curveto(412.6390625,454.51367188)(412.7484375,454.27539062)(412.8109375,453.98632812)
\curveto(412.84609375,453.80664062)(412.86367188,453.48242188)(412.86367188,453.01367188)
\lineto(412.86367188,451.60742188)
\curveto(412.86367188,450.62695312)(412.88515625,450.00585938)(412.928125,449.74414062)
\curveto(412.975,449.48632812)(413.06484375,449.23828125)(413.19765625,449)
\lineto(412.09609375,449)
\curveto(411.98671875,449.21875)(411.91640625,449.47460938)(411.88515625,449.76757812)
\closepath
\moveto(411.79726563,452.12304688)
\curveto(411.41445313,451.96679688)(410.84023438,451.83398438)(410.07460938,451.72460938)
\curveto(409.64101563,451.66210938)(409.334375,451.59179688)(409.1546875,451.51367188)
\curveto(408.975,451.43554688)(408.83632813,451.3203125)(408.73867188,451.16796875)
\curveto(408.64101563,451.01953125)(408.5921875,450.85351562)(408.5921875,450.66992188)
\curveto(408.5921875,450.38867188)(408.69765625,450.15429688)(408.90859375,449.96679688)
\curveto(409.1234375,449.77929688)(409.4359375,449.68554688)(409.84609375,449.68554688)
\curveto(410.25234375,449.68554688)(410.61367188,449.7734375)(410.93007813,449.94921875)
\curveto(411.24648438,450.12890625)(411.47890625,450.37304688)(411.62734375,450.68164062)
\curveto(411.740625,450.91992188)(411.79726563,451.27148438)(411.79726563,451.73632812)
\closepath
}
}
{
\newrgbcolor{curcolor}{0 0 0}
\pscustom[linestyle=none,fillstyle=solid,fillcolor=curcolor]
{
\newpath
\moveto(417.95546875,449)
\lineto(417.95546875,457.58984375)
\lineto(419.12148438,457.58984375)
\lineto(423.63320313,450.84570312)
\lineto(423.63320313,457.58984375)
\lineto(424.72304688,457.58984375)
\lineto(424.72304688,449)
\lineto(423.55703125,449)
\lineto(419.0453125,455.75)
\lineto(419.0453125,449)
\closepath
}
}
{
\newrgbcolor{curcolor}{0 0 0}
\pscustom[linestyle=none,fillstyle=solid,fillcolor=curcolor]
{
\newpath
\moveto(426.10585938,452.11132812)
\curveto(426.10585938,453.26367188)(426.42617188,454.1171875)(427.06679688,454.671875)
\curveto(427.60195313,455.1328125)(428.25429688,455.36328125)(429.02382813,455.36328125)
\curveto(429.87929688,455.36328125)(430.57851563,455.08203125)(431.12148438,454.51953125)
\curveto(431.66445313,453.9609375)(431.9359375,453.1875)(431.9359375,452.19921875)
\curveto(431.9359375,451.3984375)(431.81484375,450.76757812)(431.57265625,450.30664062)
\curveto(431.334375,449.84960938)(430.98476563,449.49414062)(430.52382813,449.24023438)
\curveto(430.06679688,448.98632812)(429.56679688,448.859375)(429.02382813,448.859375)
\curveto(428.15273438,448.859375)(427.44765625,449.13867188)(426.90859375,449.69726562)
\curveto(426.3734375,450.25585938)(426.10585938,451.06054688)(426.10585938,452.11132812)
\closepath
\moveto(427.18984375,452.11132812)
\curveto(427.18984375,451.31445312)(427.36367188,450.71679688)(427.71132813,450.31835938)
\curveto(428.05898438,449.92382812)(428.49648438,449.7265625)(429.02382813,449.7265625)
\curveto(429.54726563,449.7265625)(429.9828125,449.92578125)(430.33046875,450.32421875)
\curveto(430.678125,450.72265625)(430.85195313,451.33007812)(430.85195313,452.14648438)
\curveto(430.85195313,452.91601562)(430.67617188,453.49804688)(430.32460938,453.89257812)
\curveto(429.97695313,454.29101562)(429.54335938,454.49023438)(429.02382813,454.49023438)
\curveto(428.49648438,454.49023438)(428.05898438,454.29296875)(427.71132813,453.8984375)
\curveto(427.36367188,453.50390625)(427.18984375,452.90820312)(427.18984375,452.11132812)
\closepath
}
}
{
\newrgbcolor{curcolor}{0 0 0}
\pscustom[linestyle=none,fillstyle=solid,fillcolor=curcolor]
{
\newpath
\moveto(437.209375,449)
\lineto(437.209375,449.78515625)
\curveto(436.81484375,449.16796875)(436.23476563,448.859375)(435.46914063,448.859375)
\curveto(434.97304688,448.859375)(434.51601563,448.99609375)(434.09804688,449.26953125)
\curveto(433.68398438,449.54296875)(433.36171875,449.92382812)(433.13125,450.41210938)
\curveto(432.9046875,450.90429688)(432.79140625,451.46875)(432.79140625,452.10546875)
\curveto(432.79140625,452.7265625)(432.89492188,453.2890625)(433.10195313,453.79296875)
\curveto(433.30898438,454.30078125)(433.61953125,454.68945312)(434.03359375,454.95898438)
\curveto(434.44765625,455.22851562)(434.91054688,455.36328125)(435.42226563,455.36328125)
\curveto(435.79726563,455.36328125)(436.13125,455.28320312)(436.42421875,455.12304688)
\curveto(436.7171875,454.96679688)(436.95546875,454.76171875)(437.1390625,454.5078125)
\lineto(437.1390625,457.58984375)
\lineto(438.18789063,457.58984375)
\lineto(438.18789063,449)
\closepath
\moveto(433.87539063,452.10546875)
\curveto(433.87539063,451.30859375)(434.04335938,450.71289062)(434.37929688,450.31835938)
\curveto(434.71523438,449.92382812)(435.11171875,449.7265625)(435.56875,449.7265625)
\curveto(436.0296875,449.7265625)(436.4203125,449.9140625)(436.740625,450.2890625)
\curveto(437.06484375,450.66796875)(437.22695313,451.24414062)(437.22695313,452.01757812)
\curveto(437.22695313,452.86914062)(437.06289063,453.49414062)(436.73476563,453.89257812)
\curveto(436.40664063,454.29101562)(436.00234375,454.49023438)(435.521875,454.49023438)
\curveto(435.053125,454.49023438)(434.66054688,454.29882812)(434.34414063,453.91601562)
\curveto(434.03164063,453.53320312)(433.87539063,452.9296875)(433.87539063,452.10546875)
\closepath
}
}
{
\newrgbcolor{curcolor}{0 0 0}
\pscustom[linestyle=none,fillstyle=solid,fillcolor=curcolor]
{
\newpath
\moveto(444.10585938,451.00390625)
\lineto(445.19570313,450.86914062)
\curveto(445.02382813,450.23242188)(444.70546875,449.73828125)(444.240625,449.38671875)
\curveto(443.77578125,449.03515625)(443.18203125,448.859375)(442.459375,448.859375)
\curveto(441.54921875,448.859375)(440.8265625,449.13867188)(440.29140625,449.69726562)
\curveto(439.76015625,450.25976562)(439.49453125,451.046875)(439.49453125,452.05859375)
\curveto(439.49453125,453.10546875)(439.7640625,453.91796875)(440.303125,454.49609375)
\curveto(440.8421875,455.07421875)(441.54140625,455.36328125)(442.40078125,455.36328125)
\curveto(443.2328125,455.36328125)(443.9125,455.08007812)(444.43984375,454.51367188)
\curveto(444.9671875,453.94726562)(445.23085938,453.15039062)(445.23085938,452.12304688)
\curveto(445.23085938,452.06054688)(445.22890625,451.96679688)(445.225,451.84179688)
\lineto(440.584375,451.84179688)
\curveto(440.6234375,451.15820312)(440.81679688,450.63476562)(441.16445313,450.27148438)
\curveto(441.51210938,449.90820312)(441.94570313,449.7265625)(442.46523438,449.7265625)
\curveto(442.85195313,449.7265625)(443.18203125,449.828125)(443.45546875,450.03125)
\curveto(443.72890625,450.234375)(443.94570313,450.55859375)(444.10585938,451.00390625)
\closepath
\moveto(440.64296875,452.70898438)
\lineto(444.11757813,452.70898438)
\curveto(444.07070313,453.23242188)(443.93789063,453.625)(443.71914063,453.88671875)
\curveto(443.38320313,454.29296875)(442.94765625,454.49609375)(442.4125,454.49609375)
\curveto(441.928125,454.49609375)(441.51992188,454.33398438)(441.18789063,454.00976562)
\curveto(440.85976563,453.68554688)(440.678125,453.25195312)(440.64296875,452.70898438)
\closepath
}
}
{
\newrgbcolor{curcolor}{0 0 0}
\pscustom[linestyle=none,fillstyle=solid,fillcolor=curcolor]
{
\newpath
\moveto(449.5609375,453.23632812)
\curveto(449.5609375,454.25195312)(449.66445313,455.06835938)(449.87148438,455.68554688)
\curveto(450.08242188,456.30664062)(450.39296875,456.78515625)(450.803125,457.12109375)
\curveto(451.2171875,457.45703125)(451.73671875,457.625)(452.36171875,457.625)
\curveto(452.82265625,457.625)(453.22695313,457.53125)(453.57460938,457.34375)
\curveto(453.92226563,457.16015625)(454.209375,456.89257812)(454.4359375,456.54101562)
\curveto(454.6625,456.19335938)(454.84023438,455.76757812)(454.96914063,455.26367188)
\curveto(455.09804688,454.76367188)(455.1625,454.08789062)(455.1625,453.23632812)
\curveto(455.1625,452.22851562)(455.05898438,451.4140625)(454.85195313,450.79296875)
\curveto(454.64492188,450.17578125)(454.334375,449.69726562)(453.9203125,449.35742188)
\curveto(453.51015625,449.02148438)(452.990625,448.85351562)(452.36171875,448.85351562)
\curveto(451.53359375,448.85351562)(450.88320313,449.15039062)(450.41054688,449.74414062)
\curveto(449.84414063,450.45898438)(449.5609375,451.62304688)(449.5609375,453.23632812)
\closepath
\moveto(450.64492188,453.23632812)
\curveto(450.64492188,451.82617188)(450.80898438,450.88671875)(451.13710938,450.41796875)
\curveto(451.46914063,449.953125)(451.87734375,449.72070312)(452.36171875,449.72070312)
\curveto(452.84609375,449.72070312)(453.25234375,449.95507812)(453.58046875,450.42382812)
\curveto(453.9125,450.89257812)(454.07851563,451.83007812)(454.07851563,453.23632812)
\curveto(454.07851563,454.65039062)(453.9125,455.58984375)(453.58046875,456.0546875)
\curveto(453.25234375,456.51953125)(452.8421875,456.75195312)(452.35,456.75195312)
\curveto(451.865625,456.75195312)(451.47890625,456.546875)(451.18984375,456.13671875)
\curveto(450.8265625,455.61328125)(450.64492188,454.64648438)(450.64492188,453.23632812)
\closepath
}
}
{
\newrgbcolor{curcolor}{0 0 0}
\pscustom[linestyle=none,fillstyle=solid,fillcolor=curcolor]
{
\newpath
\moveto(457.21914063,446.47460938)
\lineto(456.46328125,446.47460938)
\curveto(457.63125,448.34960938)(458.21523438,450.2265625)(458.21523438,452.10546875)
\curveto(458.21523438,452.83984375)(458.13125,453.56835938)(457.96328125,454.29101562)
\curveto(457.83046875,454.87695312)(457.64492188,455.43945312)(457.40664063,455.97851562)
\curveto(457.25429688,456.33007812)(456.93984375,456.91601562)(456.46328125,457.73632812)
\lineto(457.21914063,457.73632812)
\curveto(457.95351563,456.75585938)(458.49648438,455.77148438)(458.84804688,454.78320312)
\curveto(459.14882813,453.93164062)(459.29921875,453.04101562)(459.29921875,452.11132812)
\curveto(459.29921875,451.05664062)(459.09609375,450.03710938)(458.68984375,449.05273438)
\curveto(458.2875,448.06835938)(457.79726563,447.20898438)(457.21914063,446.47460938)
\closepath
}
}
{
\newrgbcolor{curcolor}{0 0 0}
\pscustom[linestyle=none,fillstyle=solid,fillcolor=curcolor]
{
\newpath
\moveto(72.61015625,404)
\lineto(72.61015625,412.58984375)
\lineto(73.77617188,412.58984375)
\lineto(78.28789063,405.84570312)
\lineto(78.28789063,412.58984375)
\lineto(79.37773438,412.58984375)
\lineto(79.37773438,404)
\lineto(78.21171875,404)
\lineto(73.7,410.75)
\lineto(73.7,404)
\closepath
}
}
{
\newrgbcolor{curcolor}{0 0 0}
\pscustom[linestyle=none,fillstyle=solid,fillcolor=curcolor]
{
\newpath
\moveto(85.41289063,406.00390625)
\lineto(86.50273438,405.86914062)
\curveto(86.33085938,405.23242188)(86.0125,404.73828125)(85.54765625,404.38671875)
\curveto(85.0828125,404.03515625)(84.4890625,403.859375)(83.76640625,403.859375)
\curveto(82.85625,403.859375)(82.13359375,404.13867188)(81.5984375,404.69726562)
\curveto(81.0671875,405.25976562)(80.8015625,406.046875)(80.8015625,407.05859375)
\curveto(80.8015625,408.10546875)(81.07109375,408.91796875)(81.61015625,409.49609375)
\curveto(82.14921875,410.07421875)(82.8484375,410.36328125)(83.7078125,410.36328125)
\curveto(84.53984375,410.36328125)(85.21953125,410.08007812)(85.746875,409.51367188)
\curveto(86.27421875,408.94726562)(86.53789063,408.15039062)(86.53789063,407.12304688)
\curveto(86.53789063,407.06054688)(86.5359375,406.96679688)(86.53203125,406.84179688)
\lineto(81.89140625,406.84179688)
\curveto(81.93046875,406.15820312)(82.12382813,405.63476562)(82.47148438,405.27148438)
\curveto(82.81914063,404.90820312)(83.25273438,404.7265625)(83.77226563,404.7265625)
\curveto(84.15898438,404.7265625)(84.4890625,404.828125)(84.7625,405.03125)
\curveto(85.0359375,405.234375)(85.25273438,405.55859375)(85.41289063,406.00390625)
\closepath
\moveto(81.95,407.70898438)
\lineto(85.42460938,407.70898438)
\curveto(85.37773438,408.23242188)(85.24492188,408.625)(85.02617188,408.88671875)
\curveto(84.69023438,409.29296875)(84.2546875,409.49609375)(83.71953125,409.49609375)
\curveto(83.23515625,409.49609375)(82.82695313,409.33398438)(82.49492188,409.00976562)
\curveto(82.16679688,408.68554688)(81.98515625,408.25195312)(81.95,407.70898438)
\closepath
}
}
{
\newrgbcolor{curcolor}{0 0 0}
\pscustom[linestyle=none,fillstyle=solid,fillcolor=curcolor]
{
\newpath
\moveto(88.97539063,404)
\lineto(87.07109375,410.22265625)
\lineto(88.1609375,410.22265625)
\lineto(89.15117188,406.63085938)
\lineto(89.5203125,405.29492188)
\curveto(89.5359375,405.36132812)(89.64335938,405.7890625)(89.84257813,406.578125)
\lineto(90.8328125,410.22265625)
\lineto(91.91679688,410.22265625)
\lineto(92.8484375,406.61328125)
\lineto(93.15898438,405.42382812)
\lineto(93.51640625,406.625)
\lineto(94.5828125,410.22265625)
\lineto(95.60820313,410.22265625)
\lineto(93.66289063,404)
\lineto(92.5671875,404)
\lineto(91.57695313,407.7265625)
\lineto(91.33671875,408.78710938)
\lineto(90.07695313,404)
\closepath
}
}
{
\newrgbcolor{curcolor}{0 0 0}
\pscustom[linestyle=none,fillstyle=solid,fillcolor=curcolor]
{
\newpath
\moveto(99.27617188,404)
\lineto(99.27617188,405.0546875)
\lineto(103.6765625,410.55664062)
\curveto(103.9890625,410.94726562)(104.2859375,411.28710938)(104.5671875,411.57617188)
\lineto(99.77421875,411.57617188)
\lineto(99.77421875,412.58984375)
\lineto(105.9265625,412.58984375)
\lineto(105.9265625,411.57617188)
\lineto(101.10429688,405.6171875)
\lineto(100.5828125,405.01367188)
\lineto(106.0671875,405.01367188)
\lineto(106.0671875,404)
\closepath
}
}
{
\newrgbcolor{curcolor}{0 0 0}
\pscustom[linestyle=none,fillstyle=solid,fillcolor=curcolor]
{
\newpath
\moveto(107.35039063,404)
\lineto(107.35039063,412.58984375)
\lineto(113.1453125,412.58984375)
\lineto(113.1453125,411.57617188)
\lineto(108.48710938,411.57617188)
\lineto(108.48710938,408.91601562)
\lineto(112.51835938,408.91601562)
\lineto(112.51835938,407.90234375)
\lineto(108.48710938,407.90234375)
\lineto(108.48710938,404)
\closepath
}
}
{
\newrgbcolor{curcolor}{0 0 0}
\pscustom[linestyle=none,fillstyle=solid,fillcolor=curcolor]
{
\newpath
\moveto(114.23515625,406.75976562)
\lineto(115.30742188,406.85351562)
\curveto(115.35820313,406.42382812)(115.47539063,406.0703125)(115.65898438,405.79296875)
\curveto(115.84648438,405.51953125)(116.13554688,405.296875)(116.52617188,405.125)
\curveto(116.91679688,404.95703125)(117.35625,404.87304688)(117.84453125,404.87304688)
\curveto(118.278125,404.87304688)(118.6609375,404.9375)(118.99296875,405.06640625)
\curveto(119.325,405.1953125)(119.57109375,405.37109375)(119.73125,405.59375)
\curveto(119.8953125,405.8203125)(119.97734375,406.06640625)(119.97734375,406.33203125)
\curveto(119.97734375,406.6015625)(119.89921875,406.8359375)(119.74296875,407.03515625)
\curveto(119.58671875,407.23828125)(119.32890625,407.40820312)(118.96953125,407.54492188)
\curveto(118.7390625,407.63476562)(118.22929688,407.7734375)(117.44023438,407.9609375)
\curveto(116.65117188,408.15234375)(116.0984375,408.33203125)(115.78203125,408.5)
\curveto(115.371875,408.71484375)(115.06523438,408.98046875)(114.86210938,409.296875)
\curveto(114.66289063,409.6171875)(114.56328125,409.97460938)(114.56328125,410.36914062)
\curveto(114.56328125,410.80273438)(114.68632813,411.20703125)(114.93242188,411.58203125)
\curveto(115.17851563,411.9609375)(115.53789063,412.24804688)(116.01054688,412.44335938)
\curveto(116.48320313,412.63867188)(117.00859375,412.73632812)(117.58671875,412.73632812)
\curveto(118.2234375,412.73632812)(118.78398438,412.6328125)(119.26835938,412.42578125)
\curveto(119.75664063,412.22265625)(120.13164063,411.921875)(120.39335938,411.5234375)
\curveto(120.65507813,411.125)(120.79570313,410.67382812)(120.81523438,410.16992188)
\lineto(119.72539063,410.08789062)
\curveto(119.66679688,410.63085938)(119.46757813,411.04101562)(119.12773438,411.31835938)
\curveto(118.79179688,411.59570312)(118.29375,411.734375)(117.63359375,411.734375)
\curveto(116.94609375,411.734375)(116.44414063,411.60742188)(116.12773438,411.35351562)
\curveto(115.81523438,411.10351562)(115.65898438,410.80078125)(115.65898438,410.4453125)
\curveto(115.65898438,410.13671875)(115.7703125,409.8828125)(115.99296875,409.68359375)
\curveto(116.21171875,409.484375)(116.78203125,409.27929688)(117.70390625,409.06835938)
\curveto(118.6296875,408.86132812)(119.26445313,408.6796875)(119.60820313,408.5234375)
\curveto(120.10820313,408.29296875)(120.47734375,408)(120.715625,407.64453125)
\curveto(120.95390625,407.29296875)(121.07304688,406.88671875)(121.07304688,406.42578125)
\curveto(121.07304688,405.96875)(120.9421875,405.53710938)(120.68046875,405.13085938)
\curveto(120.41875,404.72851562)(120.04179688,404.4140625)(119.54960938,404.1875)
\curveto(119.06132813,403.96484375)(118.51054688,403.85351562)(117.89726563,403.85351562)
\curveto(117.11992188,403.85351562)(116.46757813,403.96679688)(115.94023438,404.19335938)
\curveto(115.41679688,404.41992188)(115.0046875,404.75976562)(114.70390625,405.21289062)
\curveto(114.40703125,405.66992188)(114.25078125,406.18554688)(114.23515625,406.75976562)
\closepath
}
}
{
\newrgbcolor{curcolor}{1 0 0}
\pscustom[linewidth=1,linecolor=curcolor]
{
\newpath
\moveto(130,407.9)
\lineto(172.2,407.9)
\moveto(55.3,290.7)
\lineto(141.9,292.4)
\lineto(228.5,139.3)
\lineto(315.2,140.3)
\lineto(401.8,137.3)
\lineto(488.4,139.3)
\lineto(575,140.3)
}
}
{
\newrgbcolor{curcolor}{0 0 0}
\pscustom[linestyle=none,fillstyle=solid,fillcolor=curcolor]
{
\newpath
\moveto(77.60820313,390.18359375)
\curveto(77.60820313,391.609375)(77.99101563,392.72460938)(78.75664063,393.52929688)
\curveto(79.52226563,394.33789062)(80.51054688,394.7421875)(81.72148438,394.7421875)
\curveto(82.51445313,394.7421875)(83.22929688,394.55273438)(83.86601563,394.17382812)
\curveto(84.50273438,393.79492188)(84.98710938,393.265625)(85.31914063,392.5859375)
\curveto(85.65507813,391.91015625)(85.82304688,391.14257812)(85.82304688,390.28320312)
\curveto(85.82304688,389.41210938)(85.64726563,388.6328125)(85.29570313,387.9453125)
\curveto(84.94414063,387.2578125)(84.44609375,386.73632812)(83.8015625,386.38085938)
\curveto(83.15703125,386.02929688)(82.46171875,385.85351562)(81.715625,385.85351562)
\curveto(80.90703125,385.85351562)(80.184375,386.04882812)(79.54765625,386.43945312)
\curveto(78.9109375,386.83007812)(78.42851563,387.36328125)(78.10039063,388.0390625)
\curveto(77.77226563,388.71484375)(77.60820313,389.4296875)(77.60820313,390.18359375)
\closepath
\moveto(78.78007813,390.16601562)
\curveto(78.78007813,389.13085938)(79.05742188,388.31445312)(79.61210938,387.71679688)
\curveto(80.17070313,387.12304688)(80.86992188,386.82617188)(81.70976563,386.82617188)
\curveto(82.56523438,386.82617188)(83.26835938,387.12695312)(83.81914063,387.72851562)
\curveto(84.37382813,388.33007812)(84.65117188,389.18359375)(84.65117188,390.2890625)
\curveto(84.65117188,390.98828125)(84.53203125,391.59765625)(84.29375,392.1171875)
\curveto(84.059375,392.640625)(83.71367188,393.04492188)(83.25664063,393.33007812)
\curveto(82.80351563,393.61914062)(82.29375,393.76367188)(81.72734375,393.76367188)
\curveto(80.92265625,393.76367188)(80.22929688,393.48632812)(79.64726563,392.93164062)
\curveto(79.06914063,392.38085938)(78.78007813,391.45898438)(78.78007813,390.16601562)
\closepath
}
}
{
\newrgbcolor{curcolor}{0 0 0}
\pscustom[linestyle=none,fillstyle=solid,fillcolor=curcolor]
{
\newpath
\moveto(87.1296875,386)
\lineto(87.1296875,394.58984375)
\lineto(88.184375,394.58984375)
\lineto(88.184375,386)
\closepath
}
}
{
\newrgbcolor{curcolor}{0 0 0}
\pscustom[linestyle=none,fillstyle=solid,fillcolor=curcolor]
{
\newpath
\moveto(93.85625,386)
\lineto(93.85625,386.78515625)
\curveto(93.46171875,386.16796875)(92.88164063,385.859375)(92.11601563,385.859375)
\curveto(91.61992188,385.859375)(91.16289063,385.99609375)(90.74492188,386.26953125)
\curveto(90.33085938,386.54296875)(90.00859375,386.92382812)(89.778125,387.41210938)
\curveto(89.5515625,387.90429688)(89.43828125,388.46875)(89.43828125,389.10546875)
\curveto(89.43828125,389.7265625)(89.54179688,390.2890625)(89.74882813,390.79296875)
\curveto(89.95585938,391.30078125)(90.26640625,391.68945312)(90.68046875,391.95898438)
\curveto(91.09453125,392.22851562)(91.55742188,392.36328125)(92.06914063,392.36328125)
\curveto(92.44414063,392.36328125)(92.778125,392.28320312)(93.07109375,392.12304688)
\curveto(93.3640625,391.96679688)(93.60234375,391.76171875)(93.7859375,391.5078125)
\lineto(93.7859375,394.58984375)
\lineto(94.83476563,394.58984375)
\lineto(94.83476563,386)
\closepath
\moveto(90.52226563,389.10546875)
\curveto(90.52226563,388.30859375)(90.69023438,387.71289062)(91.02617188,387.31835938)
\curveto(91.36210938,386.92382812)(91.75859375,386.7265625)(92.215625,386.7265625)
\curveto(92.6765625,386.7265625)(93.0671875,386.9140625)(93.3875,387.2890625)
\curveto(93.71171875,387.66796875)(93.87382813,388.24414062)(93.87382813,389.01757812)
\curveto(93.87382813,389.86914062)(93.70976563,390.49414062)(93.38164063,390.89257812)
\curveto(93.05351563,391.29101562)(92.64921875,391.49023438)(92.16875,391.49023438)
\curveto(91.7,391.49023438)(91.30742188,391.29882812)(90.99101563,390.91601562)
\curveto(90.67851563,390.53320312)(90.52226563,389.9296875)(90.52226563,389.10546875)
\closepath
}
}
{
\newrgbcolor{curcolor}{0 0 0}
\pscustom[linestyle=none,fillstyle=solid,fillcolor=curcolor]
{
\newpath
\moveto(99.27617188,386)
\lineto(99.27617188,387.0546875)
\lineto(103.6765625,392.55664062)
\curveto(103.9890625,392.94726562)(104.2859375,393.28710938)(104.5671875,393.57617188)
\lineto(99.77421875,393.57617188)
\lineto(99.77421875,394.58984375)
\lineto(105.9265625,394.58984375)
\lineto(105.9265625,393.57617188)
\lineto(101.10429688,387.6171875)
\lineto(100.5828125,387.01367188)
\lineto(106.0671875,387.01367188)
\lineto(106.0671875,386)
\closepath
}
}
{
\newrgbcolor{curcolor}{0 0 0}
\pscustom[linestyle=none,fillstyle=solid,fillcolor=curcolor]
{
\newpath
\moveto(107.35039063,386)
\lineto(107.35039063,394.58984375)
\lineto(113.1453125,394.58984375)
\lineto(113.1453125,393.57617188)
\lineto(108.48710938,393.57617188)
\lineto(108.48710938,390.91601562)
\lineto(112.51835938,390.91601562)
\lineto(112.51835938,389.90234375)
\lineto(108.48710938,389.90234375)
\lineto(108.48710938,386)
\closepath
}
}
{
\newrgbcolor{curcolor}{0 0 0}
\pscustom[linestyle=none,fillstyle=solid,fillcolor=curcolor]
{
\newpath
\moveto(114.23515625,388.75976562)
\lineto(115.30742188,388.85351562)
\curveto(115.35820313,388.42382812)(115.47539063,388.0703125)(115.65898438,387.79296875)
\curveto(115.84648438,387.51953125)(116.13554688,387.296875)(116.52617188,387.125)
\curveto(116.91679688,386.95703125)(117.35625,386.87304688)(117.84453125,386.87304688)
\curveto(118.278125,386.87304688)(118.6609375,386.9375)(118.99296875,387.06640625)
\curveto(119.325,387.1953125)(119.57109375,387.37109375)(119.73125,387.59375)
\curveto(119.8953125,387.8203125)(119.97734375,388.06640625)(119.97734375,388.33203125)
\curveto(119.97734375,388.6015625)(119.89921875,388.8359375)(119.74296875,389.03515625)
\curveto(119.58671875,389.23828125)(119.32890625,389.40820312)(118.96953125,389.54492188)
\curveto(118.7390625,389.63476562)(118.22929688,389.7734375)(117.44023438,389.9609375)
\curveto(116.65117188,390.15234375)(116.0984375,390.33203125)(115.78203125,390.5)
\curveto(115.371875,390.71484375)(115.06523438,390.98046875)(114.86210938,391.296875)
\curveto(114.66289063,391.6171875)(114.56328125,391.97460938)(114.56328125,392.36914062)
\curveto(114.56328125,392.80273438)(114.68632813,393.20703125)(114.93242188,393.58203125)
\curveto(115.17851563,393.9609375)(115.53789063,394.24804688)(116.01054688,394.44335938)
\curveto(116.48320313,394.63867188)(117.00859375,394.73632812)(117.58671875,394.73632812)
\curveto(118.2234375,394.73632812)(118.78398438,394.6328125)(119.26835938,394.42578125)
\curveto(119.75664063,394.22265625)(120.13164063,393.921875)(120.39335938,393.5234375)
\curveto(120.65507813,393.125)(120.79570313,392.67382812)(120.81523438,392.16992188)
\lineto(119.72539063,392.08789062)
\curveto(119.66679688,392.63085938)(119.46757813,393.04101562)(119.12773438,393.31835938)
\curveto(118.79179688,393.59570312)(118.29375,393.734375)(117.63359375,393.734375)
\curveto(116.94609375,393.734375)(116.44414063,393.60742188)(116.12773438,393.35351562)
\curveto(115.81523438,393.10351562)(115.65898438,392.80078125)(115.65898438,392.4453125)
\curveto(115.65898438,392.13671875)(115.7703125,391.8828125)(115.99296875,391.68359375)
\curveto(116.21171875,391.484375)(116.78203125,391.27929688)(117.70390625,391.06835938)
\curveto(118.6296875,390.86132812)(119.26445313,390.6796875)(119.60820313,390.5234375)
\curveto(120.10820313,390.29296875)(120.47734375,390)(120.715625,389.64453125)
\curveto(120.95390625,389.29296875)(121.07304688,388.88671875)(121.07304688,388.42578125)
\curveto(121.07304688,387.96875)(120.9421875,387.53710938)(120.68046875,387.13085938)
\curveto(120.41875,386.72851562)(120.04179688,386.4140625)(119.54960938,386.1875)
\curveto(119.06132813,385.96484375)(118.51054688,385.85351562)(117.89726563,385.85351562)
\curveto(117.11992188,385.85351562)(116.46757813,385.96679688)(115.94023438,386.19335938)
\curveto(115.41679688,386.41992188)(115.0046875,386.75976562)(114.70390625,387.21289062)
\curveto(114.40703125,387.66992188)(114.25078125,388.18554688)(114.23515625,388.75976562)
\closepath
}
}
{
\newrgbcolor{curcolor}{0 0 1}
\pscustom[linewidth=1,linecolor=curcolor]
{
\newpath
\moveto(130,389.9)
\lineto(172.2,389.9)
\moveto(55.3,285.2)
\lineto(141.9,300)
\lineto(228.5,113.7)
\lineto(315.2,112.6)
\lineto(401.8,111.5)
\lineto(488.4,114.7)
\lineto(575,115.8)
}
}
{
\newrgbcolor{curcolor}{0 0 0}
\pscustom[linewidth=1,linecolor=curcolor]
{
\newpath
\moveto(55.3,425.9)
\lineto(55.3,57.6)
\lineto(575,57.6)
\lineto(575,425.9)
\closepath
}
}
\end{pspicture}
}
    \captionsetup{width=0.75\linewidth}
    \caption{Bandwidth improves almost as much as latency when accessing ARC data from a remote node
        with task migration.
        Note that unlike most other graphs in this work this uses 2 MB reads,
        but the results are consistent with 1 MB reads tested with previous versions of my prototype.}
    \label{fig:bandwidth}
\end{figure}

This is a surprising result, as the theoretical memory bandwidth differences would indicate a much smaller improvement.
However these numbers are consistent with other research into NUMA systems, 
as the full bandwidth of the interconnect is shared between the nodes and thus highly variable
depending on what other process running at the same time happen to require\cite{bergstrom_stream,li_characterization_2013,song_evaluation_2017}.
These bandwidth results are also consistent with
other tests performed on this machine, such as a single-threaded STREAM benchmark or the Intel Memory Latency
Checker.
Both show a significant reductions reduction in bandwidth for remote node memory access, as discussed 
previously (Section \ref{chapter:measuringnuma}).
Bandwidth, like latency, is restored to same level as if the ARC data and the program were originally
located on the exact same NUMA node (Figure \ref{fig:bandwidthoptimal}).

\begin{figure}[H]
    \centering
    \resizebox{0.75\linewidth}{!}{%LaTeX with PSTricks extensions
%%Creator: Inkscape 1.0.2-2 (e86c870879, 2021-01-15)
%%Please note this file requires PSTricks extensions
\psset{xunit=.5pt,yunit=.5pt,runit=.5pt}
\begin{pspicture}(600,480)
{
\newrgbcolor{curcolor}{0 0 0}
\pscustom[linewidth=1,linecolor=curcolor]
{
\newpath
\moveto(55.3,57.6)
\lineto(64.3,57.6)
\moveto(575,57.6)
\lineto(566,57.6)
}
}
{
\newrgbcolor{curcolor}{0 0 0}
\pscustom[linestyle=none,fillstyle=solid,fillcolor=curcolor]
{
\newpath
\moveto(46.3671875,54.71367187)
\lineto(46.3671875,53.7)
\lineto(40.68945312,53.7)
\curveto(40.68164062,53.95390625)(40.72265625,54.19804687)(40.8125,54.43242187)
\curveto(40.95703125,54.81914062)(41.1875,55.2)(41.50390625,55.575)
\curveto(41.82421875,55.95)(42.28515625,56.38359375)(42.88671875,56.87578125)
\curveto(43.8203125,57.64140625)(44.45117188,58.246875)(44.77929688,58.6921875)
\curveto(45.10742188,59.14140625)(45.27148438,59.56523437)(45.27148438,59.96367187)
\curveto(45.27148438,60.38164062)(45.12109375,60.73320312)(44.8203125,61.01835937)
\curveto(44.5234375,61.30742187)(44.13476562,61.45195312)(43.65429688,61.45195312)
\curveto(43.14648438,61.45195312)(42.74023438,61.29960937)(42.43554688,60.99492187)
\curveto(42.13085938,60.69023437)(41.9765625,60.26835937)(41.97265625,59.72929687)
\lineto(40.88867188,59.840625)
\curveto(40.96289062,60.64921875)(41.2421875,61.26445312)(41.7265625,61.68632812)
\curveto(42.2109375,62.11210937)(42.86132812,62.325)(43.67773438,62.325)
\curveto(44.50195312,62.325)(45.15429688,62.09648437)(45.63476562,61.63945312)
\curveto(46.11523438,61.18242187)(46.35546875,60.61601562)(46.35546875,59.94023437)
\curveto(46.35546875,59.59648437)(46.28515625,59.25859375)(46.14453125,58.9265625)
\curveto(46.00390625,58.59453125)(45.76953125,58.24492187)(45.44140625,57.87773437)
\curveto(45.1171875,57.51054687)(44.57617188,57.00664062)(43.81835938,56.36601562)
\curveto(43.18554688,55.83476562)(42.77929688,55.4734375)(42.59960938,55.28203125)
\curveto(42.41992188,55.09453125)(42.27148438,54.90507812)(42.15429688,54.71367187)
\closepath
}
}
{
\newrgbcolor{curcolor}{0 0 0}
\pscustom[linewidth=1,linecolor=curcolor]
{
\newpath
\moveto(55.3,241.8)
\lineto(64.3,241.8)
\moveto(575,241.8)
\lineto(566,241.8)
}
}
{
\newrgbcolor{curcolor}{0 0 0}
\pscustom[linestyle=none,fillstyle=solid,fillcolor=curcolor]
{
\newpath
\moveto(44.20507812,237.9)
\lineto(44.20507812,239.95664063)
\lineto(40.47851562,239.95664063)
\lineto(40.47851562,240.9234375)
\lineto(44.3984375,246.48984375)
\lineto(45.25976562,246.48984375)
\lineto(45.25976562,240.9234375)
\lineto(46.41992188,240.9234375)
\lineto(46.41992188,239.95664063)
\lineto(45.25976562,239.95664063)
\lineto(45.25976562,237.9)
\closepath
\moveto(44.20507812,240.9234375)
\lineto(44.20507812,244.79648438)
\lineto(41.515625,240.9234375)
\closepath
}
}
{
\newrgbcolor{curcolor}{0 0 0}
\pscustom[linewidth=1,linecolor=curcolor]
{
\newpath
\moveto(55.3,425.9)
\lineto(64.3,425.9)
\moveto(575,425.9)
\lineto(566,425.9)
}
}
{
\newrgbcolor{curcolor}{0 0 0}
\pscustom[linestyle=none,fillstyle=solid,fillcolor=curcolor]
{
\newpath
\moveto(42.44726562,426.65820312)
\curveto(42.00976562,426.81835938)(41.68554688,427.046875)(41.47460938,427.34375)
\curveto(41.26367188,427.640625)(41.15820312,427.99609375)(41.15820312,428.41015625)
\curveto(41.15820312,429.03515625)(41.3828125,429.56054688)(41.83203125,429.98632812)
\curveto(42.28125,430.41210938)(42.87890625,430.625)(43.625,430.625)
\curveto(44.375,430.625)(44.97851562,430.40625)(45.43554688,429.96875)
\curveto(45.89257812,429.53515625)(46.12109375,429.00585938)(46.12109375,428.38085938)
\curveto(46.12109375,427.98242188)(46.015625,427.63476562)(45.8046875,427.33789062)
\curveto(45.59765625,427.04492188)(45.28125,426.81835938)(44.85546875,426.65820312)
\curveto(45.3828125,426.48632812)(45.78320312,426.20898438)(46.05664062,425.82617188)
\curveto(46.33398438,425.44335938)(46.47265625,424.98632812)(46.47265625,424.45507812)
\curveto(46.47265625,423.72070312)(46.21289062,423.10351562)(45.69335938,422.60351562)
\curveto(45.17382812,422.10351562)(44.49023438,421.85351562)(43.64257812,421.85351562)
\curveto(42.79492188,421.85351562)(42.11132812,422.10351562)(41.59179688,422.60351562)
\curveto(41.07226562,423.10742188)(40.8125,423.734375)(40.8125,424.484375)
\curveto(40.8125,425.04296875)(40.953125,425.50976562)(41.234375,425.88476562)
\curveto(41.51953125,426.26367188)(41.92382812,426.52148438)(42.44726562,426.65820312)
\closepath
\moveto(42.23632812,428.4453125)
\curveto(42.23632812,428.0390625)(42.3671875,427.70703125)(42.62890625,427.44921875)
\curveto(42.890625,427.19140625)(43.23046875,427.0625)(43.6484375,427.0625)
\curveto(44.0546875,427.0625)(44.38671875,427.18945312)(44.64453125,427.44335938)
\curveto(44.90625,427.70117188)(45.03710938,428.015625)(45.03710938,428.38671875)
\curveto(45.03710938,428.7734375)(44.90234375,429.09765625)(44.6328125,429.359375)
\curveto(44.3671875,429.625)(44.03515625,429.7578125)(43.63671875,429.7578125)
\curveto(43.234375,429.7578125)(42.90039062,429.62890625)(42.63476562,429.37109375)
\curveto(42.36914062,429.11328125)(42.23632812,428.8046875)(42.23632812,428.4453125)
\closepath
\moveto(41.89648438,424.47851562)
\curveto(41.89648438,424.17773438)(41.96679688,423.88671875)(42.10742188,423.60546875)
\curveto(42.25195312,423.32421875)(42.46484375,423.10546875)(42.74609375,422.94921875)
\curveto(43.02734375,422.796875)(43.33007812,422.72070312)(43.65429688,422.72070312)
\curveto(44.15820312,422.72070312)(44.57421875,422.8828125)(44.90234375,423.20703125)
\curveto(45.23046875,423.53125)(45.39453125,423.94335938)(45.39453125,424.44335938)
\curveto(45.39453125,424.95117188)(45.22460938,425.37109375)(44.88476562,425.703125)
\curveto(44.54882812,426.03515625)(44.12695312,426.20117188)(43.61914062,426.20117188)
\curveto(43.12304688,426.20117188)(42.7109375,426.03710938)(42.3828125,425.70898438)
\curveto(42.05859375,425.38085938)(41.89648438,424.97070312)(41.89648438,424.47851562)
\closepath
}
}
{
\newrgbcolor{curcolor}{0 0 0}
\pscustom[linewidth=1,linecolor=curcolor]
{
\newpath
\moveto(55.3,57.6)
\lineto(55.3,66.6)
\moveto(55.3,425.9)
\lineto(55.3,416.9)
}
}
{
\newrgbcolor{curcolor}{0 0 0}
\pscustom[linestyle=none,fillstyle=solid,fillcolor=curcolor]
{
\newpath
\moveto(46.33222656,36.71367187)
\lineto(46.33222656,35.7)
\lineto(40.65449219,35.7)
\curveto(40.64667969,35.95390625)(40.68769531,36.19804687)(40.77753906,36.43242187)
\curveto(40.92207031,36.81914062)(41.15253906,37.2)(41.46894531,37.575)
\curveto(41.78925781,37.95)(42.25019531,38.38359375)(42.85175781,38.87578125)
\curveto(43.78535156,39.64140625)(44.41621094,40.246875)(44.74433594,40.6921875)
\curveto(45.07246094,41.14140625)(45.23652344,41.56523437)(45.23652344,41.96367187)
\curveto(45.23652344,42.38164062)(45.08613281,42.73320312)(44.78535156,43.01835937)
\curveto(44.48847656,43.30742187)(44.09980469,43.45195312)(43.61933594,43.45195312)
\curveto(43.11152344,43.45195312)(42.70527344,43.29960937)(42.40058594,42.99492187)
\curveto(42.09589844,42.69023437)(41.94160156,42.26835937)(41.93769531,41.72929687)
\lineto(40.85371094,41.840625)
\curveto(40.92792969,42.64921875)(41.20722656,43.26445312)(41.69160156,43.68632812)
\curveto(42.17597656,44.11210937)(42.82636719,44.325)(43.64277344,44.325)
\curveto(44.46699219,44.325)(45.11933594,44.09648437)(45.59980469,43.63945312)
\curveto(46.08027344,43.18242187)(46.32050781,42.61601562)(46.32050781,41.94023437)
\curveto(46.32050781,41.59648437)(46.25019531,41.25859375)(46.10957031,40.9265625)
\curveto(45.96894531,40.59453125)(45.73457031,40.24492187)(45.40644531,39.87773437)
\curveto(45.08222656,39.51054687)(44.54121094,39.00664062)(43.78339844,38.36601562)
\curveto(43.15058594,37.83476562)(42.74433594,37.4734375)(42.56464844,37.28203125)
\curveto(42.38496094,37.09453125)(42.23652344,36.90507812)(42.11933594,36.71367187)
\closepath
}
}
{
\newrgbcolor{curcolor}{0 0 0}
\pscustom[linestyle=none,fillstyle=solid,fillcolor=curcolor]
{
\newpath
\moveto(47.46308594,37.95)
\lineto(48.57050781,38.04375)
\curveto(48.65253906,37.5046875)(48.84199219,37.0984375)(49.13886719,36.825)
\curveto(49.43964844,36.55546875)(49.80097656,36.42070312)(50.22285156,36.42070312)
\curveto(50.73066406,36.42070312)(51.16035156,36.61210937)(51.51191406,36.99492187)
\curveto(51.86347656,37.37773437)(52.03925781,37.88554687)(52.03925781,38.51835937)
\curveto(52.03925781,39.11992187)(51.86933594,39.59453125)(51.52949219,39.9421875)
\curveto(51.19355469,40.28984375)(50.75214844,40.46367187)(50.20527344,40.46367187)
\curveto(49.86542969,40.46367187)(49.55878906,40.38554687)(49.28535156,40.22929687)
\curveto(49.01191406,40.07695312)(48.79707031,39.87773437)(48.64082031,39.63164062)
\lineto(47.65058594,39.76054687)
\lineto(48.48261719,44.17265625)
\lineto(52.75410156,44.17265625)
\lineto(52.75410156,43.16484375)
\lineto(49.32636719,43.16484375)
\lineto(48.86347656,40.85625)
\curveto(49.37910156,41.215625)(49.92011719,41.3953125)(50.48652344,41.3953125)
\curveto(51.23652344,41.3953125)(51.86933594,41.13554687)(52.38496094,40.61601562)
\curveto(52.90058594,40.09648437)(53.15839844,39.42851562)(53.15839844,38.61210937)
\curveto(53.15839844,37.83476562)(52.93183594,37.16289062)(52.47871094,36.59648437)
\curveto(51.92792969,35.90117187)(51.17597656,35.55351562)(50.22285156,35.55351562)
\curveto(49.44160156,35.55351562)(48.80292969,35.77226562)(48.30683594,36.20976562)
\curveto(47.81464844,36.64726562)(47.53339844,37.22734375)(47.46308594,37.95)
\closepath
}
}
{
\newrgbcolor{curcolor}{0 0 0}
\pscustom[linestyle=none,fillstyle=solid,fillcolor=curcolor]
{
\newpath
\moveto(54.13691406,39.93632812)
\curveto(54.13691406,40.95195312)(54.24042969,41.76835937)(54.44746094,42.38554687)
\curveto(54.65839844,43.00664062)(54.96894531,43.48515625)(55.37910156,43.82109375)
\curveto(55.79316406,44.15703125)(56.31269531,44.325)(56.93769531,44.325)
\curveto(57.39863281,44.325)(57.80292969,44.23125)(58.15058594,44.04375)
\curveto(58.49824219,43.86015625)(58.78535156,43.59257812)(59.01191406,43.24101562)
\curveto(59.23847656,42.89335937)(59.41621094,42.46757812)(59.54511719,41.96367187)
\curveto(59.67402344,41.46367187)(59.73847656,40.78789062)(59.73847656,39.93632812)
\curveto(59.73847656,38.92851562)(59.63496094,38.1140625)(59.42792969,37.49296875)
\curveto(59.22089844,36.87578125)(58.91035156,36.39726562)(58.49628906,36.05742187)
\curveto(58.08613281,35.72148437)(57.56660156,35.55351562)(56.93769531,35.55351562)
\curveto(56.10957031,35.55351562)(55.45917969,35.85039062)(54.98652344,36.44414062)
\curveto(54.42011719,37.15898437)(54.13691406,38.32304687)(54.13691406,39.93632812)
\closepath
\moveto(55.22089844,39.93632812)
\curveto(55.22089844,38.52617187)(55.38496094,37.58671875)(55.71308594,37.11796875)
\curveto(56.04511719,36.653125)(56.45332031,36.42070312)(56.93769531,36.42070312)
\curveto(57.42207031,36.42070312)(57.82832031,36.65507812)(58.15644531,37.12382812)
\curveto(58.48847656,37.59257812)(58.65449219,38.53007812)(58.65449219,39.93632812)
\curveto(58.65449219,41.35039062)(58.48847656,42.28984375)(58.15644531,42.7546875)
\curveto(57.82832031,43.21953125)(57.41816406,43.45195312)(56.92597656,43.45195312)
\curveto(56.44160156,43.45195312)(56.05488281,43.246875)(55.76582031,42.83671875)
\curveto(55.40253906,42.31328125)(55.22089844,41.34648437)(55.22089844,39.93632812)
\closepath
}
}
{
\newrgbcolor{curcolor}{0 0 0}
\pscustom[linestyle=none,fillstyle=solid,fillcolor=curcolor]
{
\newpath
\moveto(61.20332031,35.7)
\lineto(61.20332031,44.28984375)
\lineto(62.91425781,44.28984375)
\lineto(64.94746094,38.2078125)
\curveto(65.13496094,37.64140625)(65.27167969,37.21757812)(65.35761719,36.93632812)
\curveto(65.45527344,37.24882812)(65.60761719,37.7078125)(65.81464844,38.31328125)
\lineto(67.87128906,44.28984375)
\lineto(69.40058594,44.28984375)
\lineto(69.40058594,35.7)
\lineto(68.30488281,35.7)
\lineto(68.30488281,42.88945312)
\lineto(65.80878906,35.7)
\lineto(64.78339844,35.7)
\lineto(62.29902344,43.0125)
\lineto(62.29902344,35.7)
\closepath
}
}
{
\newrgbcolor{curcolor}{0 0 0}
\pscustom[linewidth=1,linecolor=curcolor]
{
\newpath
\moveto(141.9,57.6)
\lineto(141.9,66.6)
\moveto(141.9,425.9)
\lineto(141.9,416.9)
}
}
{
\newrgbcolor{curcolor}{0 0 0}
\pscustom[linestyle=none,fillstyle=solid,fillcolor=curcolor]
{
\newpath
\moveto(127.38925781,37.95)
\lineto(128.49667969,38.04375)
\curveto(128.57871094,37.5046875)(128.76816406,37.0984375)(129.06503906,36.825)
\curveto(129.36582031,36.55546875)(129.72714844,36.42070312)(130.14902344,36.42070312)
\curveto(130.65683594,36.42070312)(131.08652344,36.61210937)(131.43808594,36.99492187)
\curveto(131.78964844,37.37773437)(131.96542969,37.88554687)(131.96542969,38.51835937)
\curveto(131.96542969,39.11992187)(131.79550781,39.59453125)(131.45566406,39.9421875)
\curveto(131.11972656,40.28984375)(130.67832031,40.46367187)(130.13144531,40.46367187)
\curveto(129.79160156,40.46367187)(129.48496094,40.38554687)(129.21152344,40.22929687)
\curveto(128.93808594,40.07695312)(128.72324219,39.87773437)(128.56699219,39.63164062)
\lineto(127.57675781,39.76054687)
\lineto(128.40878906,44.17265625)
\lineto(132.68027344,44.17265625)
\lineto(132.68027344,43.16484375)
\lineto(129.25253906,43.16484375)
\lineto(128.78964844,40.85625)
\curveto(129.30527344,41.215625)(129.84628906,41.3953125)(130.41269531,41.3953125)
\curveto(131.16269531,41.3953125)(131.79550781,41.13554687)(132.31113281,40.61601562)
\curveto(132.82675781,40.09648437)(133.08457031,39.42851562)(133.08457031,38.61210937)
\curveto(133.08457031,37.83476562)(132.85800781,37.16289062)(132.40488281,36.59648437)
\curveto(131.85410156,35.90117187)(131.10214844,35.55351562)(130.14902344,35.55351562)
\curveto(129.36777344,35.55351562)(128.72910156,35.77226562)(128.23300781,36.20976562)
\curveto(127.74082031,36.64726562)(127.45957031,37.22734375)(127.38925781,37.95)
\closepath
}
}
{
\newrgbcolor{curcolor}{0 0 0}
\pscustom[linestyle=none,fillstyle=solid,fillcolor=curcolor]
{
\newpath
\moveto(134.06308594,39.93632812)
\curveto(134.06308594,40.95195312)(134.16660156,41.76835937)(134.37363281,42.38554687)
\curveto(134.58457031,43.00664062)(134.89511719,43.48515625)(135.30527344,43.82109375)
\curveto(135.71933594,44.15703125)(136.23886719,44.325)(136.86386719,44.325)
\curveto(137.32480469,44.325)(137.72910156,44.23125)(138.07675781,44.04375)
\curveto(138.42441406,43.86015625)(138.71152344,43.59257812)(138.93808594,43.24101562)
\curveto(139.16464844,42.89335937)(139.34238281,42.46757812)(139.47128906,41.96367187)
\curveto(139.60019531,41.46367187)(139.66464844,40.78789062)(139.66464844,39.93632812)
\curveto(139.66464844,38.92851562)(139.56113281,38.1140625)(139.35410156,37.49296875)
\curveto(139.14707031,36.87578125)(138.83652344,36.39726562)(138.42246094,36.05742187)
\curveto(138.01230469,35.72148437)(137.49277344,35.55351562)(136.86386719,35.55351562)
\curveto(136.03574219,35.55351562)(135.38535156,35.85039062)(134.91269531,36.44414062)
\curveto(134.34628906,37.15898437)(134.06308594,38.32304687)(134.06308594,39.93632812)
\closepath
\moveto(135.14707031,39.93632812)
\curveto(135.14707031,38.52617187)(135.31113281,37.58671875)(135.63925781,37.11796875)
\curveto(135.97128906,36.653125)(136.37949219,36.42070312)(136.86386719,36.42070312)
\curveto(137.34824219,36.42070312)(137.75449219,36.65507812)(138.08261719,37.12382812)
\curveto(138.41464844,37.59257812)(138.58066406,38.53007812)(138.58066406,39.93632812)
\curveto(138.58066406,41.35039062)(138.41464844,42.28984375)(138.08261719,42.7546875)
\curveto(137.75449219,43.21953125)(137.34433594,43.45195312)(136.85214844,43.45195312)
\curveto(136.36777344,43.45195312)(135.98105469,43.246875)(135.69199219,42.83671875)
\curveto(135.32871094,42.31328125)(135.14707031,41.34648437)(135.14707031,39.93632812)
\closepath
}
}
{
\newrgbcolor{curcolor}{0 0 0}
\pscustom[linestyle=none,fillstyle=solid,fillcolor=curcolor]
{
\newpath
\moveto(140.73691406,39.93632812)
\curveto(140.73691406,40.95195312)(140.84042969,41.76835937)(141.04746094,42.38554687)
\curveto(141.25839844,43.00664062)(141.56894531,43.48515625)(141.97910156,43.82109375)
\curveto(142.39316406,44.15703125)(142.91269531,44.325)(143.53769531,44.325)
\curveto(143.99863281,44.325)(144.40292969,44.23125)(144.75058594,44.04375)
\curveto(145.09824219,43.86015625)(145.38535156,43.59257812)(145.61191406,43.24101562)
\curveto(145.83847656,42.89335937)(146.01621094,42.46757812)(146.14511719,41.96367187)
\curveto(146.27402344,41.46367187)(146.33847656,40.78789062)(146.33847656,39.93632812)
\curveto(146.33847656,38.92851562)(146.23496094,38.1140625)(146.02792969,37.49296875)
\curveto(145.82089844,36.87578125)(145.51035156,36.39726562)(145.09628906,36.05742187)
\curveto(144.68613281,35.72148437)(144.16660156,35.55351562)(143.53769531,35.55351562)
\curveto(142.70957031,35.55351562)(142.05917969,35.85039062)(141.58652344,36.44414062)
\curveto(141.02011719,37.15898437)(140.73691406,38.32304687)(140.73691406,39.93632812)
\closepath
\moveto(141.82089844,39.93632812)
\curveto(141.82089844,38.52617187)(141.98496094,37.58671875)(142.31308594,37.11796875)
\curveto(142.64511719,36.653125)(143.05332031,36.42070312)(143.53769531,36.42070312)
\curveto(144.02207031,36.42070312)(144.42832031,36.65507812)(144.75644531,37.12382812)
\curveto(145.08847656,37.59257812)(145.25449219,38.53007812)(145.25449219,39.93632812)
\curveto(145.25449219,41.35039062)(145.08847656,42.28984375)(144.75644531,42.7546875)
\curveto(144.42832031,43.21953125)(144.01816406,43.45195312)(143.52597656,43.45195312)
\curveto(143.04160156,43.45195312)(142.65488281,43.246875)(142.36582031,42.83671875)
\curveto(142.00253906,42.31328125)(141.82089844,41.34648437)(141.82089844,39.93632812)
\closepath
}
}
{
\newrgbcolor{curcolor}{0 0 0}
\pscustom[linestyle=none,fillstyle=solid,fillcolor=curcolor]
{
\newpath
\moveto(147.80332031,35.7)
\lineto(147.80332031,44.28984375)
\lineto(149.51425781,44.28984375)
\lineto(151.54746094,38.2078125)
\curveto(151.73496094,37.64140625)(151.87167969,37.21757812)(151.95761719,36.93632812)
\curveto(152.05527344,37.24882812)(152.20761719,37.7078125)(152.41464844,38.31328125)
\lineto(154.47128906,44.28984375)
\lineto(156.00058594,44.28984375)
\lineto(156.00058594,35.7)
\lineto(154.90488281,35.7)
\lineto(154.90488281,42.88945312)
\lineto(152.40878906,35.7)
\lineto(151.38339844,35.7)
\lineto(148.89902344,43.0125)
\lineto(148.89902344,35.7)
\closepath
}
}
{
\newrgbcolor{curcolor}{0 0 0}
\pscustom[linewidth=1,linecolor=curcolor]
{
\newpath
\moveto(228.5,57.6)
\lineto(228.5,66.6)
\moveto(228.5,425.9)
\lineto(228.5,416.9)
}
}
{
\newrgbcolor{curcolor}{0 0 0}
\pscustom[linestyle=none,fillstyle=solid,fillcolor=curcolor]
{
\newpath
\moveto(224.96679688,35.7)
\lineto(223.91210938,35.7)
\lineto(223.91210938,42.42070312)
\curveto(223.65820312,42.17851562)(223.32421875,41.93632812)(222.91015625,41.69414062)
\curveto(222.5,41.45195312)(222.13085938,41.2703125)(221.80273438,41.14921875)
\lineto(221.80273438,42.16875)
\curveto(222.39257812,42.44609375)(222.90820312,42.78203125)(223.34960938,43.1765625)
\curveto(223.79101562,43.57109375)(224.10351562,43.95390625)(224.28710938,44.325)
\lineto(224.96679688,44.325)
\closepath
}
}
{
\newrgbcolor{curcolor}{0 0 0}
\pscustom[linestyle=none,fillstyle=solid,fillcolor=curcolor]
{
\newpath
\moveto(232.11523438,39.06914062)
\lineto(232.11523438,40.07695312)
\lineto(235.75390625,40.0828125)
\lineto(235.75390625,36.8953125)
\curveto(235.1953125,36.45)(234.61914062,36.1140625)(234.02539062,35.8875)
\curveto(233.43164062,35.66484375)(232.82226562,35.55351562)(232.19726562,35.55351562)
\curveto(231.35351562,35.55351562)(230.5859375,35.73320312)(229.89453125,36.09257812)
\curveto(229.20703125,36.45585937)(228.6875,36.97929687)(228.3359375,37.66289062)
\curveto(227.984375,38.34648437)(227.80859375,39.11015625)(227.80859375,39.95390625)
\curveto(227.80859375,40.78984375)(227.98242188,41.56914062)(228.33007812,42.29179687)
\curveto(228.68164062,43.01835937)(229.18554688,43.55742187)(229.84179688,43.90898437)
\curveto(230.49804688,44.26054687)(231.25390625,44.43632812)(232.109375,44.43632812)
\curveto(232.73046875,44.43632812)(233.29101562,44.33476562)(233.79101562,44.13164062)
\curveto(234.29492188,43.93242187)(234.68945312,43.653125)(234.97460938,43.29375)
\curveto(235.25976562,42.934375)(235.4765625,42.465625)(235.625,41.8875)
\lineto(234.59960938,41.60625)
\curveto(234.47070312,42.04375)(234.31054688,42.3875)(234.11914062,42.6375)
\curveto(233.92773438,42.8875)(233.65429688,43.08671875)(233.29882812,43.23515625)
\curveto(232.94335938,43.3875)(232.54882812,43.46367187)(232.11523438,43.46367187)
\curveto(231.59570312,43.46367187)(231.14648438,43.38359375)(230.76757812,43.2234375)
\curveto(230.38867188,43.0671875)(230.08203125,42.86015625)(229.84765625,42.60234375)
\curveto(229.6171875,42.34453125)(229.4375,42.06132812)(229.30859375,41.75273437)
\curveto(229.08984375,41.22148437)(228.98046875,40.6453125)(228.98046875,40.02421875)
\curveto(228.98046875,39.25859375)(229.11132812,38.61796875)(229.37304688,38.10234375)
\curveto(229.63867188,37.58671875)(230.0234375,37.20390625)(230.52734375,36.95390625)
\curveto(231.03125,36.70390625)(231.56640625,36.57890625)(232.1328125,36.57890625)
\curveto(232.625,36.57890625)(233.10546875,36.67265625)(233.57421875,36.86015625)
\curveto(234.04296875,37.0515625)(234.3984375,37.2546875)(234.640625,37.46953125)
\lineto(234.640625,39.06914062)
\closepath
}
}
{
\newrgbcolor{curcolor}{0 0 0}
\pscustom[linewidth=1,linecolor=curcolor]
{
\newpath
\moveto(315.2,57.6)
\lineto(315.2,66.6)
\moveto(315.2,425.9)
\lineto(315.2,416.9)
}
}
{
\newrgbcolor{curcolor}{0 0 0}
\pscustom[linestyle=none,fillstyle=solid,fillcolor=curcolor]
{
\newpath
\moveto(313.23710937,36.71367187)
\lineto(313.23710937,35.7)
\lineto(307.559375,35.7)
\curveto(307.5515625,35.95390625)(307.59257812,36.19804687)(307.68242187,36.43242187)
\curveto(307.82695312,36.81914062)(308.05742187,37.2)(308.37382812,37.575)
\curveto(308.69414062,37.95)(309.15507812,38.38359375)(309.75664062,38.87578125)
\curveto(310.69023437,39.64140625)(311.32109375,40.246875)(311.64921875,40.6921875)
\curveto(311.97734375,41.14140625)(312.14140625,41.56523437)(312.14140625,41.96367187)
\curveto(312.14140625,42.38164062)(311.99101562,42.73320312)(311.69023437,43.01835937)
\curveto(311.39335937,43.30742187)(311.0046875,43.45195312)(310.52421875,43.45195312)
\curveto(310.01640625,43.45195312)(309.61015625,43.29960937)(309.30546875,42.99492187)
\curveto(309.00078125,42.69023437)(308.84648437,42.26835937)(308.84257812,41.72929687)
\lineto(307.75859375,41.840625)
\curveto(307.8328125,42.64921875)(308.11210937,43.26445312)(308.59648437,43.68632812)
\curveto(309.08085937,44.11210937)(309.73125,44.325)(310.54765625,44.325)
\curveto(311.371875,44.325)(312.02421875,44.09648437)(312.5046875,43.63945312)
\curveto(312.98515625,43.18242187)(313.22539062,42.61601562)(313.22539062,41.94023437)
\curveto(313.22539062,41.59648437)(313.15507812,41.25859375)(313.01445312,40.9265625)
\curveto(312.87382812,40.59453125)(312.63945312,40.24492187)(312.31132812,39.87773437)
\curveto(311.98710937,39.51054687)(311.44609375,39.00664062)(310.68828125,38.36601562)
\curveto(310.05546875,37.83476562)(309.64921875,37.4734375)(309.46953125,37.28203125)
\curveto(309.28984375,37.09453125)(309.14140625,36.90507812)(309.02421875,36.71367187)
\closepath
}
}
{
\newrgbcolor{curcolor}{0 0 0}
\pscustom[linestyle=none,fillstyle=solid,fillcolor=curcolor]
{
\newpath
\moveto(318.81523437,39.06914062)
\lineto(318.81523437,40.07695312)
\lineto(322.45390625,40.0828125)
\lineto(322.45390625,36.8953125)
\curveto(321.8953125,36.45)(321.31914062,36.1140625)(320.72539062,35.8875)
\curveto(320.13164062,35.66484375)(319.52226562,35.55351562)(318.89726562,35.55351562)
\curveto(318.05351562,35.55351562)(317.2859375,35.73320312)(316.59453125,36.09257812)
\curveto(315.90703125,36.45585937)(315.3875,36.97929687)(315.0359375,37.66289062)
\curveto(314.684375,38.34648437)(314.50859375,39.11015625)(314.50859375,39.95390625)
\curveto(314.50859375,40.78984375)(314.68242187,41.56914062)(315.03007812,42.29179687)
\curveto(315.38164062,43.01835937)(315.88554687,43.55742187)(316.54179687,43.90898437)
\curveto(317.19804687,44.26054687)(317.95390625,44.43632812)(318.809375,44.43632812)
\curveto(319.43046875,44.43632812)(319.99101562,44.33476562)(320.49101562,44.13164062)
\curveto(320.99492187,43.93242187)(321.38945312,43.653125)(321.67460937,43.29375)
\curveto(321.95976562,42.934375)(322.1765625,42.465625)(322.325,41.8875)
\lineto(321.29960937,41.60625)
\curveto(321.17070312,42.04375)(321.01054687,42.3875)(320.81914062,42.6375)
\curveto(320.62773437,42.8875)(320.35429687,43.08671875)(319.99882812,43.23515625)
\curveto(319.64335937,43.3875)(319.24882812,43.46367187)(318.81523437,43.46367187)
\curveto(318.29570312,43.46367187)(317.84648437,43.38359375)(317.46757812,43.2234375)
\curveto(317.08867187,43.0671875)(316.78203125,42.86015625)(316.54765625,42.60234375)
\curveto(316.3171875,42.34453125)(316.1375,42.06132812)(316.00859375,41.75273437)
\curveto(315.78984375,41.22148437)(315.68046875,40.6453125)(315.68046875,40.02421875)
\curveto(315.68046875,39.25859375)(315.81132812,38.61796875)(316.07304687,38.10234375)
\curveto(316.33867187,37.58671875)(316.7234375,37.20390625)(317.22734375,36.95390625)
\curveto(317.73125,36.70390625)(318.26640625,36.57890625)(318.8328125,36.57890625)
\curveto(319.325,36.57890625)(319.80546875,36.67265625)(320.27421875,36.86015625)
\curveto(320.74296875,37.0515625)(321.0984375,37.2546875)(321.340625,37.46953125)
\lineto(321.340625,39.06914062)
\closepath
}
}
{
\newrgbcolor{curcolor}{0 0 0}
\pscustom[linewidth=1,linecolor=curcolor]
{
\newpath
\moveto(401.8,57.6)
\lineto(401.8,66.6)
\moveto(401.8,425.9)
\lineto(401.8,416.9)
}
}
{
\newrgbcolor{curcolor}{0 0 0}
\pscustom[linestyle=none,fillstyle=solid,fillcolor=curcolor]
{
\newpath
\moveto(397.675,35.7)
\lineto(397.675,37.75664062)
\lineto(393.9484375,37.75664062)
\lineto(393.9484375,38.7234375)
\lineto(397.86835938,44.28984375)
\lineto(398.7296875,44.28984375)
\lineto(398.7296875,38.7234375)
\lineto(399.88984375,38.7234375)
\lineto(399.88984375,37.75664062)
\lineto(398.7296875,37.75664062)
\lineto(398.7296875,35.7)
\closepath
\moveto(397.675,38.7234375)
\lineto(397.675,42.59648437)
\lineto(394.98554688,38.7234375)
\closepath
}
}
{
\newrgbcolor{curcolor}{0 0 0}
\pscustom[linestyle=none,fillstyle=solid,fillcolor=curcolor]
{
\newpath
\moveto(405.41523438,39.06914062)
\lineto(405.41523438,40.07695312)
\lineto(409.05390625,40.0828125)
\lineto(409.05390625,36.8953125)
\curveto(408.4953125,36.45)(407.91914063,36.1140625)(407.32539063,35.8875)
\curveto(406.73164063,35.66484375)(406.12226563,35.55351562)(405.49726563,35.55351562)
\curveto(404.65351563,35.55351562)(403.8859375,35.73320312)(403.19453125,36.09257812)
\curveto(402.50703125,36.45585937)(401.9875,36.97929687)(401.6359375,37.66289062)
\curveto(401.284375,38.34648437)(401.10859375,39.11015625)(401.10859375,39.95390625)
\curveto(401.10859375,40.78984375)(401.28242188,41.56914062)(401.63007813,42.29179687)
\curveto(401.98164063,43.01835937)(402.48554688,43.55742187)(403.14179688,43.90898437)
\curveto(403.79804688,44.26054687)(404.55390625,44.43632812)(405.409375,44.43632812)
\curveto(406.03046875,44.43632812)(406.59101563,44.33476562)(407.09101563,44.13164062)
\curveto(407.59492188,43.93242187)(407.98945313,43.653125)(408.27460938,43.29375)
\curveto(408.55976563,42.934375)(408.7765625,42.465625)(408.925,41.8875)
\lineto(407.89960938,41.60625)
\curveto(407.77070313,42.04375)(407.61054688,42.3875)(407.41914063,42.6375)
\curveto(407.22773438,42.8875)(406.95429688,43.08671875)(406.59882813,43.23515625)
\curveto(406.24335938,43.3875)(405.84882813,43.46367187)(405.41523438,43.46367187)
\curveto(404.89570313,43.46367187)(404.44648438,43.38359375)(404.06757813,43.2234375)
\curveto(403.68867188,43.0671875)(403.38203125,42.86015625)(403.14765625,42.60234375)
\curveto(402.9171875,42.34453125)(402.7375,42.06132812)(402.60859375,41.75273437)
\curveto(402.38984375,41.22148437)(402.28046875,40.6453125)(402.28046875,40.02421875)
\curveto(402.28046875,39.25859375)(402.41132813,38.61796875)(402.67304688,38.10234375)
\curveto(402.93867188,37.58671875)(403.3234375,37.20390625)(403.82734375,36.95390625)
\curveto(404.33125,36.70390625)(404.86640625,36.57890625)(405.4328125,36.57890625)
\curveto(405.925,36.57890625)(406.40546875,36.67265625)(406.87421875,36.86015625)
\curveto(407.34296875,37.0515625)(407.6984375,37.2546875)(407.940625,37.46953125)
\lineto(407.940625,39.06914062)
\closepath
}
}
{
\newrgbcolor{curcolor}{0 0 0}
\pscustom[linewidth=1,linecolor=curcolor]
{
\newpath
\moveto(488.4,57.6)
\lineto(488.4,66.6)
\moveto(488.4,425.9)
\lineto(488.4,416.9)
}
}
{
\newrgbcolor{curcolor}{0 0 0}
\pscustom[linestyle=none,fillstyle=solid,fillcolor=curcolor]
{
\newpath
\moveto(482.5171875,40.35820312)
\curveto(482.0796875,40.51835937)(481.75546875,40.746875)(481.54453125,41.04375)
\curveto(481.33359375,41.340625)(481.228125,41.69609375)(481.228125,42.11015625)
\curveto(481.228125,42.73515625)(481.45273437,43.26054687)(481.90195312,43.68632812)
\curveto(482.35117187,44.11210937)(482.94882812,44.325)(483.69492187,44.325)
\curveto(484.44492187,44.325)(485.0484375,44.10625)(485.50546875,43.66875)
\curveto(485.9625,43.23515625)(486.19101562,42.70585937)(486.19101562,42.08085937)
\curveto(486.19101562,41.68242187)(486.08554687,41.33476562)(485.87460937,41.03789062)
\curveto(485.66757812,40.74492187)(485.35117187,40.51835937)(484.92539062,40.35820312)
\curveto(485.45273437,40.18632812)(485.853125,39.90898437)(486.1265625,39.52617187)
\curveto(486.40390625,39.14335937)(486.54257812,38.68632812)(486.54257812,38.15507812)
\curveto(486.54257812,37.42070312)(486.2828125,36.80351562)(485.76328125,36.30351562)
\curveto(485.24375,35.80351562)(484.56015625,35.55351562)(483.7125,35.55351562)
\curveto(482.86484375,35.55351562)(482.18125,35.80351562)(481.66171875,36.30351562)
\curveto(481.1421875,36.80742187)(480.88242187,37.434375)(480.88242187,38.184375)
\curveto(480.88242187,38.74296875)(481.02304687,39.20976562)(481.30429687,39.58476562)
\curveto(481.58945312,39.96367187)(481.99375,40.22148437)(482.5171875,40.35820312)
\closepath
\moveto(482.30625,42.1453125)
\curveto(482.30625,41.7390625)(482.43710937,41.40703125)(482.69882812,41.14921875)
\curveto(482.96054687,40.89140625)(483.30039062,40.7625)(483.71835937,40.7625)
\curveto(484.12460937,40.7625)(484.45664062,40.88945312)(484.71445312,41.14335937)
\curveto(484.97617187,41.40117187)(485.10703125,41.715625)(485.10703125,42.08671875)
\curveto(485.10703125,42.4734375)(484.97226562,42.79765625)(484.70273437,43.059375)
\curveto(484.43710937,43.325)(484.10507812,43.4578125)(483.70664062,43.4578125)
\curveto(483.30429687,43.4578125)(482.9703125,43.32890625)(482.7046875,43.07109375)
\curveto(482.4390625,42.81328125)(482.30625,42.5046875)(482.30625,42.1453125)
\closepath
\moveto(481.96640625,38.17851562)
\curveto(481.96640625,37.87773437)(482.03671875,37.58671875)(482.17734375,37.30546875)
\curveto(482.321875,37.02421875)(482.53476562,36.80546875)(482.81601562,36.64921875)
\curveto(483.09726562,36.496875)(483.4,36.42070312)(483.72421875,36.42070312)
\curveto(484.228125,36.42070312)(484.64414062,36.5828125)(484.97226562,36.90703125)
\curveto(485.30039062,37.23125)(485.46445312,37.64335937)(485.46445312,38.14335937)
\curveto(485.46445312,38.65117187)(485.29453125,39.07109375)(484.9546875,39.403125)
\curveto(484.61875,39.73515625)(484.196875,39.90117187)(483.6890625,39.90117187)
\curveto(483.19296875,39.90117187)(482.78085937,39.73710937)(482.45273437,39.40898437)
\curveto(482.12851562,39.08085937)(481.96640625,38.67070312)(481.96640625,38.17851562)
\closepath
}
}
{
\newrgbcolor{curcolor}{0 0 0}
\pscustom[linestyle=none,fillstyle=solid,fillcolor=curcolor]
{
\newpath
\moveto(492.01523437,39.06914062)
\lineto(492.01523437,40.07695312)
\lineto(495.65390625,40.0828125)
\lineto(495.65390625,36.8953125)
\curveto(495.0953125,36.45)(494.51914062,36.1140625)(493.92539062,35.8875)
\curveto(493.33164062,35.66484375)(492.72226562,35.55351562)(492.09726562,35.55351562)
\curveto(491.25351562,35.55351562)(490.4859375,35.73320312)(489.79453125,36.09257812)
\curveto(489.10703125,36.45585937)(488.5875,36.97929687)(488.2359375,37.66289062)
\curveto(487.884375,38.34648437)(487.70859375,39.11015625)(487.70859375,39.95390625)
\curveto(487.70859375,40.78984375)(487.88242187,41.56914062)(488.23007812,42.29179687)
\curveto(488.58164062,43.01835937)(489.08554687,43.55742187)(489.74179687,43.90898437)
\curveto(490.39804687,44.26054687)(491.15390625,44.43632812)(492.009375,44.43632812)
\curveto(492.63046875,44.43632812)(493.19101562,44.33476562)(493.69101562,44.13164062)
\curveto(494.19492187,43.93242187)(494.58945312,43.653125)(494.87460937,43.29375)
\curveto(495.15976562,42.934375)(495.3765625,42.465625)(495.525,41.8875)
\lineto(494.49960937,41.60625)
\curveto(494.37070312,42.04375)(494.21054687,42.3875)(494.01914062,42.6375)
\curveto(493.82773437,42.8875)(493.55429687,43.08671875)(493.19882812,43.23515625)
\curveto(492.84335937,43.3875)(492.44882812,43.46367187)(492.01523437,43.46367187)
\curveto(491.49570312,43.46367187)(491.04648437,43.38359375)(490.66757812,43.2234375)
\curveto(490.28867187,43.0671875)(489.98203125,42.86015625)(489.74765625,42.60234375)
\curveto(489.5171875,42.34453125)(489.3375,42.06132812)(489.20859375,41.75273437)
\curveto(488.98984375,41.22148437)(488.88046875,40.6453125)(488.88046875,40.02421875)
\curveto(488.88046875,39.25859375)(489.01132812,38.61796875)(489.27304687,38.10234375)
\curveto(489.53867187,37.58671875)(489.9234375,37.20390625)(490.42734375,36.95390625)
\curveto(490.93125,36.70390625)(491.46640625,36.57890625)(492.0328125,36.57890625)
\curveto(492.525,36.57890625)(493.00546875,36.67265625)(493.47421875,36.86015625)
\curveto(493.94296875,37.0515625)(494.2984375,37.2546875)(494.540625,37.46953125)
\lineto(494.540625,39.06914062)
\closepath
}
}
{
\newrgbcolor{curcolor}{0 0 0}
\pscustom[linewidth=1,linecolor=curcolor]
{
\newpath
\moveto(575,57.6)
\lineto(575,66.6)
\moveto(575,425.9)
\lineto(575,416.9)
}
}
{
\newrgbcolor{curcolor}{0 0 0}
\pscustom[linestyle=none,fillstyle=solid,fillcolor=curcolor]
{
\newpath
\moveto(568.12988281,35.7)
\lineto(567.07519531,35.7)
\lineto(567.07519531,42.42070312)
\curveto(566.82128906,42.17851562)(566.48730469,41.93632812)(566.07324219,41.69414062)
\curveto(565.66308594,41.45195312)(565.29394531,41.2703125)(564.96582031,41.14921875)
\lineto(564.96582031,42.16875)
\curveto(565.55566406,42.44609375)(566.07128906,42.78203125)(566.51269531,43.1765625)
\curveto(566.95410156,43.57109375)(567.26660156,43.95390625)(567.45019531,44.325)
\lineto(568.12988281,44.325)
\closepath
}
}
{
\newrgbcolor{curcolor}{0 0 0}
\pscustom[linestyle=none,fillstyle=solid,fillcolor=curcolor]
{
\newpath
\moveto(576.30371094,42.18632812)
\lineto(575.25488281,42.10429687)
\curveto(575.16113281,42.51835937)(575.02832031,42.81914062)(574.85644531,43.00664062)
\curveto(574.57128906,43.30742187)(574.21972656,43.4578125)(573.80175781,43.4578125)
\curveto(573.46582031,43.4578125)(573.17089844,43.3640625)(572.91699219,43.1765625)
\curveto(572.58496094,42.934375)(572.32324219,42.58085937)(572.13183594,42.11601562)
\curveto(571.94042969,41.65117187)(571.84082031,40.9890625)(571.83300781,40.1296875)
\curveto(572.08691406,40.51640625)(572.39746094,40.80351562)(572.76464844,40.99101562)
\curveto(573.13183594,41.17851562)(573.51660156,41.27226562)(573.91894531,41.27226562)
\curveto(574.62207031,41.27226562)(575.21972656,41.0125)(575.71191406,40.49296875)
\curveto(576.20800781,39.97734375)(576.45605469,39.309375)(576.45605469,38.4890625)
\curveto(576.45605469,37.95)(576.33886719,37.44804687)(576.10449219,36.98320312)
\curveto(575.87402344,36.52226562)(575.55566406,36.16875)(575.14941406,35.92265625)
\curveto(574.74316406,35.6765625)(574.28222656,35.55351562)(573.76660156,35.55351562)
\curveto(572.88769531,35.55351562)(572.17089844,35.87578125)(571.61621094,36.5203125)
\curveto(571.06152344,37.16875)(570.78417969,38.23515625)(570.78417969,39.71953125)
\curveto(570.78417969,41.3796875)(571.09082031,42.58671875)(571.70410156,43.340625)
\curveto(572.23925781,43.996875)(572.95996094,44.325)(573.86621094,44.325)
\curveto(574.54199219,44.325)(575.09472656,44.13554687)(575.52441406,43.75664062)
\curveto(575.95800781,43.37773437)(576.21777344,42.85429687)(576.30371094,42.18632812)
\closepath
\moveto(571.99707031,38.48320312)
\curveto(571.99707031,38.11992187)(572.07324219,37.77226562)(572.22558594,37.44023437)
\curveto(572.38183594,37.10820312)(572.59863281,36.85429687)(572.87597656,36.67851562)
\curveto(573.15332031,36.50664062)(573.44433594,36.42070312)(573.74902344,36.42070312)
\curveto(574.19433594,36.42070312)(574.57714844,36.60039062)(574.89746094,36.95976562)
\curveto(575.21777344,37.31914062)(575.37792969,37.80742187)(575.37792969,38.42460937)
\curveto(575.37792969,39.01835937)(575.21972656,39.48515625)(574.90332031,39.825)
\curveto(574.58691406,40.16875)(574.18847656,40.340625)(573.70800781,40.340625)
\curveto(573.23144531,40.340625)(572.82714844,40.16875)(572.49511719,39.825)
\curveto(572.16308594,39.48515625)(571.99707031,39.03789062)(571.99707031,38.48320312)
\closepath
}
}
{
\newrgbcolor{curcolor}{0 0 0}
\pscustom[linestyle=none,fillstyle=solid,fillcolor=curcolor]
{
\newpath
\moveto(581.95214844,39.06914062)
\lineto(581.95214844,40.07695312)
\lineto(585.59082031,40.0828125)
\lineto(585.59082031,36.8953125)
\curveto(585.03222656,36.45)(584.45605469,36.1140625)(583.86230469,35.8875)
\curveto(583.26855469,35.66484375)(582.65917969,35.55351562)(582.03417969,35.55351562)
\curveto(581.19042969,35.55351562)(580.42285156,35.73320312)(579.73144531,36.09257812)
\curveto(579.04394531,36.45585937)(578.52441406,36.97929687)(578.17285156,37.66289062)
\curveto(577.82128906,38.34648437)(577.64550781,39.11015625)(577.64550781,39.95390625)
\curveto(577.64550781,40.78984375)(577.81933594,41.56914062)(578.16699219,42.29179687)
\curveto(578.51855469,43.01835937)(579.02246094,43.55742187)(579.67871094,43.90898437)
\curveto(580.33496094,44.26054687)(581.09082031,44.43632812)(581.94628906,44.43632812)
\curveto(582.56738281,44.43632812)(583.12792969,44.33476562)(583.62792969,44.13164062)
\curveto(584.13183594,43.93242187)(584.52636719,43.653125)(584.81152344,43.29375)
\curveto(585.09667969,42.934375)(585.31347656,42.465625)(585.46191406,41.8875)
\lineto(584.43652344,41.60625)
\curveto(584.30761719,42.04375)(584.14746094,42.3875)(583.95605469,42.6375)
\curveto(583.76464844,42.8875)(583.49121094,43.08671875)(583.13574219,43.23515625)
\curveto(582.78027344,43.3875)(582.38574219,43.46367187)(581.95214844,43.46367187)
\curveto(581.43261719,43.46367187)(580.98339844,43.38359375)(580.60449219,43.2234375)
\curveto(580.22558594,43.0671875)(579.91894531,42.86015625)(579.68457031,42.60234375)
\curveto(579.45410156,42.34453125)(579.27441406,42.06132812)(579.14550781,41.75273437)
\curveto(578.92675781,41.22148437)(578.81738281,40.6453125)(578.81738281,40.02421875)
\curveto(578.81738281,39.25859375)(578.94824219,38.61796875)(579.20996094,38.10234375)
\curveto(579.47558594,37.58671875)(579.86035156,37.20390625)(580.36425781,36.95390625)
\curveto(580.86816406,36.70390625)(581.40332031,36.57890625)(581.96972656,36.57890625)
\curveto(582.46191406,36.57890625)(582.94238281,36.67265625)(583.41113281,36.86015625)
\curveto(583.87988281,37.0515625)(584.23535156,37.2546875)(584.47753906,37.46953125)
\lineto(584.47753906,39.06914062)
\closepath
}
}
{
\newrgbcolor{curcolor}{0 0 0}
\pscustom[linewidth=1,linecolor=curcolor]
{
\newpath
\moveto(55.3,425.9)
\lineto(55.3,57.6)
\lineto(575,57.6)
\lineto(575,425.9)
\closepath
}
}
{
\newrgbcolor{curcolor}{0 0 0}
\pscustom[linestyle=none,fillstyle=solid,fillcolor=curcolor]
{
\newpath
\moveto(16.3,195.56035156)
\lineto(7.71015625,195.56035156)
\lineto(7.71015625,198.78300781)
\curveto(7.71015625,199.43925781)(7.79804688,199.96464844)(7.97382813,200.35917969)
\curveto(8.14570313,200.75761719)(8.41328125,201.06816406)(8.7765625,201.29082031)
\curveto(9.1359375,201.51738281)(9.51289063,201.63066406)(9.90742188,201.63066406)
\curveto(10.27460938,201.63066406)(10.6203125,201.53105469)(10.94453125,201.33183594)
\curveto(11.26875,201.13261719)(11.53046875,200.83183594)(11.7296875,200.42949219)
\curveto(11.88203125,200.94902344)(12.14179688,201.34746094)(12.50898438,201.62480469)
\curveto(12.87617188,201.90605469)(13.30976563,202.04667969)(13.80976563,202.04667969)
\curveto(14.21210938,202.04667969)(14.58710938,201.96074219)(14.93476563,201.78886719)
\curveto(15.27851563,201.62089844)(15.54414063,201.41191406)(15.73164063,201.16191406)
\curveto(15.91914063,200.91191406)(16.06171875,200.59746094)(16.159375,200.21855469)
\curveto(16.253125,199.84355469)(16.3,199.38261719)(16.3,198.83574219)
\closepath
\moveto(11.31953125,196.69707031)
\lineto(11.31953125,198.55449219)
\curveto(11.31953125,199.05839844)(11.28632813,199.41972656)(11.21992188,199.63847656)
\curveto(11.13398438,199.92753906)(10.99140625,200.14433594)(10.7921875,200.28886719)
\curveto(10.59296875,200.43730469)(10.34296875,200.51152344)(10.0421875,200.51152344)
\curveto(9.75703125,200.51152344)(9.50703125,200.44316406)(9.2921875,200.30644531)
\curveto(9.0734375,200.16972656)(8.925,199.97441406)(8.846875,199.72050781)
\curveto(8.76484375,199.46660156)(8.72382813,199.03105469)(8.72382813,198.41386719)
\lineto(8.72382813,196.69707031)
\closepath
\moveto(15.28632813,196.69707031)
\lineto(15.28632813,198.83574219)
\curveto(15.28632813,199.20292969)(15.27265625,199.46074219)(15.2453125,199.60917969)
\curveto(15.1984375,199.87089844)(15.1203125,200.08964844)(15.0109375,200.26542969)
\curveto(14.9015625,200.44121094)(14.74335938,200.58574219)(14.53632813,200.69902344)
\curveto(14.32539063,200.81230469)(14.08320313,200.86894531)(13.80976563,200.86894531)
\curveto(13.48945313,200.86894531)(13.21210938,200.78691406)(12.97773438,200.62285156)
\curveto(12.73945313,200.45878906)(12.5734375,200.23027344)(12.4796875,199.93730469)
\curveto(12.38203125,199.64824219)(12.33320313,199.23027344)(12.33320313,198.68339844)
\lineto(12.33320313,196.69707031)
\closepath
}
}
{
\newrgbcolor{curcolor}{0 0 0}
\pscustom[linestyle=none,fillstyle=solid,fillcolor=curcolor]
{
\newpath
\moveto(15.53242188,207.53691406)
\curveto(15.86445313,207.14628906)(16.09882813,206.76933594)(16.23554688,206.40605469)
\curveto(16.37226563,206.04667969)(16.440625,205.65996094)(16.440625,205.24589844)
\curveto(16.440625,204.56230469)(16.27460938,204.03691406)(15.94257813,203.66972656)
\curveto(15.60664063,203.30253906)(15.17890625,203.11894531)(14.659375,203.11894531)
\curveto(14.3546875,203.11894531)(14.07734375,203.18730469)(13.82734375,203.32402344)
\curveto(13.5734375,203.46464844)(13.3703125,203.64628906)(13.21796875,203.86894531)
\curveto(13.065625,204.09550781)(12.95039063,204.34941406)(12.87226563,204.63066406)
\curveto(12.81757813,204.83769531)(12.76484375,205.15019531)(12.7140625,205.56816406)
\curveto(12.6125,206.41972656)(12.49140625,207.04667969)(12.35078125,207.44902344)
\curveto(12.20625,207.45292969)(12.11445313,207.45488281)(12.07539063,207.45488281)
\curveto(11.64570313,207.45488281)(11.34296875,207.35527344)(11.1671875,207.15605469)
\curveto(10.92890625,206.88652344)(10.80976563,206.48613281)(10.80976563,205.95488281)
\curveto(10.80976563,205.45878906)(10.89765625,205.09160156)(11.0734375,204.85332031)
\curveto(11.2453125,204.61894531)(11.55195313,204.44511719)(11.99335938,204.33183594)
\lineto(11.85273438,203.30058594)
\curveto(11.41132813,203.39433594)(11.05585938,203.54863281)(10.78632813,203.76347656)
\curveto(10.51289063,203.97832031)(10.30390625,204.28886719)(10.159375,204.69511719)
\curveto(10.0109375,205.10136719)(9.93671875,205.57207031)(9.93671875,206.10722656)
\curveto(9.93671875,206.63847656)(9.99921875,207.07011719)(10.12421875,207.40214844)
\curveto(10.24921875,207.73417969)(10.40742188,207.97832031)(10.59882813,208.13457031)
\curveto(10.78632813,208.29082031)(11.02460938,208.40019531)(11.31367188,208.46269531)
\curveto(11.49335938,208.49785156)(11.81757813,208.51542969)(12.28632813,208.51542969)
\lineto(13.69257813,208.51542969)
\curveto(14.67304688,208.51542969)(15.29414063,208.53691406)(15.55585938,208.57988281)
\curveto(15.81367188,208.62675781)(16.06171875,208.71660156)(16.3,208.84941406)
\lineto(16.3,207.74785156)
\curveto(16.08125,207.63847656)(15.82539063,207.56816406)(15.53242188,207.53691406)
\closepath
\moveto(13.17695313,207.44902344)
\curveto(13.33320313,207.06621094)(13.46601563,206.49199219)(13.57539063,205.72636719)
\curveto(13.63789063,205.29277344)(13.70820313,204.98613281)(13.78632813,204.80644531)
\curveto(13.86445313,204.62675781)(13.9796875,204.48808594)(14.13203125,204.39042969)
\curveto(14.28046875,204.29277344)(14.44648438,204.24394531)(14.63007813,204.24394531)
\curveto(14.91132813,204.24394531)(15.14570313,204.34941406)(15.33320313,204.56035156)
\curveto(15.52070313,204.77519531)(15.61445313,205.08769531)(15.61445313,205.49785156)
\curveto(15.61445313,205.90410156)(15.5265625,206.26542969)(15.35078125,206.58183594)
\curveto(15.17109375,206.89824219)(14.92695313,207.13066406)(14.61835938,207.27910156)
\curveto(14.38007813,207.39238281)(14.02851563,207.44902344)(13.56367188,207.44902344)
\closepath
}
}
{
\newrgbcolor{curcolor}{0 0 0}
\pscustom[linestyle=none,fillstyle=solid,fillcolor=curcolor]
{
\newpath
\moveto(16.3,210.15019531)
\lineto(10.07734375,210.15019531)
\lineto(10.07734375,211.09941406)
\lineto(10.96210938,211.09941406)
\curveto(10.27851563,211.55644531)(9.93671875,212.21660156)(9.93671875,213.07988281)
\curveto(9.93671875,213.45488281)(10.00507813,213.79863281)(10.14179688,214.11113281)
\curveto(10.27460938,214.42753906)(10.45039063,214.66386719)(10.66914063,214.82011719)
\curveto(10.88789063,214.97636719)(11.14765625,215.08574219)(11.4484375,215.14824219)
\curveto(11.64375,215.18730469)(11.98554688,215.20683594)(12.47382813,215.20683594)
\lineto(16.3,215.20683594)
\lineto(16.3,214.15214844)
\lineto(12.51484375,214.15214844)
\curveto(12.08515625,214.15214844)(11.76484375,214.11113281)(11.55390625,214.02910156)
\curveto(11.3390625,213.94707031)(11.16914063,213.80058594)(11.04414063,213.58964844)
\curveto(10.91523438,213.38261719)(10.85078125,213.13847656)(10.85078125,212.85722656)
\curveto(10.85078125,212.40800781)(10.99335938,212.01933594)(11.27851563,211.69121094)
\curveto(11.56367188,211.36699219)(12.1046875,211.20488281)(12.9015625,211.20488281)
\lineto(16.3,211.20488281)
\closepath
}
}
{
\newrgbcolor{curcolor}{0 0 0}
\pscustom[linestyle=none,fillstyle=solid,fillcolor=curcolor]
{
\newpath
\moveto(16.3,220.86113281)
\lineto(15.51484375,220.86113281)
\curveto(16.13203125,220.46660156)(16.440625,219.88652344)(16.440625,219.12089844)
\curveto(16.440625,218.62480469)(16.30390625,218.16777344)(16.03046875,217.74980469)
\curveto(15.75703125,217.33574219)(15.37617188,217.01347656)(14.88789063,216.78300781)
\curveto(14.39570313,216.55644531)(13.83125,216.44316406)(13.19453125,216.44316406)
\curveto(12.5734375,216.44316406)(12.0109375,216.54667969)(11.50703125,216.75371094)
\curveto(10.99921875,216.96074219)(10.61054688,217.27128906)(10.34101563,217.68535156)
\curveto(10.07148438,218.09941406)(9.93671875,218.56230469)(9.93671875,219.07402344)
\curveto(9.93671875,219.44902344)(10.01679688,219.78300781)(10.17695313,220.07597656)
\curveto(10.33320313,220.36894531)(10.53828125,220.60722656)(10.7921875,220.79082031)
\lineto(7.71015625,220.79082031)
\lineto(7.71015625,221.83964844)
\lineto(16.3,221.83964844)
\closepath
\moveto(13.19453125,217.52714844)
\curveto(13.99140625,217.52714844)(14.58710938,217.69511719)(14.98164063,218.03105469)
\curveto(15.37617188,218.36699219)(15.5734375,218.76347656)(15.5734375,219.22050781)
\curveto(15.5734375,219.68144531)(15.3859375,220.07207031)(15.0109375,220.39238281)
\curveto(14.63203125,220.71660156)(14.05585938,220.87871094)(13.28242188,220.87871094)
\curveto(12.43085938,220.87871094)(11.80585938,220.71464844)(11.40742188,220.38652344)
\curveto(11.00898438,220.05839844)(10.80976563,219.65410156)(10.80976563,219.17363281)
\curveto(10.80976563,218.70488281)(11.00117188,218.31230469)(11.38398438,217.99589844)
\curveto(11.76679688,217.68339844)(12.3703125,217.52714844)(13.19453125,217.52714844)
\closepath
}
}
{
\newrgbcolor{curcolor}{0 0 0}
\pscustom[linestyle=none,fillstyle=solid,fillcolor=curcolor]
{
\newpath
\moveto(16.3,224.64628906)
\lineto(10.07734375,222.74199219)
\lineto(10.07734375,223.83183594)
\lineto(13.66914063,224.82207031)
\lineto(15.00507813,225.19121094)
\curveto(14.93867188,225.20683594)(14.5109375,225.31425781)(13.721875,225.51347656)
\lineto(10.07734375,226.50371094)
\lineto(10.07734375,227.58769531)
\lineto(13.68671875,228.51933594)
\lineto(14.87617188,228.82988281)
\lineto(13.675,229.18730469)
\lineto(10.07734375,230.25371094)
\lineto(10.07734375,231.27910156)
\lineto(16.3,229.33378906)
\lineto(16.3,228.23808594)
\lineto(12.5734375,227.24785156)
\lineto(11.51289063,227.00761719)
\lineto(16.3,225.74785156)
\closepath
}
}
{
\newrgbcolor{curcolor}{0 0 0}
\pscustom[linestyle=none,fillstyle=solid,fillcolor=curcolor]
{
\newpath
\moveto(8.92304688,232.16972656)
\lineto(7.71015625,232.16972656)
\lineto(7.71015625,233.22441406)
\lineto(8.92304688,233.22441406)
\closepath
\moveto(16.3,232.16972656)
\lineto(10.07734375,232.16972656)
\lineto(10.07734375,233.22441406)
\lineto(16.3,233.22441406)
\closepath
}
}
{
\newrgbcolor{curcolor}{0 0 0}
\pscustom[linestyle=none,fillstyle=solid,fillcolor=curcolor]
{
\newpath
\moveto(16.3,238.86699219)
\lineto(15.51484375,238.86699219)
\curveto(16.13203125,238.47246094)(16.440625,237.89238281)(16.440625,237.12675781)
\curveto(16.440625,236.63066406)(16.30390625,236.17363281)(16.03046875,235.75566406)
\curveto(15.75703125,235.34160156)(15.37617188,235.01933594)(14.88789063,234.78886719)
\curveto(14.39570313,234.56230469)(13.83125,234.44902344)(13.19453125,234.44902344)
\curveto(12.5734375,234.44902344)(12.0109375,234.55253906)(11.50703125,234.75957031)
\curveto(10.99921875,234.96660156)(10.61054688,235.27714844)(10.34101563,235.69121094)
\curveto(10.07148438,236.10527344)(9.93671875,236.56816406)(9.93671875,237.07988281)
\curveto(9.93671875,237.45488281)(10.01679688,237.78886719)(10.17695313,238.08183594)
\curveto(10.33320313,238.37480469)(10.53828125,238.61308594)(10.7921875,238.79667969)
\lineto(7.71015625,238.79667969)
\lineto(7.71015625,239.84550781)
\lineto(16.3,239.84550781)
\closepath
\moveto(13.19453125,235.53300781)
\curveto(13.99140625,235.53300781)(14.58710938,235.70097656)(14.98164063,236.03691406)
\curveto(15.37617188,236.37285156)(15.5734375,236.76933594)(15.5734375,237.22636719)
\curveto(15.5734375,237.68730469)(15.3859375,238.07792969)(15.0109375,238.39824219)
\curveto(14.63203125,238.72246094)(14.05585938,238.88457031)(13.28242188,238.88457031)
\curveto(12.43085938,238.88457031)(11.80585938,238.72050781)(11.40742188,238.39238281)
\curveto(11.00898438,238.06425781)(10.80976563,237.65996094)(10.80976563,237.17949219)
\curveto(10.80976563,236.71074219)(11.00117188,236.31816406)(11.38398438,236.00175781)
\curveto(11.76679688,235.68925781)(12.3703125,235.53300781)(13.19453125,235.53300781)
\closepath
}
}
{
\newrgbcolor{curcolor}{0 0 0}
\pscustom[linestyle=none,fillstyle=solid,fillcolor=curcolor]
{
\newpath
\moveto(15.35664063,243.80644531)
\lineto(16.28828125,243.95878906)
\curveto(16.35078125,243.66191406)(16.38203125,243.39628906)(16.38203125,243.16191406)
\curveto(16.38203125,242.77910156)(16.32148438,242.48222656)(16.20039063,242.27128906)
\curveto(16.07929688,242.06035156)(15.92109375,241.91191406)(15.72578125,241.82597656)
\curveto(15.5265625,241.74003906)(15.11054688,241.69707031)(14.47773438,241.69707031)
\lineto(10.89765625,241.69707031)
\lineto(10.89765625,240.92363281)
\lineto(10.07734375,240.92363281)
\lineto(10.07734375,241.69707031)
\lineto(8.53632813,241.69707031)
\lineto(7.90351563,242.74589844)
\lineto(10.07734375,242.74589844)
\lineto(10.07734375,243.80644531)
\lineto(10.89765625,243.80644531)
\lineto(10.89765625,242.74589844)
\lineto(14.53632813,242.74589844)
\curveto(14.83710938,242.74589844)(15.03046875,242.76347656)(15.11640625,242.79863281)
\curveto(15.20234375,242.83769531)(15.27070313,242.89824219)(15.32148438,242.98027344)
\curveto(15.37226563,243.06621094)(15.39765625,243.18730469)(15.39765625,243.34355469)
\curveto(15.39765625,243.46074219)(15.38398438,243.61503906)(15.35664063,243.80644531)
\closepath
}
}
{
\newrgbcolor{curcolor}{0 0 0}
\pscustom[linestyle=none,fillstyle=solid,fillcolor=curcolor]
{
\newpath
\moveto(16.3,244.83769531)
\lineto(7.71015625,244.83769531)
\lineto(7.71015625,245.89238281)
\lineto(10.7921875,245.89238281)
\curveto(10.221875,246.38457031)(9.93671875,247.00566406)(9.93671875,247.75566406)
\curveto(9.93671875,248.21660156)(10.02851563,248.61699219)(10.21210938,248.95683594)
\curveto(10.39179688,249.29667969)(10.64179688,249.53886719)(10.96210938,249.68339844)
\curveto(11.28242188,249.83183594)(11.74726563,249.90605469)(12.35664063,249.90605469)
\lineto(16.3,249.90605469)
\lineto(16.3,248.85136719)
\lineto(12.35664063,248.85136719)
\curveto(11.82929688,248.85136719)(11.44648438,248.73613281)(11.20820313,248.50566406)
\curveto(10.96601563,248.27910156)(10.84492188,247.95683594)(10.84492188,247.53886719)
\curveto(10.84492188,247.22636719)(10.92695313,246.93144531)(11.09101563,246.65410156)
\curveto(11.25117188,246.38066406)(11.46992188,246.18535156)(11.74726563,246.06816406)
\curveto(12.02460938,245.95097656)(12.40742188,245.89238281)(12.89570313,245.89238281)
\lineto(16.3,245.89238281)
\closepath
}
}
{
\newrgbcolor{curcolor}{0 0 0}
\pscustom[linestyle=none,fillstyle=solid,fillcolor=curcolor]
{
\newpath
\moveto(18.82539063,256.86113281)
\curveto(18.09101563,256.27910156)(17.23164063,255.78691406)(16.24726563,255.38457031)
\curveto(15.26289063,254.98222656)(14.24335938,254.78105469)(13.18867188,254.78105469)
\curveto(12.25898438,254.78105469)(11.36835938,254.93144531)(10.51679688,255.23222656)
\curveto(9.52851563,255.58378906)(8.54414063,256.12675781)(7.56367188,256.86113281)
\lineto(7.56367188,257.61699219)
\curveto(8.37617188,257.14433594)(8.95625,256.83183594)(9.30390625,256.67949219)
\curveto(9.84296875,256.44121094)(10.40546875,256.25371094)(10.99140625,256.11699219)
\curveto(11.721875,255.94902344)(12.45625,255.86503906)(13.19453125,255.86503906)
\curveto(15.0734375,255.86503906)(16.95039063,256.44902344)(18.82539063,257.61699219)
\closepath
}
}
{
\newrgbcolor{curcolor}{0 0 0}
\pscustom[linestyle=none,fillstyle=solid,fillcolor=curcolor]
{
\newpath
\moveto(12.93085938,262.99589844)
\lineto(11.92304688,262.99589844)
\lineto(11.9171875,266.63457031)
\lineto(15.1046875,266.63457031)
\curveto(15.55,266.07597656)(15.8859375,265.49980469)(16.1125,264.90605469)
\curveto(16.33515625,264.31230469)(16.44648438,263.70292969)(16.44648438,263.07792969)
\curveto(16.44648438,262.23417969)(16.26679688,261.46660156)(15.90742188,260.77519531)
\curveto(15.54414063,260.08769531)(15.02070313,259.56816406)(14.33710938,259.21660156)
\curveto(13.65351563,258.86503906)(12.88984375,258.68925781)(12.04609375,258.68925781)
\curveto(11.21015625,258.68925781)(10.43085938,258.86308594)(9.70820313,259.21074219)
\curveto(8.98164063,259.56230469)(8.44257813,260.06621094)(8.09101563,260.72246094)
\curveto(7.73945313,261.37871094)(7.56367188,262.13457031)(7.56367188,262.99003906)
\curveto(7.56367188,263.61113281)(7.66523438,264.17167969)(7.86835938,264.67167969)
\curveto(8.06757813,265.17558594)(8.346875,265.57011719)(8.70625,265.85527344)
\curveto(9.065625,266.14042969)(9.534375,266.35722656)(10.1125,266.50566406)
\lineto(10.39375,265.48027344)
\curveto(9.95625,265.35136719)(9.6125,265.19121094)(9.3625,264.99980469)
\curveto(9.1125,264.80839844)(8.91328125,264.53496094)(8.76484375,264.17949219)
\curveto(8.6125,263.82402344)(8.53632813,263.42949219)(8.53632813,262.99589844)
\curveto(8.53632813,262.47636719)(8.61640625,262.02714844)(8.7765625,261.64824219)
\curveto(8.9328125,261.26933594)(9.13984375,260.96269531)(9.39765625,260.72832031)
\curveto(9.65546875,260.49785156)(9.93867188,260.31816406)(10.24726563,260.18925781)
\curveto(10.77851563,259.97050781)(11.3546875,259.86113281)(11.97578125,259.86113281)
\curveto(12.74140625,259.86113281)(13.38203125,259.99199219)(13.89765625,260.25371094)
\curveto(14.41328125,260.51933594)(14.79609375,260.90410156)(15.04609375,261.40800781)
\curveto(15.29609375,261.91191406)(15.42109375,262.44707031)(15.42109375,263.01347656)
\curveto(15.42109375,263.50566406)(15.32734375,263.98613281)(15.13984375,264.45488281)
\curveto(14.9484375,264.92363281)(14.7453125,265.27910156)(14.53046875,265.52128906)
\lineto(12.93085938,265.52128906)
\closepath
}
}
{
\newrgbcolor{curcolor}{0 0 0}
\pscustom[linestyle=none,fillstyle=solid,fillcolor=curcolor]
{
\newpath
\moveto(16.3,268.26347656)
\lineto(7.71015625,268.26347656)
\lineto(7.71015625,271.48613281)
\curveto(7.71015625,272.14238281)(7.79804688,272.66777344)(7.97382813,273.06230469)
\curveto(8.14570313,273.46074219)(8.41328125,273.77128906)(8.7765625,273.99394531)
\curveto(9.1359375,274.22050781)(9.51289063,274.33378906)(9.90742188,274.33378906)
\curveto(10.27460938,274.33378906)(10.6203125,274.23417969)(10.94453125,274.03496094)
\curveto(11.26875,273.83574219)(11.53046875,273.53496094)(11.7296875,273.13261719)
\curveto(11.88203125,273.65214844)(12.14179688,274.05058594)(12.50898438,274.32792969)
\curveto(12.87617188,274.60917969)(13.30976563,274.74980469)(13.80976563,274.74980469)
\curveto(14.21210938,274.74980469)(14.58710938,274.66386719)(14.93476563,274.49199219)
\curveto(15.27851563,274.32402344)(15.54414063,274.11503906)(15.73164063,273.86503906)
\curveto(15.91914063,273.61503906)(16.06171875,273.30058594)(16.159375,272.92167969)
\curveto(16.253125,272.54667969)(16.3,272.08574219)(16.3,271.53886719)
\closepath
\moveto(11.31953125,269.40019531)
\lineto(11.31953125,271.25761719)
\curveto(11.31953125,271.76152344)(11.28632813,272.12285156)(11.21992188,272.34160156)
\curveto(11.13398438,272.63066406)(10.99140625,272.84746094)(10.7921875,272.99199219)
\curveto(10.59296875,273.14042969)(10.34296875,273.21464844)(10.0421875,273.21464844)
\curveto(9.75703125,273.21464844)(9.50703125,273.14628906)(9.2921875,273.00957031)
\curveto(9.0734375,272.87285156)(8.925,272.67753906)(8.846875,272.42363281)
\curveto(8.76484375,272.16972656)(8.72382813,271.73417969)(8.72382813,271.11699219)
\lineto(8.72382813,269.40019531)
\closepath
\moveto(15.28632813,269.40019531)
\lineto(15.28632813,271.53886719)
\curveto(15.28632813,271.90605469)(15.27265625,272.16386719)(15.2453125,272.31230469)
\curveto(15.1984375,272.57402344)(15.1203125,272.79277344)(15.0109375,272.96855469)
\curveto(14.9015625,273.14433594)(14.74335938,273.28886719)(14.53632813,273.40214844)
\curveto(14.32539063,273.51542969)(14.08320313,273.57207031)(13.80976563,273.57207031)
\curveto(13.48945313,273.57207031)(13.21210938,273.49003906)(12.97773438,273.32597656)
\curveto(12.73945313,273.16191406)(12.5734375,272.93339844)(12.4796875,272.64042969)
\curveto(12.38203125,272.35136719)(12.33320313,271.93339844)(12.33320313,271.38652344)
\lineto(12.33320313,269.40019531)
\closepath
}
}
{
\newrgbcolor{curcolor}{0 0 0}
\pscustom[linestyle=none,fillstyle=solid,fillcolor=curcolor]
{
\newpath
\moveto(16.44648438,275.38847656)
\lineto(7.56367188,277.87871094)
\lineto(7.56367188,278.72246094)
\lineto(16.44648438,276.23808594)
\closepath
}
}
{
\newrgbcolor{curcolor}{0 0 0}
\pscustom[linestyle=none,fillstyle=solid,fillcolor=curcolor]
{
\newpath
\moveto(14.44257813,279.09160156)
\lineto(14.27851563,280.13457031)
\curveto(14.69648438,280.19316406)(15.01679688,280.35527344)(15.23945313,280.62089844)
\curveto(15.46210938,280.89042969)(15.5734375,281.26542969)(15.5734375,281.74589844)
\curveto(15.5734375,282.23027344)(15.47578125,282.58964844)(15.28046875,282.82402344)
\curveto(15.08125,283.05839844)(14.84882813,283.17558594)(14.58320313,283.17558594)
\curveto(14.34492188,283.17558594)(14.15742188,283.07207031)(14.02070313,282.86503906)
\curveto(13.92695313,282.72050781)(13.8078125,282.36113281)(13.66328125,281.78691406)
\curveto(13.46796875,281.01347656)(13.3,280.47636719)(13.159375,280.17558594)
\curveto(13.01484375,279.87871094)(12.81757813,279.65214844)(12.56757813,279.49589844)
\curveto(12.31367188,279.34355469)(12.034375,279.26738281)(11.7296875,279.26738281)
\curveto(11.45234375,279.26738281)(11.19648438,279.32988281)(10.96210938,279.45488281)
\curveto(10.72382813,279.58378906)(10.5265625,279.75761719)(10.3703125,279.97636719)
\curveto(10.24921875,280.14042969)(10.14765625,280.36308594)(10.065625,280.64433594)
\curveto(9.9796875,280.92949219)(9.93671875,281.23417969)(9.93671875,281.55839844)
\curveto(9.93671875,282.04667969)(10.00703125,282.47441406)(10.14765625,282.84160156)
\curveto(10.28828125,283.21269531)(10.4796875,283.48613281)(10.721875,283.66191406)
\curveto(10.96015625,283.83769531)(11.28046875,283.95878906)(11.6828125,284.02519531)
\lineto(11.8234375,282.99394531)
\curveto(11.503125,282.94707031)(11.253125,282.81035156)(11.0734375,282.58378906)
\curveto(10.89375,282.36113281)(10.80390625,282.04472656)(10.80390625,281.63457031)
\curveto(10.80390625,281.15019531)(10.88398438,280.80449219)(11.04414063,280.59746094)
\curveto(11.20429688,280.39042969)(11.39179688,280.28691406)(11.60664063,280.28691406)
\curveto(11.74335938,280.28691406)(11.86640625,280.32988281)(11.97578125,280.41582031)
\curveto(12.0890625,280.50175781)(12.1828125,280.63652344)(12.25703125,280.82011719)
\curveto(12.29609375,280.92558594)(12.3859375,281.23613281)(12.5265625,281.75175781)
\curveto(12.72578125,282.49785156)(12.88984375,283.01738281)(13.01875,283.31035156)
\curveto(13.14375,283.60722656)(13.32734375,283.83964844)(13.56953125,284.00761719)
\curveto(13.81171875,284.17558594)(14.1125,284.25957031)(14.471875,284.25957031)
\curveto(14.8234375,284.25957031)(15.15546875,284.15605469)(15.46796875,283.94902344)
\curveto(15.7765625,283.74589844)(16.01679688,283.45097656)(16.18867188,283.06425781)
\curveto(16.35664063,282.67753906)(16.440625,282.24003906)(16.440625,281.75175781)
\curveto(16.440625,280.94316406)(16.27265625,280.32597656)(15.93671875,279.90019531)
\curveto(15.60078125,279.47832031)(15.10273438,279.20878906)(14.44257813,279.09160156)
\closepath
}
}
{
\newrgbcolor{curcolor}{0 0 0}
\pscustom[linestyle=none,fillstyle=solid,fillcolor=curcolor]
{
\newpath
\moveto(18.82539063,286.20488281)
\lineto(18.82539063,285.44902344)
\curveto(16.95039063,286.61699219)(15.0734375,287.20097656)(13.19453125,287.20097656)
\curveto(12.46015625,287.20097656)(11.73164063,287.11699219)(11.00898438,286.94902344)
\curveto(10.42304688,286.81621094)(9.86054688,286.63066406)(9.32148438,286.39238281)
\curveto(8.96992188,286.24003906)(8.38398438,285.92558594)(7.56367188,285.44902344)
\lineto(7.56367188,286.20488281)
\curveto(8.54414063,286.93925781)(9.52851563,287.48222656)(10.51679688,287.83378906)
\curveto(11.36835938,288.13457031)(12.25898438,288.28496094)(13.18867188,288.28496094)
\curveto(14.24335938,288.28496094)(15.26289063,288.08183594)(16.24726563,287.67558594)
\curveto(17.23164063,287.27324219)(18.09101563,286.78300781)(18.82539063,286.20488281)
\closepath
}
}
{
\newrgbcolor{curcolor}{0 0 0}
\pscustom[linestyle=none,fillstyle=solid,fillcolor=curcolor]
{
\newpath
\moveto(293.07753906,8.7)
\lineto(293.07753906,17.28984375)
\lineto(298.87246094,17.28984375)
\lineto(298.87246094,16.27617187)
\lineto(294.21425781,16.27617187)
\lineto(294.21425781,13.61601562)
\lineto(298.24550781,13.61601562)
\lineto(298.24550781,12.60234375)
\lineto(294.21425781,12.60234375)
\lineto(294.21425781,8.7)
\closepath
}
}
{
\newrgbcolor{curcolor}{0 0 0}
\pscustom[linestyle=none,fillstyle=solid,fillcolor=curcolor]
{
\newpath
\moveto(300.22011719,16.07695312)
\lineto(300.22011719,17.28984375)
\lineto(301.27480469,17.28984375)
\lineto(301.27480469,16.07695312)
\closepath
\moveto(300.22011719,8.7)
\lineto(300.22011719,14.92265625)
\lineto(301.27480469,14.92265625)
\lineto(301.27480469,8.7)
\closepath
}
}
{
\newrgbcolor{curcolor}{0 0 0}
\pscustom[linestyle=none,fillstyle=solid,fillcolor=curcolor]
{
\newpath
\moveto(302.85683594,8.7)
\lineto(302.85683594,17.28984375)
\lineto(303.91152344,17.28984375)
\lineto(303.91152344,8.7)
\closepath
}
}
{
\newrgbcolor{curcolor}{0 0 0}
\pscustom[linestyle=none,fillstyle=solid,fillcolor=curcolor]
{
\newpath
\moveto(309.80605469,10.70390625)
\lineto(310.89589844,10.56914062)
\curveto(310.72402344,9.93242187)(310.40566406,9.43828125)(309.94082031,9.08671875)
\curveto(309.47597656,8.73515625)(308.88222656,8.559375)(308.15957031,8.559375)
\curveto(307.24941406,8.559375)(306.52675781,8.83867187)(305.99160156,9.39726562)
\curveto(305.46035156,9.95976562)(305.19472656,10.746875)(305.19472656,11.75859375)
\curveto(305.19472656,12.80546875)(305.46425781,13.61796875)(306.00332031,14.19609375)
\curveto(306.54238281,14.77421875)(307.24160156,15.06328125)(308.10097656,15.06328125)
\curveto(308.93300781,15.06328125)(309.61269531,14.78007812)(310.14003906,14.21367187)
\curveto(310.66738281,13.64726562)(310.93105469,12.85039062)(310.93105469,11.82304687)
\curveto(310.93105469,11.76054687)(310.92910156,11.66679687)(310.92519531,11.54179687)
\lineto(306.28457031,11.54179687)
\curveto(306.32363281,10.85820312)(306.51699219,10.33476562)(306.86464844,9.97148437)
\curveto(307.21230469,9.60820312)(307.64589844,9.4265625)(308.16542969,9.4265625)
\curveto(308.55214844,9.4265625)(308.88222656,9.528125)(309.15566406,9.73125)
\curveto(309.42910156,9.934375)(309.64589844,10.25859375)(309.80605469,10.70390625)
\closepath
\moveto(306.34316406,12.40898437)
\lineto(309.81777344,12.40898437)
\curveto(309.77089844,12.93242187)(309.63808594,13.325)(309.41933594,13.58671875)
\curveto(309.08339844,13.99296875)(308.64785156,14.19609375)(308.11269531,14.19609375)
\curveto(307.62832031,14.19609375)(307.22011719,14.03398437)(306.88808594,13.70976562)
\curveto(306.55996094,13.38554687)(306.37832031,12.95195312)(306.34316406,12.40898437)
\closepath
}
}
{
\newrgbcolor{curcolor}{0 0 0}
\pscustom[linestyle=none,fillstyle=solid,fillcolor=curcolor]
{
\newpath
\moveto(315.30214844,11.45976562)
\lineto(316.37441406,11.55351562)
\curveto(316.42519531,11.12382812)(316.54238281,10.7703125)(316.72597656,10.49296875)
\curveto(316.91347656,10.21953125)(317.20253906,9.996875)(317.59316406,9.825)
\curveto(317.98378906,9.65703125)(318.42324219,9.57304687)(318.91152344,9.57304687)
\curveto(319.34511719,9.57304687)(319.72792969,9.6375)(320.05996094,9.76640625)
\curveto(320.39199219,9.8953125)(320.63808594,10.07109375)(320.79824219,10.29375)
\curveto(320.96230469,10.5203125)(321.04433594,10.76640625)(321.04433594,11.03203125)
\curveto(321.04433594,11.3015625)(320.96621094,11.5359375)(320.80996094,11.73515625)
\curveto(320.65371094,11.93828125)(320.39589844,12.10820312)(320.03652344,12.24492187)
\curveto(319.80605469,12.33476562)(319.29628906,12.4734375)(318.50722656,12.6609375)
\curveto(317.71816406,12.85234375)(317.16542969,13.03203125)(316.84902344,13.2)
\curveto(316.43886719,13.41484375)(316.13222656,13.68046875)(315.92910156,13.996875)
\curveto(315.72988281,14.3171875)(315.63027344,14.67460937)(315.63027344,15.06914062)
\curveto(315.63027344,15.50273437)(315.75332031,15.90703125)(315.99941406,16.28203125)
\curveto(316.24550781,16.6609375)(316.60488281,16.94804687)(317.07753906,17.14335937)
\curveto(317.55019531,17.33867187)(318.07558594,17.43632812)(318.65371094,17.43632812)
\curveto(319.29042969,17.43632812)(319.85097656,17.3328125)(320.33535156,17.12578125)
\curveto(320.82363281,16.92265625)(321.19863281,16.621875)(321.46035156,16.2234375)
\curveto(321.72207031,15.825)(321.86269531,15.37382812)(321.88222656,14.86992187)
\lineto(320.79238281,14.78789062)
\curveto(320.73378906,15.33085937)(320.53457031,15.74101562)(320.19472656,16.01835937)
\curveto(319.85878906,16.29570312)(319.36074219,16.434375)(318.70058594,16.434375)
\curveto(318.01308594,16.434375)(317.51113281,16.30742187)(317.19472656,16.05351562)
\curveto(316.88222656,15.80351562)(316.72597656,15.50078125)(316.72597656,15.1453125)
\curveto(316.72597656,14.83671875)(316.83730469,14.5828125)(317.05996094,14.38359375)
\curveto(317.27871094,14.184375)(317.84902344,13.97929687)(318.77089844,13.76835937)
\curveto(319.69667969,13.56132812)(320.33144531,13.3796875)(320.67519531,13.2234375)
\curveto(321.17519531,12.99296875)(321.54433594,12.7)(321.78261719,12.34453125)
\curveto(322.02089844,11.99296875)(322.14003906,11.58671875)(322.14003906,11.12578125)
\curveto(322.14003906,10.66875)(322.00917969,10.23710937)(321.74746094,9.83085937)
\curveto(321.48574219,9.42851562)(321.10878906,9.1140625)(320.61660156,8.8875)
\curveto(320.12832031,8.66484375)(319.57753906,8.55351562)(318.96425781,8.55351562)
\curveto(318.18691406,8.55351562)(317.53457031,8.66679687)(317.00722656,8.89335937)
\curveto(316.48378906,9.11992187)(316.07167969,9.45976562)(315.77089844,9.91289062)
\curveto(315.47402344,10.36992187)(315.31777344,10.88554687)(315.30214844,11.45976562)
\closepath
}
}
{
\newrgbcolor{curcolor}{0 0 0}
\pscustom[linestyle=none,fillstyle=solid,fillcolor=curcolor]
{
\newpath
\moveto(323.56386719,16.07695312)
\lineto(323.56386719,17.28984375)
\lineto(324.61855469,17.28984375)
\lineto(324.61855469,16.07695312)
\closepath
\moveto(323.56386719,8.7)
\lineto(323.56386719,14.92265625)
\lineto(324.61855469,14.92265625)
\lineto(324.61855469,8.7)
\closepath
}
}
{
\newrgbcolor{curcolor}{0 0 0}
\pscustom[linestyle=none,fillstyle=solid,fillcolor=curcolor]
{
\newpath
\moveto(325.66738281,8.7)
\lineto(325.66738281,9.55546875)
\lineto(329.62832031,14.10234375)
\curveto(329.17910156,14.07890625)(328.78261719,14.0671875)(328.43886719,14.0671875)
\lineto(325.90175781,14.0671875)
\lineto(325.90175781,14.92265625)
\lineto(330.98769531,14.92265625)
\lineto(330.98769531,14.22539062)
\lineto(327.61855469,10.27617187)
\lineto(326.96816406,9.55546875)
\curveto(327.44082031,9.590625)(327.88417969,9.60820312)(328.29824219,9.60820312)
\lineto(331.17519531,9.60820312)
\lineto(331.17519531,8.7)
\closepath
}
}
{
\newrgbcolor{curcolor}{0 0 0}
\pscustom[linestyle=none,fillstyle=solid,fillcolor=curcolor]
{
\newpath
\moveto(336.48378906,10.70390625)
\lineto(337.57363281,10.56914062)
\curveto(337.40175781,9.93242187)(337.08339844,9.43828125)(336.61855469,9.08671875)
\curveto(336.15371094,8.73515625)(335.55996094,8.559375)(334.83730469,8.559375)
\curveto(333.92714844,8.559375)(333.20449219,8.83867187)(332.66933594,9.39726562)
\curveto(332.13808594,9.95976562)(331.87246094,10.746875)(331.87246094,11.75859375)
\curveto(331.87246094,12.80546875)(332.14199219,13.61796875)(332.68105469,14.19609375)
\curveto(333.22011719,14.77421875)(333.91933594,15.06328125)(334.77871094,15.06328125)
\curveto(335.61074219,15.06328125)(336.29042969,14.78007812)(336.81777344,14.21367187)
\curveto(337.34511719,13.64726562)(337.60878906,12.85039062)(337.60878906,11.82304687)
\curveto(337.60878906,11.76054687)(337.60683594,11.66679687)(337.60292969,11.54179687)
\lineto(332.96230469,11.54179687)
\curveto(333.00136719,10.85820312)(333.19472656,10.33476562)(333.54238281,9.97148437)
\curveto(333.89003906,9.60820312)(334.32363281,9.4265625)(334.84316406,9.4265625)
\curveto(335.22988281,9.4265625)(335.55996094,9.528125)(335.83339844,9.73125)
\curveto(336.10683594,9.934375)(336.32363281,10.25859375)(336.48378906,10.70390625)
\closepath
\moveto(333.02089844,12.40898437)
\lineto(336.49550781,12.40898437)
\curveto(336.44863281,12.93242187)(336.31582031,13.325)(336.09707031,13.58671875)
\curveto(335.76113281,13.99296875)(335.32558594,14.19609375)(334.79042969,14.19609375)
\curveto(334.30605469,14.19609375)(333.89785156,14.03398437)(333.56582031,13.70976562)
\curveto(333.23769531,13.38554687)(333.05605469,12.95195312)(333.02089844,12.40898437)
\closepath
}
}
{
\newrgbcolor{curcolor}{0 0 0}
\pscustom[linestyle=none,fillstyle=solid,fillcolor=curcolor]
{
\newpath
\moveto(189.25234375,449)
\lineto(189.25234375,457.58984375)
\lineto(192.475,457.58984375)
\curveto(193.13125,457.58984375)(193.65664063,457.50195312)(194.05117188,457.32617188)
\curveto(194.44960938,457.15429688)(194.76015625,456.88671875)(194.9828125,456.5234375)
\curveto(195.209375,456.1640625)(195.32265625,455.78710938)(195.32265625,455.39257812)
\curveto(195.32265625,455.02539062)(195.22304688,454.6796875)(195.02382813,454.35546875)
\curveto(194.82460938,454.03125)(194.52382813,453.76953125)(194.12148438,453.5703125)
\curveto(194.64101563,453.41796875)(195.03945313,453.15820312)(195.31679688,452.79101562)
\curveto(195.59804688,452.42382812)(195.73867188,451.99023438)(195.73867188,451.49023438)
\curveto(195.73867188,451.08789062)(195.65273438,450.71289062)(195.48085938,450.36523438)
\curveto(195.31289063,450.02148438)(195.10390625,449.75585938)(194.85390625,449.56835938)
\curveto(194.60390625,449.38085938)(194.28945313,449.23828125)(193.91054688,449.140625)
\curveto(193.53554688,449.046875)(193.07460938,449)(192.52773438,449)
\closepath
\moveto(190.3890625,453.98046875)
\lineto(192.24648438,453.98046875)
\curveto(192.75039063,453.98046875)(193.11171875,454.01367188)(193.33046875,454.08007812)
\curveto(193.61953125,454.16601562)(193.83632813,454.30859375)(193.98085938,454.5078125)
\curveto(194.12929688,454.70703125)(194.20351563,454.95703125)(194.20351563,455.2578125)
\curveto(194.20351563,455.54296875)(194.13515625,455.79296875)(193.9984375,456.0078125)
\curveto(193.86171875,456.2265625)(193.66640625,456.375)(193.4125,456.453125)
\curveto(193.15859375,456.53515625)(192.72304688,456.57617188)(192.10585938,456.57617188)
\lineto(190.3890625,456.57617188)
\closepath
\moveto(190.3890625,450.01367188)
\lineto(192.52773438,450.01367188)
\curveto(192.89492188,450.01367188)(193.15273438,450.02734375)(193.30117188,450.0546875)
\curveto(193.56289063,450.1015625)(193.78164063,450.1796875)(193.95742188,450.2890625)
\curveto(194.13320313,450.3984375)(194.27773438,450.55664062)(194.39101563,450.76367188)
\curveto(194.50429688,450.97460938)(194.5609375,451.21679688)(194.5609375,451.49023438)
\curveto(194.5609375,451.81054688)(194.47890625,452.08789062)(194.31484375,452.32226562)
\curveto(194.15078125,452.56054688)(193.92226563,452.7265625)(193.62929688,452.8203125)
\curveto(193.34023438,452.91796875)(192.92226563,452.96679688)(192.37539063,452.96679688)
\lineto(190.3890625,452.96679688)
\closepath
}
}
{
\newrgbcolor{curcolor}{0 0 0}
\pscustom[linestyle=none,fillstyle=solid,fillcolor=curcolor]
{
\newpath
\moveto(201.22890625,449.76757812)
\curveto(200.83828125,449.43554688)(200.46132813,449.20117188)(200.09804688,449.06445312)
\curveto(199.73867188,448.92773438)(199.35195313,448.859375)(198.93789063,448.859375)
\curveto(198.25429688,448.859375)(197.72890625,449.02539062)(197.36171875,449.35742188)
\curveto(196.99453125,449.69335938)(196.8109375,450.12109375)(196.8109375,450.640625)
\curveto(196.8109375,450.9453125)(196.87929688,451.22265625)(197.01601563,451.47265625)
\curveto(197.15664063,451.7265625)(197.33828125,451.9296875)(197.5609375,452.08203125)
\curveto(197.7875,452.234375)(198.04140625,452.34960938)(198.32265625,452.42773438)
\curveto(198.5296875,452.48242188)(198.8421875,452.53515625)(199.26015625,452.5859375)
\curveto(200.11171875,452.6875)(200.73867188,452.80859375)(201.14101563,452.94921875)
\curveto(201.14492188,453.09375)(201.146875,453.18554688)(201.146875,453.22460938)
\curveto(201.146875,453.65429688)(201.04726563,453.95703125)(200.84804688,454.1328125)
\curveto(200.57851563,454.37109375)(200.178125,454.49023438)(199.646875,454.49023438)
\curveto(199.15078125,454.49023438)(198.78359375,454.40234375)(198.5453125,454.2265625)
\curveto(198.3109375,454.0546875)(198.13710938,453.74804688)(198.02382813,453.30664062)
\lineto(196.99257813,453.44726562)
\curveto(197.08632813,453.88867188)(197.240625,454.24414062)(197.45546875,454.51367188)
\curveto(197.6703125,454.78710938)(197.98085938,454.99609375)(198.38710938,455.140625)
\curveto(198.79335938,455.2890625)(199.2640625,455.36328125)(199.79921875,455.36328125)
\curveto(200.33046875,455.36328125)(200.76210938,455.30078125)(201.09414063,455.17578125)
\curveto(201.42617188,455.05078125)(201.6703125,454.89257812)(201.8265625,454.70117188)
\curveto(201.9828125,454.51367188)(202.0921875,454.27539062)(202.1546875,453.98632812)
\curveto(202.18984375,453.80664062)(202.20742188,453.48242188)(202.20742188,453.01367188)
\lineto(202.20742188,451.60742188)
\curveto(202.20742188,450.62695312)(202.22890625,450.00585938)(202.271875,449.74414062)
\curveto(202.31875,449.48632812)(202.40859375,449.23828125)(202.54140625,449)
\lineto(201.43984375,449)
\curveto(201.33046875,449.21875)(201.26015625,449.47460938)(201.22890625,449.76757812)
\closepath
\moveto(201.14101563,452.12304688)
\curveto(200.75820313,451.96679688)(200.18398438,451.83398438)(199.41835938,451.72460938)
\curveto(198.98476563,451.66210938)(198.678125,451.59179688)(198.4984375,451.51367188)
\curveto(198.31875,451.43554688)(198.18007813,451.3203125)(198.08242188,451.16796875)
\curveto(197.98476563,451.01953125)(197.9359375,450.85351562)(197.9359375,450.66992188)
\curveto(197.9359375,450.38867188)(198.04140625,450.15429688)(198.25234375,449.96679688)
\curveto(198.4671875,449.77929688)(198.7796875,449.68554688)(199.18984375,449.68554688)
\curveto(199.59609375,449.68554688)(199.95742188,449.7734375)(200.27382813,449.94921875)
\curveto(200.59023438,450.12890625)(200.82265625,450.37304688)(200.97109375,450.68164062)
\curveto(201.084375,450.91992188)(201.14101563,451.27148438)(201.14101563,451.73632812)
\closepath
}
}
{
\newrgbcolor{curcolor}{0 0 0}
\pscustom[linestyle=none,fillstyle=solid,fillcolor=curcolor]
{
\newpath
\moveto(203.8421875,449)
\lineto(203.8421875,455.22265625)
\lineto(204.79140625,455.22265625)
\lineto(204.79140625,454.33789062)
\curveto(205.2484375,455.02148438)(205.90859375,455.36328125)(206.771875,455.36328125)
\curveto(207.146875,455.36328125)(207.490625,455.29492188)(207.803125,455.15820312)
\curveto(208.11953125,455.02539062)(208.35585938,454.84960938)(208.51210938,454.63085938)
\curveto(208.66835938,454.41210938)(208.77773438,454.15234375)(208.84023438,453.8515625)
\curveto(208.87929688,453.65625)(208.89882813,453.31445312)(208.89882813,452.82617188)
\lineto(208.89882813,449)
\lineto(207.84414063,449)
\lineto(207.84414063,452.78515625)
\curveto(207.84414063,453.21484375)(207.803125,453.53515625)(207.72109375,453.74609375)
\curveto(207.6390625,453.9609375)(207.49257813,454.13085938)(207.28164063,454.25585938)
\curveto(207.07460938,454.38476562)(206.83046875,454.44921875)(206.54921875,454.44921875)
\curveto(206.1,454.44921875)(205.71132813,454.30664062)(205.38320313,454.02148438)
\curveto(205.05898438,453.73632812)(204.896875,453.1953125)(204.896875,452.3984375)
\lineto(204.896875,449)
\closepath
}
}
{
\newrgbcolor{curcolor}{0 0 0}
\pscustom[linestyle=none,fillstyle=solid,fillcolor=curcolor]
{
\newpath
\moveto(214.553125,449)
\lineto(214.553125,449.78515625)
\curveto(214.15859375,449.16796875)(213.57851563,448.859375)(212.81289063,448.859375)
\curveto(212.31679688,448.859375)(211.85976563,448.99609375)(211.44179688,449.26953125)
\curveto(211.02773438,449.54296875)(210.70546875,449.92382812)(210.475,450.41210938)
\curveto(210.2484375,450.90429688)(210.13515625,451.46875)(210.13515625,452.10546875)
\curveto(210.13515625,452.7265625)(210.23867188,453.2890625)(210.44570313,453.79296875)
\curveto(210.65273438,454.30078125)(210.96328125,454.68945312)(211.37734375,454.95898438)
\curveto(211.79140625,455.22851562)(212.25429688,455.36328125)(212.76601563,455.36328125)
\curveto(213.14101563,455.36328125)(213.475,455.28320312)(213.76796875,455.12304688)
\curveto(214.0609375,454.96679688)(214.29921875,454.76171875)(214.4828125,454.5078125)
\lineto(214.4828125,457.58984375)
\lineto(215.53164063,457.58984375)
\lineto(215.53164063,449)
\closepath
\moveto(211.21914063,452.10546875)
\curveto(211.21914063,451.30859375)(211.38710938,450.71289062)(211.72304688,450.31835938)
\curveto(212.05898438,449.92382812)(212.45546875,449.7265625)(212.9125,449.7265625)
\curveto(213.3734375,449.7265625)(213.7640625,449.9140625)(214.084375,450.2890625)
\curveto(214.40859375,450.66796875)(214.57070313,451.24414062)(214.57070313,452.01757812)
\curveto(214.57070313,452.86914062)(214.40664063,453.49414062)(214.07851563,453.89257812)
\curveto(213.75039063,454.29101562)(213.34609375,454.49023438)(212.865625,454.49023438)
\curveto(212.396875,454.49023438)(212.00429688,454.29882812)(211.68789063,453.91601562)
\curveto(211.37539063,453.53320312)(211.21914063,452.9296875)(211.21914063,452.10546875)
\closepath
}
}
{
\newrgbcolor{curcolor}{0 0 0}
\pscustom[linestyle=none,fillstyle=solid,fillcolor=curcolor]
{
\newpath
\moveto(218.33828125,449)
\lineto(216.43398438,455.22265625)
\lineto(217.52382813,455.22265625)
\lineto(218.5140625,451.63085938)
\lineto(218.88320313,450.29492188)
\curveto(218.89882813,450.36132812)(219.00625,450.7890625)(219.20546875,451.578125)
\lineto(220.19570313,455.22265625)
\lineto(221.2796875,455.22265625)
\lineto(222.21132813,451.61328125)
\lineto(222.521875,450.42382812)
\lineto(222.87929688,451.625)
\lineto(223.94570313,455.22265625)
\lineto(224.97109375,455.22265625)
\lineto(223.02578125,449)
\lineto(221.93007813,449)
\lineto(220.93984375,452.7265625)
\lineto(220.69960938,453.78710938)
\lineto(219.43984375,449)
\closepath
}
}
{
\newrgbcolor{curcolor}{0 0 0}
\pscustom[linestyle=none,fillstyle=solid,fillcolor=curcolor]
{
\newpath
\moveto(225.86171875,456.37695312)
\lineto(225.86171875,457.58984375)
\lineto(226.91640625,457.58984375)
\lineto(226.91640625,456.37695312)
\closepath
\moveto(225.86171875,449)
\lineto(225.86171875,455.22265625)
\lineto(226.91640625,455.22265625)
\lineto(226.91640625,449)
\closepath
}
}
{
\newrgbcolor{curcolor}{0 0 0}
\pscustom[linestyle=none,fillstyle=solid,fillcolor=curcolor]
{
\newpath
\moveto(232.55898438,449)
\lineto(232.55898438,449.78515625)
\curveto(232.16445313,449.16796875)(231.584375,448.859375)(230.81875,448.859375)
\curveto(230.32265625,448.859375)(229.865625,448.99609375)(229.44765625,449.26953125)
\curveto(229.03359375,449.54296875)(228.71132813,449.92382812)(228.48085938,450.41210938)
\curveto(228.25429688,450.90429688)(228.14101563,451.46875)(228.14101563,452.10546875)
\curveto(228.14101563,452.7265625)(228.24453125,453.2890625)(228.4515625,453.79296875)
\curveto(228.65859375,454.30078125)(228.96914063,454.68945312)(229.38320313,454.95898438)
\curveto(229.79726563,455.22851562)(230.26015625,455.36328125)(230.771875,455.36328125)
\curveto(231.146875,455.36328125)(231.48085938,455.28320312)(231.77382813,455.12304688)
\curveto(232.06679688,454.96679688)(232.30507813,454.76171875)(232.48867188,454.5078125)
\lineto(232.48867188,457.58984375)
\lineto(233.5375,457.58984375)
\lineto(233.5375,449)
\closepath
\moveto(229.225,452.10546875)
\curveto(229.225,451.30859375)(229.39296875,450.71289062)(229.72890625,450.31835938)
\curveto(230.06484375,449.92382812)(230.46132813,449.7265625)(230.91835938,449.7265625)
\curveto(231.37929688,449.7265625)(231.76992188,449.9140625)(232.09023438,450.2890625)
\curveto(232.41445313,450.66796875)(232.5765625,451.24414062)(232.5765625,452.01757812)
\curveto(232.5765625,452.86914062)(232.4125,453.49414062)(232.084375,453.89257812)
\curveto(231.75625,454.29101562)(231.35195313,454.49023438)(230.87148438,454.49023438)
\curveto(230.40273438,454.49023438)(230.01015625,454.29882812)(229.69375,453.91601562)
\curveto(229.38125,453.53320312)(229.225,452.9296875)(229.225,452.10546875)
\closepath
}
}
{
\newrgbcolor{curcolor}{0 0 0}
\pscustom[linestyle=none,fillstyle=solid,fillcolor=curcolor]
{
\newpath
\moveto(237.4984375,449.94335938)
\lineto(237.65078125,449.01171875)
\curveto(237.35390625,448.94921875)(237.08828125,448.91796875)(236.85390625,448.91796875)
\curveto(236.47109375,448.91796875)(236.17421875,448.97851562)(235.96328125,449.09960938)
\curveto(235.75234375,449.22070312)(235.60390625,449.37890625)(235.51796875,449.57421875)
\curveto(235.43203125,449.7734375)(235.3890625,450.18945312)(235.3890625,450.82226562)
\lineto(235.3890625,454.40234375)
\lineto(234.615625,454.40234375)
\lineto(234.615625,455.22265625)
\lineto(235.3890625,455.22265625)
\lineto(235.3890625,456.76367188)
\lineto(236.43789063,457.39648438)
\lineto(236.43789063,455.22265625)
\lineto(237.4984375,455.22265625)
\lineto(237.4984375,454.40234375)
\lineto(236.43789063,454.40234375)
\lineto(236.43789063,450.76367188)
\curveto(236.43789063,450.46289062)(236.45546875,450.26953125)(236.490625,450.18359375)
\curveto(236.5296875,450.09765625)(236.59023438,450.02929688)(236.67226563,449.97851562)
\curveto(236.75820313,449.92773438)(236.87929688,449.90234375)(237.03554688,449.90234375)
\curveto(237.15273438,449.90234375)(237.30703125,449.91601562)(237.4984375,449.94335938)
\closepath
}
}
{
\newrgbcolor{curcolor}{0 0 0}
\pscustom[linestyle=none,fillstyle=solid,fillcolor=curcolor]
{
\newpath
\moveto(238.5296875,449)
\lineto(238.5296875,457.58984375)
\lineto(239.584375,457.58984375)
\lineto(239.584375,454.5078125)
\curveto(240.0765625,455.078125)(240.69765625,455.36328125)(241.44765625,455.36328125)
\curveto(241.90859375,455.36328125)(242.30898438,455.27148438)(242.64882813,455.08789062)
\curveto(242.98867188,454.90820312)(243.23085938,454.65820312)(243.37539063,454.33789062)
\curveto(243.52382813,454.01757812)(243.59804688,453.55273438)(243.59804688,452.94335938)
\lineto(243.59804688,449)
\lineto(242.54335938,449)
\lineto(242.54335938,452.94335938)
\curveto(242.54335938,453.47070312)(242.428125,453.85351562)(242.19765625,454.09179688)
\curveto(241.97109375,454.33398438)(241.64882813,454.45507812)(241.23085938,454.45507812)
\curveto(240.91835938,454.45507812)(240.6234375,454.37304688)(240.34609375,454.20898438)
\curveto(240.07265625,454.04882812)(239.87734375,453.83007812)(239.76015625,453.55273438)
\curveto(239.64296875,453.27539062)(239.584375,452.89257812)(239.584375,452.40429688)
\lineto(239.584375,449)
\closepath
}
}
{
\newrgbcolor{curcolor}{0 0 0}
\pscustom[linestyle=none,fillstyle=solid,fillcolor=curcolor]
{
\newpath
\moveto(248.3265625,453.18359375)
\curveto(248.3265625,454.609375)(248.709375,455.72460938)(249.475,456.52929688)
\curveto(250.240625,457.33789062)(251.22890625,457.7421875)(252.43984375,457.7421875)
\curveto(253.2328125,457.7421875)(253.94765625,457.55273438)(254.584375,457.17382812)
\curveto(255.22109375,456.79492188)(255.70546875,456.265625)(256.0375,455.5859375)
\curveto(256.3734375,454.91015625)(256.54140625,454.14257812)(256.54140625,453.28320312)
\curveto(256.54140625,452.41210938)(256.365625,451.6328125)(256.0140625,450.9453125)
\curveto(255.6625,450.2578125)(255.16445313,449.73632812)(254.51992188,449.38085938)
\curveto(253.87539063,449.02929688)(253.18007813,448.85351562)(252.43398438,448.85351562)
\curveto(251.62539063,448.85351562)(250.90273438,449.04882812)(250.26601563,449.43945312)
\curveto(249.62929688,449.83007812)(249.146875,450.36328125)(248.81875,451.0390625)
\curveto(248.490625,451.71484375)(248.3265625,452.4296875)(248.3265625,453.18359375)
\closepath
\moveto(249.4984375,453.16601562)
\curveto(249.4984375,452.13085938)(249.77578125,451.31445312)(250.33046875,450.71679688)
\curveto(250.8890625,450.12304688)(251.58828125,449.82617188)(252.428125,449.82617188)
\curveto(253.28359375,449.82617188)(253.98671875,450.12695312)(254.5375,450.72851562)
\curveto(255.0921875,451.33007812)(255.36953125,452.18359375)(255.36953125,453.2890625)
\curveto(255.36953125,453.98828125)(255.25039063,454.59765625)(255.01210938,455.1171875)
\curveto(254.77773438,455.640625)(254.43203125,456.04492188)(253.975,456.33007812)
\curveto(253.521875,456.61914062)(253.01210938,456.76367188)(252.44570313,456.76367188)
\curveto(251.64101563,456.76367188)(250.94765625,456.48632812)(250.365625,455.93164062)
\curveto(249.7875,455.38085938)(249.4984375,454.45898438)(249.4984375,453.16601562)
\closepath
}
}
{
\newrgbcolor{curcolor}{0 0 0}
\pscustom[linestyle=none,fillstyle=solid,fillcolor=curcolor]
{
\newpath
\moveto(257.87148438,446.61523438)
\lineto(257.87148438,455.22265625)
\lineto(258.83242188,455.22265625)
\lineto(258.83242188,454.4140625)
\curveto(259.05898438,454.73046875)(259.31484375,454.96679688)(259.6,455.12304688)
\curveto(259.88515625,455.28320312)(260.23085938,455.36328125)(260.63710938,455.36328125)
\curveto(261.16835938,455.36328125)(261.63710938,455.2265625)(262.04335938,454.953125)
\curveto(262.44960938,454.6796875)(262.75625,454.29296875)(262.96328125,453.79296875)
\curveto(263.1703125,453.296875)(263.27382813,452.75195312)(263.27382813,452.15820312)
\curveto(263.27382813,451.52148438)(263.15859375,450.94726562)(262.928125,450.43554688)
\curveto(262.7015625,449.92773438)(262.36953125,449.53710938)(261.93203125,449.26367188)
\curveto(261.4984375,448.99414062)(261.04140625,448.859375)(260.5609375,448.859375)
\curveto(260.209375,448.859375)(259.89296875,448.93359375)(259.61171875,449.08203125)
\curveto(259.334375,449.23046875)(259.10585938,449.41796875)(258.92617188,449.64453125)
\lineto(258.92617188,446.61523438)
\closepath
\moveto(258.8265625,452.07617188)
\curveto(258.8265625,451.27539062)(258.98867188,450.68359375)(259.31289063,450.30078125)
\curveto(259.63710938,449.91796875)(260.0296875,449.7265625)(260.490625,449.7265625)
\curveto(260.959375,449.7265625)(261.35976563,449.92382812)(261.69179688,450.31835938)
\curveto(262.02773438,450.71679688)(262.19570313,451.33203125)(262.19570313,452.1640625)
\curveto(262.19570313,452.95703125)(262.03164063,453.55078125)(261.70351563,453.9453125)
\curveto(261.37929688,454.33984375)(260.990625,454.53710938)(260.5375,454.53710938)
\curveto(260.08828125,454.53710938)(259.68984375,454.32617188)(259.3421875,453.90429688)
\curveto(258.9984375,453.48632812)(258.8265625,452.87695312)(258.8265625,452.07617188)
\closepath
}
}
{
\newrgbcolor{curcolor}{0 0 0}
\pscustom[linestyle=none,fillstyle=solid,fillcolor=curcolor]
{
\newpath
\moveto(266.84804688,449.94335938)
\lineto(267.00039063,449.01171875)
\curveto(266.70351563,448.94921875)(266.43789063,448.91796875)(266.20351563,448.91796875)
\curveto(265.82070313,448.91796875)(265.52382813,448.97851562)(265.31289063,449.09960938)
\curveto(265.10195313,449.22070312)(264.95351563,449.37890625)(264.86757813,449.57421875)
\curveto(264.78164063,449.7734375)(264.73867188,450.18945312)(264.73867188,450.82226562)
\lineto(264.73867188,454.40234375)
\lineto(263.96523438,454.40234375)
\lineto(263.96523438,455.22265625)
\lineto(264.73867188,455.22265625)
\lineto(264.73867188,456.76367188)
\lineto(265.7875,457.39648438)
\lineto(265.7875,455.22265625)
\lineto(266.84804688,455.22265625)
\lineto(266.84804688,454.40234375)
\lineto(265.7875,454.40234375)
\lineto(265.7875,450.76367188)
\curveto(265.7875,450.46289062)(265.80507813,450.26953125)(265.84023438,450.18359375)
\curveto(265.87929688,450.09765625)(265.93984375,450.02929688)(266.021875,449.97851562)
\curveto(266.1078125,449.92773438)(266.22890625,449.90234375)(266.38515625,449.90234375)
\curveto(266.50234375,449.90234375)(266.65664063,449.91601562)(266.84804688,449.94335938)
\closepath
}
}
{
\newrgbcolor{curcolor}{0 0 0}
\pscustom[linestyle=none,fillstyle=solid,fillcolor=curcolor]
{
\newpath
\moveto(267.88515625,456.37695312)
\lineto(267.88515625,457.58984375)
\lineto(268.93984375,457.58984375)
\lineto(268.93984375,456.37695312)
\closepath
\moveto(267.88515625,449)
\lineto(267.88515625,455.22265625)
\lineto(268.93984375,455.22265625)
\lineto(268.93984375,449)
\closepath
}
}
{
\newrgbcolor{curcolor}{0 0 0}
\pscustom[linestyle=none,fillstyle=solid,fillcolor=curcolor]
{
\newpath
\moveto(270.5453125,449)
\lineto(270.5453125,455.22265625)
\lineto(271.48867188,455.22265625)
\lineto(271.48867188,454.34960938)
\curveto(271.68398438,454.65429688)(271.94375,454.8984375)(272.26796875,455.08203125)
\curveto(272.5921875,455.26953125)(272.96132813,455.36328125)(273.37539063,455.36328125)
\curveto(273.83632813,455.36328125)(274.21328125,455.26757812)(274.50625,455.07617188)
\curveto(274.803125,454.88476562)(275.01210938,454.6171875)(275.13320313,454.2734375)
\curveto(275.62539063,455)(276.26601563,455.36328125)(277.05507813,455.36328125)
\curveto(277.67226563,455.36328125)(278.146875,455.19140625)(278.47890625,454.84765625)
\curveto(278.8109375,454.5078125)(278.97695313,453.98242188)(278.97695313,453.27148438)
\lineto(278.97695313,449)
\lineto(277.928125,449)
\lineto(277.928125,452.91992188)
\curveto(277.928125,453.34179688)(277.89296875,453.64453125)(277.82265625,453.828125)
\curveto(277.75625,454.015625)(277.63320313,454.16601562)(277.45351563,454.27929688)
\curveto(277.27382813,454.39257812)(277.06289063,454.44921875)(276.82070313,454.44921875)
\curveto(276.38320313,454.44921875)(276.01992188,454.30273438)(275.73085938,454.00976562)
\curveto(275.44179688,453.72070312)(275.29726563,453.25585938)(275.29726563,452.61523438)
\lineto(275.29726563,449)
\lineto(274.24257813,449)
\lineto(274.24257813,453.04296875)
\curveto(274.24257813,453.51171875)(274.15664063,453.86328125)(273.98476563,454.09765625)
\curveto(273.81289063,454.33203125)(273.53164063,454.44921875)(273.14101563,454.44921875)
\curveto(272.84414063,454.44921875)(272.56875,454.37109375)(272.31484375,454.21484375)
\curveto(272.06484375,454.05859375)(271.88320313,453.83007812)(271.76992188,453.52929688)
\curveto(271.65664063,453.22851562)(271.6,452.79492188)(271.6,452.22851562)
\lineto(271.6,449)
\closepath
}
}
{
\newrgbcolor{curcolor}{0 0 0}
\pscustom[linestyle=none,fillstyle=solid,fillcolor=curcolor]
{
\newpath
\moveto(284.60195313,449.76757812)
\curveto(284.21132813,449.43554688)(283.834375,449.20117188)(283.47109375,449.06445312)
\curveto(283.11171875,448.92773438)(282.725,448.859375)(282.3109375,448.859375)
\curveto(281.62734375,448.859375)(281.10195313,449.02539062)(280.73476563,449.35742188)
\curveto(280.36757813,449.69335938)(280.18398438,450.12109375)(280.18398438,450.640625)
\curveto(280.18398438,450.9453125)(280.25234375,451.22265625)(280.3890625,451.47265625)
\curveto(280.5296875,451.7265625)(280.71132813,451.9296875)(280.93398438,452.08203125)
\curveto(281.16054688,452.234375)(281.41445313,452.34960938)(281.69570313,452.42773438)
\curveto(281.90273438,452.48242188)(282.21523438,452.53515625)(282.63320313,452.5859375)
\curveto(283.48476563,452.6875)(284.11171875,452.80859375)(284.5140625,452.94921875)
\curveto(284.51796875,453.09375)(284.51992188,453.18554688)(284.51992188,453.22460938)
\curveto(284.51992188,453.65429688)(284.4203125,453.95703125)(284.22109375,454.1328125)
\curveto(283.9515625,454.37109375)(283.55117188,454.49023438)(283.01992188,454.49023438)
\curveto(282.52382813,454.49023438)(282.15664063,454.40234375)(281.91835938,454.2265625)
\curveto(281.68398438,454.0546875)(281.51015625,453.74804688)(281.396875,453.30664062)
\lineto(280.365625,453.44726562)
\curveto(280.459375,453.88867188)(280.61367188,454.24414062)(280.82851563,454.51367188)
\curveto(281.04335938,454.78710938)(281.35390625,454.99609375)(281.76015625,455.140625)
\curveto(282.16640625,455.2890625)(282.63710938,455.36328125)(283.17226563,455.36328125)
\curveto(283.70351563,455.36328125)(284.13515625,455.30078125)(284.4671875,455.17578125)
\curveto(284.79921875,455.05078125)(285.04335938,454.89257812)(285.19960938,454.70117188)
\curveto(285.35585938,454.51367188)(285.46523438,454.27539062)(285.52773438,453.98632812)
\curveto(285.56289063,453.80664062)(285.58046875,453.48242188)(285.58046875,453.01367188)
\lineto(285.58046875,451.60742188)
\curveto(285.58046875,450.62695312)(285.60195313,450.00585938)(285.64492188,449.74414062)
\curveto(285.69179688,449.48632812)(285.78164063,449.23828125)(285.91445313,449)
\lineto(284.81289063,449)
\curveto(284.70351563,449.21875)(284.63320313,449.47460938)(284.60195313,449.76757812)
\closepath
\moveto(284.5140625,452.12304688)
\curveto(284.13125,451.96679688)(283.55703125,451.83398438)(282.79140625,451.72460938)
\curveto(282.3578125,451.66210938)(282.05117188,451.59179688)(281.87148438,451.51367188)
\curveto(281.69179688,451.43554688)(281.553125,451.3203125)(281.45546875,451.16796875)
\curveto(281.3578125,451.01953125)(281.30898438,450.85351562)(281.30898438,450.66992188)
\curveto(281.30898438,450.38867188)(281.41445313,450.15429688)(281.62539063,449.96679688)
\curveto(281.84023438,449.77929688)(282.15273438,449.68554688)(282.56289063,449.68554688)
\curveto(282.96914063,449.68554688)(283.33046875,449.7734375)(283.646875,449.94921875)
\curveto(283.96328125,450.12890625)(284.19570313,450.37304688)(284.34414063,450.68164062)
\curveto(284.45742188,450.91992188)(284.5140625,451.27148438)(284.5140625,451.73632812)
\closepath
}
}
{
\newrgbcolor{curcolor}{0 0 0}
\pscustom[linestyle=none,fillstyle=solid,fillcolor=curcolor]
{
\newpath
\moveto(287.19179688,449)
\lineto(287.19179688,457.58984375)
\lineto(288.24648438,457.58984375)
\lineto(288.24648438,449)
\closepath
}
}
{
\newrgbcolor{curcolor}{0 0 0}
\pscustom[linestyle=none,fillstyle=solid,fillcolor=curcolor]
{
\newpath
\moveto(298.46523438,450.01367188)
\lineto(298.46523438,449)
\lineto(292.7875,449)
\curveto(292.7796875,449.25390625)(292.82070313,449.49804688)(292.91054688,449.73242188)
\curveto(293.05507813,450.11914062)(293.28554688,450.5)(293.60195313,450.875)
\curveto(293.92226563,451.25)(294.38320313,451.68359375)(294.98476563,452.17578125)
\curveto(295.91835938,452.94140625)(296.54921875,453.546875)(296.87734375,453.9921875)
\curveto(297.20546875,454.44140625)(297.36953125,454.86523438)(297.36953125,455.26367188)
\curveto(297.36953125,455.68164062)(297.21914063,456.03320312)(296.91835938,456.31835938)
\curveto(296.62148438,456.60742188)(296.2328125,456.75195312)(295.75234375,456.75195312)
\curveto(295.24453125,456.75195312)(294.83828125,456.59960938)(294.53359375,456.29492188)
\curveto(294.22890625,455.99023438)(294.07460938,455.56835938)(294.07070313,455.02929688)
\lineto(292.98671875,455.140625)
\curveto(293.0609375,455.94921875)(293.34023438,456.56445312)(293.82460938,456.98632812)
\curveto(294.30898438,457.41210938)(294.959375,457.625)(295.77578125,457.625)
\curveto(296.6,457.625)(297.25234375,457.39648438)(297.7328125,456.93945312)
\curveto(298.21328125,456.48242188)(298.45351563,455.91601562)(298.45351563,455.24023438)
\curveto(298.45351563,454.89648438)(298.38320313,454.55859375)(298.24257813,454.2265625)
\curveto(298.10195313,453.89453125)(297.86757813,453.54492188)(297.53945313,453.17773438)
\curveto(297.21523438,452.81054688)(296.67421875,452.30664062)(295.91640625,451.66601562)
\curveto(295.28359375,451.13476562)(294.87734375,450.7734375)(294.69765625,450.58203125)
\curveto(294.51796875,450.39453125)(294.36953125,450.20507812)(294.25234375,450.01367188)
\closepath
}
}
{
\newrgbcolor{curcolor}{0 0 0}
\pscustom[linestyle=none,fillstyle=solid,fillcolor=curcolor]
{
\newpath
\moveto(299.98867188,449)
\lineto(299.98867188,457.58984375)
\lineto(301.69960938,457.58984375)
\lineto(303.7328125,451.5078125)
\curveto(303.9203125,450.94140625)(304.05703125,450.51757812)(304.14296875,450.23632812)
\curveto(304.240625,450.54882812)(304.39296875,451.0078125)(304.6,451.61328125)
\lineto(306.65664063,457.58984375)
\lineto(308.1859375,457.58984375)
\lineto(308.1859375,449)
\lineto(307.09023438,449)
\lineto(307.09023438,456.18945312)
\lineto(304.59414063,449)
\lineto(303.56875,449)
\lineto(301.084375,456.3125)
\lineto(301.084375,449)
\closepath
}
}
{
\newrgbcolor{curcolor}{0 0 0}
\pscustom[linestyle=none,fillstyle=solid,fillcolor=curcolor]
{
\newpath
\moveto(313.37148438,449)
\lineto(313.37148438,457.58984375)
\lineto(317.18007813,457.58984375)
\curveto(317.94570313,457.58984375)(318.52773438,457.51171875)(318.92617188,457.35546875)
\curveto(319.32460938,457.203125)(319.64296875,456.93164062)(319.88125,456.54101562)
\curveto(320.11953125,456.15039062)(320.23867188,455.71875)(320.23867188,455.24609375)
\curveto(320.23867188,454.63671875)(320.04140625,454.12304688)(319.646875,453.70507812)
\curveto(319.25234375,453.28710938)(318.64296875,453.02148438)(317.81875,452.90820312)
\curveto(318.11953125,452.76367188)(318.34804688,452.62109375)(318.50429688,452.48046875)
\curveto(318.83632813,452.17578125)(319.15078125,451.79492188)(319.44765625,451.33789062)
\lineto(320.94179688,449)
\lineto(319.51210938,449)
\lineto(318.37539063,450.78710938)
\curveto(318.04335938,451.30273438)(317.76992188,451.69726562)(317.55507813,451.97070312)
\curveto(317.34023438,452.24414062)(317.146875,452.43554688)(316.975,452.54492188)
\curveto(316.80703125,452.65429688)(316.63515625,452.73046875)(316.459375,452.7734375)
\curveto(316.33046875,452.80078125)(316.11953125,452.81445312)(315.8265625,452.81445312)
\lineto(314.50820313,452.81445312)
\lineto(314.50820313,449)
\closepath
\moveto(314.50820313,453.79882812)
\lineto(316.9515625,453.79882812)
\curveto(317.47109375,453.79882812)(317.87734375,453.8515625)(318.1703125,453.95703125)
\curveto(318.46328125,454.06640625)(318.6859375,454.23828125)(318.83828125,454.47265625)
\curveto(318.990625,454.7109375)(319.06679688,454.96875)(319.06679688,455.24609375)
\curveto(319.06679688,455.65234375)(318.91835938,455.98632812)(318.62148438,456.24804688)
\curveto(318.32851563,456.50976562)(317.86367188,456.640625)(317.22695313,456.640625)
\lineto(314.50820313,456.640625)
\closepath
}
}
{
\newrgbcolor{curcolor}{0 0 0}
\pscustom[linestyle=none,fillstyle=solid,fillcolor=curcolor]
{
\newpath
\moveto(326.14492188,451.00390625)
\lineto(327.23476563,450.86914062)
\curveto(327.06289063,450.23242188)(326.74453125,449.73828125)(326.2796875,449.38671875)
\curveto(325.81484375,449.03515625)(325.22109375,448.859375)(324.4984375,448.859375)
\curveto(323.58828125,448.859375)(322.865625,449.13867188)(322.33046875,449.69726562)
\curveto(321.79921875,450.25976562)(321.53359375,451.046875)(321.53359375,452.05859375)
\curveto(321.53359375,453.10546875)(321.803125,453.91796875)(322.3421875,454.49609375)
\curveto(322.88125,455.07421875)(323.58046875,455.36328125)(324.43984375,455.36328125)
\curveto(325.271875,455.36328125)(325.9515625,455.08007812)(326.47890625,454.51367188)
\curveto(327.00625,453.94726562)(327.26992188,453.15039062)(327.26992188,452.12304688)
\curveto(327.26992188,452.06054688)(327.26796875,451.96679688)(327.2640625,451.84179688)
\lineto(322.6234375,451.84179688)
\curveto(322.6625,451.15820312)(322.85585938,450.63476562)(323.20351563,450.27148438)
\curveto(323.55117188,449.90820312)(323.98476563,449.7265625)(324.50429688,449.7265625)
\curveto(324.89101563,449.7265625)(325.22109375,449.828125)(325.49453125,450.03125)
\curveto(325.76796875,450.234375)(325.98476563,450.55859375)(326.14492188,451.00390625)
\closepath
\moveto(322.68203125,452.70898438)
\lineto(326.15664063,452.70898438)
\curveto(326.10976563,453.23242188)(325.97695313,453.625)(325.75820313,453.88671875)
\curveto(325.42226563,454.29296875)(324.98671875,454.49609375)(324.4515625,454.49609375)
\curveto(323.9671875,454.49609375)(323.55898438,454.33398438)(323.22695313,454.00976562)
\curveto(322.89882813,453.68554688)(322.7171875,453.25195312)(322.68203125,452.70898438)
\closepath
}
}
{
\newrgbcolor{curcolor}{0 0 0}
\pscustom[linestyle=none,fillstyle=solid,fillcolor=curcolor]
{
\newpath
\moveto(332.61953125,449.76757812)
\curveto(332.22890625,449.43554688)(331.85195313,449.20117188)(331.48867188,449.06445312)
\curveto(331.12929688,448.92773438)(330.74257813,448.859375)(330.32851563,448.859375)
\curveto(329.64492188,448.859375)(329.11953125,449.02539062)(328.75234375,449.35742188)
\curveto(328.38515625,449.69335938)(328.2015625,450.12109375)(328.2015625,450.640625)
\curveto(328.2015625,450.9453125)(328.26992188,451.22265625)(328.40664063,451.47265625)
\curveto(328.54726563,451.7265625)(328.72890625,451.9296875)(328.9515625,452.08203125)
\curveto(329.178125,452.234375)(329.43203125,452.34960938)(329.71328125,452.42773438)
\curveto(329.9203125,452.48242188)(330.2328125,452.53515625)(330.65078125,452.5859375)
\curveto(331.50234375,452.6875)(332.12929688,452.80859375)(332.53164063,452.94921875)
\curveto(332.53554688,453.09375)(332.5375,453.18554688)(332.5375,453.22460938)
\curveto(332.5375,453.65429688)(332.43789063,453.95703125)(332.23867188,454.1328125)
\curveto(331.96914063,454.37109375)(331.56875,454.49023438)(331.0375,454.49023438)
\curveto(330.54140625,454.49023438)(330.17421875,454.40234375)(329.9359375,454.2265625)
\curveto(329.7015625,454.0546875)(329.52773438,453.74804688)(329.41445313,453.30664062)
\lineto(328.38320313,453.44726562)
\curveto(328.47695313,453.88867188)(328.63125,454.24414062)(328.84609375,454.51367188)
\curveto(329.0609375,454.78710938)(329.37148438,454.99609375)(329.77773438,455.140625)
\curveto(330.18398438,455.2890625)(330.6546875,455.36328125)(331.18984375,455.36328125)
\curveto(331.72109375,455.36328125)(332.15273438,455.30078125)(332.48476563,455.17578125)
\curveto(332.81679688,455.05078125)(333.0609375,454.89257812)(333.2171875,454.70117188)
\curveto(333.3734375,454.51367188)(333.4828125,454.27539062)(333.5453125,453.98632812)
\curveto(333.58046875,453.80664062)(333.59804688,453.48242188)(333.59804688,453.01367188)
\lineto(333.59804688,451.60742188)
\curveto(333.59804688,450.62695312)(333.61953125,450.00585938)(333.6625,449.74414062)
\curveto(333.709375,449.48632812)(333.79921875,449.23828125)(333.93203125,449)
\lineto(332.83046875,449)
\curveto(332.72109375,449.21875)(332.65078125,449.47460938)(332.61953125,449.76757812)
\closepath
\moveto(332.53164063,452.12304688)
\curveto(332.14882813,451.96679688)(331.57460938,451.83398438)(330.80898438,451.72460938)
\curveto(330.37539063,451.66210938)(330.06875,451.59179688)(329.8890625,451.51367188)
\curveto(329.709375,451.43554688)(329.57070313,451.3203125)(329.47304688,451.16796875)
\curveto(329.37539063,451.01953125)(329.3265625,450.85351562)(329.3265625,450.66992188)
\curveto(329.3265625,450.38867188)(329.43203125,450.15429688)(329.64296875,449.96679688)
\curveto(329.8578125,449.77929688)(330.1703125,449.68554688)(330.58046875,449.68554688)
\curveto(330.98671875,449.68554688)(331.34804688,449.7734375)(331.66445313,449.94921875)
\curveto(331.98085938,450.12890625)(332.21328125,450.37304688)(332.36171875,450.68164062)
\curveto(332.475,450.91992188)(332.53164063,451.27148438)(332.53164063,451.73632812)
\closepath
}
}
{
\newrgbcolor{curcolor}{0 0 0}
\pscustom[linestyle=none,fillstyle=solid,fillcolor=curcolor]
{
\newpath
\moveto(339.26992188,449)
\lineto(339.26992188,449.78515625)
\curveto(338.87539063,449.16796875)(338.2953125,448.859375)(337.5296875,448.859375)
\curveto(337.03359375,448.859375)(336.5765625,448.99609375)(336.15859375,449.26953125)
\curveto(335.74453125,449.54296875)(335.42226563,449.92382812)(335.19179688,450.41210938)
\curveto(334.96523438,450.90429688)(334.85195313,451.46875)(334.85195313,452.10546875)
\curveto(334.85195313,452.7265625)(334.95546875,453.2890625)(335.1625,453.79296875)
\curveto(335.36953125,454.30078125)(335.68007813,454.68945312)(336.09414063,454.95898438)
\curveto(336.50820313,455.22851562)(336.97109375,455.36328125)(337.4828125,455.36328125)
\curveto(337.8578125,455.36328125)(338.19179688,455.28320312)(338.48476563,455.12304688)
\curveto(338.77773438,454.96679688)(339.01601563,454.76171875)(339.19960938,454.5078125)
\lineto(339.19960938,457.58984375)
\lineto(340.2484375,457.58984375)
\lineto(340.2484375,449)
\closepath
\moveto(335.9359375,452.10546875)
\curveto(335.9359375,451.30859375)(336.10390625,450.71289062)(336.43984375,450.31835938)
\curveto(336.77578125,449.92382812)(337.17226563,449.7265625)(337.62929688,449.7265625)
\curveto(338.09023438,449.7265625)(338.48085938,449.9140625)(338.80117188,450.2890625)
\curveto(339.12539063,450.66796875)(339.2875,451.24414062)(339.2875,452.01757812)
\curveto(339.2875,452.86914062)(339.1234375,453.49414062)(338.7953125,453.89257812)
\curveto(338.4671875,454.29101562)(338.06289063,454.49023438)(337.58242188,454.49023438)
\curveto(337.11367188,454.49023438)(336.72109375,454.29882812)(336.4046875,453.91601562)
\curveto(336.0921875,453.53320312)(335.9359375,452.9296875)(335.9359375,452.10546875)
\closepath
}
}
{
\newrgbcolor{curcolor}{0 0 0}
\pscustom[linestyle=none,fillstyle=solid,fillcolor=curcolor]
{
\newpath
\moveto(341.48476563,450.85742188)
\lineto(342.52773438,451.02148438)
\curveto(342.58632813,450.60351562)(342.7484375,450.28320312)(343.0140625,450.06054688)
\curveto(343.28359375,449.83789062)(343.65859375,449.7265625)(344.1390625,449.7265625)
\curveto(344.6234375,449.7265625)(344.9828125,449.82421875)(345.2171875,450.01953125)
\curveto(345.4515625,450.21875)(345.56875,450.45117188)(345.56875,450.71679688)
\curveto(345.56875,450.95507812)(345.46523438,451.14257812)(345.25820313,451.27929688)
\curveto(345.11367188,451.37304688)(344.75429688,451.4921875)(344.18007813,451.63671875)
\curveto(343.40664063,451.83203125)(342.86953125,452)(342.56875,452.140625)
\curveto(342.271875,452.28515625)(342.0453125,452.48242188)(341.8890625,452.73242188)
\curveto(341.73671875,452.98632812)(341.66054688,453.265625)(341.66054688,453.5703125)
\curveto(341.66054688,453.84765625)(341.72304688,454.10351562)(341.84804688,454.33789062)
\curveto(341.97695313,454.57617188)(342.15078125,454.7734375)(342.36953125,454.9296875)
\curveto(342.53359375,455.05078125)(342.75625,455.15234375)(343.0375,455.234375)
\curveto(343.32265625,455.3203125)(343.62734375,455.36328125)(343.9515625,455.36328125)
\curveto(344.43984375,455.36328125)(344.86757813,455.29296875)(345.23476563,455.15234375)
\curveto(345.60585938,455.01171875)(345.87929688,454.8203125)(346.05507813,454.578125)
\curveto(346.23085938,454.33984375)(346.35195313,454.01953125)(346.41835938,453.6171875)
\lineto(345.38710938,453.4765625)
\curveto(345.34023438,453.796875)(345.20351563,454.046875)(344.97695313,454.2265625)
\curveto(344.75429688,454.40625)(344.43789063,454.49609375)(344.02773438,454.49609375)
\curveto(343.54335938,454.49609375)(343.19765625,454.41601562)(342.990625,454.25585938)
\curveto(342.78359375,454.09570312)(342.68007813,453.90820312)(342.68007813,453.69335938)
\curveto(342.68007813,453.55664062)(342.72304688,453.43359375)(342.80898438,453.32421875)
\curveto(342.89492188,453.2109375)(343.0296875,453.1171875)(343.21328125,453.04296875)
\curveto(343.31875,453.00390625)(343.62929688,452.9140625)(344.14492188,452.7734375)
\curveto(344.89101563,452.57421875)(345.41054688,452.41015625)(345.70351563,452.28125)
\curveto(346.00039063,452.15625)(346.2328125,451.97265625)(346.40078125,451.73046875)
\curveto(346.56875,451.48828125)(346.65273438,451.1875)(346.65273438,450.828125)
\curveto(346.65273438,450.4765625)(346.54921875,450.14453125)(346.3421875,449.83203125)
\curveto(346.1390625,449.5234375)(345.84414063,449.28320312)(345.45742188,449.11132812)
\curveto(345.07070313,448.94335938)(344.63320313,448.859375)(344.14492188,448.859375)
\curveto(343.33632813,448.859375)(342.71914063,449.02734375)(342.29335938,449.36328125)
\curveto(341.87148438,449.69921875)(341.60195313,450.19726562)(341.48476563,450.85742188)
\closepath
}
}
{
\newrgbcolor{curcolor}{0 0 0}
\pscustom[linestyle=none,fillstyle=solid,fillcolor=curcolor]
{
\newpath
\moveto(351.49257813,449)
\lineto(351.49257813,454.40234375)
\lineto(350.5609375,454.40234375)
\lineto(350.5609375,455.22265625)
\lineto(351.49257813,455.22265625)
\lineto(351.49257813,455.88476562)
\curveto(351.49257813,456.30273438)(351.5296875,456.61328125)(351.60390625,456.81640625)
\curveto(351.70546875,457.08984375)(351.88320313,457.31054688)(352.13710938,457.47851562)
\curveto(352.39492188,457.65039062)(352.75429688,457.73632812)(353.21523438,457.73632812)
\curveto(353.51210938,457.73632812)(353.84023438,457.70117188)(354.19960938,457.63085938)
\lineto(354.04140625,456.7109375)
\curveto(353.82265625,456.75)(353.615625,456.76953125)(353.4203125,456.76953125)
\curveto(353.1,456.76953125)(352.8734375,456.70117188)(352.740625,456.56445312)
\curveto(352.6078125,456.42773438)(352.54140625,456.171875)(352.54140625,455.796875)
\lineto(352.54140625,455.22265625)
\lineto(353.75429688,455.22265625)
\lineto(353.75429688,454.40234375)
\lineto(352.54140625,454.40234375)
\lineto(352.54140625,449)
\closepath
}
}
{
\newrgbcolor{curcolor}{0 0 0}
\pscustom[linestyle=none,fillstyle=solid,fillcolor=curcolor]
{
\newpath
\moveto(354.58046875,456.37695312)
\lineto(354.58046875,457.58984375)
\lineto(355.63515625,457.58984375)
\lineto(355.63515625,456.37695312)
\closepath
\moveto(354.58046875,449)
\lineto(354.58046875,455.22265625)
\lineto(355.63515625,455.22265625)
\lineto(355.63515625,449)
\closepath
}
}
{
\newrgbcolor{curcolor}{0 0 0}
\pscustom[linestyle=none,fillstyle=solid,fillcolor=curcolor]
{
\newpath
\moveto(356.84804688,452.11132812)
\curveto(356.84804688,453.26367188)(357.16835938,454.1171875)(357.80898438,454.671875)
\curveto(358.34414063,455.1328125)(358.99648438,455.36328125)(359.76601563,455.36328125)
\curveto(360.62148438,455.36328125)(361.32070313,455.08203125)(361.86367188,454.51953125)
\curveto(362.40664063,453.9609375)(362.678125,453.1875)(362.678125,452.19921875)
\curveto(362.678125,451.3984375)(362.55703125,450.76757812)(362.31484375,450.30664062)
\curveto(362.0765625,449.84960938)(361.72695313,449.49414062)(361.26601563,449.24023438)
\curveto(360.80898438,448.98632812)(360.30898438,448.859375)(359.76601563,448.859375)
\curveto(358.89492188,448.859375)(358.18984375,449.13867188)(357.65078125,449.69726562)
\curveto(357.115625,450.25585938)(356.84804688,451.06054688)(356.84804688,452.11132812)
\closepath
\moveto(357.93203125,452.11132812)
\curveto(357.93203125,451.31445312)(358.10585938,450.71679688)(358.45351563,450.31835938)
\curveto(358.80117188,449.92382812)(359.23867188,449.7265625)(359.76601563,449.7265625)
\curveto(360.28945313,449.7265625)(360.725,449.92578125)(361.07265625,450.32421875)
\curveto(361.4203125,450.72265625)(361.59414063,451.33007812)(361.59414063,452.14648438)
\curveto(361.59414063,452.91601562)(361.41835938,453.49804688)(361.06679688,453.89257812)
\curveto(360.71914063,454.29101562)(360.28554688,454.49023438)(359.76601563,454.49023438)
\curveto(359.23867188,454.49023438)(358.80117188,454.29296875)(358.45351563,453.8984375)
\curveto(358.10585938,453.50390625)(357.93203125,452.90820312)(357.93203125,452.11132812)
\closepath
}
}
{
\newrgbcolor{curcolor}{0 0 0}
\pscustom[linestyle=none,fillstyle=solid,fillcolor=curcolor]
{
\newpath
\moveto(369.2640625,446.47460938)
\curveto(368.68203125,447.20898438)(368.18984375,448.06835938)(367.7875,449.05273438)
\curveto(367.38515625,450.03710938)(367.18398438,451.05664062)(367.18398438,452.11132812)
\curveto(367.18398438,453.04101562)(367.334375,453.93164062)(367.63515625,454.78320312)
\curveto(367.98671875,455.77148438)(368.5296875,456.75585938)(369.2640625,457.73632812)
\lineto(370.01992188,457.73632812)
\curveto(369.54726563,456.92382812)(369.23476563,456.34375)(369.08242188,455.99609375)
\curveto(368.84414063,455.45703125)(368.65664063,454.89453125)(368.51992188,454.30859375)
\curveto(368.35195313,453.578125)(368.26796875,452.84375)(368.26796875,452.10546875)
\curveto(368.26796875,450.2265625)(368.85195313,448.34960938)(370.01992188,446.47460938)
\closepath
}
}
{
\newrgbcolor{curcolor}{0 0 0}
\pscustom[linestyle=none,fillstyle=solid,fillcolor=curcolor]
{
\newpath
\moveto(371.37929688,449)
\lineto(371.37929688,457.58984375)
\lineto(374.33828125,457.58984375)
\curveto(375.00625,457.58984375)(375.51601563,457.54882812)(375.86757813,457.46679688)
\curveto(376.35976563,457.35351562)(376.7796875,457.1484375)(377.12734375,456.8515625)
\curveto(377.58046875,456.46875)(377.91835938,455.97851562)(378.14101563,455.38085938)
\curveto(378.36757813,454.78710938)(378.48085938,454.10742188)(378.48085938,453.34179688)
\curveto(378.48085938,452.68945312)(378.4046875,452.11132812)(378.25234375,451.60742188)
\curveto(378.1,451.10351562)(377.9046875,450.68554688)(377.66640625,450.35351562)
\curveto(377.428125,450.02539062)(377.16640625,449.765625)(376.88125,449.57421875)
\curveto(376.6,449.38671875)(376.25820313,449.24414062)(375.85585938,449.14648438)
\curveto(375.45742188,449.04882812)(374.9984375,449)(374.47890625,449)
\closepath
\moveto(372.51601563,450.01367188)
\lineto(374.35,450.01367188)
\curveto(374.91640625,450.01367188)(375.35976563,450.06640625)(375.68007813,450.171875)
\curveto(376.00429688,450.27734375)(376.26210938,450.42578125)(376.45351563,450.6171875)
\curveto(376.72304688,450.88671875)(376.93203125,451.24804688)(377.08046875,451.70117188)
\curveto(377.2328125,452.15820312)(377.30898438,452.7109375)(377.30898438,453.359375)
\curveto(377.30898438,454.2578125)(377.16054688,454.94726562)(376.86367188,455.42773438)
\curveto(376.57070313,455.91210938)(376.21328125,456.23632812)(375.79140625,456.40039062)
\curveto(375.48671875,456.51757812)(374.99648438,456.57617188)(374.32070313,456.57617188)
\lineto(372.51601563,456.57617188)
\closepath
}
}
{
\newrgbcolor{curcolor}{0 0 0}
\pscustom[linestyle=none,fillstyle=solid,fillcolor=curcolor]
{
\newpath
\moveto(383.97109375,449.76757812)
\curveto(383.58046875,449.43554688)(383.20351563,449.20117188)(382.84023438,449.06445312)
\curveto(382.48085938,448.92773438)(382.09414063,448.859375)(381.68007813,448.859375)
\curveto(380.99648438,448.859375)(380.47109375,449.02539062)(380.10390625,449.35742188)
\curveto(379.73671875,449.69335938)(379.553125,450.12109375)(379.553125,450.640625)
\curveto(379.553125,450.9453125)(379.62148438,451.22265625)(379.75820313,451.47265625)
\curveto(379.89882813,451.7265625)(380.08046875,451.9296875)(380.303125,452.08203125)
\curveto(380.5296875,452.234375)(380.78359375,452.34960938)(381.06484375,452.42773438)
\curveto(381.271875,452.48242188)(381.584375,452.53515625)(382.00234375,452.5859375)
\curveto(382.85390625,452.6875)(383.48085938,452.80859375)(383.88320313,452.94921875)
\curveto(383.88710938,453.09375)(383.8890625,453.18554688)(383.8890625,453.22460938)
\curveto(383.8890625,453.65429688)(383.78945313,453.95703125)(383.59023438,454.1328125)
\curveto(383.32070313,454.37109375)(382.9203125,454.49023438)(382.3890625,454.49023438)
\curveto(381.89296875,454.49023438)(381.52578125,454.40234375)(381.2875,454.2265625)
\curveto(381.053125,454.0546875)(380.87929688,453.74804688)(380.76601563,453.30664062)
\lineto(379.73476563,453.44726562)
\curveto(379.82851563,453.88867188)(379.9828125,454.24414062)(380.19765625,454.51367188)
\curveto(380.4125,454.78710938)(380.72304688,454.99609375)(381.12929688,455.140625)
\curveto(381.53554688,455.2890625)(382.00625,455.36328125)(382.54140625,455.36328125)
\curveto(383.07265625,455.36328125)(383.50429688,455.30078125)(383.83632813,455.17578125)
\curveto(384.16835938,455.05078125)(384.4125,454.89257812)(384.56875,454.70117188)
\curveto(384.725,454.51367188)(384.834375,454.27539062)(384.896875,453.98632812)
\curveto(384.93203125,453.80664062)(384.94960938,453.48242188)(384.94960938,453.01367188)
\lineto(384.94960938,451.60742188)
\curveto(384.94960938,450.62695312)(384.97109375,450.00585938)(385.0140625,449.74414062)
\curveto(385.0609375,449.48632812)(385.15078125,449.23828125)(385.28359375,449)
\lineto(384.18203125,449)
\curveto(384.07265625,449.21875)(384.00234375,449.47460938)(383.97109375,449.76757812)
\closepath
\moveto(383.88320313,452.12304688)
\curveto(383.50039063,451.96679688)(382.92617188,451.83398438)(382.16054688,451.72460938)
\curveto(381.72695313,451.66210938)(381.4203125,451.59179688)(381.240625,451.51367188)
\curveto(381.0609375,451.43554688)(380.92226563,451.3203125)(380.82460938,451.16796875)
\curveto(380.72695313,451.01953125)(380.678125,450.85351562)(380.678125,450.66992188)
\curveto(380.678125,450.38867188)(380.78359375,450.15429688)(380.99453125,449.96679688)
\curveto(381.209375,449.77929688)(381.521875,449.68554688)(381.93203125,449.68554688)
\curveto(382.33828125,449.68554688)(382.69960938,449.7734375)(383.01601563,449.94921875)
\curveto(383.33242188,450.12890625)(383.56484375,450.37304688)(383.71328125,450.68164062)
\curveto(383.8265625,450.91992188)(383.88320313,451.27148438)(383.88320313,451.73632812)
\closepath
}
}
{
\newrgbcolor{curcolor}{0 0 0}
\pscustom[linestyle=none,fillstyle=solid,fillcolor=curcolor]
{
\newpath
\moveto(388.88710938,449.94335938)
\lineto(389.03945313,449.01171875)
\curveto(388.74257813,448.94921875)(388.47695313,448.91796875)(388.24257813,448.91796875)
\curveto(387.85976563,448.91796875)(387.56289063,448.97851562)(387.35195313,449.09960938)
\curveto(387.14101563,449.22070312)(386.99257813,449.37890625)(386.90664063,449.57421875)
\curveto(386.82070313,449.7734375)(386.77773438,450.18945312)(386.77773438,450.82226562)
\lineto(386.77773438,454.40234375)
\lineto(386.00429688,454.40234375)
\lineto(386.00429688,455.22265625)
\lineto(386.77773438,455.22265625)
\lineto(386.77773438,456.76367188)
\lineto(387.8265625,457.39648438)
\lineto(387.8265625,455.22265625)
\lineto(388.88710938,455.22265625)
\lineto(388.88710938,454.40234375)
\lineto(387.8265625,454.40234375)
\lineto(387.8265625,450.76367188)
\curveto(387.8265625,450.46289062)(387.84414063,450.26953125)(387.87929688,450.18359375)
\curveto(387.91835938,450.09765625)(387.97890625,450.02929688)(388.0609375,449.97851562)
\curveto(388.146875,449.92773438)(388.26796875,449.90234375)(388.42421875,449.90234375)
\curveto(388.54140625,449.90234375)(388.69570313,449.91601562)(388.88710938,449.94335938)
\closepath
}
}
{
\newrgbcolor{curcolor}{0 0 0}
\pscustom[linestyle=none,fillstyle=solid,fillcolor=curcolor]
{
\newpath
\moveto(393.97890625,449.76757812)
\curveto(393.58828125,449.43554688)(393.21132813,449.20117188)(392.84804688,449.06445312)
\curveto(392.48867188,448.92773438)(392.10195313,448.859375)(391.68789063,448.859375)
\curveto(391.00429688,448.859375)(390.47890625,449.02539062)(390.11171875,449.35742188)
\curveto(389.74453125,449.69335938)(389.5609375,450.12109375)(389.5609375,450.640625)
\curveto(389.5609375,450.9453125)(389.62929688,451.22265625)(389.76601563,451.47265625)
\curveto(389.90664063,451.7265625)(390.08828125,451.9296875)(390.3109375,452.08203125)
\curveto(390.5375,452.234375)(390.79140625,452.34960938)(391.07265625,452.42773438)
\curveto(391.2796875,452.48242188)(391.5921875,452.53515625)(392.01015625,452.5859375)
\curveto(392.86171875,452.6875)(393.48867188,452.80859375)(393.89101563,452.94921875)
\curveto(393.89492188,453.09375)(393.896875,453.18554688)(393.896875,453.22460938)
\curveto(393.896875,453.65429688)(393.79726563,453.95703125)(393.59804688,454.1328125)
\curveto(393.32851563,454.37109375)(392.928125,454.49023438)(392.396875,454.49023438)
\curveto(391.90078125,454.49023438)(391.53359375,454.40234375)(391.2953125,454.2265625)
\curveto(391.0609375,454.0546875)(390.88710938,453.74804688)(390.77382813,453.30664062)
\lineto(389.74257813,453.44726562)
\curveto(389.83632813,453.88867188)(389.990625,454.24414062)(390.20546875,454.51367188)
\curveto(390.4203125,454.78710938)(390.73085938,454.99609375)(391.13710938,455.140625)
\curveto(391.54335938,455.2890625)(392.0140625,455.36328125)(392.54921875,455.36328125)
\curveto(393.08046875,455.36328125)(393.51210938,455.30078125)(393.84414063,455.17578125)
\curveto(394.17617188,455.05078125)(394.4203125,454.89257812)(394.5765625,454.70117188)
\curveto(394.7328125,454.51367188)(394.8421875,454.27539062)(394.9046875,453.98632812)
\curveto(394.93984375,453.80664062)(394.95742188,453.48242188)(394.95742188,453.01367188)
\lineto(394.95742188,451.60742188)
\curveto(394.95742188,450.62695312)(394.97890625,450.00585938)(395.021875,449.74414062)
\curveto(395.06875,449.48632812)(395.15859375,449.23828125)(395.29140625,449)
\lineto(394.18984375,449)
\curveto(394.08046875,449.21875)(394.01015625,449.47460938)(393.97890625,449.76757812)
\closepath
\moveto(393.89101563,452.12304688)
\curveto(393.50820313,451.96679688)(392.93398438,451.83398438)(392.16835938,451.72460938)
\curveto(391.73476563,451.66210938)(391.428125,451.59179688)(391.2484375,451.51367188)
\curveto(391.06875,451.43554688)(390.93007813,451.3203125)(390.83242188,451.16796875)
\curveto(390.73476563,451.01953125)(390.6859375,450.85351562)(390.6859375,450.66992188)
\curveto(390.6859375,450.38867188)(390.79140625,450.15429688)(391.00234375,449.96679688)
\curveto(391.2171875,449.77929688)(391.5296875,449.68554688)(391.93984375,449.68554688)
\curveto(392.34609375,449.68554688)(392.70742188,449.7734375)(393.02382813,449.94921875)
\curveto(393.34023438,450.12890625)(393.57265625,450.37304688)(393.72109375,450.68164062)
\curveto(393.834375,450.91992188)(393.89101563,451.27148438)(393.89101563,451.73632812)
\closepath
}
}
{
\newrgbcolor{curcolor}{0 0 0}
\pscustom[linestyle=none,fillstyle=solid,fillcolor=curcolor]
{
\newpath
\moveto(400.04921875,449)
\lineto(400.04921875,457.58984375)
\lineto(401.21523438,457.58984375)
\lineto(405.72695313,450.84570312)
\lineto(405.72695313,457.58984375)
\lineto(406.81679688,457.58984375)
\lineto(406.81679688,449)
\lineto(405.65078125,449)
\lineto(401.1390625,455.75)
\lineto(401.1390625,449)
\closepath
}
}
{
\newrgbcolor{curcolor}{0 0 0}
\pscustom[linestyle=none,fillstyle=solid,fillcolor=curcolor]
{
\newpath
\moveto(408.19960938,452.11132812)
\curveto(408.19960938,453.26367188)(408.51992188,454.1171875)(409.16054688,454.671875)
\curveto(409.69570313,455.1328125)(410.34804688,455.36328125)(411.11757813,455.36328125)
\curveto(411.97304688,455.36328125)(412.67226563,455.08203125)(413.21523438,454.51953125)
\curveto(413.75820313,453.9609375)(414.0296875,453.1875)(414.0296875,452.19921875)
\curveto(414.0296875,451.3984375)(413.90859375,450.76757812)(413.66640625,450.30664062)
\curveto(413.428125,449.84960938)(413.07851563,449.49414062)(412.61757813,449.24023438)
\curveto(412.16054688,448.98632812)(411.66054688,448.859375)(411.11757813,448.859375)
\curveto(410.24648438,448.859375)(409.54140625,449.13867188)(409.00234375,449.69726562)
\curveto(408.4671875,450.25585938)(408.19960938,451.06054688)(408.19960938,452.11132812)
\closepath
\moveto(409.28359375,452.11132812)
\curveto(409.28359375,451.31445312)(409.45742188,450.71679688)(409.80507813,450.31835938)
\curveto(410.15273438,449.92382812)(410.59023438,449.7265625)(411.11757813,449.7265625)
\curveto(411.64101563,449.7265625)(412.0765625,449.92578125)(412.42421875,450.32421875)
\curveto(412.771875,450.72265625)(412.94570313,451.33007812)(412.94570313,452.14648438)
\curveto(412.94570313,452.91601562)(412.76992188,453.49804688)(412.41835938,453.89257812)
\curveto(412.07070313,454.29101562)(411.63710938,454.49023438)(411.11757813,454.49023438)
\curveto(410.59023438,454.49023438)(410.15273438,454.29296875)(409.80507813,453.8984375)
\curveto(409.45742188,453.50390625)(409.28359375,452.90820312)(409.28359375,452.11132812)
\closepath
}
}
{
\newrgbcolor{curcolor}{0 0 0}
\pscustom[linestyle=none,fillstyle=solid,fillcolor=curcolor]
{
\newpath
\moveto(419.303125,449)
\lineto(419.303125,449.78515625)
\curveto(418.90859375,449.16796875)(418.32851563,448.859375)(417.56289063,448.859375)
\curveto(417.06679688,448.859375)(416.60976563,448.99609375)(416.19179688,449.26953125)
\curveto(415.77773438,449.54296875)(415.45546875,449.92382812)(415.225,450.41210938)
\curveto(414.9984375,450.90429688)(414.88515625,451.46875)(414.88515625,452.10546875)
\curveto(414.88515625,452.7265625)(414.98867188,453.2890625)(415.19570313,453.79296875)
\curveto(415.40273438,454.30078125)(415.71328125,454.68945312)(416.12734375,454.95898438)
\curveto(416.54140625,455.22851562)(417.00429688,455.36328125)(417.51601563,455.36328125)
\curveto(417.89101563,455.36328125)(418.225,455.28320312)(418.51796875,455.12304688)
\curveto(418.8109375,454.96679688)(419.04921875,454.76171875)(419.2328125,454.5078125)
\lineto(419.2328125,457.58984375)
\lineto(420.28164063,457.58984375)
\lineto(420.28164063,449)
\closepath
\moveto(415.96914063,452.10546875)
\curveto(415.96914063,451.30859375)(416.13710938,450.71289062)(416.47304688,450.31835938)
\curveto(416.80898438,449.92382812)(417.20546875,449.7265625)(417.6625,449.7265625)
\curveto(418.1234375,449.7265625)(418.5140625,449.9140625)(418.834375,450.2890625)
\curveto(419.15859375,450.66796875)(419.32070313,451.24414062)(419.32070313,452.01757812)
\curveto(419.32070313,452.86914062)(419.15664063,453.49414062)(418.82851563,453.89257812)
\curveto(418.50039063,454.29101562)(418.09609375,454.49023438)(417.615625,454.49023438)
\curveto(417.146875,454.49023438)(416.75429688,454.29882812)(416.43789063,453.91601562)
\curveto(416.12539063,453.53320312)(415.96914063,452.9296875)(415.96914063,452.10546875)
\closepath
}
}
{
\newrgbcolor{curcolor}{0 0 0}
\pscustom[linestyle=none,fillstyle=solid,fillcolor=curcolor]
{
\newpath
\moveto(426.19960938,451.00390625)
\lineto(427.28945313,450.86914062)
\curveto(427.11757813,450.23242188)(426.79921875,449.73828125)(426.334375,449.38671875)
\curveto(425.86953125,449.03515625)(425.27578125,448.859375)(424.553125,448.859375)
\curveto(423.64296875,448.859375)(422.9203125,449.13867188)(422.38515625,449.69726562)
\curveto(421.85390625,450.25976562)(421.58828125,451.046875)(421.58828125,452.05859375)
\curveto(421.58828125,453.10546875)(421.8578125,453.91796875)(422.396875,454.49609375)
\curveto(422.9359375,455.07421875)(423.63515625,455.36328125)(424.49453125,455.36328125)
\curveto(425.3265625,455.36328125)(426.00625,455.08007812)(426.53359375,454.51367188)
\curveto(427.0609375,453.94726562)(427.32460938,453.15039062)(427.32460938,452.12304688)
\curveto(427.32460938,452.06054688)(427.32265625,451.96679688)(427.31875,451.84179688)
\lineto(422.678125,451.84179688)
\curveto(422.7171875,451.15820312)(422.91054688,450.63476562)(423.25820313,450.27148438)
\curveto(423.60585938,449.90820312)(424.03945313,449.7265625)(424.55898438,449.7265625)
\curveto(424.94570313,449.7265625)(425.27578125,449.828125)(425.54921875,450.03125)
\curveto(425.82265625,450.234375)(426.03945313,450.55859375)(426.19960938,451.00390625)
\closepath
\moveto(422.73671875,452.70898438)
\lineto(426.21132813,452.70898438)
\curveto(426.16445313,453.23242188)(426.03164063,453.625)(425.81289063,453.88671875)
\curveto(425.47695313,454.29296875)(425.04140625,454.49609375)(424.50625,454.49609375)
\curveto(424.021875,454.49609375)(423.61367188,454.33398438)(423.28164063,454.00976562)
\curveto(422.95351563,453.68554688)(422.771875,453.25195312)(422.73671875,452.70898438)
\closepath
}
}
{
\newrgbcolor{curcolor}{0 0 0}
\pscustom[linestyle=none,fillstyle=solid,fillcolor=curcolor]
{
\newpath
\moveto(431.6546875,453.23632812)
\curveto(431.6546875,454.25195312)(431.75820313,455.06835938)(431.96523438,455.68554688)
\curveto(432.17617188,456.30664062)(432.48671875,456.78515625)(432.896875,457.12109375)
\curveto(433.3109375,457.45703125)(433.83046875,457.625)(434.45546875,457.625)
\curveto(434.91640625,457.625)(435.32070313,457.53125)(435.66835938,457.34375)
\curveto(436.01601563,457.16015625)(436.303125,456.89257812)(436.5296875,456.54101562)
\curveto(436.75625,456.19335938)(436.93398438,455.76757812)(437.06289063,455.26367188)
\curveto(437.19179688,454.76367188)(437.25625,454.08789062)(437.25625,453.23632812)
\curveto(437.25625,452.22851562)(437.15273438,451.4140625)(436.94570313,450.79296875)
\curveto(436.73867188,450.17578125)(436.428125,449.69726562)(436.0140625,449.35742188)
\curveto(435.60390625,449.02148438)(435.084375,448.85351562)(434.45546875,448.85351562)
\curveto(433.62734375,448.85351562)(432.97695313,449.15039062)(432.50429688,449.74414062)
\curveto(431.93789063,450.45898438)(431.6546875,451.62304688)(431.6546875,453.23632812)
\closepath
\moveto(432.73867188,453.23632812)
\curveto(432.73867188,451.82617188)(432.90273438,450.88671875)(433.23085938,450.41796875)
\curveto(433.56289063,449.953125)(433.97109375,449.72070312)(434.45546875,449.72070312)
\curveto(434.93984375,449.72070312)(435.34609375,449.95507812)(435.67421875,450.42382812)
\curveto(436.00625,450.89257812)(436.17226563,451.83007812)(436.17226563,453.23632812)
\curveto(436.17226563,454.65039062)(436.00625,455.58984375)(435.67421875,456.0546875)
\curveto(435.34609375,456.51953125)(434.9359375,456.75195312)(434.44375,456.75195312)
\curveto(433.959375,456.75195312)(433.57265625,456.546875)(433.28359375,456.13671875)
\curveto(432.9203125,455.61328125)(432.73867188,454.64648438)(432.73867188,453.23632812)
\closepath
}
}
{
\newrgbcolor{curcolor}{0 0 0}
\pscustom[linestyle=none,fillstyle=solid,fillcolor=curcolor]
{
\newpath
\moveto(439.31289063,446.47460938)
\lineto(438.55703125,446.47460938)
\curveto(439.725,448.34960938)(440.30898438,450.2265625)(440.30898438,452.10546875)
\curveto(440.30898438,452.83984375)(440.225,453.56835938)(440.05703125,454.29101562)
\curveto(439.92421875,454.87695312)(439.73867188,455.43945312)(439.50039063,455.97851562)
\curveto(439.34804688,456.33007812)(439.03359375,456.91601562)(438.55703125,457.73632812)
\lineto(439.31289063,457.73632812)
\curveto(440.04726563,456.75585938)(440.59023438,455.77148438)(440.94179688,454.78320312)
\curveto(441.24257813,453.93164062)(441.39296875,453.04101562)(441.39296875,452.11132812)
\curveto(441.39296875,451.05664062)(441.18984375,450.03710938)(440.78359375,449.05273438)
\curveto(440.38125,448.06835938)(439.89101563,447.20898438)(439.31289063,446.47460938)
\closepath
}
}
{
\newrgbcolor{curcolor}{0 0 0}
\pscustom[linestyle=none,fillstyle=solid,fillcolor=curcolor]
{
\newpath
\moveto(116.61992187,404)
\lineto(116.61992187,412.58984375)
\lineto(117.7859375,412.58984375)
\lineto(122.29765625,405.84570312)
\lineto(122.29765625,412.58984375)
\lineto(123.3875,412.58984375)
\lineto(123.3875,404)
\lineto(122.22148437,404)
\lineto(117.70976562,410.75)
\lineto(117.70976562,404)
\closepath
}
}
{
\newrgbcolor{curcolor}{0 0 0}
\pscustom[linestyle=none,fillstyle=solid,fillcolor=curcolor]
{
\newpath
\moveto(129.42265625,406.00390625)
\lineto(130.5125,405.86914062)
\curveto(130.340625,405.23242188)(130.02226562,404.73828125)(129.55742187,404.38671875)
\curveto(129.09257812,404.03515625)(128.49882812,403.859375)(127.77617187,403.859375)
\curveto(126.86601562,403.859375)(126.14335937,404.13867188)(125.60820312,404.69726562)
\curveto(125.07695312,405.25976562)(124.81132812,406.046875)(124.81132812,407.05859375)
\curveto(124.81132812,408.10546875)(125.08085937,408.91796875)(125.61992187,409.49609375)
\curveto(126.15898437,410.07421875)(126.85820312,410.36328125)(127.71757812,410.36328125)
\curveto(128.54960937,410.36328125)(129.22929687,410.08007812)(129.75664062,409.51367188)
\curveto(130.28398437,408.94726562)(130.54765625,408.15039062)(130.54765625,407.12304688)
\curveto(130.54765625,407.06054688)(130.54570312,406.96679688)(130.54179687,406.84179688)
\lineto(125.90117187,406.84179688)
\curveto(125.94023437,406.15820312)(126.13359375,405.63476562)(126.48125,405.27148438)
\curveto(126.82890625,404.90820312)(127.2625,404.7265625)(127.78203125,404.7265625)
\curveto(128.16875,404.7265625)(128.49882812,404.828125)(128.77226562,405.03125)
\curveto(129.04570312,405.234375)(129.2625,405.55859375)(129.42265625,406.00390625)
\closepath
\moveto(125.95976562,407.70898438)
\lineto(129.434375,407.70898438)
\curveto(129.3875,408.23242188)(129.2546875,408.625)(129.0359375,408.88671875)
\curveto(128.7,409.29296875)(128.26445312,409.49609375)(127.72929687,409.49609375)
\curveto(127.24492187,409.49609375)(126.83671875,409.33398438)(126.5046875,409.00976562)
\curveto(126.1765625,408.68554688)(125.99492187,408.25195312)(125.95976562,407.70898438)
\closepath
}
}
{
\newrgbcolor{curcolor}{0 0 0}
\pscustom[linestyle=none,fillstyle=solid,fillcolor=curcolor]
{
\newpath
\moveto(132.98515625,404)
\lineto(131.08085937,410.22265625)
\lineto(132.17070312,410.22265625)
\lineto(133.1609375,406.63085938)
\lineto(133.53007812,405.29492188)
\curveto(133.54570312,405.36132812)(133.653125,405.7890625)(133.85234375,406.578125)
\lineto(134.84257812,410.22265625)
\lineto(135.9265625,410.22265625)
\lineto(136.85820312,406.61328125)
\lineto(137.16875,405.42382812)
\lineto(137.52617187,406.625)
\lineto(138.59257812,410.22265625)
\lineto(139.61796875,410.22265625)
\lineto(137.67265625,404)
\lineto(136.57695312,404)
\lineto(135.58671875,407.7265625)
\lineto(135.34648437,408.78710938)
\lineto(134.08671875,404)
\closepath
}
}
{
\newrgbcolor{curcolor}{0 0 0}
\pscustom[linestyle=none,fillstyle=solid,fillcolor=curcolor]
{
\newpath
\moveto(143.2859375,404)
\lineto(143.2859375,405.0546875)
\lineto(147.68632812,410.55664062)
\curveto(147.99882812,410.94726562)(148.29570312,411.28710938)(148.57695312,411.57617188)
\lineto(143.78398437,411.57617188)
\lineto(143.78398437,412.58984375)
\lineto(149.93632812,412.58984375)
\lineto(149.93632812,411.57617188)
\lineto(145.1140625,405.6171875)
\lineto(144.59257812,405.01367188)
\lineto(150.07695312,405.01367188)
\lineto(150.07695312,404)
\closepath
}
}
{
\newrgbcolor{curcolor}{0 0 0}
\pscustom[linestyle=none,fillstyle=solid,fillcolor=curcolor]
{
\newpath
\moveto(151.36015625,404)
\lineto(151.36015625,412.58984375)
\lineto(157.15507812,412.58984375)
\lineto(157.15507812,411.57617188)
\lineto(152.496875,411.57617188)
\lineto(152.496875,408.91601562)
\lineto(156.528125,408.91601562)
\lineto(156.528125,407.90234375)
\lineto(152.496875,407.90234375)
\lineto(152.496875,404)
\closepath
}
}
{
\newrgbcolor{curcolor}{0 0 0}
\pscustom[linestyle=none,fillstyle=solid,fillcolor=curcolor]
{
\newpath
\moveto(158.24492187,406.75976562)
\lineto(159.3171875,406.85351562)
\curveto(159.36796875,406.42382812)(159.48515625,406.0703125)(159.66875,405.79296875)
\curveto(159.85625,405.51953125)(160.1453125,405.296875)(160.5359375,405.125)
\curveto(160.9265625,404.95703125)(161.36601562,404.87304688)(161.85429687,404.87304688)
\curveto(162.28789062,404.87304688)(162.67070312,404.9375)(163.00273437,405.06640625)
\curveto(163.33476562,405.1953125)(163.58085937,405.37109375)(163.74101562,405.59375)
\curveto(163.90507812,405.8203125)(163.98710937,406.06640625)(163.98710937,406.33203125)
\curveto(163.98710937,406.6015625)(163.90898437,406.8359375)(163.75273437,407.03515625)
\curveto(163.59648437,407.23828125)(163.33867187,407.40820312)(162.97929687,407.54492188)
\curveto(162.74882812,407.63476562)(162.2390625,407.7734375)(161.45,407.9609375)
\curveto(160.6609375,408.15234375)(160.10820312,408.33203125)(159.79179687,408.5)
\curveto(159.38164062,408.71484375)(159.075,408.98046875)(158.871875,409.296875)
\curveto(158.67265625,409.6171875)(158.57304687,409.97460938)(158.57304687,410.36914062)
\curveto(158.57304687,410.80273438)(158.69609375,411.20703125)(158.9421875,411.58203125)
\curveto(159.18828125,411.9609375)(159.54765625,412.24804688)(160.0203125,412.44335938)
\curveto(160.49296875,412.63867188)(161.01835937,412.73632812)(161.59648437,412.73632812)
\curveto(162.23320312,412.73632812)(162.79375,412.6328125)(163.278125,412.42578125)
\curveto(163.76640625,412.22265625)(164.14140625,411.921875)(164.403125,411.5234375)
\curveto(164.66484375,411.125)(164.80546875,410.67382812)(164.825,410.16992188)
\lineto(163.73515625,410.08789062)
\curveto(163.6765625,410.63085938)(163.47734375,411.04101562)(163.1375,411.31835938)
\curveto(162.8015625,411.59570312)(162.30351562,411.734375)(161.64335937,411.734375)
\curveto(160.95585937,411.734375)(160.45390625,411.60742188)(160.1375,411.35351562)
\curveto(159.825,411.10351562)(159.66875,410.80078125)(159.66875,410.4453125)
\curveto(159.66875,410.13671875)(159.78007812,409.8828125)(160.00273437,409.68359375)
\curveto(160.22148437,409.484375)(160.79179687,409.27929688)(161.71367187,409.06835938)
\curveto(162.63945312,408.86132812)(163.27421875,408.6796875)(163.61796875,408.5234375)
\curveto(164.11796875,408.29296875)(164.48710937,408)(164.72539062,407.64453125)
\curveto(164.96367187,407.29296875)(165.0828125,406.88671875)(165.0828125,406.42578125)
\curveto(165.0828125,405.96875)(164.95195312,405.53710938)(164.69023437,405.13085938)
\curveto(164.42851562,404.72851562)(164.0515625,404.4140625)(163.559375,404.1875)
\curveto(163.07109375,403.96484375)(162.5203125,403.85351562)(161.90703125,403.85351562)
\curveto(161.1296875,403.85351562)(160.47734375,403.96679688)(159.95,404.19335938)
\curveto(159.4265625,404.41992188)(159.01445312,404.75976562)(158.71367187,405.21289062)
\curveto(158.41679687,405.66992188)(158.26054687,406.18554688)(158.24492187,406.75976562)
\closepath
}
}
{
\newrgbcolor{curcolor}{0 0 0}
\pscustom[linestyle=none,fillstyle=solid,fillcolor=curcolor]
{
\newpath
\moveto(169.96953125,404)
\lineto(169.96953125,412.58984375)
\lineto(172.92851562,412.58984375)
\curveto(173.59648437,412.58984375)(174.10625,412.54882812)(174.4578125,412.46679688)
\curveto(174.95,412.35351562)(175.36992187,412.1484375)(175.71757812,411.8515625)
\curveto(176.17070312,411.46875)(176.50859375,410.97851562)(176.73125,410.38085938)
\curveto(176.9578125,409.78710938)(177.07109375,409.10742188)(177.07109375,408.34179688)
\curveto(177.07109375,407.68945312)(176.99492187,407.11132812)(176.84257812,406.60742188)
\curveto(176.69023437,406.10351562)(176.49492187,405.68554688)(176.25664062,405.35351562)
\curveto(176.01835937,405.02539062)(175.75664062,404.765625)(175.47148437,404.57421875)
\curveto(175.19023437,404.38671875)(174.8484375,404.24414062)(174.44609375,404.14648438)
\curveto(174.04765625,404.04882812)(173.58867187,404)(173.06914062,404)
\closepath
\moveto(171.10625,405.01367188)
\lineto(172.94023437,405.01367188)
\curveto(173.50664062,405.01367188)(173.95,405.06640625)(174.2703125,405.171875)
\curveto(174.59453125,405.27734375)(174.85234375,405.42578125)(175.04375,405.6171875)
\curveto(175.31328125,405.88671875)(175.52226562,406.24804688)(175.67070312,406.70117188)
\curveto(175.82304687,407.15820312)(175.89921875,407.7109375)(175.89921875,408.359375)
\curveto(175.89921875,409.2578125)(175.75078125,409.94726562)(175.45390625,410.42773438)
\curveto(175.1609375,410.91210938)(174.80351562,411.23632812)(174.38164062,411.40039062)
\curveto(174.07695312,411.51757812)(173.58671875,411.57617188)(172.9109375,411.57617188)
\lineto(171.10625,411.57617188)
\closepath
}
}
{
\newrgbcolor{curcolor}{0 0 0}
\pscustom[linestyle=none,fillstyle=solid,fillcolor=curcolor]
{
\newpath
\moveto(178.50664062,411.37695312)
\lineto(178.50664062,412.58984375)
\lineto(179.56132812,412.58984375)
\lineto(179.56132812,411.37695312)
\closepath
\moveto(178.50664062,404)
\lineto(178.50664062,410.22265625)
\lineto(179.56132812,410.22265625)
\lineto(179.56132812,404)
\closepath
}
}
{
\newrgbcolor{curcolor}{0 0 0}
\pscustom[linestyle=none,fillstyle=solid,fillcolor=curcolor]
{
\newpath
\moveto(181.41875,404)
\lineto(181.41875,409.40234375)
\lineto(180.48710937,409.40234375)
\lineto(180.48710937,410.22265625)
\lineto(181.41875,410.22265625)
\lineto(181.41875,410.88476562)
\curveto(181.41875,411.30273438)(181.45585937,411.61328125)(181.53007812,411.81640625)
\curveto(181.63164062,412.08984375)(181.809375,412.31054688)(182.06328125,412.47851562)
\curveto(182.32109375,412.65039062)(182.68046875,412.73632812)(183.14140625,412.73632812)
\curveto(183.43828125,412.73632812)(183.76640625,412.70117188)(184.12578125,412.63085938)
\lineto(183.96757812,411.7109375)
\curveto(183.74882812,411.75)(183.54179687,411.76953125)(183.34648437,411.76953125)
\curveto(183.02617187,411.76953125)(182.79960937,411.70117188)(182.66679687,411.56445312)
\curveto(182.53398437,411.42773438)(182.46757812,411.171875)(182.46757812,410.796875)
\lineto(182.46757812,410.22265625)
\lineto(183.68046875,410.22265625)
\lineto(183.68046875,409.40234375)
\lineto(182.46757812,409.40234375)
\lineto(182.46757812,404)
\closepath
}
}
{
\newrgbcolor{curcolor}{0 0 0}
\pscustom[linestyle=none,fillstyle=solid,fillcolor=curcolor]
{
\newpath
\moveto(184.5359375,404)
\lineto(184.5359375,409.40234375)
\lineto(183.60429687,409.40234375)
\lineto(183.60429687,410.22265625)
\lineto(184.5359375,410.22265625)
\lineto(184.5359375,410.88476562)
\curveto(184.5359375,411.30273438)(184.57304687,411.61328125)(184.64726562,411.81640625)
\curveto(184.74882812,412.08984375)(184.9265625,412.31054688)(185.18046875,412.47851562)
\curveto(185.43828125,412.65039062)(185.79765625,412.73632812)(186.25859375,412.73632812)
\curveto(186.55546875,412.73632812)(186.88359375,412.70117188)(187.24296875,412.63085938)
\lineto(187.08476562,411.7109375)
\curveto(186.86601562,411.75)(186.65898437,411.76953125)(186.46367187,411.76953125)
\curveto(186.14335937,411.76953125)(185.91679687,411.70117188)(185.78398437,411.56445312)
\curveto(185.65117187,411.42773438)(185.58476562,411.171875)(185.58476562,410.796875)
\lineto(185.58476562,410.22265625)
\lineto(186.79765625,410.22265625)
\lineto(186.79765625,409.40234375)
\lineto(185.58476562,409.40234375)
\lineto(185.58476562,404)
\closepath
}
}
{
\newrgbcolor{curcolor}{0 0 0}
\pscustom[linestyle=none,fillstyle=solid,fillcolor=curcolor]
{
\newpath
\moveto(191.87773437,406.00390625)
\lineto(192.96757812,405.86914062)
\curveto(192.79570312,405.23242188)(192.47734375,404.73828125)(192.0125,404.38671875)
\curveto(191.54765625,404.03515625)(190.95390625,403.859375)(190.23125,403.859375)
\curveto(189.32109375,403.859375)(188.5984375,404.13867188)(188.06328125,404.69726562)
\curveto(187.53203125,405.25976562)(187.26640625,406.046875)(187.26640625,407.05859375)
\curveto(187.26640625,408.10546875)(187.5359375,408.91796875)(188.075,409.49609375)
\curveto(188.6140625,410.07421875)(189.31328125,410.36328125)(190.17265625,410.36328125)
\curveto(191.0046875,410.36328125)(191.684375,410.08007812)(192.21171875,409.51367188)
\curveto(192.7390625,408.94726562)(193.00273437,408.15039062)(193.00273437,407.12304688)
\curveto(193.00273437,407.06054688)(193.00078125,406.96679688)(192.996875,406.84179688)
\lineto(188.35625,406.84179688)
\curveto(188.3953125,406.15820312)(188.58867187,405.63476562)(188.93632812,405.27148438)
\curveto(189.28398437,404.90820312)(189.71757812,404.7265625)(190.23710937,404.7265625)
\curveto(190.62382812,404.7265625)(190.95390625,404.828125)(191.22734375,405.03125)
\curveto(191.50078125,405.234375)(191.71757812,405.55859375)(191.87773437,406.00390625)
\closepath
\moveto(188.41484375,407.70898438)
\lineto(191.88945312,407.70898438)
\curveto(191.84257812,408.23242188)(191.70976562,408.625)(191.49101562,408.88671875)
\curveto(191.15507812,409.29296875)(190.71953125,409.49609375)(190.184375,409.49609375)
\curveto(189.7,409.49609375)(189.29179687,409.33398438)(188.95976562,409.00976562)
\curveto(188.63164062,408.68554688)(188.45,408.25195312)(188.41484375,407.70898438)
\closepath
}
}
{
\newrgbcolor{curcolor}{0 0 0}
\pscustom[linestyle=none,fillstyle=solid,fillcolor=curcolor]
{
\newpath
\moveto(194.28007812,404)
\lineto(194.28007812,410.22265625)
\lineto(195.22929687,410.22265625)
\lineto(195.22929687,409.27929688)
\curveto(195.47148437,409.72070312)(195.69414062,410.01171875)(195.89726562,410.15234375)
\curveto(196.10429687,410.29296875)(196.33085937,410.36328125)(196.57695312,410.36328125)
\curveto(196.93242187,410.36328125)(197.29375,410.25)(197.6609375,410.0234375)
\lineto(197.29765625,409.04492188)
\curveto(197.03984375,409.19726562)(196.78203125,409.2734375)(196.52421875,409.2734375)
\curveto(196.29375,409.2734375)(196.08671875,409.203125)(195.903125,409.0625)
\curveto(195.71953125,408.92578125)(195.58867187,408.734375)(195.51054687,408.48828125)
\curveto(195.39335937,408.11328125)(195.33476562,407.703125)(195.33476562,407.2578125)
\lineto(195.33476562,404)
\closepath
}
}
{
\newrgbcolor{curcolor}{0 0 0}
\pscustom[linestyle=none,fillstyle=solid,fillcolor=curcolor]
{
\newpath
\moveto(202.54765625,406.00390625)
\lineto(203.6375,405.86914062)
\curveto(203.465625,405.23242188)(203.14726562,404.73828125)(202.68242187,404.38671875)
\curveto(202.21757812,404.03515625)(201.62382812,403.859375)(200.90117187,403.859375)
\curveto(199.99101562,403.859375)(199.26835937,404.13867188)(198.73320312,404.69726562)
\curveto(198.20195312,405.25976562)(197.93632812,406.046875)(197.93632812,407.05859375)
\curveto(197.93632812,408.10546875)(198.20585937,408.91796875)(198.74492187,409.49609375)
\curveto(199.28398437,410.07421875)(199.98320312,410.36328125)(200.84257812,410.36328125)
\curveto(201.67460937,410.36328125)(202.35429687,410.08007812)(202.88164062,409.51367188)
\curveto(203.40898437,408.94726562)(203.67265625,408.15039062)(203.67265625,407.12304688)
\curveto(203.67265625,407.06054688)(203.67070312,406.96679688)(203.66679687,406.84179688)
\lineto(199.02617187,406.84179688)
\curveto(199.06523437,406.15820312)(199.25859375,405.63476562)(199.60625,405.27148438)
\curveto(199.95390625,404.90820312)(200.3875,404.7265625)(200.90703125,404.7265625)
\curveto(201.29375,404.7265625)(201.62382812,404.828125)(201.89726562,405.03125)
\curveto(202.17070312,405.234375)(202.3875,405.55859375)(202.54765625,406.00390625)
\closepath
\moveto(199.08476562,407.70898438)
\lineto(202.559375,407.70898438)
\curveto(202.5125,408.23242188)(202.3796875,408.625)(202.1609375,408.88671875)
\curveto(201.825,409.29296875)(201.38945312,409.49609375)(200.85429687,409.49609375)
\curveto(200.36992187,409.49609375)(199.96171875,409.33398438)(199.6296875,409.00976562)
\curveto(199.3015625,408.68554688)(199.11992187,408.25195312)(199.08476562,407.70898438)
\closepath
}
}
{
\newrgbcolor{curcolor}{0 0 0}
\pscustom[linestyle=none,fillstyle=solid,fillcolor=curcolor]
{
\newpath
\moveto(204.96171875,404)
\lineto(204.96171875,410.22265625)
\lineto(205.9109375,410.22265625)
\lineto(205.9109375,409.33789062)
\curveto(206.36796875,410.02148438)(207.028125,410.36328125)(207.89140625,410.36328125)
\curveto(208.26640625,410.36328125)(208.61015625,410.29492188)(208.92265625,410.15820312)
\curveto(209.2390625,410.02539062)(209.47539062,409.84960938)(209.63164062,409.63085938)
\curveto(209.78789062,409.41210938)(209.89726562,409.15234375)(209.95976562,408.8515625)
\curveto(209.99882812,408.65625)(210.01835937,408.31445312)(210.01835937,407.82617188)
\lineto(210.01835937,404)
\lineto(208.96367187,404)
\lineto(208.96367187,407.78515625)
\curveto(208.96367187,408.21484375)(208.92265625,408.53515625)(208.840625,408.74609375)
\curveto(208.75859375,408.9609375)(208.61210937,409.13085938)(208.40117187,409.25585938)
\curveto(208.19414062,409.38476562)(207.95,409.44921875)(207.66875,409.44921875)
\curveto(207.21953125,409.44921875)(206.83085937,409.30664062)(206.50273437,409.02148438)
\curveto(206.17851562,408.73632812)(206.01640625,408.1953125)(206.01640625,407.3984375)
\lineto(206.01640625,404)
\closepath
}
}
{
\newrgbcolor{curcolor}{0 0 0}
\pscustom[linestyle=none,fillstyle=solid,fillcolor=curcolor]
{
\newpath
\moveto(213.93828125,404.94335938)
\lineto(214.090625,404.01171875)
\curveto(213.79375,403.94921875)(213.528125,403.91796875)(213.29375,403.91796875)
\curveto(212.9109375,403.91796875)(212.6140625,403.97851562)(212.403125,404.09960938)
\curveto(212.1921875,404.22070312)(212.04375,404.37890625)(211.9578125,404.57421875)
\curveto(211.871875,404.7734375)(211.82890625,405.18945312)(211.82890625,405.82226562)
\lineto(211.82890625,409.40234375)
\lineto(211.05546875,409.40234375)
\lineto(211.05546875,410.22265625)
\lineto(211.82890625,410.22265625)
\lineto(211.82890625,411.76367188)
\lineto(212.87773437,412.39648438)
\lineto(212.87773437,410.22265625)
\lineto(213.93828125,410.22265625)
\lineto(213.93828125,409.40234375)
\lineto(212.87773437,409.40234375)
\lineto(212.87773437,405.76367188)
\curveto(212.87773437,405.46289062)(212.8953125,405.26953125)(212.93046875,405.18359375)
\curveto(212.96953125,405.09765625)(213.03007812,405.02929688)(213.11210937,404.97851562)
\curveto(213.19804687,404.92773438)(213.31914062,404.90234375)(213.47539062,404.90234375)
\curveto(213.59257812,404.90234375)(213.746875,404.91601562)(213.93828125,404.94335938)
\closepath
}
}
{
\newrgbcolor{curcolor}{0 0 0}
\pscustom[linestyle=none,fillstyle=solid,fillcolor=curcolor]
{
\newpath
\moveto(218.4265625,404)
\lineto(218.4265625,412.58984375)
\lineto(219.59257812,412.58984375)
\lineto(224.10429687,405.84570312)
\lineto(224.10429687,412.58984375)
\lineto(225.19414062,412.58984375)
\lineto(225.19414062,404)
\lineto(224.028125,404)
\lineto(219.51640625,410.75)
\lineto(219.51640625,404)
\closepath
}
}
{
\newrgbcolor{curcolor}{0 0 0}
\pscustom[linestyle=none,fillstyle=solid,fillcolor=curcolor]
{
\newpath
\moveto(226.57695312,407.11132812)
\curveto(226.57695312,408.26367188)(226.89726562,409.1171875)(227.53789062,409.671875)
\curveto(228.07304687,410.1328125)(228.72539062,410.36328125)(229.49492187,410.36328125)
\curveto(230.35039062,410.36328125)(231.04960937,410.08203125)(231.59257812,409.51953125)
\curveto(232.13554687,408.9609375)(232.40703125,408.1875)(232.40703125,407.19921875)
\curveto(232.40703125,406.3984375)(232.2859375,405.76757812)(232.04375,405.30664062)
\curveto(231.80546875,404.84960938)(231.45585937,404.49414062)(230.99492187,404.24023438)
\curveto(230.53789062,403.98632812)(230.03789062,403.859375)(229.49492187,403.859375)
\curveto(228.62382812,403.859375)(227.91875,404.13867188)(227.3796875,404.69726562)
\curveto(226.84453125,405.25585938)(226.57695312,406.06054688)(226.57695312,407.11132812)
\closepath
\moveto(227.6609375,407.11132812)
\curveto(227.6609375,406.31445312)(227.83476562,405.71679688)(228.18242187,405.31835938)
\curveto(228.53007812,404.92382812)(228.96757812,404.7265625)(229.49492187,404.7265625)
\curveto(230.01835937,404.7265625)(230.45390625,404.92578125)(230.8015625,405.32421875)
\curveto(231.14921875,405.72265625)(231.32304687,406.33007812)(231.32304687,407.14648438)
\curveto(231.32304687,407.91601562)(231.14726562,408.49804688)(230.79570312,408.89257812)
\curveto(230.44804687,409.29101562)(230.01445312,409.49023438)(229.49492187,409.49023438)
\curveto(228.96757812,409.49023438)(228.53007812,409.29296875)(228.18242187,408.8984375)
\curveto(227.83476562,408.50390625)(227.6609375,407.90820312)(227.6609375,407.11132812)
\closepath
}
}
{
\newrgbcolor{curcolor}{0 0 0}
\pscustom[linestyle=none,fillstyle=solid,fillcolor=curcolor]
{
\newpath
\moveto(237.68046875,404)
\lineto(237.68046875,404.78515625)
\curveto(237.2859375,404.16796875)(236.70585937,403.859375)(235.94023437,403.859375)
\curveto(235.44414062,403.859375)(234.98710937,403.99609375)(234.56914062,404.26953125)
\curveto(234.15507812,404.54296875)(233.8328125,404.92382812)(233.60234375,405.41210938)
\curveto(233.37578125,405.90429688)(233.2625,406.46875)(233.2625,407.10546875)
\curveto(233.2625,407.7265625)(233.36601562,408.2890625)(233.57304687,408.79296875)
\curveto(233.78007812,409.30078125)(234.090625,409.68945312)(234.5046875,409.95898438)
\curveto(234.91875,410.22851562)(235.38164062,410.36328125)(235.89335937,410.36328125)
\curveto(236.26835937,410.36328125)(236.60234375,410.28320312)(236.8953125,410.12304688)
\curveto(237.18828125,409.96679688)(237.4265625,409.76171875)(237.61015625,409.5078125)
\lineto(237.61015625,412.58984375)
\lineto(238.65898437,412.58984375)
\lineto(238.65898437,404)
\closepath
\moveto(234.34648437,407.10546875)
\curveto(234.34648437,406.30859375)(234.51445312,405.71289062)(234.85039062,405.31835938)
\curveto(235.18632812,404.92382812)(235.5828125,404.7265625)(236.03984375,404.7265625)
\curveto(236.50078125,404.7265625)(236.89140625,404.9140625)(237.21171875,405.2890625)
\curveto(237.5359375,405.66796875)(237.69804687,406.24414062)(237.69804687,407.01757812)
\curveto(237.69804687,407.86914062)(237.53398437,408.49414062)(237.20585937,408.89257812)
\curveto(236.87773437,409.29101562)(236.4734375,409.49023438)(235.99296875,409.49023438)
\curveto(235.52421875,409.49023438)(235.13164062,409.29882812)(234.81523437,408.91601562)
\curveto(234.50273437,408.53320312)(234.34648437,407.9296875)(234.34648437,407.10546875)
\closepath
}
}
{
\newrgbcolor{curcolor}{0 0 0}
\pscustom[linestyle=none,fillstyle=solid,fillcolor=curcolor]
{
\newpath
\moveto(244.57695312,406.00390625)
\lineto(245.66679687,405.86914062)
\curveto(245.49492187,405.23242188)(245.1765625,404.73828125)(244.71171875,404.38671875)
\curveto(244.246875,404.03515625)(243.653125,403.859375)(242.93046875,403.859375)
\curveto(242.0203125,403.859375)(241.29765625,404.13867188)(240.7625,404.69726562)
\curveto(240.23125,405.25976562)(239.965625,406.046875)(239.965625,407.05859375)
\curveto(239.965625,408.10546875)(240.23515625,408.91796875)(240.77421875,409.49609375)
\curveto(241.31328125,410.07421875)(242.0125,410.36328125)(242.871875,410.36328125)
\curveto(243.70390625,410.36328125)(244.38359375,410.08007812)(244.9109375,409.51367188)
\curveto(245.43828125,408.94726562)(245.70195312,408.15039062)(245.70195312,407.12304688)
\curveto(245.70195312,407.06054688)(245.7,406.96679688)(245.69609375,406.84179688)
\lineto(241.05546875,406.84179688)
\curveto(241.09453125,406.15820312)(241.28789062,405.63476562)(241.63554687,405.27148438)
\curveto(241.98320312,404.90820312)(242.41679687,404.7265625)(242.93632812,404.7265625)
\curveto(243.32304687,404.7265625)(243.653125,404.828125)(243.9265625,405.03125)
\curveto(244.2,405.234375)(244.41679687,405.55859375)(244.57695312,406.00390625)
\closepath
\moveto(241.1140625,407.70898438)
\lineto(244.58867187,407.70898438)
\curveto(244.54179687,408.23242188)(244.40898437,408.625)(244.19023437,408.88671875)
\curveto(243.85429687,409.29296875)(243.41875,409.49609375)(242.88359375,409.49609375)
\curveto(242.39921875,409.49609375)(241.99101562,409.33398438)(241.65898437,409.00976562)
\curveto(241.33085937,408.68554688)(241.14921875,408.25195312)(241.1140625,407.70898438)
\closepath
}
}
{
\newrgbcolor{curcolor}{1 0 0}
\pscustom[linewidth=1,linecolor=curcolor]
{
\newpath
\moveto(254.5,407.9)
\lineto(296.7,407.9)
\moveto(55.3,290.7)
\lineto(141.9,292.4)
\lineto(228.5,139.3)
\lineto(315.2,140.3)
\lineto(401.8,137.3)
\lineto(488.4,139.3)
\lineto(575,140.3)
}
}
{
\newrgbcolor{curcolor}{0 0 0}
\pscustom[linestyle=none,fillstyle=solid,fillcolor=curcolor]
{
\newpath
\moveto(135.40507812,390.18359375)
\curveto(135.40507812,391.609375)(135.78789062,392.72460938)(136.55351562,393.52929688)
\curveto(137.31914062,394.33789062)(138.30742187,394.7421875)(139.51835937,394.7421875)
\curveto(140.31132812,394.7421875)(141.02617187,394.55273438)(141.66289062,394.17382812)
\curveto(142.29960937,393.79492188)(142.78398437,393.265625)(143.11601562,392.5859375)
\curveto(143.45195312,391.91015625)(143.61992187,391.14257812)(143.61992187,390.28320312)
\curveto(143.61992187,389.41210938)(143.44414062,388.6328125)(143.09257812,387.9453125)
\curveto(142.74101562,387.2578125)(142.24296875,386.73632812)(141.5984375,386.38085938)
\curveto(140.95390625,386.02929688)(140.25859375,385.85351562)(139.5125,385.85351562)
\curveto(138.70390625,385.85351562)(137.98125,386.04882812)(137.34453125,386.43945312)
\curveto(136.7078125,386.83007812)(136.22539062,387.36328125)(135.89726562,388.0390625)
\curveto(135.56914062,388.71484375)(135.40507812,389.4296875)(135.40507812,390.18359375)
\closepath
\moveto(136.57695312,390.16601562)
\curveto(136.57695312,389.13085938)(136.85429687,388.31445312)(137.40898437,387.71679688)
\curveto(137.96757812,387.12304688)(138.66679687,386.82617188)(139.50664062,386.82617188)
\curveto(140.36210937,386.82617188)(141.06523437,387.12695312)(141.61601562,387.72851562)
\curveto(142.17070312,388.33007812)(142.44804687,389.18359375)(142.44804687,390.2890625)
\curveto(142.44804687,390.98828125)(142.32890625,391.59765625)(142.090625,392.1171875)
\curveto(141.85625,392.640625)(141.51054687,393.04492188)(141.05351562,393.33007812)
\curveto(140.60039062,393.61914062)(140.090625,393.76367188)(139.52421875,393.76367188)
\curveto(138.71953125,393.76367188)(138.02617187,393.48632812)(137.44414062,392.93164062)
\curveto(136.86601562,392.38085938)(136.57695312,391.45898438)(136.57695312,390.16601562)
\closepath
}
}
{
\newrgbcolor{curcolor}{0 0 0}
\pscustom[linestyle=none,fillstyle=solid,fillcolor=curcolor]
{
\newpath
\moveto(144.9265625,386)
\lineto(144.9265625,394.58984375)
\lineto(145.98125,394.58984375)
\lineto(145.98125,386)
\closepath
}
}
{
\newrgbcolor{curcolor}{0 0 0}
\pscustom[linestyle=none,fillstyle=solid,fillcolor=curcolor]
{
\newpath
\moveto(151.653125,386)
\lineto(151.653125,386.78515625)
\curveto(151.25859375,386.16796875)(150.67851562,385.859375)(149.91289062,385.859375)
\curveto(149.41679687,385.859375)(148.95976562,385.99609375)(148.54179687,386.26953125)
\curveto(148.12773437,386.54296875)(147.80546875,386.92382812)(147.575,387.41210938)
\curveto(147.3484375,387.90429688)(147.23515625,388.46875)(147.23515625,389.10546875)
\curveto(147.23515625,389.7265625)(147.33867187,390.2890625)(147.54570312,390.79296875)
\curveto(147.75273437,391.30078125)(148.06328125,391.68945312)(148.47734375,391.95898438)
\curveto(148.89140625,392.22851562)(149.35429687,392.36328125)(149.86601562,392.36328125)
\curveto(150.24101562,392.36328125)(150.575,392.28320312)(150.86796875,392.12304688)
\curveto(151.1609375,391.96679688)(151.39921875,391.76171875)(151.5828125,391.5078125)
\lineto(151.5828125,394.58984375)
\lineto(152.63164062,394.58984375)
\lineto(152.63164062,386)
\closepath
\moveto(148.31914062,389.10546875)
\curveto(148.31914062,388.30859375)(148.48710937,387.71289062)(148.82304687,387.31835938)
\curveto(149.15898437,386.92382812)(149.55546875,386.7265625)(150.0125,386.7265625)
\curveto(150.4734375,386.7265625)(150.8640625,386.9140625)(151.184375,387.2890625)
\curveto(151.50859375,387.66796875)(151.67070312,388.24414062)(151.67070312,389.01757812)
\curveto(151.67070312,389.86914062)(151.50664062,390.49414062)(151.17851562,390.89257812)
\curveto(150.85039062,391.29101562)(150.44609375,391.49023438)(149.965625,391.49023438)
\curveto(149.496875,391.49023438)(149.10429687,391.29882812)(148.78789062,390.91601562)
\curveto(148.47539062,390.53320312)(148.31914062,389.9296875)(148.31914062,389.10546875)
\closepath
}
}
{
\newrgbcolor{curcolor}{0 0 0}
\pscustom[linestyle=none,fillstyle=solid,fillcolor=curcolor]
{
\newpath
\moveto(157.07304687,386)
\lineto(157.07304687,387.0546875)
\lineto(161.4734375,392.55664062)
\curveto(161.7859375,392.94726562)(162.0828125,393.28710938)(162.3640625,393.57617188)
\lineto(157.57109375,393.57617188)
\lineto(157.57109375,394.58984375)
\lineto(163.7234375,394.58984375)
\lineto(163.7234375,393.57617188)
\lineto(158.90117187,387.6171875)
\lineto(158.3796875,387.01367188)
\lineto(163.8640625,387.01367188)
\lineto(163.8640625,386)
\closepath
}
}
{
\newrgbcolor{curcolor}{0 0 0}
\pscustom[linestyle=none,fillstyle=solid,fillcolor=curcolor]
{
\newpath
\moveto(165.14726562,386)
\lineto(165.14726562,394.58984375)
\lineto(170.9421875,394.58984375)
\lineto(170.9421875,393.57617188)
\lineto(166.28398437,393.57617188)
\lineto(166.28398437,390.91601562)
\lineto(170.31523437,390.91601562)
\lineto(170.31523437,389.90234375)
\lineto(166.28398437,389.90234375)
\lineto(166.28398437,386)
\closepath
}
}
{
\newrgbcolor{curcolor}{0 0 0}
\pscustom[linestyle=none,fillstyle=solid,fillcolor=curcolor]
{
\newpath
\moveto(172.03203125,388.75976562)
\lineto(173.10429687,388.85351562)
\curveto(173.15507812,388.42382812)(173.27226562,388.0703125)(173.45585937,387.79296875)
\curveto(173.64335937,387.51953125)(173.93242187,387.296875)(174.32304687,387.125)
\curveto(174.71367187,386.95703125)(175.153125,386.87304688)(175.64140625,386.87304688)
\curveto(176.075,386.87304688)(176.4578125,386.9375)(176.78984375,387.06640625)
\curveto(177.121875,387.1953125)(177.36796875,387.37109375)(177.528125,387.59375)
\curveto(177.6921875,387.8203125)(177.77421875,388.06640625)(177.77421875,388.33203125)
\curveto(177.77421875,388.6015625)(177.69609375,388.8359375)(177.53984375,389.03515625)
\curveto(177.38359375,389.23828125)(177.12578125,389.40820312)(176.76640625,389.54492188)
\curveto(176.5359375,389.63476562)(176.02617187,389.7734375)(175.23710937,389.9609375)
\curveto(174.44804687,390.15234375)(173.8953125,390.33203125)(173.57890625,390.5)
\curveto(173.16875,390.71484375)(172.86210937,390.98046875)(172.65898437,391.296875)
\curveto(172.45976562,391.6171875)(172.36015625,391.97460938)(172.36015625,392.36914062)
\curveto(172.36015625,392.80273438)(172.48320312,393.20703125)(172.72929687,393.58203125)
\curveto(172.97539062,393.9609375)(173.33476562,394.24804688)(173.80742187,394.44335938)
\curveto(174.28007812,394.63867188)(174.80546875,394.73632812)(175.38359375,394.73632812)
\curveto(176.0203125,394.73632812)(176.58085937,394.6328125)(177.06523437,394.42578125)
\curveto(177.55351562,394.22265625)(177.92851562,393.921875)(178.19023437,393.5234375)
\curveto(178.45195312,393.125)(178.59257812,392.67382812)(178.61210937,392.16992188)
\lineto(177.52226562,392.08789062)
\curveto(177.46367187,392.63085938)(177.26445312,393.04101562)(176.92460937,393.31835938)
\curveto(176.58867187,393.59570312)(176.090625,393.734375)(175.43046875,393.734375)
\curveto(174.74296875,393.734375)(174.24101562,393.60742188)(173.92460937,393.35351562)
\curveto(173.61210937,393.10351562)(173.45585937,392.80078125)(173.45585937,392.4453125)
\curveto(173.45585937,392.13671875)(173.5671875,391.8828125)(173.78984375,391.68359375)
\curveto(174.00859375,391.484375)(174.57890625,391.27929688)(175.50078125,391.06835938)
\curveto(176.4265625,390.86132812)(177.06132812,390.6796875)(177.40507812,390.5234375)
\curveto(177.90507812,390.29296875)(178.27421875,390)(178.5125,389.64453125)
\curveto(178.75078125,389.29296875)(178.86992187,388.88671875)(178.86992187,388.42578125)
\curveto(178.86992187,387.96875)(178.7390625,387.53710938)(178.47734375,387.13085938)
\curveto(178.215625,386.72851562)(177.83867187,386.4140625)(177.34648437,386.1875)
\curveto(176.85820312,385.96484375)(176.30742187,385.85351562)(175.69414062,385.85351562)
\curveto(174.91679687,385.85351562)(174.26445312,385.96679688)(173.73710937,386.19335938)
\curveto(173.21367187,386.41992188)(172.8015625,386.75976562)(172.50078125,387.21289062)
\curveto(172.20390625,387.66992188)(172.04765625,388.18554688)(172.03203125,388.75976562)
\closepath
}
}
{
\newrgbcolor{curcolor}{0 0 0}
\pscustom[linestyle=none,fillstyle=solid,fillcolor=curcolor]
{
\newpath
\moveto(183.36992187,388.75976562)
\lineto(184.4421875,388.85351562)
\curveto(184.49296875,388.42382812)(184.61015625,388.0703125)(184.79375,387.79296875)
\curveto(184.98125,387.51953125)(185.2703125,387.296875)(185.6609375,387.125)
\curveto(186.0515625,386.95703125)(186.49101562,386.87304688)(186.97929687,386.87304688)
\curveto(187.41289062,386.87304688)(187.79570312,386.9375)(188.12773437,387.06640625)
\curveto(188.45976562,387.1953125)(188.70585937,387.37109375)(188.86601562,387.59375)
\curveto(189.03007812,387.8203125)(189.11210937,388.06640625)(189.11210937,388.33203125)
\curveto(189.11210937,388.6015625)(189.03398437,388.8359375)(188.87773437,389.03515625)
\curveto(188.72148437,389.23828125)(188.46367187,389.40820312)(188.10429687,389.54492188)
\curveto(187.87382812,389.63476562)(187.3640625,389.7734375)(186.575,389.9609375)
\curveto(185.7859375,390.15234375)(185.23320312,390.33203125)(184.91679687,390.5)
\curveto(184.50664062,390.71484375)(184.2,390.98046875)(183.996875,391.296875)
\curveto(183.79765625,391.6171875)(183.69804687,391.97460938)(183.69804687,392.36914062)
\curveto(183.69804687,392.80273438)(183.82109375,393.20703125)(184.0671875,393.58203125)
\curveto(184.31328125,393.9609375)(184.67265625,394.24804688)(185.1453125,394.44335938)
\curveto(185.61796875,394.63867188)(186.14335937,394.73632812)(186.72148437,394.73632812)
\curveto(187.35820312,394.73632812)(187.91875,394.6328125)(188.403125,394.42578125)
\curveto(188.89140625,394.22265625)(189.26640625,393.921875)(189.528125,393.5234375)
\curveto(189.78984375,393.125)(189.93046875,392.67382812)(189.95,392.16992188)
\lineto(188.86015625,392.08789062)
\curveto(188.8015625,392.63085938)(188.60234375,393.04101562)(188.2625,393.31835938)
\curveto(187.9265625,393.59570312)(187.42851562,393.734375)(186.76835937,393.734375)
\curveto(186.08085937,393.734375)(185.57890625,393.60742188)(185.2625,393.35351562)
\curveto(184.95,393.10351562)(184.79375,392.80078125)(184.79375,392.4453125)
\curveto(184.79375,392.13671875)(184.90507812,391.8828125)(185.12773437,391.68359375)
\curveto(185.34648437,391.484375)(185.91679687,391.27929688)(186.83867187,391.06835938)
\curveto(187.76445312,390.86132812)(188.39921875,390.6796875)(188.74296875,390.5234375)
\curveto(189.24296875,390.29296875)(189.61210937,390)(189.85039062,389.64453125)
\curveto(190.08867187,389.29296875)(190.2078125,388.88671875)(190.2078125,388.42578125)
\curveto(190.2078125,387.96875)(190.07695312,387.53710938)(189.81523437,387.13085938)
\curveto(189.55351562,386.72851562)(189.1765625,386.4140625)(188.684375,386.1875)
\curveto(188.19609375,385.96484375)(187.6453125,385.85351562)(187.03203125,385.85351562)
\curveto(186.2546875,385.85351562)(185.60234375,385.96679688)(185.075,386.19335938)
\curveto(184.5515625,386.41992188)(184.13945312,386.75976562)(183.83867187,387.21289062)
\curveto(183.54179687,387.66992188)(183.38554687,388.18554688)(183.36992187,388.75976562)
\closepath
}
}
{
\newrgbcolor{curcolor}{0 0 0}
\pscustom[linestyle=none,fillstyle=solid,fillcolor=curcolor]
{
\newpath
\moveto(195.68632812,386.76757812)
\curveto(195.29570312,386.43554688)(194.91875,386.20117188)(194.55546875,386.06445312)
\curveto(194.19609375,385.92773438)(193.809375,385.859375)(193.3953125,385.859375)
\curveto(192.71171875,385.859375)(192.18632812,386.02539062)(191.81914062,386.35742188)
\curveto(191.45195312,386.69335938)(191.26835937,387.12109375)(191.26835937,387.640625)
\curveto(191.26835937,387.9453125)(191.33671875,388.22265625)(191.4734375,388.47265625)
\curveto(191.6140625,388.7265625)(191.79570312,388.9296875)(192.01835937,389.08203125)
\curveto(192.24492187,389.234375)(192.49882812,389.34960938)(192.78007812,389.42773438)
\curveto(192.98710937,389.48242188)(193.29960937,389.53515625)(193.71757812,389.5859375)
\curveto(194.56914062,389.6875)(195.19609375,389.80859375)(195.5984375,389.94921875)
\curveto(195.60234375,390.09375)(195.60429687,390.18554688)(195.60429687,390.22460938)
\curveto(195.60429687,390.65429688)(195.5046875,390.95703125)(195.30546875,391.1328125)
\curveto(195.0359375,391.37109375)(194.63554687,391.49023438)(194.10429687,391.49023438)
\curveto(193.60820312,391.49023438)(193.24101562,391.40234375)(193.00273437,391.2265625)
\curveto(192.76835937,391.0546875)(192.59453125,390.74804688)(192.48125,390.30664062)
\lineto(191.45,390.44726562)
\curveto(191.54375,390.88867188)(191.69804687,391.24414062)(191.91289062,391.51367188)
\curveto(192.12773437,391.78710938)(192.43828125,391.99609375)(192.84453125,392.140625)
\curveto(193.25078125,392.2890625)(193.72148437,392.36328125)(194.25664062,392.36328125)
\curveto(194.78789062,392.36328125)(195.21953125,392.30078125)(195.5515625,392.17578125)
\curveto(195.88359375,392.05078125)(196.12773437,391.89257812)(196.28398437,391.70117188)
\curveto(196.44023437,391.51367188)(196.54960937,391.27539062)(196.61210937,390.98632812)
\curveto(196.64726562,390.80664062)(196.66484375,390.48242188)(196.66484375,390.01367188)
\lineto(196.66484375,388.60742188)
\curveto(196.66484375,387.62695312)(196.68632812,387.00585938)(196.72929687,386.74414062)
\curveto(196.77617187,386.48632812)(196.86601562,386.23828125)(196.99882812,386)
\lineto(195.89726562,386)
\curveto(195.78789062,386.21875)(195.71757812,386.47460938)(195.68632812,386.76757812)
\closepath
\moveto(195.5984375,389.12304688)
\curveto(195.215625,388.96679688)(194.64140625,388.83398438)(193.87578125,388.72460938)
\curveto(193.4421875,388.66210938)(193.13554687,388.59179688)(192.95585937,388.51367188)
\curveto(192.77617187,388.43554688)(192.6375,388.3203125)(192.53984375,388.16796875)
\curveto(192.4421875,388.01953125)(192.39335937,387.85351562)(192.39335937,387.66992188)
\curveto(192.39335937,387.38867188)(192.49882812,387.15429688)(192.70976562,386.96679688)
\curveto(192.92460937,386.77929688)(193.23710937,386.68554688)(193.64726562,386.68554688)
\curveto(194.05351562,386.68554688)(194.41484375,386.7734375)(194.73125,386.94921875)
\curveto(195.04765625,387.12890625)(195.28007812,387.37304688)(195.42851562,387.68164062)
\curveto(195.54179687,387.91992188)(195.5984375,388.27148438)(195.5984375,388.73632812)
\closepath
}
}
{
\newrgbcolor{curcolor}{0 0 0}
\pscustom[linestyle=none,fillstyle=solid,fillcolor=curcolor]
{
\newpath
\moveto(198.29960937,386)
\lineto(198.29960937,392.22265625)
\lineto(199.24296875,392.22265625)
\lineto(199.24296875,391.34960938)
\curveto(199.43828125,391.65429688)(199.69804687,391.8984375)(200.02226562,392.08203125)
\curveto(200.34648437,392.26953125)(200.715625,392.36328125)(201.1296875,392.36328125)
\curveto(201.590625,392.36328125)(201.96757812,392.26757812)(202.26054687,392.07617188)
\curveto(202.55742187,391.88476562)(202.76640625,391.6171875)(202.8875,391.2734375)
\curveto(203.3796875,392)(204.0203125,392.36328125)(204.809375,392.36328125)
\curveto(205.4265625,392.36328125)(205.90117187,392.19140625)(206.23320312,391.84765625)
\curveto(206.56523437,391.5078125)(206.73125,390.98242188)(206.73125,390.27148438)
\lineto(206.73125,386)
\lineto(205.68242187,386)
\lineto(205.68242187,389.91992188)
\curveto(205.68242187,390.34179688)(205.64726562,390.64453125)(205.57695312,390.828125)
\curveto(205.51054687,391.015625)(205.3875,391.16601562)(205.2078125,391.27929688)
\curveto(205.028125,391.39257812)(204.8171875,391.44921875)(204.575,391.44921875)
\curveto(204.1375,391.44921875)(203.77421875,391.30273438)(203.48515625,391.00976562)
\curveto(203.19609375,390.72070312)(203.0515625,390.25585938)(203.0515625,389.61523438)
\lineto(203.0515625,386)
\lineto(201.996875,386)
\lineto(201.996875,390.04296875)
\curveto(201.996875,390.51171875)(201.9109375,390.86328125)(201.7390625,391.09765625)
\curveto(201.5671875,391.33203125)(201.2859375,391.44921875)(200.8953125,391.44921875)
\curveto(200.5984375,391.44921875)(200.32304687,391.37109375)(200.06914062,391.21484375)
\curveto(199.81914062,391.05859375)(199.6375,390.83007812)(199.52421875,390.52929688)
\curveto(199.4109375,390.22851562)(199.35429687,389.79492188)(199.35429687,389.22851562)
\lineto(199.35429687,386)
\closepath
}
}
{
\newrgbcolor{curcolor}{0 0 0}
\pscustom[linestyle=none,fillstyle=solid,fillcolor=curcolor]
{
\newpath
\moveto(212.55546875,388.00390625)
\lineto(213.6453125,387.86914062)
\curveto(213.4734375,387.23242188)(213.15507812,386.73828125)(212.69023437,386.38671875)
\curveto(212.22539062,386.03515625)(211.63164062,385.859375)(210.90898437,385.859375)
\curveto(209.99882812,385.859375)(209.27617187,386.13867188)(208.74101562,386.69726562)
\curveto(208.20976562,387.25976562)(207.94414062,388.046875)(207.94414062,389.05859375)
\curveto(207.94414062,390.10546875)(208.21367187,390.91796875)(208.75273437,391.49609375)
\curveto(209.29179687,392.07421875)(209.99101562,392.36328125)(210.85039062,392.36328125)
\curveto(211.68242187,392.36328125)(212.36210937,392.08007812)(212.88945312,391.51367188)
\curveto(213.41679687,390.94726562)(213.68046875,390.15039062)(213.68046875,389.12304688)
\curveto(213.68046875,389.06054688)(213.67851562,388.96679688)(213.67460937,388.84179688)
\lineto(209.03398437,388.84179688)
\curveto(209.07304687,388.15820312)(209.26640625,387.63476562)(209.6140625,387.27148438)
\curveto(209.96171875,386.90820312)(210.3953125,386.7265625)(210.91484375,386.7265625)
\curveto(211.3015625,386.7265625)(211.63164062,386.828125)(211.90507812,387.03125)
\curveto(212.17851562,387.234375)(212.3953125,387.55859375)(212.55546875,388.00390625)
\closepath
\moveto(209.09257812,389.70898438)
\lineto(212.5671875,389.70898438)
\curveto(212.5203125,390.23242188)(212.3875,390.625)(212.16875,390.88671875)
\curveto(211.8328125,391.29296875)(211.39726562,391.49609375)(210.86210937,391.49609375)
\curveto(210.37773437,391.49609375)(209.96953125,391.33398438)(209.6375,391.00976562)
\curveto(209.309375,390.68554688)(209.12773437,390.25195312)(209.09257812,389.70898438)
\closepath
}
}
{
\newrgbcolor{curcolor}{0 0 0}
\pscustom[linestyle=none,fillstyle=solid,fillcolor=curcolor]
{
\newpath
\moveto(218.4265625,386)
\lineto(218.4265625,394.58984375)
\lineto(219.59257812,394.58984375)
\lineto(224.10429687,387.84570312)
\lineto(224.10429687,394.58984375)
\lineto(225.19414062,394.58984375)
\lineto(225.19414062,386)
\lineto(224.028125,386)
\lineto(219.51640625,392.75)
\lineto(219.51640625,386)
\closepath
}
}
{
\newrgbcolor{curcolor}{0 0 0}
\pscustom[linestyle=none,fillstyle=solid,fillcolor=curcolor]
{
\newpath
\moveto(226.57695312,389.11132812)
\curveto(226.57695312,390.26367188)(226.89726562,391.1171875)(227.53789062,391.671875)
\curveto(228.07304687,392.1328125)(228.72539062,392.36328125)(229.49492187,392.36328125)
\curveto(230.35039062,392.36328125)(231.04960937,392.08203125)(231.59257812,391.51953125)
\curveto(232.13554687,390.9609375)(232.40703125,390.1875)(232.40703125,389.19921875)
\curveto(232.40703125,388.3984375)(232.2859375,387.76757812)(232.04375,387.30664062)
\curveto(231.80546875,386.84960938)(231.45585937,386.49414062)(230.99492187,386.24023438)
\curveto(230.53789062,385.98632812)(230.03789062,385.859375)(229.49492187,385.859375)
\curveto(228.62382812,385.859375)(227.91875,386.13867188)(227.3796875,386.69726562)
\curveto(226.84453125,387.25585938)(226.57695312,388.06054688)(226.57695312,389.11132812)
\closepath
\moveto(227.6609375,389.11132812)
\curveto(227.6609375,388.31445312)(227.83476562,387.71679688)(228.18242187,387.31835938)
\curveto(228.53007812,386.92382812)(228.96757812,386.7265625)(229.49492187,386.7265625)
\curveto(230.01835937,386.7265625)(230.45390625,386.92578125)(230.8015625,387.32421875)
\curveto(231.14921875,387.72265625)(231.32304687,388.33007812)(231.32304687,389.14648438)
\curveto(231.32304687,389.91601562)(231.14726562,390.49804688)(230.79570312,390.89257812)
\curveto(230.44804687,391.29101562)(230.01445312,391.49023438)(229.49492187,391.49023438)
\curveto(228.96757812,391.49023438)(228.53007812,391.29296875)(228.18242187,390.8984375)
\curveto(227.83476562,390.50390625)(227.6609375,389.90820312)(227.6609375,389.11132812)
\closepath
}
}
{
\newrgbcolor{curcolor}{0 0 0}
\pscustom[linestyle=none,fillstyle=solid,fillcolor=curcolor]
{
\newpath
\moveto(237.68046875,386)
\lineto(237.68046875,386.78515625)
\curveto(237.2859375,386.16796875)(236.70585937,385.859375)(235.94023437,385.859375)
\curveto(235.44414062,385.859375)(234.98710937,385.99609375)(234.56914062,386.26953125)
\curveto(234.15507812,386.54296875)(233.8328125,386.92382812)(233.60234375,387.41210938)
\curveto(233.37578125,387.90429688)(233.2625,388.46875)(233.2625,389.10546875)
\curveto(233.2625,389.7265625)(233.36601562,390.2890625)(233.57304687,390.79296875)
\curveto(233.78007812,391.30078125)(234.090625,391.68945312)(234.5046875,391.95898438)
\curveto(234.91875,392.22851562)(235.38164062,392.36328125)(235.89335937,392.36328125)
\curveto(236.26835937,392.36328125)(236.60234375,392.28320312)(236.8953125,392.12304688)
\curveto(237.18828125,391.96679688)(237.4265625,391.76171875)(237.61015625,391.5078125)
\lineto(237.61015625,394.58984375)
\lineto(238.65898437,394.58984375)
\lineto(238.65898437,386)
\closepath
\moveto(234.34648437,389.10546875)
\curveto(234.34648437,388.30859375)(234.51445312,387.71289062)(234.85039062,387.31835938)
\curveto(235.18632812,386.92382812)(235.5828125,386.7265625)(236.03984375,386.7265625)
\curveto(236.50078125,386.7265625)(236.89140625,386.9140625)(237.21171875,387.2890625)
\curveto(237.5359375,387.66796875)(237.69804687,388.24414062)(237.69804687,389.01757812)
\curveto(237.69804687,389.86914062)(237.53398437,390.49414062)(237.20585937,390.89257812)
\curveto(236.87773437,391.29101562)(236.4734375,391.49023438)(235.99296875,391.49023438)
\curveto(235.52421875,391.49023438)(235.13164062,391.29882812)(234.81523437,390.91601562)
\curveto(234.50273437,390.53320312)(234.34648437,389.9296875)(234.34648437,389.10546875)
\closepath
}
}
{
\newrgbcolor{curcolor}{0 0 0}
\pscustom[linestyle=none,fillstyle=solid,fillcolor=curcolor]
{
\newpath
\moveto(244.57695312,388.00390625)
\lineto(245.66679687,387.86914062)
\curveto(245.49492187,387.23242188)(245.1765625,386.73828125)(244.71171875,386.38671875)
\curveto(244.246875,386.03515625)(243.653125,385.859375)(242.93046875,385.859375)
\curveto(242.0203125,385.859375)(241.29765625,386.13867188)(240.7625,386.69726562)
\curveto(240.23125,387.25976562)(239.965625,388.046875)(239.965625,389.05859375)
\curveto(239.965625,390.10546875)(240.23515625,390.91796875)(240.77421875,391.49609375)
\curveto(241.31328125,392.07421875)(242.0125,392.36328125)(242.871875,392.36328125)
\curveto(243.70390625,392.36328125)(244.38359375,392.08007812)(244.9109375,391.51367188)
\curveto(245.43828125,390.94726562)(245.70195312,390.15039062)(245.70195312,389.12304688)
\curveto(245.70195312,389.06054688)(245.7,388.96679688)(245.69609375,388.84179688)
\lineto(241.05546875,388.84179688)
\curveto(241.09453125,388.15820312)(241.28789062,387.63476562)(241.63554687,387.27148438)
\curveto(241.98320312,386.90820312)(242.41679687,386.7265625)(242.93632812,386.7265625)
\curveto(243.32304687,386.7265625)(243.653125,386.828125)(243.9265625,387.03125)
\curveto(244.2,387.234375)(244.41679687,387.55859375)(244.57695312,388.00390625)
\closepath
\moveto(241.1140625,389.70898438)
\lineto(244.58867187,389.70898438)
\curveto(244.54179687,390.23242188)(244.40898437,390.625)(244.19023437,390.88671875)
\curveto(243.85429687,391.29296875)(243.41875,391.49609375)(242.88359375,391.49609375)
\curveto(242.39921875,391.49609375)(241.99101562,391.33398438)(241.65898437,391.00976562)
\curveto(241.33085937,390.68554688)(241.14921875,390.25195312)(241.1140625,389.70898438)
\closepath
}
}
{
\newrgbcolor{curcolor}{0 0 1}
\pscustom[linewidth=1,linecolor=curcolor]
{
\newpath
\moveto(254.5,389.9)
\lineto(296.7,389.9)
\moveto(55.3,273)
\lineto(141.9,309.4)
\lineto(228.5,112.6)
\lineto(315.2,139.3)
\lineto(401.8,138.3)
\lineto(488.4,141.2)
\lineto(575,141.2)
}
}
{
\newrgbcolor{curcolor}{0 0 0}
\pscustom[linewidth=1,linecolor=curcolor]
{
\newpath
\moveto(55.3,425.9)
\lineto(55.3,57.6)
\lineto(575,57.6)
\lineto(575,425.9)
\closepath
}
}
\end{pspicture}
}
    \captionsetup{width=0.75\linewidth}
    \caption{Bandwidth, as with latency, is identical when on a different node with my task migration prototype
        as compared on reads from the same NUMA node.}
    \label{fig:bandwidthoptimal}
\end{figure}

\section{Impact of Process Size}
The improvements seen with task migration enabled are not universal, however, and 
the actual improvement visible to user programs is highly dependent on the size of the program.
This is because when a process is moved to a different NUMA node all of its other memory allocations must be moved with it in order to see the full benefits of local memory access.
Otherwise, the process will continue to cross the interconnect in order to access any memory that it allocated previously,
and continue to incur the latency and bandwidth penalties associated accessing memory from a different NUMA node.
By creating variants of my simple read program of various sizes, I was able to measure the difference
in runtime improvement that resulted from differences in program memory size.

In order to test different process sizes, a large array was added to the program, that was then filled with data so that it would not be 
optimized away by the compiler.
The size and existence of this array was determined by the \texttt{HUGE} macro, 
defined on the \texttt{gcc} command line with \texttt{-D}, 
so that different sized programs could be built from the exact same source code.

Given that accessing ARC data for a file 1 GB or larger across the interconnect causes an approximately 10\% increase in
runtime, which improves even more as the file size does, as discussed previously,
transferring a larger process closer to that same size will likely have a similar impact, 
reducing the improvement possible from task migration.
With that knowledge, I can attempt a theoretical model of the performance improvement depending on the size of the process and the file.
\begin{align*}
    I &= \text{Performance Improvement} \\
    M &= \text{Maximum Possible Improvement} \\
    P &= \text{Process Size in Memory} \\
    F &= \text{File Size} \\
    R &= \text{Reduction Factor}
\intertext{Given these variables, I would expect performance improvement to be determined by something like the following:}
    I &= M-\left ( \frac{P}{F}\times R \right )
\end{align*}

From this theoretical model I would expect no improvement if the process and the file are the same size, 
and to see the improvement slowly reduced as the size of the process in memory increases.
Fundamentally, what causes the performance improvement visible in these results is moving less data across the NUMA interconnect.
When the process is larger, the amount of data that needs to moved increases, reducing the possible improvement by some reduction factor $R$.

\begin{figure}[H]
    \centering
    \resizebox{0.75\linewidth}{!}{%LaTeX with PSTricks extensions
%%Creator: Inkscape 1.0.2-2 (e86c870879, 2021-01-15)
%%Please note this file requires PSTricks extensions
\psset{xunit=.5pt,yunit=.5pt,runit=.5pt}
\begin{pspicture}(600,480)
{
\newrgbcolor{curcolor}{0 0 0}
\pscustom[linewidth=1,linecolor=curcolor]
{
\newpath
\moveto(85.9,123.2)
\lineto(242.7,219.7)
}
}
{
\newrgbcolor{curcolor}{0 0 0}
\pscustom[linewidth=1,linecolor=curcolor]
{
\newpath
\moveto(514.1,164)
\lineto(242.7,219.7)
}
}
{
\newrgbcolor{curcolor}{0 0 0}
\pscustom[linewidth=1,linecolor=curcolor]
{
\newpath
\moveto(85.9,123.2)
\lineto(85.9,316)
}
}
{
\newrgbcolor{curcolor}{0 0 0}
\pscustom[linewidth=1,linecolor=curcolor]
{
\newpath
\moveto(242.7,219.7)
\lineto(242.7,335.5)
}
}
{
\newrgbcolor{curcolor}{0 0 0}
\pscustom[linewidth=1,linecolor=curcolor]
{
\newpath
\moveto(514.1,164)
\lineto(514.1,303.4)
}
}
{
\newrgbcolor{curcolor}{0 0 0}
\pscustom[linewidth=1,linecolor=curcolor]
{
\newpath
\moveto(85.9,123.2)
\lineto(91.2,126.5)
}
}
{
\newrgbcolor{curcolor}{0 0 0}
\pscustom[linewidth=1,linecolor=curcolor]
{
\newpath
\moveto(242.7,219.7)
\lineto(237.3,216.4)
}
}
{
\newrgbcolor{curcolor}{0 0 0}
\pscustom[linestyle=none,fillstyle=solid,fillcolor=curcolor]
{
\newpath
\moveto(71.93222656,113.81367188)
\lineto(71.93222656,112.8)
\lineto(66.25449219,112.8)
\curveto(66.24667969,113.05390625)(66.28769531,113.29804688)(66.37753906,113.53242188)
\curveto(66.52207031,113.91914063)(66.75253906,114.3)(67.06894531,114.675)
\curveto(67.38925781,115.05)(67.85019531,115.48359375)(68.45175781,115.97578125)
\curveto(69.38535156,116.74140625)(70.01621094,117.346875)(70.34433594,117.7921875)
\curveto(70.67246094,118.24140625)(70.83652344,118.66523438)(70.83652344,119.06367188)
\curveto(70.83652344,119.48164063)(70.68613281,119.83320313)(70.38535156,120.11835938)
\curveto(70.08847656,120.40742188)(69.69980469,120.55195313)(69.21933594,120.55195313)
\curveto(68.71152344,120.55195313)(68.30527344,120.39960938)(68.00058594,120.09492188)
\curveto(67.69589844,119.79023438)(67.54160156,119.36835938)(67.53769531,118.82929688)
\lineto(66.45371094,118.940625)
\curveto(66.52792969,119.74921875)(66.80722656,120.36445313)(67.29160156,120.78632813)
\curveto(67.77597656,121.21210938)(68.42636719,121.425)(69.24277344,121.425)
\curveto(70.06699219,121.425)(70.71933594,121.19648438)(71.19980469,120.73945313)
\curveto(71.68027344,120.28242188)(71.92050781,119.71601563)(71.92050781,119.04023438)
\curveto(71.92050781,118.69648438)(71.85019531,118.35859375)(71.70957031,118.0265625)
\curveto(71.56894531,117.69453125)(71.33457031,117.34492188)(71.00644531,116.97773438)
\curveto(70.68222656,116.61054688)(70.14121094,116.10664063)(69.38339844,115.46601563)
\curveto(68.75058594,114.93476563)(68.34433594,114.5734375)(68.16464844,114.38203125)
\curveto(67.98496094,114.19453125)(67.83652344,114.00507813)(67.71933594,113.81367188)
\closepath
}
}
{
\newrgbcolor{curcolor}{0 0 0}
\pscustom[linestyle=none,fillstyle=solid,fillcolor=curcolor]
{
\newpath
\moveto(73.06308594,115.05)
\lineto(74.17050781,115.14375)
\curveto(74.25253906,114.6046875)(74.44199219,114.1984375)(74.73886719,113.925)
\curveto(75.03964844,113.65546875)(75.40097656,113.52070313)(75.82285156,113.52070313)
\curveto(76.33066406,113.52070313)(76.76035156,113.71210938)(77.11191406,114.09492188)
\curveto(77.46347656,114.47773438)(77.63925781,114.98554688)(77.63925781,115.61835938)
\curveto(77.63925781,116.21992188)(77.46933594,116.69453125)(77.12949219,117.0421875)
\curveto(76.79355469,117.38984375)(76.35214844,117.56367188)(75.80527344,117.56367188)
\curveto(75.46542969,117.56367188)(75.15878906,117.48554688)(74.88535156,117.32929688)
\curveto(74.61191406,117.17695313)(74.39707031,116.97773438)(74.24082031,116.73164063)
\lineto(73.25058594,116.86054688)
\lineto(74.08261719,121.27265625)
\lineto(78.35410156,121.27265625)
\lineto(78.35410156,120.26484375)
\lineto(74.92636719,120.26484375)
\lineto(74.46347656,117.95625)
\curveto(74.97910156,118.315625)(75.52011719,118.4953125)(76.08652344,118.4953125)
\curveto(76.83652344,118.4953125)(77.46933594,118.23554688)(77.98496094,117.71601563)
\curveto(78.50058594,117.19648438)(78.75839844,116.52851563)(78.75839844,115.71210938)
\curveto(78.75839844,114.93476563)(78.53183594,114.26289063)(78.07871094,113.69648438)
\curveto(77.52792969,113.00117188)(76.77597656,112.65351563)(75.82285156,112.65351563)
\curveto(75.04160156,112.65351563)(74.40292969,112.87226563)(73.90683594,113.30976563)
\curveto(73.41464844,113.74726563)(73.13339844,114.32734375)(73.06308594,115.05)
\closepath
}
}
{
\newrgbcolor{curcolor}{0 0 0}
\pscustom[linestyle=none,fillstyle=solid,fillcolor=curcolor]
{
\newpath
\moveto(79.73691406,117.03632813)
\curveto(79.73691406,118.05195313)(79.84042969,118.86835938)(80.04746094,119.48554688)
\curveto(80.25839844,120.10664063)(80.56894531,120.58515625)(80.97910156,120.92109375)
\curveto(81.39316406,121.25703125)(81.91269531,121.425)(82.53769531,121.425)
\curveto(82.99863281,121.425)(83.40292969,121.33125)(83.75058594,121.14375)
\curveto(84.09824219,120.96015625)(84.38535156,120.69257813)(84.61191406,120.34101563)
\curveto(84.83847656,119.99335938)(85.01621094,119.56757813)(85.14511719,119.06367188)
\curveto(85.27402344,118.56367188)(85.33847656,117.88789063)(85.33847656,117.03632813)
\curveto(85.33847656,116.02851563)(85.23496094,115.2140625)(85.02792969,114.59296875)
\curveto(84.82089844,113.97578125)(84.51035156,113.49726563)(84.09628906,113.15742188)
\curveto(83.68613281,112.82148438)(83.16660156,112.65351563)(82.53769531,112.65351563)
\curveto(81.70957031,112.65351563)(81.05917969,112.95039063)(80.58652344,113.54414063)
\curveto(80.02011719,114.25898438)(79.73691406,115.42304688)(79.73691406,117.03632813)
\closepath
\moveto(80.82089844,117.03632813)
\curveto(80.82089844,115.62617188)(80.98496094,114.68671875)(81.31308594,114.21796875)
\curveto(81.64511719,113.753125)(82.05332031,113.52070313)(82.53769531,113.52070313)
\curveto(83.02207031,113.52070313)(83.42832031,113.75507813)(83.75644531,114.22382813)
\curveto(84.08847656,114.69257813)(84.25449219,115.63007813)(84.25449219,117.03632813)
\curveto(84.25449219,118.45039063)(84.08847656,119.38984375)(83.75644531,119.8546875)
\curveto(83.42832031,120.31953125)(83.01816406,120.55195313)(82.52597656,120.55195313)
\curveto(82.04160156,120.55195313)(81.65488281,120.346875)(81.36582031,119.93671875)
\curveto(81.00253906,119.41328125)(80.82089844,118.44648438)(80.82089844,117.03632813)
\closepath
}
}
{
\newrgbcolor{curcolor}{0 0 0}
\pscustom[linestyle=none,fillstyle=solid,fillcolor=curcolor]
{
\newpath
\moveto(86.80332031,112.8)
\lineto(86.80332031,121.38984375)
\lineto(88.51425781,121.38984375)
\lineto(90.54746094,115.3078125)
\curveto(90.73496094,114.74140625)(90.87167969,114.31757813)(90.95761719,114.03632813)
\curveto(91.05527344,114.34882813)(91.20761719,114.8078125)(91.41464844,115.41328125)
\lineto(93.47128906,121.38984375)
\lineto(95.00058594,121.38984375)
\lineto(95.00058594,112.8)
\lineto(93.90488281,112.8)
\lineto(93.90488281,119.98945313)
\lineto(91.40878906,112.8)
\lineto(90.38339844,112.8)
\lineto(87.89902344,120.1125)
\lineto(87.89902344,112.8)
\closepath
}
}
{
\newrgbcolor{curcolor}{0 0 0}
\pscustom[linewidth=1,linecolor=curcolor]
{
\newpath
\moveto(131.1,114)
\lineto(136.5,117.2)
}
}
{
\newrgbcolor{curcolor}{0 0 0}
\pscustom[linewidth=1,linecolor=curcolor]
{
\newpath
\moveto(287.9,210.4)
\lineto(282.6,207.1)
}
}
{
\newrgbcolor{curcolor}{0 0 0}
\pscustom[linestyle=none,fillstyle=solid,fillcolor=curcolor]
{
\newpath
\moveto(111.68925781,105.75)
\lineto(112.79667969,105.84375)
\curveto(112.87871094,105.3046875)(113.06816406,104.8984375)(113.36503906,104.625)
\curveto(113.66582031,104.35546875)(114.02714844,104.22070312)(114.44902344,104.22070312)
\curveto(114.95683594,104.22070312)(115.38652344,104.41210938)(115.73808594,104.79492188)
\curveto(116.08964844,105.17773438)(116.26542969,105.68554688)(116.26542969,106.31835938)
\curveto(116.26542969,106.91992188)(116.09550781,107.39453125)(115.75566406,107.7421875)
\curveto(115.41972656,108.08984375)(114.97832031,108.26367188)(114.43144531,108.26367188)
\curveto(114.09160156,108.26367188)(113.78496094,108.18554688)(113.51152344,108.02929688)
\curveto(113.23808594,107.87695312)(113.02324219,107.67773438)(112.86699219,107.43164062)
\lineto(111.87675781,107.56054688)
\lineto(112.70878906,111.97265625)
\lineto(116.98027344,111.97265625)
\lineto(116.98027344,110.96484375)
\lineto(113.55253906,110.96484375)
\lineto(113.08964844,108.65625)
\curveto(113.60527344,109.015625)(114.14628906,109.1953125)(114.71269531,109.1953125)
\curveto(115.46269531,109.1953125)(116.09550781,108.93554688)(116.61113281,108.41601562)
\curveto(117.12675781,107.89648438)(117.38457031,107.22851562)(117.38457031,106.41210938)
\curveto(117.38457031,105.63476562)(117.15800781,104.96289062)(116.70488281,104.39648438)
\curveto(116.15410156,103.70117188)(115.40214844,103.35351562)(114.44902344,103.35351562)
\curveto(113.66777344,103.35351562)(113.02910156,103.57226562)(112.53300781,104.00976562)
\curveto(112.04082031,104.44726562)(111.75957031,105.02734375)(111.68925781,105.75)
\closepath
}
}
{
\newrgbcolor{curcolor}{0 0 0}
\pscustom[linestyle=none,fillstyle=solid,fillcolor=curcolor]
{
\newpath
\moveto(118.36308594,107.73632812)
\curveto(118.36308594,108.75195312)(118.46660156,109.56835938)(118.67363281,110.18554688)
\curveto(118.88457031,110.80664062)(119.19511719,111.28515625)(119.60527344,111.62109375)
\curveto(120.01933594,111.95703125)(120.53886719,112.125)(121.16386719,112.125)
\curveto(121.62480469,112.125)(122.02910156,112.03125)(122.37675781,111.84375)
\curveto(122.72441406,111.66015625)(123.01152344,111.39257812)(123.23808594,111.04101562)
\curveto(123.46464844,110.69335938)(123.64238281,110.26757812)(123.77128906,109.76367188)
\curveto(123.90019531,109.26367188)(123.96464844,108.58789062)(123.96464844,107.73632812)
\curveto(123.96464844,106.72851562)(123.86113281,105.9140625)(123.65410156,105.29296875)
\curveto(123.44707031,104.67578125)(123.13652344,104.19726562)(122.72246094,103.85742188)
\curveto(122.31230469,103.52148438)(121.79277344,103.35351562)(121.16386719,103.35351562)
\curveto(120.33574219,103.35351562)(119.68535156,103.65039062)(119.21269531,104.24414062)
\curveto(118.64628906,104.95898438)(118.36308594,106.12304688)(118.36308594,107.73632812)
\closepath
\moveto(119.44707031,107.73632812)
\curveto(119.44707031,106.32617188)(119.61113281,105.38671875)(119.93925781,104.91796875)
\curveto(120.27128906,104.453125)(120.67949219,104.22070312)(121.16386719,104.22070312)
\curveto(121.64824219,104.22070312)(122.05449219,104.45507812)(122.38261719,104.92382812)
\curveto(122.71464844,105.39257812)(122.88066406,106.33007812)(122.88066406,107.73632812)
\curveto(122.88066406,109.15039062)(122.71464844,110.08984375)(122.38261719,110.5546875)
\curveto(122.05449219,111.01953125)(121.64433594,111.25195312)(121.15214844,111.25195312)
\curveto(120.66777344,111.25195312)(120.28105469,111.046875)(119.99199219,110.63671875)
\curveto(119.62871094,110.11328125)(119.44707031,109.14648438)(119.44707031,107.73632812)
\closepath
}
}
{
\newrgbcolor{curcolor}{0 0 0}
\pscustom[linestyle=none,fillstyle=solid,fillcolor=curcolor]
{
\newpath
\moveto(125.03691406,107.73632812)
\curveto(125.03691406,108.75195312)(125.14042969,109.56835938)(125.34746094,110.18554688)
\curveto(125.55839844,110.80664062)(125.86894531,111.28515625)(126.27910156,111.62109375)
\curveto(126.69316406,111.95703125)(127.21269531,112.125)(127.83769531,112.125)
\curveto(128.29863281,112.125)(128.70292969,112.03125)(129.05058594,111.84375)
\curveto(129.39824219,111.66015625)(129.68535156,111.39257812)(129.91191406,111.04101562)
\curveto(130.13847656,110.69335938)(130.31621094,110.26757812)(130.44511719,109.76367188)
\curveto(130.57402344,109.26367188)(130.63847656,108.58789062)(130.63847656,107.73632812)
\curveto(130.63847656,106.72851562)(130.53496094,105.9140625)(130.32792969,105.29296875)
\curveto(130.12089844,104.67578125)(129.81035156,104.19726562)(129.39628906,103.85742188)
\curveto(128.98613281,103.52148438)(128.46660156,103.35351562)(127.83769531,103.35351562)
\curveto(127.00957031,103.35351562)(126.35917969,103.65039062)(125.88652344,104.24414062)
\curveto(125.32011719,104.95898438)(125.03691406,106.12304688)(125.03691406,107.73632812)
\closepath
\moveto(126.12089844,107.73632812)
\curveto(126.12089844,106.32617188)(126.28496094,105.38671875)(126.61308594,104.91796875)
\curveto(126.94511719,104.453125)(127.35332031,104.22070312)(127.83769531,104.22070312)
\curveto(128.32207031,104.22070312)(128.72832031,104.45507812)(129.05644531,104.92382812)
\curveto(129.38847656,105.39257812)(129.55449219,106.33007812)(129.55449219,107.73632812)
\curveto(129.55449219,109.15039062)(129.38847656,110.08984375)(129.05644531,110.5546875)
\curveto(128.72832031,111.01953125)(128.31816406,111.25195312)(127.82597656,111.25195312)
\curveto(127.34160156,111.25195312)(126.95488281,111.046875)(126.66582031,110.63671875)
\curveto(126.30253906,110.11328125)(126.12089844,109.14648438)(126.12089844,107.73632812)
\closepath
}
}
{
\newrgbcolor{curcolor}{0 0 0}
\pscustom[linestyle=none,fillstyle=solid,fillcolor=curcolor]
{
\newpath
\moveto(132.10332031,103.5)
\lineto(132.10332031,112.08984375)
\lineto(133.81425781,112.08984375)
\lineto(135.84746094,106.0078125)
\curveto(136.03496094,105.44140625)(136.17167969,105.01757812)(136.25761719,104.73632812)
\curveto(136.35527344,105.04882812)(136.50761719,105.5078125)(136.71464844,106.11328125)
\lineto(138.77128906,112.08984375)
\lineto(140.30058594,112.08984375)
\lineto(140.30058594,103.5)
\lineto(139.20488281,103.5)
\lineto(139.20488281,110.68945312)
\lineto(136.70878906,103.5)
\lineto(135.68339844,103.5)
\lineto(133.19902344,110.8125)
\lineto(133.19902344,103.5)
\closepath
}
}
{
\newrgbcolor{curcolor}{0 0 0}
\pscustom[linewidth=1,linecolor=curcolor]
{
\newpath
\moveto(176.4,104.7)
\lineto(181.7,108)
}
}
{
\newrgbcolor{curcolor}{0 0 0}
\pscustom[linewidth=1,linecolor=curcolor]
{
\newpath
\moveto(333.1,201.1)
\lineto(327.7,197.8)
}
}
{
\newrgbcolor{curcolor}{0 0 0}
\pscustom[linestyle=none,fillstyle=solid,fillcolor=curcolor]
{
\newpath
\moveto(167.96679688,94.2)
\lineto(166.91210938,94.2)
\lineto(166.91210938,100.92070312)
\curveto(166.65820312,100.67851562)(166.32421875,100.43632812)(165.91015625,100.19414062)
\curveto(165.5,99.95195312)(165.13085938,99.7703125)(164.80273438,99.64921875)
\lineto(164.80273438,100.66875)
\curveto(165.39257812,100.94609375)(165.90820312,101.28203125)(166.34960938,101.6765625)
\curveto(166.79101562,102.07109375)(167.10351562,102.45390625)(167.28710938,102.825)
\lineto(167.96679688,102.825)
\closepath
}
}
{
\newrgbcolor{curcolor}{0 0 0}
\pscustom[linestyle=none,fillstyle=solid,fillcolor=curcolor]
{
\newpath
\moveto(175.11523438,97.56914062)
\lineto(175.11523438,98.57695312)
\lineto(178.75390625,98.5828125)
\lineto(178.75390625,95.3953125)
\curveto(178.1953125,94.95)(177.61914062,94.6140625)(177.02539062,94.3875)
\curveto(176.43164062,94.16484375)(175.82226562,94.05351562)(175.19726562,94.05351562)
\curveto(174.35351562,94.05351562)(173.5859375,94.23320312)(172.89453125,94.59257812)
\curveto(172.20703125,94.95585937)(171.6875,95.47929687)(171.3359375,96.16289062)
\curveto(170.984375,96.84648437)(170.80859375,97.61015625)(170.80859375,98.45390625)
\curveto(170.80859375,99.28984375)(170.98242188,100.06914062)(171.33007812,100.79179687)
\curveto(171.68164062,101.51835937)(172.18554688,102.05742187)(172.84179688,102.40898437)
\curveto(173.49804688,102.76054687)(174.25390625,102.93632812)(175.109375,102.93632812)
\curveto(175.73046875,102.93632812)(176.29101562,102.83476562)(176.79101562,102.63164062)
\curveto(177.29492188,102.43242187)(177.68945312,102.153125)(177.97460938,101.79375)
\curveto(178.25976562,101.434375)(178.4765625,100.965625)(178.625,100.3875)
\lineto(177.59960938,100.10625)
\curveto(177.47070312,100.54375)(177.31054688,100.8875)(177.11914062,101.1375)
\curveto(176.92773438,101.3875)(176.65429688,101.58671875)(176.29882812,101.73515625)
\curveto(175.94335938,101.8875)(175.54882812,101.96367187)(175.11523438,101.96367187)
\curveto(174.59570312,101.96367187)(174.14648438,101.88359375)(173.76757812,101.7234375)
\curveto(173.38867188,101.5671875)(173.08203125,101.36015625)(172.84765625,101.10234375)
\curveto(172.6171875,100.84453125)(172.4375,100.56132812)(172.30859375,100.25273437)
\curveto(172.08984375,99.72148437)(171.98046875,99.1453125)(171.98046875,98.52421875)
\curveto(171.98046875,97.75859375)(172.11132812,97.11796875)(172.37304688,96.60234375)
\curveto(172.63867188,96.08671875)(173.0234375,95.70390625)(173.52734375,95.45390625)
\curveto(174.03125,95.20390625)(174.56640625,95.07890625)(175.1328125,95.07890625)
\curveto(175.625,95.07890625)(176.10546875,95.17265625)(176.57421875,95.36015625)
\curveto(177.04296875,95.5515625)(177.3984375,95.7546875)(177.640625,95.96953125)
\lineto(177.640625,97.56914062)
\closepath
}
}
{
\newrgbcolor{curcolor}{0 0 0}
\pscustom[linewidth=1,linecolor=curcolor]
{
\newpath
\moveto(221.7,95.4)
\lineto(227,98.7)
}
}
{
\newrgbcolor{curcolor}{0 0 0}
\pscustom[linewidth=1,linecolor=curcolor]
{
\newpath
\moveto(378.3,191.8)
\lineto(373,188.5)
}
}
{
\newrgbcolor{curcolor}{0 0 0}
\pscustom[linestyle=none,fillstyle=solid,fillcolor=curcolor]
{
\newpath
\moveto(214.73710937,85.91367187)
\lineto(214.73710937,84.9)
\lineto(209.059375,84.9)
\curveto(209.0515625,85.15390625)(209.09257812,85.39804687)(209.18242187,85.63242187)
\curveto(209.32695312,86.01914062)(209.55742187,86.4)(209.87382812,86.775)
\curveto(210.19414062,87.15)(210.65507812,87.58359375)(211.25664062,88.07578125)
\curveto(212.19023437,88.84140625)(212.82109375,89.446875)(213.14921875,89.8921875)
\curveto(213.47734375,90.34140625)(213.64140625,90.76523437)(213.64140625,91.16367187)
\curveto(213.64140625,91.58164062)(213.49101562,91.93320312)(213.19023437,92.21835937)
\curveto(212.89335937,92.50742187)(212.5046875,92.65195312)(212.02421875,92.65195312)
\curveto(211.51640625,92.65195312)(211.11015625,92.49960937)(210.80546875,92.19492187)
\curveto(210.50078125,91.89023437)(210.34648437,91.46835937)(210.34257812,90.92929687)
\lineto(209.25859375,91.040625)
\curveto(209.3328125,91.84921875)(209.61210937,92.46445312)(210.09648437,92.88632812)
\curveto(210.58085937,93.31210937)(211.23125,93.525)(212.04765625,93.525)
\curveto(212.871875,93.525)(213.52421875,93.29648437)(214.0046875,92.83945312)
\curveto(214.48515625,92.38242187)(214.72539062,91.81601562)(214.72539062,91.14023437)
\curveto(214.72539062,90.79648437)(214.65507812,90.45859375)(214.51445312,90.1265625)
\curveto(214.37382812,89.79453125)(214.13945312,89.44492187)(213.81132812,89.07773437)
\curveto(213.48710937,88.71054687)(212.94609375,88.20664062)(212.18828125,87.56601562)
\curveto(211.55546875,87.03476562)(211.14921875,86.6734375)(210.96953125,86.48203125)
\curveto(210.78984375,86.29453125)(210.64140625,86.10507812)(210.52421875,85.91367187)
\closepath
}
}
{
\newrgbcolor{curcolor}{0 0 0}
\pscustom[linestyle=none,fillstyle=solid,fillcolor=curcolor]
{
\newpath
\moveto(220.31523437,88.26914062)
\lineto(220.31523437,89.27695312)
\lineto(223.95390625,89.2828125)
\lineto(223.95390625,86.0953125)
\curveto(223.3953125,85.65)(222.81914062,85.3140625)(222.22539062,85.0875)
\curveto(221.63164062,84.86484375)(221.02226562,84.75351562)(220.39726562,84.75351562)
\curveto(219.55351562,84.75351562)(218.7859375,84.93320312)(218.09453125,85.29257812)
\curveto(217.40703125,85.65585937)(216.8875,86.17929687)(216.5359375,86.86289062)
\curveto(216.184375,87.54648437)(216.00859375,88.31015625)(216.00859375,89.15390625)
\curveto(216.00859375,89.98984375)(216.18242187,90.76914062)(216.53007812,91.49179687)
\curveto(216.88164062,92.21835937)(217.38554687,92.75742187)(218.04179687,93.10898437)
\curveto(218.69804687,93.46054687)(219.45390625,93.63632812)(220.309375,93.63632812)
\curveto(220.93046875,93.63632812)(221.49101562,93.53476562)(221.99101562,93.33164062)
\curveto(222.49492187,93.13242187)(222.88945312,92.853125)(223.17460937,92.49375)
\curveto(223.45976562,92.134375)(223.6765625,91.665625)(223.825,91.0875)
\lineto(222.79960937,90.80625)
\curveto(222.67070312,91.24375)(222.51054687,91.5875)(222.31914062,91.8375)
\curveto(222.12773437,92.0875)(221.85429687,92.28671875)(221.49882812,92.43515625)
\curveto(221.14335937,92.5875)(220.74882812,92.66367187)(220.31523437,92.66367187)
\curveto(219.79570312,92.66367187)(219.34648437,92.58359375)(218.96757812,92.4234375)
\curveto(218.58867187,92.2671875)(218.28203125,92.06015625)(218.04765625,91.80234375)
\curveto(217.8171875,91.54453125)(217.6375,91.26132812)(217.50859375,90.95273437)
\curveto(217.28984375,90.42148437)(217.18046875,89.8453125)(217.18046875,89.22421875)
\curveto(217.18046875,88.45859375)(217.31132812,87.81796875)(217.57304687,87.30234375)
\curveto(217.83867187,86.78671875)(218.2234375,86.40390625)(218.72734375,86.15390625)
\curveto(219.23125,85.90390625)(219.76640625,85.77890625)(220.3328125,85.77890625)
\curveto(220.825,85.77890625)(221.30546875,85.87265625)(221.77421875,86.06015625)
\curveto(222.24296875,86.2515625)(222.5984375,86.4546875)(222.840625,86.66953125)
\lineto(222.840625,88.26914062)
\closepath
}
}
{
\newrgbcolor{curcolor}{0 0 0}
\pscustom[linewidth=1,linecolor=curcolor]
{
\newpath
\moveto(266.9,86.1)
\lineto(272.3,89.4)
}
}
{
\newrgbcolor{curcolor}{0 0 0}
\pscustom[linewidth=1,linecolor=curcolor]
{
\newpath
\moveto(423.6,182.6)
\lineto(418.3,179.3)
}
}
{
\newrgbcolor{curcolor}{0 0 0}
\pscustom[linestyle=none,fillstyle=solid,fillcolor=curcolor]
{
\newpath
\moveto(257.875,75.7)
\lineto(257.875,77.75664062)
\lineto(254.1484375,77.75664062)
\lineto(254.1484375,78.7234375)
\lineto(258.06835938,84.28984375)
\lineto(258.9296875,84.28984375)
\lineto(258.9296875,78.7234375)
\lineto(260.08984375,78.7234375)
\lineto(260.08984375,77.75664062)
\lineto(258.9296875,77.75664062)
\lineto(258.9296875,75.7)
\closepath
\moveto(257.875,78.7234375)
\lineto(257.875,82.59648437)
\lineto(255.18554688,78.7234375)
\closepath
}
}
{
\newrgbcolor{curcolor}{0 0 0}
\pscustom[linestyle=none,fillstyle=solid,fillcolor=curcolor]
{
\newpath
\moveto(265.61523438,79.06914062)
\lineto(265.61523438,80.07695312)
\lineto(269.25390625,80.0828125)
\lineto(269.25390625,76.8953125)
\curveto(268.6953125,76.45)(268.11914062,76.1140625)(267.52539062,75.8875)
\curveto(266.93164062,75.66484375)(266.32226562,75.55351562)(265.69726562,75.55351562)
\curveto(264.85351562,75.55351562)(264.0859375,75.73320312)(263.39453125,76.09257812)
\curveto(262.70703125,76.45585937)(262.1875,76.97929687)(261.8359375,77.66289062)
\curveto(261.484375,78.34648437)(261.30859375,79.11015625)(261.30859375,79.95390625)
\curveto(261.30859375,80.78984375)(261.48242188,81.56914062)(261.83007812,82.29179687)
\curveto(262.18164062,83.01835937)(262.68554688,83.55742187)(263.34179688,83.90898437)
\curveto(263.99804688,84.26054687)(264.75390625,84.43632812)(265.609375,84.43632812)
\curveto(266.23046875,84.43632812)(266.79101562,84.33476562)(267.29101562,84.13164062)
\curveto(267.79492188,83.93242187)(268.18945312,83.653125)(268.47460938,83.29375)
\curveto(268.75976562,82.934375)(268.9765625,82.465625)(269.125,81.8875)
\lineto(268.09960938,81.60625)
\curveto(267.97070312,82.04375)(267.81054688,82.3875)(267.61914062,82.6375)
\curveto(267.42773438,82.8875)(267.15429688,83.08671875)(266.79882812,83.23515625)
\curveto(266.44335938,83.3875)(266.04882812,83.46367187)(265.61523438,83.46367187)
\curveto(265.09570312,83.46367187)(264.64648438,83.38359375)(264.26757812,83.2234375)
\curveto(263.88867188,83.0671875)(263.58203125,82.86015625)(263.34765625,82.60234375)
\curveto(263.1171875,82.34453125)(262.9375,82.06132812)(262.80859375,81.75273437)
\curveto(262.58984375,81.22148437)(262.48046875,80.6453125)(262.48046875,80.02421875)
\curveto(262.48046875,79.25859375)(262.61132812,78.61796875)(262.87304688,78.10234375)
\curveto(263.13867188,77.58671875)(263.5234375,77.20390625)(264.02734375,76.95390625)
\curveto(264.53125,76.70390625)(265.06640625,76.57890625)(265.6328125,76.57890625)
\curveto(266.125,76.57890625)(266.60546875,76.67265625)(267.07421875,76.86015625)
\curveto(267.54296875,77.0515625)(267.8984375,77.2546875)(268.140625,77.46953125)
\lineto(268.140625,79.06914062)
\closepath
}
}
{
\newrgbcolor{curcolor}{0 0 0}
\pscustom[linewidth=1,linecolor=curcolor]
{
\newpath
\moveto(312.1,76.8)
\lineto(317.4,80.1)
}
}
{
\newrgbcolor{curcolor}{0 0 0}
\pscustom[linewidth=1,linecolor=curcolor]
{
\newpath
\moveto(468.9,173.3)
\lineto(463.5,170)
}
}
{
\newrgbcolor{curcolor}{0 0 0}
\pscustom[linestyle=none,fillstyle=solid,fillcolor=curcolor]
{
\newpath
\moveto(301.3171875,71.05820312)
\curveto(300.8796875,71.21835937)(300.55546875,71.446875)(300.34453125,71.74375)
\curveto(300.13359375,72.040625)(300.028125,72.39609375)(300.028125,72.81015625)
\curveto(300.028125,73.43515625)(300.25273437,73.96054687)(300.70195312,74.38632812)
\curveto(301.15117187,74.81210937)(301.74882812,75.025)(302.49492187,75.025)
\curveto(303.24492187,75.025)(303.8484375,74.80625)(304.30546875,74.36875)
\curveto(304.7625,73.93515625)(304.99101562,73.40585937)(304.99101562,72.78085937)
\curveto(304.99101562,72.38242187)(304.88554687,72.03476562)(304.67460937,71.73789062)
\curveto(304.46757812,71.44492187)(304.15117187,71.21835937)(303.72539062,71.05820312)
\curveto(304.25273437,70.88632812)(304.653125,70.60898437)(304.9265625,70.22617187)
\curveto(305.20390625,69.84335937)(305.34257812,69.38632812)(305.34257812,68.85507812)
\curveto(305.34257812,68.12070312)(305.0828125,67.50351562)(304.56328125,67.00351562)
\curveto(304.04375,66.50351562)(303.36015625,66.25351562)(302.5125,66.25351562)
\curveto(301.66484375,66.25351562)(300.98125,66.50351562)(300.46171875,67.00351562)
\curveto(299.9421875,67.50742187)(299.68242187,68.134375)(299.68242187,68.884375)
\curveto(299.68242187,69.44296875)(299.82304687,69.90976562)(300.10429687,70.28476562)
\curveto(300.38945312,70.66367187)(300.79375,70.92148437)(301.3171875,71.05820312)
\closepath
\moveto(301.10625,72.8453125)
\curveto(301.10625,72.4390625)(301.23710937,72.10703125)(301.49882812,71.84921875)
\curveto(301.76054687,71.59140625)(302.10039062,71.4625)(302.51835937,71.4625)
\curveto(302.92460937,71.4625)(303.25664062,71.58945312)(303.51445312,71.84335937)
\curveto(303.77617187,72.10117187)(303.90703125,72.415625)(303.90703125,72.78671875)
\curveto(303.90703125,73.1734375)(303.77226562,73.49765625)(303.50273437,73.759375)
\curveto(303.23710937,74.025)(302.90507812,74.1578125)(302.50664062,74.1578125)
\curveto(302.10429687,74.1578125)(301.7703125,74.02890625)(301.5046875,73.77109375)
\curveto(301.2390625,73.51328125)(301.10625,73.2046875)(301.10625,72.8453125)
\closepath
\moveto(300.76640625,68.87851562)
\curveto(300.76640625,68.57773437)(300.83671875,68.28671875)(300.97734375,68.00546875)
\curveto(301.121875,67.72421875)(301.33476562,67.50546875)(301.61601562,67.34921875)
\curveto(301.89726562,67.196875)(302.2,67.12070312)(302.52421875,67.12070312)
\curveto(303.028125,67.12070312)(303.44414062,67.2828125)(303.77226562,67.60703125)
\curveto(304.10039062,67.93125)(304.26445312,68.34335937)(304.26445312,68.84335937)
\curveto(304.26445312,69.35117187)(304.09453125,69.77109375)(303.7546875,70.103125)
\curveto(303.41875,70.43515625)(302.996875,70.60117187)(302.4890625,70.60117187)
\curveto(301.99296875,70.60117187)(301.58085937,70.43710937)(301.25273437,70.10898437)
\curveto(300.92851562,69.78085937)(300.76640625,69.37070312)(300.76640625,68.87851562)
\closepath
}
}
{
\newrgbcolor{curcolor}{0 0 0}
\pscustom[linestyle=none,fillstyle=solid,fillcolor=curcolor]
{
\newpath
\moveto(310.81523437,69.76914062)
\lineto(310.81523437,70.77695312)
\lineto(314.45390625,70.7828125)
\lineto(314.45390625,67.5953125)
\curveto(313.8953125,67.15)(313.31914062,66.8140625)(312.72539062,66.5875)
\curveto(312.13164062,66.36484375)(311.52226562,66.25351562)(310.89726562,66.25351562)
\curveto(310.05351562,66.25351562)(309.2859375,66.43320312)(308.59453125,66.79257812)
\curveto(307.90703125,67.15585937)(307.3875,67.67929687)(307.0359375,68.36289062)
\curveto(306.684375,69.04648437)(306.50859375,69.81015625)(306.50859375,70.65390625)
\curveto(306.50859375,71.48984375)(306.68242187,72.26914062)(307.03007812,72.99179687)
\curveto(307.38164062,73.71835937)(307.88554687,74.25742187)(308.54179687,74.60898437)
\curveto(309.19804687,74.96054687)(309.95390625,75.13632812)(310.809375,75.13632812)
\curveto(311.43046875,75.13632812)(311.99101562,75.03476562)(312.49101562,74.83164062)
\curveto(312.99492187,74.63242187)(313.38945312,74.353125)(313.67460937,73.99375)
\curveto(313.95976562,73.634375)(314.1765625,73.165625)(314.325,72.5875)
\lineto(313.29960937,72.30625)
\curveto(313.17070312,72.74375)(313.01054687,73.0875)(312.81914062,73.3375)
\curveto(312.62773437,73.5875)(312.35429687,73.78671875)(311.99882812,73.93515625)
\curveto(311.64335937,74.0875)(311.24882812,74.16367187)(310.81523437,74.16367187)
\curveto(310.29570312,74.16367187)(309.84648437,74.08359375)(309.46757812,73.9234375)
\curveto(309.08867187,73.7671875)(308.78203125,73.56015625)(308.54765625,73.30234375)
\curveto(308.3171875,73.04453125)(308.1375,72.76132812)(308.00859375,72.45273437)
\curveto(307.78984375,71.92148437)(307.68046875,71.3453125)(307.68046875,70.72421875)
\curveto(307.68046875,69.95859375)(307.81132812,69.31796875)(308.07304687,68.80234375)
\curveto(308.33867187,68.28671875)(308.7234375,67.90390625)(309.22734375,67.65390625)
\curveto(309.73125,67.40390625)(310.26640625,67.27890625)(310.8328125,67.27890625)
\curveto(311.325,67.27890625)(311.80546875,67.37265625)(312.27421875,67.56015625)
\curveto(312.74296875,67.7515625)(313.0984375,67.9546875)(313.340625,68.16953125)
\lineto(313.340625,69.76914062)
\closepath
}
}
{
\newrgbcolor{curcolor}{0 0 0}
\pscustom[linewidth=1,linecolor=curcolor]
{
\newpath
\moveto(357.3,67.6)
\lineto(362.7,70.8)
}
}
{
\newrgbcolor{curcolor}{0 0 0}
\pscustom[linewidth=1,linecolor=curcolor]
{
\newpath
\moveto(514.1,164)
\lineto(508.8,160.7)
}
}
{
\newrgbcolor{curcolor}{0 0 0}
\pscustom[linestyle=none,fillstyle=solid,fillcolor=curcolor]
{
\newpath
\moveto(345.52988281,57.1)
\lineto(344.47519531,57.1)
\lineto(344.47519531,63.82070313)
\curveto(344.22128906,63.57851563)(343.88730469,63.33632813)(343.47324219,63.09414063)
\curveto(343.06308594,62.85195313)(342.69394531,62.6703125)(342.36582031,62.54921875)
\lineto(342.36582031,63.56875)
\curveto(342.95566406,63.84609375)(343.47128906,64.18203125)(343.91269531,64.5765625)
\curveto(344.35410156,64.97109375)(344.66660156,65.35390625)(344.85019531,65.725)
\lineto(345.52988281,65.725)
\closepath
}
}
{
\newrgbcolor{curcolor}{0 0 0}
\pscustom[linestyle=none,fillstyle=solid,fillcolor=curcolor]
{
\newpath
\moveto(353.70371094,63.58632813)
\lineto(352.65488281,63.50429688)
\curveto(352.56113281,63.91835938)(352.42832031,64.21914063)(352.25644531,64.40664063)
\curveto(351.97128906,64.70742188)(351.61972656,64.8578125)(351.20175781,64.8578125)
\curveto(350.86582031,64.8578125)(350.57089844,64.7640625)(350.31699219,64.5765625)
\curveto(349.98496094,64.334375)(349.72324219,63.98085938)(349.53183594,63.51601563)
\curveto(349.34042969,63.05117188)(349.24082031,62.3890625)(349.23300781,61.5296875)
\curveto(349.48691406,61.91640625)(349.79746094,62.20351563)(350.16464844,62.39101563)
\curveto(350.53183594,62.57851563)(350.91660156,62.67226563)(351.31894531,62.67226563)
\curveto(352.02207031,62.67226563)(352.61972656,62.4125)(353.11191406,61.89296875)
\curveto(353.60800781,61.37734375)(353.85605469,60.709375)(353.85605469,59.8890625)
\curveto(353.85605469,59.35)(353.73886719,58.84804688)(353.50449219,58.38320313)
\curveto(353.27402344,57.92226563)(352.95566406,57.56875)(352.54941406,57.32265625)
\curveto(352.14316406,57.0765625)(351.68222656,56.95351563)(351.16660156,56.95351563)
\curveto(350.28769531,56.95351563)(349.57089844,57.27578125)(349.01621094,57.9203125)
\curveto(348.46152344,58.56875)(348.18417969,59.63515625)(348.18417969,61.11953125)
\curveto(348.18417969,62.7796875)(348.49082031,63.98671875)(349.10410156,64.740625)
\curveto(349.63925781,65.396875)(350.35996094,65.725)(351.26621094,65.725)
\curveto(351.94199219,65.725)(352.49472656,65.53554688)(352.92441406,65.15664063)
\curveto(353.35800781,64.77773438)(353.61777344,64.25429688)(353.70371094,63.58632813)
\closepath
\moveto(349.39707031,59.88320313)
\curveto(349.39707031,59.51992188)(349.47324219,59.17226563)(349.62558594,58.84023438)
\curveto(349.78183594,58.50820313)(349.99863281,58.25429688)(350.27597656,58.07851563)
\curveto(350.55332031,57.90664063)(350.84433594,57.82070313)(351.14902344,57.82070313)
\curveto(351.59433594,57.82070313)(351.97714844,58.00039063)(352.29746094,58.35976563)
\curveto(352.61777344,58.71914063)(352.77792969,59.20742188)(352.77792969,59.82460938)
\curveto(352.77792969,60.41835938)(352.61972656,60.88515625)(352.30332031,61.225)
\curveto(351.98691406,61.56875)(351.58847656,61.740625)(351.10800781,61.740625)
\curveto(350.63144531,61.740625)(350.22714844,61.56875)(349.89511719,61.225)
\curveto(349.56308594,60.88515625)(349.39707031,60.43789063)(349.39707031,59.88320313)
\closepath
}
}
{
\newrgbcolor{curcolor}{0 0 0}
\pscustom[linestyle=none,fillstyle=solid,fillcolor=curcolor]
{
\newpath
\moveto(359.35214844,60.46914063)
\lineto(359.35214844,61.47695313)
\lineto(362.99082031,61.4828125)
\lineto(362.99082031,58.2953125)
\curveto(362.43222656,57.85)(361.85605469,57.5140625)(361.26230469,57.2875)
\curveto(360.66855469,57.06484375)(360.05917969,56.95351563)(359.43417969,56.95351563)
\curveto(358.59042969,56.95351563)(357.82285156,57.13320313)(357.13144531,57.49257813)
\curveto(356.44394531,57.85585938)(355.92441406,58.37929688)(355.57285156,59.06289063)
\curveto(355.22128906,59.74648438)(355.04550781,60.51015625)(355.04550781,61.35390625)
\curveto(355.04550781,62.18984375)(355.21933594,62.96914063)(355.56699219,63.69179688)
\curveto(355.91855469,64.41835938)(356.42246094,64.95742188)(357.07871094,65.30898438)
\curveto(357.73496094,65.66054688)(358.49082031,65.83632813)(359.34628906,65.83632813)
\curveto(359.96738281,65.83632813)(360.52792969,65.73476563)(361.02792969,65.53164063)
\curveto(361.53183594,65.33242188)(361.92636719,65.053125)(362.21152344,64.69375)
\curveto(362.49667969,64.334375)(362.71347656,63.865625)(362.86191406,63.2875)
\lineto(361.83652344,63.00625)
\curveto(361.70761719,63.44375)(361.54746094,63.7875)(361.35605469,64.0375)
\curveto(361.16464844,64.2875)(360.89121094,64.48671875)(360.53574219,64.63515625)
\curveto(360.18027344,64.7875)(359.78574219,64.86367188)(359.35214844,64.86367188)
\curveto(358.83261719,64.86367188)(358.38339844,64.78359375)(358.00449219,64.6234375)
\curveto(357.62558594,64.4671875)(357.31894531,64.26015625)(357.08457031,64.00234375)
\curveto(356.85410156,63.74453125)(356.67441406,63.46132813)(356.54550781,63.15273438)
\curveto(356.32675781,62.62148438)(356.21738281,62.0453125)(356.21738281,61.42421875)
\curveto(356.21738281,60.65859375)(356.34824219,60.01796875)(356.60996094,59.50234375)
\curveto(356.87558594,58.98671875)(357.26035156,58.60390625)(357.76425781,58.35390625)
\curveto(358.26816406,58.10390625)(358.80332031,57.97890625)(359.36972656,57.97890625)
\curveto(359.86191406,57.97890625)(360.34238281,58.07265625)(360.81113281,58.26015625)
\curveto(361.27988281,58.4515625)(361.63535156,58.6546875)(361.87753906,58.86953125)
\lineto(361.87753906,60.46914063)
\closepath
}
}
{
\newrgbcolor{curcolor}{0 0 0}
\pscustom[linewidth=1,linecolor=curcolor]
{
\newpath
\moveto(357.3,67.6)
\lineto(348.1,69.5)
}
}
{
\newrgbcolor{curcolor}{0 0 0}
\pscustom[linewidth=1,linecolor=curcolor]
{
\newpath
\moveto(85.9,123.2)
\lineto(95.1,121.3)
}
}
{
\newrgbcolor{curcolor}{0 0 0}
\pscustom[linestyle=none,fillstyle=solid,fillcolor=curcolor]
{
\newpath
\moveto(370.37070312,59.9)
\lineto(369.31601562,59.9)
\lineto(369.31601562,66.62070312)
\curveto(369.06210937,66.37851562)(368.728125,66.13632812)(368.3140625,65.89414062)
\curveto(367.90390625,65.65195312)(367.53476562,65.4703125)(367.20664062,65.34921875)
\lineto(367.20664062,66.36875)
\curveto(367.79648437,66.64609375)(368.31210937,66.98203125)(368.75351562,67.3765625)
\curveto(369.19492187,67.77109375)(369.50742187,68.15390625)(369.69101562,68.525)
\lineto(370.37070312,68.525)
\closepath
}
}
{
\newrgbcolor{curcolor}{0 0 0}
\pscustom[linestyle=none,fillstyle=solid,fillcolor=curcolor]
{
\newpath
\moveto(373.46445312,59.9)
\lineto(373.46445312,68.48984375)
\lineto(375.17539062,68.48984375)
\lineto(377.20859375,62.4078125)
\curveto(377.39609375,61.84140625)(377.5328125,61.41757812)(377.61875,61.13632812)
\curveto(377.71640625,61.44882812)(377.86875,61.9078125)(378.07578125,62.51328125)
\lineto(380.13242187,68.48984375)
\lineto(381.66171875,68.48984375)
\lineto(381.66171875,59.9)
\lineto(380.56601562,59.9)
\lineto(380.56601562,67.08945312)
\lineto(378.06992187,59.9)
\lineto(377.04453125,59.9)
\lineto(374.56015625,67.2125)
\lineto(374.56015625,59.9)
\closepath
}
}
{
\newrgbcolor{curcolor}{0 0 0}
\pscustom[linewidth=1,linecolor=curcolor]
{
\newpath
\moveto(373,77.2)
\lineto(363.8,79.1)
}
}
{
\newrgbcolor{curcolor}{0 0 0}
\pscustom[linewidth=1,linecolor=curcolor]
{
\newpath
\moveto(101.5,132.9)
\lineto(110.8,131)
}
}
{
\newrgbcolor{curcolor}{0 0 0}
\pscustom[linestyle=none,fillstyle=solid,fillcolor=curcolor]
{
\newpath
\moveto(387.64101563,70.51367188)
\lineto(387.64101563,69.5)
\lineto(381.96328125,69.5)
\curveto(381.95546875,69.75390625)(381.99648438,69.99804688)(382.08632813,70.23242188)
\curveto(382.23085938,70.61914062)(382.46132813,71)(382.77773438,71.375)
\curveto(383.09804688,71.75)(383.55898438,72.18359375)(384.16054688,72.67578125)
\curveto(385.09414063,73.44140625)(385.725,74.046875)(386.053125,74.4921875)
\curveto(386.38125,74.94140625)(386.5453125,75.36523438)(386.5453125,75.76367188)
\curveto(386.5453125,76.18164062)(386.39492188,76.53320312)(386.09414063,76.81835938)
\curveto(385.79726563,77.10742188)(385.40859375,77.25195312)(384.928125,77.25195312)
\curveto(384.4203125,77.25195312)(384.0140625,77.09960938)(383.709375,76.79492188)
\curveto(383.4046875,76.49023438)(383.25039063,76.06835938)(383.24648438,75.52929688)
\lineto(382.1625,75.640625)
\curveto(382.23671875,76.44921875)(382.51601563,77.06445312)(383.00039063,77.48632812)
\curveto(383.48476563,77.91210938)(384.13515625,78.125)(384.9515625,78.125)
\curveto(385.77578125,78.125)(386.428125,77.89648438)(386.90859375,77.43945312)
\curveto(387.3890625,76.98242188)(387.62929688,76.41601562)(387.62929688,75.74023438)
\curveto(387.62929688,75.39648438)(387.55898438,75.05859375)(387.41835938,74.7265625)
\curveto(387.27773438,74.39453125)(387.04335938,74.04492188)(386.71523438,73.67773438)
\curveto(386.39101563,73.31054688)(385.85,72.80664062)(385.0921875,72.16601562)
\curveto(384.459375,71.63476562)(384.053125,71.2734375)(383.8734375,71.08203125)
\curveto(383.69375,70.89453125)(383.5453125,70.70507812)(383.428125,70.51367188)
\closepath
}
}
{
\newrgbcolor{curcolor}{0 0 0}
\pscustom[linestyle=none,fillstyle=solid,fillcolor=curcolor]
{
\newpath
\moveto(389.16445313,69.5)
\lineto(389.16445313,78.08984375)
\lineto(390.87539063,78.08984375)
\lineto(392.90859375,72.0078125)
\curveto(393.09609375,71.44140625)(393.2328125,71.01757812)(393.31875,70.73632812)
\curveto(393.41640625,71.04882812)(393.56875,71.5078125)(393.77578125,72.11328125)
\lineto(395.83242188,78.08984375)
\lineto(397.36171875,78.08984375)
\lineto(397.36171875,69.5)
\lineto(396.26601563,69.5)
\lineto(396.26601563,76.68945312)
\lineto(393.76992188,69.5)
\lineto(392.74453125,69.5)
\lineto(390.26015625,76.8125)
\lineto(390.26015625,69.5)
\closepath
}
}
{
\newrgbcolor{curcolor}{0 0 0}
\pscustom[linewidth=1,linecolor=curcolor]
{
\newpath
\moveto(388.7,86.9)
\lineto(379.5,88.7)
}
}
{
\newrgbcolor{curcolor}{0 0 0}
\pscustom[linewidth=1,linecolor=curcolor]
{
\newpath
\moveto(117.2,142.5)
\lineto(126.5,140.6)
}
}
{
\newrgbcolor{curcolor}{0 0 0}
\pscustom[linestyle=none,fillstyle=solid,fillcolor=curcolor]
{
\newpath
\moveto(401.07890625,79.2)
\lineto(401.07890625,81.25664062)
\lineto(397.35234375,81.25664062)
\lineto(397.35234375,82.2234375)
\lineto(401.27226562,87.78984375)
\lineto(402.13359375,87.78984375)
\lineto(402.13359375,82.2234375)
\lineto(403.29375,82.2234375)
\lineto(403.29375,81.25664062)
\lineto(402.13359375,81.25664062)
\lineto(402.13359375,79.2)
\closepath
\moveto(401.07890625,82.2234375)
\lineto(401.07890625,86.09648437)
\lineto(398.38945312,82.2234375)
\closepath
}
}
{
\newrgbcolor{curcolor}{0 0 0}
\pscustom[linestyle=none,fillstyle=solid,fillcolor=curcolor]
{
\newpath
\moveto(404.76445312,79.2)
\lineto(404.76445312,87.78984375)
\lineto(406.47539062,87.78984375)
\lineto(408.50859375,81.7078125)
\curveto(408.69609375,81.14140625)(408.8328125,80.71757812)(408.91875,80.43632812)
\curveto(409.01640625,80.74882812)(409.16875,81.2078125)(409.37578125,81.81328125)
\lineto(411.43242187,87.78984375)
\lineto(412.96171875,87.78984375)
\lineto(412.96171875,79.2)
\lineto(411.86601562,79.2)
\lineto(411.86601562,86.38945312)
\lineto(409.36992187,79.2)
\lineto(408.34453125,79.2)
\lineto(405.86015625,86.5125)
\lineto(405.86015625,79.2)
\closepath
}
}
{
\newrgbcolor{curcolor}{0 0 0}
\pscustom[linewidth=1,linecolor=curcolor]
{
\newpath
\moveto(404.4,96.5)
\lineto(395.1,98.4)
}
}
{
\newrgbcolor{curcolor}{0 0 0}
\pscustom[linewidth=1,linecolor=curcolor]
{
\newpath
\moveto(132.9,152.2)
\lineto(142.1,150.3)
}
}
{
\newrgbcolor{curcolor}{0 0 0}
\pscustom[linestyle=none,fillstyle=solid,fillcolor=curcolor]
{
\newpath
\moveto(415.02109375,93.45820313)
\curveto(414.58359375,93.61835938)(414.259375,93.846875)(414.0484375,94.14375)
\curveto(413.8375,94.440625)(413.73203125,94.79609375)(413.73203125,95.21015625)
\curveto(413.73203125,95.83515625)(413.95664062,96.36054688)(414.40585937,96.78632813)
\curveto(414.85507812,97.21210938)(415.45273437,97.425)(416.19882812,97.425)
\curveto(416.94882812,97.425)(417.55234375,97.20625)(418.009375,96.76875)
\curveto(418.46640625,96.33515625)(418.69492187,95.80585938)(418.69492187,95.18085938)
\curveto(418.69492187,94.78242188)(418.58945312,94.43476563)(418.37851562,94.13789063)
\curveto(418.17148437,93.84492188)(417.85507812,93.61835938)(417.42929687,93.45820313)
\curveto(417.95664062,93.28632813)(418.35703125,93.00898438)(418.63046875,92.62617188)
\curveto(418.9078125,92.24335938)(419.04648437,91.78632813)(419.04648437,91.25507813)
\curveto(419.04648437,90.52070313)(418.78671875,89.90351563)(418.2671875,89.40351563)
\curveto(417.74765625,88.90351563)(417.0640625,88.65351563)(416.21640625,88.65351563)
\curveto(415.36875,88.65351563)(414.68515625,88.90351563)(414.165625,89.40351563)
\curveto(413.64609375,89.90742188)(413.38632812,90.534375)(413.38632812,91.284375)
\curveto(413.38632812,91.84296875)(413.52695312,92.30976563)(413.80820312,92.68476563)
\curveto(414.09335937,93.06367188)(414.49765625,93.32148438)(415.02109375,93.45820313)
\closepath
\moveto(414.81015625,95.2453125)
\curveto(414.81015625,94.8390625)(414.94101562,94.50703125)(415.20273437,94.24921875)
\curveto(415.46445312,93.99140625)(415.80429687,93.8625)(416.22226562,93.8625)
\curveto(416.62851562,93.8625)(416.96054687,93.98945313)(417.21835937,94.24335938)
\curveto(417.48007812,94.50117188)(417.6109375,94.815625)(417.6109375,95.18671875)
\curveto(417.6109375,95.5734375)(417.47617187,95.89765625)(417.20664062,96.159375)
\curveto(416.94101562,96.425)(416.60898437,96.5578125)(416.21054687,96.5578125)
\curveto(415.80820312,96.5578125)(415.47421875,96.42890625)(415.20859375,96.17109375)
\curveto(414.94296875,95.91328125)(414.81015625,95.6046875)(414.81015625,95.2453125)
\closepath
\moveto(414.4703125,91.27851563)
\curveto(414.4703125,90.97773438)(414.540625,90.68671875)(414.68125,90.40546875)
\curveto(414.82578125,90.12421875)(415.03867187,89.90546875)(415.31992187,89.74921875)
\curveto(415.60117187,89.596875)(415.90390625,89.52070313)(416.228125,89.52070313)
\curveto(416.73203125,89.52070313)(417.14804687,89.6828125)(417.47617187,90.00703125)
\curveto(417.80429687,90.33125)(417.96835937,90.74335938)(417.96835937,91.24335938)
\curveto(417.96835937,91.75117188)(417.7984375,92.17109375)(417.45859375,92.503125)
\curveto(417.12265625,92.83515625)(416.70078125,93.00117188)(416.19296875,93.00117188)
\curveto(415.696875,93.00117188)(415.28476562,92.83710938)(414.95664062,92.50898438)
\curveto(414.63242187,92.18085938)(414.4703125,91.77070313)(414.4703125,91.27851563)
\closepath
}
}
{
\newrgbcolor{curcolor}{0 0 0}
\pscustom[linestyle=none,fillstyle=solid,fillcolor=curcolor]
{
\newpath
\moveto(420.46445312,88.8)
\lineto(420.46445312,97.38984375)
\lineto(422.17539062,97.38984375)
\lineto(424.20859375,91.3078125)
\curveto(424.39609375,90.74140625)(424.5328125,90.31757813)(424.61875,90.03632813)
\curveto(424.71640625,90.34882813)(424.86875,90.8078125)(425.07578125,91.41328125)
\lineto(427.13242187,97.38984375)
\lineto(428.66171875,97.38984375)
\lineto(428.66171875,88.8)
\lineto(427.56601562,88.8)
\lineto(427.56601562,95.98945313)
\lineto(425.06992187,88.8)
\lineto(424.04453125,88.8)
\lineto(421.56015625,96.1125)
\lineto(421.56015625,88.8)
\closepath
}
}
{
\newrgbcolor{curcolor}{0 0 0}
\pscustom[linewidth=1,linecolor=curcolor]
{
\newpath
\moveto(420.1,106.1)
\lineto(410.8,108)
}
}
{
\newrgbcolor{curcolor}{0 0 0}
\pscustom[linewidth=1,linecolor=curcolor]
{
\newpath
\moveto(148.6,161.8)
\lineto(157.8,159.9)
}
}
{
\newrgbcolor{curcolor}{0 0 0}
\pscustom[linestyle=none,fillstyle=solid,fillcolor=curcolor]
{
\newpath
\moveto(433.07070313,98.4)
\lineto(432.01601563,98.4)
\lineto(432.01601563,105.12070312)
\curveto(431.76210938,104.87851562)(431.428125,104.63632812)(431.0140625,104.39414062)
\curveto(430.60390625,104.15195312)(430.23476563,103.9703125)(429.90664063,103.84921875)
\lineto(429.90664063,104.86875)
\curveto(430.49648438,105.14609375)(431.01210938,105.48203125)(431.45351563,105.8765625)
\curveto(431.89492188,106.27109375)(432.20742188,106.65390625)(432.39101563,107.025)
\lineto(433.07070313,107.025)
\closepath
}
}
{
\newrgbcolor{curcolor}{0 0 0}
\pscustom[linestyle=none,fillstyle=solid,fillcolor=curcolor]
{
\newpath
\moveto(441.24453125,104.88632812)
\lineto(440.19570313,104.80429687)
\curveto(440.10195313,105.21835937)(439.96914063,105.51914062)(439.79726563,105.70664062)
\curveto(439.51210938,106.00742187)(439.16054688,106.1578125)(438.74257813,106.1578125)
\curveto(438.40664063,106.1578125)(438.11171875,106.0640625)(437.8578125,105.8765625)
\curveto(437.52578125,105.634375)(437.2640625,105.28085937)(437.07265625,104.81601562)
\curveto(436.88125,104.35117187)(436.78164063,103.6890625)(436.77382813,102.8296875)
\curveto(437.02773438,103.21640625)(437.33828125,103.50351562)(437.70546875,103.69101562)
\curveto(438.07265625,103.87851562)(438.45742188,103.97226562)(438.85976563,103.97226562)
\curveto(439.56289063,103.97226562)(440.16054688,103.7125)(440.65273438,103.19296875)
\curveto(441.14882813,102.67734375)(441.396875,102.009375)(441.396875,101.1890625)
\curveto(441.396875,100.65)(441.2796875,100.14804687)(441.0453125,99.68320312)
\curveto(440.81484375,99.22226562)(440.49648438,98.86875)(440.09023438,98.62265625)
\curveto(439.68398438,98.3765625)(439.22304688,98.25351562)(438.70742188,98.25351562)
\curveto(437.82851563,98.25351562)(437.11171875,98.57578125)(436.55703125,99.2203125)
\curveto(436.00234375,99.86875)(435.725,100.93515625)(435.725,102.41953125)
\curveto(435.725,104.0796875)(436.03164063,105.28671875)(436.64492188,106.040625)
\curveto(437.18007813,106.696875)(437.90078125,107.025)(438.80703125,107.025)
\curveto(439.4828125,107.025)(440.03554688,106.83554687)(440.46523438,106.45664062)
\curveto(440.89882813,106.07773437)(441.15859375,105.55429687)(441.24453125,104.88632812)
\closepath
\moveto(436.93789063,101.18320312)
\curveto(436.93789063,100.81992187)(437.0140625,100.47226562)(437.16640625,100.14023437)
\curveto(437.32265625,99.80820312)(437.53945313,99.55429687)(437.81679688,99.37851562)
\curveto(438.09414063,99.20664062)(438.38515625,99.12070312)(438.68984375,99.12070312)
\curveto(439.13515625,99.12070312)(439.51796875,99.30039062)(439.83828125,99.65976562)
\curveto(440.15859375,100.01914062)(440.31875,100.50742187)(440.31875,101.12460937)
\curveto(440.31875,101.71835937)(440.16054688,102.18515625)(439.84414063,102.525)
\curveto(439.52773438,102.86875)(439.12929688,103.040625)(438.64882813,103.040625)
\curveto(438.17226563,103.040625)(437.76796875,102.86875)(437.4359375,102.525)
\curveto(437.10390625,102.18515625)(436.93789063,101.73789062)(436.93789063,101.18320312)
\closepath
}
}
{
\newrgbcolor{curcolor}{0 0 0}
\pscustom[linestyle=none,fillstyle=solid,fillcolor=curcolor]
{
\newpath
\moveto(442.83828125,98.4)
\lineto(442.83828125,106.98984375)
\lineto(444.54921875,106.98984375)
\lineto(446.58242188,100.9078125)
\curveto(446.76992188,100.34140625)(446.90664063,99.91757812)(446.99257813,99.63632812)
\curveto(447.09023438,99.94882812)(447.24257813,100.4078125)(447.44960938,101.01328125)
\lineto(449.50625,106.98984375)
\lineto(451.03554688,106.98984375)
\lineto(451.03554688,98.4)
\lineto(449.93984375,98.4)
\lineto(449.93984375,105.58945312)
\lineto(447.44375,98.4)
\lineto(446.41835938,98.4)
\lineto(443.93398438,105.7125)
\lineto(443.93398438,98.4)
\closepath
}
}
{
\newrgbcolor{curcolor}{0 0 0}
\pscustom[linewidth=1,linecolor=curcolor]
{
\newpath
\moveto(435.7,115.8)
\lineto(426.5,117.7)
}
}
{
\newrgbcolor{curcolor}{0 0 0}
\pscustom[linewidth=1,linecolor=curcolor]
{
\newpath
\moveto(164.3,171.5)
\lineto(173.5,169.6)
}
}
{
\newrgbcolor{curcolor}{0 0 0}
\pscustom[linestyle=none,fillstyle=solid,fillcolor=curcolor]
{
\newpath
\moveto(444.80390625,110.36757813)
\lineto(445.85859375,110.50820313)
\curveto(445.9796875,109.91054688)(446.18476563,109.47890625)(446.47382813,109.21328125)
\curveto(446.76679688,108.9515625)(447.12226563,108.82070313)(447.54023438,108.82070313)
\curveto(448.03632813,108.82070313)(448.45429688,108.99257813)(448.79414063,109.33632813)
\curveto(449.13789063,109.68007813)(449.30976563,110.10585938)(449.30976563,110.61367188)
\curveto(449.30976563,111.09804688)(449.1515625,111.49648438)(448.83515625,111.80898438)
\curveto(448.51875,112.12539063)(448.11640625,112.28359375)(447.628125,112.28359375)
\curveto(447.42890625,112.28359375)(447.18085938,112.24453125)(446.88398438,112.16640625)
\lineto(447.00117188,113.0921875)
\curveto(447.07148438,113.084375)(447.128125,113.08046875)(447.17109375,113.08046875)
\curveto(447.6203125,113.08046875)(448.02460938,113.19765625)(448.38398438,113.43203125)
\curveto(448.74335938,113.66640625)(448.92304688,114.02773438)(448.92304688,114.51601563)
\curveto(448.92304688,114.90273438)(448.7921875,115.22304688)(448.53046875,115.47695313)
\curveto(448.26875,115.73085938)(447.93085938,115.8578125)(447.51679688,115.8578125)
\curveto(447.10664063,115.8578125)(446.76484375,115.72890625)(446.49140625,115.47109375)
\curveto(446.21796875,115.21328125)(446.0421875,114.8265625)(445.9640625,114.3109375)
\lineto(444.909375,114.4984375)
\curveto(445.03828125,115.20546875)(445.33125,115.75234375)(445.78828125,116.1390625)
\curveto(446.2453125,116.5296875)(446.81367188,116.725)(447.49335938,116.725)
\curveto(447.96210938,116.725)(448.39375,116.6234375)(448.78828125,116.4203125)
\curveto(449.1828125,116.22109375)(449.48359375,115.94765625)(449.690625,115.6)
\curveto(449.9015625,115.25234375)(450.00703125,114.88320313)(450.00703125,114.49257813)
\curveto(450.00703125,114.12148438)(449.90742188,113.78359375)(449.70820313,113.47890625)
\curveto(449.50898438,113.17421875)(449.2140625,112.93203125)(448.8234375,112.75234375)
\curveto(449.33125,112.63515625)(449.72578125,112.39101563)(450.00703125,112.01992188)
\curveto(450.28828125,111.65273438)(450.42890625,111.19179688)(450.42890625,110.63710938)
\curveto(450.42890625,109.88710938)(450.15546875,109.25039063)(449.60859375,108.72695313)
\curveto(449.06171875,108.20742188)(448.3703125,107.94765625)(447.534375,107.94765625)
\curveto(446.78046875,107.94765625)(446.15351563,108.17226563)(445.65351563,108.62148438)
\curveto(445.15742188,109.07070313)(444.87421875,109.65273438)(444.80390625,110.36757813)
\closepath
}
}
{
\newrgbcolor{curcolor}{0 0 0}
\pscustom[linestyle=none,fillstyle=solid,fillcolor=curcolor]
{
\newpath
\moveto(457.01484375,109.11367188)
\lineto(457.01484375,108.1)
\lineto(451.33710938,108.1)
\curveto(451.32929688,108.35390625)(451.3703125,108.59804688)(451.46015625,108.83242188)
\curveto(451.6046875,109.21914063)(451.83515625,109.6)(452.1515625,109.975)
\curveto(452.471875,110.35)(452.9328125,110.78359375)(453.534375,111.27578125)
\curveto(454.46796875,112.04140625)(455.09882813,112.646875)(455.42695313,113.0921875)
\curveto(455.75507813,113.54140625)(455.91914063,113.96523438)(455.91914063,114.36367188)
\curveto(455.91914063,114.78164063)(455.76875,115.13320313)(455.46796875,115.41835938)
\curveto(455.17109375,115.70742188)(454.78242188,115.85195313)(454.30195313,115.85195313)
\curveto(453.79414063,115.85195313)(453.38789063,115.69960938)(453.08320313,115.39492188)
\curveto(452.77851563,115.09023438)(452.62421875,114.66835938)(452.6203125,114.12929688)
\lineto(451.53632813,114.240625)
\curveto(451.61054688,115.04921875)(451.88984375,115.66445313)(452.37421875,116.08632813)
\curveto(452.85859375,116.51210938)(453.50898438,116.725)(454.32539063,116.725)
\curveto(455.14960938,116.725)(455.80195313,116.49648438)(456.28242188,116.03945313)
\curveto(456.76289063,115.58242188)(457.003125,115.01601563)(457.003125,114.34023438)
\curveto(457.003125,113.99648438)(456.9328125,113.65859375)(456.7921875,113.3265625)
\curveto(456.6515625,112.99453125)(456.4171875,112.64492188)(456.0890625,112.27773438)
\curveto(455.76484375,111.91054688)(455.22382813,111.40664063)(454.46601563,110.76601563)
\curveto(453.83320313,110.23476563)(453.42695313,109.8734375)(453.24726563,109.68203125)
\curveto(453.06757813,109.49453125)(452.91914063,109.30507813)(452.80195313,109.11367188)
\closepath
}
}
{
\newrgbcolor{curcolor}{0 0 0}
\pscustom[linestyle=none,fillstyle=solid,fillcolor=curcolor]
{
\newpath
\moveto(458.53828125,108.1)
\lineto(458.53828125,116.68984375)
\lineto(460.24921875,116.68984375)
\lineto(462.28242188,110.6078125)
\curveto(462.46992188,110.04140625)(462.60664063,109.61757813)(462.69257813,109.33632813)
\curveto(462.79023438,109.64882813)(462.94257813,110.1078125)(463.14960938,110.71328125)
\lineto(465.20625,116.68984375)
\lineto(466.73554688,116.68984375)
\lineto(466.73554688,108.1)
\lineto(465.63984375,108.1)
\lineto(465.63984375,115.28945313)
\lineto(463.14375,108.1)
\lineto(462.11835938,108.1)
\lineto(459.63398438,115.4125)
\lineto(459.63398438,108.1)
\closepath
}
}
{
\newrgbcolor{curcolor}{0 0 0}
\pscustom[linewidth=1,linecolor=curcolor]
{
\newpath
\moveto(451.4,125.4)
\lineto(442.2,127.3)
}
}
{
\newrgbcolor{curcolor}{0 0 0}
\pscustom[linewidth=1,linecolor=curcolor]
{
\newpath
\moveto(179.9,181.1)
\lineto(189.2,179.2)
}
}
{
\newrgbcolor{curcolor}{0 0 0}
\pscustom[linestyle=none,fillstyle=solid,fillcolor=curcolor]
{
\newpath
\moveto(465.97070312,124.18632812)
\lineto(464.921875,124.10429687)
\curveto(464.828125,124.51835937)(464.6953125,124.81914062)(464.5234375,125.00664062)
\curveto(464.23828125,125.30742187)(463.88671875,125.4578125)(463.46875,125.4578125)
\curveto(463.1328125,125.4578125)(462.83789062,125.3640625)(462.58398438,125.1765625)
\curveto(462.25195312,124.934375)(461.99023438,124.58085937)(461.79882812,124.11601562)
\curveto(461.60742188,123.65117187)(461.5078125,122.9890625)(461.5,122.1296875)
\curveto(461.75390625,122.51640625)(462.06445312,122.80351562)(462.43164062,122.99101562)
\curveto(462.79882812,123.17851562)(463.18359375,123.27226562)(463.5859375,123.27226562)
\curveto(464.2890625,123.27226562)(464.88671875,123.0125)(465.37890625,122.49296875)
\curveto(465.875,121.97734375)(466.12304688,121.309375)(466.12304688,120.4890625)
\curveto(466.12304688,119.95)(466.00585938,119.44804687)(465.77148438,118.98320312)
\curveto(465.54101562,118.52226562)(465.22265625,118.16875)(464.81640625,117.92265625)
\curveto(464.41015625,117.6765625)(463.94921875,117.55351562)(463.43359375,117.55351562)
\curveto(462.5546875,117.55351562)(461.83789062,117.87578125)(461.28320312,118.5203125)
\curveto(460.72851562,119.16875)(460.45117188,120.23515625)(460.45117188,121.71953125)
\curveto(460.45117188,123.3796875)(460.7578125,124.58671875)(461.37109375,125.340625)
\curveto(461.90625,125.996875)(462.62695312,126.325)(463.53320312,126.325)
\curveto(464.20898438,126.325)(464.76171875,126.13554687)(465.19140625,125.75664062)
\curveto(465.625,125.37773437)(465.88476562,124.85429687)(465.97070312,124.18632812)
\closepath
\moveto(461.6640625,120.48320312)
\curveto(461.6640625,120.11992187)(461.74023438,119.77226562)(461.89257812,119.44023437)
\curveto(462.04882812,119.10820312)(462.265625,118.85429687)(462.54296875,118.67851562)
\curveto(462.8203125,118.50664062)(463.11132812,118.42070312)(463.41601562,118.42070312)
\curveto(463.86132812,118.42070312)(464.24414062,118.60039062)(464.56445312,118.95976562)
\curveto(464.88476562,119.31914062)(465.04492188,119.80742187)(465.04492188,120.42460937)
\curveto(465.04492188,121.01835937)(464.88671875,121.48515625)(464.5703125,121.825)
\curveto(464.25390625,122.16875)(463.85546875,122.340625)(463.375,122.340625)
\curveto(462.8984375,122.340625)(462.49414062,122.16875)(462.16210938,121.825)
\curveto(461.83007812,121.48515625)(461.6640625,121.03789062)(461.6640625,120.48320312)
\closepath
}
}
{
\newrgbcolor{curcolor}{0 0 0}
\pscustom[linestyle=none,fillstyle=solid,fillcolor=curcolor]
{
\newpath
\moveto(470.55273438,117.7)
\lineto(470.55273438,119.75664062)
\lineto(466.82617188,119.75664062)
\lineto(466.82617188,120.7234375)
\lineto(470.74609375,126.28984375)
\lineto(471.60742188,126.28984375)
\lineto(471.60742188,120.7234375)
\lineto(472.76757812,120.7234375)
\lineto(472.76757812,119.75664062)
\lineto(471.60742188,119.75664062)
\lineto(471.60742188,117.7)
\closepath
\moveto(470.55273438,120.7234375)
\lineto(470.55273438,124.59648437)
\lineto(467.86328125,120.7234375)
\closepath
}
}
{
\newrgbcolor{curcolor}{0 0 0}
\pscustom[linestyle=none,fillstyle=solid,fillcolor=curcolor]
{
\newpath
\moveto(474.23828125,117.7)
\lineto(474.23828125,126.28984375)
\lineto(475.94921875,126.28984375)
\lineto(477.98242188,120.2078125)
\curveto(478.16992188,119.64140625)(478.30664062,119.21757812)(478.39257812,118.93632812)
\curveto(478.49023438,119.24882812)(478.64257812,119.7078125)(478.84960938,120.31328125)
\lineto(480.90625,126.28984375)
\lineto(482.43554688,126.28984375)
\lineto(482.43554688,117.7)
\lineto(481.33984375,117.7)
\lineto(481.33984375,124.88945312)
\lineto(478.84375,117.7)
\lineto(477.81835938,117.7)
\lineto(475.33398438,125.0125)
\lineto(475.33398438,117.7)
\closepath
}
}
{
\newrgbcolor{curcolor}{0 0 0}
\pscustom[linewidth=1,linecolor=curcolor]
{
\newpath
\moveto(467.1,135.1)
\lineto(457.9,137)
}
}
{
\newrgbcolor{curcolor}{0 0 0}
\pscustom[linewidth=1,linecolor=curcolor]
{
\newpath
\moveto(195.6,190.7)
\lineto(204.9,188.8)
}
}
{
\newrgbcolor{curcolor}{0 0 0}
\pscustom[linestyle=none,fillstyle=solid,fillcolor=curcolor]
{
\newpath
\moveto(480.07070313,127.4)
\lineto(479.01601563,127.4)
\lineto(479.01601563,134.12070312)
\curveto(478.76210938,133.87851562)(478.428125,133.63632812)(478.0140625,133.39414062)
\curveto(477.60390625,133.15195312)(477.23476563,132.9703125)(476.90664063,132.84921875)
\lineto(476.90664063,133.86875)
\curveto(477.49648438,134.14609375)(478.01210938,134.48203125)(478.45351563,134.8765625)
\curveto(478.89492188,135.27109375)(479.20742188,135.65390625)(479.39101563,136.025)
\lineto(480.07070313,136.025)
\closepath
}
}
{
\newrgbcolor{curcolor}{0 0 0}
\pscustom[linestyle=none,fillstyle=solid,fillcolor=curcolor]
{
\newpath
\moveto(488.31484375,128.41367187)
\lineto(488.31484375,127.4)
\lineto(482.63710938,127.4)
\curveto(482.62929688,127.65390625)(482.6703125,127.89804687)(482.76015625,128.13242187)
\curveto(482.9046875,128.51914062)(483.13515625,128.9)(483.4515625,129.275)
\curveto(483.771875,129.65)(484.2328125,130.08359375)(484.834375,130.57578125)
\curveto(485.76796875,131.34140625)(486.39882813,131.946875)(486.72695313,132.3921875)
\curveto(487.05507813,132.84140625)(487.21914063,133.26523437)(487.21914063,133.66367187)
\curveto(487.21914063,134.08164062)(487.06875,134.43320312)(486.76796875,134.71835937)
\curveto(486.47109375,135.00742187)(486.08242188,135.15195312)(485.60195313,135.15195312)
\curveto(485.09414063,135.15195312)(484.68789063,134.99960937)(484.38320313,134.69492187)
\curveto(484.07851563,134.39023437)(483.92421875,133.96835937)(483.9203125,133.42929687)
\lineto(482.83632813,133.540625)
\curveto(482.91054688,134.34921875)(483.18984375,134.96445312)(483.67421875,135.38632812)
\curveto(484.15859375,135.81210937)(484.80898438,136.025)(485.62539063,136.025)
\curveto(486.44960938,136.025)(487.10195313,135.79648437)(487.58242188,135.33945312)
\curveto(488.06289063,134.88242187)(488.303125,134.31601562)(488.303125,133.64023437)
\curveto(488.303125,133.29648437)(488.2328125,132.95859375)(488.0921875,132.6265625)
\curveto(487.9515625,132.29453125)(487.7171875,131.94492187)(487.3890625,131.57773437)
\curveto(487.06484375,131.21054687)(486.52382813,130.70664062)(485.76601563,130.06601562)
\curveto(485.13320313,129.53476562)(484.72695313,129.1734375)(484.54726563,128.98203125)
\curveto(484.36757813,128.79453125)(484.21914063,128.60507812)(484.10195313,128.41367187)
\closepath
}
}
{
\newrgbcolor{curcolor}{0 0 0}
\pscustom[linestyle=none,fillstyle=solid,fillcolor=curcolor]
{
\newpath
\moveto(491.06875,132.05820312)
\curveto(490.63125,132.21835937)(490.30703125,132.446875)(490.09609375,132.74375)
\curveto(489.88515625,133.040625)(489.7796875,133.39609375)(489.7796875,133.81015625)
\curveto(489.7796875,134.43515625)(490.00429688,134.96054687)(490.45351563,135.38632812)
\curveto(490.90273438,135.81210937)(491.50039063,136.025)(492.24648438,136.025)
\curveto(492.99648438,136.025)(493.6,135.80625)(494.05703125,135.36875)
\curveto(494.5140625,134.93515625)(494.74257813,134.40585937)(494.74257813,133.78085937)
\curveto(494.74257813,133.38242187)(494.63710938,133.03476562)(494.42617188,132.73789062)
\curveto(494.21914063,132.44492187)(493.90273438,132.21835937)(493.47695313,132.05820312)
\curveto(494.00429688,131.88632812)(494.4046875,131.60898437)(494.678125,131.22617187)
\curveto(494.95546875,130.84335937)(495.09414063,130.38632812)(495.09414063,129.85507812)
\curveto(495.09414063,129.12070312)(494.834375,128.50351562)(494.31484375,128.00351562)
\curveto(493.7953125,127.50351562)(493.11171875,127.25351562)(492.2640625,127.25351562)
\curveto(491.41640625,127.25351562)(490.7328125,127.50351562)(490.21328125,128.00351562)
\curveto(489.69375,128.50742187)(489.43398438,129.134375)(489.43398438,129.884375)
\curveto(489.43398438,130.44296875)(489.57460938,130.90976562)(489.85585938,131.28476562)
\curveto(490.14101563,131.66367187)(490.5453125,131.92148437)(491.06875,132.05820312)
\closepath
\moveto(490.8578125,133.8453125)
\curveto(490.8578125,133.4390625)(490.98867188,133.10703125)(491.25039063,132.84921875)
\curveto(491.51210938,132.59140625)(491.85195313,132.4625)(492.26992188,132.4625)
\curveto(492.67617188,132.4625)(493.00820313,132.58945312)(493.26601563,132.84335937)
\curveto(493.52773438,133.10117187)(493.65859375,133.415625)(493.65859375,133.78671875)
\curveto(493.65859375,134.1734375)(493.52382813,134.49765625)(493.25429688,134.759375)
\curveto(492.98867188,135.025)(492.65664063,135.1578125)(492.25820313,135.1578125)
\curveto(491.85585938,135.1578125)(491.521875,135.02890625)(491.25625,134.77109375)
\curveto(490.990625,134.51328125)(490.8578125,134.2046875)(490.8578125,133.8453125)
\closepath
\moveto(490.51796875,129.87851562)
\curveto(490.51796875,129.57773437)(490.58828125,129.28671875)(490.72890625,129.00546875)
\curveto(490.8734375,128.72421875)(491.08632813,128.50546875)(491.36757813,128.34921875)
\curveto(491.64882813,128.196875)(491.9515625,128.12070312)(492.27578125,128.12070312)
\curveto(492.7796875,128.12070312)(493.19570313,128.2828125)(493.52382813,128.60703125)
\curveto(493.85195313,128.93125)(494.01601563,129.34335937)(494.01601563,129.84335937)
\curveto(494.01601563,130.35117187)(493.84609375,130.77109375)(493.50625,131.103125)
\curveto(493.1703125,131.43515625)(492.7484375,131.60117187)(492.240625,131.60117187)
\curveto(491.74453125,131.60117187)(491.33242188,131.43710937)(491.00429688,131.10898437)
\curveto(490.68007813,130.78085937)(490.51796875,130.37070312)(490.51796875,129.87851562)
\closepath
}
}
{
\newrgbcolor{curcolor}{0 0 0}
\pscustom[linestyle=none,fillstyle=solid,fillcolor=curcolor]
{
\newpath
\moveto(496.51210938,127.4)
\lineto(496.51210938,135.98984375)
\lineto(498.22304688,135.98984375)
\lineto(500.25625,129.9078125)
\curveto(500.44375,129.34140625)(500.58046875,128.91757812)(500.66640625,128.63632812)
\curveto(500.7640625,128.94882812)(500.91640625,129.4078125)(501.1234375,130.01328125)
\lineto(503.18007813,135.98984375)
\lineto(504.709375,135.98984375)
\lineto(504.709375,127.4)
\lineto(503.61367188,127.4)
\lineto(503.61367188,134.58945312)
\lineto(501.11757813,127.4)
\lineto(500.0921875,127.4)
\lineto(497.6078125,134.7125)
\lineto(497.6078125,127.4)
\closepath
}
}
{
\newrgbcolor{curcolor}{0 0 0}
\pscustom[linewidth=1,linecolor=curcolor]
{
\newpath
\moveto(482.8,144.7)
\lineto(473.5,146.6)
}
}
{
\newrgbcolor{curcolor}{0 0 0}
\pscustom[linewidth=1,linecolor=curcolor]
{
\newpath
\moveto(211.3,200.4)
\lineto(220.5,198.5)
}
}
{
\newrgbcolor{curcolor}{0 0 0}
\pscustom[linestyle=none,fillstyle=solid,fillcolor=curcolor]
{
\newpath
\moveto(497.34101563,138.01367188)
\lineto(497.34101563,137)
\lineto(491.66328125,137)
\curveto(491.65546875,137.25390625)(491.69648438,137.49804688)(491.78632813,137.73242188)
\curveto(491.93085938,138.11914062)(492.16132813,138.5)(492.47773438,138.875)
\curveto(492.79804688,139.25)(493.25898438,139.68359375)(493.86054688,140.17578125)
\curveto(494.79414063,140.94140625)(495.425,141.546875)(495.753125,141.9921875)
\curveto(496.08125,142.44140625)(496.2453125,142.86523438)(496.2453125,143.26367188)
\curveto(496.2453125,143.68164062)(496.09492188,144.03320312)(495.79414063,144.31835938)
\curveto(495.49726563,144.60742188)(495.10859375,144.75195312)(494.628125,144.75195312)
\curveto(494.1203125,144.75195312)(493.7140625,144.59960938)(493.409375,144.29492188)
\curveto(493.1046875,143.99023438)(492.95039063,143.56835938)(492.94648438,143.02929688)
\lineto(491.8625,143.140625)
\curveto(491.93671875,143.94921875)(492.21601563,144.56445312)(492.70039063,144.98632812)
\curveto(493.18476563,145.41210938)(493.83515625,145.625)(494.6515625,145.625)
\curveto(495.47578125,145.625)(496.128125,145.39648438)(496.60859375,144.93945312)
\curveto(497.0890625,144.48242188)(497.32929688,143.91601562)(497.32929688,143.24023438)
\curveto(497.32929688,142.89648438)(497.25898438,142.55859375)(497.11835938,142.2265625)
\curveto(496.97773438,141.89453125)(496.74335938,141.54492188)(496.41523438,141.17773438)
\curveto(496.09101563,140.81054688)(495.55,140.30664062)(494.7921875,139.66601562)
\curveto(494.159375,139.13476562)(493.753125,138.7734375)(493.5734375,138.58203125)
\curveto(493.39375,138.39453125)(493.2453125,138.20507812)(493.128125,138.01367188)
\closepath
}
}
{
\newrgbcolor{curcolor}{0 0 0}
\pscustom[linestyle=none,fillstyle=solid,fillcolor=curcolor]
{
\newpath
\moveto(498.471875,139.25)
\lineto(499.57929688,139.34375)
\curveto(499.66132813,138.8046875)(499.85078125,138.3984375)(500.14765625,138.125)
\curveto(500.4484375,137.85546875)(500.80976563,137.72070312)(501.23164063,137.72070312)
\curveto(501.73945313,137.72070312)(502.16914063,137.91210938)(502.52070313,138.29492188)
\curveto(502.87226563,138.67773438)(503.04804688,139.18554688)(503.04804688,139.81835938)
\curveto(503.04804688,140.41992188)(502.878125,140.89453125)(502.53828125,141.2421875)
\curveto(502.20234375,141.58984375)(501.7609375,141.76367188)(501.2140625,141.76367188)
\curveto(500.87421875,141.76367188)(500.56757813,141.68554688)(500.29414063,141.52929688)
\curveto(500.02070313,141.37695312)(499.80585938,141.17773438)(499.64960938,140.93164062)
\lineto(498.659375,141.06054688)
\lineto(499.49140625,145.47265625)
\lineto(503.76289063,145.47265625)
\lineto(503.76289063,144.46484375)
\lineto(500.33515625,144.46484375)
\lineto(499.87226563,142.15625)
\curveto(500.38789063,142.515625)(500.92890625,142.6953125)(501.4953125,142.6953125)
\curveto(502.2453125,142.6953125)(502.878125,142.43554688)(503.39375,141.91601562)
\curveto(503.909375,141.39648438)(504.1671875,140.72851562)(504.1671875,139.91210938)
\curveto(504.1671875,139.13476562)(503.940625,138.46289062)(503.4875,137.89648438)
\curveto(502.93671875,137.20117188)(502.18476563,136.85351562)(501.23164063,136.85351562)
\curveto(500.45039063,136.85351562)(499.81171875,137.07226562)(499.315625,137.50976562)
\curveto(498.8234375,137.94726562)(498.5421875,138.52734375)(498.471875,139.25)
\closepath
}
}
{
\newrgbcolor{curcolor}{0 0 0}
\pscustom[linestyle=none,fillstyle=solid,fillcolor=curcolor]
{
\newpath
\moveto(510.61835938,143.48632812)
\lineto(509.56953125,143.40429688)
\curveto(509.47578125,143.81835938)(509.34296875,144.11914062)(509.17109375,144.30664062)
\curveto(508.8859375,144.60742188)(508.534375,144.7578125)(508.11640625,144.7578125)
\curveto(507.78046875,144.7578125)(507.48554688,144.6640625)(507.23164063,144.4765625)
\curveto(506.89960938,144.234375)(506.63789063,143.88085938)(506.44648438,143.41601562)
\curveto(506.25507813,142.95117188)(506.15546875,142.2890625)(506.14765625,141.4296875)
\curveto(506.4015625,141.81640625)(506.71210938,142.10351562)(507.07929688,142.29101562)
\curveto(507.44648438,142.47851562)(507.83125,142.57226562)(508.23359375,142.57226562)
\curveto(508.93671875,142.57226562)(509.534375,142.3125)(510.0265625,141.79296875)
\curveto(510.52265625,141.27734375)(510.77070313,140.609375)(510.77070313,139.7890625)
\curveto(510.77070313,139.25)(510.65351563,138.74804688)(510.41914063,138.28320312)
\curveto(510.18867188,137.82226562)(509.8703125,137.46875)(509.4640625,137.22265625)
\curveto(509.0578125,136.9765625)(508.596875,136.85351562)(508.08125,136.85351562)
\curveto(507.20234375,136.85351562)(506.48554688,137.17578125)(505.93085938,137.8203125)
\curveto(505.37617188,138.46875)(505.09882813,139.53515625)(505.09882813,141.01953125)
\curveto(505.09882813,142.6796875)(505.40546875,143.88671875)(506.01875,144.640625)
\curveto(506.55390625,145.296875)(507.27460938,145.625)(508.18085938,145.625)
\curveto(508.85664063,145.625)(509.409375,145.43554688)(509.8390625,145.05664062)
\curveto(510.27265625,144.67773438)(510.53242188,144.15429688)(510.61835938,143.48632812)
\closepath
\moveto(506.31171875,139.78320312)
\curveto(506.31171875,139.41992188)(506.38789063,139.07226562)(506.54023438,138.74023438)
\curveto(506.69648438,138.40820312)(506.91328125,138.15429688)(507.190625,137.97851562)
\curveto(507.46796875,137.80664062)(507.75898438,137.72070312)(508.06367188,137.72070312)
\curveto(508.50898438,137.72070312)(508.89179688,137.90039062)(509.21210938,138.25976562)
\curveto(509.53242188,138.61914062)(509.69257813,139.10742188)(509.69257813,139.72460938)
\curveto(509.69257813,140.31835938)(509.534375,140.78515625)(509.21796875,141.125)
\curveto(508.9015625,141.46875)(508.503125,141.640625)(508.02265625,141.640625)
\curveto(507.54609375,141.640625)(507.14179688,141.46875)(506.80976563,141.125)
\curveto(506.47773438,140.78515625)(506.31171875,140.33789062)(506.31171875,139.78320312)
\closepath
}
}
{
\newrgbcolor{curcolor}{0 0 0}
\pscustom[linestyle=none,fillstyle=solid,fillcolor=curcolor]
{
\newpath
\moveto(512.21210937,137)
\lineto(512.21210937,145.58984375)
\lineto(513.92304687,145.58984375)
\lineto(515.95625,139.5078125)
\curveto(516.14375,138.94140625)(516.28046875,138.51757812)(516.36640625,138.23632812)
\curveto(516.4640625,138.54882812)(516.61640625,139.0078125)(516.8234375,139.61328125)
\lineto(518.88007812,145.58984375)
\lineto(520.409375,145.58984375)
\lineto(520.409375,137)
\lineto(519.31367187,137)
\lineto(519.31367187,144.18945312)
\lineto(516.81757812,137)
\lineto(515.7921875,137)
\lineto(513.3078125,144.3125)
\lineto(513.3078125,137)
\closepath
}
}
{
\newrgbcolor{curcolor}{0 0 0}
\pscustom[linewidth=1,linecolor=curcolor]
{
\newpath
\moveto(498.5,154.4)
\lineto(489.2,156.2)
}
}
{
\newrgbcolor{curcolor}{0 0 0}
\pscustom[linewidth=1,linecolor=curcolor]
{
\newpath
\moveto(227,210)
\lineto(236.2,208.1)
}
}
{
\newrgbcolor{curcolor}{0 0 0}
\pscustom[linestyle=none,fillstyle=solid,fillcolor=curcolor]
{
\newpath
\moveto(507.49804688,148.95)
\lineto(508.60546875,149.04375)
\curveto(508.6875,148.5046875)(508.87695312,148.0984375)(509.17382812,147.825)
\curveto(509.47460938,147.55546875)(509.8359375,147.42070312)(510.2578125,147.42070312)
\curveto(510.765625,147.42070312)(511.1953125,147.61210937)(511.546875,147.99492187)
\curveto(511.8984375,148.37773437)(512.07421875,148.88554687)(512.07421875,149.51835937)
\curveto(512.07421875,150.11992187)(511.90429688,150.59453125)(511.56445312,150.9421875)
\curveto(511.22851562,151.28984375)(510.78710938,151.46367187)(510.24023438,151.46367187)
\curveto(509.90039062,151.46367187)(509.59375,151.38554687)(509.3203125,151.22929687)
\curveto(509.046875,151.07695312)(508.83203125,150.87773437)(508.67578125,150.63164062)
\lineto(507.68554688,150.76054687)
\lineto(508.51757812,155.17265625)
\lineto(512.7890625,155.17265625)
\lineto(512.7890625,154.16484375)
\lineto(509.36132812,154.16484375)
\lineto(508.8984375,151.85625)
\curveto(509.4140625,152.215625)(509.95507812,152.3953125)(510.52148438,152.3953125)
\curveto(511.27148438,152.3953125)(511.90429688,152.13554687)(512.41992188,151.61601562)
\curveto(512.93554688,151.09648437)(513.19335938,150.42851562)(513.19335938,149.61210937)
\curveto(513.19335938,148.83476562)(512.96679688,148.16289062)(512.51367188,147.59648437)
\curveto(511.96289062,146.90117187)(511.2109375,146.55351562)(510.2578125,146.55351562)
\curveto(509.4765625,146.55351562)(508.83789062,146.77226562)(508.34179688,147.20976562)
\curveto(507.84960938,147.64726562)(507.56835938,148.22734375)(507.49804688,148.95)
\closepath
}
}
{
\newrgbcolor{curcolor}{0 0 0}
\pscustom[linestyle=none,fillstyle=solid,fillcolor=curcolor]
{
\newpath
\moveto(518.14453125,146.7)
\lineto(517.08984375,146.7)
\lineto(517.08984375,153.42070312)
\curveto(516.8359375,153.17851562)(516.50195312,152.93632812)(516.08789062,152.69414062)
\curveto(515.67773438,152.45195312)(515.30859375,152.2703125)(514.98046875,152.14921875)
\lineto(514.98046875,153.16875)
\curveto(515.5703125,153.44609375)(516.0859375,153.78203125)(516.52734375,154.1765625)
\curveto(516.96875,154.57109375)(517.28125,154.95390625)(517.46484375,155.325)
\lineto(518.14453125,155.325)
\closepath
}
}
{
\newrgbcolor{curcolor}{0 0 0}
\pscustom[linestyle=none,fillstyle=solid,fillcolor=curcolor]
{
\newpath
\moveto(526.38867188,147.71367187)
\lineto(526.38867188,146.7)
\lineto(520.7109375,146.7)
\curveto(520.703125,146.95390625)(520.74414062,147.19804687)(520.83398438,147.43242187)
\curveto(520.97851562,147.81914062)(521.20898438,148.2)(521.52539062,148.575)
\curveto(521.84570312,148.95)(522.30664062,149.38359375)(522.90820312,149.87578125)
\curveto(523.84179688,150.64140625)(524.47265625,151.246875)(524.80078125,151.6921875)
\curveto(525.12890625,152.14140625)(525.29296875,152.56523437)(525.29296875,152.96367187)
\curveto(525.29296875,153.38164062)(525.14257812,153.73320312)(524.84179688,154.01835937)
\curveto(524.54492188,154.30742187)(524.15625,154.45195312)(523.67578125,154.45195312)
\curveto(523.16796875,154.45195312)(522.76171875,154.29960937)(522.45703125,153.99492187)
\curveto(522.15234375,153.69023437)(521.99804688,153.26835937)(521.99414062,152.72929687)
\lineto(520.91015625,152.840625)
\curveto(520.984375,153.64921875)(521.26367188,154.26445312)(521.74804688,154.68632812)
\curveto(522.23242188,155.11210937)(522.8828125,155.325)(523.69921875,155.325)
\curveto(524.5234375,155.325)(525.17578125,155.09648437)(525.65625,154.63945312)
\curveto(526.13671875,154.18242187)(526.37695312,153.61601562)(526.37695312,152.94023437)
\curveto(526.37695312,152.59648437)(526.30664062,152.25859375)(526.16601562,151.9265625)
\curveto(526.02539062,151.59453125)(525.79101562,151.24492187)(525.46289062,150.87773437)
\curveto(525.13867188,150.51054687)(524.59765625,150.00664062)(523.83984375,149.36601562)
\curveto(523.20703125,148.83476562)(522.80078125,148.4734375)(522.62109375,148.28203125)
\curveto(522.44140625,148.09453125)(522.29296875,147.90507812)(522.17578125,147.71367187)
\closepath
}
}
{
\newrgbcolor{curcolor}{0 0 0}
\pscustom[linestyle=none,fillstyle=solid,fillcolor=curcolor]
{
\newpath
\moveto(527.91210938,146.7)
\lineto(527.91210938,155.28984375)
\lineto(529.62304688,155.28984375)
\lineto(531.65625,149.2078125)
\curveto(531.84375,148.64140625)(531.98046875,148.21757812)(532.06640625,147.93632812)
\curveto(532.1640625,148.24882812)(532.31640625,148.7078125)(532.5234375,149.31328125)
\lineto(534.58007812,155.28984375)
\lineto(536.109375,155.28984375)
\lineto(536.109375,146.7)
\lineto(535.01367188,146.7)
\lineto(535.01367188,153.88945312)
\lineto(532.51757812,146.7)
\lineto(531.4921875,146.7)
\lineto(529.0078125,154.0125)
\lineto(529.0078125,146.7)
\closepath
}
}
{
\newrgbcolor{curcolor}{0 0 0}
\pscustom[linewidth=1,linecolor=curcolor]
{
\newpath
\moveto(514.1,164)
\lineto(504.9,165.9)
}
}
{
\newrgbcolor{curcolor}{0 0 0}
\pscustom[linewidth=1,linecolor=curcolor]
{
\newpath
\moveto(242.7,219.7)
\lineto(251.9,217.8)
}
}
{
\newrgbcolor{curcolor}{0 0 0}
\pscustom[linestyle=none,fillstyle=solid,fillcolor=curcolor]
{
\newpath
\moveto(527.17070313,156.3)
\lineto(526.11601563,156.3)
\lineto(526.11601563,163.02070313)
\curveto(525.86210938,162.77851563)(525.528125,162.53632813)(525.1140625,162.29414063)
\curveto(524.70390625,162.05195313)(524.33476563,161.8703125)(524.00664063,161.74921875)
\lineto(524.00664063,162.76875)
\curveto(524.59648438,163.04609375)(525.11210938,163.38203125)(525.55351563,163.7765625)
\curveto(525.99492188,164.17109375)(526.30742188,164.55390625)(526.49101563,164.925)
\lineto(527.17070313,164.925)
\closepath
}
}
{
\newrgbcolor{curcolor}{0 0 0}
\pscustom[linestyle=none,fillstyle=solid,fillcolor=curcolor]
{
\newpath
\moveto(534.31914063,159.66914063)
\lineto(534.31914063,160.67695313)
\lineto(537.9578125,160.6828125)
\lineto(537.9578125,157.4953125)
\curveto(537.39921875,157.05)(536.82304688,156.7140625)(536.22929688,156.4875)
\curveto(535.63554688,156.26484375)(535.02617188,156.15351563)(534.40117188,156.15351563)
\curveto(533.55742188,156.15351563)(532.78984375,156.33320313)(532.0984375,156.69257813)
\curveto(531.4109375,157.05585938)(530.89140625,157.57929688)(530.53984375,158.26289063)
\curveto(530.18828125,158.94648438)(530.0125,159.71015625)(530.0125,160.55390625)
\curveto(530.0125,161.38984375)(530.18632813,162.16914063)(530.53398438,162.89179688)
\curveto(530.88554688,163.61835938)(531.38945313,164.15742188)(532.04570313,164.50898438)
\curveto(532.70195313,164.86054688)(533.4578125,165.03632813)(534.31328125,165.03632813)
\curveto(534.934375,165.03632813)(535.49492188,164.93476563)(535.99492188,164.73164063)
\curveto(536.49882813,164.53242188)(536.89335938,164.253125)(537.17851563,163.89375)
\curveto(537.46367188,163.534375)(537.68046875,163.065625)(537.82890625,162.4875)
\lineto(536.80351563,162.20625)
\curveto(536.67460938,162.64375)(536.51445313,162.9875)(536.32304688,163.2375)
\curveto(536.13164063,163.4875)(535.85820313,163.68671875)(535.50273438,163.83515625)
\curveto(535.14726563,163.9875)(534.75273438,164.06367188)(534.31914063,164.06367188)
\curveto(533.79960938,164.06367188)(533.35039063,163.98359375)(532.97148438,163.8234375)
\curveto(532.59257813,163.6671875)(532.2859375,163.46015625)(532.0515625,163.20234375)
\curveto(531.82109375,162.94453125)(531.64140625,162.66132813)(531.5125,162.35273438)
\curveto(531.29375,161.82148438)(531.184375,161.2453125)(531.184375,160.62421875)
\curveto(531.184375,159.85859375)(531.31523438,159.21796875)(531.57695313,158.70234375)
\curveto(531.84257813,158.18671875)(532.22734375,157.80390625)(532.73125,157.55390625)
\curveto(533.23515625,157.30390625)(533.7703125,157.17890625)(534.33671875,157.17890625)
\curveto(534.82890625,157.17890625)(535.309375,157.27265625)(535.778125,157.46015625)
\curveto(536.246875,157.6515625)(536.60234375,157.8546875)(536.84453125,158.06953125)
\lineto(536.84453125,159.66914063)
\closepath
}
}
{
\newrgbcolor{curcolor}{0 0 0}
\pscustom[linewidth=1,linecolor=curcolor]
{
\newpath
\moveto(85.9,187.5)
\lineto(94.9,187.5)
}
}
{
\newrgbcolor{curcolor}{0 0 0}
\pscustom[linestyle=none,fillstyle=solid,fillcolor=curcolor]
{
\newpath
\moveto(50.93710938,186.178125)
\lineto(50.93710938,187.23867188)
\lineto(54.17734375,187.23867188)
\lineto(54.17734375,186.178125)
\closepath
}
}
{
\newrgbcolor{curcolor}{0 0 0}
\pscustom[linestyle=none,fillstyle=solid,fillcolor=curcolor]
{
\newpath
\moveto(59.02304688,183.6)
\lineto(57.96835938,183.6)
\lineto(57.96835938,190.32070313)
\curveto(57.71445313,190.07851563)(57.38046875,189.83632813)(56.96640625,189.59414063)
\curveto(56.55625,189.35195313)(56.18710938,189.1703125)(55.85898438,189.04921875)
\lineto(55.85898438,190.06875)
\curveto(56.44882813,190.34609375)(56.96445313,190.68203125)(57.40585938,191.0765625)
\curveto(57.84726563,191.47109375)(58.15976563,191.85390625)(58.34335938,192.225)
\lineto(59.02304688,192.225)
\closepath
}
}
{
\newrgbcolor{curcolor}{0 0 0}
\pscustom[linestyle=none,fillstyle=solid,fillcolor=curcolor]
{
\newpath
\moveto(61.72421875,185.85)
\lineto(62.83164063,185.94375)
\curveto(62.91367188,185.4046875)(63.103125,184.9984375)(63.4,184.725)
\curveto(63.70078125,184.45546875)(64.06210938,184.32070313)(64.48398438,184.32070313)
\curveto(64.99179688,184.32070313)(65.42148438,184.51210938)(65.77304688,184.89492188)
\curveto(66.12460938,185.27773438)(66.30039063,185.78554688)(66.30039063,186.41835938)
\curveto(66.30039063,187.01992188)(66.13046875,187.49453125)(65.790625,187.8421875)
\curveto(65.4546875,188.18984375)(65.01328125,188.36367188)(64.46640625,188.36367188)
\curveto(64.1265625,188.36367188)(63.81992188,188.28554688)(63.54648438,188.12929688)
\curveto(63.27304688,187.97695313)(63.05820313,187.77773438)(62.90195313,187.53164063)
\lineto(61.91171875,187.66054688)
\lineto(62.74375,192.07265625)
\lineto(67.01523438,192.07265625)
\lineto(67.01523438,191.06484375)
\lineto(63.5875,191.06484375)
\lineto(63.12460938,188.75625)
\curveto(63.64023438,189.115625)(64.18125,189.2953125)(64.74765625,189.2953125)
\curveto(65.49765625,189.2953125)(66.13046875,189.03554688)(66.64609375,188.51601563)
\curveto(67.16171875,187.99648438)(67.41953125,187.32851563)(67.41953125,186.51210938)
\curveto(67.41953125,185.73476563)(67.19296875,185.06289063)(66.73984375,184.49648438)
\curveto(66.1890625,183.80117188)(65.43710938,183.45351563)(64.48398438,183.45351563)
\curveto(63.70273438,183.45351563)(63.0640625,183.67226563)(62.56796875,184.10976563)
\curveto(62.07578125,184.54726563)(61.79453125,185.12734375)(61.72421875,185.85)
\closepath
}
}
{
\newrgbcolor{curcolor}{0 0 0}
\pscustom[linewidth=1,linecolor=curcolor]
{
\newpath
\moveto(85.9,205.9)
\lineto(94.9,205.9)
}
}
{
\newrgbcolor{curcolor}{0 0 0}
\pscustom[linestyle=none,fillstyle=solid,fillcolor=curcolor]
{
\newpath
\moveto(50.93710938,204.578125)
\lineto(50.93710938,205.63867188)
\lineto(54.17734375,205.63867188)
\lineto(54.17734375,204.578125)
\closepath
}
}
{
\newrgbcolor{curcolor}{0 0 0}
\pscustom[linestyle=none,fillstyle=solid,fillcolor=curcolor]
{
\newpath
\moveto(59.02304688,202)
\lineto(57.96835938,202)
\lineto(57.96835938,208.72070312)
\curveto(57.71445313,208.47851562)(57.38046875,208.23632812)(56.96640625,207.99414062)
\curveto(56.55625,207.75195312)(56.18710938,207.5703125)(55.85898438,207.44921875)
\lineto(55.85898438,208.46875)
\curveto(56.44882813,208.74609375)(56.96445313,209.08203125)(57.40585938,209.4765625)
\curveto(57.84726563,209.87109375)(58.15976563,210.25390625)(58.34335938,210.625)
\lineto(59.02304688,210.625)
\closepath
}
}
{
\newrgbcolor{curcolor}{0 0 0}
\pscustom[linestyle=none,fillstyle=solid,fillcolor=curcolor]
{
\newpath
\moveto(61.72421875,206.23632812)
\curveto(61.72421875,207.25195312)(61.82773438,208.06835938)(62.03476563,208.68554688)
\curveto(62.24570313,209.30664062)(62.55625,209.78515625)(62.96640625,210.12109375)
\curveto(63.38046875,210.45703125)(63.9,210.625)(64.525,210.625)
\curveto(64.9859375,210.625)(65.39023438,210.53125)(65.73789063,210.34375)
\curveto(66.08554688,210.16015625)(66.37265625,209.89257812)(66.59921875,209.54101562)
\curveto(66.82578125,209.19335938)(67.00351563,208.76757812)(67.13242188,208.26367188)
\curveto(67.26132813,207.76367188)(67.32578125,207.08789062)(67.32578125,206.23632812)
\curveto(67.32578125,205.22851562)(67.22226563,204.4140625)(67.01523438,203.79296875)
\curveto(66.80820313,203.17578125)(66.49765625,202.69726562)(66.08359375,202.35742188)
\curveto(65.6734375,202.02148438)(65.15390625,201.85351562)(64.525,201.85351562)
\curveto(63.696875,201.85351562)(63.04648438,202.15039062)(62.57382813,202.74414062)
\curveto(62.00742188,203.45898438)(61.72421875,204.62304688)(61.72421875,206.23632812)
\closepath
\moveto(62.80820313,206.23632812)
\curveto(62.80820313,204.82617188)(62.97226563,203.88671875)(63.30039063,203.41796875)
\curveto(63.63242188,202.953125)(64.040625,202.72070312)(64.525,202.72070312)
\curveto(65.009375,202.72070312)(65.415625,202.95507812)(65.74375,203.42382812)
\curveto(66.07578125,203.89257812)(66.24179688,204.83007812)(66.24179688,206.23632812)
\curveto(66.24179688,207.65039062)(66.07578125,208.58984375)(65.74375,209.0546875)
\curveto(65.415625,209.51953125)(65.00546875,209.75195312)(64.51328125,209.75195312)
\curveto(64.02890625,209.75195312)(63.6421875,209.546875)(63.353125,209.13671875)
\curveto(62.98984375,208.61328125)(62.80820313,207.64648438)(62.80820313,206.23632812)
\closepath
}
}
{
\newrgbcolor{curcolor}{0 0 0}
\pscustom[linewidth=1,linecolor=curcolor]
{
\newpath
\moveto(85.9,224.3)
\lineto(94.9,224.3)
}
}
{
\newrgbcolor{curcolor}{0 0 0}
\pscustom[linestyle=none,fillstyle=solid,fillcolor=curcolor]
{
\newpath
\moveto(57.6109375,222.978125)
\lineto(57.6109375,224.03867187)
\lineto(60.85117188,224.03867187)
\lineto(60.85117188,222.978125)
\closepath
}
}
{
\newrgbcolor{curcolor}{0 0 0}
\pscustom[linestyle=none,fillstyle=solid,fillcolor=curcolor]
{
\newpath
\moveto(61.72421875,222.65)
\lineto(62.83164063,222.74375)
\curveto(62.91367188,222.2046875)(63.103125,221.7984375)(63.4,221.525)
\curveto(63.70078125,221.25546875)(64.06210938,221.12070312)(64.48398438,221.12070312)
\curveto(64.99179688,221.12070312)(65.42148438,221.31210937)(65.77304688,221.69492187)
\curveto(66.12460938,222.07773437)(66.30039063,222.58554687)(66.30039063,223.21835937)
\curveto(66.30039063,223.81992187)(66.13046875,224.29453125)(65.790625,224.6421875)
\curveto(65.4546875,224.98984375)(65.01328125,225.16367187)(64.46640625,225.16367187)
\curveto(64.1265625,225.16367187)(63.81992188,225.08554687)(63.54648438,224.92929687)
\curveto(63.27304688,224.77695312)(63.05820313,224.57773437)(62.90195313,224.33164062)
\lineto(61.91171875,224.46054687)
\lineto(62.74375,228.87265625)
\lineto(67.01523438,228.87265625)
\lineto(67.01523438,227.86484375)
\lineto(63.5875,227.86484375)
\lineto(63.12460938,225.55625)
\curveto(63.64023438,225.915625)(64.18125,226.0953125)(64.74765625,226.0953125)
\curveto(65.49765625,226.0953125)(66.13046875,225.83554687)(66.64609375,225.31601562)
\curveto(67.16171875,224.79648437)(67.41953125,224.12851562)(67.41953125,223.31210937)
\curveto(67.41953125,222.53476562)(67.19296875,221.86289062)(66.73984375,221.29648437)
\curveto(66.1890625,220.60117187)(65.43710938,220.25351562)(64.48398438,220.25351562)
\curveto(63.70273438,220.25351562)(63.0640625,220.47226562)(62.56796875,220.90976562)
\curveto(62.07578125,221.34726562)(61.79453125,221.92734375)(61.72421875,222.65)
\closepath
}
}
{
\newrgbcolor{curcolor}{0 0 0}
\pscustom[linewidth=1,linecolor=curcolor]
{
\newpath
\moveto(85.9,242.5)
\lineto(94.9,242.5)
}
}
{
\newrgbcolor{curcolor}{0 0 0}
\pscustom[linestyle=none,fillstyle=solid,fillcolor=curcolor]
{
\newpath
\moveto(61.72421875,242.83632812)
\curveto(61.72421875,243.85195312)(61.82773438,244.66835937)(62.03476563,245.28554687)
\curveto(62.24570313,245.90664062)(62.55625,246.38515625)(62.96640625,246.72109375)
\curveto(63.38046875,247.05703125)(63.9,247.225)(64.525,247.225)
\curveto(64.9859375,247.225)(65.39023438,247.13125)(65.73789063,246.94375)
\curveto(66.08554688,246.76015625)(66.37265625,246.49257812)(66.59921875,246.14101562)
\curveto(66.82578125,245.79335937)(67.00351563,245.36757812)(67.13242188,244.86367187)
\curveto(67.26132813,244.36367187)(67.32578125,243.68789062)(67.32578125,242.83632812)
\curveto(67.32578125,241.82851562)(67.22226563,241.0140625)(67.01523438,240.39296875)
\curveto(66.80820313,239.77578125)(66.49765625,239.29726562)(66.08359375,238.95742187)
\curveto(65.6734375,238.62148437)(65.15390625,238.45351562)(64.525,238.45351562)
\curveto(63.696875,238.45351562)(63.04648438,238.75039062)(62.57382813,239.34414062)
\curveto(62.00742188,240.05898437)(61.72421875,241.22304687)(61.72421875,242.83632812)
\closepath
\moveto(62.80820313,242.83632812)
\curveto(62.80820313,241.42617187)(62.97226563,240.48671875)(63.30039063,240.01796875)
\curveto(63.63242188,239.553125)(64.040625,239.32070312)(64.525,239.32070312)
\curveto(65.009375,239.32070312)(65.415625,239.55507812)(65.74375,240.02382812)
\curveto(66.07578125,240.49257812)(66.24179688,241.43007812)(66.24179688,242.83632812)
\curveto(66.24179688,244.25039062)(66.07578125,245.18984375)(65.74375,245.6546875)
\curveto(65.415625,246.11953125)(65.00546875,246.35195312)(64.51328125,246.35195312)
\curveto(64.02890625,246.35195312)(63.6421875,246.146875)(63.353125,245.73671875)
\curveto(62.98984375,245.21328125)(62.80820313,244.24648437)(62.80820313,242.83632812)
\closepath
}
}
{
\newrgbcolor{curcolor}{0 0 0}
\pscustom[linewidth=1,linecolor=curcolor]
{
\newpath
\moveto(85.9,260.9)
\lineto(94.9,260.9)
}
}
{
\newrgbcolor{curcolor}{0 0 0}
\pscustom[linestyle=none,fillstyle=solid,fillcolor=curcolor]
{
\newpath
\moveto(61.72421875,259.25)
\lineto(62.83164063,259.34375)
\curveto(62.91367188,258.8046875)(63.103125,258.3984375)(63.4,258.125)
\curveto(63.70078125,257.85546875)(64.06210938,257.72070312)(64.48398438,257.72070312)
\curveto(64.99179688,257.72070312)(65.42148438,257.91210938)(65.77304688,258.29492188)
\curveto(66.12460938,258.67773438)(66.30039063,259.18554688)(66.30039063,259.81835938)
\curveto(66.30039063,260.41992188)(66.13046875,260.89453125)(65.790625,261.2421875)
\curveto(65.4546875,261.58984375)(65.01328125,261.76367188)(64.46640625,261.76367188)
\curveto(64.1265625,261.76367188)(63.81992188,261.68554688)(63.54648438,261.52929688)
\curveto(63.27304688,261.37695312)(63.05820313,261.17773438)(62.90195313,260.93164062)
\lineto(61.91171875,261.06054688)
\lineto(62.74375,265.47265625)
\lineto(67.01523438,265.47265625)
\lineto(67.01523438,264.46484375)
\lineto(63.5875,264.46484375)
\lineto(63.12460938,262.15625)
\curveto(63.64023438,262.515625)(64.18125,262.6953125)(64.74765625,262.6953125)
\curveto(65.49765625,262.6953125)(66.13046875,262.43554688)(66.64609375,261.91601562)
\curveto(67.16171875,261.39648438)(67.41953125,260.72851562)(67.41953125,259.91210938)
\curveto(67.41953125,259.13476562)(67.19296875,258.46289062)(66.73984375,257.89648438)
\curveto(66.1890625,257.20117188)(65.43710938,256.85351562)(64.48398438,256.85351562)
\curveto(63.70273438,256.85351562)(63.0640625,257.07226562)(62.56796875,257.50976562)
\curveto(62.07578125,257.94726562)(61.79453125,258.52734375)(61.72421875,259.25)
\closepath
}
}
{
\newrgbcolor{curcolor}{0 0 0}
\pscustom[linewidth=1,linecolor=curcolor]
{
\newpath
\moveto(85.9,279.3)
\lineto(94.9,279.3)
}
}
{
\newrgbcolor{curcolor}{0 0 0}
\pscustom[linestyle=none,fillstyle=solid,fillcolor=curcolor]
{
\newpath
\moveto(59.02304688,275.4)
\lineto(57.96835938,275.4)
\lineto(57.96835938,282.12070312)
\curveto(57.71445313,281.87851562)(57.38046875,281.63632812)(56.96640625,281.39414062)
\curveto(56.55625,281.15195312)(56.18710938,280.9703125)(55.85898438,280.84921875)
\lineto(55.85898438,281.86875)
\curveto(56.44882813,282.14609375)(56.96445313,282.48203125)(57.40585938,282.8765625)
\curveto(57.84726563,283.27109375)(58.15976563,283.65390625)(58.34335938,284.025)
\lineto(59.02304688,284.025)
\closepath
}
}
{
\newrgbcolor{curcolor}{0 0 0}
\pscustom[linestyle=none,fillstyle=solid,fillcolor=curcolor]
{
\newpath
\moveto(61.72421875,279.63632812)
\curveto(61.72421875,280.65195312)(61.82773438,281.46835937)(62.03476563,282.08554687)
\curveto(62.24570313,282.70664062)(62.55625,283.18515625)(62.96640625,283.52109375)
\curveto(63.38046875,283.85703125)(63.9,284.025)(64.525,284.025)
\curveto(64.9859375,284.025)(65.39023438,283.93125)(65.73789063,283.74375)
\curveto(66.08554688,283.56015625)(66.37265625,283.29257812)(66.59921875,282.94101562)
\curveto(66.82578125,282.59335937)(67.00351563,282.16757812)(67.13242188,281.66367187)
\curveto(67.26132813,281.16367187)(67.32578125,280.48789062)(67.32578125,279.63632812)
\curveto(67.32578125,278.62851562)(67.22226563,277.8140625)(67.01523438,277.19296875)
\curveto(66.80820313,276.57578125)(66.49765625,276.09726562)(66.08359375,275.75742187)
\curveto(65.6734375,275.42148437)(65.15390625,275.25351562)(64.525,275.25351562)
\curveto(63.696875,275.25351562)(63.04648438,275.55039062)(62.57382813,276.14414062)
\curveto(62.00742188,276.85898437)(61.72421875,278.02304687)(61.72421875,279.63632812)
\closepath
\moveto(62.80820313,279.63632812)
\curveto(62.80820313,278.22617187)(62.97226563,277.28671875)(63.30039063,276.81796875)
\curveto(63.63242188,276.353125)(64.040625,276.12070312)(64.525,276.12070312)
\curveto(65.009375,276.12070312)(65.415625,276.35507812)(65.74375,276.82382812)
\curveto(66.07578125,277.29257812)(66.24179688,278.23007812)(66.24179688,279.63632812)
\curveto(66.24179688,281.05039062)(66.07578125,281.98984375)(65.74375,282.4546875)
\curveto(65.415625,282.91953125)(65.00546875,283.15195312)(64.51328125,283.15195312)
\curveto(64.02890625,283.15195312)(63.6421875,282.946875)(63.353125,282.53671875)
\curveto(62.98984375,282.01328125)(62.80820313,281.04648437)(62.80820313,279.63632812)
\closepath
}
}
{
\newrgbcolor{curcolor}{0 0 0}
\pscustom[linewidth=1,linecolor=curcolor]
{
\newpath
\moveto(85.9,297.6)
\lineto(94.9,297.6)
}
}
{
\newrgbcolor{curcolor}{0 0 0}
\pscustom[linestyle=none,fillstyle=solid,fillcolor=curcolor]
{
\newpath
\moveto(59.02304688,293.7)
\lineto(57.96835938,293.7)
\lineto(57.96835938,300.42070312)
\curveto(57.71445313,300.17851562)(57.38046875,299.93632812)(56.96640625,299.69414062)
\curveto(56.55625,299.45195312)(56.18710938,299.2703125)(55.85898438,299.14921875)
\lineto(55.85898438,300.16875)
\curveto(56.44882813,300.44609375)(56.96445313,300.78203125)(57.40585938,301.1765625)
\curveto(57.84726563,301.57109375)(58.15976563,301.95390625)(58.34335938,302.325)
\lineto(59.02304688,302.325)
\closepath
}
}
{
\newrgbcolor{curcolor}{0 0 0}
\pscustom[linestyle=none,fillstyle=solid,fillcolor=curcolor]
{
\newpath
\moveto(61.72421875,295.95)
\lineto(62.83164063,296.04375)
\curveto(62.91367188,295.5046875)(63.103125,295.0984375)(63.4,294.825)
\curveto(63.70078125,294.55546875)(64.06210938,294.42070312)(64.48398438,294.42070312)
\curveto(64.99179688,294.42070312)(65.42148438,294.61210937)(65.77304688,294.99492187)
\curveto(66.12460938,295.37773437)(66.30039063,295.88554687)(66.30039063,296.51835937)
\curveto(66.30039063,297.11992187)(66.13046875,297.59453125)(65.790625,297.9421875)
\curveto(65.4546875,298.28984375)(65.01328125,298.46367187)(64.46640625,298.46367187)
\curveto(64.1265625,298.46367187)(63.81992188,298.38554687)(63.54648438,298.22929687)
\curveto(63.27304688,298.07695312)(63.05820313,297.87773437)(62.90195313,297.63164062)
\lineto(61.91171875,297.76054687)
\lineto(62.74375,302.17265625)
\lineto(67.01523438,302.17265625)
\lineto(67.01523438,301.16484375)
\lineto(63.5875,301.16484375)
\lineto(63.12460938,298.85625)
\curveto(63.64023438,299.215625)(64.18125,299.3953125)(64.74765625,299.3953125)
\curveto(65.49765625,299.3953125)(66.13046875,299.13554687)(66.64609375,298.61601562)
\curveto(67.16171875,298.09648437)(67.41953125,297.42851562)(67.41953125,296.61210937)
\curveto(67.41953125,295.83476562)(67.19296875,295.16289062)(66.73984375,294.59648437)
\curveto(66.1890625,293.90117187)(65.43710938,293.55351562)(64.48398438,293.55351562)
\curveto(63.70273438,293.55351562)(63.0640625,293.77226562)(62.56796875,294.20976562)
\curveto(62.07578125,294.64726562)(61.79453125,295.22734375)(61.72421875,295.95)
\closepath
}
}
{
\newrgbcolor{curcolor}{0 0 0}
\pscustom[linewidth=1,linecolor=curcolor]
{
\newpath
\moveto(85.9,316)
\lineto(94.9,316)
}
}
{
\newrgbcolor{curcolor}{0 0 0}
\pscustom[linestyle=none,fillstyle=solid,fillcolor=curcolor]
{
\newpath
\moveto(60.59335938,313.11367188)
\lineto(60.59335938,312.1)
\lineto(54.915625,312.1)
\curveto(54.9078125,312.35390625)(54.94882813,312.59804688)(55.03867188,312.83242188)
\curveto(55.18320313,313.21914063)(55.41367188,313.6)(55.73007813,313.975)
\curveto(56.05039063,314.35)(56.51132813,314.78359375)(57.11289063,315.27578125)
\curveto(58.04648438,316.04140625)(58.67734375,316.646875)(59.00546875,317.0921875)
\curveto(59.33359375,317.54140625)(59.49765625,317.96523438)(59.49765625,318.36367188)
\curveto(59.49765625,318.78164063)(59.34726563,319.13320313)(59.04648438,319.41835938)
\curveto(58.74960938,319.70742188)(58.3609375,319.85195313)(57.88046875,319.85195313)
\curveto(57.37265625,319.85195313)(56.96640625,319.69960938)(56.66171875,319.39492188)
\curveto(56.35703125,319.09023438)(56.20273438,318.66835938)(56.19882813,318.12929688)
\lineto(55.11484375,318.240625)
\curveto(55.1890625,319.04921875)(55.46835938,319.66445313)(55.95273438,320.08632813)
\curveto(56.43710938,320.51210938)(57.0875,320.725)(57.90390625,320.725)
\curveto(58.728125,320.725)(59.38046875,320.49648438)(59.8609375,320.03945313)
\curveto(60.34140625,319.58242188)(60.58164063,319.01601563)(60.58164063,318.34023438)
\curveto(60.58164063,317.99648438)(60.51132813,317.65859375)(60.37070313,317.3265625)
\curveto(60.23007813,316.99453125)(59.99570313,316.64492188)(59.66757813,316.27773438)
\curveto(59.34335938,315.91054688)(58.80234375,315.40664063)(58.04453125,314.76601563)
\curveto(57.41171875,314.23476563)(57.00546875,313.8734375)(56.82578125,313.68203125)
\curveto(56.64609375,313.49453125)(56.49765625,313.30507813)(56.38046875,313.11367188)
\closepath
}
}
{
\newrgbcolor{curcolor}{0 0 0}
\pscustom[linestyle=none,fillstyle=solid,fillcolor=curcolor]
{
\newpath
\moveto(61.72421875,316.33632813)
\curveto(61.72421875,317.35195313)(61.82773438,318.16835938)(62.03476563,318.78554688)
\curveto(62.24570313,319.40664063)(62.55625,319.88515625)(62.96640625,320.22109375)
\curveto(63.38046875,320.55703125)(63.9,320.725)(64.525,320.725)
\curveto(64.9859375,320.725)(65.39023438,320.63125)(65.73789063,320.44375)
\curveto(66.08554688,320.26015625)(66.37265625,319.99257813)(66.59921875,319.64101563)
\curveto(66.82578125,319.29335938)(67.00351563,318.86757813)(67.13242188,318.36367188)
\curveto(67.26132813,317.86367188)(67.32578125,317.18789063)(67.32578125,316.33632813)
\curveto(67.32578125,315.32851563)(67.22226563,314.5140625)(67.01523438,313.89296875)
\curveto(66.80820313,313.27578125)(66.49765625,312.79726563)(66.08359375,312.45742188)
\curveto(65.6734375,312.12148438)(65.15390625,311.95351563)(64.525,311.95351563)
\curveto(63.696875,311.95351563)(63.04648438,312.25039063)(62.57382813,312.84414063)
\curveto(62.00742188,313.55898438)(61.72421875,314.72304688)(61.72421875,316.33632813)
\closepath
\moveto(62.80820313,316.33632813)
\curveto(62.80820313,314.92617188)(62.97226563,313.98671875)(63.30039063,313.51796875)
\curveto(63.63242188,313.053125)(64.040625,312.82070313)(64.525,312.82070313)
\curveto(65.009375,312.82070313)(65.415625,313.05507813)(65.74375,313.52382813)
\curveto(66.07578125,313.99257813)(66.24179688,314.93007813)(66.24179688,316.33632813)
\curveto(66.24179688,317.75039063)(66.07578125,318.68984375)(65.74375,319.1546875)
\curveto(65.415625,319.61953125)(65.00546875,319.85195313)(64.51328125,319.85195313)
\curveto(64.02890625,319.85195313)(63.6421875,319.646875)(63.353125,319.23671875)
\curveto(62.98984375,318.71328125)(62.80820313,317.74648438)(62.80820313,316.33632813)
\closepath
}
}
{
\newrgbcolor{curcolor}{0 0 0}
\pscustom[linestyle=none,fillstyle=solid,fillcolor=curcolor]
{
\newpath
\moveto(145.86621094,439.3)
\lineto(145.86621094,447.88984375)
\lineto(149.67480469,447.88984375)
\curveto(150.44042969,447.88984375)(151.02246094,447.81171875)(151.42089844,447.65546875)
\curveto(151.81933594,447.503125)(152.13769531,447.23164063)(152.37597656,446.84101563)
\curveto(152.61425781,446.45039063)(152.73339844,446.01875)(152.73339844,445.54609375)
\curveto(152.73339844,444.93671875)(152.53613281,444.42304688)(152.14160156,444.00507813)
\curveto(151.74707031,443.58710938)(151.13769531,443.32148438)(150.31347656,443.20820313)
\curveto(150.61425781,443.06367188)(150.84277344,442.92109375)(150.99902344,442.78046875)
\curveto(151.33105469,442.47578125)(151.64550781,442.09492188)(151.94238281,441.63789063)
\lineto(153.43652344,439.3)
\lineto(152.00683594,439.3)
\lineto(150.87011719,441.08710938)
\curveto(150.53808594,441.60273438)(150.26464844,441.99726563)(150.04980469,442.27070313)
\curveto(149.83496094,442.54414063)(149.64160156,442.73554688)(149.46972656,442.84492188)
\curveto(149.30175781,442.95429688)(149.12988281,443.03046875)(148.95410156,443.0734375)
\curveto(148.82519531,443.10078125)(148.61425781,443.11445313)(148.32128906,443.11445313)
\lineto(147.00292969,443.11445313)
\lineto(147.00292969,439.3)
\closepath
\moveto(147.00292969,444.09882813)
\lineto(149.44628906,444.09882813)
\curveto(149.96582031,444.09882813)(150.37207031,444.1515625)(150.66503906,444.25703125)
\curveto(150.95800781,444.36640625)(151.18066406,444.53828125)(151.33300781,444.77265625)
\curveto(151.48535156,445.0109375)(151.56152344,445.26875)(151.56152344,445.54609375)
\curveto(151.56152344,445.95234375)(151.41308594,446.28632813)(151.11621094,446.54804688)
\curveto(150.82324219,446.80976563)(150.35839844,446.940625)(149.72167969,446.940625)
\lineto(147.00292969,446.940625)
\closepath
}
}
{
\newrgbcolor{curcolor}{0 0 0}
\pscustom[linestyle=none,fillstyle=solid,fillcolor=curcolor]
{
\newpath
\moveto(158.45800781,439.3)
\lineto(158.45800781,440.2140625)
\curveto(157.97363281,439.5109375)(157.31542969,439.159375)(156.48339844,439.159375)
\curveto(156.11621094,439.159375)(155.77246094,439.2296875)(155.45214844,439.3703125)
\curveto(155.13574219,439.5109375)(154.89941406,439.68671875)(154.74316406,439.89765625)
\curveto(154.59082031,440.1125)(154.48339844,440.37421875)(154.42089844,440.6828125)
\curveto(154.37792969,440.88984375)(154.35644531,441.21796875)(154.35644531,441.6671875)
\lineto(154.35644531,445.52265625)
\lineto(155.41113281,445.52265625)
\lineto(155.41113281,442.07148438)
\curveto(155.41113281,441.52070313)(155.43261719,441.14960938)(155.47558594,440.95820313)
\curveto(155.54199219,440.68085938)(155.68261719,440.46210938)(155.89746094,440.30195313)
\curveto(156.11230469,440.14570313)(156.37792969,440.06757813)(156.69433594,440.06757813)
\curveto(157.01074219,440.06757813)(157.30761719,440.14765625)(157.58496094,440.3078125)
\curveto(157.86230469,440.471875)(158.05761719,440.69257813)(158.17089844,440.96992188)
\curveto(158.28808594,441.25117188)(158.34667969,441.65742188)(158.34667969,442.18867188)
\lineto(158.34667969,445.52265625)
\lineto(159.40136719,445.52265625)
\lineto(159.40136719,439.3)
\closepath
}
}
{
\newrgbcolor{curcolor}{0 0 0}
\pscustom[linestyle=none,fillstyle=solid,fillcolor=curcolor]
{
\newpath
\moveto(161.05371094,439.3)
\lineto(161.05371094,445.52265625)
\lineto(162.00292969,445.52265625)
\lineto(162.00292969,444.63789063)
\curveto(162.45996094,445.32148438)(163.12011719,445.66328125)(163.98339844,445.66328125)
\curveto(164.35839844,445.66328125)(164.70214844,445.59492188)(165.01464844,445.45820313)
\curveto(165.33105469,445.32539063)(165.56738281,445.14960938)(165.72363281,444.93085938)
\curveto(165.87988281,444.71210938)(165.98925781,444.45234375)(166.05175781,444.1515625)
\curveto(166.09082031,443.95625)(166.11035156,443.61445313)(166.11035156,443.12617188)
\lineto(166.11035156,439.3)
\lineto(165.05566406,439.3)
\lineto(165.05566406,443.08515625)
\curveto(165.05566406,443.51484375)(165.01464844,443.83515625)(164.93261719,444.04609375)
\curveto(164.85058594,444.2609375)(164.70410156,444.43085938)(164.49316406,444.55585938)
\curveto(164.28613281,444.68476563)(164.04199219,444.74921875)(163.76074219,444.74921875)
\curveto(163.31152344,444.74921875)(162.92285156,444.60664063)(162.59472656,444.32148438)
\curveto(162.27050781,444.03632813)(162.10839844,443.4953125)(162.10839844,442.6984375)
\lineto(162.10839844,439.3)
\closepath
}
}
{
\newrgbcolor{curcolor}{0 0 0}
\pscustom[linestyle=none,fillstyle=solid,fillcolor=curcolor]
{
\newpath
\moveto(170.03027344,440.24335938)
\lineto(170.18261719,439.31171875)
\curveto(169.88574219,439.24921875)(169.62011719,439.21796875)(169.38574219,439.21796875)
\curveto(169.00292969,439.21796875)(168.70605469,439.27851563)(168.49511719,439.39960938)
\curveto(168.28417969,439.52070313)(168.13574219,439.67890625)(168.04980469,439.87421875)
\curveto(167.96386719,440.0734375)(167.92089844,440.48945313)(167.92089844,441.12226563)
\lineto(167.92089844,444.70234375)
\lineto(167.14746094,444.70234375)
\lineto(167.14746094,445.52265625)
\lineto(167.92089844,445.52265625)
\lineto(167.92089844,447.06367188)
\lineto(168.96972656,447.69648438)
\lineto(168.96972656,445.52265625)
\lineto(170.03027344,445.52265625)
\lineto(170.03027344,444.70234375)
\lineto(168.96972656,444.70234375)
\lineto(168.96972656,441.06367188)
\curveto(168.96972656,440.76289063)(168.98730469,440.56953125)(169.02246094,440.48359375)
\curveto(169.06152344,440.39765625)(169.12207031,440.32929688)(169.20410156,440.27851563)
\curveto(169.29003906,440.22773438)(169.41113281,440.20234375)(169.56738281,440.20234375)
\curveto(169.68457031,440.20234375)(169.83886719,440.21601563)(170.03027344,440.24335938)
\closepath
}
}
{
\newrgbcolor{curcolor}{0 0 0}
\pscustom[linestyle=none,fillstyle=solid,fillcolor=curcolor]
{
\newpath
\moveto(171.06738281,446.67695313)
\lineto(171.06738281,447.88984375)
\lineto(172.12207031,447.88984375)
\lineto(172.12207031,446.67695313)
\closepath
\moveto(171.06738281,439.3)
\lineto(171.06738281,445.52265625)
\lineto(172.12207031,445.52265625)
\lineto(172.12207031,439.3)
\closepath
}
}
{
\newrgbcolor{curcolor}{0 0 0}
\pscustom[linestyle=none,fillstyle=solid,fillcolor=curcolor]
{
\newpath
\moveto(173.72753906,439.3)
\lineto(173.72753906,445.52265625)
\lineto(174.67089844,445.52265625)
\lineto(174.67089844,444.64960938)
\curveto(174.86621094,444.95429688)(175.12597656,445.1984375)(175.45019531,445.38203125)
\curveto(175.77441406,445.56953125)(176.14355469,445.66328125)(176.55761719,445.66328125)
\curveto(177.01855469,445.66328125)(177.39550781,445.56757813)(177.68847656,445.37617188)
\curveto(177.98535156,445.18476563)(178.19433594,444.9171875)(178.31542969,444.5734375)
\curveto(178.80761719,445.3)(179.44824219,445.66328125)(180.23730469,445.66328125)
\curveto(180.85449219,445.66328125)(181.32910156,445.49140625)(181.66113281,445.14765625)
\curveto(181.99316406,444.8078125)(182.15917969,444.28242188)(182.15917969,443.57148438)
\lineto(182.15917969,439.3)
\lineto(181.11035156,439.3)
\lineto(181.11035156,443.21992188)
\curveto(181.11035156,443.64179688)(181.07519531,443.94453125)(181.00488281,444.128125)
\curveto(180.93847656,444.315625)(180.81542969,444.46601563)(180.63574219,444.57929688)
\curveto(180.45605469,444.69257813)(180.24511719,444.74921875)(180.00292969,444.74921875)
\curveto(179.56542969,444.74921875)(179.20214844,444.60273438)(178.91308594,444.30976563)
\curveto(178.62402344,444.02070313)(178.47949219,443.55585938)(178.47949219,442.91523438)
\lineto(178.47949219,439.3)
\lineto(177.42480469,439.3)
\lineto(177.42480469,443.34296875)
\curveto(177.42480469,443.81171875)(177.33886719,444.16328125)(177.16699219,444.39765625)
\curveto(176.99511719,444.63203125)(176.71386719,444.74921875)(176.32324219,444.74921875)
\curveto(176.02636719,444.74921875)(175.75097656,444.67109375)(175.49707031,444.51484375)
\curveto(175.24707031,444.35859375)(175.06542969,444.13007813)(174.95214844,443.82929688)
\curveto(174.83886719,443.52851563)(174.78222656,443.09492188)(174.78222656,442.52851563)
\lineto(174.78222656,439.3)
\closepath
}
}
{
\newrgbcolor{curcolor}{0 0 0}
\pscustom[linestyle=none,fillstyle=solid,fillcolor=curcolor]
{
\newpath
\moveto(187.98339844,441.30390625)
\lineto(189.07324219,441.16914063)
\curveto(188.90136719,440.53242188)(188.58300781,440.03828125)(188.11816406,439.68671875)
\curveto(187.65332031,439.33515625)(187.05957031,439.159375)(186.33691406,439.159375)
\curveto(185.42675781,439.159375)(184.70410156,439.43867188)(184.16894531,439.99726563)
\curveto(183.63769531,440.55976563)(183.37207031,441.346875)(183.37207031,442.35859375)
\curveto(183.37207031,443.40546875)(183.64160156,444.21796875)(184.18066406,444.79609375)
\curveto(184.71972656,445.37421875)(185.41894531,445.66328125)(186.27832031,445.66328125)
\curveto(187.11035156,445.66328125)(187.79003906,445.38007813)(188.31738281,444.81367188)
\curveto(188.84472656,444.24726563)(189.10839844,443.45039063)(189.10839844,442.42304688)
\curveto(189.10839844,442.36054688)(189.10644531,442.26679688)(189.10253906,442.14179688)
\lineto(184.46191406,442.14179688)
\curveto(184.50097656,441.45820313)(184.69433594,440.93476563)(185.04199219,440.57148438)
\curveto(185.38964844,440.20820313)(185.82324219,440.0265625)(186.34277344,440.0265625)
\curveto(186.72949219,440.0265625)(187.05957031,440.128125)(187.33300781,440.33125)
\curveto(187.60644531,440.534375)(187.82324219,440.85859375)(187.98339844,441.30390625)
\closepath
\moveto(184.52050781,443.00898438)
\lineto(187.99511719,443.00898438)
\curveto(187.94824219,443.53242188)(187.81542969,443.925)(187.59667969,444.18671875)
\curveto(187.26074219,444.59296875)(186.82519531,444.79609375)(186.29003906,444.79609375)
\curveto(185.80566406,444.79609375)(185.39746094,444.63398438)(185.06542969,444.30976563)
\curveto(184.73730469,443.98554688)(184.55566406,443.55195313)(184.52050781,443.00898438)
\closepath
}
}
{
\newrgbcolor{curcolor}{0 0 0}
\pscustom[linestyle=none,fillstyle=solid,fillcolor=curcolor]
{
\newpath
\moveto(193.86621094,439.3)
\lineto(193.86621094,447.88984375)
\lineto(197.10644531,447.88984375)
\curveto(197.67675781,447.88984375)(198.11230469,447.8625)(198.41308594,447.8078125)
\curveto(198.83496094,447.7375)(199.18847656,447.60273438)(199.47363281,447.40351563)
\curveto(199.75878906,447.20820313)(199.98730469,446.9328125)(200.15917969,446.57734375)
\curveto(200.33496094,446.221875)(200.42285156,445.83125)(200.42285156,445.40546875)
\curveto(200.42285156,444.675)(200.19042969,444.05585938)(199.72558594,443.54804688)
\curveto(199.26074219,443.04414063)(198.42089844,442.7921875)(197.20605469,442.7921875)
\lineto(195.00292969,442.7921875)
\lineto(195.00292969,439.3)
\closepath
\moveto(195.00292969,443.80585938)
\lineto(197.22363281,443.80585938)
\curveto(197.95800781,443.80585938)(198.47949219,443.94257813)(198.78808594,444.21601563)
\curveto(199.09667969,444.48945313)(199.25097656,444.87421875)(199.25097656,445.3703125)
\curveto(199.25097656,445.7296875)(199.15917969,446.03632813)(198.97558594,446.29023438)
\curveto(198.79589844,446.54804688)(198.55761719,446.71796875)(198.26074219,446.8)
\curveto(198.06933594,446.85078125)(197.71582031,446.87617188)(197.20019531,446.87617188)
\lineto(195.00292969,446.87617188)
\closepath
}
}
{
\newrgbcolor{curcolor}{0 0 0}
\pscustom[linestyle=none,fillstyle=solid,fillcolor=curcolor]
{
\newpath
\moveto(205.99511719,441.30390625)
\lineto(207.08496094,441.16914063)
\curveto(206.91308594,440.53242188)(206.59472656,440.03828125)(206.12988281,439.68671875)
\curveto(205.66503906,439.33515625)(205.07128906,439.159375)(204.34863281,439.159375)
\curveto(203.43847656,439.159375)(202.71582031,439.43867188)(202.18066406,439.99726563)
\curveto(201.64941406,440.55976563)(201.38378906,441.346875)(201.38378906,442.35859375)
\curveto(201.38378906,443.40546875)(201.65332031,444.21796875)(202.19238281,444.79609375)
\curveto(202.73144531,445.37421875)(203.43066406,445.66328125)(204.29003906,445.66328125)
\curveto(205.12207031,445.66328125)(205.80175781,445.38007813)(206.32910156,444.81367188)
\curveto(206.85644531,444.24726563)(207.12011719,443.45039063)(207.12011719,442.42304688)
\curveto(207.12011719,442.36054688)(207.11816406,442.26679688)(207.11425781,442.14179688)
\lineto(202.47363281,442.14179688)
\curveto(202.51269531,441.45820313)(202.70605469,440.93476563)(203.05371094,440.57148438)
\curveto(203.40136719,440.20820313)(203.83496094,440.0265625)(204.35449219,440.0265625)
\curveto(204.74121094,440.0265625)(205.07128906,440.128125)(205.34472656,440.33125)
\curveto(205.61816406,440.534375)(205.83496094,440.85859375)(205.99511719,441.30390625)
\closepath
\moveto(202.53222656,443.00898438)
\lineto(206.00683594,443.00898438)
\curveto(205.95996094,443.53242188)(205.82714844,443.925)(205.60839844,444.18671875)
\curveto(205.27246094,444.59296875)(204.83691406,444.79609375)(204.30175781,444.79609375)
\curveto(203.81738281,444.79609375)(203.40917969,444.63398438)(203.07714844,444.30976563)
\curveto(202.74902344,443.98554688)(202.56738281,443.55195313)(202.53222656,443.00898438)
\closepath
}
}
{
\newrgbcolor{curcolor}{0 0 0}
\pscustom[linestyle=none,fillstyle=solid,fillcolor=curcolor]
{
\newpath
\moveto(208.39746094,439.3)
\lineto(208.39746094,445.52265625)
\lineto(209.34667969,445.52265625)
\lineto(209.34667969,444.57929688)
\curveto(209.58886719,445.02070313)(209.81152344,445.31171875)(210.01464844,445.45234375)
\curveto(210.22167969,445.59296875)(210.44824219,445.66328125)(210.69433594,445.66328125)
\curveto(211.04980469,445.66328125)(211.41113281,445.55)(211.77832031,445.3234375)
\lineto(211.41503906,444.34492188)
\curveto(211.15722656,444.49726563)(210.89941406,444.5734375)(210.64160156,444.5734375)
\curveto(210.41113281,444.5734375)(210.20410156,444.503125)(210.02050781,444.3625)
\curveto(209.83691406,444.22578125)(209.70605469,444.034375)(209.62792969,443.78828125)
\curveto(209.51074219,443.41328125)(209.45214844,443.003125)(209.45214844,442.5578125)
\lineto(209.45214844,439.3)
\closepath
}
}
{
\newrgbcolor{curcolor}{0 0 0}
\pscustom[linestyle=none,fillstyle=solid,fillcolor=curcolor]
{
\newpath
\moveto(216.46582031,441.57929688)
\lineto(217.50292969,441.44453125)
\curveto(217.38964844,440.7296875)(217.09863281,440.16914063)(216.62988281,439.76289063)
\curveto(216.16503906,439.36054688)(215.59277344,439.159375)(214.91308594,439.159375)
\curveto(214.06152344,439.159375)(213.37597656,439.43671875)(212.85644531,439.99140625)
\curveto(212.34082031,440.55)(212.08300781,441.34882813)(212.08300781,442.38789063)
\curveto(212.08300781,443.05976563)(212.19433594,443.64765625)(212.41699219,444.1515625)
\curveto(212.63964844,444.65546875)(212.97753906,445.03242188)(213.43066406,445.28242188)
\curveto(213.88769531,445.53632813)(214.38378906,445.66328125)(214.91894531,445.66328125)
\curveto(215.59472656,445.66328125)(216.14746094,445.49140625)(216.57714844,445.14765625)
\curveto(217.00683594,444.8078125)(217.28222656,444.3234375)(217.40332031,443.69453125)
\lineto(216.37792969,443.53632813)
\curveto(216.28027344,443.95429688)(216.10644531,444.26875)(215.85644531,444.4796875)
\curveto(215.61035156,444.690625)(215.31152344,444.79609375)(214.95996094,444.79609375)
\curveto(214.42871094,444.79609375)(213.99707031,444.6046875)(213.66503906,444.221875)
\curveto(213.33300781,443.84296875)(213.16699219,443.24140625)(213.16699219,442.4171875)
\curveto(213.16699219,441.58125)(213.32714844,440.97382813)(213.64746094,440.59492188)
\curveto(213.96777344,440.21601563)(214.38574219,440.0265625)(214.90136719,440.0265625)
\curveto(215.31542969,440.0265625)(215.66113281,440.15351563)(215.93847656,440.40742188)
\curveto(216.21582031,440.66132813)(216.39160156,441.05195313)(216.46582031,441.57929688)
\closepath
}
}
{
\newrgbcolor{curcolor}{0 0 0}
\pscustom[linestyle=none,fillstyle=solid,fillcolor=curcolor]
{
\newpath
\moveto(222.66503906,441.30390625)
\lineto(223.75488281,441.16914063)
\curveto(223.58300781,440.53242188)(223.26464844,440.03828125)(222.79980469,439.68671875)
\curveto(222.33496094,439.33515625)(221.74121094,439.159375)(221.01855469,439.159375)
\curveto(220.10839844,439.159375)(219.38574219,439.43867188)(218.85058594,439.99726563)
\curveto(218.31933594,440.55976563)(218.05371094,441.346875)(218.05371094,442.35859375)
\curveto(218.05371094,443.40546875)(218.32324219,444.21796875)(218.86230469,444.79609375)
\curveto(219.40136719,445.37421875)(220.10058594,445.66328125)(220.95996094,445.66328125)
\curveto(221.79199219,445.66328125)(222.47167969,445.38007813)(222.99902344,444.81367188)
\curveto(223.52636719,444.24726563)(223.79003906,443.45039063)(223.79003906,442.42304688)
\curveto(223.79003906,442.36054688)(223.78808594,442.26679688)(223.78417969,442.14179688)
\lineto(219.14355469,442.14179688)
\curveto(219.18261719,441.45820313)(219.37597656,440.93476563)(219.72363281,440.57148438)
\curveto(220.07128906,440.20820313)(220.50488281,440.0265625)(221.02441406,440.0265625)
\curveto(221.41113281,440.0265625)(221.74121094,440.128125)(222.01464844,440.33125)
\curveto(222.28808594,440.534375)(222.50488281,440.85859375)(222.66503906,441.30390625)
\closepath
\moveto(219.20214844,443.00898438)
\lineto(222.67675781,443.00898438)
\curveto(222.62988281,443.53242188)(222.49707031,443.925)(222.27832031,444.18671875)
\curveto(221.94238281,444.59296875)(221.50683594,444.79609375)(220.97167969,444.79609375)
\curveto(220.48730469,444.79609375)(220.07910156,444.63398438)(219.74707031,444.30976563)
\curveto(219.41894531,443.98554688)(219.23730469,443.55195313)(219.20214844,443.00898438)
\closepath
}
}
{
\newrgbcolor{curcolor}{0 0 0}
\pscustom[linestyle=none,fillstyle=solid,fillcolor=curcolor]
{
\newpath
\moveto(225.07910156,439.3)
\lineto(225.07910156,445.52265625)
\lineto(226.02832031,445.52265625)
\lineto(226.02832031,444.63789063)
\curveto(226.48535156,445.32148438)(227.14550781,445.66328125)(228.00878906,445.66328125)
\curveto(228.38378906,445.66328125)(228.72753906,445.59492188)(229.04003906,445.45820313)
\curveto(229.35644531,445.32539063)(229.59277344,445.14960938)(229.74902344,444.93085938)
\curveto(229.90527344,444.71210938)(230.01464844,444.45234375)(230.07714844,444.1515625)
\curveto(230.11621094,443.95625)(230.13574219,443.61445313)(230.13574219,443.12617188)
\lineto(230.13574219,439.3)
\lineto(229.08105469,439.3)
\lineto(229.08105469,443.08515625)
\curveto(229.08105469,443.51484375)(229.04003906,443.83515625)(228.95800781,444.04609375)
\curveto(228.87597656,444.2609375)(228.72949219,444.43085938)(228.51855469,444.55585938)
\curveto(228.31152344,444.68476563)(228.06738281,444.74921875)(227.78613281,444.74921875)
\curveto(227.33691406,444.74921875)(226.94824219,444.60664063)(226.62011719,444.32148438)
\curveto(226.29589844,444.03632813)(226.13378906,443.4953125)(226.13378906,442.6984375)
\lineto(226.13378906,439.3)
\closepath
}
}
{
\newrgbcolor{curcolor}{0 0 0}
\pscustom[linestyle=none,fillstyle=solid,fillcolor=curcolor]
{
\newpath
\moveto(234.05566406,440.24335938)
\lineto(234.20800781,439.31171875)
\curveto(233.91113281,439.24921875)(233.64550781,439.21796875)(233.41113281,439.21796875)
\curveto(233.02832031,439.21796875)(232.73144531,439.27851563)(232.52050781,439.39960938)
\curveto(232.30957031,439.52070313)(232.16113281,439.67890625)(232.07519531,439.87421875)
\curveto(231.98925781,440.0734375)(231.94628906,440.48945313)(231.94628906,441.12226563)
\lineto(231.94628906,444.70234375)
\lineto(231.17285156,444.70234375)
\lineto(231.17285156,445.52265625)
\lineto(231.94628906,445.52265625)
\lineto(231.94628906,447.06367188)
\lineto(232.99511719,447.69648438)
\lineto(232.99511719,445.52265625)
\lineto(234.05566406,445.52265625)
\lineto(234.05566406,444.70234375)
\lineto(232.99511719,444.70234375)
\lineto(232.99511719,441.06367188)
\curveto(232.99511719,440.76289063)(233.01269531,440.56953125)(233.04785156,440.48359375)
\curveto(233.08691406,440.39765625)(233.14746094,440.32929688)(233.22949219,440.27851563)
\curveto(233.31542969,440.22773438)(233.43652344,440.20234375)(233.59277344,440.20234375)
\curveto(233.70996094,440.20234375)(233.86425781,440.21601563)(234.05566406,440.24335938)
\closepath
}
}
{
\newrgbcolor{curcolor}{0 0 0}
\pscustom[linestyle=none,fillstyle=solid,fillcolor=curcolor]
{
\newpath
\moveto(239.14746094,440.06757813)
\curveto(238.75683594,439.73554688)(238.37988281,439.50117188)(238.01660156,439.36445313)
\curveto(237.65722656,439.22773438)(237.27050781,439.159375)(236.85644531,439.159375)
\curveto(236.17285156,439.159375)(235.64746094,439.32539063)(235.28027344,439.65742188)
\curveto(234.91308594,439.99335938)(234.72949219,440.42109375)(234.72949219,440.940625)
\curveto(234.72949219,441.2453125)(234.79785156,441.52265625)(234.93457031,441.77265625)
\curveto(235.07519531,442.0265625)(235.25683594,442.2296875)(235.47949219,442.38203125)
\curveto(235.70605469,442.534375)(235.95996094,442.64960938)(236.24121094,442.72773438)
\curveto(236.44824219,442.78242188)(236.76074219,442.83515625)(237.17871094,442.8859375)
\curveto(238.03027344,442.9875)(238.65722656,443.10859375)(239.05957031,443.24921875)
\curveto(239.06347656,443.39375)(239.06542969,443.48554688)(239.06542969,443.52460938)
\curveto(239.06542969,443.95429688)(238.96582031,444.25703125)(238.76660156,444.4328125)
\curveto(238.49707031,444.67109375)(238.09667969,444.79023438)(237.56542969,444.79023438)
\curveto(237.06933594,444.79023438)(236.70214844,444.70234375)(236.46386719,444.5265625)
\curveto(236.22949219,444.3546875)(236.05566406,444.04804688)(235.94238281,443.60664063)
\lineto(234.91113281,443.74726563)
\curveto(235.00488281,444.18867188)(235.15917969,444.54414063)(235.37402344,444.81367188)
\curveto(235.58886719,445.08710938)(235.89941406,445.29609375)(236.30566406,445.440625)
\curveto(236.71191406,445.5890625)(237.18261719,445.66328125)(237.71777344,445.66328125)
\curveto(238.24902344,445.66328125)(238.68066406,445.60078125)(239.01269531,445.47578125)
\curveto(239.34472656,445.35078125)(239.58886719,445.19257813)(239.74511719,445.00117188)
\curveto(239.90136719,444.81367188)(240.01074219,444.57539063)(240.07324219,444.28632813)
\curveto(240.10839844,444.10664063)(240.12597656,443.78242188)(240.12597656,443.31367188)
\lineto(240.12597656,441.90742188)
\curveto(240.12597656,440.92695313)(240.14746094,440.30585938)(240.19042969,440.04414063)
\curveto(240.23730469,439.78632813)(240.32714844,439.53828125)(240.45996094,439.3)
\lineto(239.35839844,439.3)
\curveto(239.24902344,439.51875)(239.17871094,439.77460938)(239.14746094,440.06757813)
\closepath
\moveto(239.05957031,442.42304688)
\curveto(238.67675781,442.26679688)(238.10253906,442.13398438)(237.33691406,442.02460938)
\curveto(236.90332031,441.96210938)(236.59667969,441.89179688)(236.41699219,441.81367188)
\curveto(236.23730469,441.73554688)(236.09863281,441.6203125)(236.00097656,441.46796875)
\curveto(235.90332031,441.31953125)(235.85449219,441.15351563)(235.85449219,440.96992188)
\curveto(235.85449219,440.68867188)(235.95996094,440.45429688)(236.17089844,440.26679688)
\curveto(236.38574219,440.07929688)(236.69824219,439.98554688)(237.10839844,439.98554688)
\curveto(237.51464844,439.98554688)(237.87597656,440.0734375)(238.19238281,440.24921875)
\curveto(238.50878906,440.42890625)(238.74121094,440.67304688)(238.88964844,440.98164063)
\curveto(239.00292969,441.21992188)(239.05957031,441.57148438)(239.05957031,442.03632813)
\closepath
}
}
{
\newrgbcolor{curcolor}{0 0 0}
\pscustom[linestyle=none,fillstyle=solid,fillcolor=curcolor]
{
\newpath
\moveto(241.56738281,438.784375)
\lineto(242.59277344,438.63203125)
\curveto(242.63574219,438.315625)(242.75488281,438.08515625)(242.95019531,437.940625)
\curveto(243.21191406,437.7453125)(243.56933594,437.64765625)(244.02246094,437.64765625)
\curveto(244.51074219,437.64765625)(244.88769531,437.7453125)(245.15332031,437.940625)
\curveto(245.41894531,438.1359375)(245.59863281,438.409375)(245.69238281,438.7609375)
\curveto(245.74707031,438.97578125)(245.77246094,439.42695313)(245.76855469,440.11445313)
\curveto(245.30761719,439.57148438)(244.73339844,439.3)(244.04589844,439.3)
\curveto(243.19042969,439.3)(242.52832031,439.60859375)(242.05957031,440.22578125)
\curveto(241.59082031,440.84296875)(241.35644531,441.58320313)(241.35644531,442.44648438)
\curveto(241.35644531,443.04023438)(241.46386719,443.58710938)(241.67871094,444.08710938)
\curveto(241.89355469,444.59101563)(242.20410156,444.9796875)(242.61035156,445.253125)
\curveto(243.02050781,445.5265625)(243.50097656,445.66328125)(244.05175781,445.66328125)
\curveto(244.78613281,445.66328125)(245.39160156,445.36640625)(245.86816406,444.77265625)
\lineto(245.86816406,445.52265625)
\lineto(246.84082031,445.52265625)
\lineto(246.84082031,440.14375)
\curveto(246.84082031,439.175)(246.74121094,438.48945313)(246.54199219,438.08710938)
\curveto(246.34667969,437.68085938)(246.03417969,437.36054688)(245.60449219,437.12617188)
\curveto(245.17871094,436.89179688)(244.65332031,436.77460938)(244.02832031,436.77460938)
\curveto(243.28613281,436.77460938)(242.68652344,436.94257813)(242.22949219,437.27851563)
\curveto(241.77246094,437.61054688)(241.55175781,438.1125)(241.56738281,438.784375)
\closepath
\moveto(242.44042969,442.52265625)
\curveto(242.44042969,441.70625)(242.60253906,441.11054688)(242.92675781,440.73554688)
\curveto(243.25097656,440.36054688)(243.65722656,440.17304688)(244.14550781,440.17304688)
\curveto(244.62988281,440.17304688)(245.03613281,440.35859375)(245.36425781,440.7296875)
\curveto(245.69238281,441.1046875)(245.85644531,441.690625)(245.85644531,442.4875)
\curveto(245.85644531,443.24921875)(245.68652344,443.8234375)(245.34667969,444.21015625)
\curveto(245.01074219,444.596875)(244.60449219,444.79023438)(244.12792969,444.79023438)
\curveto(243.65917969,444.79023438)(243.26074219,444.59882813)(242.93261719,444.21601563)
\curveto(242.60449219,443.83710938)(242.44042969,443.27265625)(242.44042969,442.52265625)
\closepath
}
}
{
\newrgbcolor{curcolor}{0 0 0}
\pscustom[linestyle=none,fillstyle=solid,fillcolor=curcolor]
{
\newpath
\moveto(252.69433594,441.30390625)
\lineto(253.78417969,441.16914063)
\curveto(253.61230469,440.53242188)(253.29394531,440.03828125)(252.82910156,439.68671875)
\curveto(252.36425781,439.33515625)(251.77050781,439.159375)(251.04785156,439.159375)
\curveto(250.13769531,439.159375)(249.41503906,439.43867188)(248.87988281,439.99726563)
\curveto(248.34863281,440.55976563)(248.08300781,441.346875)(248.08300781,442.35859375)
\curveto(248.08300781,443.40546875)(248.35253906,444.21796875)(248.89160156,444.79609375)
\curveto(249.43066406,445.37421875)(250.12988281,445.66328125)(250.98925781,445.66328125)
\curveto(251.82128906,445.66328125)(252.50097656,445.38007813)(253.02832031,444.81367188)
\curveto(253.55566406,444.24726563)(253.81933594,443.45039063)(253.81933594,442.42304688)
\curveto(253.81933594,442.36054688)(253.81738281,442.26679688)(253.81347656,442.14179688)
\lineto(249.17285156,442.14179688)
\curveto(249.21191406,441.45820313)(249.40527344,440.93476563)(249.75292969,440.57148438)
\curveto(250.10058594,440.20820313)(250.53417969,440.0265625)(251.05371094,440.0265625)
\curveto(251.44042969,440.0265625)(251.77050781,440.128125)(252.04394531,440.33125)
\curveto(252.31738281,440.534375)(252.53417969,440.85859375)(252.69433594,441.30390625)
\closepath
\moveto(249.23144531,443.00898438)
\lineto(252.70605469,443.00898438)
\curveto(252.65917969,443.53242188)(252.52636719,443.925)(252.30761719,444.18671875)
\curveto(251.97167969,444.59296875)(251.53613281,444.79609375)(251.00097656,444.79609375)
\curveto(250.51660156,444.79609375)(250.10839844,444.63398438)(249.77636719,444.30976563)
\curveto(249.44824219,443.98554688)(249.26660156,443.55195313)(249.23144531,443.00898438)
\closepath
}
}
{
\newrgbcolor{curcolor}{0 0 0}
\pscustom[linestyle=none,fillstyle=solid,fillcolor=curcolor]
{
\newpath
\moveto(258.77050781,439.3)
\lineto(258.77050781,447.88984375)
\lineto(259.90722656,447.88984375)
\lineto(259.90722656,439.3)
\closepath
}
}
{
\newrgbcolor{curcolor}{0 0 0}
\pscustom[linestyle=none,fillstyle=solid,fillcolor=curcolor]
{
\newpath
\moveto(261.77636719,439.3)
\lineto(261.77636719,445.52265625)
\lineto(262.71972656,445.52265625)
\lineto(262.71972656,444.64960938)
\curveto(262.91503906,444.95429688)(263.17480469,445.1984375)(263.49902344,445.38203125)
\curveto(263.82324219,445.56953125)(264.19238281,445.66328125)(264.60644531,445.66328125)
\curveto(265.06738281,445.66328125)(265.44433594,445.56757813)(265.73730469,445.37617188)
\curveto(266.03417969,445.18476563)(266.24316406,444.9171875)(266.36425781,444.5734375)
\curveto(266.85644531,445.3)(267.49707031,445.66328125)(268.28613281,445.66328125)
\curveto(268.90332031,445.66328125)(269.37792969,445.49140625)(269.70996094,445.14765625)
\curveto(270.04199219,444.8078125)(270.20800781,444.28242188)(270.20800781,443.57148438)
\lineto(270.20800781,439.3)
\lineto(269.15917969,439.3)
\lineto(269.15917969,443.21992188)
\curveto(269.15917969,443.64179688)(269.12402344,443.94453125)(269.05371094,444.128125)
\curveto(268.98730469,444.315625)(268.86425781,444.46601563)(268.68457031,444.57929688)
\curveto(268.50488281,444.69257813)(268.29394531,444.74921875)(268.05175781,444.74921875)
\curveto(267.61425781,444.74921875)(267.25097656,444.60273438)(266.96191406,444.30976563)
\curveto(266.67285156,444.02070313)(266.52832031,443.55585938)(266.52832031,442.91523438)
\lineto(266.52832031,439.3)
\lineto(265.47363281,439.3)
\lineto(265.47363281,443.34296875)
\curveto(265.47363281,443.81171875)(265.38769531,444.16328125)(265.21582031,444.39765625)
\curveto(265.04394531,444.63203125)(264.76269531,444.74921875)(264.37207031,444.74921875)
\curveto(264.07519531,444.74921875)(263.79980469,444.67109375)(263.54589844,444.51484375)
\curveto(263.29589844,444.35859375)(263.11425781,444.13007813)(263.00097656,443.82929688)
\curveto(262.88769531,443.52851563)(262.83105469,443.09492188)(262.83105469,442.52851563)
\lineto(262.83105469,439.3)
\closepath
}
}
{
\newrgbcolor{curcolor}{0 0 0}
\pscustom[linestyle=none,fillstyle=solid,fillcolor=curcolor]
{
\newpath
\moveto(271.77246094,436.91523438)
\lineto(271.77246094,445.52265625)
\lineto(272.73339844,445.52265625)
\lineto(272.73339844,444.7140625)
\curveto(272.95996094,445.03046875)(273.21582031,445.26679688)(273.50097656,445.42304688)
\curveto(273.78613281,445.58320313)(274.13183594,445.66328125)(274.53808594,445.66328125)
\curveto(275.06933594,445.66328125)(275.53808594,445.5265625)(275.94433594,445.253125)
\curveto(276.35058594,444.9796875)(276.65722656,444.59296875)(276.86425781,444.09296875)
\curveto(277.07128906,443.596875)(277.17480469,443.05195313)(277.17480469,442.45820313)
\curveto(277.17480469,441.82148438)(277.05957031,441.24726563)(276.82910156,440.73554688)
\curveto(276.60253906,440.22773438)(276.27050781,439.83710938)(275.83300781,439.56367188)
\curveto(275.39941406,439.29414063)(274.94238281,439.159375)(274.46191406,439.159375)
\curveto(274.11035156,439.159375)(273.79394531,439.23359375)(273.51269531,439.38203125)
\curveto(273.23535156,439.53046875)(273.00683594,439.71796875)(272.82714844,439.94453125)
\lineto(272.82714844,436.91523438)
\closepath
\moveto(272.72753906,442.37617188)
\curveto(272.72753906,441.57539063)(272.88964844,440.98359375)(273.21386719,440.60078125)
\curveto(273.53808594,440.21796875)(273.93066406,440.0265625)(274.39160156,440.0265625)
\curveto(274.86035156,440.0265625)(275.26074219,440.22382813)(275.59277344,440.61835938)
\curveto(275.92871094,441.01679688)(276.09667969,441.63203125)(276.09667969,442.4640625)
\curveto(276.09667969,443.25703125)(275.93261719,443.85078125)(275.60449219,444.2453125)
\curveto(275.28027344,444.63984375)(274.89160156,444.83710938)(274.43847656,444.83710938)
\curveto(273.98925781,444.83710938)(273.59082031,444.62617188)(273.24316406,444.20429688)
\curveto(272.89941406,443.78632813)(272.72753906,443.17695313)(272.72753906,442.37617188)
\closepath
}
}
{
\newrgbcolor{curcolor}{0 0 0}
\pscustom[linestyle=none,fillstyle=solid,fillcolor=curcolor]
{
\newpath
\moveto(278.43457031,439.3)
\lineto(278.43457031,445.52265625)
\lineto(279.38378906,445.52265625)
\lineto(279.38378906,444.57929688)
\curveto(279.62597656,445.02070313)(279.84863281,445.31171875)(280.05175781,445.45234375)
\curveto(280.25878906,445.59296875)(280.48535156,445.66328125)(280.73144531,445.66328125)
\curveto(281.08691406,445.66328125)(281.44824219,445.55)(281.81542969,445.3234375)
\lineto(281.45214844,444.34492188)
\curveto(281.19433594,444.49726563)(280.93652344,444.5734375)(280.67871094,444.5734375)
\curveto(280.44824219,444.5734375)(280.24121094,444.503125)(280.05761719,444.3625)
\curveto(279.87402344,444.22578125)(279.74316406,444.034375)(279.66503906,443.78828125)
\curveto(279.54785156,443.41328125)(279.48925781,443.003125)(279.48925781,442.5578125)
\lineto(279.48925781,439.3)
\closepath
}
}
{
\newrgbcolor{curcolor}{0 0 0}
\pscustom[linestyle=none,fillstyle=solid,fillcolor=curcolor]
{
\newpath
\moveto(282.04980469,442.41132813)
\curveto(282.04980469,443.56367188)(282.37011719,444.4171875)(283.01074219,444.971875)
\curveto(283.54589844,445.4328125)(284.19824219,445.66328125)(284.96777344,445.66328125)
\curveto(285.82324219,445.66328125)(286.52246094,445.38203125)(287.06542969,444.81953125)
\curveto(287.60839844,444.2609375)(287.87988281,443.4875)(287.87988281,442.49921875)
\curveto(287.87988281,441.6984375)(287.75878906,441.06757813)(287.51660156,440.60664063)
\curveto(287.27832031,440.14960938)(286.92871094,439.79414063)(286.46777344,439.54023438)
\curveto(286.01074219,439.28632813)(285.51074219,439.159375)(284.96777344,439.159375)
\curveto(284.09667969,439.159375)(283.39160156,439.43867188)(282.85253906,439.99726563)
\curveto(282.31738281,440.55585938)(282.04980469,441.36054688)(282.04980469,442.41132813)
\closepath
\moveto(283.13378906,442.41132813)
\curveto(283.13378906,441.61445313)(283.30761719,441.01679688)(283.65527344,440.61835938)
\curveto(284.00292969,440.22382813)(284.44042969,440.0265625)(284.96777344,440.0265625)
\curveto(285.49121094,440.0265625)(285.92675781,440.22578125)(286.27441406,440.62421875)
\curveto(286.62207031,441.02265625)(286.79589844,441.63007813)(286.79589844,442.44648438)
\curveto(286.79589844,443.21601563)(286.62011719,443.79804688)(286.26855469,444.19257813)
\curveto(285.92089844,444.59101563)(285.48730469,444.79023438)(284.96777344,444.79023438)
\curveto(284.44042969,444.79023438)(284.00292969,444.59296875)(283.65527344,444.1984375)
\curveto(283.30761719,443.80390625)(283.13378906,443.20820313)(283.13378906,442.41132813)
\closepath
}
}
{
\newrgbcolor{curcolor}{0 0 0}
\pscustom[linestyle=none,fillstyle=solid,fillcolor=curcolor]
{
\newpath
\moveto(290.84472656,439.3)
\lineto(288.47753906,445.52265625)
\lineto(289.59082031,445.52265625)
\lineto(290.92675781,441.79609375)
\curveto(291.07128906,441.39375)(291.20410156,440.97578125)(291.32519531,440.5421875)
\curveto(291.41894531,440.8703125)(291.54980469,441.26484375)(291.71777344,441.72578125)
\lineto(293.10058594,445.52265625)
\lineto(294.18457031,445.52265625)
\lineto(291.82910156,439.3)
\closepath
}
}
{
\newrgbcolor{curcolor}{0 0 0}
\pscustom[linestyle=none,fillstyle=solid,fillcolor=curcolor]
{
\newpath
\moveto(299.37597656,441.30390625)
\lineto(300.46582031,441.16914063)
\curveto(300.29394531,440.53242188)(299.97558594,440.03828125)(299.51074219,439.68671875)
\curveto(299.04589844,439.33515625)(298.45214844,439.159375)(297.72949219,439.159375)
\curveto(296.81933594,439.159375)(296.09667969,439.43867188)(295.56152344,439.99726563)
\curveto(295.03027344,440.55976563)(294.76464844,441.346875)(294.76464844,442.35859375)
\curveto(294.76464844,443.40546875)(295.03417969,444.21796875)(295.57324219,444.79609375)
\curveto(296.11230469,445.37421875)(296.81152344,445.66328125)(297.67089844,445.66328125)
\curveto(298.50292969,445.66328125)(299.18261719,445.38007813)(299.70996094,444.81367188)
\curveto(300.23730469,444.24726563)(300.50097656,443.45039063)(300.50097656,442.42304688)
\curveto(300.50097656,442.36054688)(300.49902344,442.26679688)(300.49511719,442.14179688)
\lineto(295.85449219,442.14179688)
\curveto(295.89355469,441.45820313)(296.08691406,440.93476563)(296.43457031,440.57148438)
\curveto(296.78222656,440.20820313)(297.21582031,440.0265625)(297.73535156,440.0265625)
\curveto(298.12207031,440.0265625)(298.45214844,440.128125)(298.72558594,440.33125)
\curveto(298.99902344,440.534375)(299.21582031,440.85859375)(299.37597656,441.30390625)
\closepath
\moveto(295.91308594,443.00898438)
\lineto(299.38769531,443.00898438)
\curveto(299.34082031,443.53242188)(299.20800781,443.925)(298.98925781,444.18671875)
\curveto(298.65332031,444.59296875)(298.21777344,444.79609375)(297.68261719,444.79609375)
\curveto(297.19824219,444.79609375)(296.79003906,444.63398438)(296.45800781,444.30976563)
\curveto(296.12988281,443.98554688)(295.94824219,443.55195313)(295.91308594,443.00898438)
\closepath
}
}
{
\newrgbcolor{curcolor}{0 0 0}
\pscustom[linestyle=none,fillstyle=solid,fillcolor=curcolor]
{
\newpath
\moveto(301.79003906,439.3)
\lineto(301.79003906,445.52265625)
\lineto(302.73339844,445.52265625)
\lineto(302.73339844,444.64960938)
\curveto(302.92871094,444.95429688)(303.18847656,445.1984375)(303.51269531,445.38203125)
\curveto(303.83691406,445.56953125)(304.20605469,445.66328125)(304.62011719,445.66328125)
\curveto(305.08105469,445.66328125)(305.45800781,445.56757813)(305.75097656,445.37617188)
\curveto(306.04785156,445.18476563)(306.25683594,444.9171875)(306.37792969,444.5734375)
\curveto(306.87011719,445.3)(307.51074219,445.66328125)(308.29980469,445.66328125)
\curveto(308.91699219,445.66328125)(309.39160156,445.49140625)(309.72363281,445.14765625)
\curveto(310.05566406,444.8078125)(310.22167969,444.28242188)(310.22167969,443.57148438)
\lineto(310.22167969,439.3)
\lineto(309.17285156,439.3)
\lineto(309.17285156,443.21992188)
\curveto(309.17285156,443.64179688)(309.13769531,443.94453125)(309.06738281,444.128125)
\curveto(309.00097656,444.315625)(308.87792969,444.46601563)(308.69824219,444.57929688)
\curveto(308.51855469,444.69257813)(308.30761719,444.74921875)(308.06542969,444.74921875)
\curveto(307.62792969,444.74921875)(307.26464844,444.60273438)(306.97558594,444.30976563)
\curveto(306.68652344,444.02070313)(306.54199219,443.55585938)(306.54199219,442.91523438)
\lineto(306.54199219,439.3)
\lineto(305.48730469,439.3)
\lineto(305.48730469,443.34296875)
\curveto(305.48730469,443.81171875)(305.40136719,444.16328125)(305.22949219,444.39765625)
\curveto(305.05761719,444.63203125)(304.77636719,444.74921875)(304.38574219,444.74921875)
\curveto(304.08886719,444.74921875)(303.81347656,444.67109375)(303.55957031,444.51484375)
\curveto(303.30957031,444.35859375)(303.12792969,444.13007813)(303.01464844,443.82929688)
\curveto(302.90136719,443.52851563)(302.84472656,443.09492188)(302.84472656,442.52851563)
\lineto(302.84472656,439.3)
\closepath
}
}
{
\newrgbcolor{curcolor}{0 0 0}
\pscustom[linestyle=none,fillstyle=solid,fillcolor=curcolor]
{
\newpath
\moveto(316.04589844,441.30390625)
\lineto(317.13574219,441.16914063)
\curveto(316.96386719,440.53242188)(316.64550781,440.03828125)(316.18066406,439.68671875)
\curveto(315.71582031,439.33515625)(315.12207031,439.159375)(314.39941406,439.159375)
\curveto(313.48925781,439.159375)(312.76660156,439.43867188)(312.23144531,439.99726563)
\curveto(311.70019531,440.55976563)(311.43457031,441.346875)(311.43457031,442.35859375)
\curveto(311.43457031,443.40546875)(311.70410156,444.21796875)(312.24316406,444.79609375)
\curveto(312.78222656,445.37421875)(313.48144531,445.66328125)(314.34082031,445.66328125)
\curveto(315.17285156,445.66328125)(315.85253906,445.38007813)(316.37988281,444.81367188)
\curveto(316.90722656,444.24726563)(317.17089844,443.45039063)(317.17089844,442.42304688)
\curveto(317.17089844,442.36054688)(317.16894531,442.26679688)(317.16503906,442.14179688)
\lineto(312.52441406,442.14179688)
\curveto(312.56347656,441.45820313)(312.75683594,440.93476563)(313.10449219,440.57148438)
\curveto(313.45214844,440.20820313)(313.88574219,440.0265625)(314.40527344,440.0265625)
\curveto(314.79199219,440.0265625)(315.12207031,440.128125)(315.39550781,440.33125)
\curveto(315.66894531,440.534375)(315.88574219,440.85859375)(316.04589844,441.30390625)
\closepath
\moveto(312.58300781,443.00898438)
\lineto(316.05761719,443.00898438)
\curveto(316.01074219,443.53242188)(315.87792969,443.925)(315.65917969,444.18671875)
\curveto(315.32324219,444.59296875)(314.88769531,444.79609375)(314.35253906,444.79609375)
\curveto(313.86816406,444.79609375)(313.45996094,444.63398438)(313.12792969,444.30976563)
\curveto(312.79980469,443.98554688)(312.61816406,443.55195313)(312.58300781,443.00898438)
\closepath
}
}
{
\newrgbcolor{curcolor}{0 0 0}
\pscustom[linestyle=none,fillstyle=solid,fillcolor=curcolor]
{
\newpath
\moveto(318.45996094,439.3)
\lineto(318.45996094,445.52265625)
\lineto(319.40917969,445.52265625)
\lineto(319.40917969,444.63789063)
\curveto(319.86621094,445.32148438)(320.52636719,445.66328125)(321.38964844,445.66328125)
\curveto(321.76464844,445.66328125)(322.10839844,445.59492188)(322.42089844,445.45820313)
\curveto(322.73730469,445.32539063)(322.97363281,445.14960938)(323.12988281,444.93085938)
\curveto(323.28613281,444.71210938)(323.39550781,444.45234375)(323.45800781,444.1515625)
\curveto(323.49707031,443.95625)(323.51660156,443.61445313)(323.51660156,443.12617188)
\lineto(323.51660156,439.3)
\lineto(322.46191406,439.3)
\lineto(322.46191406,443.08515625)
\curveto(322.46191406,443.51484375)(322.42089844,443.83515625)(322.33886719,444.04609375)
\curveto(322.25683594,444.2609375)(322.11035156,444.43085938)(321.89941406,444.55585938)
\curveto(321.69238281,444.68476563)(321.44824219,444.74921875)(321.16699219,444.74921875)
\curveto(320.71777344,444.74921875)(320.32910156,444.60664063)(320.00097656,444.32148438)
\curveto(319.67675781,444.03632813)(319.51464844,443.4953125)(319.51464844,442.6984375)
\lineto(319.51464844,439.3)
\closepath
}
}
{
\newrgbcolor{curcolor}{0 0 0}
\pscustom[linestyle=none,fillstyle=solid,fillcolor=curcolor]
{
\newpath
\moveto(327.43652344,440.24335938)
\lineto(327.58886719,439.31171875)
\curveto(327.29199219,439.24921875)(327.02636719,439.21796875)(326.79199219,439.21796875)
\curveto(326.40917969,439.21796875)(326.11230469,439.27851563)(325.90136719,439.39960938)
\curveto(325.69042969,439.52070313)(325.54199219,439.67890625)(325.45605469,439.87421875)
\curveto(325.37011719,440.0734375)(325.32714844,440.48945313)(325.32714844,441.12226563)
\lineto(325.32714844,444.70234375)
\lineto(324.55371094,444.70234375)
\lineto(324.55371094,445.52265625)
\lineto(325.32714844,445.52265625)
\lineto(325.32714844,447.06367188)
\lineto(326.37597656,447.69648438)
\lineto(326.37597656,445.52265625)
\lineto(327.43652344,445.52265625)
\lineto(327.43652344,444.70234375)
\lineto(326.37597656,444.70234375)
\lineto(326.37597656,441.06367188)
\curveto(326.37597656,440.76289063)(326.39355469,440.56953125)(326.42871094,440.48359375)
\curveto(326.46777344,440.39765625)(326.52832031,440.32929688)(326.61035156,440.27851563)
\curveto(326.69628906,440.22773438)(326.81738281,440.20234375)(326.97363281,440.20234375)
\curveto(327.09082031,440.20234375)(327.24511719,440.21601563)(327.43652344,440.24335938)
\closepath
}
}
{
\newrgbcolor{curcolor}{0 0 0}
\pscustom[linestyle=none,fillstyle=solid,fillcolor=curcolor]
{
\newpath
\moveto(331.88964844,439.3)
\lineto(331.88964844,447.88984375)
\lineto(335.11230469,447.88984375)
\curveto(335.76855469,447.88984375)(336.29394531,447.80195313)(336.68847656,447.62617188)
\curveto(337.08691406,447.45429688)(337.39746094,447.18671875)(337.62011719,446.8234375)
\curveto(337.84667969,446.4640625)(337.95996094,446.08710938)(337.95996094,445.69257813)
\curveto(337.95996094,445.32539063)(337.86035156,444.9796875)(337.66113281,444.65546875)
\curveto(337.46191406,444.33125)(337.16113281,444.06953125)(336.75878906,443.8703125)
\curveto(337.27832031,443.71796875)(337.67675781,443.45820313)(337.95410156,443.09101563)
\curveto(338.23535156,442.72382813)(338.37597656,442.29023438)(338.37597656,441.79023438)
\curveto(338.37597656,441.38789063)(338.29003906,441.01289063)(338.11816406,440.66523438)
\curveto(337.95019531,440.32148438)(337.74121094,440.05585938)(337.49121094,439.86835938)
\curveto(337.24121094,439.68085938)(336.92675781,439.53828125)(336.54785156,439.440625)
\curveto(336.17285156,439.346875)(335.71191406,439.3)(335.16503906,439.3)
\closepath
\moveto(333.02636719,444.28046875)
\lineto(334.88378906,444.28046875)
\curveto(335.38769531,444.28046875)(335.74902344,444.31367188)(335.96777344,444.38007813)
\curveto(336.25683594,444.46601563)(336.47363281,444.60859375)(336.61816406,444.8078125)
\curveto(336.76660156,445.00703125)(336.84082031,445.25703125)(336.84082031,445.5578125)
\curveto(336.84082031,445.84296875)(336.77246094,446.09296875)(336.63574219,446.3078125)
\curveto(336.49902344,446.5265625)(336.30371094,446.675)(336.04980469,446.753125)
\curveto(335.79589844,446.83515625)(335.36035156,446.87617188)(334.74316406,446.87617188)
\lineto(333.02636719,446.87617188)
\closepath
\moveto(333.02636719,440.31367188)
\lineto(335.16503906,440.31367188)
\curveto(335.53222656,440.31367188)(335.79003906,440.32734375)(335.93847656,440.3546875)
\curveto(336.20019531,440.4015625)(336.41894531,440.4796875)(336.59472656,440.5890625)
\curveto(336.77050781,440.6984375)(336.91503906,440.85664063)(337.02832031,441.06367188)
\curveto(337.14160156,441.27460938)(337.19824219,441.51679688)(337.19824219,441.79023438)
\curveto(337.19824219,442.11054688)(337.11621094,442.38789063)(336.95214844,442.62226563)
\curveto(336.78808594,442.86054688)(336.55957031,443.0265625)(336.26660156,443.1203125)
\curveto(335.97753906,443.21796875)(335.55957031,443.26679688)(335.01269531,443.26679688)
\lineto(333.02636719,443.26679688)
\closepath
}
}
{
\newrgbcolor{curcolor}{0 0 0}
\pscustom[linestyle=none,fillstyle=solid,fillcolor=curcolor]
{
\newpath
\moveto(343.86621094,440.06757813)
\curveto(343.47558594,439.73554688)(343.09863281,439.50117188)(342.73535156,439.36445313)
\curveto(342.37597656,439.22773438)(341.98925781,439.159375)(341.57519531,439.159375)
\curveto(340.89160156,439.159375)(340.36621094,439.32539063)(339.99902344,439.65742188)
\curveto(339.63183594,439.99335938)(339.44824219,440.42109375)(339.44824219,440.940625)
\curveto(339.44824219,441.2453125)(339.51660156,441.52265625)(339.65332031,441.77265625)
\curveto(339.79394531,442.0265625)(339.97558594,442.2296875)(340.19824219,442.38203125)
\curveto(340.42480469,442.534375)(340.67871094,442.64960938)(340.95996094,442.72773438)
\curveto(341.16699219,442.78242188)(341.47949219,442.83515625)(341.89746094,442.8859375)
\curveto(342.74902344,442.9875)(343.37597656,443.10859375)(343.77832031,443.24921875)
\curveto(343.78222656,443.39375)(343.78417969,443.48554688)(343.78417969,443.52460938)
\curveto(343.78417969,443.95429688)(343.68457031,444.25703125)(343.48535156,444.4328125)
\curveto(343.21582031,444.67109375)(342.81542969,444.79023438)(342.28417969,444.79023438)
\curveto(341.78808594,444.79023438)(341.42089844,444.70234375)(341.18261719,444.5265625)
\curveto(340.94824219,444.3546875)(340.77441406,444.04804688)(340.66113281,443.60664063)
\lineto(339.62988281,443.74726563)
\curveto(339.72363281,444.18867188)(339.87792969,444.54414063)(340.09277344,444.81367188)
\curveto(340.30761719,445.08710938)(340.61816406,445.29609375)(341.02441406,445.440625)
\curveto(341.43066406,445.5890625)(341.90136719,445.66328125)(342.43652344,445.66328125)
\curveto(342.96777344,445.66328125)(343.39941406,445.60078125)(343.73144531,445.47578125)
\curveto(344.06347656,445.35078125)(344.30761719,445.19257813)(344.46386719,445.00117188)
\curveto(344.62011719,444.81367188)(344.72949219,444.57539063)(344.79199219,444.28632813)
\curveto(344.82714844,444.10664063)(344.84472656,443.78242188)(344.84472656,443.31367188)
\lineto(344.84472656,441.90742188)
\curveto(344.84472656,440.92695313)(344.86621094,440.30585938)(344.90917969,440.04414063)
\curveto(344.95605469,439.78632813)(345.04589844,439.53828125)(345.17871094,439.3)
\lineto(344.07714844,439.3)
\curveto(343.96777344,439.51875)(343.89746094,439.77460938)(343.86621094,440.06757813)
\closepath
\moveto(343.77832031,442.42304688)
\curveto(343.39550781,442.26679688)(342.82128906,442.13398438)(342.05566406,442.02460938)
\curveto(341.62207031,441.96210938)(341.31542969,441.89179688)(341.13574219,441.81367188)
\curveto(340.95605469,441.73554688)(340.81738281,441.6203125)(340.71972656,441.46796875)
\curveto(340.62207031,441.31953125)(340.57324219,441.15351563)(340.57324219,440.96992188)
\curveto(340.57324219,440.68867188)(340.67871094,440.45429688)(340.88964844,440.26679688)
\curveto(341.10449219,440.07929688)(341.41699219,439.98554688)(341.82714844,439.98554688)
\curveto(342.23339844,439.98554688)(342.59472656,440.0734375)(342.91113281,440.24921875)
\curveto(343.22753906,440.42890625)(343.45996094,440.67304688)(343.60839844,440.98164063)
\curveto(343.72167969,441.21992188)(343.77832031,441.57148438)(343.77832031,442.03632813)
\closepath
}
}
{
\newrgbcolor{curcolor}{0 0 0}
\pscustom[linestyle=none,fillstyle=solid,fillcolor=curcolor]
{
\newpath
\moveto(346.05761719,441.15742188)
\lineto(347.10058594,441.32148438)
\curveto(347.15917969,440.90351563)(347.32128906,440.58320313)(347.58691406,440.36054688)
\curveto(347.85644531,440.13789063)(348.23144531,440.0265625)(348.71191406,440.0265625)
\curveto(349.19628906,440.0265625)(349.55566406,440.12421875)(349.79003906,440.31953125)
\curveto(350.02441406,440.51875)(350.14160156,440.75117188)(350.14160156,441.01679688)
\curveto(350.14160156,441.25507813)(350.03808594,441.44257813)(349.83105469,441.57929688)
\curveto(349.68652344,441.67304688)(349.32714844,441.7921875)(348.75292969,441.93671875)
\curveto(347.97949219,442.13203125)(347.44238281,442.3)(347.14160156,442.440625)
\curveto(346.84472656,442.58515625)(346.61816406,442.78242188)(346.46191406,443.03242188)
\curveto(346.30957031,443.28632813)(346.23339844,443.565625)(346.23339844,443.8703125)
\curveto(346.23339844,444.14765625)(346.29589844,444.40351563)(346.42089844,444.63789063)
\curveto(346.54980469,444.87617188)(346.72363281,445.0734375)(346.94238281,445.2296875)
\curveto(347.10644531,445.35078125)(347.32910156,445.45234375)(347.61035156,445.534375)
\curveto(347.89550781,445.6203125)(348.20019531,445.66328125)(348.52441406,445.66328125)
\curveto(349.01269531,445.66328125)(349.44042969,445.59296875)(349.80761719,445.45234375)
\curveto(350.17871094,445.31171875)(350.45214844,445.1203125)(350.62792969,444.878125)
\curveto(350.80371094,444.63984375)(350.92480469,444.31953125)(350.99121094,443.9171875)
\lineto(349.95996094,443.7765625)
\curveto(349.91308594,444.096875)(349.77636719,444.346875)(349.54980469,444.5265625)
\curveto(349.32714844,444.70625)(349.01074219,444.79609375)(348.60058594,444.79609375)
\curveto(348.11621094,444.79609375)(347.77050781,444.71601563)(347.56347656,444.55585938)
\curveto(347.35644531,444.39570313)(347.25292969,444.20820313)(347.25292969,443.99335938)
\curveto(347.25292969,443.85664063)(347.29589844,443.73359375)(347.38183594,443.62421875)
\curveto(347.46777344,443.5109375)(347.60253906,443.4171875)(347.78613281,443.34296875)
\curveto(347.89160156,443.30390625)(348.20214844,443.2140625)(348.71777344,443.0734375)
\curveto(349.46386719,442.87421875)(349.98339844,442.71015625)(350.27636719,442.58125)
\curveto(350.57324219,442.45625)(350.80566406,442.27265625)(350.97363281,442.03046875)
\curveto(351.14160156,441.78828125)(351.22558594,441.4875)(351.22558594,441.128125)
\curveto(351.22558594,440.7765625)(351.12207031,440.44453125)(350.91503906,440.13203125)
\curveto(350.71191406,439.8234375)(350.41699219,439.58320313)(350.03027344,439.41132813)
\curveto(349.64355469,439.24335938)(349.20605469,439.159375)(348.71777344,439.159375)
\curveto(347.90917969,439.159375)(347.29199219,439.32734375)(346.86621094,439.66328125)
\curveto(346.44433594,439.99921875)(346.17480469,440.49726563)(346.05761719,441.15742188)
\closepath
}
}
{
\newrgbcolor{curcolor}{0 0 0}
\pscustom[linestyle=none,fillstyle=solid,fillcolor=curcolor]
{
\newpath
\moveto(356.73925781,441.30390625)
\lineto(357.82910156,441.16914063)
\curveto(357.65722656,440.53242188)(357.33886719,440.03828125)(356.87402344,439.68671875)
\curveto(356.40917969,439.33515625)(355.81542969,439.159375)(355.09277344,439.159375)
\curveto(354.18261719,439.159375)(353.45996094,439.43867188)(352.92480469,439.99726563)
\curveto(352.39355469,440.55976563)(352.12792969,441.346875)(352.12792969,442.35859375)
\curveto(352.12792969,443.40546875)(352.39746094,444.21796875)(352.93652344,444.79609375)
\curveto(353.47558594,445.37421875)(354.17480469,445.66328125)(355.03417969,445.66328125)
\curveto(355.86621094,445.66328125)(356.54589844,445.38007813)(357.07324219,444.81367188)
\curveto(357.60058594,444.24726563)(357.86425781,443.45039063)(357.86425781,442.42304688)
\curveto(357.86425781,442.36054688)(357.86230469,442.26679688)(357.85839844,442.14179688)
\lineto(353.21777344,442.14179688)
\curveto(353.25683594,441.45820313)(353.45019531,440.93476563)(353.79785156,440.57148438)
\curveto(354.14550781,440.20820313)(354.57910156,440.0265625)(355.09863281,440.0265625)
\curveto(355.48535156,440.0265625)(355.81542969,440.128125)(356.08886719,440.33125)
\curveto(356.36230469,440.534375)(356.57910156,440.85859375)(356.73925781,441.30390625)
\closepath
\moveto(353.27636719,443.00898438)
\lineto(356.75097656,443.00898438)
\curveto(356.70410156,443.53242188)(356.57128906,443.925)(356.35253906,444.18671875)
\curveto(356.01660156,444.59296875)(355.58105469,444.79609375)(355.04589844,444.79609375)
\curveto(354.56152344,444.79609375)(354.15332031,444.63398438)(353.82128906,444.30976563)
\curveto(353.49316406,443.98554688)(353.31152344,443.55195313)(353.27636719,443.00898438)
\closepath
}
}
{
\newrgbcolor{curcolor}{0 0 0}
\pscustom[linestyle=none,fillstyle=solid,fillcolor=curcolor]
{
\newpath
\moveto(363.19042969,439.3)
\lineto(363.19042969,440.08515625)
\curveto(362.79589844,439.46796875)(362.21582031,439.159375)(361.45019531,439.159375)
\curveto(360.95410156,439.159375)(360.49707031,439.29609375)(360.07910156,439.56953125)
\curveto(359.66503906,439.84296875)(359.34277344,440.22382813)(359.11230469,440.71210938)
\curveto(358.88574219,441.20429688)(358.77246094,441.76875)(358.77246094,442.40546875)
\curveto(358.77246094,443.0265625)(358.87597656,443.5890625)(359.08300781,444.09296875)
\curveto(359.29003906,444.60078125)(359.60058594,444.98945313)(360.01464844,445.25898438)
\curveto(360.42871094,445.52851563)(360.89160156,445.66328125)(361.40332031,445.66328125)
\curveto(361.77832031,445.66328125)(362.11230469,445.58320313)(362.40527344,445.42304688)
\curveto(362.69824219,445.26679688)(362.93652344,445.06171875)(363.12011719,444.8078125)
\lineto(363.12011719,447.88984375)
\lineto(364.16894531,447.88984375)
\lineto(364.16894531,439.3)
\closepath
\moveto(359.85644531,442.40546875)
\curveto(359.85644531,441.60859375)(360.02441406,441.01289063)(360.36035156,440.61835938)
\curveto(360.69628906,440.22382813)(361.09277344,440.0265625)(361.54980469,440.0265625)
\curveto(362.01074219,440.0265625)(362.40136719,440.2140625)(362.72167969,440.5890625)
\curveto(363.04589844,440.96796875)(363.20800781,441.54414063)(363.20800781,442.31757813)
\curveto(363.20800781,443.16914063)(363.04394531,443.79414063)(362.71582031,444.19257813)
\curveto(362.38769531,444.59101563)(361.98339844,444.79023438)(361.50292969,444.79023438)
\curveto(361.03417969,444.79023438)(360.64160156,444.59882813)(360.32519531,444.21601563)
\curveto(360.01269531,443.83320313)(359.85644531,443.2296875)(359.85644531,442.40546875)
\closepath
}
}
{
\newrgbcolor{curcolor}{0 0 0}
\pscustom[linestyle=none,fillstyle=solid,fillcolor=curcolor]
{
\newpath
\moveto(368.76855469,442.41132813)
\curveto(368.76855469,443.56367188)(369.08886719,444.4171875)(369.72949219,444.971875)
\curveto(370.26464844,445.4328125)(370.91699219,445.66328125)(371.68652344,445.66328125)
\curveto(372.54199219,445.66328125)(373.24121094,445.38203125)(373.78417969,444.81953125)
\curveto(374.32714844,444.2609375)(374.59863281,443.4875)(374.59863281,442.49921875)
\curveto(374.59863281,441.6984375)(374.47753906,441.06757813)(374.23535156,440.60664063)
\curveto(373.99707031,440.14960938)(373.64746094,439.79414063)(373.18652344,439.54023438)
\curveto(372.72949219,439.28632813)(372.22949219,439.159375)(371.68652344,439.159375)
\curveto(370.81542969,439.159375)(370.11035156,439.43867188)(369.57128906,439.99726563)
\curveto(369.03613281,440.55585938)(368.76855469,441.36054688)(368.76855469,442.41132813)
\closepath
\moveto(369.85253906,442.41132813)
\curveto(369.85253906,441.61445313)(370.02636719,441.01679688)(370.37402344,440.61835938)
\curveto(370.72167969,440.22382813)(371.15917969,440.0265625)(371.68652344,440.0265625)
\curveto(372.20996094,440.0265625)(372.64550781,440.22578125)(372.99316406,440.62421875)
\curveto(373.34082031,441.02265625)(373.51464844,441.63007813)(373.51464844,442.44648438)
\curveto(373.51464844,443.21601563)(373.33886719,443.79804688)(372.98730469,444.19257813)
\curveto(372.63964844,444.59101563)(372.20605469,444.79023438)(371.68652344,444.79023438)
\curveto(371.15917969,444.79023438)(370.72167969,444.59296875)(370.37402344,444.1984375)
\curveto(370.02636719,443.80390625)(369.85253906,443.20820313)(369.85253906,442.41132813)
\closepath
}
}
{
\newrgbcolor{curcolor}{0 0 0}
\pscustom[linestyle=none,fillstyle=solid,fillcolor=curcolor]
{
\newpath
\moveto(375.83496094,439.3)
\lineto(375.83496094,445.52265625)
\lineto(376.78417969,445.52265625)
\lineto(376.78417969,444.63789063)
\curveto(377.24121094,445.32148438)(377.90136719,445.66328125)(378.76464844,445.66328125)
\curveto(379.13964844,445.66328125)(379.48339844,445.59492188)(379.79589844,445.45820313)
\curveto(380.11230469,445.32539063)(380.34863281,445.14960938)(380.50488281,444.93085938)
\curveto(380.66113281,444.71210938)(380.77050781,444.45234375)(380.83300781,444.1515625)
\curveto(380.87207031,443.95625)(380.89160156,443.61445313)(380.89160156,443.12617188)
\lineto(380.89160156,439.3)
\lineto(379.83691406,439.3)
\lineto(379.83691406,443.08515625)
\curveto(379.83691406,443.51484375)(379.79589844,443.83515625)(379.71386719,444.04609375)
\curveto(379.63183594,444.2609375)(379.48535156,444.43085938)(379.27441406,444.55585938)
\curveto(379.06738281,444.68476563)(378.82324219,444.74921875)(378.54199219,444.74921875)
\curveto(378.09277344,444.74921875)(377.70410156,444.60664063)(377.37597656,444.32148438)
\curveto(377.05175781,444.03632813)(376.88964844,443.4953125)(376.88964844,442.6984375)
\lineto(376.88964844,439.3)
\closepath
}
}
{
\newrgbcolor{curcolor}{0 0 0}
\pscustom[linestyle=none,fillstyle=solid,fillcolor=curcolor]
{
\newpath
\moveto(385.97753906,439.3)
\lineto(385.97753906,447.88984375)
\lineto(389.21777344,447.88984375)
\curveto(389.78808594,447.88984375)(390.22363281,447.8625)(390.52441406,447.8078125)
\curveto(390.94628906,447.7375)(391.29980469,447.60273438)(391.58496094,447.40351563)
\curveto(391.87011719,447.20820313)(392.09863281,446.9328125)(392.27050781,446.57734375)
\curveto(392.44628906,446.221875)(392.53417969,445.83125)(392.53417969,445.40546875)
\curveto(392.53417969,444.675)(392.30175781,444.05585938)(391.83691406,443.54804688)
\curveto(391.37207031,443.04414063)(390.53222656,442.7921875)(389.31738281,442.7921875)
\lineto(387.11425781,442.7921875)
\lineto(387.11425781,439.3)
\closepath
\moveto(387.11425781,443.80585938)
\lineto(389.33496094,443.80585938)
\curveto(390.06933594,443.80585938)(390.59082031,443.94257813)(390.89941406,444.21601563)
\curveto(391.20800781,444.48945313)(391.36230469,444.87421875)(391.36230469,445.3703125)
\curveto(391.36230469,445.7296875)(391.27050781,446.03632813)(391.08691406,446.29023438)
\curveto(390.90722656,446.54804688)(390.66894531,446.71796875)(390.37207031,446.8)
\curveto(390.18066406,446.85078125)(389.82714844,446.87617188)(389.31152344,446.87617188)
\lineto(387.11425781,446.87617188)
\closepath
}
}
{
\newrgbcolor{curcolor}{0 0 0}
\pscustom[linestyle=none,fillstyle=solid,fillcolor=curcolor]
{
\newpath
\moveto(393.83496094,439.3)
\lineto(393.83496094,445.52265625)
\lineto(394.78417969,445.52265625)
\lineto(394.78417969,444.57929688)
\curveto(395.02636719,445.02070313)(395.24902344,445.31171875)(395.45214844,445.45234375)
\curveto(395.65917969,445.59296875)(395.88574219,445.66328125)(396.13183594,445.66328125)
\curveto(396.48730469,445.66328125)(396.84863281,445.55)(397.21582031,445.3234375)
\lineto(396.85253906,444.34492188)
\curveto(396.59472656,444.49726563)(396.33691406,444.5734375)(396.07910156,444.5734375)
\curveto(395.84863281,444.5734375)(395.64160156,444.503125)(395.45800781,444.3625)
\curveto(395.27441406,444.22578125)(395.14355469,444.034375)(395.06542969,443.78828125)
\curveto(394.94824219,443.41328125)(394.88964844,443.003125)(394.88964844,442.5578125)
\lineto(394.88964844,439.3)
\closepath
}
}
{
\newrgbcolor{curcolor}{0 0 0}
\pscustom[linestyle=none,fillstyle=solid,fillcolor=curcolor]
{
\newpath
\moveto(397.45019531,442.41132813)
\curveto(397.45019531,443.56367188)(397.77050781,444.4171875)(398.41113281,444.971875)
\curveto(398.94628906,445.4328125)(399.59863281,445.66328125)(400.36816406,445.66328125)
\curveto(401.22363281,445.66328125)(401.92285156,445.38203125)(402.46582031,444.81953125)
\curveto(403.00878906,444.2609375)(403.28027344,443.4875)(403.28027344,442.49921875)
\curveto(403.28027344,441.6984375)(403.15917969,441.06757813)(402.91699219,440.60664063)
\curveto(402.67871094,440.14960938)(402.32910156,439.79414063)(401.86816406,439.54023438)
\curveto(401.41113281,439.28632813)(400.91113281,439.159375)(400.36816406,439.159375)
\curveto(399.49707031,439.159375)(398.79199219,439.43867188)(398.25292969,439.99726563)
\curveto(397.71777344,440.55585938)(397.45019531,441.36054688)(397.45019531,442.41132813)
\closepath
\moveto(398.53417969,442.41132813)
\curveto(398.53417969,441.61445313)(398.70800781,441.01679688)(399.05566406,440.61835938)
\curveto(399.40332031,440.22382813)(399.84082031,440.0265625)(400.36816406,440.0265625)
\curveto(400.89160156,440.0265625)(401.32714844,440.22578125)(401.67480469,440.62421875)
\curveto(402.02246094,441.02265625)(402.19628906,441.63007813)(402.19628906,442.44648438)
\curveto(402.19628906,443.21601563)(402.02050781,443.79804688)(401.66894531,444.19257813)
\curveto(401.32128906,444.59101563)(400.88769531,444.79023438)(400.36816406,444.79023438)
\curveto(399.84082031,444.79023438)(399.40332031,444.59296875)(399.05566406,444.1984375)
\curveto(398.70800781,443.80390625)(398.53417969,443.20820313)(398.53417969,442.41132813)
\closepath
}
}
{
\newrgbcolor{curcolor}{0 0 0}
\pscustom[linestyle=none,fillstyle=solid,fillcolor=curcolor]
{
\newpath
\moveto(408.57714844,441.57929688)
\lineto(409.61425781,441.44453125)
\curveto(409.50097656,440.7296875)(409.20996094,440.16914063)(408.74121094,439.76289063)
\curveto(408.27636719,439.36054688)(407.70410156,439.159375)(407.02441406,439.159375)
\curveto(406.17285156,439.159375)(405.48730469,439.43671875)(404.96777344,439.99140625)
\curveto(404.45214844,440.55)(404.19433594,441.34882813)(404.19433594,442.38789063)
\curveto(404.19433594,443.05976563)(404.30566406,443.64765625)(404.52832031,444.1515625)
\curveto(404.75097656,444.65546875)(405.08886719,445.03242188)(405.54199219,445.28242188)
\curveto(405.99902344,445.53632813)(406.49511719,445.66328125)(407.03027344,445.66328125)
\curveto(407.70605469,445.66328125)(408.25878906,445.49140625)(408.68847656,445.14765625)
\curveto(409.11816406,444.8078125)(409.39355469,444.3234375)(409.51464844,443.69453125)
\lineto(408.48925781,443.53632813)
\curveto(408.39160156,443.95429688)(408.21777344,444.26875)(407.96777344,444.4796875)
\curveto(407.72167969,444.690625)(407.42285156,444.79609375)(407.07128906,444.79609375)
\curveto(406.54003906,444.79609375)(406.10839844,444.6046875)(405.77636719,444.221875)
\curveto(405.44433594,443.84296875)(405.27832031,443.24140625)(405.27832031,442.4171875)
\curveto(405.27832031,441.58125)(405.43847656,440.97382813)(405.75878906,440.59492188)
\curveto(406.07910156,440.21601563)(406.49707031,440.0265625)(407.01269531,440.0265625)
\curveto(407.42675781,440.0265625)(407.77246094,440.15351563)(408.04980469,440.40742188)
\curveto(408.32714844,440.66132813)(408.50292969,441.05195313)(408.57714844,441.57929688)
\closepath
}
}
{
\newrgbcolor{curcolor}{0 0 0}
\pscustom[linestyle=none,fillstyle=solid,fillcolor=curcolor]
{
\newpath
\moveto(414.77636719,441.30390625)
\lineto(415.86621094,441.16914063)
\curveto(415.69433594,440.53242188)(415.37597656,440.03828125)(414.91113281,439.68671875)
\curveto(414.44628906,439.33515625)(413.85253906,439.159375)(413.12988281,439.159375)
\curveto(412.21972656,439.159375)(411.49707031,439.43867188)(410.96191406,439.99726563)
\curveto(410.43066406,440.55976563)(410.16503906,441.346875)(410.16503906,442.35859375)
\curveto(410.16503906,443.40546875)(410.43457031,444.21796875)(410.97363281,444.79609375)
\curveto(411.51269531,445.37421875)(412.21191406,445.66328125)(413.07128906,445.66328125)
\curveto(413.90332031,445.66328125)(414.58300781,445.38007813)(415.11035156,444.81367188)
\curveto(415.63769531,444.24726563)(415.90136719,443.45039063)(415.90136719,442.42304688)
\curveto(415.90136719,442.36054688)(415.89941406,442.26679688)(415.89550781,442.14179688)
\lineto(411.25488281,442.14179688)
\curveto(411.29394531,441.45820313)(411.48730469,440.93476563)(411.83496094,440.57148438)
\curveto(412.18261719,440.20820313)(412.61621094,440.0265625)(413.13574219,440.0265625)
\curveto(413.52246094,440.0265625)(413.85253906,440.128125)(414.12597656,440.33125)
\curveto(414.39941406,440.534375)(414.61621094,440.85859375)(414.77636719,441.30390625)
\closepath
\moveto(411.31347656,443.00898438)
\lineto(414.78808594,443.00898438)
\curveto(414.74121094,443.53242188)(414.60839844,443.925)(414.38964844,444.18671875)
\curveto(414.05371094,444.59296875)(413.61816406,444.79609375)(413.08300781,444.79609375)
\curveto(412.59863281,444.79609375)(412.19042969,444.63398438)(411.85839844,444.30976563)
\curveto(411.53027344,443.98554688)(411.34863281,443.55195313)(411.31347656,443.00898438)
\closepath
}
}
{
\newrgbcolor{curcolor}{0 0 0}
\pscustom[linestyle=none,fillstyle=solid,fillcolor=curcolor]
{
\newpath
\moveto(416.76855469,441.15742188)
\lineto(417.81152344,441.32148438)
\curveto(417.87011719,440.90351563)(418.03222656,440.58320313)(418.29785156,440.36054688)
\curveto(418.56738281,440.13789063)(418.94238281,440.0265625)(419.42285156,440.0265625)
\curveto(419.90722656,440.0265625)(420.26660156,440.12421875)(420.50097656,440.31953125)
\curveto(420.73535156,440.51875)(420.85253906,440.75117188)(420.85253906,441.01679688)
\curveto(420.85253906,441.25507813)(420.74902344,441.44257813)(420.54199219,441.57929688)
\curveto(420.39746094,441.67304688)(420.03808594,441.7921875)(419.46386719,441.93671875)
\curveto(418.69042969,442.13203125)(418.15332031,442.3)(417.85253906,442.440625)
\curveto(417.55566406,442.58515625)(417.32910156,442.78242188)(417.17285156,443.03242188)
\curveto(417.02050781,443.28632813)(416.94433594,443.565625)(416.94433594,443.8703125)
\curveto(416.94433594,444.14765625)(417.00683594,444.40351563)(417.13183594,444.63789063)
\curveto(417.26074219,444.87617188)(417.43457031,445.0734375)(417.65332031,445.2296875)
\curveto(417.81738281,445.35078125)(418.04003906,445.45234375)(418.32128906,445.534375)
\curveto(418.60644531,445.6203125)(418.91113281,445.66328125)(419.23535156,445.66328125)
\curveto(419.72363281,445.66328125)(420.15136719,445.59296875)(420.51855469,445.45234375)
\curveto(420.88964844,445.31171875)(421.16308594,445.1203125)(421.33886719,444.878125)
\curveto(421.51464844,444.63984375)(421.63574219,444.31953125)(421.70214844,443.9171875)
\lineto(420.67089844,443.7765625)
\curveto(420.62402344,444.096875)(420.48730469,444.346875)(420.26074219,444.5265625)
\curveto(420.03808594,444.70625)(419.72167969,444.79609375)(419.31152344,444.79609375)
\curveto(418.82714844,444.79609375)(418.48144531,444.71601563)(418.27441406,444.55585938)
\curveto(418.06738281,444.39570313)(417.96386719,444.20820313)(417.96386719,443.99335938)
\curveto(417.96386719,443.85664063)(418.00683594,443.73359375)(418.09277344,443.62421875)
\curveto(418.17871094,443.5109375)(418.31347656,443.4171875)(418.49707031,443.34296875)
\curveto(418.60253906,443.30390625)(418.91308594,443.2140625)(419.42871094,443.0734375)
\curveto(420.17480469,442.87421875)(420.69433594,442.71015625)(420.98730469,442.58125)
\curveto(421.28417969,442.45625)(421.51660156,442.27265625)(421.68457031,442.03046875)
\curveto(421.85253906,441.78828125)(421.93652344,441.4875)(421.93652344,441.128125)
\curveto(421.93652344,440.7765625)(421.83300781,440.44453125)(421.62597656,440.13203125)
\curveto(421.42285156,439.8234375)(421.12792969,439.58320313)(420.74121094,439.41132813)
\curveto(420.35449219,439.24335938)(419.91699219,439.159375)(419.42871094,439.159375)
\curveto(418.62011719,439.159375)(418.00292969,439.32734375)(417.57714844,439.66328125)
\curveto(417.15527344,439.99921875)(416.88574219,440.49726563)(416.76855469,441.15742188)
\closepath
}
}
{
\newrgbcolor{curcolor}{0 0 0}
\pscustom[linestyle=none,fillstyle=solid,fillcolor=curcolor]
{
\newpath
\moveto(422.76855469,441.15742188)
\lineto(423.81152344,441.32148438)
\curveto(423.87011719,440.90351563)(424.03222656,440.58320313)(424.29785156,440.36054688)
\curveto(424.56738281,440.13789063)(424.94238281,440.0265625)(425.42285156,440.0265625)
\curveto(425.90722656,440.0265625)(426.26660156,440.12421875)(426.50097656,440.31953125)
\curveto(426.73535156,440.51875)(426.85253906,440.75117188)(426.85253906,441.01679688)
\curveto(426.85253906,441.25507813)(426.74902344,441.44257813)(426.54199219,441.57929688)
\curveto(426.39746094,441.67304688)(426.03808594,441.7921875)(425.46386719,441.93671875)
\curveto(424.69042969,442.13203125)(424.15332031,442.3)(423.85253906,442.440625)
\curveto(423.55566406,442.58515625)(423.32910156,442.78242188)(423.17285156,443.03242188)
\curveto(423.02050781,443.28632813)(422.94433594,443.565625)(422.94433594,443.8703125)
\curveto(422.94433594,444.14765625)(423.00683594,444.40351563)(423.13183594,444.63789063)
\curveto(423.26074219,444.87617188)(423.43457031,445.0734375)(423.65332031,445.2296875)
\curveto(423.81738281,445.35078125)(424.04003906,445.45234375)(424.32128906,445.534375)
\curveto(424.60644531,445.6203125)(424.91113281,445.66328125)(425.23535156,445.66328125)
\curveto(425.72363281,445.66328125)(426.15136719,445.59296875)(426.51855469,445.45234375)
\curveto(426.88964844,445.31171875)(427.16308594,445.1203125)(427.33886719,444.878125)
\curveto(427.51464844,444.63984375)(427.63574219,444.31953125)(427.70214844,443.9171875)
\lineto(426.67089844,443.7765625)
\curveto(426.62402344,444.096875)(426.48730469,444.346875)(426.26074219,444.5265625)
\curveto(426.03808594,444.70625)(425.72167969,444.79609375)(425.31152344,444.79609375)
\curveto(424.82714844,444.79609375)(424.48144531,444.71601563)(424.27441406,444.55585938)
\curveto(424.06738281,444.39570313)(423.96386719,444.20820313)(423.96386719,443.99335938)
\curveto(423.96386719,443.85664063)(424.00683594,443.73359375)(424.09277344,443.62421875)
\curveto(424.17871094,443.5109375)(424.31347656,443.4171875)(424.49707031,443.34296875)
\curveto(424.60253906,443.30390625)(424.91308594,443.2140625)(425.42871094,443.0734375)
\curveto(426.17480469,442.87421875)(426.69433594,442.71015625)(426.98730469,442.58125)
\curveto(427.28417969,442.45625)(427.51660156,442.27265625)(427.68457031,442.03046875)
\curveto(427.85253906,441.78828125)(427.93652344,441.4875)(427.93652344,441.128125)
\curveto(427.93652344,440.7765625)(427.83300781,440.44453125)(427.62597656,440.13203125)
\curveto(427.42285156,439.8234375)(427.12792969,439.58320313)(426.74121094,439.41132813)
\curveto(426.35449219,439.24335938)(425.91699219,439.159375)(425.42871094,439.159375)
\curveto(424.62011719,439.159375)(424.00292969,439.32734375)(423.57714844,439.66328125)
\curveto(423.15527344,439.99921875)(422.88574219,440.49726563)(422.76855469,441.15742188)
\closepath
}
}
{
\newrgbcolor{curcolor}{0 0 0}
\pscustom[linestyle=none,fillstyle=solid,fillcolor=curcolor]
{
\newpath
\moveto(432.27246094,442.05976563)
\lineto(433.34472656,442.15351563)
\curveto(433.39550781,441.72382813)(433.51269531,441.3703125)(433.69628906,441.09296875)
\curveto(433.88378906,440.81953125)(434.17285156,440.596875)(434.56347656,440.425)
\curveto(434.95410156,440.25703125)(435.39355469,440.17304688)(435.88183594,440.17304688)
\curveto(436.31542969,440.17304688)(436.69824219,440.2375)(437.03027344,440.36640625)
\curveto(437.36230469,440.4953125)(437.60839844,440.67109375)(437.76855469,440.89375)
\curveto(437.93261719,441.1203125)(438.01464844,441.36640625)(438.01464844,441.63203125)
\curveto(438.01464844,441.9015625)(437.93652344,442.1359375)(437.78027344,442.33515625)
\curveto(437.62402344,442.53828125)(437.36621094,442.70820313)(437.00683594,442.84492188)
\curveto(436.77636719,442.93476563)(436.26660156,443.0734375)(435.47753906,443.2609375)
\curveto(434.68847656,443.45234375)(434.13574219,443.63203125)(433.81933594,443.8)
\curveto(433.40917969,444.01484375)(433.10253906,444.28046875)(432.89941406,444.596875)
\curveto(432.70019531,444.9171875)(432.60058594,445.27460938)(432.60058594,445.66914063)
\curveto(432.60058594,446.10273438)(432.72363281,446.50703125)(432.96972656,446.88203125)
\curveto(433.21582031,447.2609375)(433.57519531,447.54804688)(434.04785156,447.74335938)
\curveto(434.52050781,447.93867188)(435.04589844,448.03632813)(435.62402344,448.03632813)
\curveto(436.26074219,448.03632813)(436.82128906,447.9328125)(437.30566406,447.72578125)
\curveto(437.79394531,447.52265625)(438.16894531,447.221875)(438.43066406,446.8234375)
\curveto(438.69238281,446.425)(438.83300781,445.97382813)(438.85253906,445.46992188)
\lineto(437.76269531,445.38789063)
\curveto(437.70410156,445.93085938)(437.50488281,446.34101563)(437.16503906,446.61835938)
\curveto(436.82910156,446.89570313)(436.33105469,447.034375)(435.67089844,447.034375)
\curveto(434.98339844,447.034375)(434.48144531,446.90742188)(434.16503906,446.65351563)
\curveto(433.85253906,446.40351563)(433.69628906,446.10078125)(433.69628906,445.7453125)
\curveto(433.69628906,445.43671875)(433.80761719,445.1828125)(434.03027344,444.98359375)
\curveto(434.24902344,444.784375)(434.81933594,444.57929688)(435.74121094,444.36835938)
\curveto(436.66699219,444.16132813)(437.30175781,443.9796875)(437.64550781,443.8234375)
\curveto(438.14550781,443.59296875)(438.51464844,443.3)(438.75292969,442.94453125)
\curveto(438.99121094,442.59296875)(439.11035156,442.18671875)(439.11035156,441.72578125)
\curveto(439.11035156,441.26875)(438.97949219,440.83710938)(438.71777344,440.43085938)
\curveto(438.45605469,440.02851563)(438.07910156,439.7140625)(437.58691406,439.4875)
\curveto(437.09863281,439.26484375)(436.54785156,439.15351563)(435.93457031,439.15351563)
\curveto(435.15722656,439.15351563)(434.50488281,439.26679688)(433.97753906,439.49335938)
\curveto(433.45410156,439.71992188)(433.04199219,440.05976563)(432.74121094,440.51289063)
\curveto(432.44433594,440.96992188)(432.28808594,441.48554688)(432.27246094,442.05976563)
\closepath
}
}
{
\newrgbcolor{curcolor}{0 0 0}
\pscustom[linestyle=none,fillstyle=solid,fillcolor=curcolor]
{
\newpath
\moveto(440.53417969,446.67695313)
\lineto(440.53417969,447.88984375)
\lineto(441.58886719,447.88984375)
\lineto(441.58886719,446.67695313)
\closepath
\moveto(440.53417969,439.3)
\lineto(440.53417969,445.52265625)
\lineto(441.58886719,445.52265625)
\lineto(441.58886719,439.3)
\closepath
}
}
{
\newrgbcolor{curcolor}{0 0 0}
\pscustom[linestyle=none,fillstyle=solid,fillcolor=curcolor]
{
\newpath
\moveto(442.63769531,439.3)
\lineto(442.63769531,440.15546875)
\lineto(446.59863281,444.70234375)
\curveto(446.14941406,444.67890625)(445.75292969,444.6671875)(445.40917969,444.6671875)
\lineto(442.87207031,444.6671875)
\lineto(442.87207031,445.52265625)
\lineto(447.95800781,445.52265625)
\lineto(447.95800781,444.82539063)
\lineto(444.58886719,440.87617188)
\lineto(443.93847656,440.15546875)
\curveto(444.41113281,440.190625)(444.85449219,440.20820313)(445.26855469,440.20820313)
\lineto(448.14550781,440.20820313)
\lineto(448.14550781,439.3)
\closepath
}
}
{
\newrgbcolor{curcolor}{0 0 0}
\pscustom[linestyle=none,fillstyle=solid,fillcolor=curcolor]
{
\newpath
\moveto(453.45410156,441.30390625)
\lineto(454.54394531,441.16914063)
\curveto(454.37207031,440.53242188)(454.05371094,440.03828125)(453.58886719,439.68671875)
\curveto(453.12402344,439.33515625)(452.53027344,439.159375)(451.80761719,439.159375)
\curveto(450.89746094,439.159375)(450.17480469,439.43867188)(449.63964844,439.99726563)
\curveto(449.10839844,440.55976563)(448.84277344,441.346875)(448.84277344,442.35859375)
\curveto(448.84277344,443.40546875)(449.11230469,444.21796875)(449.65136719,444.79609375)
\curveto(450.19042969,445.37421875)(450.88964844,445.66328125)(451.74902344,445.66328125)
\curveto(452.58105469,445.66328125)(453.26074219,445.38007813)(453.78808594,444.81367188)
\curveto(454.31542969,444.24726563)(454.57910156,443.45039063)(454.57910156,442.42304688)
\curveto(454.57910156,442.36054688)(454.57714844,442.26679688)(454.57324219,442.14179688)
\lineto(449.93261719,442.14179688)
\curveto(449.97167969,441.45820313)(450.16503906,440.93476563)(450.51269531,440.57148438)
\curveto(450.86035156,440.20820313)(451.29394531,440.0265625)(451.81347656,440.0265625)
\curveto(452.20019531,440.0265625)(452.53027344,440.128125)(452.80371094,440.33125)
\curveto(453.07714844,440.534375)(453.29394531,440.85859375)(453.45410156,441.30390625)
\closepath
\moveto(449.99121094,443.00898438)
\lineto(453.46582031,443.00898438)
\curveto(453.41894531,443.53242188)(453.28613281,443.925)(453.06738281,444.18671875)
\curveto(452.73144531,444.59296875)(452.29589844,444.79609375)(451.76074219,444.79609375)
\curveto(451.27636719,444.79609375)(450.86816406,444.63398438)(450.53613281,444.30976563)
\curveto(450.20800781,443.98554688)(450.02636719,443.55195313)(449.99121094,443.00898438)
\closepath
}
}
{
\newrgbcolor{curcolor}{0.54901963 0.01176471 0.98823529}
\pscustom[linewidth=1,linecolor=curcolor]
{
\newpath
\moveto(242.7,335.5)
\lineto(287.9,332.3)
}
}
{
\newrgbcolor{curcolor}{0.58039218 0.03137255 0.96862745}
\pscustom[linewidth=1,linecolor=curcolor]
{
\newpath
\moveto(287.9,332.3)
\lineto(333.1,319.7)
}
}
{
\newrgbcolor{curcolor}{0.63529414 0.06666667 0.93333334}
\pscustom[linewidth=1,linecolor=curcolor]
{
\newpath
\moveto(333.1,319.7)
\lineto(378.3,318.3)
}
}
{
\newrgbcolor{curcolor}{0.86274511 0.21176471 0.78823531}
\pscustom[linewidth=1,linecolor=curcolor]
{
\newpath
\moveto(378.3,318.3)
\lineto(423.6,319.7)
}
}
{
\newrgbcolor{curcolor}{0.86666667 0.21568628 0.78431374}
\pscustom[linewidth=1,linecolor=curcolor]
{
\newpath
\moveto(423.6,319.7)
\lineto(468.9,300.2)
}
}
{
\newrgbcolor{curcolor}{0.89411765 0.23137255 0.76862746}
\pscustom[linewidth=1,linecolor=curcolor]
{
\newpath
\moveto(468.9,300.2)
\lineto(514.1,303.4)
}
}
{
\newrgbcolor{curcolor}{0.50196081 0 1}
\pscustom[linewidth=1,linecolor=curcolor]
{
\newpath
\moveto(227,322.7)
\lineto(272.2,321.9)
}
}
{
\newrgbcolor{curcolor}{0.65490198 0.07843138 0.92156863}
\pscustom[linewidth=1,linecolor=curcolor]
{
\newpath
\moveto(272.2,321.9)
\lineto(317.4,316.9)
}
}
{
\newrgbcolor{curcolor}{0.73725492 0.12941177 0.87058824}
\pscustom[linewidth=1,linecolor=curcolor]
{
\newpath
\moveto(317.4,316.9)
\lineto(362.7,309.9)
}
}
{
\newrgbcolor{curcolor}{0.82745099 0.19215687 0.80784315}
\pscustom[linewidth=1,linecolor=curcolor]
{
\newpath
\moveto(362.7,309.9)
\lineto(407.9,306.1)
}
}
{
\newrgbcolor{curcolor}{0.97254902 0.28235295 0.71764708}
\pscustom[linewidth=1,linecolor=curcolor]
{
\newpath
\moveto(407.9,306.1)
\lineto(453.2,303.1)
}
}
{
\newrgbcolor{curcolor}{0.80392158 0.17254902 0.82745099}
\pscustom[linewidth=1,linecolor=curcolor]
{
\newpath
\moveto(453.2,303.1)
\lineto(498.5,273.6)
}
}
{
\newrgbcolor{curcolor}{0.58039218 0.03137255 0.96862745}
\pscustom[linewidth=1,linecolor=curcolor]
{
\newpath
\moveto(211.3,317.9)
\lineto(256.6,313.9)
}
}
{
\newrgbcolor{curcolor}{0.71372551 0.11764706 0.88235295}
\pscustom[linewidth=1,linecolor=curcolor]
{
\newpath
\moveto(256.6,313.9)
\lineto(301.7,310.5)
}
}
{
\newrgbcolor{curcolor}{0.86666667 0.21568628 0.78431374}
\pscustom[linewidth=1,linecolor=curcolor]
{
\newpath
\moveto(301.7,310.5)
\lineto(347,308)
}
}
{
\newrgbcolor{curcolor}{1 0.33333334 0.66666669}
\pscustom[linewidth=1,linecolor=curcolor]
{
\newpath
\moveto(347,308)
\lineto(392.3,306.9)
}
}
{
\newrgbcolor{curcolor}{1 0.44705883 0.5529412}
\pscustom[linewidth=1,linecolor=curcolor]
{
\newpath
\moveto(392.3,306.9)
\lineto(437.5,304.1)
}
}
{
\newrgbcolor{curcolor}{1 0.45490196 0.54509807}
\pscustom[linewidth=1,linecolor=curcolor]
{
\newpath
\moveto(437.5,304.1)
\lineto(482.8,289.7)
}
}
{
\newrgbcolor{curcolor}{0.32549021 0 1}
\pscustom[linewidth=1,linecolor=curcolor]
{
\newpath
\moveto(195.6,299)
\lineto(240.9,292.7)
}
}
{
\newrgbcolor{curcolor}{0.75686276 0.14509805 0.85490197}
\pscustom[linewidth=1,linecolor=curcolor]
{
\newpath
\moveto(240.9,292.7)
\lineto(286.1,315.9)
}
}
{
\newrgbcolor{curcolor}{1 0.36862746 0.63137257}
\pscustom[linewidth=1,linecolor=curcolor]
{
\newpath
\moveto(286.1,315.9)
\lineto(331.3,302.8)
}
}
{
\newrgbcolor{curcolor}{0.96862745 0.27843139 0.72156864}
\pscustom[linewidth=1,linecolor=curcolor]
{
\newpath
\moveto(331.3,302.8)
\lineto(376.6,286)
}
}
{
\newrgbcolor{curcolor}{1 0.43529412 0.56470591}
\pscustom[linewidth=1,linecolor=curcolor]
{
\newpath
\moveto(376.6,286)
\lineto(421.8,304.3)
}
}
{
\newrgbcolor{curcolor}{1 0.5529412 0.44705883}
\pscustom[linewidth=1,linecolor=curcolor]
{
\newpath
\moveto(421.8,304.3)
\lineto(467.1,282.6)
}
}
{
\newrgbcolor{curcolor}{0.50588238 0 1}
\pscustom[linewidth=1,linecolor=curcolor]
{
\newpath
\moveto(179.9,299.3)
\lineto(225.2,288)
}
}
{
\newrgbcolor{curcolor}{0.70588237 0.11372549 0.88627452}
\pscustom[linewidth=1,linecolor=curcolor]
{
\newpath
\moveto(225.2,288)
\lineto(270.5,297.3)
}
}
{
\newrgbcolor{curcolor}{1 0.39215687 0.60784316}
\pscustom[linewidth=1,linecolor=curcolor]
{
\newpath
\moveto(270.5,297.3)
\lineto(315.6,305.6)
}
}
{
\newrgbcolor{curcolor}{1 0.38039216 0.61960787}
\pscustom[linewidth=1,linecolor=curcolor]
{
\newpath
\moveto(315.6,305.6)
\lineto(360.9,277)
}
}
{
\newrgbcolor{curcolor}{1 0.49803922 0.50196081}
\pscustom[linewidth=1,linecolor=curcolor]
{
\newpath
\moveto(360.9,277)
\lineto(406.2,302.1)
}
}
{
\newrgbcolor{curcolor}{1 0.627451 0.37254903}
\pscustom[linewidth=1,linecolor=curcolor]
{
\newpath
\moveto(406.2,302.1)
\lineto(451.4,275.2)
}
}
{
\newrgbcolor{curcolor}{0.47450981 0 1}
\pscustom[linewidth=1,linecolor=curcolor]
{
\newpath
\moveto(164.3,296.1)
\lineto(209.5,269.3)
}
}
{
\newrgbcolor{curcolor}{0.46666667 0 1}
\pscustom[linewidth=1,linecolor=curcolor]
{
\newpath
\moveto(209.5,269.3)
\lineto(254.8,276.9)
}
}
{
\newrgbcolor{curcolor}{0.57647061 0.02745098 0.97254902}
\pscustom[linewidth=1,linecolor=curcolor]
{
\newpath
\moveto(254.8,276.9)
\lineto(300,259.8)
}
}
{
\newrgbcolor{curcolor}{0.94901961 0.26666668 0.73333335}
\pscustom[linewidth=1,linecolor=curcolor]
{
\newpath
\moveto(300,259.8)
\lineto(345.2,289)
}
}
{
\newrgbcolor{curcolor}{1 0.66274512 0.33725491}
\pscustom[linewidth=1,linecolor=curcolor]
{
\newpath
\moveto(345.2,289)
\lineto(390.5,292)
}
}
{
\newrgbcolor{curcolor}{1 0.6156863 0.38431373}
\pscustom[linewidth=1,linecolor=curcolor]
{
\newpath
\moveto(390.5,292)
\lineto(435.7,264.5)
}
}
{
\newrgbcolor{curcolor}{0.1882353 0 1}
\pscustom[linewidth=1,linecolor=curcolor]
{
\newpath
\moveto(148.6,256)
\lineto(193.8,266.6)
}
}
{
\newrgbcolor{curcolor}{0.87450981 0.21960784 0.78039217}
\pscustom[linewidth=1,linecolor=curcolor]
{
\newpath
\moveto(193.8,266.6)
\lineto(239.1,293.8)
}
}
{
\newrgbcolor{curcolor}{1 0.64313728 0.35686275}
\pscustom[linewidth=1,linecolor=curcolor]
{
\newpath
\moveto(239.1,293.8)
\lineto(284.4,302.7)
}
}
{
\newrgbcolor{curcolor}{1 0.61176473 0.3882353}
\pscustom[linewidth=1,linecolor=curcolor]
{
\newpath
\moveto(284.4,302.7)
\lineto(329.5,271.2)
}
}
{
\newrgbcolor{curcolor}{1 0.47058824 0.52941179}
\pscustom[linewidth=1,linecolor=curcolor]
{
\newpath
\moveto(329.5,271.2)
\lineto(374.8,265.6)
}
}
{
\newrgbcolor{curcolor}{1 0.66274512 0.33725491}
\pscustom[linewidth=1,linecolor=curcolor]
{
\newpath
\moveto(374.8,265.6)
\lineto(420.1,277.6)
}
}
{
\newrgbcolor{curcolor}{0.30588236 0 1}
\pscustom[linewidth=1,linecolor=curcolor]
{
\newpath
\moveto(132.9,256.3)
\lineto(178.2,256.7)
}
}
{
\newrgbcolor{curcolor}{0.95294118 0.27058825 0.72941178}
\pscustom[linewidth=1,linecolor=curcolor]
{
\newpath
\moveto(178.2,256.7)
\lineto(223.4,290.9)
}
}
{
\newrgbcolor{curcolor}{1 0.49019608 0.50980395}
\pscustom[linewidth=1,linecolor=curcolor]
{
\newpath
\moveto(223.4,290.9)
\lineto(268.7,266.5)
}
}
{
\newrgbcolor{curcolor}{1 0.34509805 0.65490198}
\pscustom[linewidth=1,linecolor=curcolor]
{
\newpath
\moveto(268.7,266.5)
\lineto(313.9,253.5)
}
}
{
\newrgbcolor{curcolor}{1 0.50196081 0.49803922}
\pscustom[linewidth=1,linecolor=curcolor]
{
\newpath
\moveto(313.9,253.5)
\lineto(359.1,268.2)
}
}
{
\newrgbcolor{curcolor}{1 0.67450982 0.32549021}
\pscustom[linewidth=1,linecolor=curcolor]
{
\newpath
\moveto(359.1,268.2)
\lineto(404.4,257.1)
}
}
{
\newrgbcolor{curcolor}{0.15686275 0 1}
\pscustom[linewidth=1,linecolor=curcolor]
{
\newpath
\moveto(117.2,221.3)
\lineto(162.5,260.2)
}
}
{
\newrgbcolor{curcolor}{1 0.36078432 0.63921571}
\pscustom[linewidth=1,linecolor=curcolor]
{
\newpath
\moveto(162.5,260.2)
\lineto(207.7,279.7)
}
}
{
\newrgbcolor{curcolor}{1 0.56078434 0.43921569}
\pscustom[linewidth=1,linecolor=curcolor]
{
\newpath
\moveto(207.7,279.7)
\lineto(253,267.7)
}
}
{
\newrgbcolor{curcolor}{1 0.56470591 0.43529412}
\pscustom[linewidth=1,linecolor=curcolor]
{
\newpath
\moveto(253,267.7)
\lineto(298.3,261.5)
}
}
{
\newrgbcolor{curcolor}{1 0.57254905 0.42745098}
\pscustom[linewidth=1,linecolor=curcolor]
{
\newpath
\moveto(298.3,261.5)
\lineto(343.4,250)
}
}
{
\newrgbcolor{curcolor}{1 0.61176473 0.3882353}
\pscustom[linewidth=1,linecolor=curcolor]
{
\newpath
\moveto(343.4,250)
\lineto(388.7,247.8)
}
}
{
\newrgbcolor{curcolor}{0.23529412 0 1}
\pscustom[linewidth=1,linecolor=curcolor]
{
\newpath
\moveto(101.5,245.6)
\lineto(146.8,222.9)
}
}
{
\newrgbcolor{curcolor}{0.65490198 0.07843138 0.92156863}
\pscustom[linewidth=1,linecolor=curcolor]
{
\newpath
\moveto(146.8,222.9)
\lineto(192.1,261.9)
}
}
{
\newrgbcolor{curcolor}{1 0.42745098 0.57254905}
\pscustom[linewidth=1,linecolor=curcolor]
{
\newpath
\moveto(192.1,261.9)
\lineto(237.3,248.8)
}
}
{
\newrgbcolor{curcolor}{1 0.65882355 0.34117648}
\pscustom[linewidth=1,linecolor=curcolor]
{
\newpath
\moveto(237.3,248.8)
\lineto(282.6,273.3)
}
}
{
\newrgbcolor{curcolor}{1 0.78431374 0.21568628}
\pscustom[linewidth=1,linecolor=curcolor]
{
\newpath
\moveto(282.6,273.3)
\lineto(327.8,246.2)
}
}
{
\newrgbcolor{curcolor}{1 0.64705884 0.35294119}
\pscustom[linewidth=1,linecolor=curcolor]
{
\newpath
\moveto(327.8,246.2)
\lineto(373,237.3)
}
}
{
\newrgbcolor{curcolor}{0 0 0.47058824}
\pscustom[linewidth=1,linecolor=curcolor]
{
\newpath
\moveto(85.9,216.5)
\lineto(131.1,179.4)
}
}
{
\newrgbcolor{curcolor}{0.08627451 0 1}
\pscustom[linewidth=1,linecolor=curcolor]
{
\newpath
\moveto(131.1,179.4)
\lineto(176.4,239.2)
}
}
{
\newrgbcolor{curcolor}{1 0.55686277 0.44313726}
\pscustom[linewidth=1,linecolor=curcolor]
{
\newpath
\moveto(176.4,239.2)
\lineto(221.7,268.8)
}
}
{
\newrgbcolor{curcolor}{1 0.78431374 0.21568628}
\pscustom[linewidth=1,linecolor=curcolor]
{
\newpath
\moveto(221.7,268.8)
\lineto(266.9,250.2)
}
}
{
\newrgbcolor{curcolor}{1 0.56862748 0.43137255}
\pscustom[linewidth=1,linecolor=curcolor]
{
\newpath
\moveto(266.9,250.2)
\lineto(312.1,222.2)
}
}
{
\newrgbcolor{curcolor}{1 0.52941179 0.47058824}
\pscustom[linewidth=1,linecolor=curcolor]
{
\newpath
\moveto(312.1,222.2)
\lineto(357.3,226.8)
}
}
{
\newrgbcolor{curcolor}{0.43529412 0 1}
\pscustom[linewidth=1,linecolor=curcolor]
{
\newpath
\moveto(242.7,335.5)
\lineto(227,322.7)
}
}
{
\newrgbcolor{curcolor}{0.45490196 0 1}
\pscustom[linewidth=1,linecolor=curcolor]
{
\newpath
\moveto(227,322.7)
\lineto(211.3,317.9)
}
}
{
\newrgbcolor{curcolor}{0.40392157 0 1}
\pscustom[linewidth=1,linecolor=curcolor]
{
\newpath
\moveto(211.3,317.9)
\lineto(195.6,299)
}
}
{
\newrgbcolor{curcolor}{0.41176471 0 1}
\pscustom[linewidth=1,linecolor=curcolor]
{
\newpath
\moveto(195.6,299)
\lineto(179.9,299.3)
}
}
{
\newrgbcolor{curcolor}{0.61176473 0.05098039 0.94901961}
\pscustom[linewidth=1,linecolor=curcolor]
{
\newpath
\moveto(179.9,299.3)
\lineto(164.3,296.1)
}
}
{
\newrgbcolor{curcolor}{0.31764707 0 1}
\pscustom[linewidth=1,linecolor=curcolor]
{
\newpath
\moveto(164.3,296.1)
\lineto(148.6,256)
}
}
{
\newrgbcolor{curcolor}{0.07058824 0 1}
\pscustom[linewidth=1,linecolor=curcolor]
{
\newpath
\moveto(148.6,256)
\lineto(132.9,256.3)
}
}
{
\newrgbcolor{curcolor}{0 0 0.84705883}
\pscustom[linewidth=1,linecolor=curcolor]
{
\newpath
\moveto(132.9,256.3)
\lineto(117.2,221.3)
}
}
{
\newrgbcolor{curcolor}{0 0 0.98039216}
\pscustom[linewidth=1,linecolor=curcolor]
{
\newpath
\moveto(117.2,221.3)
\lineto(101.5,245.6)
}
}
{
\newrgbcolor{curcolor}{0.16078432 0 1}
\pscustom[linewidth=1,linecolor=curcolor]
{
\newpath
\moveto(101.5,245.6)
\lineto(85.9,216.5)
}
}
{
\newrgbcolor{curcolor}{0.61176473 0.05098039 0.94901961}
\pscustom[linewidth=1,linecolor=curcolor]
{
\newpath
\moveto(287.9,332.3)
\lineto(272.2,321.9)
}
}
{
\newrgbcolor{curcolor}{0.62352943 0.05882353 0.94117647}
\pscustom[linewidth=1,linecolor=curcolor]
{
\newpath
\moveto(272.2,321.9)
\lineto(256.6,313.9)
}
}
{
\newrgbcolor{curcolor}{0.50196081 0 1}
\pscustom[linewidth=1,linecolor=curcolor]
{
\newpath
\moveto(256.6,313.9)
\lineto(240.9,292.7)
}
}
{
\newrgbcolor{curcolor}{0.42352942 0 1}
\pscustom[linewidth=1,linecolor=curcolor]
{
\newpath
\moveto(240.9,292.7)
\lineto(225.2,288)
}
}
{
\newrgbcolor{curcolor}{0.37254903 0 1}
\pscustom[linewidth=1,linecolor=curcolor]
{
\newpath
\moveto(225.2,288)
\lineto(209.5,269.3)
}
}
{
\newrgbcolor{curcolor}{0.34509805 0 1}
\pscustom[linewidth=1,linecolor=curcolor]
{
\newpath
\moveto(209.5,269.3)
\lineto(193.8,266.6)
}
}
{
\newrgbcolor{curcolor}{0.42745098 0 1}
\pscustom[linewidth=1,linecolor=curcolor]
{
\newpath
\moveto(193.8,266.6)
\lineto(178.2,256.7)
}
}
{
\newrgbcolor{curcolor}{0.58431375 0.03529412 0.96470588}
\pscustom[linewidth=1,linecolor=curcolor]
{
\newpath
\moveto(178.2,256.7)
\lineto(162.5,260.2)
}
}
{
\newrgbcolor{curcolor}{0.40784314 0 1}
\pscustom[linewidth=1,linecolor=curcolor]
{
\newpath
\moveto(162.5,260.2)
\lineto(146.8,222.9)
}
}
{
\newrgbcolor{curcolor}{0 0 0.56078434}
\pscustom[linewidth=1,linecolor=curcolor]
{
\newpath
\moveto(146.8,222.9)
\lineto(131.1,179.4)
}
}
{
\newrgbcolor{curcolor}{0.62352943 0.05882353 0.94117647}
\pscustom[linewidth=1,linecolor=curcolor]
{
\newpath
\moveto(333.1,319.7)
\lineto(317.4,316.9)
}
}
{
\newrgbcolor{curcolor}{0.74509805 0.13725491 0.86274511}
\pscustom[linewidth=1,linecolor=curcolor]
{
\newpath
\moveto(317.4,316.9)
\lineto(301.7,310.5)
}
}
{
\newrgbcolor{curcolor}{0.96862745 0.27843139 0.72156864}
\pscustom[linewidth=1,linecolor=curcolor]
{
\newpath
\moveto(301.7,310.5)
\lineto(286.1,315.9)
}
}
{
\newrgbcolor{curcolor}{1 0.32549021 0.67450982}
\pscustom[linewidth=1,linecolor=curcolor]
{
\newpath
\moveto(286.1,315.9)
\lineto(270.5,297.3)
}
}
{
\newrgbcolor{curcolor}{0.80392158 0.17254902 0.82745099}
\pscustom[linewidth=1,linecolor=curcolor]
{
\newpath
\moveto(270.5,297.3)
\lineto(254.8,276.9)
}
}
{
\newrgbcolor{curcolor}{0.99607843 0.29803923 0.7019608}
\pscustom[linewidth=1,linecolor=curcolor]
{
\newpath
\moveto(254.8,276.9)
\lineto(239.1,293.8)
}
}
{
\newrgbcolor{curcolor}{1 0.55686277 0.44313726}
\pscustom[linewidth=1,linecolor=curcolor]
{
\newpath
\moveto(239.1,293.8)
\lineto(223.4,290.9)
}
}
{
\newrgbcolor{curcolor}{1 0.59607846 0.40392157}
\pscustom[linewidth=1,linecolor=curcolor]
{
\newpath
\moveto(223.4,290.9)
\lineto(207.7,279.7)
}
}
{
\newrgbcolor{curcolor}{1 0.52156866 0.47843137}
\pscustom[linewidth=1,linecolor=curcolor]
{
\newpath
\moveto(207.7,279.7)
\lineto(192.1,261.9)
}
}
{
\newrgbcolor{curcolor}{1 0.35294119 0.64705884}
\pscustom[linewidth=1,linecolor=curcolor]
{
\newpath
\moveto(192.1,261.9)
\lineto(176.4,239.2)
}
}
{
\newrgbcolor{curcolor}{0.74901962 0.13725491 0.86274511}
\pscustom[linewidth=1,linecolor=curcolor]
{
\newpath
\moveto(378.3,318.3)
\lineto(362.7,309.9)
}
}
{
\newrgbcolor{curcolor}{0.85882354 0.20784314 0.79215688}
\pscustom[linewidth=1,linecolor=curcolor]
{
\newpath
\moveto(362.7,309.9)
\lineto(347,308)
}
}
{
\newrgbcolor{curcolor}{1 0.3019608 0.69803923}
\pscustom[linewidth=1,linecolor=curcolor]
{
\newpath
\moveto(347,308)
\lineto(331.3,302.8)
}
}
{
\newrgbcolor{curcolor}{1 0.43529412 0.56470591}
\pscustom[linewidth=1,linecolor=curcolor]
{
\newpath
\moveto(331.3,302.8)
\lineto(315.6,305.6)
}
}
{
\newrgbcolor{curcolor}{0.92156863 0.25098041 0.74901962}
\pscustom[linewidth=1,linecolor=curcolor]
{
\newpath
\moveto(315.6,305.6)
\lineto(300,259.8)
}
}
{
\newrgbcolor{curcolor}{1 0.3764706 0.62352943}
\pscustom[linewidth=1,linecolor=curcolor]
{
\newpath
\moveto(300,259.8)
\lineto(284.4,302.7)
}
}
{
\newrgbcolor{curcolor}{1 0.58039218 0.41960785}
\pscustom[linewidth=1,linecolor=curcolor]
{
\newpath
\moveto(284.4,302.7)
\lineto(268.7,266.5)
}
}
{
\newrgbcolor{curcolor}{1 0.45882353 0.5411765}
\pscustom[linewidth=1,linecolor=curcolor]
{
\newpath
\moveto(268.7,266.5)
\lineto(253,267.7)
}
}
{
\newrgbcolor{curcolor}{1 0.47058824 0.52941179}
\pscustom[linewidth=1,linecolor=curcolor]
{
\newpath
\moveto(253,267.7)
\lineto(237.3,248.8)
}
}
{
\newrgbcolor{curcolor}{1 0.627451 0.37254903}
\pscustom[linewidth=1,linecolor=curcolor]
{
\newpath
\moveto(237.3,248.8)
\lineto(221.7,268.8)
}
}
{
\newrgbcolor{curcolor}{0.94509804 0.26274511 0.73725492}
\pscustom[linewidth=1,linecolor=curcolor]
{
\newpath
\moveto(423.6,319.7)
\lineto(407.9,306.1)
}
}
{
\newrgbcolor{curcolor}{1 0.3137255 0.68627453}
\pscustom[linewidth=1,linecolor=curcolor]
{
\newpath
\moveto(407.9,306.1)
\lineto(392.3,306.9)
}
}
{
\newrgbcolor{curcolor}{1 0.30980393 0.6901961}
\pscustom[linewidth=1,linecolor=curcolor]
{
\newpath
\moveto(392.3,306.9)
\lineto(376.6,286)
}
}
{
\newrgbcolor{curcolor}{0.88627452 0.22745098 0.77254903}
\pscustom[linewidth=1,linecolor=curcolor]
{
\newpath
\moveto(376.6,286)
\lineto(360.9,277)
}
}
{
\newrgbcolor{curcolor}{1 0.40000001 0.60000002}
\pscustom[linewidth=1,linecolor=curcolor]
{
\newpath
\moveto(360.9,277)
\lineto(345.2,289)
}
}
{
\newrgbcolor{curcolor}{1 0.50588238 0.49411765}
\pscustom[linewidth=1,linecolor=curcolor]
{
\newpath
\moveto(345.2,289)
\lineto(329.5,271.2)
}
}
{
\newrgbcolor{curcolor}{1 0.3764706 0.62352943}
\pscustom[linewidth=1,linecolor=curcolor]
{
\newpath
\moveto(329.5,271.2)
\lineto(313.9,253.5)
}
}
{
\newrgbcolor{curcolor}{1 0.4509804 0.54901963}
\pscustom[linewidth=1,linecolor=curcolor]
{
\newpath
\moveto(313.9,253.5)
\lineto(298.3,261.5)
}
}
{
\newrgbcolor{curcolor}{1 0.75686276 0.24313726}
\pscustom[linewidth=1,linecolor=curcolor]
{
\newpath
\moveto(298.3,261.5)
\lineto(282.6,273.3)
}
}
{
\newrgbcolor{curcolor}{1 0.81960785 0.18039216}
\pscustom[linewidth=1,linecolor=curcolor]
{
\newpath
\moveto(282.6,273.3)
\lineto(266.9,250.2)
}
}
{
\newrgbcolor{curcolor}{0.89411765 0.23137255 0.76862746}
\pscustom[linewidth=1,linecolor=curcolor]
{
\newpath
\moveto(468.9,300.2)
\lineto(453.2,303.1)
}
}
{
\newrgbcolor{curcolor}{1 0.41176471 0.58823532}
\pscustom[linewidth=1,linecolor=curcolor]
{
\newpath
\moveto(453.2,303.1)
\lineto(437.5,304.1)
}
}
{
\newrgbcolor{curcolor}{1 0.57254905 0.42745098}
\pscustom[linewidth=1,linecolor=curcolor]
{
\newpath
\moveto(437.5,304.1)
\lineto(421.8,304.3)
}
}
{
\newrgbcolor{curcolor}{1 0.70588237 0.29411766}
\pscustom[linewidth=1,linecolor=curcolor]
{
\newpath
\moveto(421.8,304.3)
\lineto(406.2,302.1)
}
}
{
\newrgbcolor{curcolor}{1 0.76078433 0.23921569}
\pscustom[linewidth=1,linecolor=curcolor]
{
\newpath
\moveto(406.2,302.1)
\lineto(390.5,292)
}
}
{
\newrgbcolor{curcolor}{1 0.627451 0.37254903}
\pscustom[linewidth=1,linecolor=curcolor]
{
\newpath
\moveto(390.5,292)
\lineto(374.8,265.6)
}
}
{
\newrgbcolor{curcolor}{1 0.59215689 0.40784314}
\pscustom[linewidth=1,linecolor=curcolor]
{
\newpath
\moveto(374.8,265.6)
\lineto(359.1,268.2)
}
}
{
\newrgbcolor{curcolor}{1 0.61960787 0.38039216}
\pscustom[linewidth=1,linecolor=curcolor]
{
\newpath
\moveto(359.1,268.2)
\lineto(343.4,250)
}
}
{
\newrgbcolor{curcolor}{1 0.60000002 0.40000001}
\pscustom[linewidth=1,linecolor=curcolor]
{
\newpath
\moveto(343.4,250)
\lineto(327.8,246.2)
}
}
{
\newrgbcolor{curcolor}{1 0.53333336 0.46666667}
\pscustom[linewidth=1,linecolor=curcolor]
{
\newpath
\moveto(327.8,246.2)
\lineto(312.1,222.2)
}
}
{
\newrgbcolor{curcolor}{0.80392158 0.17254902 0.82745099}
\pscustom[linewidth=1,linecolor=curcolor]
{
\newpath
\moveto(514.1,303.4)
\lineto(498.5,273.6)
}
}
{
\newrgbcolor{curcolor}{0.87058824 0.21568628 0.78431374}
\pscustom[linewidth=1,linecolor=curcolor]
{
\newpath
\moveto(498.5,273.6)
\lineto(482.8,289.7)
}
}
{
\newrgbcolor{curcolor}{1 0.43529412 0.56470591}
\pscustom[linewidth=1,linecolor=curcolor]
{
\newpath
\moveto(482.8,289.7)
\lineto(467.1,282.6)
}
}
{
\newrgbcolor{curcolor}{1 0.47450981 0.52549022}
\pscustom[linewidth=1,linecolor=curcolor]
{
\newpath
\moveto(467.1,282.6)
\lineto(451.4,275.2)
}
}
{
\newrgbcolor{curcolor}{1 0.48235294 0.51764709}
\pscustom[linewidth=1,linecolor=curcolor]
{
\newpath
\moveto(451.4,275.2)
\lineto(435.7,264.5)
}
}
{
\newrgbcolor{curcolor}{1 0.65098041 0.34901962}
\pscustom[linewidth=1,linecolor=curcolor]
{
\newpath
\moveto(435.7,264.5)
\lineto(420.1,277.6)
}
}
{
\newrgbcolor{curcolor}{1 0.74509805 0.25490198}
\pscustom[linewidth=1,linecolor=curcolor]
{
\newpath
\moveto(420.1,277.6)
\lineto(404.4,257.1)
}
}
{
\newrgbcolor{curcolor}{1 0.66274512 0.33725491}
\pscustom[linewidth=1,linecolor=curcolor]
{
\newpath
\moveto(404.4,257.1)
\lineto(388.7,247.8)
}
}
{
\newrgbcolor{curcolor}{1 0.65882355 0.34117648}
\pscustom[linewidth=1,linecolor=curcolor]
{
\newpath
\moveto(388.7,247.8)
\lineto(373,237.3)
}
}
{
\newrgbcolor{curcolor}{1 0.64313728 0.35686275}
\pscustom[linewidth=1,linecolor=curcolor]
{
\newpath
\moveto(373,237.3)
\lineto(357.3,226.8)
}
}
{
\newrgbcolor{curcolor}{0 0 0}
\pscustom[linewidth=1,linecolor=curcolor]
{
\newpath
\moveto(514.1,164)
\lineto(357.3,67.6)
}
}
{
\newrgbcolor{curcolor}{0 0 0}
\pscustom[linewidth=1,linecolor=curcolor]
{
\newpath
\moveto(85.9,123.2)
\lineto(357.3,67.6)
}
}
{
\newrgbcolor{curcolor}{0 0 0}
\pscustom[linewidth=1,linecolor=curcolor]
{
\newpath
\moveto(357.3,67.6)
\lineto(357.3,226.8)
}
}
{
\newrgbcolor{curcolor}{0 0 0}
\pscustom[linestyle=none,fillstyle=solid,fillcolor=curcolor]
{
\newpath
\moveto(146.27753906,58.7)
\lineto(146.27753906,67.28984375)
\lineto(152.07246094,67.28984375)
\lineto(152.07246094,66.27617187)
\lineto(147.41425781,66.27617187)
\lineto(147.41425781,63.61601562)
\lineto(151.44550781,63.61601562)
\lineto(151.44550781,62.60234375)
\lineto(147.41425781,62.60234375)
\lineto(147.41425781,58.7)
\closepath
}
}
{
\newrgbcolor{curcolor}{0 0 0}
\pscustom[linestyle=none,fillstyle=solid,fillcolor=curcolor]
{
\newpath
\moveto(153.42011719,66.07695312)
\lineto(153.42011719,67.28984375)
\lineto(154.47480469,67.28984375)
\lineto(154.47480469,66.07695312)
\closepath
\moveto(153.42011719,58.7)
\lineto(153.42011719,64.92265625)
\lineto(154.47480469,64.92265625)
\lineto(154.47480469,58.7)
\closepath
}
}
{
\newrgbcolor{curcolor}{0 0 0}
\pscustom[linestyle=none,fillstyle=solid,fillcolor=curcolor]
{
\newpath
\moveto(156.05683594,58.7)
\lineto(156.05683594,67.28984375)
\lineto(157.11152344,67.28984375)
\lineto(157.11152344,58.7)
\closepath
}
}
{
\newrgbcolor{curcolor}{0 0 0}
\pscustom[linestyle=none,fillstyle=solid,fillcolor=curcolor]
{
\newpath
\moveto(163.00605469,60.70390625)
\lineto(164.09589844,60.56914062)
\curveto(163.92402344,59.93242187)(163.60566406,59.43828125)(163.14082031,59.08671875)
\curveto(162.67597656,58.73515625)(162.08222656,58.559375)(161.35957031,58.559375)
\curveto(160.44941406,58.559375)(159.72675781,58.83867187)(159.19160156,59.39726562)
\curveto(158.66035156,59.95976562)(158.39472656,60.746875)(158.39472656,61.75859375)
\curveto(158.39472656,62.80546875)(158.66425781,63.61796875)(159.20332031,64.19609375)
\curveto(159.74238281,64.77421875)(160.44160156,65.06328125)(161.30097656,65.06328125)
\curveto(162.13300781,65.06328125)(162.81269531,64.78007812)(163.34003906,64.21367187)
\curveto(163.86738281,63.64726562)(164.13105469,62.85039062)(164.13105469,61.82304687)
\curveto(164.13105469,61.76054687)(164.12910156,61.66679687)(164.12519531,61.54179687)
\lineto(159.48457031,61.54179687)
\curveto(159.52363281,60.85820312)(159.71699219,60.33476562)(160.06464844,59.97148437)
\curveto(160.41230469,59.60820312)(160.84589844,59.4265625)(161.36542969,59.4265625)
\curveto(161.75214844,59.4265625)(162.08222656,59.528125)(162.35566406,59.73125)
\curveto(162.62910156,59.934375)(162.84589844,60.25859375)(163.00605469,60.70390625)
\closepath
\moveto(159.54316406,62.40898437)
\lineto(163.01777344,62.40898437)
\curveto(162.97089844,62.93242187)(162.83808594,63.325)(162.61933594,63.58671875)
\curveto(162.28339844,63.99296875)(161.84785156,64.19609375)(161.31269531,64.19609375)
\curveto(160.82832031,64.19609375)(160.42011719,64.03398437)(160.08808594,63.70976562)
\curveto(159.75996094,63.38554687)(159.57832031,62.95195312)(159.54316406,62.40898437)
\closepath
}
}
{
\newrgbcolor{curcolor}{0 0 0}
\pscustom[linestyle=none,fillstyle=solid,fillcolor=curcolor]
{
\newpath
\moveto(168.50214844,61.45976562)
\lineto(169.57441406,61.55351562)
\curveto(169.62519531,61.12382812)(169.74238281,60.7703125)(169.92597656,60.49296875)
\curveto(170.11347656,60.21953125)(170.40253906,59.996875)(170.79316406,59.825)
\curveto(171.18378906,59.65703125)(171.62324219,59.57304687)(172.11152344,59.57304687)
\curveto(172.54511719,59.57304687)(172.92792969,59.6375)(173.25996094,59.76640625)
\curveto(173.59199219,59.8953125)(173.83808594,60.07109375)(173.99824219,60.29375)
\curveto(174.16230469,60.5203125)(174.24433594,60.76640625)(174.24433594,61.03203125)
\curveto(174.24433594,61.3015625)(174.16621094,61.5359375)(174.00996094,61.73515625)
\curveto(173.85371094,61.93828125)(173.59589844,62.10820312)(173.23652344,62.24492187)
\curveto(173.00605469,62.33476562)(172.49628906,62.4734375)(171.70722656,62.6609375)
\curveto(170.91816406,62.85234375)(170.36542969,63.03203125)(170.04902344,63.2)
\curveto(169.63886719,63.41484375)(169.33222656,63.68046875)(169.12910156,63.996875)
\curveto(168.92988281,64.3171875)(168.83027344,64.67460937)(168.83027344,65.06914062)
\curveto(168.83027344,65.50273437)(168.95332031,65.90703125)(169.19941406,66.28203125)
\curveto(169.44550781,66.6609375)(169.80488281,66.94804687)(170.27753906,67.14335937)
\curveto(170.75019531,67.33867187)(171.27558594,67.43632812)(171.85371094,67.43632812)
\curveto(172.49042969,67.43632812)(173.05097656,67.3328125)(173.53535156,67.12578125)
\curveto(174.02363281,66.92265625)(174.39863281,66.621875)(174.66035156,66.2234375)
\curveto(174.92207031,65.825)(175.06269531,65.37382812)(175.08222656,64.86992187)
\lineto(173.99238281,64.78789062)
\curveto(173.93378906,65.33085937)(173.73457031,65.74101562)(173.39472656,66.01835937)
\curveto(173.05878906,66.29570312)(172.56074219,66.434375)(171.90058594,66.434375)
\curveto(171.21308594,66.434375)(170.71113281,66.30742187)(170.39472656,66.05351562)
\curveto(170.08222656,65.80351562)(169.92597656,65.50078125)(169.92597656,65.1453125)
\curveto(169.92597656,64.83671875)(170.03730469,64.5828125)(170.25996094,64.38359375)
\curveto(170.47871094,64.184375)(171.04902344,63.97929687)(171.97089844,63.76835937)
\curveto(172.89667969,63.56132812)(173.53144531,63.3796875)(173.87519531,63.2234375)
\curveto(174.37519531,62.99296875)(174.74433594,62.7)(174.98261719,62.34453125)
\curveto(175.22089844,61.99296875)(175.34003906,61.58671875)(175.34003906,61.12578125)
\curveto(175.34003906,60.66875)(175.20917969,60.23710937)(174.94746094,59.83085937)
\curveto(174.68574219,59.42851562)(174.30878906,59.1140625)(173.81660156,58.8875)
\curveto(173.32832031,58.66484375)(172.77753906,58.55351562)(172.16425781,58.55351562)
\curveto(171.38691406,58.55351562)(170.73457031,58.66679687)(170.20722656,58.89335937)
\curveto(169.68378906,59.11992187)(169.27167969,59.45976562)(168.97089844,59.91289062)
\curveto(168.67402344,60.36992187)(168.51777344,60.88554687)(168.50214844,61.45976562)
\closepath
}
}
{
\newrgbcolor{curcolor}{0 0 0}
\pscustom[linestyle=none,fillstyle=solid,fillcolor=curcolor]
{
\newpath
\moveto(176.76386719,66.07695312)
\lineto(176.76386719,67.28984375)
\lineto(177.81855469,67.28984375)
\lineto(177.81855469,66.07695312)
\closepath
\moveto(176.76386719,58.7)
\lineto(176.76386719,64.92265625)
\lineto(177.81855469,64.92265625)
\lineto(177.81855469,58.7)
\closepath
}
}
{
\newrgbcolor{curcolor}{0 0 0}
\pscustom[linestyle=none,fillstyle=solid,fillcolor=curcolor]
{
\newpath
\moveto(178.86738281,58.7)
\lineto(178.86738281,59.55546875)
\lineto(182.82832031,64.10234375)
\curveto(182.37910156,64.07890625)(181.98261719,64.0671875)(181.63886719,64.0671875)
\lineto(179.10175781,64.0671875)
\lineto(179.10175781,64.92265625)
\lineto(184.18769531,64.92265625)
\lineto(184.18769531,64.22539062)
\lineto(180.81855469,60.27617187)
\lineto(180.16816406,59.55546875)
\curveto(180.64082031,59.590625)(181.08417969,59.60820312)(181.49824219,59.60820312)
\lineto(184.37519531,59.60820312)
\lineto(184.37519531,58.7)
\closepath
}
}
{
\newrgbcolor{curcolor}{0 0 0}
\pscustom[linestyle=none,fillstyle=solid,fillcolor=curcolor]
{
\newpath
\moveto(189.68378906,60.70390625)
\lineto(190.77363281,60.56914062)
\curveto(190.60175781,59.93242187)(190.28339844,59.43828125)(189.81855469,59.08671875)
\curveto(189.35371094,58.73515625)(188.75996094,58.559375)(188.03730469,58.559375)
\curveto(187.12714844,58.559375)(186.40449219,58.83867187)(185.86933594,59.39726562)
\curveto(185.33808594,59.95976562)(185.07246094,60.746875)(185.07246094,61.75859375)
\curveto(185.07246094,62.80546875)(185.34199219,63.61796875)(185.88105469,64.19609375)
\curveto(186.42011719,64.77421875)(187.11933594,65.06328125)(187.97871094,65.06328125)
\curveto(188.81074219,65.06328125)(189.49042969,64.78007812)(190.01777344,64.21367187)
\curveto(190.54511719,63.64726562)(190.80878906,62.85039062)(190.80878906,61.82304687)
\curveto(190.80878906,61.76054687)(190.80683594,61.66679687)(190.80292969,61.54179687)
\lineto(186.16230469,61.54179687)
\curveto(186.20136719,60.85820312)(186.39472656,60.33476562)(186.74238281,59.97148437)
\curveto(187.09003906,59.60820312)(187.52363281,59.4265625)(188.04316406,59.4265625)
\curveto(188.42988281,59.4265625)(188.75996094,59.528125)(189.03339844,59.73125)
\curveto(189.30683594,59.934375)(189.52363281,60.25859375)(189.68378906,60.70390625)
\closepath
\moveto(186.22089844,62.40898437)
\lineto(189.69550781,62.40898437)
\curveto(189.64863281,62.93242187)(189.51582031,63.325)(189.29707031,63.58671875)
\curveto(188.96113281,63.99296875)(188.52558594,64.19609375)(187.99042969,64.19609375)
\curveto(187.50605469,64.19609375)(187.09785156,64.03398437)(186.76582031,63.70976562)
\curveto(186.43769531,63.38554687)(186.25605469,62.95195312)(186.22089844,62.40898437)
\closepath
}
}
{
\newrgbcolor{curcolor}{0 0 0}
\pscustom[linestyle=none,fillstyle=solid,fillcolor=curcolor]
{
\newpath
\moveto(494.11308594,92.9)
\lineto(494.11308594,101.48984375)
\lineto(497.35332031,101.48984375)
\curveto(497.92363281,101.48984375)(498.35917969,101.4625)(498.65996094,101.4078125)
\curveto(499.08183594,101.3375)(499.43535156,101.20273437)(499.72050781,101.00351562)
\curveto(500.00566406,100.80820312)(500.23417969,100.5328125)(500.40605469,100.17734375)
\curveto(500.58183594,99.821875)(500.66972656,99.43125)(500.66972656,99.00546875)
\curveto(500.66972656,98.275)(500.43730469,97.65585937)(499.97246094,97.14804687)
\curveto(499.50761719,96.64414062)(498.66777344,96.3921875)(497.45292969,96.3921875)
\lineto(495.24980469,96.3921875)
\lineto(495.24980469,92.9)
\closepath
\moveto(495.24980469,97.40585937)
\lineto(497.47050781,97.40585937)
\curveto(498.20488281,97.40585937)(498.72636719,97.54257812)(499.03496094,97.81601562)
\curveto(499.34355469,98.08945312)(499.49785156,98.47421875)(499.49785156,98.9703125)
\curveto(499.49785156,99.3296875)(499.40605469,99.63632812)(499.22246094,99.89023437)
\curveto(499.04277344,100.14804687)(498.80449219,100.31796875)(498.50761719,100.4)
\curveto(498.31621094,100.45078125)(497.96269531,100.47617187)(497.44707031,100.47617187)
\lineto(495.24980469,100.47617187)
\closepath
}
}
{
\newrgbcolor{curcolor}{0 0 0}
\pscustom[linestyle=none,fillstyle=solid,fillcolor=curcolor]
{
\newpath
\moveto(501.97050781,92.9)
\lineto(501.97050781,99.12265625)
\lineto(502.91972656,99.12265625)
\lineto(502.91972656,98.17929687)
\curveto(503.16191406,98.62070312)(503.38457031,98.91171875)(503.58769531,99.05234375)
\curveto(503.79472656,99.19296875)(504.02128906,99.26328125)(504.26738281,99.26328125)
\curveto(504.62285156,99.26328125)(504.98417969,99.15)(505.35136719,98.9234375)
\lineto(504.98808594,97.94492187)
\curveto(504.73027344,98.09726562)(504.47246094,98.1734375)(504.21464844,98.1734375)
\curveto(503.98417969,98.1734375)(503.77714844,98.103125)(503.59355469,97.9625)
\curveto(503.40996094,97.82578125)(503.27910156,97.634375)(503.20097656,97.38828125)
\curveto(503.08378906,97.01328125)(503.02519531,96.603125)(503.02519531,96.1578125)
\lineto(503.02519531,92.9)
\closepath
}
}
{
\newrgbcolor{curcolor}{0 0 0}
\pscustom[linestyle=none,fillstyle=solid,fillcolor=curcolor]
{
\newpath
\moveto(505.58574219,96.01132812)
\curveto(505.58574219,97.16367187)(505.90605469,98.0171875)(506.54667969,98.571875)
\curveto(507.08183594,99.0328125)(507.73417969,99.26328125)(508.50371094,99.26328125)
\curveto(509.35917969,99.26328125)(510.05839844,98.98203125)(510.60136719,98.41953125)
\curveto(511.14433594,97.8609375)(511.41582031,97.0875)(511.41582031,96.09921875)
\curveto(511.41582031,95.2984375)(511.29472656,94.66757812)(511.05253906,94.20664062)
\curveto(510.81425781,93.74960937)(510.46464844,93.39414062)(510.00371094,93.14023437)
\curveto(509.54667969,92.88632812)(509.04667969,92.759375)(508.50371094,92.759375)
\curveto(507.63261719,92.759375)(506.92753906,93.03867187)(506.38847656,93.59726562)
\curveto(505.85332031,94.15585937)(505.58574219,94.96054687)(505.58574219,96.01132812)
\closepath
\moveto(506.66972656,96.01132812)
\curveto(506.66972656,95.21445312)(506.84355469,94.61679687)(507.19121094,94.21835937)
\curveto(507.53886719,93.82382812)(507.97636719,93.6265625)(508.50371094,93.6265625)
\curveto(509.02714844,93.6265625)(509.46269531,93.82578125)(509.81035156,94.22421875)
\curveto(510.15800781,94.62265625)(510.33183594,95.23007812)(510.33183594,96.04648437)
\curveto(510.33183594,96.81601562)(510.15605469,97.39804687)(509.80449219,97.79257812)
\curveto(509.45683594,98.19101562)(509.02324219,98.39023437)(508.50371094,98.39023437)
\curveto(507.97636719,98.39023437)(507.53886719,98.19296875)(507.19121094,97.7984375)
\curveto(506.84355469,97.40390625)(506.66972656,96.80820312)(506.66972656,96.01132812)
\closepath
}
}
{
\newrgbcolor{curcolor}{0 0 0}
\pscustom[linestyle=none,fillstyle=solid,fillcolor=curcolor]
{
\newpath
\moveto(516.71269531,95.17929687)
\lineto(517.74980469,95.04453125)
\curveto(517.63652344,94.3296875)(517.34550781,93.76914062)(516.87675781,93.36289062)
\curveto(516.41191406,92.96054687)(515.83964844,92.759375)(515.15996094,92.759375)
\curveto(514.30839844,92.759375)(513.62285156,93.03671875)(513.10332031,93.59140625)
\curveto(512.58769531,94.15)(512.32988281,94.94882812)(512.32988281,95.98789062)
\curveto(512.32988281,96.65976562)(512.44121094,97.24765625)(512.66386719,97.7515625)
\curveto(512.88652344,98.25546875)(513.22441406,98.63242187)(513.67753906,98.88242187)
\curveto(514.13457031,99.13632812)(514.63066406,99.26328125)(515.16582031,99.26328125)
\curveto(515.84160156,99.26328125)(516.39433594,99.09140625)(516.82402344,98.74765625)
\curveto(517.25371094,98.4078125)(517.52910156,97.9234375)(517.65019531,97.29453125)
\lineto(516.62480469,97.13632812)
\curveto(516.52714844,97.55429687)(516.35332031,97.86875)(516.10332031,98.0796875)
\curveto(515.85722656,98.290625)(515.55839844,98.39609375)(515.20683594,98.39609375)
\curveto(514.67558594,98.39609375)(514.24394531,98.2046875)(513.91191406,97.821875)
\curveto(513.57988281,97.44296875)(513.41386719,96.84140625)(513.41386719,96.0171875)
\curveto(513.41386719,95.18125)(513.57402344,94.57382812)(513.89433594,94.19492187)
\curveto(514.21464844,93.81601562)(514.63261719,93.6265625)(515.14824219,93.6265625)
\curveto(515.56230469,93.6265625)(515.90800781,93.75351562)(516.18535156,94.00742187)
\curveto(516.46269531,94.26132812)(516.63847656,94.65195312)(516.71269531,95.17929687)
\closepath
}
}
{
\newrgbcolor{curcolor}{0 0 0}
\pscustom[linestyle=none,fillstyle=solid,fillcolor=curcolor]
{
\newpath
\moveto(522.91191406,94.90390625)
\lineto(524.00175781,94.76914062)
\curveto(523.82988281,94.13242187)(523.51152344,93.63828125)(523.04667969,93.28671875)
\curveto(522.58183594,92.93515625)(521.98808594,92.759375)(521.26542969,92.759375)
\curveto(520.35527344,92.759375)(519.63261719,93.03867187)(519.09746094,93.59726562)
\curveto(518.56621094,94.15976562)(518.30058594,94.946875)(518.30058594,95.95859375)
\curveto(518.30058594,97.00546875)(518.57011719,97.81796875)(519.10917969,98.39609375)
\curveto(519.64824219,98.97421875)(520.34746094,99.26328125)(521.20683594,99.26328125)
\curveto(522.03886719,99.26328125)(522.71855469,98.98007812)(523.24589844,98.41367187)
\curveto(523.77324219,97.84726562)(524.03691406,97.05039062)(524.03691406,96.02304687)
\curveto(524.03691406,95.96054687)(524.03496094,95.86679687)(524.03105469,95.74179687)
\lineto(519.39042969,95.74179687)
\curveto(519.42949219,95.05820312)(519.62285156,94.53476562)(519.97050781,94.17148437)
\curveto(520.31816406,93.80820312)(520.75175781,93.6265625)(521.27128906,93.6265625)
\curveto(521.65800781,93.6265625)(521.98808594,93.728125)(522.26152344,93.93125)
\curveto(522.53496094,94.134375)(522.75175781,94.45859375)(522.91191406,94.90390625)
\closepath
\moveto(519.44902344,96.60898437)
\lineto(522.92363281,96.60898437)
\curveto(522.87675781,97.13242187)(522.74394531,97.525)(522.52519531,97.78671875)
\curveto(522.18925781,98.19296875)(521.75371094,98.39609375)(521.21855469,98.39609375)
\curveto(520.73417969,98.39609375)(520.32597656,98.23398437)(519.99394531,97.90976562)
\curveto(519.66582031,97.58554687)(519.48417969,97.15195312)(519.44902344,96.60898437)
\closepath
}
}
{
\newrgbcolor{curcolor}{0 0 0}
\pscustom[linestyle=none,fillstyle=solid,fillcolor=curcolor]
{
\newpath
\moveto(524.90410156,94.75742187)
\lineto(525.94707031,94.92148437)
\curveto(526.00566406,94.50351562)(526.16777344,94.18320312)(526.43339844,93.96054687)
\curveto(526.70292969,93.73789062)(527.07792969,93.6265625)(527.55839844,93.6265625)
\curveto(528.04277344,93.6265625)(528.40214844,93.72421875)(528.63652344,93.91953125)
\curveto(528.87089844,94.11875)(528.98808594,94.35117187)(528.98808594,94.61679687)
\curveto(528.98808594,94.85507812)(528.88457031,95.04257812)(528.67753906,95.17929687)
\curveto(528.53300781,95.27304687)(528.17363281,95.3921875)(527.59941406,95.53671875)
\curveto(526.82597656,95.73203125)(526.28886719,95.9)(525.98808594,96.040625)
\curveto(525.69121094,96.18515625)(525.46464844,96.38242187)(525.30839844,96.63242187)
\curveto(525.15605469,96.88632812)(525.07988281,97.165625)(525.07988281,97.4703125)
\curveto(525.07988281,97.74765625)(525.14238281,98.00351562)(525.26738281,98.23789062)
\curveto(525.39628906,98.47617187)(525.57011719,98.6734375)(525.78886719,98.8296875)
\curveto(525.95292969,98.95078125)(526.17558594,99.05234375)(526.45683594,99.134375)
\curveto(526.74199219,99.2203125)(527.04667969,99.26328125)(527.37089844,99.26328125)
\curveto(527.85917969,99.26328125)(528.28691406,99.19296875)(528.65410156,99.05234375)
\curveto(529.02519531,98.91171875)(529.29863281,98.7203125)(529.47441406,98.478125)
\curveto(529.65019531,98.23984375)(529.77128906,97.91953125)(529.83769531,97.5171875)
\lineto(528.80644531,97.3765625)
\curveto(528.75957031,97.696875)(528.62285156,97.946875)(528.39628906,98.1265625)
\curveto(528.17363281,98.30625)(527.85722656,98.39609375)(527.44707031,98.39609375)
\curveto(526.96269531,98.39609375)(526.61699219,98.31601562)(526.40996094,98.15585937)
\curveto(526.20292969,97.99570312)(526.09941406,97.80820312)(526.09941406,97.59335937)
\curveto(526.09941406,97.45664062)(526.14238281,97.33359375)(526.22832031,97.22421875)
\curveto(526.31425781,97.1109375)(526.44902344,97.0171875)(526.63261719,96.94296875)
\curveto(526.73808594,96.90390625)(527.04863281,96.8140625)(527.56425781,96.6734375)
\curveto(528.31035156,96.47421875)(528.82988281,96.31015625)(529.12285156,96.18125)
\curveto(529.41972656,96.05625)(529.65214844,95.87265625)(529.82011719,95.63046875)
\curveto(529.98808594,95.38828125)(530.07207031,95.0875)(530.07207031,94.728125)
\curveto(530.07207031,94.3765625)(529.96855469,94.04453125)(529.76152344,93.73203125)
\curveto(529.55839844,93.4234375)(529.26347656,93.18320312)(528.87675781,93.01132812)
\curveto(528.49003906,92.84335937)(528.05253906,92.759375)(527.56425781,92.759375)
\curveto(526.75566406,92.759375)(526.13847656,92.92734375)(525.71269531,93.26328125)
\curveto(525.29082031,93.59921875)(525.02128906,94.09726562)(524.90410156,94.75742187)
\closepath
}
}
{
\newrgbcolor{curcolor}{0 0 0}
\pscustom[linestyle=none,fillstyle=solid,fillcolor=curcolor]
{
\newpath
\moveto(530.90410156,94.75742187)
\lineto(531.94707031,94.92148437)
\curveto(532.00566406,94.50351562)(532.16777344,94.18320312)(532.43339844,93.96054687)
\curveto(532.70292969,93.73789062)(533.07792969,93.6265625)(533.55839844,93.6265625)
\curveto(534.04277344,93.6265625)(534.40214844,93.72421875)(534.63652344,93.91953125)
\curveto(534.87089844,94.11875)(534.98808594,94.35117187)(534.98808594,94.61679687)
\curveto(534.98808594,94.85507812)(534.88457031,95.04257812)(534.67753906,95.17929687)
\curveto(534.53300781,95.27304687)(534.17363281,95.3921875)(533.59941406,95.53671875)
\curveto(532.82597656,95.73203125)(532.28886719,95.9)(531.98808594,96.040625)
\curveto(531.69121094,96.18515625)(531.46464844,96.38242187)(531.30839844,96.63242187)
\curveto(531.15605469,96.88632812)(531.07988281,97.165625)(531.07988281,97.4703125)
\curveto(531.07988281,97.74765625)(531.14238281,98.00351562)(531.26738281,98.23789062)
\curveto(531.39628906,98.47617187)(531.57011719,98.6734375)(531.78886719,98.8296875)
\curveto(531.95292969,98.95078125)(532.17558594,99.05234375)(532.45683594,99.134375)
\curveto(532.74199219,99.2203125)(533.04667969,99.26328125)(533.37089844,99.26328125)
\curveto(533.85917969,99.26328125)(534.28691406,99.19296875)(534.65410156,99.05234375)
\curveto(535.02519531,98.91171875)(535.29863281,98.7203125)(535.47441406,98.478125)
\curveto(535.65019531,98.23984375)(535.77128906,97.91953125)(535.83769531,97.5171875)
\lineto(534.80644531,97.3765625)
\curveto(534.75957031,97.696875)(534.62285156,97.946875)(534.39628906,98.1265625)
\curveto(534.17363281,98.30625)(533.85722656,98.39609375)(533.44707031,98.39609375)
\curveto(532.96269531,98.39609375)(532.61699219,98.31601562)(532.40996094,98.15585937)
\curveto(532.20292969,97.99570312)(532.09941406,97.80820312)(532.09941406,97.59335937)
\curveto(532.09941406,97.45664062)(532.14238281,97.33359375)(532.22832031,97.22421875)
\curveto(532.31425781,97.1109375)(532.44902344,97.0171875)(532.63261719,96.94296875)
\curveto(532.73808594,96.90390625)(533.04863281,96.8140625)(533.56425781,96.6734375)
\curveto(534.31035156,96.47421875)(534.82988281,96.31015625)(535.12285156,96.18125)
\curveto(535.41972656,96.05625)(535.65214844,95.87265625)(535.82011719,95.63046875)
\curveto(535.98808594,95.38828125)(536.07207031,95.0875)(536.07207031,94.728125)
\curveto(536.07207031,94.3765625)(535.96855469,94.04453125)(535.76152344,93.73203125)
\curveto(535.55839844,93.4234375)(535.26347656,93.18320312)(534.87675781,93.01132812)
\curveto(534.49003906,92.84335937)(534.05253906,92.759375)(533.56425781,92.759375)
\curveto(532.75566406,92.759375)(532.13847656,92.92734375)(531.71269531,93.26328125)
\curveto(531.29082031,93.59921875)(531.02128906,94.09726562)(530.90410156,94.75742187)
\closepath
}
}
{
\newrgbcolor{curcolor}{0 0 0}
\pscustom[linestyle=none,fillstyle=solid,fillcolor=curcolor]
{
\newpath
\moveto(540.40800781,95.65976562)
\lineto(541.48027344,95.75351562)
\curveto(541.53105469,95.32382812)(541.64824219,94.9703125)(541.83183594,94.69296875)
\curveto(542.01933594,94.41953125)(542.30839844,94.196875)(542.69902344,94.025)
\curveto(543.08964844,93.85703125)(543.52910156,93.77304687)(544.01738281,93.77304687)
\curveto(544.45097656,93.77304687)(544.83378906,93.8375)(545.16582031,93.96640625)
\curveto(545.49785156,94.0953125)(545.74394531,94.27109375)(545.90410156,94.49375)
\curveto(546.06816406,94.7203125)(546.15019531,94.96640625)(546.15019531,95.23203125)
\curveto(546.15019531,95.5015625)(546.07207031,95.7359375)(545.91582031,95.93515625)
\curveto(545.75957031,96.13828125)(545.50175781,96.30820312)(545.14238281,96.44492187)
\curveto(544.91191406,96.53476562)(544.40214844,96.6734375)(543.61308594,96.8609375)
\curveto(542.82402344,97.05234375)(542.27128906,97.23203125)(541.95488281,97.4)
\curveto(541.54472656,97.61484375)(541.23808594,97.88046875)(541.03496094,98.196875)
\curveto(540.83574219,98.5171875)(540.73613281,98.87460937)(540.73613281,99.26914062)
\curveto(540.73613281,99.70273437)(540.85917969,100.10703125)(541.10527344,100.48203125)
\curveto(541.35136719,100.8609375)(541.71074219,101.14804687)(542.18339844,101.34335937)
\curveto(542.65605469,101.53867187)(543.18144531,101.63632812)(543.75957031,101.63632812)
\curveto(544.39628906,101.63632812)(544.95683594,101.5328125)(545.44121094,101.32578125)
\curveto(545.92949219,101.12265625)(546.30449219,100.821875)(546.56621094,100.4234375)
\curveto(546.82792969,100.025)(546.96855469,99.57382812)(546.98808594,99.06992187)
\lineto(545.89824219,98.98789062)
\curveto(545.83964844,99.53085937)(545.64042969,99.94101562)(545.30058594,100.21835937)
\curveto(544.96464844,100.49570312)(544.46660156,100.634375)(543.80644531,100.634375)
\curveto(543.11894531,100.634375)(542.61699219,100.50742187)(542.30058594,100.25351562)
\curveto(541.98808594,100.00351562)(541.83183594,99.70078125)(541.83183594,99.3453125)
\curveto(541.83183594,99.03671875)(541.94316406,98.7828125)(542.16582031,98.58359375)
\curveto(542.38457031,98.384375)(542.95488281,98.17929687)(543.87675781,97.96835937)
\curveto(544.80253906,97.76132812)(545.43730469,97.5796875)(545.78105469,97.4234375)
\curveto(546.28105469,97.19296875)(546.65019531,96.9)(546.88847656,96.54453125)
\curveto(547.12675781,96.19296875)(547.24589844,95.78671875)(547.24589844,95.32578125)
\curveto(547.24589844,94.86875)(547.11503906,94.43710937)(546.85332031,94.03085937)
\curveto(546.59160156,93.62851562)(546.21464844,93.3140625)(545.72246094,93.0875)
\curveto(545.23417969,92.86484375)(544.68339844,92.75351562)(544.07011719,92.75351562)
\curveto(543.29277344,92.75351562)(542.64042969,92.86679687)(542.11308594,93.09335937)
\curveto(541.58964844,93.31992187)(541.17753906,93.65976562)(540.87675781,94.11289062)
\curveto(540.57988281,94.56992187)(540.42363281,95.08554687)(540.40800781,95.65976562)
\closepath
}
}
{
\newrgbcolor{curcolor}{0 0 0}
\pscustom[linestyle=none,fillstyle=solid,fillcolor=curcolor]
{
\newpath
\moveto(548.66972656,100.27695312)
\lineto(548.66972656,101.48984375)
\lineto(549.72441406,101.48984375)
\lineto(549.72441406,100.27695312)
\closepath
\moveto(548.66972656,92.9)
\lineto(548.66972656,99.12265625)
\lineto(549.72441406,99.12265625)
\lineto(549.72441406,92.9)
\closepath
}
}
{
\newrgbcolor{curcolor}{0 0 0}
\pscustom[linestyle=none,fillstyle=solid,fillcolor=curcolor]
{
\newpath
\moveto(550.77324219,92.9)
\lineto(550.77324219,93.75546875)
\lineto(554.73417969,98.30234375)
\curveto(554.28496094,98.27890625)(553.88847656,98.2671875)(553.54472656,98.2671875)
\lineto(551.00761719,98.2671875)
\lineto(551.00761719,99.12265625)
\lineto(556.09355469,99.12265625)
\lineto(556.09355469,98.42539062)
\lineto(552.72441406,94.47617187)
\lineto(552.07402344,93.75546875)
\curveto(552.54667969,93.790625)(552.99003906,93.80820312)(553.40410156,93.80820312)
\lineto(556.28105469,93.80820312)
\lineto(556.28105469,92.9)
\closepath
}
}
{
\newrgbcolor{curcolor}{0 0 0}
\pscustom[linestyle=none,fillstyle=solid,fillcolor=curcolor]
{
\newpath
\moveto(561.58964844,94.90390625)
\lineto(562.67949219,94.76914062)
\curveto(562.50761719,94.13242187)(562.18925781,93.63828125)(561.72441406,93.28671875)
\curveto(561.25957031,92.93515625)(560.66582031,92.759375)(559.94316406,92.759375)
\curveto(559.03300781,92.759375)(558.31035156,93.03867187)(557.77519531,93.59726562)
\curveto(557.24394531,94.15976562)(556.97832031,94.946875)(556.97832031,95.95859375)
\curveto(556.97832031,97.00546875)(557.24785156,97.81796875)(557.78691406,98.39609375)
\curveto(558.32597656,98.97421875)(559.02519531,99.26328125)(559.88457031,99.26328125)
\curveto(560.71660156,99.26328125)(561.39628906,98.98007812)(561.92363281,98.41367187)
\curveto(562.45097656,97.84726562)(562.71464844,97.05039062)(562.71464844,96.02304687)
\curveto(562.71464844,95.96054687)(562.71269531,95.86679687)(562.70878906,95.74179687)
\lineto(558.06816406,95.74179687)
\curveto(558.10722656,95.05820312)(558.30058594,94.53476562)(558.64824219,94.17148437)
\curveto(558.99589844,93.80820312)(559.42949219,93.6265625)(559.94902344,93.6265625)
\curveto(560.33574219,93.6265625)(560.66582031,93.728125)(560.93925781,93.93125)
\curveto(561.21269531,94.134375)(561.42949219,94.45859375)(561.58964844,94.90390625)
\closepath
\moveto(558.12675781,96.60898437)
\lineto(561.60136719,96.60898437)
\curveto(561.55449219,97.13242187)(561.42167969,97.525)(561.20292969,97.78671875)
\curveto(560.86699219,98.19296875)(560.43144531,98.39609375)(559.89628906,98.39609375)
\curveto(559.41191406,98.39609375)(559.00371094,98.23398437)(558.67167969,97.90976562)
\curveto(558.34355469,97.58554687)(558.16191406,97.15195312)(558.12675781,96.60898437)
\closepath
}
}
{
\newrgbcolor{curcolor}{0 0 0}
\pscustom[linestyle=none,fillstyle=solid,fillcolor=curcolor]
{
\newpath
\moveto(31.7,182.11992187)
\lineto(23.11015625,182.11992187)
\lineto(23.11015625,183.25664062)
\lineto(31.7,183.25664062)
\closepath
}
}
{
\newrgbcolor{curcolor}{0 0 0}
\pscustom[linestyle=none,fillstyle=solid,fillcolor=curcolor]
{
\newpath
\moveto(31.7,185.12578125)
\lineto(25.47734375,185.12578125)
\lineto(25.47734375,186.06914062)
\lineto(26.35039062,186.06914062)
\curveto(26.04570312,186.26445312)(25.8015625,186.52421875)(25.61796875,186.8484375)
\curveto(25.43046875,187.17265625)(25.33671875,187.54179687)(25.33671875,187.95585937)
\curveto(25.33671875,188.41679687)(25.43242187,188.79375)(25.62382812,189.08671875)
\curveto(25.81523437,189.38359375)(26.0828125,189.59257812)(26.4265625,189.71367187)
\curveto(25.7,190.20585937)(25.33671875,190.84648437)(25.33671875,191.63554687)
\curveto(25.33671875,192.25273437)(25.50859375,192.72734375)(25.85234375,193.059375)
\curveto(26.1921875,193.39140625)(26.71757812,193.55742187)(27.42851562,193.55742187)
\lineto(31.7,193.55742187)
\lineto(31.7,192.50859375)
\lineto(27.78007812,192.50859375)
\curveto(27.35820312,192.50859375)(27.05546875,192.4734375)(26.871875,192.403125)
\curveto(26.684375,192.33671875)(26.53398437,192.21367187)(26.42070312,192.03398437)
\curveto(26.30742187,191.85429687)(26.25078125,191.64335937)(26.25078125,191.40117187)
\curveto(26.25078125,190.96367187)(26.39726562,190.60039062)(26.69023437,190.31132812)
\curveto(26.97929687,190.02226562)(27.44414062,189.87773437)(28.08476562,189.87773437)
\lineto(31.7,189.87773437)
\lineto(31.7,188.82304687)
\lineto(27.65703125,188.82304687)
\curveto(27.18828125,188.82304687)(26.83671875,188.73710937)(26.60234375,188.56523437)
\curveto(26.36796875,188.39335937)(26.25078125,188.11210937)(26.25078125,187.72148437)
\curveto(26.25078125,187.42460937)(26.32890625,187.14921875)(26.48515625,186.8953125)
\curveto(26.64140625,186.6453125)(26.86992187,186.46367187)(27.17070312,186.35039062)
\curveto(27.47148437,186.23710937)(27.90507812,186.18046875)(28.47148437,186.18046875)
\lineto(31.7,186.18046875)
\closepath
}
}
{
\newrgbcolor{curcolor}{0 0 0}
\pscustom[linestyle=none,fillstyle=solid,fillcolor=curcolor]
{
\newpath
\moveto(34.08476563,195.121875)
\lineto(25.47734375,195.121875)
\lineto(25.47734375,196.0828125)
\lineto(26.2859375,196.0828125)
\curveto(25.96953125,196.309375)(25.73320312,196.56523437)(25.57695312,196.85039062)
\curveto(25.41679687,197.13554687)(25.33671875,197.48125)(25.33671875,197.8875)
\curveto(25.33671875,198.41875)(25.4734375,198.8875)(25.746875,199.29375)
\curveto(26.0203125,199.7)(26.40703125,200.00664062)(26.90703125,200.21367187)
\curveto(27.403125,200.42070312)(27.94804687,200.52421875)(28.54179687,200.52421875)
\curveto(29.17851562,200.52421875)(29.75273437,200.40898437)(30.26445312,200.17851562)
\curveto(30.77226562,199.95195312)(31.16289062,199.61992187)(31.43632812,199.18242187)
\curveto(31.70585937,198.74882812)(31.840625,198.29179687)(31.840625,197.81132812)
\curveto(31.840625,197.45976562)(31.76640625,197.14335937)(31.61796875,196.86210937)
\curveto(31.46953125,196.58476562)(31.28203125,196.35625)(31.05546875,196.1765625)
\lineto(34.08476563,196.1765625)
\closepath
\moveto(28.62382812,196.07695312)
\curveto(29.42460937,196.07695312)(30.01640625,196.2390625)(30.39921875,196.56328125)
\curveto(30.78203125,196.8875)(30.9734375,197.28007812)(30.9734375,197.74101562)
\curveto(30.9734375,198.20976562)(30.77617187,198.61015625)(30.38164062,198.9421875)
\curveto(29.98320312,199.278125)(29.36796875,199.44609375)(28.5359375,199.44609375)
\curveto(27.74296875,199.44609375)(27.14921875,199.28203125)(26.7546875,198.95390625)
\curveto(26.36015625,198.6296875)(26.16289062,198.24101562)(26.16289062,197.78789062)
\curveto(26.16289062,197.33867187)(26.37382812,196.94023437)(26.79570312,196.59257812)
\curveto(27.21367187,196.24882812)(27.82304687,196.07695312)(28.62382812,196.07695312)
\closepath
}
}
{
\newrgbcolor{curcolor}{0 0 0}
\pscustom[linestyle=none,fillstyle=solid,fillcolor=curcolor]
{
\newpath
\moveto(31.7,201.78398437)
\lineto(25.47734375,201.78398437)
\lineto(25.47734375,202.73320312)
\lineto(26.42070312,202.73320312)
\curveto(25.97929687,202.97539062)(25.68828125,203.19804687)(25.54765625,203.40117187)
\curveto(25.40703125,203.60820312)(25.33671875,203.83476562)(25.33671875,204.08085937)
\curveto(25.33671875,204.43632812)(25.45,204.79765625)(25.6765625,205.16484375)
\lineto(26.65507812,204.8015625)
\curveto(26.50273437,204.54375)(26.4265625,204.2859375)(26.4265625,204.028125)
\curveto(26.4265625,203.79765625)(26.496875,203.590625)(26.6375,203.40703125)
\curveto(26.77421875,203.2234375)(26.965625,203.09257812)(27.21171875,203.01445312)
\curveto(27.58671875,202.89726562)(27.996875,202.83867187)(28.4421875,202.83867187)
\lineto(31.7,202.83867187)
\closepath
}
}
{
\newrgbcolor{curcolor}{0 0 0}
\pscustom[linestyle=none,fillstyle=solid,fillcolor=curcolor]
{
\newpath
\moveto(28.58867187,205.39921875)
\curveto(27.43632812,205.39921875)(26.5828125,205.71953125)(26.028125,206.36015625)
\curveto(25.5671875,206.8953125)(25.33671875,207.54765625)(25.33671875,208.3171875)
\curveto(25.33671875,209.17265625)(25.61796875,209.871875)(26.18046875,210.41484375)
\curveto(26.7390625,210.9578125)(27.5125,211.22929687)(28.50078125,211.22929687)
\curveto(29.3015625,211.22929687)(29.93242187,211.10820312)(30.39335937,210.86601562)
\curveto(30.85039062,210.62773437)(31.20585937,210.278125)(31.45976562,209.8171875)
\curveto(31.71367187,209.36015625)(31.840625,208.86015625)(31.840625,208.3171875)
\curveto(31.840625,207.44609375)(31.56132812,206.74101562)(31.00273437,206.20195312)
\curveto(30.44414062,205.66679687)(29.63945312,205.39921875)(28.58867187,205.39921875)
\closepath
\moveto(28.58867187,206.48320312)
\curveto(29.38554687,206.48320312)(29.98320312,206.65703125)(30.38164062,207.0046875)
\curveto(30.77617187,207.35234375)(30.9734375,207.78984375)(30.9734375,208.3171875)
\curveto(30.9734375,208.840625)(30.77421875,209.27617187)(30.37578125,209.62382812)
\curveto(29.97734375,209.97148437)(29.36992187,210.1453125)(28.55351562,210.1453125)
\curveto(27.78398437,210.1453125)(27.20195312,209.96953125)(26.80742187,209.61796875)
\curveto(26.40898437,209.2703125)(26.20976562,208.83671875)(26.20976562,208.3171875)
\curveto(26.20976562,207.78984375)(26.40703125,207.35234375)(26.8015625,207.0046875)
\curveto(27.19609375,206.65703125)(27.79179687,206.48320312)(28.58867187,206.48320312)
\closepath
}
}
{
\newrgbcolor{curcolor}{0 0 0}
\pscustom[linestyle=none,fillstyle=solid,fillcolor=curcolor]
{
\newpath
\moveto(31.7,214.19414062)
\lineto(25.47734375,211.82695312)
\lineto(25.47734375,212.94023437)
\lineto(29.20390625,214.27617187)
\curveto(29.60625,214.42070312)(30.02421875,214.55351562)(30.4578125,214.67460937)
\curveto(30.1296875,214.76835937)(29.73515625,214.89921875)(29.27421875,215.0671875)
\lineto(25.47734375,216.45)
\lineto(25.47734375,217.53398437)
\lineto(31.7,215.17851562)
\closepath
}
}
{
\newrgbcolor{curcolor}{0 0 0}
\pscustom[linestyle=none,fillstyle=solid,fillcolor=curcolor]
{
\newpath
\moveto(29.69609375,222.72539062)
\lineto(29.83085937,223.81523437)
\curveto(30.46757812,223.64335937)(30.96171875,223.325)(31.31328125,222.86015625)
\curveto(31.66484375,222.3953125)(31.840625,221.8015625)(31.840625,221.07890625)
\curveto(31.840625,220.16875)(31.56132812,219.44609375)(31.00273437,218.9109375)
\curveto(30.44023437,218.3796875)(29.653125,218.1140625)(28.64140625,218.1140625)
\curveto(27.59453125,218.1140625)(26.78203125,218.38359375)(26.20390625,218.92265625)
\curveto(25.62578125,219.46171875)(25.33671875,220.1609375)(25.33671875,221.0203125)
\curveto(25.33671875,221.85234375)(25.61992187,222.53203125)(26.18632812,223.059375)
\curveto(26.75273437,223.58671875)(27.54960937,223.85039062)(28.57695312,223.85039062)
\curveto(28.63945312,223.85039062)(28.73320312,223.8484375)(28.85820312,223.84453125)
\lineto(28.85820312,219.20390625)
\curveto(29.54179687,219.24296875)(30.06523437,219.43632812)(30.42851562,219.78398437)
\curveto(30.79179687,220.13164062)(30.9734375,220.56523437)(30.9734375,221.08476562)
\curveto(30.9734375,221.47148437)(30.871875,221.8015625)(30.66875,222.075)
\curveto(30.465625,222.3484375)(30.14140625,222.56523437)(29.69609375,222.72539062)
\closepath
\moveto(27.99101562,219.2625)
\lineto(27.99101562,222.73710937)
\curveto(27.46757812,222.69023437)(27.075,222.55742187)(26.81328125,222.33867187)
\curveto(26.40703125,222.00273437)(26.20390625,221.5671875)(26.20390625,221.03203125)
\curveto(26.20390625,220.54765625)(26.36601562,220.13945312)(26.69023437,219.80742187)
\curveto(27.01445312,219.47929687)(27.44804687,219.29765625)(27.99101562,219.2625)
\closepath
}
}
{
\newrgbcolor{curcolor}{0 0 0}
\pscustom[linestyle=none,fillstyle=solid,fillcolor=curcolor]
{
\newpath
\moveto(31.7,225.13945312)
\lineto(25.47734375,225.13945312)
\lineto(25.47734375,226.0828125)
\lineto(26.35039062,226.0828125)
\curveto(26.04570312,226.278125)(25.8015625,226.53789062)(25.61796875,226.86210937)
\curveto(25.43046875,227.18632812)(25.33671875,227.55546875)(25.33671875,227.96953125)
\curveto(25.33671875,228.43046875)(25.43242187,228.80742187)(25.62382812,229.10039062)
\curveto(25.81523437,229.39726562)(26.0828125,229.60625)(26.4265625,229.72734375)
\curveto(25.7,230.21953125)(25.33671875,230.86015625)(25.33671875,231.64921875)
\curveto(25.33671875,232.26640625)(25.50859375,232.74101562)(25.85234375,233.07304687)
\curveto(26.1921875,233.40507812)(26.71757812,233.57109375)(27.42851562,233.57109375)
\lineto(31.7,233.57109375)
\lineto(31.7,232.52226562)
\lineto(27.78007812,232.52226562)
\curveto(27.35820312,232.52226562)(27.05546875,232.48710937)(26.871875,232.41679687)
\curveto(26.684375,232.35039062)(26.53398437,232.22734375)(26.42070312,232.04765625)
\curveto(26.30742187,231.86796875)(26.25078125,231.65703125)(26.25078125,231.41484375)
\curveto(26.25078125,230.97734375)(26.39726562,230.6140625)(26.69023437,230.325)
\curveto(26.97929687,230.0359375)(27.44414062,229.89140625)(28.08476562,229.89140625)
\lineto(31.7,229.89140625)
\lineto(31.7,228.83671875)
\lineto(27.65703125,228.83671875)
\curveto(27.18828125,228.83671875)(26.83671875,228.75078125)(26.60234375,228.57890625)
\curveto(26.36796875,228.40703125)(26.25078125,228.12578125)(26.25078125,227.73515625)
\curveto(26.25078125,227.43828125)(26.32890625,227.16289062)(26.48515625,226.90898437)
\curveto(26.64140625,226.65898437)(26.86992187,226.47734375)(27.17070312,226.3640625)
\curveto(27.47148437,226.25078125)(27.90507812,226.19414062)(28.47148437,226.19414062)
\lineto(31.7,226.19414062)
\closepath
}
}
{
\newrgbcolor{curcolor}{0 0 0}
\pscustom[linestyle=none,fillstyle=solid,fillcolor=curcolor]
{
\newpath
\moveto(29.69609375,239.3953125)
\lineto(29.83085937,240.48515625)
\curveto(30.46757812,240.31328125)(30.96171875,239.99492187)(31.31328125,239.53007812)
\curveto(31.66484375,239.06523437)(31.840625,238.47148437)(31.840625,237.74882812)
\curveto(31.840625,236.83867187)(31.56132812,236.11601562)(31.00273437,235.58085937)
\curveto(30.44023437,235.04960937)(29.653125,234.78398437)(28.64140625,234.78398437)
\curveto(27.59453125,234.78398437)(26.78203125,235.05351562)(26.20390625,235.59257812)
\curveto(25.62578125,236.13164062)(25.33671875,236.83085937)(25.33671875,237.69023437)
\curveto(25.33671875,238.52226562)(25.61992187,239.20195312)(26.18632812,239.72929687)
\curveto(26.75273437,240.25664062)(27.54960937,240.5203125)(28.57695312,240.5203125)
\curveto(28.63945312,240.5203125)(28.73320312,240.51835937)(28.85820312,240.51445312)
\lineto(28.85820312,235.87382812)
\curveto(29.54179687,235.91289062)(30.06523437,236.10625)(30.42851562,236.45390625)
\curveto(30.79179687,236.8015625)(30.9734375,237.23515625)(30.9734375,237.7546875)
\curveto(30.9734375,238.14140625)(30.871875,238.47148437)(30.66875,238.74492187)
\curveto(30.465625,239.01835937)(30.14140625,239.23515625)(29.69609375,239.3953125)
\closepath
\moveto(27.99101562,235.93242187)
\lineto(27.99101562,239.40703125)
\curveto(27.46757812,239.36015625)(27.075,239.22734375)(26.81328125,239.00859375)
\curveto(26.40703125,238.67265625)(26.20390625,238.23710937)(26.20390625,237.70195312)
\curveto(26.20390625,237.21757812)(26.36601562,236.809375)(26.69023437,236.47734375)
\curveto(27.01445312,236.14921875)(27.44804687,235.96757812)(27.99101562,235.93242187)
\closepath
}
}
{
\newrgbcolor{curcolor}{0 0 0}
\pscustom[linestyle=none,fillstyle=solid,fillcolor=curcolor]
{
\newpath
\moveto(31.7,241.809375)
\lineto(25.47734375,241.809375)
\lineto(25.47734375,242.75859375)
\lineto(26.36210937,242.75859375)
\curveto(25.67851562,243.215625)(25.33671875,243.87578125)(25.33671875,244.7390625)
\curveto(25.33671875,245.1140625)(25.40507812,245.4578125)(25.54179687,245.7703125)
\curveto(25.67460937,246.08671875)(25.85039062,246.32304687)(26.06914062,246.47929687)
\curveto(26.28789062,246.63554687)(26.54765625,246.74492187)(26.8484375,246.80742187)
\curveto(27.04375,246.84648437)(27.38554687,246.86601562)(27.87382812,246.86601562)
\lineto(31.7,246.86601562)
\lineto(31.7,245.81132812)
\lineto(27.91484375,245.81132812)
\curveto(27.48515625,245.81132812)(27.16484375,245.7703125)(26.95390625,245.68828125)
\curveto(26.7390625,245.60625)(26.56914062,245.45976562)(26.44414062,245.24882812)
\curveto(26.31523437,245.04179687)(26.25078125,244.79765625)(26.25078125,244.51640625)
\curveto(26.25078125,244.0671875)(26.39335937,243.67851562)(26.67851562,243.35039062)
\curveto(26.96367187,243.02617187)(27.5046875,242.8640625)(28.3015625,242.8640625)
\lineto(31.7,242.8640625)
\closepath
}
}
{
\newrgbcolor{curcolor}{0 0 0}
\pscustom[linestyle=none,fillstyle=solid,fillcolor=curcolor]
{
\newpath
\moveto(30.75664062,250.7859375)
\lineto(31.68828125,250.93828125)
\curveto(31.75078125,250.64140625)(31.78203125,250.37578125)(31.78203125,250.14140625)
\curveto(31.78203125,249.75859375)(31.72148437,249.46171875)(31.60039062,249.25078125)
\curveto(31.47929687,249.03984375)(31.32109375,248.89140625)(31.12578125,248.80546875)
\curveto(30.9265625,248.71953125)(30.51054687,248.6765625)(29.87773437,248.6765625)
\lineto(26.29765625,248.6765625)
\lineto(26.29765625,247.903125)
\lineto(25.47734375,247.903125)
\lineto(25.47734375,248.6765625)
\lineto(23.93632812,248.6765625)
\lineto(23.30351562,249.72539062)
\lineto(25.47734375,249.72539062)
\lineto(25.47734375,250.7859375)
\lineto(26.29765625,250.7859375)
\lineto(26.29765625,249.72539062)
\lineto(29.93632812,249.72539062)
\curveto(30.23710937,249.72539062)(30.43046875,249.74296875)(30.51640625,249.778125)
\curveto(30.60234375,249.8171875)(30.67070312,249.87773437)(30.72148437,249.95976562)
\curveto(30.77226562,250.04570312)(30.79765625,250.16679687)(30.79765625,250.32304687)
\curveto(30.79765625,250.44023437)(30.78398437,250.59453125)(30.75664062,250.7859375)
\closepath
}
}
{
\newrgbcolor{curcolor}{0 0 0}
\pscustom[linestyle=none,fillstyle=solid,fillcolor=curcolor]
{
\newpath
\moveto(34.22539063,257.16679687)
\curveto(33.49101563,256.58476562)(32.63164062,256.09257812)(31.64726562,255.69023437)
\curveto(30.66289062,255.28789062)(29.64335937,255.08671875)(28.58867187,255.08671875)
\curveto(27.65898437,255.08671875)(26.76835937,255.23710937)(25.91679687,255.53789062)
\curveto(24.92851562,255.88945312)(23.94414062,256.43242187)(22.96367187,257.16679687)
\lineto(22.96367187,257.92265625)
\curveto(23.77617187,257.45)(24.35625,257.1375)(24.70390625,256.98515625)
\curveto(25.24296875,256.746875)(25.80546875,256.559375)(26.39140625,256.42265625)
\curveto(27.121875,256.2546875)(27.85625,256.17070312)(28.59453125,256.17070312)
\curveto(30.4734375,256.17070312)(32.35039062,256.7546875)(34.22539063,257.92265625)
\closepath
}
}
{
\newrgbcolor{curcolor}{0 0 0}
\pscustom[linestyle=none,fillstyle=solid,fillcolor=curcolor]
{
\newpath
\moveto(34.08476563,259.14726562)
\lineto(25.47734375,259.14726562)
\lineto(25.47734375,260.10820312)
\lineto(26.2859375,260.10820312)
\curveto(25.96953125,260.33476562)(25.73320312,260.590625)(25.57695312,260.87578125)
\curveto(25.41679687,261.1609375)(25.33671875,261.50664062)(25.33671875,261.91289062)
\curveto(25.33671875,262.44414062)(25.4734375,262.91289062)(25.746875,263.31914062)
\curveto(26.0203125,263.72539062)(26.40703125,264.03203125)(26.90703125,264.2390625)
\curveto(27.403125,264.44609375)(27.94804687,264.54960937)(28.54179687,264.54960937)
\curveto(29.17851562,264.54960937)(29.75273437,264.434375)(30.26445312,264.20390625)
\curveto(30.77226562,263.97734375)(31.16289062,263.6453125)(31.43632812,263.2078125)
\curveto(31.70585937,262.77421875)(31.840625,262.3171875)(31.840625,261.83671875)
\curveto(31.840625,261.48515625)(31.76640625,261.16875)(31.61796875,260.8875)
\curveto(31.46953125,260.61015625)(31.28203125,260.38164062)(31.05546875,260.20195312)
\lineto(34.08476563,260.20195312)
\closepath
\moveto(28.62382812,260.10234375)
\curveto(29.42460937,260.10234375)(30.01640625,260.26445312)(30.39921875,260.58867187)
\curveto(30.78203125,260.91289062)(30.9734375,261.30546875)(30.9734375,261.76640625)
\curveto(30.9734375,262.23515625)(30.77617187,262.63554687)(30.38164062,262.96757812)
\curveto(29.98320312,263.30351562)(29.36796875,263.47148437)(28.5359375,263.47148437)
\curveto(27.74296875,263.47148437)(27.14921875,263.30742187)(26.7546875,262.97929687)
\curveto(26.36015625,262.65507812)(26.16289062,262.26640625)(26.16289062,261.81328125)
\curveto(26.16289062,261.3640625)(26.37382812,260.965625)(26.79570312,260.61796875)
\curveto(27.21367187,260.27421875)(27.82304687,260.10234375)(28.62382812,260.10234375)
\closepath
}
}
{
\newrgbcolor{curcolor}{0 0 0}
\pscustom[linestyle=none,fillstyle=solid,fillcolor=curcolor]
{
\newpath
\moveto(29.69609375,270.08085937)
\lineto(29.83085937,271.17070312)
\curveto(30.46757812,270.99882812)(30.96171875,270.68046875)(31.31328125,270.215625)
\curveto(31.66484375,269.75078125)(31.840625,269.15703125)(31.840625,268.434375)
\curveto(31.840625,267.52421875)(31.56132812,266.8015625)(31.00273437,266.26640625)
\curveto(30.44023437,265.73515625)(29.653125,265.46953125)(28.64140625,265.46953125)
\curveto(27.59453125,265.46953125)(26.78203125,265.7390625)(26.20390625,266.278125)
\curveto(25.62578125,266.8171875)(25.33671875,267.51640625)(25.33671875,268.37578125)
\curveto(25.33671875,269.2078125)(25.61992187,269.8875)(26.18632812,270.41484375)
\curveto(26.75273437,270.9421875)(27.54960937,271.20585937)(28.57695312,271.20585937)
\curveto(28.63945312,271.20585937)(28.73320312,271.20390625)(28.85820312,271.2)
\lineto(28.85820312,266.559375)
\curveto(29.54179687,266.5984375)(30.06523437,266.79179687)(30.42851562,267.13945312)
\curveto(30.79179687,267.48710937)(30.9734375,267.92070312)(30.9734375,268.44023437)
\curveto(30.9734375,268.82695312)(30.871875,269.15703125)(30.66875,269.43046875)
\curveto(30.465625,269.70390625)(30.14140625,269.92070312)(29.69609375,270.08085937)
\closepath
\moveto(27.99101562,266.61796875)
\lineto(27.99101562,270.09257812)
\curveto(27.46757812,270.04570312)(27.075,269.91289062)(26.81328125,269.69414062)
\curveto(26.40703125,269.35820312)(26.20390625,268.92265625)(26.20390625,268.3875)
\curveto(26.20390625,267.903125)(26.36601562,267.49492187)(26.69023437,267.16289062)
\curveto(27.01445312,266.83476562)(27.44804687,266.653125)(27.99101562,266.61796875)
\closepath
}
}
{
\newrgbcolor{curcolor}{0 0 0}
\pscustom[linestyle=none,fillstyle=solid,fillcolor=curcolor]
{
\newpath
\moveto(31.7,272.48320312)
\lineto(25.47734375,272.48320312)
\lineto(25.47734375,273.43242187)
\lineto(26.42070312,273.43242187)
\curveto(25.97929687,273.67460937)(25.68828125,273.89726562)(25.54765625,274.10039062)
\curveto(25.40703125,274.30742187)(25.33671875,274.53398437)(25.33671875,274.78007812)
\curveto(25.33671875,275.13554687)(25.45,275.496875)(25.6765625,275.8640625)
\lineto(26.65507812,275.50078125)
\curveto(26.50273437,275.24296875)(26.4265625,274.98515625)(26.4265625,274.72734375)
\curveto(26.4265625,274.496875)(26.496875,274.28984375)(26.6375,274.10625)
\curveto(26.77421875,273.92265625)(26.965625,273.79179687)(27.21171875,273.71367187)
\curveto(27.58671875,273.59648437)(27.996875,273.53789062)(28.4421875,273.53789062)
\lineto(31.7,273.53789062)
\closepath
}
}
{
\newrgbcolor{curcolor}{0 0 0}
\pscustom[linestyle=none,fillstyle=solid,fillcolor=curcolor]
{
\newpath
\moveto(29.42070312,280.5515625)
\lineto(29.55546875,281.58867187)
\curveto(30.2703125,281.47539062)(30.83085937,281.184375)(31.23710937,280.715625)
\curveto(31.63945312,280.25078125)(31.840625,279.67851562)(31.840625,278.99882812)
\curveto(31.840625,278.14726562)(31.56328125,277.46171875)(31.00859375,276.9421875)
\curveto(30.45,276.4265625)(29.65117187,276.16875)(28.61210937,276.16875)
\curveto(27.94023437,276.16875)(27.35234375,276.28007812)(26.8484375,276.50273437)
\curveto(26.34453125,276.72539062)(25.96757812,277.06328125)(25.71757812,277.51640625)
\curveto(25.46367187,277.9734375)(25.33671875,278.46953125)(25.33671875,279.0046875)
\curveto(25.33671875,279.68046875)(25.50859375,280.23320312)(25.85234375,280.66289062)
\curveto(26.1921875,281.09257812)(26.6765625,281.36796875)(27.30546875,281.4890625)
\lineto(27.46367187,280.46367187)
\curveto(27.04570312,280.36601562)(26.73125,280.1921875)(26.5203125,279.9421875)
\curveto(26.309375,279.69609375)(26.20390625,279.39726562)(26.20390625,279.04570312)
\curveto(26.20390625,278.51445312)(26.3953125,278.0828125)(26.778125,277.75078125)
\curveto(27.15703125,277.41875)(27.75859375,277.25273437)(28.5828125,277.25273437)
\curveto(29.41875,277.25273437)(30.02617187,277.41289062)(30.40507812,277.73320312)
\curveto(30.78398437,278.05351562)(30.9734375,278.47148437)(30.9734375,278.98710937)
\curveto(30.9734375,279.40117187)(30.84648437,279.746875)(30.59257812,280.02421875)
\curveto(30.33867187,280.3015625)(29.94804687,280.47734375)(29.42070312,280.5515625)
\closepath
}
}
{
\newrgbcolor{curcolor}{0 0 0}
\pscustom[linestyle=none,fillstyle=solid,fillcolor=curcolor]
{
\newpath
\moveto(29.69609375,286.75078125)
\lineto(29.83085937,287.840625)
\curveto(30.46757812,287.66875)(30.96171875,287.35039062)(31.31328125,286.88554687)
\curveto(31.66484375,286.42070312)(31.840625,285.82695312)(31.840625,285.10429687)
\curveto(31.840625,284.19414062)(31.56132812,283.47148437)(31.00273437,282.93632812)
\curveto(30.44023437,282.40507812)(29.653125,282.13945312)(28.64140625,282.13945312)
\curveto(27.59453125,282.13945312)(26.78203125,282.40898437)(26.20390625,282.94804687)
\curveto(25.62578125,283.48710937)(25.33671875,284.18632812)(25.33671875,285.04570312)
\curveto(25.33671875,285.87773437)(25.61992187,286.55742187)(26.18632812,287.08476562)
\curveto(26.75273437,287.61210937)(27.54960937,287.87578125)(28.57695312,287.87578125)
\curveto(28.63945312,287.87578125)(28.73320312,287.87382812)(28.85820312,287.86992187)
\lineto(28.85820312,283.22929687)
\curveto(29.54179687,283.26835937)(30.06523437,283.46171875)(30.42851562,283.809375)
\curveto(30.79179687,284.15703125)(30.9734375,284.590625)(30.9734375,285.11015625)
\curveto(30.9734375,285.496875)(30.871875,285.82695312)(30.66875,286.10039062)
\curveto(30.465625,286.37382812)(30.14140625,286.590625)(29.69609375,286.75078125)
\closepath
\moveto(27.99101562,283.28789062)
\lineto(27.99101562,286.7625)
\curveto(27.46757812,286.715625)(27.075,286.5828125)(26.81328125,286.3640625)
\curveto(26.40703125,286.028125)(26.20390625,285.59257812)(26.20390625,285.05742187)
\curveto(26.20390625,284.57304687)(26.36601562,284.16484375)(26.69023437,283.8328125)
\curveto(27.01445312,283.5046875)(27.44804687,283.32304687)(27.99101562,283.28789062)
\closepath
}
}
{
\newrgbcolor{curcolor}{0 0 0}
\pscustom[linestyle=none,fillstyle=solid,fillcolor=curcolor]
{
\newpath
\moveto(31.7,289.16484375)
\lineto(25.47734375,289.16484375)
\lineto(25.47734375,290.1140625)
\lineto(26.36210937,290.1140625)
\curveto(25.67851562,290.57109375)(25.33671875,291.23125)(25.33671875,292.09453125)
\curveto(25.33671875,292.46953125)(25.40507812,292.81328125)(25.54179687,293.12578125)
\curveto(25.67460937,293.4421875)(25.85039062,293.67851562)(26.06914062,293.83476562)
\curveto(26.28789062,293.99101562)(26.54765625,294.10039062)(26.8484375,294.16289062)
\curveto(27.04375,294.20195312)(27.38554687,294.22148437)(27.87382812,294.22148437)
\lineto(31.7,294.22148437)
\lineto(31.7,293.16679687)
\lineto(27.91484375,293.16679687)
\curveto(27.48515625,293.16679687)(27.16484375,293.12578125)(26.95390625,293.04375)
\curveto(26.7390625,292.96171875)(26.56914062,292.81523437)(26.44414062,292.60429687)
\curveto(26.31523437,292.39726562)(26.25078125,292.153125)(26.25078125,291.871875)
\curveto(26.25078125,291.42265625)(26.39335937,291.03398437)(26.67851562,290.70585937)
\curveto(26.96367187,290.38164062)(27.5046875,290.21953125)(28.3015625,290.21953125)
\lineto(31.7,290.21953125)
\closepath
}
}
{
\newrgbcolor{curcolor}{0 0 0}
\pscustom[linestyle=none,fillstyle=solid,fillcolor=curcolor]
{
\newpath
\moveto(30.75664062,298.14140625)
\lineto(31.68828125,298.29375)
\curveto(31.75078125,297.996875)(31.78203125,297.73125)(31.78203125,297.496875)
\curveto(31.78203125,297.1140625)(31.72148437,296.8171875)(31.60039062,296.60625)
\curveto(31.47929687,296.3953125)(31.32109375,296.246875)(31.12578125,296.1609375)
\curveto(30.9265625,296.075)(30.51054687,296.03203125)(29.87773437,296.03203125)
\lineto(26.29765625,296.03203125)
\lineto(26.29765625,295.25859375)
\lineto(25.47734375,295.25859375)
\lineto(25.47734375,296.03203125)
\lineto(23.93632812,296.03203125)
\lineto(23.30351562,297.08085937)
\lineto(25.47734375,297.08085937)
\lineto(25.47734375,298.14140625)
\lineto(26.29765625,298.14140625)
\lineto(26.29765625,297.08085937)
\lineto(29.93632812,297.08085937)
\curveto(30.23710937,297.08085937)(30.43046875,297.0984375)(30.51640625,297.13359375)
\curveto(30.60234375,297.17265625)(30.67070312,297.23320312)(30.72148437,297.31523437)
\curveto(30.77226562,297.40117187)(30.79765625,297.52226562)(30.79765625,297.67851562)
\curveto(30.79765625,297.79570312)(30.78398437,297.95)(30.75664062,298.14140625)
\closepath
}
}
{
\newrgbcolor{curcolor}{0 0 0}
\pscustom[linestyle=none,fillstyle=solid,fillcolor=curcolor]
{
\newpath
\moveto(30.93242187,303.23320312)
\curveto(31.26445312,302.84257812)(31.49882812,302.465625)(31.63554687,302.10234375)
\curveto(31.77226562,301.74296875)(31.840625,301.35625)(31.840625,300.9421875)
\curveto(31.840625,300.25859375)(31.67460937,299.73320312)(31.34257812,299.36601562)
\curveto(31.00664062,298.99882812)(30.57890625,298.81523437)(30.059375,298.81523437)
\curveto(29.7546875,298.81523437)(29.47734375,298.88359375)(29.22734375,299.0203125)
\curveto(28.9734375,299.1609375)(28.7703125,299.34257812)(28.61796875,299.56523437)
\curveto(28.465625,299.79179687)(28.35039062,300.04570312)(28.27226562,300.32695312)
\curveto(28.21757812,300.53398437)(28.16484375,300.84648437)(28.1140625,301.26445312)
\curveto(28.0125,302.11601562)(27.89140625,302.74296875)(27.75078125,303.1453125)
\curveto(27.60625,303.14921875)(27.51445312,303.15117187)(27.47539062,303.15117187)
\curveto(27.04570312,303.15117187)(26.74296875,303.0515625)(26.5671875,302.85234375)
\curveto(26.32890625,302.5828125)(26.20976562,302.18242187)(26.20976562,301.65117187)
\curveto(26.20976562,301.15507812)(26.29765625,300.78789062)(26.4734375,300.54960937)
\curveto(26.6453125,300.31523437)(26.95195312,300.14140625)(27.39335937,300.028125)
\lineto(27.25273437,298.996875)
\curveto(26.81132812,299.090625)(26.45585937,299.24492187)(26.18632812,299.45976562)
\curveto(25.91289062,299.67460937)(25.70390625,299.98515625)(25.559375,300.39140625)
\curveto(25.4109375,300.79765625)(25.33671875,301.26835937)(25.33671875,301.80351562)
\curveto(25.33671875,302.33476562)(25.39921875,302.76640625)(25.52421875,303.0984375)
\curveto(25.64921875,303.43046875)(25.80742187,303.67460937)(25.99882812,303.83085937)
\curveto(26.18632812,303.98710937)(26.42460937,304.09648437)(26.71367187,304.15898437)
\curveto(26.89335937,304.19414062)(27.21757812,304.21171875)(27.68632812,304.21171875)
\lineto(29.09257812,304.21171875)
\curveto(30.07304687,304.21171875)(30.69414062,304.23320312)(30.95585937,304.27617187)
\curveto(31.21367187,304.32304687)(31.46171875,304.41289062)(31.7,304.54570312)
\lineto(31.7,303.44414062)
\curveto(31.48125,303.33476562)(31.22539062,303.26445312)(30.93242187,303.23320312)
\closepath
\moveto(28.57695312,303.1453125)
\curveto(28.73320312,302.7625)(28.86601562,302.18828125)(28.97539062,301.42265625)
\curveto(29.03789062,300.9890625)(29.10820312,300.68242187)(29.18632812,300.50273437)
\curveto(29.26445312,300.32304687)(29.3796875,300.184375)(29.53203125,300.08671875)
\curveto(29.68046875,299.9890625)(29.84648437,299.94023437)(30.03007812,299.94023437)
\curveto(30.31132812,299.94023437)(30.54570312,300.04570312)(30.73320312,300.25664062)
\curveto(30.92070312,300.47148437)(31.01445312,300.78398437)(31.01445312,301.19414062)
\curveto(31.01445312,301.60039062)(30.9265625,301.96171875)(30.75078125,302.278125)
\curveto(30.57109375,302.59453125)(30.32695312,302.82695312)(30.01835937,302.97539062)
\curveto(29.78007812,303.08867187)(29.42851562,303.1453125)(28.96367187,303.1453125)
\closepath
}
}
{
\newrgbcolor{curcolor}{0 0 0}
\pscustom[linestyle=none,fillstyle=solid,fillcolor=curcolor]
{
\newpath
\moveto(32.215625,305.653125)
\lineto(32.36796875,306.67851562)
\curveto(32.684375,306.72148437)(32.91484375,306.840625)(33.059375,307.0359375)
\curveto(33.2546875,307.29765625)(33.35234375,307.65507812)(33.35234375,308.10820312)
\curveto(33.35234375,308.59648437)(33.2546875,308.9734375)(33.059375,309.2390625)
\curveto(32.8640625,309.5046875)(32.590625,309.684375)(32.2390625,309.778125)
\curveto(32.02421875,309.8328125)(31.57304687,309.85820312)(30.88554687,309.85429687)
\curveto(31.42851562,309.39335937)(31.7,308.81914062)(31.7,308.13164062)
\curveto(31.7,307.27617187)(31.39140625,306.6140625)(30.77421875,306.1453125)
\curveto(30.15703125,305.6765625)(29.41679687,305.4421875)(28.55351562,305.4421875)
\curveto(27.95976562,305.4421875)(27.41289062,305.54960937)(26.91289062,305.76445312)
\curveto(26.40898437,305.97929687)(26.0203125,306.28984375)(25.746875,306.69609375)
\curveto(25.4734375,307.10625)(25.33671875,307.58671875)(25.33671875,308.1375)
\curveto(25.33671875,308.871875)(25.63359375,309.47734375)(26.22734375,309.95390625)
\lineto(25.47734375,309.95390625)
\lineto(25.47734375,310.9265625)
\lineto(30.85625,310.9265625)
\curveto(31.825,310.9265625)(32.51054688,310.82695312)(32.91289063,310.62773437)
\curveto(33.31914063,310.43242187)(33.63945313,310.11992187)(33.87382813,309.69023437)
\curveto(34.10820313,309.26445312)(34.22539063,308.7390625)(34.22539063,308.1140625)
\curveto(34.22539063,307.371875)(34.05742188,306.77226562)(33.72148438,306.31523437)
\curveto(33.38945313,305.85820312)(32.8875,305.6375)(32.215625,305.653125)
\closepath
\moveto(28.47734375,306.52617187)
\curveto(29.29375,306.52617187)(29.88945312,306.68828125)(30.26445312,307.0125)
\curveto(30.63945312,307.33671875)(30.82695312,307.74296875)(30.82695312,308.23125)
\curveto(30.82695312,308.715625)(30.64140625,309.121875)(30.2703125,309.45)
\curveto(29.8953125,309.778125)(29.309375,309.9421875)(28.5125,309.9421875)
\curveto(27.75078125,309.9421875)(27.1765625,309.77226562)(26.78984375,309.43242187)
\curveto(26.403125,309.09648437)(26.20976562,308.69023437)(26.20976562,308.21367187)
\curveto(26.20976562,307.74492187)(26.40117187,307.34648437)(26.78398437,307.01835937)
\curveto(27.16289062,306.69023437)(27.72734375,306.52617187)(28.47734375,306.52617187)
\closepath
}
}
{
\newrgbcolor{curcolor}{0 0 0}
\pscustom[linestyle=none,fillstyle=solid,fillcolor=curcolor]
{
\newpath
\moveto(29.69609375,316.78007812)
\lineto(29.83085937,317.86992187)
\curveto(30.46757812,317.69804687)(30.96171875,317.3796875)(31.31328125,316.91484375)
\curveto(31.66484375,316.45)(31.840625,315.85625)(31.840625,315.13359375)
\curveto(31.840625,314.2234375)(31.56132812,313.50078125)(31.00273437,312.965625)
\curveto(30.44023437,312.434375)(29.653125,312.16875)(28.64140625,312.16875)
\curveto(27.59453125,312.16875)(26.78203125,312.43828125)(26.20390625,312.97734375)
\curveto(25.62578125,313.51640625)(25.33671875,314.215625)(25.33671875,315.075)
\curveto(25.33671875,315.90703125)(25.61992187,316.58671875)(26.18632812,317.1140625)
\curveto(26.75273437,317.64140625)(27.54960937,317.90507812)(28.57695312,317.90507812)
\curveto(28.63945312,317.90507812)(28.73320312,317.903125)(28.85820312,317.89921875)
\lineto(28.85820312,313.25859375)
\curveto(29.54179687,313.29765625)(30.06523437,313.49101562)(30.42851562,313.83867187)
\curveto(30.79179687,314.18632812)(30.9734375,314.61992187)(30.9734375,315.13945312)
\curveto(30.9734375,315.52617187)(30.871875,315.85625)(30.66875,316.1296875)
\curveto(30.465625,316.403125)(30.14140625,316.61992187)(29.69609375,316.78007812)
\closepath
\moveto(27.99101562,313.3171875)
\lineto(27.99101562,316.79179687)
\curveto(27.46757812,316.74492187)(27.075,316.61210937)(26.81328125,316.39335937)
\curveto(26.40703125,316.05742187)(26.20390625,315.621875)(26.20390625,315.08671875)
\curveto(26.20390625,314.60234375)(26.36601562,314.19414062)(26.69023437,313.86210937)
\curveto(27.01445312,313.53398437)(27.44804687,313.35234375)(27.99101562,313.3171875)
\closepath
}
}
{
\newrgbcolor{curcolor}{0 0 0}
\pscustom[linestyle=none,fillstyle=solid,fillcolor=curcolor]
{
\newpath
\moveto(34.22539063,319.88554687)
\lineto(34.22539063,319.1296875)
\curveto(32.35039062,320.29765625)(30.4734375,320.88164062)(28.59453125,320.88164062)
\curveto(27.86015625,320.88164062)(27.13164062,320.79765625)(26.40898437,320.6296875)
\curveto(25.82304687,320.496875)(25.26054687,320.31132812)(24.72148437,320.07304687)
\curveto(24.36992187,319.92070312)(23.78398437,319.60625)(22.96367187,319.1296875)
\lineto(22.96367187,319.88554687)
\curveto(23.94414062,320.61992187)(24.92851562,321.16289062)(25.91679687,321.51445312)
\curveto(26.76835937,321.81523437)(27.65898437,321.965625)(28.58867187,321.965625)
\curveto(29.64335937,321.965625)(30.66289062,321.7625)(31.64726562,321.35625)
\curveto(32.63164062,320.95390625)(33.49101563,320.46367187)(34.22539063,319.88554687)
\closepath
}
}
{
\newrgbcolor{curcolor}{0 0 0.00392157}
\pscustom[linestyle=none,fillstyle=solid,fillcolor=curcolor]
{
\newpath
\moveto(522.3,207.2)
\lineto(543.9,207.2)
\lineto(543.9,208.4)
\lineto(522.3,208.4)
\closepath
}
}
{
\newrgbcolor{curcolor}{0 0 0.03529412}
\pscustom[linestyle=none,fillstyle=solid,fillcolor=curcolor]
{
\newpath
\moveto(522.3,208.3)
\lineto(543.9,208.3)
\lineto(543.9,209.5)
\lineto(522.3,209.5)
\closepath
}
}
{
\newrgbcolor{curcolor}{0 0 0.0627451}
\pscustom[linestyle=none,fillstyle=solid,fillcolor=curcolor]
{
\newpath
\moveto(522.3,209.4)
\lineto(543.9,209.4)
\lineto(543.9,210.6)
\lineto(522.3,210.6)
\closepath
}
}
{
\newrgbcolor{curcolor}{0 0 0.09411765}
\pscustom[linestyle=none,fillstyle=solid,fillcolor=curcolor]
{
\newpath
\moveto(522.3,210.5)
\lineto(543.9,210.5)
\lineto(543.9,211.7)
\lineto(522.3,211.7)
\closepath
}
}
{
\newrgbcolor{curcolor}{0 0 0.1254902}
\pscustom[linestyle=none,fillstyle=solid,fillcolor=curcolor]
{
\newpath
\moveto(522.3,211.6)
\lineto(543.9,211.6)
\lineto(543.9,212.9)
\lineto(522.3,212.9)
\closepath
}
}
{
\newrgbcolor{curcolor}{0 0 0.16078432}
\pscustom[linestyle=none,fillstyle=solid,fillcolor=curcolor]
{
\newpath
\moveto(522.3,212.8)
\lineto(543.9,212.8)
\lineto(543.9,214)
\lineto(522.3,214)
\closepath
}
}
{
\newrgbcolor{curcolor}{0 0 0.1882353}
\pscustom[linestyle=none,fillstyle=solid,fillcolor=curcolor]
{
\newpath
\moveto(522.3,213.9)
\lineto(543.9,213.9)
\lineto(543.9,215.1)
\lineto(522.3,215.1)
\closepath
}
}
{
\newrgbcolor{curcolor}{0 0 0.21960784}
\pscustom[linestyle=none,fillstyle=solid,fillcolor=curcolor]
{
\newpath
\moveto(522.3,215)
\lineto(543.9,215)
\lineto(543.9,216.2)
\lineto(522.3,216.2)
\closepath
}
}
{
\newrgbcolor{curcolor}{0 0 0.25098041}
\pscustom[linestyle=none,fillstyle=solid,fillcolor=curcolor]
{
\newpath
\moveto(522.3,216.1)
\lineto(543.9,216.1)
\lineto(543.9,217.3)
\lineto(522.3,217.3)
\closepath
}
}
{
\newrgbcolor{curcolor}{0 0 0.28235295}
\pscustom[linestyle=none,fillstyle=solid,fillcolor=curcolor]
{
\newpath
\moveto(522.3,217.2)
\lineto(543.9,217.2)
\lineto(543.9,218.5)
\lineto(522.3,218.5)
\closepath
}
}
{
\newrgbcolor{curcolor}{0 0 0.3137255}
\pscustom[linestyle=none,fillstyle=solid,fillcolor=curcolor]
{
\newpath
\moveto(522.3,218.4)
\lineto(543.9,218.4)
\lineto(543.9,219.6)
\lineto(522.3,219.6)
\closepath
}
}
{
\newrgbcolor{curcolor}{0 0 0.34509805}
\pscustom[linestyle=none,fillstyle=solid,fillcolor=curcolor]
{
\newpath
\moveto(522.3,219.5)
\lineto(543.9,219.5)
\lineto(543.9,220.7)
\lineto(522.3,220.7)
\closepath
}
}
{
\newrgbcolor{curcolor}{0 0 0.3764706}
\pscustom[linestyle=none,fillstyle=solid,fillcolor=curcolor]
{
\newpath
\moveto(522.3,220.6)
\lineto(543.9,220.6)
\lineto(543.9,221.8)
\lineto(522.3,221.8)
\closepath
}
}
{
\newrgbcolor{curcolor}{0 0 0.40784314}
\pscustom[linestyle=none,fillstyle=solid,fillcolor=curcolor]
{
\newpath
\moveto(522.3,221.7)
\lineto(543.9,221.7)
\lineto(543.9,222.9)
\lineto(522.3,222.9)
\closepath
}
}
{
\newrgbcolor{curcolor}{0 0 0.43921569}
\pscustom[linestyle=none,fillstyle=solid,fillcolor=curcolor]
{
\newpath
\moveto(522.3,222.8)
\lineto(543.9,222.8)
\lineto(543.9,224.1)
\lineto(522.3,224.1)
\closepath
}
}
{
\newrgbcolor{curcolor}{0 0 0.47058824}
\pscustom[linestyle=none,fillstyle=solid,fillcolor=curcolor]
{
\newpath
\moveto(522.3,224)
\lineto(543.9,224)
\lineto(543.9,225.2)
\lineto(522.3,225.2)
\closepath
}
}
{
\newrgbcolor{curcolor}{0 0 0.50196081}
\pscustom[linestyle=none,fillstyle=solid,fillcolor=curcolor]
{
\newpath
\moveto(522.3,225.1)
\lineto(543.9,225.1)
\lineto(543.9,226.3)
\lineto(522.3,226.3)
\closepath
}
}
{
\newrgbcolor{curcolor}{0 0 0.53333336}
\pscustom[linestyle=none,fillstyle=solid,fillcolor=curcolor]
{
\newpath
\moveto(522.3,226.2)
\lineto(543.9,226.2)
\lineto(543.9,227.4)
\lineto(522.3,227.4)
\closepath
}
}
{
\newrgbcolor{curcolor}{0 0 0.56470591}
\pscustom[linestyle=none,fillstyle=solid,fillcolor=curcolor]
{
\newpath
\moveto(522.3,227.3)
\lineto(543.9,227.3)
\lineto(543.9,228.5)
\lineto(522.3,228.5)
\closepath
}
}
{
\newrgbcolor{curcolor}{0 0 0.59607846}
\pscustom[linestyle=none,fillstyle=solid,fillcolor=curcolor]
{
\newpath
\moveto(522.3,228.4)
\lineto(543.9,228.4)
\lineto(543.9,229.7)
\lineto(522.3,229.7)
\closepath
}
}
{
\newrgbcolor{curcolor}{0 0 0.627451}
\pscustom[linestyle=none,fillstyle=solid,fillcolor=curcolor]
{
\newpath
\moveto(522.3,229.6)
\lineto(543.9,229.6)
\lineto(543.9,230.8)
\lineto(522.3,230.8)
\closepath
}
}
{
\newrgbcolor{curcolor}{0 0 0.65882355}
\pscustom[linestyle=none,fillstyle=solid,fillcolor=curcolor]
{
\newpath
\moveto(522.3,230.7)
\lineto(543.9,230.7)
\lineto(543.9,231.9)
\lineto(522.3,231.9)
\closepath
}
}
{
\newrgbcolor{curcolor}{0 0 0.6901961}
\pscustom[linestyle=none,fillstyle=solid,fillcolor=curcolor]
{
\newpath
\moveto(522.3,231.8)
\lineto(543.9,231.8)
\lineto(543.9,233)
\lineto(522.3,233)
\closepath
}
}
{
\newrgbcolor{curcolor}{0 0 0.72156864}
\pscustom[linestyle=none,fillstyle=solid,fillcolor=curcolor]
{
\newpath
\moveto(522.3,232.9)
\lineto(543.9,232.9)
\lineto(543.9,234.1)
\lineto(522.3,234.1)
\closepath
}
}
{
\newrgbcolor{curcolor}{0 0 0.74901962}
\pscustom[linestyle=none,fillstyle=solid,fillcolor=curcolor]
{
\newpath
\moveto(522.3,234)
\lineto(543.9,234)
\lineto(543.9,235.3)
\lineto(522.3,235.3)
\closepath
}
}
{
\newrgbcolor{curcolor}{0 0 0.78431374}
\pscustom[linestyle=none,fillstyle=solid,fillcolor=curcolor]
{
\newpath
\moveto(522.3,235.2)
\lineto(543.9,235.2)
\lineto(543.9,236.4)
\lineto(522.3,236.4)
\closepath
}
}
{
\newrgbcolor{curcolor}{0 0 0.81568629}
\pscustom[linestyle=none,fillstyle=solid,fillcolor=curcolor]
{
\newpath
\moveto(522.3,236.3)
\lineto(543.9,236.3)
\lineto(543.9,237.5)
\lineto(522.3,237.5)
\closepath
}
}
{
\newrgbcolor{curcolor}{0 0 0.84705883}
\pscustom[linestyle=none,fillstyle=solid,fillcolor=curcolor]
{
\newpath
\moveto(522.3,237.4)
\lineto(543.9,237.4)
\lineto(543.9,238.6)
\lineto(522.3,238.6)
\closepath
}
}
{
\newrgbcolor{curcolor}{0 0 0.87450981}
\pscustom[linestyle=none,fillstyle=solid,fillcolor=curcolor]
{
\newpath
\moveto(522.3,238.5)
\lineto(543.9,238.5)
\lineto(543.9,239.7)
\lineto(522.3,239.7)
\closepath
}
}
{
\newrgbcolor{curcolor}{0 0 0.90588236}
\pscustom[linestyle=none,fillstyle=solid,fillcolor=curcolor]
{
\newpath
\moveto(522.3,239.6)
\lineto(543.9,239.6)
\lineto(543.9,240.9)
\lineto(522.3,240.9)
\closepath
}
}
{
\newrgbcolor{curcolor}{0 0 0.94117647}
\pscustom[linestyle=none,fillstyle=solid,fillcolor=curcolor]
{
\newpath
\moveto(522.3,240.8)
\lineto(543.9,240.8)
\lineto(543.9,242)
\lineto(522.3,242)
\closepath
}
}
{
\newrgbcolor{curcolor}{0 0 0.97254902}
\pscustom[linestyle=none,fillstyle=solid,fillcolor=curcolor]
{
\newpath
\moveto(522.3,241.9)
\lineto(543.9,241.9)
\lineto(543.9,243.1)
\lineto(522.3,243.1)
\closepath
}
}
{
\newrgbcolor{curcolor}{0 0 1}
\pscustom[linestyle=none,fillstyle=solid,fillcolor=curcolor]
{
\newpath
\moveto(522.3,243)
\lineto(543.9,243)
\lineto(543.9,244.2)
\lineto(522.3,244.2)
\closepath
}
}
{
\newrgbcolor{curcolor}{0.02352941 0 1}
\pscustom[linestyle=none,fillstyle=solid,fillcolor=curcolor]
{
\newpath
\moveto(522.3,244.1)
\lineto(543.9,244.1)
\lineto(543.9,245.3)
\lineto(522.3,245.3)
\closepath
}
}
{
\newrgbcolor{curcolor}{0.05098039 0 1}
\pscustom[linestyle=none,fillstyle=solid,fillcolor=curcolor]
{
\newpath
\moveto(522.3,245.2)
\lineto(543.9,245.2)
\lineto(543.9,246.5)
\lineto(522.3,246.5)
\closepath
}
}
{
\newrgbcolor{curcolor}{0.07450981 0 1}
\pscustom[linestyle=none,fillstyle=solid,fillcolor=curcolor]
{
\newpath
\moveto(522.3,246.4)
\lineto(543.9,246.4)
\lineto(543.9,247.6)
\lineto(522.3,247.6)
\closepath
}
}
{
\newrgbcolor{curcolor}{0.09803922 0 1}
\pscustom[linestyle=none,fillstyle=solid,fillcolor=curcolor]
{
\newpath
\moveto(522.3,247.5)
\lineto(543.9,247.5)
\lineto(543.9,248.7)
\lineto(522.3,248.7)
\closepath
}
}
{
\newrgbcolor{curcolor}{0.12156863 0 1}
\pscustom[linestyle=none,fillstyle=solid,fillcolor=curcolor]
{
\newpath
\moveto(522.3,248.6)
\lineto(543.9,248.6)
\lineto(543.9,249.8)
\lineto(522.3,249.8)
\closepath
}
}
{
\newrgbcolor{curcolor}{0.14901961 0 1}
\pscustom[linestyle=none,fillstyle=solid,fillcolor=curcolor]
{
\newpath
\moveto(522.3,249.7)
\lineto(543.9,249.7)
\lineto(543.9,250.9)
\lineto(522.3,250.9)
\closepath
}
}
{
\newrgbcolor{curcolor}{0.17254902 0 1}
\pscustom[linestyle=none,fillstyle=solid,fillcolor=curcolor]
{
\newpath
\moveto(522.3,250.8)
\lineto(543.9,250.8)
\lineto(543.9,252.1)
\lineto(522.3,252.1)
\closepath
}
}
{
\newrgbcolor{curcolor}{0.19607843 0 1}
\pscustom[linestyle=none,fillstyle=solid,fillcolor=curcolor]
{
\newpath
\moveto(522.3,252)
\lineto(543.9,252)
\lineto(543.9,253.2)
\lineto(522.3,253.2)
\closepath
}
}
{
\newrgbcolor{curcolor}{0.21960784 0 1}
\pscustom[linestyle=none,fillstyle=solid,fillcolor=curcolor]
{
\newpath
\moveto(522.3,253.1)
\lineto(543.9,253.1)
\lineto(543.9,254.3)
\lineto(522.3,254.3)
\closepath
}
}
{
\newrgbcolor{curcolor}{0.24705882 0 1}
\pscustom[linestyle=none,fillstyle=solid,fillcolor=curcolor]
{
\newpath
\moveto(522.3,254.2)
\lineto(543.9,254.2)
\lineto(543.9,255.4)
\lineto(522.3,255.4)
\closepath
}
}
{
\newrgbcolor{curcolor}{0.27058825 0 1}
\pscustom[linestyle=none,fillstyle=solid,fillcolor=curcolor]
{
\newpath
\moveto(522.3,255.3)
\lineto(543.9,255.3)
\lineto(543.9,256.5)
\lineto(522.3,256.5)
\closepath
}
}
{
\newrgbcolor{curcolor}{0.29411766 0 1}
\pscustom[linestyle=none,fillstyle=solid,fillcolor=curcolor]
{
\newpath
\moveto(522.3,256.4)
\lineto(543.9,256.4)
\lineto(543.9,257.7)
\lineto(522.3,257.7)
\closepath
}
}
{
\newrgbcolor{curcolor}{0.31764707 0 1}
\pscustom[linestyle=none,fillstyle=solid,fillcolor=curcolor]
{
\newpath
\moveto(522.3,257.6)
\lineto(543.9,257.6)
\lineto(543.9,258.8)
\lineto(522.3,258.8)
\closepath
}
}
{
\newrgbcolor{curcolor}{0.34509805 0 1}
\pscustom[linestyle=none,fillstyle=solid,fillcolor=curcolor]
{
\newpath
\moveto(522.3,258.7)
\lineto(543.9,258.7)
\lineto(543.9,259.9)
\lineto(522.3,259.9)
\closepath
}
}
{
\newrgbcolor{curcolor}{0.36862746 0 1}
\pscustom[linestyle=none,fillstyle=solid,fillcolor=curcolor]
{
\newpath
\moveto(522.3,259.8)
\lineto(543.9,259.8)
\lineto(543.9,261)
\lineto(522.3,261)
\closepath
}
}
{
\newrgbcolor{curcolor}{0.39215687 0 1}
\pscustom[linestyle=none,fillstyle=solid,fillcolor=curcolor]
{
\newpath
\moveto(522.3,260.9)
\lineto(543.9,260.9)
\lineto(543.9,262.1)
\lineto(522.3,262.1)
\closepath
}
}
{
\newrgbcolor{curcolor}{0.41568628 0 1}
\pscustom[linestyle=none,fillstyle=solid,fillcolor=curcolor]
{
\newpath
\moveto(522.3,262)
\lineto(543.9,262)
\lineto(543.9,263.3)
\lineto(522.3,263.3)
\closepath
}
}
{
\newrgbcolor{curcolor}{0.44313726 0 1}
\pscustom[linestyle=none,fillstyle=solid,fillcolor=curcolor]
{
\newpath
\moveto(522.3,263.2)
\lineto(543.9,263.2)
\lineto(543.9,264.4)
\lineto(522.3,264.4)
\closepath
}
}
{
\newrgbcolor{curcolor}{0.46666667 0 1}
\pscustom[linestyle=none,fillstyle=solid,fillcolor=curcolor]
{
\newpath
\moveto(522.3,264.3)
\lineto(543.9,264.3)
\lineto(543.9,265.5)
\lineto(522.3,265.5)
\closepath
}
}
{
\newrgbcolor{curcolor}{0.49019608 0 1}
\pscustom[linestyle=none,fillstyle=solid,fillcolor=curcolor]
{
\newpath
\moveto(522.3,265.4)
\lineto(543.9,265.4)
\lineto(543.9,266.6)
\lineto(522.3,266.6)
\closepath
}
}
{
\newrgbcolor{curcolor}{0.51372552 0 1}
\pscustom[linestyle=none,fillstyle=solid,fillcolor=curcolor]
{
\newpath
\moveto(522.3,266.5)
\lineto(543.9,266.5)
\lineto(543.9,267.7)
\lineto(522.3,267.7)
\closepath
}
}
{
\newrgbcolor{curcolor}{0.53725493 0.00392157 0.99607843}
\pscustom[linestyle=none,fillstyle=solid,fillcolor=curcolor]
{
\newpath
\moveto(522.3,267.6)
\lineto(543.9,267.6)
\lineto(543.9,268.9)
\lineto(522.3,268.9)
\closepath
}
}
{
\newrgbcolor{curcolor}{0.56470591 0.01960784 0.98039216}
\pscustom[linestyle=none,fillstyle=solid,fillcolor=curcolor]
{
\newpath
\moveto(522.3,268.8)
\lineto(543.9,268.8)
\lineto(543.9,270)
\lineto(522.3,270)
\closepath
}
}
{
\newrgbcolor{curcolor}{0.58823532 0.03529412 0.96470588}
\pscustom[linestyle=none,fillstyle=solid,fillcolor=curcolor]
{
\newpath
\moveto(522.3,269.9)
\lineto(543.9,269.9)
\lineto(543.9,271.1)
\lineto(522.3,271.1)
\closepath
}
}
{
\newrgbcolor{curcolor}{0.61176473 0.05098039 0.94901961}
\pscustom[linestyle=none,fillstyle=solid,fillcolor=curcolor]
{
\newpath
\moveto(522.3,271)
\lineto(543.9,271)
\lineto(543.9,272.2)
\lineto(522.3,272.2)
\closepath
}
}
{
\newrgbcolor{curcolor}{0.63529414 0.06666667 0.93333334}
\pscustom[linestyle=none,fillstyle=solid,fillcolor=curcolor]
{
\newpath
\moveto(522.3,272.1)
\lineto(543.9,272.1)
\lineto(543.9,273.3)
\lineto(522.3,273.3)
\closepath
}
}
{
\newrgbcolor{curcolor}{0.65882355 0.08235294 0.91764706}
\pscustom[linestyle=none,fillstyle=solid,fillcolor=curcolor]
{
\newpath
\moveto(522.3,273.2)
\lineto(543.9,273.2)
\lineto(543.9,274.5)
\lineto(522.3,274.5)
\closepath
}
}
{
\newrgbcolor{curcolor}{0.68627453 0.09803922 0.90196079}
\pscustom[linestyle=none,fillstyle=solid,fillcolor=curcolor]
{
\newpath
\moveto(522.3,274.4)
\lineto(543.9,274.4)
\lineto(543.9,275.6)
\lineto(522.3,275.6)
\closepath
}
}
{
\newrgbcolor{curcolor}{0.70980394 0.11372549 0.88627452}
\pscustom[linestyle=none,fillstyle=solid,fillcolor=curcolor]
{
\newpath
\moveto(522.3,275.5)
\lineto(543.9,275.5)
\lineto(543.9,276.7)
\lineto(522.3,276.7)
\closepath
}
}
{
\newrgbcolor{curcolor}{0.73333335 0.12941177 0.87058824}
\pscustom[linestyle=none,fillstyle=solid,fillcolor=curcolor]
{
\newpath
\moveto(522.3,276.6)
\lineto(543.9,276.6)
\lineto(543.9,277.8)
\lineto(522.3,277.8)
\closepath
}
}
{
\newrgbcolor{curcolor}{0.75686276 0.14509805 0.85490197}
\pscustom[linestyle=none,fillstyle=solid,fillcolor=curcolor]
{
\newpath
\moveto(522.3,277.7)
\lineto(543.9,277.7)
\lineto(543.9,279)
\lineto(522.3,279)
\closepath
}
}
{
\newrgbcolor{curcolor}{0.78431374 0.16078432 0.8392157}
\pscustom[linestyle=none,fillstyle=solid,fillcolor=curcolor]
{
\newpath
\moveto(522.3,278.9)
\lineto(543.9,278.9)
\lineto(543.9,280.1)
\lineto(522.3,280.1)
\closepath
}
}
{
\newrgbcolor{curcolor}{0.80784315 0.17647059 0.82352942}
\pscustom[linestyle=none,fillstyle=solid,fillcolor=curcolor]
{
\newpath
\moveto(522.3,280)
\lineto(543.9,280)
\lineto(543.9,281.2)
\lineto(522.3,281.2)
\closepath
}
}
{
\newrgbcolor{curcolor}{0.83137256 0.19215687 0.80784315}
\pscustom[linestyle=none,fillstyle=solid,fillcolor=curcolor]
{
\newpath
\moveto(522.3,281.1)
\lineto(543.9,281.1)
\lineto(543.9,282.3)
\lineto(522.3,282.3)
\closepath
}
}
{
\newrgbcolor{curcolor}{0.85490197 0.20784314 0.79215688}
\pscustom[linestyle=none,fillstyle=solid,fillcolor=curcolor]
{
\newpath
\moveto(522.3,282.2)
\lineto(543.9,282.2)
\lineto(543.9,283.4)
\lineto(522.3,283.4)
\closepath
}
}
{
\newrgbcolor{curcolor}{0.87843138 0.22352941 0.7764706}
\pscustom[linestyle=none,fillstyle=solid,fillcolor=curcolor]
{
\newpath
\moveto(522.3,283.3)
\lineto(543.9,283.3)
\lineto(543.9,284.6)
\lineto(522.3,284.6)
\closepath
}
}
{
\newrgbcolor{curcolor}{0.90588236 0.23921569 0.76078433}
\pscustom[linestyle=none,fillstyle=solid,fillcolor=curcolor]
{
\newpath
\moveto(522.3,284.5)
\lineto(543.9,284.5)
\lineto(543.9,285.7)
\lineto(522.3,285.7)
\closepath
}
}
{
\newrgbcolor{curcolor}{0.92941177 0.25490198 0.74509805}
\pscustom[linestyle=none,fillstyle=solid,fillcolor=curcolor]
{
\newpath
\moveto(522.3,285.6)
\lineto(543.9,285.6)
\lineto(543.9,286.8)
\lineto(522.3,286.8)
\closepath
}
}
{
\newrgbcolor{curcolor}{0.95294118 0.27058825 0.72941178}
\pscustom[linestyle=none,fillstyle=solid,fillcolor=curcolor]
{
\newpath
\moveto(522.3,286.7)
\lineto(543.9,286.7)
\lineto(543.9,287.9)
\lineto(522.3,287.9)
\closepath
}
}
{
\newrgbcolor{curcolor}{0.97647059 0.28627452 0.71372551}
\pscustom[linestyle=none,fillstyle=solid,fillcolor=curcolor]
{
\newpath
\moveto(522.3,287.8)
\lineto(543.9,287.8)
\lineto(543.9,289)
\lineto(522.3,289)
\closepath
}
}
{
\newrgbcolor{curcolor}{1 0.3019608 0.69803923}
\pscustom[linestyle=none,fillstyle=solid,fillcolor=curcolor]
{
\newpath
\moveto(522.3,288.9)
\lineto(543.9,288.9)
\lineto(543.9,290.2)
\lineto(522.3,290.2)
\closepath
}
}
{
\newrgbcolor{curcolor}{1 0.31764707 0.68235296}
\pscustom[linestyle=none,fillstyle=solid,fillcolor=curcolor]
{
\newpath
\moveto(522.3,290.1)
\lineto(543.9,290.1)
\lineto(543.9,291.3)
\lineto(522.3,291.3)
\closepath
}
}
{
\newrgbcolor{curcolor}{1 0.33333334 0.66666669}
\pscustom[linestyle=none,fillstyle=solid,fillcolor=curcolor]
{
\newpath
\moveto(522.3,291.2)
\lineto(543.9,291.2)
\lineto(543.9,292.4)
\lineto(522.3,292.4)
\closepath
}
}
{
\newrgbcolor{curcolor}{1 0.34901962 0.65098041}
\pscustom[linestyle=none,fillstyle=solid,fillcolor=curcolor]
{
\newpath
\moveto(522.3,292.3)
\lineto(543.9,292.3)
\lineto(543.9,293.5)
\lineto(522.3,293.5)
\closepath
}
}
{
\newrgbcolor{curcolor}{1 0.36470589 0.63529414}
\pscustom[linestyle=none,fillstyle=solid,fillcolor=curcolor]
{
\newpath
\moveto(522.3,293.4)
\lineto(543.9,293.4)
\lineto(543.9,294.6)
\lineto(522.3,294.6)
\closepath
}
}
{
\newrgbcolor{curcolor}{1 0.38039216 0.61960787}
\pscustom[linestyle=none,fillstyle=solid,fillcolor=curcolor]
{
\newpath
\moveto(522.3,294.5)
\lineto(543.9,294.5)
\lineto(543.9,295.8)
\lineto(522.3,295.8)
\closepath
}
}
{
\newrgbcolor{curcolor}{1 0.39607844 0.60392159}
\pscustom[linestyle=none,fillstyle=solid,fillcolor=curcolor]
{
\newpath
\moveto(522.3,295.7)
\lineto(543.9,295.7)
\lineto(543.9,296.9)
\lineto(522.3,296.9)
\closepath
}
}
{
\newrgbcolor{curcolor}{1 0.41176471 0.58823532}
\pscustom[linestyle=none,fillstyle=solid,fillcolor=curcolor]
{
\newpath
\moveto(522.3,296.8)
\lineto(543.9,296.8)
\lineto(543.9,298)
\lineto(522.3,298)
\closepath
}
}
{
\newrgbcolor{curcolor}{1 0.42745098 0.57254905}
\pscustom[linestyle=none,fillstyle=solid,fillcolor=curcolor]
{
\newpath
\moveto(522.3,297.9)
\lineto(543.9,297.9)
\lineto(543.9,299.1)
\lineto(522.3,299.1)
\closepath
}
}
{
\newrgbcolor{curcolor}{1 0.44313726 0.55686277}
\pscustom[linestyle=none,fillstyle=solid,fillcolor=curcolor]
{
\newpath
\moveto(522.3,299)
\lineto(543.9,299)
\lineto(543.9,300.2)
\lineto(522.3,300.2)
\closepath
}
}
{
\newrgbcolor{curcolor}{1 0.45882353 0.5411765}
\pscustom[linestyle=none,fillstyle=solid,fillcolor=curcolor]
{
\newpath
\moveto(522.3,300.1)
\lineto(543.9,300.1)
\lineto(543.9,301.4)
\lineto(522.3,301.4)
\closepath
}
}
{
\newrgbcolor{curcolor}{1 0.47450981 0.52549022}
\pscustom[linestyle=none,fillstyle=solid,fillcolor=curcolor]
{
\newpath
\moveto(522.3,301.3)
\lineto(543.9,301.3)
\lineto(543.9,302.5)
\lineto(522.3,302.5)
\closepath
}
}
{
\newrgbcolor{curcolor}{1 0.49019608 0.50980395}
\pscustom[linestyle=none,fillstyle=solid,fillcolor=curcolor]
{
\newpath
\moveto(522.3,302.4)
\lineto(543.9,302.4)
\lineto(543.9,303.6)
\lineto(522.3,303.6)
\closepath
}
}
{
\newrgbcolor{curcolor}{1 0.50588238 0.49411765}
\pscustom[linestyle=none,fillstyle=solid,fillcolor=curcolor]
{
\newpath
\moveto(522.3,303.5)
\lineto(543.9,303.5)
\lineto(543.9,304.7)
\lineto(522.3,304.7)
\closepath
}
}
{
\newrgbcolor{curcolor}{1 0.52156866 0.47843137}
\pscustom[linestyle=none,fillstyle=solid,fillcolor=curcolor]
{
\newpath
\moveto(522.3,304.6)
\lineto(543.9,304.6)
\lineto(543.9,305.8)
\lineto(522.3,305.8)
\closepath
}
}
{
\newrgbcolor{curcolor}{1 0.53333336 0.46666667}
\pscustom[linestyle=none,fillstyle=solid,fillcolor=curcolor]
{
\newpath
\moveto(522.3,305.7)
\lineto(543.9,305.7)
\lineto(543.9,307)
\lineto(522.3,307)
\closepath
}
}
{
\newrgbcolor{curcolor}{1 0.5529412 0.44705883}
\pscustom[linestyle=none,fillstyle=solid,fillcolor=curcolor]
{
\newpath
\moveto(522.3,306.9)
\lineto(543.9,306.9)
\lineto(543.9,308.1)
\lineto(522.3,308.1)
\closepath
}
}
{
\newrgbcolor{curcolor}{1 0.56862748 0.43137255}
\pscustom[linestyle=none,fillstyle=solid,fillcolor=curcolor]
{
\newpath
\moveto(522.3,308)
\lineto(543.9,308)
\lineto(543.9,309.2)
\lineto(522.3,309.2)
\closepath
}
}
{
\newrgbcolor{curcolor}{1 0.58431375 0.41568628}
\pscustom[linestyle=none,fillstyle=solid,fillcolor=curcolor]
{
\newpath
\moveto(522.3,309.1)
\lineto(543.9,309.1)
\lineto(543.9,310.3)
\lineto(522.3,310.3)
\closepath
}
}
{
\newrgbcolor{curcolor}{1 0.59607846 0.40392157}
\pscustom[linestyle=none,fillstyle=solid,fillcolor=curcolor]
{
\newpath
\moveto(522.3,310.2)
\lineto(543.9,310.2)
\lineto(543.9,311.4)
\lineto(522.3,311.4)
\closepath
}
}
{
\newrgbcolor{curcolor}{1 0.61176473 0.3882353}
\pscustom[linestyle=none,fillstyle=solid,fillcolor=curcolor]
{
\newpath
\moveto(522.3,311.3)
\lineto(543.9,311.3)
\lineto(543.9,312.6)
\lineto(522.3,312.6)
\closepath
}
}
{
\newrgbcolor{curcolor}{1 0.63137257 0.36862746}
\pscustom[linestyle=none,fillstyle=solid,fillcolor=curcolor]
{
\newpath
\moveto(522.3,312.5)
\lineto(543.9,312.5)
\lineto(543.9,313.7)
\lineto(522.3,313.7)
\closepath
}
}
{
\newrgbcolor{curcolor}{1 0.64705884 0.35294119}
\pscustom[linestyle=none,fillstyle=solid,fillcolor=curcolor]
{
\newpath
\moveto(522.3,313.6)
\lineto(543.9,313.6)
\lineto(543.9,314.8)
\lineto(522.3,314.8)
\closepath
}
}
{
\newrgbcolor{curcolor}{1 0.65882355 0.34117648}
\pscustom[linestyle=none,fillstyle=solid,fillcolor=curcolor]
{
\newpath
\moveto(522.3,314.7)
\lineto(543.9,314.7)
\lineto(543.9,315.9)
\lineto(522.3,315.9)
\closepath
}
}
{
\newrgbcolor{curcolor}{1 0.67450982 0.32549021}
\pscustom[linestyle=none,fillstyle=solid,fillcolor=curcolor]
{
\newpath
\moveto(522.3,315.8)
\lineto(543.9,315.8)
\lineto(543.9,317)
\lineto(522.3,317)
\closepath
}
}
{
\newrgbcolor{curcolor}{1 0.6901961 0.30980393}
\pscustom[linestyle=none,fillstyle=solid,fillcolor=curcolor]
{
\newpath
\moveto(522.3,316.9)
\lineto(543.9,316.9)
\lineto(543.9,318.2)
\lineto(522.3,318.2)
\closepath
}
}
{
\newrgbcolor{curcolor}{1 0.70980394 0.29019609}
\pscustom[linestyle=none,fillstyle=solid,fillcolor=curcolor]
{
\newpath
\moveto(522.3,318.1)
\lineto(543.9,318.1)
\lineto(543.9,319.3)
\lineto(522.3,319.3)
\closepath
}
}
{
\newrgbcolor{curcolor}{1 0.72156864 0.27843139}
\pscustom[linestyle=none,fillstyle=solid,fillcolor=curcolor]
{
\newpath
\moveto(522.3,319.2)
\lineto(543.9,319.2)
\lineto(543.9,320.4)
\lineto(522.3,320.4)
\closepath
}
}
{
\newrgbcolor{curcolor}{1 0.73725492 0.26274511}
\pscustom[linestyle=none,fillstyle=solid,fillcolor=curcolor]
{
\newpath
\moveto(522.3,320.3)
\lineto(543.9,320.3)
\lineto(543.9,321.5)
\lineto(522.3,321.5)
\closepath
}
}
{
\newrgbcolor{curcolor}{1 0.75294119 0.24705882}
\pscustom[linestyle=none,fillstyle=solid,fillcolor=curcolor]
{
\newpath
\moveto(522.3,321.4)
\lineto(543.9,321.4)
\lineto(543.9,322.6)
\lineto(522.3,322.6)
\closepath
}
}
{
\newrgbcolor{curcolor}{1 0.76862746 0.23137255}
\pscustom[linestyle=none,fillstyle=solid,fillcolor=curcolor]
{
\newpath
\moveto(522.3,322.5)
\lineto(543.9,322.5)
\lineto(543.9,323.8)
\lineto(522.3,323.8)
\closepath
}
}
{
\newrgbcolor{curcolor}{1 0.78431374 0.21568628}
\pscustom[linestyle=none,fillstyle=solid,fillcolor=curcolor]
{
\newpath
\moveto(522.3,323.7)
\lineto(543.9,323.7)
\lineto(543.9,324.9)
\lineto(522.3,324.9)
\closepath
}
}
{
\newrgbcolor{curcolor}{1 0.80000001 0.2}
\pscustom[linestyle=none,fillstyle=solid,fillcolor=curcolor]
{
\newpath
\moveto(522.3,324.8)
\lineto(543.9,324.8)
\lineto(543.9,326)
\lineto(522.3,326)
\closepath
}
}
{
\newrgbcolor{curcolor}{1 0.81568629 0.18431373}
\pscustom[linestyle=none,fillstyle=solid,fillcolor=curcolor]
{
\newpath
\moveto(522.3,325.9)
\lineto(543.9,325.9)
\lineto(543.9,327.1)
\lineto(522.3,327.1)
\closepath
}
}
{
\newrgbcolor{curcolor}{1 0.83137256 0.16862746}
\pscustom[linestyle=none,fillstyle=solid,fillcolor=curcolor]
{
\newpath
\moveto(522.3,327)
\lineto(543.9,327)
\lineto(543.9,328.2)
\lineto(522.3,328.2)
\closepath
}
}
{
\newrgbcolor{curcolor}{1 0.84705883 0.15294118}
\pscustom[linestyle=none,fillstyle=solid,fillcolor=curcolor]
{
\newpath
\moveto(522.3,328.1)
\lineto(543.9,328.1)
\lineto(543.9,329.4)
\lineto(522.3,329.4)
\closepath
}
}
{
\newrgbcolor{curcolor}{1 0.86274511 0.13725491}
\pscustom[linestyle=none,fillstyle=solid,fillcolor=curcolor]
{
\newpath
\moveto(522.3,329.3)
\lineto(543.9,329.3)
\lineto(543.9,330.5)
\lineto(522.3,330.5)
\closepath
}
}
{
\newrgbcolor{curcolor}{1 0.87843138 0.12156863}
\pscustom[linestyle=none,fillstyle=solid,fillcolor=curcolor]
{
\newpath
\moveto(522.3,330.4)
\lineto(543.9,330.4)
\lineto(543.9,331.6)
\lineto(522.3,331.6)
\closepath
}
}
{
\newrgbcolor{curcolor}{1 0.89411765 0.10588235}
\pscustom[linestyle=none,fillstyle=solid,fillcolor=curcolor]
{
\newpath
\moveto(522.3,331.5)
\lineto(543.9,331.5)
\lineto(543.9,332.7)
\lineto(522.3,332.7)
\closepath
}
}
{
\newrgbcolor{curcolor}{1 0.90980393 0.09019608}
\pscustom[linestyle=none,fillstyle=solid,fillcolor=curcolor]
{
\newpath
\moveto(522.3,332.6)
\lineto(543.9,332.6)
\lineto(543.9,333.8)
\lineto(522.3,333.8)
\closepath
}
}
{
\newrgbcolor{curcolor}{1 0.9254902 0.07450981}
\pscustom[linestyle=none,fillstyle=solid,fillcolor=curcolor]
{
\newpath
\moveto(522.3,333.7)
\lineto(543.9,333.7)
\lineto(543.9,335)
\lineto(522.3,335)
\closepath
}
}
{
\newrgbcolor{curcolor}{1 0.94117647 0.05882353}
\pscustom[linestyle=none,fillstyle=solid,fillcolor=curcolor]
{
\newpath
\moveto(522.3,334.9)
\lineto(543.9,334.9)
\lineto(543.9,336.1)
\lineto(522.3,336.1)
\closepath
}
}
{
\newrgbcolor{curcolor}{1 0.95686275 0.04313726}
\pscustom[linestyle=none,fillstyle=solid,fillcolor=curcolor]
{
\newpath
\moveto(522.3,336)
\lineto(543.9,336)
\lineto(543.9,337.2)
\lineto(522.3,337.2)
\closepath
}
}
{
\newrgbcolor{curcolor}{1 0.97254902 0.02745098}
\pscustom[linestyle=none,fillstyle=solid,fillcolor=curcolor]
{
\newpath
\moveto(522.3,337.1)
\lineto(543.9,337.1)
\lineto(543.9,338.3)
\lineto(522.3,338.3)
\closepath
}
}
{
\newrgbcolor{curcolor}{1 0.98823529 0.01176471}
\pscustom[linestyle=none,fillstyle=solid,fillcolor=curcolor]
{
\newpath
\moveto(522.3,338.2)
\lineto(543.9,338.2)
\lineto(543.9,339.4)
\lineto(522.3,339.4)
\closepath
}
}
{
\newrgbcolor{curcolor}{1 1 0.02352941}
\pscustom[linestyle=none,fillstyle=solid,fillcolor=curcolor]
{
\newpath
\moveto(522.3,339.3)
\lineto(543.9,339.3)
\lineto(543.9,340.6)
\lineto(522.3,340.6)
\closepath
}
}
{
\newrgbcolor{curcolor}{1 1 0.12941177}
\pscustom[linestyle=none,fillstyle=solid,fillcolor=curcolor]
{
\newpath
\moveto(522.3,340.5)
\lineto(543.9,340.5)
\lineto(543.9,341.7)
\lineto(522.3,341.7)
\closepath
}
}
{
\newrgbcolor{curcolor}{1 1 0.22352941}
\pscustom[linestyle=none,fillstyle=solid,fillcolor=curcolor]
{
\newpath
\moveto(522.3,341.6)
\lineto(543.9,341.6)
\lineto(543.9,342.8)
\lineto(522.3,342.8)
\closepath
}
}
{
\newrgbcolor{curcolor}{1 1 0.32156864}
\pscustom[linestyle=none,fillstyle=solid,fillcolor=curcolor]
{
\newpath
\moveto(522.3,342.7)
\lineto(543.9,342.7)
\lineto(543.9,343.9)
\lineto(522.3,343.9)
\closepath
}
}
{
\newrgbcolor{curcolor}{1 1 0.41568628}
\pscustom[linestyle=none,fillstyle=solid,fillcolor=curcolor]
{
\newpath
\moveto(522.3,343.8)
\lineto(543.9,343.8)
\lineto(543.9,345)
\lineto(522.3,345)
\closepath
}
}
{
\newrgbcolor{curcolor}{1 1 0.51372552}
\pscustom[linestyle=none,fillstyle=solid,fillcolor=curcolor]
{
\newpath
\moveto(522.3,344.9)
\lineto(543.9,344.9)
\lineto(543.9,346.2)
\lineto(522.3,346.2)
\closepath
}
}
{
\newrgbcolor{curcolor}{1 1 0.6156863}
\pscustom[linestyle=none,fillstyle=solid,fillcolor=curcolor]
{
\newpath
\moveto(522.3,346.1)
\lineto(543.9,346.1)
\lineto(543.9,347.3)
\lineto(522.3,347.3)
\closepath
}
}
{
\newrgbcolor{curcolor}{1 1 0.71372551}
\pscustom[linestyle=none,fillstyle=solid,fillcolor=curcolor]
{
\newpath
\moveto(522.3,347.2)
\lineto(543.9,347.2)
\lineto(543.9,348.4)
\lineto(522.3,348.4)
\closepath
}
}
{
\newrgbcolor{curcolor}{1 1 0.80784315}
\pscustom[linestyle=none,fillstyle=solid,fillcolor=curcolor]
{
\newpath
\moveto(522.3,348.3)
\lineto(543.9,348.3)
\lineto(543.9,349.5)
\lineto(522.3,349.5)
\closepath
}
}
{
\newrgbcolor{curcolor}{1 1 0.90588236}
\pscustom[linestyle=none,fillstyle=solid,fillcolor=curcolor]
{
\newpath
\moveto(522.3,349.4)
\lineto(543.9,349.4)
\lineto(543.9,350.6)
\lineto(522.3,350.6)
\closepath
}
}
{
\newrgbcolor{curcolor}{0 0 0}
\pscustom[linewidth=1,linecolor=curcolor]
{
\newpath
\moveto(522.3,207.2)
\lineto(543.9,207.2)
\lineto(543.9,350.6)
\lineto(522.3,350.6)
\closepath
}
}
{
\newrgbcolor{curcolor}{0 0 0}
\pscustom[linewidth=1,linecolor=curcolor]
{
\newpath
\moveto(543.9,207.2)
\lineto(534.9,207.2)
}
}
{
\newrgbcolor{curcolor}{0 0 0}
\pscustom[linestyle=none,fillstyle=solid,fillcolor=curcolor]
{
\newpath
\moveto(552.58085938,205.878125)
\lineto(552.58085938,206.93867188)
\lineto(555.82109375,206.93867188)
\lineto(555.82109375,205.878125)
\closepath
}
}
{
\newrgbcolor{curcolor}{0 0 0}
\pscustom[linestyle=none,fillstyle=solid,fillcolor=curcolor]
{
\newpath
\moveto(560.66679688,203.3)
\lineto(559.61210938,203.3)
\lineto(559.61210938,210.02070313)
\curveto(559.35820313,209.77851563)(559.02421875,209.53632813)(558.61015625,209.29414063)
\curveto(558.2,209.05195313)(557.83085938,208.8703125)(557.50273438,208.74921875)
\lineto(557.50273438,209.76875)
\curveto(558.09257813,210.04609375)(558.60820313,210.38203125)(559.04960938,210.7765625)
\curveto(559.49101563,211.17109375)(559.80351563,211.55390625)(559.98710938,211.925)
\lineto(560.66679688,211.925)
\closepath
}
}
{
\newrgbcolor{curcolor}{0 0 0}
\pscustom[linestyle=none,fillstyle=solid,fillcolor=curcolor]
{
\newpath
\moveto(563.36796875,205.55)
\lineto(564.47539063,205.64375)
\curveto(564.55742188,205.1046875)(564.746875,204.6984375)(565.04375,204.425)
\curveto(565.34453125,204.15546875)(565.70585938,204.02070313)(566.12773438,204.02070313)
\curveto(566.63554688,204.02070313)(567.06523438,204.21210938)(567.41679688,204.59492188)
\curveto(567.76835938,204.97773438)(567.94414063,205.48554688)(567.94414063,206.11835938)
\curveto(567.94414063,206.71992188)(567.77421875,207.19453125)(567.434375,207.5421875)
\curveto(567.0984375,207.88984375)(566.65703125,208.06367188)(566.11015625,208.06367188)
\curveto(565.7703125,208.06367188)(565.46367188,207.98554688)(565.19023438,207.82929688)
\curveto(564.91679688,207.67695313)(564.70195313,207.47773438)(564.54570313,207.23164063)
\lineto(563.55546875,207.36054688)
\lineto(564.3875,211.77265625)
\lineto(568.65898438,211.77265625)
\lineto(568.65898438,210.76484375)
\lineto(565.23125,210.76484375)
\lineto(564.76835938,208.45625)
\curveto(565.28398438,208.815625)(565.825,208.9953125)(566.39140625,208.9953125)
\curveto(567.14140625,208.9953125)(567.77421875,208.73554688)(568.28984375,208.21601563)
\curveto(568.80546875,207.69648438)(569.06328125,207.02851563)(569.06328125,206.21210938)
\curveto(569.06328125,205.43476563)(568.83671875,204.76289063)(568.38359375,204.19648438)
\curveto(567.8328125,203.50117188)(567.08085938,203.15351563)(566.12773438,203.15351563)
\curveto(565.34648438,203.15351563)(564.7078125,203.37226563)(564.21171875,203.80976563)
\curveto(563.71953125,204.24726563)(563.43828125,204.82734375)(563.36796875,205.55)
\closepath
}
}
{
\newrgbcolor{curcolor}{0 0 0}
\pscustom[linewidth=1,linecolor=curcolor]
{
\newpath
\moveto(522.3,207.2)
\lineto(531.3,207.2)
\moveto(543.9,227.6)
\lineto(534.9,227.6)
}
}
{
\newrgbcolor{curcolor}{0 0 0}
\pscustom[linestyle=none,fillstyle=solid,fillcolor=curcolor]
{
\newpath
\moveto(552.58085938,226.278125)
\lineto(552.58085938,227.33867187)
\lineto(555.82109375,227.33867187)
\lineto(555.82109375,226.278125)
\closepath
}
}
{
\newrgbcolor{curcolor}{0 0 0}
\pscustom[linestyle=none,fillstyle=solid,fillcolor=curcolor]
{
\newpath
\moveto(560.66679688,223.7)
\lineto(559.61210938,223.7)
\lineto(559.61210938,230.42070312)
\curveto(559.35820313,230.17851562)(559.02421875,229.93632812)(558.61015625,229.69414062)
\curveto(558.2,229.45195312)(557.83085938,229.2703125)(557.50273438,229.14921875)
\lineto(557.50273438,230.16875)
\curveto(558.09257813,230.44609375)(558.60820313,230.78203125)(559.04960938,231.1765625)
\curveto(559.49101563,231.57109375)(559.80351563,231.95390625)(559.98710938,232.325)
\lineto(560.66679688,232.325)
\closepath
}
}
{
\newrgbcolor{curcolor}{0 0 0}
\pscustom[linestyle=none,fillstyle=solid,fillcolor=curcolor]
{
\newpath
\moveto(563.36796875,227.93632812)
\curveto(563.36796875,228.95195312)(563.47148438,229.76835937)(563.67851563,230.38554687)
\curveto(563.88945313,231.00664062)(564.2,231.48515625)(564.61015625,231.82109375)
\curveto(565.02421875,232.15703125)(565.54375,232.325)(566.16875,232.325)
\curveto(566.6296875,232.325)(567.03398438,232.23125)(567.38164063,232.04375)
\curveto(567.72929688,231.86015625)(568.01640625,231.59257812)(568.24296875,231.24101562)
\curveto(568.46953125,230.89335937)(568.64726563,230.46757812)(568.77617188,229.96367187)
\curveto(568.90507813,229.46367187)(568.96953125,228.78789062)(568.96953125,227.93632812)
\curveto(568.96953125,226.92851562)(568.86601563,226.1140625)(568.65898438,225.49296875)
\curveto(568.45195313,224.87578125)(568.14140625,224.39726562)(567.72734375,224.05742187)
\curveto(567.3171875,223.72148437)(566.79765625,223.55351562)(566.16875,223.55351562)
\curveto(565.340625,223.55351562)(564.69023438,223.85039062)(564.21757813,224.44414062)
\curveto(563.65117188,225.15898437)(563.36796875,226.32304687)(563.36796875,227.93632812)
\closepath
\moveto(564.45195313,227.93632812)
\curveto(564.45195313,226.52617187)(564.61601563,225.58671875)(564.94414063,225.11796875)
\curveto(565.27617188,224.653125)(565.684375,224.42070312)(566.16875,224.42070312)
\curveto(566.653125,224.42070312)(567.059375,224.65507812)(567.3875,225.12382812)
\curveto(567.71953125,225.59257812)(567.88554688,226.53007812)(567.88554688,227.93632812)
\curveto(567.88554688,229.35039062)(567.71953125,230.28984375)(567.3875,230.7546875)
\curveto(567.059375,231.21953125)(566.64921875,231.45195312)(566.15703125,231.45195312)
\curveto(565.67265625,231.45195312)(565.2859375,231.246875)(564.996875,230.83671875)
\curveto(564.63359375,230.31328125)(564.45195313,229.34648437)(564.45195313,227.93632812)
\closepath
}
}
{
\newrgbcolor{curcolor}{0 0 0}
\pscustom[linewidth=1,linecolor=curcolor]
{
\newpath
\moveto(522.3,227.6)
\lineto(531.3,227.6)
\moveto(543.9,248.1)
\lineto(534.9,248.1)
}
}
{
\newrgbcolor{curcolor}{0 0 0}
\pscustom[linestyle=none,fillstyle=solid,fillcolor=curcolor]
{
\newpath
\moveto(552.58085938,246.778125)
\lineto(552.58085938,247.83867187)
\lineto(555.82109375,247.83867187)
\lineto(555.82109375,246.778125)
\closepath
}
}
{
\newrgbcolor{curcolor}{0 0 0}
\pscustom[linestyle=none,fillstyle=solid,fillcolor=curcolor]
{
\newpath
\moveto(556.69414063,246.45)
\lineto(557.8015625,246.54375)
\curveto(557.88359375,246.0046875)(558.07304688,245.5984375)(558.36992188,245.325)
\curveto(558.67070313,245.05546875)(559.03203125,244.92070312)(559.45390625,244.92070312)
\curveto(559.96171875,244.92070312)(560.39140625,245.11210937)(560.74296875,245.49492187)
\curveto(561.09453125,245.87773437)(561.2703125,246.38554687)(561.2703125,247.01835937)
\curveto(561.2703125,247.61992187)(561.10039063,248.09453125)(560.76054688,248.4421875)
\curveto(560.42460938,248.78984375)(559.98320313,248.96367187)(559.43632813,248.96367187)
\curveto(559.09648438,248.96367187)(558.78984375,248.88554687)(558.51640625,248.72929687)
\curveto(558.24296875,248.57695312)(558.028125,248.37773437)(557.871875,248.13164062)
\lineto(556.88164063,248.26054687)
\lineto(557.71367188,252.67265625)
\lineto(561.98515625,252.67265625)
\lineto(561.98515625,251.66484375)
\lineto(558.55742188,251.66484375)
\lineto(558.09453125,249.35625)
\curveto(558.61015625,249.715625)(559.15117188,249.8953125)(559.71757813,249.8953125)
\curveto(560.46757813,249.8953125)(561.10039063,249.63554687)(561.61601563,249.11601562)
\curveto(562.13164063,248.59648437)(562.38945313,247.92851562)(562.38945313,247.11210937)
\curveto(562.38945313,246.33476562)(562.16289063,245.66289062)(561.70976563,245.09648437)
\curveto(561.15898438,244.40117187)(560.40703125,244.05351562)(559.45390625,244.05351562)
\curveto(558.67265625,244.05351562)(558.03398438,244.27226562)(557.53789063,244.70976562)
\curveto(557.04570313,245.14726562)(556.76445313,245.72734375)(556.69414063,246.45)
\closepath
}
}
{
\newrgbcolor{curcolor}{0 0 0}
\pscustom[linewidth=1,linecolor=curcolor]
{
\newpath
\moveto(522.3,248.1)
\lineto(531.3,248.1)
\moveto(543.9,268.6)
\lineto(534.9,268.6)
}
}
{
\newrgbcolor{curcolor}{0 0 0}
\pscustom[linestyle=none,fillstyle=solid,fillcolor=curcolor]
{
\newpath
\moveto(552.69804688,268.93632812)
\curveto(552.69804688,269.95195312)(552.8015625,270.76835937)(553.00859375,271.38554687)
\curveto(553.21953125,272.00664062)(553.53007813,272.48515625)(553.94023438,272.82109375)
\curveto(554.35429688,273.15703125)(554.87382813,273.325)(555.49882813,273.325)
\curveto(555.95976563,273.325)(556.3640625,273.23125)(556.71171875,273.04375)
\curveto(557.059375,272.86015625)(557.34648438,272.59257812)(557.57304688,272.24101562)
\curveto(557.79960938,271.89335937)(557.97734375,271.46757812)(558.10625,270.96367187)
\curveto(558.23515625,270.46367187)(558.29960938,269.78789062)(558.29960938,268.93632812)
\curveto(558.29960938,267.92851562)(558.19609375,267.1140625)(557.9890625,266.49296875)
\curveto(557.78203125,265.87578125)(557.47148438,265.39726562)(557.05742188,265.05742187)
\curveto(556.64726563,264.72148437)(556.12773438,264.55351562)(555.49882813,264.55351562)
\curveto(554.67070313,264.55351562)(554.0203125,264.85039062)(553.54765625,265.44414062)
\curveto(552.98125,266.15898437)(552.69804688,267.32304687)(552.69804688,268.93632812)
\closepath
\moveto(553.78203125,268.93632812)
\curveto(553.78203125,267.52617187)(553.94609375,266.58671875)(554.27421875,266.11796875)
\curveto(554.60625,265.653125)(555.01445313,265.42070312)(555.49882813,265.42070312)
\curveto(555.98320313,265.42070312)(556.38945313,265.65507812)(556.71757813,266.12382812)
\curveto(557.04960938,266.59257812)(557.215625,267.53007812)(557.215625,268.93632812)
\curveto(557.215625,270.35039062)(557.04960938,271.28984375)(556.71757813,271.7546875)
\curveto(556.38945313,272.21953125)(555.97929688,272.45195312)(555.48710938,272.45195312)
\curveto(555.00273438,272.45195312)(554.61601563,272.246875)(554.32695313,271.83671875)
\curveto(553.96367188,271.31328125)(553.78203125,270.34648437)(553.78203125,268.93632812)
\closepath
}
}
{
\newrgbcolor{curcolor}{0 0 0}
\pscustom[linewidth=1,linecolor=curcolor]
{
\newpath
\moveto(522.3,268.6)
\lineto(531.3,268.6)
\moveto(543.9,289.1)
\lineto(534.9,289.1)
}
}
{
\newrgbcolor{curcolor}{0 0 0}
\pscustom[linestyle=none,fillstyle=solid,fillcolor=curcolor]
{
\newpath
\moveto(552.69804688,287.45)
\lineto(553.80546875,287.54375)
\curveto(553.8875,287.0046875)(554.07695313,286.5984375)(554.37382813,286.325)
\curveto(554.67460938,286.05546875)(555.0359375,285.92070312)(555.4578125,285.92070312)
\curveto(555.965625,285.92070312)(556.3953125,286.11210937)(556.746875,286.49492187)
\curveto(557.0984375,286.87773437)(557.27421875,287.38554687)(557.27421875,288.01835937)
\curveto(557.27421875,288.61992187)(557.10429688,289.09453125)(556.76445313,289.4421875)
\curveto(556.42851563,289.78984375)(555.98710938,289.96367187)(555.44023438,289.96367187)
\curveto(555.10039063,289.96367187)(554.79375,289.88554687)(554.5203125,289.72929687)
\curveto(554.246875,289.57695312)(554.03203125,289.37773437)(553.87578125,289.13164062)
\lineto(552.88554688,289.26054687)
\lineto(553.71757813,293.67265625)
\lineto(557.9890625,293.67265625)
\lineto(557.9890625,292.66484375)
\lineto(554.56132813,292.66484375)
\lineto(554.0984375,290.35625)
\curveto(554.6140625,290.715625)(555.15507813,290.8953125)(555.72148438,290.8953125)
\curveto(556.47148438,290.8953125)(557.10429688,290.63554687)(557.61992188,290.11601562)
\curveto(558.13554688,289.59648437)(558.39335938,288.92851562)(558.39335938,288.11210937)
\curveto(558.39335938,287.33476562)(558.16679688,286.66289062)(557.71367188,286.09648437)
\curveto(557.16289063,285.40117187)(556.4109375,285.05351562)(555.4578125,285.05351562)
\curveto(554.6765625,285.05351562)(554.03789063,285.27226562)(553.54179688,285.70976562)
\curveto(553.04960938,286.14726562)(552.76835938,286.72734375)(552.69804688,287.45)
\closepath
}
}
{
\newrgbcolor{curcolor}{0 0 0}
\pscustom[linewidth=1,linecolor=curcolor]
{
\newpath
\moveto(522.3,289.1)
\lineto(531.3,289.1)
\moveto(543.9,309.6)
\lineto(534.9,309.6)
}
}
{
\newrgbcolor{curcolor}{0 0 0}
\pscustom[linestyle=none,fillstyle=solid,fillcolor=curcolor]
{
\newpath
\moveto(556.67070313,305.7)
\lineto(555.61601563,305.7)
\lineto(555.61601563,312.42070312)
\curveto(555.36210938,312.17851562)(555.028125,311.93632812)(554.6140625,311.69414062)
\curveto(554.20390625,311.45195312)(553.83476563,311.2703125)(553.50664063,311.14921875)
\lineto(553.50664063,312.16875)
\curveto(554.09648438,312.44609375)(554.61210938,312.78203125)(555.05351563,313.1765625)
\curveto(555.49492188,313.57109375)(555.80742188,313.95390625)(555.99101563,314.325)
\lineto(556.67070313,314.325)
\closepath
}
}
{
\newrgbcolor{curcolor}{0 0 0}
\pscustom[linestyle=none,fillstyle=solid,fillcolor=curcolor]
{
\newpath
\moveto(559.371875,309.93632812)
\curveto(559.371875,310.95195312)(559.47539063,311.76835937)(559.68242188,312.38554687)
\curveto(559.89335938,313.00664062)(560.20390625,313.48515625)(560.6140625,313.82109375)
\curveto(561.028125,314.15703125)(561.54765625,314.325)(562.17265625,314.325)
\curveto(562.63359375,314.325)(563.03789063,314.23125)(563.38554688,314.04375)
\curveto(563.73320313,313.86015625)(564.0203125,313.59257812)(564.246875,313.24101562)
\curveto(564.4734375,312.89335937)(564.65117188,312.46757812)(564.78007813,311.96367187)
\curveto(564.90898438,311.46367187)(564.9734375,310.78789062)(564.9734375,309.93632812)
\curveto(564.9734375,308.92851562)(564.86992188,308.1140625)(564.66289063,307.49296875)
\curveto(564.45585938,306.87578125)(564.1453125,306.39726562)(563.73125,306.05742187)
\curveto(563.32109375,305.72148437)(562.8015625,305.55351562)(562.17265625,305.55351562)
\curveto(561.34453125,305.55351562)(560.69414063,305.85039062)(560.22148438,306.44414062)
\curveto(559.65507813,307.15898437)(559.371875,308.32304687)(559.371875,309.93632812)
\closepath
\moveto(560.45585938,309.93632812)
\curveto(560.45585938,308.52617187)(560.61992188,307.58671875)(560.94804688,307.11796875)
\curveto(561.28007813,306.653125)(561.68828125,306.42070312)(562.17265625,306.42070312)
\curveto(562.65703125,306.42070312)(563.06328125,306.65507812)(563.39140625,307.12382812)
\curveto(563.7234375,307.59257812)(563.88945313,308.53007812)(563.88945313,309.93632812)
\curveto(563.88945313,311.35039062)(563.7234375,312.28984375)(563.39140625,312.7546875)
\curveto(563.06328125,313.21953125)(562.653125,313.45195312)(562.1609375,313.45195312)
\curveto(561.6765625,313.45195312)(561.28984375,313.246875)(561.00078125,312.83671875)
\curveto(560.6375,312.31328125)(560.45585938,311.34648437)(560.45585938,309.93632812)
\closepath
}
}
{
\newrgbcolor{curcolor}{0 0 0}
\pscustom[linewidth=1,linecolor=curcolor]
{
\newpath
\moveto(522.3,309.6)
\lineto(531.3,309.6)
\moveto(543.9,330.1)
\lineto(534.9,330.1)
}
}
{
\newrgbcolor{curcolor}{0 0 0}
\pscustom[linestyle=none,fillstyle=solid,fillcolor=curcolor]
{
\newpath
\moveto(556.67070313,326.2)
\lineto(555.61601563,326.2)
\lineto(555.61601563,332.92070312)
\curveto(555.36210938,332.67851562)(555.028125,332.43632812)(554.6140625,332.19414062)
\curveto(554.20390625,331.95195312)(553.83476563,331.7703125)(553.50664063,331.64921875)
\lineto(553.50664063,332.66875)
\curveto(554.09648438,332.94609375)(554.61210938,333.28203125)(555.05351563,333.6765625)
\curveto(555.49492188,334.07109375)(555.80742188,334.45390625)(555.99101563,334.825)
\lineto(556.67070313,334.825)
\closepath
}
}
{
\newrgbcolor{curcolor}{0 0 0}
\pscustom[linestyle=none,fillstyle=solid,fillcolor=curcolor]
{
\newpath
\moveto(559.371875,328.45)
\lineto(560.47929688,328.54375)
\curveto(560.56132813,328.0046875)(560.75078125,327.5984375)(561.04765625,327.325)
\curveto(561.3484375,327.05546875)(561.70976563,326.92070312)(562.13164063,326.92070312)
\curveto(562.63945313,326.92070312)(563.06914063,327.11210937)(563.42070313,327.49492187)
\curveto(563.77226563,327.87773437)(563.94804688,328.38554687)(563.94804688,329.01835937)
\curveto(563.94804688,329.61992187)(563.778125,330.09453125)(563.43828125,330.4421875)
\curveto(563.10234375,330.78984375)(562.6609375,330.96367187)(562.1140625,330.96367187)
\curveto(561.77421875,330.96367187)(561.46757813,330.88554687)(561.19414063,330.72929687)
\curveto(560.92070313,330.57695312)(560.70585938,330.37773437)(560.54960938,330.13164062)
\lineto(559.559375,330.26054687)
\lineto(560.39140625,334.67265625)
\lineto(564.66289063,334.67265625)
\lineto(564.66289063,333.66484375)
\lineto(561.23515625,333.66484375)
\lineto(560.77226563,331.35625)
\curveto(561.28789063,331.715625)(561.82890625,331.8953125)(562.3953125,331.8953125)
\curveto(563.1453125,331.8953125)(563.778125,331.63554687)(564.29375,331.11601562)
\curveto(564.809375,330.59648437)(565.0671875,329.92851562)(565.0671875,329.11210937)
\curveto(565.0671875,328.33476562)(564.840625,327.66289062)(564.3875,327.09648437)
\curveto(563.83671875,326.40117187)(563.08476563,326.05351562)(562.13164063,326.05351562)
\curveto(561.35039063,326.05351562)(560.71171875,326.27226562)(560.215625,326.70976562)
\curveto(559.7234375,327.14726562)(559.4421875,327.72734375)(559.371875,328.45)
\closepath
}
}
{
\newrgbcolor{curcolor}{0 0 0}
\pscustom[linewidth=1,linecolor=curcolor]
{
\newpath
\moveto(522.3,330.1)
\lineto(531.3,330.1)
\moveto(543.9,350.6)
\lineto(534.9,350.6)
}
}
{
\newrgbcolor{curcolor}{0 0 0}
\pscustom[linestyle=none,fillstyle=solid,fillcolor=curcolor]
{
\newpath
\moveto(558.24101563,347.71367187)
\lineto(558.24101563,346.7)
\lineto(552.56328125,346.7)
\curveto(552.55546875,346.95390625)(552.59648438,347.19804687)(552.68632813,347.43242187)
\curveto(552.83085938,347.81914062)(553.06132813,348.2)(553.37773438,348.575)
\curveto(553.69804688,348.95)(554.15898438,349.38359375)(554.76054688,349.87578125)
\curveto(555.69414063,350.64140625)(556.325,351.246875)(556.653125,351.6921875)
\curveto(556.98125,352.14140625)(557.1453125,352.56523437)(557.1453125,352.96367187)
\curveto(557.1453125,353.38164062)(556.99492188,353.73320312)(556.69414063,354.01835937)
\curveto(556.39726563,354.30742187)(556.00859375,354.45195312)(555.528125,354.45195312)
\curveto(555.0203125,354.45195312)(554.6140625,354.29960937)(554.309375,353.99492187)
\curveto(554.0046875,353.69023437)(553.85039063,353.26835937)(553.84648438,352.72929687)
\lineto(552.7625,352.840625)
\curveto(552.83671875,353.64921875)(553.11601563,354.26445312)(553.60039063,354.68632812)
\curveto(554.08476563,355.11210937)(554.73515625,355.325)(555.5515625,355.325)
\curveto(556.37578125,355.325)(557.028125,355.09648437)(557.50859375,354.63945312)
\curveto(557.9890625,354.18242187)(558.22929688,353.61601562)(558.22929688,352.94023437)
\curveto(558.22929688,352.59648437)(558.15898438,352.25859375)(558.01835938,351.9265625)
\curveto(557.87773438,351.59453125)(557.64335938,351.24492187)(557.31523438,350.87773437)
\curveto(556.99101563,350.51054687)(556.45,350.00664062)(555.6921875,349.36601562)
\curveto(555.059375,348.83476562)(554.653125,348.4734375)(554.4734375,348.28203125)
\curveto(554.29375,348.09453125)(554.1453125,347.90507812)(554.028125,347.71367187)
\closepath
}
}
{
\newrgbcolor{curcolor}{0 0 0}
\pscustom[linestyle=none,fillstyle=solid,fillcolor=curcolor]
{
\newpath
\moveto(559.371875,350.93632812)
\curveto(559.371875,351.95195312)(559.47539063,352.76835937)(559.68242188,353.38554687)
\curveto(559.89335938,354.00664062)(560.20390625,354.48515625)(560.6140625,354.82109375)
\curveto(561.028125,355.15703125)(561.54765625,355.325)(562.17265625,355.325)
\curveto(562.63359375,355.325)(563.03789063,355.23125)(563.38554688,355.04375)
\curveto(563.73320313,354.86015625)(564.0203125,354.59257812)(564.246875,354.24101562)
\curveto(564.4734375,353.89335937)(564.65117188,353.46757812)(564.78007813,352.96367187)
\curveto(564.90898438,352.46367187)(564.9734375,351.78789062)(564.9734375,350.93632812)
\curveto(564.9734375,349.92851562)(564.86992188,349.1140625)(564.66289063,348.49296875)
\curveto(564.45585938,347.87578125)(564.1453125,347.39726562)(563.73125,347.05742187)
\curveto(563.32109375,346.72148437)(562.8015625,346.55351562)(562.17265625,346.55351562)
\curveto(561.34453125,346.55351562)(560.69414063,346.85039062)(560.22148438,347.44414062)
\curveto(559.65507813,348.15898437)(559.371875,349.32304687)(559.371875,350.93632812)
\closepath
\moveto(560.45585938,350.93632812)
\curveto(560.45585938,349.52617187)(560.61992188,348.58671875)(560.94804688,348.11796875)
\curveto(561.28007813,347.653125)(561.68828125,347.42070312)(562.17265625,347.42070312)
\curveto(562.65703125,347.42070312)(563.06328125,347.65507812)(563.39140625,348.12382812)
\curveto(563.7234375,348.59257812)(563.88945313,349.53007812)(563.88945313,350.93632812)
\curveto(563.88945313,352.35039062)(563.7234375,353.28984375)(563.39140625,353.7546875)
\curveto(563.06328125,354.21953125)(562.653125,354.45195312)(562.1609375,354.45195312)
\curveto(561.6765625,354.45195312)(561.28984375,354.246875)(561.00078125,353.83671875)
\curveto(560.6375,353.31328125)(560.45585938,352.34648437)(560.45585938,350.93632812)
\closepath
}
}
{
\newrgbcolor{curcolor}{0 0 0}
\pscustom[linewidth=1,linecolor=curcolor]
{
\newpath
\moveto(522.3,350.6)
\lineto(531.3,350.6)
}
}
\end{pspicture}
}
    \captionsetup{width=0.75\linewidth}
    \caption{The Effect of Process Size on the Percentage Improvement in Runtime.
        Measuring using the simple read program sized to various process sizes, with a read size of 1M.}
    \label{fig:ProcessSize}
\end{figure}

Reductions in runtime improvement are immediately visible with larger processes, and by a much larger reduction factor than I expected 
(Figure \ref{fig:ProcessSize}).
At 1G in size, runtime improvement is reduced to 2-5\% for files larger than 1 GB
and no improvement is visible with a 1 GB file and a process of the same size.
Regardless of where it runs, a program that has 1 GB in memory and reads a 1 GB file stored on another NUMA node
must still move 1 GB of data across the interconnect regardless of whether Task Migration is enabled.
This aspect of the model was correct, but the prediction for smaller processes was not.
Smaller process sizes do slowly regain the performance improvement that was visible in the earlier tests,
as they get smaller, but do so much more slowly than expected, indicating that the reduction factor is quite high.
(Figure \ref{fig:smallresults}).
This performance improvement is also highly variable across test runs, resulting in a graph that is difficult
to make definitive conclusions about.
This is likely due to the loop over the large memory area that had to be added to the program in order to
ensure it would not remove it as unnecessary when compiled (Listing \ref{lst:simpleread}, line 39).

\section{Impact of Read Size}
\label{chapter:readsizeimpact}
The size of the process is not the only factor in determining how the runtime of the process will be affected by task migration.
Another important factor is the size of the individual file reads that the process uses to sequentially read through it.

\begin{figure}[H]
    \centering
    \resizebox{0.75\linewidth}{!}{%LaTeX with PSTricks extensions
%%Creator: Inkscape 1.0.2-2 (e86c870879, 2021-01-15)
%%Please note this file requires PSTricks extensions
\psset{xunit=.5pt,yunit=.5pt,runit=.5pt}
\begin{pspicture}(600,480)
{
\newrgbcolor{curcolor}{0 0 0}
\pscustom[linewidth=1,linecolor=curcolor]
{
\newpath
\moveto(85.9,123.2)
\lineto(242.7,219.7)
}
}
{
\newrgbcolor{curcolor}{0 0 0}
\pscustom[linewidth=1,linecolor=curcolor]
{
\newpath
\moveto(514.1,164)
\lineto(242.7,219.7)
}
}
{
\newrgbcolor{curcolor}{0 0 0}
\pscustom[linewidth=1,linecolor=curcolor]
{
\newpath
\moveto(85.9,123.2)
\lineto(85.9,316)
}
}
{
\newrgbcolor{curcolor}{0 0 0}
\pscustom[linewidth=1,linecolor=curcolor]
{
\newpath
\moveto(242.7,219.7)
\lineto(242.7,327.9)
}
}
{
\newrgbcolor{curcolor}{0 0 0}
\pscustom[linewidth=1,linecolor=curcolor]
{
\newpath
\moveto(514.1,164)
\lineto(514.1,304.9)
}
}
{
\newrgbcolor{curcolor}{0 0 0}
\pscustom[linewidth=1,linecolor=curcolor]
{
\newpath
\moveto(85.9,123.2)
\lineto(91.2,126.5)
}
}
{
\newrgbcolor{curcolor}{0 0 0}
\pscustom[linewidth=1,linecolor=curcolor]
{
\newpath
\moveto(242.7,219.7)
\lineto(237.3,216.4)
}
}
{
\newrgbcolor{curcolor}{0 0 0}
\pscustom[linestyle=none,fillstyle=solid,fillcolor=curcolor]
{
\newpath
\moveto(71.93222656,113.81367188)
\lineto(71.93222656,112.8)
\lineto(66.25449219,112.8)
\curveto(66.24667969,113.05390625)(66.28769531,113.29804688)(66.37753906,113.53242188)
\curveto(66.52207031,113.91914063)(66.75253906,114.3)(67.06894531,114.675)
\curveto(67.38925781,115.05)(67.85019531,115.48359375)(68.45175781,115.97578125)
\curveto(69.38535156,116.74140625)(70.01621094,117.346875)(70.34433594,117.7921875)
\curveto(70.67246094,118.24140625)(70.83652344,118.66523438)(70.83652344,119.06367188)
\curveto(70.83652344,119.48164063)(70.68613281,119.83320313)(70.38535156,120.11835938)
\curveto(70.08847656,120.40742188)(69.69980469,120.55195313)(69.21933594,120.55195313)
\curveto(68.71152344,120.55195313)(68.30527344,120.39960938)(68.00058594,120.09492188)
\curveto(67.69589844,119.79023438)(67.54160156,119.36835938)(67.53769531,118.82929688)
\lineto(66.45371094,118.940625)
\curveto(66.52792969,119.74921875)(66.80722656,120.36445313)(67.29160156,120.78632813)
\curveto(67.77597656,121.21210938)(68.42636719,121.425)(69.24277344,121.425)
\curveto(70.06699219,121.425)(70.71933594,121.19648438)(71.19980469,120.73945313)
\curveto(71.68027344,120.28242188)(71.92050781,119.71601563)(71.92050781,119.04023438)
\curveto(71.92050781,118.69648438)(71.85019531,118.35859375)(71.70957031,118.0265625)
\curveto(71.56894531,117.69453125)(71.33457031,117.34492188)(71.00644531,116.97773438)
\curveto(70.68222656,116.61054688)(70.14121094,116.10664063)(69.38339844,115.46601563)
\curveto(68.75058594,114.93476563)(68.34433594,114.5734375)(68.16464844,114.38203125)
\curveto(67.98496094,114.19453125)(67.83652344,114.00507813)(67.71933594,113.81367188)
\closepath
}
}
{
\newrgbcolor{curcolor}{0 0 0}
\pscustom[linestyle=none,fillstyle=solid,fillcolor=curcolor]
{
\newpath
\moveto(73.06308594,115.05)
\lineto(74.17050781,115.14375)
\curveto(74.25253906,114.6046875)(74.44199219,114.1984375)(74.73886719,113.925)
\curveto(75.03964844,113.65546875)(75.40097656,113.52070313)(75.82285156,113.52070313)
\curveto(76.33066406,113.52070313)(76.76035156,113.71210938)(77.11191406,114.09492188)
\curveto(77.46347656,114.47773438)(77.63925781,114.98554688)(77.63925781,115.61835938)
\curveto(77.63925781,116.21992188)(77.46933594,116.69453125)(77.12949219,117.0421875)
\curveto(76.79355469,117.38984375)(76.35214844,117.56367188)(75.80527344,117.56367188)
\curveto(75.46542969,117.56367188)(75.15878906,117.48554688)(74.88535156,117.32929688)
\curveto(74.61191406,117.17695313)(74.39707031,116.97773438)(74.24082031,116.73164063)
\lineto(73.25058594,116.86054688)
\lineto(74.08261719,121.27265625)
\lineto(78.35410156,121.27265625)
\lineto(78.35410156,120.26484375)
\lineto(74.92636719,120.26484375)
\lineto(74.46347656,117.95625)
\curveto(74.97910156,118.315625)(75.52011719,118.4953125)(76.08652344,118.4953125)
\curveto(76.83652344,118.4953125)(77.46933594,118.23554688)(77.98496094,117.71601563)
\curveto(78.50058594,117.19648438)(78.75839844,116.52851563)(78.75839844,115.71210938)
\curveto(78.75839844,114.93476563)(78.53183594,114.26289063)(78.07871094,113.69648438)
\curveto(77.52792969,113.00117188)(76.77597656,112.65351563)(75.82285156,112.65351563)
\curveto(75.04160156,112.65351563)(74.40292969,112.87226563)(73.90683594,113.30976563)
\curveto(73.41464844,113.74726563)(73.13339844,114.32734375)(73.06308594,115.05)
\closepath
}
}
{
\newrgbcolor{curcolor}{0 0 0}
\pscustom[linestyle=none,fillstyle=solid,fillcolor=curcolor]
{
\newpath
\moveto(79.73691406,117.03632813)
\curveto(79.73691406,118.05195313)(79.84042969,118.86835938)(80.04746094,119.48554688)
\curveto(80.25839844,120.10664063)(80.56894531,120.58515625)(80.97910156,120.92109375)
\curveto(81.39316406,121.25703125)(81.91269531,121.425)(82.53769531,121.425)
\curveto(82.99863281,121.425)(83.40292969,121.33125)(83.75058594,121.14375)
\curveto(84.09824219,120.96015625)(84.38535156,120.69257813)(84.61191406,120.34101563)
\curveto(84.83847656,119.99335938)(85.01621094,119.56757813)(85.14511719,119.06367188)
\curveto(85.27402344,118.56367188)(85.33847656,117.88789063)(85.33847656,117.03632813)
\curveto(85.33847656,116.02851563)(85.23496094,115.2140625)(85.02792969,114.59296875)
\curveto(84.82089844,113.97578125)(84.51035156,113.49726563)(84.09628906,113.15742188)
\curveto(83.68613281,112.82148438)(83.16660156,112.65351563)(82.53769531,112.65351563)
\curveto(81.70957031,112.65351563)(81.05917969,112.95039063)(80.58652344,113.54414063)
\curveto(80.02011719,114.25898438)(79.73691406,115.42304688)(79.73691406,117.03632813)
\closepath
\moveto(80.82089844,117.03632813)
\curveto(80.82089844,115.62617188)(80.98496094,114.68671875)(81.31308594,114.21796875)
\curveto(81.64511719,113.753125)(82.05332031,113.52070313)(82.53769531,113.52070313)
\curveto(83.02207031,113.52070313)(83.42832031,113.75507813)(83.75644531,114.22382813)
\curveto(84.08847656,114.69257813)(84.25449219,115.63007813)(84.25449219,117.03632813)
\curveto(84.25449219,118.45039063)(84.08847656,119.38984375)(83.75644531,119.8546875)
\curveto(83.42832031,120.31953125)(83.01816406,120.55195313)(82.52597656,120.55195313)
\curveto(82.04160156,120.55195313)(81.65488281,120.346875)(81.36582031,119.93671875)
\curveto(81.00253906,119.41328125)(80.82089844,118.44648438)(80.82089844,117.03632813)
\closepath
}
}
{
\newrgbcolor{curcolor}{0 0 0}
\pscustom[linestyle=none,fillstyle=solid,fillcolor=curcolor]
{
\newpath
\moveto(86.80332031,112.8)
\lineto(86.80332031,121.38984375)
\lineto(88.51425781,121.38984375)
\lineto(90.54746094,115.3078125)
\curveto(90.73496094,114.74140625)(90.87167969,114.31757813)(90.95761719,114.03632813)
\curveto(91.05527344,114.34882813)(91.20761719,114.8078125)(91.41464844,115.41328125)
\lineto(93.47128906,121.38984375)
\lineto(95.00058594,121.38984375)
\lineto(95.00058594,112.8)
\lineto(93.90488281,112.8)
\lineto(93.90488281,119.98945313)
\lineto(91.40878906,112.8)
\lineto(90.38339844,112.8)
\lineto(87.89902344,120.1125)
\lineto(87.89902344,112.8)
\closepath
}
}
{
\newrgbcolor{curcolor}{0 0 0}
\pscustom[linewidth=1,linecolor=curcolor]
{
\newpath
\moveto(131.1,114)
\lineto(136.5,117.2)
}
}
{
\newrgbcolor{curcolor}{0 0 0}
\pscustom[linewidth=1,linecolor=curcolor]
{
\newpath
\moveto(287.9,210.4)
\lineto(282.6,207.1)
}
}
{
\newrgbcolor{curcolor}{0 0 0}
\pscustom[linestyle=none,fillstyle=solid,fillcolor=curcolor]
{
\newpath
\moveto(111.68925781,105.75)
\lineto(112.79667969,105.84375)
\curveto(112.87871094,105.3046875)(113.06816406,104.8984375)(113.36503906,104.625)
\curveto(113.66582031,104.35546875)(114.02714844,104.22070312)(114.44902344,104.22070312)
\curveto(114.95683594,104.22070312)(115.38652344,104.41210938)(115.73808594,104.79492188)
\curveto(116.08964844,105.17773438)(116.26542969,105.68554688)(116.26542969,106.31835938)
\curveto(116.26542969,106.91992188)(116.09550781,107.39453125)(115.75566406,107.7421875)
\curveto(115.41972656,108.08984375)(114.97832031,108.26367188)(114.43144531,108.26367188)
\curveto(114.09160156,108.26367188)(113.78496094,108.18554688)(113.51152344,108.02929688)
\curveto(113.23808594,107.87695312)(113.02324219,107.67773438)(112.86699219,107.43164062)
\lineto(111.87675781,107.56054688)
\lineto(112.70878906,111.97265625)
\lineto(116.98027344,111.97265625)
\lineto(116.98027344,110.96484375)
\lineto(113.55253906,110.96484375)
\lineto(113.08964844,108.65625)
\curveto(113.60527344,109.015625)(114.14628906,109.1953125)(114.71269531,109.1953125)
\curveto(115.46269531,109.1953125)(116.09550781,108.93554688)(116.61113281,108.41601562)
\curveto(117.12675781,107.89648438)(117.38457031,107.22851562)(117.38457031,106.41210938)
\curveto(117.38457031,105.63476562)(117.15800781,104.96289062)(116.70488281,104.39648438)
\curveto(116.15410156,103.70117188)(115.40214844,103.35351562)(114.44902344,103.35351562)
\curveto(113.66777344,103.35351562)(113.02910156,103.57226562)(112.53300781,104.00976562)
\curveto(112.04082031,104.44726562)(111.75957031,105.02734375)(111.68925781,105.75)
\closepath
}
}
{
\newrgbcolor{curcolor}{0 0 0}
\pscustom[linestyle=none,fillstyle=solid,fillcolor=curcolor]
{
\newpath
\moveto(118.36308594,107.73632812)
\curveto(118.36308594,108.75195312)(118.46660156,109.56835938)(118.67363281,110.18554688)
\curveto(118.88457031,110.80664062)(119.19511719,111.28515625)(119.60527344,111.62109375)
\curveto(120.01933594,111.95703125)(120.53886719,112.125)(121.16386719,112.125)
\curveto(121.62480469,112.125)(122.02910156,112.03125)(122.37675781,111.84375)
\curveto(122.72441406,111.66015625)(123.01152344,111.39257812)(123.23808594,111.04101562)
\curveto(123.46464844,110.69335938)(123.64238281,110.26757812)(123.77128906,109.76367188)
\curveto(123.90019531,109.26367188)(123.96464844,108.58789062)(123.96464844,107.73632812)
\curveto(123.96464844,106.72851562)(123.86113281,105.9140625)(123.65410156,105.29296875)
\curveto(123.44707031,104.67578125)(123.13652344,104.19726562)(122.72246094,103.85742188)
\curveto(122.31230469,103.52148438)(121.79277344,103.35351562)(121.16386719,103.35351562)
\curveto(120.33574219,103.35351562)(119.68535156,103.65039062)(119.21269531,104.24414062)
\curveto(118.64628906,104.95898438)(118.36308594,106.12304688)(118.36308594,107.73632812)
\closepath
\moveto(119.44707031,107.73632812)
\curveto(119.44707031,106.32617188)(119.61113281,105.38671875)(119.93925781,104.91796875)
\curveto(120.27128906,104.453125)(120.67949219,104.22070312)(121.16386719,104.22070312)
\curveto(121.64824219,104.22070312)(122.05449219,104.45507812)(122.38261719,104.92382812)
\curveto(122.71464844,105.39257812)(122.88066406,106.33007812)(122.88066406,107.73632812)
\curveto(122.88066406,109.15039062)(122.71464844,110.08984375)(122.38261719,110.5546875)
\curveto(122.05449219,111.01953125)(121.64433594,111.25195312)(121.15214844,111.25195312)
\curveto(120.66777344,111.25195312)(120.28105469,111.046875)(119.99199219,110.63671875)
\curveto(119.62871094,110.11328125)(119.44707031,109.14648438)(119.44707031,107.73632812)
\closepath
}
}
{
\newrgbcolor{curcolor}{0 0 0}
\pscustom[linestyle=none,fillstyle=solid,fillcolor=curcolor]
{
\newpath
\moveto(125.03691406,107.73632812)
\curveto(125.03691406,108.75195312)(125.14042969,109.56835938)(125.34746094,110.18554688)
\curveto(125.55839844,110.80664062)(125.86894531,111.28515625)(126.27910156,111.62109375)
\curveto(126.69316406,111.95703125)(127.21269531,112.125)(127.83769531,112.125)
\curveto(128.29863281,112.125)(128.70292969,112.03125)(129.05058594,111.84375)
\curveto(129.39824219,111.66015625)(129.68535156,111.39257812)(129.91191406,111.04101562)
\curveto(130.13847656,110.69335938)(130.31621094,110.26757812)(130.44511719,109.76367188)
\curveto(130.57402344,109.26367188)(130.63847656,108.58789062)(130.63847656,107.73632812)
\curveto(130.63847656,106.72851562)(130.53496094,105.9140625)(130.32792969,105.29296875)
\curveto(130.12089844,104.67578125)(129.81035156,104.19726562)(129.39628906,103.85742188)
\curveto(128.98613281,103.52148438)(128.46660156,103.35351562)(127.83769531,103.35351562)
\curveto(127.00957031,103.35351562)(126.35917969,103.65039062)(125.88652344,104.24414062)
\curveto(125.32011719,104.95898438)(125.03691406,106.12304688)(125.03691406,107.73632812)
\closepath
\moveto(126.12089844,107.73632812)
\curveto(126.12089844,106.32617188)(126.28496094,105.38671875)(126.61308594,104.91796875)
\curveto(126.94511719,104.453125)(127.35332031,104.22070312)(127.83769531,104.22070312)
\curveto(128.32207031,104.22070312)(128.72832031,104.45507812)(129.05644531,104.92382812)
\curveto(129.38847656,105.39257812)(129.55449219,106.33007812)(129.55449219,107.73632812)
\curveto(129.55449219,109.15039062)(129.38847656,110.08984375)(129.05644531,110.5546875)
\curveto(128.72832031,111.01953125)(128.31816406,111.25195312)(127.82597656,111.25195312)
\curveto(127.34160156,111.25195312)(126.95488281,111.046875)(126.66582031,110.63671875)
\curveto(126.30253906,110.11328125)(126.12089844,109.14648438)(126.12089844,107.73632812)
\closepath
}
}
{
\newrgbcolor{curcolor}{0 0 0}
\pscustom[linestyle=none,fillstyle=solid,fillcolor=curcolor]
{
\newpath
\moveto(132.10332031,103.5)
\lineto(132.10332031,112.08984375)
\lineto(133.81425781,112.08984375)
\lineto(135.84746094,106.0078125)
\curveto(136.03496094,105.44140625)(136.17167969,105.01757812)(136.25761719,104.73632812)
\curveto(136.35527344,105.04882812)(136.50761719,105.5078125)(136.71464844,106.11328125)
\lineto(138.77128906,112.08984375)
\lineto(140.30058594,112.08984375)
\lineto(140.30058594,103.5)
\lineto(139.20488281,103.5)
\lineto(139.20488281,110.68945312)
\lineto(136.70878906,103.5)
\lineto(135.68339844,103.5)
\lineto(133.19902344,110.8125)
\lineto(133.19902344,103.5)
\closepath
}
}
{
\newrgbcolor{curcolor}{0 0 0}
\pscustom[linewidth=1,linecolor=curcolor]
{
\newpath
\moveto(176.4,104.7)
\lineto(181.7,108)
}
}
{
\newrgbcolor{curcolor}{0 0 0}
\pscustom[linewidth=1,linecolor=curcolor]
{
\newpath
\moveto(333.1,201.1)
\lineto(327.7,197.8)
}
}
{
\newrgbcolor{curcolor}{0 0 0}
\pscustom[linestyle=none,fillstyle=solid,fillcolor=curcolor]
{
\newpath
\moveto(167.96679688,94.2)
\lineto(166.91210938,94.2)
\lineto(166.91210938,100.92070312)
\curveto(166.65820312,100.67851562)(166.32421875,100.43632812)(165.91015625,100.19414062)
\curveto(165.5,99.95195312)(165.13085938,99.7703125)(164.80273438,99.64921875)
\lineto(164.80273438,100.66875)
\curveto(165.39257812,100.94609375)(165.90820312,101.28203125)(166.34960938,101.6765625)
\curveto(166.79101562,102.07109375)(167.10351562,102.45390625)(167.28710938,102.825)
\lineto(167.96679688,102.825)
\closepath
}
}
{
\newrgbcolor{curcolor}{0 0 0}
\pscustom[linestyle=none,fillstyle=solid,fillcolor=curcolor]
{
\newpath
\moveto(175.11523438,97.56914062)
\lineto(175.11523438,98.57695312)
\lineto(178.75390625,98.5828125)
\lineto(178.75390625,95.3953125)
\curveto(178.1953125,94.95)(177.61914062,94.6140625)(177.02539062,94.3875)
\curveto(176.43164062,94.16484375)(175.82226562,94.05351562)(175.19726562,94.05351562)
\curveto(174.35351562,94.05351562)(173.5859375,94.23320312)(172.89453125,94.59257812)
\curveto(172.20703125,94.95585937)(171.6875,95.47929687)(171.3359375,96.16289062)
\curveto(170.984375,96.84648437)(170.80859375,97.61015625)(170.80859375,98.45390625)
\curveto(170.80859375,99.28984375)(170.98242188,100.06914062)(171.33007812,100.79179687)
\curveto(171.68164062,101.51835937)(172.18554688,102.05742187)(172.84179688,102.40898437)
\curveto(173.49804688,102.76054687)(174.25390625,102.93632812)(175.109375,102.93632812)
\curveto(175.73046875,102.93632812)(176.29101562,102.83476562)(176.79101562,102.63164062)
\curveto(177.29492188,102.43242187)(177.68945312,102.153125)(177.97460938,101.79375)
\curveto(178.25976562,101.434375)(178.4765625,100.965625)(178.625,100.3875)
\lineto(177.59960938,100.10625)
\curveto(177.47070312,100.54375)(177.31054688,100.8875)(177.11914062,101.1375)
\curveto(176.92773438,101.3875)(176.65429688,101.58671875)(176.29882812,101.73515625)
\curveto(175.94335938,101.8875)(175.54882812,101.96367187)(175.11523438,101.96367187)
\curveto(174.59570312,101.96367187)(174.14648438,101.88359375)(173.76757812,101.7234375)
\curveto(173.38867188,101.5671875)(173.08203125,101.36015625)(172.84765625,101.10234375)
\curveto(172.6171875,100.84453125)(172.4375,100.56132812)(172.30859375,100.25273437)
\curveto(172.08984375,99.72148437)(171.98046875,99.1453125)(171.98046875,98.52421875)
\curveto(171.98046875,97.75859375)(172.11132812,97.11796875)(172.37304688,96.60234375)
\curveto(172.63867188,96.08671875)(173.0234375,95.70390625)(173.52734375,95.45390625)
\curveto(174.03125,95.20390625)(174.56640625,95.07890625)(175.1328125,95.07890625)
\curveto(175.625,95.07890625)(176.10546875,95.17265625)(176.57421875,95.36015625)
\curveto(177.04296875,95.5515625)(177.3984375,95.7546875)(177.640625,95.96953125)
\lineto(177.640625,97.56914062)
\closepath
}
}
{
\newrgbcolor{curcolor}{0 0 0}
\pscustom[linewidth=1,linecolor=curcolor]
{
\newpath
\moveto(221.7,95.4)
\lineto(227,98.7)
}
}
{
\newrgbcolor{curcolor}{0 0 0}
\pscustom[linewidth=1,linecolor=curcolor]
{
\newpath
\moveto(378.3,191.8)
\lineto(373,188.5)
}
}
{
\newrgbcolor{curcolor}{0 0 0}
\pscustom[linestyle=none,fillstyle=solid,fillcolor=curcolor]
{
\newpath
\moveto(214.73710937,85.91367187)
\lineto(214.73710937,84.9)
\lineto(209.059375,84.9)
\curveto(209.0515625,85.15390625)(209.09257812,85.39804687)(209.18242187,85.63242187)
\curveto(209.32695312,86.01914062)(209.55742187,86.4)(209.87382812,86.775)
\curveto(210.19414062,87.15)(210.65507812,87.58359375)(211.25664062,88.07578125)
\curveto(212.19023437,88.84140625)(212.82109375,89.446875)(213.14921875,89.8921875)
\curveto(213.47734375,90.34140625)(213.64140625,90.76523437)(213.64140625,91.16367187)
\curveto(213.64140625,91.58164062)(213.49101562,91.93320312)(213.19023437,92.21835937)
\curveto(212.89335937,92.50742187)(212.5046875,92.65195312)(212.02421875,92.65195312)
\curveto(211.51640625,92.65195312)(211.11015625,92.49960937)(210.80546875,92.19492187)
\curveto(210.50078125,91.89023437)(210.34648437,91.46835937)(210.34257812,90.92929687)
\lineto(209.25859375,91.040625)
\curveto(209.3328125,91.84921875)(209.61210937,92.46445312)(210.09648437,92.88632812)
\curveto(210.58085937,93.31210937)(211.23125,93.525)(212.04765625,93.525)
\curveto(212.871875,93.525)(213.52421875,93.29648437)(214.0046875,92.83945312)
\curveto(214.48515625,92.38242187)(214.72539062,91.81601562)(214.72539062,91.14023437)
\curveto(214.72539062,90.79648437)(214.65507812,90.45859375)(214.51445312,90.1265625)
\curveto(214.37382812,89.79453125)(214.13945312,89.44492187)(213.81132812,89.07773437)
\curveto(213.48710937,88.71054687)(212.94609375,88.20664062)(212.18828125,87.56601562)
\curveto(211.55546875,87.03476562)(211.14921875,86.6734375)(210.96953125,86.48203125)
\curveto(210.78984375,86.29453125)(210.64140625,86.10507812)(210.52421875,85.91367187)
\closepath
}
}
{
\newrgbcolor{curcolor}{0 0 0}
\pscustom[linestyle=none,fillstyle=solid,fillcolor=curcolor]
{
\newpath
\moveto(220.31523437,88.26914062)
\lineto(220.31523437,89.27695312)
\lineto(223.95390625,89.2828125)
\lineto(223.95390625,86.0953125)
\curveto(223.3953125,85.65)(222.81914062,85.3140625)(222.22539062,85.0875)
\curveto(221.63164062,84.86484375)(221.02226562,84.75351562)(220.39726562,84.75351562)
\curveto(219.55351562,84.75351562)(218.7859375,84.93320312)(218.09453125,85.29257812)
\curveto(217.40703125,85.65585937)(216.8875,86.17929687)(216.5359375,86.86289062)
\curveto(216.184375,87.54648437)(216.00859375,88.31015625)(216.00859375,89.15390625)
\curveto(216.00859375,89.98984375)(216.18242187,90.76914062)(216.53007812,91.49179687)
\curveto(216.88164062,92.21835937)(217.38554687,92.75742187)(218.04179687,93.10898437)
\curveto(218.69804687,93.46054687)(219.45390625,93.63632812)(220.309375,93.63632812)
\curveto(220.93046875,93.63632812)(221.49101562,93.53476562)(221.99101562,93.33164062)
\curveto(222.49492187,93.13242187)(222.88945312,92.853125)(223.17460937,92.49375)
\curveto(223.45976562,92.134375)(223.6765625,91.665625)(223.825,91.0875)
\lineto(222.79960937,90.80625)
\curveto(222.67070312,91.24375)(222.51054687,91.5875)(222.31914062,91.8375)
\curveto(222.12773437,92.0875)(221.85429687,92.28671875)(221.49882812,92.43515625)
\curveto(221.14335937,92.5875)(220.74882812,92.66367187)(220.31523437,92.66367187)
\curveto(219.79570312,92.66367187)(219.34648437,92.58359375)(218.96757812,92.4234375)
\curveto(218.58867187,92.2671875)(218.28203125,92.06015625)(218.04765625,91.80234375)
\curveto(217.8171875,91.54453125)(217.6375,91.26132812)(217.50859375,90.95273437)
\curveto(217.28984375,90.42148437)(217.18046875,89.8453125)(217.18046875,89.22421875)
\curveto(217.18046875,88.45859375)(217.31132812,87.81796875)(217.57304687,87.30234375)
\curveto(217.83867187,86.78671875)(218.2234375,86.40390625)(218.72734375,86.15390625)
\curveto(219.23125,85.90390625)(219.76640625,85.77890625)(220.3328125,85.77890625)
\curveto(220.825,85.77890625)(221.30546875,85.87265625)(221.77421875,86.06015625)
\curveto(222.24296875,86.2515625)(222.5984375,86.4546875)(222.840625,86.66953125)
\lineto(222.840625,88.26914062)
\closepath
}
}
{
\newrgbcolor{curcolor}{0 0 0}
\pscustom[linewidth=1,linecolor=curcolor]
{
\newpath
\moveto(266.9,86.1)
\lineto(272.3,89.4)
}
}
{
\newrgbcolor{curcolor}{0 0 0}
\pscustom[linewidth=1,linecolor=curcolor]
{
\newpath
\moveto(423.6,182.6)
\lineto(418.3,179.3)
}
}
{
\newrgbcolor{curcolor}{0 0 0}
\pscustom[linestyle=none,fillstyle=solid,fillcolor=curcolor]
{
\newpath
\moveto(257.875,75.7)
\lineto(257.875,77.75664062)
\lineto(254.1484375,77.75664062)
\lineto(254.1484375,78.7234375)
\lineto(258.06835938,84.28984375)
\lineto(258.9296875,84.28984375)
\lineto(258.9296875,78.7234375)
\lineto(260.08984375,78.7234375)
\lineto(260.08984375,77.75664062)
\lineto(258.9296875,77.75664062)
\lineto(258.9296875,75.7)
\closepath
\moveto(257.875,78.7234375)
\lineto(257.875,82.59648437)
\lineto(255.18554688,78.7234375)
\closepath
}
}
{
\newrgbcolor{curcolor}{0 0 0}
\pscustom[linestyle=none,fillstyle=solid,fillcolor=curcolor]
{
\newpath
\moveto(265.61523438,79.06914062)
\lineto(265.61523438,80.07695312)
\lineto(269.25390625,80.0828125)
\lineto(269.25390625,76.8953125)
\curveto(268.6953125,76.45)(268.11914062,76.1140625)(267.52539062,75.8875)
\curveto(266.93164062,75.66484375)(266.32226562,75.55351562)(265.69726562,75.55351562)
\curveto(264.85351562,75.55351562)(264.0859375,75.73320312)(263.39453125,76.09257812)
\curveto(262.70703125,76.45585937)(262.1875,76.97929687)(261.8359375,77.66289062)
\curveto(261.484375,78.34648437)(261.30859375,79.11015625)(261.30859375,79.95390625)
\curveto(261.30859375,80.78984375)(261.48242188,81.56914062)(261.83007812,82.29179687)
\curveto(262.18164062,83.01835937)(262.68554688,83.55742187)(263.34179688,83.90898437)
\curveto(263.99804688,84.26054687)(264.75390625,84.43632812)(265.609375,84.43632812)
\curveto(266.23046875,84.43632812)(266.79101562,84.33476562)(267.29101562,84.13164062)
\curveto(267.79492188,83.93242187)(268.18945312,83.653125)(268.47460938,83.29375)
\curveto(268.75976562,82.934375)(268.9765625,82.465625)(269.125,81.8875)
\lineto(268.09960938,81.60625)
\curveto(267.97070312,82.04375)(267.81054688,82.3875)(267.61914062,82.6375)
\curveto(267.42773438,82.8875)(267.15429688,83.08671875)(266.79882812,83.23515625)
\curveto(266.44335938,83.3875)(266.04882812,83.46367187)(265.61523438,83.46367187)
\curveto(265.09570312,83.46367187)(264.64648438,83.38359375)(264.26757812,83.2234375)
\curveto(263.88867188,83.0671875)(263.58203125,82.86015625)(263.34765625,82.60234375)
\curveto(263.1171875,82.34453125)(262.9375,82.06132812)(262.80859375,81.75273437)
\curveto(262.58984375,81.22148437)(262.48046875,80.6453125)(262.48046875,80.02421875)
\curveto(262.48046875,79.25859375)(262.61132812,78.61796875)(262.87304688,78.10234375)
\curveto(263.13867188,77.58671875)(263.5234375,77.20390625)(264.02734375,76.95390625)
\curveto(264.53125,76.70390625)(265.06640625,76.57890625)(265.6328125,76.57890625)
\curveto(266.125,76.57890625)(266.60546875,76.67265625)(267.07421875,76.86015625)
\curveto(267.54296875,77.0515625)(267.8984375,77.2546875)(268.140625,77.46953125)
\lineto(268.140625,79.06914062)
\closepath
}
}
{
\newrgbcolor{curcolor}{0 0 0}
\pscustom[linewidth=1,linecolor=curcolor]
{
\newpath
\moveto(312.1,76.8)
\lineto(317.4,80.1)
}
}
{
\newrgbcolor{curcolor}{0 0 0}
\pscustom[linewidth=1,linecolor=curcolor]
{
\newpath
\moveto(468.9,173.3)
\lineto(463.5,170)
}
}
{
\newrgbcolor{curcolor}{0 0 0}
\pscustom[linestyle=none,fillstyle=solid,fillcolor=curcolor]
{
\newpath
\moveto(301.3171875,71.05820312)
\curveto(300.8796875,71.21835937)(300.55546875,71.446875)(300.34453125,71.74375)
\curveto(300.13359375,72.040625)(300.028125,72.39609375)(300.028125,72.81015625)
\curveto(300.028125,73.43515625)(300.25273437,73.96054687)(300.70195312,74.38632812)
\curveto(301.15117187,74.81210937)(301.74882812,75.025)(302.49492187,75.025)
\curveto(303.24492187,75.025)(303.8484375,74.80625)(304.30546875,74.36875)
\curveto(304.7625,73.93515625)(304.99101562,73.40585937)(304.99101562,72.78085937)
\curveto(304.99101562,72.38242187)(304.88554687,72.03476562)(304.67460937,71.73789062)
\curveto(304.46757812,71.44492187)(304.15117187,71.21835937)(303.72539062,71.05820312)
\curveto(304.25273437,70.88632812)(304.653125,70.60898437)(304.9265625,70.22617187)
\curveto(305.20390625,69.84335937)(305.34257812,69.38632812)(305.34257812,68.85507812)
\curveto(305.34257812,68.12070312)(305.0828125,67.50351562)(304.56328125,67.00351562)
\curveto(304.04375,66.50351562)(303.36015625,66.25351562)(302.5125,66.25351562)
\curveto(301.66484375,66.25351562)(300.98125,66.50351562)(300.46171875,67.00351562)
\curveto(299.9421875,67.50742187)(299.68242187,68.134375)(299.68242187,68.884375)
\curveto(299.68242187,69.44296875)(299.82304687,69.90976562)(300.10429687,70.28476562)
\curveto(300.38945312,70.66367187)(300.79375,70.92148437)(301.3171875,71.05820312)
\closepath
\moveto(301.10625,72.8453125)
\curveto(301.10625,72.4390625)(301.23710937,72.10703125)(301.49882812,71.84921875)
\curveto(301.76054687,71.59140625)(302.10039062,71.4625)(302.51835937,71.4625)
\curveto(302.92460937,71.4625)(303.25664062,71.58945312)(303.51445312,71.84335937)
\curveto(303.77617187,72.10117187)(303.90703125,72.415625)(303.90703125,72.78671875)
\curveto(303.90703125,73.1734375)(303.77226562,73.49765625)(303.50273437,73.759375)
\curveto(303.23710937,74.025)(302.90507812,74.1578125)(302.50664062,74.1578125)
\curveto(302.10429687,74.1578125)(301.7703125,74.02890625)(301.5046875,73.77109375)
\curveto(301.2390625,73.51328125)(301.10625,73.2046875)(301.10625,72.8453125)
\closepath
\moveto(300.76640625,68.87851562)
\curveto(300.76640625,68.57773437)(300.83671875,68.28671875)(300.97734375,68.00546875)
\curveto(301.121875,67.72421875)(301.33476562,67.50546875)(301.61601562,67.34921875)
\curveto(301.89726562,67.196875)(302.2,67.12070312)(302.52421875,67.12070312)
\curveto(303.028125,67.12070312)(303.44414062,67.2828125)(303.77226562,67.60703125)
\curveto(304.10039062,67.93125)(304.26445312,68.34335937)(304.26445312,68.84335937)
\curveto(304.26445312,69.35117187)(304.09453125,69.77109375)(303.7546875,70.103125)
\curveto(303.41875,70.43515625)(302.996875,70.60117187)(302.4890625,70.60117187)
\curveto(301.99296875,70.60117187)(301.58085937,70.43710937)(301.25273437,70.10898437)
\curveto(300.92851562,69.78085937)(300.76640625,69.37070312)(300.76640625,68.87851562)
\closepath
}
}
{
\newrgbcolor{curcolor}{0 0 0}
\pscustom[linestyle=none,fillstyle=solid,fillcolor=curcolor]
{
\newpath
\moveto(310.81523437,69.76914062)
\lineto(310.81523437,70.77695312)
\lineto(314.45390625,70.7828125)
\lineto(314.45390625,67.5953125)
\curveto(313.8953125,67.15)(313.31914062,66.8140625)(312.72539062,66.5875)
\curveto(312.13164062,66.36484375)(311.52226562,66.25351562)(310.89726562,66.25351562)
\curveto(310.05351562,66.25351562)(309.2859375,66.43320312)(308.59453125,66.79257812)
\curveto(307.90703125,67.15585937)(307.3875,67.67929687)(307.0359375,68.36289062)
\curveto(306.684375,69.04648437)(306.50859375,69.81015625)(306.50859375,70.65390625)
\curveto(306.50859375,71.48984375)(306.68242187,72.26914062)(307.03007812,72.99179687)
\curveto(307.38164062,73.71835937)(307.88554687,74.25742187)(308.54179687,74.60898437)
\curveto(309.19804687,74.96054687)(309.95390625,75.13632812)(310.809375,75.13632812)
\curveto(311.43046875,75.13632812)(311.99101562,75.03476562)(312.49101562,74.83164062)
\curveto(312.99492187,74.63242187)(313.38945312,74.353125)(313.67460937,73.99375)
\curveto(313.95976562,73.634375)(314.1765625,73.165625)(314.325,72.5875)
\lineto(313.29960937,72.30625)
\curveto(313.17070312,72.74375)(313.01054687,73.0875)(312.81914062,73.3375)
\curveto(312.62773437,73.5875)(312.35429687,73.78671875)(311.99882812,73.93515625)
\curveto(311.64335937,74.0875)(311.24882812,74.16367187)(310.81523437,74.16367187)
\curveto(310.29570312,74.16367187)(309.84648437,74.08359375)(309.46757812,73.9234375)
\curveto(309.08867187,73.7671875)(308.78203125,73.56015625)(308.54765625,73.30234375)
\curveto(308.3171875,73.04453125)(308.1375,72.76132812)(308.00859375,72.45273437)
\curveto(307.78984375,71.92148437)(307.68046875,71.3453125)(307.68046875,70.72421875)
\curveto(307.68046875,69.95859375)(307.81132812,69.31796875)(308.07304687,68.80234375)
\curveto(308.33867187,68.28671875)(308.7234375,67.90390625)(309.22734375,67.65390625)
\curveto(309.73125,67.40390625)(310.26640625,67.27890625)(310.8328125,67.27890625)
\curveto(311.325,67.27890625)(311.80546875,67.37265625)(312.27421875,67.56015625)
\curveto(312.74296875,67.7515625)(313.0984375,67.9546875)(313.340625,68.16953125)
\lineto(313.340625,69.76914062)
\closepath
}
}
{
\newrgbcolor{curcolor}{0 0 0}
\pscustom[linewidth=1,linecolor=curcolor]
{
\newpath
\moveto(357.3,67.6)
\lineto(362.7,70.8)
}
}
{
\newrgbcolor{curcolor}{0 0 0}
\pscustom[linewidth=1,linecolor=curcolor]
{
\newpath
\moveto(514.1,164)
\lineto(508.8,160.7)
}
}
{
\newrgbcolor{curcolor}{0 0 0}
\pscustom[linestyle=none,fillstyle=solid,fillcolor=curcolor]
{
\newpath
\moveto(345.52988281,57.1)
\lineto(344.47519531,57.1)
\lineto(344.47519531,63.82070313)
\curveto(344.22128906,63.57851563)(343.88730469,63.33632813)(343.47324219,63.09414063)
\curveto(343.06308594,62.85195313)(342.69394531,62.6703125)(342.36582031,62.54921875)
\lineto(342.36582031,63.56875)
\curveto(342.95566406,63.84609375)(343.47128906,64.18203125)(343.91269531,64.5765625)
\curveto(344.35410156,64.97109375)(344.66660156,65.35390625)(344.85019531,65.725)
\lineto(345.52988281,65.725)
\closepath
}
}
{
\newrgbcolor{curcolor}{0 0 0}
\pscustom[linestyle=none,fillstyle=solid,fillcolor=curcolor]
{
\newpath
\moveto(353.70371094,63.58632813)
\lineto(352.65488281,63.50429688)
\curveto(352.56113281,63.91835938)(352.42832031,64.21914063)(352.25644531,64.40664063)
\curveto(351.97128906,64.70742188)(351.61972656,64.8578125)(351.20175781,64.8578125)
\curveto(350.86582031,64.8578125)(350.57089844,64.7640625)(350.31699219,64.5765625)
\curveto(349.98496094,64.334375)(349.72324219,63.98085938)(349.53183594,63.51601563)
\curveto(349.34042969,63.05117188)(349.24082031,62.3890625)(349.23300781,61.5296875)
\curveto(349.48691406,61.91640625)(349.79746094,62.20351563)(350.16464844,62.39101563)
\curveto(350.53183594,62.57851563)(350.91660156,62.67226563)(351.31894531,62.67226563)
\curveto(352.02207031,62.67226563)(352.61972656,62.4125)(353.11191406,61.89296875)
\curveto(353.60800781,61.37734375)(353.85605469,60.709375)(353.85605469,59.8890625)
\curveto(353.85605469,59.35)(353.73886719,58.84804688)(353.50449219,58.38320313)
\curveto(353.27402344,57.92226563)(352.95566406,57.56875)(352.54941406,57.32265625)
\curveto(352.14316406,57.0765625)(351.68222656,56.95351563)(351.16660156,56.95351563)
\curveto(350.28769531,56.95351563)(349.57089844,57.27578125)(349.01621094,57.9203125)
\curveto(348.46152344,58.56875)(348.18417969,59.63515625)(348.18417969,61.11953125)
\curveto(348.18417969,62.7796875)(348.49082031,63.98671875)(349.10410156,64.740625)
\curveto(349.63925781,65.396875)(350.35996094,65.725)(351.26621094,65.725)
\curveto(351.94199219,65.725)(352.49472656,65.53554688)(352.92441406,65.15664063)
\curveto(353.35800781,64.77773438)(353.61777344,64.25429688)(353.70371094,63.58632813)
\closepath
\moveto(349.39707031,59.88320313)
\curveto(349.39707031,59.51992188)(349.47324219,59.17226563)(349.62558594,58.84023438)
\curveto(349.78183594,58.50820313)(349.99863281,58.25429688)(350.27597656,58.07851563)
\curveto(350.55332031,57.90664063)(350.84433594,57.82070313)(351.14902344,57.82070313)
\curveto(351.59433594,57.82070313)(351.97714844,58.00039063)(352.29746094,58.35976563)
\curveto(352.61777344,58.71914063)(352.77792969,59.20742188)(352.77792969,59.82460938)
\curveto(352.77792969,60.41835938)(352.61972656,60.88515625)(352.30332031,61.225)
\curveto(351.98691406,61.56875)(351.58847656,61.740625)(351.10800781,61.740625)
\curveto(350.63144531,61.740625)(350.22714844,61.56875)(349.89511719,61.225)
\curveto(349.56308594,60.88515625)(349.39707031,60.43789063)(349.39707031,59.88320313)
\closepath
}
}
{
\newrgbcolor{curcolor}{0 0 0}
\pscustom[linestyle=none,fillstyle=solid,fillcolor=curcolor]
{
\newpath
\moveto(359.35214844,60.46914063)
\lineto(359.35214844,61.47695313)
\lineto(362.99082031,61.4828125)
\lineto(362.99082031,58.2953125)
\curveto(362.43222656,57.85)(361.85605469,57.5140625)(361.26230469,57.2875)
\curveto(360.66855469,57.06484375)(360.05917969,56.95351563)(359.43417969,56.95351563)
\curveto(358.59042969,56.95351563)(357.82285156,57.13320313)(357.13144531,57.49257813)
\curveto(356.44394531,57.85585938)(355.92441406,58.37929688)(355.57285156,59.06289063)
\curveto(355.22128906,59.74648438)(355.04550781,60.51015625)(355.04550781,61.35390625)
\curveto(355.04550781,62.18984375)(355.21933594,62.96914063)(355.56699219,63.69179688)
\curveto(355.91855469,64.41835938)(356.42246094,64.95742188)(357.07871094,65.30898438)
\curveto(357.73496094,65.66054688)(358.49082031,65.83632813)(359.34628906,65.83632813)
\curveto(359.96738281,65.83632813)(360.52792969,65.73476563)(361.02792969,65.53164063)
\curveto(361.53183594,65.33242188)(361.92636719,65.053125)(362.21152344,64.69375)
\curveto(362.49667969,64.334375)(362.71347656,63.865625)(362.86191406,63.2875)
\lineto(361.83652344,63.00625)
\curveto(361.70761719,63.44375)(361.54746094,63.7875)(361.35605469,64.0375)
\curveto(361.16464844,64.2875)(360.89121094,64.48671875)(360.53574219,64.63515625)
\curveto(360.18027344,64.7875)(359.78574219,64.86367188)(359.35214844,64.86367188)
\curveto(358.83261719,64.86367188)(358.38339844,64.78359375)(358.00449219,64.6234375)
\curveto(357.62558594,64.4671875)(357.31894531,64.26015625)(357.08457031,64.00234375)
\curveto(356.85410156,63.74453125)(356.67441406,63.46132813)(356.54550781,63.15273438)
\curveto(356.32675781,62.62148438)(356.21738281,62.0453125)(356.21738281,61.42421875)
\curveto(356.21738281,60.65859375)(356.34824219,60.01796875)(356.60996094,59.50234375)
\curveto(356.87558594,58.98671875)(357.26035156,58.60390625)(357.76425781,58.35390625)
\curveto(358.26816406,58.10390625)(358.80332031,57.97890625)(359.36972656,57.97890625)
\curveto(359.86191406,57.97890625)(360.34238281,58.07265625)(360.81113281,58.26015625)
\curveto(361.27988281,58.4515625)(361.63535156,58.6546875)(361.87753906,58.86953125)
\lineto(361.87753906,60.46914063)
\closepath
}
}
{
\newrgbcolor{curcolor}{0 0 0}
\pscustom[linewidth=1,linecolor=curcolor]
{
\newpath
\moveto(357.3,67.6)
\lineto(348.1,69.5)
}
}
{
\newrgbcolor{curcolor}{0 0 0}
\pscustom[linewidth=1,linecolor=curcolor]
{
\newpath
\moveto(85.9,123.2)
\lineto(95.1,121.3)
}
}
{
\newrgbcolor{curcolor}{0 0 0}
\pscustom[linestyle=none,fillstyle=solid,fillcolor=curcolor]
{
\newpath
\moveto(370.37070312,59.9)
\lineto(369.31601562,59.9)
\lineto(369.31601562,66.62070312)
\curveto(369.06210937,66.37851562)(368.728125,66.13632812)(368.3140625,65.89414062)
\curveto(367.90390625,65.65195312)(367.53476562,65.4703125)(367.20664062,65.34921875)
\lineto(367.20664062,66.36875)
\curveto(367.79648437,66.64609375)(368.31210937,66.98203125)(368.75351562,67.3765625)
\curveto(369.19492187,67.77109375)(369.50742187,68.15390625)(369.69101562,68.525)
\lineto(370.37070312,68.525)
\closepath
}
}
{
\newrgbcolor{curcolor}{0 0 0}
\pscustom[linestyle=none,fillstyle=solid,fillcolor=curcolor]
{
\newpath
\moveto(373.45273437,59.9)
\lineto(373.45273437,68.48984375)
\lineto(374.58945312,68.48984375)
\lineto(374.58945312,64.23007812)
\lineto(378.85507812,68.48984375)
\lineto(380.39609375,68.48984375)
\lineto(376.79257812,65.009375)
\lineto(380.55429687,59.9)
\lineto(379.05429687,59.9)
\lineto(375.99570312,64.24765625)
\lineto(374.58945312,62.8765625)
\lineto(374.58945312,59.9)
\closepath
}
}
{
\newrgbcolor{curcolor}{0 0 0}
\pscustom[linewidth=1,linecolor=curcolor]
{
\newpath
\moveto(369.4,75)
\lineto(360.2,76.9)
}
}
{
\newrgbcolor{curcolor}{0 0 0}
\pscustom[linewidth=1,linecolor=curcolor]
{
\newpath
\moveto(97.9,130.7)
\lineto(107.2,128.8)
}
}
{
\newrgbcolor{curcolor}{0 0 0}
\pscustom[linestyle=none,fillstyle=solid,fillcolor=curcolor]
{
\newpath
\moveto(383.94101562,68.31367188)
\lineto(383.94101562,67.3)
\lineto(378.26328125,67.3)
\curveto(378.25546875,67.55390625)(378.29648437,67.79804688)(378.38632812,68.03242188)
\curveto(378.53085937,68.41914063)(378.76132812,68.8)(379.07773437,69.175)
\curveto(379.39804687,69.55)(379.85898437,69.98359375)(380.46054687,70.47578125)
\curveto(381.39414062,71.24140625)(382.025,71.846875)(382.353125,72.2921875)
\curveto(382.68125,72.74140625)(382.8453125,73.16523438)(382.8453125,73.56367188)
\curveto(382.8453125,73.98164063)(382.69492187,74.33320313)(382.39414062,74.61835938)
\curveto(382.09726562,74.90742188)(381.70859375,75.05195313)(381.228125,75.05195313)
\curveto(380.7203125,75.05195313)(380.3140625,74.89960938)(380.009375,74.59492188)
\curveto(379.7046875,74.29023438)(379.55039062,73.86835938)(379.54648437,73.32929688)
\lineto(378.4625,73.440625)
\curveto(378.53671875,74.24921875)(378.81601562,74.86445313)(379.30039062,75.28632813)
\curveto(379.78476562,75.71210938)(380.43515625,75.925)(381.2515625,75.925)
\curveto(382.07578125,75.925)(382.728125,75.69648438)(383.20859375,75.23945313)
\curveto(383.6890625,74.78242188)(383.92929687,74.21601563)(383.92929687,73.54023438)
\curveto(383.92929687,73.19648438)(383.85898437,72.85859375)(383.71835937,72.5265625)
\curveto(383.57773437,72.19453125)(383.34335937,71.84492188)(383.01523437,71.47773438)
\curveto(382.69101562,71.11054688)(382.15,70.60664063)(381.3921875,69.96601563)
\curveto(380.759375,69.43476563)(380.353125,69.0734375)(380.1734375,68.88203125)
\curveto(379.99375,68.69453125)(379.8453125,68.50507813)(379.728125,68.31367188)
\closepath
}
}
{
\newrgbcolor{curcolor}{0 0 0}
\pscustom[linestyle=none,fillstyle=solid,fillcolor=curcolor]
{
\newpath
\moveto(385.45273437,67.3)
\lineto(385.45273437,75.88984375)
\lineto(386.58945312,75.88984375)
\lineto(386.58945312,71.63007813)
\lineto(390.85507812,75.88984375)
\lineto(392.39609375,75.88984375)
\lineto(388.79257812,72.409375)
\lineto(392.55429687,67.3)
\lineto(391.05429687,67.3)
\lineto(387.99570312,71.64765625)
\lineto(386.58945312,70.2765625)
\lineto(386.58945312,67.3)
\closepath
}
}
{
\newrgbcolor{curcolor}{0 0 0}
\pscustom[linewidth=1,linecolor=curcolor]
{
\newpath
\moveto(381.5,82.4)
\lineto(372.2,84.3)
}
}
{
\newrgbcolor{curcolor}{0 0 0}
\pscustom[linewidth=1,linecolor=curcolor]
{
\newpath
\moveto(110,138.1)
\lineto(119.2,136.2)
}
}
{
\newrgbcolor{curcolor}{0 0 0}
\pscustom[linestyle=none,fillstyle=solid,fillcolor=curcolor]
{
\newpath
\moveto(393.87890625,74.7)
\lineto(393.87890625,76.75664062)
\lineto(390.15234375,76.75664062)
\lineto(390.15234375,77.7234375)
\lineto(394.07226562,83.28984375)
\lineto(394.93359375,83.28984375)
\lineto(394.93359375,77.7234375)
\lineto(396.09375,77.7234375)
\lineto(396.09375,76.75664062)
\lineto(394.93359375,76.75664062)
\lineto(394.93359375,74.7)
\closepath
\moveto(393.87890625,77.7234375)
\lineto(393.87890625,81.59648437)
\lineto(391.18945312,77.7234375)
\closepath
}
}
{
\newrgbcolor{curcolor}{0 0 0}
\pscustom[linestyle=none,fillstyle=solid,fillcolor=curcolor]
{
\newpath
\moveto(397.55273438,74.7)
\lineto(397.55273438,83.28984375)
\lineto(398.68945312,83.28984375)
\lineto(398.68945312,79.03007812)
\lineto(402.95507812,83.28984375)
\lineto(404.49609375,83.28984375)
\lineto(400.89257812,79.809375)
\lineto(404.65429688,74.7)
\lineto(403.15429688,74.7)
\lineto(400.09570312,79.04765625)
\lineto(398.68945312,77.6765625)
\lineto(398.68945312,74.7)
\closepath
}
}
{
\newrgbcolor{curcolor}{0 0 0}
\pscustom[linewidth=1,linecolor=curcolor]
{
\newpath
\moveto(393.5,89.8)
\lineto(384.3,91.7)
}
}
{
\newrgbcolor{curcolor}{0 0 0}
\pscustom[linewidth=1,linecolor=curcolor]
{
\newpath
\moveto(122,145.5)
\lineto(131.3,143.6)
}
}
{
\newrgbcolor{curcolor}{0 0 0}
\pscustom[linestyle=none,fillstyle=solid,fillcolor=curcolor]
{
\newpath
\moveto(404.22109375,86.75820313)
\curveto(403.78359375,86.91835938)(403.459375,87.146875)(403.2484375,87.44375)
\curveto(403.0375,87.740625)(402.93203125,88.09609375)(402.93203125,88.51015625)
\curveto(402.93203125,89.13515625)(403.15664063,89.66054688)(403.60585938,90.08632813)
\curveto(404.05507813,90.51210938)(404.65273438,90.725)(405.39882813,90.725)
\curveto(406.14882813,90.725)(406.75234375,90.50625)(407.209375,90.06875)
\curveto(407.66640625,89.63515625)(407.89492188,89.10585938)(407.89492188,88.48085938)
\curveto(407.89492188,88.08242188)(407.78945313,87.73476563)(407.57851563,87.43789063)
\curveto(407.37148438,87.14492188)(407.05507813,86.91835938)(406.62929688,86.75820313)
\curveto(407.15664063,86.58632813)(407.55703125,86.30898438)(407.83046875,85.92617188)
\curveto(408.1078125,85.54335938)(408.24648438,85.08632813)(408.24648438,84.55507813)
\curveto(408.24648438,83.82070313)(407.98671875,83.20351563)(407.4671875,82.70351563)
\curveto(406.94765625,82.20351563)(406.2640625,81.95351563)(405.41640625,81.95351563)
\curveto(404.56875,81.95351563)(403.88515625,82.20351563)(403.365625,82.70351563)
\curveto(402.84609375,83.20742188)(402.58632813,83.834375)(402.58632813,84.584375)
\curveto(402.58632813,85.14296875)(402.72695313,85.60976563)(403.00820313,85.98476563)
\curveto(403.29335938,86.36367188)(403.69765625,86.62148438)(404.22109375,86.75820313)
\closepath
\moveto(404.01015625,88.5453125)
\curveto(404.01015625,88.1390625)(404.14101563,87.80703125)(404.40273438,87.54921875)
\curveto(404.66445313,87.29140625)(405.00429688,87.1625)(405.42226563,87.1625)
\curveto(405.82851563,87.1625)(406.16054688,87.28945313)(406.41835938,87.54335938)
\curveto(406.68007813,87.80117188)(406.8109375,88.115625)(406.8109375,88.48671875)
\curveto(406.8109375,88.8734375)(406.67617188,89.19765625)(406.40664063,89.459375)
\curveto(406.14101563,89.725)(405.80898438,89.8578125)(405.41054688,89.8578125)
\curveto(405.00820313,89.8578125)(404.67421875,89.72890625)(404.40859375,89.47109375)
\curveto(404.14296875,89.21328125)(404.01015625,88.9046875)(404.01015625,88.5453125)
\closepath
\moveto(403.6703125,84.57851563)
\curveto(403.6703125,84.27773438)(403.740625,83.98671875)(403.88125,83.70546875)
\curveto(404.02578125,83.42421875)(404.23867188,83.20546875)(404.51992188,83.04921875)
\curveto(404.80117188,82.896875)(405.10390625,82.82070313)(405.428125,82.82070313)
\curveto(405.93203125,82.82070313)(406.34804688,82.9828125)(406.67617188,83.30703125)
\curveto(407.00429688,83.63125)(407.16835938,84.04335938)(407.16835938,84.54335938)
\curveto(407.16835938,85.05117188)(406.9984375,85.47109375)(406.65859375,85.803125)
\curveto(406.32265625,86.13515625)(405.90078125,86.30117188)(405.39296875,86.30117188)
\curveto(404.896875,86.30117188)(404.48476563,86.13710938)(404.15664063,85.80898438)
\curveto(403.83242188,85.48085938)(403.6703125,85.07070313)(403.6703125,84.57851563)
\closepath
}
}
{
\newrgbcolor{curcolor}{0 0 0}
\pscustom[linestyle=none,fillstyle=solid,fillcolor=curcolor]
{
\newpath
\moveto(409.65273438,82.1)
\lineto(409.65273438,90.68984375)
\lineto(410.78945313,90.68984375)
\lineto(410.78945313,86.43007813)
\lineto(415.05507813,90.68984375)
\lineto(416.59609375,90.68984375)
\lineto(412.99257813,87.209375)
\lineto(416.75429688,82.1)
\lineto(415.25429688,82.1)
\lineto(412.19570313,86.44765625)
\lineto(410.78945313,85.0765625)
\lineto(410.78945313,82.1)
\closepath
}
}
{
\newrgbcolor{curcolor}{0 0 0}
\pscustom[linewidth=1,linecolor=curcolor]
{
\newpath
\moveto(405.6,97.2)
\lineto(396.3,99.1)
}
}
{
\newrgbcolor{curcolor}{0 0 0}
\pscustom[linewidth=1,linecolor=curcolor]
{
\newpath
\moveto(134.1,152.9)
\lineto(143.4,151)
}
}
{
\newrgbcolor{curcolor}{0 0 0}
\pscustom[linestyle=none,fillstyle=solid,fillcolor=curcolor]
{
\newpath
\moveto(418.57070313,89.5)
\lineto(417.51601563,89.5)
\lineto(417.51601563,96.22070312)
\curveto(417.26210938,95.97851562)(416.928125,95.73632812)(416.5140625,95.49414062)
\curveto(416.10390625,95.25195312)(415.73476563,95.0703125)(415.40664063,94.94921875)
\lineto(415.40664063,95.96875)
\curveto(415.99648438,96.24609375)(416.51210938,96.58203125)(416.95351563,96.9765625)
\curveto(417.39492188,97.37109375)(417.70742188,97.75390625)(417.89101563,98.125)
\lineto(418.57070313,98.125)
\closepath
}
}
{
\newrgbcolor{curcolor}{0 0 0}
\pscustom[linestyle=none,fillstyle=solid,fillcolor=curcolor]
{
\newpath
\moveto(426.74453125,95.98632812)
\lineto(425.69570313,95.90429688)
\curveto(425.60195313,96.31835938)(425.46914063,96.61914062)(425.29726563,96.80664062)
\curveto(425.01210938,97.10742188)(424.66054688,97.2578125)(424.24257813,97.2578125)
\curveto(423.90664063,97.2578125)(423.61171875,97.1640625)(423.3578125,96.9765625)
\curveto(423.02578125,96.734375)(422.7640625,96.38085938)(422.57265625,95.91601562)
\curveto(422.38125,95.45117188)(422.28164063,94.7890625)(422.27382813,93.9296875)
\curveto(422.52773438,94.31640625)(422.83828125,94.60351562)(423.20546875,94.79101562)
\curveto(423.57265625,94.97851562)(423.95742188,95.07226562)(424.35976563,95.07226562)
\curveto(425.06289063,95.07226562)(425.66054688,94.8125)(426.15273438,94.29296875)
\curveto(426.64882813,93.77734375)(426.896875,93.109375)(426.896875,92.2890625)
\curveto(426.896875,91.75)(426.7796875,91.24804688)(426.5453125,90.78320312)
\curveto(426.31484375,90.32226562)(425.99648438,89.96875)(425.59023438,89.72265625)
\curveto(425.18398438,89.4765625)(424.72304688,89.35351562)(424.20742188,89.35351562)
\curveto(423.32851563,89.35351562)(422.61171875,89.67578125)(422.05703125,90.3203125)
\curveto(421.50234375,90.96875)(421.225,92.03515625)(421.225,93.51953125)
\curveto(421.225,95.1796875)(421.53164063,96.38671875)(422.14492188,97.140625)
\curveto(422.68007813,97.796875)(423.40078125,98.125)(424.30703125,98.125)
\curveto(424.9828125,98.125)(425.53554688,97.93554688)(425.96523438,97.55664062)
\curveto(426.39882813,97.17773438)(426.65859375,96.65429688)(426.74453125,95.98632812)
\closepath
\moveto(422.43789063,92.28320312)
\curveto(422.43789063,91.91992188)(422.5140625,91.57226562)(422.66640625,91.24023438)
\curveto(422.82265625,90.90820312)(423.03945313,90.65429688)(423.31679688,90.47851562)
\curveto(423.59414063,90.30664062)(423.88515625,90.22070312)(424.18984375,90.22070312)
\curveto(424.63515625,90.22070312)(425.01796875,90.40039062)(425.33828125,90.75976562)
\curveto(425.65859375,91.11914062)(425.81875,91.60742188)(425.81875,92.22460938)
\curveto(425.81875,92.81835938)(425.66054688,93.28515625)(425.34414063,93.625)
\curveto(425.02773438,93.96875)(424.62929688,94.140625)(424.14882813,94.140625)
\curveto(423.67226563,94.140625)(423.26796875,93.96875)(422.9359375,93.625)
\curveto(422.60390625,93.28515625)(422.43789063,92.83789062)(422.43789063,92.28320312)
\closepath
}
}
{
\newrgbcolor{curcolor}{0 0 0}
\pscustom[linestyle=none,fillstyle=solid,fillcolor=curcolor]
{
\newpath
\moveto(428.3265625,89.5)
\lineto(428.3265625,98.08984375)
\lineto(429.46328125,98.08984375)
\lineto(429.46328125,93.83007812)
\lineto(433.72890625,98.08984375)
\lineto(435.26992188,98.08984375)
\lineto(431.66640625,94.609375)
\lineto(435.428125,89.5)
\lineto(433.928125,89.5)
\lineto(430.86953125,93.84765625)
\lineto(429.46328125,92.4765625)
\lineto(429.46328125,89.5)
\closepath
}
}
{
\newrgbcolor{curcolor}{0 0 0}
\pscustom[linewidth=1,linecolor=curcolor]
{
\newpath
\moveto(417.7,104.7)
\lineto(408.4,106.5)
}
}
{
\newrgbcolor{curcolor}{0 0 0}
\pscustom[linewidth=1,linecolor=curcolor]
{
\newpath
\moveto(146.2,160.3)
\lineto(155.4,158.4)
}
}
{
\newrgbcolor{curcolor}{0 0 0}
\pscustom[linestyle=none,fillstyle=solid,fillcolor=curcolor]
{
\newpath
\moveto(426.70390625,99.26757812)
\lineto(427.75859375,99.40820312)
\curveto(427.8796875,98.81054688)(428.08476562,98.37890625)(428.37382812,98.11328125)
\curveto(428.66679687,97.8515625)(429.02226562,97.72070312)(429.44023437,97.72070312)
\curveto(429.93632812,97.72070312)(430.35429687,97.89257812)(430.69414062,98.23632812)
\curveto(431.03789062,98.58007812)(431.20976562,99.00585938)(431.20976562,99.51367188)
\curveto(431.20976562,99.99804688)(431.0515625,100.39648438)(430.73515625,100.70898438)
\curveto(430.41875,101.02539062)(430.01640625,101.18359375)(429.528125,101.18359375)
\curveto(429.32890625,101.18359375)(429.08085937,101.14453125)(428.78398437,101.06640625)
\lineto(428.90117187,101.9921875)
\curveto(428.97148437,101.984375)(429.028125,101.98046875)(429.07109375,101.98046875)
\curveto(429.5203125,101.98046875)(429.92460937,102.09765625)(430.28398437,102.33203125)
\curveto(430.64335937,102.56640625)(430.82304687,102.92773438)(430.82304687,103.41601562)
\curveto(430.82304687,103.80273438)(430.6921875,104.12304688)(430.43046875,104.37695312)
\curveto(430.16875,104.63085938)(429.83085937,104.7578125)(429.41679687,104.7578125)
\curveto(429.00664062,104.7578125)(428.66484375,104.62890625)(428.39140625,104.37109375)
\curveto(428.11796875,104.11328125)(427.9421875,103.7265625)(427.8640625,103.2109375)
\lineto(426.809375,103.3984375)
\curveto(426.93828125,104.10546875)(427.23125,104.65234375)(427.68828125,105.0390625)
\curveto(428.1453125,105.4296875)(428.71367187,105.625)(429.39335937,105.625)
\curveto(429.86210937,105.625)(430.29375,105.5234375)(430.68828125,105.3203125)
\curveto(431.0828125,105.12109375)(431.38359375,104.84765625)(431.590625,104.5)
\curveto(431.8015625,104.15234375)(431.90703125,103.78320312)(431.90703125,103.39257812)
\curveto(431.90703125,103.02148438)(431.80742187,102.68359375)(431.60820312,102.37890625)
\curveto(431.40898437,102.07421875)(431.1140625,101.83203125)(430.7234375,101.65234375)
\curveto(431.23125,101.53515625)(431.62578125,101.29101562)(431.90703125,100.91992188)
\curveto(432.18828125,100.55273438)(432.32890625,100.09179688)(432.32890625,99.53710938)
\curveto(432.32890625,98.78710938)(432.05546875,98.15039062)(431.50859375,97.62695312)
\curveto(430.96171875,97.10742188)(430.2703125,96.84765625)(429.434375,96.84765625)
\curveto(428.68046875,96.84765625)(428.05351562,97.07226562)(427.55351562,97.52148438)
\curveto(427.05742187,97.97070312)(426.77421875,98.55273438)(426.70390625,99.26757812)
\closepath
}
}
{
\newrgbcolor{curcolor}{0 0 0}
\pscustom[linestyle=none,fillstyle=solid,fillcolor=curcolor]
{
\newpath
\moveto(438.91484375,98.01367188)
\lineto(438.91484375,97)
\lineto(433.23710937,97)
\curveto(433.22929687,97.25390625)(433.2703125,97.49804688)(433.36015625,97.73242188)
\curveto(433.5046875,98.11914062)(433.73515625,98.5)(434.0515625,98.875)
\curveto(434.371875,99.25)(434.8328125,99.68359375)(435.434375,100.17578125)
\curveto(436.36796875,100.94140625)(436.99882812,101.546875)(437.32695312,101.9921875)
\curveto(437.65507812,102.44140625)(437.81914062,102.86523438)(437.81914062,103.26367188)
\curveto(437.81914062,103.68164062)(437.66875,104.03320312)(437.36796875,104.31835938)
\curveto(437.07109375,104.60742188)(436.68242187,104.75195312)(436.20195312,104.75195312)
\curveto(435.69414062,104.75195312)(435.28789062,104.59960938)(434.98320312,104.29492188)
\curveto(434.67851562,103.99023438)(434.52421875,103.56835938)(434.5203125,103.02929688)
\lineto(433.43632812,103.140625)
\curveto(433.51054687,103.94921875)(433.78984375,104.56445312)(434.27421875,104.98632812)
\curveto(434.75859375,105.41210938)(435.40898437,105.625)(436.22539062,105.625)
\curveto(437.04960937,105.625)(437.70195312,105.39648438)(438.18242187,104.93945312)
\curveto(438.66289062,104.48242188)(438.903125,103.91601562)(438.903125,103.24023438)
\curveto(438.903125,102.89648438)(438.8328125,102.55859375)(438.6921875,102.2265625)
\curveto(438.5515625,101.89453125)(438.3171875,101.54492188)(437.9890625,101.17773438)
\curveto(437.66484375,100.81054688)(437.12382812,100.30664062)(436.36601562,99.66601562)
\curveto(435.73320312,99.13476562)(435.32695312,98.7734375)(435.14726562,98.58203125)
\curveto(434.96757812,98.39453125)(434.81914062,98.20507812)(434.70195312,98.01367188)
\closepath
}
}
{
\newrgbcolor{curcolor}{0 0 0}
\pscustom[linestyle=none,fillstyle=solid,fillcolor=curcolor]
{
\newpath
\moveto(440.4265625,97)
\lineto(440.4265625,105.58984375)
\lineto(441.56328125,105.58984375)
\lineto(441.56328125,101.33007812)
\lineto(445.82890625,105.58984375)
\lineto(447.36992187,105.58984375)
\lineto(443.76640625,102.109375)
\lineto(447.528125,97)
\lineto(446.028125,97)
\lineto(442.96953125,101.34765625)
\lineto(441.56328125,99.9765625)
\lineto(441.56328125,97)
\closepath
}
}
{
\newrgbcolor{curcolor}{0 0 0}
\pscustom[linewidth=1,linecolor=curcolor]
{
\newpath
\moveto(429.7,112.1)
\lineto(420.5,114)
}
}
{
\newrgbcolor{curcolor}{0 0 0}
\pscustom[linewidth=1,linecolor=curcolor]
{
\newpath
\moveto(158.2,167.7)
\lineto(167.5,165.9)
}
}
{
\newrgbcolor{curcolor}{0 0 0}
\pscustom[linestyle=none,fillstyle=solid,fillcolor=curcolor]
{
\newpath
\moveto(444.17070312,110.88632812)
\lineto(443.121875,110.80429687)
\curveto(443.028125,111.21835937)(442.8953125,111.51914062)(442.7234375,111.70664062)
\curveto(442.43828125,112.00742187)(442.08671875,112.1578125)(441.66875,112.1578125)
\curveto(441.3328125,112.1578125)(441.03789062,112.0640625)(440.78398437,111.8765625)
\curveto(440.45195312,111.634375)(440.19023437,111.28085937)(439.99882812,110.81601562)
\curveto(439.80742187,110.35117187)(439.7078125,109.6890625)(439.7,108.8296875)
\curveto(439.95390625,109.21640625)(440.26445312,109.50351562)(440.63164062,109.69101562)
\curveto(440.99882812,109.87851562)(441.38359375,109.97226562)(441.7859375,109.97226562)
\curveto(442.4890625,109.97226562)(443.08671875,109.7125)(443.57890625,109.19296875)
\curveto(444.075,108.67734375)(444.32304687,108.009375)(444.32304687,107.1890625)
\curveto(444.32304687,106.65)(444.20585937,106.14804687)(443.97148437,105.68320312)
\curveto(443.74101562,105.22226562)(443.42265625,104.86875)(443.01640625,104.62265625)
\curveto(442.61015625,104.3765625)(442.14921875,104.25351562)(441.63359375,104.25351562)
\curveto(440.7546875,104.25351562)(440.03789062,104.57578125)(439.48320312,105.2203125)
\curveto(438.92851562,105.86875)(438.65117187,106.93515625)(438.65117187,108.41953125)
\curveto(438.65117187,110.0796875)(438.9578125,111.28671875)(439.57109375,112.040625)
\curveto(440.10625,112.696875)(440.82695312,113.025)(441.73320312,113.025)
\curveto(442.40898437,113.025)(442.96171875,112.83554687)(443.39140625,112.45664062)
\curveto(443.825,112.07773437)(444.08476562,111.55429687)(444.17070312,110.88632812)
\closepath
\moveto(439.8640625,107.18320312)
\curveto(439.8640625,106.81992187)(439.94023437,106.47226562)(440.09257812,106.14023437)
\curveto(440.24882812,105.80820312)(440.465625,105.55429687)(440.74296875,105.37851562)
\curveto(441.0203125,105.20664062)(441.31132812,105.12070312)(441.61601562,105.12070312)
\curveto(442.06132812,105.12070312)(442.44414062,105.30039062)(442.76445312,105.65976562)
\curveto(443.08476562,106.01914062)(443.24492187,106.50742187)(443.24492187,107.12460937)
\curveto(443.24492187,107.71835937)(443.08671875,108.18515625)(442.7703125,108.525)
\curveto(442.45390625,108.86875)(442.05546875,109.040625)(441.575,109.040625)
\curveto(441.0984375,109.040625)(440.69414062,108.86875)(440.36210937,108.525)
\curveto(440.03007812,108.18515625)(439.8640625,107.73789062)(439.8640625,107.18320312)
\closepath
}
}
{
\newrgbcolor{curcolor}{0 0 0}
\pscustom[linestyle=none,fillstyle=solid,fillcolor=curcolor]
{
\newpath
\moveto(448.75273437,104.4)
\lineto(448.75273437,106.45664062)
\lineto(445.02617187,106.45664062)
\lineto(445.02617187,107.4234375)
\lineto(448.94609375,112.98984375)
\lineto(449.80742187,112.98984375)
\lineto(449.80742187,107.4234375)
\lineto(450.96757812,107.4234375)
\lineto(450.96757812,106.45664062)
\lineto(449.80742187,106.45664062)
\lineto(449.80742187,104.4)
\closepath
\moveto(448.75273437,107.4234375)
\lineto(448.75273437,111.29648437)
\lineto(446.06328125,107.4234375)
\closepath
}
}
{
\newrgbcolor{curcolor}{0 0 0}
\pscustom[linestyle=none,fillstyle=solid,fillcolor=curcolor]
{
\newpath
\moveto(452.4265625,104.4)
\lineto(452.4265625,112.98984375)
\lineto(453.56328125,112.98984375)
\lineto(453.56328125,108.73007812)
\lineto(457.82890625,112.98984375)
\lineto(459.36992187,112.98984375)
\lineto(455.76640625,109.509375)
\lineto(459.528125,104.4)
\lineto(458.028125,104.4)
\lineto(454.96953125,108.74765625)
\lineto(453.56328125,107.3765625)
\lineto(453.56328125,104.4)
\closepath
}
}
{
\newrgbcolor{curcolor}{0 0 0}
\pscustom[linewidth=1,linecolor=curcolor]
{
\newpath
\moveto(441.8,119.5)
\lineto(432.5,121.4)
}
}
{
\newrgbcolor{curcolor}{0 0 0}
\pscustom[linewidth=1,linecolor=curcolor]
{
\newpath
\moveto(170.3,175.2)
\lineto(179.5,173.3)
}
}
{
\newrgbcolor{curcolor}{0 0 0}
\pscustom[linestyle=none,fillstyle=solid,fillcolor=curcolor]
{
\newpath
\moveto(454.77070313,111.8)
\lineto(453.71601563,111.8)
\lineto(453.71601563,118.52070313)
\curveto(453.46210938,118.27851563)(453.128125,118.03632813)(452.7140625,117.79414063)
\curveto(452.30390625,117.55195313)(451.93476563,117.3703125)(451.60664063,117.24921875)
\lineto(451.60664063,118.26875)
\curveto(452.19648438,118.54609375)(452.71210938,118.88203125)(453.15351563,119.2765625)
\curveto(453.59492188,119.67109375)(453.90742188,120.05390625)(454.09101563,120.425)
\lineto(454.77070313,120.425)
\closepath
}
}
{
\newrgbcolor{curcolor}{0 0 0}
\pscustom[linestyle=none,fillstyle=solid,fillcolor=curcolor]
{
\newpath
\moveto(463.01484375,112.81367188)
\lineto(463.01484375,111.8)
\lineto(457.33710938,111.8)
\curveto(457.32929688,112.05390625)(457.3703125,112.29804688)(457.46015625,112.53242188)
\curveto(457.6046875,112.91914063)(457.83515625,113.3)(458.1515625,113.675)
\curveto(458.471875,114.05)(458.9328125,114.48359375)(459.534375,114.97578125)
\curveto(460.46796875,115.74140625)(461.09882813,116.346875)(461.42695313,116.7921875)
\curveto(461.75507813,117.24140625)(461.91914063,117.66523438)(461.91914063,118.06367188)
\curveto(461.91914063,118.48164063)(461.76875,118.83320313)(461.46796875,119.11835938)
\curveto(461.17109375,119.40742188)(460.78242188,119.55195313)(460.30195313,119.55195313)
\curveto(459.79414063,119.55195313)(459.38789063,119.39960938)(459.08320313,119.09492188)
\curveto(458.77851563,118.79023438)(458.62421875,118.36835938)(458.6203125,117.82929688)
\lineto(457.53632813,117.940625)
\curveto(457.61054688,118.74921875)(457.88984375,119.36445313)(458.37421875,119.78632813)
\curveto(458.85859375,120.21210938)(459.50898438,120.425)(460.32539063,120.425)
\curveto(461.14960938,120.425)(461.80195313,120.19648438)(462.28242188,119.73945313)
\curveto(462.76289063,119.28242188)(463.003125,118.71601563)(463.003125,118.04023438)
\curveto(463.003125,117.69648438)(462.9328125,117.35859375)(462.7921875,117.0265625)
\curveto(462.6515625,116.69453125)(462.4171875,116.34492188)(462.0890625,115.97773438)
\curveto(461.76484375,115.61054688)(461.22382813,115.10664063)(460.46601563,114.46601563)
\curveto(459.83320313,113.93476563)(459.42695313,113.5734375)(459.24726563,113.38203125)
\curveto(459.06757813,113.19453125)(458.91914063,113.00507813)(458.80195313,112.81367188)
\closepath
}
}
{
\newrgbcolor{curcolor}{0 0 0}
\pscustom[linestyle=none,fillstyle=solid,fillcolor=curcolor]
{
\newpath
\moveto(465.76875,116.45820313)
\curveto(465.33125,116.61835938)(465.00703125,116.846875)(464.79609375,117.14375)
\curveto(464.58515625,117.440625)(464.4796875,117.79609375)(464.4796875,118.21015625)
\curveto(464.4796875,118.83515625)(464.70429688,119.36054688)(465.15351563,119.78632813)
\curveto(465.60273438,120.21210938)(466.20039063,120.425)(466.94648438,120.425)
\curveto(467.69648438,120.425)(468.3,120.20625)(468.75703125,119.76875)
\curveto(469.2140625,119.33515625)(469.44257813,118.80585938)(469.44257813,118.18085938)
\curveto(469.44257813,117.78242188)(469.33710938,117.43476563)(469.12617188,117.13789063)
\curveto(468.91914063,116.84492188)(468.60273438,116.61835938)(468.17695313,116.45820313)
\curveto(468.70429688,116.28632813)(469.1046875,116.00898438)(469.378125,115.62617188)
\curveto(469.65546875,115.24335938)(469.79414063,114.78632813)(469.79414063,114.25507813)
\curveto(469.79414063,113.52070313)(469.534375,112.90351563)(469.01484375,112.40351563)
\curveto(468.4953125,111.90351563)(467.81171875,111.65351563)(466.9640625,111.65351563)
\curveto(466.11640625,111.65351563)(465.4328125,111.90351563)(464.91328125,112.40351563)
\curveto(464.39375,112.90742188)(464.13398438,113.534375)(464.13398438,114.284375)
\curveto(464.13398438,114.84296875)(464.27460938,115.30976563)(464.55585938,115.68476563)
\curveto(464.84101563,116.06367188)(465.2453125,116.32148438)(465.76875,116.45820313)
\closepath
\moveto(465.5578125,118.2453125)
\curveto(465.5578125,117.8390625)(465.68867188,117.50703125)(465.95039063,117.24921875)
\curveto(466.21210938,116.99140625)(466.55195313,116.8625)(466.96992188,116.8625)
\curveto(467.37617188,116.8625)(467.70820313,116.98945313)(467.96601563,117.24335938)
\curveto(468.22773438,117.50117188)(468.35859375,117.815625)(468.35859375,118.18671875)
\curveto(468.35859375,118.5734375)(468.22382813,118.89765625)(467.95429688,119.159375)
\curveto(467.68867188,119.425)(467.35664063,119.5578125)(466.95820313,119.5578125)
\curveto(466.55585938,119.5578125)(466.221875,119.42890625)(465.95625,119.17109375)
\curveto(465.690625,118.91328125)(465.5578125,118.6046875)(465.5578125,118.2453125)
\closepath
\moveto(465.21796875,114.27851563)
\curveto(465.21796875,113.97773438)(465.28828125,113.68671875)(465.42890625,113.40546875)
\curveto(465.5734375,113.12421875)(465.78632813,112.90546875)(466.06757813,112.74921875)
\curveto(466.34882813,112.596875)(466.6515625,112.52070313)(466.97578125,112.52070313)
\curveto(467.4796875,112.52070313)(467.89570313,112.6828125)(468.22382813,113.00703125)
\curveto(468.55195313,113.33125)(468.71601563,113.74335938)(468.71601563,114.24335938)
\curveto(468.71601563,114.75117188)(468.54609375,115.17109375)(468.20625,115.503125)
\curveto(467.8703125,115.83515625)(467.4484375,116.00117188)(466.940625,116.00117188)
\curveto(466.44453125,116.00117188)(466.03242188,115.83710938)(465.70429688,115.50898438)
\curveto(465.38007813,115.18085938)(465.21796875,114.77070313)(465.21796875,114.27851563)
\closepath
}
}
{
\newrgbcolor{curcolor}{0 0 0}
\pscustom[linestyle=none,fillstyle=solid,fillcolor=curcolor]
{
\newpath
\moveto(471.20039063,111.8)
\lineto(471.20039063,120.38984375)
\lineto(472.33710938,120.38984375)
\lineto(472.33710938,116.13007813)
\lineto(476.60273438,120.38984375)
\lineto(478.14375,120.38984375)
\lineto(474.54023438,116.909375)
\lineto(478.30195313,111.8)
\lineto(476.80195313,111.8)
\lineto(473.74335938,116.14765625)
\lineto(472.33710938,114.7765625)
\lineto(472.33710938,111.8)
\closepath
}
}
{
\newrgbcolor{curcolor}{0 0 0}
\pscustom[linewidth=1,linecolor=curcolor]
{
\newpath
\moveto(453.8,126.9)
\lineto(444.6,128.8)
}
}
{
\newrgbcolor{curcolor}{0 0 0}
\pscustom[linewidth=1,linecolor=curcolor]
{
\newpath
\moveto(182.3,182.6)
\lineto(191.6,180.7)
}
}
{
\newrgbcolor{curcolor}{0 0 0}
\pscustom[linestyle=none,fillstyle=solid,fillcolor=curcolor]
{
\newpath
\moveto(468.44101562,120.21367187)
\lineto(468.44101562,119.2)
\lineto(462.76328125,119.2)
\curveto(462.75546875,119.45390625)(462.79648437,119.69804687)(462.88632812,119.93242187)
\curveto(463.03085937,120.31914062)(463.26132812,120.7)(463.57773437,121.075)
\curveto(463.89804687,121.45)(464.35898437,121.88359375)(464.96054687,122.37578125)
\curveto(465.89414062,123.14140625)(466.525,123.746875)(466.853125,124.1921875)
\curveto(467.18125,124.64140625)(467.3453125,125.06523437)(467.3453125,125.46367187)
\curveto(467.3453125,125.88164062)(467.19492187,126.23320312)(466.89414062,126.51835937)
\curveto(466.59726562,126.80742187)(466.20859375,126.95195312)(465.728125,126.95195312)
\curveto(465.2203125,126.95195312)(464.8140625,126.79960937)(464.509375,126.49492187)
\curveto(464.2046875,126.19023437)(464.05039062,125.76835937)(464.04648437,125.22929687)
\lineto(462.9625,125.340625)
\curveto(463.03671875,126.14921875)(463.31601562,126.76445312)(463.80039062,127.18632812)
\curveto(464.28476562,127.61210937)(464.93515625,127.825)(465.7515625,127.825)
\curveto(466.57578125,127.825)(467.228125,127.59648437)(467.70859375,127.13945312)
\curveto(468.1890625,126.68242187)(468.42929687,126.11601562)(468.42929687,125.44023437)
\curveto(468.42929687,125.09648437)(468.35898437,124.75859375)(468.21835937,124.4265625)
\curveto(468.07773437,124.09453125)(467.84335937,123.74492187)(467.51523437,123.37773437)
\curveto(467.19101562,123.01054687)(466.65,122.50664062)(465.8921875,121.86601562)
\curveto(465.259375,121.33476562)(464.853125,120.9734375)(464.6734375,120.78203125)
\curveto(464.49375,120.59453125)(464.3453125,120.40507812)(464.228125,120.21367187)
\closepath
}
}
{
\newrgbcolor{curcolor}{0 0 0}
\pscustom[linestyle=none,fillstyle=solid,fillcolor=curcolor]
{
\newpath
\moveto(469.571875,121.45)
\lineto(470.67929687,121.54375)
\curveto(470.76132812,121.0046875)(470.95078125,120.5984375)(471.24765625,120.325)
\curveto(471.5484375,120.05546875)(471.90976562,119.92070312)(472.33164062,119.92070312)
\curveto(472.83945312,119.92070312)(473.26914062,120.11210937)(473.62070312,120.49492187)
\curveto(473.97226562,120.87773437)(474.14804687,121.38554687)(474.14804687,122.01835937)
\curveto(474.14804687,122.61992187)(473.978125,123.09453125)(473.63828125,123.4421875)
\curveto(473.30234375,123.78984375)(472.8609375,123.96367187)(472.3140625,123.96367187)
\curveto(471.97421875,123.96367187)(471.66757812,123.88554687)(471.39414062,123.72929687)
\curveto(471.12070312,123.57695312)(470.90585937,123.37773437)(470.74960937,123.13164062)
\lineto(469.759375,123.26054687)
\lineto(470.59140625,127.67265625)
\lineto(474.86289062,127.67265625)
\lineto(474.86289062,126.66484375)
\lineto(471.43515625,126.66484375)
\lineto(470.97226562,124.35625)
\curveto(471.48789062,124.715625)(472.02890625,124.8953125)(472.5953125,124.8953125)
\curveto(473.3453125,124.8953125)(473.978125,124.63554687)(474.49375,124.11601562)
\curveto(475.009375,123.59648437)(475.2671875,122.92851562)(475.2671875,122.11210937)
\curveto(475.2671875,121.33476562)(475.040625,120.66289062)(474.5875,120.09648437)
\curveto(474.03671875,119.40117187)(473.28476562,119.05351562)(472.33164062,119.05351562)
\curveto(471.55039062,119.05351562)(470.91171875,119.27226562)(470.415625,119.70976562)
\curveto(469.9234375,120.14726562)(469.6421875,120.72734375)(469.571875,121.45)
\closepath
}
}
{
\newrgbcolor{curcolor}{0 0 0}
\pscustom[linestyle=none,fillstyle=solid,fillcolor=curcolor]
{
\newpath
\moveto(481.71835937,125.68632812)
\lineto(480.66953125,125.60429687)
\curveto(480.57578125,126.01835937)(480.44296875,126.31914062)(480.27109375,126.50664062)
\curveto(479.9859375,126.80742187)(479.634375,126.9578125)(479.21640625,126.9578125)
\curveto(478.88046875,126.9578125)(478.58554687,126.8640625)(478.33164062,126.6765625)
\curveto(477.99960937,126.434375)(477.73789062,126.08085937)(477.54648437,125.61601562)
\curveto(477.35507812,125.15117187)(477.25546875,124.4890625)(477.24765625,123.6296875)
\curveto(477.5015625,124.01640625)(477.81210937,124.30351562)(478.17929687,124.49101562)
\curveto(478.54648437,124.67851562)(478.93125,124.77226562)(479.33359375,124.77226562)
\curveto(480.03671875,124.77226562)(480.634375,124.5125)(481.1265625,123.99296875)
\curveto(481.62265625,123.47734375)(481.87070312,122.809375)(481.87070312,121.9890625)
\curveto(481.87070312,121.45)(481.75351562,120.94804687)(481.51914062,120.48320312)
\curveto(481.28867187,120.02226562)(480.9703125,119.66875)(480.5640625,119.42265625)
\curveto(480.1578125,119.1765625)(479.696875,119.05351562)(479.18125,119.05351562)
\curveto(478.30234375,119.05351562)(477.58554687,119.37578125)(477.03085937,120.0203125)
\curveto(476.47617187,120.66875)(476.19882812,121.73515625)(476.19882812,123.21953125)
\curveto(476.19882812,124.8796875)(476.50546875,126.08671875)(477.11875,126.840625)
\curveto(477.65390625,127.496875)(478.37460937,127.825)(479.28085937,127.825)
\curveto(479.95664062,127.825)(480.509375,127.63554687)(480.9390625,127.25664062)
\curveto(481.37265625,126.87773437)(481.63242187,126.35429687)(481.71835937,125.68632812)
\closepath
\moveto(477.41171875,121.98320312)
\curveto(477.41171875,121.61992187)(477.48789062,121.27226562)(477.64023437,120.94023437)
\curveto(477.79648437,120.60820312)(478.01328125,120.35429687)(478.290625,120.17851562)
\curveto(478.56796875,120.00664062)(478.85898437,119.92070312)(479.16367187,119.92070312)
\curveto(479.60898437,119.92070312)(479.99179687,120.10039062)(480.31210937,120.45976562)
\curveto(480.63242187,120.81914062)(480.79257812,121.30742187)(480.79257812,121.92460937)
\curveto(480.79257812,122.51835937)(480.634375,122.98515625)(480.31796875,123.325)
\curveto(480.0015625,123.66875)(479.603125,123.840625)(479.12265625,123.840625)
\curveto(478.64609375,123.840625)(478.24179687,123.66875)(477.90976562,123.325)
\curveto(477.57773437,122.98515625)(477.41171875,122.53789062)(477.41171875,121.98320312)
\closepath
}
}
{
\newrgbcolor{curcolor}{0 0 0}
\pscustom[linestyle=none,fillstyle=solid,fillcolor=curcolor]
{
\newpath
\moveto(483.30039062,119.2)
\lineto(483.30039062,127.78984375)
\lineto(484.43710937,127.78984375)
\lineto(484.43710937,123.53007812)
\lineto(488.70273437,127.78984375)
\lineto(490.24375,127.78984375)
\lineto(486.64023437,124.309375)
\lineto(490.40195312,119.2)
\lineto(488.90195312,119.2)
\lineto(485.84335937,123.54765625)
\lineto(484.43710937,122.1765625)
\lineto(484.43710937,119.2)
\closepath
}
}
{
\newrgbcolor{curcolor}{0 0 0}
\pscustom[linewidth=1,linecolor=curcolor]
{
\newpath
\moveto(465.9,134.3)
\lineto(456.6,136.2)
}
}
{
\newrgbcolor{curcolor}{0 0 0}
\pscustom[linewidth=1,linecolor=curcolor]
{
\newpath
\moveto(194.4,190)
\lineto(203.7,188.1)
}
}
{
\newrgbcolor{curcolor}{0 0 0}
\pscustom[linestyle=none,fillstyle=solid,fillcolor=curcolor]
{
\newpath
\moveto(474.89804687,128.85)
\lineto(476.00546875,128.94375)
\curveto(476.0875,128.4046875)(476.27695312,127.9984375)(476.57382812,127.725)
\curveto(476.87460937,127.45546875)(477.2359375,127.32070313)(477.6578125,127.32070313)
\curveto(478.165625,127.32070313)(478.5953125,127.51210938)(478.946875,127.89492188)
\curveto(479.2984375,128.27773438)(479.47421875,128.78554688)(479.47421875,129.41835938)
\curveto(479.47421875,130.01992188)(479.30429687,130.49453125)(478.96445312,130.8421875)
\curveto(478.62851562,131.18984375)(478.18710937,131.36367188)(477.64023437,131.36367188)
\curveto(477.30039062,131.36367188)(476.99375,131.28554688)(476.7203125,131.12929688)
\curveto(476.446875,130.97695313)(476.23203125,130.77773438)(476.07578125,130.53164063)
\lineto(475.08554687,130.66054688)
\lineto(475.91757812,135.07265625)
\lineto(480.1890625,135.07265625)
\lineto(480.1890625,134.06484375)
\lineto(476.76132812,134.06484375)
\lineto(476.2984375,131.75625)
\curveto(476.8140625,132.115625)(477.35507812,132.2953125)(477.92148437,132.2953125)
\curveto(478.67148437,132.2953125)(479.30429687,132.03554688)(479.81992187,131.51601563)
\curveto(480.33554687,130.99648438)(480.59335937,130.32851563)(480.59335937,129.51210938)
\curveto(480.59335937,128.73476563)(480.36679687,128.06289063)(479.91367187,127.49648438)
\curveto(479.36289062,126.80117188)(478.6109375,126.45351563)(477.6578125,126.45351563)
\curveto(476.8765625,126.45351563)(476.23789062,126.67226563)(475.74179687,127.10976563)
\curveto(475.24960937,127.54726563)(474.96835937,128.12734375)(474.89804687,128.85)
\closepath
}
}
{
\newrgbcolor{curcolor}{0 0 0}
\pscustom[linestyle=none,fillstyle=solid,fillcolor=curcolor]
{
\newpath
\moveto(485.54453125,126.6)
\lineto(484.48984375,126.6)
\lineto(484.48984375,133.32070313)
\curveto(484.2359375,133.07851563)(483.90195312,132.83632813)(483.48789062,132.59414063)
\curveto(483.07773437,132.35195313)(482.70859375,132.1703125)(482.38046875,132.04921875)
\lineto(482.38046875,133.06875)
\curveto(482.9703125,133.34609375)(483.4859375,133.68203125)(483.92734375,134.0765625)
\curveto(484.36875,134.47109375)(484.68125,134.85390625)(484.86484375,135.225)
\lineto(485.54453125,135.225)
\closepath
}
}
{
\newrgbcolor{curcolor}{0 0 0}
\pscustom[linestyle=none,fillstyle=solid,fillcolor=curcolor]
{
\newpath
\moveto(493.78867187,127.61367188)
\lineto(493.78867187,126.6)
\lineto(488.1109375,126.6)
\curveto(488.103125,126.85390625)(488.14414062,127.09804688)(488.23398437,127.33242188)
\curveto(488.37851562,127.71914063)(488.60898437,128.1)(488.92539062,128.475)
\curveto(489.24570312,128.85)(489.70664062,129.28359375)(490.30820312,129.77578125)
\curveto(491.24179687,130.54140625)(491.87265625,131.146875)(492.20078125,131.5921875)
\curveto(492.52890625,132.04140625)(492.69296875,132.46523438)(492.69296875,132.86367188)
\curveto(492.69296875,133.28164063)(492.54257812,133.63320313)(492.24179687,133.91835938)
\curveto(491.94492187,134.20742188)(491.55625,134.35195313)(491.07578125,134.35195313)
\curveto(490.56796875,134.35195313)(490.16171875,134.19960938)(489.85703125,133.89492188)
\curveto(489.55234375,133.59023438)(489.39804687,133.16835938)(489.39414062,132.62929688)
\lineto(488.31015625,132.740625)
\curveto(488.384375,133.54921875)(488.66367187,134.16445313)(489.14804687,134.58632813)
\curveto(489.63242187,135.01210938)(490.2828125,135.225)(491.09921875,135.225)
\curveto(491.9234375,135.225)(492.57578125,134.99648438)(493.05625,134.53945313)
\curveto(493.53671875,134.08242188)(493.77695312,133.51601563)(493.77695312,132.84023438)
\curveto(493.77695312,132.49648438)(493.70664062,132.15859375)(493.56601562,131.8265625)
\curveto(493.42539062,131.49453125)(493.19101562,131.14492188)(492.86289062,130.77773438)
\curveto(492.53867187,130.41054688)(491.99765625,129.90664063)(491.23984375,129.26601563)
\curveto(490.60703125,128.73476563)(490.20078125,128.3734375)(490.02109375,128.18203125)
\curveto(489.84140625,127.99453125)(489.69296875,127.80507813)(489.57578125,127.61367188)
\closepath
}
}
{
\newrgbcolor{curcolor}{0 0 0}
\pscustom[linestyle=none,fillstyle=solid,fillcolor=curcolor]
{
\newpath
\moveto(495.30039062,126.6)
\lineto(495.30039062,135.18984375)
\lineto(496.43710937,135.18984375)
\lineto(496.43710937,130.93007813)
\lineto(500.70273437,135.18984375)
\lineto(502.24375,135.18984375)
\lineto(498.64023437,131.709375)
\lineto(502.40195312,126.6)
\lineto(500.90195312,126.6)
\lineto(497.84335937,130.94765625)
\lineto(496.43710937,129.5765625)
\lineto(496.43710937,126.6)
\closepath
}
}
{
\newrgbcolor{curcolor}{0 0 0}
\pscustom[linewidth=1,linecolor=curcolor]
{
\newpath
\moveto(478,141.7)
\lineto(468.7,143.6)
}
}
{
\newrgbcolor{curcolor}{0 0 0}
\pscustom[linewidth=1,linecolor=curcolor]
{
\newpath
\moveto(206.5,197.4)
\lineto(215.7,195.5)
}
}
{
\newrgbcolor{curcolor}{0 0 0}
\pscustom[linestyle=none,fillstyle=solid,fillcolor=curcolor]
{
\newpath
\moveto(490.97070312,134.1)
\lineto(489.91601562,134.1)
\lineto(489.91601562,140.82070313)
\curveto(489.66210938,140.57851563)(489.328125,140.33632813)(488.9140625,140.09414063)
\curveto(488.50390625,139.85195313)(488.13476562,139.6703125)(487.80664062,139.54921875)
\lineto(487.80664062,140.56875)
\curveto(488.39648438,140.84609375)(488.91210938,141.18203125)(489.35351562,141.5765625)
\curveto(489.79492188,141.97109375)(490.10742188,142.35390625)(490.29101562,142.725)
\lineto(490.97070312,142.725)
\closepath
}
}
{
\newrgbcolor{curcolor}{0 0 0}
\pscustom[linestyle=none,fillstyle=solid,fillcolor=curcolor]
{
\newpath
\moveto(494.06445312,134.1)
\lineto(494.06445312,142.68984375)
\lineto(495.77539062,142.68984375)
\lineto(497.80859375,136.6078125)
\curveto(497.99609375,136.04140625)(498.1328125,135.61757813)(498.21875,135.33632813)
\curveto(498.31640625,135.64882813)(498.46875,136.1078125)(498.67578125,136.71328125)
\lineto(500.73242188,142.68984375)
\lineto(502.26171875,142.68984375)
\lineto(502.26171875,134.1)
\lineto(501.16601562,134.1)
\lineto(501.16601562,141.28945313)
\lineto(498.66992188,134.1)
\lineto(497.64453125,134.1)
\lineto(495.16015625,141.4125)
\lineto(495.16015625,134.1)
\closepath
}
}
{
\newrgbcolor{curcolor}{0 0 0}
\pscustom[linewidth=1,linecolor=curcolor]
{
\newpath
\moveto(490,149.2)
\lineto(480.8,151.1)
}
}
{
\newrgbcolor{curcolor}{0 0 0}
\pscustom[linewidth=1,linecolor=curcolor]
{
\newpath
\moveto(218.5,204.8)
\lineto(227.8,202.9)
}
}
{
\newrgbcolor{curcolor}{0 0 0}
\pscustom[linestyle=none,fillstyle=solid,fillcolor=curcolor]
{
\newpath
\moveto(504.54101562,142.51367188)
\lineto(504.54101562,141.5)
\lineto(498.86328125,141.5)
\curveto(498.85546875,141.75390625)(498.89648438,141.99804688)(498.98632812,142.23242188)
\curveto(499.13085938,142.61914062)(499.36132812,143)(499.67773438,143.375)
\curveto(499.99804688,143.75)(500.45898438,144.18359375)(501.06054688,144.67578125)
\curveto(501.99414062,145.44140625)(502.625,146.046875)(502.953125,146.4921875)
\curveto(503.28125,146.94140625)(503.4453125,147.36523438)(503.4453125,147.76367188)
\curveto(503.4453125,148.18164062)(503.29492188,148.53320312)(502.99414062,148.81835938)
\curveto(502.69726562,149.10742188)(502.30859375,149.25195312)(501.828125,149.25195312)
\curveto(501.3203125,149.25195312)(500.9140625,149.09960938)(500.609375,148.79492188)
\curveto(500.3046875,148.49023438)(500.15039062,148.06835938)(500.14648438,147.52929688)
\lineto(499.0625,147.640625)
\curveto(499.13671875,148.44921875)(499.41601562,149.06445312)(499.90039062,149.48632812)
\curveto(500.38476562,149.91210938)(501.03515625,150.125)(501.8515625,150.125)
\curveto(502.67578125,150.125)(503.328125,149.89648438)(503.80859375,149.43945312)
\curveto(504.2890625,148.98242188)(504.52929688,148.41601562)(504.52929688,147.74023438)
\curveto(504.52929688,147.39648438)(504.45898438,147.05859375)(504.31835938,146.7265625)
\curveto(504.17773438,146.39453125)(503.94335938,146.04492188)(503.61523438,145.67773438)
\curveto(503.29101562,145.31054688)(502.75,144.80664062)(501.9921875,144.16601562)
\curveto(501.359375,143.63476562)(500.953125,143.2734375)(500.7734375,143.08203125)
\curveto(500.59375,142.89453125)(500.4453125,142.70507812)(500.328125,142.51367188)
\closepath
}
}
{
\newrgbcolor{curcolor}{0 0 0}
\pscustom[linestyle=none,fillstyle=solid,fillcolor=curcolor]
{
\newpath
\moveto(506.06445312,141.5)
\lineto(506.06445312,150.08984375)
\lineto(507.77539062,150.08984375)
\lineto(509.80859375,144.0078125)
\curveto(509.99609375,143.44140625)(510.1328125,143.01757812)(510.21875,142.73632812)
\curveto(510.31640625,143.04882812)(510.46875,143.5078125)(510.67578125,144.11328125)
\lineto(512.73242188,150.08984375)
\lineto(514.26171875,150.08984375)
\lineto(514.26171875,141.5)
\lineto(513.16601562,141.5)
\lineto(513.16601562,148.68945312)
\lineto(510.66992188,141.5)
\lineto(509.64453125,141.5)
\lineto(507.16015625,148.8125)
\lineto(507.16015625,141.5)
\closepath
}
}
{
\newrgbcolor{curcolor}{0 0 0}
\pscustom[linewidth=1,linecolor=curcolor]
{
\newpath
\moveto(502.1,156.6)
\lineto(492.8,158.5)
}
}
{
\newrgbcolor{curcolor}{0 0 0}
\pscustom[linewidth=1,linecolor=curcolor]
{
\newpath
\moveto(230.6,212.3)
\lineto(239.8,210.4)
}
}
{
\newrgbcolor{curcolor}{0 0 0}
\pscustom[linestyle=none,fillstyle=solid,fillcolor=curcolor]
{
\newpath
\moveto(514.47890625,148.9)
\lineto(514.47890625,150.95664062)
\lineto(510.75234375,150.95664062)
\lineto(510.75234375,151.9234375)
\lineto(514.67226563,157.48984375)
\lineto(515.53359375,157.48984375)
\lineto(515.53359375,151.9234375)
\lineto(516.69375,151.9234375)
\lineto(516.69375,150.95664062)
\lineto(515.53359375,150.95664062)
\lineto(515.53359375,148.9)
\closepath
\moveto(514.47890625,151.9234375)
\lineto(514.47890625,155.79648437)
\lineto(511.78945313,151.9234375)
\closepath
}
}
{
\newrgbcolor{curcolor}{0 0 0}
\pscustom[linestyle=none,fillstyle=solid,fillcolor=curcolor]
{
\newpath
\moveto(518.16445313,148.9)
\lineto(518.16445313,157.48984375)
\lineto(519.87539063,157.48984375)
\lineto(521.90859375,151.4078125)
\curveto(522.09609375,150.84140625)(522.2328125,150.41757812)(522.31875,150.13632812)
\curveto(522.41640625,150.44882812)(522.56875,150.9078125)(522.77578125,151.51328125)
\lineto(524.83242188,157.48984375)
\lineto(526.36171875,157.48984375)
\lineto(526.36171875,148.9)
\lineto(525.26601563,148.9)
\lineto(525.26601563,156.08945312)
\lineto(522.76992188,148.9)
\lineto(521.74453125,148.9)
\lineto(519.26015625,156.2125)
\lineto(519.26015625,148.9)
\closepath
}
}
{
\newrgbcolor{curcolor}{0 0 0}
\pscustom[linewidth=1,linecolor=curcolor]
{
\newpath
\moveto(514.1,164)
\lineto(504.9,165.9)
}
}
{
\newrgbcolor{curcolor}{0 0 0}
\pscustom[linewidth=1,linecolor=curcolor]
{
\newpath
\moveto(242.7,219.7)
\lineto(251.9,217.8)
}
}
{
\newrgbcolor{curcolor}{0 0 0}
\pscustom[linestyle=none,fillstyle=solid,fillcolor=curcolor]
{
\newpath
\moveto(524.82109375,160.95820313)
\curveto(524.38359375,161.11835938)(524.059375,161.346875)(523.8484375,161.64375)
\curveto(523.6375,161.940625)(523.53203125,162.29609375)(523.53203125,162.71015625)
\curveto(523.53203125,163.33515625)(523.75664063,163.86054688)(524.20585938,164.28632813)
\curveto(524.65507813,164.71210938)(525.25273438,164.925)(525.99882813,164.925)
\curveto(526.74882813,164.925)(527.35234375,164.70625)(527.809375,164.26875)
\curveto(528.26640625,163.83515625)(528.49492188,163.30585938)(528.49492188,162.68085938)
\curveto(528.49492188,162.28242188)(528.38945313,161.93476563)(528.17851563,161.63789063)
\curveto(527.97148438,161.34492188)(527.65507813,161.11835938)(527.22929688,160.95820313)
\curveto(527.75664063,160.78632813)(528.15703125,160.50898438)(528.43046875,160.12617188)
\curveto(528.7078125,159.74335938)(528.84648438,159.28632813)(528.84648438,158.75507813)
\curveto(528.84648438,158.02070313)(528.58671875,157.40351563)(528.0671875,156.90351563)
\curveto(527.54765625,156.40351563)(526.8640625,156.15351563)(526.01640625,156.15351563)
\curveto(525.16875,156.15351563)(524.48515625,156.40351563)(523.965625,156.90351563)
\curveto(523.44609375,157.40742188)(523.18632813,158.034375)(523.18632813,158.784375)
\curveto(523.18632813,159.34296875)(523.32695313,159.80976563)(523.60820313,160.18476563)
\curveto(523.89335938,160.56367188)(524.29765625,160.82148438)(524.82109375,160.95820313)
\closepath
\moveto(524.61015625,162.7453125)
\curveto(524.61015625,162.3390625)(524.74101563,162.00703125)(525.00273438,161.74921875)
\curveto(525.26445313,161.49140625)(525.60429688,161.3625)(526.02226563,161.3625)
\curveto(526.42851563,161.3625)(526.76054688,161.48945313)(527.01835938,161.74335938)
\curveto(527.28007813,162.00117188)(527.4109375,162.315625)(527.4109375,162.68671875)
\curveto(527.4109375,163.0734375)(527.27617188,163.39765625)(527.00664063,163.659375)
\curveto(526.74101563,163.925)(526.40898438,164.0578125)(526.01054688,164.0578125)
\curveto(525.60820313,164.0578125)(525.27421875,163.92890625)(525.00859375,163.67109375)
\curveto(524.74296875,163.41328125)(524.61015625,163.1046875)(524.61015625,162.7453125)
\closepath
\moveto(524.2703125,158.77851563)
\curveto(524.2703125,158.47773438)(524.340625,158.18671875)(524.48125,157.90546875)
\curveto(524.62578125,157.62421875)(524.83867188,157.40546875)(525.11992188,157.24921875)
\curveto(525.40117188,157.096875)(525.70390625,157.02070313)(526.028125,157.02070313)
\curveto(526.53203125,157.02070313)(526.94804688,157.1828125)(527.27617188,157.50703125)
\curveto(527.60429688,157.83125)(527.76835938,158.24335938)(527.76835938,158.74335938)
\curveto(527.76835938,159.25117188)(527.5984375,159.67109375)(527.25859375,160.003125)
\curveto(526.92265625,160.33515625)(526.50078125,160.50117188)(525.99296875,160.50117188)
\curveto(525.496875,160.50117188)(525.08476563,160.33710938)(524.75664063,160.00898438)
\curveto(524.43242188,159.68085938)(524.2703125,159.27070313)(524.2703125,158.77851563)
\closepath
}
}
{
\newrgbcolor{curcolor}{0 0 0}
\pscustom[linestyle=none,fillstyle=solid,fillcolor=curcolor]
{
\newpath
\moveto(530.26445313,156.3)
\lineto(530.26445313,164.88984375)
\lineto(531.97539063,164.88984375)
\lineto(534.00859375,158.8078125)
\curveto(534.19609375,158.24140625)(534.3328125,157.81757813)(534.41875,157.53632813)
\curveto(534.51640625,157.84882813)(534.66875,158.3078125)(534.87578125,158.91328125)
\lineto(536.93242188,164.88984375)
\lineto(538.46171875,164.88984375)
\lineto(538.46171875,156.3)
\lineto(537.36601563,156.3)
\lineto(537.36601563,163.48945313)
\lineto(534.86992188,156.3)
\lineto(533.84453125,156.3)
\lineto(531.36015625,163.6125)
\lineto(531.36015625,156.3)
\closepath
}
}
{
\newrgbcolor{curcolor}{0 0 0}
\pscustom[linewidth=1,linecolor=curcolor]
{
\newpath
\moveto(85.9,187.5)
\lineto(94.9,187.5)
}
}
{
\newrgbcolor{curcolor}{0 0 0}
\pscustom[linestyle=none,fillstyle=solid,fillcolor=curcolor]
{
\newpath
\moveto(50.93710938,186.178125)
\lineto(50.93710938,187.23867188)
\lineto(54.17734375,187.23867188)
\lineto(54.17734375,186.178125)
\closepath
}
}
{
\newrgbcolor{curcolor}{0 0 0}
\pscustom[linestyle=none,fillstyle=solid,fillcolor=curcolor]
{
\newpath
\moveto(59.02304688,183.6)
\lineto(57.96835938,183.6)
\lineto(57.96835938,190.32070313)
\curveto(57.71445313,190.07851563)(57.38046875,189.83632813)(56.96640625,189.59414063)
\curveto(56.55625,189.35195313)(56.18710938,189.1703125)(55.85898438,189.04921875)
\lineto(55.85898438,190.06875)
\curveto(56.44882813,190.34609375)(56.96445313,190.68203125)(57.40585938,191.0765625)
\curveto(57.84726563,191.47109375)(58.15976563,191.85390625)(58.34335938,192.225)
\lineto(59.02304688,192.225)
\closepath
}
}
{
\newrgbcolor{curcolor}{0 0 0}
\pscustom[linestyle=none,fillstyle=solid,fillcolor=curcolor]
{
\newpath
\moveto(61.72421875,187.83632813)
\curveto(61.72421875,188.85195313)(61.82773438,189.66835938)(62.03476563,190.28554688)
\curveto(62.24570313,190.90664063)(62.55625,191.38515625)(62.96640625,191.72109375)
\curveto(63.38046875,192.05703125)(63.9,192.225)(64.525,192.225)
\curveto(64.9859375,192.225)(65.39023438,192.13125)(65.73789063,191.94375)
\curveto(66.08554688,191.76015625)(66.37265625,191.49257813)(66.59921875,191.14101563)
\curveto(66.82578125,190.79335938)(67.00351563,190.36757813)(67.13242188,189.86367188)
\curveto(67.26132813,189.36367188)(67.32578125,188.68789063)(67.32578125,187.83632813)
\curveto(67.32578125,186.82851563)(67.22226563,186.0140625)(67.01523438,185.39296875)
\curveto(66.80820313,184.77578125)(66.49765625,184.29726563)(66.08359375,183.95742188)
\curveto(65.6734375,183.62148438)(65.15390625,183.45351563)(64.525,183.45351563)
\curveto(63.696875,183.45351563)(63.04648438,183.75039063)(62.57382813,184.34414063)
\curveto(62.00742188,185.05898438)(61.72421875,186.22304688)(61.72421875,187.83632813)
\closepath
\moveto(62.80820313,187.83632813)
\curveto(62.80820313,186.42617188)(62.97226563,185.48671875)(63.30039063,185.01796875)
\curveto(63.63242188,184.553125)(64.040625,184.32070313)(64.525,184.32070313)
\curveto(65.009375,184.32070313)(65.415625,184.55507813)(65.74375,185.02382813)
\curveto(66.07578125,185.49257813)(66.24179688,186.43007813)(66.24179688,187.83632813)
\curveto(66.24179688,189.25039063)(66.07578125,190.18984375)(65.74375,190.6546875)
\curveto(65.415625,191.11953125)(65.00546875,191.35195313)(64.51328125,191.35195313)
\curveto(64.02890625,191.35195313)(63.6421875,191.146875)(63.353125,190.73671875)
\curveto(62.98984375,190.21328125)(62.80820313,189.24648438)(62.80820313,187.83632813)
\closepath
}
}
{
\newrgbcolor{curcolor}{0 0 0}
\pscustom[linewidth=1,linecolor=curcolor]
{
\newpath
\moveto(85.9,209)
\lineto(94.9,209)
}
}
{
\newrgbcolor{curcolor}{0 0 0}
\pscustom[linestyle=none,fillstyle=solid,fillcolor=curcolor]
{
\newpath
\moveto(57.6109375,207.678125)
\lineto(57.6109375,208.73867188)
\lineto(60.85117188,208.73867188)
\lineto(60.85117188,207.678125)
\closepath
}
}
{
\newrgbcolor{curcolor}{0 0 0}
\pscustom[linestyle=none,fillstyle=solid,fillcolor=curcolor]
{
\newpath
\moveto(61.72421875,207.35)
\lineto(62.83164063,207.44375)
\curveto(62.91367188,206.9046875)(63.103125,206.4984375)(63.4,206.225)
\curveto(63.70078125,205.95546875)(64.06210938,205.82070313)(64.48398438,205.82070313)
\curveto(64.99179688,205.82070313)(65.42148438,206.01210938)(65.77304688,206.39492188)
\curveto(66.12460938,206.77773438)(66.30039063,207.28554688)(66.30039063,207.91835938)
\curveto(66.30039063,208.51992188)(66.13046875,208.99453125)(65.790625,209.3421875)
\curveto(65.4546875,209.68984375)(65.01328125,209.86367188)(64.46640625,209.86367188)
\curveto(64.1265625,209.86367188)(63.81992188,209.78554688)(63.54648438,209.62929688)
\curveto(63.27304688,209.47695313)(63.05820313,209.27773438)(62.90195313,209.03164063)
\lineto(61.91171875,209.16054688)
\lineto(62.74375,213.57265625)
\lineto(67.01523438,213.57265625)
\lineto(67.01523438,212.56484375)
\lineto(63.5875,212.56484375)
\lineto(63.12460938,210.25625)
\curveto(63.64023438,210.615625)(64.18125,210.7953125)(64.74765625,210.7953125)
\curveto(65.49765625,210.7953125)(66.13046875,210.53554688)(66.64609375,210.01601563)
\curveto(67.16171875,209.49648438)(67.41953125,208.82851563)(67.41953125,208.01210938)
\curveto(67.41953125,207.23476563)(67.19296875,206.56289063)(66.73984375,205.99648438)
\curveto(66.1890625,205.30117188)(65.43710938,204.95351563)(64.48398438,204.95351563)
\curveto(63.70273438,204.95351563)(63.0640625,205.17226563)(62.56796875,205.60976563)
\curveto(62.07578125,206.04726563)(61.79453125,206.62734375)(61.72421875,207.35)
\closepath
}
}
{
\newrgbcolor{curcolor}{0 0 0}
\pscustom[linewidth=1,linecolor=curcolor]
{
\newpath
\moveto(85.9,230.4)
\lineto(94.9,230.4)
}
}
{
\newrgbcolor{curcolor}{0 0 0}
\pscustom[linestyle=none,fillstyle=solid,fillcolor=curcolor]
{
\newpath
\moveto(61.72421875,230.73632812)
\curveto(61.72421875,231.75195312)(61.82773438,232.56835938)(62.03476563,233.18554688)
\curveto(62.24570313,233.80664062)(62.55625,234.28515625)(62.96640625,234.62109375)
\curveto(63.38046875,234.95703125)(63.9,235.125)(64.525,235.125)
\curveto(64.9859375,235.125)(65.39023438,235.03125)(65.73789063,234.84375)
\curveto(66.08554688,234.66015625)(66.37265625,234.39257812)(66.59921875,234.04101562)
\curveto(66.82578125,233.69335938)(67.00351563,233.26757812)(67.13242188,232.76367188)
\curveto(67.26132813,232.26367188)(67.32578125,231.58789062)(67.32578125,230.73632812)
\curveto(67.32578125,229.72851562)(67.22226563,228.9140625)(67.01523438,228.29296875)
\curveto(66.80820313,227.67578125)(66.49765625,227.19726562)(66.08359375,226.85742188)
\curveto(65.6734375,226.52148438)(65.15390625,226.35351562)(64.525,226.35351562)
\curveto(63.696875,226.35351562)(63.04648438,226.65039062)(62.57382813,227.24414062)
\curveto(62.00742188,227.95898438)(61.72421875,229.12304688)(61.72421875,230.73632812)
\closepath
\moveto(62.80820313,230.73632812)
\curveto(62.80820313,229.32617188)(62.97226563,228.38671875)(63.30039063,227.91796875)
\curveto(63.63242188,227.453125)(64.040625,227.22070312)(64.525,227.22070312)
\curveto(65.009375,227.22070312)(65.415625,227.45507812)(65.74375,227.92382812)
\curveto(66.07578125,228.39257812)(66.24179688,229.33007812)(66.24179688,230.73632812)
\curveto(66.24179688,232.15039062)(66.07578125,233.08984375)(65.74375,233.5546875)
\curveto(65.415625,234.01953125)(65.00546875,234.25195312)(64.51328125,234.25195312)
\curveto(64.02890625,234.25195312)(63.6421875,234.046875)(63.353125,233.63671875)
\curveto(62.98984375,233.11328125)(62.80820313,232.14648438)(62.80820313,230.73632812)
\closepath
}
}
{
\newrgbcolor{curcolor}{0 0 0}
\pscustom[linewidth=1,linecolor=curcolor]
{
\newpath
\moveto(85.9,251.7)
\lineto(94.9,251.7)
}
}
{
\newrgbcolor{curcolor}{0 0 0}
\pscustom[linestyle=none,fillstyle=solid,fillcolor=curcolor]
{
\newpath
\moveto(61.72421875,250.05)
\lineto(62.83164063,250.14375)
\curveto(62.91367188,249.6046875)(63.103125,249.1984375)(63.4,248.925)
\curveto(63.70078125,248.65546875)(64.06210938,248.52070313)(64.48398438,248.52070313)
\curveto(64.99179688,248.52070313)(65.42148438,248.71210938)(65.77304688,249.09492188)
\curveto(66.12460938,249.47773438)(66.30039063,249.98554688)(66.30039063,250.61835938)
\curveto(66.30039063,251.21992188)(66.13046875,251.69453125)(65.790625,252.0421875)
\curveto(65.4546875,252.38984375)(65.01328125,252.56367188)(64.46640625,252.56367188)
\curveto(64.1265625,252.56367188)(63.81992188,252.48554688)(63.54648438,252.32929688)
\curveto(63.27304688,252.17695313)(63.05820313,251.97773438)(62.90195313,251.73164063)
\lineto(61.91171875,251.86054688)
\lineto(62.74375,256.27265625)
\lineto(67.01523438,256.27265625)
\lineto(67.01523438,255.26484375)
\lineto(63.5875,255.26484375)
\lineto(63.12460938,252.95625)
\curveto(63.64023438,253.315625)(64.18125,253.4953125)(64.74765625,253.4953125)
\curveto(65.49765625,253.4953125)(66.13046875,253.23554688)(66.64609375,252.71601563)
\curveto(67.16171875,252.19648438)(67.41953125,251.52851563)(67.41953125,250.71210938)
\curveto(67.41953125,249.93476563)(67.19296875,249.26289063)(66.73984375,248.69648438)
\curveto(66.1890625,248.00117188)(65.43710938,247.65351563)(64.48398438,247.65351563)
\curveto(63.70273438,247.65351563)(63.0640625,247.87226563)(62.56796875,248.30976563)
\curveto(62.07578125,248.74726563)(61.79453125,249.32734375)(61.72421875,250.05)
\closepath
}
}
{
\newrgbcolor{curcolor}{0 0 0}
\pscustom[linewidth=1,linecolor=curcolor]
{
\newpath
\moveto(85.9,273.1)
\lineto(94.9,273.1)
}
}
{
\newrgbcolor{curcolor}{0 0 0}
\pscustom[linestyle=none,fillstyle=solid,fillcolor=curcolor]
{
\newpath
\moveto(59.02304688,269.2)
\lineto(57.96835938,269.2)
\lineto(57.96835938,275.92070312)
\curveto(57.71445313,275.67851562)(57.38046875,275.43632812)(56.96640625,275.19414062)
\curveto(56.55625,274.95195312)(56.18710938,274.7703125)(55.85898438,274.64921875)
\lineto(55.85898438,275.66875)
\curveto(56.44882813,275.94609375)(56.96445313,276.28203125)(57.40585938,276.6765625)
\curveto(57.84726563,277.07109375)(58.15976563,277.45390625)(58.34335938,277.825)
\lineto(59.02304688,277.825)
\closepath
}
}
{
\newrgbcolor{curcolor}{0 0 0}
\pscustom[linestyle=none,fillstyle=solid,fillcolor=curcolor]
{
\newpath
\moveto(61.72421875,273.43632812)
\curveto(61.72421875,274.45195312)(61.82773438,275.26835937)(62.03476563,275.88554687)
\curveto(62.24570313,276.50664062)(62.55625,276.98515625)(62.96640625,277.32109375)
\curveto(63.38046875,277.65703125)(63.9,277.825)(64.525,277.825)
\curveto(64.9859375,277.825)(65.39023438,277.73125)(65.73789063,277.54375)
\curveto(66.08554688,277.36015625)(66.37265625,277.09257812)(66.59921875,276.74101562)
\curveto(66.82578125,276.39335937)(67.00351563,275.96757812)(67.13242188,275.46367187)
\curveto(67.26132813,274.96367187)(67.32578125,274.28789062)(67.32578125,273.43632812)
\curveto(67.32578125,272.42851562)(67.22226563,271.6140625)(67.01523438,270.99296875)
\curveto(66.80820313,270.37578125)(66.49765625,269.89726562)(66.08359375,269.55742187)
\curveto(65.6734375,269.22148437)(65.15390625,269.05351562)(64.525,269.05351562)
\curveto(63.696875,269.05351562)(63.04648438,269.35039062)(62.57382813,269.94414062)
\curveto(62.00742188,270.65898437)(61.72421875,271.82304687)(61.72421875,273.43632812)
\closepath
\moveto(62.80820313,273.43632812)
\curveto(62.80820313,272.02617187)(62.97226563,271.08671875)(63.30039063,270.61796875)
\curveto(63.63242188,270.153125)(64.040625,269.92070312)(64.525,269.92070312)
\curveto(65.009375,269.92070312)(65.415625,270.15507812)(65.74375,270.62382812)
\curveto(66.07578125,271.09257812)(66.24179688,272.03007812)(66.24179688,273.43632812)
\curveto(66.24179688,274.85039062)(66.07578125,275.78984375)(65.74375,276.2546875)
\curveto(65.415625,276.71953125)(65.00546875,276.95195312)(64.51328125,276.95195312)
\curveto(64.02890625,276.95195312)(63.6421875,276.746875)(63.353125,276.33671875)
\curveto(62.98984375,275.81328125)(62.80820313,274.84648437)(62.80820313,273.43632812)
\closepath
}
}
{
\newrgbcolor{curcolor}{0 0 0}
\pscustom[linewidth=1,linecolor=curcolor]
{
\newpath
\moveto(85.9,294.6)
\lineto(94.9,294.6)
}
}
{
\newrgbcolor{curcolor}{0 0 0}
\pscustom[linestyle=none,fillstyle=solid,fillcolor=curcolor]
{
\newpath
\moveto(59.02304688,290.7)
\lineto(57.96835938,290.7)
\lineto(57.96835938,297.42070312)
\curveto(57.71445313,297.17851562)(57.38046875,296.93632812)(56.96640625,296.69414062)
\curveto(56.55625,296.45195312)(56.18710938,296.2703125)(55.85898438,296.14921875)
\lineto(55.85898438,297.16875)
\curveto(56.44882813,297.44609375)(56.96445313,297.78203125)(57.40585938,298.1765625)
\curveto(57.84726563,298.57109375)(58.15976563,298.95390625)(58.34335938,299.325)
\lineto(59.02304688,299.325)
\closepath
}
}
{
\newrgbcolor{curcolor}{0 0 0}
\pscustom[linestyle=none,fillstyle=solid,fillcolor=curcolor]
{
\newpath
\moveto(61.72421875,292.95)
\lineto(62.83164063,293.04375)
\curveto(62.91367188,292.5046875)(63.103125,292.0984375)(63.4,291.825)
\curveto(63.70078125,291.55546875)(64.06210938,291.42070312)(64.48398438,291.42070312)
\curveto(64.99179688,291.42070312)(65.42148438,291.61210937)(65.77304688,291.99492187)
\curveto(66.12460938,292.37773437)(66.30039063,292.88554687)(66.30039063,293.51835937)
\curveto(66.30039063,294.11992187)(66.13046875,294.59453125)(65.790625,294.9421875)
\curveto(65.4546875,295.28984375)(65.01328125,295.46367187)(64.46640625,295.46367187)
\curveto(64.1265625,295.46367187)(63.81992188,295.38554687)(63.54648438,295.22929687)
\curveto(63.27304688,295.07695312)(63.05820313,294.87773437)(62.90195313,294.63164062)
\lineto(61.91171875,294.76054687)
\lineto(62.74375,299.17265625)
\lineto(67.01523438,299.17265625)
\lineto(67.01523438,298.16484375)
\lineto(63.5875,298.16484375)
\lineto(63.12460938,295.85625)
\curveto(63.64023438,296.215625)(64.18125,296.3953125)(64.74765625,296.3953125)
\curveto(65.49765625,296.3953125)(66.13046875,296.13554687)(66.64609375,295.61601562)
\curveto(67.16171875,295.09648437)(67.41953125,294.42851562)(67.41953125,293.61210937)
\curveto(67.41953125,292.83476562)(67.19296875,292.16289062)(66.73984375,291.59648437)
\curveto(66.1890625,290.90117187)(65.43710938,290.55351562)(64.48398438,290.55351562)
\curveto(63.70273438,290.55351562)(63.0640625,290.77226562)(62.56796875,291.20976562)
\curveto(62.07578125,291.64726562)(61.79453125,292.22734375)(61.72421875,292.95)
\closepath
}
}
{
\newrgbcolor{curcolor}{0 0 0}
\pscustom[linewidth=1,linecolor=curcolor]
{
\newpath
\moveto(85.9,316)
\lineto(94.9,316)
}
}
{
\newrgbcolor{curcolor}{0 0 0}
\pscustom[linestyle=none,fillstyle=solid,fillcolor=curcolor]
{
\newpath
\moveto(60.59335938,313.11367188)
\lineto(60.59335938,312.1)
\lineto(54.915625,312.1)
\curveto(54.9078125,312.35390625)(54.94882813,312.59804688)(55.03867188,312.83242188)
\curveto(55.18320313,313.21914063)(55.41367188,313.6)(55.73007813,313.975)
\curveto(56.05039063,314.35)(56.51132813,314.78359375)(57.11289063,315.27578125)
\curveto(58.04648438,316.04140625)(58.67734375,316.646875)(59.00546875,317.0921875)
\curveto(59.33359375,317.54140625)(59.49765625,317.96523438)(59.49765625,318.36367188)
\curveto(59.49765625,318.78164063)(59.34726563,319.13320313)(59.04648438,319.41835938)
\curveto(58.74960938,319.70742188)(58.3609375,319.85195313)(57.88046875,319.85195313)
\curveto(57.37265625,319.85195313)(56.96640625,319.69960938)(56.66171875,319.39492188)
\curveto(56.35703125,319.09023438)(56.20273438,318.66835938)(56.19882813,318.12929688)
\lineto(55.11484375,318.240625)
\curveto(55.1890625,319.04921875)(55.46835938,319.66445313)(55.95273438,320.08632813)
\curveto(56.43710938,320.51210938)(57.0875,320.725)(57.90390625,320.725)
\curveto(58.728125,320.725)(59.38046875,320.49648438)(59.8609375,320.03945313)
\curveto(60.34140625,319.58242188)(60.58164063,319.01601563)(60.58164063,318.34023438)
\curveto(60.58164063,317.99648438)(60.51132813,317.65859375)(60.37070313,317.3265625)
\curveto(60.23007813,316.99453125)(59.99570313,316.64492188)(59.66757813,316.27773438)
\curveto(59.34335938,315.91054688)(58.80234375,315.40664063)(58.04453125,314.76601563)
\curveto(57.41171875,314.23476563)(57.00546875,313.8734375)(56.82578125,313.68203125)
\curveto(56.64609375,313.49453125)(56.49765625,313.30507813)(56.38046875,313.11367188)
\closepath
}
}
{
\newrgbcolor{curcolor}{0 0 0}
\pscustom[linestyle=none,fillstyle=solid,fillcolor=curcolor]
{
\newpath
\moveto(61.72421875,316.33632813)
\curveto(61.72421875,317.35195313)(61.82773438,318.16835938)(62.03476563,318.78554688)
\curveto(62.24570313,319.40664063)(62.55625,319.88515625)(62.96640625,320.22109375)
\curveto(63.38046875,320.55703125)(63.9,320.725)(64.525,320.725)
\curveto(64.9859375,320.725)(65.39023438,320.63125)(65.73789063,320.44375)
\curveto(66.08554688,320.26015625)(66.37265625,319.99257813)(66.59921875,319.64101563)
\curveto(66.82578125,319.29335938)(67.00351563,318.86757813)(67.13242188,318.36367188)
\curveto(67.26132813,317.86367188)(67.32578125,317.18789063)(67.32578125,316.33632813)
\curveto(67.32578125,315.32851563)(67.22226563,314.5140625)(67.01523438,313.89296875)
\curveto(66.80820313,313.27578125)(66.49765625,312.79726563)(66.08359375,312.45742188)
\curveto(65.6734375,312.12148438)(65.15390625,311.95351563)(64.525,311.95351563)
\curveto(63.696875,311.95351563)(63.04648438,312.25039063)(62.57382813,312.84414063)
\curveto(62.00742188,313.55898438)(61.72421875,314.72304688)(61.72421875,316.33632813)
\closepath
\moveto(62.80820313,316.33632813)
\curveto(62.80820313,314.92617188)(62.97226563,313.98671875)(63.30039063,313.51796875)
\curveto(63.63242188,313.053125)(64.040625,312.82070313)(64.525,312.82070313)
\curveto(65.009375,312.82070313)(65.415625,313.05507813)(65.74375,313.52382813)
\curveto(66.07578125,313.99257813)(66.24179688,314.93007813)(66.24179688,316.33632813)
\curveto(66.24179688,317.75039063)(66.07578125,318.68984375)(65.74375,319.1546875)
\curveto(65.415625,319.61953125)(65.00546875,319.85195313)(64.51328125,319.85195313)
\curveto(64.02890625,319.85195313)(63.6421875,319.646875)(63.353125,319.23671875)
\curveto(62.98984375,318.71328125)(62.80820313,317.74648438)(62.80820313,316.33632813)
\closepath
}
}
{
\newrgbcolor{curcolor}{0 0 0}
\pscustom[linestyle=none,fillstyle=solid,fillcolor=curcolor]
{
\newpath
\moveto(168.75585938,441.55)
\lineto(169.86328125,441.64375)
\curveto(169.9453125,441.1046875)(170.13476562,440.6984375)(170.43164062,440.425)
\curveto(170.73242188,440.15546875)(171.09375,440.02070313)(171.515625,440.02070313)
\curveto(172.0234375,440.02070313)(172.453125,440.21210938)(172.8046875,440.59492188)
\curveto(173.15625,440.97773438)(173.33203125,441.48554688)(173.33203125,442.11835938)
\curveto(173.33203125,442.71992188)(173.16210938,443.19453125)(172.82226562,443.5421875)
\curveto(172.48632812,443.88984375)(172.04492188,444.06367188)(171.49804688,444.06367188)
\curveto(171.15820312,444.06367188)(170.8515625,443.98554688)(170.578125,443.82929688)
\curveto(170.3046875,443.67695313)(170.08984375,443.47773438)(169.93359375,443.23164063)
\lineto(168.94335938,443.36054688)
\lineto(169.77539062,447.77265625)
\lineto(174.046875,447.77265625)
\lineto(174.046875,446.76484375)
\lineto(170.61914062,446.76484375)
\lineto(170.15625,444.45625)
\curveto(170.671875,444.815625)(171.21289062,444.9953125)(171.77929688,444.9953125)
\curveto(172.52929688,444.9953125)(173.16210938,444.73554688)(173.67773438,444.21601563)
\curveto(174.19335938,443.69648438)(174.45117188,443.02851563)(174.45117188,442.21210938)
\curveto(174.45117188,441.43476563)(174.22460938,440.76289063)(173.77148438,440.19648438)
\curveto(173.22070312,439.50117188)(172.46875,439.15351563)(171.515625,439.15351563)
\curveto(170.734375,439.15351563)(170.09570312,439.37226563)(169.59960938,439.80976563)
\curveto(169.10742188,440.24726563)(168.82617188,440.82734375)(168.75585938,441.55)
\closepath
}
}
{
\newrgbcolor{curcolor}{0 0 0}
\pscustom[linestyle=none,fillstyle=solid,fillcolor=curcolor]
{
\newpath
\moveto(175.4296875,443.53632813)
\curveto(175.4296875,444.55195313)(175.53320312,445.36835938)(175.74023438,445.98554688)
\curveto(175.95117188,446.60664063)(176.26171875,447.08515625)(176.671875,447.42109375)
\curveto(177.0859375,447.75703125)(177.60546875,447.925)(178.23046875,447.925)
\curveto(178.69140625,447.925)(179.09570312,447.83125)(179.44335938,447.64375)
\curveto(179.79101562,447.46015625)(180.078125,447.19257813)(180.3046875,446.84101563)
\curveto(180.53125,446.49335938)(180.70898438,446.06757813)(180.83789062,445.56367188)
\curveto(180.96679688,445.06367188)(181.03125,444.38789063)(181.03125,443.53632813)
\curveto(181.03125,442.52851563)(180.92773438,441.7140625)(180.72070312,441.09296875)
\curveto(180.51367188,440.47578125)(180.203125,439.99726563)(179.7890625,439.65742188)
\curveto(179.37890625,439.32148438)(178.859375,439.15351563)(178.23046875,439.15351563)
\curveto(177.40234375,439.15351563)(176.75195312,439.45039063)(176.27929688,440.04414063)
\curveto(175.71289062,440.75898438)(175.4296875,441.92304688)(175.4296875,443.53632813)
\closepath
\moveto(176.51367188,443.53632813)
\curveto(176.51367188,442.12617188)(176.67773438,441.18671875)(177.00585938,440.71796875)
\curveto(177.33789062,440.253125)(177.74609375,440.02070313)(178.23046875,440.02070313)
\curveto(178.71484375,440.02070313)(179.12109375,440.25507813)(179.44921875,440.72382813)
\curveto(179.78125,441.19257813)(179.94726562,442.13007813)(179.94726562,443.53632813)
\curveto(179.94726562,444.95039063)(179.78125,445.88984375)(179.44921875,446.3546875)
\curveto(179.12109375,446.81953125)(178.7109375,447.05195313)(178.21875,447.05195313)
\curveto(177.734375,447.05195313)(177.34765625,446.846875)(177.05859375,446.43671875)
\curveto(176.6953125,445.91328125)(176.51367188,444.94648438)(176.51367188,443.53632813)
\closepath
}
}
{
\newrgbcolor{curcolor}{0 0 0}
\pscustom[linestyle=none,fillstyle=solid,fillcolor=curcolor]
{
\newpath
\moveto(184.69921875,440.24335938)
\lineto(184.8515625,439.31171875)
\curveto(184.5546875,439.24921875)(184.2890625,439.21796875)(184.0546875,439.21796875)
\curveto(183.671875,439.21796875)(183.375,439.27851563)(183.1640625,439.39960938)
\curveto(182.953125,439.52070313)(182.8046875,439.67890625)(182.71875,439.87421875)
\curveto(182.6328125,440.0734375)(182.58984375,440.48945313)(182.58984375,441.12226563)
\lineto(182.58984375,444.70234375)
\lineto(181.81640625,444.70234375)
\lineto(181.81640625,445.52265625)
\lineto(182.58984375,445.52265625)
\lineto(182.58984375,447.06367188)
\lineto(183.63867188,447.69648438)
\lineto(183.63867188,445.52265625)
\lineto(184.69921875,445.52265625)
\lineto(184.69921875,444.70234375)
\lineto(183.63867188,444.70234375)
\lineto(183.63867188,441.06367188)
\curveto(183.63867188,440.76289063)(183.65625,440.56953125)(183.69140625,440.48359375)
\curveto(183.73046875,440.39765625)(183.79101562,440.32929688)(183.87304688,440.27851563)
\curveto(183.95898438,440.22773438)(184.08007812,440.20234375)(184.23632812,440.20234375)
\curveto(184.35351562,440.20234375)(184.5078125,440.21601563)(184.69921875,440.24335938)
\closepath
}
}
{
\newrgbcolor{curcolor}{0 0 0}
\pscustom[linestyle=none,fillstyle=solid,fillcolor=curcolor]
{
\newpath
\moveto(185.73046875,439.3)
\lineto(185.73046875,447.88984375)
\lineto(186.78515625,447.88984375)
\lineto(186.78515625,444.8078125)
\curveto(187.27734375,445.378125)(187.8984375,445.66328125)(188.6484375,445.66328125)
\curveto(189.109375,445.66328125)(189.50976562,445.57148438)(189.84960938,445.38789063)
\curveto(190.18945312,445.20820313)(190.43164062,444.95820313)(190.57617188,444.63789063)
\curveto(190.72460938,444.31757813)(190.79882812,443.85273438)(190.79882812,443.24335938)
\lineto(190.79882812,439.3)
\lineto(189.74414062,439.3)
\lineto(189.74414062,443.24335938)
\curveto(189.74414062,443.77070313)(189.62890625,444.15351563)(189.3984375,444.39179688)
\curveto(189.171875,444.63398438)(188.84960938,444.75507813)(188.43164062,444.75507813)
\curveto(188.11914062,444.75507813)(187.82421875,444.67304688)(187.546875,444.50898438)
\curveto(187.2734375,444.34882813)(187.078125,444.13007813)(186.9609375,443.85273438)
\curveto(186.84375,443.57539063)(186.78515625,443.19257813)(186.78515625,442.70429688)
\lineto(186.78515625,439.3)
\closepath
}
}
{
\newrgbcolor{curcolor}{0 0 0}
\pscustom[linestyle=none,fillstyle=solid,fillcolor=curcolor]
{
\newpath
\moveto(195.87304688,439.3)
\lineto(195.87304688,447.88984375)
\lineto(199.11328125,447.88984375)
\curveto(199.68359375,447.88984375)(200.11914062,447.8625)(200.41992188,447.8078125)
\curveto(200.84179688,447.7375)(201.1953125,447.60273438)(201.48046875,447.40351563)
\curveto(201.765625,447.20820313)(201.99414062,446.9328125)(202.16601562,446.57734375)
\curveto(202.34179688,446.221875)(202.4296875,445.83125)(202.4296875,445.40546875)
\curveto(202.4296875,444.675)(202.19726562,444.05585938)(201.73242188,443.54804688)
\curveto(201.26757812,443.04414063)(200.42773438,442.7921875)(199.21289062,442.7921875)
\lineto(197.00976562,442.7921875)
\lineto(197.00976562,439.3)
\closepath
\moveto(197.00976562,443.80585938)
\lineto(199.23046875,443.80585938)
\curveto(199.96484375,443.80585938)(200.48632812,443.94257813)(200.79492188,444.21601563)
\curveto(201.10351562,444.48945313)(201.2578125,444.87421875)(201.2578125,445.3703125)
\curveto(201.2578125,445.7296875)(201.16601562,446.03632813)(200.98242188,446.29023438)
\curveto(200.80273438,446.54804688)(200.56445312,446.71796875)(200.26757812,446.8)
\curveto(200.07617188,446.85078125)(199.72265625,446.87617188)(199.20703125,446.87617188)
\lineto(197.00976562,446.87617188)
\closepath
}
}
{
\newrgbcolor{curcolor}{0 0 0}
\pscustom[linestyle=none,fillstyle=solid,fillcolor=curcolor]
{
\newpath
\moveto(208.00195312,441.30390625)
\lineto(209.09179688,441.16914063)
\curveto(208.91992188,440.53242188)(208.6015625,440.03828125)(208.13671875,439.68671875)
\curveto(207.671875,439.33515625)(207.078125,439.159375)(206.35546875,439.159375)
\curveto(205.4453125,439.159375)(204.72265625,439.43867188)(204.1875,439.99726563)
\curveto(203.65625,440.55976563)(203.390625,441.346875)(203.390625,442.35859375)
\curveto(203.390625,443.40546875)(203.66015625,444.21796875)(204.19921875,444.79609375)
\curveto(204.73828125,445.37421875)(205.4375,445.66328125)(206.296875,445.66328125)
\curveto(207.12890625,445.66328125)(207.80859375,445.38007813)(208.3359375,444.81367188)
\curveto(208.86328125,444.24726563)(209.12695312,443.45039063)(209.12695312,442.42304688)
\curveto(209.12695312,442.36054688)(209.125,442.26679688)(209.12109375,442.14179688)
\lineto(204.48046875,442.14179688)
\curveto(204.51953125,441.45820313)(204.71289062,440.93476563)(205.06054688,440.57148438)
\curveto(205.40820312,440.20820313)(205.84179688,440.0265625)(206.36132812,440.0265625)
\curveto(206.74804688,440.0265625)(207.078125,440.128125)(207.3515625,440.33125)
\curveto(207.625,440.534375)(207.84179688,440.85859375)(208.00195312,441.30390625)
\closepath
\moveto(204.5390625,443.00898438)
\lineto(208.01367188,443.00898438)
\curveto(207.96679688,443.53242188)(207.83398438,443.925)(207.61523438,444.18671875)
\curveto(207.27929688,444.59296875)(206.84375,444.79609375)(206.30859375,444.79609375)
\curveto(205.82421875,444.79609375)(205.41601562,444.63398438)(205.08398438,444.30976563)
\curveto(204.75585938,443.98554688)(204.57421875,443.55195313)(204.5390625,443.00898438)
\closepath
}
}
{
\newrgbcolor{curcolor}{0 0 0}
\pscustom[linestyle=none,fillstyle=solid,fillcolor=curcolor]
{
\newpath
\moveto(210.40429688,439.3)
\lineto(210.40429688,445.52265625)
\lineto(211.35351562,445.52265625)
\lineto(211.35351562,444.57929688)
\curveto(211.59570312,445.02070313)(211.81835938,445.31171875)(212.02148438,445.45234375)
\curveto(212.22851562,445.59296875)(212.45507812,445.66328125)(212.70117188,445.66328125)
\curveto(213.05664062,445.66328125)(213.41796875,445.55)(213.78515625,445.3234375)
\lineto(213.421875,444.34492188)
\curveto(213.1640625,444.49726563)(212.90625,444.5734375)(212.6484375,444.5734375)
\curveto(212.41796875,444.5734375)(212.2109375,444.503125)(212.02734375,444.3625)
\curveto(211.84375,444.22578125)(211.71289062,444.034375)(211.63476562,443.78828125)
\curveto(211.51757812,443.41328125)(211.45898438,443.003125)(211.45898438,442.5578125)
\lineto(211.45898438,439.3)
\closepath
}
}
{
\newrgbcolor{curcolor}{0 0 0}
\pscustom[linestyle=none,fillstyle=solid,fillcolor=curcolor]
{
\newpath
\moveto(218.47265625,441.57929688)
\lineto(219.50976562,441.44453125)
\curveto(219.39648438,440.7296875)(219.10546875,440.16914063)(218.63671875,439.76289063)
\curveto(218.171875,439.36054688)(217.59960938,439.159375)(216.91992188,439.159375)
\curveto(216.06835938,439.159375)(215.3828125,439.43671875)(214.86328125,439.99140625)
\curveto(214.34765625,440.55)(214.08984375,441.34882813)(214.08984375,442.38789063)
\curveto(214.08984375,443.05976563)(214.20117188,443.64765625)(214.42382812,444.1515625)
\curveto(214.64648438,444.65546875)(214.984375,445.03242188)(215.4375,445.28242188)
\curveto(215.89453125,445.53632813)(216.390625,445.66328125)(216.92578125,445.66328125)
\curveto(217.6015625,445.66328125)(218.15429688,445.49140625)(218.58398438,445.14765625)
\curveto(219.01367188,444.8078125)(219.2890625,444.3234375)(219.41015625,443.69453125)
\lineto(218.38476562,443.53632813)
\curveto(218.28710938,443.95429688)(218.11328125,444.26875)(217.86328125,444.4796875)
\curveto(217.6171875,444.690625)(217.31835938,444.79609375)(216.96679688,444.79609375)
\curveto(216.43554688,444.79609375)(216.00390625,444.6046875)(215.671875,444.221875)
\curveto(215.33984375,443.84296875)(215.17382812,443.24140625)(215.17382812,442.4171875)
\curveto(215.17382812,441.58125)(215.33398438,440.97382813)(215.65429688,440.59492188)
\curveto(215.97460938,440.21601563)(216.39257812,440.0265625)(216.90820312,440.0265625)
\curveto(217.32226562,440.0265625)(217.66796875,440.15351563)(217.9453125,440.40742188)
\curveto(218.22265625,440.66132813)(218.3984375,441.05195313)(218.47265625,441.57929688)
\closepath
}
}
{
\newrgbcolor{curcolor}{0 0 0}
\pscustom[linestyle=none,fillstyle=solid,fillcolor=curcolor]
{
\newpath
\moveto(224.671875,441.30390625)
\lineto(225.76171875,441.16914063)
\curveto(225.58984375,440.53242188)(225.27148438,440.03828125)(224.80664062,439.68671875)
\curveto(224.34179688,439.33515625)(223.74804688,439.159375)(223.02539062,439.159375)
\curveto(222.11523438,439.159375)(221.39257812,439.43867188)(220.85742188,439.99726563)
\curveto(220.32617188,440.55976563)(220.06054688,441.346875)(220.06054688,442.35859375)
\curveto(220.06054688,443.40546875)(220.33007812,444.21796875)(220.86914062,444.79609375)
\curveto(221.40820312,445.37421875)(222.10742188,445.66328125)(222.96679688,445.66328125)
\curveto(223.79882812,445.66328125)(224.47851562,445.38007813)(225.00585938,444.81367188)
\curveto(225.53320312,444.24726563)(225.796875,443.45039063)(225.796875,442.42304688)
\curveto(225.796875,442.36054688)(225.79492188,442.26679688)(225.79101562,442.14179688)
\lineto(221.15039062,442.14179688)
\curveto(221.18945312,441.45820313)(221.3828125,440.93476563)(221.73046875,440.57148438)
\curveto(222.078125,440.20820313)(222.51171875,440.0265625)(223.03125,440.0265625)
\curveto(223.41796875,440.0265625)(223.74804688,440.128125)(224.02148438,440.33125)
\curveto(224.29492188,440.534375)(224.51171875,440.85859375)(224.671875,441.30390625)
\closepath
\moveto(221.20898438,443.00898438)
\lineto(224.68359375,443.00898438)
\curveto(224.63671875,443.53242188)(224.50390625,443.925)(224.28515625,444.18671875)
\curveto(223.94921875,444.59296875)(223.51367188,444.79609375)(222.97851562,444.79609375)
\curveto(222.49414062,444.79609375)(222.0859375,444.63398438)(221.75390625,444.30976563)
\curveto(221.42578125,443.98554688)(221.24414062,443.55195313)(221.20898438,443.00898438)
\closepath
}
}
{
\newrgbcolor{curcolor}{0 0 0}
\pscustom[linestyle=none,fillstyle=solid,fillcolor=curcolor]
{
\newpath
\moveto(227.0859375,439.3)
\lineto(227.0859375,445.52265625)
\lineto(228.03515625,445.52265625)
\lineto(228.03515625,444.63789063)
\curveto(228.4921875,445.32148438)(229.15234375,445.66328125)(230.015625,445.66328125)
\curveto(230.390625,445.66328125)(230.734375,445.59492188)(231.046875,445.45820313)
\curveto(231.36328125,445.32539063)(231.59960938,445.14960938)(231.75585938,444.93085938)
\curveto(231.91210938,444.71210938)(232.02148438,444.45234375)(232.08398438,444.1515625)
\curveto(232.12304688,443.95625)(232.14257812,443.61445313)(232.14257812,443.12617188)
\lineto(232.14257812,439.3)
\lineto(231.08789062,439.3)
\lineto(231.08789062,443.08515625)
\curveto(231.08789062,443.51484375)(231.046875,443.83515625)(230.96484375,444.04609375)
\curveto(230.8828125,444.2609375)(230.73632812,444.43085938)(230.52539062,444.55585938)
\curveto(230.31835938,444.68476563)(230.07421875,444.74921875)(229.79296875,444.74921875)
\curveto(229.34375,444.74921875)(228.95507812,444.60664063)(228.62695312,444.32148438)
\curveto(228.30273438,444.03632813)(228.140625,443.4953125)(228.140625,442.6984375)
\lineto(228.140625,439.3)
\closepath
}
}
{
\newrgbcolor{curcolor}{0 0 0}
\pscustom[linestyle=none,fillstyle=solid,fillcolor=curcolor]
{
\newpath
\moveto(236.0625,440.24335938)
\lineto(236.21484375,439.31171875)
\curveto(235.91796875,439.24921875)(235.65234375,439.21796875)(235.41796875,439.21796875)
\curveto(235.03515625,439.21796875)(234.73828125,439.27851563)(234.52734375,439.39960938)
\curveto(234.31640625,439.52070313)(234.16796875,439.67890625)(234.08203125,439.87421875)
\curveto(233.99609375,440.0734375)(233.953125,440.48945313)(233.953125,441.12226563)
\lineto(233.953125,444.70234375)
\lineto(233.1796875,444.70234375)
\lineto(233.1796875,445.52265625)
\lineto(233.953125,445.52265625)
\lineto(233.953125,447.06367188)
\lineto(235.00195312,447.69648438)
\lineto(235.00195312,445.52265625)
\lineto(236.0625,445.52265625)
\lineto(236.0625,444.70234375)
\lineto(235.00195312,444.70234375)
\lineto(235.00195312,441.06367188)
\curveto(235.00195312,440.76289063)(235.01953125,440.56953125)(235.0546875,440.48359375)
\curveto(235.09375,440.39765625)(235.15429688,440.32929688)(235.23632812,440.27851563)
\curveto(235.32226562,440.22773438)(235.44335938,440.20234375)(235.59960938,440.20234375)
\curveto(235.71679688,440.20234375)(235.87109375,440.21601563)(236.0625,440.24335938)
\closepath
}
}
{
\newrgbcolor{curcolor}{0 0 0}
\pscustom[linestyle=none,fillstyle=solid,fillcolor=curcolor]
{
\newpath
\moveto(237.09960938,446.67695313)
\lineto(237.09960938,447.88984375)
\lineto(238.15429688,447.88984375)
\lineto(238.15429688,446.67695313)
\closepath
\moveto(237.09960938,439.3)
\lineto(237.09960938,445.52265625)
\lineto(238.15429688,445.52265625)
\lineto(238.15429688,439.3)
\closepath
}
}
{
\newrgbcolor{curcolor}{0 0 0}
\pscustom[linestyle=none,fillstyle=solid,fillcolor=curcolor]
{
\newpath
\moveto(239.73632812,439.3)
\lineto(239.73632812,447.88984375)
\lineto(240.79101562,447.88984375)
\lineto(240.79101562,439.3)
\closepath
}
}
{
\newrgbcolor{curcolor}{0 0 0}
\pscustom[linestyle=none,fillstyle=solid,fillcolor=curcolor]
{
\newpath
\moveto(246.68554688,441.30390625)
\lineto(247.77539062,441.16914063)
\curveto(247.60351562,440.53242188)(247.28515625,440.03828125)(246.8203125,439.68671875)
\curveto(246.35546875,439.33515625)(245.76171875,439.159375)(245.0390625,439.159375)
\curveto(244.12890625,439.159375)(243.40625,439.43867188)(242.87109375,439.99726563)
\curveto(242.33984375,440.55976563)(242.07421875,441.346875)(242.07421875,442.35859375)
\curveto(242.07421875,443.40546875)(242.34375,444.21796875)(242.8828125,444.79609375)
\curveto(243.421875,445.37421875)(244.12109375,445.66328125)(244.98046875,445.66328125)
\curveto(245.8125,445.66328125)(246.4921875,445.38007813)(247.01953125,444.81367188)
\curveto(247.546875,444.24726563)(247.81054688,443.45039063)(247.81054688,442.42304688)
\curveto(247.81054688,442.36054688)(247.80859375,442.26679688)(247.8046875,442.14179688)
\lineto(243.1640625,442.14179688)
\curveto(243.203125,441.45820313)(243.39648438,440.93476563)(243.74414062,440.57148438)
\curveto(244.09179688,440.20820313)(244.52539062,440.0265625)(245.04492188,440.0265625)
\curveto(245.43164062,440.0265625)(245.76171875,440.128125)(246.03515625,440.33125)
\curveto(246.30859375,440.534375)(246.52539062,440.85859375)(246.68554688,441.30390625)
\closepath
\moveto(243.22265625,443.00898438)
\lineto(246.69726562,443.00898438)
\curveto(246.65039062,443.53242188)(246.51757812,443.925)(246.29882812,444.18671875)
\curveto(245.96289062,444.59296875)(245.52734375,444.79609375)(244.9921875,444.79609375)
\curveto(244.5078125,444.79609375)(244.09960938,444.63398438)(243.76757812,444.30976563)
\curveto(243.43945312,443.98554688)(243.2578125,443.55195313)(243.22265625,443.00898438)
\closepath
}
}
{
\newrgbcolor{curcolor}{0 0 0}
\pscustom[linestyle=none,fillstyle=solid,fillcolor=curcolor]
{
\newpath
\moveto(252.52148438,439.3)
\lineto(252.52148438,447.88984375)
\lineto(253.65820312,447.88984375)
\lineto(253.65820312,440.31367188)
\lineto(257.88867188,440.31367188)
\lineto(257.88867188,439.3)
\closepath
}
}
{
\newrgbcolor{curcolor}{0 0 0}
\pscustom[linestyle=none,fillstyle=solid,fillcolor=curcolor]
{
\newpath
\moveto(263.16796875,440.06757813)
\curveto(262.77734375,439.73554688)(262.40039062,439.50117188)(262.03710938,439.36445313)
\curveto(261.67773438,439.22773438)(261.29101562,439.159375)(260.87695312,439.159375)
\curveto(260.19335938,439.159375)(259.66796875,439.32539063)(259.30078125,439.65742188)
\curveto(258.93359375,439.99335938)(258.75,440.42109375)(258.75,440.940625)
\curveto(258.75,441.2453125)(258.81835938,441.52265625)(258.95507812,441.77265625)
\curveto(259.09570312,442.0265625)(259.27734375,442.2296875)(259.5,442.38203125)
\curveto(259.7265625,442.534375)(259.98046875,442.64960938)(260.26171875,442.72773438)
\curveto(260.46875,442.78242188)(260.78125,442.83515625)(261.19921875,442.8859375)
\curveto(262.05078125,442.9875)(262.67773438,443.10859375)(263.08007812,443.24921875)
\curveto(263.08398438,443.39375)(263.0859375,443.48554688)(263.0859375,443.52460938)
\curveto(263.0859375,443.95429688)(262.98632812,444.25703125)(262.78710938,444.4328125)
\curveto(262.51757812,444.67109375)(262.1171875,444.79023438)(261.5859375,444.79023438)
\curveto(261.08984375,444.79023438)(260.72265625,444.70234375)(260.484375,444.5265625)
\curveto(260.25,444.3546875)(260.07617188,444.04804688)(259.96289062,443.60664063)
\lineto(258.93164062,443.74726563)
\curveto(259.02539062,444.18867188)(259.1796875,444.54414063)(259.39453125,444.81367188)
\curveto(259.609375,445.08710938)(259.91992188,445.29609375)(260.32617188,445.440625)
\curveto(260.73242188,445.5890625)(261.203125,445.66328125)(261.73828125,445.66328125)
\curveto(262.26953125,445.66328125)(262.70117188,445.60078125)(263.03320312,445.47578125)
\curveto(263.36523438,445.35078125)(263.609375,445.19257813)(263.765625,445.00117188)
\curveto(263.921875,444.81367188)(264.03125,444.57539063)(264.09375,444.28632813)
\curveto(264.12890625,444.10664063)(264.14648438,443.78242188)(264.14648438,443.31367188)
\lineto(264.14648438,441.90742188)
\curveto(264.14648438,440.92695313)(264.16796875,440.30585938)(264.2109375,440.04414063)
\curveto(264.2578125,439.78632813)(264.34765625,439.53828125)(264.48046875,439.3)
\lineto(263.37890625,439.3)
\curveto(263.26953125,439.51875)(263.19921875,439.77460938)(263.16796875,440.06757813)
\closepath
\moveto(263.08007812,442.42304688)
\curveto(262.69726562,442.26679688)(262.12304688,442.13398438)(261.35742188,442.02460938)
\curveto(260.92382812,441.96210938)(260.6171875,441.89179688)(260.4375,441.81367188)
\curveto(260.2578125,441.73554688)(260.11914062,441.6203125)(260.02148438,441.46796875)
\curveto(259.92382812,441.31953125)(259.875,441.15351563)(259.875,440.96992188)
\curveto(259.875,440.68867188)(259.98046875,440.45429688)(260.19140625,440.26679688)
\curveto(260.40625,440.07929688)(260.71875,439.98554688)(261.12890625,439.98554688)
\curveto(261.53515625,439.98554688)(261.89648438,440.0734375)(262.21289062,440.24921875)
\curveto(262.52929688,440.42890625)(262.76171875,440.67304688)(262.91015625,440.98164063)
\curveto(263.0234375,441.21992188)(263.08007812,441.57148438)(263.08007812,442.03632813)
\closepath
}
}
{
\newrgbcolor{curcolor}{0 0 0}
\pscustom[linestyle=none,fillstyle=solid,fillcolor=curcolor]
{
\newpath
\moveto(268.08398438,440.24335938)
\lineto(268.23632812,439.31171875)
\curveto(267.93945312,439.24921875)(267.67382812,439.21796875)(267.43945312,439.21796875)
\curveto(267.05664062,439.21796875)(266.75976562,439.27851563)(266.54882812,439.39960938)
\curveto(266.33789062,439.52070313)(266.18945312,439.67890625)(266.10351562,439.87421875)
\curveto(266.01757812,440.0734375)(265.97460938,440.48945313)(265.97460938,441.12226563)
\lineto(265.97460938,444.70234375)
\lineto(265.20117188,444.70234375)
\lineto(265.20117188,445.52265625)
\lineto(265.97460938,445.52265625)
\lineto(265.97460938,447.06367188)
\lineto(267.0234375,447.69648438)
\lineto(267.0234375,445.52265625)
\lineto(268.08398438,445.52265625)
\lineto(268.08398438,444.70234375)
\lineto(267.0234375,444.70234375)
\lineto(267.0234375,441.06367188)
\curveto(267.0234375,440.76289063)(267.04101562,440.56953125)(267.07617188,440.48359375)
\curveto(267.11523438,440.39765625)(267.17578125,440.32929688)(267.2578125,440.27851563)
\curveto(267.34375,440.22773438)(267.46484375,440.20234375)(267.62109375,440.20234375)
\curveto(267.73828125,440.20234375)(267.89257812,440.21601563)(268.08398438,440.24335938)
\closepath
}
}
{
\newrgbcolor{curcolor}{0 0 0}
\pscustom[linestyle=none,fillstyle=solid,fillcolor=curcolor]
{
\newpath
\moveto(273.375,441.30390625)
\lineto(274.46484375,441.16914063)
\curveto(274.29296875,440.53242188)(273.97460938,440.03828125)(273.50976562,439.68671875)
\curveto(273.04492188,439.33515625)(272.45117188,439.159375)(271.72851562,439.159375)
\curveto(270.81835938,439.159375)(270.09570312,439.43867188)(269.56054688,439.99726563)
\curveto(269.02929688,440.55976563)(268.76367188,441.346875)(268.76367188,442.35859375)
\curveto(268.76367188,443.40546875)(269.03320312,444.21796875)(269.57226562,444.79609375)
\curveto(270.11132812,445.37421875)(270.81054688,445.66328125)(271.66992188,445.66328125)
\curveto(272.50195312,445.66328125)(273.18164062,445.38007813)(273.70898438,444.81367188)
\curveto(274.23632812,444.24726563)(274.5,443.45039063)(274.5,442.42304688)
\curveto(274.5,442.36054688)(274.49804688,442.26679688)(274.49414062,442.14179688)
\lineto(269.85351562,442.14179688)
\curveto(269.89257812,441.45820313)(270.0859375,440.93476563)(270.43359375,440.57148438)
\curveto(270.78125,440.20820313)(271.21484375,440.0265625)(271.734375,440.0265625)
\curveto(272.12109375,440.0265625)(272.45117188,440.128125)(272.72460938,440.33125)
\curveto(272.99804688,440.534375)(273.21484375,440.85859375)(273.375,441.30390625)
\closepath
\moveto(269.91210938,443.00898438)
\lineto(273.38671875,443.00898438)
\curveto(273.33984375,443.53242188)(273.20703125,443.925)(272.98828125,444.18671875)
\curveto(272.65234375,444.59296875)(272.21679688,444.79609375)(271.68164062,444.79609375)
\curveto(271.19726562,444.79609375)(270.7890625,444.63398438)(270.45703125,444.30976563)
\curveto(270.12890625,443.98554688)(269.94726562,443.55195313)(269.91210938,443.00898438)
\closepath
}
}
{
\newrgbcolor{curcolor}{0 0 0}
\pscustom[linestyle=none,fillstyle=solid,fillcolor=curcolor]
{
\newpath
\moveto(275.7890625,439.3)
\lineto(275.7890625,445.52265625)
\lineto(276.73828125,445.52265625)
\lineto(276.73828125,444.63789063)
\curveto(277.1953125,445.32148438)(277.85546875,445.66328125)(278.71875,445.66328125)
\curveto(279.09375,445.66328125)(279.4375,445.59492188)(279.75,445.45820313)
\curveto(280.06640625,445.32539063)(280.30273438,445.14960938)(280.45898438,444.93085938)
\curveto(280.61523438,444.71210938)(280.72460938,444.45234375)(280.78710938,444.1515625)
\curveto(280.82617188,443.95625)(280.84570312,443.61445313)(280.84570312,443.12617188)
\lineto(280.84570312,439.3)
\lineto(279.79101562,439.3)
\lineto(279.79101562,443.08515625)
\curveto(279.79101562,443.51484375)(279.75,443.83515625)(279.66796875,444.04609375)
\curveto(279.5859375,444.2609375)(279.43945312,444.43085938)(279.22851562,444.55585938)
\curveto(279.02148438,444.68476563)(278.77734375,444.74921875)(278.49609375,444.74921875)
\curveto(278.046875,444.74921875)(277.65820312,444.60664063)(277.33007812,444.32148438)
\curveto(277.00585938,444.03632813)(276.84375,443.4953125)(276.84375,442.6984375)
\lineto(276.84375,439.3)
\closepath
}
}
{
\newrgbcolor{curcolor}{0 0 0}
\pscustom[linestyle=none,fillstyle=solid,fillcolor=curcolor]
{
\newpath
\moveto(286.5234375,441.57929688)
\lineto(287.56054688,441.44453125)
\curveto(287.44726562,440.7296875)(287.15625,440.16914063)(286.6875,439.76289063)
\curveto(286.22265625,439.36054688)(285.65039062,439.159375)(284.97070312,439.159375)
\curveto(284.11914062,439.159375)(283.43359375,439.43671875)(282.9140625,439.99140625)
\curveto(282.3984375,440.55)(282.140625,441.34882813)(282.140625,442.38789063)
\curveto(282.140625,443.05976563)(282.25195312,443.64765625)(282.47460938,444.1515625)
\curveto(282.69726562,444.65546875)(283.03515625,445.03242188)(283.48828125,445.28242188)
\curveto(283.9453125,445.53632813)(284.44140625,445.66328125)(284.9765625,445.66328125)
\curveto(285.65234375,445.66328125)(286.20507812,445.49140625)(286.63476562,445.14765625)
\curveto(287.06445312,444.8078125)(287.33984375,444.3234375)(287.4609375,443.69453125)
\lineto(286.43554688,443.53632813)
\curveto(286.33789062,443.95429688)(286.1640625,444.26875)(285.9140625,444.4796875)
\curveto(285.66796875,444.690625)(285.36914062,444.79609375)(285.01757812,444.79609375)
\curveto(284.48632812,444.79609375)(284.0546875,444.6046875)(283.72265625,444.221875)
\curveto(283.390625,443.84296875)(283.22460938,443.24140625)(283.22460938,442.4171875)
\curveto(283.22460938,441.58125)(283.38476562,440.97382813)(283.70507812,440.59492188)
\curveto(284.02539062,440.21601563)(284.44335938,440.0265625)(284.95898438,440.0265625)
\curveto(285.37304688,440.0265625)(285.71875,440.15351563)(285.99609375,440.40742188)
\curveto(286.2734375,440.66132813)(286.44921875,441.05195313)(286.5234375,441.57929688)
\closepath
}
}
{
\newrgbcolor{curcolor}{0 0 0}
\pscustom[linestyle=none,fillstyle=solid,fillcolor=curcolor]
{
\newpath
\moveto(288.41601562,436.90351563)
\lineto(288.29882812,437.89375)
\curveto(288.52929688,437.83125)(288.73046875,437.8)(288.90234375,437.8)
\curveto(289.13671875,437.8)(289.32421875,437.8390625)(289.46484375,437.9171875)
\curveto(289.60546875,437.9953125)(289.72070312,438.1046875)(289.81054688,438.2453125)
\curveto(289.87695312,438.35078125)(289.984375,438.6125)(290.1328125,439.03046875)
\curveto(290.15234375,439.0890625)(290.18359375,439.175)(290.2265625,439.28828125)
\lineto(287.86523438,445.52265625)
\lineto(289.00195312,445.52265625)
\lineto(290.296875,441.91914063)
\curveto(290.46484375,441.46210938)(290.61523438,440.98164063)(290.74804688,440.47773438)
\curveto(290.86914062,440.96210938)(291.01367188,441.43476563)(291.18164062,441.89570313)
\lineto(292.51171875,445.52265625)
\lineto(293.56640625,445.52265625)
\lineto(291.19921875,439.19453125)
\curveto(290.9453125,438.5109375)(290.74804688,438.04023438)(290.60742188,437.78242188)
\curveto(290.41992188,437.43476563)(290.20507812,437.18085938)(289.96289062,437.02070313)
\curveto(289.72070312,436.85664063)(289.43164062,436.77460938)(289.09570312,436.77460938)
\curveto(288.89257812,436.77460938)(288.66601562,436.81757813)(288.41601562,436.90351563)
\closepath
}
}
{
\newrgbcolor{curcolor}{0 0 0}
\pscustom[linestyle=none,fillstyle=solid,fillcolor=curcolor]
{
\newpath
\moveto(297.93164062,439.3)
\lineto(297.93164062,447.88984375)
\lineto(301.171875,447.88984375)
\curveto(301.7421875,447.88984375)(302.17773438,447.8625)(302.47851562,447.8078125)
\curveto(302.90039062,447.7375)(303.25390625,447.60273438)(303.5390625,447.40351563)
\curveto(303.82421875,447.20820313)(304.05273438,446.9328125)(304.22460938,446.57734375)
\curveto(304.40039062,446.221875)(304.48828125,445.83125)(304.48828125,445.40546875)
\curveto(304.48828125,444.675)(304.25585938,444.05585938)(303.79101562,443.54804688)
\curveto(303.32617188,443.04414063)(302.48632812,442.7921875)(301.27148438,442.7921875)
\lineto(299.06835938,442.7921875)
\lineto(299.06835938,439.3)
\closepath
\moveto(299.06835938,443.80585938)
\lineto(301.2890625,443.80585938)
\curveto(302.0234375,443.80585938)(302.54492188,443.94257813)(302.85351562,444.21601563)
\curveto(303.16210938,444.48945313)(303.31640625,444.87421875)(303.31640625,445.3703125)
\curveto(303.31640625,445.7296875)(303.22460938,446.03632813)(303.04101562,446.29023438)
\curveto(302.86132812,446.54804688)(302.62304688,446.71796875)(302.32617188,446.8)
\curveto(302.13476562,446.85078125)(301.78125,446.87617188)(301.265625,446.87617188)
\lineto(299.06835938,446.87617188)
\closepath
}
}
{
\newrgbcolor{curcolor}{0 0 0}
\pscustom[linestyle=none,fillstyle=solid,fillcolor=curcolor]
{
\newpath
\moveto(310.06054688,441.30390625)
\lineto(311.15039062,441.16914063)
\curveto(310.97851562,440.53242188)(310.66015625,440.03828125)(310.1953125,439.68671875)
\curveto(309.73046875,439.33515625)(309.13671875,439.159375)(308.4140625,439.159375)
\curveto(307.50390625,439.159375)(306.78125,439.43867188)(306.24609375,439.99726563)
\curveto(305.71484375,440.55976563)(305.44921875,441.346875)(305.44921875,442.35859375)
\curveto(305.44921875,443.40546875)(305.71875,444.21796875)(306.2578125,444.79609375)
\curveto(306.796875,445.37421875)(307.49609375,445.66328125)(308.35546875,445.66328125)
\curveto(309.1875,445.66328125)(309.8671875,445.38007813)(310.39453125,444.81367188)
\curveto(310.921875,444.24726563)(311.18554688,443.45039063)(311.18554688,442.42304688)
\curveto(311.18554688,442.36054688)(311.18359375,442.26679688)(311.1796875,442.14179688)
\lineto(306.5390625,442.14179688)
\curveto(306.578125,441.45820313)(306.77148438,440.93476563)(307.11914062,440.57148438)
\curveto(307.46679688,440.20820313)(307.90039062,440.0265625)(308.41992188,440.0265625)
\curveto(308.80664062,440.0265625)(309.13671875,440.128125)(309.41015625,440.33125)
\curveto(309.68359375,440.534375)(309.90039062,440.85859375)(310.06054688,441.30390625)
\closepath
\moveto(306.59765625,443.00898438)
\lineto(310.07226562,443.00898438)
\curveto(310.02539062,443.53242188)(309.89257812,443.925)(309.67382812,444.18671875)
\curveto(309.33789062,444.59296875)(308.90234375,444.79609375)(308.3671875,444.79609375)
\curveto(307.8828125,444.79609375)(307.47460938,444.63398438)(307.14257812,444.30976563)
\curveto(306.81445312,443.98554688)(306.6328125,443.55195313)(306.59765625,443.00898438)
\closepath
}
}
{
\newrgbcolor{curcolor}{0 0 0}
\pscustom[linestyle=none,fillstyle=solid,fillcolor=curcolor]
{
\newpath
\moveto(312.46289062,439.3)
\lineto(312.46289062,445.52265625)
\lineto(313.41210938,445.52265625)
\lineto(313.41210938,444.57929688)
\curveto(313.65429688,445.02070313)(313.87695312,445.31171875)(314.08007812,445.45234375)
\curveto(314.28710938,445.59296875)(314.51367188,445.66328125)(314.75976562,445.66328125)
\curveto(315.11523438,445.66328125)(315.4765625,445.55)(315.84375,445.3234375)
\lineto(315.48046875,444.34492188)
\curveto(315.22265625,444.49726563)(314.96484375,444.5734375)(314.70703125,444.5734375)
\curveto(314.4765625,444.5734375)(314.26953125,444.503125)(314.0859375,444.3625)
\curveto(313.90234375,444.22578125)(313.77148438,444.034375)(313.69335938,443.78828125)
\curveto(313.57617188,443.41328125)(313.51757812,443.003125)(313.51757812,442.5578125)
\lineto(313.51757812,439.3)
\closepath
}
}
{
\newrgbcolor{curcolor}{0 0 0}
\pscustom[linestyle=none,fillstyle=solid,fillcolor=curcolor]
{
\newpath
\moveto(320.53125,441.57929688)
\lineto(321.56835938,441.44453125)
\curveto(321.45507812,440.7296875)(321.1640625,440.16914063)(320.6953125,439.76289063)
\curveto(320.23046875,439.36054688)(319.65820312,439.159375)(318.97851562,439.159375)
\curveto(318.12695312,439.159375)(317.44140625,439.43671875)(316.921875,439.99140625)
\curveto(316.40625,440.55)(316.1484375,441.34882813)(316.1484375,442.38789063)
\curveto(316.1484375,443.05976563)(316.25976562,443.64765625)(316.48242188,444.1515625)
\curveto(316.70507812,444.65546875)(317.04296875,445.03242188)(317.49609375,445.28242188)
\curveto(317.953125,445.53632813)(318.44921875,445.66328125)(318.984375,445.66328125)
\curveto(319.66015625,445.66328125)(320.21289062,445.49140625)(320.64257812,445.14765625)
\curveto(321.07226562,444.8078125)(321.34765625,444.3234375)(321.46875,443.69453125)
\lineto(320.44335938,443.53632813)
\curveto(320.34570312,443.95429688)(320.171875,444.26875)(319.921875,444.4796875)
\curveto(319.67578125,444.690625)(319.37695312,444.79609375)(319.02539062,444.79609375)
\curveto(318.49414062,444.79609375)(318.0625,444.6046875)(317.73046875,444.221875)
\curveto(317.3984375,443.84296875)(317.23242188,443.24140625)(317.23242188,442.4171875)
\curveto(317.23242188,441.58125)(317.39257812,440.97382813)(317.71289062,440.59492188)
\curveto(318.03320312,440.21601563)(318.45117188,440.0265625)(318.96679688,440.0265625)
\curveto(319.38085938,440.0265625)(319.7265625,440.15351563)(320.00390625,440.40742188)
\curveto(320.28125,440.66132813)(320.45703125,441.05195313)(320.53125,441.57929688)
\closepath
}
}
{
\newrgbcolor{curcolor}{0 0 0}
\pscustom[linestyle=none,fillstyle=solid,fillcolor=curcolor]
{
\newpath
\moveto(326.73046875,441.30390625)
\lineto(327.8203125,441.16914063)
\curveto(327.6484375,440.53242188)(327.33007812,440.03828125)(326.86523438,439.68671875)
\curveto(326.40039062,439.33515625)(325.80664062,439.159375)(325.08398438,439.159375)
\curveto(324.17382812,439.159375)(323.45117188,439.43867188)(322.91601562,439.99726563)
\curveto(322.38476562,440.55976563)(322.11914062,441.346875)(322.11914062,442.35859375)
\curveto(322.11914062,443.40546875)(322.38867188,444.21796875)(322.92773438,444.79609375)
\curveto(323.46679688,445.37421875)(324.16601562,445.66328125)(325.02539062,445.66328125)
\curveto(325.85742188,445.66328125)(326.53710938,445.38007813)(327.06445312,444.81367188)
\curveto(327.59179688,444.24726563)(327.85546875,443.45039063)(327.85546875,442.42304688)
\curveto(327.85546875,442.36054688)(327.85351562,442.26679688)(327.84960938,442.14179688)
\lineto(323.20898438,442.14179688)
\curveto(323.24804688,441.45820313)(323.44140625,440.93476563)(323.7890625,440.57148438)
\curveto(324.13671875,440.20820313)(324.5703125,440.0265625)(325.08984375,440.0265625)
\curveto(325.4765625,440.0265625)(325.80664062,440.128125)(326.08007812,440.33125)
\curveto(326.35351562,440.534375)(326.5703125,440.85859375)(326.73046875,441.30390625)
\closepath
\moveto(323.26757812,443.00898438)
\lineto(326.7421875,443.00898438)
\curveto(326.6953125,443.53242188)(326.5625,443.925)(326.34375,444.18671875)
\curveto(326.0078125,444.59296875)(325.57226562,444.79609375)(325.03710938,444.79609375)
\curveto(324.55273438,444.79609375)(324.14453125,444.63398438)(323.8125,444.30976563)
\curveto(323.484375,443.98554688)(323.30273438,443.55195313)(323.26757812,443.00898438)
\closepath
}
}
{
\newrgbcolor{curcolor}{0 0 0}
\pscustom[linestyle=none,fillstyle=solid,fillcolor=curcolor]
{
\newpath
\moveto(329.14453125,439.3)
\lineto(329.14453125,445.52265625)
\lineto(330.09375,445.52265625)
\lineto(330.09375,444.63789063)
\curveto(330.55078125,445.32148438)(331.2109375,445.66328125)(332.07421875,445.66328125)
\curveto(332.44921875,445.66328125)(332.79296875,445.59492188)(333.10546875,445.45820313)
\curveto(333.421875,445.32539063)(333.65820312,445.14960938)(333.81445312,444.93085938)
\curveto(333.97070312,444.71210938)(334.08007812,444.45234375)(334.14257812,444.1515625)
\curveto(334.18164062,443.95625)(334.20117188,443.61445313)(334.20117188,443.12617188)
\lineto(334.20117188,439.3)
\lineto(333.14648438,439.3)
\lineto(333.14648438,443.08515625)
\curveto(333.14648438,443.51484375)(333.10546875,443.83515625)(333.0234375,444.04609375)
\curveto(332.94140625,444.2609375)(332.79492188,444.43085938)(332.58398438,444.55585938)
\curveto(332.37695312,444.68476563)(332.1328125,444.74921875)(331.8515625,444.74921875)
\curveto(331.40234375,444.74921875)(331.01367188,444.60664063)(330.68554688,444.32148438)
\curveto(330.36132812,444.03632813)(330.19921875,443.4953125)(330.19921875,442.6984375)
\lineto(330.19921875,439.3)
\closepath
}
}
{
\newrgbcolor{curcolor}{0 0 0}
\pscustom[linestyle=none,fillstyle=solid,fillcolor=curcolor]
{
\newpath
\moveto(338.12109375,440.24335938)
\lineto(338.2734375,439.31171875)
\curveto(337.9765625,439.24921875)(337.7109375,439.21796875)(337.4765625,439.21796875)
\curveto(337.09375,439.21796875)(336.796875,439.27851563)(336.5859375,439.39960938)
\curveto(336.375,439.52070313)(336.2265625,439.67890625)(336.140625,439.87421875)
\curveto(336.0546875,440.0734375)(336.01171875,440.48945313)(336.01171875,441.12226563)
\lineto(336.01171875,444.70234375)
\lineto(335.23828125,444.70234375)
\lineto(335.23828125,445.52265625)
\lineto(336.01171875,445.52265625)
\lineto(336.01171875,447.06367188)
\lineto(337.06054688,447.69648438)
\lineto(337.06054688,445.52265625)
\lineto(338.12109375,445.52265625)
\lineto(338.12109375,444.70234375)
\lineto(337.06054688,444.70234375)
\lineto(337.06054688,441.06367188)
\curveto(337.06054688,440.76289063)(337.078125,440.56953125)(337.11328125,440.48359375)
\curveto(337.15234375,440.39765625)(337.21289062,440.32929688)(337.29492188,440.27851563)
\curveto(337.38085938,440.22773438)(337.50195312,440.20234375)(337.65820312,440.20234375)
\curveto(337.77539062,440.20234375)(337.9296875,440.21601563)(338.12109375,440.24335938)
\closepath
}
}
{
\newrgbcolor{curcolor}{0 0 0}
\pscustom[linestyle=none,fillstyle=solid,fillcolor=curcolor]
{
\newpath
\moveto(343.21289062,440.06757813)
\curveto(342.82226562,439.73554688)(342.4453125,439.50117188)(342.08203125,439.36445313)
\curveto(341.72265625,439.22773438)(341.3359375,439.159375)(340.921875,439.159375)
\curveto(340.23828125,439.159375)(339.71289062,439.32539063)(339.34570312,439.65742188)
\curveto(338.97851562,439.99335938)(338.79492188,440.42109375)(338.79492188,440.940625)
\curveto(338.79492188,441.2453125)(338.86328125,441.52265625)(339,441.77265625)
\curveto(339.140625,442.0265625)(339.32226562,442.2296875)(339.54492188,442.38203125)
\curveto(339.77148438,442.534375)(340.02539062,442.64960938)(340.30664062,442.72773438)
\curveto(340.51367188,442.78242188)(340.82617188,442.83515625)(341.24414062,442.8859375)
\curveto(342.09570312,442.9875)(342.72265625,443.10859375)(343.125,443.24921875)
\curveto(343.12890625,443.39375)(343.13085938,443.48554688)(343.13085938,443.52460938)
\curveto(343.13085938,443.95429688)(343.03125,444.25703125)(342.83203125,444.4328125)
\curveto(342.5625,444.67109375)(342.16210938,444.79023438)(341.63085938,444.79023438)
\curveto(341.13476562,444.79023438)(340.76757812,444.70234375)(340.52929688,444.5265625)
\curveto(340.29492188,444.3546875)(340.12109375,444.04804688)(340.0078125,443.60664063)
\lineto(338.9765625,443.74726563)
\curveto(339.0703125,444.18867188)(339.22460938,444.54414063)(339.43945312,444.81367188)
\curveto(339.65429688,445.08710938)(339.96484375,445.29609375)(340.37109375,445.440625)
\curveto(340.77734375,445.5890625)(341.24804688,445.66328125)(341.78320312,445.66328125)
\curveto(342.31445312,445.66328125)(342.74609375,445.60078125)(343.078125,445.47578125)
\curveto(343.41015625,445.35078125)(343.65429688,445.19257813)(343.81054688,445.00117188)
\curveto(343.96679688,444.81367188)(344.07617188,444.57539063)(344.13867188,444.28632813)
\curveto(344.17382812,444.10664063)(344.19140625,443.78242188)(344.19140625,443.31367188)
\lineto(344.19140625,441.90742188)
\curveto(344.19140625,440.92695313)(344.21289062,440.30585938)(344.25585938,440.04414063)
\curveto(344.30273438,439.78632813)(344.39257812,439.53828125)(344.52539062,439.3)
\lineto(343.42382812,439.3)
\curveto(343.31445312,439.51875)(343.24414062,439.77460938)(343.21289062,440.06757813)
\closepath
\moveto(343.125,442.42304688)
\curveto(342.7421875,442.26679688)(342.16796875,442.13398438)(341.40234375,442.02460938)
\curveto(340.96875,441.96210938)(340.66210938,441.89179688)(340.48242188,441.81367188)
\curveto(340.30273438,441.73554688)(340.1640625,441.6203125)(340.06640625,441.46796875)
\curveto(339.96875,441.31953125)(339.91992188,441.15351563)(339.91992188,440.96992188)
\curveto(339.91992188,440.68867188)(340.02539062,440.45429688)(340.23632812,440.26679688)
\curveto(340.45117188,440.07929688)(340.76367188,439.98554688)(341.17382812,439.98554688)
\curveto(341.58007812,439.98554688)(341.94140625,440.0734375)(342.2578125,440.24921875)
\curveto(342.57421875,440.42890625)(342.80664062,440.67304688)(342.95507812,440.98164063)
\curveto(343.06835938,441.21992188)(343.125,441.57148438)(343.125,442.03632813)
\closepath
}
}
{
\newrgbcolor{curcolor}{0 0 0}
\pscustom[linestyle=none,fillstyle=solid,fillcolor=curcolor]
{
\newpath
\moveto(345.6328125,438.784375)
\lineto(346.65820312,438.63203125)
\curveto(346.70117188,438.315625)(346.8203125,438.08515625)(347.015625,437.940625)
\curveto(347.27734375,437.7453125)(347.63476562,437.64765625)(348.08789062,437.64765625)
\curveto(348.57617188,437.64765625)(348.953125,437.7453125)(349.21875,437.940625)
\curveto(349.484375,438.1359375)(349.6640625,438.409375)(349.7578125,438.7609375)
\curveto(349.8125,438.97578125)(349.83789062,439.42695313)(349.83398438,440.11445313)
\curveto(349.37304688,439.57148438)(348.79882812,439.3)(348.11132812,439.3)
\curveto(347.25585938,439.3)(346.59375,439.60859375)(346.125,440.22578125)
\curveto(345.65625,440.84296875)(345.421875,441.58320313)(345.421875,442.44648438)
\curveto(345.421875,443.04023438)(345.52929688,443.58710938)(345.74414062,444.08710938)
\curveto(345.95898438,444.59101563)(346.26953125,444.9796875)(346.67578125,445.253125)
\curveto(347.0859375,445.5265625)(347.56640625,445.66328125)(348.1171875,445.66328125)
\curveto(348.8515625,445.66328125)(349.45703125,445.36640625)(349.93359375,444.77265625)
\lineto(349.93359375,445.52265625)
\lineto(350.90625,445.52265625)
\lineto(350.90625,440.14375)
\curveto(350.90625,439.175)(350.80664062,438.48945313)(350.60742188,438.08710938)
\curveto(350.41210938,437.68085938)(350.09960938,437.36054688)(349.66992188,437.12617188)
\curveto(349.24414062,436.89179688)(348.71875,436.77460938)(348.09375,436.77460938)
\curveto(347.3515625,436.77460938)(346.75195312,436.94257813)(346.29492188,437.27851563)
\curveto(345.83789062,437.61054688)(345.6171875,438.1125)(345.6328125,438.784375)
\closepath
\moveto(346.50585938,442.52265625)
\curveto(346.50585938,441.70625)(346.66796875,441.11054688)(346.9921875,440.73554688)
\curveto(347.31640625,440.36054688)(347.72265625,440.17304688)(348.2109375,440.17304688)
\curveto(348.6953125,440.17304688)(349.1015625,440.35859375)(349.4296875,440.7296875)
\curveto(349.7578125,441.1046875)(349.921875,441.690625)(349.921875,442.4875)
\curveto(349.921875,443.24921875)(349.75195312,443.8234375)(349.41210938,444.21015625)
\curveto(349.07617188,444.596875)(348.66992188,444.79023438)(348.19335938,444.79023438)
\curveto(347.72460938,444.79023438)(347.32617188,444.59882813)(346.99804688,444.21601563)
\curveto(346.66992188,443.83710938)(346.50585938,443.27265625)(346.50585938,442.52265625)
\closepath
}
}
{
\newrgbcolor{curcolor}{0 0 0}
\pscustom[linestyle=none,fillstyle=solid,fillcolor=curcolor]
{
\newpath
\moveto(356.75976562,441.30390625)
\lineto(357.84960938,441.16914063)
\curveto(357.67773438,440.53242188)(357.359375,440.03828125)(356.89453125,439.68671875)
\curveto(356.4296875,439.33515625)(355.8359375,439.159375)(355.11328125,439.159375)
\curveto(354.203125,439.159375)(353.48046875,439.43867188)(352.9453125,439.99726563)
\curveto(352.4140625,440.55976563)(352.1484375,441.346875)(352.1484375,442.35859375)
\curveto(352.1484375,443.40546875)(352.41796875,444.21796875)(352.95703125,444.79609375)
\curveto(353.49609375,445.37421875)(354.1953125,445.66328125)(355.0546875,445.66328125)
\curveto(355.88671875,445.66328125)(356.56640625,445.38007813)(357.09375,444.81367188)
\curveto(357.62109375,444.24726563)(357.88476562,443.45039063)(357.88476562,442.42304688)
\curveto(357.88476562,442.36054688)(357.8828125,442.26679688)(357.87890625,442.14179688)
\lineto(353.23828125,442.14179688)
\curveto(353.27734375,441.45820313)(353.47070312,440.93476563)(353.81835938,440.57148438)
\curveto(354.16601562,440.20820313)(354.59960938,440.0265625)(355.11914062,440.0265625)
\curveto(355.50585938,440.0265625)(355.8359375,440.128125)(356.109375,440.33125)
\curveto(356.3828125,440.534375)(356.59960938,440.85859375)(356.75976562,441.30390625)
\closepath
\moveto(353.296875,443.00898438)
\lineto(356.77148438,443.00898438)
\curveto(356.72460938,443.53242188)(356.59179688,443.925)(356.37304688,444.18671875)
\curveto(356.03710938,444.59296875)(355.6015625,444.79609375)(355.06640625,444.79609375)
\curveto(354.58203125,444.79609375)(354.17382812,444.63398438)(353.84179688,444.30976563)
\curveto(353.51367188,443.98554688)(353.33203125,443.55195313)(353.296875,443.00898438)
\closepath
}
}
{
\newrgbcolor{curcolor}{0 0 0}
\pscustom[linestyle=none,fillstyle=solid,fillcolor=curcolor]
{
\newpath
\moveto(362.8359375,439.3)
\lineto(362.8359375,447.88984375)
\lineto(363.97265625,447.88984375)
\lineto(363.97265625,439.3)
\closepath
}
}
{
\newrgbcolor{curcolor}{0 0 0}
\pscustom[linestyle=none,fillstyle=solid,fillcolor=curcolor]
{
\newpath
\moveto(365.84179688,439.3)
\lineto(365.84179688,445.52265625)
\lineto(366.78515625,445.52265625)
\lineto(366.78515625,444.64960938)
\curveto(366.98046875,444.95429688)(367.24023438,445.1984375)(367.56445312,445.38203125)
\curveto(367.88867188,445.56953125)(368.2578125,445.66328125)(368.671875,445.66328125)
\curveto(369.1328125,445.66328125)(369.50976562,445.56757813)(369.80273438,445.37617188)
\curveto(370.09960938,445.18476563)(370.30859375,444.9171875)(370.4296875,444.5734375)
\curveto(370.921875,445.3)(371.5625,445.66328125)(372.3515625,445.66328125)
\curveto(372.96875,445.66328125)(373.44335938,445.49140625)(373.77539062,445.14765625)
\curveto(374.10742188,444.8078125)(374.2734375,444.28242188)(374.2734375,443.57148438)
\lineto(374.2734375,439.3)
\lineto(373.22460938,439.3)
\lineto(373.22460938,443.21992188)
\curveto(373.22460938,443.64179688)(373.18945312,443.94453125)(373.11914062,444.128125)
\curveto(373.05273438,444.315625)(372.9296875,444.46601563)(372.75,444.57929688)
\curveto(372.5703125,444.69257813)(372.359375,444.74921875)(372.1171875,444.74921875)
\curveto(371.6796875,444.74921875)(371.31640625,444.60273438)(371.02734375,444.30976563)
\curveto(370.73828125,444.02070313)(370.59375,443.55585938)(370.59375,442.91523438)
\lineto(370.59375,439.3)
\lineto(369.5390625,439.3)
\lineto(369.5390625,443.34296875)
\curveto(369.5390625,443.81171875)(369.453125,444.16328125)(369.28125,444.39765625)
\curveto(369.109375,444.63203125)(368.828125,444.74921875)(368.4375,444.74921875)
\curveto(368.140625,444.74921875)(367.86523438,444.67109375)(367.61132812,444.51484375)
\curveto(367.36132812,444.35859375)(367.1796875,444.13007813)(367.06640625,443.82929688)
\curveto(366.953125,443.52851563)(366.89648438,443.09492188)(366.89648438,442.52851563)
\lineto(366.89648438,439.3)
\closepath
}
}
{
\newrgbcolor{curcolor}{0 0 0}
\pscustom[linestyle=none,fillstyle=solid,fillcolor=curcolor]
{
\newpath
\moveto(375.83789062,436.91523438)
\lineto(375.83789062,445.52265625)
\lineto(376.79882812,445.52265625)
\lineto(376.79882812,444.7140625)
\curveto(377.02539062,445.03046875)(377.28125,445.26679688)(377.56640625,445.42304688)
\curveto(377.8515625,445.58320313)(378.19726562,445.66328125)(378.60351562,445.66328125)
\curveto(379.13476562,445.66328125)(379.60351562,445.5265625)(380.00976562,445.253125)
\curveto(380.41601562,444.9796875)(380.72265625,444.59296875)(380.9296875,444.09296875)
\curveto(381.13671875,443.596875)(381.24023438,443.05195313)(381.24023438,442.45820313)
\curveto(381.24023438,441.82148438)(381.125,441.24726563)(380.89453125,440.73554688)
\curveto(380.66796875,440.22773438)(380.3359375,439.83710938)(379.8984375,439.56367188)
\curveto(379.46484375,439.29414063)(379.0078125,439.159375)(378.52734375,439.159375)
\curveto(378.17578125,439.159375)(377.859375,439.23359375)(377.578125,439.38203125)
\curveto(377.30078125,439.53046875)(377.07226562,439.71796875)(376.89257812,439.94453125)
\lineto(376.89257812,436.91523438)
\closepath
\moveto(376.79296875,442.37617188)
\curveto(376.79296875,441.57539063)(376.95507812,440.98359375)(377.27929688,440.60078125)
\curveto(377.60351562,440.21796875)(377.99609375,440.0265625)(378.45703125,440.0265625)
\curveto(378.92578125,440.0265625)(379.32617188,440.22382813)(379.65820312,440.61835938)
\curveto(379.99414062,441.01679688)(380.16210938,441.63203125)(380.16210938,442.4640625)
\curveto(380.16210938,443.25703125)(379.99804688,443.85078125)(379.66992188,444.2453125)
\curveto(379.34570312,444.63984375)(378.95703125,444.83710938)(378.50390625,444.83710938)
\curveto(378.0546875,444.83710938)(377.65625,444.62617188)(377.30859375,444.20429688)
\curveto(376.96484375,443.78632813)(376.79296875,443.17695313)(376.79296875,442.37617188)
\closepath
}
}
{
\newrgbcolor{curcolor}{0 0 0}
\pscustom[linestyle=none,fillstyle=solid,fillcolor=curcolor]
{
\newpath
\moveto(382.5,439.3)
\lineto(382.5,445.52265625)
\lineto(383.44921875,445.52265625)
\lineto(383.44921875,444.57929688)
\curveto(383.69140625,445.02070313)(383.9140625,445.31171875)(384.1171875,445.45234375)
\curveto(384.32421875,445.59296875)(384.55078125,445.66328125)(384.796875,445.66328125)
\curveto(385.15234375,445.66328125)(385.51367188,445.55)(385.88085938,445.3234375)
\lineto(385.51757812,444.34492188)
\curveto(385.25976562,444.49726563)(385.00195312,444.5734375)(384.74414062,444.5734375)
\curveto(384.51367188,444.5734375)(384.30664062,444.503125)(384.12304688,444.3625)
\curveto(383.93945312,444.22578125)(383.80859375,444.034375)(383.73046875,443.78828125)
\curveto(383.61328125,443.41328125)(383.5546875,443.003125)(383.5546875,442.5578125)
\lineto(383.5546875,439.3)
\closepath
}
}
{
\newrgbcolor{curcolor}{0 0 0}
\pscustom[linestyle=none,fillstyle=solid,fillcolor=curcolor]
{
\newpath
\moveto(386.11523438,442.41132813)
\curveto(386.11523438,443.56367188)(386.43554688,444.4171875)(387.07617188,444.971875)
\curveto(387.61132812,445.4328125)(388.26367188,445.66328125)(389.03320312,445.66328125)
\curveto(389.88867188,445.66328125)(390.58789062,445.38203125)(391.13085938,444.81953125)
\curveto(391.67382812,444.2609375)(391.9453125,443.4875)(391.9453125,442.49921875)
\curveto(391.9453125,441.6984375)(391.82421875,441.06757813)(391.58203125,440.60664063)
\curveto(391.34375,440.14960938)(390.99414062,439.79414063)(390.53320312,439.54023438)
\curveto(390.07617188,439.28632813)(389.57617188,439.159375)(389.03320312,439.159375)
\curveto(388.16210938,439.159375)(387.45703125,439.43867188)(386.91796875,439.99726563)
\curveto(386.3828125,440.55585938)(386.11523438,441.36054688)(386.11523438,442.41132813)
\closepath
\moveto(387.19921875,442.41132813)
\curveto(387.19921875,441.61445313)(387.37304688,441.01679688)(387.72070312,440.61835938)
\curveto(388.06835938,440.22382813)(388.50585938,440.0265625)(389.03320312,440.0265625)
\curveto(389.55664062,440.0265625)(389.9921875,440.22578125)(390.33984375,440.62421875)
\curveto(390.6875,441.02265625)(390.86132812,441.63007813)(390.86132812,442.44648438)
\curveto(390.86132812,443.21601563)(390.68554688,443.79804688)(390.33398438,444.19257813)
\curveto(389.98632812,444.59101563)(389.55273438,444.79023438)(389.03320312,444.79023438)
\curveto(388.50585938,444.79023438)(388.06835938,444.59296875)(387.72070312,444.1984375)
\curveto(387.37304688,443.80390625)(387.19921875,443.20820313)(387.19921875,442.41132813)
\closepath
}
}
{
\newrgbcolor{curcolor}{0 0 0}
\pscustom[linestyle=none,fillstyle=solid,fillcolor=curcolor]
{
\newpath
\moveto(394.91015625,439.3)
\lineto(392.54296875,445.52265625)
\lineto(393.65625,445.52265625)
\lineto(394.9921875,441.79609375)
\curveto(395.13671875,441.39375)(395.26953125,440.97578125)(395.390625,440.5421875)
\curveto(395.484375,440.8703125)(395.61523438,441.26484375)(395.78320312,441.72578125)
\lineto(397.16601562,445.52265625)
\lineto(398.25,445.52265625)
\lineto(395.89453125,439.3)
\closepath
}
}
{
\newrgbcolor{curcolor}{0 0 0}
\pscustom[linestyle=none,fillstyle=solid,fillcolor=curcolor]
{
\newpath
\moveto(403.44140625,441.30390625)
\lineto(404.53125,441.16914063)
\curveto(404.359375,440.53242188)(404.04101562,440.03828125)(403.57617188,439.68671875)
\curveto(403.11132812,439.33515625)(402.51757812,439.159375)(401.79492188,439.159375)
\curveto(400.88476562,439.159375)(400.16210938,439.43867188)(399.62695312,439.99726563)
\curveto(399.09570312,440.55976563)(398.83007812,441.346875)(398.83007812,442.35859375)
\curveto(398.83007812,443.40546875)(399.09960938,444.21796875)(399.63867188,444.79609375)
\curveto(400.17773438,445.37421875)(400.87695312,445.66328125)(401.73632812,445.66328125)
\curveto(402.56835938,445.66328125)(403.24804688,445.38007813)(403.77539062,444.81367188)
\curveto(404.30273438,444.24726563)(404.56640625,443.45039063)(404.56640625,442.42304688)
\curveto(404.56640625,442.36054688)(404.56445312,442.26679688)(404.56054688,442.14179688)
\lineto(399.91992188,442.14179688)
\curveto(399.95898438,441.45820313)(400.15234375,440.93476563)(400.5,440.57148438)
\curveto(400.84765625,440.20820313)(401.28125,440.0265625)(401.80078125,440.0265625)
\curveto(402.1875,440.0265625)(402.51757812,440.128125)(402.79101562,440.33125)
\curveto(403.06445312,440.534375)(403.28125,440.85859375)(403.44140625,441.30390625)
\closepath
\moveto(399.97851562,443.00898438)
\lineto(403.453125,443.00898438)
\curveto(403.40625,443.53242188)(403.2734375,443.925)(403.0546875,444.18671875)
\curveto(402.71875,444.59296875)(402.28320312,444.79609375)(401.74804688,444.79609375)
\curveto(401.26367188,444.79609375)(400.85546875,444.63398438)(400.5234375,444.30976563)
\curveto(400.1953125,443.98554688)(400.01367188,443.55195313)(399.97851562,443.00898438)
\closepath
}
}
{
\newrgbcolor{curcolor}{0 0 0}
\pscustom[linestyle=none,fillstyle=solid,fillcolor=curcolor]
{
\newpath
\moveto(405.85546875,439.3)
\lineto(405.85546875,445.52265625)
\lineto(406.79882812,445.52265625)
\lineto(406.79882812,444.64960938)
\curveto(406.99414062,444.95429688)(407.25390625,445.1984375)(407.578125,445.38203125)
\curveto(407.90234375,445.56953125)(408.27148438,445.66328125)(408.68554688,445.66328125)
\curveto(409.14648438,445.66328125)(409.5234375,445.56757813)(409.81640625,445.37617188)
\curveto(410.11328125,445.18476563)(410.32226562,444.9171875)(410.44335938,444.5734375)
\curveto(410.93554688,445.3)(411.57617188,445.66328125)(412.36523438,445.66328125)
\curveto(412.98242188,445.66328125)(413.45703125,445.49140625)(413.7890625,445.14765625)
\curveto(414.12109375,444.8078125)(414.28710938,444.28242188)(414.28710938,443.57148438)
\lineto(414.28710938,439.3)
\lineto(413.23828125,439.3)
\lineto(413.23828125,443.21992188)
\curveto(413.23828125,443.64179688)(413.203125,443.94453125)(413.1328125,444.128125)
\curveto(413.06640625,444.315625)(412.94335938,444.46601563)(412.76367188,444.57929688)
\curveto(412.58398438,444.69257813)(412.37304688,444.74921875)(412.13085938,444.74921875)
\curveto(411.69335938,444.74921875)(411.33007812,444.60273438)(411.04101562,444.30976563)
\curveto(410.75195312,444.02070313)(410.60742188,443.55585938)(410.60742188,442.91523438)
\lineto(410.60742188,439.3)
\lineto(409.55273438,439.3)
\lineto(409.55273438,443.34296875)
\curveto(409.55273438,443.81171875)(409.46679688,444.16328125)(409.29492188,444.39765625)
\curveto(409.12304688,444.63203125)(408.84179688,444.74921875)(408.45117188,444.74921875)
\curveto(408.15429688,444.74921875)(407.87890625,444.67109375)(407.625,444.51484375)
\curveto(407.375,444.35859375)(407.19335938,444.13007813)(407.08007812,443.82929688)
\curveto(406.96679688,443.52851563)(406.91015625,443.09492188)(406.91015625,442.52851563)
\lineto(406.91015625,439.3)
\closepath
}
}
{
\newrgbcolor{curcolor}{0 0 0}
\pscustom[linestyle=none,fillstyle=solid,fillcolor=curcolor]
{
\newpath
\moveto(420.11132812,441.30390625)
\lineto(421.20117188,441.16914063)
\curveto(421.02929688,440.53242188)(420.7109375,440.03828125)(420.24609375,439.68671875)
\curveto(419.78125,439.33515625)(419.1875,439.159375)(418.46484375,439.159375)
\curveto(417.5546875,439.159375)(416.83203125,439.43867188)(416.296875,439.99726563)
\curveto(415.765625,440.55976563)(415.5,441.346875)(415.5,442.35859375)
\curveto(415.5,443.40546875)(415.76953125,444.21796875)(416.30859375,444.79609375)
\curveto(416.84765625,445.37421875)(417.546875,445.66328125)(418.40625,445.66328125)
\curveto(419.23828125,445.66328125)(419.91796875,445.38007813)(420.4453125,444.81367188)
\curveto(420.97265625,444.24726563)(421.23632812,443.45039063)(421.23632812,442.42304688)
\curveto(421.23632812,442.36054688)(421.234375,442.26679688)(421.23046875,442.14179688)
\lineto(416.58984375,442.14179688)
\curveto(416.62890625,441.45820313)(416.82226562,440.93476563)(417.16992188,440.57148438)
\curveto(417.51757812,440.20820313)(417.95117188,440.0265625)(418.47070312,440.0265625)
\curveto(418.85742188,440.0265625)(419.1875,440.128125)(419.4609375,440.33125)
\curveto(419.734375,440.534375)(419.95117188,440.85859375)(420.11132812,441.30390625)
\closepath
\moveto(416.6484375,443.00898438)
\lineto(420.12304688,443.00898438)
\curveto(420.07617188,443.53242188)(419.94335938,443.925)(419.72460938,444.18671875)
\curveto(419.38867188,444.59296875)(418.953125,444.79609375)(418.41796875,444.79609375)
\curveto(417.93359375,444.79609375)(417.52539062,444.63398438)(417.19335938,444.30976563)
\curveto(416.86523438,443.98554688)(416.68359375,443.55195313)(416.6484375,443.00898438)
\closepath
}
}
{
\newrgbcolor{curcolor}{0 0 0}
\pscustom[linestyle=none,fillstyle=solid,fillcolor=curcolor]
{
\newpath
\moveto(422.52539062,439.3)
\lineto(422.52539062,445.52265625)
\lineto(423.47460938,445.52265625)
\lineto(423.47460938,444.63789063)
\curveto(423.93164062,445.32148438)(424.59179688,445.66328125)(425.45507812,445.66328125)
\curveto(425.83007812,445.66328125)(426.17382812,445.59492188)(426.48632812,445.45820313)
\curveto(426.80273438,445.32539063)(427.0390625,445.14960938)(427.1953125,444.93085938)
\curveto(427.3515625,444.71210938)(427.4609375,444.45234375)(427.5234375,444.1515625)
\curveto(427.5625,443.95625)(427.58203125,443.61445313)(427.58203125,443.12617188)
\lineto(427.58203125,439.3)
\lineto(426.52734375,439.3)
\lineto(426.52734375,443.08515625)
\curveto(426.52734375,443.51484375)(426.48632812,443.83515625)(426.40429688,444.04609375)
\curveto(426.32226562,444.2609375)(426.17578125,444.43085938)(425.96484375,444.55585938)
\curveto(425.7578125,444.68476563)(425.51367188,444.74921875)(425.23242188,444.74921875)
\curveto(424.78320312,444.74921875)(424.39453125,444.60664063)(424.06640625,444.32148438)
\curveto(423.7421875,444.03632813)(423.58007812,443.4953125)(423.58007812,442.6984375)
\lineto(423.58007812,439.3)
\closepath
}
}
{
\newrgbcolor{curcolor}{0 0 0}
\pscustom[linestyle=none,fillstyle=solid,fillcolor=curcolor]
{
\newpath
\moveto(431.50195312,440.24335938)
\lineto(431.65429688,439.31171875)
\curveto(431.35742188,439.24921875)(431.09179688,439.21796875)(430.85742188,439.21796875)
\curveto(430.47460938,439.21796875)(430.17773438,439.27851563)(429.96679688,439.39960938)
\curveto(429.75585938,439.52070313)(429.60742188,439.67890625)(429.52148438,439.87421875)
\curveto(429.43554688,440.0734375)(429.39257812,440.48945313)(429.39257812,441.12226563)
\lineto(429.39257812,444.70234375)
\lineto(428.61914062,444.70234375)
\lineto(428.61914062,445.52265625)
\lineto(429.39257812,445.52265625)
\lineto(429.39257812,447.06367188)
\lineto(430.44140625,447.69648438)
\lineto(430.44140625,445.52265625)
\lineto(431.50195312,445.52265625)
\lineto(431.50195312,444.70234375)
\lineto(430.44140625,444.70234375)
\lineto(430.44140625,441.06367188)
\curveto(430.44140625,440.76289063)(430.45898438,440.56953125)(430.49414062,440.48359375)
\curveto(430.53320312,440.39765625)(430.59375,440.32929688)(430.67578125,440.27851563)
\curveto(430.76171875,440.22773438)(430.8828125,440.20234375)(431.0390625,440.20234375)
\curveto(431.15625,440.20234375)(431.31054688,440.21601563)(431.50195312,440.24335938)
\closepath
}
}
{
\newrgbcolor{curcolor}{0.24705882 0 1}
\pscustom[linewidth=1,linecolor=curcolor]
{
\newpath
\moveto(242.7,327.9)
\lineto(287.9,315.1)
}
}
{
\newrgbcolor{curcolor}{0.71764708 0.11764706 0.88235295}
\pscustom[linewidth=1,linecolor=curcolor]
{
\newpath
\moveto(287.9,315.1)
\lineto(333.1,348.1)
}
}
{
\newrgbcolor{curcolor}{1 0.36470589 0.63529414}
\pscustom[linewidth=1,linecolor=curcolor]
{
\newpath
\moveto(333.1,348.1)
\lineto(378.3,327.9)
}
}
{
\newrgbcolor{curcolor}{1 0.3882353 0.61176473}
\pscustom[linewidth=1,linecolor=curcolor]
{
\newpath
\moveto(378.3,327.9)
\lineto(423.6,332.8)
}
}
{
\newrgbcolor{curcolor}{1 0.49803922 0.50196081}
\pscustom[linewidth=1,linecolor=curcolor]
{
\newpath
\moveto(423.6,332.8)
\lineto(468.9,323.6)
}
}
{
\newrgbcolor{curcolor}{1 0.42745098 0.57254905}
\pscustom[linewidth=1,linecolor=curcolor]
{
\newpath
\moveto(468.9,323.6)
\lineto(514.1,304.9)
}
}
{
\newrgbcolor{curcolor}{0.3764706 0 1}
\pscustom[linewidth=1,linecolor=curcolor]
{
\newpath
\moveto(230.6,354.5)
\lineto(275.9,284.4)
}
}
{
\newrgbcolor{curcolor}{0.45490196 0 1}
\pscustom[linewidth=1,linecolor=curcolor]
{
\newpath
\moveto(275.9,284.4)
\lineto(321,342.5)
}
}
{
\newrgbcolor{curcolor}{1 0.35686275 0.64313728}
\pscustom[linewidth=1,linecolor=curcolor]
{
\newpath
\moveto(321,342.5)
\lineto(366.3,317.7)
\lineto(411.6,324.1)
}
}
{
\newrgbcolor{curcolor}{1 0.32156864 0.67843139}
\pscustom[linewidth=1,linecolor=curcolor]
{
\newpath
\moveto(411.6,324.1)
\lineto(456.8,294.5)
}
}
{
\newrgbcolor{curcolor}{0.78431374 0.16078432 0.8392157}
\pscustom[linewidth=1,linecolor=curcolor]
{
\newpath
\moveto(456.8,294.5)
\lineto(502.1,285.2)
}
}
{
\newrgbcolor{curcolor}{0.10196079 0 1}
\pscustom[linewidth=1,linecolor=curcolor]
{
\newpath
\moveto(218.5,319.5)
\lineto(263.8,282)
}
}
{
\newrgbcolor{curcolor}{0.64705884 0.07450981 0.9254902}
\pscustom[linewidth=1,linecolor=curcolor]
{
\newpath
\moveto(263.8,282)
\lineto(309,345.6)
}
}
{
\newrgbcolor{curcolor}{1 0.627451 0.37254903}
\pscustom[linewidth=1,linecolor=curcolor]
{
\newpath
\moveto(309,345.6)
\lineto(354.2,334.6)
}
}
{
\newrgbcolor{curcolor}{1 0.64705884 0.35294119}
\pscustom[linewidth=1,linecolor=curcolor]
{
\newpath
\moveto(354.2,334.6)
\lineto(399.5,329.8)
}
}
{
\newrgbcolor{curcolor}{1 0.59215689 0.40784314}
\pscustom[linewidth=1,linecolor=curcolor]
{
\newpath
\moveto(399.5,329.8)
\lineto(444.8,309)
}
}
{
\newrgbcolor{curcolor}{1 0.54509807 0.45490196}
\pscustom[linewidth=1,linecolor=curcolor]
{
\newpath
\moveto(444.8,309)
\lineto(490,305)
}
}
{
\newrgbcolor{curcolor}{0.45490196 0 1}
\pscustom[linewidth=1,linecolor=curcolor]
{
\newpath
\moveto(206.5,299.1)
\lineto(251.7,316.7)
}
}
{
\newrgbcolor{curcolor}{1 0.45490196 0.54509807}
\pscustom[linewidth=1,linecolor=curcolor]
{
\newpath
\moveto(251.7,316.7)
\lineto(297,345.2)
}
}
{
\newrgbcolor{curcolor}{1 0.72549021 0.27450982}
\pscustom[linewidth=1,linecolor=curcolor]
{
\newpath
\moveto(297,345.2)
\lineto(342.2,332.9)
}
}
{
\newrgbcolor{curcolor}{1 0.71372551 0.28627452}
\pscustom[linewidth=1,linecolor=curcolor]
{
\newpath
\moveto(342.2,332.9)
\lineto(387.4,325.2)
}
}
{
\newrgbcolor{curcolor}{1 0.75294119 0.24705882}
\pscustom[linewidth=1,linecolor=curcolor]
{
\newpath
\moveto(387.4,325.2)
\lineto(432.7,319.4)
}
}
{
\newrgbcolor{curcolor}{1 0.72941178 0.27058825}
\pscustom[linewidth=1,linecolor=curcolor]
{
\newpath
\moveto(432.7,319.4)
\lineto(478,303.5)
}
}
{
\newrgbcolor{curcolor}{0 0 0.59607846}
\pscustom[linewidth=1,linecolor=curcolor]
{
\newpath
\moveto(194.4,274.5)
\lineto(239.7,262.9)
}
}
{
\newrgbcolor{curcolor}{0.63921571 0.07058824 0.92941177}
\pscustom[linewidth=1,linecolor=curcolor]
{
\newpath
\moveto(239.7,262.9)
\lineto(284.9,334.7)
}
}
{
\newrgbcolor{curcolor}{1 0.72549021 0.27450982}
\pscustom[linewidth=1,linecolor=curcolor]
{
\newpath
\moveto(284.9,334.7)
\lineto(330.1,328.7)
}
}
{
\newrgbcolor{curcolor}{1 0.66666669 0.33333334}
\pscustom[linewidth=1,linecolor=curcolor]
{
\newpath
\moveto(330.1,328.7)
\lineto(375.4,308.3)
}
}
{
\newrgbcolor{curcolor}{1 0.60392159 0.39607844}
\pscustom[linewidth=1,linecolor=curcolor]
{
\newpath
\moveto(375.4,308.3)
\lineto(420.6,302)
}
}
{
\newrgbcolor{curcolor}{1 0.66274512 0.33725491}
\pscustom[linewidth=1,linecolor=curcolor]
{
\newpath
\moveto(420.6,302)
\lineto(465.9,297.8)
}
}
{
\newrgbcolor{curcolor}{0.24705882 0 1}
\pscustom[linewidth=1,linecolor=curcolor]
{
\newpath
\moveto(182.3,292.2)
\lineto(227.6,276.7)
}
}
{
\newrgbcolor{curcolor}{0.75294119 0.14117648 0.85882354}
\pscustom[linewidth=1,linecolor=curcolor]
{
\newpath
\moveto(227.6,276.7)
\lineto(272.9,315.2)
}
}
{
\newrgbcolor{curcolor}{1 0.58039218 0.41960785}
\pscustom[linewidth=1,linecolor=curcolor]
{
\newpath
\moveto(272.9,315.2)
\lineto(318,314.7)
}
}
{
\newrgbcolor{curcolor}{1 0.66274512 0.33725491}
\pscustom[linewidth=1,linecolor=curcolor]
{
\newpath
\moveto(318,314.7)
\lineto(363.3,307.3)
}
}
{
\newrgbcolor{curcolor}{1 0.61960787 0.38039216}
\pscustom[linewidth=1,linecolor=curcolor]
{
\newpath
\moveto(363.3,307.3)
\lineto(408.6,290.5)
}
}
{
\newrgbcolor{curcolor}{1 0.59607846 0.40392157}
\pscustom[linewidth=1,linecolor=curcolor]
{
\newpath
\moveto(408.6,290.5)
\lineto(453.8,285.6)
}
}
{
\newrgbcolor{curcolor}{1 0.45490196 0.54509807}
\pscustom[linewidth=1,linecolor=curcolor]
{
\newpath
\moveto(170.3,356.9)
\lineto(215.6,278.8)
}
}
{
\newrgbcolor{curcolor}{1 0.30980393 0.6901961}
\pscustom[linewidth=1,linecolor=curcolor]
{
\newpath
\moveto(215.6,278.8)
\lineto(260.8,320.1)
}
}
{
\newrgbcolor{curcolor}{1 0.74901962 0.25098041}
\pscustom[linewidth=1,linecolor=curcolor]
{
\newpath
\moveto(260.8,320.1)
\lineto(306,316.6)
}
}
{
\newrgbcolor{curcolor}{1 0.78823531 0.21176471}
\pscustom[linewidth=1,linecolor=curcolor]
{
\newpath
\moveto(306,316.6)
\lineto(351.2,306.6)
}
}
{
\newrgbcolor{curcolor}{1 0.82745099 0.17254902}
\pscustom[linewidth=1,linecolor=curcolor]
{
\newpath
\moveto(351.2,306.6)
\lineto(396.5,303.2)
}
}
{
\newrgbcolor{curcolor}{1 0.76862746 0.23137255}
\pscustom[linewidth=1,linecolor=curcolor]
{
\newpath
\moveto(396.5,303.2)
\lineto(441.8,280.2)
}
}
{
\newrgbcolor{curcolor}{0.67058825 0.09019608 0.90980393}
\pscustom[linewidth=1,linecolor=curcolor]
{
\newpath
\moveto(158.2,287.8)
\lineto(203.5,286.4)
}
}
{
\newrgbcolor{curcolor}{1 0.36078432 0.63921571}
\pscustom[linewidth=1,linecolor=curcolor]
{
\newpath
\moveto(203.5,286.4)
\lineto(248.8,304.2)
}
}
{
\newrgbcolor{curcolor}{1 0.3019608 0.69803923}
\pscustom[linewidth=1,linecolor=curcolor]
{
\newpath
\moveto(248.8,304.2)
\lineto(294,260.2)
}
}
{
\newrgbcolor{curcolor}{0.55686277 0.01568628 0.98431373}
\pscustom[linewidth=1,linecolor=curcolor]
{
\newpath
\moveto(294,260.2)
\lineto(339.2,248.9)
}
}
{
\newrgbcolor{curcolor}{0.48627451 0 1}
\pscustom[linewidth=1,linecolor=curcolor]
{
\newpath
\moveto(339.2,248.9)
\lineto(384.4,236)
}
}
{
\newrgbcolor{curcolor}{0.50980395 0 1}
\pscustom[linewidth=1,linecolor=curcolor]
{
\newpath
\moveto(384.4,236)
\lineto(429.7,232.3)
}
}
{
\newrgbcolor{curcolor}{0.27843139 0 1}
\pscustom[linewidth=1,linecolor=curcolor]
{
\newpath
\moveto(146.2,238.3)
\lineto(191.4,288.6)
}
}
{
\newrgbcolor{curcolor}{0.71764708 0.12156863 0.87843138}
\pscustom[linewidth=1,linecolor=curcolor]
{
\newpath
\moveto(191.4,288.6)
\lineto(236.7,256)
}
}
{
\newrgbcolor{curcolor}{0.30588236 0 1}
\pscustom[linewidth=1,linecolor=curcolor]
{
\newpath
\moveto(236.7,256)
\lineto(282,236.1)
}
}
{
\newrgbcolor{curcolor}{0.29019609 0 1}
\pscustom[linewidth=1,linecolor=curcolor]
{
\newpath
\moveto(282,236.1)
\lineto(327.1,236.4)
}
}
{
\newrgbcolor{curcolor}{0.39215687 0 1}
\pscustom[linewidth=1,linecolor=curcolor]
{
\newpath
\moveto(327.1,236.4)
\lineto(372.4,225.9)
}
}
{
\newrgbcolor{curcolor}{0.33333334 0 1}
\pscustom[linewidth=1,linecolor=curcolor]
{
\newpath
\moveto(372.4,225.9)
\lineto(417.7,213)
}
}
{
\newrgbcolor{curcolor}{0.07450981 0 1}
\pscustom[linewidth=1,linecolor=curcolor]
{
\newpath
\moveto(134.1,251.6)
\lineto(179.4,243.8)
}
}
{
\newrgbcolor{curcolor}{0.11764706 0 1}
\pscustom[linewidth=1,linecolor=curcolor]
{
\newpath
\moveto(179.4,243.8)
\lineto(224.6,236.6)
}
}
{
\newrgbcolor{curcolor}{0.16078432 0 1}
\pscustom[linewidth=1,linecolor=curcolor]
{
\newpath
\moveto(224.6,236.6)
\lineto(269.9,228.9)
}
}
{
\newrgbcolor{curcolor}{0.28235295 0 1}
\pscustom[linewidth=1,linecolor=curcolor]
{
\newpath
\moveto(269.9,228.9)
\lineto(315.1,228)
}
}
{
\newrgbcolor{curcolor}{0.38039216 0 1}
\pscustom[linewidth=1,linecolor=curcolor]
{
\newpath
\moveto(315.1,228)
\lineto(360.3,218.6)
}
}
{
\newrgbcolor{curcolor}{0.32156864 0 1}
\pscustom[linewidth=1,linecolor=curcolor]
{
\newpath
\moveto(360.3,218.6)
\lineto(405.6,204.4)
}
}
{
\newrgbcolor{curcolor}{0.41176471 0 1}
\pscustom[linewidth=1,linecolor=curcolor]
{
\newpath
\moveto(122,254.3)
\lineto(167.3,254.1)
}
}
{
\newrgbcolor{curcolor}{0.36862746 0 1}
\pscustom[linewidth=1,linecolor=curcolor]
{
\newpath
\moveto(167.3,254.1)
\lineto(212.6,232.2)
}
}
{
\newrgbcolor{curcolor}{0.25882354 0 1}
\pscustom[linewidth=1,linecolor=curcolor]
{
\newpath
\moveto(212.6,232.2)
\lineto(257.8,226.7)
}
}
{
\newrgbcolor{curcolor}{0.30588236 0 1}
\pscustom[linewidth=1,linecolor=curcolor]
{
\newpath
\moveto(257.8,226.7)
\lineto(303,217.5)
}
}
{
\newrgbcolor{curcolor}{0.28235295 0 1}
\pscustom[linewidth=1,linecolor=curcolor]
{
\newpath
\moveto(303,217.5)
\lineto(348.3,206.2)
}
}
{
\newrgbcolor{curcolor}{0.16862746 0 1}
\pscustom[linewidth=1,linecolor=curcolor]
{
\newpath
\moveto(348.3,206.2)
\lineto(393.5,189.3)
}
}
{
\newrgbcolor{curcolor}{0.35686275 0 1}
\pscustom[linewidth=1,linecolor=curcolor]
{
\newpath
\moveto(110,249.2)
\lineto(155.2,239.9)
}
}
{
\newrgbcolor{curcolor}{0.30980393 0 1}
\pscustom[linewidth=1,linecolor=curcolor]
{
\newpath
\moveto(155.2,239.9)
\lineto(200.5,226.7)
}
}
{
\newrgbcolor{curcolor}{0.20784314 0 1}
\pscustom[linewidth=1,linecolor=curcolor]
{
\newpath
\moveto(200.5,226.7)
\lineto(245.8,213.1)
}
}
{
\newrgbcolor{curcolor}{0.18431373 0 1}
\pscustom[linewidth=1,linecolor=curcolor]
{
\newpath
\moveto(245.8,213.1)
\lineto(291,206)
}
}
{
\newrgbcolor{curcolor}{0.28627452 0 1}
\pscustom[linewidth=1,linecolor=curcolor]
{
\newpath
\moveto(291,206)
\lineto(336.2,203.1)
\lineto(381.5,187.4)
}
}
{
\newrgbcolor{curcolor}{0.23921569 0 1}
\pscustom[linewidth=1,linecolor=curcolor]
{
\newpath
\moveto(97.9,227.3)
\lineto(143.2,237.2)
}
}
{
\newrgbcolor{curcolor}{0.36470589 0 1}
\pscustom[linewidth=1,linecolor=curcolor]
{
\newpath
\moveto(143.2,237.2)
\lineto(188.4,219.2)
}
}
{
\newrgbcolor{curcolor}{0.26666668 0 1}
\pscustom[linewidth=1,linecolor=curcolor]
{
\newpath
\moveto(188.4,219.2)
\lineto(233.7,210.5)
}
}
{
\newrgbcolor{curcolor}{0.25882354 0 1}
\pscustom[linewidth=1,linecolor=curcolor]
{
\newpath
\moveto(233.7,210.5)
\lineto(279,200.1)
}
}
{
\newrgbcolor{curcolor}{0.32941177 0 1}
\pscustom[linewidth=1,linecolor=curcolor]
{
\newpath
\moveto(279,200.1)
\lineto(324.1,197.5)
}
}
{
\newrgbcolor{curcolor}{0.38039216 0 1}
\pscustom[linewidth=1,linecolor=curcolor]
{
\newpath
\moveto(324.1,197.5)
\lineto(369.4,186)
}
}
{
\newrgbcolor{curcolor}{0.13725491 0 1}
\pscustom[linewidth=1,linecolor=curcolor]
{
\newpath
\moveto(85.9,227)
\lineto(131.1,214.4)
}
}
{
\newrgbcolor{curcolor}{0.22745098 0 1}
\pscustom[linewidth=1,linecolor=curcolor]
{
\newpath
\moveto(131.1,214.4)
\lineto(176.4,215.8)
}
}
{
\newrgbcolor{curcolor}{0.36470589 0 1}
\pscustom[linewidth=1,linecolor=curcolor]
{
\newpath
\moveto(176.4,215.8)
\lineto(221.7,207.1)
}
}
{
\newrgbcolor{curcolor}{0.30980393 0 1}
\pscustom[linewidth=1,linecolor=curcolor]
{
\newpath
\moveto(221.7,207.1)
\lineto(266.9,192.7)
}
}
{
\newrgbcolor{curcolor}{0.34509805 0 1}
\pscustom[linewidth=1,linecolor=curcolor]
{
\newpath
\moveto(266.9,192.7)
\lineto(312.1,191.4)
}
}
{
\newrgbcolor{curcolor}{0.38431373 0 1}
\pscustom[linewidth=1,linecolor=curcolor]
{
\newpath
\moveto(312.1,191.4)
\lineto(357.3,177.5)
}
}
{
\newrgbcolor{curcolor}{0.7019608 0.10980392 0.89019608}
\pscustom[linewidth=1,linecolor=curcolor]
{
\newpath
\moveto(242.7,327.9)
\lineto(230.6,354.5)
}
}
{
\newrgbcolor{curcolor}{0.78039217 0.16078432 0.8392157}
\pscustom[linewidth=1,linecolor=curcolor]
{
\newpath
\moveto(230.6,354.5)
\lineto(218.5,319.5)
}
}
{
\newrgbcolor{curcolor}{0.28627452 0 1}
\pscustom[linewidth=1,linecolor=curcolor]
{
\newpath
\moveto(218.5,319.5)
\lineto(206.5,299.1)
}
}
{
\newrgbcolor{curcolor}{0 0 0.89803922}
\pscustom[linewidth=1,linecolor=curcolor]
{
\newpath
\moveto(206.5,299.1)
\lineto(194.4,274.5)
}
}
{
\newrgbcolor{curcolor}{0.01960784 0 1}
\pscustom[linewidth=1,linecolor=curcolor]
{
\newpath
\moveto(194.4,274.5)
\lineto(182.3,292.2)
}
}
{
\newrgbcolor{curcolor}{1 0.42745098 0.57254905}
\pscustom[linewidth=1,linecolor=curcolor]
{
\newpath
\moveto(182.3,292.2)
\lineto(170.3,356.9)
}
}
{
\newrgbcolor{curcolor}{1 0.50980395 0.49019608}
\pscustom[linewidth=1,linecolor=curcolor]
{
\newpath
\moveto(170.3,356.9)
\lineto(158.2,287.8)
}
}
{
\newrgbcolor{curcolor}{0.0627451 0 1}
\pscustom[linewidth=1,linecolor=curcolor]
{
\newpath
\moveto(158.2,287.8)
\lineto(146.2,238.3)
}
}
{
\newrgbcolor{curcolor}{0 0 0.74901962}
\pscustom[linewidth=1,linecolor=curcolor]
{
\newpath
\moveto(146.2,238.3)
\lineto(134.1,251.6)
}
}
{
\newrgbcolor{curcolor}{0.18039216 0 1}
\pscustom[linewidth=1,linecolor=curcolor]
{
\newpath
\moveto(134.1,251.6)
\lineto(122,254.3)
}
}
{
\newrgbcolor{curcolor}{0.32941177 0 1}
\pscustom[linewidth=1,linecolor=curcolor]
{
\newpath
\moveto(122,254.3)
\lineto(110,249.2)
}
}
{
\newrgbcolor{curcolor}{0.18039216 0 1}
\pscustom[linewidth=1,linecolor=curcolor]
{
\newpath
\moveto(110,249.2)
\lineto(97.9,227.3)
}
}
{
\newrgbcolor{curcolor}{0.09411765 0 1}
\pscustom[linewidth=1,linecolor=curcolor]
{
\newpath
\moveto(97.9,227.3)
\lineto(85.9,227)
}
}
{
\newrgbcolor{curcolor}{0 0 0.89803922}
\pscustom[linewidth=1,linecolor=curcolor]
{
\newpath
\moveto(287.9,315.1)
\lineto(275.9,284.4)
}
}
{
\newrgbcolor{curcolor}{0 0 0.6156863}
\pscustom[linewidth=1,linecolor=curcolor]
{
\newpath
\moveto(275.9,284.4)
\lineto(263.8,282)
}
}
{
\newrgbcolor{curcolor}{0.27058825 0 1}
\pscustom[linewidth=1,linecolor=curcolor]
{
\newpath
\moveto(263.8,282)
\lineto(251.7,316.7)
}
}
{
\newrgbcolor{curcolor}{0.21960784 0 1}
\pscustom[linewidth=1,linecolor=curcolor]
{
\newpath
\moveto(251.7,316.7)
\lineto(239.7,262.9)
}
}
{
\newrgbcolor{curcolor}{0 0 0.89019608}
\pscustom[linewidth=1,linecolor=curcolor]
{
\newpath
\moveto(239.7,262.9)
\lineto(227.6,276.7)
}
}
{
\newrgbcolor{curcolor}{0.28627452 0 1}
\pscustom[linewidth=1,linecolor=curcolor]
{
\newpath
\moveto(227.6,276.7)
\lineto(215.6,278.8)
}
}
{
\newrgbcolor{curcolor}{0.58431375 0.03529412 0.96470588}
\pscustom[linewidth=1,linecolor=curcolor]
{
\newpath
\moveto(215.6,278.8)
\lineto(203.5,286.4)
}
}
{
\newrgbcolor{curcolor}{0.88627452 0.22745098 0.77254903}
\pscustom[linewidth=1,linecolor=curcolor]
{
\newpath
\moveto(203.5,286.4)
\lineto(191.4,288.6)
}
}
{
\newrgbcolor{curcolor}{0.54901963 0.01176471 0.98823529}
\pscustom[linewidth=1,linecolor=curcolor]
{
\newpath
\moveto(191.4,288.6)
\lineto(179.4,243.8)
}
}
{
\newrgbcolor{curcolor}{0.30980393 0 1}
\pscustom[linewidth=1,linecolor=curcolor]
{
\newpath
\moveto(179.4,243.8)
\lineto(167.3,254.1)
}
}
{
\newrgbcolor{curcolor}{0.43921569 0 1}
\pscustom[linewidth=1,linecolor=curcolor]
{
\newpath
\moveto(167.3,254.1)
\lineto(155.2,239.9)
}
}
{
\newrgbcolor{curcolor}{0.41176471 0 1}
\pscustom[linewidth=1,linecolor=curcolor]
{
\newpath
\moveto(155.2,239.9)
\lineto(143.2,237.2)
}
}
{
\newrgbcolor{curcolor}{0.28627452 0 1}
\pscustom[linewidth=1,linecolor=curcolor]
{
\newpath
\moveto(143.2,237.2)
\lineto(131.1,214.4)
}
}
{
\newrgbcolor{curcolor}{1 0.4627451 0.53725493}
\pscustom[linewidth=1,linecolor=curcolor]
{
\newpath
\moveto(333.1,348.1)
\lineto(321,342.5)
}
}
{
\newrgbcolor{curcolor}{1 0.55686277 0.44313726}
\pscustom[linewidth=1,linecolor=curcolor]
{
\newpath
\moveto(321,342.5)
\lineto(309,345.6)
}
}
{
\newrgbcolor{curcolor}{1 0.69411767 0.30588236}
\pscustom[linewidth=1,linecolor=curcolor]
{
\newpath
\moveto(309,345.6)
\lineto(297,345.2)
}
}
{
\newrgbcolor{curcolor}{1 0.72549021 0.27450982}
\pscustom[linewidth=1,linecolor=curcolor]
{
\newpath
\moveto(297,345.2)
\lineto(284.9,334.7)
}
}
{
\newrgbcolor{curcolor}{1 0.60784316 0.39215687}
\pscustom[linewidth=1,linecolor=curcolor]
{
\newpath
\moveto(284.9,334.7)
\lineto(272.9,315.2)
\lineto(260.8,320.1)
}
}
{
\newrgbcolor{curcolor}{1 0.63921571 0.36078432}
\pscustom[linewidth=1,linecolor=curcolor]
{
\newpath
\moveto(260.8,320.1)
\lineto(248.8,304.2)
}
}
{
\newrgbcolor{curcolor}{0.92941177 0.25490198 0.74509805}
\pscustom[linewidth=1,linecolor=curcolor]
{
\newpath
\moveto(248.8,304.2)
\lineto(236.7,256)
}
}
{
\newrgbcolor{curcolor}{0.28627452 0 1}
\pscustom[linewidth=1,linecolor=curcolor]
{
\newpath
\moveto(236.7,256)
\lineto(224.6,236.6)
}
}
{
\newrgbcolor{curcolor}{0.17647059 0 1}
\pscustom[linewidth=1,linecolor=curcolor]
{
\newpath
\moveto(224.6,236.6)
\lineto(212.6,232.2)
}
}
{
\newrgbcolor{curcolor}{0.23529412 0 1}
\pscustom[linewidth=1,linecolor=curcolor]
{
\newpath
\moveto(212.6,232.2)
\lineto(200.5,226.7)
}
}
{
\newrgbcolor{curcolor}{0.25882354 0 1}
\pscustom[linewidth=1,linecolor=curcolor]
{
\newpath
\moveto(200.5,226.7)
\lineto(188.4,219.2)
}
}
{
\newrgbcolor{curcolor}{0.30980393 0 1}
\pscustom[linewidth=1,linecolor=curcolor]
{
\newpath
\moveto(188.4,219.2)
\lineto(176.4,215.8)
}
}
{
\newrgbcolor{curcolor}{0.93333334 0.25490198 0.74509805}
\pscustom[linewidth=1,linecolor=curcolor]
{
\newpath
\moveto(378.3,327.9)
\lineto(366.3,317.7)
}
}
{
\newrgbcolor{curcolor}{1 0.42352942 0.57647061}
\pscustom[linewidth=1,linecolor=curcolor]
{
\newpath
\moveto(366.3,317.7)
\lineto(354.2,334.6)
}
}
{
\newrgbcolor{curcolor}{1 0.65882355 0.34117648}
\pscustom[linewidth=1,linecolor=curcolor]
{
\newpath
\moveto(354.2,334.6)
\lineto(342.2,332.9)
}
}
{
\newrgbcolor{curcolor}{1 0.72549021 0.27450982}
\pscustom[linewidth=1,linecolor=curcolor]
{
\newpath
\moveto(342.2,332.9)
\lineto(330.1,328.7)
}
}
{
\newrgbcolor{curcolor}{1 0.7019608 0.29803923}
\pscustom[linewidth=1,linecolor=curcolor]
{
\newpath
\moveto(330.1,328.7)
\lineto(318,314.7)
}
}
{
\newrgbcolor{curcolor}{1 0.72156864 0.27843139}
\pscustom[linewidth=1,linecolor=curcolor]
{
\newpath
\moveto(318,314.7)
\lineto(306,316.6)
}
}
{
\newrgbcolor{curcolor}{1 0.41176471 0.58823532}
\pscustom[linewidth=1,linecolor=curcolor]
{
\newpath
\moveto(306,316.6)
\lineto(294,260.2)
}
}
{
\newrgbcolor{curcolor}{0.3764706 0 1}
\pscustom[linewidth=1,linecolor=curcolor]
{
\newpath
\moveto(294,260.2)
\lineto(282,236.1)
}
}
{
\newrgbcolor{curcolor}{0.17647059 0 1}
\pscustom[linewidth=1,linecolor=curcolor]
{
\newpath
\moveto(282,236.1)
\lineto(269.9,228.9)
}
}
{
\newrgbcolor{curcolor}{0.24313726 0 1}
\pscustom[linewidth=1,linecolor=curcolor]
{
\newpath
\moveto(269.9,228.9)
\lineto(257.8,226.7)
}
}
{
\newrgbcolor{curcolor}{0.23137255 0 1}
\pscustom[linewidth=1,linecolor=curcolor]
{
\newpath
\moveto(257.8,226.7)
\lineto(245.8,213.1)
}
}
{
\newrgbcolor{curcolor}{0.21568628 0 1}
\pscustom[linewidth=1,linecolor=curcolor]
{
\newpath
\moveto(245.8,213.1)
\lineto(233.7,210.5)
}
}
{
\newrgbcolor{curcolor}{0.32156864 0 1}
\pscustom[linewidth=1,linecolor=curcolor]
{
\newpath
\moveto(233.7,210.5)
\lineto(221.7,207.1)
}
}
{
\newrgbcolor{curcolor}{1 0.49019608 0.50980395}
\pscustom[linewidth=1,linecolor=curcolor]
{
\newpath
\moveto(423.6,332.8)
\lineto(411.6,324.1)
}
}
{
\newrgbcolor{curcolor}{1 0.58039218 0.41960785}
\pscustom[linewidth=1,linecolor=curcolor]
{
\newpath
\moveto(411.6,324.1)
\lineto(399.5,329.8)
}
}
{
\newrgbcolor{curcolor}{1 0.70588237 0.29411766}
\pscustom[linewidth=1,linecolor=curcolor]
{
\newpath
\moveto(399.5,329.8)
\lineto(387.4,325.2)
}
}
{
\newrgbcolor{curcolor}{1 0.65490198 0.34509805}
\pscustom[linewidth=1,linecolor=curcolor]
{
\newpath
\moveto(387.4,325.2)
\lineto(375.4,308.3)
}
}
{
\newrgbcolor{curcolor}{1 0.627451 0.37254903}
\pscustom[linewidth=1,linecolor=curcolor]
{
\newpath
\moveto(375.4,308.3)
\lineto(363.3,307.3)
}
}
{
\newrgbcolor{curcolor}{1 0.72941178 0.27058825}
\pscustom[linewidth=1,linecolor=curcolor]
{
\newpath
\moveto(363.3,307.3)
\lineto(351.2,306.6)
}
}
{
\newrgbcolor{curcolor}{1 0.39215687 0.60784316}
\pscustom[linewidth=1,linecolor=curcolor]
{
\newpath
\moveto(351.2,306.6)
\lineto(339.2,248.9)
}
}
{
\newrgbcolor{curcolor}{0.47058824 0 1}
\pscustom[linewidth=1,linecolor=curcolor]
{
\newpath
\moveto(339.2,248.9)
\lineto(327.1,236.4)
}
}
{
\newrgbcolor{curcolor}{0.39607844 0 1}
\pscustom[linewidth=1,linecolor=curcolor]
{
\newpath
\moveto(327.1,236.4)
\lineto(315.1,228)
}
}
{
\newrgbcolor{curcolor}{0.34509805 0 1}
\pscustom[linewidth=1,linecolor=curcolor]
{
\newpath
\moveto(315.1,228)
\lineto(303,217.5)
}
}
{
\newrgbcolor{curcolor}{0.25882354 0 1}
\pscustom[linewidth=1,linecolor=curcolor]
{
\newpath
\moveto(303,217.5)
\lineto(291,206)
}
}
{
\newrgbcolor{curcolor}{0.22745098 0 1}
\pscustom[linewidth=1,linecolor=curcolor]
{
\newpath
\moveto(291,206)
\lineto(279,200.1)
}
}
{
\newrgbcolor{curcolor}{0.24705882 0 1}
\pscustom[linewidth=1,linecolor=curcolor]
{
\newpath
\moveto(279,200.1)
\lineto(266.9,192.7)
}
}
{
\newrgbcolor{curcolor}{1 0.32941177 0.67058825}
\pscustom[linewidth=1,linecolor=curcolor]
{
\newpath
\moveto(468.9,323.6)
\lineto(456.8,294.5)
}
}
{
\newrgbcolor{curcolor}{1 0.33333334 0.66666669}
\pscustom[linewidth=1,linecolor=curcolor]
{
\newpath
\moveto(456.8,294.5)
\lineto(444.8,309)
}
}
{
\newrgbcolor{curcolor}{1 0.64313728 0.35686275}
\pscustom[linewidth=1,linecolor=curcolor]
{
\newpath
\moveto(444.8,309)
\lineto(432.7,319.4)
}
}
{
\newrgbcolor{curcolor}{1 0.7019608 0.29803923}
\pscustom[linewidth=1,linecolor=curcolor]
{
\newpath
\moveto(432.7,319.4)
\lineto(420.6,302)
}
}
{
\newrgbcolor{curcolor}{1 0.59215689 0.40784314}
\pscustom[linewidth=1,linecolor=curcolor]
{
\newpath
\moveto(420.6,302)
\lineto(408.6,290.5)
}
}
{
\newrgbcolor{curcolor}{1 0.71764708 0.28235295}
\pscustom[linewidth=1,linecolor=curcolor]
{
\newpath
\moveto(408.6,290.5)
\lineto(396.5,303.2)
}
}
{
\newrgbcolor{curcolor}{1 0.40784314 0.59215689}
\pscustom[linewidth=1,linecolor=curcolor]
{
\newpath
\moveto(396.5,303.2)
\lineto(384.4,236)
}
}
{
\newrgbcolor{curcolor}{0.40784314 0 1}
\pscustom[linewidth=1,linecolor=curcolor]
{
\newpath
\moveto(384.4,236)
\lineto(372.4,225.9)
}
}
{
\newrgbcolor{curcolor}{0.38039216 0 1}
\pscustom[linewidth=1,linecolor=curcolor]
{
\newpath
\moveto(372.4,225.9)
\lineto(360.3,218.6)
}
}
{
\newrgbcolor{curcolor}{0.32156864 0 1}
\pscustom[linewidth=1,linecolor=curcolor]
{
\newpath
\moveto(360.3,218.6)
\lineto(348.3,206.2)
}
}
{
\newrgbcolor{curcolor}{0.3137255 0 1}
\pscustom[linewidth=1,linecolor=curcolor]
{
\newpath
\moveto(348.3,206.2)
\lineto(336.2,203.1)
}
}
{
\newrgbcolor{curcolor}{0.3882353 0 1}
\pscustom[linewidth=1,linecolor=curcolor]
{
\newpath
\moveto(336.2,203.1)
\lineto(324.1,197.5)
}
}
{
\newrgbcolor{curcolor}{0.42352942 0 1}
\pscustom[linewidth=1,linecolor=curcolor]
{
\newpath
\moveto(324.1,197.5)
\lineto(312.1,191.4)
}
}
{
\newrgbcolor{curcolor}{0.93333334 0.25882354 0.74117649}
\pscustom[linewidth=1,linecolor=curcolor]
{
\newpath
\moveto(514.1,304.9)
\lineto(502.1,285.2)
}
}
{
\newrgbcolor{curcolor}{1 0.37254903 0.627451}
\pscustom[linewidth=1,linecolor=curcolor]
{
\newpath
\moveto(502.1,285.2)
\lineto(490,305)
}
}
{
\newrgbcolor{curcolor}{1 0.63137257 0.36862746}
\pscustom[linewidth=1,linecolor=curcolor]
{
\newpath
\moveto(490,305)
\lineto(478,303.5)
}
}
{
\newrgbcolor{curcolor}{1 0.6901961 0.30980393}
\pscustom[linewidth=1,linecolor=curcolor]
{
\newpath
\moveto(478,303.5)
\lineto(465.9,297.8)
}
}
{
\newrgbcolor{curcolor}{1 0.66666669 0.33333334}
\pscustom[linewidth=1,linecolor=curcolor]
{
\newpath
\moveto(465.9,297.8)
\lineto(453.8,285.6)
}
}
{
\newrgbcolor{curcolor}{1 0.64705884 0.35294119}
\pscustom[linewidth=1,linecolor=curcolor]
{
\newpath
\moveto(453.8,285.6)
\lineto(441.8,280.2)
}
}
{
\newrgbcolor{curcolor}{1 0.34509805 0.65490198}
\pscustom[linewidth=1,linecolor=curcolor]
{
\newpath
\moveto(441.8,280.2)
\lineto(429.7,232.3)
}
}
{
\newrgbcolor{curcolor}{0.43529412 0 1}
\pscustom[linewidth=1,linecolor=curcolor]
{
\newpath
\moveto(429.7,232.3)
\lineto(417.7,213)
}
}
{
\newrgbcolor{curcolor}{0.27450982 0 1}
\pscustom[linewidth=1,linecolor=curcolor]
{
\newpath
\moveto(417.7,213)
\lineto(405.6,204.4)
}
}
{
\newrgbcolor{curcolor}{0.16862746 0 1}
\pscustom[linewidth=1,linecolor=curcolor]
{
\newpath
\moveto(405.6,204.4)
\lineto(393.5,189.3)
}
}
{
\newrgbcolor{curcolor}{0.14117648 0 1}
\pscustom[linewidth=1,linecolor=curcolor]
{
\newpath
\moveto(393.5,189.3)
\lineto(381.5,187.4)
}
}
{
\newrgbcolor{curcolor}{0.28235295 0 1}
\pscustom[linewidth=1,linecolor=curcolor]
{
\newpath
\moveto(381.5,187.4)
\lineto(369.4,186)
}
}
{
\newrgbcolor{curcolor}{0.34117648 0 1}
\pscustom[linewidth=1,linecolor=curcolor]
{
\newpath
\moveto(369.4,186)
\lineto(357.3,177.5)
}
}
{
\newrgbcolor{curcolor}{0 0 0}
\pscustom[linewidth=1,linecolor=curcolor]
{
\newpath
\moveto(514.1,164)
\lineto(357.3,67.6)
}
}
{
\newrgbcolor{curcolor}{0 0 0}
\pscustom[linewidth=1,linecolor=curcolor]
{
\newpath
\moveto(85.9,123.2)
\lineto(357.3,67.6)
}
}
{
\newrgbcolor{curcolor}{0 0 0}
\pscustom[linewidth=1,linecolor=curcolor]
{
\newpath
\moveto(357.3,67.6)
\lineto(357.3,177.5)
}
}
{
\newrgbcolor{curcolor}{0 0 0}
\pscustom[linestyle=none,fillstyle=solid,fillcolor=curcolor]
{
\newpath
\moveto(146.27753906,58.7)
\lineto(146.27753906,67.28984375)
\lineto(152.07246094,67.28984375)
\lineto(152.07246094,66.27617187)
\lineto(147.41425781,66.27617187)
\lineto(147.41425781,63.61601562)
\lineto(151.44550781,63.61601562)
\lineto(151.44550781,62.60234375)
\lineto(147.41425781,62.60234375)
\lineto(147.41425781,58.7)
\closepath
}
}
{
\newrgbcolor{curcolor}{0 0 0}
\pscustom[linestyle=none,fillstyle=solid,fillcolor=curcolor]
{
\newpath
\moveto(153.42011719,66.07695312)
\lineto(153.42011719,67.28984375)
\lineto(154.47480469,67.28984375)
\lineto(154.47480469,66.07695312)
\closepath
\moveto(153.42011719,58.7)
\lineto(153.42011719,64.92265625)
\lineto(154.47480469,64.92265625)
\lineto(154.47480469,58.7)
\closepath
}
}
{
\newrgbcolor{curcolor}{0 0 0}
\pscustom[linestyle=none,fillstyle=solid,fillcolor=curcolor]
{
\newpath
\moveto(156.05683594,58.7)
\lineto(156.05683594,67.28984375)
\lineto(157.11152344,67.28984375)
\lineto(157.11152344,58.7)
\closepath
}
}
{
\newrgbcolor{curcolor}{0 0 0}
\pscustom[linestyle=none,fillstyle=solid,fillcolor=curcolor]
{
\newpath
\moveto(163.00605469,60.70390625)
\lineto(164.09589844,60.56914062)
\curveto(163.92402344,59.93242187)(163.60566406,59.43828125)(163.14082031,59.08671875)
\curveto(162.67597656,58.73515625)(162.08222656,58.559375)(161.35957031,58.559375)
\curveto(160.44941406,58.559375)(159.72675781,58.83867187)(159.19160156,59.39726562)
\curveto(158.66035156,59.95976562)(158.39472656,60.746875)(158.39472656,61.75859375)
\curveto(158.39472656,62.80546875)(158.66425781,63.61796875)(159.20332031,64.19609375)
\curveto(159.74238281,64.77421875)(160.44160156,65.06328125)(161.30097656,65.06328125)
\curveto(162.13300781,65.06328125)(162.81269531,64.78007812)(163.34003906,64.21367187)
\curveto(163.86738281,63.64726562)(164.13105469,62.85039062)(164.13105469,61.82304687)
\curveto(164.13105469,61.76054687)(164.12910156,61.66679687)(164.12519531,61.54179687)
\lineto(159.48457031,61.54179687)
\curveto(159.52363281,60.85820312)(159.71699219,60.33476562)(160.06464844,59.97148437)
\curveto(160.41230469,59.60820312)(160.84589844,59.4265625)(161.36542969,59.4265625)
\curveto(161.75214844,59.4265625)(162.08222656,59.528125)(162.35566406,59.73125)
\curveto(162.62910156,59.934375)(162.84589844,60.25859375)(163.00605469,60.70390625)
\closepath
\moveto(159.54316406,62.40898437)
\lineto(163.01777344,62.40898437)
\curveto(162.97089844,62.93242187)(162.83808594,63.325)(162.61933594,63.58671875)
\curveto(162.28339844,63.99296875)(161.84785156,64.19609375)(161.31269531,64.19609375)
\curveto(160.82832031,64.19609375)(160.42011719,64.03398437)(160.08808594,63.70976562)
\curveto(159.75996094,63.38554687)(159.57832031,62.95195312)(159.54316406,62.40898437)
\closepath
}
}
{
\newrgbcolor{curcolor}{0 0 0}
\pscustom[linestyle=none,fillstyle=solid,fillcolor=curcolor]
{
\newpath
\moveto(168.50214844,61.45976562)
\lineto(169.57441406,61.55351562)
\curveto(169.62519531,61.12382812)(169.74238281,60.7703125)(169.92597656,60.49296875)
\curveto(170.11347656,60.21953125)(170.40253906,59.996875)(170.79316406,59.825)
\curveto(171.18378906,59.65703125)(171.62324219,59.57304687)(172.11152344,59.57304687)
\curveto(172.54511719,59.57304687)(172.92792969,59.6375)(173.25996094,59.76640625)
\curveto(173.59199219,59.8953125)(173.83808594,60.07109375)(173.99824219,60.29375)
\curveto(174.16230469,60.5203125)(174.24433594,60.76640625)(174.24433594,61.03203125)
\curveto(174.24433594,61.3015625)(174.16621094,61.5359375)(174.00996094,61.73515625)
\curveto(173.85371094,61.93828125)(173.59589844,62.10820312)(173.23652344,62.24492187)
\curveto(173.00605469,62.33476562)(172.49628906,62.4734375)(171.70722656,62.6609375)
\curveto(170.91816406,62.85234375)(170.36542969,63.03203125)(170.04902344,63.2)
\curveto(169.63886719,63.41484375)(169.33222656,63.68046875)(169.12910156,63.996875)
\curveto(168.92988281,64.3171875)(168.83027344,64.67460937)(168.83027344,65.06914062)
\curveto(168.83027344,65.50273437)(168.95332031,65.90703125)(169.19941406,66.28203125)
\curveto(169.44550781,66.6609375)(169.80488281,66.94804687)(170.27753906,67.14335937)
\curveto(170.75019531,67.33867187)(171.27558594,67.43632812)(171.85371094,67.43632812)
\curveto(172.49042969,67.43632812)(173.05097656,67.3328125)(173.53535156,67.12578125)
\curveto(174.02363281,66.92265625)(174.39863281,66.621875)(174.66035156,66.2234375)
\curveto(174.92207031,65.825)(175.06269531,65.37382812)(175.08222656,64.86992187)
\lineto(173.99238281,64.78789062)
\curveto(173.93378906,65.33085937)(173.73457031,65.74101562)(173.39472656,66.01835937)
\curveto(173.05878906,66.29570312)(172.56074219,66.434375)(171.90058594,66.434375)
\curveto(171.21308594,66.434375)(170.71113281,66.30742187)(170.39472656,66.05351562)
\curveto(170.08222656,65.80351562)(169.92597656,65.50078125)(169.92597656,65.1453125)
\curveto(169.92597656,64.83671875)(170.03730469,64.5828125)(170.25996094,64.38359375)
\curveto(170.47871094,64.184375)(171.04902344,63.97929687)(171.97089844,63.76835937)
\curveto(172.89667969,63.56132812)(173.53144531,63.3796875)(173.87519531,63.2234375)
\curveto(174.37519531,62.99296875)(174.74433594,62.7)(174.98261719,62.34453125)
\curveto(175.22089844,61.99296875)(175.34003906,61.58671875)(175.34003906,61.12578125)
\curveto(175.34003906,60.66875)(175.20917969,60.23710937)(174.94746094,59.83085937)
\curveto(174.68574219,59.42851562)(174.30878906,59.1140625)(173.81660156,58.8875)
\curveto(173.32832031,58.66484375)(172.77753906,58.55351562)(172.16425781,58.55351562)
\curveto(171.38691406,58.55351562)(170.73457031,58.66679687)(170.20722656,58.89335937)
\curveto(169.68378906,59.11992187)(169.27167969,59.45976562)(168.97089844,59.91289062)
\curveto(168.67402344,60.36992187)(168.51777344,60.88554687)(168.50214844,61.45976562)
\closepath
}
}
{
\newrgbcolor{curcolor}{0 0 0}
\pscustom[linestyle=none,fillstyle=solid,fillcolor=curcolor]
{
\newpath
\moveto(176.76386719,66.07695312)
\lineto(176.76386719,67.28984375)
\lineto(177.81855469,67.28984375)
\lineto(177.81855469,66.07695312)
\closepath
\moveto(176.76386719,58.7)
\lineto(176.76386719,64.92265625)
\lineto(177.81855469,64.92265625)
\lineto(177.81855469,58.7)
\closepath
}
}
{
\newrgbcolor{curcolor}{0 0 0}
\pscustom[linestyle=none,fillstyle=solid,fillcolor=curcolor]
{
\newpath
\moveto(178.86738281,58.7)
\lineto(178.86738281,59.55546875)
\lineto(182.82832031,64.10234375)
\curveto(182.37910156,64.07890625)(181.98261719,64.0671875)(181.63886719,64.0671875)
\lineto(179.10175781,64.0671875)
\lineto(179.10175781,64.92265625)
\lineto(184.18769531,64.92265625)
\lineto(184.18769531,64.22539062)
\lineto(180.81855469,60.27617187)
\lineto(180.16816406,59.55546875)
\curveto(180.64082031,59.590625)(181.08417969,59.60820312)(181.49824219,59.60820312)
\lineto(184.37519531,59.60820312)
\lineto(184.37519531,58.7)
\closepath
}
}
{
\newrgbcolor{curcolor}{0 0 0}
\pscustom[linestyle=none,fillstyle=solid,fillcolor=curcolor]
{
\newpath
\moveto(189.68378906,60.70390625)
\lineto(190.77363281,60.56914062)
\curveto(190.60175781,59.93242187)(190.28339844,59.43828125)(189.81855469,59.08671875)
\curveto(189.35371094,58.73515625)(188.75996094,58.559375)(188.03730469,58.559375)
\curveto(187.12714844,58.559375)(186.40449219,58.83867187)(185.86933594,59.39726562)
\curveto(185.33808594,59.95976562)(185.07246094,60.746875)(185.07246094,61.75859375)
\curveto(185.07246094,62.80546875)(185.34199219,63.61796875)(185.88105469,64.19609375)
\curveto(186.42011719,64.77421875)(187.11933594,65.06328125)(187.97871094,65.06328125)
\curveto(188.81074219,65.06328125)(189.49042969,64.78007812)(190.01777344,64.21367187)
\curveto(190.54511719,63.64726562)(190.80878906,62.85039062)(190.80878906,61.82304687)
\curveto(190.80878906,61.76054687)(190.80683594,61.66679687)(190.80292969,61.54179687)
\lineto(186.16230469,61.54179687)
\curveto(186.20136719,60.85820312)(186.39472656,60.33476562)(186.74238281,59.97148437)
\curveto(187.09003906,59.60820312)(187.52363281,59.4265625)(188.04316406,59.4265625)
\curveto(188.42988281,59.4265625)(188.75996094,59.528125)(189.03339844,59.73125)
\curveto(189.30683594,59.934375)(189.52363281,60.25859375)(189.68378906,60.70390625)
\closepath
\moveto(186.22089844,62.40898437)
\lineto(189.69550781,62.40898437)
\curveto(189.64863281,62.93242187)(189.51582031,63.325)(189.29707031,63.58671875)
\curveto(188.96113281,63.99296875)(188.52558594,64.19609375)(187.99042969,64.19609375)
\curveto(187.50605469,64.19609375)(187.09785156,64.03398437)(186.76582031,63.70976562)
\curveto(186.43769531,63.38554687)(186.25605469,62.95195312)(186.22089844,62.40898437)
\closepath
}
}
{
\newrgbcolor{curcolor}{0 0 0}
\pscustom[linestyle=none,fillstyle=solid,fillcolor=curcolor]
{
\newpath
\moveto(501.46074219,92.9)
\lineto(501.46074219,101.48984375)
\lineto(505.26933594,101.48984375)
\curveto(506.03496094,101.48984375)(506.61699219,101.41171875)(507.01542969,101.25546875)
\curveto(507.41386719,101.103125)(507.73222656,100.83164062)(507.97050781,100.44101562)
\curveto(508.20878906,100.05039062)(508.32792969,99.61875)(508.32792969,99.14609375)
\curveto(508.32792969,98.53671875)(508.13066406,98.02304687)(507.73613281,97.60507812)
\curveto(507.34160156,97.18710937)(506.73222656,96.92148437)(505.90800781,96.80820312)
\curveto(506.20878906,96.66367187)(506.43730469,96.52109375)(506.59355469,96.38046875)
\curveto(506.92558594,96.07578125)(507.24003906,95.69492187)(507.53691406,95.23789062)
\lineto(509.03105469,92.9)
\lineto(507.60136719,92.9)
\lineto(506.46464844,94.68710937)
\curveto(506.13261719,95.20273437)(505.85917969,95.59726562)(505.64433594,95.87070312)
\curveto(505.42949219,96.14414062)(505.23613281,96.33554687)(505.06425781,96.44492187)
\curveto(504.89628906,96.55429687)(504.72441406,96.63046875)(504.54863281,96.6734375)
\curveto(504.41972656,96.70078125)(504.20878906,96.71445312)(503.91582031,96.71445312)
\lineto(502.59746094,96.71445312)
\lineto(502.59746094,92.9)
\closepath
\moveto(502.59746094,97.69882812)
\lineto(505.04082031,97.69882812)
\curveto(505.56035156,97.69882812)(505.96660156,97.7515625)(506.25957031,97.85703125)
\curveto(506.55253906,97.96640625)(506.77519531,98.13828125)(506.92753906,98.37265625)
\curveto(507.07988281,98.6109375)(507.15605469,98.86875)(507.15605469,99.14609375)
\curveto(507.15605469,99.55234375)(507.00761719,99.88632812)(506.71074219,100.14804687)
\curveto(506.41777344,100.40976562)(505.95292969,100.540625)(505.31621094,100.540625)
\lineto(502.59746094,100.540625)
\closepath
}
}
{
\newrgbcolor{curcolor}{0 0 0}
\pscustom[linestyle=none,fillstyle=solid,fillcolor=curcolor]
{
\newpath
\moveto(514.23417969,94.90390625)
\lineto(515.32402344,94.76914062)
\curveto(515.15214844,94.13242187)(514.83378906,93.63828125)(514.36894531,93.28671875)
\curveto(513.90410156,92.93515625)(513.31035156,92.759375)(512.58769531,92.759375)
\curveto(511.67753906,92.759375)(510.95488281,93.03867187)(510.41972656,93.59726562)
\curveto(509.88847656,94.15976562)(509.62285156,94.946875)(509.62285156,95.95859375)
\curveto(509.62285156,97.00546875)(509.89238281,97.81796875)(510.43144531,98.39609375)
\curveto(510.97050781,98.97421875)(511.66972656,99.26328125)(512.52910156,99.26328125)
\curveto(513.36113281,99.26328125)(514.04082031,98.98007812)(514.56816406,98.41367187)
\curveto(515.09550781,97.84726562)(515.35917969,97.05039062)(515.35917969,96.02304687)
\curveto(515.35917969,95.96054687)(515.35722656,95.86679687)(515.35332031,95.74179687)
\lineto(510.71269531,95.74179687)
\curveto(510.75175781,95.05820312)(510.94511719,94.53476562)(511.29277344,94.17148437)
\curveto(511.64042969,93.80820312)(512.07402344,93.6265625)(512.59355469,93.6265625)
\curveto(512.98027344,93.6265625)(513.31035156,93.728125)(513.58378906,93.93125)
\curveto(513.85722656,94.134375)(514.07402344,94.45859375)(514.23417969,94.90390625)
\closepath
\moveto(510.77128906,96.60898437)
\lineto(514.24589844,96.60898437)
\curveto(514.19902344,97.13242187)(514.06621094,97.525)(513.84746094,97.78671875)
\curveto(513.51152344,98.19296875)(513.07597656,98.39609375)(512.54082031,98.39609375)
\curveto(512.05644531,98.39609375)(511.64824219,98.23398437)(511.31621094,97.90976562)
\curveto(510.98808594,97.58554687)(510.80644531,97.15195312)(510.77128906,96.60898437)
\closepath
}
}
{
\newrgbcolor{curcolor}{0 0 0}
\pscustom[linestyle=none,fillstyle=solid,fillcolor=curcolor]
{
\newpath
\moveto(520.70878906,93.66757812)
\curveto(520.31816406,93.33554687)(519.94121094,93.10117187)(519.57792969,92.96445312)
\curveto(519.21855469,92.82773437)(518.83183594,92.759375)(518.41777344,92.759375)
\curveto(517.73417969,92.759375)(517.20878906,92.92539062)(516.84160156,93.25742187)
\curveto(516.47441406,93.59335937)(516.29082031,94.02109375)(516.29082031,94.540625)
\curveto(516.29082031,94.8453125)(516.35917969,95.12265625)(516.49589844,95.37265625)
\curveto(516.63652344,95.6265625)(516.81816406,95.8296875)(517.04082031,95.98203125)
\curveto(517.26738281,96.134375)(517.52128906,96.24960937)(517.80253906,96.32773437)
\curveto(518.00957031,96.38242187)(518.32207031,96.43515625)(518.74003906,96.4859375)
\curveto(519.59160156,96.5875)(520.21855469,96.70859375)(520.62089844,96.84921875)
\curveto(520.62480469,96.99375)(520.62675781,97.08554687)(520.62675781,97.12460937)
\curveto(520.62675781,97.55429687)(520.52714844,97.85703125)(520.32792969,98.0328125)
\curveto(520.05839844,98.27109375)(519.65800781,98.39023437)(519.12675781,98.39023437)
\curveto(518.63066406,98.39023437)(518.26347656,98.30234375)(518.02519531,98.1265625)
\curveto(517.79082031,97.9546875)(517.61699219,97.64804687)(517.50371094,97.20664062)
\lineto(516.47246094,97.34726562)
\curveto(516.56621094,97.78867187)(516.72050781,98.14414062)(516.93535156,98.41367187)
\curveto(517.15019531,98.68710937)(517.46074219,98.89609375)(517.86699219,99.040625)
\curveto(518.27324219,99.1890625)(518.74394531,99.26328125)(519.27910156,99.26328125)
\curveto(519.81035156,99.26328125)(520.24199219,99.20078125)(520.57402344,99.07578125)
\curveto(520.90605469,98.95078125)(521.15019531,98.79257812)(521.30644531,98.60117187)
\curveto(521.46269531,98.41367187)(521.57207031,98.17539062)(521.63457031,97.88632812)
\curveto(521.66972656,97.70664062)(521.68730469,97.38242187)(521.68730469,96.91367187)
\lineto(521.68730469,95.50742187)
\curveto(521.68730469,94.52695312)(521.70878906,93.90585937)(521.75175781,93.64414062)
\curveto(521.79863281,93.38632812)(521.88847656,93.13828125)(522.02128906,92.9)
\lineto(520.91972656,92.9)
\curveto(520.81035156,93.11875)(520.74003906,93.37460937)(520.70878906,93.66757812)
\closepath
\moveto(520.62089844,96.02304687)
\curveto(520.23808594,95.86679687)(519.66386719,95.73398437)(518.89824219,95.62460937)
\curveto(518.46464844,95.56210937)(518.15800781,95.49179687)(517.97832031,95.41367187)
\curveto(517.79863281,95.33554687)(517.65996094,95.2203125)(517.56230469,95.06796875)
\curveto(517.46464844,94.91953125)(517.41582031,94.75351562)(517.41582031,94.56992187)
\curveto(517.41582031,94.28867187)(517.52128906,94.05429687)(517.73222656,93.86679687)
\curveto(517.94707031,93.67929687)(518.25957031,93.58554687)(518.66972656,93.58554687)
\curveto(519.07597656,93.58554687)(519.43730469,93.6734375)(519.75371094,93.84921875)
\curveto(520.07011719,94.02890625)(520.30253906,94.27304687)(520.45097656,94.58164062)
\curveto(520.56425781,94.81992187)(520.62089844,95.17148437)(520.62089844,95.63632812)
\closepath
}
}
{
\newrgbcolor{curcolor}{0 0 0}
\pscustom[linestyle=none,fillstyle=solid,fillcolor=curcolor]
{
\newpath
\moveto(527.35917969,92.9)
\lineto(527.35917969,93.68515625)
\curveto(526.96464844,93.06796875)(526.38457031,92.759375)(525.61894531,92.759375)
\curveto(525.12285156,92.759375)(524.66582031,92.89609375)(524.24785156,93.16953125)
\curveto(523.83378906,93.44296875)(523.51152344,93.82382812)(523.28105469,94.31210937)
\curveto(523.05449219,94.80429687)(522.94121094,95.36875)(522.94121094,96.00546875)
\curveto(522.94121094,96.6265625)(523.04472656,97.1890625)(523.25175781,97.69296875)
\curveto(523.45878906,98.20078125)(523.76933594,98.58945312)(524.18339844,98.85898437)
\curveto(524.59746094,99.12851562)(525.06035156,99.26328125)(525.57207031,99.26328125)
\curveto(525.94707031,99.26328125)(526.28105469,99.18320312)(526.57402344,99.02304687)
\curveto(526.86699219,98.86679687)(527.10527344,98.66171875)(527.28886719,98.4078125)
\lineto(527.28886719,101.48984375)
\lineto(528.33769531,101.48984375)
\lineto(528.33769531,92.9)
\closepath
\moveto(524.02519531,96.00546875)
\curveto(524.02519531,95.20859375)(524.19316406,94.61289062)(524.52910156,94.21835937)
\curveto(524.86503906,93.82382812)(525.26152344,93.6265625)(525.71855469,93.6265625)
\curveto(526.17949219,93.6265625)(526.57011719,93.8140625)(526.89042969,94.1890625)
\curveto(527.21464844,94.56796875)(527.37675781,95.14414062)(527.37675781,95.91757812)
\curveto(527.37675781,96.76914062)(527.21269531,97.39414062)(526.88457031,97.79257812)
\curveto(526.55644531,98.19101562)(526.15214844,98.39023437)(525.67167969,98.39023437)
\curveto(525.20292969,98.39023437)(524.81035156,98.19882812)(524.49394531,97.81601562)
\curveto(524.18144531,97.43320312)(524.02519531,96.8296875)(524.02519531,96.00546875)
\closepath
}
}
{
\newrgbcolor{curcolor}{0 0 0}
\pscustom[linestyle=none,fillstyle=solid,fillcolor=curcolor]
{
\newpath
\moveto(533.07792969,95.65976562)
\lineto(534.15019531,95.75351562)
\curveto(534.20097656,95.32382812)(534.31816406,94.9703125)(534.50175781,94.69296875)
\curveto(534.68925781,94.41953125)(534.97832031,94.196875)(535.36894531,94.025)
\curveto(535.75957031,93.85703125)(536.19902344,93.77304687)(536.68730469,93.77304687)
\curveto(537.12089844,93.77304687)(537.50371094,93.8375)(537.83574219,93.96640625)
\curveto(538.16777344,94.0953125)(538.41386719,94.27109375)(538.57402344,94.49375)
\curveto(538.73808594,94.7203125)(538.82011719,94.96640625)(538.82011719,95.23203125)
\curveto(538.82011719,95.5015625)(538.74199219,95.7359375)(538.58574219,95.93515625)
\curveto(538.42949219,96.13828125)(538.17167969,96.30820312)(537.81230469,96.44492187)
\curveto(537.58183594,96.53476562)(537.07207031,96.6734375)(536.28300781,96.8609375)
\curveto(535.49394531,97.05234375)(534.94121094,97.23203125)(534.62480469,97.4)
\curveto(534.21464844,97.61484375)(533.90800781,97.88046875)(533.70488281,98.196875)
\curveto(533.50566406,98.5171875)(533.40605469,98.87460937)(533.40605469,99.26914062)
\curveto(533.40605469,99.70273437)(533.52910156,100.10703125)(533.77519531,100.48203125)
\curveto(534.02128906,100.8609375)(534.38066406,101.14804687)(534.85332031,101.34335937)
\curveto(535.32597656,101.53867187)(535.85136719,101.63632812)(536.42949219,101.63632812)
\curveto(537.06621094,101.63632812)(537.62675781,101.5328125)(538.11113281,101.32578125)
\curveto(538.59941406,101.12265625)(538.97441406,100.821875)(539.23613281,100.4234375)
\curveto(539.49785156,100.025)(539.63847656,99.57382812)(539.65800781,99.06992187)
\lineto(538.56816406,98.98789062)
\curveto(538.50957031,99.53085937)(538.31035156,99.94101562)(537.97050781,100.21835937)
\curveto(537.63457031,100.49570312)(537.13652344,100.634375)(536.47636719,100.634375)
\curveto(535.78886719,100.634375)(535.28691406,100.50742187)(534.97050781,100.25351562)
\curveto(534.65800781,100.00351562)(534.50175781,99.70078125)(534.50175781,99.3453125)
\curveto(534.50175781,99.03671875)(534.61308594,98.7828125)(534.83574219,98.58359375)
\curveto(535.05449219,98.384375)(535.62480469,98.17929687)(536.54667969,97.96835937)
\curveto(537.47246094,97.76132812)(538.10722656,97.5796875)(538.45097656,97.4234375)
\curveto(538.95097656,97.19296875)(539.32011719,96.9)(539.55839844,96.54453125)
\curveto(539.79667969,96.19296875)(539.91582031,95.78671875)(539.91582031,95.32578125)
\curveto(539.91582031,94.86875)(539.78496094,94.43710937)(539.52324219,94.03085937)
\curveto(539.26152344,93.62851562)(538.88457031,93.3140625)(538.39238281,93.0875)
\curveto(537.90410156,92.86484375)(537.35332031,92.75351562)(536.74003906,92.75351562)
\curveto(535.96269531,92.75351562)(535.31035156,92.86679687)(534.78300781,93.09335937)
\curveto(534.25957031,93.31992187)(533.84746094,93.65976562)(533.54667969,94.11289062)
\curveto(533.24980469,94.56992187)(533.09355469,95.08554687)(533.07792969,95.65976562)
\closepath
}
}
{
\newrgbcolor{curcolor}{0 0 0}
\pscustom[linestyle=none,fillstyle=solid,fillcolor=curcolor]
{
\newpath
\moveto(541.33964844,100.27695312)
\lineto(541.33964844,101.48984375)
\lineto(542.39433594,101.48984375)
\lineto(542.39433594,100.27695312)
\closepath
\moveto(541.33964844,92.9)
\lineto(541.33964844,99.12265625)
\lineto(542.39433594,99.12265625)
\lineto(542.39433594,92.9)
\closepath
}
}
{
\newrgbcolor{curcolor}{0 0 0}
\pscustom[linestyle=none,fillstyle=solid,fillcolor=curcolor]
{
\newpath
\moveto(543.44316406,92.9)
\lineto(543.44316406,93.75546875)
\lineto(547.40410156,98.30234375)
\curveto(546.95488281,98.27890625)(546.55839844,98.2671875)(546.21464844,98.2671875)
\lineto(543.67753906,98.2671875)
\lineto(543.67753906,99.12265625)
\lineto(548.76347656,99.12265625)
\lineto(548.76347656,98.42539062)
\lineto(545.39433594,94.47617187)
\lineto(544.74394531,93.75546875)
\curveto(545.21660156,93.790625)(545.65996094,93.80820312)(546.07402344,93.80820312)
\lineto(548.95097656,93.80820312)
\lineto(548.95097656,92.9)
\closepath
}
}
{
\newrgbcolor{curcolor}{0 0 0}
\pscustom[linestyle=none,fillstyle=solid,fillcolor=curcolor]
{
\newpath
\moveto(554.25957031,94.90390625)
\lineto(555.34941406,94.76914062)
\curveto(555.17753906,94.13242187)(554.85917969,93.63828125)(554.39433594,93.28671875)
\curveto(553.92949219,92.93515625)(553.33574219,92.759375)(552.61308594,92.759375)
\curveto(551.70292969,92.759375)(550.98027344,93.03867187)(550.44511719,93.59726562)
\curveto(549.91386719,94.15976562)(549.64824219,94.946875)(549.64824219,95.95859375)
\curveto(549.64824219,97.00546875)(549.91777344,97.81796875)(550.45683594,98.39609375)
\curveto(550.99589844,98.97421875)(551.69511719,99.26328125)(552.55449219,99.26328125)
\curveto(553.38652344,99.26328125)(554.06621094,98.98007812)(554.59355469,98.41367187)
\curveto(555.12089844,97.84726562)(555.38457031,97.05039062)(555.38457031,96.02304687)
\curveto(555.38457031,95.96054687)(555.38261719,95.86679687)(555.37871094,95.74179687)
\lineto(550.73808594,95.74179687)
\curveto(550.77714844,95.05820312)(550.97050781,94.53476562)(551.31816406,94.17148437)
\curveto(551.66582031,93.80820312)(552.09941406,93.6265625)(552.61894531,93.6265625)
\curveto(553.00566406,93.6265625)(553.33574219,93.728125)(553.60917969,93.93125)
\curveto(553.88261719,94.134375)(554.09941406,94.45859375)(554.25957031,94.90390625)
\closepath
\moveto(550.79667969,96.60898437)
\lineto(554.27128906,96.60898437)
\curveto(554.22441406,97.13242187)(554.09160156,97.525)(553.87285156,97.78671875)
\curveto(553.53691406,98.19296875)(553.10136719,98.39609375)(552.56621094,98.39609375)
\curveto(552.08183594,98.39609375)(551.67363281,98.23398437)(551.34160156,97.90976562)
\curveto(551.01347656,97.58554687)(550.83183594,97.15195312)(550.79667969,96.60898437)
\closepath
}
}
{
\newrgbcolor{curcolor}{0 0 0}
\pscustom[linestyle=none,fillstyle=solid,fillcolor=curcolor]
{
\newpath
\moveto(31.7,182.11992187)
\lineto(23.11015625,182.11992187)
\lineto(23.11015625,183.25664062)
\lineto(31.7,183.25664062)
\closepath
}
}
{
\newrgbcolor{curcolor}{0 0 0}
\pscustom[linestyle=none,fillstyle=solid,fillcolor=curcolor]
{
\newpath
\moveto(31.7,185.12578125)
\lineto(25.47734375,185.12578125)
\lineto(25.47734375,186.06914062)
\lineto(26.35039062,186.06914062)
\curveto(26.04570312,186.26445312)(25.8015625,186.52421875)(25.61796875,186.8484375)
\curveto(25.43046875,187.17265625)(25.33671875,187.54179687)(25.33671875,187.95585937)
\curveto(25.33671875,188.41679687)(25.43242187,188.79375)(25.62382812,189.08671875)
\curveto(25.81523437,189.38359375)(26.0828125,189.59257812)(26.4265625,189.71367187)
\curveto(25.7,190.20585937)(25.33671875,190.84648437)(25.33671875,191.63554687)
\curveto(25.33671875,192.25273437)(25.50859375,192.72734375)(25.85234375,193.059375)
\curveto(26.1921875,193.39140625)(26.71757812,193.55742187)(27.42851562,193.55742187)
\lineto(31.7,193.55742187)
\lineto(31.7,192.50859375)
\lineto(27.78007812,192.50859375)
\curveto(27.35820312,192.50859375)(27.05546875,192.4734375)(26.871875,192.403125)
\curveto(26.684375,192.33671875)(26.53398437,192.21367187)(26.42070312,192.03398437)
\curveto(26.30742187,191.85429687)(26.25078125,191.64335937)(26.25078125,191.40117187)
\curveto(26.25078125,190.96367187)(26.39726562,190.60039062)(26.69023437,190.31132812)
\curveto(26.97929687,190.02226562)(27.44414062,189.87773437)(28.08476562,189.87773437)
\lineto(31.7,189.87773437)
\lineto(31.7,188.82304687)
\lineto(27.65703125,188.82304687)
\curveto(27.18828125,188.82304687)(26.83671875,188.73710937)(26.60234375,188.56523437)
\curveto(26.36796875,188.39335937)(26.25078125,188.11210937)(26.25078125,187.72148437)
\curveto(26.25078125,187.42460937)(26.32890625,187.14921875)(26.48515625,186.8953125)
\curveto(26.64140625,186.6453125)(26.86992187,186.46367187)(27.17070312,186.35039062)
\curveto(27.47148437,186.23710937)(27.90507812,186.18046875)(28.47148437,186.18046875)
\lineto(31.7,186.18046875)
\closepath
}
}
{
\newrgbcolor{curcolor}{0 0 0}
\pscustom[linestyle=none,fillstyle=solid,fillcolor=curcolor]
{
\newpath
\moveto(34.08476563,195.121875)
\lineto(25.47734375,195.121875)
\lineto(25.47734375,196.0828125)
\lineto(26.2859375,196.0828125)
\curveto(25.96953125,196.309375)(25.73320312,196.56523437)(25.57695312,196.85039062)
\curveto(25.41679687,197.13554687)(25.33671875,197.48125)(25.33671875,197.8875)
\curveto(25.33671875,198.41875)(25.4734375,198.8875)(25.746875,199.29375)
\curveto(26.0203125,199.7)(26.40703125,200.00664062)(26.90703125,200.21367187)
\curveto(27.403125,200.42070312)(27.94804687,200.52421875)(28.54179687,200.52421875)
\curveto(29.17851562,200.52421875)(29.75273437,200.40898437)(30.26445312,200.17851562)
\curveto(30.77226562,199.95195312)(31.16289062,199.61992187)(31.43632812,199.18242187)
\curveto(31.70585937,198.74882812)(31.840625,198.29179687)(31.840625,197.81132812)
\curveto(31.840625,197.45976562)(31.76640625,197.14335937)(31.61796875,196.86210937)
\curveto(31.46953125,196.58476562)(31.28203125,196.35625)(31.05546875,196.1765625)
\lineto(34.08476563,196.1765625)
\closepath
\moveto(28.62382812,196.07695312)
\curveto(29.42460937,196.07695312)(30.01640625,196.2390625)(30.39921875,196.56328125)
\curveto(30.78203125,196.8875)(30.9734375,197.28007812)(30.9734375,197.74101562)
\curveto(30.9734375,198.20976562)(30.77617187,198.61015625)(30.38164062,198.9421875)
\curveto(29.98320312,199.278125)(29.36796875,199.44609375)(28.5359375,199.44609375)
\curveto(27.74296875,199.44609375)(27.14921875,199.28203125)(26.7546875,198.95390625)
\curveto(26.36015625,198.6296875)(26.16289062,198.24101562)(26.16289062,197.78789062)
\curveto(26.16289062,197.33867187)(26.37382812,196.94023437)(26.79570312,196.59257812)
\curveto(27.21367187,196.24882812)(27.82304687,196.07695312)(28.62382812,196.07695312)
\closepath
}
}
{
\newrgbcolor{curcolor}{0 0 0}
\pscustom[linestyle=none,fillstyle=solid,fillcolor=curcolor]
{
\newpath
\moveto(31.7,201.78398437)
\lineto(25.47734375,201.78398437)
\lineto(25.47734375,202.73320312)
\lineto(26.42070312,202.73320312)
\curveto(25.97929687,202.97539062)(25.68828125,203.19804687)(25.54765625,203.40117187)
\curveto(25.40703125,203.60820312)(25.33671875,203.83476562)(25.33671875,204.08085937)
\curveto(25.33671875,204.43632812)(25.45,204.79765625)(25.6765625,205.16484375)
\lineto(26.65507812,204.8015625)
\curveto(26.50273437,204.54375)(26.4265625,204.2859375)(26.4265625,204.028125)
\curveto(26.4265625,203.79765625)(26.496875,203.590625)(26.6375,203.40703125)
\curveto(26.77421875,203.2234375)(26.965625,203.09257812)(27.21171875,203.01445312)
\curveto(27.58671875,202.89726562)(27.996875,202.83867187)(28.4421875,202.83867187)
\lineto(31.7,202.83867187)
\closepath
}
}
{
\newrgbcolor{curcolor}{0 0 0}
\pscustom[linestyle=none,fillstyle=solid,fillcolor=curcolor]
{
\newpath
\moveto(28.58867187,205.39921875)
\curveto(27.43632812,205.39921875)(26.5828125,205.71953125)(26.028125,206.36015625)
\curveto(25.5671875,206.8953125)(25.33671875,207.54765625)(25.33671875,208.3171875)
\curveto(25.33671875,209.17265625)(25.61796875,209.871875)(26.18046875,210.41484375)
\curveto(26.7390625,210.9578125)(27.5125,211.22929687)(28.50078125,211.22929687)
\curveto(29.3015625,211.22929687)(29.93242187,211.10820312)(30.39335937,210.86601562)
\curveto(30.85039062,210.62773437)(31.20585937,210.278125)(31.45976562,209.8171875)
\curveto(31.71367187,209.36015625)(31.840625,208.86015625)(31.840625,208.3171875)
\curveto(31.840625,207.44609375)(31.56132812,206.74101562)(31.00273437,206.20195312)
\curveto(30.44414062,205.66679687)(29.63945312,205.39921875)(28.58867187,205.39921875)
\closepath
\moveto(28.58867187,206.48320312)
\curveto(29.38554687,206.48320312)(29.98320312,206.65703125)(30.38164062,207.0046875)
\curveto(30.77617187,207.35234375)(30.9734375,207.78984375)(30.9734375,208.3171875)
\curveto(30.9734375,208.840625)(30.77421875,209.27617187)(30.37578125,209.62382812)
\curveto(29.97734375,209.97148437)(29.36992187,210.1453125)(28.55351562,210.1453125)
\curveto(27.78398437,210.1453125)(27.20195312,209.96953125)(26.80742187,209.61796875)
\curveto(26.40898437,209.2703125)(26.20976562,208.83671875)(26.20976562,208.3171875)
\curveto(26.20976562,207.78984375)(26.40703125,207.35234375)(26.8015625,207.0046875)
\curveto(27.19609375,206.65703125)(27.79179687,206.48320312)(28.58867187,206.48320312)
\closepath
}
}
{
\newrgbcolor{curcolor}{0 0 0}
\pscustom[linestyle=none,fillstyle=solid,fillcolor=curcolor]
{
\newpath
\moveto(31.7,214.19414062)
\lineto(25.47734375,211.82695312)
\lineto(25.47734375,212.94023437)
\lineto(29.20390625,214.27617187)
\curveto(29.60625,214.42070312)(30.02421875,214.55351562)(30.4578125,214.67460937)
\curveto(30.1296875,214.76835937)(29.73515625,214.89921875)(29.27421875,215.0671875)
\lineto(25.47734375,216.45)
\lineto(25.47734375,217.53398437)
\lineto(31.7,215.17851562)
\closepath
}
}
{
\newrgbcolor{curcolor}{0 0 0}
\pscustom[linestyle=none,fillstyle=solid,fillcolor=curcolor]
{
\newpath
\moveto(29.69609375,222.72539062)
\lineto(29.83085937,223.81523437)
\curveto(30.46757812,223.64335937)(30.96171875,223.325)(31.31328125,222.86015625)
\curveto(31.66484375,222.3953125)(31.840625,221.8015625)(31.840625,221.07890625)
\curveto(31.840625,220.16875)(31.56132812,219.44609375)(31.00273437,218.9109375)
\curveto(30.44023437,218.3796875)(29.653125,218.1140625)(28.64140625,218.1140625)
\curveto(27.59453125,218.1140625)(26.78203125,218.38359375)(26.20390625,218.92265625)
\curveto(25.62578125,219.46171875)(25.33671875,220.1609375)(25.33671875,221.0203125)
\curveto(25.33671875,221.85234375)(25.61992187,222.53203125)(26.18632812,223.059375)
\curveto(26.75273437,223.58671875)(27.54960937,223.85039062)(28.57695312,223.85039062)
\curveto(28.63945312,223.85039062)(28.73320312,223.8484375)(28.85820312,223.84453125)
\lineto(28.85820312,219.20390625)
\curveto(29.54179687,219.24296875)(30.06523437,219.43632812)(30.42851562,219.78398437)
\curveto(30.79179687,220.13164062)(30.9734375,220.56523437)(30.9734375,221.08476562)
\curveto(30.9734375,221.47148437)(30.871875,221.8015625)(30.66875,222.075)
\curveto(30.465625,222.3484375)(30.14140625,222.56523437)(29.69609375,222.72539062)
\closepath
\moveto(27.99101562,219.2625)
\lineto(27.99101562,222.73710937)
\curveto(27.46757812,222.69023437)(27.075,222.55742187)(26.81328125,222.33867187)
\curveto(26.40703125,222.00273437)(26.20390625,221.5671875)(26.20390625,221.03203125)
\curveto(26.20390625,220.54765625)(26.36601562,220.13945312)(26.69023437,219.80742187)
\curveto(27.01445312,219.47929687)(27.44804687,219.29765625)(27.99101562,219.2625)
\closepath
}
}
{
\newrgbcolor{curcolor}{0 0 0}
\pscustom[linestyle=none,fillstyle=solid,fillcolor=curcolor]
{
\newpath
\moveto(31.7,225.13945312)
\lineto(25.47734375,225.13945312)
\lineto(25.47734375,226.0828125)
\lineto(26.35039062,226.0828125)
\curveto(26.04570312,226.278125)(25.8015625,226.53789062)(25.61796875,226.86210937)
\curveto(25.43046875,227.18632812)(25.33671875,227.55546875)(25.33671875,227.96953125)
\curveto(25.33671875,228.43046875)(25.43242187,228.80742187)(25.62382812,229.10039062)
\curveto(25.81523437,229.39726562)(26.0828125,229.60625)(26.4265625,229.72734375)
\curveto(25.7,230.21953125)(25.33671875,230.86015625)(25.33671875,231.64921875)
\curveto(25.33671875,232.26640625)(25.50859375,232.74101562)(25.85234375,233.07304687)
\curveto(26.1921875,233.40507812)(26.71757812,233.57109375)(27.42851562,233.57109375)
\lineto(31.7,233.57109375)
\lineto(31.7,232.52226562)
\lineto(27.78007812,232.52226562)
\curveto(27.35820312,232.52226562)(27.05546875,232.48710937)(26.871875,232.41679687)
\curveto(26.684375,232.35039062)(26.53398437,232.22734375)(26.42070312,232.04765625)
\curveto(26.30742187,231.86796875)(26.25078125,231.65703125)(26.25078125,231.41484375)
\curveto(26.25078125,230.97734375)(26.39726562,230.6140625)(26.69023437,230.325)
\curveto(26.97929687,230.0359375)(27.44414062,229.89140625)(28.08476562,229.89140625)
\lineto(31.7,229.89140625)
\lineto(31.7,228.83671875)
\lineto(27.65703125,228.83671875)
\curveto(27.18828125,228.83671875)(26.83671875,228.75078125)(26.60234375,228.57890625)
\curveto(26.36796875,228.40703125)(26.25078125,228.12578125)(26.25078125,227.73515625)
\curveto(26.25078125,227.43828125)(26.32890625,227.16289062)(26.48515625,226.90898437)
\curveto(26.64140625,226.65898437)(26.86992187,226.47734375)(27.17070312,226.3640625)
\curveto(27.47148437,226.25078125)(27.90507812,226.19414062)(28.47148437,226.19414062)
\lineto(31.7,226.19414062)
\closepath
}
}
{
\newrgbcolor{curcolor}{0 0 0}
\pscustom[linestyle=none,fillstyle=solid,fillcolor=curcolor]
{
\newpath
\moveto(29.69609375,239.3953125)
\lineto(29.83085937,240.48515625)
\curveto(30.46757812,240.31328125)(30.96171875,239.99492187)(31.31328125,239.53007812)
\curveto(31.66484375,239.06523437)(31.840625,238.47148437)(31.840625,237.74882812)
\curveto(31.840625,236.83867187)(31.56132812,236.11601562)(31.00273437,235.58085937)
\curveto(30.44023437,235.04960937)(29.653125,234.78398437)(28.64140625,234.78398437)
\curveto(27.59453125,234.78398437)(26.78203125,235.05351562)(26.20390625,235.59257812)
\curveto(25.62578125,236.13164062)(25.33671875,236.83085937)(25.33671875,237.69023437)
\curveto(25.33671875,238.52226562)(25.61992187,239.20195312)(26.18632812,239.72929687)
\curveto(26.75273437,240.25664062)(27.54960937,240.5203125)(28.57695312,240.5203125)
\curveto(28.63945312,240.5203125)(28.73320312,240.51835937)(28.85820312,240.51445312)
\lineto(28.85820312,235.87382812)
\curveto(29.54179687,235.91289062)(30.06523437,236.10625)(30.42851562,236.45390625)
\curveto(30.79179687,236.8015625)(30.9734375,237.23515625)(30.9734375,237.7546875)
\curveto(30.9734375,238.14140625)(30.871875,238.47148437)(30.66875,238.74492187)
\curveto(30.465625,239.01835937)(30.14140625,239.23515625)(29.69609375,239.3953125)
\closepath
\moveto(27.99101562,235.93242187)
\lineto(27.99101562,239.40703125)
\curveto(27.46757812,239.36015625)(27.075,239.22734375)(26.81328125,239.00859375)
\curveto(26.40703125,238.67265625)(26.20390625,238.23710937)(26.20390625,237.70195312)
\curveto(26.20390625,237.21757812)(26.36601562,236.809375)(26.69023437,236.47734375)
\curveto(27.01445312,236.14921875)(27.44804687,235.96757812)(27.99101562,235.93242187)
\closepath
}
}
{
\newrgbcolor{curcolor}{0 0 0}
\pscustom[linestyle=none,fillstyle=solid,fillcolor=curcolor]
{
\newpath
\moveto(31.7,241.809375)
\lineto(25.47734375,241.809375)
\lineto(25.47734375,242.75859375)
\lineto(26.36210937,242.75859375)
\curveto(25.67851562,243.215625)(25.33671875,243.87578125)(25.33671875,244.7390625)
\curveto(25.33671875,245.1140625)(25.40507812,245.4578125)(25.54179687,245.7703125)
\curveto(25.67460937,246.08671875)(25.85039062,246.32304687)(26.06914062,246.47929687)
\curveto(26.28789062,246.63554687)(26.54765625,246.74492187)(26.8484375,246.80742187)
\curveto(27.04375,246.84648437)(27.38554687,246.86601562)(27.87382812,246.86601562)
\lineto(31.7,246.86601562)
\lineto(31.7,245.81132812)
\lineto(27.91484375,245.81132812)
\curveto(27.48515625,245.81132812)(27.16484375,245.7703125)(26.95390625,245.68828125)
\curveto(26.7390625,245.60625)(26.56914062,245.45976562)(26.44414062,245.24882812)
\curveto(26.31523437,245.04179687)(26.25078125,244.79765625)(26.25078125,244.51640625)
\curveto(26.25078125,244.0671875)(26.39335937,243.67851562)(26.67851562,243.35039062)
\curveto(26.96367187,243.02617187)(27.5046875,242.8640625)(28.3015625,242.8640625)
\lineto(31.7,242.8640625)
\closepath
}
}
{
\newrgbcolor{curcolor}{0 0 0}
\pscustom[linestyle=none,fillstyle=solid,fillcolor=curcolor]
{
\newpath
\moveto(30.75664062,250.7859375)
\lineto(31.68828125,250.93828125)
\curveto(31.75078125,250.64140625)(31.78203125,250.37578125)(31.78203125,250.14140625)
\curveto(31.78203125,249.75859375)(31.72148437,249.46171875)(31.60039062,249.25078125)
\curveto(31.47929687,249.03984375)(31.32109375,248.89140625)(31.12578125,248.80546875)
\curveto(30.9265625,248.71953125)(30.51054687,248.6765625)(29.87773437,248.6765625)
\lineto(26.29765625,248.6765625)
\lineto(26.29765625,247.903125)
\lineto(25.47734375,247.903125)
\lineto(25.47734375,248.6765625)
\lineto(23.93632812,248.6765625)
\lineto(23.30351562,249.72539062)
\lineto(25.47734375,249.72539062)
\lineto(25.47734375,250.7859375)
\lineto(26.29765625,250.7859375)
\lineto(26.29765625,249.72539062)
\lineto(29.93632812,249.72539062)
\curveto(30.23710937,249.72539062)(30.43046875,249.74296875)(30.51640625,249.778125)
\curveto(30.60234375,249.8171875)(30.67070312,249.87773437)(30.72148437,249.95976562)
\curveto(30.77226562,250.04570312)(30.79765625,250.16679687)(30.79765625,250.32304687)
\curveto(30.79765625,250.44023437)(30.78398437,250.59453125)(30.75664062,250.7859375)
\closepath
}
}
{
\newrgbcolor{curcolor}{0 0 0}
\pscustom[linestyle=none,fillstyle=solid,fillcolor=curcolor]
{
\newpath
\moveto(34.22539063,257.16679687)
\curveto(33.49101563,256.58476562)(32.63164062,256.09257812)(31.64726562,255.69023437)
\curveto(30.66289062,255.28789062)(29.64335937,255.08671875)(28.58867187,255.08671875)
\curveto(27.65898437,255.08671875)(26.76835937,255.23710937)(25.91679687,255.53789062)
\curveto(24.92851562,255.88945312)(23.94414062,256.43242187)(22.96367187,257.16679687)
\lineto(22.96367187,257.92265625)
\curveto(23.77617187,257.45)(24.35625,257.1375)(24.70390625,256.98515625)
\curveto(25.24296875,256.746875)(25.80546875,256.559375)(26.39140625,256.42265625)
\curveto(27.121875,256.2546875)(27.85625,256.17070312)(28.59453125,256.17070312)
\curveto(30.4734375,256.17070312)(32.35039062,256.7546875)(34.22539063,257.92265625)
\closepath
}
}
{
\newrgbcolor{curcolor}{0 0 0}
\pscustom[linestyle=none,fillstyle=solid,fillcolor=curcolor]
{
\newpath
\moveto(34.08476563,259.14726562)
\lineto(25.47734375,259.14726562)
\lineto(25.47734375,260.10820312)
\lineto(26.2859375,260.10820312)
\curveto(25.96953125,260.33476562)(25.73320312,260.590625)(25.57695312,260.87578125)
\curveto(25.41679687,261.1609375)(25.33671875,261.50664062)(25.33671875,261.91289062)
\curveto(25.33671875,262.44414062)(25.4734375,262.91289062)(25.746875,263.31914062)
\curveto(26.0203125,263.72539062)(26.40703125,264.03203125)(26.90703125,264.2390625)
\curveto(27.403125,264.44609375)(27.94804687,264.54960937)(28.54179687,264.54960937)
\curveto(29.17851562,264.54960937)(29.75273437,264.434375)(30.26445312,264.20390625)
\curveto(30.77226562,263.97734375)(31.16289062,263.6453125)(31.43632812,263.2078125)
\curveto(31.70585937,262.77421875)(31.840625,262.3171875)(31.840625,261.83671875)
\curveto(31.840625,261.48515625)(31.76640625,261.16875)(31.61796875,260.8875)
\curveto(31.46953125,260.61015625)(31.28203125,260.38164062)(31.05546875,260.20195312)
\lineto(34.08476563,260.20195312)
\closepath
\moveto(28.62382812,260.10234375)
\curveto(29.42460937,260.10234375)(30.01640625,260.26445312)(30.39921875,260.58867187)
\curveto(30.78203125,260.91289062)(30.9734375,261.30546875)(30.9734375,261.76640625)
\curveto(30.9734375,262.23515625)(30.77617187,262.63554687)(30.38164062,262.96757812)
\curveto(29.98320312,263.30351562)(29.36796875,263.47148437)(28.5359375,263.47148437)
\curveto(27.74296875,263.47148437)(27.14921875,263.30742187)(26.7546875,262.97929687)
\curveto(26.36015625,262.65507812)(26.16289062,262.26640625)(26.16289062,261.81328125)
\curveto(26.16289062,261.3640625)(26.37382812,260.965625)(26.79570312,260.61796875)
\curveto(27.21367187,260.27421875)(27.82304687,260.10234375)(28.62382812,260.10234375)
\closepath
}
}
{
\newrgbcolor{curcolor}{0 0 0}
\pscustom[linestyle=none,fillstyle=solid,fillcolor=curcolor]
{
\newpath
\moveto(29.69609375,270.08085937)
\lineto(29.83085937,271.17070312)
\curveto(30.46757812,270.99882812)(30.96171875,270.68046875)(31.31328125,270.215625)
\curveto(31.66484375,269.75078125)(31.840625,269.15703125)(31.840625,268.434375)
\curveto(31.840625,267.52421875)(31.56132812,266.8015625)(31.00273437,266.26640625)
\curveto(30.44023437,265.73515625)(29.653125,265.46953125)(28.64140625,265.46953125)
\curveto(27.59453125,265.46953125)(26.78203125,265.7390625)(26.20390625,266.278125)
\curveto(25.62578125,266.8171875)(25.33671875,267.51640625)(25.33671875,268.37578125)
\curveto(25.33671875,269.2078125)(25.61992187,269.8875)(26.18632812,270.41484375)
\curveto(26.75273437,270.9421875)(27.54960937,271.20585937)(28.57695312,271.20585937)
\curveto(28.63945312,271.20585937)(28.73320312,271.20390625)(28.85820312,271.2)
\lineto(28.85820312,266.559375)
\curveto(29.54179687,266.5984375)(30.06523437,266.79179687)(30.42851562,267.13945312)
\curveto(30.79179687,267.48710937)(30.9734375,267.92070312)(30.9734375,268.44023437)
\curveto(30.9734375,268.82695312)(30.871875,269.15703125)(30.66875,269.43046875)
\curveto(30.465625,269.70390625)(30.14140625,269.92070312)(29.69609375,270.08085937)
\closepath
\moveto(27.99101562,266.61796875)
\lineto(27.99101562,270.09257812)
\curveto(27.46757812,270.04570312)(27.075,269.91289062)(26.81328125,269.69414062)
\curveto(26.40703125,269.35820312)(26.20390625,268.92265625)(26.20390625,268.3875)
\curveto(26.20390625,267.903125)(26.36601562,267.49492187)(26.69023437,267.16289062)
\curveto(27.01445312,266.83476562)(27.44804687,266.653125)(27.99101562,266.61796875)
\closepath
}
}
{
\newrgbcolor{curcolor}{0 0 0}
\pscustom[linestyle=none,fillstyle=solid,fillcolor=curcolor]
{
\newpath
\moveto(31.7,272.48320312)
\lineto(25.47734375,272.48320312)
\lineto(25.47734375,273.43242187)
\lineto(26.42070312,273.43242187)
\curveto(25.97929687,273.67460937)(25.68828125,273.89726562)(25.54765625,274.10039062)
\curveto(25.40703125,274.30742187)(25.33671875,274.53398437)(25.33671875,274.78007812)
\curveto(25.33671875,275.13554687)(25.45,275.496875)(25.6765625,275.8640625)
\lineto(26.65507812,275.50078125)
\curveto(26.50273437,275.24296875)(26.4265625,274.98515625)(26.4265625,274.72734375)
\curveto(26.4265625,274.496875)(26.496875,274.28984375)(26.6375,274.10625)
\curveto(26.77421875,273.92265625)(26.965625,273.79179687)(27.21171875,273.71367187)
\curveto(27.58671875,273.59648437)(27.996875,273.53789062)(28.4421875,273.53789062)
\lineto(31.7,273.53789062)
\closepath
}
}
{
\newrgbcolor{curcolor}{0 0 0}
\pscustom[linestyle=none,fillstyle=solid,fillcolor=curcolor]
{
\newpath
\moveto(29.42070312,280.5515625)
\lineto(29.55546875,281.58867187)
\curveto(30.2703125,281.47539062)(30.83085937,281.184375)(31.23710937,280.715625)
\curveto(31.63945312,280.25078125)(31.840625,279.67851562)(31.840625,278.99882812)
\curveto(31.840625,278.14726562)(31.56328125,277.46171875)(31.00859375,276.9421875)
\curveto(30.45,276.4265625)(29.65117187,276.16875)(28.61210937,276.16875)
\curveto(27.94023437,276.16875)(27.35234375,276.28007812)(26.8484375,276.50273437)
\curveto(26.34453125,276.72539062)(25.96757812,277.06328125)(25.71757812,277.51640625)
\curveto(25.46367187,277.9734375)(25.33671875,278.46953125)(25.33671875,279.0046875)
\curveto(25.33671875,279.68046875)(25.50859375,280.23320312)(25.85234375,280.66289062)
\curveto(26.1921875,281.09257812)(26.6765625,281.36796875)(27.30546875,281.4890625)
\lineto(27.46367187,280.46367187)
\curveto(27.04570312,280.36601562)(26.73125,280.1921875)(26.5203125,279.9421875)
\curveto(26.309375,279.69609375)(26.20390625,279.39726562)(26.20390625,279.04570312)
\curveto(26.20390625,278.51445312)(26.3953125,278.0828125)(26.778125,277.75078125)
\curveto(27.15703125,277.41875)(27.75859375,277.25273437)(28.5828125,277.25273437)
\curveto(29.41875,277.25273437)(30.02617187,277.41289062)(30.40507812,277.73320312)
\curveto(30.78398437,278.05351562)(30.9734375,278.47148437)(30.9734375,278.98710937)
\curveto(30.9734375,279.40117187)(30.84648437,279.746875)(30.59257812,280.02421875)
\curveto(30.33867187,280.3015625)(29.94804687,280.47734375)(29.42070312,280.5515625)
\closepath
}
}
{
\newrgbcolor{curcolor}{0 0 0}
\pscustom[linestyle=none,fillstyle=solid,fillcolor=curcolor]
{
\newpath
\moveto(29.69609375,286.75078125)
\lineto(29.83085937,287.840625)
\curveto(30.46757812,287.66875)(30.96171875,287.35039062)(31.31328125,286.88554687)
\curveto(31.66484375,286.42070312)(31.840625,285.82695312)(31.840625,285.10429687)
\curveto(31.840625,284.19414062)(31.56132812,283.47148437)(31.00273437,282.93632812)
\curveto(30.44023437,282.40507812)(29.653125,282.13945312)(28.64140625,282.13945312)
\curveto(27.59453125,282.13945312)(26.78203125,282.40898437)(26.20390625,282.94804687)
\curveto(25.62578125,283.48710937)(25.33671875,284.18632812)(25.33671875,285.04570312)
\curveto(25.33671875,285.87773437)(25.61992187,286.55742187)(26.18632812,287.08476562)
\curveto(26.75273437,287.61210937)(27.54960937,287.87578125)(28.57695312,287.87578125)
\curveto(28.63945312,287.87578125)(28.73320312,287.87382812)(28.85820312,287.86992187)
\lineto(28.85820312,283.22929687)
\curveto(29.54179687,283.26835937)(30.06523437,283.46171875)(30.42851562,283.809375)
\curveto(30.79179687,284.15703125)(30.9734375,284.590625)(30.9734375,285.11015625)
\curveto(30.9734375,285.496875)(30.871875,285.82695312)(30.66875,286.10039062)
\curveto(30.465625,286.37382812)(30.14140625,286.590625)(29.69609375,286.75078125)
\closepath
\moveto(27.99101562,283.28789062)
\lineto(27.99101562,286.7625)
\curveto(27.46757812,286.715625)(27.075,286.5828125)(26.81328125,286.3640625)
\curveto(26.40703125,286.028125)(26.20390625,285.59257812)(26.20390625,285.05742187)
\curveto(26.20390625,284.57304687)(26.36601562,284.16484375)(26.69023437,283.8328125)
\curveto(27.01445312,283.5046875)(27.44804687,283.32304687)(27.99101562,283.28789062)
\closepath
}
}
{
\newrgbcolor{curcolor}{0 0 0}
\pscustom[linestyle=none,fillstyle=solid,fillcolor=curcolor]
{
\newpath
\moveto(31.7,289.16484375)
\lineto(25.47734375,289.16484375)
\lineto(25.47734375,290.1140625)
\lineto(26.36210937,290.1140625)
\curveto(25.67851562,290.57109375)(25.33671875,291.23125)(25.33671875,292.09453125)
\curveto(25.33671875,292.46953125)(25.40507812,292.81328125)(25.54179687,293.12578125)
\curveto(25.67460937,293.4421875)(25.85039062,293.67851562)(26.06914062,293.83476562)
\curveto(26.28789062,293.99101562)(26.54765625,294.10039062)(26.8484375,294.16289062)
\curveto(27.04375,294.20195312)(27.38554687,294.22148437)(27.87382812,294.22148437)
\lineto(31.7,294.22148437)
\lineto(31.7,293.16679687)
\lineto(27.91484375,293.16679687)
\curveto(27.48515625,293.16679687)(27.16484375,293.12578125)(26.95390625,293.04375)
\curveto(26.7390625,292.96171875)(26.56914062,292.81523437)(26.44414062,292.60429687)
\curveto(26.31523437,292.39726562)(26.25078125,292.153125)(26.25078125,291.871875)
\curveto(26.25078125,291.42265625)(26.39335937,291.03398437)(26.67851562,290.70585937)
\curveto(26.96367187,290.38164062)(27.5046875,290.21953125)(28.3015625,290.21953125)
\lineto(31.7,290.21953125)
\closepath
}
}
{
\newrgbcolor{curcolor}{0 0 0}
\pscustom[linestyle=none,fillstyle=solid,fillcolor=curcolor]
{
\newpath
\moveto(30.75664062,298.14140625)
\lineto(31.68828125,298.29375)
\curveto(31.75078125,297.996875)(31.78203125,297.73125)(31.78203125,297.496875)
\curveto(31.78203125,297.1140625)(31.72148437,296.8171875)(31.60039062,296.60625)
\curveto(31.47929687,296.3953125)(31.32109375,296.246875)(31.12578125,296.1609375)
\curveto(30.9265625,296.075)(30.51054687,296.03203125)(29.87773437,296.03203125)
\lineto(26.29765625,296.03203125)
\lineto(26.29765625,295.25859375)
\lineto(25.47734375,295.25859375)
\lineto(25.47734375,296.03203125)
\lineto(23.93632812,296.03203125)
\lineto(23.30351562,297.08085937)
\lineto(25.47734375,297.08085937)
\lineto(25.47734375,298.14140625)
\lineto(26.29765625,298.14140625)
\lineto(26.29765625,297.08085937)
\lineto(29.93632812,297.08085937)
\curveto(30.23710937,297.08085937)(30.43046875,297.0984375)(30.51640625,297.13359375)
\curveto(30.60234375,297.17265625)(30.67070312,297.23320312)(30.72148437,297.31523437)
\curveto(30.77226562,297.40117187)(30.79765625,297.52226562)(30.79765625,297.67851562)
\curveto(30.79765625,297.79570312)(30.78398437,297.95)(30.75664062,298.14140625)
\closepath
}
}
{
\newrgbcolor{curcolor}{0 0 0}
\pscustom[linestyle=none,fillstyle=solid,fillcolor=curcolor]
{
\newpath
\moveto(30.93242187,303.23320312)
\curveto(31.26445312,302.84257812)(31.49882812,302.465625)(31.63554687,302.10234375)
\curveto(31.77226562,301.74296875)(31.840625,301.35625)(31.840625,300.9421875)
\curveto(31.840625,300.25859375)(31.67460937,299.73320312)(31.34257812,299.36601562)
\curveto(31.00664062,298.99882812)(30.57890625,298.81523437)(30.059375,298.81523437)
\curveto(29.7546875,298.81523437)(29.47734375,298.88359375)(29.22734375,299.0203125)
\curveto(28.9734375,299.1609375)(28.7703125,299.34257812)(28.61796875,299.56523437)
\curveto(28.465625,299.79179687)(28.35039062,300.04570312)(28.27226562,300.32695312)
\curveto(28.21757812,300.53398437)(28.16484375,300.84648437)(28.1140625,301.26445312)
\curveto(28.0125,302.11601562)(27.89140625,302.74296875)(27.75078125,303.1453125)
\curveto(27.60625,303.14921875)(27.51445312,303.15117187)(27.47539062,303.15117187)
\curveto(27.04570312,303.15117187)(26.74296875,303.0515625)(26.5671875,302.85234375)
\curveto(26.32890625,302.5828125)(26.20976562,302.18242187)(26.20976562,301.65117187)
\curveto(26.20976562,301.15507812)(26.29765625,300.78789062)(26.4734375,300.54960937)
\curveto(26.6453125,300.31523437)(26.95195312,300.14140625)(27.39335937,300.028125)
\lineto(27.25273437,298.996875)
\curveto(26.81132812,299.090625)(26.45585937,299.24492187)(26.18632812,299.45976562)
\curveto(25.91289062,299.67460937)(25.70390625,299.98515625)(25.559375,300.39140625)
\curveto(25.4109375,300.79765625)(25.33671875,301.26835937)(25.33671875,301.80351562)
\curveto(25.33671875,302.33476562)(25.39921875,302.76640625)(25.52421875,303.0984375)
\curveto(25.64921875,303.43046875)(25.80742187,303.67460937)(25.99882812,303.83085937)
\curveto(26.18632812,303.98710937)(26.42460937,304.09648437)(26.71367187,304.15898437)
\curveto(26.89335937,304.19414062)(27.21757812,304.21171875)(27.68632812,304.21171875)
\lineto(29.09257812,304.21171875)
\curveto(30.07304687,304.21171875)(30.69414062,304.23320312)(30.95585937,304.27617187)
\curveto(31.21367187,304.32304687)(31.46171875,304.41289062)(31.7,304.54570312)
\lineto(31.7,303.44414062)
\curveto(31.48125,303.33476562)(31.22539062,303.26445312)(30.93242187,303.23320312)
\closepath
\moveto(28.57695312,303.1453125)
\curveto(28.73320312,302.7625)(28.86601562,302.18828125)(28.97539062,301.42265625)
\curveto(29.03789062,300.9890625)(29.10820312,300.68242187)(29.18632812,300.50273437)
\curveto(29.26445312,300.32304687)(29.3796875,300.184375)(29.53203125,300.08671875)
\curveto(29.68046875,299.9890625)(29.84648437,299.94023437)(30.03007812,299.94023437)
\curveto(30.31132812,299.94023437)(30.54570312,300.04570312)(30.73320312,300.25664062)
\curveto(30.92070312,300.47148437)(31.01445312,300.78398437)(31.01445312,301.19414062)
\curveto(31.01445312,301.60039062)(30.9265625,301.96171875)(30.75078125,302.278125)
\curveto(30.57109375,302.59453125)(30.32695312,302.82695312)(30.01835937,302.97539062)
\curveto(29.78007812,303.08867187)(29.42851562,303.1453125)(28.96367187,303.1453125)
\closepath
}
}
{
\newrgbcolor{curcolor}{0 0 0}
\pscustom[linestyle=none,fillstyle=solid,fillcolor=curcolor]
{
\newpath
\moveto(32.215625,305.653125)
\lineto(32.36796875,306.67851562)
\curveto(32.684375,306.72148437)(32.91484375,306.840625)(33.059375,307.0359375)
\curveto(33.2546875,307.29765625)(33.35234375,307.65507812)(33.35234375,308.10820312)
\curveto(33.35234375,308.59648437)(33.2546875,308.9734375)(33.059375,309.2390625)
\curveto(32.8640625,309.5046875)(32.590625,309.684375)(32.2390625,309.778125)
\curveto(32.02421875,309.8328125)(31.57304687,309.85820312)(30.88554687,309.85429687)
\curveto(31.42851562,309.39335937)(31.7,308.81914062)(31.7,308.13164062)
\curveto(31.7,307.27617187)(31.39140625,306.6140625)(30.77421875,306.1453125)
\curveto(30.15703125,305.6765625)(29.41679687,305.4421875)(28.55351562,305.4421875)
\curveto(27.95976562,305.4421875)(27.41289062,305.54960937)(26.91289062,305.76445312)
\curveto(26.40898437,305.97929687)(26.0203125,306.28984375)(25.746875,306.69609375)
\curveto(25.4734375,307.10625)(25.33671875,307.58671875)(25.33671875,308.1375)
\curveto(25.33671875,308.871875)(25.63359375,309.47734375)(26.22734375,309.95390625)
\lineto(25.47734375,309.95390625)
\lineto(25.47734375,310.9265625)
\lineto(30.85625,310.9265625)
\curveto(31.825,310.9265625)(32.51054688,310.82695312)(32.91289063,310.62773437)
\curveto(33.31914063,310.43242187)(33.63945313,310.11992187)(33.87382813,309.69023437)
\curveto(34.10820313,309.26445312)(34.22539063,308.7390625)(34.22539063,308.1140625)
\curveto(34.22539063,307.371875)(34.05742188,306.77226562)(33.72148438,306.31523437)
\curveto(33.38945313,305.85820312)(32.8875,305.6375)(32.215625,305.653125)
\closepath
\moveto(28.47734375,306.52617187)
\curveto(29.29375,306.52617187)(29.88945312,306.68828125)(30.26445312,307.0125)
\curveto(30.63945312,307.33671875)(30.82695312,307.74296875)(30.82695312,308.23125)
\curveto(30.82695312,308.715625)(30.64140625,309.121875)(30.2703125,309.45)
\curveto(29.8953125,309.778125)(29.309375,309.9421875)(28.5125,309.9421875)
\curveto(27.75078125,309.9421875)(27.1765625,309.77226562)(26.78984375,309.43242187)
\curveto(26.403125,309.09648437)(26.20976562,308.69023437)(26.20976562,308.21367187)
\curveto(26.20976562,307.74492187)(26.40117187,307.34648437)(26.78398437,307.01835937)
\curveto(27.16289062,306.69023437)(27.72734375,306.52617187)(28.47734375,306.52617187)
\closepath
}
}
{
\newrgbcolor{curcolor}{0 0 0}
\pscustom[linestyle=none,fillstyle=solid,fillcolor=curcolor]
{
\newpath
\moveto(29.69609375,316.78007812)
\lineto(29.83085937,317.86992187)
\curveto(30.46757812,317.69804687)(30.96171875,317.3796875)(31.31328125,316.91484375)
\curveto(31.66484375,316.45)(31.840625,315.85625)(31.840625,315.13359375)
\curveto(31.840625,314.2234375)(31.56132812,313.50078125)(31.00273437,312.965625)
\curveto(30.44023437,312.434375)(29.653125,312.16875)(28.64140625,312.16875)
\curveto(27.59453125,312.16875)(26.78203125,312.43828125)(26.20390625,312.97734375)
\curveto(25.62578125,313.51640625)(25.33671875,314.215625)(25.33671875,315.075)
\curveto(25.33671875,315.90703125)(25.61992187,316.58671875)(26.18632812,317.1140625)
\curveto(26.75273437,317.64140625)(27.54960937,317.90507812)(28.57695312,317.90507812)
\curveto(28.63945312,317.90507812)(28.73320312,317.903125)(28.85820312,317.89921875)
\lineto(28.85820312,313.25859375)
\curveto(29.54179687,313.29765625)(30.06523437,313.49101562)(30.42851562,313.83867187)
\curveto(30.79179687,314.18632812)(30.9734375,314.61992187)(30.9734375,315.13945312)
\curveto(30.9734375,315.52617187)(30.871875,315.85625)(30.66875,316.1296875)
\curveto(30.465625,316.403125)(30.14140625,316.61992187)(29.69609375,316.78007812)
\closepath
\moveto(27.99101562,313.3171875)
\lineto(27.99101562,316.79179687)
\curveto(27.46757812,316.74492187)(27.075,316.61210937)(26.81328125,316.39335937)
\curveto(26.40703125,316.05742187)(26.20390625,315.621875)(26.20390625,315.08671875)
\curveto(26.20390625,314.60234375)(26.36601562,314.19414062)(26.69023437,313.86210937)
\curveto(27.01445312,313.53398437)(27.44804687,313.35234375)(27.99101562,313.3171875)
\closepath
}
}
{
\newrgbcolor{curcolor}{0 0 0}
\pscustom[linestyle=none,fillstyle=solid,fillcolor=curcolor]
{
\newpath
\moveto(34.22539063,319.88554687)
\lineto(34.22539063,319.1296875)
\curveto(32.35039062,320.29765625)(30.4734375,320.88164062)(28.59453125,320.88164062)
\curveto(27.86015625,320.88164062)(27.13164062,320.79765625)(26.40898437,320.6296875)
\curveto(25.82304687,320.496875)(25.26054687,320.31132812)(24.72148437,320.07304687)
\curveto(24.36992187,319.92070312)(23.78398437,319.60625)(22.96367187,319.1296875)
\lineto(22.96367187,319.88554687)
\curveto(23.94414062,320.61992187)(24.92851562,321.16289062)(25.91679687,321.51445312)
\curveto(26.76835937,321.81523437)(27.65898437,321.965625)(28.58867187,321.965625)
\curveto(29.64335937,321.965625)(30.66289062,321.7625)(31.64726562,321.35625)
\curveto(32.63164062,320.95390625)(33.49101563,320.46367187)(34.22539063,319.88554687)
\closepath
}
}
{
\newrgbcolor{curcolor}{0 0 0.00392157}
\pscustom[linestyle=none,fillstyle=solid,fillcolor=curcolor]
{
\newpath
\moveto(522.3,207.2)
\lineto(543.9,207.2)
\lineto(543.9,208.4)
\lineto(522.3,208.4)
\closepath
}
}
{
\newrgbcolor{curcolor}{0 0 0.03529412}
\pscustom[linestyle=none,fillstyle=solid,fillcolor=curcolor]
{
\newpath
\moveto(522.3,208.3)
\lineto(543.9,208.3)
\lineto(543.9,209.5)
\lineto(522.3,209.5)
\closepath
}
}
{
\newrgbcolor{curcolor}{0 0 0.0627451}
\pscustom[linestyle=none,fillstyle=solid,fillcolor=curcolor]
{
\newpath
\moveto(522.3,209.4)
\lineto(543.9,209.4)
\lineto(543.9,210.6)
\lineto(522.3,210.6)
\closepath
}
}
{
\newrgbcolor{curcolor}{0 0 0.09411765}
\pscustom[linestyle=none,fillstyle=solid,fillcolor=curcolor]
{
\newpath
\moveto(522.3,210.5)
\lineto(543.9,210.5)
\lineto(543.9,211.7)
\lineto(522.3,211.7)
\closepath
}
}
{
\newrgbcolor{curcolor}{0 0 0.1254902}
\pscustom[linestyle=none,fillstyle=solid,fillcolor=curcolor]
{
\newpath
\moveto(522.3,211.6)
\lineto(543.9,211.6)
\lineto(543.9,212.9)
\lineto(522.3,212.9)
\closepath
}
}
{
\newrgbcolor{curcolor}{0 0 0.16078432}
\pscustom[linestyle=none,fillstyle=solid,fillcolor=curcolor]
{
\newpath
\moveto(522.3,212.8)
\lineto(543.9,212.8)
\lineto(543.9,214)
\lineto(522.3,214)
\closepath
}
}
{
\newrgbcolor{curcolor}{0 0 0.1882353}
\pscustom[linestyle=none,fillstyle=solid,fillcolor=curcolor]
{
\newpath
\moveto(522.3,213.9)
\lineto(543.9,213.9)
\lineto(543.9,215.1)
\lineto(522.3,215.1)
\closepath
}
}
{
\newrgbcolor{curcolor}{0 0 0.21960784}
\pscustom[linestyle=none,fillstyle=solid,fillcolor=curcolor]
{
\newpath
\moveto(522.3,215)
\lineto(543.9,215)
\lineto(543.9,216.2)
\lineto(522.3,216.2)
\closepath
}
}
{
\newrgbcolor{curcolor}{0 0 0.25098041}
\pscustom[linestyle=none,fillstyle=solid,fillcolor=curcolor]
{
\newpath
\moveto(522.3,216.1)
\lineto(543.9,216.1)
\lineto(543.9,217.3)
\lineto(522.3,217.3)
\closepath
}
}
{
\newrgbcolor{curcolor}{0 0 0.28235295}
\pscustom[linestyle=none,fillstyle=solid,fillcolor=curcolor]
{
\newpath
\moveto(522.3,217.2)
\lineto(543.9,217.2)
\lineto(543.9,218.5)
\lineto(522.3,218.5)
\closepath
}
}
{
\newrgbcolor{curcolor}{0 0 0.3137255}
\pscustom[linestyle=none,fillstyle=solid,fillcolor=curcolor]
{
\newpath
\moveto(522.3,218.4)
\lineto(543.9,218.4)
\lineto(543.9,219.6)
\lineto(522.3,219.6)
\closepath
}
}
{
\newrgbcolor{curcolor}{0 0 0.34509805}
\pscustom[linestyle=none,fillstyle=solid,fillcolor=curcolor]
{
\newpath
\moveto(522.3,219.5)
\lineto(543.9,219.5)
\lineto(543.9,220.7)
\lineto(522.3,220.7)
\closepath
}
}
{
\newrgbcolor{curcolor}{0 0 0.3764706}
\pscustom[linestyle=none,fillstyle=solid,fillcolor=curcolor]
{
\newpath
\moveto(522.3,220.6)
\lineto(543.9,220.6)
\lineto(543.9,221.8)
\lineto(522.3,221.8)
\closepath
}
}
{
\newrgbcolor{curcolor}{0 0 0.40784314}
\pscustom[linestyle=none,fillstyle=solid,fillcolor=curcolor]
{
\newpath
\moveto(522.3,221.7)
\lineto(543.9,221.7)
\lineto(543.9,222.9)
\lineto(522.3,222.9)
\closepath
}
}
{
\newrgbcolor{curcolor}{0 0 0.43921569}
\pscustom[linestyle=none,fillstyle=solid,fillcolor=curcolor]
{
\newpath
\moveto(522.3,222.8)
\lineto(543.9,222.8)
\lineto(543.9,224.1)
\lineto(522.3,224.1)
\closepath
}
}
{
\newrgbcolor{curcolor}{0 0 0.47058824}
\pscustom[linestyle=none,fillstyle=solid,fillcolor=curcolor]
{
\newpath
\moveto(522.3,224)
\lineto(543.9,224)
\lineto(543.9,225.2)
\lineto(522.3,225.2)
\closepath
}
}
{
\newrgbcolor{curcolor}{0 0 0.50196081}
\pscustom[linestyle=none,fillstyle=solid,fillcolor=curcolor]
{
\newpath
\moveto(522.3,225.1)
\lineto(543.9,225.1)
\lineto(543.9,226.3)
\lineto(522.3,226.3)
\closepath
}
}
{
\newrgbcolor{curcolor}{0 0 0.53333336}
\pscustom[linestyle=none,fillstyle=solid,fillcolor=curcolor]
{
\newpath
\moveto(522.3,226.2)
\lineto(543.9,226.2)
\lineto(543.9,227.4)
\lineto(522.3,227.4)
\closepath
}
}
{
\newrgbcolor{curcolor}{0 0 0.56470591}
\pscustom[linestyle=none,fillstyle=solid,fillcolor=curcolor]
{
\newpath
\moveto(522.3,227.3)
\lineto(543.9,227.3)
\lineto(543.9,228.5)
\lineto(522.3,228.5)
\closepath
}
}
{
\newrgbcolor{curcolor}{0 0 0.59607846}
\pscustom[linestyle=none,fillstyle=solid,fillcolor=curcolor]
{
\newpath
\moveto(522.3,228.4)
\lineto(543.9,228.4)
\lineto(543.9,229.7)
\lineto(522.3,229.7)
\closepath
}
}
{
\newrgbcolor{curcolor}{0 0 0.627451}
\pscustom[linestyle=none,fillstyle=solid,fillcolor=curcolor]
{
\newpath
\moveto(522.3,229.6)
\lineto(543.9,229.6)
\lineto(543.9,230.8)
\lineto(522.3,230.8)
\closepath
}
}
{
\newrgbcolor{curcolor}{0 0 0.65882355}
\pscustom[linestyle=none,fillstyle=solid,fillcolor=curcolor]
{
\newpath
\moveto(522.3,230.7)
\lineto(543.9,230.7)
\lineto(543.9,231.9)
\lineto(522.3,231.9)
\closepath
}
}
{
\newrgbcolor{curcolor}{0 0 0.6901961}
\pscustom[linestyle=none,fillstyle=solid,fillcolor=curcolor]
{
\newpath
\moveto(522.3,231.8)
\lineto(543.9,231.8)
\lineto(543.9,233)
\lineto(522.3,233)
\closepath
}
}
{
\newrgbcolor{curcolor}{0 0 0.72156864}
\pscustom[linestyle=none,fillstyle=solid,fillcolor=curcolor]
{
\newpath
\moveto(522.3,232.9)
\lineto(543.9,232.9)
\lineto(543.9,234.1)
\lineto(522.3,234.1)
\closepath
}
}
{
\newrgbcolor{curcolor}{0 0 0.74901962}
\pscustom[linestyle=none,fillstyle=solid,fillcolor=curcolor]
{
\newpath
\moveto(522.3,234)
\lineto(543.9,234)
\lineto(543.9,235.3)
\lineto(522.3,235.3)
\closepath
}
}
{
\newrgbcolor{curcolor}{0 0 0.78431374}
\pscustom[linestyle=none,fillstyle=solid,fillcolor=curcolor]
{
\newpath
\moveto(522.3,235.2)
\lineto(543.9,235.2)
\lineto(543.9,236.4)
\lineto(522.3,236.4)
\closepath
}
}
{
\newrgbcolor{curcolor}{0 0 0.81568629}
\pscustom[linestyle=none,fillstyle=solid,fillcolor=curcolor]
{
\newpath
\moveto(522.3,236.3)
\lineto(543.9,236.3)
\lineto(543.9,237.5)
\lineto(522.3,237.5)
\closepath
}
}
{
\newrgbcolor{curcolor}{0 0 0.84705883}
\pscustom[linestyle=none,fillstyle=solid,fillcolor=curcolor]
{
\newpath
\moveto(522.3,237.4)
\lineto(543.9,237.4)
\lineto(543.9,238.6)
\lineto(522.3,238.6)
\closepath
}
}
{
\newrgbcolor{curcolor}{0 0 0.87450981}
\pscustom[linestyle=none,fillstyle=solid,fillcolor=curcolor]
{
\newpath
\moveto(522.3,238.5)
\lineto(543.9,238.5)
\lineto(543.9,239.7)
\lineto(522.3,239.7)
\closepath
}
}
{
\newrgbcolor{curcolor}{0 0 0.90588236}
\pscustom[linestyle=none,fillstyle=solid,fillcolor=curcolor]
{
\newpath
\moveto(522.3,239.6)
\lineto(543.9,239.6)
\lineto(543.9,240.9)
\lineto(522.3,240.9)
\closepath
}
}
{
\newrgbcolor{curcolor}{0 0 0.94117647}
\pscustom[linestyle=none,fillstyle=solid,fillcolor=curcolor]
{
\newpath
\moveto(522.3,240.8)
\lineto(543.9,240.8)
\lineto(543.9,242)
\lineto(522.3,242)
\closepath
}
}
{
\newrgbcolor{curcolor}{0 0 0.97254902}
\pscustom[linestyle=none,fillstyle=solid,fillcolor=curcolor]
{
\newpath
\moveto(522.3,241.9)
\lineto(543.9,241.9)
\lineto(543.9,243.1)
\lineto(522.3,243.1)
\closepath
}
}
{
\newrgbcolor{curcolor}{0 0 1}
\pscustom[linestyle=none,fillstyle=solid,fillcolor=curcolor]
{
\newpath
\moveto(522.3,243)
\lineto(543.9,243)
\lineto(543.9,244.2)
\lineto(522.3,244.2)
\closepath
}
}
{
\newrgbcolor{curcolor}{0.02352941 0 1}
\pscustom[linestyle=none,fillstyle=solid,fillcolor=curcolor]
{
\newpath
\moveto(522.3,244.1)
\lineto(543.9,244.1)
\lineto(543.9,245.3)
\lineto(522.3,245.3)
\closepath
}
}
{
\newrgbcolor{curcolor}{0.05098039 0 1}
\pscustom[linestyle=none,fillstyle=solid,fillcolor=curcolor]
{
\newpath
\moveto(522.3,245.2)
\lineto(543.9,245.2)
\lineto(543.9,246.5)
\lineto(522.3,246.5)
\closepath
}
}
{
\newrgbcolor{curcolor}{0.07450981 0 1}
\pscustom[linestyle=none,fillstyle=solid,fillcolor=curcolor]
{
\newpath
\moveto(522.3,246.4)
\lineto(543.9,246.4)
\lineto(543.9,247.6)
\lineto(522.3,247.6)
\closepath
}
}
{
\newrgbcolor{curcolor}{0.09803922 0 1}
\pscustom[linestyle=none,fillstyle=solid,fillcolor=curcolor]
{
\newpath
\moveto(522.3,247.5)
\lineto(543.9,247.5)
\lineto(543.9,248.7)
\lineto(522.3,248.7)
\closepath
}
}
{
\newrgbcolor{curcolor}{0.12156863 0 1}
\pscustom[linestyle=none,fillstyle=solid,fillcolor=curcolor]
{
\newpath
\moveto(522.3,248.6)
\lineto(543.9,248.6)
\lineto(543.9,249.8)
\lineto(522.3,249.8)
\closepath
}
}
{
\newrgbcolor{curcolor}{0.14901961 0 1}
\pscustom[linestyle=none,fillstyle=solid,fillcolor=curcolor]
{
\newpath
\moveto(522.3,249.7)
\lineto(543.9,249.7)
\lineto(543.9,250.9)
\lineto(522.3,250.9)
\closepath
}
}
{
\newrgbcolor{curcolor}{0.17254902 0 1}
\pscustom[linestyle=none,fillstyle=solid,fillcolor=curcolor]
{
\newpath
\moveto(522.3,250.8)
\lineto(543.9,250.8)
\lineto(543.9,252.1)
\lineto(522.3,252.1)
\closepath
}
}
{
\newrgbcolor{curcolor}{0.19607843 0 1}
\pscustom[linestyle=none,fillstyle=solid,fillcolor=curcolor]
{
\newpath
\moveto(522.3,252)
\lineto(543.9,252)
\lineto(543.9,253.2)
\lineto(522.3,253.2)
\closepath
}
}
{
\newrgbcolor{curcolor}{0.21960784 0 1}
\pscustom[linestyle=none,fillstyle=solid,fillcolor=curcolor]
{
\newpath
\moveto(522.3,253.1)
\lineto(543.9,253.1)
\lineto(543.9,254.3)
\lineto(522.3,254.3)
\closepath
}
}
{
\newrgbcolor{curcolor}{0.24705882 0 1}
\pscustom[linestyle=none,fillstyle=solid,fillcolor=curcolor]
{
\newpath
\moveto(522.3,254.2)
\lineto(543.9,254.2)
\lineto(543.9,255.4)
\lineto(522.3,255.4)
\closepath
}
}
{
\newrgbcolor{curcolor}{0.27058825 0 1}
\pscustom[linestyle=none,fillstyle=solid,fillcolor=curcolor]
{
\newpath
\moveto(522.3,255.3)
\lineto(543.9,255.3)
\lineto(543.9,256.5)
\lineto(522.3,256.5)
\closepath
}
}
{
\newrgbcolor{curcolor}{0.29411766 0 1}
\pscustom[linestyle=none,fillstyle=solid,fillcolor=curcolor]
{
\newpath
\moveto(522.3,256.4)
\lineto(543.9,256.4)
\lineto(543.9,257.7)
\lineto(522.3,257.7)
\closepath
}
}
{
\newrgbcolor{curcolor}{0.31764707 0 1}
\pscustom[linestyle=none,fillstyle=solid,fillcolor=curcolor]
{
\newpath
\moveto(522.3,257.6)
\lineto(543.9,257.6)
\lineto(543.9,258.8)
\lineto(522.3,258.8)
\closepath
}
}
{
\newrgbcolor{curcolor}{0.34509805 0 1}
\pscustom[linestyle=none,fillstyle=solid,fillcolor=curcolor]
{
\newpath
\moveto(522.3,258.7)
\lineto(543.9,258.7)
\lineto(543.9,259.9)
\lineto(522.3,259.9)
\closepath
}
}
{
\newrgbcolor{curcolor}{0.36862746 0 1}
\pscustom[linestyle=none,fillstyle=solid,fillcolor=curcolor]
{
\newpath
\moveto(522.3,259.8)
\lineto(543.9,259.8)
\lineto(543.9,261)
\lineto(522.3,261)
\closepath
}
}
{
\newrgbcolor{curcolor}{0.39215687 0 1}
\pscustom[linestyle=none,fillstyle=solid,fillcolor=curcolor]
{
\newpath
\moveto(522.3,260.9)
\lineto(543.9,260.9)
\lineto(543.9,262.1)
\lineto(522.3,262.1)
\closepath
}
}
{
\newrgbcolor{curcolor}{0.41568628 0 1}
\pscustom[linestyle=none,fillstyle=solid,fillcolor=curcolor]
{
\newpath
\moveto(522.3,262)
\lineto(543.9,262)
\lineto(543.9,263.3)
\lineto(522.3,263.3)
\closepath
}
}
{
\newrgbcolor{curcolor}{0.44313726 0 1}
\pscustom[linestyle=none,fillstyle=solid,fillcolor=curcolor]
{
\newpath
\moveto(522.3,263.2)
\lineto(543.9,263.2)
\lineto(543.9,264.4)
\lineto(522.3,264.4)
\closepath
}
}
{
\newrgbcolor{curcolor}{0.46666667 0 1}
\pscustom[linestyle=none,fillstyle=solid,fillcolor=curcolor]
{
\newpath
\moveto(522.3,264.3)
\lineto(543.9,264.3)
\lineto(543.9,265.5)
\lineto(522.3,265.5)
\closepath
}
}
{
\newrgbcolor{curcolor}{0.49019608 0 1}
\pscustom[linestyle=none,fillstyle=solid,fillcolor=curcolor]
{
\newpath
\moveto(522.3,265.4)
\lineto(543.9,265.4)
\lineto(543.9,266.6)
\lineto(522.3,266.6)
\closepath
}
}
{
\newrgbcolor{curcolor}{0.51372552 0 1}
\pscustom[linestyle=none,fillstyle=solid,fillcolor=curcolor]
{
\newpath
\moveto(522.3,266.5)
\lineto(543.9,266.5)
\lineto(543.9,267.7)
\lineto(522.3,267.7)
\closepath
}
}
{
\newrgbcolor{curcolor}{0.53725493 0.00392157 0.99607843}
\pscustom[linestyle=none,fillstyle=solid,fillcolor=curcolor]
{
\newpath
\moveto(522.3,267.6)
\lineto(543.9,267.6)
\lineto(543.9,268.9)
\lineto(522.3,268.9)
\closepath
}
}
{
\newrgbcolor{curcolor}{0.56470591 0.01960784 0.98039216}
\pscustom[linestyle=none,fillstyle=solid,fillcolor=curcolor]
{
\newpath
\moveto(522.3,268.8)
\lineto(543.9,268.8)
\lineto(543.9,270)
\lineto(522.3,270)
\closepath
}
}
{
\newrgbcolor{curcolor}{0.58823532 0.03529412 0.96470588}
\pscustom[linestyle=none,fillstyle=solid,fillcolor=curcolor]
{
\newpath
\moveto(522.3,269.9)
\lineto(543.9,269.9)
\lineto(543.9,271.1)
\lineto(522.3,271.1)
\closepath
}
}
{
\newrgbcolor{curcolor}{0.61176473 0.05098039 0.94901961}
\pscustom[linestyle=none,fillstyle=solid,fillcolor=curcolor]
{
\newpath
\moveto(522.3,271)
\lineto(543.9,271)
\lineto(543.9,272.2)
\lineto(522.3,272.2)
\closepath
}
}
{
\newrgbcolor{curcolor}{0.63529414 0.06666667 0.93333334}
\pscustom[linestyle=none,fillstyle=solid,fillcolor=curcolor]
{
\newpath
\moveto(522.3,272.1)
\lineto(543.9,272.1)
\lineto(543.9,273.3)
\lineto(522.3,273.3)
\closepath
}
}
{
\newrgbcolor{curcolor}{0.65882355 0.08235294 0.91764706}
\pscustom[linestyle=none,fillstyle=solid,fillcolor=curcolor]
{
\newpath
\moveto(522.3,273.2)
\lineto(543.9,273.2)
\lineto(543.9,274.5)
\lineto(522.3,274.5)
\closepath
}
}
{
\newrgbcolor{curcolor}{0.68627453 0.09803922 0.90196079}
\pscustom[linestyle=none,fillstyle=solid,fillcolor=curcolor]
{
\newpath
\moveto(522.3,274.4)
\lineto(543.9,274.4)
\lineto(543.9,275.6)
\lineto(522.3,275.6)
\closepath
}
}
{
\newrgbcolor{curcolor}{0.70980394 0.11372549 0.88627452}
\pscustom[linestyle=none,fillstyle=solid,fillcolor=curcolor]
{
\newpath
\moveto(522.3,275.5)
\lineto(543.9,275.5)
\lineto(543.9,276.7)
\lineto(522.3,276.7)
\closepath
}
}
{
\newrgbcolor{curcolor}{0.73333335 0.12941177 0.87058824}
\pscustom[linestyle=none,fillstyle=solid,fillcolor=curcolor]
{
\newpath
\moveto(522.3,276.6)
\lineto(543.9,276.6)
\lineto(543.9,277.8)
\lineto(522.3,277.8)
\closepath
}
}
{
\newrgbcolor{curcolor}{0.75686276 0.14509805 0.85490197}
\pscustom[linestyle=none,fillstyle=solid,fillcolor=curcolor]
{
\newpath
\moveto(522.3,277.7)
\lineto(543.9,277.7)
\lineto(543.9,279)
\lineto(522.3,279)
\closepath
}
}
{
\newrgbcolor{curcolor}{0.78431374 0.16078432 0.8392157}
\pscustom[linestyle=none,fillstyle=solid,fillcolor=curcolor]
{
\newpath
\moveto(522.3,278.9)
\lineto(543.9,278.9)
\lineto(543.9,280.1)
\lineto(522.3,280.1)
\closepath
}
}
{
\newrgbcolor{curcolor}{0.80784315 0.17647059 0.82352942}
\pscustom[linestyle=none,fillstyle=solid,fillcolor=curcolor]
{
\newpath
\moveto(522.3,280)
\lineto(543.9,280)
\lineto(543.9,281.2)
\lineto(522.3,281.2)
\closepath
}
}
{
\newrgbcolor{curcolor}{0.83137256 0.19215687 0.80784315}
\pscustom[linestyle=none,fillstyle=solid,fillcolor=curcolor]
{
\newpath
\moveto(522.3,281.1)
\lineto(543.9,281.1)
\lineto(543.9,282.3)
\lineto(522.3,282.3)
\closepath
}
}
{
\newrgbcolor{curcolor}{0.85490197 0.20784314 0.79215688}
\pscustom[linestyle=none,fillstyle=solid,fillcolor=curcolor]
{
\newpath
\moveto(522.3,282.2)
\lineto(543.9,282.2)
\lineto(543.9,283.4)
\lineto(522.3,283.4)
\closepath
}
}
{
\newrgbcolor{curcolor}{0.87843138 0.22352941 0.7764706}
\pscustom[linestyle=none,fillstyle=solid,fillcolor=curcolor]
{
\newpath
\moveto(522.3,283.3)
\lineto(543.9,283.3)
\lineto(543.9,284.6)
\lineto(522.3,284.6)
\closepath
}
}
{
\newrgbcolor{curcolor}{0.90588236 0.23921569 0.76078433}
\pscustom[linestyle=none,fillstyle=solid,fillcolor=curcolor]
{
\newpath
\moveto(522.3,284.5)
\lineto(543.9,284.5)
\lineto(543.9,285.7)
\lineto(522.3,285.7)
\closepath
}
}
{
\newrgbcolor{curcolor}{0.92941177 0.25490198 0.74509805}
\pscustom[linestyle=none,fillstyle=solid,fillcolor=curcolor]
{
\newpath
\moveto(522.3,285.6)
\lineto(543.9,285.6)
\lineto(543.9,286.8)
\lineto(522.3,286.8)
\closepath
}
}
{
\newrgbcolor{curcolor}{0.95294118 0.27058825 0.72941178}
\pscustom[linestyle=none,fillstyle=solid,fillcolor=curcolor]
{
\newpath
\moveto(522.3,286.7)
\lineto(543.9,286.7)
\lineto(543.9,287.9)
\lineto(522.3,287.9)
\closepath
}
}
{
\newrgbcolor{curcolor}{0.97647059 0.28627452 0.71372551}
\pscustom[linestyle=none,fillstyle=solid,fillcolor=curcolor]
{
\newpath
\moveto(522.3,287.8)
\lineto(543.9,287.8)
\lineto(543.9,289)
\lineto(522.3,289)
\closepath
}
}
{
\newrgbcolor{curcolor}{1 0.3019608 0.69803923}
\pscustom[linestyle=none,fillstyle=solid,fillcolor=curcolor]
{
\newpath
\moveto(522.3,288.9)
\lineto(543.9,288.9)
\lineto(543.9,290.2)
\lineto(522.3,290.2)
\closepath
}
}
{
\newrgbcolor{curcolor}{1 0.31764707 0.68235296}
\pscustom[linestyle=none,fillstyle=solid,fillcolor=curcolor]
{
\newpath
\moveto(522.3,290.1)
\lineto(543.9,290.1)
\lineto(543.9,291.3)
\lineto(522.3,291.3)
\closepath
}
}
{
\newrgbcolor{curcolor}{1 0.33333334 0.66666669}
\pscustom[linestyle=none,fillstyle=solid,fillcolor=curcolor]
{
\newpath
\moveto(522.3,291.2)
\lineto(543.9,291.2)
\lineto(543.9,292.4)
\lineto(522.3,292.4)
\closepath
}
}
{
\newrgbcolor{curcolor}{1 0.34901962 0.65098041}
\pscustom[linestyle=none,fillstyle=solid,fillcolor=curcolor]
{
\newpath
\moveto(522.3,292.3)
\lineto(543.9,292.3)
\lineto(543.9,293.5)
\lineto(522.3,293.5)
\closepath
}
}
{
\newrgbcolor{curcolor}{1 0.36470589 0.63529414}
\pscustom[linestyle=none,fillstyle=solid,fillcolor=curcolor]
{
\newpath
\moveto(522.3,293.4)
\lineto(543.9,293.4)
\lineto(543.9,294.6)
\lineto(522.3,294.6)
\closepath
}
}
{
\newrgbcolor{curcolor}{1 0.38039216 0.61960787}
\pscustom[linestyle=none,fillstyle=solid,fillcolor=curcolor]
{
\newpath
\moveto(522.3,294.5)
\lineto(543.9,294.5)
\lineto(543.9,295.8)
\lineto(522.3,295.8)
\closepath
}
}
{
\newrgbcolor{curcolor}{1 0.39607844 0.60392159}
\pscustom[linestyle=none,fillstyle=solid,fillcolor=curcolor]
{
\newpath
\moveto(522.3,295.7)
\lineto(543.9,295.7)
\lineto(543.9,296.9)
\lineto(522.3,296.9)
\closepath
}
}
{
\newrgbcolor{curcolor}{1 0.41176471 0.58823532}
\pscustom[linestyle=none,fillstyle=solid,fillcolor=curcolor]
{
\newpath
\moveto(522.3,296.8)
\lineto(543.9,296.8)
\lineto(543.9,298)
\lineto(522.3,298)
\closepath
}
}
{
\newrgbcolor{curcolor}{1 0.42745098 0.57254905}
\pscustom[linestyle=none,fillstyle=solid,fillcolor=curcolor]
{
\newpath
\moveto(522.3,297.9)
\lineto(543.9,297.9)
\lineto(543.9,299.1)
\lineto(522.3,299.1)
\closepath
}
}
{
\newrgbcolor{curcolor}{1 0.44313726 0.55686277}
\pscustom[linestyle=none,fillstyle=solid,fillcolor=curcolor]
{
\newpath
\moveto(522.3,299)
\lineto(543.9,299)
\lineto(543.9,300.2)
\lineto(522.3,300.2)
\closepath
}
}
{
\newrgbcolor{curcolor}{1 0.45882353 0.5411765}
\pscustom[linestyle=none,fillstyle=solid,fillcolor=curcolor]
{
\newpath
\moveto(522.3,300.1)
\lineto(543.9,300.1)
\lineto(543.9,301.4)
\lineto(522.3,301.4)
\closepath
}
}
{
\newrgbcolor{curcolor}{1 0.47450981 0.52549022}
\pscustom[linestyle=none,fillstyle=solid,fillcolor=curcolor]
{
\newpath
\moveto(522.3,301.3)
\lineto(543.9,301.3)
\lineto(543.9,302.5)
\lineto(522.3,302.5)
\closepath
}
}
{
\newrgbcolor{curcolor}{1 0.49019608 0.50980395}
\pscustom[linestyle=none,fillstyle=solid,fillcolor=curcolor]
{
\newpath
\moveto(522.3,302.4)
\lineto(543.9,302.4)
\lineto(543.9,303.6)
\lineto(522.3,303.6)
\closepath
}
}
{
\newrgbcolor{curcolor}{1 0.50588238 0.49411765}
\pscustom[linestyle=none,fillstyle=solid,fillcolor=curcolor]
{
\newpath
\moveto(522.3,303.5)
\lineto(543.9,303.5)
\lineto(543.9,304.7)
\lineto(522.3,304.7)
\closepath
}
}
{
\newrgbcolor{curcolor}{1 0.52156866 0.47843137}
\pscustom[linestyle=none,fillstyle=solid,fillcolor=curcolor]
{
\newpath
\moveto(522.3,304.6)
\lineto(543.9,304.6)
\lineto(543.9,305.8)
\lineto(522.3,305.8)
\closepath
}
}
{
\newrgbcolor{curcolor}{1 0.53333336 0.46666667}
\pscustom[linestyle=none,fillstyle=solid,fillcolor=curcolor]
{
\newpath
\moveto(522.3,305.7)
\lineto(543.9,305.7)
\lineto(543.9,307)
\lineto(522.3,307)
\closepath
}
}
{
\newrgbcolor{curcolor}{1 0.5529412 0.44705883}
\pscustom[linestyle=none,fillstyle=solid,fillcolor=curcolor]
{
\newpath
\moveto(522.3,306.9)
\lineto(543.9,306.9)
\lineto(543.9,308.1)
\lineto(522.3,308.1)
\closepath
}
}
{
\newrgbcolor{curcolor}{1 0.56862748 0.43137255}
\pscustom[linestyle=none,fillstyle=solid,fillcolor=curcolor]
{
\newpath
\moveto(522.3,308)
\lineto(543.9,308)
\lineto(543.9,309.2)
\lineto(522.3,309.2)
\closepath
}
}
{
\newrgbcolor{curcolor}{1 0.58431375 0.41568628}
\pscustom[linestyle=none,fillstyle=solid,fillcolor=curcolor]
{
\newpath
\moveto(522.3,309.1)
\lineto(543.9,309.1)
\lineto(543.9,310.3)
\lineto(522.3,310.3)
\closepath
}
}
{
\newrgbcolor{curcolor}{1 0.59607846 0.40392157}
\pscustom[linestyle=none,fillstyle=solid,fillcolor=curcolor]
{
\newpath
\moveto(522.3,310.2)
\lineto(543.9,310.2)
\lineto(543.9,311.4)
\lineto(522.3,311.4)
\closepath
}
}
{
\newrgbcolor{curcolor}{1 0.61176473 0.3882353}
\pscustom[linestyle=none,fillstyle=solid,fillcolor=curcolor]
{
\newpath
\moveto(522.3,311.3)
\lineto(543.9,311.3)
\lineto(543.9,312.6)
\lineto(522.3,312.6)
\closepath
}
}
{
\newrgbcolor{curcolor}{1 0.63137257 0.36862746}
\pscustom[linestyle=none,fillstyle=solid,fillcolor=curcolor]
{
\newpath
\moveto(522.3,312.5)
\lineto(543.9,312.5)
\lineto(543.9,313.7)
\lineto(522.3,313.7)
\closepath
}
}
{
\newrgbcolor{curcolor}{1 0.64705884 0.35294119}
\pscustom[linestyle=none,fillstyle=solid,fillcolor=curcolor]
{
\newpath
\moveto(522.3,313.6)
\lineto(543.9,313.6)
\lineto(543.9,314.8)
\lineto(522.3,314.8)
\closepath
}
}
{
\newrgbcolor{curcolor}{1 0.65882355 0.34117648}
\pscustom[linestyle=none,fillstyle=solid,fillcolor=curcolor]
{
\newpath
\moveto(522.3,314.7)
\lineto(543.9,314.7)
\lineto(543.9,315.9)
\lineto(522.3,315.9)
\closepath
}
}
{
\newrgbcolor{curcolor}{1 0.67450982 0.32549021}
\pscustom[linestyle=none,fillstyle=solid,fillcolor=curcolor]
{
\newpath
\moveto(522.3,315.8)
\lineto(543.9,315.8)
\lineto(543.9,317)
\lineto(522.3,317)
\closepath
}
}
{
\newrgbcolor{curcolor}{1 0.6901961 0.30980393}
\pscustom[linestyle=none,fillstyle=solid,fillcolor=curcolor]
{
\newpath
\moveto(522.3,316.9)
\lineto(543.9,316.9)
\lineto(543.9,318.2)
\lineto(522.3,318.2)
\closepath
}
}
{
\newrgbcolor{curcolor}{1 0.70980394 0.29019609}
\pscustom[linestyle=none,fillstyle=solid,fillcolor=curcolor]
{
\newpath
\moveto(522.3,318.1)
\lineto(543.9,318.1)
\lineto(543.9,319.3)
\lineto(522.3,319.3)
\closepath
}
}
{
\newrgbcolor{curcolor}{1 0.72156864 0.27843139}
\pscustom[linestyle=none,fillstyle=solid,fillcolor=curcolor]
{
\newpath
\moveto(522.3,319.2)
\lineto(543.9,319.2)
\lineto(543.9,320.4)
\lineto(522.3,320.4)
\closepath
}
}
{
\newrgbcolor{curcolor}{1 0.73725492 0.26274511}
\pscustom[linestyle=none,fillstyle=solid,fillcolor=curcolor]
{
\newpath
\moveto(522.3,320.3)
\lineto(543.9,320.3)
\lineto(543.9,321.5)
\lineto(522.3,321.5)
\closepath
}
}
{
\newrgbcolor{curcolor}{1 0.75294119 0.24705882}
\pscustom[linestyle=none,fillstyle=solid,fillcolor=curcolor]
{
\newpath
\moveto(522.3,321.4)
\lineto(543.9,321.4)
\lineto(543.9,322.6)
\lineto(522.3,322.6)
\closepath
}
}
{
\newrgbcolor{curcolor}{1 0.76862746 0.23137255}
\pscustom[linestyle=none,fillstyle=solid,fillcolor=curcolor]
{
\newpath
\moveto(522.3,322.5)
\lineto(543.9,322.5)
\lineto(543.9,323.8)
\lineto(522.3,323.8)
\closepath
}
}
{
\newrgbcolor{curcolor}{1 0.78431374 0.21568628}
\pscustom[linestyle=none,fillstyle=solid,fillcolor=curcolor]
{
\newpath
\moveto(522.3,323.7)
\lineto(543.9,323.7)
\lineto(543.9,324.9)
\lineto(522.3,324.9)
\closepath
}
}
{
\newrgbcolor{curcolor}{1 0.80000001 0.2}
\pscustom[linestyle=none,fillstyle=solid,fillcolor=curcolor]
{
\newpath
\moveto(522.3,324.8)
\lineto(543.9,324.8)
\lineto(543.9,326)
\lineto(522.3,326)
\closepath
}
}
{
\newrgbcolor{curcolor}{1 0.81568629 0.18431373}
\pscustom[linestyle=none,fillstyle=solid,fillcolor=curcolor]
{
\newpath
\moveto(522.3,325.9)
\lineto(543.9,325.9)
\lineto(543.9,327.1)
\lineto(522.3,327.1)
\closepath
}
}
{
\newrgbcolor{curcolor}{1 0.83137256 0.16862746}
\pscustom[linestyle=none,fillstyle=solid,fillcolor=curcolor]
{
\newpath
\moveto(522.3,327)
\lineto(543.9,327)
\lineto(543.9,328.2)
\lineto(522.3,328.2)
\closepath
}
}
{
\newrgbcolor{curcolor}{1 0.84705883 0.15294118}
\pscustom[linestyle=none,fillstyle=solid,fillcolor=curcolor]
{
\newpath
\moveto(522.3,328.1)
\lineto(543.9,328.1)
\lineto(543.9,329.4)
\lineto(522.3,329.4)
\closepath
}
}
{
\newrgbcolor{curcolor}{1 0.86274511 0.13725491}
\pscustom[linestyle=none,fillstyle=solid,fillcolor=curcolor]
{
\newpath
\moveto(522.3,329.3)
\lineto(543.9,329.3)
\lineto(543.9,330.5)
\lineto(522.3,330.5)
\closepath
}
}
{
\newrgbcolor{curcolor}{1 0.87843138 0.12156863}
\pscustom[linestyle=none,fillstyle=solid,fillcolor=curcolor]
{
\newpath
\moveto(522.3,330.4)
\lineto(543.9,330.4)
\lineto(543.9,331.6)
\lineto(522.3,331.6)
\closepath
}
}
{
\newrgbcolor{curcolor}{1 0.89411765 0.10588235}
\pscustom[linestyle=none,fillstyle=solid,fillcolor=curcolor]
{
\newpath
\moveto(522.3,331.5)
\lineto(543.9,331.5)
\lineto(543.9,332.7)
\lineto(522.3,332.7)
\closepath
}
}
{
\newrgbcolor{curcolor}{1 0.90980393 0.09019608}
\pscustom[linestyle=none,fillstyle=solid,fillcolor=curcolor]
{
\newpath
\moveto(522.3,332.6)
\lineto(543.9,332.6)
\lineto(543.9,333.8)
\lineto(522.3,333.8)
\closepath
}
}
{
\newrgbcolor{curcolor}{1 0.9254902 0.07450981}
\pscustom[linestyle=none,fillstyle=solid,fillcolor=curcolor]
{
\newpath
\moveto(522.3,333.7)
\lineto(543.9,333.7)
\lineto(543.9,335)
\lineto(522.3,335)
\closepath
}
}
{
\newrgbcolor{curcolor}{1 0.94117647 0.05882353}
\pscustom[linestyle=none,fillstyle=solid,fillcolor=curcolor]
{
\newpath
\moveto(522.3,334.9)
\lineto(543.9,334.9)
\lineto(543.9,336.1)
\lineto(522.3,336.1)
\closepath
}
}
{
\newrgbcolor{curcolor}{1 0.95686275 0.04313726}
\pscustom[linestyle=none,fillstyle=solid,fillcolor=curcolor]
{
\newpath
\moveto(522.3,336)
\lineto(543.9,336)
\lineto(543.9,337.2)
\lineto(522.3,337.2)
\closepath
}
}
{
\newrgbcolor{curcolor}{1 0.97254902 0.02745098}
\pscustom[linestyle=none,fillstyle=solid,fillcolor=curcolor]
{
\newpath
\moveto(522.3,337.1)
\lineto(543.9,337.1)
\lineto(543.9,338.3)
\lineto(522.3,338.3)
\closepath
}
}
{
\newrgbcolor{curcolor}{1 0.98823529 0.01176471}
\pscustom[linestyle=none,fillstyle=solid,fillcolor=curcolor]
{
\newpath
\moveto(522.3,338.2)
\lineto(543.9,338.2)
\lineto(543.9,339.4)
\lineto(522.3,339.4)
\closepath
}
}
{
\newrgbcolor{curcolor}{1 1 0.02352941}
\pscustom[linestyle=none,fillstyle=solid,fillcolor=curcolor]
{
\newpath
\moveto(522.3,339.3)
\lineto(543.9,339.3)
\lineto(543.9,340.6)
\lineto(522.3,340.6)
\closepath
}
}
{
\newrgbcolor{curcolor}{1 1 0.12941177}
\pscustom[linestyle=none,fillstyle=solid,fillcolor=curcolor]
{
\newpath
\moveto(522.3,340.5)
\lineto(543.9,340.5)
\lineto(543.9,341.7)
\lineto(522.3,341.7)
\closepath
}
}
{
\newrgbcolor{curcolor}{1 1 0.22352941}
\pscustom[linestyle=none,fillstyle=solid,fillcolor=curcolor]
{
\newpath
\moveto(522.3,341.6)
\lineto(543.9,341.6)
\lineto(543.9,342.8)
\lineto(522.3,342.8)
\closepath
}
}
{
\newrgbcolor{curcolor}{1 1 0.32156864}
\pscustom[linestyle=none,fillstyle=solid,fillcolor=curcolor]
{
\newpath
\moveto(522.3,342.7)
\lineto(543.9,342.7)
\lineto(543.9,343.9)
\lineto(522.3,343.9)
\closepath
}
}
{
\newrgbcolor{curcolor}{1 1 0.41568628}
\pscustom[linestyle=none,fillstyle=solid,fillcolor=curcolor]
{
\newpath
\moveto(522.3,343.8)
\lineto(543.9,343.8)
\lineto(543.9,345)
\lineto(522.3,345)
\closepath
}
}
{
\newrgbcolor{curcolor}{1 1 0.51372552}
\pscustom[linestyle=none,fillstyle=solid,fillcolor=curcolor]
{
\newpath
\moveto(522.3,344.9)
\lineto(543.9,344.9)
\lineto(543.9,346.2)
\lineto(522.3,346.2)
\closepath
}
}
{
\newrgbcolor{curcolor}{1 1 0.6156863}
\pscustom[linestyle=none,fillstyle=solid,fillcolor=curcolor]
{
\newpath
\moveto(522.3,346.1)
\lineto(543.9,346.1)
\lineto(543.9,347.3)
\lineto(522.3,347.3)
\closepath
}
}
{
\newrgbcolor{curcolor}{1 1 0.71372551}
\pscustom[linestyle=none,fillstyle=solid,fillcolor=curcolor]
{
\newpath
\moveto(522.3,347.2)
\lineto(543.9,347.2)
\lineto(543.9,348.4)
\lineto(522.3,348.4)
\closepath
}
}
{
\newrgbcolor{curcolor}{1 1 0.80784315}
\pscustom[linestyle=none,fillstyle=solid,fillcolor=curcolor]
{
\newpath
\moveto(522.3,348.3)
\lineto(543.9,348.3)
\lineto(543.9,349.5)
\lineto(522.3,349.5)
\closepath
}
}
{
\newrgbcolor{curcolor}{1 1 0.90588236}
\pscustom[linestyle=none,fillstyle=solid,fillcolor=curcolor]
{
\newpath
\moveto(522.3,349.4)
\lineto(543.9,349.4)
\lineto(543.9,350.6)
\lineto(522.3,350.6)
\closepath
}
}
{
\newrgbcolor{curcolor}{0 0 0}
\pscustom[linewidth=1,linecolor=curcolor]
{
\newpath
\moveto(522.3,207.2)
\lineto(543.9,207.2)
\lineto(543.9,350.6)
\lineto(522.3,350.6)
\closepath
}
}
{
\newrgbcolor{curcolor}{0 0 0}
\pscustom[linewidth=1,linecolor=curcolor]
{
\newpath
\moveto(543.9,207.2)
\lineto(534.9,207.2)
}
}
{
\newrgbcolor{curcolor}{0 0 0}
\pscustom[linestyle=none,fillstyle=solid,fillcolor=curcolor]
{
\newpath
\moveto(552.58085938,205.878125)
\lineto(552.58085938,206.93867188)
\lineto(555.82109375,206.93867188)
\lineto(555.82109375,205.878125)
\closepath
}
}
{
\newrgbcolor{curcolor}{0 0 0}
\pscustom[linestyle=none,fillstyle=solid,fillcolor=curcolor]
{
\newpath
\moveto(560.66679688,203.3)
\lineto(559.61210938,203.3)
\lineto(559.61210938,210.02070313)
\curveto(559.35820313,209.77851563)(559.02421875,209.53632813)(558.61015625,209.29414063)
\curveto(558.2,209.05195313)(557.83085938,208.8703125)(557.50273438,208.74921875)
\lineto(557.50273438,209.76875)
\curveto(558.09257813,210.04609375)(558.60820313,210.38203125)(559.04960938,210.7765625)
\curveto(559.49101563,211.17109375)(559.80351563,211.55390625)(559.98710938,211.925)
\lineto(560.66679688,211.925)
\closepath
}
}
{
\newrgbcolor{curcolor}{0 0 0}
\pscustom[linestyle=none,fillstyle=solid,fillcolor=curcolor]
{
\newpath
\moveto(563.36796875,207.53632813)
\curveto(563.36796875,208.55195313)(563.47148438,209.36835938)(563.67851563,209.98554688)
\curveto(563.88945313,210.60664063)(564.2,211.08515625)(564.61015625,211.42109375)
\curveto(565.02421875,211.75703125)(565.54375,211.925)(566.16875,211.925)
\curveto(566.6296875,211.925)(567.03398438,211.83125)(567.38164063,211.64375)
\curveto(567.72929688,211.46015625)(568.01640625,211.19257813)(568.24296875,210.84101563)
\curveto(568.46953125,210.49335938)(568.64726563,210.06757813)(568.77617188,209.56367188)
\curveto(568.90507813,209.06367188)(568.96953125,208.38789063)(568.96953125,207.53632813)
\curveto(568.96953125,206.52851563)(568.86601563,205.7140625)(568.65898438,205.09296875)
\curveto(568.45195313,204.47578125)(568.14140625,203.99726563)(567.72734375,203.65742188)
\curveto(567.3171875,203.32148438)(566.79765625,203.15351563)(566.16875,203.15351563)
\curveto(565.340625,203.15351563)(564.69023438,203.45039063)(564.21757813,204.04414063)
\curveto(563.65117188,204.75898438)(563.36796875,205.92304688)(563.36796875,207.53632813)
\closepath
\moveto(564.45195313,207.53632813)
\curveto(564.45195313,206.12617188)(564.61601563,205.18671875)(564.94414063,204.71796875)
\curveto(565.27617188,204.253125)(565.684375,204.02070313)(566.16875,204.02070313)
\curveto(566.653125,204.02070313)(567.059375,204.25507813)(567.3875,204.72382813)
\curveto(567.71953125,205.19257813)(567.88554688,206.13007813)(567.88554688,207.53632813)
\curveto(567.88554688,208.95039063)(567.71953125,209.88984375)(567.3875,210.3546875)
\curveto(567.059375,210.81953125)(566.64921875,211.05195313)(566.15703125,211.05195313)
\curveto(565.67265625,211.05195313)(565.2859375,210.846875)(564.996875,210.43671875)
\curveto(564.63359375,209.91328125)(564.45195313,208.94648438)(564.45195313,207.53632813)
\closepath
}
}
{
\newrgbcolor{curcolor}{0 0 0}
\pscustom[linewidth=1,linecolor=curcolor]
{
\newpath
\moveto(522.3,207.2)
\lineto(531.3,207.2)
\moveto(543.9,231.1)
\lineto(534.9,231.1)
}
}
{
\newrgbcolor{curcolor}{0 0 0}
\pscustom[linestyle=none,fillstyle=solid,fillcolor=curcolor]
{
\newpath
\moveto(552.58085938,229.778125)
\lineto(552.58085938,230.83867187)
\lineto(555.82109375,230.83867187)
\lineto(555.82109375,229.778125)
\closepath
}
}
{
\newrgbcolor{curcolor}{0 0 0}
\pscustom[linestyle=none,fillstyle=solid,fillcolor=curcolor]
{
\newpath
\moveto(556.69414063,229.45)
\lineto(557.8015625,229.54375)
\curveto(557.88359375,229.0046875)(558.07304688,228.5984375)(558.36992188,228.325)
\curveto(558.67070313,228.05546875)(559.03203125,227.92070312)(559.45390625,227.92070312)
\curveto(559.96171875,227.92070312)(560.39140625,228.11210937)(560.74296875,228.49492187)
\curveto(561.09453125,228.87773437)(561.2703125,229.38554687)(561.2703125,230.01835937)
\curveto(561.2703125,230.61992187)(561.10039063,231.09453125)(560.76054688,231.4421875)
\curveto(560.42460938,231.78984375)(559.98320313,231.96367187)(559.43632813,231.96367187)
\curveto(559.09648438,231.96367187)(558.78984375,231.88554687)(558.51640625,231.72929687)
\curveto(558.24296875,231.57695312)(558.028125,231.37773437)(557.871875,231.13164062)
\lineto(556.88164063,231.26054687)
\lineto(557.71367188,235.67265625)
\lineto(561.98515625,235.67265625)
\lineto(561.98515625,234.66484375)
\lineto(558.55742188,234.66484375)
\lineto(558.09453125,232.35625)
\curveto(558.61015625,232.715625)(559.15117188,232.8953125)(559.71757813,232.8953125)
\curveto(560.46757813,232.8953125)(561.10039063,232.63554687)(561.61601563,232.11601562)
\curveto(562.13164063,231.59648437)(562.38945313,230.92851562)(562.38945313,230.11210937)
\curveto(562.38945313,229.33476562)(562.16289063,228.66289062)(561.70976563,228.09648437)
\curveto(561.15898438,227.40117187)(560.40703125,227.05351562)(559.45390625,227.05351562)
\curveto(558.67265625,227.05351562)(558.03398438,227.27226562)(557.53789063,227.70976562)
\curveto(557.04570313,228.14726562)(556.76445313,228.72734375)(556.69414063,229.45)
\closepath
}
}
{
\newrgbcolor{curcolor}{0 0 0}
\pscustom[linewidth=1,linecolor=curcolor]
{
\newpath
\moveto(522.3,231.1)
\lineto(531.3,231.1)
\moveto(543.9,255)
\lineto(534.9,255)
}
}
{
\newrgbcolor{curcolor}{0 0 0}
\pscustom[linestyle=none,fillstyle=solid,fillcolor=curcolor]
{
\newpath
\moveto(552.69804688,255.33632812)
\curveto(552.69804688,256.35195312)(552.8015625,257.16835938)(553.00859375,257.78554688)
\curveto(553.21953125,258.40664063)(553.53007813,258.88515625)(553.94023438,259.22109375)
\curveto(554.35429688,259.55703125)(554.87382813,259.725)(555.49882813,259.725)
\curveto(555.95976563,259.725)(556.3640625,259.63125)(556.71171875,259.44375)
\curveto(557.059375,259.26015625)(557.34648438,258.99257813)(557.57304688,258.64101563)
\curveto(557.79960938,258.29335938)(557.97734375,257.86757813)(558.10625,257.36367188)
\curveto(558.23515625,256.86367188)(558.29960938,256.18789062)(558.29960938,255.33632812)
\curveto(558.29960938,254.32851562)(558.19609375,253.5140625)(557.9890625,252.89296875)
\curveto(557.78203125,252.27578125)(557.47148438,251.79726562)(557.05742188,251.45742187)
\curveto(556.64726563,251.12148437)(556.12773438,250.95351562)(555.49882813,250.95351562)
\curveto(554.67070313,250.95351562)(554.0203125,251.25039062)(553.54765625,251.84414062)
\curveto(552.98125,252.55898437)(552.69804688,253.72304687)(552.69804688,255.33632812)
\closepath
\moveto(553.78203125,255.33632812)
\curveto(553.78203125,253.92617187)(553.94609375,252.98671875)(554.27421875,252.51796875)
\curveto(554.60625,252.053125)(555.01445313,251.82070312)(555.49882813,251.82070312)
\curveto(555.98320313,251.82070312)(556.38945313,252.05507812)(556.71757813,252.52382812)
\curveto(557.04960938,252.99257812)(557.215625,253.93007812)(557.215625,255.33632812)
\curveto(557.215625,256.75039062)(557.04960938,257.68984375)(556.71757813,258.1546875)
\curveto(556.38945313,258.61953125)(555.97929688,258.85195313)(555.48710938,258.85195313)
\curveto(555.00273438,258.85195313)(554.61601563,258.646875)(554.32695313,258.23671875)
\curveto(553.96367188,257.71328125)(553.78203125,256.74648437)(553.78203125,255.33632812)
\closepath
}
}
{
\newrgbcolor{curcolor}{0 0 0}
\pscustom[linewidth=1,linecolor=curcolor]
{
\newpath
\moveto(522.3,255)
\lineto(531.3,255)
\moveto(543.9,278.9)
\lineto(534.9,278.9)
}
}
{
\newrgbcolor{curcolor}{0 0 0}
\pscustom[linestyle=none,fillstyle=solid,fillcolor=curcolor]
{
\newpath
\moveto(552.69804688,277.25)
\lineto(553.80546875,277.34375)
\curveto(553.8875,276.8046875)(554.07695313,276.3984375)(554.37382813,276.125)
\curveto(554.67460938,275.85546875)(555.0359375,275.72070312)(555.4578125,275.72070312)
\curveto(555.965625,275.72070312)(556.3953125,275.91210938)(556.746875,276.29492188)
\curveto(557.0984375,276.67773438)(557.27421875,277.18554688)(557.27421875,277.81835938)
\curveto(557.27421875,278.41992188)(557.10429688,278.89453125)(556.76445313,279.2421875)
\curveto(556.42851563,279.58984375)(555.98710938,279.76367188)(555.44023438,279.76367188)
\curveto(555.10039063,279.76367188)(554.79375,279.68554688)(554.5203125,279.52929688)
\curveto(554.246875,279.37695312)(554.03203125,279.17773438)(553.87578125,278.93164062)
\lineto(552.88554688,279.06054688)
\lineto(553.71757813,283.47265625)
\lineto(557.9890625,283.47265625)
\lineto(557.9890625,282.46484375)
\lineto(554.56132813,282.46484375)
\lineto(554.0984375,280.15625)
\curveto(554.6140625,280.515625)(555.15507813,280.6953125)(555.72148438,280.6953125)
\curveto(556.47148438,280.6953125)(557.10429688,280.43554688)(557.61992188,279.91601562)
\curveto(558.13554688,279.39648438)(558.39335938,278.72851562)(558.39335938,277.91210938)
\curveto(558.39335938,277.13476562)(558.16679688,276.46289062)(557.71367188,275.89648438)
\curveto(557.16289063,275.20117188)(556.4109375,274.85351562)(555.4578125,274.85351562)
\curveto(554.6765625,274.85351562)(554.03789063,275.07226562)(553.54179688,275.50976562)
\curveto(553.04960938,275.94726562)(552.76835938,276.52734375)(552.69804688,277.25)
\closepath
}
}
{
\newrgbcolor{curcolor}{0 0 0}
\pscustom[linewidth=1,linecolor=curcolor]
{
\newpath
\moveto(522.3,278.9)
\lineto(531.3,278.9)
\moveto(543.9,302.8)
\lineto(534.9,302.8)
}
}
{
\newrgbcolor{curcolor}{0 0 0}
\pscustom[linestyle=none,fillstyle=solid,fillcolor=curcolor]
{
\newpath
\moveto(556.67070313,298.9)
\lineto(555.61601563,298.9)
\lineto(555.61601563,305.62070312)
\curveto(555.36210938,305.37851562)(555.028125,305.13632812)(554.6140625,304.89414062)
\curveto(554.20390625,304.65195312)(553.83476563,304.4703125)(553.50664063,304.34921875)
\lineto(553.50664063,305.36875)
\curveto(554.09648438,305.64609375)(554.61210938,305.98203125)(555.05351563,306.3765625)
\curveto(555.49492188,306.77109375)(555.80742188,307.15390625)(555.99101563,307.525)
\lineto(556.67070313,307.525)
\closepath
}
}
{
\newrgbcolor{curcolor}{0 0 0}
\pscustom[linestyle=none,fillstyle=solid,fillcolor=curcolor]
{
\newpath
\moveto(559.371875,303.13632812)
\curveto(559.371875,304.15195312)(559.47539063,304.96835937)(559.68242188,305.58554687)
\curveto(559.89335938,306.20664062)(560.20390625,306.68515625)(560.6140625,307.02109375)
\curveto(561.028125,307.35703125)(561.54765625,307.525)(562.17265625,307.525)
\curveto(562.63359375,307.525)(563.03789063,307.43125)(563.38554688,307.24375)
\curveto(563.73320313,307.06015625)(564.0203125,306.79257812)(564.246875,306.44101562)
\curveto(564.4734375,306.09335937)(564.65117188,305.66757812)(564.78007813,305.16367187)
\curveto(564.90898438,304.66367187)(564.9734375,303.98789062)(564.9734375,303.13632812)
\curveto(564.9734375,302.12851562)(564.86992188,301.3140625)(564.66289063,300.69296875)
\curveto(564.45585938,300.07578125)(564.1453125,299.59726562)(563.73125,299.25742187)
\curveto(563.32109375,298.92148437)(562.8015625,298.75351562)(562.17265625,298.75351562)
\curveto(561.34453125,298.75351562)(560.69414063,299.05039062)(560.22148438,299.64414062)
\curveto(559.65507813,300.35898437)(559.371875,301.52304687)(559.371875,303.13632812)
\closepath
\moveto(560.45585938,303.13632812)
\curveto(560.45585938,301.72617187)(560.61992188,300.78671875)(560.94804688,300.31796875)
\curveto(561.28007813,299.853125)(561.68828125,299.62070312)(562.17265625,299.62070312)
\curveto(562.65703125,299.62070312)(563.06328125,299.85507812)(563.39140625,300.32382812)
\curveto(563.7234375,300.79257812)(563.88945313,301.73007812)(563.88945313,303.13632812)
\curveto(563.88945313,304.55039062)(563.7234375,305.48984375)(563.39140625,305.9546875)
\curveto(563.06328125,306.41953125)(562.653125,306.65195312)(562.1609375,306.65195312)
\curveto(561.6765625,306.65195312)(561.28984375,306.446875)(561.00078125,306.03671875)
\curveto(560.6375,305.51328125)(560.45585938,304.54648437)(560.45585938,303.13632812)
\closepath
}
}
{
\newrgbcolor{curcolor}{0 0 0}
\pscustom[linewidth=1,linecolor=curcolor]
{
\newpath
\moveto(522.3,302.8)
\lineto(531.3,302.8)
\moveto(543.9,326.7)
\lineto(534.9,326.7)
}
}
{
\newrgbcolor{curcolor}{0 0 0}
\pscustom[linestyle=none,fillstyle=solid,fillcolor=curcolor]
{
\newpath
\moveto(556.67070313,322.8)
\lineto(555.61601563,322.8)
\lineto(555.61601563,329.52070313)
\curveto(555.36210938,329.27851563)(555.028125,329.03632813)(554.6140625,328.79414063)
\curveto(554.20390625,328.55195313)(553.83476563,328.3703125)(553.50664063,328.24921875)
\lineto(553.50664063,329.26875)
\curveto(554.09648438,329.54609375)(554.61210938,329.88203125)(555.05351563,330.2765625)
\curveto(555.49492188,330.67109375)(555.80742188,331.05390625)(555.99101563,331.425)
\lineto(556.67070313,331.425)
\closepath
}
}
{
\newrgbcolor{curcolor}{0 0 0}
\pscustom[linestyle=none,fillstyle=solid,fillcolor=curcolor]
{
\newpath
\moveto(559.371875,325.05)
\lineto(560.47929688,325.14375)
\curveto(560.56132813,324.6046875)(560.75078125,324.1984375)(561.04765625,323.925)
\curveto(561.3484375,323.65546875)(561.70976563,323.52070313)(562.13164063,323.52070313)
\curveto(562.63945313,323.52070313)(563.06914063,323.71210938)(563.42070313,324.09492188)
\curveto(563.77226563,324.47773438)(563.94804688,324.98554688)(563.94804688,325.61835938)
\curveto(563.94804688,326.21992188)(563.778125,326.69453125)(563.43828125,327.0421875)
\curveto(563.10234375,327.38984375)(562.6609375,327.56367188)(562.1140625,327.56367188)
\curveto(561.77421875,327.56367188)(561.46757813,327.48554688)(561.19414063,327.32929688)
\curveto(560.92070313,327.17695313)(560.70585938,326.97773438)(560.54960938,326.73164063)
\lineto(559.559375,326.86054688)
\lineto(560.39140625,331.27265625)
\lineto(564.66289063,331.27265625)
\lineto(564.66289063,330.26484375)
\lineto(561.23515625,330.26484375)
\lineto(560.77226563,327.95625)
\curveto(561.28789063,328.315625)(561.82890625,328.4953125)(562.3953125,328.4953125)
\curveto(563.1453125,328.4953125)(563.778125,328.23554688)(564.29375,327.71601563)
\curveto(564.809375,327.19648438)(565.0671875,326.52851563)(565.0671875,325.71210938)
\curveto(565.0671875,324.93476563)(564.840625,324.26289063)(564.3875,323.69648438)
\curveto(563.83671875,323.00117188)(563.08476563,322.65351563)(562.13164063,322.65351563)
\curveto(561.35039063,322.65351563)(560.71171875,322.87226563)(560.215625,323.30976563)
\curveto(559.7234375,323.74726563)(559.4421875,324.32734375)(559.371875,325.05)
\closepath
}
}
{
\newrgbcolor{curcolor}{0 0 0}
\pscustom[linewidth=1,linecolor=curcolor]
{
\newpath
\moveto(522.3,326.7)
\lineto(531.3,326.7)
\moveto(543.9,350.6)
\lineto(534.9,350.6)
}
}
{
\newrgbcolor{curcolor}{0 0 0}
\pscustom[linestyle=none,fillstyle=solid,fillcolor=curcolor]
{
\newpath
\moveto(558.24101563,347.71367187)
\lineto(558.24101563,346.7)
\lineto(552.56328125,346.7)
\curveto(552.55546875,346.95390625)(552.59648438,347.19804687)(552.68632813,347.43242187)
\curveto(552.83085938,347.81914062)(553.06132813,348.2)(553.37773438,348.575)
\curveto(553.69804688,348.95)(554.15898438,349.38359375)(554.76054688,349.87578125)
\curveto(555.69414063,350.64140625)(556.325,351.246875)(556.653125,351.6921875)
\curveto(556.98125,352.14140625)(557.1453125,352.56523437)(557.1453125,352.96367187)
\curveto(557.1453125,353.38164062)(556.99492188,353.73320312)(556.69414063,354.01835937)
\curveto(556.39726563,354.30742187)(556.00859375,354.45195312)(555.528125,354.45195312)
\curveto(555.0203125,354.45195312)(554.6140625,354.29960937)(554.309375,353.99492187)
\curveto(554.0046875,353.69023437)(553.85039063,353.26835937)(553.84648438,352.72929687)
\lineto(552.7625,352.840625)
\curveto(552.83671875,353.64921875)(553.11601563,354.26445312)(553.60039063,354.68632812)
\curveto(554.08476563,355.11210937)(554.73515625,355.325)(555.5515625,355.325)
\curveto(556.37578125,355.325)(557.028125,355.09648437)(557.50859375,354.63945312)
\curveto(557.9890625,354.18242187)(558.22929688,353.61601562)(558.22929688,352.94023437)
\curveto(558.22929688,352.59648437)(558.15898438,352.25859375)(558.01835938,351.9265625)
\curveto(557.87773438,351.59453125)(557.64335938,351.24492187)(557.31523438,350.87773437)
\curveto(556.99101563,350.51054687)(556.45,350.00664062)(555.6921875,349.36601562)
\curveto(555.059375,348.83476562)(554.653125,348.4734375)(554.4734375,348.28203125)
\curveto(554.29375,348.09453125)(554.1453125,347.90507812)(554.028125,347.71367187)
\closepath
}
}
{
\newrgbcolor{curcolor}{0 0 0}
\pscustom[linestyle=none,fillstyle=solid,fillcolor=curcolor]
{
\newpath
\moveto(559.371875,350.93632812)
\curveto(559.371875,351.95195312)(559.47539063,352.76835937)(559.68242188,353.38554687)
\curveto(559.89335938,354.00664062)(560.20390625,354.48515625)(560.6140625,354.82109375)
\curveto(561.028125,355.15703125)(561.54765625,355.325)(562.17265625,355.325)
\curveto(562.63359375,355.325)(563.03789063,355.23125)(563.38554688,355.04375)
\curveto(563.73320313,354.86015625)(564.0203125,354.59257812)(564.246875,354.24101562)
\curveto(564.4734375,353.89335937)(564.65117188,353.46757812)(564.78007813,352.96367187)
\curveto(564.90898438,352.46367187)(564.9734375,351.78789062)(564.9734375,350.93632812)
\curveto(564.9734375,349.92851562)(564.86992188,349.1140625)(564.66289063,348.49296875)
\curveto(564.45585938,347.87578125)(564.1453125,347.39726562)(563.73125,347.05742187)
\curveto(563.32109375,346.72148437)(562.8015625,346.55351562)(562.17265625,346.55351562)
\curveto(561.34453125,346.55351562)(560.69414063,346.85039062)(560.22148438,347.44414062)
\curveto(559.65507813,348.15898437)(559.371875,349.32304687)(559.371875,350.93632812)
\closepath
\moveto(560.45585938,350.93632812)
\curveto(560.45585938,349.52617187)(560.61992188,348.58671875)(560.94804688,348.11796875)
\curveto(561.28007813,347.653125)(561.68828125,347.42070312)(562.17265625,347.42070312)
\curveto(562.65703125,347.42070312)(563.06328125,347.65507812)(563.39140625,348.12382812)
\curveto(563.7234375,348.59257812)(563.88945313,349.53007812)(563.88945313,350.93632812)
\curveto(563.88945313,352.35039062)(563.7234375,353.28984375)(563.39140625,353.7546875)
\curveto(563.06328125,354.21953125)(562.653125,354.45195312)(562.1609375,354.45195312)
\curveto(561.6765625,354.45195312)(561.28984375,354.246875)(561.00078125,353.83671875)
\curveto(560.6375,353.31328125)(560.45585938,352.34648437)(560.45585938,350.93632812)
\closepath
}
}
{
\newrgbcolor{curcolor}{0 0 0}
\pscustom[linewidth=1,linecolor=curcolor]
{
\newpath
\moveto(522.3,350.6)
\lineto(531.3,350.6)
}
}
\end{pspicture}
}
    \captionsetup{width=0.75\linewidth}
    \caption{Impact of Read Size}
    \label{fig:readsize}
\end{figure}

Performance improvement from this prototype only occurs when a program reads 128 KB or more at a time (Figure \ref{fig:readsize}).
This seems to indicate that the ARC or some layer below it may be handling read sizes of 
less than one block differently than those reading one block or larger.
Once at this read size the performance improvement increases to the 10-15\% threshold which is the maximum possible
as shown by previous testing (Figure \ref{fig:OldZFS}).

When reading 1-2 MB at a time, there appears to be a peak where significantly more performance is regained 
(Figure \ref{fig:readsize}).
This ideal case seems to be caused by prefetching within ZFS, which does not consider any reads that are above 1 MB
\cite[{module/zfs/dmu\_zfetch.c}]{zfs}.
Within the DMU, all reads smaller than this are tracked and organized into sequential streams in order to predict
what reads will happen next and ensure that the next blocks in a sequential read are already available 
(Section \ref{chapter:prefetch}).
This prefetching code seems to be a bottleneck in the ZFS code for sequential reads,
as it does a significant amount of accounting around every read operation in order to detect sequences.
However, it only calls \texttt{arc\_read} from the context of the original read 
on blocks it detects will be next in the sequence in order to load them into the ARC.
Thus, this is simply a property of ZFS itself and not related to my changes, though it is an interesting result.

\chapter{Conclusion}

As this work has shown, Non-Uniform Memory Access has a significant impact on ZFS,
a filesystem that has significant in memory caches to speed up operations.
While reading files from memory is still significantly faster than reading them from disk,
ensuring that programs and files end up on the same NUMA node provides significant improvements to read latency and bandwidth.
In an attempt to improve ZFS's performance in these situations, 
I have prototyped one way to reduce accesses to ARC data stored on a different node.
This solution requires invalidating CPU caches by moving a process to a CPU on a different
NUMA node, but it does provide a performance improvement when the files being read are very large, over 1 GB,
and they are being read in large amounts at a time, over 128K.
Because a process is often smaller than the files that it is reading, moving the process instead of its memory 
can reduce the amount of memory moving across the interconnect, limiting the latency impact.

While NUMA balancing is a known approach already implemented in the Linux kernel to mitigate this problem for a process's
own memory, it does not consider external caches that the program accesses such as the ARC in ZFS.
Task migration is an alternative solution that may provide some benefit in very specific situations,
especially when the amount of memory being accessed is very large and the process's other memory allocations are very small.
Future research will determine if these two approaches can work together to improve the performance of a system.
NUMA balancing, if applied to the ARC as well, would likely show very similar results to those presented in this work.

These results should not be taken as proof that task migration is always the better approach.
For situations like the ones tested in this work, with large in-memory structures, 
it may be valuable for operating systems to consider the relative size of a process and the memory it believes that process will
access from another node before deciding which to move when it finds a program accessing large amount of memory from a different
NUMA node. 

Another important contribution of this work is to show that there is much room for improvement in large in-memory caches such as
the ARC when it comes to NUMA.
While ZFS does take NUMA into account when first placing data into the ARC, it is clear that there is still much more that could be done 
for later processes reading this data who might not be on the same node, as my performance measurements of ZFS have shown.
As the large systems that have necessitated a NUMA architecture are also the kinds of systems where a filesystem like ZFS is often
used to store large amounts of data, this improvement would be beneficial for many of its users.

\section{Future Work}

\subsection{Read Size Limitations}
The impact of read size (Figure \ref{fig:readsize}) is an unfortunate wrinkle in otherwise very promising results.
The impact of NUMA on smaller read sizes seems more limited than with larger sizes, so this may in fact be a reflection of the limited
gains that can be made with smaller read sizes.
As discussed previously, the inflection point at 128 KB would suggest that the ARC is handling reads smaller than one block differently,
as 128K is the default maximum block size (Section \ref{chapter:readsizeimpact}).
Also the 1 MB inflection point for best improvement overlapping with the point at which the prefetcher is disabled indicates that
its behavior needs to be taken into account and likely modified.
Like the ARC, the prefetcher likely needs to be aware of which NUMA node the program accessing a particular file is running
on and work to allocate ARC memory closest to that program.

\subsection{Impact on Arbitrary Programs on Long-Running Systems}
The testing environment for these improvements has been, by necessity, rather academic in order to eliminate the influence of other factors.
However, no one reboots these machines every time they need to run a program, and so testing this against arbitrary programs with 
an ARC of a long-running system would be useful future work to prove or disprove whether it can improve real-world application
run times without impacting other programs on the system too much.
It may be that micro-managing the scheduler in this way leads to unbalanced systems over time,
as programs that access the same files are all forced to run on the same node.

The impact of this management style may also be positive, as the scheduler has no knowledge of NUMA memory accesses, 
and might schedule applications poorly when they are frequently requesting data from another node.
As Megiddo and Modha write in their original paper on the ARC, ``real-life workloads possess a great deal of
richness and variation, and do not admit a one-size-fits-all characterization'' \cite{megiddo_dharmendra_ARC}.

\subsection{Interactions with NUMA Balancing and Swapping}
The current prototype implementation assumes blocks don't move in memory, 
which may not be the case when NUMA balancing is enabled, or when swapping occurs.
Retrieving the exact location of the physical buffer before checking it against the current node would partially relieve these issues, 
but it does not solve the problem of potentially splitting a buffer across multiple nodes, which would require handling
each buffer of an ABD separately when considering whether to move a process.

This could be partially handled by locating the page of ARC header's physical buffer at read time, but doing in the standard read code path is likely to be a very expensive operation 
that could reduce the amount of possible improvement by taking significantly more time just to determine this information.
This approach would allow NUMA balancing to be re-enabled and work with task migration.

\subsection{More Complex NUMA Topologies}
Due to only having access to one system with Non-Uniform Memory Access, it is difficult to make 

\subsection{Applicability to More General File Caching Systems}
The general idea of an in-memory cache is a common feature of filesystems, with most on Linux using the page cache.
The Linux page cache uses two CLOCK lists that approximate two LRU caches, one for pages accessed once,
and another for pages accessed multiple times.
While this approach appears superficially similar to the ARC, it does not have the dynamically resizing capabilities of
the ARC that allow it to learn from past cache misses through a ghost cache (Section \ref{chapter:ARC}).
While the ARC is a smarter caching system, the page cache operates with the same general principles,
and so the ideas presented here may be just as applicable to the page cache.

\subsection{Global View of the ARC and of Files}
The current implementation is greedy, considering only one block at a time and believing every block to be equally important, an equal indicator that a process is going to read an entire file.
It also does not consider the size of the file when deciding to move a process,
which results have shown is a critical factor in whether moving a process is actually helpful.

With input from the prefetching code, it could be made smarter, 
only moving a process after it shows itself to be likely to read most of a file.
However, waiting for a certain amount of access also guarantees that those first few accesses are going to be high latency if from a remote node, and might reduce the impact of this improvement on overall average latency and runtime.

The ARC is specifically designed to be concerned only about blocks, and leaves the higher level concepts of
files to the layers above it, specifically the DMU and ZPL.
Thus the process migration decision really should be happening at one of these higher levels,
probably the DMU as that is where prefetching happens.
However, ensuring that the DMU knows where the data it's getting from the ARC is would likely be difficult,
as the DMU is designed to know very little about how the ARC operates, with good reason as it is a clear
abstraction on top of a complex caching layer.





%%-- If you want to add some appendices uncomment \appendix below
%% All this actually does is start calling the "chapters" "appendices"
\appendix
\chapter{Source Code}
\label{sourcecode}

\section{Automatic Testing Code}
\singlespacing
\lstinputlisting[caption={Automatic Testing Python Script},label={lst:autotest},language=Python]{code/autotest.py}

\lstinputlisting[caption={Automatic Testing rc.local},label={lst:rc.local},language=sh]{code/rc.local}

\lstinputlisting[caption={Simple Read Test Script},label={lst:simplereadtest},language=sh]{code/simpleread.sh}

\lstinputlisting[caption={Fio Test Script},label={lst:fiotest},language=sh]{code/fio.sh}

\lstinputlisting[caption={Simple Read Program},label={lst:simpleread},language=C]{code/simpleread.c}

\section{Linux Changes}

\lstinputlisting[caption={Linux Kernel Patch to export do\_migrate\_pages, tested against version 5.4.65},label={lst:linuxpatch},language=diff]{code/linux.patch}

\section{ZFS Changes}
\lstinputlisting[caption={SPL Migrate, module/spl/spl-migrate.c, for ZFS 0.8.5},label={lst:splmigrate},language=C]{code/spl-migrate.c}

\lstinputlisting[caption={ARC Patch for ZFS 0.8.5},label={lst:arcpatch}, language=diff]{code/arc.patch}

\lstinputlisting[caption={ZFS Header and Makefile Changes, for ZFS 0.8.5},label={lst:headermake},language=diff]{code/headermake.patch}

\doublespacing

\chapter{Low-Level Memory Allocation in the SPL}
My initial research into ZFS led me to dive into the very low level details of how memory allocation
works in ZFS, as it is actually rather different from how memory allocation works in other Linux kernel modules.
My attempts to make these memory allocations more NUMA-aware had a substantially negative effect on the performance of ZFS.
This is likely because I had to substantially constrain the scheduling of the memory allocation threads.
I ended up taking my project in an entirely different direction after this, and thus these details of how this part of 
ZFS works are mostly irrelevant to my work, so I have confined it to this appendix.

\subsection{kmem}
The SPL contains a wrapper on kmem alloc and free, which attempts to emulate a more Solaris-like allocation process for small allocations,
preventing them from failing, and instead retrying repeatedly until it succeeds
\cite[{module/spl/spl-kmem.c}]{zfs}.
It also includes debug functions to track memory allocations and find memory leaks.
These functions call the per-NUMA node \texttt{kmalloc\_node}, but with no NUMA node preference, 
and no such option exposed to the ZFS module to control its allocations.
However, this function respects the calling thread's memory policy, which defaults to preferring allocations from the local NUMA node.

For larger allocations, ZFS uses \texttt{vmem\_alloc} to allocate chunks of virtual memory, which uses Linux's \texttt{\_\_vmalloc} pretty much directly, 
and again uses a loop to prevent failure, which the code claims is unlikely with \texttt{\_\_vmalloc} \cite[{module/spl/spl-kmem.c}]{zfs}.

\subsection{SLAB Allocator}
The SPL also implements it own slab allocator for  \texttt{kmem\_cache}, instead of just providing a wrapper as it does for other kinds of memory 
allocation \cite[{module/spl/spl-kmem-cache.c}]{zfs}. 
This allows it to diverge significantly from the Linux implementation.
Unlike the Linux implementation, which, according to the comments in spl-kmem-cache was inspired by Solaris' version, it supports destructors for cache
objects (which have been removed from Linux), and bases its allocations on virtual addresses, removing the need for large chunks of contiguous memory
when large allocations are requested (when objects smaller than 16KB are requested, it falls back to the Linux slab allocator).
It also implements cache expiration, returning objects that have not been accessed in a few seconds back to the cache 
(Linux used to only do so in the case of a memory shortage, though it now has expiration after 2 seconds \cite{lameter_slab_2014}).
Finally, the SPL slab allocator implements a optimization known as a ``cache magazine.''
These retain a per-cpu cache of recently-freed objects, which can be reused without locking.

Internally, the slab allocator supports using Linux's \texttt{\_\_get\_free\_pages} with some specific flags, 
but does not use it except in deadlock situations, instead preferring \texttt{\_\_vmalloc} to allocate objects in virtual memory.
It also uses an SPL task queue for handling memory allocation and cache expiration.

While kmalloc\_node can be used to enforce memory node binding, there is no direct way to influence \texttt{\_\_get\_free\_pages}
or \texttt{\_\_vmalloc}.
However, both functions respect the current thread's memory policy.
\texttt{\_\_vmalloc} technically has a similar function, \texttt{\_\_vmalloc\_node}, but this symbol is not exported to modules \cite[{mm/vmalloc.c}]{linux}.
It seems that ZFS is using this function in an unusual way, allocating memory from the HIGHMEM zone, while the expected use of the function, per the
more standard (and exported) \texttt{vmalloc} and \texttt{vmalloc\_node} functions, allocates from the NORMAL zone, and the comment above the function states 
``Any use of gfp flags outside of [the NORMAL zone] should be consulted with [memory management] people.'' \cite[{mm/vmalloc.c}]{linux}.
In newer versions of Linux \texttt{\_\_vmalloc\_node} is exported but only with a specific testing configuration that regular kernels are unlikely to use
\cite{newlinux}.

Due to the nature of a slab allocator, which subdivides large memory allocations for smaller objects,
relying on memory policy is not enough.
Linux's slab allocator was made NUMA-aware by having per-node and per-cpu lists, though this leads to a large increase in memory usage by this


\addcontentsline{toc}{chapter}{Bibliography}
\printbibliography

%% Vita is optional
%\makeThesisVita{vita}


\end{document}
